\documentclass[crop=false,class=book]{standalone}
%----------------------------Preamble-------------------------------%
%---------------------------Packages----------------------------%
\usepackage{geometry}
\geometry{b5paper, margin=1.0in}
\usepackage[T1]{fontenc}
\usepackage{graphicx, float}            % Graphics/Images.
\usepackage{natbib}                     % For bibliographies.
\bibliographystyle{agsm}                % Bibliography style.
\usepackage[french, english]{babel}     % Language typesetting.
\usepackage[dvipsnames]{xcolor}         % Color names.
\usepackage{listings}                   % Verbatim-Like Tools.
\usepackage{mathtools, esint, mathrsfs} % amsmath and integrals.
\usepackage{amsthm, amsfonts, amssymb}  % Fonts and theorems.
\usepackage{tcolorbox}                  % Frames around theorems.
\usepackage{upgreek}                    % Non-Italic Greek.
\usepackage{fmtcount, etoolbox}         % For the \book{} command.
\usepackage[newparttoc]{titlesec}       % Formatting chapter, etc.
\usepackage{titletoc}                   % Allows \book in toc.
\usepackage[nottoc]{tocbibind}          % Bibliography in toc.
\usepackage[titles]{tocloft}            % ToC formatting.
\usepackage{pgfplots, tikz}             % Drawing/graphing tools.
\usepackage{imakeidx}                   % Used for index.
\usetikzlibrary{
    calc,                   % Calculating right angles and more.
    angles,                 % Drawing angles within triangles.
    arrows.meta,            % Latex and Stealth arrows.
    quotes,                 % Adding labels to angles.
    positioning,            % Relative positioning of nodes.
    decorations.markings,   % Adding arrows in the middle of a line.
    patterns,
    arrows
}                                       % Libraries for tikz.
\pgfplotsset{compat=1.9}                % Version of pgfplots.
\usepackage[font=scriptsize,
            labelformat=simple,
            labelsep=colon]{subcaption} % Subfigure captions.
\usepackage[font={scriptsize},
            hypcap=true,
            labelsep=colon]{caption}    % Figure captions.
\usepackage[pdftex,
            pdfauthor={Ryan Maguire},
            pdftitle={Mathematics and Physics},
            pdfsubject={Mathematics, Physics, Science},
            pdfkeywords={Mathematics, Physics, Computer Science, Biology},
            pdfproducer={LaTeX},
            pdfcreator={pdflatex}]{hyperref}
\hypersetup{
    colorlinks=true,
    linkcolor=blue,
    filecolor=magenta,
    urlcolor=Cerulean,
    citecolor=SkyBlue
}                           % Colors for hyperref.
\usepackage[toc,acronym,nogroupskip,nopostdot]{glossaries}
\usepackage{glossary-mcols}
%------------------------Theorem Styles-------------------------%
\theoremstyle{plain}
\newtheorem{theorem}{Theorem}[section]

% Define theorem style for default spacing and normal font.
\newtheoremstyle{normal}
    {\topsep}               % Amount of space above the theorem.
    {\topsep}               % Amount of space below the theorem.
    {}                      % Font used for body of theorem.
    {}                      % Measure of space to indent.
    {\bfseries}             % Font of the header of the theorem.
    {}                      % Punctuation between head and body.
    {.5em}                  % Space after theorem head.
    {}

% Italic header environment.
\newtheoremstyle{thmit}{\topsep}{\topsep}{}{}{\itshape}{}{0.5em}{}

% Define environments with italic headers.
\theoremstyle{thmit}
\newtheorem*{solution}{Solution}

% Define default environments.
\theoremstyle{normal}
\newtheorem{example}{Example}[section]
\newtheorem{definition}{Definition}[section]
\newtheorem{problem}{Problem}[section]

% Define framed environment.
\tcbuselibrary{most}
\newtcbtheorem[use counter*=theorem]{ftheorem}{Theorem}{%
    before=\par\vspace{2ex},
    boxsep=0.5\topsep,
    after=\par\vspace{2ex},
    colback=green!5,
    colframe=green!35!black,
    fonttitle=\bfseries\upshape%
}{thm}

\newtcbtheorem[auto counter, number within=section]{faxiom}{Axiom}{%
    before=\par\vspace{2ex},
    boxsep=0.5\topsep,
    after=\par\vspace{2ex},
    colback=Apricot!5,
    colframe=Apricot!35!black,
    fonttitle=\bfseries\upshape%
}{ax}

\newtcbtheorem[use counter*=definition]{fdefinition}{Definition}{%
    before=\par\vspace{2ex},
    boxsep=0.5\topsep,
    after=\par\vspace{2ex},
    colback=blue!5!white,
    colframe=blue!75!black,
    fonttitle=\bfseries\upshape%
}{def}

\newtcbtheorem[use counter*=example]{fexample}{Example}{%
    before=\par\vspace{2ex},
    boxsep=0.5\topsep,
    after=\par\vspace{2ex},
    colback=red!5!white,
    colframe=red!75!black,
    fonttitle=\bfseries\upshape%
}{ex}

\newtcbtheorem[auto counter, number within=section]{fnotation}{Notation}{%
    before=\par\vspace{2ex},
    boxsep=0.5\topsep,
    after=\par\vspace{2ex},
    colback=SeaGreen!5!white,
    colframe=SeaGreen!75!black,
    fonttitle=\bfseries\upshape%
}{not}

\newtcbtheorem[use counter*=remark]{fremark}{Remark}{%
    fonttitle=\bfseries\upshape,
    colback=Goldenrod!5!white,
    colframe=Goldenrod!75!black}{ex}

\newenvironment{bproof}{\textit{Proof.}}{\hfill$\square$}
\tcolorboxenvironment{bproof}{%
    blanker,
    breakable,
    left=3mm,
    before skip=5pt,
    after skip=10pt,
    borderline west={0.6mm}{0pt}{green!80!black}
}

\AtEndEnvironment{lexample}{$\hfill\textcolor{red}{\blacksquare}$}
\newtcbtheorem[use counter*=example]{lexample}{Example}{%
    empty,
    title={Example~\theexample},
    boxed title style={%
        empty,
        size=minimal,
        toprule=2pt,
        top=0.5\topsep,
    },
    coltitle=red,
    fonttitle=\bfseries,
    parbox=false,
    boxsep=0pt,
    before=\par\vspace{2ex},
    left=0pt,
    right=0pt,
    top=3ex,
    bottom=1ex,
    before=\par\vspace{2ex},
    after=\par\vspace{2ex},
    breakable,
    pad at break*=0mm,
    vfill before first,
    overlay unbroken={%
        \draw[red, line width=2pt]
            ([yshift=-1.2ex]title.south-|frame.west) to
            ([yshift=-1.2ex]title.south-|frame.east);
        },
    overlay first={%
        \draw[red, line width=2pt]
            ([yshift=-1.2ex]title.south-|frame.west) to
            ([yshift=-1.2ex]title.south-|frame.east);
    },
}{ex}

\AtEndEnvironment{ldefinition}{$\hfill\textcolor{Blue}{\blacksquare}$}
\newtcbtheorem[use counter*=definition]{ldefinition}{Definition}{%
    empty,
    title={Definition~\thedefinition:~{#1}},
    boxed title style={%
        empty,
        size=minimal,
        toprule=2pt,
        top=0.5\topsep,
    },
    coltitle=Blue,
    fonttitle=\bfseries,
    parbox=false,
    boxsep=0pt,
    before=\par\vspace{2ex},
    left=0pt,
    right=0pt,
    top=3ex,
    bottom=0pt,
    before=\par\vspace{2ex},
    after=\par\vspace{1ex},
    breakable,
    pad at break*=0mm,
    vfill before first,
    overlay unbroken={%
        \draw[Blue, line width=2pt]
            ([yshift=-1.2ex]title.south-|frame.west) to
            ([yshift=-1.2ex]title.south-|frame.east);
        },
    overlay first={%
        \draw[Blue, line width=2pt]
            ([yshift=-1.2ex]title.south-|frame.west) to
            ([yshift=-1.2ex]title.south-|frame.east);
    },
}{def}

\AtEndEnvironment{ltheorem}{$\hfill\textcolor{Green}{\blacksquare}$}
\newtcbtheorem[use counter*=theorem]{ltheorem}{Theorem}{%
    empty,
    title={Theorem~\thetheorem:~{#1}},
    boxed title style={%
        empty,
        size=minimal,
        toprule=2pt,
        top=0.5\topsep,
    },
    coltitle=Green,
    fonttitle=\bfseries,
    parbox=false,
    boxsep=0pt,
    before=\par\vspace{2ex},
    left=0pt,
    right=0pt,
    top=3ex,
    bottom=-1.5ex,
    breakable,
    pad at break*=0mm,
    vfill before first,
    overlay unbroken={%
        \draw[Green, line width=2pt]
            ([yshift=-1.2ex]title.south-|frame.west) to
            ([yshift=-1.2ex]title.south-|frame.east);},
    overlay first={%
        \draw[Green, line width=2pt]
            ([yshift=-1.2ex]title.south-|frame.west) to
            ([yshift=-1.2ex]title.south-|frame.east);
    }
}{thm}

%--------------------Declared Math Operators--------------------%
\DeclareMathOperator{\adjoint}{adj}         % Adjoint.
\DeclareMathOperator{\Card}{Card}           % Cardinality.
\DeclareMathOperator{\curl}{curl}           % Curl.
\DeclareMathOperator{\diam}{diam}           % Diameter.
\DeclareMathOperator{\dist}{dist}           % Distance.
\DeclareMathOperator{\Div}{div}             % Divergence.
\DeclareMathOperator{\Erf}{Erf}             % Error Function.
\DeclareMathOperator{\Erfc}{Erfc}           % Complementary Error Function.
\DeclareMathOperator{\Ext}{Ext}             % Exterior.
\DeclareMathOperator{\GCD}{GCD}             % Greatest common denominator.
\DeclareMathOperator{\grad}{grad}           % Gradient
\DeclareMathOperator{\Ima}{Im}              % Image.
\DeclareMathOperator{\Int}{Int}             % Interior.
\DeclareMathOperator{\LC}{LC}               % Leading coefficient.
\DeclareMathOperator{\LCM}{LCM}             % Least common multiple.
\DeclareMathOperator{\LM}{LM}               % Leading monomial.
\DeclareMathOperator{\LT}{LT}               % Leading term.
\DeclareMathOperator{\Mod}{mod}             % Modulus.
\DeclareMathOperator{\Mon}{Mon}             % Monomial.
\DeclareMathOperator{\multideg}{mutlideg}   % Multi-Degree (Graphs).
\DeclareMathOperator{\nul}{nul}             % Null space of operator.
\DeclareMathOperator{\Ord}{Ord}             % Ordinal of ordered set.
\DeclareMathOperator{\Prin}{Prin}           % Principal value.
\DeclareMathOperator{\proj}{proj}           % Projection.
\DeclareMathOperator{\Refl}{Refl}           % Reflection operator.
\DeclareMathOperator{\rk}{rk}               % Rank of operator.
\DeclareMathOperator{\sgn}{sgn}             % Sign of a number.
\DeclareMathOperator{\sinc}{sinc}           % Sinc function.
\DeclareMathOperator{\Span}{Span}           % Span of a set.
\DeclareMathOperator{\Spec}{Spec}           % Spectrum.
\DeclareMathOperator{\supp}{supp}           % Support
\DeclareMathOperator{\Tr}{Tr}               % Trace of matrix.
%--------------------Declared Math Symbols--------------------%
\DeclareMathSymbol{\minus}{\mathbin}{AMSa}{"39} % Unary minus sign.
%------------------------New Commands---------------------------%
\DeclarePairedDelimiter\norm{\lVert}{\rVert}
\DeclarePairedDelimiter\ceil{\lceil}{\rceil}
\DeclarePairedDelimiter\floor{\lfloor}{\rfloor}
\newcommand*\diff{\mathop{}\!\mathrm{d}}
\newcommand*\Diff[1]{\mathop{}\!\mathrm{d^#1}}
\renewcommand*{\glstextformat}[1]{\textcolor{RoyalBlue}{#1}}
\renewcommand{\glsnamefont}[1]{\textbf{#1}}
\renewcommand\labelitemii{$\circ$}
\renewcommand\thesubfigure{%
    \arabic{chapter}.\arabic{figure}.\arabic{subfigure}}
\addto\captionsenglish{\renewcommand{\figurename}{Fig.}}
\numberwithin{equation}{section}

\renewcommand{\vector}[1]{\boldsymbol{\mathrm{#1}}}

\newcommand{\uvector}[1]{\boldsymbol{\hat{\mathrm{#1}}}}
\newcommand{\topspace}[2][]{(#2,\tau_{#1})}
\newcommand{\measurespace}[2][]{(#2,\varSigma_{#1},\mu_{#1})}
\newcommand{\measurablespace}[2][]{(#2,\varSigma_{#1})}
\newcommand{\manifold}[2][]{(#2,\tau_{#1},\mathcal{A}_{#1})}
\newcommand{\tanspace}[2]{T_{#1}{#2}}
\newcommand{\cotanspace}[2]{T_{#1}^{*}{#2}}
\newcommand{\Ckspace}[3][\mathbb{R}]{C^{#2}(#3,#1)}
\newcommand{\funcspace}[2][\mathbb{R}]{\mathcal{F}(#2,#1)}
\newcommand{\smoothvecf}[1]{\mathfrak{X}(#1)}
\newcommand{\smoothonef}[1]{\mathfrak{X}^{*}(#1)}
\newcommand{\bracket}[2]{[#1,#2]}

%------------------------Book Command---------------------------%
\makeatletter
\renewcommand\@pnumwidth{1cm}
\newcounter{book}
\renewcommand\thebook{\@Roman\c@book}
\newcommand\book{%
    \if@openright
        \cleardoublepage
    \else
        \clearpage
    \fi
    \thispagestyle{plain}%
    \if@twocolumn
        \onecolumn
        \@tempswatrue
    \else
        \@tempswafalse
    \fi
    \null\vfil
    \secdef\@book\@sbook
}
\def\@book[#1]#2{%
    \refstepcounter{book}
    \addcontentsline{toc}{book}{\bookname\ \thebook:\hspace{1em}#1}
    \markboth{}{}
    {\centering
     \interlinepenalty\@M
     \normalfont
     \huge\bfseries\bookname\nobreakspace\thebook
     \par
     \vskip 20\p@
     \Huge\bfseries#2\par}%
    \@endbook}
\def\@sbook#1{%
    {\centering
     \interlinepenalty \@M
     \normalfont
     \Huge\bfseries#1\par}%
    \@endbook}
\def\@endbook{
    \vfil\newpage
        \if@twoside
            \if@openright
                \null
                \thispagestyle{empty}%
                \newpage
            \fi
        \fi
        \if@tempswa
            \twocolumn
        \fi
}
\newcommand*\l@book[2]{%
    \ifnum\c@tocdepth >-3\relax
        \addpenalty{-\@highpenalty}%
        \addvspace{2.25em\@plus\p@}%
        \setlength\@tempdima{3em}%
        \begingroup
            \parindent\z@\rightskip\@pnumwidth
            \parfillskip -\@pnumwidth
            {
                \leavevmode
                \Large\bfseries#1\hfill\hb@xt@\@pnumwidth{\hss#2}
            }
            \par
            \nobreak
            \global\@nobreaktrue
            \everypar{\global\@nobreakfalse\everypar{}}%
        \endgroup
    \fi}
\newcommand\bookname{Book}
\renewcommand{\thebook}{\texorpdfstring{\Numberstring{book}}{book}}
\providecommand*{\toclevel@book}{-2}
\makeatother
\titleformat{\part}[display]
    {\Large\bfseries}
    {\partname\nobreakspace\thepart}
    {0mm}
    {\Huge\bfseries}
\titlecontents{part}[0pt]
    {\large\bfseries}
    {\partname\ \thecontentslabel: \quad}
    {}
    {\hfill\contentspage}
\titlecontents{chapter}[0pt]
    {\bfseries}
    {\chaptername\ \thecontentslabel:\quad}
    {}
    {\hfill\contentspage}
\newglossarystyle{longpara}{%
    \setglossarystyle{long}%
    \renewenvironment{theglossary}{%
        \begin{longtable}[l]{{p{0.25\hsize}p{0.65\hsize}}}
    }{\end{longtable}}%
    \renewcommand{\glossentry}[2]{%
        \glstarget{##1}{\glossentryname{##1}}%
        &\glossentrydesc{##1}{~##2.}
        \tabularnewline%
        \tabularnewline
    }%
}
\newglossary[not-glg]{notation}{not-gls}{not-glo}{Notation}
\newcommand*{\newnotation}[4][]{%
    \newglossaryentry{#2}{type=notation, name={\textbf{#3}, },
                          text={#4}, description={#4},#1}%
}
%--------------------------LENGTHS------------------------------%
% Spacings for the Table of Contents.
\addtolength{\cftsecnumwidth}{1ex}
\addtolength{\cftsubsecindent}{1ex}
\addtolength{\cftsubsecnumwidth}{1ex}
\addtolength{\cftfignumwidth}{1ex}
\addtolength{\cfttabnumwidth}{1ex}

% Indent and paragraph spacing.
\setlength{\parindent}{0em}
\setlength{\parskip}{0em}
%----------------------------GLOSSARY-------------------------------%
\makeglossaries
\loadglsentries{../../../glossary}
\loadglsentries{../../../acronym}
%--------------------------Main Document----------------------------%
\begin{document}
\chapter{UML General Education}
    \section{Western Civilization}
        \subsection{Notes for Exam I}
            \begin{itemize}[noitemsep]
                \item \textbf{Cuneiform:}
                    First of writing. Means wedge shaped in latin.
                \item \textbf{Monarchy:}
                    2700 BC in Sumer. Means ``Rule by one.''
                \item \textbf{Polytheistic:}
                    Mesopotamians worshipped many gods.
                \item \textbf{Anthropomorphic:}
                    Gods resembled humans
                \item \textbf{Ziggurat:}
                    Mesopotamian Religious Building.
                \item \textbf{Hieroglyphics:}
                    Egyptian writing system
                \item \textbf{Monotheism}:
                    Belief in one god, first practiced
                    by Israelites.
                \item \textbf{Polis:}
                    Greek city-state. Major city (capital) and
                    surrounding area with 5-10k people
                \item \textbf{Hoplites:}
                    Greek soldiers. Farmers when not fighting.
                    Infantry/foot soldier.
                \item \textbf{Phalanx:}
                    Greek infantry formation during battle.
                \item \textbf{Battle of Marathon:}
                    Persians land at Attica, defeated by smaller
                    Athenian army.
                \item \textbf{Battle of Thermopolae:}
                    7,000 greeks defend mountains. Persians find
                    alternate route. All but 300 spartans leave.
                    The movie 300 happens.
                \item \textbf{Democracy:}
                    ``Rule of the people,'' ordinary people
                    set agenda and vote.
                \item \textbf{Pericles:}
                    Athenian Politician/General who
                    promoted democracy.
                \item \textbf{Homer:}
                    Greek poet, wrote Illiad and Odyssey.
                \item \textbf{Rhetoric:}
                    The art of speaking and argumentation.
                \item \textbf{Herodotus:}
                    Father of history. Wrote the Histories.
                    Interviewed people for their stories.
                \item \textbf{Plato:}
                    Disciple of Socrates. Athenian philosopher.
                    Theory of forms. Created an academy in Athens.
                \item \textbf{Aristotle:}
                    Emphasized importance of scientific observation.
                    Wrote politics. Lyceum was his school.
                \item \textbf{Hippocrates:}
                    Father of medicine. Namesake of Hippocratic
                    oath. Illness does not come from religion.
                \item \textbf{Hellenistic:}
                    Spread and transformation of Greek culture and
                    civilization through Alexander the Great.
                \item \textbf{Alexandria, Egypt:}
                    Major center of Hellenistic Greece. 331 BC.
                    Had museums and libraries.
                \item \textbf{Paterfamilias:}
                    Roman families were patriarchal, governed
                    by oldest living male.
                \item \textbf{Republic:}
                    Commonwealth, or public thing. Not democratic
                    in Rome.
                \item \textbf{Cicero:}
                    Greco-Roman, trained to be a lawyer,
                    also studied oratory and philosophy.
                    Served as Latin Conduit
                    of Greek Thought.
                \item \textbf{Hannibal:}
                    Carthaginian leader during the second punic war.
                    Had elephants.
                \item \textbf{Spartacus:}
                    Roman slave who led slave revolt known as the
                    Third Servile War against Rome. Not successful.
                \item \textbf{Caesar:}
                    General of Roman Army. Conquered Gaul (France).
                    Also a politician. Upon learning the senate
                    wants him removed from power, he invades
                    Italy after crossing the Rubicon.
                    Becomes dictator for life.
                \item \textbf{Cleopatra:}
                    Final Hellenistic queen of Egypt. Caesar’s
                    lover. Had a kid together. Goes to war
                    with Octavian.
                    Loses at Actium.
                \item \textbf{Pax Romana:}
                    Roman Peace. Era of peace established by
                    Augustus Ceasar, Julius’ nephew.
                \item \textbf{Aeneid:}
                    Major poetic piece during the Pax Romana
                    written by Virgil.
                \item \textbf{Marcus Aurelius:}
                    Emperor from 161 A.D. to 180 A.D. Called the
                    enlightened emperor. Devoted to Stoic
                    Philosophy. Wrote the Meditations.
                \item \textbf{Diocletian:}
                    Ended era of crisis that struck Rome for about
                    50 years. Created beneficial reforms. Created
                    tetrarchy - 4 emperors instead of 1.
                \item \textbf{Messiah:}
                    Anointed one. Savior. 
                \item \textbf{Jesus of Nazareth:}
                    Itinerant rabbi preached coming of
                    ``Kingdom of God.'' Gains large following.
                    12 apostles/disciples. The bible.
                \item \textbf{Gospel:}
                    ``Good news.'' Writings preached by Jews about
                    Jesus and the coming of the ``Kingdom of God.''
                \item \textbf{Paul of Tarsus:}
                    Hated Jesus then loved Jesus after meeting him.
                    Brought the Gospel to Rome’s non-Jews.
                \item \textbf{Baptism:}
                    Rite of initiation. Pouring water on person.
                \item \textbf{The Eucharist:}
                    Who wants some yum yums, I brought snacks.
                \item \textbf{Martyrs:}
                    Christians who died as witnesses to their faith.
                    Dying for a cause.
                \item \textbf{Emperor Constantine the Great:}
                    Went from pagan to christian. Stopped the
                    persecution of christians.
                \item \textbf{Council of Nicaea:}
                    Group of people deciding whether Jesus was a
                    person of a god. They decided he was god-like.
                \item \textbf{Augustine of Hippo:}
                    Hippo is a city in North Africa. Famous convert
                    to christianity. Church Father. Wrote
                    Confessions and City of God.  
            \end{itemize}
        \subsection{Notes for Exam II}
            \begin{itemize}[noitemsep]
                \item \textbf{Atilla:}
                    Important leader of the huns. Attacked Gaul and
                    Italy from 451 to 452. Spoke with Pope Leo the
                    Great and did not destroy Rome.
                \item \textbf{The Franks:}
                    Germanic group the settled in France and Germany
                    after the dissolution of the Roman Empire.
                    Originally came from the area of the North Sea.
                    First major ruler was Clovis who founded the
                    Merovingian Dynasty.
                \item \textbf{Angles and Saxons:}
                    Germanic tribe from the North Sea, raided
                    Britannia and eventually permanently settled
                    the southeastern coast of modern day Britain.
                \item \textbf{Constantinople:}
                    Capital of Eastern Rome (Byzantium).
                    Modern day Istanbul.
                \item \textbf{Hagia Sophia:}
                    ``Holy Wisdom,'' in greek. Super cool church
                    that is a Wonder in Civ 5. I think it gives +5
                    Happiness in the city it is built in.
                    Constructed under Justinian. Built in 537.
                \item \textbf{Gregory the Great:}
                    A pope in late antiquity. Self described
                    ``Servant of the servants of God.'' Supreme
                    spiritual leader of the latin church. Enforced
                    celibacy amongst the priests.
                \item \textbf{Monasticism:}
                    Solitary asceticism as a way of life away from
                    the world. Christians were obsessed with
                    asceticism, or self-denial. Things like
                    fasting/starving, abstinence, shit like that.
                    People would flee to the desert and seek
                    salvation. There were two types of Monks.
                    Hermits who live alone, and communal monks.
                \item \textbf{Muhammed:}
                    The prophet of Islam. Kind of a dick.
                \item \textbf{Mosque:}
                A church for the follows of Islam.
                \item \textbf{Quran:}
                    The bible for Muslims.
                \item \textbf{The Caliphs:}
                    The successors to Muhammed upon his death. One
                    of them was Ali, who was assassinated in 661.
                    His death caused the Sunni-Shia scism.
                \item \textbf{Greek Fire:}
                    A weapon used by the Byzantium’s. Used to spray
                    fire onto an enemy ship.
                \item \textbf{Icons:}
                    Devotional images that were a popular form of
                    worship amongst christians.
                \item \textbf{Charles Martel:}
                    Carolingian who stopped the Muslim conquest in
                    France at the Battle of Poiters.
                \item \textbf{Charlemagne:}
                    French for Charles the Great. A Carolingian
                    ruler in Modern day France. Ruled from Aachen.
                    Eventually became emperor. 
                \item \textbf{Vikings:}
                    Northern invaders of Danish and Norwegian
                    descent. They traveled in longboats and raided
                    the coasts of France.
                \item \textbf{Serfs:}
                    Laboring/working/peasant class in the early
                    middle ages.
                \item \textbf{Vassal:}
                    Lord’s man. Fought for a lord and served him on
                    a voluntary basis. The lord, in exchange,
                    provided him with food, clothing, shelter, arms,
                    and a fief (Piece of land)
                \item \textbf{Guild:}
                    Institutions in the middle ages formed by
                    merchants and artisans to promote economic
                    regulations and self-protection.
                \item \textbf{Usury:}
                    A loan is provided and then repaid in
                    installments with interest. The church banned
                    this because of reasons.
                \item \textbf{King Alfred the Great:}
                    Ruler of Wessex who raised an army in 878 and
                    defeated them. Became the first King of England.
                \item \textbf{Battle of Hastings:}
                    Decisive battle near Hastings, England where
                    William the Conquerer defeated Harold the Saxons
                    forces.
                \item \textbf{William the Conquerer:}
                    The new king of England following the norman
                    conquest of England which includes the Battle of
                    Hastings.
                \item \textbf{Magna Carta:}
                    Document signed by King John agreeing to not be
                    a total prick all the time.
                \item \textbf{Reconquista:}
                    The Christian reconquering of modern day
                    Spain from the Muslims.
                \item \textbf{Cardinals:}
                    Prelates of the Roman Church known for their
                    red robes and hats.
                \item \textbf{Great Schism:}
                    The Latin and Greek churches were no longer in
                    communion and broke apart. Still true today.
                \item \textbf{Investiture Conflict:}
                    Emperors appointed bishops. Pope Gregory VII was
                    like ``Aw hell nah.''
                    Emperor Henry IV said ``lol, k.''
                    The pope excommunicates him. 
                \item \textbf{Thomas Becket:}
                    Archbishop of Canterbury. The king wanted
                    crime-committing clergy to be tried in secular
                    courts. This guy said no. He was assassinated.
                \item \textbf{Francis of Assisi:}
                    Founded the order of Franciscans, a group of
                    “friars,” who went out into the world to serve
                    their lord. Son of a wealthy cloth merchant.
                \item \textbf{Dominicans:}
                    Group of friars founded by the Spanish priest
                    and missionary Dominican Guzman
                \item \textbf{Medieval Inquisition:}
                    Courts set up to find and punish heretics.
                    Seriously messed up stuff went down.
                \item \textbf{Transubstantiation:}
                    When the priest does priestly things to wine and
                    bread it literally becomes Jesus’s body.
                    Christians enter an age of cannibalism.
                \item \textbf{Purgatory:}
                    The place you go to when you’re not good enough
                    for heaven, yet not a complete dick.
                \item \textbf{University:}
                    A place where students go to learn and pay way
                    to fucking much to do so.
                \item \textbf{Scholasticism:}
                    Western Europeans rediscovered Aristotle’s work
                    and used logic in their intellectual pursuits.
                \item \textbf{Thomas Aquinas:}
                    Celebrated Dominican scholastic friar.
                    Wrote the Summa Theologica.
                \item \textbf{Troubadours:}
                    Musicians and poets who performed at courts
                \item \textbf{Chivalry:}
                    The code of knights. Basically, don’t be a dick
                \item \textbf{Vernacular Languages:}
                    The common languages. French, Italian, and
                    Spanish had descended from Latin. English from
                    Germanic language.
                \item \textbf{Gothic:}
                    A style of architecture that has its origins
                    in 13th century France.
                \item \textbf{Eleanor of Aquitaine:}
                    French landowner (Lady form of lord). Became
                    wife of king Louis VII and later king Henry II.
                \item \textbf{Black Death:}
                    Disease the came from Asia and killed a
                    lot of people.
                \item \textbf{Medici:}
                    Banking family from Florence. Became the
                    Papal bankers.
                \item \textbf{Hanseatic League:}
                    Conglomeration of 100+ cities that came
                    together for financial purposes.
                \item \textbf{Hundred Years War:}
                    War between England and France that lasted
                    116 years. 
                \item \textbf{Joan of Arc:}
                    French teenage girl who had a vision from
                    God to lead the french to victory in
                    the 100 years war.
                \item \textbf{Ferdinand and Isabella:}
                    Spanish leaders who completed the Reconquista.
                \item \textbf{Spanish Inquisition:}
                    Spanish religious court used to get rid of
                    Jewish practices in the land.
            \end{itemize}
    \section{College Writing}
        \subsection{Vocabulary}
            \begin{itemize}[noitemsep]
                \item \textbf{Abiding:} Of a feeling or memory lasting
                      a long time; enduring.
                \item \textbf{Abhorrence:} A feeling of repulsion;
                      disgusted, loathing.
                \item \textbf{Acquiesce:} Accept something reluctantly
                      but without protest.
                \item \textbf{Addle:} Make unable to think clearly; confuse.
                \item \textbf{Adulation:} Obsequious flattery;
                      excessive admiration or praise.
                \item \textbf{Affable:} Friendly, good-natured, or easy
                      to talk to.
                \item \textbf{Amalgamate:} Bring or combine together or
                      with something else.
                \item \textbf{Amble:} Walk or move at a slow, relaxed pace.
                \item \textbf{Amorous:} Inclined toward or displaying
                      love; expressive of or exciting sexual love or romance.
                \item \textbf{Antithetical:} Directly opposed or
                      contrasted; utually incompatible.
                \item \textbf{Aphorism:} A pithy observation that contains
                      a general truth.
                \item \textbf{Apoplectic:} Overcoe with anger;
                      extremely indignant.
                \item \textbf{Aquiver:} Quivering; trembling.
                \item \textbf{Arrogate:} Take or claim something
                      without justification.
                \item \textbf{Balmy:} (Of the weather) pleasantly warm.
                \item \textbf{Bathetic:} Producing an unintentional
                      effect of anticlimax.
                \item \textbf{Benison:} A blessing.
                \item \textbf{Bilk:} Obtain or withhold money from someone
                      by deceit or without justification; cheat or defraud.
                \item \textbf{Castigation:} Verbal punishment.
                \item \textbf{Cerebrate:} Think deeply about something; ponder.
                \item \textbf{Coeval:} Having the same age or date of
                      origin; contemporary.
                \item \textbf{Cogent:} (Of an argument or case) clear,
                      logical, and convincing.
                \item \textbf{Complicity:} The state of being involved
                      with others in an illegal activity or wrongdoing.
                \item \textbf{Comport:} Conduct oneself; behave.
                \item \textbf{Conniption:} A fit of extreme anger
                      or excitement.
                \item \textbf{Consternation:} Feelings of anxiety or
                      dismay, typically at something unexpected.
                \item \textbf{Copse:} A small group of trees.
                \item \textbf{Corpulent:} Having a large bulky body.
                \item \textbf{Couth:} Cultured, refined, and well mannered.
                \item \textbf{Crepuscular:} Of, resembling, or relating
                      to twilight.
                \item \textbf{Crestfallen:} Sad and disappointed.
                \item \textbf{Denigrate:} Criticize unfairly; disparage.
                \item \textbf{Distrait:} Distracted or absentminded.
                \item \textbf{Enervate:} Cause someone to feel drained of
                      energy or vitality; weaken.
                \item \textbf{Enrapture:} Give intense pleasure or joy to.
                \item \textbf{Eristic:} Of or characterized by debate
                      or argument.
                \item \textbf{Expiation:} The act of making amends
                      or reparation for guilt or wrongdoing; atonement.
                \item \textbf{Facetious:} Treating serious issues with
                      deliberately inappropriate humor; flippant.
                \item \textbf{Faineant:} Lazy, idle, do-nothing,
                      ineffectual, inactive. A person who has these features.
                \item \textbf{Fervor:} Intense and passionate feeling.
                \item \textbf{Fogment:} Instigate or stir up an undesirable
                      or violent sentiment or course of action.
                \item \textbf{Frivolity:} Lack of seriousness;
                      lightheartedness.
                \item \textbf{Fugacious:} Tending to disappear
                \item \textbf{Gallivant:} Go around from one place to
                      another in the pursuit of pleasure or entertainment.
                \item \textbf{Gehenna:} A place or state of misery; hell.
                \item \textbf{Gelid:} Icy; Extremely cold.
                \item \textbf{Genteel:} Polite, refined, or respectable, often
                      in an affected or ostentatious way.
                \item \textbf{Haptic:} Relating to the sense of touch;
                      tactile.
                \item \textbf{Hibernal:} Of, characteristic of, or
                      occuring in winter.
                \item \textbf{Surfeit:} An excessive amount
                \item \textbf{Haggard:} Looking exhausting and unwell
                \item \textbf{Perfunctory:} Carried out with minimum effort
                \item \textbf{Specious:} Superficially plausible, but
                      actually wrong
                \item \textbf{Paltry:} Small or meager.
                \item \textbf{Impertinent:} Not showing proper respect; rude.
                \item \textbf{Prescient:} Having or showing knowledge of
                      events before they take place.
                \item \textbf{Puritanical:} Practicing or affecting strict
                      religious or moral behavior.
                \item \textbf{Sentient:} Able to perceive or feel things.
                \item \textbf{Sordid:} Involving ignoble actions and motives;
                      arousing moral distaste and contempt.
                \item \textbf{Itinerant:} Traveling from place to place.
                \item \textbf{Quiescent:} In a state or period of inactivity
                      or dormancy.
                \item \textbf{Lethargic:} Sluggish or apathetic.
                \item \textbf{Apathetic:} Showing or feeling no interest,
                      enthusiasm, or concern.
            \end{itemize}
    \section{Genesis}
    Chapter 1
    \begin{enumerate}
        \item Genesis 1:1 In the beginning God created the heaven and the earth.
        \item Genesis 1:3 And God said, Let there be light: and there was light.
        \item God creates light and darkness, calls them day and light, all on the first day.
        \item On the second day he creates a `Firmament' called Heaven.
        \item On the third day, Land, seas, grass, herb yielding seed, and the fruit tree.
        \item On the fourth day, `Lights' in heaven for `signs' and seasons.
              Two great lights, greater (sun) and lesser (moon), and stars.
        \item On the fifth day, god creates animals (Fowls, great whales, etc.), blesses them
              and says be fruitful and multiply (a.k.a., To fuck).
        \item Genesis 1:24 And God said, Let the earth bring forth the living creature after
              his kind, cattle, and creeping thing, and beast of the earth
              after his kind: and it was so.
        \item Genesis 1:25 And God made the beast of the earth after his kind,
              and cattle after their kind, and everything that creepeth on the
              earth after his kid: and God saw that it was good.
        \item Genesis 1:27 So God created man in his own image, in the image of
              God created he him; male and female created he him.
        \item Genesis 1:28 So God blessed them, and God said unto them, Be fruitful,
              and multiply, and replenish the earth, and subdue it: and have dominion
              over the fish of the sea, and over the fowl of the air, and over every
              living thing that moveth upon the earth.
    \end{enumerate}
    Chapter 2
    \begin{enumerate}
        \item God finishes on the seventh day and rests,
              and sanctifies the day.
    \end{enumerate}
\end{document}