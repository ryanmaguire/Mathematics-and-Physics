\documentclass[crop=false,class=article,oneside]{standalone}
%----------------------------Preamble-------------------------------%
%---------------------------Packages----------------------------%
\usepackage{geometry}
\geometry{b5paper, margin=1.0in}
\usepackage[T1]{fontenc}
\usepackage{graphicx, float}            % Graphics/Images.
\usepackage{natbib}                     % For bibliographies.
\bibliographystyle{agsm}                % Bibliography style.
\usepackage[french, english]{babel}     % Language typesetting.
\usepackage[dvipsnames]{xcolor}         % Color names.
\usepackage{listings}                   % Verbatim-Like Tools.
\usepackage{mathtools, esint, mathrsfs} % amsmath and integrals.
\usepackage{amsthm, amsfonts, amssymb}  % Fonts and theorems.
\usepackage{tcolorbox}                  % Frames around theorems.
\usepackage{upgreek}                    % Non-Italic Greek.
\usepackage{fmtcount, etoolbox}         % For the \book{} command.
\usepackage[newparttoc]{titlesec}       % Formatting chapter, etc.
\usepackage{titletoc}                   % Allows \book in toc.
\usepackage[nottoc]{tocbibind}          % Bibliography in toc.
\usepackage[titles]{tocloft}            % ToC formatting.
\usepackage{pgfplots, tikz}             % Drawing/graphing tools.
\usepackage{imakeidx}                   % Used for index.
\usetikzlibrary{
    calc,                   % Calculating right angles and more.
    angles,                 % Drawing angles within triangles.
    arrows.meta,            % Latex and Stealth arrows.
    quotes,                 % Adding labels to angles.
    positioning,            % Relative positioning of nodes.
    decorations.markings,   % Adding arrows in the middle of a line.
    patterns,
    arrows
}                                       % Libraries for tikz.
\pgfplotsset{compat=1.9}                % Version of pgfplots.
\usepackage[font=scriptsize,
            labelformat=simple,
            labelsep=colon]{subcaption} % Subfigure captions.
\usepackage[font={scriptsize},
            hypcap=true,
            labelsep=colon]{caption}    % Figure captions.
\usepackage[pdftex,
            pdfauthor={Ryan Maguire},
            pdftitle={Mathematics and Physics},
            pdfsubject={Mathematics, Physics, Science},
            pdfkeywords={Mathematics, Physics, Computer Science, Biology},
            pdfproducer={LaTeX},
            pdfcreator={pdflatex}]{hyperref}
\hypersetup{
    colorlinks=true,
    linkcolor=blue,
    filecolor=magenta,
    urlcolor=Cerulean,
    citecolor=SkyBlue
}                           % Colors for hyperref.
\usepackage[toc,acronym,nogroupskip,nopostdot]{glossaries}
\usepackage{glossary-mcols}
%------------------------Theorem Styles-------------------------%
\theoremstyle{plain}
\newtheorem{theorem}{Theorem}[section]

% Define theorem style for default spacing and normal font.
\newtheoremstyle{normal}
    {\topsep}               % Amount of space above the theorem.
    {\topsep}               % Amount of space below the theorem.
    {}                      % Font used for body of theorem.
    {}                      % Measure of space to indent.
    {\bfseries}             % Font of the header of the theorem.
    {}                      % Punctuation between head and body.
    {.5em}                  % Space after theorem head.
    {}

% Italic header environment.
\newtheoremstyle{thmit}{\topsep}{\topsep}{}{}{\itshape}{}{0.5em}{}

% Define environments with italic headers.
\theoremstyle{thmit}
\newtheorem*{solution}{Solution}

% Define default environments.
\theoremstyle{normal}
\newtheorem{example}{Example}[section]
\newtheorem{definition}{Definition}[section]
\newtheorem{problem}{Problem}[section]

% Define framed environment.
\tcbuselibrary{most}
\newtcbtheorem[use counter*=theorem]{ftheorem}{Theorem}{%
    before=\par\vspace{2ex},
    boxsep=0.5\topsep,
    after=\par\vspace{2ex},
    colback=green!5,
    colframe=green!35!black,
    fonttitle=\bfseries\upshape%
}{thm}

\newtcbtheorem[auto counter, number within=section]{faxiom}{Axiom}{%
    before=\par\vspace{2ex},
    boxsep=0.5\topsep,
    after=\par\vspace{2ex},
    colback=Apricot!5,
    colframe=Apricot!35!black,
    fonttitle=\bfseries\upshape%
}{ax}

\newtcbtheorem[use counter*=definition]{fdefinition}{Definition}{%
    before=\par\vspace{2ex},
    boxsep=0.5\topsep,
    after=\par\vspace{2ex},
    colback=blue!5!white,
    colframe=blue!75!black,
    fonttitle=\bfseries\upshape%
}{def}

\newtcbtheorem[use counter*=example]{fexample}{Example}{%
    before=\par\vspace{2ex},
    boxsep=0.5\topsep,
    after=\par\vspace{2ex},
    colback=red!5!white,
    colframe=red!75!black,
    fonttitle=\bfseries\upshape%
}{ex}

\newtcbtheorem[auto counter, number within=section]{fnotation}{Notation}{%
    before=\par\vspace{2ex},
    boxsep=0.5\topsep,
    after=\par\vspace{2ex},
    colback=SeaGreen!5!white,
    colframe=SeaGreen!75!black,
    fonttitle=\bfseries\upshape%
}{not}

\newtcbtheorem[use counter*=remark]{fremark}{Remark}{%
    fonttitle=\bfseries\upshape,
    colback=Goldenrod!5!white,
    colframe=Goldenrod!75!black}{ex}

\newenvironment{bproof}{\textit{Proof.}}{\hfill$\square$}
\tcolorboxenvironment{bproof}{%
    blanker,
    breakable,
    left=3mm,
    before skip=5pt,
    after skip=10pt,
    borderline west={0.6mm}{0pt}{green!80!black}
}

\AtEndEnvironment{lexample}{$\hfill\textcolor{red}{\blacksquare}$}
\newtcbtheorem[use counter*=example]{lexample}{Example}{%
    empty,
    title={Example~\theexample},
    boxed title style={%
        empty,
        size=minimal,
        toprule=2pt,
        top=0.5\topsep,
    },
    coltitle=red,
    fonttitle=\bfseries,
    parbox=false,
    boxsep=0pt,
    before=\par\vspace{2ex},
    left=0pt,
    right=0pt,
    top=3ex,
    bottom=1ex,
    before=\par\vspace{2ex},
    after=\par\vspace{2ex},
    breakable,
    pad at break*=0mm,
    vfill before first,
    overlay unbroken={%
        \draw[red, line width=2pt]
            ([yshift=-1.2ex]title.south-|frame.west) to
            ([yshift=-1.2ex]title.south-|frame.east);
        },
    overlay first={%
        \draw[red, line width=2pt]
            ([yshift=-1.2ex]title.south-|frame.west) to
            ([yshift=-1.2ex]title.south-|frame.east);
    },
}{ex}

\AtEndEnvironment{ldefinition}{$\hfill\textcolor{Blue}{\blacksquare}$}
\newtcbtheorem[use counter*=definition]{ldefinition}{Definition}{%
    empty,
    title={Definition~\thedefinition:~{#1}},
    boxed title style={%
        empty,
        size=minimal,
        toprule=2pt,
        top=0.5\topsep,
    },
    coltitle=Blue,
    fonttitle=\bfseries,
    parbox=false,
    boxsep=0pt,
    before=\par\vspace{2ex},
    left=0pt,
    right=0pt,
    top=3ex,
    bottom=0pt,
    before=\par\vspace{2ex},
    after=\par\vspace{1ex},
    breakable,
    pad at break*=0mm,
    vfill before first,
    overlay unbroken={%
        \draw[Blue, line width=2pt]
            ([yshift=-1.2ex]title.south-|frame.west) to
            ([yshift=-1.2ex]title.south-|frame.east);
        },
    overlay first={%
        \draw[Blue, line width=2pt]
            ([yshift=-1.2ex]title.south-|frame.west) to
            ([yshift=-1.2ex]title.south-|frame.east);
    },
}{def}

\AtEndEnvironment{ltheorem}{$\hfill\textcolor{Green}{\blacksquare}$}
\newtcbtheorem[use counter*=theorem]{ltheorem}{Theorem}{%
    empty,
    title={Theorem~\thetheorem:~{#1}},
    boxed title style={%
        empty,
        size=minimal,
        toprule=2pt,
        top=0.5\topsep,
    },
    coltitle=Green,
    fonttitle=\bfseries,
    parbox=false,
    boxsep=0pt,
    before=\par\vspace{2ex},
    left=0pt,
    right=0pt,
    top=3ex,
    bottom=-1.5ex,
    breakable,
    pad at break*=0mm,
    vfill before first,
    overlay unbroken={%
        \draw[Green, line width=2pt]
            ([yshift=-1.2ex]title.south-|frame.west) to
            ([yshift=-1.2ex]title.south-|frame.east);},
    overlay first={%
        \draw[Green, line width=2pt]
            ([yshift=-1.2ex]title.south-|frame.west) to
            ([yshift=-1.2ex]title.south-|frame.east);
    }
}{thm}

%--------------------Declared Math Operators--------------------%
\DeclareMathOperator{\adjoint}{adj}         % Adjoint.
\DeclareMathOperator{\Card}{Card}           % Cardinality.
\DeclareMathOperator{\curl}{curl}           % Curl.
\DeclareMathOperator{\diam}{diam}           % Diameter.
\DeclareMathOperator{\dist}{dist}           % Distance.
\DeclareMathOperator{\Div}{div}             % Divergence.
\DeclareMathOperator{\Erf}{Erf}             % Error Function.
\DeclareMathOperator{\Erfc}{Erfc}           % Complementary Error Function.
\DeclareMathOperator{\Ext}{Ext}             % Exterior.
\DeclareMathOperator{\GCD}{GCD}             % Greatest common denominator.
\DeclareMathOperator{\grad}{grad}           % Gradient
\DeclareMathOperator{\Ima}{Im}              % Image.
\DeclareMathOperator{\Int}{Int}             % Interior.
\DeclareMathOperator{\LC}{LC}               % Leading coefficient.
\DeclareMathOperator{\LCM}{LCM}             % Least common multiple.
\DeclareMathOperator{\LM}{LM}               % Leading monomial.
\DeclareMathOperator{\LT}{LT}               % Leading term.
\DeclareMathOperator{\Mod}{mod}             % Modulus.
\DeclareMathOperator{\Mon}{Mon}             % Monomial.
\DeclareMathOperator{\multideg}{mutlideg}   % Multi-Degree (Graphs).
\DeclareMathOperator{\nul}{nul}             % Null space of operator.
\DeclareMathOperator{\Ord}{Ord}             % Ordinal of ordered set.
\DeclareMathOperator{\Prin}{Prin}           % Principal value.
\DeclareMathOperator{\proj}{proj}           % Projection.
\DeclareMathOperator{\Refl}{Refl}           % Reflection operator.
\DeclareMathOperator{\rk}{rk}               % Rank of operator.
\DeclareMathOperator{\sgn}{sgn}             % Sign of a number.
\DeclareMathOperator{\sinc}{sinc}           % Sinc function.
\DeclareMathOperator{\Span}{Span}           % Span of a set.
\DeclareMathOperator{\Spec}{Spec}           % Spectrum.
\DeclareMathOperator{\supp}{supp}           % Support
\DeclareMathOperator{\Tr}{Tr}               % Trace of matrix.
%--------------------Declared Math Symbols--------------------%
\DeclareMathSymbol{\minus}{\mathbin}{AMSa}{"39} % Unary minus sign.
%------------------------New Commands---------------------------%
\DeclarePairedDelimiter\norm{\lVert}{\rVert}
\DeclarePairedDelimiter\ceil{\lceil}{\rceil}
\DeclarePairedDelimiter\floor{\lfloor}{\rfloor}
\newcommand*\diff{\mathop{}\!\mathrm{d}}
\newcommand*\Diff[1]{\mathop{}\!\mathrm{d^#1}}
\renewcommand*{\glstextformat}[1]{\textcolor{RoyalBlue}{#1}}
\renewcommand{\glsnamefont}[1]{\textbf{#1}}
\renewcommand\labelitemii{$\circ$}
\renewcommand\thesubfigure{%
    \arabic{chapter}.\arabic{figure}.\arabic{subfigure}}
\addto\captionsenglish{\renewcommand{\figurename}{Fig.}}
\numberwithin{equation}{section}

\renewcommand{\vector}[1]{\boldsymbol{\mathrm{#1}}}

\newcommand{\uvector}[1]{\boldsymbol{\hat{\mathrm{#1}}}}
\newcommand{\topspace}[2][]{(#2,\tau_{#1})}
\newcommand{\measurespace}[2][]{(#2,\varSigma_{#1},\mu_{#1})}
\newcommand{\measurablespace}[2][]{(#2,\varSigma_{#1})}
\newcommand{\manifold}[2][]{(#2,\tau_{#1},\mathcal{A}_{#1})}
\newcommand{\tanspace}[2]{T_{#1}{#2}}
\newcommand{\cotanspace}[2]{T_{#1}^{*}{#2}}
\newcommand{\Ckspace}[3][\mathbb{R}]{C^{#2}(#3,#1)}
\newcommand{\funcspace}[2][\mathbb{R}]{\mathcal{F}(#2,#1)}
\newcommand{\smoothvecf}[1]{\mathfrak{X}(#1)}
\newcommand{\smoothonef}[1]{\mathfrak{X}^{*}(#1)}
\newcommand{\bracket}[2]{[#1,#2]}

%------------------------Book Command---------------------------%
\makeatletter
\renewcommand\@pnumwidth{1cm}
\newcounter{book}
\renewcommand\thebook{\@Roman\c@book}
\newcommand\book{%
    \if@openright
        \cleardoublepage
    \else
        \clearpage
    \fi
    \thispagestyle{plain}%
    \if@twocolumn
        \onecolumn
        \@tempswatrue
    \else
        \@tempswafalse
    \fi
    \null\vfil
    \secdef\@book\@sbook
}
\def\@book[#1]#2{%
    \refstepcounter{book}
    \addcontentsline{toc}{book}{\bookname\ \thebook:\hspace{1em}#1}
    \markboth{}{}
    {\centering
     \interlinepenalty\@M
     \normalfont
     \huge\bfseries\bookname\nobreakspace\thebook
     \par
     \vskip 20\p@
     \Huge\bfseries#2\par}%
    \@endbook}
\def\@sbook#1{%
    {\centering
     \interlinepenalty \@M
     \normalfont
     \Huge\bfseries#1\par}%
    \@endbook}
\def\@endbook{
    \vfil\newpage
        \if@twoside
            \if@openright
                \null
                \thispagestyle{empty}%
                \newpage
            \fi
        \fi
        \if@tempswa
            \twocolumn
        \fi
}
\newcommand*\l@book[2]{%
    \ifnum\c@tocdepth >-3\relax
        \addpenalty{-\@highpenalty}%
        \addvspace{2.25em\@plus\p@}%
        \setlength\@tempdima{3em}%
        \begingroup
            \parindent\z@\rightskip\@pnumwidth
            \parfillskip -\@pnumwidth
            {
                \leavevmode
                \Large\bfseries#1\hfill\hb@xt@\@pnumwidth{\hss#2}
            }
            \par
            \nobreak
            \global\@nobreaktrue
            \everypar{\global\@nobreakfalse\everypar{}}%
        \endgroup
    \fi}
\newcommand\bookname{Book}
\renewcommand{\thebook}{\texorpdfstring{\Numberstring{book}}{book}}
\providecommand*{\toclevel@book}{-2}
\makeatother
\titleformat{\part}[display]
    {\Large\bfseries}
    {\partname\nobreakspace\thepart}
    {0mm}
    {\Huge\bfseries}
\titlecontents{part}[0pt]
    {\large\bfseries}
    {\partname\ \thecontentslabel: \quad}
    {}
    {\hfill\contentspage}
\titlecontents{chapter}[0pt]
    {\bfseries}
    {\chaptername\ \thecontentslabel:\quad}
    {}
    {\hfill\contentspage}
\newglossarystyle{longpara}{%
    \setglossarystyle{long}%
    \renewenvironment{theglossary}{%
        \begin{longtable}[l]{{p{0.25\hsize}p{0.65\hsize}}}
    }{\end{longtable}}%
    \renewcommand{\glossentry}[2]{%
        \glstarget{##1}{\glossentryname{##1}}%
        &\glossentrydesc{##1}{~##2.}
        \tabularnewline%
        \tabularnewline
    }%
}
\newglossary[not-glg]{notation}{not-gls}{not-glo}{Notation}
\newcommand*{\newnotation}[4][]{%
    \newglossaryentry{#2}{type=notation, name={\textbf{#3}, },
                          text={#4}, description={#4},#1}%
}
%--------------------------LENGTHS------------------------------%
% Spacings for the Table of Contents.
\addtolength{\cftsecnumwidth}{1ex}
\addtolength{\cftsubsecindent}{1ex}
\addtolength{\cftsubsecnumwidth}{1ex}
\addtolength{\cftfignumwidth}{1ex}
\addtolength{\cfttabnumwidth}{1ex}

% Indent and paragraph spacing.
\setlength{\parindent}{0em}
\setlength{\parskip}{0em}
\graphicspath{{../../../../images/}}    % Path to Image Folder.
%--------------------------Main Document----------------------------%
\begin{document}
    \ifx\ifphysicscourseselectromagnetismI\undefined
        \section*{Miscellaneous Materials}
        \setcounter{section}{1}
    \fi
    \subsection{To Do List}
        \begin{enumerate}
            \item Derive geometry, all of MTR83, MTR86, and User Guide. Recreate figures.
            \item Read Van de Huulst and Chandrasakhr, Chylek, Intro to Matched Filters,
                  Chebyshev doc, Elliot 84, Understanding FFTs and Windows,
                  Hanning and Hamming windows, Characteristics of different
                  smoothing windows, NAIF Tutorials, scattering papers, etc.
                  Learn Fresnel Diffraction and Fourier Optics. Understand noise
                  and computational limits. Learn Mie Theory (Jackson, Whittet).
            \item Learn about the $\alpha$ of a plot.
            \item Make interactive plot for rev007 E/I. Process all bands.
                  Document results.
            \item Think about user interactive parts.
            \begin{itemize}
                \item Geometry
                \begin{itemize}
                    \item Mindfully choose the RSR you want. Radial coverage, opening angle, etc.
                \end{itemize}
                \item Phase and power retreival
                \begin{itemize}
                    \item Picking start and end SPM
                    \item Choosing nots
                    \item Choosing order of residual frequency fit
                \end{itemize}
                \item Diffraction reconstruction
                \begin{itemize}
                    \item Choose resolution, window type, and range.
                \end{itemize}
                \item Look at Essam's new directory structure.
                \item Look at Essam's diffraction profiles. Try to recreate his corrected profiles using the diffraction reconstruction code.
            \end{itemize}
            \item Document why the FFT suceeded and why the
                  integral failed for certain revs where the cubic was
                  non negligible. Note that the quartic term was not
                  the problem, and that undersampling is not the
                  problem either. It is the fact that the FFT for
                  some reason swaps the sign of odd terms. Also not
                  that it is not a window size or a B problem, but
                  rather only dependent on when psi has large odd
                  polynomial terms. Document when cubic and quartic
                  terms in $\psi$ are needed.
            \item Document in detail the normalization factor that
                  is used for the window width. Include approximate
                  normalization factors that assume psi is equal to
                  the fresnel approximation and show why the factor
                  is need. That is, for large resolutions
                  (Small windows) you'd reconstructions scaled
                  to near zero.
            \item Document the ideal, or optimal, resolution to compare
                  with Essam based on comparisons with $L_{1}$,
                  $L_{2}$, and $L_{\infty}$ for various revs, including
                  old high resolution data sets from various idl .sav
                  files. Include documentation on the effect of using
                  different windows for these comparisons. Show that
                  the best match uses a window (KBMD20) that is not
                  mentioned in MTR86.
            \item Document problems with 10km reconstruction.
                  250m spacing input, 2.5km spacing. etc.
            \item Document swap of $\rho$ with $\rho_{0}$, how this fixes
                  the cubic, and the solution found to fix integration.
            \item Document the phase problem. Why is the phase
                  negative? T runs clockwise? A reflection in time due
                  to the problem being modelled as if the Earth sends
                  a signal to Cassini (Time goes backwards)?
            \item Look into FFTW.
            \item Figure out threshold optical depth.
            \item Learn how to use the NAIF toolkit and learn
                  how to read and plot Cassini trajectory files.
            \item Plot the geometry of the Voyager RSS encounter.
            \item Use Dick Simpson's Fortran code on Rev 282.
            \item Learn rgf\_read\_rsr\_hdr.pro. 
                  This routine reads and RSR header and returns
                  the first SFDU.
            \item Learn cubic and quadratic splines. Learn knots.
            \item Learn how to read RSR files and calc sky freq.
            \item Learn to read and write PDS3 files.
            \item Look into Rev123. Try to find F ring.
            \item Learn to retrieve phase. Look into high-pass filtering
                  to remove linear phase shifts and poly fitting.
            \item Learn how to retrieve frequencies from noisy data.
                  Take sum of a few sines, add noise, and try to remove
                  noise with FFT's.
            \item Document conversion factors:
                  $S=\frac{3}{11}X$, $Ka=\frac{19}{5}X$
            \item Learn how to use co-added FFT's to get power.
            \item Learn to use FFT's to compute phase.
            \item Document Loss of Signal.
            \item Try running JG\_rss\_occs\_v3.pro.
            \item Look for F-Ring in Rev133.
            \item Document why $\psi$ can not, in general, be
                  approximated by a sum of Chebyshev functions.
                  That is, note that the coefficients will indeed be
                  functions. Convolution doesn't apply.
            \item Document what v2 diffraction reconstruction did.
                  npoints parameter for FFT. etc.
            \item Document the Lagrange Interpolation method for
                  derivatives and why it's been removed. That is,
                  the failure of convergence in Newton-Raphson that
                  non-zero dphi causes.
            \item Learn how to use decimate (MATLAB, Python, IDL).
            \item Try 100m reconstruction of 123E. Look for F ring.
            \item Document in depth comparisons against Essam of Rev007.
                  Include geometry, power, normalization curves,
                  frequency offset, phase, and diffraction
                  reconstruction.
            \item Document window width as a function of radius.
                  Document min and max restraints on radius.
            \item Make note: Time tags are at the start of a second.
                  So left side binning. To do midpoint computation,
                  add $\Delta{t}/2$ to each value.
            \item Try using rgf\_spm2uniform\_rho\_km.pro. Document
                  $\overline{I}^{2}+\overline{Q}^{2}$ vs.
                  $\overline{I^{2}+Q^{2}}$.
            \item Look into the geometry of the 2017/07/23 periapse
                  ring occultation event.
            \item Document 1khz and 16khz reconstructions.
            \item Document the feasibility of high resolution
                  reconstruction in Rev125 S band.
            \item Learn how to calculate the Saturn pole direction.
                  Learn about .bsp files.
            \item Look for waves in the D ring to compare with VIMs data.
                  If these are dusty rings, the radio data shouldn't
                  see it. Range 73000 to inner C ring. Compare with
                  Colleen's results.
            \item Essam assumes uniform power across rings, but there is
                  an antenna beam patter. Learn about this. Try this
                  problem on a square well. The wave is a sinc function
                  in terms of the angle from the on-axis ray.
                  The first null is proportional to $\lambda/d$, where
                  $d$ is the antenna diameter. Ask prof French about
                  more accurate numbers and models.
            \item Find the noise limit for reconstructing Rev133.
            \item Document disk swapping in Rev133. In Unix: ps au.
                  Document variations in points processed per second.
                  Note is should decrease with window size, but
                  doesn't.
            \item Document the differences in v4 and v5.
            \item Look into better integration techniques than simple
                  Riemann sums.
            \item Learn about phase-locked loops. Read Paul Schinder's
                  codes.
            \item Add these regions (Try finding waves):
            \begin{itemize}
                \item Mi 4:1 m=2  74890 km  1.1$\pm$0.5 g/$\textrm{cm}^2$  0.16$\pm$0.09  0.18$\pm$0.13 g/$\textrm{cm}^2$
                \item W74.66 m=-7 74666 km  0.7$\pm$0.2 g/$\textrm{cm}^2$  0.09$\pm$0.03  0.15$\pm$0.09 g/$\textrm{cm}^2$
                \item W74.93 m=-4 74936 km  0.3$\pm$0.1 g/$\textrm{cm}^2$  0.05$\pm$0.02  0.21$\pm$0.17 g/$\textrm{cm}^2$
                \item W74.94 m=-9 74941km,  1.3$\pm$0.5 g/$\textrm{cm}^2$, 0.17$\pm$0.07  0.14$\pm$0.07 g$\textrm{cm}^2$
                \item W76.02 m=-9 76018km,  1.3$\pm$0.7 g/$\textrm{cm}^2$, 0.05$\pm$0.04  0.05$\pm$0.05 g$\textrm{cm}^2$
                \item W76.23 m=-8 76237.5km 1.6$\pm$1.0 g/$\textrm{cm}^2$, 0.17$\pm$0.06  0.15$\pm$0.09 g$\textrm{cm}^2$
                \item W76.44 m=-2 76435.5km 1.0$\pm$1.4 g/$\textrm{cm}^2$, 0.04$\pm$0.02  0.07$\pm$0.05 g$\textrm{cm}^2$
            \end{itemize}
            \begin{itemize}
                \begin{multicols}{2}
                    \item Kuiper Gap 119400 $\pm$30km
                    \item Jeffreys Gap 118950 $\pm$40km
                    \item Russell gap 118610 $\pm$40km
                    \item Bessel-Barnard 120270 $\pm$50km
                    \item Strange Ringlet 117910 $\pm$30km
                    \item Herschel Gap 118240 $\pm$140km
                \end{multicols}
            \end{itemize}
            \item Document how to use rdcol in IDL.
            \item Document the real and imaginary parts of $T$ and $\hat{T}$
                  comparing with Essam.
            \item Document V5 reconstructions of Titan and Maxwell Rev7.
            \item Try computing power as
                  $\sigma(\langle{P}\rangle)/\langle{P}\rangle$,
                  where $\langle\rangle$ represents the average value,
                  and $\langle{P}\rangle=\langle{I^{2}+Q^{2}}\rangle$.
                  Document differences between this and MTR86.
            \item Document source of computation time. The need
                  to perform Newton-Raphson, the slowness of
                  Lagrange Interpolation, psi calculation, FFT vs integral.
            
        \end{enumerate}
To be objective about when a signal was lost, we adopted the criterion that a signal is lost if its level permanently falls below 3-sigma (99\% confidence level).
Loss of Signal data sets:
The data are on the SOPC in:
/data/RS\_Share/s101-rev293-SA-data/OCC/RSR. The data are on the rodan cluster in: /rssg/data/CAS/2017\_258/OCC/RSR. The filenames (on casrss2 and rodan cluster are):
\begin{itemize}
\begin{multicols}{2}
    \item 43X: SC1SAOI2017258\_0302XMMX43RD.2A1
    \item 43S: SC1SAOI2017258\_0302XMMS43RD.2B1
    \item 35X: SC1SAOI2017258\_0330XMMX35RV.1N1
    \item 35K: SC1SAOI2017258\_0330XMMK35RV.1N1
    \item 74X: SC1SAOI2017258\_0459XMMX74RP.3A1
\end{multicols}
\end{itemize}
\begin{figure}
    \centering
    \captionsetup{type=table}
    \begin{subfigure}[b]{0.49\textwidth}
        \centering
        \caption{Essam LOS Times}
        \begin{tabular}{c c c} 
            \hline
             & SPM      & ERT (UTC)    \\ 
            \hline
            S43: & 42942.85 & 11:55:42.85  \\
            X43: & 42934.70 & 11:55:34.70  \\
            K35: & 42931.81 & 11:55:31.81  \\
            X35: & 42931.12 & 11:55:31.12  \\
            X74: & 42930.06 & 11:55:30.06  \\
            \hline
        \end{tabular}
    \end{subfigure}
    \begin{subfigure}[b]{0.49\textwidth}
        \centering
        \captionsetup{type=table}
        \caption{Loss of Signal Times}
        \begin{tabular}{c c c} 
            \hline
            & SPM      & Difference (Seconds)    \\ 
            \hline
            S43: & 42942.355 & -0.496    \\
            X43: & 42934.517 & -0.184    \\
            K35: & 42932.106 & +0.297    \\
            X35: & 42930.832 & -0.289    \\
            X74: & 42929.897 & -0.160    \\
            \hline
        \end{tabular}
    \end{subfigure}
\end{figure}
Starting at about 11:40 we see systematic increase in the signal frequency with the residual from its value away from LOS reaching a peak value of about 24 Hz at Ka-band close to LOS (see attached figure).  It's also clearly detectable at X-band with peak residual of about 6 Hz. The peak frequency residual is hardly detectable at S-band. The frequency change is definitely not due to the ionosphere. The frequency scaling appears to be proportional to frequency (inversely proportional to wavelength) hence is likely due to neutral media or deviation of spacecraft velocity from trajectory prediction.  The frequency increase appears to eliminates the latter since atmospheric drag is expected to slightly slow down Cassini, hence decreasing not increasing the received Doppler shifted frequency. If true, it would appear that Cassini signals began to sense the upper neutral atmosphere, but I'm not sure since Cassini altitude was too high.
\end{document}