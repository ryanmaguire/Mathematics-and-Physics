\documentclass[crop=false,class=article,oneside]{standalone}
%----------------------------Preamble-------------------------------%
%---------------------------Packages----------------------------%
\usepackage{geometry}
\geometry{b5paper, margin=1.0in}
\usepackage[T1]{fontenc}
\usepackage{graphicx, float}            % Graphics/Images.
\usepackage{natbib}                     % For bibliographies.
\bibliographystyle{agsm}                % Bibliography style.
\usepackage[french, english]{babel}     % Language typesetting.
\usepackage[dvipsnames]{xcolor}         % Color names.
\usepackage{listings}                   % Verbatim-Like Tools.
\usepackage{mathtools, esint, mathrsfs} % amsmath and integrals.
\usepackage{amsthm, amsfonts, amssymb}  % Fonts and theorems.
\usepackage{tcolorbox}                  % Frames around theorems.
\usepackage{upgreek}                    % Non-Italic Greek.
\usepackage{fmtcount, etoolbox}         % For the \book{} command.
\usepackage[newparttoc]{titlesec}       % Formatting chapter, etc.
\usepackage{titletoc}                   % Allows \book in toc.
\usepackage[nottoc]{tocbibind}          % Bibliography in toc.
\usepackage[titles]{tocloft}            % ToC formatting.
\usepackage{pgfplots, tikz}             % Drawing/graphing tools.
\usepackage{imakeidx}                   % Used for index.
\usetikzlibrary{
    calc,                   % Calculating right angles and more.
    angles,                 % Drawing angles within triangles.
    arrows.meta,            % Latex and Stealth arrows.
    quotes,                 % Adding labels to angles.
    positioning,            % Relative positioning of nodes.
    decorations.markings,   % Adding arrows in the middle of a line.
    patterns,
    arrows
}                                       % Libraries for tikz.
\pgfplotsset{compat=1.9}                % Version of pgfplots.
\usepackage[font=scriptsize,
            labelformat=simple,
            labelsep=colon]{subcaption} % Subfigure captions.
\usepackage[font={scriptsize},
            hypcap=true,
            labelsep=colon]{caption}    % Figure captions.
\usepackage[pdftex,
            pdfauthor={Ryan Maguire},
            pdftitle={Mathematics and Physics},
            pdfsubject={Mathematics, Physics, Science},
            pdfkeywords={Mathematics, Physics, Computer Science, Biology},
            pdfproducer={LaTeX},
            pdfcreator={pdflatex}]{hyperref}
\hypersetup{
    colorlinks=true,
    linkcolor=blue,
    filecolor=magenta,
    urlcolor=Cerulean,
    citecolor=SkyBlue
}                           % Colors for hyperref.
\usepackage[toc,acronym,nogroupskip,nopostdot]{glossaries}
\usepackage{glossary-mcols}
%------------------------Theorem Styles-------------------------%
\theoremstyle{plain}
\newtheorem{theorem}{Theorem}[section]

% Define theorem style for default spacing and normal font.
\newtheoremstyle{normal}
    {\topsep}               % Amount of space above the theorem.
    {\topsep}               % Amount of space below the theorem.
    {}                      % Font used for body of theorem.
    {}                      % Measure of space to indent.
    {\bfseries}             % Font of the header of the theorem.
    {}                      % Punctuation between head and body.
    {.5em}                  % Space after theorem head.
    {}

% Italic header environment.
\newtheoremstyle{thmit}{\topsep}{\topsep}{}{}{\itshape}{}{0.5em}{}

% Define environments with italic headers.
\theoremstyle{thmit}
\newtheorem*{solution}{Solution}

% Define default environments.
\theoremstyle{normal}
\newtheorem{example}{Example}[section]
\newtheorem{definition}{Definition}[section]
\newtheorem{problem}{Problem}[section]

% Define framed environment.
\tcbuselibrary{most}
\newtcbtheorem[use counter*=theorem]{ftheorem}{Theorem}{%
    before=\par\vspace{2ex},
    boxsep=0.5\topsep,
    after=\par\vspace{2ex},
    colback=green!5,
    colframe=green!35!black,
    fonttitle=\bfseries\upshape%
}{thm}

\newtcbtheorem[auto counter, number within=section]{faxiom}{Axiom}{%
    before=\par\vspace{2ex},
    boxsep=0.5\topsep,
    after=\par\vspace{2ex},
    colback=Apricot!5,
    colframe=Apricot!35!black,
    fonttitle=\bfseries\upshape%
}{ax}

\newtcbtheorem[use counter*=definition]{fdefinition}{Definition}{%
    before=\par\vspace{2ex},
    boxsep=0.5\topsep,
    after=\par\vspace{2ex},
    colback=blue!5!white,
    colframe=blue!75!black,
    fonttitle=\bfseries\upshape%
}{def}

\newtcbtheorem[use counter*=example]{fexample}{Example}{%
    before=\par\vspace{2ex},
    boxsep=0.5\topsep,
    after=\par\vspace{2ex},
    colback=red!5!white,
    colframe=red!75!black,
    fonttitle=\bfseries\upshape%
}{ex}

\newtcbtheorem[auto counter, number within=section]{fnotation}{Notation}{%
    before=\par\vspace{2ex},
    boxsep=0.5\topsep,
    after=\par\vspace{2ex},
    colback=SeaGreen!5!white,
    colframe=SeaGreen!75!black,
    fonttitle=\bfseries\upshape%
}{not}

\newtcbtheorem[use counter*=remark]{fremark}{Remark}{%
    fonttitle=\bfseries\upshape,
    colback=Goldenrod!5!white,
    colframe=Goldenrod!75!black}{ex}

\newenvironment{bproof}{\textit{Proof.}}{\hfill$\square$}
\tcolorboxenvironment{bproof}{%
    blanker,
    breakable,
    left=3mm,
    before skip=5pt,
    after skip=10pt,
    borderline west={0.6mm}{0pt}{green!80!black}
}

\AtEndEnvironment{lexample}{$\hfill\textcolor{red}{\blacksquare}$}
\newtcbtheorem[use counter*=example]{lexample}{Example}{%
    empty,
    title={Example~\theexample},
    boxed title style={%
        empty,
        size=minimal,
        toprule=2pt,
        top=0.5\topsep,
    },
    coltitle=red,
    fonttitle=\bfseries,
    parbox=false,
    boxsep=0pt,
    before=\par\vspace{2ex},
    left=0pt,
    right=0pt,
    top=3ex,
    bottom=1ex,
    before=\par\vspace{2ex},
    after=\par\vspace{2ex},
    breakable,
    pad at break*=0mm,
    vfill before first,
    overlay unbroken={%
        \draw[red, line width=2pt]
            ([yshift=-1.2ex]title.south-|frame.west) to
            ([yshift=-1.2ex]title.south-|frame.east);
        },
    overlay first={%
        \draw[red, line width=2pt]
            ([yshift=-1.2ex]title.south-|frame.west) to
            ([yshift=-1.2ex]title.south-|frame.east);
    },
}{ex}

\AtEndEnvironment{ldefinition}{$\hfill\textcolor{Blue}{\blacksquare}$}
\newtcbtheorem[use counter*=definition]{ldefinition}{Definition}{%
    empty,
    title={Definition~\thedefinition:~{#1}},
    boxed title style={%
        empty,
        size=minimal,
        toprule=2pt,
        top=0.5\topsep,
    },
    coltitle=Blue,
    fonttitle=\bfseries,
    parbox=false,
    boxsep=0pt,
    before=\par\vspace{2ex},
    left=0pt,
    right=0pt,
    top=3ex,
    bottom=0pt,
    before=\par\vspace{2ex},
    after=\par\vspace{1ex},
    breakable,
    pad at break*=0mm,
    vfill before first,
    overlay unbroken={%
        \draw[Blue, line width=2pt]
            ([yshift=-1.2ex]title.south-|frame.west) to
            ([yshift=-1.2ex]title.south-|frame.east);
        },
    overlay first={%
        \draw[Blue, line width=2pt]
            ([yshift=-1.2ex]title.south-|frame.west) to
            ([yshift=-1.2ex]title.south-|frame.east);
    },
}{def}

\AtEndEnvironment{ltheorem}{$\hfill\textcolor{Green}{\blacksquare}$}
\newtcbtheorem[use counter*=theorem]{ltheorem}{Theorem}{%
    empty,
    title={Theorem~\thetheorem:~{#1}},
    boxed title style={%
        empty,
        size=minimal,
        toprule=2pt,
        top=0.5\topsep,
    },
    coltitle=Green,
    fonttitle=\bfseries,
    parbox=false,
    boxsep=0pt,
    before=\par\vspace{2ex},
    left=0pt,
    right=0pt,
    top=3ex,
    bottom=-1.5ex,
    breakable,
    pad at break*=0mm,
    vfill before first,
    overlay unbroken={%
        \draw[Green, line width=2pt]
            ([yshift=-1.2ex]title.south-|frame.west) to
            ([yshift=-1.2ex]title.south-|frame.east);},
    overlay first={%
        \draw[Green, line width=2pt]
            ([yshift=-1.2ex]title.south-|frame.west) to
            ([yshift=-1.2ex]title.south-|frame.east);
    }
}{thm}

%--------------------Declared Math Operators--------------------%
\DeclareMathOperator{\adjoint}{adj}         % Adjoint.
\DeclareMathOperator{\Card}{Card}           % Cardinality.
\DeclareMathOperator{\curl}{curl}           % Curl.
\DeclareMathOperator{\diam}{diam}           % Diameter.
\DeclareMathOperator{\dist}{dist}           % Distance.
\DeclareMathOperator{\Div}{div}             % Divergence.
\DeclareMathOperator{\Erf}{Erf}             % Error Function.
\DeclareMathOperator{\Erfc}{Erfc}           % Complementary Error Function.
\DeclareMathOperator{\Ext}{Ext}             % Exterior.
\DeclareMathOperator{\GCD}{GCD}             % Greatest common denominator.
\DeclareMathOperator{\grad}{grad}           % Gradient
\DeclareMathOperator{\Ima}{Im}              % Image.
\DeclareMathOperator{\Int}{Int}             % Interior.
\DeclareMathOperator{\LC}{LC}               % Leading coefficient.
\DeclareMathOperator{\LCM}{LCM}             % Least common multiple.
\DeclareMathOperator{\LM}{LM}               % Leading monomial.
\DeclareMathOperator{\LT}{LT}               % Leading term.
\DeclareMathOperator{\Mod}{mod}             % Modulus.
\DeclareMathOperator{\Mon}{Mon}             % Monomial.
\DeclareMathOperator{\multideg}{mutlideg}   % Multi-Degree (Graphs).
\DeclareMathOperator{\nul}{nul}             % Null space of operator.
\DeclareMathOperator{\Ord}{Ord}             % Ordinal of ordered set.
\DeclareMathOperator{\Prin}{Prin}           % Principal value.
\DeclareMathOperator{\proj}{proj}           % Projection.
\DeclareMathOperator{\Refl}{Refl}           % Reflection operator.
\DeclareMathOperator{\rk}{rk}               % Rank of operator.
\DeclareMathOperator{\sgn}{sgn}             % Sign of a number.
\DeclareMathOperator{\sinc}{sinc}           % Sinc function.
\DeclareMathOperator{\Span}{Span}           % Span of a set.
\DeclareMathOperator{\Spec}{Spec}           % Spectrum.
\DeclareMathOperator{\supp}{supp}           % Support
\DeclareMathOperator{\Tr}{Tr}               % Trace of matrix.
%--------------------Declared Math Symbols--------------------%
\DeclareMathSymbol{\minus}{\mathbin}{AMSa}{"39} % Unary minus sign.
%------------------------New Commands---------------------------%
\DeclarePairedDelimiter\norm{\lVert}{\rVert}
\DeclarePairedDelimiter\ceil{\lceil}{\rceil}
\DeclarePairedDelimiter\floor{\lfloor}{\rfloor}
\newcommand*\diff{\mathop{}\!\mathrm{d}}
\newcommand*\Diff[1]{\mathop{}\!\mathrm{d^#1}}
\renewcommand*{\glstextformat}[1]{\textcolor{RoyalBlue}{#1}}
\renewcommand{\glsnamefont}[1]{\textbf{#1}}
\renewcommand\labelitemii{$\circ$}
\renewcommand\thesubfigure{%
    \arabic{chapter}.\arabic{figure}.\arabic{subfigure}}
\addto\captionsenglish{\renewcommand{\figurename}{Fig.}}
\numberwithin{equation}{section}

\renewcommand{\vector}[1]{\boldsymbol{\mathrm{#1}}}

\newcommand{\uvector}[1]{\boldsymbol{\hat{\mathrm{#1}}}}
\newcommand{\topspace}[2][]{(#2,\tau_{#1})}
\newcommand{\measurespace}[2][]{(#2,\varSigma_{#1},\mu_{#1})}
\newcommand{\measurablespace}[2][]{(#2,\varSigma_{#1})}
\newcommand{\manifold}[2][]{(#2,\tau_{#1},\mathcal{A}_{#1})}
\newcommand{\tanspace}[2]{T_{#1}{#2}}
\newcommand{\cotanspace}[2]{T_{#1}^{*}{#2}}
\newcommand{\Ckspace}[3][\mathbb{R}]{C^{#2}(#3,#1)}
\newcommand{\funcspace}[2][\mathbb{R}]{\mathcal{F}(#2,#1)}
\newcommand{\smoothvecf}[1]{\mathfrak{X}(#1)}
\newcommand{\smoothonef}[1]{\mathfrak{X}^{*}(#1)}
\newcommand{\bracket}[2]{[#1,#2]}

%------------------------Book Command---------------------------%
\makeatletter
\renewcommand\@pnumwidth{1cm}
\newcounter{book}
\renewcommand\thebook{\@Roman\c@book}
\newcommand\book{%
    \if@openright
        \cleardoublepage
    \else
        \clearpage
    \fi
    \thispagestyle{plain}%
    \if@twocolumn
        \onecolumn
        \@tempswatrue
    \else
        \@tempswafalse
    \fi
    \null\vfil
    \secdef\@book\@sbook
}
\def\@book[#1]#2{%
    \refstepcounter{book}
    \addcontentsline{toc}{book}{\bookname\ \thebook:\hspace{1em}#1}
    \markboth{}{}
    {\centering
     \interlinepenalty\@M
     \normalfont
     \huge\bfseries\bookname\nobreakspace\thebook
     \par
     \vskip 20\p@
     \Huge\bfseries#2\par}%
    \@endbook}
\def\@sbook#1{%
    {\centering
     \interlinepenalty \@M
     \normalfont
     \Huge\bfseries#1\par}%
    \@endbook}
\def\@endbook{
    \vfil\newpage
        \if@twoside
            \if@openright
                \null
                \thispagestyle{empty}%
                \newpage
            \fi
        \fi
        \if@tempswa
            \twocolumn
        \fi
}
\newcommand*\l@book[2]{%
    \ifnum\c@tocdepth >-3\relax
        \addpenalty{-\@highpenalty}%
        \addvspace{2.25em\@plus\p@}%
        \setlength\@tempdima{3em}%
        \begingroup
            \parindent\z@\rightskip\@pnumwidth
            \parfillskip -\@pnumwidth
            {
                \leavevmode
                \Large\bfseries#1\hfill\hb@xt@\@pnumwidth{\hss#2}
            }
            \par
            \nobreak
            \global\@nobreaktrue
            \everypar{\global\@nobreakfalse\everypar{}}%
        \endgroup
    \fi}
\newcommand\bookname{Book}
\renewcommand{\thebook}{\texorpdfstring{\Numberstring{book}}{book}}
\providecommand*{\toclevel@book}{-2}
\makeatother
\titleformat{\part}[display]
    {\Large\bfseries}
    {\partname\nobreakspace\thepart}
    {0mm}
    {\Huge\bfseries}
\titlecontents{part}[0pt]
    {\large\bfseries}
    {\partname\ \thecontentslabel: \quad}
    {}
    {\hfill\contentspage}
\titlecontents{chapter}[0pt]
    {\bfseries}
    {\chaptername\ \thecontentslabel:\quad}
    {}
    {\hfill\contentspage}
\newglossarystyle{longpara}{%
    \setglossarystyle{long}%
    \renewenvironment{theglossary}{%
        \begin{longtable}[l]{{p{0.25\hsize}p{0.65\hsize}}}
    }{\end{longtable}}%
    \renewcommand{\glossentry}[2]{%
        \glstarget{##1}{\glossentryname{##1}}%
        &\glossentrydesc{##1}{~##2.}
        \tabularnewline%
        \tabularnewline
    }%
}
\newglossary[not-glg]{notation}{not-gls}{not-glo}{Notation}
\newcommand*{\newnotation}[4][]{%
    \newglossaryentry{#2}{type=notation, name={\textbf{#3}, },
                          text={#4}, description={#4},#1}%
}
%--------------------------LENGTHS------------------------------%
% Spacings for the Table of Contents.
\addtolength{\cftsecnumwidth}{1ex}
\addtolength{\cftsubsecindent}{1ex}
\addtolength{\cftsubsecnumwidth}{1ex}
\addtolength{\cftfignumwidth}{1ex}
\addtolength{\cfttabnumwidth}{1ex}

% Indent and paragraph spacing.
\setlength{\parindent}{0em}
\setlength{\parskip}{0em}
%--------------------------Main Document----------------------------%
\begin{document}
    \ifx\ifresearchnotesosthemathematicsofcassini\undefined
        \section*{Atmospheric Occultations}
        \setcounter{section}{1}
        \renewcommand\thefigure{\arabic{section}.\arabic{figure}}
        \renewcommand\thesubfigure{%
            \arabic{section}.\arabic{figure}.\arabic{subfigure}}
    \fi
    \subsection{Titan Geometry}
        \begin{figure}[H]
        	\centering
        	\captionsetup{type=figure}
        	\begin{subfigure}[b]{0.49\textwidth}
        	    \centering
        	    \captionsetup{type=figure}
        	    \resizebox{\textwidth}{!}{
                    \subimport{../../../../tikz/}
                              {Titan_Occultation_Geometry}}
            	\subcaption{Geometry of an Occultation of Titan}
        	    \label{fig:math_titan_geom_vec}
            \end{subfigure}
            \begin{subfigure}[b]{0.49\textwidth}
                \centering
                \captionsetup{type=figure}
                \resizebox{\textwidth}{!}{
                \subimport{../../../../tikz/}
                          {Titan_Bending_Angle_Geometry}}
                \subcaption{Geometry of the Bending Angle}
                \label{fig:math_geo_bending_angle}
            \end{subfigure}
            \caption{Various Geometries for Titan}
        \end{figure}
        The following definitions are used:
        \begin{enumerate}
            \begin{multicols}{2}
                \item $O$ is the center of Titan.
                \item $E$ is the Earth.
                \item $C$ is the Cassini spacecraft.
                \item $\mathbf{r}_{E}=\overrightarrow{OE}$
                \item $\mathbf{v}_{E}=\dot{\mathbf{r}}_{E}$
                \item $\mathbf{r}_{S}=\overrightarrow{OC}$
                \item $\mathbf{v}_{S}=\dot{\mathbf{r}}_{S}$
                \item $\mathbf{p}_{in}$ is the projection of
                      $O$ onto $\overline{AC}$
                \item $\mathbf{p}_{out}$ is the projection
                      of $O$ onto $\overline{EA}$
                \item $\alpha$ is the bending angle
                      ($\pi-\angle EAC$)
                \item $\hat{\mathbf{n}}_{in}$ is the
                      direction of the emission.
                \item $\hat{\mathbf{n}}_{out}$ is the
                      direction of the reception.
                \item The ray plane lies in the plane $OEC$
                \item $\phi=\angle{AOC}$
                \item $\theta=\angle{ACO}$
                \item $\beta=\angle{OAC}$
            \end{multicols}

            % Multicols adds 6pt a vspace.
            \vspace{-6pt}
            \item $A$ is the intersection of the lines
                  starting at $C$ and $E$, parallel to
                  $\hat{\mathbf{n}}_{in}$ and
                  $\hat{\mathbf{n}}_{out}$, respectively.
        \end{enumerate}

        % Replace the 6pt vspace removed to below the list.
        \vspace{6pt}
        Where $\dot{\mathbf{r}}$ denotes the time derivative
        of $\mathbf{r}$. The following assumptions are made:
        \begin{enumerate}
            \begin{multicols}{2}
                \item $\angle{OAE}=\angle{OAC}$
                \item $A$ lies in the plane $OEC$
            \end{multicols}
        \end{enumerate}
        \begin{theorem}
            \label{theorem:ray_plane_perp_to_r_e_cross_r_s}
            The ray plane is perpendicular to
            $\hat{\mathbf{z}}%
             =\frac{\mathbf{r}_{S}\times
             \mathbf{r}_{E}}{\norm{\mathbf{r}_{S}\times
             \mathbf{r}_{E}}}$
        \end{theorem}
        \begin{proof}
            As the ray plane is the plane $OEC$,
            $\mathbf{r}_{S}$ and $\mathbf{r}_{E}$ lie parallel
            to this plane. Moreover, during an occultation,
            $\mathbf{r}_{S}$ and $\mathbf{r}_{E}$ are not
            parallel and therefore $OEC$ is uniquely determined
            by $\mathbf{r}_{E}$, $\mathbf{r}_{S}$, and the point
            $O$. But
            $\hat{\mathbf{z}}%
             =\frac{\mathbf{r}_{S}\times
             \mathbf{r}_{E}}{\norm{\mathbf{r}_{S}\times
             \mathbf{r}_{E}}}$
            is perpendicular to both $\mathbf{r}_{E}$ and
            $\mathbf{r}_{S}$. Therefore $\hat{\mathbf{z}}$ is
            perpendicular to the ray plane.
        \end{proof}
        \begin{theorem}
            \label{theorem:r_e_dot_p_out_equal_p_out_square}
            $\mathbf{r}_{E}\cdot\mathbf{p}_{out}%
             =\norm{\mathbf{p}_{out}}^{2}$
        \end{theorem}
        \begin{proof}
            $\mathbf{p}_{out}$ is the projection of the $O$
            onto $\overline{EA}$. But $\overline{EA}$ lies
            parallel to $\hat{\mathbf{n}}_{out}$, and
            therefore $\mathbf{p}_{out}$ and
            $\hat{\mathbf{n}}_{out}$ are orthogonal,
            and thus
            $\mathbf{p}_{out}\cdot\hat{\mathbf{n}}_{out}=0$.
            Moreoever,
            $\mathbf{r}_{E}%
             =\mathbf{p}_{out}+(\mathbf{r}_{E}\cdot
             \hat{\mathbf{n}}_{out}) \hat{\mathbf{n}}_{out}$.
            But then:
            \begin{align*}
                \mathbf{p}_{out}\cdot \mathbf{r}_{E}
                &=\mathbf{p}_{out}\cdot\big(
                    \mathbf{p}_{out}
                    +(\mathbf{r}_{E}\cdot
                    \hat{\mathbf{n}}_{out})
                    \hat{\mathbf{n}}_{out}
                \big)\\
                \Rightarrow\mathbf{p}_{out}\cdot \mathbf{r}_{E}
                &=\mathbf{p}_{out}\cdot\mathbf{p}_{out}
                 +(\mathbf{r}_{E}\cdot\hat{\mathbf{n}}_{out})
                  \mathbf{p}_{out}\cdot \hat{\mathbf{n}}_{out}\\
                \Rightarrow\mathbf{p}_{out}\cdot\mathbf{r}_{E}
                &=\mathbf{p}_{out}\cdot\mathbf{p}_{out}
            \end{align*}
            Therefore
            $\mathbf{p}_{out}\cdot\mathbf{r}_{E}%
             =\norm{\mathbf{p}_{out}}^{2}$
        \end{proof}
        \begin{theorem}
            $\alpha%
             =\cos^{-1}(\hat{\mathbf{n}}_{in}
              \cdot\hat{\mathbf{n}}_{out})$
        \end{theorem}
        \begin{proof}
            By definition,
            $\alpha=\pi-\angle{EAC}$.
            But $\hat{\mathbf{n}}_{out}$ lies parallel to
            $\overrightarrow{AE}$, and $-\hat{\mathbf{n}}_{in}$
            lies parallel to $\overrightarrow{AC}$. Therefore:
            \begin{equation*}
                -\hat{\mathbf{n}}_{out}\cdot
                 \hat{\mathbf{n}}_{in}
                =\hat{\mathbf{n}}_{out}\cdot
                 (-\hat{\mathbf{n}}_{in})
                =\norm{\hat{\mathbf{n}}_{out}}
                 \norm{-\hat{\mathbf{n}}_{in}}\cos(\angle{EAC})
            \end{equation*}
            But $\hat{\mathbf{n}}_{in}$ and
            $\hat{\mathbf{n}}_{out}$ are unit vectors,
            and therefore
            $\norm{\hat{\mathbf{n}}_{out}}%
             =\norm{-\hat{\mathbf{n}}_{in}}=1$.
            Therefore:
            \begin{equation*}
                \angle EAC
                =\cos^{-1}(
                    -\hat{\mathbf{n}}_{out}\cdot
                    \hat{\mathbf{n}}_{in}
                )
            \end{equation*}
            But $\alpha=\pi-\angle{EAC}$,
            and $\cos^{-1}(-x)=\pi-\cos^{-1}(x)$.
            Therefore:
            \begin{equation*}
                \alpha=\pi-\angle{EAC}
                =\pi-\big(
                    \pi-\cos^{-1}(\hat{\mathbf{n}}_{out}\cdot
                    \hat{\mathbf{n}}_{in})
                \big)
                =\cos^{-1}(\hat{\mathbf{n}}_{out}\cdot
                 \hat{\mathbf{n}}_{in})
            \end{equation*}
        \end{proof}
        \begin{theorem}
            $\theta%
             =\cos^{-1}\big(%
                  \frac{(-\mathbf{r}_{S})\cdot%
                  \hat{\mathbf{n}}_{in}}{\norm{\mathbf{r}_{S}}}%
              \big)$
        \end{theorem}
        \begin{proof}
            For $\theta=\angle OCA$.
            But $\hat{\mathbf{n}}_{in}$ is parallel with
            $\overrightarrow{CA}$, and $(-\mathbf{r}_{S})$
            is parallel with $\overrightarrow{CO}$.
            Therefore:
            \begin{align*}
                (-\mathbf{r}_{S})\cdot\hat{\mathbf{n}}_{in}
                &=\norm{(-\mathbf{r}_{S})}
                  \norm{\hat{\mathbf{n}}_{in}}\cos(\theta)\\
                \Rightarrow\theta
                &=\cos^{-1}\bigg(
                    \frac{%
                        (-\mathbf{r}_{S})\cdot
                        \hat{\mathbf{n}}_{in}
                    }{\norm{\mathbf{r}_{S}}}
                \bigg)
            \end{align*}
        \end{proof}
        \begin{theorem}
            $\beta%
             =\pi-\frac{1}{2}\cos^{-1}%
              \bigg(%
                  \frac{\mathbf{r}_{s}\cdot \mathbf{r}_{E}}
                  {%
                    \norm{\mathbf{r}_{s}}%
                    \norm{\mathbf{r}_{E}}%
                  }%
              \bigg)%
             -\frac{1}{2}\cos^{-1}%
              \bigg(%
                  \frac{%
                      \mathbf{r}_{E}\cdot%
                      \hat{\mathbf{n}}_{out}%
                  }{\norm{\mathbf{r}_{E}}}%
              \bigg)%
              -\frac{1}{2}\cos^{-1}\bigg(%
                  \frac{%
                      (-\mathbf{r}_{s})\cdot%
                      \hat{\mathbf{n}}_{in}%
                  }{\norm{\mathbf{r}_{s}}}%
              \bigg)$
        \end{theorem}
        \begin{proof}
            The sum of the angles in $OEAC$ is $2\pi$.
            But
            $\angle{OAE}=\angle{OAC}=\phi$, and therefore:
            \begin{align*}
                2\beta&=\angle{EAC}\\
                \Rightarrow
                2\pi
                &=2\beta+\angle{AEO}
                 +\angle{EOC}+\angle{OCA}\\
                \Rightarrow\beta
                &=\pi-\frac{\angle{AEO}}{2}
                 -\frac{\angle EOC}{2}-\frac{\angle OCA}{2}
            \end{align*}
            But:
            \begin{align*}
                (-\hat{\mathbf{n}}_{out})\cdot
                (-\hat{\mathbf{r}}_{E})
                &=\norm{\mathbf{r}_{E}}\cos(\angle AEO)\\
                \Rightarrow\angle{AEO}
                &=\cos^{-1}\bigg(
                    \frac{
                        \hat{\mathbf{n}}_{out}\cdot
                        \mathbf{r}_{E}
                    }{\norm{\mathbf{r}_{E}}}
                \bigg)
            \end{align*}
            Also:
            \begin{align*}
                \mathbf{r}_{E}\cdot\mathbf{r}_{S}
                &=\norm{\mathbf{r}_{E}}
                  \norm{\mathbf{r}_{S}}\cos(\angle EOC)\\
                \Rightarrow\angle{EOC}
                &=\cos^{-1}
                  \bigg(
                      \frac{
                          \mathbf{r}_{E}\cdot
                          \mathbf{r}_{S}
                      }{
                          \norm{\mathbf{r}_{E}}
                          \norm{\mathbf{r}_{S}}
                      }
                  \bigg)
            \end{align*}
            But
            $\angle{OCA}%
             =\theta%
             =\cos^{-1}%
              \big(\frac{%
                       (-\mathbf{r}_{S})\cdot%
                       \hat{\mathbf{n}}_{in}%
                   }{\norm{\mathbf{r}_{S}}}%
              \big)$.
            Therefore:
            \begin{equation*}
                \beta=\pi-\frac{1}{2}\cos^{-1}
                    \bigg(
                        \frac{
                            \mathbf{r}_{s}\cdot
                            \mathbf{r}_{E}
                        }{
                            \norm{\mathbf{r}_{s}}
                            \norm{\mathbf{r}_{E}}
                        }
                    \bigg)
                    -\frac{1}{2}\cos^{-1}
                    \bigg(
                        \frac{
                            \mathbf{r}_{E}\cdot
                            \hat{\mathbf{n}}_{out}
                        }{
                            \norm{\mathbf{r}_{E}}
                        }
                    \bigg)
                    -\frac{1}{2}\cos^{-1}
                    \bigg(
                        \frac{
                            (-\mathbf{r}_{s})\cdot
                            \hat{\mathbf{n}}_{in}
                        }{
                            \norm{\mathbf{r}_{s}}
                        }
                    \bigg)
            \end{equation*}
        \end{proof}
        \begin{theorem}
            $\alpha=\pi-2\beta$
        \end{theorem}
        \begin{proof}
            $\alpha$ and $\angle EAC$ are supplementary to
            the ray $\overrightarrow{CA}$, and therefore
            $\alpha+\angle EAC=\pi$.
            But $\angle{EAC}=\angle{EAC}+\angle{OAC}=2\beta$.
            Therefore $\alpha+2\beta=\pi$.
            Thus, $\alpha=\pi-2\beta$.
        \end{proof}
        \begin{theorem}
            $\theta=\frac{\pi}{2}+\frac{\alpha}{2}-\phi$
        \end{theorem}
        \begin{proof}
            As the angles of a triangle sum to $\pi$,
            $\theta+\beta+\phi=\pi$. But 
            $\alpha=\pi-2\beta\Rightarrow\beta%
             =\frac{\pi}{2}-\frac{\alpha}{2}$.
            So we have:
            \begin{align*}
                \theta+\phi+\beta
                &=\pi\\
                \Rightarrow
                \theta+\phi+\frac{\pi}{2}-\frac{\alpha}{2}
                &=\pi\\
                \Rightarrow\theta
                &=\frac{\pi}{2}+\frac{\alpha}{2}-\phi
            \end{align*}
        \end{proof}
        \begin{theorem}
            $\phi%
             =\frac{1}{2}\cos^{-1}%
                 \bigg(%
                     \frac{%
                         \mathbf{r}_{s}\cdot%
                         \mathbf{r}_{E}%
                     }{%
                         \norm{\mathbf{r}_{s}}%
                         \norm{\mathbf{r}_{E}}}%
                 \bigg)%
             +\frac{1}{2}\cos^{-1}%
                 \bigg(%
                     \frac{%
                         \mathbf{r}_{E}\cdot%
                         \hat{\mathbf{n}}_{out}%
                     }{%
                         \norm{\mathbf{r}_{E}}}%
                 \bigg)%
             -\frac{1}{2}\cos^{-1}%
                  \bigg(%
                      \frac{%
                          (-\mathbf{r}_{s})\cdot%
                          \hat{\mathbf{n}}_{in}%
                      }{\norm{\mathbf{r}_{s}}}%
                  \bigg)$
        \end{theorem}
        \begin{proof}
            For:
            \begin{align*}
                \pi&=\beta+\theta+\phi\\
                \theta
                &=\cos^{-1}
                    \bigg(
                        \frac{
                            (-\mathbf{r}_{S})\cdot
                            \hat{\mathbf{n}}_{in}
                        }{\norm{\mathbf{r}_{S}}}
                    \bigg)\\
                \beta&= \pi-\frac{1}{2}\cos^{-1}
                    \bigg(
                        \frac{
                            \mathbf{r}_{s}\cdot
                            \mathbf{r}_{E}
                        }{
                            \norm{\mathbf{r}_{s}}
                            \norm{\mathbf{r}_{E}}
                        }
                    \bigg)
                    -\frac{1}{2}\cos^{-1}
                        \bigg(
                            \frac{
                                \mathbf{r}_{E}\cdot
                                \hat{\mathbf{n}}_{out}
                            }{\norm{\mathbf{r}_{E}}}
                        \bigg)
                        -\frac{1}{2}\cos^{-1}
                            \bigg(
                                \frac{
                                    (-\mathbf{r}_{s})\cdot
                                    \hat{\mathbf{n}}_{in}
                                }{\norm{\mathbf{r}_{s}}}
                            \bigg)\\
                \Rightarrow\phi
                &=\frac{1}{2}\cos^{-1}
                    \bigg(
                        \frac{
                            \mathbf{r}_{s}\cdot
                            \mathbf{r}_{E}
                        }{
                            \norm{\mathbf{r}_{s}}
                            \norm{\mathbf{r}_{E}}
                        }
                    \bigg)
                 +\frac{1}{2}\cos^{-1}
                     \bigg(
                         \frac{
                             \mathbf{r}_{E}\cdot
                             \hat{\mathbf{n}}_{out}
                         }{
                             \norm{\mathbf{r}_{E}}
                         }
                     \bigg)
                 -\frac{1}{2}\cos^{-1}
                     \bigg(
                         \frac{
                             (-\mathbf{r}_{s})\cdot
                             \hat{\mathbf{n}}_{in}
                         }{
                             \norm{\mathbf{r}_{s}}
                         }
                     \bigg)
            \end{align*}
        \end{proof}
        \begin{theorem}
            \label{%
                theorem:impact_parameter_p_%
                closed_form_solution
            }
            $p=\norm{\mathbf{p}_{in}}%
              =\norm{\mathbf{r}_{S}}%
               \cos(\phi-\frac{\alpha}{2})$
        \end{theorem}
        \begin{proof}
            As $P$ is the orthogonal projection of $O$
            onto $\overline{CA}$,
            $\angle{OPC}=\frac{\pi}{2}$. But then:
            \begin{equation*}
                |\overline{OP}|
                =|\overline{OC}|\sin(\angle{OCP})
            \end{equation*}
            But $|\overline{OP}|=\norm{\mathbf{p}_{in}}$,
            $|\overline{OC}|=\norm{\mathbf{r}_{S}}$,
            and $\angle{OCP}=\theta$. Therefore:
            \begin{equation*}
                \norm{\mathbf{p}_{in}}
                =\norm{\mathbf{r}_{S}}\sin(\theta)
            \end{equation*}
            But $\theta=\frac{\pi}{2}+\frac{\alpha}{2}-\phi$,
            and $\sin(\frac{\pi}{2}+x)=\cos(x)$. Therefore:
            \begin{equation*}
                \norm{\mathbf{p}_{in}}
                =\norm{\mathbf{r}_{S}}\cos
                    \big(\frac{\alpha}{2}-\phi\big)
            \end{equation*}
        \end{proof}
        \begin{theorem}
            $\norm{\mathbf{p}_{in}}%
             =|\overline{OA}|\sin(\beta)$
        \end{theorem}
        \begin{proof}
            For $\overline{OP}$ is perpendicular to
            $\overline{CA}$, and therefore $\Delta OPA$
            is a right-angled triangle, and $\overline{OA}$
            is the hypotenuse. Moreoever
            $\angle{PAO}=\beta$. But then:
            \begin{align*}
                |\overline{OP}|
                &=|\overline{OA}|\sin(\angle PAO)\\
                \Rightarrow|OP|
                &=|\overline{OA}|\sin(\beta)
            \end{align*}
            But $\norm{\mathbf{p}_{in}}=|\overline{OP}|$, and
            thus $\norm{\mathbf{p}_{in}}=|OA|\sin(\beta)$
        \end{proof}
        \begin{theorem}
            \label{theorem:p_out_equals_p_in}
            $\norm{\mathbf{p}_{in}}=\norm{\mathbf{p}_{out}}$
        \end{theorem}
        \begin{proof}
            For $\angle OAE=\angle{OAC}=\beta$, and thus:
            \begin{equation*}
                \norm{\mathbf{p}}_{out}
                =|\overline{OA}|\sin(\angle OAE)
                =|\overline{OA}|\sin(\beta)
                =\norm{\mathbf{p}}_{in}
            \end{equation*}
        \end{proof}
\end{document}