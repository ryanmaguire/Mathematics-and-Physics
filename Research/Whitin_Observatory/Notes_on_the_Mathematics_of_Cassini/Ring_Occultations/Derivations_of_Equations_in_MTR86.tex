\documentclass[crop=false,class=article,oneside]{standalone}
%----------------------------Preamble-------------------------------%
%---------------------------Packages----------------------------%
\usepackage{geometry}
\geometry{b5paper, margin=1.0in}
\usepackage[T1]{fontenc}
\usepackage{graphicx, float}            % Graphics/Images.
\usepackage{natbib}                     % For bibliographies.
\bibliographystyle{agsm}                % Bibliography style.
\usepackage[french, english]{babel}     % Language typesetting.
\usepackage[dvipsnames]{xcolor}         % Color names.
\usepackage{listings, lstlinebgrd}      % Verbatim-Like Tools.
\usepackage{mathtools, esint, mathrsfs} % amsmath and integrals.
\usepackage{amsthm, amsfonts}           % Fonts and theorems.
\usepackage{tabularx}
\usepackage{tcolorbox}                  % Frames around theorems.
\usepackage{upgreek}                    % Non-Italic Greek.
\usepackage{paracol}                    % Two-column styling.
\usepackage{wrapfig}                    % Wrap text around figure.
\usepackage{fmtcount, etoolbox}         % For the \book{} command.
\usepackage[newparttoc]{titlesec}       % Formatting chapter, etc.
\usepackage{titletoc}                   % Allows \book in toc.
\usepackage[nottoc]{tocbibind}          % Bibliography in toc.
\usepackage[titles]{tocloft}            % ToC formatting.
\usepackage{multicol, enumitem}         % Multi-column/enumerate.
\usepackage{import}                     % Import external files.
\usepackage{pgfplots, tikz}             % Drawing/graphing tools.
\usetikzlibrary{
    calc,                   % Calculating right angles and more.
    angles,                 % Drawing angles within triangles.
    arrows.meta,            % Latex and Stealth arrows.
    quotes,                 % Adding labels to angles.
    positioning,            % Relative positioning of nodes.
    decorations.markings,   % Adding arrows in the middle of a line.
    patterns,
    arrows,
    shapes,
    shapes.geometric,
    cd,
    hobby,
    babel
}                                       % Libraries for tikz.
\pgfplotsset{compat=1.9}                % Version of pgfplots.
\usepackage[font=scriptsize,
            labelformat=simple,
            labelsep=colon]{subcaption} % Subfigure captions.
\usepackage[font={scriptsize},
            hypcap=true,
            labelsep=colon]{caption}    % Figure captions.
\usepackage{hyperref}                   % Allows for hyperlinks.
\hypersetup{
    colorlinks=true,
    linkcolor=blue,
    filecolor=magenta,
    urlcolor=Cerulean,
    citecolor=SkyBlue
}                           % Colors for hyperref.
\usepackage[toc,acronym,nogroupskip]{glossaries} % Glossaries and acronyms.
\usepackage[subpreambles=false]{standalone}      % Complileable sub files.

% Various font stuff from kiwi.
% Use this for Times text and Computer Modern math
%\usepackage{times}

% Quite nice
%\usepackage[charter, greekfamily=, greekuppercase=italicized]{mathdesign}
%\usepackage[utopia, greekuppercase=italicized]{mathdesign}    % Math is narrower

% Use this for Times text and math
%\usepackage{newtxtext}
%\usepackage[libertine,cmintegrals]{newtxmath}
%\usepackage{fix-cm}

%\usepackage{txfontsb}
% or
%\usepackage{mathptmx}

%\usepackage[scaled=0.92]{helvet}
%\renewcommand{\rmdefault}{ptm}

%\usepackage{mathpazo}    % add possibly `sc` and `osf` options
%\usepackage{eulervm}

%\usepackage{fourier}
%\renewcommand{\rmdefault}{ptm}
%\usepackage{mathptm}

%\usepackage{fontspec}
%\setmainfont{lmodern}

%\usepackage[varg]{txfonts}
%\usepackage{fouriernc}
%\usepackage{mathpazo}

%\usepackage{bookman}
%\usepackage[scaled]{uarial}
%\usepackage[scaled]{helvet}
%\renewcommand*\familydefault{\sfdefault}
%\usepackage[math]{anttor}

%\newcommand\fgeorgia{\fontfamily{jvn}\selectfont}
%\newcommand\ftimes{\fontfamily{ptm}\selectfont}
%\newcommand\fhelvetica{\fontfamily{phv}\selectfont}
%\newcommand\fcourier{\fontfamily{pcr}\selectfont}
%\newcommand\fbookman{\fontfamily{pbk}\selectfont}
%\newcommand\fnewcentury{\fontfamily{pnc}\selectfont}
%\newcommand\fpalatino{\fontfamily{ppl}\selectfont}
%\newcommand\favantgarde{\fontfamily{pag}\selectfont}
%\newcommand\fnormal{\normalfont}
%\newcommand\fsize[1]{\ifnum#1>0\fontsize{#1}{#1}\selectfont\else\normalsize\fi}
%------------------------Theorem Styles-------------------------%
% Define theorem style for default spacing and normal font.
\newtheoremstyle{normal}
    {\topsep}               % Amount of space above the theorem.
    {\topsep}               % Amount of space below the theorem.
    {}                      % Font used for body of theorem.
    {}                      % Measure of space to indent.
    {\bfseries}             % Font of the header of the theorem.
    {}                      % Punctuation between head and body.
    {.5em}                  % Space after theorem head.
    {}

% Define theorem style for default spacing with italicized font.
\newtheoremstyle{normalit}{\topsep}{\topsep}
                {\itshape}{}{\bfseries}{}{.5em}{}

% Italic header environment.
\newtheoremstyle{thmit}{\topsep}{\topsep}{}{}{\itshape}{}{0.5em}{}

% Define italicized environments.
\theoremstyle{normalit}
\newtheorem{theorem}{Theorem}[section]
\newtheorem{lemma}{Lemma}[section]
\newtheorem{corollary}{Corollary}[section]
\newtheorem{proposition}{Proposition}[section]
\newtheorem*{theorem*}{Theorem}

% Define environments with italic headers.
\theoremstyle{thmit}
\newtheorem*{solution}{Solution}
\newtheorem*{fsolution}{Solution}

% Define default environments.
\theoremstyle{normal}
\newtheorem{example}{Example}[section]
\newtheorem{definition}{Definition}[section]
\newtheorem{problem}{Problem}[section]
\newtheorem{question}{Question}[section]
\newtheorem{remark}{Remark}[section]
\newtheorem{properties}{Properties}[section]
\newtheorem{notation}{Notation}[section]
\newtheorem{axiom}{Axiom}[section]
\newtheorem*{properties*}{Properties}
\newtheorem*{remark*}{Remark}
\newtheorem*{definition*}{Definition}
\theoremstyle{plain}

% Define framed environment.
\tcbuselibrary{most}
\newtcbtheorem[use counter*=theorem]{ftheorem}{Theorem}%
    {colback=green!5,colframe=green!35!black,
     fonttitle=\bfseries\upshape}{th}

\newtcbtheorem[use counter*=example]{fdefinition}{Definition}%
    {fonttitle=\bfseries\upshape,
     colback=blue!5!white,colframe=blue!75!black}{def}

\newtcbtheorem[use counter*=example]{fexample}{Example}%
    {fonttitle=\bfseries\upshape,
     colback=red!5!white,colframe=red!75!black}{ex}

\newtcbtheorem[use counter*=notation]{fnotation}{Notation}%
    {fonttitle=\bfseries\upshape,
     colback=SeaGreen!5!white,colframe=SeaGreen!75!black}{ex}

\newtcbtheorem[use counter*=corollary]{fcorollary}{Corollary}%
    {fonttitle=\bfseries\upshape,
     colback=Orchid!5!white,colframe=Orchid!75!black}{ex}

\newenvironment{bproof}{\textit{Proof.}}{\hfill$\square$}
\tcolorboxenvironment{bproof}{blanker,breakable,left=5mm,
                             before skip=10pt,after skip=10pt,
                             borderline west={1mm}{0pt}{red}}
\tcolorboxenvironment{fsolution}
    {enhanced jigsaw,colframe=cyan,interior hidden,breakable}

%--------------------Declared Math Operators--------------------%
\DeclareMathOperator{\Refl}{Refl}           % Reflection operator.
\DeclareMathOperator{\Span}{Span}           % Span of a set of vectors.
\DeclareMathOperator{\Card}{Card}           % Cardinality of set.
\DeclareMathOperator{\Ord}{Ord}             % Ordinal of ordered set.
\DeclareMathOperator{\Tr}{Tr}               % Trace of matrix.
\DeclareMathOperator{\adjoint}{adj}         % Adjoint of matrix.
\DeclareMathOperator{\rk}{rk}               % Rank of operator.
\DeclareMathOperator{\nul}{nul}             % Null space of operator.
\DeclareMathOperator{\sgn}{sgn}             % Sign of a number.
\DeclareMathOperator{\multideg}{mutlideg}   % Multi-Degree (Graphs).
\DeclareMathOperator{\GCD}{GCD}             % Greatest common denominator.
\DeclareMathOperator{\LM}{LM}               % Leading monomial
\DeclareMathOperator{\LC}{LC}               % Leading coefficient.
\DeclareMathOperator{\LT}{LT}               % Leading term.
\DeclareMathOperator{\LCM}{LCM}             % Least common multiple.
\DeclareMathOperator{\Mon}{Mon}             % Monomial.
\DeclareMathOperator{\Spec}{Spec}           % Spectrum.
\DeclareMathOperator{\proj}{proj}           % Projection.
\DeclareMathOperator{\comp}{comp}           % Component.
\DeclareMathOperator{\sinc}{sinc}           % Sinc function.
\DeclareMathOperator{\Ima}{Im}              % Image of operator.
\DeclareMathOperator{\Prin}{Prin}           % Principal value.
\DeclareMathOperator{\Mod}{mod}             % Modulus.
%------------------------New Commands---------------------------%
\DeclarePairedDelimiter\norm{\lVert}{\rVert}
\DeclarePairedDelimiter\ceil{\lceil}{\rceil}
\DeclarePairedDelimiter\floor{\lfloor}{\rfloor}
\newcommand*\diff{\mathop{}\!\mathrm{d}}
\newcommand*\Diff[1]{\mathop{}\!\mathrm{d^#1}}
\renewcommand{\mod}{\ \Mod}
\renewcommand*{\glstextformat}[1]{\textcolor{RoyalBlue}{#1}}
\renewcommand{\glsnamefont}[1]{\textbf{#1}}
\renewcommand\labelitemii{$\circ$}
\renewcommand\thesubfigure{\arabic{chapter}.\arabic{figure}}
\renewcommand\thesubfigure{%
    \arabic{chapter}.\arabic{figure}.\arabic{subfigure}}
\addto\captionsenglish{\renewcommand{\figurename}{Fig.}}
%------------------------Book Command---------------------------%
\makeatletter
\renewcommand\@pnumwidth{1cm}
\newcounter{book}
\renewcommand\thebook{\@Roman\c@book}
\newcommand\book{%
    \if@openright
        \cleardoublepage
    \else
        \clearpage
    \fi
    \thispagestyle{plain}%
    \if@twocolumn
        \onecolumn
        \@tempswatrue
    \else
        \@tempswafalse
    \fi
    \null\vfil
    \secdef\@book\@sbook
}
\def\@book[#1]#2{%
    \ifnum \c@secnumdepth >-3\relax
        \refstepcounter{book}%
        \addcontentsline{toc}{book}{
            \bookname\ \thebook:\hspace{1em}#1
        }
    \else
        \addcontentsline{toc}{book}{#1}%
    \fi
    \markboth{}{}%
    {\centering
     \interlinepenalty \@M
     \normalfont
     \ifnum \c@secnumdepth >-2\relax
       \huge\bfseries \bookname\nobreakspace\thebook
       \par
       \vskip 20\p@
     \fi
     \Huge \bfseries #2\par}%
    \@endbook}
\def\@sbook#1{%
    {\centering
     \interlinepenalty \@M
     \normalfont
     \Huge \bfseries #1\par}%
    \@endbook}
\def\@endbook{
    \vfil\newpage
        \if@twoside
            \if@openright
                \null
                \thispagestyle{empty}%
                \newpage
            \fi
        \fi
        \if@tempswa
            \twocolumn
        \fi
}
\newcommand*\l@book[2]{%
    \ifnum \c@tocdepth >-2\relax
        \addpenalty{-\@highpenalty}%
        \addvspace{2.25em \@plus\p@}%
        \setlength\@tempdima{3em}%
        \begingroup
            \parindent \z@ \rightskip \@pnumwidth
            \parfillskip -\@pnumwidth
            {
                \leavevmode
                \Large \bfseries #1\hfil \hb@xt@\@pnumwidth{
                    \hss #2
                }
            }
            \par
            \nobreak
            \global\@nobreaktrue
            \everypar{\global\@nobreakfalse\everypar{}}%
        \endgroup
    \fi}
\newcommand\bookname{Book}
\renewcommand{\thebook}{\texorpdfstring{\Numberstring{book}}{book}}
\providecommand*{\toclevel@book}{-2}
\makeatother
\titlecontents{chapter}[0pt]
    {\bfseries}
    {\chaptername\ \thecontentslabel:\quad}
    {}
    {\hfill\contentspage}
\titleformat{\part}[display]
    {\Large\bfseries}
    {\partname\nobreakspace\thepart}
    {0mm}
    {\Huge\bfseries}
    \titlecontents{part}[0pt]
    {\large\bfseries}
    {\partname\ \thecontentslabel: \quad}
    {}
    {\hfill\contentspage}
\newcommand{\MarkRightAngle}[4][.3cm]
    {\coordinate (tempa) at ($(#3)!#1!(#2)$);
     \coordinate (tempb) at ($(#3)!#1!(#4)$);
     \coordinate (tempc) at ($(tempa)!0.5!(tempb)$);%midpoint
     \draw (tempa) -- ($(#3)!2!(tempc)$) -- (tempb);}
%--------------------------LENGTHS------------------------------%
% Spacings for the Table of Contents.
\addtolength{\cftsecnumwidth}{1ex}
\addtolength{\cftsubsecindent}{1ex}
\addtolength{\cftsubsecnumwidth}{1ex}
\addtolength{\cftfignumwidth}{1ex}
\addtolength{\cfttabnumwidth}{1ex}

% Spacing for multi-column and enumerate environments.
\setlength{\multicolsep}{6pt}
\setlist[enumerate]{itemsep=0pt,topsep=3pt}

% Indent and paragraph spacing.
\setlength{\parindent}{0em}
\setlength{\parskip}{0em}
%--------------------------Main Document----------------------------%
\begin{document}
    \ifx\ifresearchnotesosthemathematicsofcassini\undefined
        \section*{Ring Occultations}
        \setcounter{section}{1}
        \renewcommand\thesubfigure{%
            \arabic{section}.\arabic{figure}.\arabic{subfigure}%
        }
    \fi
    \subsection{Derivations of Equations in MTR86}
        \subsubsection{Equation 4a}
            Let $\hat{\mathbf{u}}$ be the unit
            vector pointing from Earth to Voyager.
            Let $\hat{\mathbf{z}}$ be the spin axis of
            Saturn. To make the arguments easier,
            we assume the line from Earth to Saturn
            and the line from Earth to Voyager are
            parallel (We assume Saturn is infinitely far away).
            Let
            $B=\sin^{-1}(\hat{\mathbf{z}}\cdot \hat{\mathbf{u}})$,
            let
            $\hat{\mathbf{y}}%
             =(\hat{\mathbf{u}}\times%
              \hat{\mathbf{z}})/%
              \norm{\hat{\mathbf{u}}\times%
              \hat{\mathbf{z}}}$,
            and
            $\hat{\mathbf{x}}%
             =\hat{\mathbf{y}}\times \hat{\mathbf{z}}$.
            We take the origin as Saturn's center. So
            $\hat{\mathbf{u}}%
             =\cos(B)\hat{\mathbf{x}}+\sin(B)\hat{\mathbf{z}}$.
            Let $\boldsymbol{\rho}_{0}$ be the vector pointing
            from Saturn to the ring intercept point,
            and let $\boldsymbol{\rho}$ be a vector in
            the ring plane. Let $\phi_{0}$ and
            $\phi$ be the angles made with
            $\boldsymbol{\rho}_{0}$ and $\boldsymbol{\rho}$
            to the $x$ axis, respectively. Then
            $\boldsymbol{\rho}_{0}%
             =\rho_{0}\big(\cos(\phi_0)\hat{\mathbf{x}}+%
              \sin(\phi_{0})\hat{\mathbf{y}}\big)$
            and
            $\boldsymbol{\rho}%
             =\rho\big(\cos(\phi)\hat{\mathbf{x}}+%
              \sin(\phi)\hat{\mathbf{y}}\big)$.
            Let $\mathbf{R}_{c}$ be the vector pointing
            from Saturn to Voyager. Let $D$ be the
            distance from the ring intercept point
            to Voyager. Then
            $\mathbf{R}_{c}%
             =\boldsymbol{\rho}_{0}+D\hat{\mathbf{u}}$.
            We thus have the following:
            \begin{align*}
                \mathbf{R}_{c}
                &=\big(
                    \rho_{0}\cos(\phi_0)+D\cos(B)
                \big)\hat{\mathbf{x}}+
                \rho_{0}\sin(\phi_{0})\hat{\mathbf{y}}+
                D\sin(B)\hat{\mathbf{z}}
                &
                \boldsymbol{\rho}
                &=\rho\cos(\phi)\hat{\mathbf{x}}+
                \rho\sin(\phi)\hat{\mathbf{y}}\\
                \boldsymbol{\rho}_{0}
                &=\rho_{0}\cos(\phi_0)\hat{\mathbf{x}}+
                \rho_{0}\sin(\phi_{0})\hat{\mathbf{y}}
                &
                \hat{\mathbf{u}}
                &=\cos(B)\hat{\mathbf{x}}+\sin(B)\hat{\mathbf{z}}
            \end{align*}
            We wish to compute
            $\hat{\mathbf{u}}\cdot%
             \boldsymbol{\rho}+%
             \norm{\mathbf{R}_{c}-\boldsymbol{\rho}}$.
             We have
            \begin{equation*}
                \big(
                    \cos(B)\hat{\mathbf{x}}+\sin(B)\hat{\mathbf{z}}
                \big)\cdot\big(
                    \rho(\cos(\phi)\hat{\mathbf{x}}+
                    \sin(\phi)\hat{\mathbf{y}})
                \big)=\rho\cos(B)\cos(\phi)
            \end{equation*}
            And
            $\norm{\mathbf{R}_{c}-\boldsymbol{\rho}}%
             =\sqrt{(\mathbf{R}_{c}-%
                    \boldsymbol{\rho})\cdot(\mathbf{R}_{c}-%
                    \boldsymbol{\rho})}$.
            So we have
            $\sqrt{\norm{\mathbf{R}_{c}}^2+%
             \norm{\boldsymbol{\rho}}^2-%
             2\mathbf{R}_{c}\cdot\boldsymbol{\rho}}$.
            But:
            \begin{align*}
                \norm{\mathbf{R}_{c}}^{2}
                &=\rho_{0}^2\cos^2(\phi_0)+
                  2D\rho_{0}\cos(\phi_0)\cos(B)+D^2\cos^2(B)+
                  \rho_{0}^2\sin^2(\phi_0)+D^2\sin^2(B)\\
                &=\rho_{0}^2+D^2+2D\rho_{0}\cos(\phi_0)\cos(B)
            \end{align*}
            And $\norm{\boldsymbol{\rho}}^{2}=\rho^{2}$.
            But also:
            \begin{align*}
                \mathbf{R}_{c}\cdot\boldsymbol{\rho}
                &=\rho\cos(\phi)\big(
                      \rho_{0}\cos(\phi_{0})+D\cos(B)
                  \big)+\rho\rho_{0}\sin(\phi)\sin(\phi_{0})\\
                &=\rho\rho_{0}\big(
                      \cos(\phi)\cos(\phi_{0})+
                      \sin(\phi)\sin(\phi_{0})
                  \big)+\rho D\cos(B)\\
                &=\rho\rho_{0}\cos(\phi-\phi_{0})+\rho D\cos(B)
            \end{align*}
            So we have:
            \begin{align*}
                \norm{\mathbf{R}_{c}-\boldsymbol{\rho}}^{2}
                &=
                \rho_{0}^{2}+D^{2}+
                \rho^{2}+2D\rho_{0}\cos(\phi_{0})\cos(B)-
                2\rho\rho_{0}\cos(\phi-\phi_{0})-2\rho D\cos(B)\\
                &=
                \rho^2+\rho_{0}^2+D^2-
                2\rho\rho_{0}\cos(\phi-\phi_{0})
                + 2D\cos(B)\big(\rho_{0}\cos(\phi_0)-
                \rho\cos(\phi)\big)\\
                \Rightarrow
                \hat{\mathbf{u}}\cdot\boldsymbol{\rho}+
                \norm{\mathbf{R}_{c}-\boldsymbol{\rho}}
                &=\rho\cos(B)\cos(\phi)\\
                &\phantom{=}+
                \sqrt{\rho^2+\rho_{0}^2+D^2-
                      2\rho\rho_{0}\cos(\phi - \phi_{0})+
                      2D\cos(B)\big(\rho_{0}\cos(\phi_0)-
                      \rho\cos(\phi)\big)}
            \end{align*}
            Now the definition of $\hat{T}$ is:
            \begin{equation*}
                \hat{T}=\frac{E_{c}}
                             {E_{0}}
                e^{-ik\hat{\mathbf{u}}\cdot\mathbf{R}_{c}}
            \end{equation*}
            So:
            \begin{equation*}
                \psi=
                k\bigg(
                    \norm{\mathbf{R}_{c}-\rho}^2+
                    \hat{\mathbf{u}}\cdot\boldsymbol{\rho}-
                    \hat{\mathbf{u}}\cdot \mathbf{R}_{c}
                \bigg)
            \end{equation*}
            But:
            \begin{equation*}
                \hat{\mathbf{u}}\cdot \mathbf{R}_{c}
                =\rho_{0}\cos(\phi_0)\cos(B) + D\cos^2(B)+
                D\sin^2(B)=\rho_{0}\cos(\phi_0)\cos(B)+D    
            \end{equation*}
            So we have:
            \begin{align*}
                \frac{\psi}{k}
                &=\phantom{+}\rho\cos(\phi)\cos(B)-
                \rho_0\cos(\phi_0)\cos(B)-D\\
                &\phantom{=}+\sqrt{\rho^2+\rho_{0}^2+D^2-
                2\rho\rho_{0}\cos(\phi-\phi_{0})+
                2D\cos(B)\big(\rho_{0}\cos(\phi_0)-
                \rho\cos(\phi)\big)}\\
                &=\phantom{+}\cos(B)\big(\rho\cos(\phi)-
                \rho_{0}\cos(\phi_{0})\big)-D\\
                &\phantom{=}+\sqrt{\rho^2+\rho_{0}^2+D^2-
                2\rho\rho_{0}\cos(\phi-\phi_{0})+
                2D\cos(B)\big(\rho_{0}\cos(\phi_0)-
                \rho\cos(\phi)\big)}
            \end{align*}
            Let's define the following:
            \begin{align*}
                \xi&=\frac{\cos(B)\big(\rho_{0}\cos(\phi_{0})-
                           \rho\cos(\phi)\big)}
                          {D}
                &
                \eta&=\frac{\rho_{0}^2+\rho^2-
                            2\rho\rho_{0}\cos(\phi-\phi_{0})}
                           {D^2}
            \end{align*}
            Then we have:
            \begin{align*}
                \psi
                &=k\big(\sqrt{D^2+D^2\eta+2D^2\xi}-D-D\xi\big)\\
                &=kD\big(\sqrt{1+\eta+2\xi}-(1+\xi)\big)
            \end{align*}
\end{document}