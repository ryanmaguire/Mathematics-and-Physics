\chapter{Probability}
    \subsection{Probability and Statistics}
        Probability is defined in sample spaces. The sample space of
        a random experiment is the set of all
        possible outcomes for the experiment.
        \begin{example}
            The sample space for tossing two coings randomly is
            $S=\{(H,H),(H,T),(T,H),(T,T)\}$.
        \end{example}
        \begin{definition}
            An event is an element of a sample space.
        \end{definition}
        \begin{definition}
            A probability function is a function
            $P:S\rightarrow[0,1]$, where $S$ is a sample
            space, such that:
            \begin{enumerate}
                \item For all $E\subset{S}$, $0\leq{P(E)}\leq{1}$.
                \item $P(S)=1$ and $P(\emptyset)=0$.
                \item If $A_{n}$ is a set of mutually disjoint
                      events in $S$, then
                      $P(\bigcup_{n=1}^{\infty}A_{n})%
                       =\sum_{n=1}^{\infty}P(A_{n})$
            \end{enumerate}
        \end{definition}
        The uniform probability model on a finite sample space with
        $n$ events gives the probability $1/n$ to every event.
        This is a useful model for many problems, such as flipping
        coins, or tossing dice. Given a collection of events, the probability
        is then the total number of times these events occur divided by
        the total number of possible events. If we take $k$ elements from
        a set of $n$, and if the ordering does not matter, the total number of
        ways to choose the $k$ elements is the number of permutations of $k$
        elements. This is:
        \begin{equation*}
            P(n,k)=\frac{n!}{(n-k)!}
        \end{equation*}
        If the ordering does matter, this is the number of combinations
        that are possible. This is:
        \begin{equation*}
            C(n,r)=\binom{n}{k}=\frac{n!}{k!(n-k)!}
        \end{equation*}
        Here $\binom{n}{k}$ is the \textit{binomial coefficient}.
        \begin{example}
            Suppose 10 men and 8 women are to be selected from to form
            a committee of 5 people. What is the probability that there
            are exactly 3 men? There are $\binom{10}{3}$ ways to
            select 3 men, and $\binom{8}{2}$ ways to select 2 women.
            There is a total of $\binom{18}{5}$ ways to select 5 people.
            Thus, we have:
            \begin{equation*}
                P=\frac{\binom{10}{3}\binom{8}{2}}{\binom{18}{5}}
                =\frac{20}{51}
            \end{equation*}
        \end{example}
        The probability point function, or probability mass function,
        is the function $Q(X)=P(X=x)$. The probability distribution
        function is given by
        $F_{X}(x)=P_{X}(X\leq{x})$. This can be expressed in
        terms of the probability mass function as
        $F_{X}(x)=\sum_{x<X}Q(x)$. For continuous random variables,
        the probability density function is defined as
        $f_{X}(x)=F_{X}'(x)$. In the study of statistics, we choose a sample
        from the total population which is intended to represent the
        entire population as closely as possible.
        \begin{definition}
            The arithmetic mean of a finite set
            $\{x_{1},\hdots,x_{N}\}$ is:
            \begin{equation*}
                \overline{x}=\frac{1}{N}\sum_{n=1}^{N}x_{n}
            \end{equation*}
        \end{definition}
        If $f_{i}$ is the probability of $x_{i}$, then the arithmetic mean is:
        \begin{equation*}
            \overline{x}=
            \frac{\sum_{n=1}^{N}x_{n}f_{n}}{\sum_{n=1}^{N}f_{n}}
        \end{equation*}
        The arithmetic mean is strongly effected by extreme values, or outliers.
        The deviation of $x_{i}$ from the mean $\overline{x}$
        is $d_{i}=x_{i}-\overline{x}$. This has the special property:
        \begin{equation*}
            \sum_{n=1}^{N}d_{n}
            =\sum_{n=1}^{N}(x_{i}-\overline{x})
            =0
        \end{equation*}
        \begin{theorem}
            If $x_{n}$ and $y_{n}$ are finite sequences of
            $N$ elements, and $z_{n}=x_{n}+y_{n}$, then
            $\overline{z}=\overline{x}+\overline{z}$.
        \end{theorem}
        \begin{definition}
            The geometric mean of a finite set
            $\{x_{1},\hdots,x_{N}\}$ is:
            \begin{equation*}
                g=\sqrt[N]{\prod_{n=1}^{N}x_{n}}
            \end{equation*}
            Where
            $\prod_{n=1}^{N}x_{n}=x_{1}\cdot{x_{2}}\cdots{x_{N}}$
        \end{definition}
        Logarithms can help calculate geometric means:
        \begin{theorem}
            Given a finite set $\{x_{1},\hdots,x_{N}\}$,
            the geometric mean is:
            \begin{equation*}
                g=\exp\Big(\frac{1}{N}\sum_{n=1}^{N}x_{n}\Big)
                =\exp(\overline{x})
            \end{equation*}
        \end{theorem}
        Given an initial deposit $A$ in a bank with yearly interest $q$,
        after $n$ years the account balance is given by the
        Compound Interest Formula:
        \begin{equation*}
            M=A(1+q)^{n}
        \end{equation*}
        \begin{definition}
            The harmonic mean of $\{x_{1},\hdots,x_{N}\}$
            is:
            \begin{equation*}
                h=\frac{1}{\frac{1}{N}\sum_{n=1}^{N}\frac{1}{x_{n}}}
                =\frac{n}{\sum_{n=1}^{N}\frac{1}{x_{n}}}
            \end{equation*}
        \end{definition}
        \begin{theorem}
            Given a set of numbers $\{x_{1},\hdots,x_{n}\}$,
            the harmonic mean $h$, and the geometric mean $g$,
            $h\leq{g}\leq\overline{x}$. Equality holds if
            and only if all of the $x_{i}$ are the same.
        \end{theorem}
        \begin{definition}
            The root-mean square (\textit{rms}), or the
            quadratic mean, of a data set $\{x_{1},\hdots,x_{N}\}$
            is:
            \begin{equation*}
                rms=\sqrt{\overline{x^{2}}}
                =\sqrt{\frac{1}{N}\sum_{n=1}^{N}x_{n}^2}
            \end{equation*}
        \end{definition}
        \begin{theorem}
            The quadratic mean of two numbers is greater
            than the geometric mean. That is:
            \begin{equation*}
                \sqrt{ab}\leq\sqrt{\frac{a^{2}+b^{2}}{2}}
            \end{equation*}
        \end{theorem}
        \begin{proof}
            For $(a-b)^{2}\geq{0}$, and thus
            $a^{2}+b^{2}-2ab\geq{0}$. Therefore, etc.
        \end{proof}
        The variation of data is a measurement of how much
        the data spreads about the average of the data.
        \begin{definition}
            The range of a finite data set is the maximum value
            minus the minimum value.
        \end{definition}
        The $n^{th}$ percentile of a data set is the value such that
        $n\%$ of the data set lies below said value, and
        $(100-n)\%$ lies above it. Percentiles are usually split into
        quartiles to better represent data sets. The interquartile range
        is the difference between the $75\%$ and the $25\%$ marks.
        The average deviation of a data set is:
        \begin{equation*}
            \overline{|\overline{x}-x_{i}|}
            =\frac{1}{N}\sum_{n=1}^{N}|\overline{x}-x_{i}|
        \end{equation*}
        Note the need for the absolute value signs. For without them,
        the average would always be zero.
        \begin{definition}
            The standard deviation of a finite data set
            $\{x_{1},\hdots,x_{N}\}$ is:
            \begin{equation*}
                \sigma=
                \sqrt{\frac{1}{N}\sum_{n=1}^{N}(\overline{x}-x_{n})^{2}}
            \end{equation*}
        \end{definition}
        \begin{definition}
            The variance of a data set is the square of the
            standard deviation. $V=\sigma^{2}$.
        \end{definition}
        \begin{theorem}
            The variance is equal to:
            \begin{equation*}
                V=\overline{x^{2}}-\overline{x}^{2}
            \end{equation*}
        \end{theorem}
        For a normal, or bell curve, the interval between
        $\overline{x}$ and $\overline{x}\pm\sigma$ contains
        roughly $68\%$ of the entire data set. $2\sigma$
        contains about $95\%$ of the data, and
        $3\sigma$ contains $99.5\%$.
        The $\chi$ square statistic is defined by:
        \begin{equation*}
            \chi^{2}=\frac{1}{\sigma^{2}}
                \sum_{n=1}^{N}(\overline{x}-x_{n})^{2}
        \end{equation*}
        The $\chi$ square distribution is:
        \begin{equation*}
            Y=Y_{0}\chi{\nu-2}e^{-\frac{1}{2}\chi^{2}}
        \end{equation*}
        Where $\nu=n-1$ is the umber of degrees of freedom,
        and $Y_{0}$ is a constant to make the area under the
        curve equal 1.