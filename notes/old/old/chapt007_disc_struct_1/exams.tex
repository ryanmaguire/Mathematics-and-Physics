\documentclass[../main.tex]{subfiles}
\begin{document}
\section{Exams}
%
\subsection{Practice Exam I}
%
\begin{problem}
Let $S = \{1,2,3,4,5,6,7,8\}$.
\begin{enumerate}
\item How many subsets of $S$ are there with exactly three elements?
\item How many subsets of $S$ contain exactly one even number and two odd numbers?
\item How many subsets of $S$ contain exactly three elements, at most one of which is even?
\item How many subsets of $S$ are there with exactly three elements satisfying the condition that the subset contains the numbers $1$ or $8$ (Or both)?
\item How many subsets of $S$ are there with exactly three elements which satisfy the property that two of its elements sum to $9$?
\end{enumerate}
\end{problem}
\begin{proof}[Solution]
\ 
\begin{enumerate}
\begin{multicols}{3}
\item $\binom{8}{3} = \frac{8!}{3!(8-3)!} = 56$
\item $\binom{4}{1}\binom{4}{2} = 24$
\item $\binom{4}{1} \binom{4}{2} + \binom{4}{0} \binom{4}{3} = 28$.
\end{multicols}
\item $\underbrace{\underset{\textrm{Contains $1$}}{\binom{7}{2}}} + \underbrace{\underset{\textrm{Contains $8$}}{\binom{7}{2}}} - \underbrace{\underset{\textrm{Contains Both}}{\binom{6}{1}}} = 36$
\item We have $8+1=7+2=6+3=5+4= 9$. There are $4$ pairs with $\binom{6}{1}$ subsets per pair. So we have $4\cdot 6 = 24$.
\end{enumerate}
\end{proof}

\begin{problem}
A club has $20$ members.
\begin{enumerate}
\item In how many different ways can the club select a president, vice-president, and secretary?
\item In how many different ways can a social committee with four members be elected?
\item The club contains 12 men and 8 women. In how many different ways can a committee of four people be selected if the committee must have two men and two women?
\item In how many different ways can a four person committee be selected from the club members if one person is designated as the leader of the committee?
\end{enumerate}
\end{problem}
\begin{proof}[Solution]
\
\begin{enumerate}
\begin{multicols}{2}
\item $P(20,3) = \frac{20!}{(20-3)!} = 6840$
\item $\binom{20}{4} = \frac{20!}{4!(20-4)!} = 4845$
\item $\binom{12}{2}\binom{8}{2} = 1848$
\item $\binom{19}{3} = 969$
\end{multicols}
\end{enumerate}
\end{proof}

\begin{problem}
Let $A = \{1,2\}$, $B = \{2,4,5\}$, and $U = \{1,2,3,4,5,6,7\}$. Compute the following:
\begin{enumerate}
\begin{multicols}{4}
\item $A\times B$
\item $A^3$
\item $\mathcal{P}(B)$.
\item $A\setminus B$
\item $A \oplus B$
\item $A\cap B^c$
\item $|\{(x,y)\in U^2:x\ne y\}|$
\item $A\cup B$.
\end{multicols}
\end{enumerate}
\end{problem}
\begin{proof}
\
\begin{enumerate}
\item $\{(1,2),(1,4),(1,5),(2,2),(2,4),(2,5)\}$
\item $\{(1,1,1),(1,1,2),(1,2,1),(1,2,2),(2,1,1),(2,1,2),(2,2,1),(2,2,2)\}$
\item $\big\{\emptyset,\{2\},\{4\},\{5\},\{2,4\},\{2,5\},\{4,5\},\{2,4,5\}\big\}$
\begin{multicols}{3}
\item $\{1\}$
\item $\{1,4,5\}$
\item $\{1\}$
\item $7^2-7 = 42$.
\item $\{1,2,4,5\}$
\end{multicols}
\end{enumerate}
\end{proof}

\begin{problem}
\
\begin{enumerate}
\item Calculate $\sum_{i=0}^{2}\sum_{j=1}^{3}(3i-j)$
\item Simplify $\prod_{i=1}^{n} \frac{2i}{i+1}$
\item Let $A_{i} = \{i,i+1\}$ for $i\in \mathbb{N}$. Find $\cup_{i=1}^{n} A_{i}$.
\end{enumerate}
\end{problem}
\begin{proof}[Solution]
\
\begin{enumerate}
\item $\sum_{i=0}^{2}\sum_{j=1}^{3}(3i-j) = \sum_{i=0}^{2}\big((3i-1)+(3i-2)+(3i-3)\big) = \sum_{i=0}^{2}\big(9i-6\big) = -6+3+12 = 9$.
\item $\prod_{i=1}^{n} \frac{2i}{i+1}= \frac{2^n}{n+1}$. We prove by induction. The base case of $n=1$ is by the definition of $\prod_{i=1}^{1}\frac{2i}{i+1}$. Suppose it is true for $n\in \mathbb{N}$. Then $\prod_{i=1}^{n+1} \frac{2i}{i+1} = \frac{2(n+1)}{(n+1)+1}\prod_{i=1}^{n}\frac{2i}{i+1}$. But from the induction hypothesis, $\prod_{i=1}^{n}\frac{2i}{i+1} = \frac{2^n}{n+1}$. So we have $\frac{2n}{(n+1)+1}\frac{2^n}{n+1} = \frac{2^{n+1}}{(n+1)+n}$. Therefore $\prod_{i=1}^{n+1} \frac{2i}{i+1} = \frac{2^{n+1}}{(n+1)+1}$. Therefore, etc.
\item $\cup_{i=1}^{n} A_i = \mathbb{Z}_{n+1}$. We prove by induction. The base case is by definition of $A_1$. Suppose it is true for $n\in \mathbb{N}$. Then $\cup_{i=1}^{n+1}A_i = \big(\cup_{i=1}^{n}A_{i}\big) \cup A_{n+1}$. From the induction hypothesis, $\cup_{i=1}^{n} = \mathbb{Z}_{n+1}$. So we have $\mathbb{Z}_{n+1} \cup \{n+1,n+2\} = \mathbb{Z}_{n+2}$. Thus $\cup_{i=1}^{n+1} A_{i} = \mathbb{Z}_{n+2}$. Therefore, etc.
\end{enumerate}
\end{proof}

\begin{problem}
\
\begin{enumerate}
\item Compute the binary representation of $75$.
\item Convert $1100101_2$ to decimal.
\end{enumerate}
\end{problem}
\begin{proof}[Solution]
\
\begin{enumerate}
\item $75 = 2\cdot 37+1$. $37 = 2\cdot 18+1$. $18 = 2\cdot 9 + 0$. $9 = 2\cdot 4 + 1$. $4 = 2\cdot 2+0$. $2 = 2\cdot 1+0$, $1= 2\cdot 0 + 1$. So, we have $75 = 1001011_2$.
\item $1100101_2 = 1+4+32+64 = 101$.
\end{enumerate}
\end{proof}

\begin{problem}
A woman has eight friends. Answer the following questions with explanation and/or clear computations:
\begin{enumerate}
\item In how many different ways can she invite four of her friends to dinner?
\item Two of her friends dislike each other. If she invites one friend, she can't invite the other. How many ways can she invite her friends?
\item Five are her friends are men and three are women. How many ways can she invite four of her friends if she wants two men and two women.
\item The eight friends consist of two married couples and four single people. How many ways can she invite four friends over to dinner under the condition that if she invites a husband or wife, she must invite the spouse. 
\end{enumerate}
\end{problem}
\begin{proof}[Solution]
\
\begin{enumerate}
\begin{multicols}{3}
\item $\binom{8}{4} = 70$.
\item $\binom{6}{3} + \binom{6}{3} + \binom{6}{4} = 55$
\item $\binom{5}{2}\binom{3}{2} = 30$
\end{multicols}
\item $\underbrace{\underset{\textrm{Both Couples}}{\binom{4}{0}}}+\underbrace{\underset{\textrm{One Couple}}{\binom{4}{2}}}+\underbrace{\underset{\textrm{Other Couple}}{\binom{4}{2}}}+\underbrace{\underset{\textrm{No Couples}}{\binom{4}{4}}} = 14 $
\end{enumerate}
\end{proof}

\begin{problem}
A bit string is a sequence of numbers consisting of $0's$ and $1's$. Answer the following questions with explanation and/or clear computations:
\begin{enumerate}
\item How many strings of length $5$ are there?
\item How many strings of length $6$ have an even number of $1$'s?
\item How many strings of length 5 begin with $0$ or end with $1$ (Or both)?
\item How many strings of length $6$ contain exactly three $1$'s?
\item How many strings of length $6$ contain at least three $1$'s?
\item How many strings of length $6$ are palindromic (Read the same from left to right as from right to left)?
\end{enumerate}
\end{problem}
\begin{proof}[Solution]
\
\begin{enumerate}
\begin{multicols}{2}
\item $2^5 = 32$.
\item $\binom{6}{0}+\binom{6}{2}+\binom{6}{4}+\binom{6}{6} = 32$
\end{multicols}
\item $\underbrace{\underset{\textrm{Starts and ends with $0$}}{2^3}}+\underbrace{\underset{\textrm{Starts with $0$ and ends with $1$}}{2^3}}+\underbrace{\underset{\textrm{Starts and ends with $1$}}{2^3}} {= 32}$
\begin{multicols}{2}
\item $\binom{6}{3} = 20$
\item $\binom{6}{3}+\binom{6}{4}+\binom{6}{5}+\binom{6}{6} = 42$
\end{multicols}
\item $2^3 = 8$. 
\end{enumerate}
\end{proof}

\begin{problem}
Expand $(2x-5y)^4$.
\end{problem}
\begin{proof}[Solution]
The Binomial Theorem says $(a+b)^n = \sum_{k=0}^{n} \binom{n}{k}a^{n-k}b^{k}$. Here we have $a=2x$, $b= -5y$, and $n=4$. So $(2x-5y)^4 = 2^4 x^4-4\cdot 2^3 x^3\cdot 5y + 6\cdot 2^2 x^2 \cdot 5^2 y^2 - 4\cdot 2x\cdot 5^3y^3 + 5^4y^4 = 16x^4-160x^3y + 600x^2y^2 - 1000xy^3+625y^4$
\end{proof}

\begin{problem}
Find the coefficient of $x^3y^6$ in $(x-10y)^9$.
\end{problem}
\begin{proof}[Solution]
We have $(x-10y)^9 = \sum_{k=0}^{9} \binom{9}{k}x^{n-k}(-10y)^{k}$. We want $k=6$. This gives the coefficient $\binom{9}{6}(-10)^6 = 84,000,000$.
\end{proof}

\begin{problem}
Let $A$ and $B$ be finite sets. Suppose $|A\cup B| = 50$, $|A| = 37$, and $|A\cap B| = 20$. Compute the following:
\begin{enumerate}
\begin{multicols}{4}
\item $|B|$
\item $|A\setminus B|$
\item $|B\setminus A|$
\item $|A\oplus B|$.
\end{multicols}
\end{enumerate}
\end{problem}
\begin{proof}[Solution]
\
\begin{enumerate}
\begin{multicols}{2}
\item $|B| = |A\cup B|-|A|+|A\cap B| = 33$.
\item $|A\setminus B| = |A|-|A\cap B| = 17$.
\item $|B\setminus A| = |B| - |A\cap B| = 13$.
\item $|A\oplus B| = |A\cup B|- |A\cap B| = 30$.
\end{multicols}
\end{enumerate}
\end{proof}
%
\subsection{Exam I}
%
\begin{problem}
Let $A = \{1,3,5\}$, $B = \{3,5,6,7\}$, $C = \{2,7\}$, and let the universe set be $U = \{1,2,3,4,5,6,7,8\}$. Compute the following:
\begin{enumerate}
\begin{multicols}{6}
\item $A\times C$
\item $C^2$
\item $\mathcal{P}(C)$
\item $A^c \setminus C$
\item $A\oplus B$
\item $A\cup B$
\end{multicols}
\end{enumerate}
\end{problem}
\begin{proof}[Solution]
\
\begin{enumerate}
\item $A\times C = \{(1,2),(1,7),(3,2),(3,7),(5,2),(5,7)\}$.
\item $\{(2,2),(2,7),(7,2),(7,7)\}$
\begin{multicols}{2}
\item $\big\{\emptyset, \{2\}, \{7\}, \{2,7\}\big\}$
\item $\{4,6,8\}$
\item $\{1,6,7\}$
\item $\{1,3,5,6,7\}$
\end{multicols}
\end{enumerate}
\end{proof}

\begin{problem}
A special deck of cards contains $28$ cards. Each card comes in one of four colors (Red, yellow, grean, or blue), and there are seven cards for each color labelled with integers $1$ to $7$. Answer the following questions with an explanation and/or computations.
\begin{enumerate}
\item How many ways can a hand of four cards be selected?
\item How many ways can a hand of four cards be selected such that each card is a different color.
\item How many ways can a hand of four cards be selected such that no cards cards have the same colors or numbers.
\item How many ways can a hand of four cards be selected such that there is one red card and three yellow cards. 
\item How many ways can four cards be drawn such that all cards are the same color?
\end{enumerate}
\end{problem}
\begin{proof}[Solution]
\
\begin{enumerate}
\begin{multicols}{3}
\item $\binom{28}{4} = 20,475$
\item $\frac{28 \cdot 21 \cdot 14 \cdot 7}{4!} = 2401$
\item $\frac{28\cdot 18 \cdot 10 \cdot 4}{4!} = 840$
\item $\binom{7}{1}\binom{7}{3} = 245$
\item $4\cdot \binom{7}{4} = 140$
\end{multicols}
\end{enumerate}
\end{proof}

\begin{problem}
A true false quiz has $6$ questions. Answer the following:
\begin{enumerate}
\item How many ways are there to fill out the quiz so that there are exactly four true answer?
\item How many different ways are there to fill out the quiz so that there are at most two false answers?
\end{enumerate}
\end{problem}
\begin{proof}[Solution]
\
\begin{enumerate}
\begin{multicols}{2}
\item $\binom{6}{2} = \frac{6!}{2!(6-4)!}=15$
\item $\binom{6}{4}+\binom{6}{5}+\binom{6}{6} = 22$
\end{multicols}
\end{enumerate}
\end{proof}

\begin{problem}
Find the coefficient of $x^2y^3$ in $(7x-10y)^5$.
\end{problem}
\begin{proof}[Solution]
$(7x-10y)^5 = \sum_{k=0}^{5}\binom{5}{k}(7x)^{5-k}(-10y)^{k}$. So, we have $k = 3$. The coefficient is $\binom{5}{3}7^2(-10)^3 = -490,000$.
\end{proof}

\begin{problem}
Let $A,B$ be finite sets such that $|A\cap B| =10$, $|A| = 22$, and $|B| = 15$. Compute the following:
\begin{enumerate}
\begin{multicols}{4}
\item $|A\cup B|$
\item $|A\setminus B|$
\item $|A\oplus B|$
\item $|A\times B|$
\end{multicols}
\end{enumerate}
\end{problem}
\begin{proof}[Solution]
\
\begin{enumerate}
\begin{multicols}{2}
\item $|A\cup B| = |A|+|B|-|A\cap B| = 27$
\item $|A\setminus B| = |A|-|A\cap B| = 12$
\item $|A\oplus B| = |A\cup B|-|A\cap B| = 17$
\item $|A\times B| = 22\cdot 15 = 330$.
\end{multicols}
\end{enumerate}
\end{proof}

\begin{problem}
Expand and simplify $\sum_{i=0}^{3}\sum_{j=1}^{2}(2x^i-x^j)$
\end{problem}
\begin{proof}[Solution]
$4+4x^3$
\end{proof}

\begin{problem}
Find a formula for $|A\cup B\cup C|$, where $A,B,C$ are finite sets.
\end{problem}
\begin{proof}[Solution]
$|A\cup B\cup C| = |A|+|B|+|C|-|A\cap B|-|A\cap C|-|B\cap C| +|A\cap B \cap C|$
\end{proof}
%
\subsection{Practice Exam II}
%
\begin{problem} Let $a,b$ be real numbers. Write the contrapositive of the following: If $a<\frac{1}{2}$ and $b< \frac{1}{2}$, then $a+b<1$. 
\end{problem}
\begin{proof}[Solution]
If $a+b \geq 1$, then $a\geq \frac{1}{2}$ or $b\geq \frac{1}{2}$.
\end{proof}

\begin{remark}
The contrapositive is logically equivalent to the original statement. That is, the original statement is true if and only if the contrapositive is true.
\end{remark}

\begin{problem}
Is the converse of the statement in problem $1$ true?
\end{problem}
\begin{proof}[Solution]
The converse of $p\rightarrow q$ is $q\rightarrow p$. The converse is if $a+b <1$, then $a<\frac{1}{2}$ and $b< \frac{1}{2}$. This is not true, for $2-2 = 0 < 1$, but $2\not<\frac{1}{2}$
\end{proof}

\begin{problem}
Make a true table for $p,q,p\land q, p\lor q, (p\land q)\lor (\neg p\land q)$
\end{problem}
\begin{proof}[Solution]
\begin{center}
 \begin{tabular}{c c c c c} 
 \hline
 $p$ & $q$ & $p\land q$ & $p\lor q$ & $(p\land q)\lor(\neg p\land q)$ \\ [0.5ex] 
 \hline
 0 & 0 & 0 & 0 & 0\\ 
 0 & 1 & 0 & 1 & 1  \\
 1 & 0 & 0 & 1 & 0 \\
 1 & 1 & 1 & 1 & 1 \\
 \hline
\end{tabular}
\end{center}
\end{proof}

\begin{problem}
Prove that $(p\land q)\lor(\neg p\land q)$ is equivalent to $q$.
\end{problem}
\begin{proof}[Solution]
By the distributive law, $(p\land q)\lor(\neg p\land q) \Leftrightarrow (p\lor \neg p)\land q \Leftrightarrow 1\land q \Leftrightarrow q$.
\end{proof}

\begin{problem}
Prove that if $n$ is an integer, then $n^2+6n$ is even if and only if $n$ is even.
\end{problem}
\begin{proof}[Solution]
Suppose $n$ is even. Then $n=2k$ for some integer $k$. Then $n^2+6n = 4k^2+12k = 2\big(2k^2+6k)$. Thus $n^2+6n$ is even. Suppose $n^2+6n$ is even and that $n$ is odd. Then $n=2k+1$ for some integer $k$. But then $n^2+6n = 2\big(k^2+8k)+1$, and thus is odd. A contradiction. Therefore $n$ is even.
\end{proof}

\begin{problem}
Prove by contradiction that if $7n^2+2n + 1$ is even, then $n$ odd.
\end{problem}
\begin{proof}
Suppose $n$ is even. Then $n=2k$ for some integer $k$. But then $7n^2+2n+1 = 28k^2+14k+1 = 2(14k^2+7k)+1$, which is an odd number, a contradiction. Therefore $n$ is odd.
\end{proof}

\begin{problem}
Express the following in English. Determine whether it is true or false.
\begin{enumerate}
\begin{multicols}{4}
\item $\exists_{x\in \mathbb{Z}}$\footnotesize{(x is even and prime)}
\item $\forall_{n\in \mathbb{P}}(n^2>0)$.
\item $\forall_{x\in \mathbb{P}}\exists_{y\in \mathbb{Q}}(xy=1)$
\item $\exists_{y\in \mathbb{Q}}\forall_{x\in \mathbb{P}}(xy=1)$.
\end{multicols}
\end{enumerate}
\end{problem}
\begin{proof}[Solution]
\
\begin{enumerate}
\item There exists an integer $x$ such that $x$ is even and prime. This is true for $2$ is such an integer.
\item For all positive integers $n$, $n^2>0$. This is true of all real numbers.
\item For all positive integers $x$ there is a rational number $y$ such that $xy=1$. This is true for let $y=\frac{1}{x}$.
\item There exists a rational number $y$ such that for all positive integers $x$, $xy=1$. This is false. For if $x_1 y = 1$ and $x_2 y = 1$, we have $(x_1-x_2)y = 0$. But $x_1 \ne x_2$, so $y=0$. A contradiction as $x_1y=1$. Thus there is no such number.
\end{enumerate}
\end{proof}

\begin{problem}
Prove the following: $\sum_{k=1}^{n} k^2 = \frac{n(n+1)(2n+1)}{6}$
\end{problem}
\begin{proof}[Solution]
We prove by induction. The base case says $1 = \frac{1(1+1)(1+2)}{6}$, which is true. Suppose it is true for some $n\in \mathbb{N}$. Then $\sum_{k=1}^{n+1}k^2 = (n+1)^2+\sum_{k=1}^{n} k^2$. But $\sum_{k=1}^{n} k^2 = \frac{n(n+1)(2n+1)}{6}$ by hypothesis. Thus $\sum_{k=1}^{n+1} k^2 = (n+1)^2+\frac{n(n+1)(2n+1)}{6} = \frac{6(n+1)^2+n(n+1)(2n+1)}{6} = \frac{(n+1)\big(6(n+1)+n(2n+1)\big)}{6} = \frac{n+1}{6}\big(2n^2+7n+6\big) = \frac{(n+1)(n+2)(2n+3)}{6} = \frac{(n+1)\big((n+1)+1\big)\big(2(n+1)+1\big)}{6}$. Therefore, etc.
\end{proof}

\begin{problem}
Does $\neg p\land (p\rightarrow q)$ imply $\neg q$?
\end{problem}
\begin{proof}[Solution]
No it does not.
\end{proof}

\begin{problem}
Let $U = \mathcal{P}(\{1,2,3,4,5\})$, and consider the following propositions over $U$:
\begin{enumerate}
\item[] $p(A):	A\cap\{1,2,3\} = \emptyset$
\item[] $q(A):	A\subset \{1,2,5\}$
\item[] $r(A):	|A| = 2$
\end{enumerate}
Find the truth sets of the following $p,r, q\land r, q\lor p$.
\end{problem}
\begin{proof}[Solution]
\
\begin{enumerate}
\item $T_{p} = \mathcal{P}(\{4,5\})$
\item $T_{r} = \{\{1,2\},\{1,3\},\{1,4\},\{1,5\},\{2,3\},\{2,4\},\{2,5\},\{3,4\},\{3,5\},\{4,5\}\}$.
\item $T_{q\land r} = \{1,3\},\{1,5\},\{3,5\}$.
\item $T_{q\lor p} = \{\emptyset, \{1\},\{3\},\{5\},\{1,3\},\{1,5\},\{3,5\},\{4,\},\{4,5\},\{1,3,5\}$.
\end{enumerate}
\end{proof}

\begin{problem}
Consider the following propositions:
\begin{enumerate}
\item[] $p(n):	n^2 = 100$
\item[] $q(n):	2n-20 = 0$
\item[] $r(n):	n^4 = 10,000$
\end{enumerate}
\begin{enumerate}
\item Suppose $p,q,r$ are over $\mathbb{R}$. Which are equivalent? Given an example of a proposition which implies another example.
\item Suppose $p,q,r$ are over $\mathbb{C}$. Which are equivalent? Given an example of a proposition which implies another example.
\end{enumerate}
\end{problem}

\begin{proof}[Solution]
\
\begin{enumerate}
\item Over $\mathbb{R}$, we have $T_{p} = \{-10,10\}$, $T_{q} = \{10\}$, and $T_{r} = \{-10,10\}$. So $p$ and $r$ are equivalent since $T_{p} = T_{r}$. Also $q\Rightarrow p$ and $q\rightarrow r$ since $T_{q} \subset T_{p}$ and $T_{q}\subset T_{r}$.
\item Over $\mathbb{C}$, we have $T_{p} = \{-10,10\}$, $T_{q} = \{10\}$, $T_{r} = \{-10i,10i,-10,10\}$. So none of the propositions are equivalent, however $q\Rightarrow p \rightarrow r$.
\end{enumerate}
\end{proof}

\begin{problem}
\label{problem:discrete structures_make_a_truth_table_for_p_and_1_and_more}
Make a truth table for $p\land 1$, $p\rightarrow p\lor q$, $p\lor q \rightarrow q$, $p\lor 0 \leftrightarrow \neg p$. Determine which are tautologies, contradictions, or neither.
\end{problem}
\begin{proof}[Solution]
%
\begin{table}[H]
    \centering
    \begin{tabular}{c c c c c c c c c c c c c} 
         \hline 
         $p$ & $q$ & $\neg p$ & $\neg q$ & $0$ & $1$ & $p\lor q$ & $p \lor 0$ & $p\land 1$ & $p\rightarrow p\lor q$ & $p\lor q \rightarrow q$ & $p\lor 0 \leftrightarrow \neg p$.\\ [0.5ex] 
         \hline
         0 & 0 & 1 & 1 & 0 & 1 & 0 & 0 & 0 & 1 & 1 & 0 \\
         1 & 0 & 0 & 1 & 0 & 1 & 1 & 1 & 1 & 1 & 0 & 0 \\
         0 & 1 & 1 & 0 & 0 & 1 & 1 & 0 & 0 & 1 & 1 & 0 \\
         1 & 1 & 0 & 0 & 0 & 1 & 1 & 1 & 1 & 1 & 1 & 0 \\
         \hline
    \end{tabular}
    \caption{Truth Table for Problem \ref{problem:discrete structures_make_a_truth_table_for_p_and_1_and_more}}
    \label{tab:my_label}
\end{table}
%
We see that $p\rightarrow p\lor q$ is a tautology, $p\lor 0 \leftrightarrow \neg p$ is a contradiction, $p\land 1$ is neither, and $p\lor q \rightarrow q$ is neither. 
\end{proof}

\begin{problem}
Consider the following propositions over $\mathbb{Z}$:
\begin{enumerate}
\item[] $p(n):	n$ is a square.
\item[] $q(n):	n$ is even.
\item[] $r(n):	n$ is divisible by $4$.
\item[] $s(n):	n<0$.
\end{enumerate}
\begin{enumerate}
\item Express ``Every integer that is an even square is divisible by 4" symbolically.
\item Write $\neg(\exists_{n\in \mathbb{Z}}(p\land s))$ in English.
\item Rewrite $2$ using the $\forall$ symbol.
\end{enumerate}
\end{problem}
\begin{proof}[Solution]
\
\begin{enumerate}
\item $\forall_{n\in \mathbb{Z}}(p\land q \rightarrow r)$.
\item There is no integer $n$ such that $n$ is a square and $n<0$.
\item $\forall_{n\in \mathbb{R}}(\neg(p\land s))$.
\end{enumerate}
\end{proof}
%
\subsection{Exam II}
%
\begin{problem}
Let $n$ be an integer. Prove the $n$ is odd if and only if $n^2+2n+5$ is even.
\end{problem}
\begin{proof}[Solution]
Suppose $n$ is odd. Then there is a $k\in \mathbb{Z}$ such that $n = 2k+1$. So $n^2+2n+5 = (2k+1)^2+2(2k+1)+5 = 4k^2+4k+1+4k+2+5 = 4k^2+8k+8 = 2(2k^2+4k+4)$, which is even. Thus if $n$ is odd, then $n^2+2n+5$ is even. Let $n^2+2n+5$ be even and suppose $n$ is even. Then $n = 2k$. But then $n^2+2n+5 = 4k^2+4k+5 = 2(2k^2+2k+2)+1$. But this is an odd number, a contradiction. Therefore, $n$ is not an even number.
\end{proof}

\begin{problem}
Prove that for all $n\in \mathbb{N}$, $\sum_{k=1}^{n} 3k(k+1) = n(n+1)(n+1)$.
\end{problem}
\begin{proof}[Solution]
We prove by induction. The base case is $n=1$, and we have $3\cdot 1(1+1) = 6 = 1\cdot(1+1)\cdot(1+2) = 1\cdot 2 \cdot 3 = 6$, which is true. Suppose it is true for some $n\in \mathbb{N}$. Then $\sum_{k=1}^{n+1} 3k(k+1) = 3(n+1)(n+2) +\sum_{k=1}^{n} 3k(k+1) = 3(n+1)(n+2) +n(n+1)(n+2) = (n+3)(n+1)(n+2) = (n+1)\big((n+1)+1\big)\big((n+1)+2\big)$.
\end{proof}

\begin{problem}
\label{problem:discrete_structures_exam_2_p_and_q_iff_not_p_or_not_q}
Consider $(p\land q) \leftrightarrow (\neg q \lor \neg p)$.
\begin{enumerate}
\item Write down the truth table for $(p\land q)\leftrightarrow (\neg q \lor \neg p)$ and $\neg(q\lor p)$.
\item Determine whether this proposition is a tautology, contradiction, or neither.
\item Is $\neg(q\lor p)$ equivalent to $\neg q \lor \neg p$.
\end{enumerate}
\end{problem}
\begin{proof}[Solution]
%
\begin{table}[H]
    \centering
    \begin{tabular}{c c c c c c c c c} 
         \hline
         $p$ & $q$ & $p\land q$ & $p \lor q$ & $\neg q$ & $\neg p$ & $\neg q\lor \neg p$ & $\neg(q\lor p)$ & $(p\land q)\leftrightarrow(\neg q \lor \neg p)$ \\ [0.5ex] 
         \hline
         0 & 0 & 0 & 0 & 1 & 1  &   1 & 1  & 0	\\ 
         0 & 1 & 0 & 1 & 0 & 1  &   1 & 0  & 0	\\
         1 & 0 & 0 & 1 & 1 & 0  &   1 & 0  & 0	\\
         1 & 1 & 1 & 1 & 0 & 0  &	0 & 0  & 0	\\
         \hline
    \end{tabular}
    \caption{Truth Table for Problem \ref{problem:discrete_structures_exam_2_p_and_q_iff_not_p_or_not_q}}
    \label{tab:discrete_structures_Exam_II_Problem_3}
\end{table}
%
The Proposition is a tautology. Also, $\neg q \lor \neg p$ is not equivalent to $\neg(q \lor p)$.
\end{proof}

\begin{problem}
Consider the following propositions of $\mathbb{N}$.
\begin{enumerate}
\item[] $p(n): 0\leq n \leq 5$
\item[] $q(n): n$ is a positive prime number and $n<7$.
\item[] $r(n):  n^2 \leq 9$
\item[] $s(m,n): 2m+n = 4$
\end{enumerate}
\begin{enumerate}
\item Translate the following statement into English: $\forall_{n\in \mathbb{Z}}\exists_{m\in \mathbb{Z}}(s(m,n))$
\item Translate the following symbolically: There exists an integer $n$ so that $n^2>9$ or $0\leq n \leq 5$.
\item The Down the truth sets for $p(n),q(n),r(n)$.
\item Does one of the propositions imply another?
\item Is the statement in part $1$ true?
\end{enumerate}
\end{problem}
\begin{proof}[Solution]
\
\begin{enumerate}
\item For all integers $n$ there exists an integer $m$ so that $2m+n = 4$. 
\item $\exists_{n\in \mathbb{Z}}(\neg r(n)\lor p(n))$
\item
\begin{enumerate}
\item $T_{p} = \{0,1,2,3,4,5\}$
\item $T_{q} = \{2,3,5\}$
\item $\{,3,2,1,0,1,2,3\}$
\end{enumerate}
\item Yes, $q(n)\Rightarrow P(n)$ because $T_{q}\subset T_{p}$.
\item False. If $n = 1$ then $2m+1$ is odd for all $n\in \mathbb{Z}$, yet $4$ is even.
\end{enumerate}
\end{proof}

\begin{problem}
Let $n$ be an integer.
\begin{enumerate}
\item Write the contrapositive of : If $n^2 \geq 1$, then $n\geq 1$. Is this true?
\item Write the converse of: If $n^2 \geq 1$, then $n\geq 1$. Is this true?
\end{enumerate}
\end{problem}
\begin{proof}[Solution]
\
\begin{enumerate}
\item If $n < 1$, then $n^2 <1$. This is false. If $n = -5$, then $n<1$, yet $n^2 = 25 >1$.
\item If $n\geq 1$, then $n^2 \geq 1$. This is true, for is $n\geq 1$, then squaring both sides we get $n^2 \geq 1^2 = 1$. 
\end{enumerate}
\end{proof}

\begin{problem}
Let $U$ be a set with $n$ elements. How many different propositions of $U$ can you list without listing two that are equivalent?
\end{problem}
\begin{proof}[Solution]
Two propositions $p$ and $q$ are equivalent if $T_p = T_q$. There are $|\mathcal{P}(U)| = 2^n$ possible truth sets. So there are $2^n$ non-equivalent propositions over $U$.
\end{proof}
%
\subsection{Practice Exam III}
%
\begin{problem}
Let $A= \{1,2,3\}$, $B = \{3,4,5,6\}$, and $C = \{1,2,4,6\}$. Let the universe set be $U = \{1,2,3,4,5,6,7\}$. 
\begin{enumerate}
\item Find all minsets generated by $A,B,C$.
\item Show that the nonempty minsets form a partition of $U$.
\item How many different sets in $\mathcal{P}(U)$ can be generated by $A,B,C$ via any combination of union, intersection, and complement?
\item Express $\{1,2,3,5,7\}$ and $\{5,6,7\}$ in minset normal form, if possible. If not, show why it can't be done.
\end{enumerate}
\end{problem}
\begin{proof}[Solution]
%
\begin{table}[H]
    \centering
    \begin{tabular}{c c c c c}
        \hline
        $A$ & $B$ & $C$ & &  \\ [0.5ex] 
        \hline
        $0$ & $0$ & $0$ & $A^c\cap B^c\cap C^c$ & $\{7\}$\\
        $0$ & $0$ & $1$ & $A^c\cap B^c\cap C$ & $\emptyset$\\
        $0$ & $1$ & $0$ & $A^c\cap B\cap C^c$ & $\{5\}$\\
        $0$ & $1$ & $1$ & $A^c\cap B \cap C$ & $\{4,6\}$\\
        $1$ & $0$ & $0$ & $A\cap B^c\cap C^c$ & $\emptyset$\\
        $1$ & $0$ & $1$ & $A\cap B^c\cap C$ & $\{1,2\}$\\
        $1$ & $1$ & $0$ & $A\cap B\cap C^c$ & $\{3\}$\\
        $1$ & $1$ & $1$ & $A\cap B \cap C$ & $\emptyset$\\
        \hline
    \end{tabular}
    \caption{The Minsets of $A,B,$ and $C$.}
    \label{tab:discrete_structures_practice_exam_III_Problem_1}
\end{table}
%
The non-empty sets form a partition of $U$ for their union is $U$, and there are no overaps. There are $5$ non-empty minsets, so $2^{5} = 32$ subsets of $\mathcal{P}(U)$ can be generated from the minsets. $\{1,2,3,5,7\} = (A^c\cap B^c \cap C^c)\cup (A^c\cap B\cap C^c)\cup (A\cap B^c\cap C)\cup (A\cap B\cap C^c)$. $\{5,6,7\}$ cannot be expressed as a union of minsets, for $6$ lies in the minset $\{4,6\}$, and thus $4$ would need to be included, but it is not.
\end{proof}

\begin{problem}
Let $A$ and $B$ be subsets of $U$. 
\begin{enumerate}
\item List all minsets generated by $A, B$.
\item Express the sets $A$ and $A^c\cup B$ in minset normal form.
\end{enumerate}
\end{problem}
\begin{proof}[Solution]
%
\begin{table}[H]
    \centering
    \begin{tabular}{c c c}
        \hline
        $A$ & $B$ & \\ [0.5ex]
        \hline
        $0$ & $0$ & $A^c \cap B^c$\\
        $0$ & $1$ & $A^c \cap B$\\
        $1$ & $0$ & $A\cap B^c$\\
        $1$ & $1$ & $A\cap B$\\
        \hline
    \end{tabular}
    \caption{Minsets of $A$ and $B$.}
    \label{tab:discrete_structures_exam_II_problem_blah}
\end{table}
%
\begin{align*}
    A &= A\cap U = A\cap(B\cup B^c) = (A\cap B)\cup(A\cap B^c)\\
    A^c \cup B &= (A^c \cap U)\cup (B\cap U)\\
    &= \big(A^c\cap (B\cup B^c)\big)\cup\big(B\cap (A\cup A^c)\big)\\
    &= (A^c\cap B)\cup(A^c\cap B^c)\cup (B\cap A)
\end{align*}
\end{proof}

\begin{problem}
Let:
\begin{equation*}
    A = \begin{pmatrix} 1 & 4 \\ 0 & 5 \\ 2 & -3 \end{pmatrix} \quad B = \begin{pmatrix} 2 & 10 \\ -1 & 0 \\ 5 & -1 \end{pmatrix} \quad C = \begin{pmatrix} 1 & 3 \\ -2 & 4\end{pmatrix} \quad
    D = \begin{pmatrix} 3 & 0 \\ 0 & 2 \end{pmatrix} \quad F = \begin{pmatrix} 1 & 3 \\ -2 & -6 \end{pmatrix}
\end{equation*}
Compute the following, if possible.
\begin{enumerate}
\begin{multicols}{5}
\item $AB$
\item $3A-2B$
\item $7A + 2C$
\item $A^2$
\item $C^2$
\item $D^4$
\item $AC+BC$
\item $CA$
\item $C^{-1}$
\item $C^{-2}$
\item $F^{-1}$.
\item $XC=A$
\end{multicols}
\end{enumerate}
\end{problem}
\begin{proof}
\
\begin{enumerate}
\item $AB$ is not possible because $A$ and $B$ are both $3\times 2$ matrices. The number of columns of $A$ must be the same as the number of rows of $B$.
\item $3A-2B = \begin{pmatrix}-1 & -8\\ 2 & 15\\ -4 & -7 \end{pmatrix}$
\item $7A+2C$ is not possible because $A$ and $C$ are of different dimensions.
\item $A^2$ is not possible because $A$ is not a square matrix.
\item $C^2 = \begin{pmatrix} -5 & 15 \\ -10 & 10\end{pmatrix}$.
\item $D^4 = \begin{pmatrix} 81 & 0 \\ 0 & 16\end{pmatrix}$. 
\item $AC+BC = (A+B)C = \begin{pmatrix} 3 & 14 \\ -1 & 5 \\ 7 & -4 \end{pmatrix} \begin{pmatrix} 1 & 3 \\ -2 & 4 \end{pmatrix} = \begin{pmatrix} -25 & 65 \\ -11 & 17 \\ 15 & 5 \end{pmatrix}$
\item $C$ is a $2\times 2$ and $A$ is a $3\times 2$, so multiplication is undefined.
\item $\det(C) = 10$, so $C^{-1} = \frac{1}{10}\begin{pmatrix} 4 & -3 \\ 2 & 1 \end{pmatrix}$
\item $C^{-2} = (C^2)^{-1} = \frac{1}{100}\begin{pmatrix}10 & -15 \\ 10 & -5\end{pmatrix}$
\item $\det(F) = 0$, so $F^{-1}$ does not exist. 
\item $X = AC^{-1} = \frac{1}{10}\begin{pmatrix}12 & 1\\10 & 5\\ 2 & -9\end{pmatrix}$.
\end{enumerate}
\end{proof}

\begin{problem}
Convert:
\begin{align}
\nonumber 4x - y &= 10\\
\nonumber 3x - 2y &= 3
\end{align}
Into $AX = B$, and then solve.
\end{problem}
\begin{proof}[Solution]
We have $\begin{bmatrix} 4 & =1 \\ 3 & -2 \end{bmatrix} \begin{bmatrix} x \\ y \end{bmatrix} = \begin{bmatrix} 10 \\ 3 \end{bmatrix}$. The solution is $X = A^{-1}B$. The inverse of $\begin{bmatrix} 4 & -1 \\ 3 & -2 \end{bmatrix}$ is $\frac{-1}{5}\begin{bmatrix} -2 & -3 \\ 1 & 4\end{bmatrix}$. So, $X = \frac{1}{5} \begin{bmatrix} 17 \\ 18 \end{bmatrix}$
\end{proof}

\begin{problem}
Prove or disprove the following for $n\times n$ matrices $A, B$.
\begin{enumerate}
\item $AB = BA$
\item $(A+B)(A-B) = A^2 - B^2$
\item If $A^2 = AB$, and $\det(A) = 4$, then $A=B$.
\item If $AB = 0$, then $A=0$ or $B=0$
\end{enumerate}
\end{problem}
\begin{proof}[Solution]
\
\begin{enumerate}
\item False, for $\begin{bmatrix} 0 & 1 \\ 0 & 1\end{bmatrix} \begin{bmatrix} 1 & 1 \\ 0 & 0 \end{bmatrix} = \begin{bmatrix} 0 & 0 \\ 0 & 0\end{bmatrix}$, but $\begin{bmatrix} 1 & 1 \\ 0 & 0 \end{bmatrix}\begin{bmatrix} 0 & 1 \\ 0 & 1\end{bmatrix}  = \begin{bmatrix} 0 & 2 \\ 0 & 0 \end{bmatrix}$.
\item False, for $(A+B)(A-B) = A(A-B) + B(A-B) = A^2-AB+BA - B^2 =d (A^2-B^2) + (BA-AB)$. Since $AB$ may not necessarily be equal to $BA$, $BA-AB$ may not zero.
\item True. If $\det(A) = 4$, then $A^{-1}$ exists. Then $A^2 = AB$, and thus $A^{-1}A^2  = A^{-1}AB \Leftrightarrow A = B$.
\item False. For $\begin{bmatrix} 1 & 0 \\ 0 & 0 \end{bmatrix} \begin{bmatrix} 0 & 0 \\ 0 & 1 \end{bmatrix} = 0$.
\end{enumerate}
\end{proof}
%
\subsection{Exam III}
%
\begin{problem}
Prove that for sets $A,B,C$, $(A\cap B)\times C \subset B\times C$.
\end{problem}
\begin{proof}[Solution]
For let $(x,y) \in (A\cap B)\times C$. Then $x\in A\cap B$ and $y\in C$. But if $x\in A\cap B$, then $x\in B$. But if $x\in B$ and $y\in C$, then $(x,y) \in B\times C$. Therefore, $(A\cap B)\times C \subset B\times C$. 
\end{proof}

\begin{problem}
For set $A,B\subset U$, prove that $A\cup (A\cap B)^c = U$, and find the dual of this.
\end{problem}
\begin{proof}[Solution]
For Let $x\in U$. If $x\in A$, then $x\in A\cup (A\cap B)^c$. Suppose not. Then $x\in A^c$. But if $x\in A^c$, then $x\notin A\cap B$. But if $x\notin A\cap B$, then $x\in (A\cap B)^c$. But then $x\in A\cup (A\cap B)^c$. Therefore $U\subset A\cup (A\cap B)^c$. But $A\cup (A\cap B)^c \subset U$, as $U$ is the universe set. Therefore $A\cup (A\cap B)^c = U$. The dual is $A\cap (A\cup B)^c = \emptyset$
\end{proof}

\begin{problem}
Convert:

\begin{align}
\nonumber 4x-6y &= 5 \\
\nonumber 3x-7y &= -7
\end{align}
Into the form $AX = B$. Find $A^{-1}$, and then solve for $X$.
\end{problem}
\begin{proof}[Solution]
We have $\begin{bmatrix} 4 & -6 \\ 3 & -7 \end{bmatrix} \begin{bmatrix} x \\ y \end{bmatrix} = \begin{bmatrix} 5 \\ -7 \end{bmatrix}$. The inverse of $A$ is $\frac{-1}{10}\begin{bmatrix} -7 & 6 \\ -3 & 4\end{bmatrix}$. So, $X = A^{-1}B = \frac{1}{10} \begin{bmatrix}83 \\ 47 \end{bmatrix}$.
\end{proof}

\begin{problem}
Let $A = \begin{pmatrix} 1 & 0 & -2 \\ 5 & 3 & 0 \end{pmatrix}, B = \begin{pmatrix} 0 & 2 \\ 4 & -10 \\ 8 & -6 \end{pmatrix}, C = \begin{pmatrix} 7 & -1 \\ -2 & 5 \\ -4 & 3 \end{pmatrix}, E = \begin{pmatrix} 1 & -3 \\ 2 & 5 \end{pmatrix}$. Compute the following, if possible:
\begin{enumerate}
\begin{multicols}{6}
\item $3A+5C$
\item $2B-3C$
\item $CE$
\item $EC$
\item $E^2$
\item $BA+2CA$
\end{multicols}
\end{enumerate}
\end{problem}
\begin{proof}[Solution]
\
\begin{enumerate}
\item $A$ and $C$ do not have the same dimensions, so this can't be done.
\item $2B - 3C = \begin{pmatrix} -21 & 7 \\ 14 & -35 \\ 28 & -21 \end{pmatrix}$
\item $CE = \begin{pmatrix} 5 & -26 \\ 8 & 31 \\ 2 & 27 \end{pmatrix}$
\item $EC$ can't be done as $E$ is a $2\times 2$ and $C$ is a $3\times 2$.
\item $E^2 = \begin{pmatrix} -5 & -18 \\ 12 & 19 \end{pmatrix}$
\item $BA + 2CA = \begin{pmatrix} 13 & 0 & -28 \\ 0 & 0 & 0 \\ 0 & 0 & 0 \end{pmatrix}$
\end{enumerate}
\end{proof}

\begin{problem}
Let $A = \{1,3,5\}$, $B = \{2,3,4,5\}$, and $U = \{1,2,3,4,5,6\}$.
\begin{enumerate}
\item Find all minsets generated by $A$ and $B$.
\item How many different sets in the power set of $U$ can be generated by $A,B$ by any combination of union, intersection, and complement?
\item Express $\{2,4,6\}$ in minset normal form.
\item Find the maxsets generated by $A$ and $B$.
\item Express $\{1,3,5\}$ in maxset normal form.
\end{enumerate}
\end{problem}
\begin{proof}[Solution]
\
\begin{enumerate}
\item $A^c\cap B^c = \{6\}$, $A\cap B^c = \{1\}$, $A^c \cap B = \{2,4\}$, $A\cap B = \{3,5\}$.
\item There are $4$ non-empty minsets, so $2^4 = 16$. 
\item $\{2,4,6\} = (A^c \cap B)\cup (A^c \cap B^c)$.
\item $A\cup B = \{1,2,3,4,5\}, A^c \cup B = \{2,3,4,5,6\}, A\cup B^c = \{1,3,5,6\}, A^c \cup B^c = \{1,2,4,6\}$.
\item $\{1,3,5\} = (A\cup B) \cap (A\cup B^c)$.
\end{enumerate}
\end{proof}
%
\subsection{Final Exam}
%
\begin{problem}
Let $A = \{1,2\}$, $B = \{2,3,4,6\}$, $C = \{4,6,7\}$, $U = \{1,2,3,4,5,6,7,8\}$. Compute the following:
\begin{enumerate}
\begin{multicols}{4}
\item $A^c \cap B$
\item $A\cup C$
\item $B\oplus C$
\item $B\setminus C$
\item $A\times C$
\item $A^3$
\item $\mathcal{P}(C)$
\end{multicols}
\end{enumerate}
\end{problem}
\begin{proof}[Solution]
\
\begin{enumerate}
\begin{multicols}{2}
\item $A^c \cap B = \{3,4,6\}$.
\item $A \cup C = \{1,2,4,6,7\}$
\item $B\oplus C = \{2,3,7\}$
\item $B\setminus C = \{2,3\}$
\end{multicols}
\item $A\times C = \{(1,4),(1,6),(1,7),(2,4),(2,6),(2,7)\}$
\item $A^3 = \{(1,1,1),(1,1,2),(1,2,1),(1,2,2),(2,1,1),(2,1,2),(2,2,1),(2,2,2)\}$.
\item $\mathcal{P}(C) = \{\emptyset, \{4\},\{6\},\{7\},\{4,6\},\{4,7\},\{6,7\},\{4,6,7\}\}$.
\end{enumerate}
\end{proof}

\begin{problem}
Prove the following is false: If $A\cap B = A \cap C$, then $B = C$.
\end{problem}
\begin{proof}
For let $A = \{1\}$, $B = \{1,2\}$, and $C = \mathbb{R}$. Then $A\cap B = \{1\}$, $A\cap C = \{1\}$, but $B \ne C$.
\end{proof}

\begin{problem}
Ten students are competing for a scholarship.
\begin{enumerate}
\item If there are three scholarships worth $\$2000$, how many ways can they be distributed?
\item If there are two scholarships worth $\$5000$ and three worth $\$2000$, how many ways can they be distributed?
\item Suppose that the group of ten students consists of six freshmen and four sophomores. In how many different ways can four equal scholarships be distributed if at least two of the scholarships should be awarded to freshmen?
\item Suppose the group of ten students consists of six freshmen and four sophomores. In how many different ways can two scholarships of $\$5000$ and two scholarships of $\$2000$ be distributed if at least three of the scholarships will be awarded to freshmen?
\end{enumerate}
\end{problem}
\begin{proof}[Solution]
\
\begin{enumerate}
\item $\binom{10}{3} = \frac{10!}{3!(10-3)!} = 120$.
\item $\binom{10}{2}\binom{8}{3} = 2520$.
\item If $2$ scholarships are awarded to freshmen, we have $\binom{6}{2}\binom{4}{2} = 90$. If $3$ scholarships are awards to freshmen, we have $\binom{6}{3}\binom{4}{1} = 80$. If $4$ scholarships are awarded to freshmen, we have $\binom{6}{4}\binom{4}{0} = 15$. Adding them together, we get $185$.
\item Hi
\end{enumerate}
\end{proof}
\end{document}