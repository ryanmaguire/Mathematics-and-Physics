\documentclass[../main.tex]{subfiles}
\begin{document}
\section{Quizzes}
\subsection{Quiz I}
%
\begin{problem}
Let $A = \{1,2,3,7,8\}$, $B = \{1,3,5\}$, $C = \{2,4,8\}$, and let $U = \{1,2,3,4,5,6,7,8,9,10\}$. Evaluate the following:
\begin{enumerate}
\begin{multicols}{6}
\item $A\cup C$
\item $A\cap B$
\item $A \oplus C$
\item $A^c$
\item $A\setminus B$
\item $B\times C$
\end{multicols}
\end{enumerate}
\end{problem}
\begin{proof}[Solution]
\
\begin{enumerate}
\begin{multicols}{2}
\item $A\cup C = \{1,2,3,4,7,8\}$.
\item $A\cap B = \{1,3\}$.
\item $A\oplus C = \{1,3,4,7\}$.
\item $A^c = \{4,5,6,9,10\}$.
\end{multicols}
\item $A\setminus B = \{2,7,8\}$.
\item $B \times C = \{(1,2),(1,4),(1,8),(3,2),(2,4),(3,8),(5,2),(5,4),(5,8)\}$.
\end{enumerate}
\end{proof}

\begin{problem}
Let $S = \{1,3,5\}$. Find $\mathcal{P}(S)$.
\end{problem}
\begin{proof}[Solution]
$\mathcal{P}(S) = \{\emptyset,\{1\},\{3\},\{5\},\{1,3\},\{1,5\},\{3,5\},\{1,3,5\}\}$
\end{proof}

\begin{problem}
Let $S = \{7k-3: k \in \mathbb{N}, k < 5\}$. List all of the elements of $S$.
\end{problem}
\begin{proof}[Solution]
$S = \{-3,4,11,18,25\}$
\end{proof}

\begin{problem}
Express $49$ in binary.
\end{problem}
\begin{proof}[Solution]
\
\begin{enumerate}
\item[] $49 = 2\cdot 24+1$
\item[] $24 = 2\cdot 12+0$
\item[] $12 = 2\cdot 6+0$
\item[] $6 = 2\cdot 3 + 0$
\item $3 = 2\cdot 1 + 1$
\item $1 = 2\cdot 0 + 1$
\end{enumerate}
So, $49 = 110001_{2}$
\end{proof} 
%
\subsection{Quiz II}
%
\begin{problem}
Calculate $\sum_{k=-1}^{3} (2^k+1)$.
\end{problem}
\begin{proof}[Solution]
$\sum_{k=-1}^{3}(2^k+1) = (2^{-1}+1) + (2^0+1) + (2^1+1)+(2^2+1) + (2^3+1) = 20 +\frac{1}{2} = \frac{41}{2}$.
\end{proof}

\begin{problem}
Three men and three women are to be seated in a row.
\begin{enumerate}
\item How many different ways can the six people be seated?
\item How many different ways can the six people be seated if it is required that the genders alternate.
\end{enumerate}
\end{problem}
\begin{proof}[Solution]
\
\begin{enumerate}
\item $6! = 6\cdot 5 \cdot 5 \cdot 4 \cdot 3 \cdot 2 \cdot 1 = 720$
\item It is $MWMWMW$, so $3\cdot 3 \cdot 2 \cdot 2 \cdot 1 \cdot 1 = 72$. Or, there are $6$ ways to seat the first person, $3$ ways to seat the second person, $2$ ways to seat the third person, $2$ ways to seat the fourth person, and $1$ way to seat the last two. So, $6\cdot 3 \cdot 2 \cdot 2 \cdot 1 \cdot 1 = 72$.
\end{enumerate}
\end{proof}

\begin{problem}
Calculate $P(7;3)$
\end{problem}
\begin{proof}[Solution]
$P(7;3) = \frac{7!}{(7-3)!} = 7\cdot 6 \cdot 5 = 210$.
\end{proof}

\begin{problem}
Let $A$ be a set such that $|A| = n$.
\begin{enumerate}
\item Calculate $|A^4|$.
\item Calculate $|\{(a,b,c,d)\in A:\ \textrm{Each Term is Different}\}|$.
\end{enumerate}
\end{problem}
\begin{proof}[Solution]
\
\begin{enumerate}
\item $|A^4| = |A\times A \times A \times A| = n^4$.
\item $n \cdot (n-1)\cdot (n-2)\cdot (n-3) = P(n;4)$
\end{enumerate}
\end{proof}
%
\subsection{Quiz III}
%
\begin{problem}
Let $p,q,r$ be the following propositions:
\begin{enumerate}
\begin{multicols}{3}
\item[] $p(x):	x = 1$
\item[] $p(x):	x = -1$
\item[] $r(x):	x^2 = 1$
\end{multicols}
\end{enumerate}
\begin{enumerate}
\item Express ``If $x^2 = 1$, then $x=1$ and $x=-1$," in symbolic form.
\item Write the converse of this in English, and symbolically.
\item Express $\neg p \land \neg r$ in English.
\item Express $r\leftrightarrow (q\lor p)$ in English.
\end{enumerate}
\end{problem}
\begin{proof}[Solution]
\
\begin{enumerate}
\item $r\rightarrow (p\land q)$
\item $(p\land q) \rightarrow r$. If $x=1$ and $x=-1$, then $x^2 = 1$.
\item $x\ne = 1$ and $x^2 \ne 1$>
\item $x^2 = 1$ if and only if $x=1$ or $x=-1$.
\end{enumerate}
\end{proof}

\begin{problem}
\label{discrete_structures_quiz_3_problem_2}
Make a truth table for $(p\lor \neg q)\land r$.
\end{problem}
\begin{proof}[Solution]
\
\begin{table}[H]
    \centering
    \begin{tabular}{c c c c c c}
        \hline
        $p$ & $q$ & $r$ & $\neg q$ & $p\lor \neg q$ & $(p\lor \neg q)\land r$ \\ [0.5ex]
        \hline
        $0$ & $0$ & $0$ & $1$ & $1$ & $0$\\
        $0$ & $0$ & $1$ & $1$ & $1$ & $1$\\
        $0$ & $1$ & $0$ & $0$ & $0$ & $0$\\
        $0$ & $1$ & $1$ & $0$ & $0$ & $0$\\
        $1$ & $0$ & $0$ & $1$ & $1$ & $0$\\
        $1$ & $0$ & $1$ & $1$ & $1$ & $1$\\
        $1$ & $1$ & $0$ & $0$ & $1$ & $0$\\
        $1$ & $1$ & $1$ & $0$ & $1$ & $1$\\
        \hline
    \end{tabular}
    \caption{Truth Table for Problem \ref{discrete_structures_quiz_3_problem_2}}
    \label{tab:discrete_structures_final_exam_problem}
\end{table}
\end{proof}
%
\subsection{Quiz IV}
%
\begin{problem}
Prove directly that $a\rightarrow b, \neg c\rightarrow \neg b, \neg c \Rightarrow \neg a$.
\end{problem}
\begin{proof}[Solution]
For if $a\rightarrow b$, then $\neg b \rightarrow \neg a$. But $\neg c \rightarrow \neg b$. But if $\neg c \rightarrow \neg b$ and $\neg b \rightarrow \neg a$, then $\neg c \rightarrow \neg a$. Thus $a\rightarrow b, \neg c \rightarrow \neg b, \neg c \Rightarrow \neg a$.
\end{proof}

\begin{problem}
Prove indirectly that $a\rightarrow b, \neg c \rightarrow \neg b, \neg c \Rightarrow \neg a$.
\end{problem}
\begin{proof}[Solution]
For if $\neg c \rightarrow \neg b$, then $b\rightarrow c$. But if $a\rightarrow b$ and $b\rightarrow c$, then $a\rightarrow c$. Therefore $a\rightarrow c$. But if $a\rightarrow c$, then $\neg c \rightarrow \neg a$. Therefore, $a\rightarrow b, \neg c \rightarrow \neg b, \neg c \Rightarrow \neg a$.
\end{proof}

\begin{problem}
Let $U = \{1,2,3,4,5,6,7,8,9,10\}$. Consider the following propositions over $U$:
\begin{enumerate}
\begin{multicols}{3}
\item[] $p(n):	n$ is prime.
\item[] $q(n):	n$ is odd.
\item[] $r(n):	n\leq 7$.
\end{multicols}
\end{enumerate}
\begin{enumerate}
\item Find the truth sets for $p,q,r$.
\item Which of these propositions implies one of the others?
\item Find the truth set of $q\land r$.
\end{enumerate}
\end{problem}

\begin{proof}[Solution]
\
\begin{enumerate}
\item 
\begin{enumerate}
\begin{multicols}{3}
\item[] $T_{p} = \{2,3,5,7\}$
\item[] $T_{q} = \{1,3,5,7,9\}$
\item[] $T_{r} = \{1,2,3,4,5,6,7\}$
\end{multicols}
\end{enumerate}
\begin{multicols}{2}
\item $p\Rightarrow r$, since $T_{p}\subset T_{r}$.
\item $T_{q\land r} = \{1,3,5,7\}$.
\end{multicols}
\end{enumerate}
\end{proof}
%
\subsection{Quiz V}
%
\begin{problem}
Let $p,q,r$ be the following propositions over $\mathbb{N}$:
\begin{enumerate}
\begin{multicols}{4}
\item[] $p(n):	n^2+3n = 1$
\item[] $q(n):	n$ is prime.
\item[] $r(n):	n$ is even.
\item[] $s(m,n):	m$ divides $n$
\end{multicols}
\end{enumerate}
\begin{enumerate}
\item Express ``There exists a solution for $n^2+3n = 10$ that is prime," symbolically.
\item Express $\forall_{n\in \mathbb{Z}}(q\rightarrow \neg r)$ in English.
\item Express $\forall_{n\in \mathbb{N}}\exists_{m\in \mathbb{N}}(s(m,n))$ in English.
\end{enumerate}
\end{problem}
\begin{proof}[Solution]
\
\begin{enumerate}
\item $\exists_{n\in T_{q\land p}}$.
\item For every integer $n$, if $n$ is prime, then $n$ is not an even number.
\item For every positive integer $n$, there exists a positive integer $m$ such that $m$ divides $n$. 
\end{enumerate}
\end{proof}

\begin{problem}
Prove $\sum_{k=1}^{n} 10k = 5n(n+1)$ using mathematical induction. 
\end{problem}
\begin{proof}[Solution]
The base case is $n=1$, so $10 = 5\cdot 1(1+1) = 5\cdot 2 = 10$, which is true. Suppose this is true for some $n\in \mathbb{N}$. Then $\sum_{k=1}^{n+1} 10k = 10(n+1) + \sum_{k=1}^{n} 10k$. By hypothesis, $\sum_{k=1}^{n} 10k = 5n(n+1)$, so $\sum_{k=1}^{n+1}10k = 10(n+1)+5n(n+1) = (n+1)(10+5n) = 5(n+1)(n+2) = 4(n+1)((n+1)+1)$. This proves the induction step.
\end{proof}
\end{document}