\documentclass[../main.tex]{subfiles}
\begin{document}
\section{North Shore Placement Exam}
%
\subsection{Order of Operations}
%
First we start with the fundamental properties of addition and multiplication:
%
\begin{properties}
\label{property:North_Shore_Arithmetic_Properties}
\
\begin{enumerate}[itemsep=0pt]
    \item \label{property:North_Shore_Arithmetic_Properties_Com_Add} $a+b = b+a$ \hfill [Commutativity of Addition]
    \item \label{property:north_shore_arithmetic_properties_assoc_add} $a+(b+c) = (a+b)+c$ \hfill [Associativity of Addition]
    \item \label{property:north_shore_arithmetic_properties_comm_mult} $a\cdot b = b \cdot a$ \hfill [Commutativity of Multiplication]
    \item \label{property:north_shore_arithmetic_properties_assoc_mult} $a\cdot (b\cdot c) = (a\cdot b)\cdot c$ \hfill [Associativity of Multiplication]
    \item \label{property:north_shore_arithmetic_properties_add_idenity} $a+0 = a$ \hfill [Identity Property of Addition]
    \item \label{property:north_shore_arithmetic_properties_mult_identity} $a\cdot 1 = a$ \hfill [Identity Property of Multiplication]
    \item \label{property:north_shore_arithmetic_properties_mult_inverse} If $a\ne 0$, then $\frac{a}{a} = 1$ \hfill [Inverse Property of Multiplication]
    \item \label{property:north_shore_arithmetic_properties_add_inverse} $a + (-a) = 0$ \hfill [Inverse Property of Addition]
    \item \label{property:north_shore_arithmetic_properties_distributive_property} $a\cdot (b+c) = a\cdot b + a\cdot c$ \hfill [Distributive Property of Multiplication over Addition]
\end{enumerate}
\end{properties}
%
The following are the fundamental properties of exponents.
%
\begin{properties}
\
\label{property:North_Shore_Exponent_Rules}
\begin{enumerate}[itemsep=0pt]
\item \label{property:north_shore_distributive_property_of_expo} $(x\cdot y)^n = x^n \cdot y^n$ \hfill [Distributive Property of Exponents]
\item \label{property:north_shore_inverse_proerty_of_expo} $x^{-n} = \frac{1}{x^n}$ \hfill [Inverse Property of Exponents]
\item \label{property:north_shore_power_property_of_expo} $(x^n)^m = x^{n\cdot m}$ \hfill [Power Property of Exponents]
\item \label{property:north_shore_product_property_of_expo} $x^{n} x^{m} = x^{n+m}$ \hfill [Multiplicative Property of Exponents]
\end{enumerate}
\end{properties}
%
The order of operations \gls{pemdas} tells one how to simplify expressions.
\begin{enumerate}[itemsep=0pt]
\label{North_Shore_PEMDAS}
    \item \textbf{P}arenthesis
    \item \textbf{E}xponents
    \item \textbf{M}ultiplication or \textbf{D}ivision, in the order they appear from left to right.
    \item \textbf{A}ddition or \textbf{S}ubtraction, in the order they appear from left to right.
\end{enumerate}
%
\clearpage
%
\begin{example}
\
\begin{enumerate}[itemsep=0pt]
\item $2\cdot 3 + 4 \underset{\textrm{\tiny Multiplication}}{=} 6+4 \underset{\textrm{\tiny Addition}}{=} \boxed{10}$
\item $3\cdot 3 + 2^4 \underset{\textrm{\tiny Exponents}}{=} 3\cdot 3 + 16 \underset{\textrm{\tiny Multiplication}}{=} 9+16 \underset{\textrm{\tiny Addition}}{=} \boxed{25}$
\item $(4+1)^2 - 17\cdot 3 \underset{\textrm{\tiny Parenthesis}}{=} 5^2 - 17\cdot 3 \underset{\textrm{\tiny Exponents}}{=} 25 - 17 \cdot 3 \underset{\textrm{\tiny Multiplication}}{=} 25 - 51 \underset{\textrm{\tiny Subtraction}}{=} \boxed{-26}$
\item $(1+1)^{(2+3)}\cdot 5 - 2\cdot 3 \underset{\textrm{\tiny Parenthesis}}{=} 2^5\cdot 5 - 2\cdot 3 \underset{\textrm{\tiny Exponents}}{=} 32\cdot 5 - 2\cdot 3 \underset{\textrm{\tiny Multiplication}}{=} 160 - 6\underset{\textrm{\tiny Subtraction}}{=} \boxed{154}$
\item $(2+2) - (16-11)^{(1+1)} \underset{\textrm{\tiny Parenthesis}}{=} 4 - 5^2 \underset{\textrm{\tiny Exponents}}{=} 4 - 25 \underset{\textrm{\tiny Subtraction}}{=} \boxed{-21}$
\end{enumerate}
\end{example}
%
\begin{remark}
\label{remark:North_Shore_Radicals_Def}
Many problems involve radicals, $\sqrt{\textrm{ }}$. This uses the following definitions:
\vspace{-1ex}
\begin{equation*}
\sqrt{x}=x^{\frac{1}{2}}\quad\quad\quad\quad\quad\quad \sqrt[n]{x}=x^{\frac{1}{n}}
\end{equation*}
\end{remark}
%
\vspace{-4ex}
%
\begin{example}
\
\begin{enumerate}
\begin{multicols}{4}
    \item $\sqrt{x} = x^{\frac{1}{2}}$
    \item $\sqrt[3]{x} = x^{\frac{1}{3}}$
    \item $\sqrt[5]{x} = x^{\frac{1}{5}}$
    \item $\sqrt[27]{x} = x^{\frac{1}{27}}$
    \end{multicols}
\end{enumerate}
\end{example}
%
\begin{remark}
\label{remark:North_Shore_Exponent_and_Radical_Def}
Radicals can be combined with exponents. These are defined by:
\begin{equation*}
    \sqrt[n]{x^m} = x^{\frac{m}{n}}
\end{equation*}
\end{remark}
%
\vspace{-4ex}
%
\begin{example}
\
\begin{enumerate}
    \begin{multicols}{4}
        \item $\sqrt{x^3} = x^{\frac{3}{2}}$
        \item $\sqrt[3]{x^2} = x^{\frac{2}{3}}$
        \item $\sqrt[15]{x^3} = x^{\frac{1}{5}}$
        \item $\sqrt[3]{x^3} = x$
    \end{multicols}
\end{enumerate}
\end{example}
%
\begin{theorem}
\label{thm:North_Shore_square_root_of_product_of_positive_reals}
If $a$ and $b$ are positive real numbers, then $\sqrt{a\cdot b} = \sqrt{a}\cdot \sqrt{b}$
\end{theorem}
%
\begin{example}
\
\begin{enumerate}
    \begin{multicols}{3}
    \item $\sqrt{6} = \sqrt{2}\cdot \sqrt{3}$
    \item $\sqrt{8} = 2\sqrt{2}$
    \item $\sqrt{1,\!000} = 10\sqrt{10}$
    \end{multicols}
\end{enumerate}
\end{example}
%
\begin{theorem}
\label{thm:North_Shore_square_root_of_negative_real}
If $x$ is a $\mathbf{positive}$ number, and $i$ is the imaginary unit $(i^2 = -1)$, then:
\begin{equation*}
    \sqrt{-x} = i\sqrt{x} 
\end{equation*}
\end{theorem}
%
\begin{example}
\
\begin{enumerate}
    \begin{multicols}{4}
    \item $\sqrt{-7} = i\sqrt{7}$
    \item $\sqrt{-9} = 3i$
    \item $\sqrt{-4} = 2i$
    \item $\sqrt{-2} = i\sqrt{2}$
    \end{multicols}
\end{enumerate}
\end{example}
%
\begin{theorem}
\label{thm:North_Shore_odd_nth_root_of_real}
If $n$ is an $\mathbf{odd}$ number, and $x$ is a $\mathbf{real}$ number, then
\begin{equation*}
    \sqrt[n]{-x} = -\sqrt[n]{x}
\end{equation*}
\end{theorem}
%
\begin{example}
\
\begin{enumerate}
    \begin{multicols}{4}
    \item $\sqrt[3]{-7} = -\sqrt[3]{7}$
    \item $\sqrt[5]{-9} = -\sqrt[5]{9}$
    \item $\sqrt[17]{-1} = -1$
    \item $\sqrt[13]{-2} = -\sqrt[13]{2}$
    \end{multicols}
\end{enumerate}
\end{example}
%
\begin{remark}
\label{remark:North_Shore_non_zero_raised_to_zero}
For all non-zero  real numbers, $x^0 = 1$. $0^0$ is left undefined.
\end{remark}
%
\clearpage
%
\begin{example}
\
\begin{enumerate}
    \begin{multicols}{4}
    \item $\pi^0 = 1$
    \item $(-5)^0 = 1$
    \item $(\frac{1}{2})^0 = 1$
    \item $0^0$ is undefined.
    \end{multicols}
\end{enumerate}
\end{example}
%
\begin{definition}
\label{definiiton:North_Shore_Absolute_Value_Def}
The absolute value of a real number $x$ is defined as $|x| = \begin{cases} x, & x \geq 0 \\ -x, & x<0\end{cases}$
\end{definition}
%
\begin{example}
\
\begin{enumerate}[itemsep=0pt]
\begin{multicols}{5}
\item[1.] $|3| = 3$
\item[6.] $|24| = 24$
\item[2.] $|-3| = 3$
\item[7.] $|-24| = 24$
\item[3.] $|\pi| = \pi$
\item[8.] $|\frac{1}{2}| = \frac{1}{2}$
\item[4.] $|-\pi| = \pi$
\item[9.] $|-\frac{1}{2}| = \frac{1}{2}$
\item[5.] $|0| = 0$
\item[10.] $|-1| = 1$
\end{multicols}
\end{enumerate}
\end{example}
%
\begin{remark}
\label{remark:North_Shore_Rational_Expressions}
Expressions like $\frac{a+b}{c+d}$ should be treated as equivalent to $(a+b)\olddiv (c+d)$.
\end{remark}
%
\subsubsection{Problems}
%
\begin{problem}
Simplify $3^2 + 5 - \sqrt{4} + 4^0$
\end{problem}
\begin{proof}[Solution]
\begin{flalign*}
3^{2} + 5 - \sqrt{4} + 4^{0} &= 3^{2}+5-4^{\frac{1}{2}} & \tag{Remark \ref{remark:North_Shore_Radicals_Def}}\\
&= 9+5-2+1 & \tag{Exponentiation}\\
&= 14-2+1 & \tag{Addition}\\
&= 12+1 & \tag{Subtraction}\\
&= \boxed{13} & \tag{Addition}
\end{flalign*}
\end{proof}
%
\begin{problem}
Simplify $(5+1)(4-2)-3$
\end{problem}
\begin{proof}[Solution]
\begin{flalign*}
    (5+1)(4-2)-3 &= 6\cdot 2 - 3 &\tag{Parenthesis}\\
    &= 12-3 & \tag{Multiplication}\\
    &= \boxed{9} & \tag{Subtraction}
\end{flalign*}
\end{proof}
%
\begin{problem}
Simplify $3\cdot 7^2$
\end{problem}
\begin{proof}[Solution]
\begin{flalign*}
    3\cdot 7^{2} &= 3\cdot 49 & \tag{Exponentiation}\\
    &= \boxed{147} & \tag{Multiplication}
\end{flalign*}
\end{proof}
%
\clearpage
%
\begin{problem}
Simplify $2(7+3)^2$
\end{problem}
\begin{proof}[Solution]
\begin{flalign*}
    2\cdot(7+3)^{2} &= 2\cdot 10^{2} & \tag{Parenthesis}\\
    &= 2\cdot 100 & \tag{Exponentiation}\\
    &= \boxed{200} & \tag{Multiplication}
\end{flalign*}
\end{proof}
%
\begin{problem}
Simplify $49 \olddiv 7 -2\cdot 2$.
\end{problem}
\begin{proof}[Solution]
\begin{flalign*}
    49 \olddiv 7 - 2 \cdot 2 &= 7-2\cdot 2 & \tag{Division}\\
    &= 7-4 & \tag{Multiplication}\\
    &= \boxed{3} & \tag{Subtraction}
\end{flalign*}
\end{proof}
%
\begin{problem}
Simplify $9\olddiv 3 \cdot 5 - 8\olddiv 2 + 27$.
\end{problem}
\begin{proof}[Solution]
\begin{flalign*}
    9 \olddiv 3 \cdot 5 - 8 \olddiv 2 + 27 &= 3\cdot 5 - 4+27 & \tag{Multiplication/Division from Left to Right}\\
    &= 15 - 4+27 & \tag{Multiplication}\\
    &= 11+27 & \tag{Subtraction}\\
    &= \boxed{38} & \tag{Addition}
\end{flalign*}
\end{proof}
%
\begin{problem}
Simplify $3+2(5)-|-7|$.
\end{problem}
\begin{proof}[Solution]
\begin{flalign*}
    2+2(5)- |-7| &= 3+2(5)-7 & \tag{Definition \ref{definiiton:North_Shore_Absolute_Value_Def}}\\
    &= 3+10+7 & \tag{Multiplication}\\
    &= 13+7 & \tag{Addition}\\
    &= \boxed{20} & \tag{Addition}
\end{flalign*}
\end{proof}
%
\begin{problem}
Simplify $\frac{5\cdot 5 - 4(4)}{2^2-1}$.
\end{problem}
\begin{proof}[Solution]
\begin{flalign*}
    \frac{5\cdot 5 - 4(4)}{2^{2}-1} &= \big(5\cdot 5 - 4(4)\big)\olddiv \big(2^{2}-1\big) & \tag{Remark \ref{remark:North_Shore_Rational_Expressions}}\\
    &= \big(5\cdot 5 - 4(4)\big)\olddiv\big(4-1\big) & \tag{Exponentiation Inside Parenthesis}\\
    &= \big(25-16\big)\olddiv \big(4-1\big) & \tag{Multiplication Inside Parenthesis}\\
    &= 9 \olddiv 3 & \tag{Parenthesis}\\
    &= \boxed{3} & \tag{Division}
\end{flalign*}
\end{proof}
%
\begin{problem}
Simplify $\frac{4^2 - 5^2}{(4-5)^2}$
\end{problem}
\begin{proof}[Solution]
\begin{flalign*}
    \frac{4^2-5^2}{(4-5)^2} &= (4^2-5^2)\olddiv (4-5)^2 & \tag{Remark \ref{remark:North_Shore_Rational_Expressions}}\\
    &= (16-25)\olddiv (-1)^2 & \tag{Parenthesis}\\
    &= -9 \olddiv 1 & \tag{Parenthesis}\\
    &= \boxed{-9}
\end{flalign*}
\end{proof}
%
\begin{problem}
Simplify $-5^2$
\end{problem}
\begin{proof}[Solution]
\begin{flalign*}
    -5^2 &= \boxed{-25} & \tag{Exponentiation}
\end{flalign*}
\end{proof}
%
\begin{problem}
Simplify $48 \olddiv 2(3+9)$
\end{problem}
\begin{proof}[Solution]
\begin{flalign*}
    48 \olddiv 2(3+9) &= 48 \olddiv 2(12) & \tag{Parenthesis}\\
    &= 24(12) & \tag{Division}\\
    &= \boxed{288} & \tag{Multiplication}
\end{flalign*}
\end{proof}
%
\begin{problem}
Simplify $6\cdot (3+2)^{2} + 14$
\end{problem}
\begin{proof}[Solution]
\begin{flalign*}
    6\cdot(3+2)^{2} + 14 &= 6\cdot 5^{2} + 14 & \tag{Parenthesis}\\
    &= 6\cdot 25 + 14 & \tag{Exponentiation}\\
    &= 150+14 & \tag{Multiplication}\\
    &= \boxed{164}
\end{flalign*}
\end{proof}
%
\begin{problem}
Simplify $(5+1)^{(3-2)}$
\end{problem}
\begin{proof}[Solution]
\begin{flalign*}
    (5+1)^{(3-2)} &= 6^{1} & \tag{Parenthesis}\\
    &= \boxed{6} & \tag{Exponentiation}
\end{flalign*}
\end{proof}
%
\clearpage
%
\subsection{Scientific Notation}
%
Non-zero real numbers can be written as $r\times 10^{n}$, where $1\leq |r|<10$ and $n$ is an integer.
%
\begin{example}
\
\begin{enumerate}[itemsep=0pt]
\begin{multicols}{3}
    \item[1.] $12,\! 345 = 1.2345\times 10^{4}$
    \item[4.] $10 = 1\times 10^{1}$
    \item[2.] $0.01 = 1\times 10^{-2}$
    \item[5.] $124 = 1.24 \times 10^{2}$
    \item[3.] $36.24 = 3.624\times 10^{1}$
    \item[6.] $0.000314 = 3.14\times 10^{-4}$
\end{multicols}
\end{enumerate}
\end{example}
%
\begin{remark}
Several constants in chemistry/physics are written in scientific notation.
\begin{enumerate}[itemsep=0pt]
    \item Planck's Constant: $h = 6.626\times 10^{-34} J\cdot s$
    \item Universal Gravitation Constant: $G = 6.67 \times 10^{-11} Nm^{2}kg^{-2}$
    \item Avogadro's Number: $N_{A} = 6.0221\times 10^{23}\ \textrm{mol}^{-1}$
    \item Speed of Light: $c = 2.998\times 10^{8}\textrm{ms}^{-1}$
\end{enumerate}
\end{remark}
%
\subsubsection{Problems}
%
\begin{problem}
Write the following in scientific notation:
\begin{enumerate}
    \begin{multicols}{4}
    \item[1.] $350,\! 000,\! 000$
    \item[3.] $120,\!500,\!000,\!000$
    \item[2.] $0.0000000523$
    \item[4.] $10.01$
    \end{multicols}
\end{enumerate}
\end{problem}
\begin{proof}[Solution]
\
\begin{enumerate}
    \begin{multicols}{4}
    \item $3.5 \times 10^8$
    \item $5.23 \times 10^{-8}$
    \item $1.205 \times 10^{11}$
    \item $1.001 \times 10^{1}$
    \end{multicols}
\end{enumerate}
\end{proof}
%
\begin{problem}
Write in expanded form:
\begin{enumerate}
    \begin{multicols}{3}
    \item $6.02\times 10^{15}$
    \item $3.0 \times 10^{8}$
    \item $1.819\times 10^{-9}$
    \end{multicols}
\end{enumerate}
\end{problem}
\begin{proof}[Solution]
\
\begin{enumerate}
\begin{multicols}{3}
\item $6,\!020,\!000,\!000,\!000,\!000$
\item $300,\!000,\!000$
\item $0.000000001819$
\end{multicols}
\end{enumerate}
\end{proof}
%
\begin{problem}
Simplify:
\begin{enumerate}
    \begin{multicols}{2}
    \item[1.] $(3\times 10^3)(5\times 10^6)$
    \item[3.] $(3\times 10^{-4})^{2}$
    \item[2.] $\frac{6\times 10^9}{3\times 10^4}$
    \item[4.] $\frac{(3.2\times 10^{5})(2\times 10^{-3})}{2\times 10^{-5}}$
    \end{multicols}
\end{enumerate}
\end{problem}
\begin{proof}[Solution]
\
\begin{enumerate}
    \item $(3\times 10^3)(5\times 10^6) = 3\times 5 \times 10^3 \times 10^6 = 15 \times 10^{6+3} = 15\times 10^9 = \boxed{1.5\times 10^{10}}$
    \item $\frac{6\times 10^9}{3\times 10^4} =  \frac{6}{3}\times \frac{10^9}{10^4} = 2\times 10^{9-4} = \boxed{2\times 10^5}$
    \item $(3\times 10^{-4})^2 = 3^2 \times (10^{-4})^2 = \boxed{9\times 10^{-8}}$
    \item $\frac{(3.2\times 10^{5})(2\times 10^{-2})}{2\times 10^{-5}} = 3.2\times 10^{5} \times \frac{2\times 10^{-3}}{2\times 10^{-5}} = 3.2 \times 10^{5}\times 10^{2} = \boxed{3.2\times 10^{7}}$
\end{enumerate}
\end{proof}
%
\subsection{Substitution}
%
\subsubsection{Problems}
%
\begin{problem}
Solve $xyz-4z$ for $x=3, y=-4, z=2$.
\end{problem}
\begin{proof}[Solution]
\begin{flalign*}
    xyz-4z\big|_{x=3,y=-4,z=2} &= (3)(-4)(2) - 4(2) & \tag{Substitution}\\
    &= -24 - 8 & \tag{Multiplication}\\
    &= \boxed{-32}& \tag{Subtration}
\end{flalign*}
\end{proof}
%
\begin{problem}
Solve $2x-y$ for $x=3, y=-4, z=2$.
\end{problem}
\begin{proof}[Solution]
\begin{flalign*}
    2x-y\big|_{x=3,y=-4,z=2} &= 2(3) - (-4) & \tag{Substitution}\\
    &= 6+4 & \tag{Multiplication}\\
    &= \boxed{10} & \tag{Addition}
\end{flalign*}
\end{proof}
%
\begin{problem}
Solve $x(y-3z)$ for $x=3, y=-4, z=2$.
\end{problem}
\begin{proof}[Solution]
\begin{flalign*}
    x(y-3z)\big|_{x=3,y=-4,z=2} &= (3)\big((-4)-3(2)\big) & \tag{Substitution}\\
    &= 3(-10) & \tag{Parenthesis}\\
    &= \boxed{-30} & \tag{Multiplication}
\end{flalign*}
\end{proof}
%
\begin{problem}
Solve $\frac{5x - z}{xy}$ for $x=3, y=-4, z=2$.
\end{problem}
\begin{proof}[Solution]
\begin{flalign*}
    \frac{5x-z}{xy}\big|_{x=3,y=-4,z=2} &= \frac{5(3)-(2)}{(3)(-4)} & \tag{Substitution}\\
    &= \frac{15-2}{-12} & \tag{Multiplication}\\
    &= \boxed{-\frac{13}{12}} & \tag{Subtraction}
\end{flalign*}
\end{proof}
%
\begin{problem}
Solve $3y^2 - 2x+4z$ for $x=3, y=-4, z=2$.
\end{problem}
\begin{proof}[Solution]
\begin{flalign*}
    3y^2-2x+4z\big|_{x=3,y=-4,z=2} &= 3(-4)^2 - 2(3)+4(2) & \tag{Substitution}\\
    &= 3(16) - 2(3) + 4(2) & \tag{Exponentiation}\\
    &= 48 - 6 + 8 & \tag{Multiplication}\\
    &= \boxed{50} & \tag{Addition/Subtraction}
\end{flalign*}
\end{proof}
%
\clearpage
%
\begin{problem}
Solve $x+y+z$ for $x = 1 , y =2, z = 3$
\end{problem}
\begin{proof}[Solution]
\begin{flalign*}
    x+y+z\big|_{x=1,y=2,z=3} &= (1)+(2)+(3) & \tag{Substitution}\\
    &= \boxed{6} & \tag{Addition}
\end{flalign*}
\end{proof}
%
\begin{problem}
Solve $(x+1)(y-2)$ for $x = 1 , y =2, z = 3$
\end{problem}
\begin{proof}[Solution]
\begin{flalign*}
    (x+1)(y-2)\big|_{x=1,y=2,z=3} &= \big((1)+1\big)\big((2)-2\big) & \tag{Substitution}\\
    &= 2\cdot 0 & \tag{Parenthesis}\\
    &= \boxed{0} & \tag{Multipliation}
\end{flalign*}
\end{proof}
%
\begin{problem}
Solve $x^2+y^2 - z^2$ for $x = 1 , y =2, z = 3$
\end{problem}
\begin{proof}[Solution]
\begin{flalign*}
    x^2+y^2-z^2\big|_{x=1,y=2,z=3} &= (1)^2+(2)^2-(3)^2 & \tag{Substitution}\\
    &= 1+4-9 & \tag{Exponentiation}\\
    &= \boxed{-4} & \tag{Addition/Subtraction}
\end{flalign*}
\end{proof}
%
\begin{problem}
Solve $\frac{z+1}{y(x+1)}$ for $x = 1 , y =2, z = 3$
\end{problem}
\begin{proof}[Solution]
\begin{flalign*}
    \frac{z+1}{y(x+1)}\big|_{x=1,y=2,z=3} &= \frac{(3)+1}{(2)\big((1)+1\big)} &\tag{Substitution}\\
    &= \frac{4}{2(2)} & \tag{Parenthesis}\\
    &= \frac{4}{4} & \tag{Multiplication}\\
    &= \boxed{1} & \tag{Division}
\end{flalign*}
\end{proof}
%
\begin{problem}
Solve $xy+xz+yz$ for $x = 0, y = 3, z = -1$
\end{problem}
\begin{proof}[Solution]
\begin{flalign*}
    xy+xz+yz\big|_{x=0,y=3,z=-1} &= (0)(3) + (0)(-1) + (3)(-1) & \tag{substitution}\\
    &= 0+0+(-3) & \tag{Multiplication}\\
    &= \boxed{-3} & \tag{Addition}
\end{flalign*}
\end{proof}
%
\begin{problem}
Solve $y^x + z^y$ for $x = 0, y = 3, z = -1$
\end{problem}
\begin{proof}[Solution]
\begin{flalign*}
    y^{x}+z^{y}\big|_{x=0,y=3,z=-1} &= (3)^{(0)} + (-1)^{(3)} & \tag{Substitution}\\
    &= 1 + (-1) & \tag{Exponentiation}\\
    &= \boxed{0} & \tag{Addition}
\end{flalign*}
\end{proof}
%
\begin{problem}
Solve $\frac{y+z}{x}$ for $x = 0, y = 3, z = -1$
\end{problem}
\begin{proof}[Solution]
\begin{flalign*}
    \frac{y+z}{x}\big|_{x=0,y=3,z=-1} &= \frac{(3)+(-1)}{(0)} & \tag{Substitution}\\
    &= \boxed{Undefined} & \tag{Division by Zero}
\end{flalign*}
\end{proof}
%
\begin{problem}
Solve $xy^{z^y} + y$ for $x = 0, y = 3, z = -1$
\end{problem}
\begin{proof}[Solution]
\begin{flalign*}
    xy^{z^{y}}+y\big|_{x=0,y=3,z=-1} &= (0)(3)^{(-1)^{(3)}} + 3  & \tag{Substitution}\\
    &= 0\cdot 3^{-1}+3 & \tag{Exponentiation}\\
    &= 0\cdot \frac{1}{3}+3 &\tag{Exponentiation}\\
    &= 0 + 3 &\tag{Multiplication}\\
    &= \boxed{3} &\tag{Addition}
\end{flalign*}
\end{proof}
%
\subsection{Linear Equations in One Variable}
%
\subsubsection{Problems}
%
\begin{problem}
Solve for $x$: $6x - 48 = 6$
\end{problem}
\begin{proof}[Solution]
\begin{flalign*}
6x-48 &= 6 \\
\Rightarrow 6x &= 54 & \tag{Add $48$ to Both Sides}\\
\Rightarrow x &= \frac{54}{6} & \tag{Divide Both Sides by $6$}\\
\Rightarrow x &= \boxed{9} & \tag{Division}
\end{flalign*}
\end{proof}
%
\clearpage
%
\begin{problem}
Solve for $x$: $\frac{2}{3}x - 5 = x-3$
\end{problem}
\begin{proof}[Solution]
\begin{flalign*}
\frac{2}{3}x - 5 &= x-3 \\
\Rightarrow 2x - 15 &= 3x-9 & \tag{Multiply Both Sides by $3$}\\
\Rightarrow -x - 15 &= -9 & \tag{Subtract $3x$ from Both Sides}\\
\Rightarrow -x &= 6 & \tag{Add $15$ to Both Sides}\\
\Rightarrow x &= \boxed{-6} & \tag{Multiply Both Sides by $-1$}
\end{flalign*}
\end{proof}
%
\begin{problem}
Solve for $x$: $50 -x - (3x+2) = 0$
\end{problem}
\begin{proof}[Solution]
\begin{flalign*}
    50 -x - (3x+2) &= 0 \\
    \Rightarrow 50 - x - 3x - 2 &= 0 & \tag{Distribute the Minus Sign}\\
    \Rightarrow 48 - 4x &= 0 & \tag{Simplify the Left-Hand Side}\\
    \Rightarrow 4x &= 48 & \tag{Add $4x$ to Both Sides}\\
    \Rightarrow x &= \frac{48}{4} & \tag{Divide Both Sides by $4$}\\
    \Rightarrow x &= \boxed{12} & \tag{Division}
\end{flalign*}
\end{proof}
%
\begin{problem}
Solve for $x$: $8 - 4(x-1) = 2+3(4-x)$
\end{problem}
\begin{proof}[Solution]
\begin{flalign*}
    8 - 4(x-1) &= 2+3(4-x)\\
    \Rightarrow 8 - 4x + 4 &= 2+12 - 3x & \tag{Simplify Both Sides}\\
    \Rightarrow 12-4x &= 14 - 3x & \tag{Simplify Both Sides}\\
    \Rightarrow 12 &= 14+x & \tag{Add $4x$ to Both Sides}\\
    \Rightarrow x &= \boxed{-2} & \tag{Subtract $14$ from Both Sides}
\end{flalign*}
\end{proof}
%
\begin{problem}
Solve $x+1 = 1$ for $x$.
\end{problem}
\begin{proof}[Solution]
\begin{flalign*}
    x+1 &= 1 \\
    \Rightarrow x &= \boxed{0} & \tag{Subtract $1$ from Both Sides}
\end{flalign*}
\end{proof}
%
\clearpage
%
\begin{problem}
Solve $4(x-1) + x = 0$ for $x$.
\end{problem}
\begin{proof}[Solution]
\begin{flalign*}
    4(x-1) + x &= 0\\
    \Rightarrow 5x - 4 &= 0 & \tag{Simplify the Left-Hand Side}\\
    \Rightarrow 5x &= 4 & \tag{Add $4$ to Both Sides}\\
    \Rightarrow x &= \boxed{\frac{4}{5}} & \tag{Divide Both Sides by $5$}
\end{flalign*}
\end{proof}
%
\begin{problem}
Solve $1-x + 10 = 7$ for $x$.
\end{problem}
\begin{proof}[Solution]
\begin{flalign*}
    1-x+10 &= 7 \\
    \Rightarrow 11-x &=7 & \tag{Simplify the Left-Hand Side}\\
    \Rightarrow 11 &= 7+x & \tag{Add $x$ to both sides}\\
    \Rightarrow x &= \boxed{4} & \tag{Subtract $7$ from Both Sides}
\end{flalign*}
\end{proof}
%
\subsection{Formulas}
%
\subsubsection{Problems}
%
\begin{problem}
Solve $PV = nRT$ for $T$.
\end{problem}
\begin{proof}[Solution]
This is the Ideal Gas Law from chemistry: $\boxed{T = \frac{PV}{nR}}$
\end{proof}
%
\begin{problem}
Solve $y=3x+2$ for $x$.
\end{problem}
\begin{proof}[Solution]
$y = 3x+2 \Rightarrow y-2 = 3x \Rightarrow \boxed{x =  \frac{y-2}{3}}$
\end{proof}
%
\begin{problem}
Solve $C = 2\pi r$ for $r$.
\end{problem}
\begin{proof}[Solution]
This is the formula for the circumference $C$ of a circle of radius $r$: $\boxed{r = \frac{C}{2\pi}}$
\end{proof}
%
\begin{problem}
Solve $\frac{x}{2}+\frac{y}{5} = 1$ for $y$.
\end{problem}
\begin{proof}[Solution]
$\frac{x}{2}+\frac{y}{5} = 1 \Rightarrow = \frac{y}{5} = 1-\frac{x}{2} \Rightarrow \boxed{y = 5 - \frac{5}{2}x}$.
\end{proof}
%
\begin{problem}
Solve $y = hx+4x$ for $x$.
\end{problem}
\begin{proof}[Solution]
$y = hx+4x \Rightarrow x(h+4) = y \Rightarrow \boxed{x= \frac{y}{h+4}}$
\end{proof}
%
\subsection{Word Problems}
%
\subsubsection{Problems}
%
\begin{problem}
$y$ is $5$ more than twice that of $x$, their sum is $35$. Find $x$ and $y$.
\end{problem}
\begin{proof}[Solution]
We have $y = 5+2x$ and $x+y = 35$. Substituting $y$, we get $x+2x+5 = 35\Rightarrow 3x+5 = 35\Rightarrow 3x = 30 \Rightarrow \boxed{x = 10}$. But $y = 2+2x = 5+2(10) \Rightarrow \boxed{y= 25}$.
\end{proof}
%
\begin{problem}
Ms. Jones invested $\$18,\!000$ in two accounts, one pays $6\%$ and the other $8\%$. Her total interest was $\$1,\!290$. How much did she have in each account?
\end{problem}
\begin{proof}[Solution]
Let $x$ be the amount in the $6\%$ account and $y$ be the amount in the $8\%$ account. Then $x+y = 18,\!000$ and $\frac{6}{100}x + \frac{8}{100}y = 1,\!290$. Solving the first equation for $y$, we have $y= 18,\!000 - x$. Substituting in the second equation, we get $\frac{6}{100}x + \frac{8}{100}(18,\!000 - x) = 1,\!290$. So we have $\frac{8 \cdot 18,000}{100} - \frac{2}{100}x = 1,\!290 \Rightarrow \frac{2}{100}x = 1,\!440 - 1,\!290 \Rightarrow \frac{x}{50} = 150 \Rightarrow \boxed{x = 7,\!500}$. But $y = 18,\!000 - x \Rightarrow y= 18,\!000 - 7,\!500 \Rightarrow \boxed{y = 10,\!500}$. Ms. Jones put \boxed{\$7,\!500\textrm{ in the }6\%\textrm{ account}}, and \boxed{10,\!500\textrm{ into the }8\%\textrm{ account}}.
\end{proof}
%
\begin{problem}
How many liters of $40\%$ and $16\%$ solution must be mixed to obtain $20$ liters of $22\%$ solution?
\end{problem}
\begin{proof}[Solution]
Let $x$ be the amount of $40\%$ solution and $y$ be the amount of $16\%$ solution. Then $x+y = 20$, and $\frac{40}{100}x +\frac{16}{100}y = \frac{22}{100}20\Rightarrow 40x+16y = 440$. Solving the first equation for $y$, we get $y = 20 - x$. Substituting, we have: $40x + 16(20-x) = 440\Rightarrow 24x + 320 = 440 \Rightarrow 24x = 120\Rightarrow x=\frac{120}{24}\Rightarrow \boxed{x = 5}$. But $y = 20 - x = 20 - 5 \Rightarrow \boxed{y = 15}$. So there are \boxed{5\textrm{ liters of }40\%} solution and \boxed{15\textrm{ liters of }16\%} solution.
\end{proof}
%
\begin{problem}
Sheila bought burgers and fries for her children and some friends. The burgers cost $\$2.05$ each and the fries are $\$0.85$ each. She bought a total of $14$ items for a total cost of $\$19.10$. How many of each did she buy?
\end{problem}
\begin{proof}[Solution]
Let $x$ be the number of burgers and $y$ be the number of fries. Then $2.05x + 0.85y = 19.10$, and $x+y = 14$. Using this second equation and solving for $y$, we get $y = 14 - x$. Substituting this into the first equation we get $2.05x + 0.85(14-x) = 19.10$. So $(2.05 - 0.85)x + 14\cdot 0.85 = 19.10 \Rightarrow 1.20x + 11.90 = 19.10\Rightarrow 1.20x = 7.20 \Rightarrow \boxed{x = 6}$. But $y = 14 - x \Rightarrow \boxed{y = 8}$. Sheila bought \boxed{6\textrm{ burgers}} and \boxed{8\textrm{ fries}}.
\end{proof}
%
\subsection{Inequalities}
%
There are two main rules for dealing with inequalities:
%
\begin{properties}\label{property:north_shore_properties_of_inequalities}
\
\begin{enumerate}[itemsep=0pt]
    \item \label{property:north_shore_additive_property_inequals}If $c$ is real and $a<b$, then $a+c<b+c$ \hfill [Additive Property of Inequalities]
    \item \label{property:north_shore_multiplicative_property_inequals}If $c$ is $\mathbf{positive}$ and $a<b$, then $ac < bc$. \hfill [Multiplicative Property of Inequalities]
    \item \label{property:north_shore_inverse_property_inequals}If $c$ is $\mathbf{negative}$ and $a<b$, then $bc < ac$. \hfill [Inverse Property of Inequalities]
\end{enumerate}
\end{properties}
%
\subsubsection{Problems}
%
\begin{problem}
Solve for $x$: $2x-7 \geq 3$
\end{problem}
\begin{proof}[Solution]
    \begin{flalign*}
        2x-7&\geq 3\\
        \Rightarrow 2x &\geq 10 & \tag{Add $7$ to Both Sides, property \ref{property:north_shore_properties_of_inequalities} part \ref{property:north_shore_additive_property_inequals}}\\
        \Rightarrow x &\geq \frac{10}{2} & \tag{Divide Both Sides by $2$, property \ref{property:north_shore_properties_of_inequalities} part \ref{property:north_shore_multiplicative_property_inequals}}\\
        \Rightarrow x &\geq 5 & \tag{Division}
    \end{flalign*}
\end{proof}
%
\begin{problem}
Solve for $x$: $-5(2x+3) <2x - 3$
\end{problem}
\begin{proof}[Solution]
\begin{flalign*}
-5(2x+3) &< 2x-3\\
\Rightarrow -10x - 15 &< 2x - 3 & \tag{Simplify the Left-Hand Side}\\
\Rightarrow -15 + 3 &< 2x + 10x & \tag{Add $10x+3$ to Both Sides, property \ref{property:north_shore_properties_of_inequalities} part \ref{property:north_shore_additive_property_inequals}}\\
\Rightarrow -12 &< 12 x & \tag{Simplify Both Sides}\\
\Rightarrow x&>-1 & \tag{Divide Both Sides by $12$, property \ref{property:north_shore_properties_of_inequalities} part \ref{property:north_shore_multiplicative_property_inequals}}
\end{flalign*}
\end{proof}
%
\begin{problem}
Solve for $x$: $3(x-4) - (x+1) \leq -12$
\end{problem}
\begin{proof}[Solution]
    \begin{flalign*}
        3(x-4) - (x+1) &\leq -12\\
        \Rightarrow 3x-12 - x - 1 &\leq -12 & \tag{Simplify the Left-Hand Side}\\
        \Rightarrow 2x - 13 &\leq -12 & \tag{Simplify the Left-Hand Side}\\
        \Rightarrow 2x &\leq 1 & \tag{Add $13$ to Both Sides, property \ref{property:north_shore_properties_of_inequalities} part \ref{property:north_shore_additive_property_inequals}}\\
        \Rightarrow x &\leq 0.5 & \tag{Divide Both Sides by $2$, property \ref{property:north_shore_properties_of_inequalities} part \ref{property:north_shore_multiplicative_property_inequals}}
    \end{flalign*}
\end{proof}
%
\subsection{Exponents and Polynomials}
%
The two main rules for problem with polynomials and exponents are:
\begin{align}
    (a_1 x^2 + b_1 x + c_1) + (a_2 x^2 + b_2 x + c_2) &= x^2(a_1+a_2) + x (b_1+b_2) + (c_1+c_2)\label{equation:north_shore_additive_polynomial_property}\\
    (ax+b)(cx+d) &= acx^2 + (ad+bc)x + bd\label{equation:north_shore_multiplicative_polynomial_property}
\end{align}
%
\begin{remark}
Equation \ref{equation:north_shore_additive_polynomial_property} says that the coefficients of like terms can be added together to simplify the expression, and equation \ref{equation:north_shore_multiplicative_polynomial_property} is often called the \gls{foil} rule.
\end{remark}
%
\begin{remark}
\label{remark:north_shore_square_of_a_sum_foil}
There is a special case of \gls{foil}. We can write $(a+b)^2 = (a+b)(a+b) = a^2+ab+ba+b^2 = a^2+2ab+b^2$. This is a good formula to be familiar with:
\begin{equation}\label{equation:North_Shore_square_of_a_sum}
    (a+b)^2 = a^2+2ab+b^2
\end{equation}
\end{remark}
%
\subsubsection{Problems}
%
\begin{problem}
Simplify using only positive exponents: $(3x^0y^5z^6)(-2xy^3z^{-2})$
\end{problem}
\begin{proof}[Solution]
    \begin{flalign*}
        (3x^0 y^5 z^6)(-2xy^3z^{-2}) &= \big(3\cdot (-2)\big) \big(x^{0}\cdot x^{1}\big)\big(y^{5}\cdot y^{3}\big)\big(z^{6}\cdot z^{-2}\big) &\tag{Property \ref{property:North_Shore_Arithmetic_Properties} part \ref{property:north_shore_arithmetic_properties_assoc_mult}}\\
        &= -6 x^{0+1}y^{5+3}z^{6 - 2} & \tag{Property \ref{property:North_Shore_Exponent_Rules} part \ref{property:north_shore_product_property_of_expo}}\\
        &= -6xy^8z^4 & \tag{Simplify Exponents}
    \end{flalign*}
\end{proof}
%
\begin{problem}
Simplify using only positive exponents: $(3x^2 - 5x - 6) + (5x^2 +4x + 4)$
\end{problem}
\begin{proof}[Solution]
\begin{flalign*}
    (3x^2 - 5x - 6) + (5x^2 + 4x + 4) &= (3+5)x^2 + (-5+4)x + (-6+4) &\tag{Equation \ref{equation:north_shore_additive_polynomial_property}}\\
    &= 8x^2 - x -2 & \tag{Simplify Parenthesis}
\end{flalign*}
\end{proof}
%
\begin{problem}
Simplify using only positive exponents: $\frac{(2a^{-5}b^4 c^3)^{-2}}{(3a^{3}b^{-7}c^3)^2}$
\end{problem}
\begin{proof}[Solution]
\begin{flalign*}
    \frac{(2a^{-5}b^4 c^3)^{-2}}{(3a^{3}b^{-7}c^3)^2} &= (2a^{-5}b^{4}c^{3})^{-2}\cdot\frac{1}{(3a^3 b^{-7}c^{3})^2} & \tag{Remark \ref{remark:North_Shore_Rational_Expressions}}\\
    &= \frac{1}{(2a^{-5}b^{4}c^{3})^{2}}\cdot \frac{1}{(3a^{3}b^{-7}c^{3})^{2}} & \tag{Property \ref{property:North_Shore_Exponent_Rules} part \ref{property:north_shore_inverse_proerty_of_expo}}\\
    &= \frac{1}{2^{2}(a^{-5})^{2}(b^{4})^{2}(c^{3})^{2}}\cdot\frac{1}{3^{2}(a^{3})^{2}(b^{-7})^{2}(c^{3})^{2}} & \tag{Property \ref{property:North_Shore_Exponent_Rules} part \ref{property:north_shore_distributive_property_of_expo}}\\
    &= \frac{1}{4a^{-10}b^{8}c^{6}}\cdot \frac{1}{9a^{6}b^{-14}c^{6}} & \tag{Property \ref{property:North_Shore_Exponent_Rules} part \ref{property:north_shore_power_property_of_expo}}\\
    &= \frac{a^{10}}{4b^{8}c^{6}}\cdot \frac{b^{14}}{9a^{6}c^{6}} & \tag{Property \ref{property:North_Shore_Exponent_Rules} part \ref{property:north_shore_inverse_proerty_of_expo}}\\
    &= \frac{a^{10}b^{14}}{36a^{6}b^{8}c^{6}c^{6}} & \tag{Multiplication}\\
    &= \frac{a^{10}b^{14}}{36a^{6}b^{8}c^{12}} & \tag{Property \ref{property:North_Shore_Exponent_Rules} part \ref{property:north_shore_product_property_of_expo}}\\
    &= \frac{a^{4}b^{6}}{36 c^{12}} & \tag{Division}
\end{flalign*}
\end{proof}
%
\begin{problem}
Simplify using only positive exponents: $(-a^5b^7c^9)^4$
\end{problem}
\begin{proof}[Solution]
    \begin{flalign*}
        (-a^5 b^7 c^9)^4 &= \big((-1)\cdot a^5 b^7 c^9\big)^4 & \tag{Rewrite Expression}\\
        &= (-1)^4 (a^{5})^{4}(b^{7})^{4}(c^{9})^{4} & \tag{Property \ref{property:North_Shore_Exponent_Rules} part \ref{property:north_shore_distributive_property_of_expo}}\\
        &= a^{20}b^{28}c^{36} & \tag{Property \ref{property:North_Shore_Exponent_Rules} part \ref{property:north_shore_power_property_of_expo}}
    \end{flalign*}
\end{proof}
%
\begin{problem}
Simplify using only positive exponents: $(4x^2 y^6 z)^2 (-2x^{-2}y^3z^4)^6$
\end{problem}
\begin{proof}[Solution]
    \begin{flalign*}
        (4x^2 y^6 z)^2 (-x^{-2}y^3z^4)^6 &= \big(4^2 (x^{2})^{2} (y^{6})^{2} (z)^{2}\big)\big((x^{-2})^{6}(y^{3})^{6}(z^{4})^{6}\big) & \tag{Property \ref{property:North_Shore_Exponent_Rules} part \ref{property:north_shore_distributive_property_of_expo}}\\
        &= (16 x^{4}y^{12}z^{2})(x^{-12}y^{18}z^{24}) & \tag{Property \ref{property:North_Shore_Exponent_Rules} part \ref{property:north_shore_power_property_of_expo}}\\
        &=16 (x^{4}x^{-12})(y^{12}y^{18})(z^{2}z^{24}) & \tag{Property \ref{property:North_Shore_Arithmetic_Properties} part \ref{property:north_shore_arithmetic_properties_assoc_mult}}\\
        &= 16 x^{4-12}y^{18+12}z^{2+24} & \tag{Property \ref{property:North_Shore_Exponent_Rules} part \ref{property:north_shore_product_property_of_expo}}\\
        &= 16 x^{-8}y^{30}z^{26} & \tag{Simplify Exponents}\\ 
        &= \frac{16y^{30}z^{26}}{x^{8}} & \tag{Property \ref{property:North_Shore_Exponent_Rules} part \ref{property:north_shore_inverse_proerty_of_expo}}
    \end{flalign*}
\end{proof}
%
\begin{problem}
Simplify using only positive exponents: $\frac{24x^4 - 32x^3 + 16x^2}{8x^2}$
\end{problem}
\begin{proof}[Solution]
    \begin{flalign*}
        \frac{24x^4 - 32x^3 + 16x^2}{8x^2} &= \frac{8x^2(3x^2 - 4x + 2)}{8x^2} & \tag{Property \ref{property:North_Shore_Arithmetic_Properties} part \ref{property:north_shore_arithmetic_properties_distributive_property}}\\
        &= 3x^2 - 4x + 2 & \tag{Property \ref{property:North_Shore_Arithmetic_Properties} part \ref{property:north_shore_arithmetic_properties_mult_inverse}}
    \end{flalign*}
\end{proof}
%
\begin{problem}
Simplify using only positive exponents: $(x^2 - 5x)(2x^3 - 7)$
\end{problem}
\begin{proof}[Solution]
    \begin{flalign*}
        (x^2 - 5x)(2x^3 - 7) &= 2x^5 - 7x^2-10x^4 +35x & \tag{\gls{foil}}\\
        &= x(2x^4 -10x^3-7x + 35) & \tag{Property \ref{property:North_Shore_Arithmetic_Properties} part \ref{property:north_shore_arithmetic_properties_distributive_property}}
    \end{flalign*}
\end{proof}
%
\begin{problem}
Simplify using only positive exponents: $\frac{26 a^2 b^{-5}c^{9}}{-4a^{-6}bc^{9}}$
\end{problem}
\begin{proof}
    \begin{flalign*}
        \frac{26 a^2 b^{-5}c^{9}}{-4a^{-6}bc^{9}} &= 26a^2b^{-5}c^9\cdot \frac{1}{-4 a^{-6}bc^9} & \tag{Remark \ref{remark:North_Shore_Rational_Expressions}}\\
        &= \frac{26 a^2 c^9}{b^5} \cdot \frac{a^{6}}{-4bc^9} & \tag{Property \ref{property:North_Shore_Exponent_Rules} part \ref{property:north_shore_inverse_proerty_of_expo}}\\
        &= -\frac{26}{4}\cdot \frac{a^{2+6}c^{9}}{b^{5+1}c^9} & \tag{Property \ref{property:North_Shore_Exponent_Rules} part \ref{property:north_shore_product_property_of_expo}}\\
        &= -\frac{13a^{8}}{2b^{6}} & \tag{Property \ref{property:North_Shore_Arithmetic_Properties} part \ref{property:north_shore_arithmetic_properties_mult_inverse}}
    \end{flalign*}
\end{proof}
%
\begin{problem}
Simplify using only positive exponents: $(5a+6)^2$
\end{problem}
\begin{proof}[Solution]
    \begin{flalign*}
        (5a+6)^2 &= 25a^2 + 60a + 36 & \tag{Remark \ref{remark:north_shore_square_of_a_sum_foil} equation \ref{equation:North_Shore_square_of_a_sum}}
    \end{flalign*}
\end{proof}
%
\begin{problem}
Simplify using only positive exponents: $(5x+1)(x+3)$
\end{problem}
\begin{proof}[Solution]
    \begin{flalign*}
        (5x+1)(x+3) &= 5x^2 + 16x+3 & \tag{Equation \ref{equation:north_shore_multiplicative_polynomial_property}}
    \end{flalign*}
\end{proof}
%
\subsection{Factoring}
%
\begin{definition}
A quadratic is an equation of the form $y=ax^2+bx+c$.
\end{definition}
%
\begin{definition}
The Factorization of a quadratic $y=ax^2+bx+c$ is an equivalent expression of the form $y=\gamma(x+\alpha)(x+\beta)$.
\end{definition}
%
\begin{theorem}
\label{theorem:north_shore_factorization_of_quadratic_when_a_is_equal_to_zero}
If $y=x^2+bx+c$, and if the factorization is $y=(x+\alpha)(x+\beta)$, then:
\begin{equation*}
    b = \alpha + \beta
    \quad \quad \quad
    c = \alpha \cdot \beta
\end{equation*}
\end{theorem}
\begin{proof}
Suppose $y = x^2 + bx + c$ and $y = (x+\alpha)(x+\beta)$. By \gls{foil}, $y = (x+\alpha)(x+\beta) = x^2 +x(\alpha + \beta) + \alpha \cdot \beta$. But also $y=x^2 + bx + c$. But the coefficients of the terms must be equal. Therefore, $\alpha + \beta = b$ and $\alpha \cdot \beta = c$. This proves the theorem.
\end{proof}
%
\begin{remark}\label{remark:north_shore_example_of_using_factorization_when_a_equals_zero}
This result helps us factor quadratic equations when the leading term has a coefficient of $1$ (That is, $a=1$). If we see something like $x^2-2x + 1$ and we want to factor it, all we need to ask is ``What two numbers add to $-2$ and multiply to $1$?" Often times guessing and checking will get the answer after a few tries. For $x^2-2x+1$ We see that $(-1)+(-1) = -2$, and $(-1)\cdot (-1) = 1$. So, $x^2 - 2x +1 = (x-1)(x-1) = (x-1)^2$
\end{remark}
%
\begin{theorem}[The Difference of Squares]
\label{theorem:north_shore_difference_of_squares}
If $a$ and $b$ are real numbers, then:
\begin{equation*}
    a^2 - b^2 = (a-b)(a+b)
\end{equation*}
\end{theorem}
\begin{proof}
By \gls{foil}, $(a-b)(a+b) = a^2 - ab + ba - b^2$. But $ab = ba$, so $ab-ba = 0$. Thus $a^2 - ab + ba - b^2 = a^2 - b^2$. Therefore $(a-b)(a+b) = a^2-b^2$ This proves the theorem.
\end{proof}
%
\begin{remark}
This helps factor quadratics quickly if two squares are being subtracted.
\end{remark}
%
\begin{example}
Factor the expression: $9x^2 - 64$. Note that $9 = 3^2$, so $9x^2 = (3x)^2$. Also note that $64 = 8^2$. So we can write $9x^2 - 64 = (3x)^2 - (8)^2$. Using the difference of squares formula, we have $9x^2 - 64 = (3x)^2 - (8)^2 = (3x-8)(3x+8)$
\end{example}
%
\begin{example}
Factor the expression: $16x^4 - 81y^4$. While this looks like a ``Quartic Equation" (Equations involving $x^4$), it can be rewriten as a quadratic one. Note that $16 = 4^2$ and $81 = 9^2$. But also $x^4 = (x^2)^2$ and $y^4 = (y^2)^2$. So we have $16x^4-81y^4 = (4x^2)^2 - (9y^2)^2$. This is a difference of squares, and so we can apply the difference of squares formula. $16x^4 - 81y^4 = (4x^2)^2 - (9y^2)^2 = (4x^2 + 9y^2)(4x^2 - 9y^2)$. We're not quite done yet, for we can simplify $4x^2 - 9y^2$ as well. Note that $4x^2 = (2x)^2$, and $9y^2 = (3y)^2$. So we have $4x^2 - 9y^2 = (2x)^2 - (3y)^2 = (2x-3y)(2x+3y)$. Together, we have: $16x^4 - 81y^4 = (4x^2 + 9y^2)(2x-3y)(2x+3y)$
\end{example}
%
\begin{theorem}[The Difference of Cubes]
\label{theorem:north_shore_difference_of_cubes}
If $a$ and $b$ are real numbers, then:
\begin{equation*}
    a^3 - b^3 = (a-b)(a^2+ab+b^2)
\end{equation*}
\end{theorem}
\begin{proof}
By the distributive property, $(a-b)(a^2+ab+b^2) = a(a^2+ab+b^2) - b(a^2+ab+b^2)$. Distributing again, we get $a^3 + a^2b + ab^2 - ba^2 - ab^2 - b^3$. Rearranging, we have $(a^3 - b^3) + (a^2b - ba^2) + (ab^2 - b^2a)$. By the commutative property, $a^2b = ba^2$, and $ab^2 = b^2a$. Therefore, $(a^2b - ba^2)+(ab^2 - b^2a) = 0$. Thus, $(a-b)(a^2+ab+b^2) = a^3 - b^3$. This completes the proof.
\end{proof}
%
\begin{remark}
We can use the difference of cubes formula to factor cubic expressions.
\end{remark}
%
\begin{theorem}[The Sum of Squares]
\label{theorem:north_shore_sum_of_squares}
If $x$ and $a$ are real numbers, and if $a$ is non-zero, then there is no \textbf{real} factorization of $x^2+a^2$.
\end{theorem}
\begin{proof}
For suppose $(x+\alpha)(x+\gamma)$ is a real factorization. Then by theorem \ref{theorem:north_shore_factorization_of_quadratic_when_a_is_equal_to_zero}, $\alpha+\beta = 0$ and $\alpha\cdot \beta = a^2$. But then $\alpha = -\beta$ and therefore $-\alpha^2 = a^2$. But $a$ is real, and therefore $a^2 \geq 0$. But as $\alpha$ is real and non-zero, $\alpha^2> 0$, and thus $-\alpha^2 <0$. But $a^2 = -\alpha^2$, a contradiction as $a^2 \geq 0$. Therefore there is no real factorization.
\end{proof}
%
\begin{remark}
If we see a sum of squares we know that there is no real factorization.
\end{remark}
%
\subsubsection{Problems}
%
\begin{problem}
Factor $x^2 + 5x - 6$
\end{problem}
\begin{proof}[Solution]
By theorem \ref{theorem:north_shore_factorization_of_quadratic_when_a_is_equal_to_zero}, if $x^2+5x-6 = (x+\alpha)(x+\beta)$, then $\alpha+\beta = 5$ and $\alpha\cdot \beta = -6$. By guessing and checking a few common factors of $5$ and $6$, we get $\alpha = 6$ and $\beta = -1$. So, $x^2+5x-6 = (x+6)(x-1)$
\end{proof}
%
\begin{problem}
$x^2 - 5x - 6$.
\end{problem}
\begin{proof}[Solution]
By theorem \ref{theorem:north_shore_factorization_of_quadratic_when_a_is_equal_to_zero}, if $x^2-5x-6=(x+\alpha)(x+\beta)$, then $\alpha+\beta = -5$ and $\alpha\cdot \beta = -6$. After guessing and checking, we have $\alpha = -6$ and $\beta = 1$. So $x^2-5x-6 = (x-6)(x+1)$
\end{proof}
%
\begin{problem}
$4x^2 - 36$
\end{problem}
\begin{proof}[Solution]
Note that $4x^2-36 = 4(x^2-3^2)$. By theorem \ref{theorem:north_shore_difference_of_squares}, $x^2-3^2= (x+3)(x-3)$. Thus, $4x^2-36 = 4(x-3)(x+3)$
\end{proof}
%
\begin{problem}
$x^2 + 4$
\end{problem}
\begin{proof}
This is a sum of squares. By theorem \ref{theorem:north_shore_sum_of_squares}, there is no real factorization.
\end{proof}
%
\begin{problem}
$64x^4 - 4y^4$.
\end{problem}
\begin{proof}
Note that $64x^2 - 4y^2 = (8x^2)^2 - (2y^2)^2$. By theorem \ref{theorem:north_shore_difference_of_squares}, $(8x^2)^2 - (2y^2)^2 = (8x^2+2y^2)(8x^2 - 2y^2)$. Note that $8x^2$ and $2y^2$ share a common factor of $2$ so we can simplify this as $ 4(4x^2 +y^2)(4x^2-y^2)$. Again by the difference of squares, we have $4x^2 - y^2 = (2x)^2 - y^2 = (2x-y)(2x+y)$. Piecing this all back together, we have $64x^2 - 4y^2 = 4(4x^2 + y^2)(2x-y)(2x+y)$.
\end{proof}
%
\begin{problem}
$8x^3 - 27$
\end{problem}
\begin{proof}[Solution]
By theorem \ref{theorem:north_shore_difference_of_cubes}, $8x^3 - 27 = (2x)^3 - (3)^3 = (2x-3)(4x^2+6x+9)$
\end{proof}
%
\begin{problem}
$49y^2 + 84y + 36$
\end{problem}
\begin{proof}[Solution]
First note that $49 = 7^2$, $36 = 6^2$, and $84 = 2\cdot 6 \cdot 7$. So, $49x^2 + 84x + 36 = (7x)^2 + 2(7)(6)x + (6)^2$. But $(ax+b)^2 = a^2x^2 + 2abx + b^2$. In our problem we have that $a=7$ and $b = 6$. So $49x^2 + 84x + 36 = (7x+6)^2$.
\end{proof}
%
\begin{problem}
$12x^2 + 12x + 3$
\end{problem}
\begin{proof}[Solution]
First note that each term is divisible by $3$, so we may factor that out to get $12x^2 + 12x + 3 = 3(4x^2+4x+1)$. Next note that $4 = 2^2$, and $4 = 2(2)(1)$. So we have $12x^2+12x+3 = 3\big((2x)^2 + 2(2)(1)x + (1)^2\big)$. Similarly to the last problem we have $4x^2 + 4x + 1$ is of the form $a^2x^2 + 2abx + b^2$. So, $3\big((2x)^2+2(2)(1)(x) + (1)^2\big) = 3(2x+1)^2$. The answer is $12x^2+12x+3 = 3(2x+1)^2$.
\end{proof}
%
\subsection{Quadratic Expressions}
%
\begin{theorem}[Completing the Square]
\label{theorem:north_shore_completing_the_square}
If $y = ax^2 +bx +c$, then $y = a(x+\frac{b}{2a})^2 - \frac{b^2}{4a}+c$
\end{theorem}
\begin{proof}
Using \gls{foil}, we get $(x+\frac{b}{2a})^2 = x^2 + \frac{b}{a}x + \frac{b^2}{4a^2}$. So $a(x+\frac{b}{2a})^2 = ax^2 + bx + \frac{b^2}{4a}$. But then we have $a(x+\frac{b}{2a})^2 - \frac{b^2}{4a} + c = ax^2 + bx +c$. This proves the theorem.
\end{proof}
%
\begin{theorem}[The Quadratic Formula]
\label{theorem:north_shore_quadratic_formula_theorem}
If $ax^2 + bx + c = 0$, $a\ne 0$, then $x = \frac{-b\pm \sqrt{b^2 - 4ac}}{2a}$
\end{theorem}
\begin{proof}
By theorem \ref{theorem:north_shore_completing_the_square}, $ax^2 + bx + c = a(x+\frac{b}{2a})^2 - \frac{b^2}{4a} + c$. But $ax^2 + bx + c = 0$, so $a(x+\frac{b}{2a}x)^2 - \frac{b^2}{4a} + c = 0$. Therefore $a(x+\frac{b}{2a})^2 = \frac{b^2}{4a} - c$. But $c = \frac{4ac}{4a}$, so $a(x^2+\frac{b}{2a})^2 = \frac{b^2-4ac}{4a}$. Dividing both sides by $a$, we get $(x+\frac{b}{2a})^2 = \frac{b^2-4ac}{4a^2}$. Now we take square roots, but note that there are two possible square roots. So, $x+\frac{b}{2a} = \pm \frac{\sqrt{b^2 - 4ac}}{2a}$. Subtracting $\frac{b}{2a}$ from both sides we get $x = -\frac{b}{2a} \pm \frac{\sqrt{b^2-4ac}}{2a} = \frac{-b \pm \sqrt{b^2 - 4ac}}{2a}$
\end{proof}
%
\begin{remark}
The thing to remember is $ax^2 +bx +c = 0$ has two solutions:

\begin{equation*}
    x = \frac{-b + \sqrt{b^2 - 4ac}}{2a}
    \quad \quad \quad
    x = \frac{-b - \sqrt{b^2-4ac}}{2a}
\end{equation*}
\end{remark}
%
\begin{theorem}
\label{theorem:north_shore_zeros_of_a_factored_polynomial}
If $\gamma(x-\alpha)(x-\beta) = 0$, then either $x=\alpha$ or $x=\beta$.
\end{theorem}
%
\begin{remark}
This tells us that if we see something like $ax^2+bx+c = 0$ and if we can $\mathbf{factor}$ the expression into the form $\gamma(x-\alpha)(x-\beta)$, then we can skip the quadratic formula because we know the solution is $x = \alpha$ and $x=\beta$.
\end{remark}
%
\subsubsection{Problems}
%
\begin{problem}
Find all solutions for a: $4a^2 + 9a + 2 = 0$
\end{problem}
\begin{proof}[Solution]
By theorem \ref{theorem:north_shore_quadratic_formula_theorem}, $a = \frac{-9 \pm \sqrt{9^2 - 4\cdot 4 \cdot 2}}{2\cdot 4} = \frac{-9 \pm 7}{8}$. $a=-2$, $a=-\frac{1}{4}$
\end{proof}
%
\begin{problem}
Find all solutions for $x$: $9x^2 - 81 = 0$
\end{problem}
\begin{proof}[Solution]
By theorem \ref{theorem:north_shore_difference_of_squares}, $9x^2-81 = (3x - 9)(3x+9)$. By theorem \ref{theorem:north_shore_zeros_of_a_factored_polynomial}, $3x = 9$, or $3x = -9$. The solutions are $x = 3$, and $x= -3$.
\end{proof}
%
\begin{problem}
Find all solutions for $x$: $25x^2 - 6 = 30$.
\end{problem}
\begin{proof}[Solution]
Subtract $30$ to get $25x^2 - 36 = 0$. By theorem \ref{theorem:north_shore_difference_of_squares}, $25x^2-36 =(5x - 6)(5x+6)$. By theorem \ref{theorem:north_shore_zeros_of_a_factored_polynomial}, $5x = 6$ or $5x = -6$. The solutions are $x = \frac{6}{5}$, $x=-\frac{6}{5}$.
\end{proof}
%
\begin{problem}
Find all solutions for $x$: $3x^2 - 5x - 2 = 0$
\end{problem}
\begin{proof}[Solution]
By theorem \ref{theorem:north_shore_quadratic_formula_theorem}, $x = \frac{5 \pm \sqrt{(-5)^2 - 4(3)(-2)}}{2(3)} = \frac{5 \pm 7}{6}$. So $x = 2$ and $x = -\frac{1}{3}$
\end{proof}
%
\begin{problem}
Find all solutions for $x$: $(3x+2)^2 = 16$
\end{problem}
\begin{proof}[Solution]
Subtracting $16$ we get $(3x+2)^2 - 16$. By theorem \ref{theorem:north_shore_difference_of_squares}, $(3x+2)^2-16 = (3x+6)(3x-2)$. By theorem \ref{theorem:north_shore_zeros_of_a_factored_polynomial}, $3x = -6$ or $3x = 2$. Thus, $x = -2$, $x = \frac{2}{3}$.
\end{proof}
%
\begin{problem}
Find all solutions for $r$: $r^2 - 2r - 4 = 0$
\end{problem}
\begin{proof}[Solution]
By theorem \ref{theorem:north_shore_quadratic_formula_theorem}, $r = \frac{2 \pm \sqrt{(-2)^2 - 4(1)(-4)}}{2(1)} =  1\pm \sqrt{5}$. By theorem \ref{theorem:north_shore_zeros_of_a_factored_polynomial}, $x = 1+\sqrt{5}$ or $x=1-\sqrt{5}$. The solutions are $x = 1+\sqrt{5}$, $x=1-\sqrt{5}$
\end{proof}
%
\subsection{Rational Expressions}
%
\begin{theorem}[Cross Multiplying]
\label{theorem:north_shore_cross_multiplying}
If $a,b,c,d$ are real numbers, $b\ne 0$, $d\ne 0$, then:
\begin{equation*}
    \frac{a}{b} +\frac{c}{d} = \frac{ad+bc}{bd}
\end{equation*}
\end{theorem}
%
\begin{definition}
\label{definition:north_shore_numerator_of_rational_expression}
The numerator of a rational expression $\frac{P(x)}{Q(x)}$ is the polynomial $P(x)$.
\end{definition}
%
\begin{definition}
\label{definition:north_shore_denominator_of_rational_expression}
The denominator of a rational expression $\frac{P(x)}{Q(x)}$ is the polynomial $Q(x)$.
\end{definition}
%
\subsubsection{Problems}
%
\begin{problem}
Simplify: $\frac{4}{2a - 2} + \frac{3a}{a^2 - a}$
\end{problem}
\begin{proof}[Solution]
    \begin{flalign*}
        \frac{4}{2a-2}+\frac{3a}{a^2-a} & = \frac{2}{a-1} + \frac{3}{a-1} & \tag{Factor and Simplify Terms}\\
        &= \frac{5}{a-1} & \tag{Addition}
    \end{flalign*}
\end{proof}
%
\begin{problem}
Simplify: $\frac{3}{x^2 - 1} - \frac{4}{x^2 + 3x + 2}$
\end{problem}
\begin{proof}[Solution]
    \begin{flalign*}
        \frac{3}{x^2-1} - \frac{4}{x^2+3x+2} &= \frac{3}{(x-1)(x+1)} - \frac{4}{(x+2)(x+1)} & \tag{Factor Denominators}\\
        &= \frac{3(x+2)-4(x-1)}{(x-1)(x+2)(x+1)} & \tag{Theorem \ref{theorem:north_shore_cross_multiplying}}\\
        &= \frac{10-x}{(x+1)(x+2)(x-1)} & \tag{Simplify Numerator}
    \end{flalign*}
\end{proof}
%
\begin{problem}
Simplify: $\frac{x^3-1}{x-1}$
\end{problem}
\begin{proof}[Solution]
    \begin{flalign*}
        \frac{x^3-1}{x-1} &= \frac{(x-1)(x^2+x+1)}{x-1} & \tag{Theorem \ref{theorem:north_shore_difference_of_cubes}}\\
        &= x^2+x+1 & \tag{Division}
    \end{flalign*}
\end{proof}
%
\begin{problem}
Simplify: $\frac{\frac{2}{x} - \frac{1}{y}}{\frac{1}{xy}}$
\end{problem}
\begin{proof}[Solution]
    \begin{flalign*}
        \frac{\frac{2}{x} - \frac{1}{y}}{\frac{1}{xy}} &= xy\big(\frac{2}{x}-\frac{1}{y}\big) & \tag{Property \ref{property:North_Shore_Exponent_Rules} part \ref{property:north_shore_inverse_proerty_of_expo}}\\
        &= 2y - x & \tag{Multiplication}
    \end{flalign*}
\end{proof}
%
\begin{problem}
$\frac{2}{x-1} + \frac{1}{x+1} = \frac{5}{4}$
\end{problem}
\begin{proof}[Solution]
    \begin{flalign*}
        \frac{2}{x-1} + \frac{1}{x+1} &= \frac{2(x+1) + (x-1)}{(x+1)(x-1)} & \tag{Theorem \ref{theorem:north_shore_cross_multiplying}}\\
        &= \frac{3x+1}{(x+1)(x-1)} & \tag{Simplify the Numerator}\\
        \Rightarrow \frac{3x+1}{(x+1)(x-1)} &= \frac{5}{4} &\tag{Substitution} \\
        \Rightarrow 3x+1 &= \frac{5}{4}(x+1)(x-1) & \tag{Multiply Both Sides by $(x+1)(x-1)$}\\
        \Rightarrow 12x+4 &= 5x^2-5 & \tag{Multiply Both Sides by $4$ and Simplify}\\
        \Rightarrow 5x^2 - 12x - 9 &=0 & \tag{Subtract $12x+4$ from Both Sides}\\
        \Rightarrow x &= \frac{12 \pm \sqrt{(12)^2 - 4\cdot (5)\cdot(-9)}}{2\cdot(5)} & \tag{Theorem \ref{theorem:north_shore_quadratic_formula_theorem}}\\
        \Rightarrow x &= \frac{12 \pm \sqrt{144 + 180}}{10} & \tag{Simplify}\\
        \Rightarrow x &= \frac{12 \pm \sqrt{324}}{10} & \tag{Simplify}\\
        \Rightarrow x &= \frac{12 \pm 18}{10} & \tag{Simplify}\\
        \Rightarrow x &= 3, -\frac{3}{5} & \tag{Simplify}
    \end{flalign*}
    There are two solutions: $x=3$, $x=-\frac{3}{5}$
\end{proof}
\end{document}