\documentclass[../main.tex]{subfiles}
\begin{document}
\subsection{The Algebra-Geometry Dictionary}
%
\subsubsection{Hilbert's Nullstellensatz}
%
\begin{theorem}[The Weak Nullstellensatz Theorem]
If $k$ is an algebraically closed field and $I \subset k[x_1,\hdots ,x_n]$ is an ideal such that $\textbf{V}(I) = \emptyset$, then $I = k[x_1,\hdots ,x_n]$.
\end{theorem}
%
\begin{theorem}[Hilbert's Nullstellensatz]
If $k$ is an algebraically closed field, if $f_1,\hdots, f_s\in k[x_1,\hdots ,x_n]$, and $f\in \textbf{I}\big(\textbf{V}(f_1,\hdots, f_s)\big)$, then there is an $m\geq 1$ such that $f^m \in \langle f_1,\hdots, f_s \rangle$.
\end{theorem}
%
\subsubsection{Radical Ideals and the Ideal-Variety Correspondence}
%
\begin{theorem}
If $V$ is an affine variety, and if $f\in \textbf{I}(V)$, then $f^m\in \textbf{I}(V)$.
\end{theorem}

\begin{definition}
An ideal $I$ is said to be radical $f^m \in I$ implies $f\in I$ for some $m\geq 1$.
\end{definition}

\begin{theorem}
If $V$ is an affine variety, then $\textbf{I}(V)$ is a radical ideal.
\end{theorem}

\begin{definition}
If $I\subset k[x_1,\hdots ,x_n]$ is an ideal, then the radical of $I$ is the set $\sqrt{I} = \{f: f^m \in I, m \geq 1\}$.
\end{definition}

\begin{theorem}
If $I\subset k[x_1,\hdots ,x_n]$ is an ideal, then $\sqrt{I}$ is an ideal.
\end{theorem}

\begin{theorem}[The Strong Nullstellensatz]
If $k$ is an algebraically closed field, and if $I\subset k[x_1,\hdots ,x_n]$ is an ideal, then $\textbf{I}\big(\textbf{V}(I)\big) = \sqrt{I}$.
\end{theorem}

\begin{theorem}[The Ideal-Variety Correspondence]
If $k$ is a field, then the maps $\textrm{affine varieties} \overset{\textbf{I}}\rightarrow \textrm{ideals}$ and $\textrm{ideals} \overset{\textbf{V}}\rightarrow \textrm{affine varieties}$ are inclusion reversing and for any afffine variety $V$, $\textbf{V}\big(\textbf{I}(V)\big) = V$.
\end{theorem}

\begin{theorem}[Radical Membership Theorem]
If $k$ is a field and $I=\langle f_1,\hdots, f_s\rangle\subset k[x_1,\hdots ,x_n]$ is an ideal, then $f\in \sqrt{I}$ if and only if the constant polynomial $1$ belongs to $\langle f_1,\hdots, f_s, 1-yf\rangle$.
\end{theorem}

\begin{theorem}
If $f\in k[x_1,\hdots ,x_n]$, and $I = \langle f\rangle$, and if $f = f_1^{\alpha_1}\cdots f_s^{\alpha_s}$, then $\sqrt{I} = \langle f_1\cdots f_s\rangle$.
\end{theorem}

\begin{definition}
The reduction of a polynomial $f\in k[x_1,\hdots ,x_n]$ is the polynomial $f_{red}$ such that $\langle f_{red}\rangle = \sqrt{\langle f\rangle}$.
\end{definition}

\begin{definition}
A polynomial $f\in k[x_1,\hdots ,x_n]$ is said to be square free if $f = f_{red}$.
\end{definition}

\begin{definition}
If $f,g\in k[x_1,\hdots ,x_n]$, then $h\in k[x_1,\hdots ,x_n]$ is said to be the greatest common divisor of $f$ and $g$ if $f$ divides $f$ and $g$, and if $p$ is any polynomial that divides $f$ and $g$, then $p$ divides $h$.
\end{definition}

\begin{theorem}
If $k$ is a field such that $\mathbb{Q} \subset k$, and $I = \langle f\rangle$ for some $f\in k[x_1,\hdots ,x_n]$, then $\sqrt{I} = \langle f_{red}\rangle$, where $f_{red} = \frac{f}{GCD\big(f, \frac{\partial f}{\partial x_1}, \frac{\partial f}{\partial x_2}, \hdots, \frac{\partial f}{\partial x_n}\big)}$
\end{theorem}
%
\subsubsection{Sums, Products, and Intersections of Ideals}
%
\begin{definition}
If $I$ and $J$ are ideals of a the ring $k[x_1,\hdots ,x_n]$, then the sum of $I$ and $J$, denoted $I+J$, is the set $I+J = \{f+g: f\in I, g\in J\}$.
\end{definition}

\begin{theorem}
If $I$ and $J$ are ideals in $k[x_1,\hdots ,x_n]$, then $I+J$ is also an ideal in $k[x_1,\hdots ,x_n]$.
\end{theorem}

\begin{theorem}
If $I$ and $J$ are ideals in $k[x_1,\hdots ,x_n]$, then $I+J$ is the smallest ideal containing $I$ and $J$.
\end{theorem}

\begin{theorem}
If $f_1,\hdots, f_r \in k[x_1,\hdots ,x_n]$, then $\langle f_1,\hdots, f_r\rangle \sum_{k=1}^{r} \langle f_k\rangle$
\end{theorem}

\begin{theorem}
If $I$ and $J$ are ideals in $k[x_1,\hdots ,x_n]$, then $\textbf{V}(I+J) = \textbf{V}(I) \cap \textbf{V}(J)$.
\end{theorem}

\begin{definition}
If $I$ and $J$ are two ideals in $k[x_1,\hdots ,x_n]$, then their product, denoted $I\cdot J$, is defined to be the ideal generated by all polynomials $f\cdot g$, where $f\in I$, and $g\in J$.
\end{definition}

\begin{theorem}
If $I = \langle f_1,\hdots, f_r\rangle$ and $J = \langle g_1,\hdots, g_s\rangle$, then $I \cdot J$ is generated by the set of all products $\{f_ig_j:1\leq i\leq r, 1\leq j \leq s\}$
\end{theorem}

\begin{theorem}
If $I,J\subset k[x_1,\hdots ,x_n]$ are ideals, then $\textbf{V}(I\cdot J) = \textbf{V}(I)\cup \textbf{V}(J)$.
\end{theorem}

\begin{definition}
If $I,J\subset k[x_1,\hdots ,x_n]$ are ideals, then the intersection of $I$ and $J$, denoted $I\cap J$, is the set of polynomials in both $I$ and $J$.
\end{definition}

\begin{theorem}
If $I,J\subset k[x_1,\hdots ,x_n]$ are ideals, then $I\cap J$ is an ideal.
\end{theorem}
%
\subsubsection{Zariski Closure and Quotients of Ideals}
%
\begin{theorem}
If $S\subset k^n$, then the affine variety $\textbf{V}\big(\textbf{I}(S)\big)$ is the smallest affine variety that contains $S$.
\end{theorem}

\begin{definition}
The Zariski Closure of a subset $S$, denoted $\overline{S}$, of an affine space is the smallest affine algebraic variety containing the set. 
\end{definition}

\begin{theorem}
If $k$ is an algebraically closed field and $V = \vf\subset k^n$, then $\textbf{V}(I_{\ell})$ is the Zariski Closure of $\pi_{\ell}(V)$.
\end{theorem}

\begin{theorem}
If $V$ and $W$ are varieties such that $V\subset W$, then $W = V\cup \overline{\big(W\setminus V\big)}$.
\end{theorem}

\begin{definition}
If $I,J\subset k[x_1,\hdots ,x_n]$ are ideals, then $I:J$ is the set, $\{f\in k[x_1,\hdots ,x_n]: fg \in I\ \forall_{g\in J}\}$ and is called the ideal quotient of $I$ by $J$.
\end{definition}

\begin{theorem}
If $I,J\subset k[x_1,\hdots ,x_n]$ are ideals, then $I:J$ is an ideal.
\end{theorem}

\begin{theorem}
If $I,J\subset k[x_1,\hdots ,x_n]$ are ideals, then $\overline{\textbf{V}(I)\setminus \textbf{V}(J)} \subset \textbf{V}(I:J)$.
\end{theorem}

\begin{theorem}
If $I,J\subset k^n$ are affine varieties, then $\textbf{I}(V): \textbf{I}(W) = \textbf{I}(V\setminus)$
\end{theorem}

\begin{theorem}
If $I,J, K \subset k[x_1,\hdots ,x_n]$, then $I:k[x_1,\hdots ,x_n] = I$.
\end{theorem}

\begin{theorem}
If $I,J,K \subset k[x_1,\hdots ,x_n]$ are ideals, then $I\cdot J \subset K$ if and only if $I \subset K:J$
\end{theorem}

\begin{theorem}
If $I,J,K \subset k[x_1,\hdots ,x_n]$ are ideals, then $J\subset I$ if and only if $I:J = k[x_1,\hdots ,x_n]$
\end{theorem}

\begin{theorem}
If $I$ is an ideal, $g\in k[x_1,\hdots ,x_n]$, and if $\{h_1,\hdots, h_p\}$ is a basis of the ideal $I\cap \langle g \rangle$, then $\{h_1/g,\hdots, h_p/g\}$ is a basis of $I:\langle g\rangle$.
\end{theorem}
%
\subsubsection{Irreducible Varieties and Prime Ideals}
%
\begin{definition}
An affine variety $V\subset k^n$ is irreducible if there are no affine varieties $V_1, V_2$, such that $V = V_1\cup V_2$, $V_1,V_2\ne \emptyset$, and $V_1 \ne V, V_2 \ne V$.
\end{definition}

\begin{definition}
An ideal $I\subset k[x_1,\hdots ,x_n]$ is said to be prime if whenever $f,g\in k[x_1,\hdots ,x_n]$ and $fg\in I$, either $f\in I$ or $g\in I$.
\end{definition}

\begin{theorem}
If $V\subset k^n$ is an affine variety, then $V$ is irreducible if and only if $\textbf{I}(V)$ is a prime ideal.
\end{theorem}

\begin{definition}
An ideal $I\subset k[x_1,\hdots ,x_n]$ is said to be maximal if $I \ne k[x_1,\hdots ,x_n]$ and any ideal $J$ containing $I$ is such that either $J = I$ or $J = k[x_1,\hdots ,x_n]$.
\end{definition}

\begin{definition}
An ideal $I\subset k[x_1,\hdots ,x_n]$ is called proper if $I$ is not equal to $k[x_1,\hdots ,x_n]$.
\end{definition}

\begin{theorem}
If $k$ is a field and $I = \langle x_1-a_1,\hdots, x_n-a_n\rangle$ is and ideal where $a_1,\hdots, a_n \in k$, then $I$ is maximal.
\end{theorem}

\begin{theorem}
If $k$ is a field, then any maximal ideal is also a prime ideal.
\end{theorem}

\begin{theorem}
If $k$ is an algebraically closed field, then every maximal ideal of $k[x_1,\hdots ,x_n]$ is of the form $\langle x_1-a_1,\hdots, x_n-a_n\rangle$ for some $a_1,\hdots, a_n\in k$.
\end{theorem}

\begin{definition}
A primary decomposition of an ideal $I$ is an expression of $I$ as an intersection of primary ideals $I = \cap_{i=1}^{r} Q_{i}$.
\end{definition}

\begin{definition}
A primary decomposition of an ideal $I$ is said to be minimal $\sqrt{Q_i}$ are all distinct and $\cap_{j\ne i}Q_j \not \subset Q_i$
\end{definition}

\begin{theorem}
If $I,J$ are primary and $\sqrt{I}=\sqrt{J}$, then $I\cap J$ is primary.
\end{theorem}

\begin{theorem}[Lasker-Noether Theorem]
Every ideal $I \subset k[x_1,\hdots ,x_n]$ has a minimal primary decomposition.
\end{theorem}
\end{document}