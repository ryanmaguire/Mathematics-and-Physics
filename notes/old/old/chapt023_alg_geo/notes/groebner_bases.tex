\documentclass[../main.tex]{subfiles}
\begin{document}
\subsection{Groebner Bases}
%
\subsubsection{Introduction}

There are three problems we wish to address:

\begin{enumerate}
\item Does every Ideal $I\subset k[x_1,\hdots ,x_n]$ have a finite generating set?
\item Given $f\in k[x_1,\hdots ,x_n]$, and $I = \langle f_1,\hdots, f_s\rangle$, can we determine if $f\in I?$
\item For $f_1,\hdots, f_s \in k[x_1,\hdots ,x_n]$, can we determine what $\vf$ is?
\end{enumerate}

We've already solved this in the case of one variable, $n=1$. The case of $n\in \mathbb{N}$ where $f_1,\hdots, f_s$ are linear functions is the subject of linear algebra. Both the Eucldiean algorithm and the methods of linear algebra require a notion of ordering of terms. In the case of one variable, if $n>m$ we write $x^n>x^m$. In the case of linear algebra we usually write $x_n>x_{n-1}>\hdots > x_2 > x_1$. 
%
\subsubsection{Orderings on the Monomials in \texorpdfstring{$k[x_1,\hdots ,x_n]$}{kx}}
%
\begin{definition}
A monomial ordering on $k[x_1,\hdots, x_n]$ is any relation $\succ$ on $\mathbb{N}^n$ satisfying the following:
\begin{enumerate}
\item $\succ$ is a total ordering.
\item If $\alpha \succ \beta$ and $\gamma \in \mathbb{N}^n$, then $\alpha+\gamma \succ \beta + \gamma$.
\item $\succ$ is a well-ordering on $\mathbb{N}^n$. 
\end{enumerate}
\end{definition}
%
\begin{theorem}
An ordering $\prec$ on $\mathbb{N}^n$ is a well-ordering if and only if for any monotonically decreasing sequence $\{a_n\}_{n=1}^{\infty}$, there is an $N\in \mathbb{N}$ such that for all $n>N$, $a_n = a_N$.
\end{theorem}
\begin{proof}
For if $\prec$ is a well ordering, then $\{a_n\}_{n=1}^{\infty}$ contains a least element $x$. Suppose $a_n$ contains a strictly decreasing subsequence. But $\prec$ is a well ordering, and therefore $\{a_n\}_{n=1}^{\infty}$ contains a least element $x$. But again $\prec$ is a well ordering, and thus $\{a_n\}_{n=1}^{\infty} \setminus \{x\}$ contains a least element $y$. But then $x\prec y$, and $x$ is the least element of $\{a_n\}_{n=1}^{\infty}$. Therefore there is an $a_n$ such that $x\preceq a_n \preceq y$. But $a_n$ is strictly decreasing, and therefore $a_{n+1} \preceq x$, and thus $a_{n+2} \prec x$. But $x$ is the least element of $\{a_n\}_{n=1}^{\infty}$, a contradiction. Therefore $a_n$ contains no strictly increasing subsequence. Suppose every decreasing sequence eventually terminates. Let $E\subset \mathbb{N}^n$. Suppose there is no least element. Then we can construct a strictly decreasing sequence. But every decreasing sequence eventually terminates, a contradiction. Therefore, etc.
\end{proof}
%
\begin{definition}
If $\alpha,\beta \in \mathbb{N}^n$, then $\alpha$ is said to be lexicographically greater than $\beta$, denoted $\underset{Lex}{>}$, if the left-most entry of $\alpha-\beta$ is positive.
\end{definition}
%
\begin{theorem}
The Lexicographic Ordering is a monomial ordering.
\end{theorem}
%
\begin{definition}
The graded lexicographic ordering $\underset{GrLex}{>}$ on $\mathbb{N}^n$ is an ordering on $\mathbb{N}^n$ such that $\alpha \underset{GrLex}{>}\beta$ if and only if either $|\alpha|>|\beta|$, or $|\alpha| = |\beta|$ and $\alpha \underset{Lex}{>}\beta$.
\end{definition}
%
\begin{theorem}
The graded lexicographic ordering is a monomial ordering.
\end{theorem}
%
\begin{definition}
For $f=\sum_{\alpha} a_{\alpha} x^\alpha \in k[x_1,\hdots ,x_n]$, and $\prec$ a monomial ordering, the multidegree of $f$ is $\multideg(f) = \max\{\alpha\in\mathbb{N}^n: a_{\alpha} \ne 0\}$.
\end{definition}
%
\begin{definition}
For $f=\sum_{\alpha}a_\alpha x^\alpha \in k[x_1,\hdots ,x_n]$ and monomial order $>$, the leading coefficient of $f$ is $LC(f) =a_{\multideg(f)}\in k$
\end{definition}
%
\begin{definition}
For $f=\sum_{\alpha} a_{\alpha} x^\alpha \in k[x_1,\hdots ,x_n]$, and $\prec$ a monomial ordering, the leading monomial of $f$ is $\LM(f) = x^{\multideg(f)}$
\end{definition}
%
\begin{definition}
For $f=\sum_{\alpha} a_{\alpha} x^{\alpha} \in k[x_1,\hdots ,x_n]$, and $\prec$ a monomial ordering, the leading term of $f$ is $\LT(f) = \LC(f)\cdot \LM(f)$.
\end{definition}
%
\begin{theorem}
If $f,g\in k[x_1,\hdots ,x_n]$ are non-zero, then:
\begin{equation*}
    \multideg(fg) = \multideg(f)+\multideg(g)
\end{equation*}
\end{theorem}
%
\begin{theorem}
If $f,g\in k[x_1,\hdots ,x_n]$ are non-zero, and if $f+g \ne 0$, then $\multideg(f+g) \leq \max\{\multideg(f),\multideg(g)\}$. If $\multideg(f)\ne \multideg(g)$, then $\multideg(f+g) = \max\{\multideg(f),\multideg(g)\}$.
\end{theorem}
%
\begin{theorem}
If $>$ is a monomial ordering on $\mathbb{N}^n$, and $F = (f_1,\hdots, f_s)$ is an ordered $s-$tuple of polynomials in $k[x_1,\hdots ,x_n]$, then every $f\in k[x_1,\hdots ,x_n]$ can be written as $f = r+\sum_{k=1}^{s} a_k f_k$, where $a_k,r\in k[x_1,\hdots ,x_n]$, and either $r=0$ or $r$ is a linear combination, with coefficients in $k$, of monomials, none of which is divisible by any of $\LT(f_1),\hdots, \LT(f_s)$. We call $r$ the remainder of $f$ with respect to $F$.
\end{theorem}
%
\begin{definition}
An ideal $I\subset k[x_1,\hdots ,x_n]$ is a monomial ideal if there is a subset $A\subset \mathbb{N}^n$ such that $I$ consists of all polynomials which are finite sums of the form $\sum_{\alpha} h_{\alpha} x^\alpha$, where $h_{\alpha} \in k[x_1,\hdots ,x_n]$. 
\end{definition}
%
\begin{theorem}
If $I=\langle x^\alpha: \alpha \in A\}$ is a monomial ideal, then a monomial $x^\beta$ lies in $I$ if and only if $x^\beta$ is divisible by $x^\alpha$ for some $\alpha \in A$.
\end{theorem}
%
\begin{theorem}
If $I$ is a monomial ideal, and $f\in k[x_1,\hdots ,x_n]$, then the following are equivalent:
\begin{enumerate}
\item $f\in I$
\item Every term of $f$ lies in $I$.
\item $f$ is a $k-$linear combination of the monomials in $I$.
\end{enumerate}
\end{theorem}
%
\begin{theorem}[Dickson's Lemma]
If $I=\langle x^\alpha: \alpha \in A\rangle$ is a monomial ideal, then $I$ can be written as $\langle x^{\alpha(1)}, \hdots, x^{\alpha(s)}\rangle$, where $\alpha(1),\hdots, \alpha(s) \in A$. 
\end{theorem}
%
\begin{theorem}
If $>$ is a relation on $\mathbb{N}^n$ such that $>$ is a total ordering and for $\alpha>\beta$ and $\gamma\in \mathbb{N}^n$, $\alpha+\gamma>\beta+\gamma$, then $>$ is a well-ordering if and only if for all $\alpha \in \mathbb{N}^n$, $\alpha \geq 0$.
\end{theorem}
%
\subsubsection{The Hilbert Basis Theorem and Groebner Bases}
%
\begin{definition}
For a non-zero ideal $I\subset k[x_1,\hdots ,x_n]$, $\LT(I)$ is the set of leading terms of elements of $I$. $\langle \LT(I)\rangle$ is the ideal generated by this set.
\end{definition}

\begin{theorem}
If $I\subset k[x_1,\hdots ,x_n]$ is an ideal, then $\langle \LT(I)\rangle$ is a monomial ideal.
\end{theorem}


\begin{theorem}
If $I\subset k[x_1,\hdots ,x_n]$ is an ideal, then there are $g_1,\hdots, g_t\in I$ such that $\langle \LT(I)\rangle = \langle \LT(g_1),\hdots, \LT(g_t)\rangle$
\end{theorem}

\begin{theorem}[Hilbert Basis Theorem]
Every ideal $I\subset k[x_1,\hdots ,x_n]$ has a finite generated set.
\end{theorem}

\begin{definition}
For a monomial order $>$, a finite subset $G=\{g_1,\hdots, g_t\}$ of an ideal $I$ is said to be a Groebner Basis if $\langle \LT(g_1),\hdots, \LT(g_t)\rangle = \langle \LT(I)\rangle$
\end{definition}

\begin{theorem}
If $>$ is a monomial order, then every non-zero ideal $I\subset k[x_1,\hdots ,x_n]$ has a Groebner basis.
\end{theorem}

\begin{theorem}
If $I\subset k[x_1,\hdots ,x_n]$ is a non-zero ideal and $G$ is a Groebner Basis, then $G$ is also a generated set of $I$.
\end{theorem}

\begin{theorem}[The Ascending Chain Condition]
If $I_n$ is a sequence of ideals such that $I_{n}\subset I_{n+1}$, then there is an $N\in \mathbb{N}$ such that for all $n>N$, $I_n = I_N$.
\end{theorem}

\begin{definition}
If $I\subset k[x_1,\hdots ,x_n]$ is an ideal, then $\textbf{V}(I)$ is the set $\{\alpha \in k^n: \forall_{f\in I},f(\alpha) = 0\}$
\end{definition}

\begin{theorem}
If $I\subset k[x_1,\hdots ,x_n]$ is an ideal, then $\textbf{V}(I)$ is an affine variety.
\end{theorem}

\begin{theorem}
If $I = \langle f_1,\hdots, f_s\rangle$, then $\textbf{V}(I) = \vf$.
\end{theorem}
%
\subsubsection{Properties of Groebner Bases}
%
\begin{theorem}
If $G=\{g_1,\hdots, g_t\}$ is a Groebner basis of $I\subset k[x_1,\hdots ,x_n]$ and $f\in k[x_1,\hdots ,x_n]$, then there is a unique $r\in k[x_1,\hdots ,x_n]$ such that $r$ is not divisible by any of $\LT(g_1),\hdots, \LT(g_t)$, and there is a $g\in I$ such that $f = g+r$. 
\end{theorem}

\begin{notation}
We write $\overline{f}^{F}$ for the remainder on division of $f$ by the ordered $s-$tupbe $F = (f_1,\hdots, f_s)$.
\end{notation}

\begin{definition}
If $f,g\in k[x_1,\hdots ,x_n]$ are non-zero polynomials, $\multideg(f) = \alpha$, $\multideg(g) = \beta$, and if $\gamma = (\gamma_1,\hdots, \gamma_n)$, where $\gamma_k = \max\{\alpha_k,\beta_k\}$, then $x^y$ is the least common multiple of $\LM(f)$ and $\LM(g)$, denoted $x^y = \LCM(\LM(f),\LM(g))$.
\end{definition}

\begin{definition}
If $f,g\in k[x_1,\hdots ,x_n]$ are non-zero, then the $S-$polynomial of $f$ and $g$ is $S(f,g) = \frac{x^y}{\LT(f)}f - \frac{x^y}{\LT(g)}g$
\end{definition}

\begin{theorem}[Buchberger's Criterion]
If $I$ is a polynomial ideal, then a basis $G=\{g_1,\hdots, g_t\}$ for $I$ is a Groebner basis for $I$ if and only if for all pairs $i\ne j$, the remainder on division of $S(g_i,g_j)$ by $G$ is zero.
\end{theorem}
\end{document}