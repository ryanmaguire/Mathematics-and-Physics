\documentclass[../main.tex]{subfiles}
\begin{document}
\subsection{Polynomials and Rational Functions on a Variety}
%
\subsubsection{Polynomial Mappings}
%
\begin{definition}
If $V\subset k^m$, $W\subset k^n$ are affine varieties, a function $\phi:V\rightarrow W$ is said to be a polynomial mapping if there exist polynomials $f_1,\hdots, f_n\in k[x_1,\hdots, x_m]$ such that $\phi(a_1,\hdots, a_m) = \big(f_1(a_1,\hdots, a_m),\hdots, f_n(a_1,\hdots, a_m)\big)$ for all $(a_1,\hdots, a_m) \in V$. We say that $(f_1,\hdots, f_n)$ represents $\phi$.
\end{definition}

\begin{theorem}
If $V\subset k^m$ is an affine variety, then $f,g\in k[x_1,\hdots, x_m]$ represent the same polnyomial on $V$ if and only if $f-g \in \textbf{I}(V)$.
\end{theorem}

\begin{theorem}
If $V\subset k^m$ is an affine variety, then $(f_1,\hdots, f_n)$ and $(g_1,\hdots, g_n)$ represent the same polynomial mapping if and only if $f_i-g_i \in \textbf{I}(V)$ for $1\leq i \leq n$.
\end{theorem}

\begin{notation}
The set of polynomial mappings from $V$ to $k$ is denoted $k[V]$.
\end{notation}

\begin{theorem}
If $V\subset k^n$ is an affine variety, the the following are equivalent:
\begin{enumerate}
\item $V$ is irreducible.
\item $\textbf{I}(V)$ is a prime ideal.
\item $k[V]$ is an integral domain.
\end{enumerate}
\end{theorem}
%
\subsubsection{Quotients of Polynomial Rings}
%
\begin{definition}
If $I\subset k[x_1,\hdots ,x_n]$ is an ideal, if $f,g\in k[x_1,\hdots ,x_n]$, then $f$ and $g$ are congruent modulo $I$, denoted $f \equiv g \mod I$, if $f-g \in I$.
\end{definition}

\begin{theorem}
If $I\subset k[x_1,\hdots ,x_n]$ is an ideal, then the congruence modulo $I$ is an equivalence relation on $k[x_1,\hdots ,x_n]$.
\end{theorem}

\begin{theorem}
There exists a bijection from the set of distinct polynomial functions $\phi:V\rightarrow k$ and the set of equivalence classes of polynomials under congruence modulo $\textbf{I}(V)$.
\end{theorem}

\begin{definition}
The quotient of $k[x_1,\hdots ,x_n]$ modulo $I$, denoted $k[x_1,\hdots ,x_n]/I$, is the set of equivalence classes for congruence modulo $I$.
\end{definition}

\begin{theorem}
If $I\subset k[x_1,\hdots ,x_n]$ is an ideal, then $k[x_1,\hdots ,x_n]/I$ is a commutative ring under the sum and product operations.
\end{theorem}

\begin{definition}
If $R,S$ are commutative rings, then $\phi:R\rightarrow S$ is a ring isomorphism if $\phi$ is a bijection and:
\begin{enumerate}
\item For all $a,b\in R$, $\phi(a+b) = \phi(a)+\phi(b)$
\item For all $a,b\in R$, $\phi(ab) = \phi(a)\phi(b)$.
\end{enumerate}
\end{definition}

\begin{theorem}
If $I\subset k[x_1,\hdots ,x_n]$ is an ideal, then there is a bijection between the ideals in the quotient ring $k[x_1,\hdots ,x_n]/I$ and the ideals of $k[x_1,\hdots ,x_n]$ that contain $I$.
\end{theorem}

\begin{theorem}
If $I\subset k[x_1,\hdots ,x_n]$ is an ideal, then every ideal of $k[x_1,\hdots ,x_n]/I$ is finitely generated.
\end{theorem}
%
\subsubsection{The Coordinate Ring of an Affine Variety}
%
\begin{definition}
The coordinate ring of an affine variety $V\subset k^n$ is the ring $k[V]$.
\end{definition}

\begin{definition}
If $V\subset k^n$ is an affine variety, and if $J = \langle \phi_1,\hdots, \phi_s\rangle \subset k[V]$, then $\textbf{V}_{V}(J) = \{x\in V:\forall_{\phi \in J}, \phi(x) = 0 \}$ is called the subvariety of $V$.
\end{definition}

\begin{theorem}
If $V\subset k^n$ is an affine variety and if $J\subset k[V]$ is an ideal, then $W = \textbf{V}_{V}(J)$ is an affine variety in $k^n$ contained in $V$.
\end{theorem}

\begin{theorem}
If $V\subset k^n$ is an affine variety, and if $W\subset V$, then $\textbf{V}_{V}(W)$ is an ideal of $k[V]$.
\end{theorem}

\begin{definition}
If $V$ is an irreducible variety in $k^n$, then the function field, denoted $QF(k[V])$, on $V$ is the quotient field of $k[V]$.
\end{definition}

\begin{definition}
If $V\subset k^m$ and $W\subset k^n$ are irreducible affine varieties, then a rational mapping is a function $\phi$ such that $\phi(x_1,\hdots, x_m) = \bigg(\frac{f_1(x_1,\hdots, x_m)}{g_1(x_1,\hdots, x_m)}, \hdots, \frac{f_n(x_1,\hdots, x_m)}{g_n(x_1,\hdots, x_m)}\bigg)$.
\end{definition}

\begin{theorem}
Two rational mappings $\phi,\psi:V\rightarrow W$ are equal if and only if there is a proper subvariety $V'\subset V$ such that $\phi$ and $\psi$ are defined on $V\setminus V'$ and $\phi(p) = \psi(p)$ for all $p \in V\setminus V'$.
\end{theorem}

\begin{theorem}[The Closure Theorem]
If $k$ is an algebraically closed field, $V=\textbf{V}(I)$, $V\ne \emptyset$, then there is an affine variety $W\underset{Proper}\subset \textbf{V}(I_{\ell})$ such that $\textbf{V}(I_{\ell})\setminus W \subset \pi_{\ell}(V)$.
\end{theorem}
\end{document}