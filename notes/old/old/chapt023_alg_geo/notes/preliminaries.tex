\documentclass[../main.tex]{subfiles}
\begin{document}
\subsection{Preliminaries}
%
\subsubsection{Groups}
%
\begin{definition}
A binary operation on a set $S$ is a function $*:S\times S \rightarrow S$.
\end{definition}
%
\begin{definition}
A group is a set $G$ and a binary operation $*$ defined on $G$ for which the following are true:
\begin{enumerate}
\item For all $a,b,c\in G$, $a*(b*c) = (a*b)*c$ \hfill [Associativity]
\item There is an $e\in G$ such that for all $a\in G$, $a*e=e*a = a$ \hfill [Existence of Neutral Element]
\item For all $a\in G$, there is a $b\in G$ such that $a*b = b*a = e$. We write $b=a^{-1}$ \hfill [Existence of Inverse Elements]
\end{enumerate}
We denote a group by $\langle G,*\rangle$.
\end{definition}
%
\begin{theorem}
If $\langle G, *\rangle$ is a group with neutral element $e$, then $e$ is unique.
\end{theorem}
%
\begin{theorem}
If $\langle G,*\rangle$ is a group and $a\in G$, then $a^{-1}$ is unique.
\end{theorem}
%
\begin{theorem}
If $p$ is prime and $\mathbb{F}_p = \{0,1,\hdots, p-1\}$, then $\mathbb{F}_p\setminus \{0\}$ is a group under multiplication modulo p.
\end{theorem}
%
\begin{remark}
$\mathbb{Z}$ is NOT a group under the operation of multiplication $(\cdot)$. Multiplicative inverses may not be integers.
\end{remark}
%
\begin{definition}
An injective function $f:A\rightarrow B$ is a function such that $\forall_{a,b\in A}$, $f(a) = f(b)$ if and only if $a=b$.
\end{definition}
%
\begin{definition}
A surjective functions $f:A\rightarrow B$ is a function such that $\forall_{b\in B}, \exists_{a\in A}: f(a) = b$.
\end{definition}
%
\begin{definition}
A bijective function $f:A\rightarrow B$ is a function that is both injective and surjective.
\end{definition}
%
\begin{definition}
A permutation on a set $S$ is a bijective function $\sigma:S\rightarrow S$.
\end{definition}
%
\begin{definition}
The composition of functions $f:A\rightarrow B$ and $g:B\rightarrow C$ is $g\circ f:A\rightarrow C$ defined by $x\mapsto g(f(x))$.
\end{definition}
%
\begin{definition}
In a group $\langle G,*\rangle$, a subgroup is a set $H\subset G$ such that the following are true:
\begin{enumerate}
\item $e\in H$ \hfill [Existence of Neutral Element]
\item If $a,b\in H$, then $a*b \in H$ \hfill [Closure Under Binary Operation]
\item If $a\in H$, then $a^{-1}\in H$ \hfill [Existence of Inverse Elements]
\end{enumerate}
\end{definition}
%
\subsubsection{Fields and Rings}
%
\begin{definition}
A field is a set $k$ with two binary operations $+$ and $\cdot$ such that the following are true:
\begin{enumerate}
\item For all $a,b,c\in k$, $a+(b+c)=(a+b)+c$ \hfill [Associativity of Addition]
\item For all $a,b,c\in k$, $a\cdot(b\cdot c) = (a\cdot b)\cdot c$ \hfill [Associativity of Multiplication]
\item For all $a,b\in k$, $a+b=b+a$ \hfill [Commutativity of Addition]
\item For all $a,b\in k$, $a\cdot b = b\cdot a$ \hfill [Commutativity of Multiplication]
\item There is a $0 \in k$ such that for all $a\in k$, $a+0=0+a = a$ \hfill [Existence of Additive Identity]
\item There is a $1\in k$ such that for all $a\in k$, $1\cdot a=a\cdot 1 = a$ \hfill [Existence of Multiplicative Identity]
\item For all $a\in k$ there is a $b\in k$ such that $a+b=0$ \hfill [Existence of Additive Inverse]
\item For all $a\in k$, $a\ne 0$, there is a $b\in k$ such that $a\cdot b = 1$ \hfill [Existence of Multiplicative Inverses]
\item For all $a,b,c\in k$, $a\cdot(b+c) = a\cdot b + a\cdot c$ \hfill [Multiplication Distributes Over Addition]
\end{enumerate}
\end{definition}
%
\begin{remark}
We usually omit the multiplication symbol $\cdot$ and just write $ab$ instead of $a\cdot b$.
\end{remark}
%
\begin{remark}
If $k$ is a field, then $\langle k, + \rangle$ is an Abelian group.
\end{remark}
%
\begin{theorem}
If $k$ is a field and $a\in k$, then $a\cdot 0 = 0$.
\end{theorem}
%
\begin{corollary}
If $k$ is a field and $0=1$, then every element of $k$ is $0$.
\end{corollary}
%
\begin{remark}
This makes the "Zero Field," rather boring.
\end{remark}
%
\begin{theorem}
If $-1$ is the additive inverse of $1$, then $(-1)^2 = 1$.
\end{theorem}
%
\begin{definition}
A ring is a set $R$ with two binary operations $+$ and $\cdot$ such that the following are true:
\begin{enumerate}
\item For all $a,b,c\in k$, $a+(b+c)=(a+b)+c$ \hfill [Associativity of Addition]
\item For all $a,b,c\in k$, $a\cdot(b\cdot c) = (a\cdot b)\cdot c$ \hfill [Associativity of Multiplication]
\item For all $a,b\in k$, $a+b=b+a$ \hfill [Commutativity of Addition]
\item There is a $0 \in k$ such that for all $a\in k$, $a+0=0+a = a$ \hfill [Existence of Additive Identity]
\item For all $a\in k$ there is a $b\in k$ such that $a+b=0$ \hfill [Existence of Additive Inverse]
\item For all $a,b,c\in k$, $a\cdot(b+c) = a\cdot b + a\cdot c$ and $(b+c)\cdot a = b\cdot a + c\cdot a$ \hfill [Multiplication Distributes Over Addition]
\end{enumerate}
\end{definition}
%
\begin{definition}
A commutative ring with unity is a ring such that multiplication is commutative and there exists a multiplicative identity, usually denoted as $1$.
\end{definition}
%
\begin{remark}
In both a ring and in a field, $+$ is usually called addition and $\cdot$ is usually called multiplication.
\end{remark}
%
\begin{corollary}
If $R$ is a ring and $a\in R$, then $a\cdot 0 = 0\cdot a = 0$.
\end{corollary}
%
\begin{definition}
An integral domain is a commutative ring $R$ such that $\forall_{a,b\in R}:ab=0$, either $a=0$ or $b=0$.
\end{definition}
%
\begin{definition}
In a ring $R$, a divisor of zero is an element $a\in R$ such that $\exists_{b\in R,b\ne 0}$ where $ab=0$.
\end{definition}
%
\begin{theorem}
If $R$ is a ring, then $a\in R$ is a divisor of zero if and only if $f:R\rightarrow R$, $f(x) = a x$ is not injective.
\end{theorem}
%
\begin{theorem}
Any field $k$ is an integral domain.
\end{theorem}
%
\begin{definition}
In a commutative ring $R$, an ideal is a subset $I\subset R$ such that the following are true:
\begin{enumerate}
\item $0\in I$ \hfill [Existence of Additive Inverse]
\item $\forall_{a,b\in I}$, $a+b\in I$ \hfill [Closure Under Addition]
\item $\forall_{a\in I,b\in R}$, $a b \in I$ \hfill [Absorption Property]
\end{enumerate}
\end{definition}
%
\subsubsection{Determinants}
%
The elementary definitions from linear algebra are presumed (i.e. What a matrix is, what a determinant is, etc.) Recall we defined $S_n$ to be the set of all permutations of the set $\mathbb{Z}_n$.

\begin{definition}
For any $\sigma \in S_n$, the $n\times n$ matrix that is obtained by permuting the columns of the identity matrix $I$ by $\sigma$ is called the permutation matrix of $\sigma$.
\end{definition}

\begin{example}
Consider the permutation on $\mathbb{Z}_3$ defined by the cycle $1\rightarrow 3 \rightarrow 2 \rightarrow 1$. We can make this a matrix equation as follows: $\begin{bmatrix} 0 & 0 & 1 \\ 1 & 0 & 0 \\ 0 & 1 & 0 \end{bmatrix} \begin{bmatrix} 1 \\ 2 \\ 3 \end{bmatrix} = \begin{bmatrix} 3 \\ 1 \\ 2 \end{bmatrix}$ The matrix $\begin{bmatrix} 0 & 0 & 1 \\ 1 & 0 & 0 \\ 0 & 1 & 0 \end{bmatrix}$ is obtained by permuting the columns of the identity matrix $I = \begin{bmatrix} 1 & 0 & 0 \\ 0 & 1 & 0 \\ 0 & 0 & 1 \end{bmatrix}$ by $\sigma$.
\end{example}

\begin{definition}
For any $\sigma\in S_n$, then the sign of $\sigma$ is defined as $\sgn(\sigma) = \det(P_{\sigma})$.
\end{definition}

\begin{remark}
From the way $P_{\sigma}$ is defined, $\sgn(\sigma) = \det(P_{\sigma}) = \pm 1$, depending on $\sigma$.
\end{remark}

\begin{theorem}
If $A=(a_{ij})$ is an $n\times n$ matrix, then $\det(A) = \underset{\sigma \in S_n}\sum \sgn(\sigma) \prod_{k=1}^{n} a_{k\sigma(k)}$.
\end{theorem}
\end{document}