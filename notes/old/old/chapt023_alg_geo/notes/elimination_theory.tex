\documentclass[../main.tex]{subfiles}
\begin{document}
\subsection{Elimination Theory}
%
\subsubsection{The Elimination and Extension Theorems}
%
\begin{definition}
If $I = \langle f_1,\hdots, f_s\rangle \subset k[x_1,\hdots ,x_n]$, the $\ell-$th elimination ideal, denoted $I_{\ell}$, is the ideal defined as $I_{\ell} = I \cap k[x_{\ell+1},\hdots, x_n]$.
\end{definition}

\begin{theorem}
For $\ell \in \mathbb{Z}_{n-1}$, if $I = \langle f_1,\hdots, f_s\rangle\subset k[x_1,\hdots ,x_n]$ is an ideal, then $I_{\ell}$ is an ideal of $k[x_1,\hdots ,x_n]$.
\end{theorem}

\begin{theorem}[The Elimination Theorem]
If $I\subset k[x_1,\hdots ,x_n]$ is an ideal and $G$ is a Groebner Basis of $I$ with respect to the lexicographic ordering $x_1>x_2>\hdots > x_n$, then for all $\ell \in \mathbb{Z}_{n}$, $G_{\ell} = G\cap k[x_{\ell+1},\hdots, x_n]$ is a Groebner Basis of $I_{\ell}$.
\end{theorem}

\begin{theorem}[The Extension Theorem]
If $I = \langle f_1,\hdots, g_s\rangle \subset \mathbb{C}[x_1,\hdots ,x_n]$, and if $I_1$ is the first elimination ideal of $I$, and if for all $i\in \mathbb{Z}_s$ $f_i = g(x_2,\hdots, x_n)x_1^{N_i}+h$, where the degree of the $x_1$ component of $h$ is less than $N_i$, and if $(a_2,\hdots, a_n)\notin \textbf{V}(g_1,\hdots, g_s)$, then there is an $a_1 \in \mathbb{C}$ such that $(a_1,\hdots, a_n)\in \textbf{V}(I)$
\end{theorem}

\begin{remark}
The requirement that we work in $\mathbb{C}$ is crucial. This theorem does not hold in $\mathbb{R}$. 
\end{remark}

\begin{theorem}
If $I = \langle f_1,\hdots, f_s\rangle \subset \mathbb{C}[x_1,\hdots, x_n]$ if for some $i$, $f_i$ is of the form $f_i = cx_1^N + g(x_1,\hdots, x_n)$, where the degree of the $x_1$ term in $g$ is less than $N$, and $c \ne 0$, and if $(a_2,\hdots, a_n) \in \textbf{V}(I_{1})$, then there is an $a_1 \in \mathbb{C}$ such that $(a_1,\hdots, a_n) \in \textbf{V}(I)$.
\end{theorem}
%
\subsubsection{The Geometry of Elimination}
%
\begin{definition}
The projectiom map $\pi_{\ell}: \mathbb{C}^n \rightarrow \mathbb{C}^{n-\ell}$ is defined as $\pi_{\ell}(a_1,\hdots, a_n) = (a_{\ell+1},\hdots, a_n)$.
\end{definition}

\begin{theorem}
If $V=\vf \subset \mathbb{C}^n$, and $I_{\ell}$ is the $\ell-$th elimination ideal of $\langle f_1,\hdots, f_s\rangle$, then $\pi_{\ell}(V) \subset \textbf{V}(I_{\ell})$
\end{theorem}

\begin{theorem}
If $V = \vf \subset \mathbb{C}^n$, and $G_{\ell}$ is as defined in the extension theorem, then $\textbf{V}(I_{\ell}) = \pi_{\ell}(V)\cup G_{\ell}$
\end{theorem}

\begin{theorem}[The First Closure Theorem]
If $V = \vf \subset \mathbb{C}^n$ and $I_{\ell}$ is the $\ell-$th elimination ideal of $\langle f_1,\hdots, f_s\rangle$, then $\textbf{V}(I_{\ell})$ is the smallest affine variety containing $\pi_{\ell}(V)\subset \mathbb{C}^{n-\ell}$.
\end{theorem}

\begin{theorem}[The Second Closure Theorem]
If $V = \vf \subset \mathbb{C}^n$, $V\ne \emptyset$, and if $I_{\ell}$ is the $\ell-$th elimination ideal of $\langle f_1,\hdots, f_s\rangle$, then there is an affine variety $W\underset{Proper}{\subset} \textbf{V}(I_{\ell})$ such that $\textbf{V}(I_{\ell})\setminus W \subset \pi_{\ell}(V)$.
\end{theorem}

\begin{theorem}
If $V = \vf\subset \mathbb{C}^n$ and if for some $i$, $f_i$ is of the form $f_i = cx_1^N + g$, where the $x_1$ terms in $g$ are of degree less than $N$, and $c\ne 0$, then $\pi_{1}(V) = \textbf{V}(I_{1})$.
\end{theorem}
%
\subsubsection{Implicitization}
%
\begin{definition}
A polynomial parametrization is a finite set of equations:
%
\begin{equation*}
    x_k = f_k(t_1,\hdots, t_m)\in k[t_1,\hdots, t_m]    
\end{equation*}
%
The function $F:k^m\rightarrow k^n$ is the image defined by $(t_1,\hdots, t_m)\mapsto (x_1,\hdots, x_n)$
\end{definition}

\begin{theorem}[The Polynomial Implicitization Theorem]
If $k$ is an infinite field and $F:k^m\rightarrow k^n$ is a function determined by some polynomial parametrization, and if $I$ is an ideal $I = \langle x_1-f_1,\hdots, x_n - f_n\rangle \subset k[t_1,\hdots, t_m, x_1,\hdots, x_n]$, then $\textbf{V}(I_m)$ is the smallest variety in $k^n$ containing $F(k^n)$, where $I_m$ is the $m^{th}$ elimination ideal.
\end{theorem}

\begin{definition}
A rational parametrization is a set of $n$ equation $x_k = \frac{f_k(t_1,\hdots, t_m)}{g_k(t_1,\hdots, t_m)}$, where $f_1,\hdots, f_n, g_1,\hdots, g_n\in k[t_1,\hdots, t_m]$
\end{definition}

\begin{theorem}[Rational Implicitization]
If $k$ is an infinite field, $f_k, g_k, k=1,2,\hdots, n$ are a rational parametrization, $W = \vg$, and if $F:k^m\setminus W \rightarrow k^n$ is the function determined by the rational parametrization, if $J = \langle g_1 x_1 - g_1,\hdots, g_n x_n - g_n, 1-gy\rangle \subset k[y,t_1,\hdots, g_m, x_1,\hdots, x_n]$, where $g = g_1\cdots g_n$, and if $J_{m+1}$ is the $(m+1)^{th}$ elimination ideal, then $\textbf{V}(J_{m+1})$ is the smallest variety in $k^n$ containing $F(K^m\setminus W)$.
\end{theorem}
%
\subsubsection{Singular Points and Envelopes}
%
\begin{definition}
A singular point on an affine variety $\textbf{V}(f)$ is a point $x\in k$ such that there exists no tangent line at $x$.
\end{definition}

\begin{remark}
For curves in the plane, this usually happens when either the curve intersects itself or has a kink in it.
\end{remark}

\begin{definition}
If $k\in \mathbb{N}$, if $(a,b)\in \textbf{V}(f)$, and if $L$ is a line through $(a,b)$, then $L$ meets $\textbf{V}(f)$ with multiplicity $k$ at $(a,b)$ if $L$ can be linearly parametrized in $x$ and $y$ so that $t=0$ is a root of multiplicity $k$ of the polynomial $g(t) = f(a+ct,b+dt)$.
\end{definition}

\begin{theorem}
If $f\in k[x,y]$, $(a,b) \in \textbf{V}(f)$, and if $\nabla f(a,b) \ne (0,0)$, then there is a unique line through $(a,b)$ which meets $\textbf{V}(f)$ with multiplicity $k\geq 2$.
\end{theorem}

\begin{theorem}
If $f\in k[x,y]$, $(a,b) \in \textbf{V}(f)$, and if $\nabla f(a,b) = 0$, then every line through $(a,b)$ meets $\textbf{V}(f)$ with multiplicity $k \geq 2$.
\end{theorem}

\begin{definition}
If $f\in k[x,y]$, $(a,b) \in \textbf{V}(f)$, and if $\nabla f(a,b) \ne (0,0)$, then the tangent line of $\textbf{V}(f)$ at $(a,b)$ is the unique line through $(a,b)$ with multiplicity $k\geq 2$. We say that $(a,b)$ is a non-singular point of $\textbf{V}(f)$.
\end{definition}

\begin{definition}
If $f\in k[x,y]$, $(a,b) \in \textbf{V}(f)$, and if $\nabla f(a,b) = (0,0)$, then we say that $(a,b)$ is a singular point of $\textbf{V}(f)$.
\end{definition}

\begin{definition}
If $\textbf{V}(F_t)$ is a family of curves in $\mathbb{R}^2$, its envelope consists of all points $(x,y) \in \mathbb{R}^2$ such that $F(x,y,t) = 0$ and $\frac{\partial}{\partial t}F(x,y,t) = 0$ for some $t\in \mathbb{R}$.
\end{definition}
%
\subsubsection{Unique Factorization and Resultants}
%
\begin{definition}
If $k$ is a field, then a polynomial $f\in k[x_1,\hdots ,x_n]$ is said to be irreducible if $f$ is non-constant and is not the product of two non-constant polynomials in $k[x_1,\hdots ,x_n]$.
\end{definition}

\begin{theorem}
Every non-constant polynomial $f\in k[x_1,\hdots ,x_n]$ can be written as a product of polynomials which are irreducible over $k$
\end{theorem}

\begin{theorem}
If $f,g\in k[x_1,\hdots ,x_n]$ have positive degree in $x_1$, then $f$ and $g$ have a common factor in $k[x_1,\hdots ,x_n]$ of positive degree in $x_1$ if and only if they have a common factor in $k(x_2,\hdots, x_n)[x_1]$
\end{theorem}

\begin{theorem}
Every non-constant $f\in k[x_1,\hdots ,x_n]$ can be written as a product $f = f_1\cdots f_r$ of irreducibles of $k$. Furthermore, if $f = g_1\cdots g_s$, where the $g_k$ are irreducible, then $r=s$ and there are constants $\alpha_1,\hdots, \alpha_n$ such that $\{f_1,\hdots, f_r\} = \{\alpha_1 g_1, \hdots, \alpha_r g_r\}$.
\end{theorem}

\begin{theorem}
If $f,g \in k[x]$ are polynomials of degree $\ell>0$ and $m>0$, respectively, then $f$ and $g$ have a common factor if and only if there are polynomials $A,B\in k[x]$ such that $A$ and $B$ are not both zero, $A$ has degree at most $m-1$ and $B$ has degree at most $\ell-1$, and $Af+Bg = 0$.
\end{theorem}

\begin{definition}
If $f = a_0 x^{\ell} +\hdots + a_{\ell}$ and $g = b_0 x^m + \hdots b_m$, then the Sylvester Matrix is defined as follows:\\$
\left({\begin{matrix} a_0 & 0 & 0 & 0 & b_0 & 0 & 0 & 0 \\ a_1 & a_0 & 0 & 0 & b_1 & b_0 & 0 & 0 \\ \vdots & \vdots & \ddots & 0 & \vdots & \vdots & \ddots & 0 \\ \vdots & \vdots & \ddots & a_{0} & \vdots & \vdots & \ddots & b_0 \\ a_{\ell} & \hdots & \hdots & a_{1} & b_{m} & \hdots & \hdots & 0 \\ 0 & a_{\ell} & \hdots & \vdots & 0 & b_{m} & \hdots & \vdots\\ 0 & 0 & \ddots & 0 & 0 & \hdots & \ddots & 0 \\ 0 & \hdots & \hdots & a_{\ell} & 0 & \hdots & \hdots & b_{m} \end{matrix}}\right)$ 
\end{definition}

\begin{theorem}
If $f,g \in k[x]$, then the resultant of $f$ and $g$ is the determinant of the Sylvester matrix of $f$ and $g$.
\end{theorem}

\begin{theorem}
If $f,g\in k[x]$ are polynomials of positive degree, then the resultant of $f$ and $g$ is an integer polynomial in the coefficients of $f$ and $g$.
\end{theorem}

\begin{theorem}
If $f,g\in k[x]$ are polynomials of positive degree, then $f$ and $g$ have a common factor if and only if their resultant is zero.
\end{theorem}

\begin{theorem}
If $f,g\in k[x]$ are of positive degree, then there are polynomials $A,B \in k[x]$ such that $Af + Bg = Resultant(f,g)$
\end{theorem}
\end{document}