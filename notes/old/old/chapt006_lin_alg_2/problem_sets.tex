\documentclass[../main.tex]{subfiles}
\begin{document}
\section{Problem Sets}
%
\subsection{Problem Set I}
%
\begin{problem}
Find the point on the line $y=4x$ which is closest to the point $(2,5)$.
\end{problem}
\begin{proof}[Solution 1]
Given a vector $\mathbf{v}$ that is parallel to the line $y$, we know that the vector $\mathbf{w}$ from $(2,5)$ to the point $(x,y)$ that minimizes the distance from $y=4x$ to the point $(2,5)$ will satisfy $\langle \mathbf{v}, \mathbf{w}\rangle = 0$. That is:
\begin{equation*}
    \big\langle (1,4), (2-x,5-y)\big\rangle = 0\Rightarrow 2-x+4(5-y) = 0 \Rightarrow 22 - x - 4 y = 0    
\end{equation*}
But $y = 4x$, and thus $22-17x = 0 \Rightarrow x= \frac{22}{17}$. The point of least distance is $\frac{22}{17}(1,4)$.
\end{proof}
\begin{proof}[Solution 2]
This point is the projection of the vector $(2,5)^T$ onto $(1,4)^T$. That is:
\begin{equation*}
    \mathbf{P} = \frac{\begin{bmatrix}2 & 5 \end{bmatrix} \begin{bmatrix} 1 \\ 4 \end{bmatrix}}{\begin{bmatrix} 1 & 4 \end{bmatrix} \begin{bmatrix} 1 \\ 4 \end{bmatrix}} \begin{bmatrix} 1 \\ 4 \end{bmatrix} = \frac{22}{17} \begin{bmatrix} 1 \\ 4\end{bmatrix}
\end{equation*}
\end{proof}
%
\begin{problem}
Show that $\mathbf{x}\mathbf{y}^T + \mathbf{y}\mathbf{x}^T$ is symmetric.
\end{problem}
\begin{proof}[Solution]
Recall that a matrix is symmetric if it is equal to its transpose. Thus, we must show $A = A^T$. But for any $n\times n$ matrices $A$ and $B$, $(A+B)^T = A^T + B^T$, and $(AB)^T = B^T A^T$, and $(A^T)^T = A$. Thus, given our matrix $A= \mathbf{x}\mathbf{y}^T + \mathbf{y}\mathbf{x}^T$, we have that $A^T = (\mathbf{x}\mathbf{y}^T + \mathbf{y}\mathbf{x}^T)^T = (\mathbf{x}\mathbf{y}^T)^T + (\mathbf{y}\mathbf{x}^T)^T = (\mathbf{y}^T)^T\mathbf{x}^T + (\mathbf{x}^T)^T\mathbf{y}^T = \mathbf{y}\mathbf{x}^T + \mathbf{x}\mathbf{y}^T = \mathbf{x}\mathbf{y}^T + \mathbf{y}\mathbf{x}^T = A$
\end{proof}
%
\begin{problem}
Compute the product $\begin{bmatrix} 2 & -1 \\ 3 & 1\end{bmatrix} \begin{bmatrix} -1 & 2 & 3 & 1 \\ 2 & -2 & 1 & -1 \end{bmatrix}$
\end{problem}
\begin{proof}[Solution]
\begin{align*}
    &\begin{bmatrix} 2 & -1 \\ 3 & 1\end{bmatrix} \begin{bmatrix} -1 & 2 & 3 & 1 \\ 2 & -2 & 1 & -1 \end{bmatrix}\\
    =&\begin{bmatrix} 2(-1)+(-1)2 & 2\cdot 2 + (-1)(-2) & 2\cdot 3 + (-1)1 & 2\cdot 1 + (-1)(-1) \\ 3(-1)+1\cdot 2 & 3\cdot 2 + 1(-2) & 3\cdot 3 + 1\cdot 1 & 3\cdot 1 + 1(-1)\end{bmatrix}\\
    =&\begin{bmatrix} -4 & 6 & 5 &3 \\ -1 & 4 & 10 & 2\end{bmatrix}
\end{align*}
\end{proof}
%
\begin{problem}
Find the equation of the plane that passes through $P_{1}(2,2,1),P_{2}(2,3,2)$, and $P_{3}(-1,3,1)$.
\end{problem}
\begin{proof}[Solution]
It suffices to find a vector normal to this plane. We have that:
\begin{align*}
    \overrightarrow{P_1P_2} &= (0,1,1)^T\\
    \overrightarrow{P_1P_3} &= (-3,1,0)^T
\end{align*}
Then both vectors are parallel to the plane, and thus $\overrightarrow{P_1P_2}\times \overrightarrow{P_1P_3}=(-1,3,3)^T$ is perpendicular to the plane. Suppose $Q=(x,y,z)$ is a point in the plane. Then the relative position vector $P_1 Q = (x-2,y-2,z-1)^T$ is orthogonal to $(-1,3,3)^T$. Thus:
\begin{align*}
    (x-2,y-2,z-1)(-1,3,3)^T &= 0\\
    \Rightarrow 2-x+3y-6+3z-3 &= 0\\
    \Rightarrow x-3y-3z +7 &= 0   
\end{align*}
This is the equation of the plane.
\end{proof}
%
\begin{problem}
Let $S$ be the subspace of $\mathbb{R}^3$ spanned by $\mathbf{x}_1 = (1,-1,2)^T$ and $\mathbf{x}_2 = (-1,2,2)^T$. Find a basis for $S^{\perp}$
\end{problem}
\begin{proof}[Solution]
We seek a vector in $\mathbf{x}_3\in\mathbb{R}^3$ such that $\langle \mathbf{x}_3, \mathbf{x}_{1,2}\rangle = 0$. Or in other words:
%
\begin{equation*}
    \begin{bmatrix} 1 & -1 & 2 \\ 0 & 1 & 4 \end{bmatrix} \begin{bmatrix} x_1 \\ x_2 \\ x_3 \end{bmatrix} = 0    
\end{equation*}
%
This gives the following:
%
\begin{align*}
x_1 - x_2 + 2x_3 &= 0\\
	x_2 + 4x_3 &= 0
\end{align*}
So $x_2 = -4x_3$, $x_1=-6x_3$. Thus $\mathbf{x}_3 = x_3(-6,-4,1)^T$. $x_3$ is a free variable.
\end{proof}
%
\begin{problem}
For the matrix $A = \begin{bmatrix} 1 & 2 & 2 \\ -1 & -1 & 0 \end{bmatrix}$, find a basis for the following:
\begin{enumerate}
\begin{multicols}{4}
\item $R(A^T)$
\item $N(A)$
\item $R(A)$
\item $N(A^T)$
\end{multicols}
\end{enumerate}
\end{problem}
\begin{proof}[Solution] In order,
\begin{enumerate}
\item Putting $A$ into row-echelon form, we get $\begin{bmatrix} 1 & 1 & 0 \\ 0 & 1 & 2 \end{bmatrix}$. Thus, reading off the rows gives us a basis for $R(A^T)$. That is, $(1,1,0)^T$ and $(0,1,2)^T$.
\item $N(A) = \{x\in \mathbb{R}^3: Ax = 0\}$. Thus:
\begin{align*}
    x_1 + 2x_2 + 2x_3 &= 0\\
    x_2 + 2x_3 &= 0
\end{align*}
This gives us a basis of $x_3(2,-2,1)^T$, where $x_3$ is a free variable.
\item Putting $A^T$ into row echelon form gives us $\begin{bmatrix} 1 & 0 \\ 0 & -1 \\ 0 & 0 \end{bmatrix}$. Reading off the non-zero row vectors gives us a basis: $(1,0)^T, (0,-1)^T$.
\item $N(A^T)= \{x\in \mathbb{R}^2: A^T x = 0\}$. Putting $A^T$ into row echelon form reduces the problem to $\begin{bmatrix} 1 & 0 \\ 0 & -1 \\ 0 & 0 \end{bmatrix} \begin{bmatrix} x_1 \\ x_2 \end{bmatrix} = 0$. This gives $x_1 = 0$ and $-x_2 = 0$. So, $N(A^T) = \{0\}$.
\end{enumerate}
\end{proof}
%
\subsection{Problem Set II}
%
\begin{problem}
Find a point on the line $y=5x$ that is closest to the point $(1,3)$.
\end{problem}
\begin{proof}[Solution]
Pick a point on the line, say $\mathbf{w} = (1,5)^T$. The point $P$ is the projection of $\mathbf{v} = (1,3)^T$ onto the line $y=5x$, and thus:
\begin{equation*}
    P = \frac{v^T w}{w^T w} = \frac{\begin{bmatrix}1 & 5 \end{bmatrix}\begin{bmatrix}1 \\ 5\end{bmatrix}}{\begin{bmatrix}1 & 5 \end{bmatrix}\begin{bmatrix}1 \\ 5\end{bmatrix}}(1,5)^T = \frac{8}{13}(1,5)^T
\end{equation*}
\end{proof}
%
\begin{problem}
Is $A = xy^T - yx^T$ symmetric? ($x$ and $y$ are $n\times 1$ vectors)
\end{problem}
\begin{proof}[Solution]
In general, no. For if it were, then $A-A^T = 0$. But:
\begin{align*}
    A-A^T &= xy^T - yx^T - (xy^T-yx^T)^T & &= 2xy^T-2yx^T\\
    &= xy^T - yx^T -[(xy^T)^T-(yx^T)^T] & &= 2A\\
    &= xy^T - yx^T - [yx^T - xy^T] & \Rightarrow A+A^{T}&=0 
\end{align*}
But $A-A^{T} = 0$, and therefore $2A = 0$. But then:
\begin{equation*}
    xy^T - yx^T = 0 \Rightarrow xy^T = yx^T    
\end{equation*}
As this is not, in general, true, $A$ is not necessarily symmetric.
\end{proof}
%
\begin{problem}
Compute the product $\begin{bmatrix} -1 & 3 \\ 4 & 2 \end{bmatrix} \begin{bmatrix} -1 & 1 & 2 & -2 \\ 2 & 3 & 1 & 1 \end{bmatrix}$
\end{problem}
\begin{proof}[Solution]
\begin{align*}
    \begin{bmatrix} -1 & 3 \\ 4 & 2 \end{bmatrix} \begin{bmatrix} -1 & 1 & 2 & -2 \\ 2 & 3 & 1 & 1 \end{bmatrix} &= \begin{bmatrix} 1+6 & -1 + 9 & -2 +3 & 2 + 3 \\ -4 + 4 & 4+6 & 8 + 2 & -8 + 2 \end{bmatrix}\\
    &= \begin{bmatrix} 7 & 8 & 1 & 5 \\ 0 & 10 & 10 & -6 \end{bmatrix}
\end{align*}
\end{proof}
%
\begin{problem}
Find the equation of the plane that passes through the points $P_1 = (2,2,2),\ P_2 = (2,3,4),\ P_3 = (-1,3,3)$.
\end{problem}
\begin{proof}[Solution]
$\overrightarrow{P_1P_2} = (0,1,2)^{T}$, $\overrightarrow{P_1 P_3} = (-3,1,1)^{T}$. So:
%
\begin{equation*}
    \overrightarrow{N} = \begin{vmatrix} \hat{\mathbf{i}} & \hat{\mathbf{j}} & \hat{\mathbf{k}} \\ 0 & 1 & 2 \\ -3 & 1 & 1 \end{vmatrix} = \hat{\mathbf{i}}(1-2) + \hat{\mathbf{j}}(0+6) + \hat{\mathbf{k}}(0+3)=\begin{bmatrix}-1 \\ -6 \\ 3\end{bmatrix}   
\end{equation*}
%
We get the equation of the plane from the fact that for some point $P=(x,y,z)$ in the plane, $\langle \overrightarrow{P_1P}, \overrightarrow{N}\rangle = 0$. Thus, $x + 6y - 3z =0$
\end{proof}
%
\clearpage
%
\begin{problem}
Let $S$ be the subspace of $\mathbb{R}^3$ spanned by $\mathbf{x}_1 = (2,1,2)^T$ and $\mathbf{x}_2 = (-2,-1,3)^T$. Find a basis for $S^{\perp}$.
\end{problem}
\begin{proof}[Solution]
Let $A = \begin{bmatrix} 2 & 1 & 2 \\ -2 & -1 & 3\end{bmatrix}$. Then $S^{\perp} = N(A)$. So, it suffices to find a basis for $N(A)$. Putting $A$ into row echelon form, we get $\begin{bmatrix} 2 & 1 & 2 \\ 0 & 0 & 5 \end{bmatrix}$. Solving for $Ax = 0$, we get:
\begin{align*}
     2x_1 + x_2 + 2x_3 &= 0\\ 
     5x_3 &= 0    
\end{align*}
So $x_3 = 0$, and $x_2 = - 2x_1$. $S^{\perp}$ has the basis $x_1(1,-2,0)^T$, where $x_1$ is a free variable.
\end{proof}
%
\begin{problem}
Let $A = \begin{bmatrix} 2 & 3 & 4 \\ -2 & -2 & 0 \end{bmatrix}$. Find a basis for the following:
\begin{enumerate}
\begin{multicols}{4}
\item $R(A^T)$
\item $N(A)$
\item $R(A)$
\item $N(A^T)$
\end{multicols}
\end{enumerate}
\end{problem}
\begin{proof}[Solution]
In order,
\begin{enumerate}
\item Putting $A$ into row echelon form, we get $\begin{bmatrix} 1 & 1 & 0 \\ 0 & 1 & 4 \end{bmatrix}$. Thus, $(1,1,0)^T$ and $(0,1,4)^T$ form a basis for $R(A^T)$.
\item $N(A) = \{x\in \mathbb{R}^3: Ax = 0\}$. So, we have:
\begin{align*}
    x_1 + x_2 &= 0\\
    x_2 + 4x_3 &= 0    
\end{align*}
This leads to $x_3(4,-4,1)^T$, where $x_3$ is a free variable.
\item Putting $A^T$ into row echelon form, $\begin{bmatrix} 1 & 0 \\ 0 & 1 \\ 0 & 0 \end{bmatrix}$. This gives a basis of $(1,0)^T$ and $(0,1)^T$.
\item $N(A^T) = \{x\in \mathbb{R}^2: A^Tx = 0\}$. This gives $\begin{bmatrix} 1 & 0 \\ 0 & 1 \\ 0 & 0 \end{bmatrix} \begin{bmatrix}x_1 \\ x_2 \end{bmatrix} = 0$. So, $x_1 = 0$ and $x_2 = 0$. Thus, $x=0$. $N(A^T) = 0$.
\end{enumerate}
\end{proof}
%
\subsection{Problem Set III}
%
\begin{problem}
Let $A,B,C$ be $n\times n$ matrices. Is $A = BC^T + CB^T$ symmetric?
\end{problem}
\begin{proof}[Solution]
A matrix is symmetric if $A = A^T$. If $A = BC^T+CB^T$, then:
\begin{align*}
    A^T &= (BC^T+CB^T)^T & &= (C^T)^TB^T + (B^T)^TC^T & &= BC^T + CB^T\\
    &=  (BC^T)^T + (CB^T)^T & &= CB^T + BC^T & &= A
\end{align*}
$A$ is symmetric.
\end{proof}
%
\clearpage
%
\begin{problem}
Compute $\norm{x}_1, \norm{x}_2, \norm{x}_3$ for $x = (2,-3,1)^T$
\end{problem}
\begin{proof}[Solution]
By definition, for $x\in \mathbb{R}^n$, $\norm{x}_p = (\sum_{k=1}^{n}|x_k|^p)^{1/p}$. So we have the following:
\begin{enumerate}
\item $\norm{x}_1 = |2|+|-3|+|1| = 2+3+1 = 6$
\item $\norm{x}_2 = (|2|^2+|-3|^2+|1|^2)^{1/2} = (4+9+1)^{1/2} = \sqrt{14}$
\item $\norm{x}_3 = (|2|^3+|-3|^3+|1|^3)^{1/3} = (8+27+1)^{1/3} = \sqrt[3]{36}$
\end{enumerate}
\end{proof}
%
\begin{problem}
For the matrix $A = \begin{bmatrix} 2 & -2 & 4 \\ -1 & 1 & -2 \end{bmatrix}$, find a basis for the following:
\begin{enumerate}
\begin{multicols}{4}
\item $R(A^T)$
\item $N(A)$
\item $R(A)$
\item $N(A^T)$
\end{multicols}
\end{enumerate}
\end{problem}
\begin{proof}[Solution]
In order,
\begin{enumerate}
\item Putting $A$ into row echelon form, we get $\begin{bmatrix} 1 & -1 & 2 \\ 0 & 0 & 0 \end{bmatrix}$. $(1,-1,2)^T$ is a basis for $R(A^{T})$.
\item $N(A) = \{x\in \mathbb{R}^3: Ax = 0\}$. Using the row echelon form we get:
\begin{equation*}
    \begin{bmatrix} 1 & -1 & 2 \\ 0 & 0 & 0 \end{bmatrix} \begin{bmatrix} x_1 \\ x_2 \end{bmatrix} = 0
\end{equation*}
So $x_1 - x_2 + 2x_3 = 0$. This gives us two free variables, and we get $x_2(1,1,0)^T + x_3(-2,0,1)^T$ as a basis.
\item Putting $A^T$ into row echelon form, we get $\begin{bmatrix} -2 & 1 \\ 0 & 0 \\ 0 & 0 \end{bmatrix}$. $(-2,1)^T$ is a basis for $R(A)$.
\item $N(A^T) = \{x\in \mathbb{R}^2: A^T x = 0\}$. So:
%
\begin{equation*}
    \begin{bmatrix} -2 & 1 \\ 0 & 0 \\ 0 & 0 \end{bmatrix} \begin{bmatrix} x_1 \\ x_2 \end{bmatrix} = 0    
\end{equation*}
%
So, $-2x_1 + x_2 = -$, or $x_2 = 2x_1$. $(1,2)^T$ is a basis for $N(A^{T})$.
\end{enumerate}
\end{proof}
%
\begin{problem}
Find the least-squares solution to the following system:
\begin{align*}
    x_1 - x_2 &=2\\
    x_1 + x_2 &= 0\\
    x_1 + 2x_2 &=-1
\end{align*}
\end{problem}
\begin{proof}[Solution]
We want the solution to $A^T A x = A^T b$, where $A = \begin{bmatrix} 1 & -1 \\ 1 & 1 \\ 1 & 2 \end{bmatrix}$, and $b = \begin{bmatrix} 2\\0\\-1\end{bmatrix}$. Computing $A^T A$, we get $\begin{bmatrix} 3 & 1 \\ 1 & 9 \end{bmatrix} \begin{bmatrix} x_1 \\ x_2 \end{bmatrix} = \begin{bmatrix} 1 \\ -6 \end{bmatrix}$. The solution is $x = \frac{1}{26}(15,-19)^T$
\end{proof}
%
\begin{problem}
Let $\theta\in\mathbb{R}$ and let $\mathbf{x}_1 = (\cos(\theta), \sin(\theta))^{T}$, $\mathbf{x}_2 = (-\sin(\theta), \cos(\theta))^{T}$. Show that $\{\mathbf{x}_1,\mathbf{x}_2\}$ is an orthonormal basis for $\mathbb{R}^2$. Given the vector $\mathbf{y} = (-2, 3)^{T}$, write it as a linear combination $\mathbf{y} = c_1 \mathbf{x}_1+c_2\mathbf{x}_2$
\end{problem}
\begin{proof}[Solution]
They are orthogonal as $\mathbf{x}_1^T \mathbf{x}_2 = -\cos(\theta)\sin(\theta) + \cos(\theta)\sin(\theta) = 0$. They are orthonormal as $\norm{\mathbf{x}_1} = \sqrt{\cos^2(\theta)+\sin^2(\theta)} = 1$ and $\norm{\mathbf{x}_2} = \sqrt{(-\sin^2(\theta))+\cos^2(\theta)}=1$. Thus they are an orthonormal basis. We want $\mathbf{y} = c_1 \mathbf{x}_1 + c_2 \mathbf{x}_2$, so $\langle \mathbf{y}, \mathbf{x}_{1}\rangle = c_{1}$ and $\langle \mathbf{y},\mathbf{x}_{2}\rangle = c_{2}$. Therefore, $c_1 = -2\cos(\theta)+3\sin(\theta)$ and $c_2 = 2\sin(\theta)+3\cos(\theta)$. Thus, $\mathbf{y} = (-2\cos(\theta)+3\sin(\theta))\mathbf{x}_1 + (2\sin(\theta)+3\cos(\theta)\mathbf{x}_2)$
\end{proof}
%
\subsection{Problem Set IV}
%
\begin{problem}
Find the eigenvalues and associated eigenspaces of $A = \begin{bmatrix}4 & 5 \\ 2 & 1 \end{bmatrix}$
\end{problem}
\begin{proof}[Solution]
We need to compute $\det(A-\lambda I)=0$. This gives us:
%
\begin{equation*}
    \begin{vmatrix} 4-\lambda & 5 \\ 2 & 1-\lambda \end{vmatrix} = (4-\lambda)(1-\lambda)-10 = 0
\end{equation*}
The solutions to this are $\lambda_1 = 6, \lambda_2 = -1$. Solving $Ax = \lambda x$ yields the eigenspaces. We have:
\begin{align*}
    \begin{bmatrix} 4 & 5 \\ 2 & 1 \end{bmatrix} \begin{bmatrix} x_1 \\ x_2 \end{bmatrix} &= -\begin{bmatrix} x_1 \\ x_2 \end{bmatrix}\\
    \begin{bmatrix} 4 & 5 \\ 2 & 1 \end{bmatrix} \begin{bmatrix} x_1 \\ x_2 \end{bmatrix} &= 6\begin{bmatrix} x_1 \\ x_2 \end{bmatrix}
\end{align*}
These give solutions $x_2(-1,1)^T$ and $x_2 (\frac{5}{2},1)^T$, where $x_2$ is a free variable.
\end{proof}
%
\begin{problem}
Show that for a $2\times 2$ matrix $A$, $\lambda^2 - \Tr(A)\lambda + \det(A) = 0$, where $\lambda$ is an eigenvalue of $A$.
\end{problem}
\begin{proof}[Solution]
For we have that $\det(A-\lambda I) = 0$. But:
\begin{align*}
    \det(A-\lambda I) &= \begin{vmatrix} a-\lambda & b \\ c & d-\lambda \end{vmatrix} & &= \lambda^2 - (a+d)\lambda + ad - bc\\
    &= (a-\lambda)(d-\lambda) - bc & &= \lambda^{2} - \Tr(A)\lambda + \det(A)
\end{align*}
Thus, $\lambda^2 - \Tr(A) \lambda + \det(A) = 0$.
\end{proof}
%
\begin{problem}
Find the eigenvalues and associated eigenspaces for $A = \begin{bmatrix} 1 & 1 & 1 \\ 0 & 2 & 1 \\ 0 & 0 & 3\end{bmatrix}$
\end{problem}
\begin{proof}[Solution]
Recall that the determinant expansion can be done along any row. Thus:
%
\begin{align*}
    \det(A-\lambda I) &= \begin{vmatrix} 1-\lambda & 1 & 1 \\ 0 & 2-\lambda & 1 \\ 0 & 0 & 3-\lambda \end{vmatrix}\\
    &= 0\begin{vmatrix} 1 & 1 \\ 2-\lambda & 1 \end{vmatrix} - 0 \begin{vmatrix} 1-\lambda & 1 \\ 0 & 1 \end{vmatrix} + (3-\lambda)\begin{vmatrix} 1-\lambda & 1 \\ 0 & 2-\lambda\end{vmatrix}\\
    &= (3-\lambda)(1-\lambda)(2-\lambda)    
\end{align*}
The solutions are $\lambda_1 = 1,\ \lambda_2 = 2,\ \lambda_3 = 3$. The eigenspaces correspond to the solutions of the equation $Ax = \lambda x$. Thus we get: \begin{equation*}
    \begin{bmatrix} 1 & 1 & 1 \\ 0 & 2 & 1 \\ 0 & 0 & 3 \end{bmatrix}\begin{bmatrix} x \\ y \\ z \end{bmatrix} = \lambda \begin{bmatrix}x \\ y \\ z\end{bmatrix}    
\end{equation*}
This gives 3 different equations for each value of $\lambda$.
\begin{enumerate}
    \item $\begin{bmatrix} 1 & 1 & 1 \\ 0 & 2 & 1 \\ 0 & 0 & 3 \end{bmatrix}\begin{bmatrix} x \\ y \\ z \end{bmatrix} = \ \  \begin{bmatrix}x \\ y \\ z\end{bmatrix}\hspace{2.48 cm} \begin{bmatrix} x \\ y \\ z \end{bmatrix} = (1,0,0)^T$
    \item $\begin{bmatrix} 1 & 1 & 1 \\ 0 & 2 & 1 \\ 0 & 0 & 3 \end{bmatrix}\begin{bmatrix} x \\ y \\ z \end{bmatrix} = 2\begin{bmatrix}x \\ y \\ z\end{bmatrix}\hspace{2.54 cm}\begin{bmatrix} x \\ y \\ z \end{bmatrix} = (1,1,0)^T$
    \item $\begin{bmatrix} 1 & 1 & 1 \\ 0 & 2 & 1 \\ 0 & 0 & 3 \end{bmatrix}\begin{bmatrix} x \\ y \\ z \end{bmatrix} = 3\begin{bmatrix}x \\ y \\ z\end{bmatrix}\hspace{2.54 cm} \begin{bmatrix} x \\ y \\ z \end{bmatrix} = (1,1,1)^T$
\end{enumerate}
\end{proof}
%
\subsection{Problem Set V}
%
\begin{problem}
Factor the matrix $\begin{bmatrix} 4 & 2 \\ 2 & 1 \end{bmatrix}$ into the form $PDP^T$, where $D$ is a diagonal matrix and $P$ is an orthogonal matrix.
\end{problem}
\begin{proof}[Solution]
The eigenvectors of $A$ are the solutions to $(4-\lambda)(1-\lambda)-4=0$, which gives $\lambda_1 = 0$, and $\lambda_2 = 5$. The eigenvectors are solutions to:
%
\begin{equation*}
    \begin{bmatrix} 4 & 2 \\ 2 & 1 \end{bmatrix} \begin{bmatrix} x \\ y \end{bmatrix} = \lambda \begin{bmatrix} x \\ y \end{bmatrix}
\end{equation*}
%
Which gives us $\frac{1}{\sqrt{5}}(2,1)^T$ and $\frac{1}{\sqrt{5}}(-1,2)^T$. Thus:
%
\begin{equation*}
    P = \frac{1}{\sqrt{5}}\begin{bmatrix} -1 & 2 \\ 2 & 1 \end{bmatrix}\quad\quad D = \begin{bmatrix} 0 & 0 \\ 0 & 5 \end{bmatrix}\quad\quad P^{T} = \frac{1}{\sqrt{5}}\begin{bmatrix} -1 & 2 \\ 2 & 1 \end{bmatrix}
\end{equation*}
\end{proof}
%
\begin{problem}
Solve the differential equation $Y'(t) = \begin{bmatrix} 4 & 2 \\ 2 & 1 \end{bmatrix} Y(t)$ with $Y(0) = \begin{bmatrix} -1 \\ 4 \end{bmatrix}$
\end{problem}
\begin{proof}[Solution]
We know from the previous problem that the eigenvalues and eigenvectors are distinct, and thus $Y(t) = \alpha V_1 e^{\lambda_1 t} + \beta V_2 e^{\lambda_2 t}$ where $\lambda_{i}$ are the distinct eigenvalues, and $V_{i}$ are the distinct eigenvectors. Solving for the initial condition:
%
\begin{align*}
    \frac{1}{\sqrt{5}}\begin{bmatrix} 2 & -1 \\ 1 & 2 \end{bmatrix}  \begin{bmatrix} \alpha \\ \beta \end{bmatrix} &= \begin{bmatrix} -1 \\ 4 \end{bmatrix}\\    
    \Rightarrow \begin{bmatrix} \alpha \\ \beta \end{bmatrix} &= \frac{1}{\sqrt{5}}\begin{bmatrix} -1 & 2 \\ 2 & 1 \end{bmatrix} \begin{bmatrix} -1 \\ 4 \end{bmatrix}\\
    &= \frac{1}{\sqrt{5}} \begin{bmatrix} 9 \\ 2 \end{bmatrix}
\end{align*}
Thus, $Y(t) = \frac{9}{5}(-1,2)^T + \frac{2}{5} (2,1)^T e^{5t}$ 
\end{proof}
%
\begin{problem}
Solve the following:
\begin{enumerate}
\item Let $A$ be an $n\times n$ complex Hermitian matrix such that $A^4=I$. What are the possible eigenvalues of $A$?
\item If $A$ is an $n\times n$ complex matrix and $A^4 = I$, what are the possible eigenvalues?
\end{enumerate}
\end{problem}
%
\begin{problem}
Using the method of least squares, find the line in $\mathbb{R}^2$ that best fits the data points: $(2,1),\ (3,2),\ (4,2),\ (5,3)$
\end{problem}
\begin{proof}[Solution]
We want a line $y=mx+b$ that best fits the points. Setting up the problem, we get:
%
\begin{equation*}
    \begin{bmatrix} 1 & 2 \\ 1 & 3 \\ 1 & 4 \\ 1 & 5 \end{bmatrix} \begin{bmatrix} b \\ m \end{bmatrix} = \begin{bmatrix} 1 \\ 2 \\ 2 \\ 3\end{bmatrix}   
\end{equation*}
%
This has no solution. Denoted $A$ as the left-most matrix, we compute $A^T$:
%
\begin{equation*}
    A^T = \begin{bmatrix} 1 & 1 & 1 & 1 \\ 2 & 3 & 4 & 5 \end{bmatrix}   
\end{equation*}
%
So $A^T A = \begin{bmatrix} 4 & 14 \\ 14 & 54 \end{bmatrix}$. We now solve $A^{T}AX$:
%
\begin{equation*}
    \begin{bmatrix} 4 & 14 \\ 14 & 54 \end{bmatrix} \begin{bmatrix} b \\ m \end{bmatrix} =  A^T \begin{bmatrix} 1 \\ 2 \\ 2 \\ 3 \end{bmatrix} = \begin{bmatrix} 8 \\ 31 \end{bmatrix}   
\end{equation*}
%
The solution gives us $y = 0.6x-0.1$
\end{proof}
%
\begin{problem}
Find the projection matrix $P$ that projects $\mathbb{R}^4$ onto the line through the origin spanned by the vector $(2,1,-1,-1)$.
\end{problem}
%
\begin{problem}
Consider the rotation matrix $R = \begin{bmatrix} \frac{-4}{9} & \frac{-7}{9} & \frac{4}{9} \\ \frac{1}{9} & \frac{4}{9} & \frac{8}{9} \\ \frac{-8}{9} & \frac{4}{9} & \frac{-1}{9} \end{bmatrix}$ Compute the axis vector $\textbf{u}$ and both the sine and cosine of the counterclockwise angle $\theta$ such that $R = R_{\theta,\textbf{u}}$
\end{problem}
%
\begin{problem}
Find an orthonormal basis for the column space of the matrix:
%
\begin{equation*}
    A = \begin{bmatrix} 1 & 1 & 1 \\ 0 & 3 & 1 \\ 2 & 2 & 2 \\ 2 & 4 & 3 \\ -1 & 2 & 0 \end{bmatrix}
\end{equation*}
\end{problem}
\begin{proof}[Solution]
We use the Gram-Schmidt procedure to do this. Take $(1,0,2,2,-1)$ and normalize it, giving us:
%
\begin{equation*}
    e_{1} = \frac{1}{\sqrt{10}}(1,0,2,2,-1)^T    
\end{equation*}
%
We then compute:
%
\begin{align*}
    (1,3,2,4,2)^T &- \frac{(1,3,2,4,2)^T(1,0,2,2,-1)}{(1,0,2,2,-1)^T (1,0,2,2,-1)}(1,0,2,2,-1)^T\\
    &= (1,3,2,4,2)^T-\frac{11}{10}(1,0,2,2,-1)\\
    &=(-\frac{1}{10},3,-\frac{2}{10},\frac{18}{10},\frac{33}{10})\\
    &= \frac{1}{10}(-1,30,-2,18,33)
\end{align*}
%
Thus:
%
\begin{equation*}
    e_{2} = \frac{ \frac{1}{10}(-1,30,-2,18,33)}{\norm{ \frac{1}{10}(-1,30,-2,18,33)}} = \frac{1}{\sqrt{2318}}(-1,30,-2,18,33)
\end{equation*}
%
Finishing off, we compute:
%
\begin{align*}
    \mathbf{v}_{3} &= (1,1,2,3,0)^T - \frac{(1,1,2,3,0)^T(1,0,2,2,-1)}{10}(1,0,2,2,-1)^T\\
    &- \frac{(1,1,2,3,0)^T(1,3,2,4,2)}{34}(1,3,2,4,2,0)^T     
\end{align*}
Finally,
%
\begin{equation*}
    e_3 = \frac{\textbf{v}_3}{\norm{\textbf{v}_3}}
\end{equation*}
\end{proof}
%
\begin{problem}
Eliminate crossterms, classify, and sketch the graph of the conic section $6x^2 - 4xy+3y^2 = 1$
\end{problem}
%
\subsection{Problem Set VI}:
%
\begin{problem}
Let $\begin{bmatrix*}[r] 1 & 0 & 3 & \vline & 1 \\ 0 & 1 & -2 & \vline & 3 \\ 1 & 2 & 0 & \vline & 0 \end{bmatrix*}$ be an augmented matrix.
\begin{enumerate}
\item Solve the system using Gaussian elimination.
\item Express $\begin{bmatrix} 1 \\ 3 \\ 0\end{bmatrix}$ as a linear combination of the column vectors of the coefficient matrix.
\item Use elementary matrices to find the LU decomposition of the coefficient matrix.
\end{enumerate}
\end{problem}
\begin{proof}[Solution]
In order,
\begin{enumerate}
\item Swap rows $2$ and $3$ to get $\begin{bmatrix}[ccc|c] 1 & 0 & 3 & 1 \\ 1 & 2 & 0 & 0 \\ 0 & 1 & -2 & 3 \end{bmatrix}$. Subtract $1$ from $2$ to obtain $\begin{bmatrix}[ccc|c] 1 & 0 & 3 & 1 \\ 0 & 2 & -3 & -1 \\ 0 & 1 & -2 & 3 \end{bmatrix}$. Divide row $2$ be $2$, $\begin{bmatrix}[ccc|c] 1& 0 & 3 & 1 \\ 0 & 1 & \frac{-3}{2} & \frac{-1}{2} \\ 0 & 1 & -2 & 3\end{bmatrix}$. Subtract $2$ from $3$, $\begin{bmatrix}[ccc|c] 1 & 0 & 3 & 1 \\ 0 & 1 & \frac{-3}{2} & \frac{-1}{2} \\ 0 & 0 & \frac{-1}{2} & \frac{7}{2} \end{bmatrix}$. Multiply row $3$ be $-2$, $\begin{bmatrix}[ccc|c] 1 & 0 & 3 & 1 \\ 0 & 1 & \frac{-3}{2} & \frac{-1}{2} \\ 0 & 0 & 1 & -7 \end{bmatrix}$. Subtract $3$ of row $3$ from row $1$ to get $\begin{bmatrix}[ccc|c] 1 & 0 & 0 & 22 \\ 0 & 1 & \frac{-3}{2} & \frac{-1}{2} \\ 0 & 0 & 1 & -7 \end{bmatrix}$. Finally, add $\frac{3}{2}$ of row $3$ to row 2, $\begin{bmatrix}[ccc|c] 1&0&0& 22 \\ 0&1&0&-11 \\ 0 & 0 & 1 & -7 \end{bmatrix}$
\item $\begin{bmatrix} 1 \\ 3 \\ 0 \end{bmatrix} = 22 \begin{bmatrix} 1 \\ 0 \\ 1 \end{bmatrix} - 11\begin{bmatrix} 0 \\ 1 \\ 2 \end{bmatrix} -7 \begin{bmatrix} 3 \\ -2 \\ 0 \end{bmatrix}$
\item $A = \begin{bmatrix} 1 & 0 & 0 \\ 0 & 1 & 0 \\ 1 & 2 & 1 \end{bmatrix} \begin{bmatrix} 1 & 0 & 3 \\ 0 & 1 & -2 \\ 0 & 0 & 1 \end{bmatrix}$
\end{enumerate}
\end{proof} 
%
\begin{problem}
Let $A = \begin{bmatrix} 1 & 0 & 0 \\ 2 & 1 & 0 \\ 3 & 4 & 1 \end{bmatrix}$, $B=\begin{bmatrix}1 & 0 & 0 \\ -2 & 1 & 0 \\ 5 & -4 & 1 \end{bmatrix}$, and $C = \begin{bmatrix} 2 & 3 \\ -1 & 0 \\ 1 & 1 \end{bmatrix}$. 
\begin{enumerate}
\begin{multicols}{3}
\item Solve $AC+BC$
\item Solve $AB$
\item Does $A = B^{-1}$?
\end{multicols}
\end{enumerate}
\end{problem}
\begin{proof}[Solution]
In order,
\begin{enumerate}
\item $AC+BC = (A+B)C = \begin{bmatrix} 2 & 0 & 0 \\ 0 & 2 & 0 \\ 8 & 0 & 2 \end{bmatrix} \begin{bmatrix} 2 & 3 \\ -1 & 0 \\ 1 & 1 \end{bmatrix} = \begin{bmatrix} 4 & 6 \\ -2 & 0 \\ 18 & 26 \end{bmatrix}$
\item $AB = \begin{bmatrix} 1 & 1 & 0 \\ 0 & -15 & 0 \\ 0 & 0 & 1 \end{bmatrix}$
\item No, as if $A=B^{-1}$ then $AB=I$.
\end{enumerate}
\end{proof}
%
\begin{problem}
If $A$ and $B$ are $n\times n$ invertible matrices, what is $(AB)^{-1}$?
\end{problem}
\begin{proof}[Solution]
As $A^{-1}$ and $B^{-1}$ exist, and as $A$ and $B$ are of the same dimension, $B^{-1}A^{-1}$ exists. But $(B^{-1}A^{-1})(AB) = B^{-1}(A^{-1}A)B = B^{-1}IB = B^{-1}B = I$. Thus, as inverses are unique, $(AB)^{-1} = B^{-1}A^{-1}$.
\end{proof}
%
\begin{problem}
What are the solutions of:
\begin{enumerate}
\item $\begin{bmatrix}[cccc|c] 1 & 1 & 0 & 0 & -1 \\ 0 & 1 & 0 & 0 & 3 \\ 0 & 0 & 1 & 1 & 2 \\ 0 & 0 & 1 & 1 & 1 \end{bmatrix}$
\item $\begin{bmatrix}[cccc|c] 1 & 1 & 0 & 0 & -1 \\ 0 & 1 & 0 & 0 & 3 \\ 0 & 0 & 1 & 1 & 1 \\ 0 & 0 & 1 & 1 & 1 \end{bmatrix}$
\end{enumerate}
\end{problem}
\begin{proof}[Solution]
In order,
\begin{enumerate}
\item No solution as the bottom two rows say $x_3 + x_4 = 2$ and $x_3 + x_4 = 1$. An impossibility.
\item The entire space $S = \{(-4,3,x,1-x):x\in \mathbb{R}\}$.
\end{enumerate}
\end{proof}
%
\begin{problem}
If $A$ and $B$ are $n\times n$ matrices, what is $(A+B)^2$?
\end{problem}
\begin{proof}[Solution]
$(A+B)^2 =(A+B)(A+B) = A(A+B)+B(A+B)=A^2+AB+BA+B^2$. Note: It is not true in general that $AB=BA$, and thus we cannot simplify further.
\end{proof}
%
\begin{problem}
If $A$ and $A^T$ are $n\times n$ invertible matrices, show that $(A^T)^{-1} = (A^{-1})^T$
\end{problem}
\begin{proof}[Solution]
For $A^T(A^{-1})^T = ((A^{-1})^T A^T)^T = (A^{-1}A)^T = I^T = I$. Thus, as inverses are unique, $(A^T)^{-1} = (A^{-1})^T$
\end{proof}
%
\begin{problem}
If $A,B,$ and $C$ are $n\times n$ invertible matrices, then solve the following equations for $X$:
\begin{enumerate}
\item $XA+B=C$
\item $AX+B=X$
\item $XA+C=X$
\end{enumerate}
\end{problem}
\begin{proof}
In order,
\begin{enumerate}
\item $XA +B=C\Rightarrow XA = C-B \Rightarrow X = (C-A)A^{-1}$
\item $AX+B = X\Rightarrow AX-X=-B \Rightarrow (A-I)X=-B \Rightarrow X = -(A-I)^{-1}B$
\item $X = -C(I-A)^{-1}$
\end{enumerate}
\end{proof}
%
\subsection{Problem Set VII}
%
\begin{problem}
Determine the basis of the given vector space over the given field.
\begin{enumerate}
\item $V=\mathbb{R}$ over $K=\mathbb{R}$
\item $V=\mathbb{C}$ over $K=\mathbb{C}$
\item $V=\mathbb{C}$ over $K=\mathbb{R}$
\end{enumerate}
\end{problem}
\begin{proof}[Solution]
In order,
\begin{enumerate}
\item The set $\{1\}$ is a basis. Let $r \in \mathbb{R}$. Then $r=1\cdot r$.
\item The set $\{(1,0)\}$ is a basis. Let $z\in \mathbb{Z}$. Then $z\cdot(1,0) = z$
\item The set $\{(1,0),(0,1)\}$ is a basis. Let $z=a+bi\in \mathbb{Z}$. Then $z = a(1,0)+b(0,1)$.
\end{enumerate}
\end{proof}
%
\begin{problem}
What is the nullspace of an $n\times n$ matrix $A$ with real entires?
\end{problem}
\begin{proof}[Solution]
The nullspace is the set $N(A) = \{X\in \mathbb{R}^n: AX = 0\}$
\end{proof}
%
\begin{problem}
Let $A=\begin{bmatrix} 1 & 2 & 3 & 4 \\ -1 & -1 & -4 & -2 \\ 3 & 4 & 11 & 8 \end{bmatrix}$ and suppose it can be row reduced to $\begin{bmatrix} 1 & 0 & 5 & 0 \\ 0 & 1 & -1 & 2 \\ 0 & 0 & 0 & 0 \end{bmatrix}$. What is the rank of $A$?
\end{problem}
\begin{proof}[Solution]
The rank is the dimension of the space spanned by the column vectors of the matrix. Using the row-reduced form, we see that these columns span $\mathbb{R}^2$ and thus the matrix has rank $2$.
\end{proof}
%
\begin{problem}
What is the rank-nullity theorem?
\end{problem}
\begin{proof}[Solution]
For an $n\times n$ matrix $A$, $Rk(A)+Nul(A) = n$.
\end{proof}
%
\subsection{Problem Set VIII}
%
\begin{problem}
Let $T:\mathbb{R}^3\rightarrow \mathbb{R}^2$ be defined by $T\begin{bmatrix} x_1 \\ x_2 \\ x_3 \end{bmatrix} = \begin{bmatrix} x_3 \\ x_1+x_2 \end{bmatrix}$.
\begin{enumerate}
\item Determine $\ker(T)$.
\item Determine the dimensions of $\ker(T)$.
\item Using the Nullity Theorem, determine the dimension of im$(T)$.
\end{enumerate}
\end{problem}
\begin{proof}[Solution]
In order,
\begin{enumerate}
\item $\ker(T)= \{x\in \mathbb{R}^3: Tx = 0\}$. So, $x = (x_1,x_2,x_3)^T$ such that $Tx = 0$. Thus $x_3 = 0$ and $x_1+x_2 = 0$. $\ker(T) = \{(x,-x,0):x\in \mathbb{R}\}$.
\item This is a line through the origin, so the dimension is $1$. 
\item The Nullity Theorem states that $\dim(\ker(T))+\dim(im(T)) = \dim(\mathbb{R}^3) = 3$. Thus $\dim(im(T)) = 2$.
\end{enumerate}
\end{proof}
%
\begin{problem}
Using the linear transformation of the previous problem, determine the matrix representation of $T$ in the standard basis of $\mathbb{R}^3$.
\end{problem}
\begin{proof}[Solution]
$Te_1 = (0,1)^T$, $T e_2 = (0,1)^T$, and $Te_3 = (1,0)^T$. The matrix representation is $T=\begin{bmatrix} 0 & 0 & 1 \\ 1 & 1 & 0 \end{bmatrix}$
\end{proof}
%
\begin{problem}
Let $P_n$ be the set of all polynomials with real coefficients of degree less than $n$. The standard basis is $\{1,x,\hdots, \ x^{n-1}\}$. Let $D:P_3 \rightarrow P_2$ be defined by $D(p) = 5\frac{dp}{dx}$. Determine the matrix representation of $D$ with respect to the standard basis.
\end{problem}
\begin{proof}[Solution]
We need only check how $D$ acts on the basis vectors. $D(1) = 0+0x$, $D(x) = 1+0x$, $D(x^2) = 0+2x$. So, we have $D = \begin{bmatrix} 0 & 2 & 0 \\ 1 & 0 & 0 \end{bmatrix}$
\end{proof}
%
\begin{problem}
Let $V$ be a vector space over $\mathbb{R}$ and let $S$ be a subspace of $V$.
\begin{enumerate}
\item Define $S^{\perp}$.
If $S = $Span$\{ (1,2,1)^T, (1-1,2)^T\}$, what is $S^{\perp}$?
\end{enumerate}
\end{problem}
\begin{proof}[Solution]
In order,
\begin{enumerate}
\item $S^{\perp} = \{x\in V: \forall y\in S, x^T y = 0\}$.
\item $S^{\perp} = \{x\in \mathbb{R}^3: x^T (1,2,1)^T = x^T(1,-1,2)^T = 0\}$. Let $x = (x_1,x_2,x_3)^T \in S^{\perp}$. Then $x_1+2x_2+x_3=0$ and $x_1-x_2+2x_3 = 0$. We may write this as $\begin{bmatrix} 1 & 2 & 1 \\ 1 & -1 & 2 \end{bmatrix} \begin{bmatrix} x_1 \\ x_2\\ x_3 \end{bmatrix} = \begin{bmatrix} 0 \\ 0 \end{bmatrix}$. In row echelon form we have $\begin{bmatrix} 1 & 0 & \frac{5}{3} \\ 0 & 1 & \frac{-1}{3} \end{bmatrix}$. We thus get $x_1 = - \frac{5}{3}x_3$ and $x_2 = \frac{1}{3} x_3$.  $S^{\perp} = \{x(-\frac{5}{3}, \frac{1}{3}, 1)^T: x\in \mathbb{R}\}$.
\end{enumerate}
\end{proof}
%
\begin{problem}
\begin{enumerate}
\item Let $V$ be a vector space over $\mathbb{R}$. Define an inner product.
\item What is the difference between the standard dot product in $\mathbb{R}^n$ and an inner product? Can a vector space have more than one inner product?
\item If $\langle x,y \rangle = xy$, what is $\norm{x}$?
\end{enumerate}
\end{problem}
\begin{proof}[Solution]
In order,
\begin{enumerate}
\item An inner product is a function from $\mathbb{R}\times \mathbb{R}\rightarrow \mathbb{R}$ with the following properties:
\begin{enumerate}
\item $\langle ax+by,z\rangle = a\langle x,z\rangle+b\langle y,z\rangle$
\item $\langle x,y\rangle = \langle y,x \rangle$
\item $\langle x,x\rangle \geq 0$.
\end{enumerate}
\item An inner product is a generalization of the standard dot product. The dot product is itself an inner product, but not all inner products are dot products. There are infinitely many inner products for $\mathbb{R}$. Let $n\in \mathbb{N}$ be arbitrary, then $\langle x,y \rangle = nxy$ is an inner product.
\item $\norm{x} = \sqrt{\langle x,x \rangle } = \sqrt{x^2}= |x|$.
\end{enumerate}
\end{proof}
%
\begin{problem}
Let $V = C[-1,1]$ and let $\langle f,g\rangle = \int_{-1}^{1} f(x)g(x)dx$.
\begin{enumerate}
\item Show that $f(x)=1$ and $g(x) = x$ are orthogonal with respect to this inner product.
\item Determine $\norm{f}$ and $\norm{g}$.
\item Show that $f$ and $g$ satisfy the Pythagorean Law.
\end{enumerate}
\end{problem}
\begin{proof}[Solution]
In order,
\begin{enumerate}
\item $\langle 1,x\rangle = \int_{-1}^{1} xdx = \frac{x^2}{2}\big|_{-1}^{1} = \frac{1^2}{2}-\frac{(-1)^2}{2} = \frac{1}{2}-\frac{1}{2} = 0$.
\item $\norm{1} = \sqrt{ \int_{-1}^{1} dx} = \sqrt{2}$. $\norm{x} = \sqrt{\int_{-1}^{1}x^2dx} = \sqrt{\frac{2}{3}}$
\item $\norm{1+x}^2 = \langle 1+x,1+x\rangle = \langle 1,1\rangle + 2\langle 1,x \rangle + \langle x,x\rangle = \norm{1}^2 + \norm{x}^2$
\end{enumerate}
\end{proof}
%
\begin{problem}
Let $V$ be any inner product space. State and prove the Pythagorean Theorem for inner product spaces.
\end{problem}
\begin{proof}[Solution]
The Pythagorean Theorem for Inner Product Spaces state that if $V$ is an inner product space with inner product $\langle, \rangle$, and if $\langle x,y\rangle = 0$, then $\norm{x}^2+\norm{y}^2 = \norm{x+y}^2$. For $\norm{x+y}^2 = \langle x+y,x+y\rangle = \langle x,x\rangle + 2\langle x,y\rangle +\langle y,y\rangle$. But as $x$ and $y$ are orthogonal, $\langle x,y \rangle = 0$. Thus $\norm{x+y}^2 = \langle x,x\rangle + \langle y,y\rangle = \norm{x}^2+\norm{y}^2$. $\norm{x+y}^2 =\norm{x}^2+\norm{y}^2$.
\end{proof}
%
\begin{problem}
Prove that if $V$ is an inner product space and $S$ is a subspace of $V$, then $S^{\perp}$ is a subspace of $V$.
\end{problem}
\begin{proof}[Solution]
We must check that $0\in S^{\perp}$ and that $S^{\perp}$ is closed under addition and scalar multiplication.
\begin{enumerate}
\item For all $x\in S$, $\langle 0,x \rangle = 0$, and thus $0\in S^{\perp}$.
\item If $x,y\in S^{\perp}$ and $z\in S$, then $\langle x+y,z\rangle = \langle x,z\rangle + \langle y,z\rangle = 0+0=0$. Thus $x+y\in S^{\perp}$.
\item If $x\in S^{\perp}$, $y\in S$, and $\alpha$ is a scalar, then $\langle \alpha x,y \rangle = \alpha \langle x,y \rangle = \alpha \cdot 0 = 0$. Thus $\alpha x \in S^{\perp}$. $S^{\perp}$ is a subspace.
\end{enumerate}
\end{proof}
\end{document}