%------------------------------------------------------------------------------%
\begingroup
    \ifcsname\PATH\endcsname
        \newcommand{\PATH}{books/Foundations/ZFC}
        \newcommand{\OLDPATH}{\PATH}
    \else
        \newcommand{\OLDPATH}{\PATH}
        \renewcommand{\PATH}{books/Foundations/ZFC}
    \fi
    \chapter{Zermelo-Fraenkel Set Theory}
        \label{chapt:Zermelo_Fraenkel_Set_Theory}%
        We'll develop mathematics from an axiomatic view built on set theory,
        adopting as truths the few postulates of Zermelo and Fraenkel. We'll
        then add the axiom of choice and proceed from there to define many
        familiar concepts and prove some basic results that are often taken for
        granted. The existence of many types of sets will be proven, rather than
        accepting these things as trivial truths.
        %------------------------------------------------------------------------------%
\section{The Axioms of Zermelo and Fraenkel}
    The first thing to do is define what \textit{sets}\index{Set} are.
    \begin{fdefinition}{Sets}{Sets}
        A \gls{set} is a collection of objects (or elements), none of which is
        the set itself.\index{Set}
    \end{fdefinition}
    If we wish to stand on a truly solid foundation, it seems we're off to a bad
    start. In defining sets we've used the words \textit{collection} and
    \textit{objects}, neither of which have been defined. This is the problem
    found in Chapt.~\ref{chapt:Logic} when defining connectives. To begin
    stating definitions and theorems we need the existence of a \textit{thing}.
    Sets act as our thing. We know they exist, but we don't know how to define
    them all to well. Nevertheless, we can describe how they behave and what
    they can do, as well as how to obtain new sets from pre-existing ones.
    \begin{fnotation}{Element Notation}{Element_Notation}
        If $A$ is a \gls{set} and if $x$ is an element\index{Set!Element of} of
        $A$, then we denote this by writing
        \glslink{containmentsymb}{$x\in{A}$}. If $x$ is not an element
        of $A$, we write $x\notin{A}$.\index{Containment $\in$}
    \end{fnotation}
    We do not yet know that sets exist. Pedagogically it seems poor to wait
    for examples, so we'll speak loosely for the moment so we may familiarize
    ourselves with the notation.
    \newpage
    \begin{fexample}{Using Element Notation}{Using_Element_Notation}
        Given a set $A$ that contains only a few objects, we can represent $A$
        by listing out the elements, separated by commas, and enclosing them in
        braces. Suppose $A$ is the set that contains three distinct objects
        labelled $a$, $b$, and $c$. We then write:
        \begin{equation}
            A=\big\{\,a,\,b,\,c\,\big\}
        \end{equation}
        If we are told that there is a fourth object $d$ that is different from
        $a$, $b$, and $c$, then we can use the notation defined in
        Not.~\ref{not:Element_Notation} to write the following:
        \par
        \begin{subequations}
            \begin{minipage}[b]{0.49\textwidth}
                \centering
                \begin{equation}
                    a\in{A}
                \end{equation}
            \end{minipage}
            \hfill
            \begin{minipage}[b]{0.49\textwidth}
                \centering
                \begin{equation}
                    d\notin{A}
                \end{equation}
            \end{minipage}
        \end{subequations}
        \par\vspace{2.5ex}
        The notation $a\in{A}$ should be read as \textit{a is an element of A},
        or \textit{a is contained in A}, or simply \textit{a is in A}.
        Similarly, the notation $d\notin{A}$ should be read as
        \textit{d is not an element of A}, or \textit{d is not contained in A}.
        \par\hfill\par
        $A$ is an example of a \textit{finite} set\index{Set!Finite}, moreover
        it contains only three elements. For larger sets we rely on other
        methods to write them down. One such means is to indicate a pattern and
        use an ellipses to show that it goes on. Such a description is vague and
        lacks rigor, but can be helpful when the pattern is obvious. The set of
        all \textit{natural} numbers\index{Natural Numbers}, or non-negative
        integers (denoted \gls{mathbbN}) can be loosely represented by writing:
        \begin{equation}
            \label{eqn:Natural_Numbers_Ellipses}%
            \mathbb{N}=\big\{\,0,\,1,\,2,\,3,\,4,\,5,\,\dots\,\big\}
        \end{equation}
        Using our developed notation, we can write:
        \par
        \begin{subequations}
            \begin{minipage}[b]{0.49\textwidth}
                \centering
                \begin{equation}
                    23\in\mathbb{N}
                \end{equation}
            \end{minipage}
            \hfill
            \begin{minipage}[b]{0.49\textwidth}
                \centering
                \begin{equation}
                    \minus{4}\notin\mathbb{N}
                \end{equation}
            \end{minipage}
        \end{subequations}
        \par\vspace{2.5ex}
        Letting \gls{mathbbZn} denote all non-negative integers between 0 and
        $n-1$, we have:
        \begin{equation}
            \label{eqn:Z_n_Ellipses}%
            \mathbb{Z}_{n}=\big\{0,\,1,\,2,\,\dots,\,n-1\,\big\}
        \end{equation}
        Thus $17\in\mathbb{Z}_{18}$ but $19\notin\mathbb{Z}_{18}$. Lastly, we
        present the integers\index{Integers} (\gls{mathbbZ}).
        \begin{equation}
            \label{eqn:Integers_Ellipses}%
            \mathbb{Z}=\big\{\,\dots,\,\minus{3},\,\minus{2},\,\minus{1},
                             \,0,\,1,\,2,\,3,\dots\,\big\}
        \end{equation}
    \end{fexample}
    In our definition of a set (Def.~\ref{def:Sets}) we explicitly required
    that sets cannot contain themselves. That is, if $A$ is a set, then
    $A\notin{A}$. This requirement was introduced to avoid paradoxes discovered
    by Bertrand Russell\index{Russell, Bertrand} in 1901. Allow us to neglect
    this requirement for a moment and reveal why it is essential. Recall from
    logic that a system of mathematics is inconsistent if one can prove a
    contradiction within the theory. In Naive Set Theory\index{Naive Set Theory}
    we allow the \textit{axiom of unrestricted comprehension}%
    \index{Axiom!of Unrestricted Comprehension}. This allows us to construct
    sets as any definable collection. That is, if we have a
    proposition\index{Proposition} $P$, then we can define a set $A$ as the set
    of all objects that satisfy $P$. We can write:
    \begin{equation}
        A=\big\{\,x\;|\;P(x)\,\big\}
    \end{equation}
    Problems with such a loose definition arise instantly. Let $P$ be the
    proposition \textit{true if x is a set, false otherwise}. Then
    $A=\{\,x\;|\;P(x)\,\}$ can be read in plain English as the
    \textit{set of all sets}\index{Set!of All Sets}. A natural question would be
    whether or not $A$ then contains itself. That is, is $A\in{A}$? Russell's
    paradox arises by defining proper sets to be sets $B$ such that
    $B\notin{B}$, and improper sets to be sets $B$ such that $B\in{B}$. Using
    the \textit{Law of the Excluded Middle}\index{Law of the Excluded Middle}
    (which we will prove later), one has that every set is either proper or
    improper.
    \begin{ftheorem}{Russell's Paradox}{Russells_Paradox}
        Naive Set Theory is inconsistent.\index{Russell's Paradox}
        \index{Theorem!Russell's Paradox}
    \end{ftheorem}
    \begin{bproof}
        For let $P$ be the proposition \textit{true if} $x\notin{x}$,
        \textit{false otherwise}. Let $A$ be the set defined by this
        proposition:
        \begin{equation}
            A=\big\{\,x\;|\;P(x)\,\big\}
        \end{equation}
        That is, $A$ is the set of all sets that do not contain themselves.
        Suppose $A\in{A}$. If $A\in{A}$ then $P(A)$ is true. That is, $A$ is a
        proper set. But proper sets do not contain themselves and $A\in{A}$, a
        contradiction. Thus $A\notin{A}$. But if $A\notin{A}$ than $P(A)$ is
        false. But if $P(A)$ is false, than $A$ is an improper set. But then
        $A\in{A}$, a contradiction as $A\notin{A}$. Thus $A\in{A}$ if and only
        if $A\notin{A}$, a contradiction. Therefore, Naive Set Theory is
        inconsistent.
    \end{bproof}
    Our development of Zermelo-Fraenkel Set Theory%
    \index{Zermelo-Fraenkel Set Theory} is to avoid this paradox and attempt to
    develop a consistent system of mathematics. The proof of Russell's Paradox
    (Thm.~\ref{thm:Russells_Paradox}) relied on the
    \textit{Law of the Excluded Middle}\index{Law of the Excluded Middle} which
    states that, given a proposition $P$, either $P$ is true or its negation is
    true. Thus we have shown that the axiom of unrestricted
    comprehension\index{Axiom!of Unrestricted Comprehension} and the law of the
    excluded middle are not compatible. This is quite unfortunate as the law of
    the excluded middle is essential in mathematics as it allows one to prove
    things via contradiction\index{Proof!by Contradiction}. That is, given some
    statement we assume the opposite is true and arrive at a contradiction thus
    showing the negation of our statement is false. We then invoke the law of
    the excluded middle to show that our original statement is true. The axioms
    of Zermelo and Fraenkel, together with the axiom of choice (a system
    commonly abbreviated as \gls{ZFC}) are able to prove the validity of the law
    of the excluded middle. That is, if ZFC is consistent, then so is the law of
    the excluded middle. This is one of the reasons for studying ZFC in detail.
    \par\hfill\par
    The first collection of axioms were proposed in 1908 by
    Ernst Zermelo\index{Zermelo, Ernst}. Subtle problems were pointed out by
    Abraham Fraenkel\index{Fraenkel, Abraham} in 1920, and in 1921 the system of
    Zermelo-Fraenkel Set Theory\index{Zermelo-Fraenkel Set Theory} came to be.
    The requirement that a set does not contain itself, which is equivalent to
    the \textit{axiom of regularity}\index{Axiom!of Regularity}, is sufficient
    to avoid Russell's paradox. We will prove the equivalence of this axiom with
    our definition once we have obtained the law of the excluded middle.
    \subsection{Subsets and Equality}
        To delve more into set theory it would be convenient to know that at
        least \textit{one} set exists. The axiom of the empty
        set\index{Axiom!of the Empty Set} gives us such an existence.
        \begin{faxiom}{Axiom of the Empty Set}{Axiom_of_the_Empty_Set}
            There exists a set \gls{emptysetsymb} (the \gls{empty set}) such
            that for all $x$ it is true that $x\notin\emptyset$.
            \index{Empty Set}\index{Set!Empty}
            \begin{equation*}
                \exists_{\emptyset}:\forall_{x}\big(\neg(x\in\emptyset)\big)
            \end{equation*}
        \end{faxiom}
        The empty set is the set that contains no elements. As such some choose
        to write $\emptyset=\{\}$. Note that this is different from the
        set $\{\emptyset\}$ since the empty set contains no elements whereas
        $\{\emptyset\}$ contains one elements (it contains the empty set).
        Indeed, the equality of $\emptyset$ and $\{\emptyset\}$ would violate
        our requirement that sets do not contain themselves. Any set that
        contains \textit{something} is called non-empty.
        \begin{fdefinition}{Non-Empty Set}{Non_Empty_Set}
            A \gls{non-empty set} is a \gls{set} $A$ such that there exists an
            $x$ such that $x\in{A}$.\index{Set!Non-Empty}
        \end{fdefinition}
        The terminology is somewhat redundant, and essentially every set we deal
        with is non-empty. Indeed, there is only one empty set
        (see Thm.~\ref{thm:Empty_Set_is_Unique}). Thus, every other set one
        thinks of is non-empty.
        \begin{example}
            Using the notation from Ex.~\ref{ex:Using_Element_Notation}, the set
            of all natural numbers (\gls{mathbbN}) and the set of all integers
            (\gls{mathbbZ}) are non-empty since $0\in\mathbb{N}$ and
            $0\in\mathbb{Z}$. If $n$ is a positive integer, then the set
            of integers between $0$ and $n-1$ (\gls{mathbbZn}) is non-empty as
            well since $0\in\mathbb{Z}_{n}$. Note that there is some ambiguity
            behind the meaning of $\mathbb{Z}_{0}$. This stems from our
            \textit{dot dot dot} definition and it is unclear what this should
            mean for $n=0$. When we rigorously define this notation we will see
            that $\mathbb{Z}_{0}$ is empty. That is, we will say
            $k\in\mathbb{Z}_{n}$ if $k\in\mathbb{N}$ and $k<n$, a description
            that will be justified by the axiom schema of
            specification\index{Axiom!Schema of Specification}. Thus, for
            $\mathbb{Z}_{0}$ we seek an integer $k\in\mathbb{N}$ such that
            $k<0$. But there are no such integers, and thus the set is empty.
        \end{example}
        \begin{example}
            It's possible to write down some formula for a set that ultimately
            leads to the empty set. For consider the \textit{set of all}
            \textit{rational numbers whose square is two}. This set turns out to
            be empty since there is no rational that satisfies this criterion.
            That is, $\sqrt{2}$ is known to be an irrational number. Thus, the
            set specified by our proposition is the empty set.
        \end{example}
        \begin{example}
            Going in the other direction, it is possible to write a formula for
            a set that appears empty, but is indeed not. The set of all
            $p\textrm{-Sylow}$ subgroups\index{Group!Subgroup!Sylow} of a
            non-empty finite group (Discussed in Book~\ref{book:Algebra}) is a
            non-empty set, but there's no reason to believe so from the start.
        \end{example}
        A set is entirely determined by its elements, and as such repetition and
        order cannot be accounted for. Thus the sets $\{a,b\}$ and $\{a,a,b\}$
        must be considered the same since they contain precisely the same
        elements. This will be made clear once equality has been defined. In a
        similar manner, sets have no sense of order and thus $\{a,b\}$ and
        $\{b,a\}$ are equivalent. It then becomes a task to invent some new
        object that does have a notion of order. To do this requires the concept
        of a \textit{function}\index{Function}, and it is our current aim to
        develop this topic.
        \par\hfill\par
        To rigorously show that the examples in the previous paragraph are equal
        requires a definition of equality. This is the
        \textit{axiom of extensionality}\index{Axiom!of Extensionality}. First,
        we define the familiar symbol for equality (\gls{equalsymb}) in
        terms of containment (\gls{containmentsymb}).
        \begin{fnotation}{Equality}{Equality}
            If $A$ and $B$ are sets, then \glslink{equalsymb}{$A=B$} if and
            only if for all sets $C$, $C\in{A}$ if and only if $C\in{B}$, and
            for all sets $D$, $A\in{D}$ if and only if $B\in{D}$.
            \begin{equation*}
                \forall_{A}\forall_{B}\Big((A=B)\Longleftrightarrow\big(
                    \forall_{C}(C\in{A}\Leftrightarrow{C}\in{B})
                    \land\forall_{D}(A\in{D}\Leftrightarrow{B}\in{D})\big)\Big)
            \end{equation*}
        \end{fnotation}
        \begin{example}
            Consider the set of all planets in the solar system, and consider
            the set of the eight largest objects in the solar system other than
            the sun. These two sets are equal since the eight largest objects
            (other than the sun) are the eight planets (sorry Pluto), and the
            set of planets form the eight largest objects. The tricky part is
            to check that for any set one can name, it is true that if the set
            of planets lies in the set, then the set of the eight largest
            objects not equal to the sun lie in this set as well, and vice
            versa. This is almost impossible, and seemingly redundant, and so we
            rely on the \textit{axiom of extensionality} to ease the
            demonstration of equality.
        \end{example}
        The axiom of extensionality says that to check for equality it suffices
        to show that for all $C$, $C\in{A}$ if and only if $C\in{B}$.
        That is, there is no need to check that for all $D$, $A\in{D}$ if and
        only if $B\in{D}$. For simplicity, the axiom of extensionality may be
        taken as the definition of
        equality\index{Equality}\index{Set!Equal Sets}.
        \begin{faxiom}{Axiom of Extensionality}{Axiom_of_Extensionality}
            If $A$ and $B$ are sets, and if for all $x$ it is true that
            $x\in{A}$ if and only if $x\in{B}$, then $A=B$. That is, $A$ and $B$
            are equal sets.\index{Axiom!of Extensionality}
            \begin{equation*}
                \forall_{A}\forall_{B}\Big(\forall_{x}(x\in{A}\Leftrightarrow
                x\in{B})\Longleftrightarrow\big(A=B\big)\Big)
            \end{equation*}
        \end{faxiom}
        \begin{example}
            Returning to our example of planets, we have seen that the set of
            all planets and the set of the eight largest objects other than the
            sun contain precisely the same elements. By the axiom of
            extensionality, we thus have equality amongst these two.
        \end{example}
        \begin{example}
            Let \gls{mathbbR} denote the real numbers, let \gls{greaterthan} and
            \gls{leq} denote the usual notions of \textit{greater than} and
            \textit{less than or equal to}, respectively. Define $A$ and $B$ by:
            \par
            \begin{subequations}
                \begin{minipage}[b]{0.49\textwidth}
                    \centering
                    \begin{equation}
                        A=\{\,x\in\mathbb{R}\;|\;x>0\,\}
                    \end{equation}
                \end{minipage}
                \hfill
                \begin{minipage}[b]{0.49\textwidth}
                    \centering
                    \begin{equation}
                        B=\{\,y\in\mathbb{R}\;|\;y\not\leq{0}\,\}
                    \end{equation}
                \end{minipage}
            \end{subequations}
            \par\vspace{2.5ex}
            This notation will be justified by the specification axiom
            (see Ax.~\ref{ax:Axiom_Schema_of_Specification}). Then for any real
            number $x\in\mathbb{R}$, we have that $x\in{A}$ if and only if
            $x>0$. That is, $A$ is the set of all positive real numbers. But if
            $x>0$, then $x\ne{0}$ and $x$ is non-negative, so $x\not\leq{0}$.
            Thus $x$ satisfies the criterion for membership of $B$, and
            therefore $x\in{B}$. Similarly, $y\in{B}$ if and only if
            $y\not\leq{0}$ and this is just another way of stating that $y>0$,
            and hence $y\in{A}$. By the axiom of extensionality, $A=B$.
        \end{example}
        We'll restate the definition of equality\index{Equality} using the
        language of subsets\index{Set!Subset}, lessening the effort required in
        proving various things are equal. The notions are equivalent. Subsets
        are sets that are defined in terms of another given set by simply
        removing some (or none, or all) of the elements.
        \begin{fdefinition}{Subsets}{Subsets}
            A \gls{subset} of a \gls{set} $B$ is a set $A$ such that for all
            $x\in{A}$ it is true that $x\in{B}$. If $A$ is a subset of $B$ we
            write \glslink{subseteq}{$A\subseteq{B}$}. Otherwise, we write
            $A\nsubseteq{B}$.\index{Set!Subset}
            \begin{equation*}
                \forall_{A}\forall_{B}\Big(\big(A\subseteq{B}\big)
                \Longleftrightarrow
                \forall_{x}\big(x\in{A}\Rightarrow{x}\in{B}\big)\Big)
            \end{equation*}
        \end{fdefinition}
        We can often visualize sets as blobs in the plane. Using such a visual,
        we can envision subsets as well (Fig.~\ref{fig:Subset_Blobs}). Given a
        blob $B$, a subset of $B$ is another blob $A$ that is entirely contained
        within $B$.
        \begin{figure}[H]
            \centering
            %--------------------------------Dependencies----------------------------------%
%   tikz                                                                       %
%-------------------------------Main Document----------------------------------%
\begin{tikzpicture}[line width=0.2mm, scale=1.2]

    % Coordinates for the bigger blob.
    \coordinate (P1) at ( 0.0, -2.0);
    \coordinate (P2) at ( 1.0, -1.0);
    \coordinate (P3) at ( 1.5,  1.0);
    \coordinate (P4) at ( 0.0,  2.0);
    \coordinate (P5) at (-3.0,  0.0);

    % Coordinates for the inner blob.
    \coordinate (Q1) at ( 0.0, -1.0);
    \coordinate (Q2) at ( 1.0,  0.0);
    \coordinate (Q3) at ( 0.5,  0.5);
    \coordinate (Q4) at (-0.5,  0.5);
    \coordinate (Q5) at (-1.0,  0.0);

    % Coordindates to label things.
    \coordinate (A) at (-0.1, -0.2);
    \coordinate (B) at (-1.5,  0.5);

    % Draw the bigger blob.
    \draw[fill=red, opacity=0.4] (P1) to [out=0,    in=-120] (P2)
                                      to [out=60,   in=-45]  (P3)
                                      to [out=135,  in=0]    (P4)
                                      to [out=-180, in=70]   (P5)
                                      to [out=-110, in=-180] cycle;

    % Draw the inner blob.
    \draw[fill=cyan, opacity=0.8] (Q1) to [out=0,    in=-120]  (Q2)
                                       to [out=60,   in=20]    (Q3)
                                       to [out=-160, in=45]    (Q4)
                                       to [out=-135, in=90]    (Q5)
                                       to [out=-90,  in=180]   cycle;

    % Labels for the two blobs.
    \node at (A) {$A$};
    \node at (B) {$B$};
\end{tikzpicture}

            \caption{Visualizing Subsets as Blobs}
            \label{fig:Subset_Blobs}
        \end{figure}
        \begin{example}
            Consider the set of natural numbers $\mathbb{N}$ and the set of
            integers $\mathbb{Z}$ (loosely defined in
            Eqn.~\ref{eqn:Natural_Numbers_Ellipses} and
            Eqn.~\ref{eqn:Integers_Ellipses}, respectively). It can be seen that
            every natural number is also an integer, and thus we have:
            \begin{equation}
                \mathbb{N}\subseteq\mathbb{Z}
            \end{equation}
            Letting \gls{mathbbQ} denote the rational
            numbers\index{Rational Numbers} $p/q$, where $p,q\in\mathbb{Z}$ and
            $q$ is non-zero, we can see that $\mathbb{Q}$ contains $\mathbb{Z}$
            as a subset. That is, setting $q=1$ and allowing $p$ to vary over
            $\mathbb{Z}$ gives us every integer. Thus:
            \begin{equation}
                \mathbb{Z}\subseteq\mathbb{Q}
            \end{equation}
            We can continue with the real numbers (\gls{mathbbR}) and the
            complex numbers (\gls{mathbbC}) as well, creating a chain of
            subsets:
            \begin{equation}
                \mathbb{N}\subseteq\mathbb{Z}\subseteq\mathbb{Q}
                \subseteq\mathbb{R}\subseteq\mathbb{C}
            \end{equation}
        \end{example}
        \begin{example}
            Let $\mathbb{Z}_{n}$ be the set of all integers $k\in\mathbb{N}$
            such that $k<n$. If $m,n\in\mathbb{N}$ and $m<n$ we see that:
            \begin{equation}
                \mathbb{Z}_{m}\subseteq\mathbb{Z}_{n}
            \end{equation}
            This is because $\mathbb{Z}_{m}$ is the set of all $k\in\mathbb{N}$
            such that $k<m$. But since $\mathbb{Z}_{n}$ is the set of all
            $k\in\mathbb{N}$ such that $k<n$, and since $m<n$, if
            $k\in\mathbb{Z}_{m}$ then $k<m$, and thus $k<n$, which implies that
            $k\in\mathbb{Z}_{n}$.
        \end{example}
        It is important to note the distinction between the symbols for
        containment (\gls{containmentsymb}) for for subset\index{Set!Subset}
        (\gls{subseteq}). The symbol $\in$ is used to denote that some object
        $x$ is an \textit{element}\index{Set!Element of} of some set. That is,
        $x\in{A}$ indicates that $x$ is an element of $A$. This does not
        necessarily imply $x\subseteq{A}$, but this \textit{does} imply that
        $\{x\}\subseteq{A}$. That is, if $x\in{A}$, then the set that contains
        only $x$ is a subset of $A$. Moreover, the notions are not mutually
        exclusive. It is possible for $A$ to be a set such that $x\in{A}$ and
        $x\subseteq{A}$. For let $A=\{\emptyset\}$. For any set $A$ it is true
        that $\emptyset\subseteq{A}$ (see Thm.~\ref{thm:Emptyset_Is_Subset}).
        But from how $A$ is defined, we have that $\emptyset\in{A}$. Thus it is
        true that both $\emptyset\in{A}$ and $\emptyset\subseteq{A}$.
        \begin{fexample}{Elementary Examples of Subsets}
                        {Elementary_Examples_of_Subsets}
            Let $A$ and $B$ be the sets defined by:
            \par
            \begin{subequations}
                \begin{minipage}[b]{0.49\textwidth}
                    \centering
                    \begin{equation}
                        A=\big\{\,a,\,b,\,c\,\big\}
                    \end{equation}
                \end{minipage}
                \hfill
                \begin{minipage}[b]{0.49\textwidth}
                    \centering
                    \begin{equation}
                        B=\big\{\,a,\,b,\,c,\,d\,\big\}
                    \end{equation}
                \end{minipage}
            \end{subequations}
            \par\vspace{2.5ex}
            where we assume that $a$, $b$, $c$, and $d$ are distinct objects.
            From the definition of subsets (Def.~\ref{def:Subsets}):
            \par
            \begin{subequations}
                \begin{minipage}[b]{0.49\textwidth}
                    \centering
                    \begin{equation}
                        A\subseteq{B}
                    \end{equation}
                \end{minipage}
                \hfill
                \begin{minipage}[b]{0.49\textwidth}
                    \centering
                    \begin{equation}
                        B\nsubseteq{A}
                    \end{equation}
                \end{minipage}
            \end{subequations}
            \par\vspace{2.5ex}
            This is true since from the definition of $A$ and $B$, every element
            of $A$ is also an element of $B$. The converse of this is not true
            since there is an element of $B$ that is not an element of $A$
            (namely, the element $d$). That is, $d\in{B}$ but $d\notin{A}$ and
            therefore $B\nsubseteq{A}$.
        \end{fexample}
        The example shown in Ex.~\ref{ex:Elementary_Examples_of_Subsets} hints
        at how we can redefine equality of sets. We see that $A\subseteq{B}$,
        but $B\nsubseteq{A}$. If we have two sets $A$ and $B$ such that
        $A\subseteq{B}$ and $B\subseteq{A}$, then it would be impossible to
        discern between the two. This gives us our new definition of equality.
        We now prove this equivalence with the axiom of extensionality%
        \index{Axiom!of Extensionality} (Ax.~\ref{ax:Axiom_of_Extensionality}).
        \begin{theorem}
            \label{thm:Equivalent_Def_of_Equality}%
            If $A$ and $B$ are sets, then $A=B$ if and only if $A\subseteq{B}$
            and $B\subseteq{A}$.
        \end{theorem}
        \begin{proof}
            By the axiom of extensionality
            (Ax.~\ref{ax:Axiom_of_Extensionality}), $A=B$ if and only if for
            all $x$ it is true that $x\in{A}$ if and only if $x\in{B}$. But then
            $x\in{A}$ implies that $x\in{B}$, and thus $A\subseteq{B}$
            (Def.~\ref{def:Subsets}). But also $x\in{B}$ implies $x\in{A}$, and
            therefore $B\subseteq{A}$. Therefore if $A=B$, then $A\subseteq{B}$
            and $B\subseteq{A}$. Now if $A\subseteq{B}$ and $B\subseteq{A}$,
            then for all $x\in{A}$ it is true that $x\in{B}$ and for all
            $x\in{B}$ it is true that $x\in{A}$ (Def.~\ref{def:Subsets}), and
            therefore $x\in{A}$ if and only if $x\in{B}$. Thus if
            $A\subseteq{B}$ and $B\subseteq{A}$, then $A=B$. But it was just
            proved that if $A=B$, then $A\subseteq{B}$ and $B\subseteq{A}$.
            Therefore $A=B$ if and only if $A\subseteq{B}$ and $B\subseteq{A}$.
        \end{proof}
        With this, we can redefine the notion of
        \textit{equal sets}\index{Set!Equal Sets}.
        \begin{fdefinition}{Equal Sets}{Equal_Sets}
            \Glspl{equal set} are \glspl{set} $A$ and $B$, denoted
            \glslink{equalsymb}{$A=B$}, such that $A\subseteq{B}$ and
            $B\subseteq{A}$.\index{Set!Equality}
            \begin{equation*}
                \forall_{A}\forall_{B}\Big((A=B)\Longleftrightarrow
                \big((A\subseteq{B})\land(B\subseteq{A})\big)\Big)
            \end{equation*}
        \end{fdefinition}
        Def.~\ref{def:Equal_Sets} is justified by
        Thm.~\ref{thm:Equivalent_Def_of_Equality}, and thus there is no
        contradiction with the axiom of extensionality
        (Ax.~\ref{ax:Axiom_of_Extensionality}). If $A$ and $B$ are not equal, we
        write $A\ne{B}$. 
        \begin{lexample}{More Examples of Subsets}{More_Examples_of_Subsets}
            Using the notation from Ex.~\ref{ex:Using_Element_Notation}, for all
            $n\in\mathbb{N}$ we have:
            \begin{equation}
                \mathbb{Z}_{n}\subseteq\mathbb{N}
            \end{equation}
            Let's define \gls{mathbbNe} and \gls{mathbbNo} to be the sets of
            even\index{Natural Numbers!Even} and odd\index{Natural Numbers!Odd}
            non-negative integers, respectively:
            \par
            \begin{subequations}
                \begin{minipage}[b]{0.49\textwidth}
                    \centering
                    \begin{equation}
                        \label{eqn:Even_Pos_Ints_Ellipses}%
                        \mathbb{N}_{e}=\big\{\,0,\,2,\,4,\,6,\,8,\,\dots\,\big\}
                    \end{equation}
                \end{minipage}
                \hfill
                \begin{minipage}[b]{0.49\textwidth}
                    \centering
                    \begin{equation}
                        \label{eqn:Odd_Pos_Ints_Ellipses}%
                        \mathbb{N}_{o}=\big\{\,1,\,3,\,5,\,7,\,9,\,\dots\,\big\}
                    \end{equation}
                \end{minipage}
            \end{subequations}
            \par\vspace{2.5ex}
            From this we see the following two expressions are true:
            \par
            \begin{subequations}
                \begin{minipage}[b]{0.49\textwidth}
                    \centering
                    \begin{equation}
                        \mathbb{N}_{o}\subseteq\mathbb{N}
                    \end{equation}
                \end{minipage}
                \hfill
                \begin{minipage}[b]{0.49\textwidth}
                    \centering
                    \begin{equation}
                        \mathbb{N}_{e}=\mathbb{N}
                    \end{equation}
                \end{minipage}
            \end{subequations}
            \par\vspace{2.5ex}
            Moreover we see that $\mathbb{N}_{o}$ and $\mathbb{N}_{e}$ have no
            elements in common. That is, they are \textit{disjoint}%
            \index{Set!Disjoint Sets}. From this, we can write:
            \par
            \begin{subequations}
                \begin{minipage}[b]{0.49\textwidth}
                    \centering
                    \begin{equation}
                        \mathbb{N}_{o}\nsubseteq\mathbb{N}_{e}
                    \end{equation}
                \end{minipage}
                \hfill
                \begin{minipage}[b]{0.49\textwidth}
                    \centering
                    \begin{equation}
                        \mathbb{N}_{e}\nsubseteq\mathbb{N}_{o}
                    \end{equation}
                \end{minipage}
            \end{subequations}
            \par\vspace{2.5ex}
            We can also think of trivial examples:
            \par
            \begin{subequations}
                \begin{minipage}[b]{0.49\textwidth}
                    \centering
                    \begin{equation}
                        \mathbb{Z}_{3}\subseteq\mathbb{Z}_{4}
                    \end{equation}
                \end{minipage}
                \hfill
                \begin{minipage}[b]{0.49\textwidth}
                    \centering
                    \begin{equation}
                        \mathbb{Z}_{4}\nsubseteq\mathbb{Z}_{3}
                    \end{equation}
                \end{minipage}
            \end{subequations}
            \par\vspace{2.5ex}
            This is because every element of $\mathbb{Z}_{3}$ is contained in
            $\mathbb{Z}_{4}$, but $3\in\mathbb{Z}_{4}$ and
            $3\notin\mathbb{Z}_{3}$.
        \end{lexample}
        It may seem like bad notation to write $3\notin\mathbb{Z}_{3}$, but
        since we want $\mathbb{Z}_{n}$ to have $n$ elements, and since we
        started counting at zero, we have that $n\notin\mathbb{Z}_{n}$ for all
        $n\in\mathbb{N}$. Such counting schemes are common in computer science,
        but there's disagreement in mathematics as to whether $0\in\mathbb{N}$
        or not. We will use the
        \textit{axiom of infinity}\index{Axiom!of Infinity} to prove the
        existence of $\mathbb{N}$, and in doing so it will be natural to define
        $\mathbb{N}$ as a set that contains $0$.
        \par\hfill\par
        While Def.~\ref{def:Equal_Sets} is indeed equivalent to the axiom of
        extensionality, this definition creates a few problems. As discussed
        previously, sets have no notion of order and cannot account for
        repetition. For let $A$, $B$, and $C$ be the sets defined by:
        \par
        \begin{subequations}
            \begin{minipage}[b]{0.31\textwidth}
                \centering
                \begin{equation}
                    A=\big\{\,a,\,b\,\big\}
                \end{equation}
            \end{minipage}
            \hfill
            \begin{minipage}[b]{0.36\textwidth}
                \centering
                \begin{equation}
                    B=\big\{\,a,\,a,\,b\,\big\}
                \end{equation}
            \end{minipage}
            \hfill
            \begin{minipage}[b]{0.31\textwidth}
                \centering
                \begin{equation}
                    C=\big\{\,b,\,a\,\big\}
                \end{equation}
            \end{minipage}
        \end{subequations}
        \par\vspace{2.5ex}
        All three of these sets are equal by both the definition of equality
        (Def.~\ref{def:Equal_Sets})\index{Set!Equal Sets} and the axiom of
        extensionality\index{Axiom!of Extensionality}. It seems clear that
        $A\subseteq{B}$, but it is also true that $B\subseteq{A}$. This is
        because $B$ contains only the elements $a$ and $b$. While $a$ is
        included twice, repetition cannot be accounted for and $B$ is entirely
        determined by $a$ and $b$. But $A$ also contains $a$ and $b$, and
        therefore $B\subseteq{A}$. By the definition of equality
        (Def.~\ref{def:Equal_Sets}), we have that $A=B$. In a similar manner,
        $A=C$. From the definition of subsets, for any set $A$ we see that
        $A\subseteq{A}$ (see Thm.~\ref{thm:Reflexivity_of_Inclusion}). It would
        be nice to distinguish between subsets that aren't the entire set
        itself. These are called proper subsets\index{Set!Subset!Proper}, and
        we can define them in terms of equality.
        \begin{fdefinition}{Proper Subsets}{Proper_Subsets}
            A \gls{proper subset} of a \gls{set} $B$ is a set $A$ such that
            $A\subseteq{B}$ and $A\ne{B}$. We write
            \glslink{subsetneq}{$A\subsetneq{B}$} to denote that $A$ is a proper
            subset of $B$.\index{Set!Subset!Proper}
            \begin{equation*}
                \forall_{A}\forall_{B}(A\subsetneq{B})
                \Longleftrightarrow\Big((A\subseteq{B})\land(A\ne{B})\Big)
            \end{equation*}
        \end{fdefinition}
        The symbols \gls{subseteq} and \gls{subsetneq} are analogous to the
        notations of inequalities that one finds in calculus: \gls{leq} and
        \gls{lessthan}. In many texts, the two symbols $\subseteq$ and $\subset$
        are taken to be identical, which may cause confusion. In an attempt to
        reduce confusion, $\subseteq$ will denote any subset, $\subsetneq$
        denotes a proper subset, and the symbol $\subset$ will be avoided.
        \begin{lexample}{Proper Subsets}{Proper_Subsets}
            Let $A$ and $B$ be sets defined as follows:
            \par
            \begin{subequations}
                \begin{minipage}[b]{0.49\textwidth}
                    \centering
                    \begin{equation}
                        A=\big\{\,a,\,b,\,c\,\big\}
                    \end{equation}
                \end{minipage}
                \hfill
                \begin{minipage}[b]{0.49\textwidth}
                    \centering
                    \begin{equation}
                        B=\big\{\,a,\,b,\,c,\,d\,\big\}
                    \end{equation}
                \end{minipage}
            \end{subequations}
            \par\vspace{2.5ex}
            Then $A\subseteq{B}$, since every element of $A$ is an element of
            $B$, but $B\nsubseteq{A}$ since $d\in{B}$ and $d\notin{A}$.
            Therefore $A\ne{B}$, and thus $A$ is a proper subset of $B$. We can
            denote this by writing $A\subsetneq{B}$.
        \end{lexample}
        Proper subsets are subsets that are missing at least one element
        (see Thm.~\ref{thm:Prop_Subset_Not_Equal}). Returning to our claim that
        $\emptyset\subseteq{A}$ for any set $A$, for any non-empty set $A$ we
        have that $\emptyset\subsetneq{A}$. This is because for non-empty sets
        there is at least one $x$ such that $x\in{A}$
        (Def.~\ref{def:Non_Empty_Set}), whereas for all $x$ it is true that
        $x\notin\emptyset$. Thus equality cannot occur, and the empty set must
        be a proper subset. It is also true that the empty set contains no
        proper subsets. The only subset of $\emptyset$ is itself.
        \begin{example}
            Returning to more concrete examples, $\mathbb{N}$ is a proper subset
            of $\mathbb{Z}$. To see this, note that $\minus{1}\in\mathbb{Z}$ but
            $\minus{1}\notin\mathbb{N}$. Indeed, none of the negative integers
            are natural numbers, but they are integers. We can write this by:
            \begin{equation}
                \mathbb{N}\subsetneq\mathbb{Z}
            \end{equation}
            Similarly, $\mathbb{Q}$ contains numbers that are not integers,
            for example $1/2$. Thus, $\mathbb{Z}$ is also a proper subset of
            $\mathbb{Q}$. Lastly, since $\sqrt{2}$ is not a rational number, the
            set of rational numbers must then be a proper subset of the set of
            real numbers.
        \end{example}
        We now introduce the \textit{axiom schema of specification}.
        \begin{faxiom}{Axiom Schema of Specification}
                      {Axiom_Schema_of_Specification}
            If $A$ is a set and if $P$ is a proposition, then there exists a set
            $B$ such that $x\in{B}$ if and only if $x\in{A}$ and $P(x)$ is true.
            We can write this as:\index{Axiom!Schema of Specification}
            \index{Proposition}
            \begin{equation*}
                B=\big\{\,x\in{A}\;|\;P(x)\,\big\}
            \end{equation*}
            Using our formal language, we have:
            \begin{equation*}
                \forall_{A}\forall_{P}\exists_{B}:
                \forall_{x}\Big((x\in{B})\Leftrightarrow
                \big((x\in{A})\land{P}(x)\big)\Big)
            \end{equation*}
        \end{faxiom}
        Ax.~\ref{ax:Axiom_Schema_of_Specification} is different from the
        inconsistent axiom of unrestricted comprehension%
        \index{Axiom!of Unrestricted Comprehension} in that we can only speak of
        elements that are already defined and contained in some other set. That
        is, this new axiom does not allow us to talk about the
        \textit{set of all sets}\index{Set!of All Sets}, and so we have avoided
        the crux of Russell's paradox.\index{Russell's Paradox}
        \par\hfill\par
        This allows us to use the Set-Builder\index{Set-Builder Notation} method
        of constructing sets. We described the natural numbers \gls{mathbbN} and
        integers \gls{mathbbZ} (From the German \textit{Zahl}) using
        Eqns.~\ref{eqn:Natural_Numbers_Ellipses} and
        \ref{eqn:Integers_Ellipses}, respectively. It would be more difficult
        (but not impossible) to describe the set of rational numbers%
        \index{Rational Numbers} in such a way. Instead, we use set-builder
        notation if it is known that \gls{mathbbQ} is contained in some larger
        set \gls{mathbbR} (the \textit{real} numbers)\index{Real Numbers}.
        \begin{equation}
            \mathbb{Q}=\Big\{\;\frac{p}{q}\in\mathbb{R}\;\big|\;
                                p,\,q\in\mathbb{Z}\textrm{ and }q\ne{0}\;\Big\}
        \end{equation}
        That is, the rational numbers are the set of all real numbers which can
        be written as the ratios of integers with non-zero denominator. The
        Axiom Schema of Specification states that this is is a valid method of
        describing sets. It is also known as the axiom of separation%
        \index{Axiom!of Separation}.
        \begin{example}
            We can describe the sets $\mathbb{Z}$, $\mathbb{N}$,
            $\mathbb{N}_{e}$, and $\mathbb{N}_{o}$ using set-builder notation if
            we assume these belong to some larger set $\mathbb{R}$. We define
            $\mathbb{Z}$ by:
            \index{Natural Numbers}\index{Natural Numbers!Even}%
            \index{Natural Numbers!Odd}\index{Integers}
            \begin{equation}
                \mathbb{Z}=
                \big\{\,n\in\mathbb{R}\;|\;n\textrm{ is an integer}\,\big\}
            \end{equation}
            From here we can define $\mathbb{N}$ by:
            \begin{equation}
                \mathbb{N}=\{\,n\in\mathbb{Z}\;|\;n\geq{0}\,\}
            \end{equation}
            Furthermore, $\mathbb{N}_{e}$ and $\mathbb{N}_{0}$ can be described
            as follows:
            \par
            \begin{subequations}
                \begin{minipage}[b]{0.495\textwidth}
                    \centering
                    \begin{equation}
                        \label{eqn:Even_Pos_Ints_Set_Builder}%
                        \mathbb{N}_{e}=
                        \big\{n\in\mathbb{N}\;|\;n\textrm{ is even}\big\}
                    \end{equation}
                \end{minipage}
                \hfill
                \begin{minipage}[b]{0.495\textwidth}
                    \centering
                    \begin{equation}
                        \label{eqn:Odd_Pos_Ints_Set_Builder}%
                        \mathbb{N}_{o}=
                        \big\{n\in\mathbb{N}\;|\;n\textrm{ is odd}\big\}
                    \end{equation}
                \end{minipage}
            \end{subequations}
            \par\vspace{2.5ex}
            Such notation is justified by the axiom schema of specification%
            \index{Axiom!Schema of Specification}.
        \end{example}
        We are not adopting these definitions since they lack rigor. These
        examples build intuition behind the notation and the axioms, but we will
        develop arithmetic axiomatically using the \textit{axiom of infinity}%
        \index{Axiom!of Infinity}.
    \subsection{Ordered Pairs and Unions}
        We now wish to solve the issue previously raised that sets do
        not have order. We'll develop a new object, called ordered pairs%
        \index{Ordered Pair}, that can distinguish such things. The definition
        we'll adopt is due to Kuratowski\index{Kuratowski, Kazimierz} and
        defines \gls{orderedpairsymb} as follows:
        \begin{equation}
            (a,\,b)=\big\{\,\{\,a\,\},\,\{\,a,\,b\,\}\,\big\}
        \end{equation}
        We now prove such a set exists within the framework of \gls{ZFC}.
        \begin{faxiom}{Axiom of Pairing}{Axiom_of_Pairing}
            If $A$ and $B$ are sets, then there exists a set $\mathcal{C}$
            such that $A\in\mathcal{C}$ and $B\in\mathcal{C}$.
            \index{Axiom!of Pairing}
            \begin{equation*}
                \forall_{A}\forall_{B}\exists_{\mathcal{C}}:
                \Big((A\in\mathcal{C})\land(B\in\mathcal{C})\Big)
            \end{equation*}
        \end{faxiom}
        The set hypothesized to exist in this axiom may be very large, we have
        no way of knowing. What we want from this is a set that contains two
        elements $A$ and $B$, and only those elements. We obtain this by
        combining pairing with specification.
        \begin{theorem}
            \label{thm:Existence_of_Set_Built_from_Two_Sets}%
            If $A$ and $B$ are sets, then there exists a set $D$ such that
            for all $x$ it is true that $x\in{D}$ if and only if $x=A$ or
            $x=B$. That is:
            \begin{equation}
                D=\{\,A,\,B\,\}
            \end{equation}
        \end{theorem}
        \begin{proof}
            By the axiom of pairing (Ax.~\ref{ax:Axiom_of_Pairing}) there
            exists a set $\mathcal{C}$ such that $A\in\mathcal{C}$ and
            $B\in\mathcal{C}$. Let $P$ be the proposition
            \textit{true if} $x=A$ \textit{or} $x=B$, \textit{false otherwise}.
            By the axiom schema of specification
            (Ax.~\ref{ax:Axiom_Schema_of_Specification}), there is a set
            $D$ such that:
            \begin{equation}
                D=\{\,x\in\mathcal{C}\;|\;P(x)\,\}
            \end{equation}
            That is, $x\in{D}$ if and only if $x\in\mathcal{C}$ and $P(x)$ is
            true. But then $x\in{D}$ if and only if $x\in\mathcal{C}$ and $x=A$
            or $x\in\mathcal{C}$ and $x=B$. But $A\in\mathcal{C}$ and
            $B\in\mathcal{C}$, and thus $P(x)$ implies $x\in\mathcal{C}$. Thus,
            $x\in{D}$ if and only if $P(x)$ is true. That is, $x\in{D}$ if and
            only if $x=A$ or $x=B$.
        \end{proof}
        By the axiom of extensionality\index{Axiom!of Extensionality}
        (Ax.~\ref{ax:Axiom_of_Extensionality}), the set hypothesized in
        Thm.~\ref{thm:Existence_of_Set_Built_from_Two_Sets} is unique, and thus
        there is no trouble in \textit{defining} the symbol $\{A,B\}$ to be the
        unique set that contains the elements $A$ and $B$ and only those
        elements. That is, we develop the new notation:
        \begin{fnotation}{Finite Set Notation}{Finite_Set_Notation}
            If $A$ and $B$ are sets, then $\{A,B\}$ is the unique set such that
            for all $x$, $x\in\{A,B\}$ if and only if $x=A$ or $x=B$.
            \index{Set!Finite}
            \begin{equation*}
                \forall_{x}\Big(\big(x\in\{A,B\}\big)
                \Longleftrightarrow\big((x=A)\lor(x=B)\big)\Big)
            \end{equation*}
        \end{fnotation}
        \begin{theorem}
            \label{thm:Existence_of_Set_Containing_Set}%
            If $A$ is a set, then there is a set $B$ such that $x\in{B}$ if
            and only if $x=A$. That is, there exists a set $B$ such that:
            \begin{equation}
                B=\{\,A\,\}
            \end{equation}
        \end{theorem}
        \begin{proof}
            For since $A$ is a set, by
            Thm.~\ref{thm:Existence_of_Set_Built_from_Two_Sets} there exists
            a set $B=\{A,\,A\}$. But then $x\in{B}$ if and only if $x=A$.
        \end{proof}
        We can apply Not.~\ref{not:Finite_Set_Notation} to a single set $A$ and
        similarly define what the notation $\{A\}$ means. With this, we can now
        prove the existence of ordered pairs.
        \begin{ltheorem}{Existence of Ordered Pairs}{Existence_of_Ordered_Pairs}
            If $A$ and $B$ are sets, then there is a set $(A,\,B)$ such that
            for all $x$ it is true that $x\in(A,\,B)$ if only if $x=\{A\}$
            or $x=\{A,B\}$.\index{Ordered Pair}
        \end{ltheorem}
        \begin{proof}
            For by Thm.~\ref{thm:Existence_of_Set_Containing_Set}, there is
            a set $\{A\}$ such that $x\in\{A\}$ if and only if $x=A$.
            But by Thm.~\ref{thm:Existence_of_Set_Built_from_Two_Sets}, there
            is a set $\{A,\,B\}$ such that $x\in\{A,\,B\}$ and if only
            if $x=A$ or $x=B$. But again by
            Thm.~\ref{thm:Existence_of_Set_Built_from_Two_Sets}, since
            $\{A\}$ and $\{A,\,B\}$ are sets, there is a set $(A,\,B)$ such
            that $x\in(A,\,B)$ if and only if $x=\{A\}$ or $x=\{A,\,B\}$.
        \end{proof}
        \begin{fdefinition}{Ordered Pairs}{Ordered_Pairs}
            The \gls{ordered pair} of a \gls{set} $x$ with respect to a set
            $y$ is the set:\index{Ordered Pair}
            \begin{equation*}
                (x,\,y)=\big\{\,\{\,x\,\},\,\{\,x,\,y\,\}\,\big\}
            \end{equation*}
            Using our formal language:
            \begin{equation*}
                \forall_{x}\forall_{y}\forall_{z}\Big(
                    \big(z\in(x,y)\big)\Longleftrightarrow
                    \big((z=\{x\})\lor(z=\{x,y\})\big)
                \Big)
            \end{equation*}
        \end{fdefinition}
        Thm.~\ref{thm:Existence_of_Ordered_Pairs} asserts the existence of
        ordered pairs\index{Ordered Pair}, as defined by Kuratowski%.
        \index{Kuratowski, Kazimierz}, and allows us to present
        Def.~\ref{def:Ordered_Pairs} in a way that is consistent with ZFC.
        Kuratowski first put forward this definition in 1921 and it does
        precisely what we want it to do and orders elements. That is, if $x$ and
        $y$ are distinct, then $(x,\,y)\ne(y,\,x)$. The caveat with this
        definition is the following reduction:
        \begin{equation}
            (x,\,x)
            =\big\{\,\{\,x\,\},\,\{\,x,\,x\,\}\,\big\}
            =\big\{\,\{\,x\,\},\{\,x\,\}\,\big\}
            =\big\{\,\{\,x\,\}\,\big\}
        \end{equation}
        Prior to Kuratowski there existed a definition due to Norbert Wiener%
        \index{Wiener, Norbert}, put forward in 1914. His definition grew out of
        Bertrand Russell's\index{Russell, Bertrand} Type
        Theory\index{Type Theory} which was an attempt to rid set theory of the
        paradoxes he discovered. Wiener writes:
        \begin{equation}
            (x,\,y)_{W}=\Big\{\,\big\{\,\{\,x\,\},\,\emptyset\,\big\},\,
                                \big\{\,\{\,y\,\}\,\big\}\Big\}
        \end{equation}
        Returning to Kuratowski's definition (Def.~\ref{def:Ordered_Pairs}),
        consider the ordered pair $(1,\,2)$, where we take for granted that
        $1\ne{2}$. We have:
        \begin{equation}
            (1,\,2)=\big\{\,\{\,1\,\},\,\{\,1,\,2\,\}\,\big\}
        \end{equation}
        Swapping and computing $(2,\,1)$, we obtain:
        \begin{equation}
            (2,\,1)=\big\{\,\{\,2\,\},\,\{\,2,\,1\,\}\,\big\}
        \end{equation}
        We know that sets cannot distinguish order, so
        $\{\,1,\,2\,\}=\{\,2,\,1\,\}$. Thus:
        \par
        \begin{subequations}
            \begin{minipage}[b]{0.49\textwidth}
                \centering
                \begin{equation}
                    (1,\,2)=\big\{\,\{\,1\,\},\,\{\,1,\,2\,\}\,\big\}
                \end{equation}
            \end{minipage}
            \hfill
            \begin{minipage}[b]{0.49\textwidth}
                \centering
                \begin{equation}
                    (2,\,1)=\big\{\,\{\,2\,\},\,\{\,1,\,2\,\}\,\big\}
                \end{equation}
            \end{minipage}
        \end{subequations}
        \par\vspace{2.5ex}
        Combining these equations, we now have that:
        \begin{equation}
            (1,\,2)\ne(2,\,1)
        \end{equation}
        To see this, note that both sets contain the element $\{1,\,2\}$, but
        $\{1\}$ is an element of $(1,\,2)$ and not an element of $(2,\,1)$,
        and thus $(1,\,2)\nsubseteq(2,\,1)$. Similarly, $\{2\}$ is an element
        of $(2,\,1)$ but not an element of $(1,\,2)$, and therefore
        $(2,\,1)\nsubseteq(1,\,2)$. From the definition of equality
        (Def.~\ref{def:Equal_Sets}), we have that these sets are not equal.
        \par\hfill\par
        There's is an unfortunate doubling of notation that occurs in
        mathematics, and $(a,b)$ has two common meanings. The first meaning is
        the ordered pair which we've just defined, and the second is the
        \textit{open interval}\index{Interval!Open} defined in the context of a
        \textit{partially ordered set}\index{Partially Ordered Set}.
        The most common example is when discussing the real numbers
        $\mathbb{R}$, $(a,b)$ denotes the set of all real numbers $x$ such that
        $a<x$ and $x<b$. Hopefully it will be clear what the notation means when
        a theorem or example is being presented, but we will be explicit when
        ambiguity can arise.
        \par\hfill\par
        The natural thing from here is to construct the
        \textit{Cartesian Product}\index{Cartesian Product} of two sets. This is
        the set of all ordered pairs\index{Ordered Pair} $(a,\,b)$ where $a$
        belongs to some set $A$ and $b$ belongs to another set $B$. To prove
        such a set exists requires two more axioms.
        \newpage
        \begin{faxiom}{Axiom of Union}{Axiom_of_Union}
            If $\mathcal{O}$ is a set, then there exists a set $\mathcal{F}$
            such that, for all $A$ such that $A\in\mathcal{O}$ and for all
            $x$ such that $x\in{A}$, it is true that $x\in\mathcal{F}$.%
            \index{Axiom!of Union}
            \begin{equation*}
                \forall_{\mathcal{O}}\exists_{\mathcal{F}}:\forall_{x}\Big(
                \big(\exists_{A\in\mathcal{O}}:x\in{A}\big)
                \Rightarrow{x}\in\mathcal{F}\Big)
            \end{equation*}
        \end{faxiom}
        This states that, given a collection of sets $\mathcal{O}$, there exists
        a larger set which contains the elements of the constituent sets of
        $\mathcal{O}$. Similar to the axiom of pairing, $\mathcal{F}$ may be
        much larger than desired and we must invoke the axiom schema of
        specification to arrive at the \textit{union} over a collection.
        \begin{ltheorem}{Existence of the Union of Sets}{Existence_of_Unions}
            If $\mathcal{O}$ is a set, then there exists a set \gls{unionO} such
            that for all $x$ it is true that $x\in\bigcup\mathcal{O}$ if and
            only if there is a set $A\in\mathcal{O}$ such that $x\in{A}$.
        \end{ltheorem}
        \begin{proof}
            For by the axiom of union (Ax.~\ref{ax:Axiom_of_Union}), there
            exists a set $\mathcal{F}$ such that for all $A\in\mathcal{O}$
            and for all $x\in{A}$ it is true that $x\in\mathcal{F}$. Let
            $P$ be the proposition \textit{true if there exists a set}
            $A\in\mathcal{O}$ \textit{such that} $x\in{A}$,
            \textit{false otherwise}. Then, by the axiom schema of specification
            (Ax.~\ref{ax:Axiom_Schema_of_Specification}) there exists a set
            $\bigcup\mathcal{O}$ such that:
            \begin{equation}
                \bigcup\mathcal{O}=\big\{\,x\in\mathcal{F}\;|\;P(x)\,\big\}
            \end{equation}
            But then $x\in\bigcup\mathcal{O}$ if and only if $x\in\mathcal{F}$
            and $P(x)$ is true. But if $P(x)$ is true, then $x\in\mathcal{F}$,
            and thus $x\in\bigcup\mathcal{O}$ if and only if there is a set
            $A\in\mathcal{O}$ such that $x\in{A}$.
        \end{proof}
        One question that arises is
        \textit{what happens if our collection is empty}? That is, if
        $\mathcal{O}=\emptyset$, is there any meaning behind the equation:
        \begin{equation}
            \mathcal{F}=\bigcup\emptyset
        \end{equation}
        There is, and $\mathcal{F}$ will be the empty set. That is,
        $\mathcal{F}=\emptyset$. This is true in a vacuous sense and can be
        proved via contradiction with the law of the excluded middle.
        We define the set $\mathcal{F}$ described in
        Thm.~\ref{thm:Existence_of_Unions} as the \textit{union}%
        \index{Union (Sets)} over the set $\mathcal{O}$. The set $\mathcal{O}$
        is often called the index set\index{Index Set} for which we take the
        union over.
        \begin{fdefinition}{Union over a Set}{Union_over_a_Set}
            The \gls{union over a set} $\mathcal{O}$ is the set:
            \index{Set!Union}\index{Union!over a Collection}
            \begin{equation*}
                \bigcup_{\mathcal{U}\in\mathcal{O}}\mathcal{U}
                =\big\{\,x\;|\;\textrm{There exists a set }
                         \mathcal{U}\in\mathcal{O}\textrm{ such that }
                         x\in\mathcal{U}\big\}
            \end{equation*}
            Using our formal language:
            \begin{equation*}
                \forall_{\mathcal{O}}\forall_{x}\bigg(
                    \Big(x\in\bigcup_{\mathcal{U}\in\mathcal{O}}\mathcal{U}\Big)
                    \Longleftrightarrow\big(\exists_{\mathcal{U}\in\mathcal{O}}:
                    x\in\mathcal{U}\big)\bigg)
            \end{equation*}
        \end{fdefinition}
        There are two ways to write unions for arbitrary collections, and we
        will make use of both depending on scenario. The first manner we have
        already seen in Thm.~\ref{thm:Existence_of_Unions} where, given a
        collection $\mathcal{O}$, we wrote \gls{unionO} to denote the union over
        $\mathcal{O}$. The second is depicted in
        Def.~\ref{def:Union_over_a_Set}. That is, given $\mathcal{O}$ we write:
        \begin{equation}
            \bigcup\mathcal{O}=\bigcup_{\mathcal{U}\in\mathcal{O}}\mathcal{U}
        \end{equation}
        This alternative notation can be useful when we are using various
        indexing tricks to solve problems, or when combining unions with the
        various other set operations such as differences and intersections.
        \par\hfill\par
        The notion of union is very convenient if we already have a collection
        of sets defined, but it would be nice to form the union over two given
        sets without considering them as part of a larger collection. This can
        be done by combining the axiom of union\index{Axiom!of Union} with
        pairing\index{Axiom!of Pairing}.
        \begin{theorem}
            \label{thm:Union_of_Two_Sets_Exists}%
            If $A$ and $B$ are sets, then there exists a set \gls{AcupB} such
            that $x\in{A}\cup{B}$ if and only if either $x\in{A}$ or $x\in{B}$.
        \end{theorem}
        \begin{proof}
            For by Thm.~\ref{thm:Existence_of_Set_Built_from_Two_Sets} there
            exists a set $\mathcal{O}$ such that $y\in\mathcal{O}$ if and only
            if $y=A$ or $y=B$. That is, $\mathcal{O}=\{A,\,B\}$. But by
            Thm.~\ref{thm:Existence_of_Unions} there exists a set $A\cup{B}$
            such that $x\in{A}\cup{B}$ if and only if there exists a set
            $\mathcal{U}\in\mathcal{O}$ such that $x\in\mathcal{U}$. But then
            $x\in{A}\cup{B}$ if and only if either $x\in{A}$ or $x\in{B}$.
        \end{proof}
        This allows us to define our first \textit{operation} of two sets.
        \begin{fdefinition}{Union of Two Sets}{Union_of_Two_Sets}
            The \gls{union of two sets} $A$ and $B$ is the set \gls{AcupB}
            defined by:\index{Union!of Two Sets}
            \begin{equation*}
                A\cup{B}=\big\{\,x\;|\;x\in{A}\textrm{ or }x\in{B}\,\big\}
            \end{equation*}
            That is:
            \begin{equation*}
                \forall_{A}\forall_{B}\forall_{x}\Big(
                    \big(x\in{A}\cup{B}\big)
                    \Longleftrightarrow
                    \big((x\in{A})\lor(x\in{B})\big)
                \Big)
            \end{equation*}
        \end{fdefinition}
        In our definition of the union over a collection and the union of two
        sets we have slightly abused our set-builder
        notation\index{Set-Builder Notation}. The axiom schema of
        specification\index{Axiom!Schema of Specification} allows us to write a
        set as $A=\{\,x\in{B}\,|\,P(x)\,\}$ given some set $B$ that is already
        known to exists, and some proposition\index{Proposition} $P$. These two
        definitions (Defs.~\ref{def:Union_over_a_Set} and
        \ref{def:Union_of_Two_Sets}) are justified by the theorems we have
        proven, and so there is no contradiction.
        \begin{example}
            \label{ex:Union_of_Zm_and_Zn}%
            Again using the notation found in Eqn.~\ref{eqn:Z_n_Ellipses}, if we
            let $\mathbb{Z}_{n}$ denote the integers between $0$ and $n-1$, we
            have the following: If $m$ is less than $n$, then:
            \begin{equation}
                \mathbb{Z}_{m}\cup\mathbb{Z}_{n}=\mathbb{Z}_{n}
            \end{equation}
            This is because every element of $\mathbb{Z}_{m}$ is already
            and element of $\mathbb{Z}_{n}$, and thus taking the union adds
            nothing new to $\mathbb{Z}_{n}$, so the resulting set is
            $\mathbb{Z}_{m}$.
        \end{example}
        \begin{example}
            Denoting the even and odd non-negative integers by \gls{mathbbNe}
            and \gls{mathbbNo}, respectively, we see that:
            \index{Natural Numbers!Even}\index{Natural Numbers!Odd}
            \begin{equation}
                \mathbb{N}_{e}\cup\mathbb{N}_{o}=\mathbb{N}
            \end{equation}
            This is because every non-negative integer $n\in\mathbb{N}$ is
            either even or odd, and thus either $n\in\mathbb{N}_{e}$ or
            $n\in\mathbb{N}_{o}$. Taking the union therefore gives the entire
            set $\mathbb{N}$. The union does not add anything more than
            $\mathbb{N}$ since $\mathbb{N}_{e}\subseteq\mathbb{N}$ and
            $\mathbb{N}_{o}\subseteq\mathbb{N}$.
        \end{example}
        \begin{fexample}{Union of Two Sets}{Union_of_Two_Sets}
            Let $A$ and $B$ be the sets defined by:
            \par
            \begin{subequations}
                \begin{minipage}[b]{0.49\textwidth}
                    \centering
                    \begin{equation}
                        A=\{\,a,\,b,\,c\,\}
                    \end{equation}
                \end{minipage}
                \hfill
                \begin{minipage}[b]{0.49\textwidth}
                    \centering
                    \begin{equation}
                        B=\{\,c,\,1,\,2\,\}
                    \end{equation}
                \end{minipage}
            \end{subequations}
            \par\vspace{2.5ex}
            The union of $A$ and $B$ is the set that contains all of the
            elements of $A$ and all of the elements of $B$, and only such
            elements. That is:
            \begin{equation}
                A\cup{B}=\{\,a,\,b,\,c,\,1,\,2\,\}
            \end{equation}
            Even though $c\in{A}$ and $c\in{B}$, $c$ only appears once in the
            union. This is because sets cannot account for repetition, so
            including $c$ twice would be redundant.
        \end{fexample}
        The union of two sets can again be visualized by considering blobs
        in the plane. Let $A$ and $B$ be two circles that overlap somewhere in
        the middle. The union $A\cup{B}$ can then be represented by shading in
        the region covered by either $A$ or $B$
        (see Fig.~\ref{fig:Union_of_Two_Sets}). Such a drawing is called a
        \textit{Venn diagram}\index{Venn Diagram}.
        \begin{figure}[H]
            \centering
            \captionsetup{type=figure}
            %--------------------------------Dependencies----------------------------------%
%   tikz                                                                       %
%-------------------------------Main Document----------------------------------%
\begin{tikzpicture}[line width=0.2mm]

    % Coordinates for the centers of the circles.
    \coordinate (C1) at (-1.3, 0);
    \coordinate (C2) at ( 1.3, 0);

    % Coordinates for the labels.
    \coordinate (A) at (-1.3, 1.2);
    \coordinate (B) at ( 1.3, 1.2);
    \coordinate (U) at ( 0.0, 2.5);

    % Rectangle indicating the universe set.
    \draw (-4, -3) rectangle (4, 3);

    % Fill in the circle with cyan.
    \draw[fill=cyan, draw=none] (C1) circle (2);
    \draw[fill=cyan, draw=none] (C2) circle (2);

    % Give outlines to the circles.
    \draw (C1) circle (2);
    \draw (C2) circle (2);

    % Labels.
    \node at (A) {$A$};
    \node at (B) {$B$};
    \node at (U) {$A\cup{B}$};
\end{tikzpicture}
            \caption{Venn Diagram for Union}
            \label{fig:Union_of_Two_Sets}
        \end{figure}
        Fig.~\ref{fig:Union_of_Two_Sets} can be extended to an
        arbitrary collection of sets. For the sake of simplicity, a Venn
        diagram for the union of three sets is shown in
        Fig.~\ref{fig:Union_of_Three_Sets}.
        \begin{figure}[H]
            \centering
            \captionsetup{type=figure}
            \begin{tikzpicture}
    % Coordinates for the centers of the circles.
    \coordinate (C1) at (-1.3,  0.00);
    \coordinate (C2) at ( 1.3,  0.00);
    \coordinate (C3) at ( 0.0, -2.15);

    % Coordinates for the labels.
    \coordinate (A) at (-1.3, 1.2);
    \coordinate (B) at ( 1.3, 1.2);
    \coordinate (C) at ( 0.0, -3.5);
    \coordinate (U) at ( 0.0, 2.5);

    % Rectangle indicating the universe set.
    \draw (-4, -4.5) rectangle (4, 3);

    % Fill in the circle with cyan.
    \draw[fill=cyan, draw=none] (C1) circle (2);
    \draw[fill=cyan, draw=none] (C2) circle (2);
    \draw[fill=cyan, draw=none] (C3) circle (2);

    % Give outlines to the circles.
    \draw (C1) circle (2);
    \draw (C2) circle (2);
    \draw (C3) circle (2);

    % Labels.
    \node at (A) {$A$};
    \node at (B) {$B$};
    \node at (C) {$C$};
    \node at (U)
        {$\Large{\underset{{\mathcal{U}\in\{A,B,C\}}}{\bigcup}\mathcal{U}}$};
\end{tikzpicture}
            \caption{The Union of Three Sets}
            \label{fig:Union_of_Three_Sets}
        \end{figure}
        We can combine the axiom schema of specification
        (Ax.~\ref{ax:Axiom_Schema_of_Specification}) with the existence of the
        union of two sets to define intersections. The intersection of two sets,
        denoted \gls{AcapB}, is the set consisting of all elements that lie in
        both $A$ and $B$ simultaneously.
        \begin{theorem}
            If $A$ and $B$ are sets, then there exists a set \gls{AcapB}
            such that for all $x$ it is true that $x\in{A}\cap{B}$ if and
            only if $x\in{A}$ and $x\in{B}$.
        \end{theorem}
        \begin{proof}
            For by Thm.~\ref{thm:Union_of_Two_Sets_Exists}, there exists
            a set $A\cup{B}$ such that for all $x$ is is true that
            $x\in{A}\cup{B}$ if and only if $x\in{A}$ or $x\in{B}$. Let
            $P$ be the proposition \textit{True if} $x\in{A}$ \textit{and}
            $x\in{B}$, \textit{false otherwise}. Then by the axiom schema
            of specification (Ax.~\ref{ax:Axiom_Schema_of_Specification})
            there is a set $A\cap{B}$ such that:
            \begin{equation}
                A\cap{B}=\big\{\,x\in{A}\cup{B}\;|\;P(x)\,\big\}
            \end{equation}
            But then $x\in{A}\cap{B}$ if and only if $x\in{A}\cup{B}$ and
            $x\in{A}$ and $x\in{B}$. But if $x\in{A}$ and $x\in{B}$, then
            $x\in{A}$, and thus $x\in{A}\cup{B}$
            (Def.~\ref{def:Union_of_Two_Sets}). That is, $P(x)$ implies that
            $x\in{A}\cap{B}$. Therefore, $x\in{A}\cap{B}$ if and only if $P(x)$
            is true. That is, $x\in{A}\cap{B}$ if and only if $x\in{A}$ and
            $x\in{B}$.
        \end{proof}
        \begin{fdefinition}{Intersection of Two Sets}
                           {Intersection_of_Two_Sets}
            The \gls{intersection of two sets} $A$ and $B$, denoted \gls{AcapB},
            is the set:\index{Intersection!of Two Sets}
            \begin{equation*}
                A\cap{B}
                =\big\{\,x\in{A}\cup{B}\;|\;
                    a\in{A}\textrm{ and }b\in{B}\,\big\}
            \end{equation*}
            That is:
            \begin{equation*}
                \forall_{A}\forall_{B}\forall_{x}\Big(
                    \big(x\in{A}\cap{B}\big)
                    \Leftrightarrow
                    \big((x\in{A})\land(x\in{B})\big)
                \Big)
            \end{equation*}
        \end{fdefinition}
        \begin{example}
            Using Eqn.~\ref{eqn:Z_n_Ellipses} to represent $\mathbb{Z}_{n}$,
            we can see that if $m<n$ then:
            \begin{equation}
                \mathbb{Z}_{m}\cap\mathbb{Z}_{n}=\mathbb{Z}_{m}
            \end{equation}
            Since $m<n$, every element of $\mathbb{Z}_{m}$ is contained in
            $\mathbb{Z}_{n}$. But every element of $\mathbb{Z}_{n}$ that is
            not contained in $\mathbb{Z}_{m}$ will not be in the intersection.
            This is the opposite of the pattern we saw in
            Ex.~\ref{ex:Union_of_Zm_and_Zn} when we considered the union of
            $\mathbb{Z}_{m}$ and $\mathbb{Z}_{n}$. This spells out a general
            theorem: If $A\subseteq{B}$, then $A\cap{B}=A$ and $A\cup{B}=B$
            (see Thm.~\ref{thm:Intersection_with_Subset} and
            \ref{thm:Union_With_Subset}, respectively).
        \end{example}
        \begin{fexample}{Intersections of Two Sets}
                        {Intersections_of_Two_Sets}
            If we let $A$ and $B$ be the sets defined by:
            \par
            \begin{subequations}
                \begin{minipage}[b]{0.49\textwidth}
                    \centering
                    \begin{equation}
                        A=\{\,a,\,b,\,c\,\}
                    \end{equation}
                \end{minipage}
                \hfill
                \begin{minipage}[b]{0.49\textwidth}
                    \centering
                    \begin{equation}
                        B=\{\,c,\,1,\,2\,\}
                    \end{equation}
                \end{minipage}
            \end{subequations}
            \par\vspace{2.5ex}
            we have that the intersection is then:
            \begin{equation}
                A\cap{B}=\{\,1\,\}
            \end{equation}
            This is because $1$ is the only element that appears in both $A$ and
            $B$, and is hence the only member of $A\cap{B}$.
        \end{fexample}
        \begin{example}
            Recalling our comment in Ex.~\ref{ex:More_Examples_of_Subsets},
            we claimed that the set of even integers (\gls{mathbbNe}) and the
            set of odd integers (\gls{mathbbNo}) are
            \textit{disjoint}\index{Set!Disjoint Sets}. We can now be precise
            about what this means. Since even numbers are of the form $2n$ and
            odd numbers are of the form $2n+1$, there are no natural numbers
            $k\in\mathbb{N}$ that are both even and odd. Thus:
            \begin{equation}
                \mathbb{N}_{e}\cap\mathbb{N}_{o}=\emptyset
            \end{equation}
            This is our definition of disjoint sets: Those with empty
            intersection.
        \end{example}
        \begin{fdefinition}{Disjoint Sets}{Disjoint_Sets}
            \Gls{disjoint sets} are \glspl{set} $A$ and $B$ such that
            $A\cap{B}=\emptyset$.\index{Set!Disjoint Sets}
        \end{fdefinition}
        We'll need one brief theorem about intersections to allow us to prove
        that certain sets are not equal.
        \begin{theorem}
            \label{thm:Lemma_for_Anti_Russells_Paradox}%
            If $A$ and $B$ are sets, if $x\in{B}$, and if $x\notin{A}\cap{B}$,
            then $x\notin{A}$.
        \end{theorem}
        \begin{proof}
            For if $x\notin{A}\cap{B}$ then either $x\notin{A}$ or $x\notin{B}$
            (Def.~\ref{def:Intersection_of_Two_Sets}). But $x\in{B}$, and
            therefore $x\notin{A}$.
        \end{proof}
        Similar to how unions can be visualized with Venn diagrams
        (Fig.~\ref{fig:Union_of_Two_Sets}), so can the intersection of
        two sets. We draw two circles that overlap slightly, and consider the
        region contained in both (see Fig.~\ref{fig:Intersection_of_Two_Sets}).
        \index{Venn Diagram}
        \begin{figure}[H]
            \centering
            \captionsetup{type=figure}
            %--------------------------------Dependencies----------------------------------%
%   tikz                                                                       %
%-------------------------------Main Document----------------------------------%
\begin{tikzpicture}[line width=0.2mm]
    % Coordinates for the centers of the circles.
    \coordinate (C1) at (-1.3, 0);
    \coordinate (C2) at ( 1.3, 0);

    % Coordinates for the labels.
    \coordinate (A) at (-1.3, 1.2);
    \coordinate (B) at ( 1.3, 1.2);
    \coordinate (I) at ( 0.0, 2.5);

    % Rectangle indicating the universe set.
    \draw (-4, -3) rectangle (4, 3);

    % Fill in the circle with cyan.
    \draw[fill=cyan] (0, -1.51987) arc(-49.46:49.46:2) arc(130.54:229.46:2);

    % Give outlines to the circles.
    \draw (C1) circle (2);
    \draw (C2) circle (2);

    % Labels.
    \node at (A) {$A$};
    \node at (B) {$B$};
    \node at (I) {$A\cap{B}$};
\end{tikzpicture}
            \caption{Venn Diagram for Intersection}
            \label{fig:Intersection_of_Two_Sets}
        \end{figure}
        We can extend this further and define the intersection over any
        collection of sets.
        \begin{ltheorem}{Existence of the Intersection of Sets}
                        {thm:Existence_of_Arbitrary_Intersetions}
            If $\mathcal{O}$ is a set, then there exists a set \gls{intersectO}
            such that for all $x$ it is true that $x\in\bigcap\mathcal{O}$ if
            and only if $x\in\bigcup\mathcal{O}$ and for all
            $\mathcal{U}\in\mathcal{O}$ it is true that $x\in\mathcal{U}$.
        \end{ltheorem}
        \begin{proof}
            For by Thm.~\ref{thm:Existence_of_Unions} there is a set
            $\bigcup\mathcal{O}$ such that for all $x$ it is true that
            $x\in\bigcup\mathcal{O}$ if and only if there exists a
            $\mathcal{U}\in\mathcal{O}$ such that $x\in\mathcal{U}$. Let
            $P$ be the proposition \textit{True if for all}
            $\mathcal{U}\in\mathcal{O}$ \textit{it is true that}
            $x\in\mathcal{U}$, \textit{false otherwise}. Then by the
            axiom schema of specification
            (Ax.~\ref{ax:Axiom_Schema_of_Specification}), there exists the set:
            \begin{equation}
                \bigcap\mathcal{O}
                =\Big\{\,x\in\bigcup\mathcal{O}
                    \;|\;P(x)\,\Big\}
            \end{equation}
            But then $x\in\bigcap\mathcal{O}$ if and only if
            $x\in\bigcup\mathcal{O}$ and $P(x)$ is true. That is,
            $x\in\bigcap\mathcal{O}$ if and only if $x\in\bigcup\mathcal{O}$
            and for all $\mathcal{U}\in\mathcal{O}$ it is true that
            $x\in\mathcal{U}$.
        \end{proof}
        It is common to consider some \textit{universal} set, of which all
        other sets of current consideration are drawn from. Using this the
        definition of arbitrary intersection is defined as the subset of
        this universal set such that every element of this subset is
        contained in every element of the arbitrary collection. One may then
        ask what would happen if the collection is empty. Using this
        definition the intersection would be the entire universal set
        in a vacuous sense. That is, there would be no $x$ in the universe
        that fails the definition of the intersection over an empty
        collection, and thus the intersection is everything. Letting $X$
        denote our universe, we obtain:
        \begin{equation}
            \emptyset=\bigcup_{\mathcal{U}\in\emptyset}\mathcal{U}
                \subseteq\bigcap_{\mathcal{U}\in\emptyset}\mathcal{U}
            =X
        \end{equation}
        It seems like unions should always be bigger. Indeed, for any
        non-empty collection, the intersection over the collection is a
        subset of the union over the collection. Because of this we do
        not adopt this definition of the intersection over a collection,
        but rather require in our construction the use of the union over
        the collection, and then use the axiom schema of specification to
        pick the subset of all elements of the union that belong to every
        element of the collection. Thus:
        \begin{equation}
            \bigcap_{\mathcal{U}\in\emptyset}\mathcal{U}
            \subseteq\bigcup_{\mathcal{U}\in\emptyset}\mathcal{U}
            =\emptyset
        \end{equation}
        And from this we conclude the intersection is empty as well.
        \begin{fdefinition}{Intersection Over a Collection}
                           {Intersection_Over_a_Collection}
            The \gls{intersection over a set} $\mathcal{O}$
            of sets is the set \gls{intersectO} defined by:
            \begin{equation*}
                \bigcap_{\mathcal{U}\in\mathcal{O}}\mathcal{U}
                =\Big\{\,x\in\bigcup_{\mathcal{U}\in\mathcal{O}}\mathcal{U}
                    \;\Big|\;x\in\mathcal{U}\textrm{ for all }
                    \mathcal{U}\in\mathcal{O}\,\Big\}
            \end{equation*}
            That is:
            \begin{equation*}
                \forall_{\mathcal{O}}\forall_{x}\bigg(
                    \Big(x\in\bigcap_{\mathcal{U}\in\mathcal{O}}\mathcal{U}\Big)
                    \Longleftrightarrow\Big(
                        \big(
                            x\in\bigcup_{\mathcal{U}\in\mathcal{O}}\mathcal{U}
                        \big)
                        \land\big(
                            \forall_{\mathcal{U}\in\mathcal{O}}
                            (x\in\mathcal{U})
                        \big)
                    \Big)
                \bigg)
            \end{equation*}
        \end{fdefinition}
        We can extend our Venn diagram for larger collections as well
        (see Fig.~\ref{fig:Intersection_of_Three_Sets}).
        \begin{figure}[H]
            \centering
            \begin{tikzpicture}
    % Coordinates for the centers of the circles.
    \coordinate (C1) at (-1.3,  0.00);
    \coordinate (C2) at ( 1.3,  0.00);
    \coordinate (C3) at ( 0.0, -2.15);

    \coordinate (O)  at ( 0.0000, -0.7500);
    \coordinate (P1) at ( 0.6817, -0.2697);
    \coordinate (P2) at (-0.6817, -0.2697);
    \coordinate (P3) at ( 0.0000, -1.5198);

    % Coordinates for the labels.
    \coordinate (A) at (-1.3, 1.2);
    \coordinate (B) at ( 1.3, 1.2);
    \coordinate (C) at ( 0.0, -3.5);
    \coordinate (U) at ( 0.0, 2.5);

    % Rectangle indicating the universe set.
    \draw (-4, -4.5) rectangle (4, 3.3);

    % Fill in the intersection with cyan.
    \draw[fill=cyan, draw=none] (P1) arc(70.07:109.93:2)
                                     arc(187.8:229.50:2)
                                     arc(310.54:352.25:2);

    % Give outlines to the circles.
    \draw (C1) circle (2);
    \draw (C2) circle (2);
    \draw (C3) circle (2);

    % Labels.
    \node at (A) {$A$};
    \node at (B) {$B$};
    \node at (C) {$C$};
    \node at (U)
        {\Large{$\underset{\mathcal{U}\in\{A,B,C\}}{\bigcap}\mathcal{U}$}};
\end{tikzpicture}
            \caption{The Intersection of Three Sets}
            \label{fig:Intersection_of_Three_Sets}
        \end{figure}
        Much the way we've defined what it means for two sets to be disjoint, we
        can extend this to arbitrary collections. We'll say that a collections
        of sets is mutually disjoint if all distinct elements of the collection
        have empty intersection. That is, all distinct pairs are disjoint.
        \begin{fdefinition}{Mutually Disjoint Collection of Sets}
                           {Mutually_Disjoint_Collection_of_Sets}
            A mutually disjoint collection of sets is a set $\mathcal{O}$ such
            that for all $A,B\in\mathcal{O}$ such that $A\ne{B}$ it is true that
            $A\cap{B}=\emptyset$. That is, for all distinct elements
            $A,B\in\mathcal{O}$ it is true that $A$ and $B$ are
            \glslink{disjoint sets}{disjoint}.
            \index{Mutually Disjoint Collection}
        \end{fdefinition}
        The term pairwise disjoint\index{Pairwise Disjoint} is also frequently
        used in measure theory and probability. As such, one might guess that
        the notion has a fair amount of use in these subjects. Next on the list
        of axioms is that of \textit{regularity}.
        \begin{faxiom}{Axiom of Regularity}{Axiom_of_Regularity}
            If $A$ is a non-empty set, then there is an element $B\in{A}$
            such that $A\cap{B}=\emptyset$.\index{Axiom!of Regularity}
            \begin{equation*}
                \forall_{A}(\exists_{x\in{A}})\Rightarrow
                \exists_{y}:\Big((y\in{A})\land
                \big((y\cap{A})=\emptyset\big)\Big)
            \end{equation*}
        \end{faxiom}
        This axiom is often seen as unnecessary by many working mathematicians
        and indeed it's use seems to only lie in set theory and foundations.
        That is, unlike the axioms of choice and union which are widely
        applicable to analysis and topology, regularity seems to only be useful
        to set theorists. This is not entirely true if one really pays attention
        to the details. Often we are presented some object $X$, perhaps a
        topological space, or an algebraic structure like a group, and we need
        to extend this object to a larger collection. One concrete example comes
        from topology when we have a \textit{non-compact} space $(X,\tau)$ and
        we want to find a \textit{compact} space $(\tilde{X},\tilde{\tau})$ that
        contains our original space. In the construction we add a single point
        to $(X,\tau)$ called \textit{infinity}, and label it $\infty$. But what
        if the symbol $\infty$ already belongs to $X$? How do we find a new
        object that is guarenteed not to lie in $X$? The axiom of regularity
        allows us to show that $\{X\}$ does not lie in $X$, for any set $X$, and
        thus we can take this to be our new point. This, and many similar
        constructions, rely on the axiom of regularity to guarentee that our
        reasoning is not ultimately circular.
        \par\hfill\par
        Regardless of the axioms application, its existence is vital to support
        the claim that ZFC is a good system to base mathematics on. We will
        combine this with pairing to prove that for any set $A$ it is true that
        $A\notin{A}$. That is, Zermelo-Fraenkel set theory is free of Russell's
        paradox\index{Russell's Paradox}.
        \begin{theorem}
            \label{thm:Anti_Russells_Paradox}%
            If $A$ is a set, then $A\notin{A}$.
        \end{theorem}
        \begin{proof}
            For if $A$ is a set, then $\{A\}$ is a set
            (Thm.~\ref{thm:Existence_of_Set_Containing_Set}). But since
            $A\in\{A\}$, $\{A\}$ is a non-empty set
            (Def.~\ref{def:Non_Empty_Set}). Thus by the axiom of regularity
            (Ax.~\ref{ax:Axiom_of_Regularity}) there is a set $B\in\{A\}$ such
            that $B\cap\{A\}=\emptyset$. But $B\in\{A\}$ if and only if
            $B=A$, and therefore $A\cap\{A\}=\emptyset$. Thus, by the axiom of
            the empty set (Ax.~\ref{ax:Axiom_of_the_Empty_Set}), for all $x$ it
            is true that $x\notin{A}\cap\{A\}$ and therefore
            $A\notin{A}\cap\{A\}$. But $A\in\{A\}$ and therefore
            $A\notin{A}$ (Thm.~\ref{thm:Lemma_for_Anti_Russells_Paradox}).
        \end{proof}
        \begin{theorem}
            \label{thm:Containment_NEqual_Underlying_Set}%
            If $A$ and $B$ are sets and if $A\in{B}$, then $A\ne{B}$.
        \end{theorem}
        \begin{proof}
            For $A\notin{A}$ (Thm.~\ref{thm:Anti_Russells_Paradox}) and
            $A\in{B}$ and therefore it is not true that for all $x$, $x\in{A}$
            if and only if $x\in{B}$. Therefore, by the Axiom of
            Extensionality (Ax.~\ref{ax:Axiom_of_Extensionality}), $A\ne{B}$.
        \end{proof}
        \begin{theorem}
            \label{thm:Cor_of_Containment_NEqual_Underlying_Set}%
            If $A$ is a set, then $A\ne\{A\}$.
        \end{theorem}
        \begin{proof}
            For if $A$ is a set, then $A\notin{A}$
            (Thm.~\ref{thm:Anti_Russells_Paradox}). But if $A$ is a set, then
            $\{A\}$ is a set (Thm.~\ref{thm:Existence_of_Set_Containing_Set}).
            But $A\in\{A\}$, and thus $A\ne\{A\}$
            (Thm.~\ref{thm:Containment_NEqual_Underlying_Set}).
        \end{proof}
        These quick theorems will eventually prove the well known result that
        $0\ne{1}$. It also shows us that there is no set of all
        sets\index{Set!of All Sets}.
    \subsection{The Axiom of the Power Set}
        Continuing in our goal of constructing order, we move on to the
        Cartesian product\index{Cartesian Product} of two sets $A$ and $B$. This
        is the collection of all ordered pairs $(a,b)$ such that $a\in{A}$ and
        $b\in{B}$. To prove such a set exists requires the
        \textit{axiom of the power set}.
        \begin{faxiom}{Axiom of the Power Set}{Axiom_of_the_Power_Set}
            If $A$ is a set, then there exists a set $\mathscr{P}$ such that
            for all $x\subseteq{A}$ it is true that $x\in\mathscr{P}$.
            \index{Axiom!of the Power Set}
            \begin{equation*}
                \forall_{A}\exists_{\mathscr{P}}:
                \forall_{x}\Big((x\subseteq{X})\Rightarrow(x\in\mathscr{P})\Big)
            \end{equation*}
        \end{faxiom}
        Again, much like the axiom of union and the axiom of pairing, this
        set may be bigger than we would like. We wish to find a set, called
        the \textit{power set}, that contains all of the subsets of a given
        set $A$ and nothing else. Combining the axiom of the power set
        with the axiom schema of specification gives us such existence.
        \begin{ltheorem}{Existence of the Power Set}
                        {Existence_of_the_Power_Set}
            If $A$ is a set, then there exists a set \gls{powersetsymb}
            such that for all $x$ it is true that $x\in\mathcal{P}(A)$ if and
            only if $x\subseteq{A}$.\index{Set!Power Set}
        \end{ltheorem}
        \begin{proof}
            For by the axiom of the power set
            (Ax.~\ref{ax:Axiom_of_the_Power_Set}) there is a set $\mathscr{P}$
            such that for all $x\subseteq{A}$ it is true that $x\in\mathscr{P}$.
            Let $P$ be the proposition \textit{true if} $x\subseteq{A}$,
            \textit{false otherwise}. By the axiom schema of specification
            (Ax.~\ref{ax:Axiom_Schema_of_Specification}), there is a set
            $\mathcal{P}(A)$ such that:
            \begin{equation}
                \mathcal{P}(A)=\{\,x\in\mathscr{P}(A)\;|\;P(x)\,\}
            \end{equation}
            But if $P(x)$ is true, then $x$ is a subset of $A$, and therefore
            $x\in\mathscr{P}(A)$. Thus $x\in\mathcal{P}(A)$ if and only if
            $x\subseteq{A}$.
        \end{proof}
        With this we now define the \textit{power set} of a given set.
        \newpage
        \begin{fdefinition}{Power Set}{Power_Set}
            The \gls{power set} of a \gls{set} $A$ is the set $\mathcal{P}(A)$
            defined by:\index{Power Set}\index{Set!Power Set}
            \begin{equation*}
                \mathcal{P}(A)=\{\,x\;|\;x\subseteq{X}\,\}
            \end{equation*}
            That is, the set of all subsets of $X$.
            \begin{equation*}
                \forall_{A}\forall_{B}\Big(B\in\mathcal{P}(A)\Longleftrightarrow
                    B\subseteq{A}\Big)
            \end{equation*}
        \end{fdefinition}
        Again, there is some abuse of our set-builder notation, but
        Thm.~\ref{thm:Existence_of_the_Power_Set} justifies such a definition.
        The power set of a set is a crucial construction for when one discusses
        the \textit{cardinality} of sets, denoted $\textrm{Card}(A)$. This
        describes the \textit{size} of a set in a very precise manner. A theorem
        that will eventually be proved known as
        \textit{Cantor's Theorem}\index{Theorem!Cantor's Power Set Theorem}
        shows that the power set of a set is always strictly \textit{larger}
        than the original set. That is:
        \begin{equation}
            \textrm{Card}(A)<\textrm{Card}\big(\mathcal{P}(A)\big)
        \end{equation}
        This will be made precise soon enough. The axiom of the power set allows
        us to build \textit{larger} sets from a given set. This creates a
        paradoxical heirarchy of infinities. Starting with the \textit{smallest}
        infinity, the natural numbers $\mathbb{N}$, we can create a
        significantly larger set by considering $\mathcal{P}(\mathbb{N})$. We
        can continue and consider $\mathcal{P}(\mathcal{P}(\mathbb{N}))$, and
        there's no reason to stop there. At each step we create a new, massively
        larger set. This is both unintuitive and paradoxical and as such some
        may choose to reject it. This axiom is vital in the discussion of
        topology and measure theory, and so we choose to accept it as true.
        \begin{example}
            If $A=\{1,2\}$, then the power set is:
            \begin{equation}
                \mathcal{P}(A)=\big\{\,\emptyset,\,\{1\},\,\{2\},\,
                    \{1,2\}\,\big\}
            \end{equation}
            We must consider the empty set since $\emptyset\subseteq{A}$.
            Now suppose $A=\{1,2,3\}$:
            \begin{equation}
                \mathcal{P}(A)=\big\{\,\emptyset,\,\{1\},\,\{2\},\,\{3\},\,
                    \{1,2\},\,\{1,3\},\,\{2,3\},\,\{1,2,3\}\,\big\}
            \end{equation}
            We see that a set with 2 elements has a power set with 4 elements
            and a set with 3 elements has a power set with 8. This pattern
            continues for finite sets and if $A$ has $n$ elements, then
            $\mathcal{P}(A)$ has $2^{n}$ elements.
        \end{example}
        \begin{example}
            Let $A=\{a_{1},\dots,a_{n}\}$, where all of the elements $a_{k}$ are
            distinct. To count the total number of subsets we first note that
            there is one set that contains zero elements, the empty set. Next,
            there are $n$ subsets that contain one element, these are the sets
            $\{a_{k}\}$. There are $n(n-1)/2$ sets that contain two elements,
            $\{a_{i},a_{j}\}$, such that $i\ne{j}$. Continuing, we see that
            there are $\binom{n}{k}$ subsets that contain $k$ elements, where
            $\binom{n}{k}$ is the \textit{binomial coefficient}%
            \index{Binomial Coefficient}. This is defined in terms of the
            factorial function:
            \begin{subequations}
                \begin{align}
                    \binom{n}{k}
                    &=\frac{n!}{k!(n-k)!}\\
                    &=\frac{n\cdot(n-1)\cdots{2}\cdot{1}}
                        {k\cdot(k-1)\cdots{2}\cdot{1}\cdot(n-k)
                        \cdot(n-k-1)\cdots{2}\cdot{1}}\\
                    &=\frac{n\cdot(n-1)\cdots\cdot(n-k+1)}
                        {k\cdot(k-1)\cdots{2}\cdot{1}}
                \end{align}
            \end{subequations}
            To avoid undefined ratios, we define $0!=1$. Note then that
            $\binom{n}{n}=1$. This says that the number of ways to choose $n$
            element subsets from $A$ is 1. This makes sense since the only $n$
            element subset of $A$ is the entirety of $A$! To compute the size of
            $\mathcal{P}(A)$ it now suffices to sum over all of these binomial
            coefficients. Such a task can be achieved by invoking the
            \textit{binomial theorem}\index{Theorem!Binomial Theorem}. Given
            a positive integer $n$ and two real numbers $x$ and $y$, the
            binomial theorem states that:
            \begin{subequations}
                \begin{align}
                    (x+y)^{n}
                    &=\sum_{k=0}^{n}\binom{n}{k}x^{n-k}y^{k}\\
                    &=\binom{n}{0}x^{n}+\binom{n}{1}x^{n-1}y+\cdots+
                        \binom{n}{n-1}xy^{n-1}+\binom{n}{n}y^{n}
                \end{align}
            \end{subequations}
            Here, the notation $\Sigma$ is simply shorthand for denoting a long
            sum. For example:
            \par
            \begin{subequations}
                \begin{minipage}[b]{0.49\textwidth}
                    \centering
                    \begin{equation}
                        \sum_{n=1}^{3}n=1+2+3=6
                    \end{equation}
                \end{minipage}
                \hfill
                \begin{minipage}[b]{0.49\textwidth}
                    \centering
                    \begin{equation}
                        \sum_{n=1}^{3}n^{2}=1+2^{2}+3^{2}=14
                    \end{equation}
                \end{minipage}
            \end{subequations}
            \par\vspace{2.5ex}
            Setting $x=y=1$, we obtain:
            \begin{equation}
                2^{n}=\sum_{k=0}^{n}\binom{n}{k}
            \end{equation}
            and this is precisely the number of elements of $\mathcal{P}(A)$.
        \end{example}
        \begin{example}
            When we consider the case of an \textit{infinite} set $A$ we have
            that $\mathcal{P}(A)$ is a strictly larger set and this creates a
            paradoxical heirarchy of infinities. The smallest heirarchy is
            that of the \textit{countable} infinite sets, like $\mathbb{N}$.
            Everything larger is called \textit{uncountable}. It will be
            shown that the following is true:
            \begin{equation}
                \Card\big(\mathcal{P}(\mathbb{N})\big)=
                \Card(\mathbb{R})
            \end{equation}
            where again $\mathbb{N}$ denotes the non-negative integers and
            $\mathbb{R}$ denotes the set of all \textit{real} numbers. We
            can loosely show this by using the binary representation of real
            numbers. A real number may be thought of as an infinite decimal.
            For example, $\pi=3.1415926\dots$ and $1=1.000\dots$ We can
            also represent real numbers as a sequence of zeroes and ones and
            this is the \textit{binary} representation. For
            $A\subseteq\mathbb{N}$ and let $r_{A}=0.n_{1}n_{2}\hdots$ where:
            \begin{equation}
                n_{i}=
                \begin{cases}
                    0,&i\notin{A}\\
                    1,&i\in{A}
                \end{cases}
            \end{equation}
            Thus for each $A\in\mathcal{P}(\mathbb{N})$ there is a real
            number $r_{A}$ such that $0\leq{r}_{A}\leq{1}$ that is
            associated with it, and moreover to every real number between
            zero and one there is a subset of $\mathbb{N}$ associated with
            it. The tricky numbers to see are zero and one, but note that
            $r_{\emptyset}$ is associated to 0 and $r_{\mathbb{N}}$ gets
            paired with 1. To show that $\mathbb{R}$ and
            $\mathcal{P}(\mathbb{N})$ are the same size requires us to
            refine this association so that every element of
            $\mathcal{P}(\mathbb{N})$ uniquely corresponds to an element of
            $\mathbb{R}$, and vice-versa.
        \end{example}
    \subsection{Cartesian Products and Functions}
        Previously we've introduced ordered pairs and the notion of the power
        set. We can use both of these concepts to define and prove the existence
        of \textit{Cartesian products}. Intuitively we want to define
        $A\times{B}$ to be the set of all ordered pairs $(a,b)$ where $a\in{A}$
        and $b\in{B}$:
        \begin{equation}
            A\times{B}=\big\{\,(a,\,b)\;|\;a\in{A}\textrm{ and }b\in{B}\,\big\}
        \end{equation}
        But recalling Def.~\ref{def:Ordered_Pairs}, ordered pairs are sets
        of the form $\{\{a\},\{a,b\}\}$. Thus elements of $A\times{B}$ are
        contained in the power set of the power set of $A\cup{B}$:
        \begin{equation}
            A\times{B}\subseteq\mathcal{P}\big(\mathcal{P}(A\cup{B})\big)
        \end{equation}
        We can combine the axiom of the power set with the axiom schema of
        specification to obtain the existence of the Cartesian product of
        two sets.\index{Axiom!of the Power Set}
        \index{Axiom!Schema of Specification}
        \begin{theorem}
            \label{thm:Ordered_Pair_Subset_of_Power_Set}%
            If $A$ and $B$ are sets, if $a\in{A}$ and $b\in{B}$, then
            $(a,b)\subseteq\mathcal{P}(A\cup{B})$.
        \end{theorem}
        \begin{proof}
            For if $a\in{A}$ and $b\in{B}$, then
            $(a,b)=\{\,\{\,a\,\},\,\{\,a,\,b\,\}\,\}$
            (Def.~\ref{def:Ordered_Pairs}). But if $a\in{A}$, then $a\in{A}$
            or $a\in{B}$, and thus $a\in{A}\cup{B}$
            (Def.~\ref{def:Union_of_Two_Sets}). But then
            $\{\,a\,\}\subseteq{A}\cup{B}$ (Def.~\ref{def:Subsets}). But if
            $b\in{B}$, then $b\in{A}$ or $b\in{B}$, and thus $b\in{A}\cup{B}$
            (Def.~\ref{def:Union_of_Two_Sets}). But then
            $\{\,a,\,b\,\}\subseteq{A}\cup{B}$ (Def.~\ref{def:Subsets}).
            But then $\{\,a\,\}\subseteq{A}\cup{B}$ and
            $\{\,a,\,b\,\}\subseteq{A}\cup{B}$, and thus
            $(a,b)\subseteq\mathcal{P}(A\cup{B})$ (Def.~\ref{def:Power_Set}).
        \end{proof}
        \begin{ltheorem}{Existence of the Cartesian Product}
                        {Existence_of_the_Cartesian_Product}
            If $A$ and $B$ are sets, then there exists a set $A\times{B}$
            such that, for all $z$, $z\in{A}\times{B}$ if and only if there
            is an $a\in{A}$ and $b\in{B}$ such that $z=(a,b)$.
            \index{Cartesian Product}
        \end{ltheorem}
        \begin{proof}
            For if $A$ and $B$ are sets, then $A\cup{B}$ is a set
            (Thm.~\ref{thm:Union_of_Two_Sets_Exists}). But if $A\cup{B}$ is a
            set, then $\mathcal{P}(A\cup{B})$ is a set
            (Thm.~\ref{thm:Existence_of_the_Power_Set}), where $\mathcal{P}(X)$
            denotes the power set of $X$. But if $\mathcal{P}(A\cup{B})$ is a
            set, then $\mathcal{P}(\mathcal{P}(A\cup{B}))$ is a set
            (Thm.~\ref{thm:Existence_of_the_Power_Set}). But then
            $z\in\mathcal{P}(\mathcal{P}(A\cup{B}))$ if and only if
            $z\subseteq\mathcal{P}(A\cup{B})$ (Def.~\ref{def:Power_Set}).
            But if $a\in{A}$ and $b\in{B}$, then
            $(a,b)\subseteq\mathcal{P}(A\cup{B})$
            (Thm.~\ref{thm:Ordered_Pair_Subset_of_Power_Set}), and therefore
            $(a,b)\in\mathcal{P}(\mathcal{P}(A\cup{B}))$. Let $P$ be the
            proposition \textit{True if there exists} $a\in{A}$ \textit{and}
            $b\in{B}$ \textit{such that} $z=(a,b)$, \textit{false otherwise}.
            Then by the axiom schema of specification
            (Ax.~\ref{ax:Axiom_Schema_of_Specification}), there exists a
            set $A\times{B}$ such that:
            \begin{equation}
                A\times{B}=
                \{\,z\in\mathcal{P}\big(\mathcal{P}(A\cup{B})\big)\;|\;
                    P(z)\,\}
            \end{equation}
            But it was proved that $P(z)$ implies that
            $z\in\mathcal{P}(\mathcal{P}(A\cup{B}))$. Thus $z\in{A}\times{B}$
            if and only if there exists $a\in{A}$ and $b\in{B}$
            such that $z=(a,b)$.
        \end{proof}
        \begin{fdefinition}{Cartesian Product of Two Sets}
                           {Cartesian_Product_of_Two_Sets}
            The \gls{Cartesian product} of two \glspl{set} $A$ and $B$ is the
            set:\index{Cartesian Product}
            \begin{equation*}
                A\times{B}
                =\{\,(a,\,b)\;|\;a\in{A}\textrm{ and }b\in{B}\,\}
            \end{equation*}
            Formally:
            \begin{equation*}
                \forall_{A}\forall_{B}\forall_{z}\Big(
                    (z\in{A}\times{B})\Longleftrightarrow
                    \big(\exists_{x\in{A}}\land\exists_{y\in{B}}:z=(x,y)\big)
                \Big)
            \end{equation*}
        \end{fdefinition}
        Note that since, in general, $(a,b)\ne(b,a)$, it is generally true that
        $A\times{B}\ne{B}\times{A}$. Indeed, equality occurs if and only if
        $A=B$ (or if either set is empty).
        \begin{fexample}{Basic Cartesian Products}{Basic_Cartesian_Products}
            Let $A$ and $B$ be sets defined as follows:
            \par
            \begin{subequations}
                \begin{minipage}[b]{0.49\textwidth}
                    \centering
                    \begin{equation}
                        A=\{\,1,\,2,\,3\,\}
                    \end{equation}
                \end{minipage}
                \hfill
                \begin{minipage}[b]{0.49\textwidth}
                    \centering
                    \begin{equation}
                        B=\{\,a,\,b\,\}
                    \end{equation}
                \end{minipage}
            \end{subequations}
            \par\vspace{2.5ex}
            Let's compute $A\times{B}$ and $B\times{A}$. From the definition
            (Def.~\ref{def:Cartesian_Product_of_Two_Sets}) we have:
            \begin{equation}
                A\times{B}=\{\,(a,b)\;|\;a\in{A}\textrm{ and }b\in{B}\,\}
            \end{equation}
            Using this, we can compute:
            \begin{equation}
                A\times{B}=\big\{\,(1,a),\,(2,a),\,(3,a),\,
                                   (1,b),\,(2,b),\,(3,b)\,\big\}
            \end{equation}
            Computing $B\times{A}$, we have:
            \begin{equation}
                B\times{A}=\big\{\,(a,\,1),\,(a,\,2),\,(a,\,3),\,
                                   (b,\,1),\,(b,\,2),\,(b,\,3)\,\big\}
            \end{equation}
            Now if we suppose that $a$ is not equal to 1, then we see that
            $(a,1)$ is a different element than $(1,a)$, and thus $A\times{B}$
            is not equal to $B\times{A}$. Next, compute $A\times{A}$:
            \begin{equation}
                \begin{split}
                    A\times{A}=\Big\{\,(1,1),\,(1,2),\,&(1,3),
                                       (2,1),\,(2,2),\,\\&(2,3),
                                       (3,1),\,(3,2),\,(3,3)\,\Big\}
                \end{split}
            \end{equation}
            And finally $B\times{B}$:
            \begin{equation}
                B\times{B}=\big\{\,(a,\,a),\,(a,\,b),
                                 \,(b,\,a),\,(b,\,b)\,\big\}
            \end{equation}
            Equality of $A\times{B}$ and $B\times{A}$ is achieved if and only
            if $A=B$, or if either set is the empty set.
        \end{fexample}
        Note that in Ex.~\ref{ex:Basic_Cartesian_Products}, the \textit{size}
        of the Cartesian product of two sets was simply the product of the
        number of elements of the constituent sets. That is, we see that $A$
        has three elements and $B$ has two elements, but also that
        $A\times{B}$ has six elements. Moreover, $A\times{A}$ has nine
        elements and $B\times{B}$ has four. This pattern holds for the
        Cartesian products of any two \textit{finite} sets.
        \index{Set!Finite}
        \par\hfill\par
        It is common to consider the Cartesian product of a set with itself.
        That is, given a set $A$, we are often interested in $A\times{A}$. We
        denote this by writing $A^{2}$. One such example is when we consider
        the set of real numbers $\mathbb{R}$. The Cartesian product
        $\mathbb{R}^{2}$ is called the \textit{Euclidean Plane}, or the
        \textit{Cartesian Plane}\index{Euclidean Plane}\index{Cartesian Plane},
        after Euclid of Alexandria\index{Euclid of Alexandria} and Ren\'{e}
        Descartes\index{Descartes, Ren\'{e}}. This is because $\mathbb{R}^{2}$
        is used to model both planar geometry and analytical geometry, of which
        Euclid and Descartes were pioneers of, respectively. The term Cartesian
        products is in honor of Ren\'{e} Descartes, as well. Let $\mathbb{R}$
        denote the set of real numbers, and let $A=\mathbb{R}$ and
        $B=\mathbb{R}$. Then we have:
        \begin{equation}
            A\times{B}=\mathbb{R}\times\mathbb{R}\equiv\mathbb{R}^{2}
        \end{equation}
        Where the symbol \gls{equiv} means that $\mathbb{R}^{2}$ is defined by
        this expression. Using the definition of Cartesian products
        (Def.~\ref{def:Cartesian_Product_of_Two_Sets}), we obtain:
        \begin{equation}
            \mathbb{R}^{2}=\{\;(x,y)\,:\,x\in\mathbb{R}
                               \textrm{ and }y\in\mathbb{R}\;\}
        \end{equation}
        That is, $\mathbb{R}^{2}$ is the set of all ordered pairs of real
        numbers. The first term is called the $x$ coordinate, and similarly the
        second term is called the $y$ coordinate. We envision this as a
        \textit{plane} of points, each one corresponding to an ordered pair
        $(x,y)$. This is depicted in Fig.~\ref{fig:Cartesian_Plane}.
        \begin{figure}[H]
            \centering
            %--------------------------------Dependencies----------------------------------%
%   amssymb                                                                    %
%   tikz                                                                       %
%       arrows.meta                                                            %
%-------------------------------Main Document----------------------------------%
\begin{tikzpicture}[%
    >=Latex,
    line width=0.2mm,
    line cap=round,
    font=\Large
]
    % Coordinates for the points.
    \coordinate (x) at (2.2, 0.0);
    \coordinate (y) at (0.0, 2.9);
    \coordinate (z) at (2.2, 2.9);

    % Draw a grid.
    \draw[style=help lines] (-0.3, -0.3) grid (7.9, 7.9);

    % Axes.
    \begin{scope}[thick]
        \draw[->] (-0.3, 0) to (8.4, 0) node [above] {$\mathbb{R}$};
        \draw[->] (0, -0.3) to (0, 8.4) node [right] {$\mathbb{R}$};
    \end{scope}

    % Draw dashed lines to the point.
    \begin{scope}[densely dashed]
        \draw (x) to (z);
        \draw (y) to (z);
    \end{scope}

    % Draw dots marking the various points.
    \draw[fill=black] (x) circle (0.6mm);
    \draw[fill=black] (y) circle (0.6mm);
    \draw[fill=black] (z) circle (0.6mm);

    \node at (x) [below=0.1]     {$x$};
    \node at (y) [left=0.1]      {$y$};
    \node at (z) [above right]   {$(x,\,y)$};
\end{tikzpicture}
            \caption{The Cartesian Plane $\mathbb{R}^{2}$}
            \label{fig:Cartesian_Plane}
        \end{figure}
        Consider further the set $\mathbb{N}^{2}$. That is, letting
        $\mathbb{N}$ denote the set of natural numbers
        (Eqn.~\ref{eqn:Natural_Numbers_Ellipses}), letting $A=\mathbb{N}$ and
        $B=\mathbb{N}$ we have:\index{Integer Lattice}
        \begin{equation}
            A\times{B}=\mathbb{N}\times\mathbb{N}\equiv\mathbb{N}^{2}
        \end{equation}
        Again using the definition of Cartesian products
        (Def.~\ref{def:Cartesian_Product_of_Two_Sets}), we have:
        \begin{equation}
            \mathbb{N}^{2}=
            \{\,(n,\,m)\;|\;n\in\mathbb{N}\textrm{ and }m\in\mathbb{N}\,\}
        \end{equation}
        We can visualize this as a subset of $\mathbb{R}^{2}$ by drawing a
        lattice of points in the Cartesian plane
        (Fig.~\ref{fig:Lattice_Cart_Prod_of_N_with_N}).
        \begin{figure}[H]
            \centering
            %--------------------------------Dependencies----------------------------------%
%   amssymb                                                                    %
%   tikz                                                                       %
%       arrows.meta                                                            %
%-------------------------------Main Document----------------------------------%
\begin{tikzpicture}[%
    >=Latex,
    line width=0.2mm,
    line cap=round
]

    % Axes.
    \begin{scope}[thick, font=\Large]
        \draw[->] (0, 0) to (8.4, 0) node [above] {$\mathbb{N}$};
        \draw[->] (0, 0) to (0, 8.4) node [right] {$\mathbb{N}$};
    \end{scope}

    \foreach\x in{1, 2, 3, 4, 5, 6, 7, 8}{
        \foreach\y in{1, 2, 3, 4, 5, 6, 7, 8}{
            \draw[fill=black] (\x, \y) circle (0.2mm);
        }
        \draw (\x, -0.1) to (\x, 0.1) node [below=1ex] {$\x$};
        \draw (-0.1, \x) to (0.1, \x) node [left=1ex]  {$\x$};
    }
\end{tikzpicture}
            \caption{The Lattice $\mathbb{N}^{2}$}
            \label{fig:Lattice_Cart_Prod_of_N_with_N}
        \end{figure}
        This can then be consider a subset of the Euclidean plane
        $\mathbb{R}^{2}$. That is, at every ordered pair of integers $(m,n)$,
        we place a point in the Euclidean plane whose $x$ coordinate is $m$ and
        whose $y$ coordinate is $n$. We can also be more abstract and general in
        our examples. Consider the following sets:
        \par
        \begin{subequations}
            \begin{minipage}[b]{0.49\textwidth}
                \centering
                \begin{equation}
                    A=\{\,\textrm{Point, Line 1, Line 2}\,\}
                \end{equation}
            \end{minipage}
            \hfill
            \begin{minipage}[b]{0.49\textwidth}
                \centering
                \begin{equation}
                    B=\{\,\textrm{Point, Line}\,\}
                \end{equation}
            \end{minipage}
        \end{subequations}
        \par\vspace{2.5ex}
        We can visually represent the Cartesian product $A\times{B}$ by
        drawing $A$ in green and $B$ in red, as shown in
        Fig.~\ref{fig:Cartesian_Product_Example}. The Cartesian Product
        $A\times{B}$ is the set formed by connecting all of the points
        from $A$ and $B$ in the plane. This is shown in blue.
        \begin{figure}[H]
            \centering
            %--------------------------------Dependencies----------------------------------%
%   tikz                                                                       %
%       arrows.meta                                                            %
%-------------------------------Main Document----------------------------------%
\begin{tikzpicture}[%
    >=Latex,
    line width=0.2mm,
    line cap=round
]

    % Draw green to indicate the set A.
    \begin{scope}[green]

        % Draw some points.
        \draw[fill=green] (1, 0) circle (0.3mm);
        \draw[fill=green] (2, 0) circle (0.3mm);
        \draw[fill=green] (5, 0) circle (0.3mm);
        \draw[fill=green] (6, 0) circle (0.3mm);
        \draw[fill=green] (7, 0) circle (0.3mm);

        % Draw lines.
        \draw (2, 0) to (5, 0);
        \draw (6, 0) to (7, 0);
    \end{scope}

    % Draw red to denote the set B.
    \begin{scope}[red]

        % Draw in some points.
        \draw[fill=red] (0, 1) circle (0.3mm);
        \draw[fill=red] (0, 2) circle (0.3mm);
        \draw[fill=red] (0, 5) circle (0.3mm);

        % Draw a line.
        \draw (0, 2) to (0, 5);
    \end{scope}

    % Use blue to mark AxB (Cartesian product).
    \begin{scope}[blue]

        % Fill in points.
        \draw[fill=blue] (1, 1) circle (0.3mm);
        \draw[fill=blue] (1, 2) circle (0.3mm);
        \draw[fill=blue] (1, 5) circle (0.3mm);
        \draw[fill=blue] (2, 1) circle (0.3mm);
        \draw[fill=blue] (5, 1) circle (0.3mm);
        \draw[fill=blue] (6, 1) circle (0.3mm);
        \draw[fill=blue] (7, 1) circle (0.3mm);
        \draw[fill=blue] (2, 2) circle (0.3mm);
        \draw[fill=blue] (2, 5) circle (0.3mm);
        \draw[fill=blue] (5, 2) circle (0.3mm);
        \draw[fill=blue] (5, 5) circle (0.3mm);
        \draw[fill=blue] (6, 2) circle (0.3mm);
        \draw[fill=blue] (7, 2) circle (0.3mm);
        \draw[fill=blue] (6, 5) circle (0.3mm);
        \draw[fill=blue] (7, 5) circle (0.3mm);

        % Draw lines.
        \draw (1, 2) to (1, 5);
        \draw (2, 1) to (5, 1);
        \draw (6, 1) to (7, 1);

        % Fill in rectangles.
        \draw[fill=blue, opacity=0.4] (2, 2) to (5, 2) to (5, 5)
                                             to (2, 5) to cycle;
        \draw[fill=blue, opacity=0.4] (6, 2) to (7, 2) to (7, 5)
                                             to (6, 5) to cycle;
        \draw (2, 2) to (5, 2) to (5, 5) to (2, 5) to cycle;
        \draw (6, 2) to (7, 2) to (7, 5) to (6, 5) to cycle;
    \end{scope}
\end{tikzpicture}
            \caption[Cartesian Product of Two Sets]
                {The Cartesian Product of Two Sets. $A$ is
                 in \textcolor{green}{Green},
                 $B$ is in \textcolor{red}{red}, and
                 $A\times{B}$ is in \textcolor{blue}{blue}.}
            \label{fig:Cartesian_Product_Example}
        \end{figure}
        Cartesian products are not \textit{associative}%
        \index{Binary Operation!Associative}. That is, given three sets $A$,
        $B$, and $C$, there is no clear way to take the Cartesian
        product\index{Cartesian Product} of these since:
        \begin{equation}
            A\times(B\times{C})\ne(A\times{B})\times{C}
        \end{equation}
        To see this, note that the elements of $A\times(B\times{C})$ are
        ordered pairs of the form $\big(a,\,(b,\,c)\big)$, whereas elements of
        $(A\times{B})\times{C}$ are of the form $\big((a,\,b),\,c\big)$. When
        we write $A\times{B}\times{C}$ we really want ordered \textit{triples}
        of the form $(a,\,b,\,c)$.
        Much the way ordered pairs have been defined, we can modify Kuratowski's
        approach and define ordered triples\index{Ordered Triple} and ordered
        $n$ tuples\index{Ordered $n$ Tuple}. Rather than doing this we will use
        the language of functions to define higher order Cartesian products.
        \begin{fdefinition}{Functions}{Function}
            A \gls{function} from a \gls{set} $A$ to a set $B$ is a \gls{subset}
            $f\subseteq{A}\times{B}$, denoted $f:A\rightarrow{B}$, such that
            for all $x\in{A}$ there is a unique $y\in{B}$ such that
            $(x,y)\in{f}$. $A$ is called the domain of $f$
            and $B$ is called the codomain.\index{Function}
        \end{fdefinition}
        We're used to hearing that a function is a rule that assigns to an
        input value $x$ some output value $f(x)$. It may seem hard to justify,
        then, why we've defined a function as a subset of the Cartesian
        product. But note the requirement that for each $x\in{A}$ there is a
        \textit{unique} $y\in{B}$ such that $(x,y)\in{f}$. We call this unique
        element the \textit{image} of $x$ under the function $f$ and write
        $y=f(x)$. The condition that there is a unique such value $y$ to each
        $x$ is called the \textit{vertical line test}\index{Vertical Line Test}
        when graphing functions of the form $f:\mathbb{R}\rightarrow\mathbb{R}$
        (Fig.~\ref{fig:Function_R_to_R_Subset_Cart_Prod}). Simply, given such
        a function, if one draws a vertical line in the plane, then it must
        intersect the graph of $f$ once and only once. This provides a
        quick means of discerning functions from non-functions.
        \begin{lexample}{The Square Function}{Square_Function}
            If we can come up with some rule that assigns to every element
            $a\in{A}$ a unique element of $B$, then we can use this rule to
            define a function $f:A\rightarrow{B}$. Such a rule often comes
            in the form of a \textit{formula}\index{Formula}. We write the
            unique element that $a$ corresponds to as $f(b)$. For example, let
            $A=\mathbb{R}$ and let $B=\mathbb{R}$. We can define a function by
            the squaring formula:
            \begin{equation}
                f(x)=x\cdot{x}=x^{2}
            \end{equation}
            Once we know that $x^{2}$ gives a unique number
            (which will require some notion of arithmetic), we can define
            the function $f:\mathbb{R}\rightarrow\mathbb{R}$ by:
            \begin{equation}
                f=\{\,(x,\,x^{2})\in\mathbb{R}^{2}\;|\;x\in\mathbb{R}\,\}
            \end{equation}
            Usually we'll define functions by their formula's, rather than
            expressing them explicitly as subsets of the Cartesian product.
        \end{lexample}
        In the field of mathematical analysis we are often concerned with
        functions involving real numbers. For the sake of intuition, let
        us consider functions of the form $f:\mathbb{R}\rightarrow\mathbb{R}$.
        Any curve that we draw left-to-right, without picking up the pencil,
        will be a valid function.
        (See Fig.~\ref{fig:Function_R_to_R_Subset_Cart_Prod}).
        \begin{figure}[H]
            \centering
            %--------------------------------Dependencies----------------------------------%
%   xcolor                                                                     %
%   amssymb                                                                    %
%   tikz                                                                       %
%       arrows.meta                                                            %
%-------------------------------Main Document----------------------------------%
\begin{tikzpicture}[%
    >=Latex,
    line width=0.2mm,
    line cap=round,
    scale=1.2
]
    % Coorindates for the curve.
    \coordinate (P1) at (-4.00, -2.00);
    \coordinate (P2) at (-2.00, -3.00);
    \coordinate (P3) at ( 0.00,  0.00);
    \coordinate (P4) at ( 2.00,  3.00);
    \coordinate (P5) at ( 4.00,  3.90);

    % Draw a green mesh indicating the Cartesian plane.
    \foreach\x in {-40, -39, ..., 39}{
        \draw[draw=green, line width=0.1mm] (\x/10, -4) to (-4, \x/10);
        \draw[draw=green, line width=0.1mm] (4, \x/10)  to (\x/10, 4);
    }
    \draw[draw=green, line width=0.1mm] (4, 4)  to (4, 4);

    \begin{scope}[thick, font=\Large]
        \draw[<->] (-4.3,  0.0) to (4.3, 0.0) node [above] {$\mathbb{R}$};
        \draw[<->] ( 0.0, -4.3) to (0.0, 4.3) node [right] {$\mathbb{R}$};
    \end{scope}

    \draw[draw=blue] (P1) to [out=-30, in=150]  (P2)
                          to [out=-30, in=210]  (P3)
                          to [out=30,  in=180]  (P4)
                          to [out=0,   in=200]  (P5);
    \draw[fill=white, draw=white] 
        (1.3, 2.0) rectangle node {$\textcolor{blue}{f}$} (1.6, 1.4);
\end{tikzpicture}
            \caption[Example of a Function $f:\mathbb{R}\rightarrow\mathbb{R}$]
                    {Example of a function $f:\mathbb{R}\rightarrow\mathbb{R}$.
                     The Cartesian product $\mathbb{R}\times\mathbb{R}$ is
                     shown in \textcolor{green!80!black}{green}, and the
                     function $f\subseteq\mathbb{R}\times\mathbb{R}$ is shown
                     in \textcolor{blue}{blue}.}
            \label{fig:Function_R_to_R_Subset_Cart_Prod}
        \end{figure}
        Let $g\subseteq\mathbb{R}\times\mathbb{R}$ be defined as follows:
        \begin{equation}
            g=\big\{\,(x,\,y)\in\mathbb{R}^{2}\;|\;y^{2}=x\,\big\}
        \end{equation}
        It is tempting to label $g$ by writing $g(x)=\sqrt{x}$, but $g$ is
        not a function for it fails two of the requirements of a function.
        Firstly, for any $x>0$, there are two values $y_{1}$ and $y_{2}$
        whose square is equal to $x$. Indeed, if $y_{1}$ is one such value,
        then setting $y_{2}=\minus{y}_{1}$ will result in a second
        distinct value. Thus $g$ does not have the uniqueness property
        required for functions. Moreover, if $x<0$, then there is no such
        value $y\in\mathbb{R}$ such that $(x,y)\in{g}$, and thus $g$ also
        lacks the existence property. In terms of the vertical line test,
        there are points $x$ such that the vertical line through
        $(x,\,0)$ intersects $g$ twice, and there are points such that the
        vertical line does not intersect at all. The graph of $g$ is shown
        in Fig.~\ref{fig:SQRT_Not_a_Function}.
        \begin{figure}[H]
            \centering
            %--------------------------------Dependencies----------------------------------%
%   xcolor                                                                     %
%   amssymb                                                                    %
%   tikz                                                                       %
%       arrows.meta                                                            %
%       patterns                                                               %
%-------------------------------Main Document----------------------------------%
\begin{tikzpicture}[%
    >=Latex,
    line width=0.2mm,
    line cap=round,
    scale=1.2
]
    % Coorindates for the curve.
    \coordinate (P1) at (-3.85, -2.00);
    \coordinate (P2) at (-2.00, -3.00);
    \coordinate (P3) at ( 0.00,  0.00);
    \coordinate (P4) at ( 2.00,  3.00);
    \coordinate (P5) at ( 3.85,  3.80);

    \draw[%
        pattern=north west lines,
        pattern color=Green!80!Black,
        opacity=0.5,
        draw=white
    ]   (-3.9, -3.9) rectangle (3.9, 3.9);

    \begin{scope}[thick, font=\Large]
        \draw[<->] (-4.2, 0) to (4.2, 0) node [above] {$\mathbb{R}$};
        \draw[<->] (0, -4.2) to (0, 4.2) node [right] {$\mathbb{R}$};
    \end{scope}

    \draw[draw=red] (3.880000, -1.969772) to (3.840000, -1.959592)
                                          to (3.800000, -1.949359)
                                          to (3.760000, -1.939072)
                                          to (3.720000, -1.928730)
                                          to (3.680000, -1.918333)
                                          to (3.640000, -1.907878)
                                          to (3.600000, -1.897367)
                                          to (3.560000, -1.886796)
                                          to (3.520000, -1.876166)
                                          to (3.480000, -1.865476)
                                          to (3.440000, -1.854724)
                                          to (3.400000, -1.843909)
                                          to (3.360000, -1.833030)
                                          to (3.320000, -1.822087)
                                          to (3.280000, -1.811077)
                                          to (3.240000, -1.800000)
                                          to (3.200000, -1.788854)
                                          to (3.160000, -1.777639)
                                          to (3.120000, -1.766352)
                                          to (3.080000, -1.754993)
                                          to (3.040000, -1.743560)
                                          to (3.000000, -1.732051)
                                          to (2.960000, -1.720465)
                                          to (2.920000, -1.708801)
                                          to (2.880000, -1.697056)
                                          to (2.840000, -1.685230)
                                          to (2.800000, -1.673320)
                                          to (2.760000, -1.661325)
                                          to (2.720000, -1.649242)
                                          to (2.680000, -1.637071)
                                          to (2.640000, -1.624808)
                                          to (2.600000, -1.612452)
                                          to (2.560000, -1.600000)
                                          to (2.520000, -1.587451)
                                          to (2.480000, -1.574802)
                                          to (2.440000, -1.562050)
                                          to (2.400000, -1.549193)
                                          to (2.360000, -1.536229)
                                          to (2.320000, -1.523155)
                                          to (2.280000, -1.509967)
                                          to (2.240000, -1.496663)
                                          to (2.200000, -1.483240)
                                          to (2.160000, -1.469694)
                                          to (2.120000, -1.456022)
                                          to (2.080000, -1.442221)
                                          to (2.040000, -1.428286)
                                          to (2.000000, -1.414214)
                                          to (1.960000, -1.400000)
                                          to (1.920000, -1.385641)
                                          to (1.880000, -1.371131)
                                          to (1.840000, -1.356466)
                                          to (1.800000, -1.341641)
                                          to (1.760000, -1.326650)
                                          to (1.720000, -1.311488)
                                          to (1.680000, -1.296148)
                                          to (1.640000, -1.280625)
                                          to (1.600000, -1.264911)
                                          to (1.560000, -1.249000)
                                          to (1.520000, -1.232883)
                                          to (1.480000, -1.216553)
                                          to (1.440000, -1.200000)
                                          to (1.400000, -1.183216)
                                          to (1.360000, -1.166190)
                                          to (1.320000, -1.148913)
                                          to (1.280000, -1.131371)
                                          to (1.240000, -1.113553)
                                          to (1.200000, -1.095445)
                                          to (1.160000, -1.077033)
                                          to (1.120000, -1.058301)
                                          to (1.080000, -1.039230)
                                          to (1.040000, -1.019804)
                                          to (1.000000, -1.000000)
                                          to (0.960000, -0.979796)
                                          to (0.920000, -0.959166)
                                          to (0.880000, -0.938083)
                                          to (0.840000, -0.916515)
                                          to (0.800000, -0.894427)
                                          to (0.760000, -0.871780)
                                          to (0.720000, -0.848528)
                                          to (0.680000, -0.824621)
                                          to (0.640000, -0.800000)
                                          to (0.600000, -0.774597)
                                          to (0.560000, -0.748331)
                                          to (0.520000, -0.721110)
                                          to (0.480000, -0.692820)
                                          to (0.440000, -0.663325)
                                          to (0.400000, -0.632456)
                                          to (0.360000, -0.600000)
                                          to (0.320000, -0.565685)
                                          to (0.280000, -0.529150)
                                          to (0.240000, -0.489898)
                                          to (0.200000, -0.447214)
                                          to (0.160000, -0.400000)
                                          to (0.120000, -0.346410)
                                          to (0.080000, -0.282843)
                                          to (0.040000, -0.200000)
                                          to (0.000000, 0.000000) 
                                          to (0.040000, 0.200000)
                                          to (0.080000, 0.282843)
                                          to (0.120000, 0.346410)
                                          to (0.160000, 0.400000)
                                          to (0.200000, 0.447214)
                                          to (0.240000, 0.489898)
                                          to (0.280000, 0.529150)
                                          to (0.320000, 0.565685)
                                          to (0.360000, 0.600000)
                                          to (0.400000, 0.632456)
                                          to (0.440000, 0.663325)
                                          to (0.480000, 0.692820)
                                          to (0.520000, 0.721110)
                                          to (0.560000, 0.748331)
                                          to (0.600000, 0.774597)
                                          to (0.640000, 0.800000)
                                          to (0.680000, 0.824621)
                                          to (0.720000, 0.848528)
                                          to (0.760000, 0.871780)
                                          to (0.800000, 0.894427)
                                          to (0.840000, 0.916515)
                                          to (0.880000, 0.938083)
                                          to (0.920000, 0.959166)
                                          to (0.960000, 0.979796)
                                          to (1.000000, 1.000000)
                                          to (1.040000, 1.019804)
                                          to (1.080000, 1.039230)
                                          to (1.120000, 1.058301)
                                          to (1.160000, 1.077033)
                                          to (1.200000, 1.095445)
                                          to (1.240000, 1.113553)
                                          to (1.280000, 1.131371)
                                          to (1.320000, 1.148913)
                                          to (1.360000, 1.166190)
                                          to (1.400000, 1.183216)
                                          to (1.440000, 1.200000)
                                          to (1.480000, 1.216553)
                                          to (1.520000, 1.232883)
                                          to (1.560000, 1.249000)
                                          to (1.600000, 1.264911)
                                          to (1.640000, 1.280625)
                                          to (1.680000, 1.296148)
                                          to (1.720000, 1.311488)
                                          to (1.760000, 1.326650)
                                          to (1.800000, 1.341641)
                                          to (1.840000, 1.356466)
                                          to (1.880000, 1.371131)
                                          to (1.920000, 1.385641)
                                          to (1.960000, 1.400000)
                                          to (2.000000, 1.414214)
                                          to (2.040000, 1.428286)
                                          to (2.080000, 1.442221)
                                          to (2.120000, 1.456022)
                                          to (2.160000, 1.469694)
                                          to (2.200000, 1.483240)
                                          to (2.240000, 1.496663)
                                          to (2.280000, 1.509967)
                                          to (2.320000, 1.523155)
                                          to (2.360000, 1.536229)
                                          to (2.400000, 1.549193)
                                          to (2.440000, 1.562050)
                                          to (2.480000, 1.574802)
                                          to (2.520000, 1.587451)
                                          to (2.560000, 1.600000)
                                          to (2.600000, 1.612452)
                                          to (2.640000, 1.624808)
                                          to (2.680000, 1.637071)
                                          to (2.720000, 1.649242)
                                          to (2.760000, 1.661325)
                                          to (2.800000, 1.673320)
                                          to (2.840000, 1.685230)
                                          to (2.880000, 1.697056)
                                          to (2.920000, 1.708801)
                                          to (2.960000, 1.720465)
                                          to (3.000000, 1.732051)
                                          to (3.040000, 1.743560)
                                          to (3.080000, 1.754993)
                                          to (3.120000, 1.766352)
                                          to (3.160000, 1.777639)
                                          to (3.200000, 1.788854)
                                          to (3.240000, 1.800000)
                                          to (3.280000, 1.811077)
                                          to (3.320000, 1.822087)
                                          to (3.360000, 1.833030)
                                          to (3.400000, 1.843909)
                                          to (3.440000, 1.854724)
                                          to (3.480000, 1.865476)
                                          to (3.520000, 1.876166)
                                          to (3.560000, 1.886796)
                                          to (3.600000, 1.897367)
                                          to (3.640000, 1.907878)
                                          to (3.680000, 1.918333)
                                          to (3.720000, 1.928730)
                                          to (3.760000, 1.939072)
                                          to (3.800000, 1.949359)
                                          to (3.840000, 1.959592)
                                          to (3.880000, 1.969772);
    \draw[fill=white, draw=white] 
        (1.3, 2.0) rectangle node {$\textcolor{red}{g}$} (1.6, 1.5);
\end{tikzpicture}
            \caption[Example of a Non-Function]
                {$g\subseteq\mathbb{R}\times\mathbb{R}$ is not a function
                 since it fails the vertical line test.}
            \label{fig:SQRT_Not_a_Function}
        \end{figure}
        We need not only consider functions of the form
        $f:\mathbb{R}\rightarrow\mathbb{R}$, nor are we restricted to function
        liked $f:\mathcal{U}\rightarrow\mathcal{V}$ where $\mathcal{U}$ and
        $\mathcal{V}$ are subsets of $\mathbb{R}$, and we can allow for
        arbitrary abstract functions. Let $A$ and $B$ be defined as follows:
        \index{Function}
        \par
        \begin{subequations}
            \begin{minipage}[b]{0.49\textwidth}
                \centering
                \begin{equation}
                    A=\{\,1,\,2,\,3,\,4\,\}
                \end{equation}
            \end{minipage}
            \hfill
            \begin{minipage}[b]{0.49\textwidth}
                \centering
                \begin{equation}
                    B=\{\,a,\,b,\,c\,\}
                \end{equation}
            \end{minipage}
        \end{subequations}
        \par\vspace{2.5ex}
        Similar to the vertical line test\index{Vertical Line Test}, we can
        devise a visual to discerning functions from non-functions for abstract
        sets. We represent the elements of $A$ and $B$ as points in some blob
        in the plane, and then draw arrows between the points
        $x\in{A}$ and $y\in{b}$ indicating that $(x,y)\in{f}$. This allows us to
        determine if a given $f\subseteq{A}\times{B}$ is a functions ore not.
        Every point in $A$ must be mapped to a unique point in $B$. That is,
        every point in $A$ must have one and only one arrow connecting it to
        some point in $B$. Examples of valid functions are shown in
        Fig.~\ref{fig:Abstract_Functions}, and non-functions are shown in
        Fig.~\ref{fig:Abstract_Non_Functions}.
        \begin{figure}[H]
            \centering
            \begin{subfigure}[b]{0.49\textwidth}
                \centering
                \resizebox{\textwidth}{!}{%
                    %--------------------------------Dependencies----------------------------------%
%   tikz                                                                       %
%       arrows.meta                                                            %
%-------------------------------Main Document----------------------------------%
\begin{tikzpicture}[%
    >=latex,
    line width=0.2mm,
    line cap=round,
    scale=1.2
]
    % Coorindates.
    \coordinate (a) at ( 1.5,  0.75);
    \coordinate (b) at ( 1.5, -0.00);
    \coordinate (c) at ( 1.5, -0.75);
    \coordinate (1) at (-1.5,  1.20);
    \coordinate (2) at (-1.5,  0.40);
    \coordinate (3) at (-1.5, -0.40);
    \coordinate (4) at (-1.5, -1.20);
    \coordinate (A) at (-1.5,  2.50);
    \coordinate (B) at ( 1.5,  2.50);

    % Ellipses representing the sets A and B.
    \draw[thick] (-1.5, 0.0) ellipse (1 and 2);
    \draw[thick] ( 1.5, 0.0) ellipse (1 and 2);

    % Draw circles for the various points.
    \draw[fill=black] (a) circle (0.4mm);
    \draw[fill=black] (b) circle (0.4mm);
    \draw[fill=black] (c) circle (0.4mm);
    \draw[fill=black] (1) circle (0.4mm);
    \draw[fill=black] (2) circle (0.4mm);
    \draw[fill=black] (3) circle (0.4mm);
    \draw[fill=black] (4) circle (0.4mm);

    % Draw paths indicating mappings.
    \begin{scope}[->]
        \draw[shorten >=0.8mm] (1) to (c);
        \draw[shorten >=0.8mm] (2) to (a);
        \draw[shorten >=0.8mm] (3) to (b);
        \draw[shorten >=0.8mm] (4) to (a);
    \end{scope}

    % Labels.
    \node at (A)         {$A$};
    \node at (B)         {$B$};
    \node at (a) [right] {$a$};
    \node at (b) [right] {$b$};
    \node at (c) [right] {$c$};
    \node at (1) [left]  {$1$};
    \node at (2) [left]  {$2$};
    \node at (3) [left]  {$3$};
    \node at (4) [left]  {$4$};
\end{tikzpicture}
                }
                \subcaption{A Valid Function.}
            \end{subfigure}
            \begin{subfigure}[b]{0.49\textwidth}
                \centering
                \resizebox{\textwidth}{!}{%
                    %--------------------------------Dependencies----------------------------------%
%   tikz                                                                       %
%       arrows.meta                                                            %
%-------------------------------Main Document----------------------------------%
\begin{tikzpicture}[%
    >=latex,
    line width=0.2mm,
    line cap=round,
    scale=1.2
]
    % Coorindates.
    \coordinate (a) at ( 1.5,  0.75);
    \coordinate (b) at ( 1.5, -0.00);
    \coordinate (c) at ( 1.5, -0.75);
    \coordinate (1) at (-1.5,  1.20);
    \coordinate (2) at (-1.5,  0.40);
    \coordinate (3) at (-1.5, -0.40);
    \coordinate (4) at (-1.5, -1.20);
    \coordinate (A) at (-1.5,  2.50);
    \coordinate (B) at ( 1.5,  2.50);

    % Ellipses representing the sets A and B.
    \draw[thick] (-1.5, 0.0) ellipse (1 and 2);
    \draw[thick] ( 1.5, 0.0) ellipse (1 and 2);

    % Draw circles for the various points.
    \draw[fill=black] (a) circle (0.4mm);
    \draw[fill=black] (b) circle (0.4mm);
    \draw[fill=black] (c) circle (0.4mm);
    \draw[fill=black] (1) circle (0.4mm);
    \draw[fill=black] (2) circle (0.4mm);
    \draw[fill=black] (3) circle (0.4mm);
    \draw[fill=black] (4) circle (0.4mm);

    % Draw paths indicating mappings.
    \begin{scope}[->]
        \draw[shorten >=0.8mm] (1) to (a);
        \draw[shorten >=0.8mm] (2) to (a);
        \draw[shorten >=0.8mm] (3) to (a);
        \draw[shorten >=0.8mm] (4) to (a);
    \end{scope}

    % Labels.
    \node at (A)         {$A$};
    \node at (B)         {$B$};
    \node at (a) [right] {$a$};
    \node at (b) [right] {$b$};
    \node at (c) [right] {$c$};
    \node at (1) [left]  {$1$};
    \node at (2) [left]  {$2$};
    \node at (3) [left]  {$3$};
    \node at (4) [left]  {$4$};
\end{tikzpicture}
                }
                \subcaption{Another Valid Function.}
            \end{subfigure}
            \caption{Visual for Abstract Functions}
            \label{fig:Abstract_Functions}
        \end{figure}
        \begin{figure}[H]
            \centering
            \begin{subfigure}[b]{0.49\textwidth}
                \centering
                \resizebox{\textwidth}{!}{%
                    %--------------------------------Dependencies----------------------------------%
%   tikz                                                                       %
%       arrows.meta                                                            %
%-------------------------------Main Document----------------------------------%
\begin{tikzpicture}[%
    >=latex,
    line width=0.2mm,
    line cap=round,
    scale=1.2
]
    % Coorindates.
    \coordinate (a) at ( 1.5,  0.75);
    \coordinate (b) at ( 1.5, -0.00);
    \coordinate (c) at ( 1.5, -0.75);
    \coordinate (1) at (-1.5,  1.20);
    \coordinate (2) at (-1.5,  0.40);
    \coordinate (3) at (-1.5, -0.40);
    \coordinate (4) at (-1.5, -1.20);
    \coordinate (A) at (-1.5,  2.50);
    \coordinate (B) at ( 1.5,  2.50);

    % Ellipses representing the sets A and B.
    \draw[thick] (-1.5, 0.0) ellipse (1 and 2);
    \draw[thick] ( 1.5, 0.0) ellipse (1 and 2);

    % Draw circles for the various points.
    \draw[fill=black] (a) circle (0.4mm);
    \draw[fill=black] (b) circle (0.4mm);
    \draw[fill=black] (c) circle (0.4mm);
    \draw[fill=black] (1) circle (0.4mm);
    \draw[fill=black] (2) circle (0.4mm);
    \draw[fill=black] (3) circle (0.4mm);
    \draw[fill=black] (4) circle (0.4mm);

    % Draw paths indicating mappings.
    \begin{scope}[->]
        \draw[shorten >=0.8mm] (1) to (a);
        \draw[shorten >=0.8mm] (2) to (b);
        \draw[shorten >=0.8mm] (3) to (c);
    \end{scope}

    % Labels.
    \node at (A)         {$A$};
    \node at (B)         {$B$};
    \node at (a) [right] {$a$};
    \node at (b) [right] {$b$};
    \node at (c) [right] {$c$};
    \node at (1) [left]  {$1$};
    \node at (2) [left]  {$2$};
    \node at (3) [left]  {$3$};
    \node at (4) [left]  {$4$};
\end{tikzpicture}
                }
                \subcaption{Fails Existence.}
            \end{subfigure}
            \begin{subfigure}[b]{0.49\textwidth}
                \centering
                \resizebox{\textwidth}{!}{%
                    %--------------------------------Dependencies----------------------------------%
%   tikz                                                                       %
%       arrows.meta                                                            %
%-------------------------------Main Document----------------------------------%
\begin{tikzpicture}[%
    >=latex,
    line width=0.2mm,
    line cap=round,
    scale=1.2
]
    % Coorindates.
    \coordinate (a) at ( 1.5,  0.75);
    \coordinate (b) at ( 1.5, -0.00);
    \coordinate (c) at ( 1.5, -0.75);
    \coordinate (1) at (-1.5,  1.20);
    \coordinate (2) at (-1.5,  0.40);
    \coordinate (3) at (-1.5, -0.40);
    \coordinate (4) at (-1.5, -1.20);
    \coordinate (A) at (-1.5,  2.50);
    \coordinate (B) at ( 1.5,  2.50);

    % Ellipses representing the sets A and B.
    \draw[thick] (-1.5, 0.0) ellipse (1 and 2);
    \draw[thick] ( 1.5, 0.0) ellipse (1 and 2);

    % Draw circles for the various points.
    \draw[fill=black] (a) circle (0.4mm);
    \draw[fill=black] (b) circle (0.4mm);
    \draw[fill=black] (c) circle (0.4mm);
    \draw[fill=black] (1) circle (0.4mm);
    \draw[fill=black] (2) circle (0.4mm);
    \draw[fill=black] (3) circle (0.4mm);
    \draw[fill=black] (4) circle (0.4mm);

    % Draw paths indicating mappings.
    \begin{scope}[->]
        \draw[shorten >=0.8mm] (1) to (a);
        \draw[shorten >=0.8mm] (2) to (a);
        \draw[shorten >=0.8mm] (2) to (b);
        \draw[shorten >=0.8mm] (3) to (c);
        \draw[shorten >=0.8mm] (4) to (c);
    \end{scope}

    % Labels.
    \node at (A)         {$A$};
    \node at (B)         {$B$};
    \node at (a) [right] {$a$};
    \node at (b) [right] {$b$};
    \node at (c) [right] {$c$};
    \node at (1) [left]  {$1$};
    \node at (2) [left]  {$2$};
    \node at (3) [left]  {$3$};
    \node at (4) [left]  {$4$};
\end{tikzpicture}
                }
                \subcaption{Fails Uniqueness.}
            \end{subfigure}
            \caption{Non-Functions}
            \label{fig:Abstract_Non_Functions}
        \end{figure}
        It is possible to count the total number of functions from $A$ to $B$.
        Since every element of $A$ needs to be mapped to some element of $B$,
        and since there are 4 elements in $A$ and 3 elements in $B$, the total
        number of functions $f:A\rightarrow{B}$ is $4^{3}=64$. On the other
        hand, the total number of subsets of $A\times{B}$ is $2^{12}=4096$
        (we will justify this when we discuss the \textit{cardinality} of
        sets). Thus, if we were to randomly pick a subset of $A\times{B}$, the
        odds are that it is almost certainly \textit{not} a function
        (1.5625\%). Thus, functions are very special subsets.
        There is a frequent need to discuss the \textit{set of all functions}
        from a given set $A$ into another set $B$. To ensure we don't create
        a function version of Russell's paradox, we prove such a set exists.
        \begin{theorem}
            If $A$ and $B$ are sets, then there exists a set $\mathcal{F}$ such
            that, for all $f$ it is true that $f\in\mathcal{F}$ if and only if
            $f$ is a function from $A$ to $B$, $f:A\rightarrow{B}$.
            \index{Function!Set of All}
        \end{theorem}
        \begin{proof}
            For if $A$ and $B$ are sets, then by
            Thm.~\ref{thm:Existence_of_the_Cartesian_Product} the set
            $A\times{B}$ exists. But by
            Thm.~\ref{thm:Existence_of_the_Power_Set}, the power set of
            $A\times{B}$, $\mathcal{P}(A\times{B})$, exists. Let $P$ be the
            proposition \textit{True if} $f$ \textit{is a function from} $A$
            \textit{to} $B$, \textit{false otherwise}. Then by axiom schema of
            specification (Ax.~\ref{ax:Axiom_Schema_of_Specification}), there is
            a set $\mathcal{F}$ such that:
            \begin{equation}
                \mathcal{F}=\big\{\,f\in\mathcal{P}(A\times{B})\;|
                    \;P(f)\,\big\}
            \end{equation}
            But then for all $f$, $f\in\mathcal{F}$ if and only if
            $f\in\mathcal{F}$ and $P(f)$ is true. But if $P(f)$ is true then
            $f$ is a function from $A$ to $B$, and thus by the definition of a
            function (Def.~\ref{def:Function}) $f\subseteq{A}\times{B}$. But
            then by the definition of the power set (Def.~\ref{def:Power_Set})
            we have that $f\in\mathcal{P}(A\times{B})$. Thus $P(f)$ implies
            $f\in\mathcal{F}$. Therefore $f\in\mathcal{F}$ if and only if
            $P(f)$. That is, $f\in\mathcal{F}$ if and only if $f$ is a function
            from $A$ to $B$.
        \end{proof}
        There is non-standard notation when discussing the set of all functions
        from a given set $A$ to a set $B$:
        \begin{fnotation}{Set of All Functions}{Set_of_All_Functions}
            If $A$ and $B$ are sets, the set of all functions from $A$ to $B$,
            $f:A\rightarrow{B}$, is denoted as either $\mathcal{F}(A,B)$ or
            $B^{A}$.
        \end{fnotation}
        The notation $B^{A}$ is common in many areas such as topology and
        algebra, especially when $A=B$. The \textit{topological space} $I^{I}$,
        which is the set of all functions from the \textit{closed unit inverval}
        \index{Interval!Closed} to itself, is often used to construct examples
        and counterexamples. In analysis the notation $\mathcal{F}(A,B)$ seems
        to be more common, in particular $\mathcal{C}(A,B)$ is often used to
        denote the set of all \textit{continuous} functions from $A$ to $B$,
        provided the word continuous has meaning. Since the notation is not
        universal nor standard across the various disciplines, an attempt will
        be made to specify what $B^{A}$ or $\mathcal{F}(A,B)$ means before using
        it in a theorem or counterexample.
        \begin{fdefinition}{Image of a Point}{Image_of_Point}
            The \gls{image of a point} of an element $x$ in a set $A$ under a
            \gls{function} $f:A\rightarrow{B}$ is the unique value $y\in{B}$
            such that $(x,y)\in{f}$. We write $y=f(x)$.
            \index{Image!of a Point}
        \end{fdefinition}
        This allows us to define functions by simply specifying what the
        image of each $x\in{A}$ is. Restating our previous claim, if we can
        define some formula such that for each $x\in{A}$ there is a unique
        $f(x)\in{B}$ such that the formula takes $x$ to $f(x)$, then we can
        define $f$ as the set of all such ordered pairs $(x,f(x))$, and this
        will be a function.
        \begin{fnotation}{Image Notation}{Image_Notation}
            If $A$ and $B$ are sets, if $f:A\rightarrow{B}$ is a function,
            if $x\in{A}$ and if $y=f(x)\in{B}$, then we denote this by
            writing $x\overset{f}{\longmapsto}{y}$ or just $x\mapsto{y}$.
        \end{fnotation}
        Throughout we will almost exclusively use the notation $y=f(x)$ rather
        than $x\mapsto{y}$. The reasons are purely aesthetic and both notations
        are common in mathematics. In a similar manner, we can define the image
        of an entire subset.
        \begin{theorem}
            If $A$ and $B$ are sets, if $f:A\rightarrow{B}$ is a function,
            and if $\mathcal{U}\subseteq{A}$, then there is a set
            $\mathcal{V}\subseteq{B}$ such that for all $y$ it is true that
            $y\in\mathcal{V}$ if and only if $y\in{B}$ and such that there is
            an $x\in\mathcal{U}$ such that $y=f(x)$.
        \end{theorem}
        \begin{proof}
            For let $P$ be the proposition \textit{True if there exists}
            $x\in\mathcal{U}$ \textit{such that} $y=f(x)$,
            \textit{false otherwise}. By the axiom schema of specification
            (Ax.~\ref{ax:Axiom_Schema_of_Specification}) there is a set
            $\mathcal{V}$ such that, for all $y$ it is true that
            $y\in\mathcal{V}$ if and only if $y\in{B}$ and $P(y)$ is true. That
            is, $y\in\mathcal{V}$ if and only if $y\in{B}$ and if there is an
            $x\in\mathcal{U}$ such that $y=f(x)$.
        \end{proof}
        \begin{fdefinition}{Image of a Subset}{Image_of_Subset}
            The \gls{set image} of a \gls{subset} $\mathcal{U}$ of a \gls{set}
            $A$ under a \gls{function} $f:A\rightarrow{B}$ is the
            set:\index{Image!of a Set}
            \begin{equation*}
                f\big(\mathcal{U}\big)=
                    \{\,y\in{B}\;|\;\textrm{There exists }x\in\mathcal{U}
                                    \textrm{ such that }y=f(x)\,\}
            \end{equation*}
            That is, the set of all points in $B$ that are the image of points
            in $\mathcal{U}$. Formally:
            \begin{equation*}
                \forall_{A}\forall_{B}\forall_{f:A\rightarrow{B}}
                \forall_{\mathcal{U}\subseteq{A}}\forall_{y}\Big(
                    \big(y\in{f}(\mathcal{U})\big)\Longleftrightarrow
                    \big(\exists_{x\in\mathcal{U}}:y=f(x)\big)\Big)
            \end{equation*}
        \end{fdefinition}
        This definition of the image of a subset was given in such a manner
        so that it only relies on the axiom schema of specification to
        justify it's existence. We could also use the notation:
        \begin{equation}
            f\big(\mathcal{U})=\{\,f(x)\in{B}\;|\;x\in\mathcal{U}\,\}
        \end{equation}
        Writing the definition of the image of a subset in such a way is
        justified by the \textit{axiom schema of replacement}%
        \index{Axiom!Schema of Replacement}, but we've not yet included this
        axiom in our system.
        \begin{example}
            If $f:\mathbb{R}\rightarrow\mathbb{R}$ is the function $f(x)=x^{2}$,
            then:
            \begin{equation}
                f(\mathbb{R})=[0,\infty)\equiv\{\,x\in\mathbb{R}\;|\;x\geq{0}\}
            \end{equation}
            This is because every non-negative real number $y$ gets mapped to
            by at least one real number (the positive square root $\sqrt{y}$).
            None of the negative numbers are the image of any element of
            $\mathbb{R}$ since the square of a real number is always
            non-negative.
        \end{example}
        We can visualize functions and images by using blobs in the plane. Given
        some sub-blob of a set $A$, the image of this will be another sub-blob
        of $B$. Note that if $f:A\rightarrow{B}$ is a function, it does
        \textbf{not} need to be true that $f(A)=B$. These are special functions
        that are called \textit{surjective}\index{Function!Surjective} and are
        discussed in Chapt.~\ref{chapt:Function_Theory}. Such a drawing of the
        general case is shown in Fig.~\ref{fig:Image_of_Point_and_Subset}.
        \begin{figure}[H]
            \centering
            \captionsetup{type=figure}
            \begin{tikzpicture}[>=Latex]
    \coordinate (U1) at (-5.0, -2.0);
    \coordinate (U2) at (-3.5, -2.0);
    \coordinate (U3) at (-0.5, -0.5);
    \coordinate (U4) at (-2.0,  2.0);
    \coordinate (U5) at (-3.3,  1.6);
    \coordinate (U6) at (-4.0,  2.0);
    \coordinate (U7) at (-5.0,  0.0);

    \coordinate (V1) at (5.0,  2.0);
    \coordinate (V2) at (4.0,  2.0);
    \coordinate (V3) at (2.0,  0.0);
    \coordinate (V4) at (2.0, -2.0);
    \coordinate (V5) at (4.0, -1.0);
    \coordinate (V6) at (5.0, -1.0);

    \coordinate (S1) at (-4.0,  0.0);
    \coordinate (S2) at (-3.0, -1.0);
    \coordinate (S3) at (-2.5,  0.0);
    \coordinate (S4) at (-3.0,  0.5);

    \coordinate (T1) at (3.0,  0.0);
    \coordinate (T2) at (3.5, -0.8);
    \coordinate (T3) at (4.5,  0.0);
    \coordinate (T4) at (3.5,  0.8);

    \coordinate (x)  at (-2.8, -0.3);
    \coordinate (fx) at (3.4,  -0.4);

    \draw[fill=blue,opacity=0.5,draw=black,thick]
        (U1)    to[out=0,  in=-150] (U2)
                to[out=30, in=-90]  (U3)
                to[out=90, in=-60]  (U4)
                to[out=120, in=-30] (U5)
                to[out=150,in=10]   (U6)
                to[out=-170,in=90]  (U7)
                to[out=-90,in=180]  cycle;

    \draw[fill=red!80!white,opacity=0.5,draw=black,thick]
        (V1)    to[out=180, in=0]       (V2)
                to[out=180, in=60]      (V3)
                to[out=-120,in=120]     (V4)
                to[out=-60, in=180]     (V5)
                to[out=0,   in=-120]    (V6)
                to[out=60,  in=0]       cycle;

    \draw[fill=blue!80!white] (S1)  to[out=-150,in=180] (S2)
                                    to[out=0,   in=-90] (S3)
                                    to[out=90,  in=-60] (S4)
                                    to[out=120, in=30]  cycle;

    \draw[fill=red!80!white] (T1)   to[out=-150,in=180] (T2)
                                    to[out=0,   in=-90] (T3)
                                    to[out=90,  in=0]   (T4)
                                    to[out=180, in=30]  cycle;

    \draw[fill=black] (x)  circle (0.3mm);
    \draw[fill=black] (fx) circle (0.3mm);
    \node at (-2.5, 1.0) {\Large{$A$}};
    \node at ( 4.6, 1.3) {\Large{$B$}};
    \node at (-3.3, 0.0) {\large{$\mathcal{U}$}};
    \node at ( 3.7, 0.2) {\large{$f(\mathcal{U})$}};
    \node at (x)  [below] {$x$};
    \node at (fx) [right] {$f(x)$};
    \draw[->,shorten >= 1.5mm,shorten <= 1.5mm]
        (x) to[out=-30,in=-150] node[below]{$f$} (fx);
\end{tikzpicture}
            \caption{Image of a Subset and of a Point under a Function}
            \label{fig:Image_of_Point_and_Subset}
        \end{figure}
        If we consider a function $f:A\rightarrow{B}$ and the image of the
        entire set $A$ we obtain the \textit{range} of $f$. That is, the range
        is the set $f(A)\subseteq{B}$. In a similar manner to the forward image
        of a function, we can define the pre-image.
        \begin{theorem}
            \label{thm:Existence_of_Pre_Image}%
            If $A$ and $B$ are sets, if $f:A\rightarrow{B}$ is a function from
            $A$ to $B$, and if $\mathcal{V}\subseteq{B}$, then there is a set
            $\mathcal{U}\subseteq{A}$ such that for all $x$ it is true that
            $x\in\mathcal{U}$ if and only if $x\in{A}$ and $f(x)\in\mathcal{V}$.
        \end{theorem}
        \begin{proof}
            For let $P$ be the proposition \textit{True if} $x\in{A}$
            \textit{and} $f(x)\in\mathcal{V}$, \textit{false otherwise}. Then
            by the axiom schema of specification
            (Ax.~\ref{ax:Axiom_Schema_of_Specification}) there is a set
            $\mathcal{U}$ such that:
            \begin{equation*}
                \mathcal{U}=\big\{\,x\in{A}\;|\;P(x)\,\big\}
            \end{equation*}
            But $P(x)$ implies $x\in{A}$ and thus $x\in\mathcal{U}$ if and only
            if $x\in{A}$ and $f(x)\in\mathcal{V}$.
        \end{proof}
        \begin{fdefinition}{Pre-Image of a Subset}{Pre_Image_of_Subset}
            The \gls{pre-image} of a \gls{subset} $\mathcal{V}\subseteq{B}$
            under a \gls{function} $f:A\rightarrow{B}$ is the set:
            \index{Pre-Image}
            \begin{equation}
                f^{\minus{1}}(\mathcal{V})
                =\big\{\,x\in{A}\;|\;f(x)\in\mathcal{V}\,\big\}
            \end{equation}
            Using our formal language:
            \begin{equation*}
                \forall_{A}\forall_{B}\forall_{f:A\rightarrow{B}}
                \forall_{\mathcal{V}\subseteq{B}}\forall_{x}\Big(
                    \big(x\in{f}^{\minus{1}}(\mathcal{V})\big)
                    \Longleftrightarrow
                    \big((x\in{A})\land(f(x)\in\mathcal{V}))\big)\Big)
            \end{equation*}
        \end{fdefinition}
        The pre-image of a set behaves a lot differently than the image, and
        this will be explored in detail when functions are discussed. The cause
        of the discrepancy is the requirement that elements of $A$ map uniquely
        to elements of $B$, but a single element in $B$ can be the image of
        many different points in $A$. This gives rise to the notion of the
        \textit{fiber} of a point in $B$.
        \begin{theorem}
            If $A$ and $B$ are sets, if $f:A\rightarrow{B}$ is a function, and
            if $b\in{A}$, then there is a set $\mathcal{U}\subseteq{A}$ such
            for all $x\in{A}$ it is true that $x\in\mathcal{U}$ if and only if
            $f(x)=b$.
        \end{theorem}
        \begin{proof}
            For by Thm.~\ref{thm:Existence_of_Set_Containing_Set}, $\{b\}$ is
            a set and $\{b\}\subseteq{B}$ (Def.~\ref{def:Subsets}). But if
            $\{b\}$ is a subset of $B$, then there is a set
            $\mathcal{U}\subseteq{A}$ such that for all $x\in{A}$ it is true
            that $x\in\mathcal{U}$ if and only if $f(x)\in\{b\}$
            (Thm.~\ref{thm:Existence_of_Pre_Image}). But $f(x)\in\{b\}$ if and
            only if $f(x)=b$ (Thm.~\ref{thm:Existence_of_Set_Containing_Set}).
            Thus, for all $x\in{A}$, $x\in\mathcal{U}$ if and only if $f(x)=b$.
        \end{proof}
        \begin{fdefinition}{Fiber of an Element}{Fiber_of_Element}
            The \gls{fiber} of an element $b$ in a \gls{set} $B$ under a
            \gls{function} $f:A\rightarrow{B}$ from a set $A$ to a set $B$ is
            the pre-image of the set $\{b\}$. That is:\index{Fiber}
            \begin{equation*}
                f^{\minus{1}}(\{b\})
                =\big\{\,a\in{A}\:|\;f(a)=b\,\big\}
            \end{equation*}
        \end{fdefinition}
        Before we end our introduction to the concept of a function, we should
        first note the other synonomous terminology that is used and give some
        historical background. In many texts, including this one, authors use
        the word \textit{map} or \textit{mapping} to denote a function, rather
        than simply use the word function. This language has its roots in
        geometry, and particularly in the study of projective geometry. One can
        consider \textit{perspectives}, which are functions that map one line in
        the Euclidean plane to another. Consider two non-parallel lines
        $\overline{AB}$ and $\overline{CD}$ and a point $P$ in the plane that
        lies on neither of these two. Given a point $x$ which lies on
        $\overline{AB}$, we map this to the line $\overline{CD}$ by drawing a
        straight line from $P$ to $x$ and marking where this intersects
        $\overline{CD}$
        (see Fig.~\ref{fig:Example_of_Map_from_Projective_Geometry}).
        \begin{figure}[H]
            \centering
            \captionsetup{type=figure}
            \begin{tikzpicture}[>=Latex]
    \coordinate (A) at (-4.0000, -1.0000);
    \coordinate (B) at ( 4.0000,  1.0000);
    \coordinate (C) at (-4.0000,  0.0000);
    \coordinate (D) at ( 4.0000, -1.0000);
    \coordinate (P) at ( 0.0000,  1.5000);
    \coordinate (x) at ( 1.0000,  0.2500);
    \coordinate (y) at ( 1.7778, -0.7222);

    \draw[<->, shorten >=-1cm, shorten <=-1cm] (A) to (B);
    \draw[<->, shorten >=-1cm, shorten <=-1cm] (C) to (D);
    \draw (P) to (y);

    \draw[fill=black] (P) circle (0.5mm);
    \draw[fill=black] (A) circle (0.5mm);
    \draw[fill=black] (B) circle (0.5mm);
    \draw[fill=black] (C) circle (0.5mm);
    \draw[fill=black] (D) circle (0.5mm);
    \draw[fill=black] (x) circle (0.5mm);
    \draw[fill=black] (y) circle (0.5mm);

    \node at (A) [below] {$A$};
    \node at (B) [above] {$B$};
    \node at (C) [above] {$C$};
    \node at (D) [below] {$D$};
    \node at (P) [right] {$P$};
    \node at (x) [above] {$x$};
    \node at (y) [below] {$y$};
\end{tikzpicture}
            \caption{Example of a Mapping from Projective Geometry}
            \label{fig:Example_of_Map_from_Projective_Geometry}
        \end{figure}
        Such constructions play a central role in geometry, and particularly
        projective geometry, and culminate in a beautiful theorem known as
        \textit{Desargues's Theorem}\index{Theorem!Desargues's}, named after
        Girard Desargues. From this historical motivation, the word function,
        map, and mapping are synonomous in the realm of mathematics. In select
        fields such as analysis and geometry the words \textit{operator} and
        \textit{transformation} are used, and occasionally the word
        \textit{graph} is used. We will try to be clear in our useage of this
        vocabularly to rid of any potential ambiguity.
    \subsection{The Axiom of Choice and Diaconescu's Theorem}
        The next two axioms to be introduced are the most controversial of those
        listed in ZFC: The \textit{axiom of infinity} and the
        \textit{axiom of choice}. While the axiom of infinity only has a
        small number of critics, the axiom of choice is far more contentious.
        Choice is equivalent to many other statements that come across in
        almost all forms of mathematics (analysis, algebra, topology, etc.).
        Many of which are theorems we would \textit{want} to be true, and so
        accepting the axiom of choice allows us to prove them. In particular,
        the axioms presented thus far can be combined with the axiom of choice
        to prove the \textit{Law of the Excluded Middle}, a result known as
        Diaconescu's theorem, and this is our current goal.
        \begin{faxiom}{Axiom of Choice}{Axiom_of_Choice}
            If $\mathcal{O}$ is a non-empty set such that for all
            $\mathcal{U}\in\mathcal{O}$ it is true that $\mathcal{U}$ is
            non-empty, and if $\bigcup\mathcal{O}$ is the union over
            $\mathcal{O}$, then there is a function
            $f:\mathcal{O}\rightarrow\mathcal{F}$ such that for all
            $x\in\mathcal{O}$ it is true that $f(x)\in{x}$.
            Formally:\index{Axiom!of Choice}
            \begin{equation*}
                \forall_{\mathcal{O}}\big((\mathcal{O}\ne\emptyset)\land
                    (\emptyset\notin\mathcal{O})\big)
                \exists_{f:\mathcal{O}\rightarrow\bigcup\mathcal{O}}\Big(
                \forall_{x\in\mathcal{O}}\big(f(x)\in{x}\big)\Big)
            \end{equation*}
        \end{faxiom}
        Such a function is called a \textit{choice function}%
        \index{Function!Choice Function}. The axiom can be made obviously true
        or obviously false depending on how we word it. To convince one of its
        validity requires talking about products\index{Product!of Sets}. The
        Cartesian product\index{Cartesian Product} has been defined using
        ordered pairs\index{Ordered Pair} as defined by
        Kuratowski\index{Kazimierz, Kuratowski} and allows us to order two
        elements. Given two sets $A$ and $B$ we can define an equivalent notion
        using the set of all functions from $\mathbb{Z}_{2}=\{0,1\}$ into
        $A\cup{B}$ with a particular property:
        \begin{equation}
            A\times{B}=
            \big\{\,f:\mathbb{Z}_{2}\rightarrow{A}\cup{B}\;|\;
                f(0)\in{A}\textrm{ and }f(1)\in{B}\,\big\}
        \end{equation}
        To see why this is equivalent note that $A\times{B}$ is the set of all
        ordered pairs whose first entry lies in $A$ and whose second entry lies
        in $B$. Given $a\in{A}$ and $b\in{B}$, let
        $f:\mathbb{Z}_{2}\rightarrow{A}\cup{B}$ be the function such that
        $f(0)=a$ and $f(1)=b$. Then we can identify the ordered pair $(a,b)$
        with $f$. Indeed, $(a,b)=(f(0),f(1))$ making our identification very
        explicit. We can now generalize to a collection of $n$ different sets
        and define the ordered $n$ tuple over a collection of $n$ sets to be the
        set of all functions from $\mathbb{Z}_{n}$ into the union over this
        collection with a similar property:
        \begin{equation}
            \prod_{k\in\mathbb{Z}_{n}}A_{k}
            =\Big\{\,f:\mathbb{Z}_{n}\rightarrow\bigcup_{k\in\mathbb{Z}_{n}}^{n}
                A_{k}\;\big|\;f(k)\in{A}_{k}\textrm{ for all }k\in\mathbb{Z}_{n}
            \Big\}
        \end{equation}
        Given the function $f$ that maps $k$ to $a_{k}\in{A}_{k}$, we identify
        this by:
        \begin{equation}
            f=(a_{0},\,a_{1},\,\dots,\,a_{k},\,\dots,\,a_{n-1})
            =(f(0),\,f(1),\,\dots,\,f(k),\,\dots,\,f(n-1))
        \end{equation}
        And thus we have a more general way of defining products. Note that
        swapping the order of the product is equivalent to changing functions,
        so two $n$ tuples are equal if and only if all of their entries are
        equal. What's nice about our function definition is that it allows one
        to define products over \textit{arbitrary} collections. This is crucial
        for topology and analysis as we often wish to speak of
        \textit{infinite dimensional} spaces that are constructed using these
        abstract products. Given a set $I$, often called the \textit{index set},
        such that for all $\mathcal{U}\in{I}$ it is true $\mathcal{U}$ is a set,
        we can form the product over $I$ by defining this to be the collection
        of all functions from $I$ into the union over $I$.
        \begin{equation}
            \prod_{i\in{I}}A_{i}
            =\big\{\,f:I\rightarrow\bigcup_{i\in{I}}A_{i}\;|\;
                f(i)\in{A}_{i}\textrm{ for all }i\in{I}\,\big\}
        \end{equation}
        The axiom of choice\index{Axiom!of Choice} is equivalent to the
        statement \textit{The infinite product of non-empty sets is non-empty}.
        These functions that identify $k$ with the $k^{th}$ set are precisely
        choice functions. Phrasing it like this we see that the axiom of choice
        is somewhat obvious. The infinite product of non-empty sets is most
        likely enormous! Claiming it's non-empty seems trivial. It is then
        unfortunate that this claim can not be proven with the other axioms
        we've developed. As stated before, the axiom of choice is equivalent to
        many other statements such as \textit{Zorn's lemma, Tychonoff's theorem,
        the well ordering theorem, every vector space has a basis, every set has
        a group structure}, and countless more. Many of these theorems have many
        applications to algebra, analysis, and topology, and since we would like
        to use them to prove other things we are forced to accept the axiom of
        choice. Many theorems in real analysis hide the use of the axiom of
        choice by constructing sequences \textit{by induction}. An attempt will
        be made to be clear whenever the axiom of choice is used in a proof.
        \par\hfill\par
        We conclude this section by presenting Diaconescu's theorem.
        \begin{ftheorem}{Diaconescu's Theorem}{Diaconescus_Theorem}
            If $P$ is a proposition on sets and if $x$ is a set, then either
            $P(x)$ is true or the negation of $P(x)$ is true. That is,
            $P\lor\neg{P}$ is true.
            \index{Theorem!Diaconescu's}
        \end{ftheorem}
        \begin{bproof}
            Let $0=\emptyset$. By
            Thm.~\ref{thm:Existence_of_Set_Containing_Set}, we have that the set
            $\{0\}$ exists. Let $1=\{0\}$. Then since $0\in{1}$, $0\ne{1}$
            (Thm.~\ref{thm:Containment_NEqual_Underlying_Set}). Since $0$ and
            $1$ are sets, by Thm.~\ref{thm:Existence_of_Set_Built_from_Two_Sets}
            we have that the set $\{0,1\}$ exists. Let $Q$ be the proposition
            \textit{true if P(x) or } $x=0$, \textit{false otherwise}. By the
            axiom schema of specification
            (Ax.~\ref{ax:Axiom_Schema_of_Specification}) there exists a set
            $\mathcal{U}$ such that:
            \begin{equation}
                \mathcal{U}=\big\{\,x\in\{\,0,\,1\,\}\;|\;Q(x)\,\big\}
            \end{equation}
            Similarly, let $R$ be the proposition \textit{true if P(x) or}
            $x=1$, \textit{false otherwise}. By the axiom schema of
            specification we have that the following set exists:
            \begin{equation}
                \mathcal{V}=\big\{\,x\in\{\,0,\,1\,\}\;|\;R(x)\,\big\}
            \end{equation}
            By Thm.~\ref{thm:Existence_of_Set_Built_from_Two_Sets}, we have that
            the set $\{\,\mathcal{U},\,\mathcal{V}\,\}$ exists. By the axiom of
            choice (Ax.~\ref{ax:Axiom_of_Choice}), there exists a function
            $f:\{\mathcal{U},\mathcal{V}\}\rightarrow%
             \bigcup\{\mathcal{U},\mathcal{V}\}$ such that
            $f(\mathcal{U})\in\mathcal{U}$ and $f(\mathcal{V})\in\mathcal{V}$.
            But then, by the definition of $\mathcal{U}$, either
            $f(\mathcal{U})=0$ or $P(x)$ is true. Similarly, either
            $f(\mathcal{V})=1$ or $P(x)$ is true. But since $0\ne{1}$, either
            $f(\mathcal{U})\ne{f}(\mathcal{V})$ or $P(x)$ is true. Again by the
            axiom of extensionality (Ax.~\ref{ax:Axiom_of_Extensionality}), and
            by the definition of $\mathcal{U}$ and $\mathcal{V}$, if $P(x)$ is
            true then $\mathcal{U}=\mathcal{V}$. But then
            $f(\mathcal{U})=f(\mathcal{V})$. But then, by the contrapositive,
            $\neg{P}(x)$ implies that $f(\mathcal{U})\ne{f}(\mathcal{V})$. But
            by extensionality, either $f(\mathcal{U})=f(\mathcal{V})$ or
            $f(\mathcal{U})\ne{f}(\mathcal{V})$, and thus either $P(x)$ or
            $\neg{P}(x)$. That is, $P\lor\neg{P}$ is true.
        \end{bproof}
        We can now prove things via \textit{proof by contradiction}. While we
        have made great efforts to justify every step of a proof thus far, we
        will often omit mention of Diaconescu's theorem as a justification for
        the law of the excluded middle and simply use it freely. We may rest
        easy knowing that we've proved it's validity within the framework of
        ZFC. There are two more axioms remaining, that of infinity and
        replacement. The axiom of infinity is best introduced when we construct
        the natural numbers, and from there build the real numbers, and thus we
        shall delay its development briefly. The axiom of replacement needs a
        notion of class and thus will also be postponed.
        \section{The Structure of Sets}
    We've developed two \textit{operations}\index{Binary Operation} on sets thus
    far, that of union and intersection. Several of the properties of these two
    give rise to a structure known as a
    \textit{Boolean algebra}\index{Boolean Algebra}. Many theorems about sets
    can thus be proven in an algebraic setting by using the structure of a
    Boolean algebra, and thus it is our current goal to prove the basics about
    unions and operations so that we may transition to algebraic proofs.
    \subsection{Basic Theorems}
        With the law of the excluded middle in our toolbelt, we can now rapidly
        prove many basic and familiar results.
        \begin{theorem}
            \label{thm:Emptyset_Is_Subset}%
            If $A$ is a set, then $\emptyset\subseteq{A}$.
        \end{theorem}
        \begin{proof}
            For if not, then there is an $x\in\emptyset$ such that
            $x\notin{A}$ (Def.~\ref{def:Subsets}). But for all $x$ it is true
            that $x\notin\emptyset$ (Def.~\ref{def:Empty_Set}), a contradiction.
            Therefore $\emptyset\subseteq{A}$.
        \end{proof}
        This short theorem allows us to prove that the empty set is unique.
        \begin{theorem}
            \label{thm:Empty_Set_is_Unique}%
            If $\emptyset'$ is a set with no elements, then
            $\emptyset=\emptyset'$.
        \end{theorem}
        \begin{proof}
            For suppose not. But $\emptyset'$ is a set, and thus
            $\emptyset\subseteq\emptyset'$ (Thm.~\ref{thm:Emptyset_Is_Subset}).
            By the definition of equality if $\emptyset\ne\emptyset'$,
            then $\emptyset'\nsubseteq\emptyset$ (Def.~\ref{def:Equal_Sets}).
            But then there is an $x$ such that $x\in\emptyset'$ and
            $x\notin\emptyset$ (Def.~\ref{def:Subsets}). But by hypothesis
            $\emptyset'$ contains no elements, a contradiction. Thus
            $\emptyset'\subseteq\emptyset$. Therefore, $\emptyset=\emptyset'$
            (Def.~\ref{def:Equal_Sets}).
        \end{proof}
        We can use this to define \textit{the} empty set.
        \begin{fdefinition}{The Empty Set}{Empty_Set}
            The \gls{empty set} is the unique \gls{set} $\emptyset$ such that
            for all $x$ it is true that $x\notin\emptyset$.\index{Empty Set}
        \end{fdefinition}
        \begin{ltheorem}{Transitivity of Inclusion}{Transitivity_of_Inclusion}
            If $A$, $B$, and $C$ are sets, if $A\subseteq{B}$, and if
            $B\subseteq{C}$, then $A\subseteq{C}$.\index{Relation!Transitive}
            \index{Transitivity!of Inclusion}
        \end{ltheorem}
        \begin{proof}
            For suppose not. Then by the definition of subset there is an
            $x\in{A}$ such that $x\notin{C}$ (Def.~\ref{def:Subsets}). But $A$
            is a subset of $B$ and thus $x\in{B}$ (Def.~\ref{def:Subsets}).
            Similarly, $B$ is a subset of $C$ and therefore $x\in{C}$
            (Def.~\ref{def:Subsets}), a contradiction.
        \end{proof}
        Containment (\gls{containmentsymb}) is not transitive. That is, if
        $A\in{B}$ and $B\in{C}$, it may not be true that $A\in{C}$, nor is it
        necessarily true that $\{A\}\in{C}$, or even $\{A\}\subseteq{C}$. For a
        simple example, let $A=\emptyset$, $B=\{A,\{A\}\}$, and
        $C=\{B\}$. Then by definition $A\in{B}$ and $B\in{C}$, but $A\notin{C}$.
        This is because for all $x$ it is true that $x\in{C}$ if and only if
        $x=B$, and since $A\in{B}$ it is necessarily true that $A\ne{B}$
        (see Thm.~\ref{thm:Containment_NEqual_Underlying_Set}), and therefore
        $A\notin{C}$. Moreover, since $\{A\}\notin\{A\}$
        (Thm.~\ref{thm:Anti_Russells_Paradox}), we have that $\{A\}\ne{B}$.
        For $\{A\}\in{B}$ by definition and $\{A\}\notin\{A\}$, and thus by
        the axiom of extensionality (Ax.~\ref{ax:Axiom_of_Extensionality}) we
        may conclude that $\{A\}\ne{B}$. From this we may then conclude that
        $\{A\}\notin{C}$ and $\{A\}\nsubseteq{C}$.
        \par\hfill\par
        It is possible to find sets such that $A\in{B}$ and $B\in{C}$, and
        furthermore such that $A\in{C}$, we need only consider a nested chain
        of sets. Let $A=\emptyset$, $B=\{A,\{A\}\}$, and
        $C=\{A,B,\{B\}\}$. Then by construction, $A\in{B}$, $B\in{C}$, and
        $A\in{C}$. Such chains are used in the von Neumann construction of the
        integers.\index{von Neumann, John}
        \par\hfill\par
        With the transitivity of inclusion we can present a few brief
        corrolaries.
        \begin{theorem}
            \label{thm:Subsets_of_Equal_Sets}%
            If $A$, $B$, and $C$ are sets, if $A=B$, and if $C\subseteq{A}$,
            then $C\subseteq{B}$.
        \end{theorem}
        \begin{proof}
            For if $A=B$, then $A\subseteq{B}$ (Def.~\ref{def:Equal_Sets}). But
            by the transitivity of inclusion, if $C\subseteq{A}$ and
            $A\subseteq{B}$, then $C\subseteq{B}$
            (Thm.~\ref{thm:Transitivity_of_Inclusion}).
        \end{proof}
        \begin{theorem}
            \label{thm:Superset_of_Equal_Sets}%
            If $A,B$, and $C$ are sets, if $A=B$, and if $A\subseteq{C}$, then
            $B\subseteq{C}$.
        \end{theorem}
        \begin{proof}
            For if $A=B$ then $B\subseteq{A}$ (Def.~\ref{def:Equal_Sets}).
            But if $B\subseteq{A}$ and $A\subseteq{C}$, then by the transitivity
            of inclusion, $B\subseteq{C}$
            (Thm.~\ref{thm:Transitivity_of_Inclusion}).
        \end{proof}
        More than just being transitive, inclusion is also reflexive.
        \begin{ltheorem}{Reflexivity of Inclusion}{Reflexivity_of_Inclusion}
            If $A$ is a set, then $A\subseteq{A}$.
            \index{Relation!Reflexive}\index{Reflexivity!of Inclusion}
        \end{ltheorem}
        \begin{proof}
            For if not, then by the definition of subset there is an $x\in{A}$
            such that $x\notin{A}$ (Def.~\ref{def:Subsets}), a contradiction.
            Therefore $A\subseteq{A}$.
        \end{proof}
        The notion of inclusion $(A\subseteq{B})$ is also
        antisymmetric\index{Relation!Antisymmetric}. That is to say, if
        $A\subseteq{B}$ and $B\subseteq{A}$, then $A=B$. This is simply the
        definition of equality (Def.~\ref{def:Equal_Sets}). A relation that is
        transitive, reflexive, and antisymetric is known as a partial
        ordering\index{Partial Order} and will be very important in the study of
        order theory and topology. The partial ordering of inclusion on the
        power set of some given set is the quintessential example of a partial
        ordering. On the other hand, the notion of containment is not reflexive.
        That is, for any set $A$ it is true that $A\notin{A}$
        (Thm.~\ref{thm:Anti_Russells_Paradox}). Indeed, this was a desired
        characteristic of containment to allow us to avoid Russell's
        Paradox\index{Russell's Paradox}.
        \begin{ltheorem}{Reflexivity of Equality}{Equality_Reflexive}
            If $A$ is a set, then $A=A$.
            \index{Relation!Reflexive}\index{Reflexivity!of Equality}
        \end{ltheorem}
        \begin{proof}
            For if $A$ is a set then $A\subseteq{A}$
            (Thm.~\ref{thm:Reflexivity_of_Inclusion}). Thus, $A=A$
            (Def.~\ref{def:Equal_Sets}).
        \end{proof}
        \begin{ltheorem}{Symmetry of Equality}{Symmetry_of_Equality}
            If $A$ and $B$ are sets and if $A=B$, then $B=A$.
            \index{Relation!Symmetric}\index{Symmetry!of Equality}
        \end{ltheorem}
        \begin{proof}
            For suppose not. If $B\ne{A}$, then either $B\nsubseteq{A}$ or
            $A\nsubseteq{B}$. But $A=B$, and thus $A\subseteq{B}$  and
            $B\subseteq{A}$ (Def.~\ref{def:Equal_Sets}),
            a contradiction. Thus, $B=A$.
        \end{proof}
        An important but non-obvious statement is that containment
        (\gls{containmentsymb}) is \textit{not} symmetric, and indeed is the
        exact opposite, it is antisymetric. That is, for any two sets $A$ and
        $B$ it impossible for both $A\in{B}$ and $B\in{A}$, for this would
        violate the axiom of regularity\index{Axiom!of Regularity}
        (Ax.~\ref{ax:Axiom_of_Regularity}). We now prove this claim rigorously.
        \begin{ltheorem}{Antisymmetry of Containment}
                        {Antisymmetry_of_Containment}
            If $A$ and $B$ are sets, and if $A\in{B}$, then $B\notin{A}$.
            \index{Relation!Antisymmetric}\index{Antisymmetry!of Containment}
        \end{ltheorem}
        \begin{proof}
            For suppose not, and suppose $B\in{A}$. But by
            Thm.~\ref{thm:Existence_of_Set_Built_from_Two_Sets}, if $A$ and $B$
            are sets, then there is a set $\{A,B\}$ such that for all $x$ it is
            true that $x\in\{A,B\}$ if and only if $x=A$ or $x=B$. But then
            $\{A,B\}$ is a non-empty set, and thus by the axiom of regularity
            (Ax.~\ref{ax:Axiom_of_Regularity}) there is an $x\in\{A,B\}$ such
            that $x\cap\{A,B\}=\emptyset$. But by hypothesis, $A\in{B}$, and
            thus $A\in{B}\cap\{A,B\}$ (Def.~\ref{def:Intersection_of_Two_Sets}),
            and therefore $x\ne{B}$. But similarly, if $B\in{A}$, then
            $B\in{A}\cap\{A,B\}$ and thus $x\ne{B}$. But $x\in\{A,B\}$ if and
            only if $x=A$ or $x=B$, a contradiction. Therefore, $B\notin{A}$.
        \end{proof}
        An instant corrolary of this is that $\{A\}\notin{A}$.
        \begin{theorem}
            \label{thm:Set_Containing_A_is_not_Element_of_A}%
            If $A$ is a set, then $\{A\}\notin{A}$.
        \end{theorem}
        \begin{proof}
            For $A\in\{A\}$. But if $A\in\{A\}$, then $\{A\}\notin{A}$
            (Thm.~\ref{thm:Antisymmetry_of_Containment}).
        \end{proof}
        We now continue developing the structure of equality.
        \begin{ltheorem}{Transitivity of Equality}{Transitivity_of_Equality}
            If $A$, $B$, and $C$ are sets, if $A=B$, and if $B=C$, then $A=C$.
        \end{ltheorem}
        \begin{proof}
            For if $B=C$, then $C\subseteq{B}$ (Def.~\ref{def:Equal_Sets}). But
            if $A=B$, then $B=A$ (Thm.~\ref{thm:Reflexivity_of_Inclusion}). But
            if $B=A$ and $C\subseteq{B}$, then $C\subseteq{A}$
            (Thm.~\ref{thm:Subsets_of_Equal_Sets}). And if $A=B$, then
            $A\subseteq{B}$ (Def.~\ref{def:Equal_Sets}). But if $B=C$ and
            $A\subseteq{B}$, then $A\subseteq{C}$
            (Thm.~\ref{thm:Subsets_of_Equal_Sets}). But it was proved that
            $C\subseteq{A}$, and thus $A=C$ (Def.~\ref{def:Equal_Sets}).
        \end{proof}
        The three properties we've proved thus far, that of reflexivity
        (Thm.~\ref{thm:Equality_Reflexive}), symmetry
        (Thm.~\ref{thm:Symmetry_of_Equality}), and transitivity
        (Thm.~\ref{thm:Transitivity_of_Equality}) are the key ingredients to
        defining \textit{equivalence relations}\index{Equivalence Relation}.
        Equivalence relations are used to model the notion of equality in more
        abstract settings and are fundamental in the study of algebra and
        topology. We'll discuss these more in
        \S~\ref{Section:ZFC:Elementary_Set_Theory:Relations}.
        \begin{theorem}
            \label{thm:Prop_Subset_Not_Equal}%
            If $A$ and $B$ are sets, and if $A\subsetneq{B}$, then there is an
            $x\in{B}$ such that $x\notin{A}$.
        \end{theorem}
        \begin{proof}
            For suppose not. Then for all $x\in{B}$ it is true that $x\in{A}$.
            But then $B\subseteq{A}$ (Def.~\ref{def:Subsets}).
            But $A\subseteq{B}$ and thus $A=B$ (Def.~\ref{def:Equal_Sets}).
            But $A\subsetneq{B}$ and therefore $A\ne{B}$, a contradiction.
        \end{proof}
        Theorem \ref{thm:Prop_Subset_Not_Equal} can be used as an equivalent
        definition of a proper subset. That is, a proper subset is a subset that
        is missing at least one element.
    \subsection{Operations on Sets}
        Similar to the arithmetic of real numbers, there are standard operations
        that can be performed on sets to obtain new sets. The four most common
        operations are union, intersection, set difference, and symmetric
        difference. As stated before, we wish to build the structure of sets in
        an algebraic manner. To do this requires the notion that the operations
        of intersection and unions are \textit{commutative},
        \textit{distributive}, have \textit{identities}, and have
        \textit{complements}.
        \begin{ltheorem}{Commutative Law of Unions}{Commutative_Law_of_Unions}
            If $A$ and $B$ are sets, then $A\cup{B}=B\cup{A}$.
        \end{ltheorem}
        \begin{proof}
            For if $x\in{A}\cup{B}$, then either $x\in{A}$ or $x\in{B}$, or both
            (Def.~\ref{def:Union_of_Two_Sets}). But then either $x\in{B}$ or
            $x\in{A}$, or both, and therefore $x\in{B}\cup{A}$
            (Def.~\ref{def:Union_of_Two_Sets}). But then for all
            $x\in{A}\cup{B}$ it is true that $x\in{B}\cup{A}$, and therefore
            $A\cup{B}\subseteq{B}\cup{A}$ (Def.~\ref{def:Subsets}). Similarly,
            $B\cup{A}\subseteq{A}\cup{B}$, and thus
            $A\cup{B}=B\cup{A}$ (Def.~\ref{def:Equal_Sets}).
        \end{proof}
        When taking the union of two sets, we obtain a \textit{larger} set, in
        a sense. Again relying on the analogy of arithmetic, given two
        non-negative integers $a$ and $b$, it is true that $a\leq{a}+b$.
        Equality is obtained if and only if either $b$ is equal to zero. The
        empty set thus acts as the \textit{zero} of unions. Also, given three
        non-negative integers $a$, $b$, and $c$, if $b\leq{c}$, then
        $a+b\leq{a}+c$. A similar result will hold for unions.
        \begin{theorem}
            \label{thm:Union_is_Bigger}%
            If $A$ and $B$ are sets, then $A\subseteq{A}\cup{B}$.
        \end{theorem}
        \begin{proof}
            For suppose not. Then there is an $x\in{A}$ such that
            $x\notin{A}\cup{B}$. But if $x\in{A}$, then $x\in{A}$ or $x\in{B}$
            and thus $x\in{A}\cup{B}$ (Def.~\ref{def:Union_of_Two_Sets}), a
            contradiction.
        \end{proof}
        \begin{theorem}
            \label{thm:Union_With_Lesser_Set_on_Right}%
            If $A$, $B$, and $C$ are sets, and if $B\subseteq{C}$, then
            $A\cup{B}\subseteq{A}\cup{C}$.
        \end{theorem}
        \begin{proof}
            For if $x\in{A}\cup{B}$, then either $x\in{A}$, or $x\in{B}$, or
            both (Def.~\ref{def:Union_of_Two_Sets}). But $B$ is a subset of $C$,
            and therefore if $x\in{B}$, then $x\in{C}$ (Def.~\ref{def:Subsets}).
            Thus, if $x\in{A}$ or $x\in{B}$, then $x\in{A}$ or $x\in{C}$, and
            therefore $x\in{A}\cup{C}$ (Def.~\ref{def:Union_of_Two_Sets}).
            Thus, $A\cup{B}\subseteq{A}\cup{C}$ (Def.~\ref{def:Subsets}).
        \end{proof}
        \begin{theorem}
            \label{thm:Union_With_Lesser_Set_on_Left}%
            If $A$, $B$, and $C$ are sets, and if $B\subseteq{C}$, then
            $B\cup{A}\subseteq{C}\cup{A}$.
        \end{theorem}
        \begin{proof}
            For $B\cup{A}=A\cup{B}$ (Thm.~\ref{thm:Commutative_Law_of_Unions}).
            But if $B\subseteq{C}$, then $A\cup{B}\subseteq{A}\cup{C}$
            (Thm.~\ref{thm:Union_With_Lesser_Set_on_Right}). And if
            $B\cup{A}=A\cup{B}$ and $A\cup{B}\subseteq{A}\cup{C}$, then
            $B\cup{A}\subseteq{A}\cup{C}$
            (Thm.~\ref{thm:Superset_of_Equal_Sets}). But $A\cup{C}=C\cup{A}$
            (Thm.\ref{thm:Commutative_Law_of_Unions}) and if $A\cup{C}=C\cup{A}$
            and $B\cup{A}\subseteq{A}\cup{C}$, then
            $B\cup{A}\subseteq{C}\cup{A}$
            (Thm.~\ref{thm:Subsets_of_Equal_Sets}).
        \end{proof}
        \begin{theorem}
            If $A$, $B$, $C$, and $D$ are sets, if $A\subseteq{C}$, and if
            $B\subseteq{D}$, then $A\cup{B}\subseteq{C}\cup{D}$.
        \end{theorem}
        \begin{proof}
            For if $B\subseteq{D}$, then $A\cup{B}\subseteq{A}\cup{D}$
            (Thm.~\ref{thm:Union_With_Lesser_Set_on_Right}). But if
            $A\subseteq{C}$, then $A\cup{D}\subseteq{C}\cup{D}$
            (Thm.~\ref{thm:Union_With_Lesser_Set_on_Left}), obtaining the
            result.
        \end{proof}
        Taking the union of subsets is redundant, as we simply obtain the larger
        set. This starts to break down the analogy between sets and arithmetic,
        since there is only one \textit{zero}. That is, there is only one number
        $b$ such that $a+b=a$, and that is $b=0$. While any subset acts as a
        \textit{zero} of a given set, the empty set has the property that it
        acts as a zero for \textit{every} set. It is the only set with this
        property, and thus the analogy with arithmetic is slightly restored.
        \begin{theorem}
            \label{thm:Union_With_Subset}%
            If $A$ and $B$ are sets, and if $A\subseteq{B}$, then $A\cup{B}=B$.
        \end{theorem}
        \begin{proof}
            For if $A$ and $B$ are sets, then $B\subseteq{A}\cup{B}$
            (Thm.~\ref{thm:Union_is_Bigger}). But if $A\subseteq{B}$, then for
            all $x\in{A}$ it is true that $x\in{B}$ (Def.~\ref{def:Subsets}).
            Thus if $x\in{A}$ or if $x\in{B}$, then $x\in{B}$. But then, for all
            $x\in{A}\cup{B}$, it is true that $x\in{B}$, and therefore
            $A\cup{B}\subseteq{B}$ (Def.~\ref{def:Subsets}). Thus, $A\cup{B}=B$
            (Def.~\ref{def:Equal_Sets}).
        \end{proof}
        \begin{theorem}
            \label{thm:Union_with_Emptyset}%
            If $A$ is a set, then $A=\emptyset\cup{A}$.
        \end{theorem}
        \begin{proof}
            For $\emptyset\subseteq{A}$ (Thm.~\ref{thm:Emptyset_Is_Subset}) and
            therefore $\emptyset\cup{A}=A$ (Thm.~\ref{thm:Union_With_Subset}).
        \end{proof}
        \begin{theorem}
            \label{thm:Empty_Set_Is_Zero_for_Unions}%
            If $A$ is a set such that, for any set $B$ it is true that
            $A\cup{B}=B$, then $A$ is the empty set.
        \end{theorem}
        \begin{proof}
            For suppose not. If $A\ne\emptyset$, then there is an $x\in{A}$
            (Def.~\ref{def:Empty_Set}). But if $A$ is a set, then $B=\{A\}$ is a
            set (Thm.~\ref{thm:Existence_of_Set_Containing_Set}). But then
            $x\in{A}\cup{B}$ (Def.~\ref{def:Union_of_Two_Sets}). But
            $x\notin{B}$, and thus $A\cup{B}\ne{B}$
            (Def.~\ref{def:Equal_Sets}), a contradiction
            since $A$ is such that for any set $B$, it is
            true that $A\cup{B}=B$. Thus, $A$ is the empty set.
        \end{proof}
        Thm.~\ref{thm:Empty_Set_Is_Zero_for_Unions} proves the assertion that
        the empty set is the zero of set union. The converse of
        Thm.~\ref{thm:Union_With_Subset} can be proved as well.
        \begin{theorem}
            \label{thm:Conv_Union_Is_Bigger}%
            If $A$ and $B$ are sets, and if $A\cup{B}\subseteq{A}$, then
            $A\cup{B}=A$.
        \end{theorem}
        \begin{proof}
            For $A\subseteq{A}\cup{B}$ (Thm.~\ref{thm:Union_is_Bigger}). But by
            hypothesis, $A\cup{B}\subseteq{A}$. But then $A=A\cup{B}$
            (Def.~\ref{def:Equal_Sets}).
        \end{proof}
        \begin{theorem}
            \label{thm:Union_is_Equal}%
            If $A$ and $B$ are sets, and if $A\cup{B}\subseteq{A}$, then
            $B\subseteq{A}$.
        \end{theorem}
        \begin{proof}
            For if $A\cup{B}\subseteq{A}$, then $A\cup{B}=A$
            (Thm.~\ref{thm:Conv_Union_Is_Bigger}). And also,
            $B\subseteq{A}\cup{B}$ (Thm.~\ref{thm:Union_is_Bigger}). But if
            $A\cup{B}=A$ and $B\subseteq{A}\cup{B}$, then $B\subseteq{A}$
            (Thm.~\ref{thm:Subsets_of_Equal_Sets}).
        \end{proof}
        Logically, and pedagogically, it would seem appropriate to demonstrate
        the fact that union is an \textit{associative} operation on sets. We
        will instead hold off on this and prove it in a purely algebraic setting
        once we've developed Boolean algebras. Boolean algebras consist of two
        operations that are commutative, distributive, contain identities and
        complements. For the case of sets we will use union and intersection as
        our operations, and the power set of some given set as the set which
        these operations act on. First, we'll state associativity of unions:
        \begin{equation}
            A\cup(B\cup{C})=(A\cup{B})\cup{C}
        \end{equation}
        This becomes intuitively clear when one examines the Venn diagram
        shown in Fig.~\ref{fig:Union_of_Three_Sets}. Using the analogy of
        arithmetic, given three real numbers $a$, $b$, and $c$, it is true that
        $a+(b+c)=(a+b)+c$. This is called the associative law of addition.
        Combining this law with the commutative law shows that the order in
        which three real numbers are added is irrelevant. Applying induction, we
        see that given any finite collection of real numbers, the order in
        which we add them is again irrelevant. The same holds true for the union
        of sets. We now move on to intersections, and return to the associative
        law in Chapter~\ref{chapt:Function_Theory},
        (Section~\ref{sec:Boolean_Algebra}).
        \begin{ltheorem}{Commutative Law of Intersections}{Commut_Law_Intersec}
            If $A$ and $B$ are sets, then $A\cap{B}=B\cap{A}$.
        \end{ltheorem}
        \begin{proof}
            For if $x\in{A}\cap{B}$, then $x\in{A}$ and
            $x\in{B}$. But then $x\in{B}$ and $x\in{A}$,
            and therefore $x\in{B}\cap{A}$
            (Def.~\ref{def:Intersection_of_Two_Sets}). But then
            for all $x\in{A}\cap{B}$ it is true that
            $x\in{B}\cap{A}$, and therefore
            $A\cup{B}\subseteq{B}\cup{A}$
            (Def.~\ref{def:Subsets}). Similarly,
            $B\cap{A}\subseteq{A}\cap{B}$, and thus
            $A\cap{B}=B\cap{A}$ (Def.~\ref{def:Equal_Sets}).
            Therefore, etc.
        \end{proof}
        \begin{theorem}
            \label{thm:Intersection_is_Smaller}%
            If $A$ snd $B$ are sets, then
            $A\cap{B}\subseteq{A}$.
        \end{theorem}
        \begin{proof}
            If $x\in{A}\cap{B}$, then $x\in{A}$ and
            $x\in{B}$, and thus $x\in{A}$. Therefore, etc.
        \end{proof}
        \begin{theorem}
            \label{thm:Intersection_with_Lesser_Set}%
            If $A$, $B$, and $C$ are sets, and if
            $B\subseteq{C}$, then
            $A\cap{B}\subseteq{A}\cap{C}$.
        \end{theorem}
        \begin{proof}
            For if $x\in{A}\cap{B}$, then $x\in{A}$ and
            $x\in{B}$ (Def.~\ref{def:Intersection_of_Two_Sets}).
            But $B$ is a subset of $C$, and thus if
            $x\in{B}$, then $x\in{C}$
            (Def.~\ref{def:Subsets}). But then $x\in{A}$ and
            $x\in{C}$, and therefore $x\in{A}\cap{C}$
            (Def.~\ref{def:Intersection_of_Two_Sets}). But
            then $A\cap{B}\subseteq{A}\cap{C}$
            (Def.~\ref{def:Subsets}). Therefore, etc.
        \end{proof}
        \begin{figure}[H]
            \centering
            \captionsetup{type=figure}
            \centering
            \documentclass[crop,class=article]{standalone}
%----------------------------Preamble-------------------------------%
\usepackage{tikz}                       % Drawing/graphing tools.
%--------------------------Main Document----------------------------%
\begin{document}
    \begin{tikzpicture}
        \draw (-3,-2.3) rectangle (2.5,2.3);
        \draw (0,0) circle (2);
        \draw[fill=cyan] (0.5,-0.3) circle (0.85);
        \node at (0.5,-0.1) {$A$};
        \node at (0,1.3) {$B$};
        \node at (-2.3,1.5) {$A\cap{B}$};
    \end{tikzpicture}
\end{document}
            \caption{Visual for Thm.~\ref{thm:Intersection_of_Subset}.}
            \label{fig:Union_Intersection_venn_diagram}
        \end{figure}
        \begin{theorem}
            \label{thm:Intersection_is_Equal}%
            If $A$ and $B$ are sets, and if
            $A=A\cap{B}$, then $A\subseteq{B}$.
        \end{theorem}
        \begin{proof}
            For suppose not. Then there is an $x\in{A}$ such
            that $x\notin{B}$. But since $A=A\cap{B}$,
            if $x\in{A}$ then $x\in{A}\cap{B}$
            (Def.~\ref{def:Equal_Sets}). But if
            $x\in{A}\cap{B}$, then $x\in{B}$
            (Thm.~\ref{thm:Intersection_is_Smaller}),
            a contradiction. Therefore, etc.
        \end{proof}
        \begin{theorem}
            \label{thm:Intersection_of_Subset}%
            If $A$ and $B$ are sets, and if
            $A\subseteq{B}$, then $A\cap{B}=A$.
        \end{theorem}
        \begin{proof}
            For $A\cap{B}\subseteq{A}$
            (Thm.~\ref{thm:Intersection_is_Smaller}). But
            since $A$ is a subset of $B$, if $x\in{A}$, then
            $x\in{B}$ (Def.~\ref{def:Subsets}). But then
            $x\in{A}\cap{B}$
            (Def.~\ref{def:Intersection_of_Two_Sets}). Therefore,
            $A\subseteq{A}\cap{B}$ (Def~\ref{def:Subsets})
            and thus $A=A\cap{B}$ (Def~\ref{def:Equal_Sets}).
            Therefore, etc.
        \end{proof}
        \begin{theorem}
            \label{thm:Conv_Intersection_is_Smaller}%
            If $A$ and $B$ are sets, and if
            $A\subseteq{A}\cap{B}$, then $A=A\cap{B}$.
        \end{theorem}
        \begin{proof}
            For $A\cap{B}\subseteq{A}$
            (Thm.~\ref{thm:Intersection_is_Smaller}). But
            by hypothesis, $A\subseteq{A}\cap{B}$, and thus
            $A=A\cap{B}$ (Def.~\ref{def:Equal_Sets}).
            Therefore, etc.
        \end{proof}
        \begin{theorem}
            If $A$ is a set, then $\emptyset\cap{A}=\emptyset$.
        \end{theorem}
        \begin{proof}
            For $\emptyset\subseteq{A}$ (Thm.~\ref{thm:Emptyset_Is_Subset}), and
            therefore $\emptyset\cap{A}=\emptyset$
            (Thm.~\ref{thm:Intersection_of_Subset}).
        \end{proof}
        % \begin{theorem}
        %     \label{thm:Redundant_Intersection}%
        %     If $A$, $B$, and $C$ are sets and if
        %     $B\subseteq{A}$, then
        %     $A\cap(B\cap{C})=B\cap{C}$.
        % \end{theorem}
        % \begin{proof}
        %     For $A\cup(B\cup{C})\subseteq{B}\cup{C}$
        %     (Thm.~\ref{thm:Intersection_is_Smaller}). But
        %     $A\cap(B\cap{C})=(A\cap{B})\cap{C}$
        %     (Thm.~\ref{thm:Assoc_Law_Intersec}).
        %     And since $B$ is a subset of $A$,
        %     $A\cap{B}=A$
        %     (Thm.~\ref{thm:Intersection_of_Subset}),
        %     and thus $(A\cap{B})\cap{C}=B\cap{C}$. Thus,
        %     $B\cap{C}=A\cap(B\cap{C})$
        %     (Thm.~\ref{thm:Transitivity_of_Equality}).
        %     Therefore, etc.
        % \end{proof}
        \begin{theorem}
            \label{thm:First_Pseudo_Dist_Law_Union}%
            If $A$, $B$, and $C$ are sets, then
            $(B\cap{C})\subseteq(A\cup{B})\cap(A\cup{C})$.
        \end{theorem}
        \begin{proof}
            For $B\subseteq{A}\cup{B}$
            (Thm.~\ref{thm:Union_is_Bigger}). But then
            $B\cap{C}\subseteq(A\cup{B})\cap{C}$
            (Thm.~\ref{thm:Intersection_with_Lesser_Set}).
            But $C\subseteq{A}\cup{C}$
            (Thm.~\ref{thm:Union_is_Bigger}), and thus
            $(A\cup{B})\cap{C}%
            \subseteq(A\cup{B})\cap{A}\cup{C}$
            (Thm.~\ref{thm:Intersection_with_Lesser_Set}).
            But it was just proved that
            $B\cap{C}\subseteq(A\cup{B})\cap{C}$, and
            therefore by transivity,
            $(B\cap{C})\subseteq(A\cup{B})\cap(A\cup{C})$
            (Thm.~\ref{thm:Transitivity_of_Inclusion}).
            Therefore, etc.
        \end{proof}
        \begin{ftheorem}{Distributive Law of Unions}
            {Distributive_Law_Union}
            If $A$, $B$, and $C$ are sets, then:
            \begin{equation*}
                A\cup(B\cap{C})=(A\cup{B})\cap(A\cup{C})
            \end{equation*}
        \end{ftheorem}
        \begin{bproof}
            For $(B\cap{C})\subseteq(A\cup{B})\cap(A\cup{C})$
            (Thm.~\ref{thm:First_Pseudo_Dist_Law_Union}).
            But then:
            \begin{equation}
                A\cup(B\cap{C})\subseteq
                A\cup\Big((A\cup{B})\cap(A\cup{C})\Big)
            \end{equation}
            But $A\cup((A\cup{B})\cap(A\cup{C}))%
                =(A\cup{B})\cap(A\cup{C})$, and therefore:
            \begin{equation}
                A\cup(B\cap{C})\subseteq
                (A\cup{B})\cap(A\cup{C})
            \end{equation}
        \end{bproof}
        We can represent this pictorially using Venn diagrams.
        \begin{figure}[H]
            \centering
            \captionsetup{type=figure}
            \begin{tikzpicture}
    % Coordinates for the points.
    \coordinate (A) at (-2.6, 0.0);
    \coordinate (B) at ( 0.0, 0.0);
    \coordinate (C) at ( 2.6, 0.0);

    % Fill in the circle A.
    \draw[fill=cyan, draw=none] (A) circle (2);

    % Fill in the overlap between B and C.
    \draw[fill=cyan] (1.3, -1.51987) arc(-49.46:49.46:2) arc(130.54:229.46:2);

    % Draw circles outlining the regions A, B, and C.
    \draw[draw=black] (A) circle (2);
    \draw[draw=black] (B) circle (2);
    \draw[draw=black] (C) circle (2);

    % Add some labels.
    \node at (A) [above=1cm] {$A$};
    \node at (B) [above=1cm] {$B$};
    \node at (C) [above=1cm] {$C$};
\end{tikzpicture}
            \caption{Venn Diagram for the Distributive Law of Unions}
            \label{fig:Venn_Diagram_Distributive_Law_of_Union}
        \end{figure}
        \begin{ftheorem}{Distributive Law of Intersections}
            {Distributive_Law_Intersections}
            If $A$, $B$, and $C$ are sets, then:
            \begin{equation*}
                A\cap(B\cup{C})=(A\cap{B})\cup(A\cap{C})
            \end{equation*}
        \end{ftheorem}
        \begin{bproof}
            For by the distributive law of unions
            (Thm.~\ref{thm:Distributive_Law_Union}), we have:
            \begin{equation}
                    (A\cap{B})\cup(A\cap{C})
                    =\Big((A\cap{B})\cup{A}\Big)\cap\Big((A\cap{B})\cup{C}\Big)
            \end{equation}
            But $A\cap{B}\subseteq{A}$ (Thm.~\ref{thm:Intersection_is_Smaller})
            and thus $(A\cap{B})\cup{A}=A$ (Thm.~\ref{thm:Union_With_Subset}).
            Thus:
            \begin{equation}
                \Big((A\cap{B})\cup{A}\Big)\cap\Big((A\cap{B})\cup{C}\Big)
                =A\cap\Big((A\cap{B})\cup{C}\Big)
            \end{equation}
            But by the commutative law of unions
            (Thm.~\ref{thm:Commutative_Law_of_Unions}),
            $(A\cap{B})\cup{C}=C\cup(A\cap{B})$. Applying the distributive law
            again, we obtain:
            \begin{equation}
                A\cap\Big(C\cup(A\cap{B})\Big)
                =A\cap\Big((C\cup{A})\cap(C\cup{B})\Big)
            \end{equation}
            But by the associative law of intersections
            (Thm.~\ref{thm:Associative_Law_of_Intersections}) we have:
            \begin{equation}
                A\cap\Big((C\cup{A})\cap(C\cup{B})\Big)
                =\Big(A\cap(C\cup{A})\Big)\cap(C\cup{B})
            \end{equation}
            But $A\subseteq{C}\cup{A}$ (Thm.~\ref{thm:Union_is_Bigger}), and
            thus $A\cap(C\cup{A})=A$
            (Thm.~\ref{thm:Intersection_with_Lesser_Set}). But then:
            \begin{equation}
                \Big(A\cap(C\cup{A})\Big)\cap(C\cup{B})
                =A\cap(C\cup{B})
            \end{equation}
            By the transitivity of equality
            (Thm.~\ref{thm:Transitivity_of_Equality}) we conclude the result.
        \end{bproof}
        \begin{figure}[H]
            \centering
            \captionsetup{type=figure}
            \begin{tikzpicture}
                % Coordinates for the points.
                \coordinate (B) at (-2.6, 0.0);
                \coordinate (A) at ( 0.0, 0.0);
                \coordinate (C) at ( 2.6, 0.0);
            
                % Fill in the overlap between B and A.
                \draw[fill=cyan] (-1.3, -1.51987) arc(-49.46:49.46:2) arc(130.54:229.46:2);
            
                % Fill in the overlap between A and C.
                \draw[fill=cyan] (1.3, -1.51987) arc(-49.46:49.46:2) arc(130.54:229.46:2);
            
                % Draw circles outlining the regions A, B, and C.
                \draw[draw=black] (A) circle (2);
                \draw[draw=black] (B) circle (2);
                \draw[draw=black] (C) circle (2);
            
                % Add some labels.
                \node at (A) [above=1cm] {$A$};
                \node at (B) [above=1cm] {$B$};
                \node at (C) [above=1cm] {$C$};
            \end{tikzpicture}
            \caption{Venn Diagram for the Distributive Law of Intersections}
            \label{fig:Venn_Diagram_Distributive_Law_of_Intersections}
        \end{figure}
        Continuing with this analogy, we discuss set difference.
        \begin{theorem}
            \label{thm:Existence_of_Set_Difference}%
            If $A$ and $B$ are sets, then there is a set $C$ such that for all
            $x$ it is true that $x\in{C}$ if and only if $x\in{A}$ and
            $x\notin{B}$.
        \end{theorem}
        \begin{proof}
            For let $P$ be the proposition \textit{true if} $x\notin{B}$,
            \textit{false otherwise}. Then by the axiom schema of specification
            (Ax.~\ref{ax:Axiom_Schema_of_Specification}), there is a set $C$
            such that:
            \begin{equation}
                C=\{\,x\in{A}\;|\;P(x)\,\}
            \end{equation}
            But then $x\in{C}$ if and only if $x\in{A}$ and $P(x)$ is true.
            But $P(x)$ is true if and only if $x\notin{B}$. Thus $x\in{C}$ if
            and only if $x\in{A}$ and $x\notin{B}$.
        \end{proof}
        The set described in Thm.~\ref{thm:Existence_of_Set_Difference} is known
        as the set difference of $B$ with respect to $A$. We take a moment to
        discuss some of it's properties.
        \begin{fdefinition}{Set Difference}{Set_Difference}
            The set difference of a set $B$ with respect to a set $A$ is the
            set:
            \begin{equation*}
                A\setminus{B}=\{\,x\in{A}\;|\;x\notin{B}\,\}
            \end{equation*}
        \end{fdefinition}
        \begin{example}
            Let $\mathbb{Z}$ denote the integers, and $\mathbb{N}$ denote the
            set of natural numbers. Then $\mathbb{Z}\setminus\mathbb{N}$ is the
            set of all negative integers. Flipping this, we see that
            $\mathbb{N}\setminus\mathbb{Z}$ is the empty set since there are no
            non-negative integers that are not also integers.
        \end{example}
        \begin{example}
            Letting $\mathbb{R}$ denote the real numbers and $\mathbb{Q}$ denote
            the rationals, $\mathbb{R}\setminus\mathbb{Q}$ is the set of all
            \textit{irrational} numbers. Famous examples include $\sqrt{2}$,
            $\pi$, and $e$ (sometimes known as Euler's constant).
        \end{example}
        \begin{example}
            If we let $A$ and $B$ be defined as:
            \par
            \begin{subequations}
                \begin{minipage}[b]{0.49\textwidth}
                    \centering
                    \begin{equation}
                        A=\{\,a,\,b,\,c\,\}
                    \end{equation}
                \end{minipage}
                \hfill
                \begin{minipage}[b]{0.49\textwidth}
                    \centering
                    \begin{equation}
                        B=\{\,b,\,c,\,d\,\}
                    \end{equation}
                \end{minipage}
            \end{subequations}
            \par\vspace{2.5ex}
            Then we can compute directly the set difference between the two:
            \par
            \begin{subequations}
                \begin{minipage}[b]{0.49\textwidth}
                    \centering
                    \begin{equation}
                        A\setminus{B}=\{\,a\,\}
                    \end{equation}
                \end{minipage}
                \hfill
                \begin{minipage}[b]{0.49\textwidth}
                    \centering
                    \begin{equation}
                        B\setminus{A}=\{\,d\,\}
                    \end{equation}
                \end{minipage}
            \end{subequations}
            \par\vspace{2.5ex}
            Thus the symmetric difference between sets is not a
            \textit{commutative} set operation.
        \end{example}
        \begin{figure}[H]
            \centering
            \captionsetup{type=figure}
            %--------------------------------Dependencies----------------------------------%
%   tikz                                                                       %
%-------------------------------Main Document----------------------------------%
\begin{tikzpicture}[line width=0.2mm]

    % Coordinates for the centers of the circles.
    \coordinate (C1) at (-1.3, 0);
    \coordinate (C2) at ( 1.3, 0);

    % Coordinates for the labels.
    \coordinate (A) at (-1.3, 1.2);
    \coordinate (B) at ( 1.3, 1.2);
    \coordinate (U) at ( 0.0, 2.5);

    % Rectangle indicating the universe set.
    \draw (-4, -3) rectangle (4, 3);

    % Draw the two circles.
    \draw[fill=cyan]              (C1) circle (2);
    \draw[fill=white, draw=black] (C2) circle (2);

    % Add an outline to the left circle.
    \draw (C1) circle (2);

    % Labels.
    \node at (A) {$A$};
    \node at (B) {$B$};
    \node at (U) {$A\setminus{B}$};
\end{tikzpicture}
            \caption{Venn Diagram for Set Difference}
            \label{fig:Set_Diff_Venn_Diagram}
        \end{figure} 
        \begin{theorem}
            \label{thm:Set_Difference_of_Set_with_Self}%
            If $A$ is a set, then $A\setminus{A}=\emptyset$.
        \end{theorem}
        \begin{proof}
            For suppose not. But since $A\setminus{A}$ is a set,
            $\emptyset\subseteq{A}\setminus{A}
            $(Thm.~\ref{thm:Emptyset_Is_Subset}),and thus if
            $A\setminus{A}\ne\emptyset$ then $A\setminus{A}\nsubseteq\emptyset$
            (Def.~\ref{def:Equal_Sets}). But then there is an
            $x\in{A}\setminus{A}$ such that $x\notin\emptyset$. But if
            $x\in{A}\setminus{A}$, then $x\in{A}$ and $x\notin{A}$
            (Def.~\ref{def:Set_Difference}), a contradiction. Thus,
            $A\setminus{A}=\emptyset$.
        \end{proof}
        \begin{theorem}
            \label{thm:Set_Difference_is_Subset}%
            If $A$ and $B$ are sets, then $A\setminus{B}\subseteq{A}$.
        \end{theorem}
        \begin{proof}
            For suppose not. Then there is an $x\in{A}\setminus{B}$ such that
            $x\notin{A}$. But $x\in{A}\setminus{B}$ if and only if $x\in{A}$ and
            $x\notin{B}$ (Def.~\ref{def:Set_Difference}), and thus $x\in{A}$, a
            contradiction.
        \end{proof}
        \begin{theorem}
            \label{thm:Set_Difference_of_Set_and_Empty}%
            If $A$ is a set, then $A\setminus\emptyset=A$.
        \end{theorem}
        \begin{proof}
            For since $A$ and $\emptyset$ are sets,
            $A\setminus\emptyset\subseteq{A}$
            (Thm.~\ref{thm:Set_Difference_is_Subset}). Suppose they are not
            equal. Then there is an $x\in{A}$ such that
            $x\notin{A}\setminus\emptyset$ (Def.~\ref{def:Equal_Sets}). But if
            $x\in{A}$ and $x\notin{A}\setminus\emptyset$, then $x\in\emptyset$
            (Def.~\ref{def:Set_Difference}) a contradiction since for all $x$ it
            is true that $x\notin\emptyset$ (Def.~\ref{def:Empty_Set}). Thus,
            they are equal.
        \end{proof}
        \begin{example}
            More than being a non-commutative operation, set difference is not
            \textit{associative} either. For let $A$ be any non-empty set and
            let $A=B=C$. Then:
            \begin{equation}
                A\setminus(A\setminus{A})=A\setminus\emptyset=A
            \end{equation}
            Flipping this around, we have:
            \begin{equation}
                (A\setminus{A})\setminus{A}=\emptyset\setminus{A}=\emptyset
            \end{equation}
            But since $A$ is a non-empty set, $A\ne\emptyset$, thus showing
            that set difference is not associative.
        \end{example}
        \begin{ldefinition}{Symmetric Difference}{Symmetric_Difference}
            The symmetric difference of $A$ and $B$, denoted $A\ominus{B}$, is
            the set:
            \begin{equation}
                A\ominus{B}
                =(A\cup{B})\setminus(A\cap{B})
            \end{equation}
        \end{ldefinition}
        While set difference appears similar to subtraction, the two have their
        differences. For any two real numbers $a$ and $b$, it is always true
        that $b=a-(a-b)$. For sets this is not true. For let $A$ be the empty
        set, and let $B$ be non-empty. Then
        $A\setminus(A\setminus{B})=\emptyset$, which is not $B$. Set differences
        can not be easily simplified. The notion is not associative, nor is it
        commutative. If there is a larger \textit{universe} set, then set
        difference can be related to intersection.
        \begin{theorem}
            \label{thm:Set_Difference_As_Intersection}%
            If $A$, $B$, and $C$ are sets, and if $A\subseteq{C}$ and
            $B\subseteq{C}$, then:
            \begin{equation}
                B\setminus{A}=B\cap(C\setminus{A})
            \end{equation}
        \end{theorem}
        \begin{proof}
            For if $x\in{B}\setminus{A}$, then
            $x\in{B}$ and $x\notin{A}$. But
            $B\subseteq{C}$, and thus if $x\in{B}$, then
            $x\in{C}$. But if $x\notin{A}$, then
            $x\in{C}\setminus{A}$. Therefore
            $B\setminus{A}\subseteq{B}\cap(C\setminus{A})$.
            Similarly,
            $B\cap(C\setminus{A})\subseteq{B}\setminus{A}$,
            and therefore
            $B\setminus{A}={B}\cap(C\setminus{A})$.
        \end{proof}
        We can draw a Vann diagram for the symmetric difference, see
        Fig.~\ref{fig:Sym_Diff_Venn_Diagram}.
        \begin{figure}[H]
            \centering
            \captionsetup{type=figure}
            %--------------------------------Dependencies----------------------------------%
%   tikz                                                                       %
%-------------------------------Main Document----------------------------------%
\begin{tikzpicture}[line width=0.2mm]

    % Coordinates for the centers of the circles.
    \coordinate (C1) at (-1.3, 0);
    \coordinate (C2) at ( 1.3, 0);

    % Coordinates for the labels.
    \coordinate (A) at (-1.3, 1.2);
    \coordinate (B) at ( 1.3, 1.2);
    \coordinate (S) at ( 0.0, 2.5);

    % Rectangle indicating the universe set.
    \draw (-4, -3) rectangle (4, 3);

    % Fill in the circle with cyan.
    \draw[fill=cyan, draw=none] (C1) circle (2);
    \draw[fill=cyan, draw=none] (C2) circle (2);

    % Fill in the circle with cyan.
    \draw[fill=white, draw=none] (0, -1.51987) arc(-49.46:49.46:2)
                                               arc(130.54:229.46:2);

    % Give outlines to the circles.
    \draw (C1) circle (2);
    \draw (C2) circle (2);

    % Labels.
    \node at (A) {$A$};
    \node at (B) {$B$};
    \node at (S) {$A\ominus{B}$};
\end{tikzpicture}
            \caption{Venn Diagram for Symmetric Difference}
            \label{fig:Sym_Diff_Venn_Diagram}
        \end{figure}
        The concept of set difference can then be used to define the
        concept of complement.
        Thm.~\ref{thm:Set_Difference_As_Intersection} can then be
        translated into the notation of complements as follows:
        \begin{theorem}
            If $A$, $B$, and $\Omega$ are sets,
            $A,B\subseteq\Omega$, and if $A^{C}$ is the
            complement of $A$ with respect to $\Omega$, then:
            \begin{equation}
                B\setminus{A}=B\cap{A}^{C}
            \end{equation}
        \end{theorem}
        \begin{proof}
            By the definition of complement,
            $A^{C}=\Omega\setminus{A}$.
            As $A\subseteq\Omega$ and $B\subseteq\Omega$, by
            Thm.~\ref{thm:Set_Difference_As_Intersection},
            $B\setminus{A}=B\cap(\Omega\setminus{A})$,
            and therefore $B\setminus{A}=B\cap{A}^{C}$.
        \end{proof}
        The main result about complements are known as
        DeMorgan's Laws. The laws relate unions and
        intersections by means of complements. The general
        laws hold for arbitrary unions and arbitrary
        intersections, as will be shown later.
        \begin{ftheorem}{DeMorgan's Laws}{MEASURE_DEMORGAN}
            If $A$, $B$, and $\Omega$ are sets, if
            $A\subseteq\Omega$ and $B\subseteq\Omega$, then:
            \begin{subequations}
                \begin{align}
                    \big(A\cap{B}\big)^{C}
                    &=A^{C}\cup{B}^{C}\\
                    \big(A\cup{B}\big)^{C}
                    &=A^{C}\cap{B}^{C}
                \end{align}
            \end{subequations}
        \end{ftheorem}
        With this, we can prove some results about
        set differences.
        \begin{theorem}
            If $A$ and $B$ are sets, then:
            \begin{equation}
                A=\big(A\cap{B}\big)
                    \cup\big(A\setminus{B}\big)
            \end{equation}
        \end{theorem}
        \begin{proof}
            For let $\Omega=A\cup{B}$. Then
            $A\subseteq\Omega$ and $B\subseteq\Omega$,
            and thus:
            \begin{subequations}
                \begin{align}
                    \big(A\cap{B})\cup\big(A\setminus{B}\big)
                    &=\big(A\cap{B}\big)
                        \cup\big(A\cap{B}^{C}\big)\\
                    &=A\cap(B\cup{B}^{C})\\
                    &=A\cap\Omega
                \end{align}
            \end{subequations}
            But by Thm.~\ref{thm:Intersection_is_Smaller},
            $A\cap\Omega=A$. Therefore, etc.
        \end{proof}
        \begin{theorem}
            If $A$, $B$, and $C$ are sets, then:
            \begin{equation}
                A\cap\big(B\setminus{C}\big)
                =\big(A\cap{B}\big)\cap\big(A\setminus{C}\big)
            \end{equation}
        \end{theorem}
        \begin{proof}
            For:
            \begin{subequations}
                \begin{align}
                    A\cap\big(B\setminus{C}\big)
                    &=A\cap\big(B\cap{C}^{C}\big)\\
                    &=\big(A\cap{A}\big)
                        \cap\big(B\cap{C}^{C}\big)\\
                    &=\big(A\cap{B}\big)
                        \cap\big(A\cap{C}^{C}\big)\\
                    &=\big(A\cap{B}\big)
                        \cap\big(A\setminus{C}\big)
                \end{align}
            \end{subequations}
        \end{proof}
        Intersections do distribute over set differences.
        \begin{theorem}
            If $A$, $B$, and $C$ are sets, then:
            \begin{equation}
                A\cap(B\setminus{C})=
                (A\cap{B})\setminus(A\cap{C})
            \end{equation}
        \end{theorem}
        \begin{proof}
            For:
            \begin{subequations}
                \begin{align}
                    \big(A\cap{B}\big)\setminus
                        \big(A\cap{C}\big)
                    &=\big(A\cap{B}\big)
                        \cap\big(A\cap{C}\big)^{C}\\
                    &=\big(A\cap{B}\big)
                        \cap\big(A^{C}\cup{C}^{C}\big)\\
                    &=\big[\big(A\cap{B}\big)\cap{A}^{C}\big]
                        \cup\big[\big({A}\cap{B}\big)
                        \cap{C}^{C}\big]\\
                    &=\big[\big(A\cap{A}^{C}\big)\cap{B}\big]
                        \cup\big[\big(A\cap{B}\big)
                        \cap{C}^{C}\big]\\
                    &=\emptyset\cup\big[\big(A\cap{B}\big)
                        \cap{C}^{C}\big]\\
                    &=\big(A\cap{B}\big)\cap{C}^{C}\\
                    &=A\cap\big(B\cap{C}^{C}\big)\\
                    &=A\cap\big(B\setminus{C}\big)
                \end{align}
            \end{subequations}
            Therefore, etc.
        \end{proof}
        Unions do not, however. For let $A$ be non-empty
        and let $A=B=C$. Then $A\cup(B\setminus{C})=A$, but
        $(A\cup{B})\setminus(A\cup{C})=\emptyset$.
        \begin{theorem}
            If $A$ and $B$ are sets and $A\subset B$,
            then $B\setminus(B\setminus A)=A$.
        \end{theorem}
        \begin{proof}
            For:
            \begin{align}
                Yo
            \end{align}
            $[x\in B\setminus(B\setminus{A})]%
            \Rightarrow[x\in{B}\land{x}\notin%
            \{x\in{B}:x\notin{A}\}]%
            \Rightarrow[x\in{A}\subset{B}]$.
            $[x\in{A}]\Rightarrow[x\notin{B}\setminus{A}]%
            \Rightarrow[x\in{B}\setminus(B\setminus{A})]$.
        \end{proof}
        The previous theorem shows that $(A^C)^{C}=A$.
        % Untrue garbage.
        % If $A$ and $B$ are sets, and if $C\subseteq{A}\cup{B}$, then
        % either $C\subseteq{A}$ or $C\subseteq{B}$, or both. It is
        % possible that $C\subseteq{A}\cup{B}$ and yet $C$ and $B$ have no
        % elements in common, as long as $C\subseteq{A}$. As an example,
        % take $A$ and $B$ to be disjoint sets. Then $A\subseteq{A}\cup{B}$,
        % yet $A$ and $B$ have no elements in common. If
        % $C\subseteq{A}\cap{B}$, then it must be true that
        % $C\subseteq{A}$ and $C\subseteq{B}$.
        % As with the notions of unions and intersections, set differences and
        % symmetric differences can be visualized using Venn diagrams.
        \begin{theorem}
            \label{thm:MEASURE_THEORY_SET_DIFFERENCE_AS_INTERSECTION}
            If $A$, $B$, and $C$ are sets, and if $A\subseteq{C}$
            and $B\subseteq{C}$, then:
            \begin{equation}
                B\setminus{A}=B\cap(C\setminus{A})
            \end{equation}
        \end{theorem}
        \begin{proof}
            For if $x\in{B}\setminus{A}$, then
            $x\in{B}$ and $x\notin{A}$. But
            $B\subseteq{C}$, and thus if $x\in{B}$, then $x\in{C}$.
            But if $x\notin{A}$, then $x\in{C}\setminus{A}$. Therefore
            $B\setminus{A}\subseteq{B}\cap(C\setminus{A})$.
            Similarly, $B\cap(C\setminus{A})\subseteq{B}\setminus{A}$,
            and therefore $B\setminus{A}={B}\cap(C\setminus{A})$.
        \end{proof}
        While set difference appears similar to subtraction that one finds in
        basic arithmetic, the two have their differences. For any two real
        numbers $a$ and $b$, $b=a-(a-b)$. For sets this is not true. For let $A$
        be the empty set, and let $B$ be non-empty. Then
        $A\setminus(A\setminus{B})=\emptyset$, which is not $B$.
        Also, while it may seems convincing that
        $A\setminus(B\setminus{A})=A\setminus{B}$, this is not true. For
        let $A$ be a non-empty set and let $B=A$. Then
        $A\setminus(B\setminus{A})=A$, but $A\setminus{B}=\emptyset$.
        The concept of set difference can then be used to define the
        concept of complement.
        %------------------------------------------------------------------------------%
\section{Relations}
    \label{Section:ZFC:Elementary_Set_Theory:Relations}%
    \begin{fdefinition}{Relation on a Set}{Relation_on_a_Set}
        A \gls{relation} on a \gls{set} $A$ is a \gls{subset} $R$ of the
        \gls{Cartesian product} $A\times{A}$.
    \end{fdefinition}
    We use a special notation for relations on a set.
    \begin{fnotation}{Relation Notation}{Relation_Notation}
        If $A$ is a set, if $R$ is a relation on $A$, and if $(a,b)\in{R}$, we
        write $aRb$.
        \begin{equation*}
            \forall_{x}\forall_{y}\big(aRb\big)\Leftrightarrow
            \big((a,b)\in{R}\big)
        \end{equation*}
    \end{fnotation}
    For a relation $R$ it is not necessary true that $aRb$ implies $bRa$, nor is
    it necessarily true that $aRa$. These are called symmetric and reflexive
    relations, respectively.
    \begin{lexample}{Examples of Relations}{Examples_of_Relations}
        Let $A=\mathbb{R}$ and consider the relation of equality. That is, let
        $R_{=}\subseteq\mathbb{R}^{2}$ be defined by:
        \begin{equation}
            R_{=}=\{\,(x,y)\in\mathbb{R}^{2}\;|\;x=y\,\}
        \end{equation}
        Then $R_{=}$ is a relation on $\mathbb{R}^{2}$. Rather than writing
        $(x,y)\in{R_{=}}$ or $xR_{=}y$ we commonly write $x=y$. Note that this
        relation is defined entirely by the \textit{diagonal} of the Cartesian
        product $\mathbb{R}\times\mathbb{R}$. Another simple relation is that of
        ordering. Let $R_{<}$ be defined as follows:
        \begin{equation}
            R_{<}=\{\,(x,y)\in\mathbb{R}^{2}\;|\;x<y\,\}
        \end{equation}
        This is also a relation since it is a subset of the Cartesian product,
        but it is a slightly more complicated one. There are many
        \textit{off-diagonal} elements of this relation.
    \end{lexample}
    \begin{theorem}
        If $B$ is a set, if $A\subseteq{B}$, and if $R$ is a relation on $B$,
        then there is a relation $R_{A}$ such that $R_{A}$ is a relation on
        $A$ and $R_{A}\subseteq{R}$.
    \end{theorem}
    \begin{proof}
        For let $P$ be the proposition \textit{True if} $(x,y)\in{A}^{2}$,
        \textit{false otherwise}. By the axiom schema of specification
        (Ax.~\ref{ax:Axiom_Schema_of_Specification}) there is a set:
        \begin{equation}
            R_{A}=\big\{\,(x,y)\in{R}\;|\;P\big((x,y)\big)\,\big\}
        \end{equation}
        But then $(x,y)\in{R}_{A}$ if and only if $(x,y)\in{R}$ and
        $(x,y)\in{A}^{2}$.
    \end{proof}
    This set is called the \textit{restriction} of $R$ to the subset $A$.
    \begin{fdefinition}{Restriction of a Relation}{Restriction_of_a_Relation}
        The restriction of a relation $R$ on a set $B$ to a subset $A$ is the
        set $R_{A}$ defined by:
        \begin{equation*}
            R_{A}=\big\{\,(x,y)\in{R}\;|\;(x,y)\in{A}^{2}\,\big\}
        \end{equation*}
    \end{fdefinition}
    There are many basic properties that relations have, and we prove them now.
    \begin{theorem}
        \label{thm:Cartesian_Product_Is_Relation}%
        If $A$ is a set, then $A\times{A}$ is a relation on $A$.
    \end{theorem}
    \begin{proof}
        For if $A$ is a set, then
        $A\times{A}\subseteq{A}\times{A}$. Therefore, etc.
    \end{proof}
    \begin{theorem}
        \label{thm:Empty_Set_Is_Relation}%
        If $A$ is a set, and then $\emptyset$ is a relation
        on $A$.
    \end{theorem}
    \begin{proof}
        For if $A$ is a set, then
        $\emptyset\subseteq{A}\times{A}$. Therefore, etc.
    \end{proof}
    \begin{theorem}
        Set inclusion $\subseteq$ is a relation. Proper set inclusion
        $\subsetneq$ is a relation. These define partial orderings.
    \end{theorem}
    \begin{fdefinition}{Domain of a Relation}{Domain_of_a_Relation}
        The \gls{domain (relation)} of a \gls{relation} $R$ on a \gls{set} $A$
        is the set:
        \begin{equation*}
            \textrm{dom}(R)=\big\{a\in{A}\;|\;\exists{b}\in{A}
                \textrm{ such that }aRb\big\}
        \end{equation*}
    \end{fdefinition}
    \begin{fdefinition}{Range of a Relation}{Range_of_a_Relation}
        The \gls{range (relation)} of a \gls{relation} $R$ on a \gls{set} $A$ is
        the set:
        \begin{equation*}
            \textrm{ran}(R)=\big\{b\in{A}\;|\;\exists{a}\in{A}
                \textrm{ such that }aRb\big\}
        \end{equation*}
    \end{fdefinition}
    \begin{fdefinition}{Field of a Relation}{Field_of_a_Relation}
        The \gls{field (relation)} of a \gls{relation} $R$ on a set $A$ is the
        set:
        \begin{equation*}
            \textrm{field}(R)=\textrm{dom}(R)\cup\textrm{ran}(R)
        \end{equation*}
        Where $\textrm{dom}(R)$ is the \gls{domain (relation)} of $R$ and
        $\textrm{ran}(R)$ is the \gls{range (relation)} of $R$.
    \end{fdefinition}
    These provide the two most basic examples of relations on a
    set. The empty set is the relation that says no two elements
    are related. Indeed, even single elements are unrelated to
    themselves. The second, the entire Cartesian product
    $A\times{A}$, says that everything is related. These are the
    two extreme cases, but provide useful examples and
    counterexamples in various contexts. More useful is that the
    union and intersection of relations is also a relation. We
    prove this now.
    \begin{theorem}
        \label{thm:Intersection_of_Relations_Is_Relation}%
        If $A$ is a set and if $R_{1}$ and $R_{2}$ are relations
        on $A$, then $R_{1}\cap{R}_{2}$ is a relation on $A$.
    \end{theorem}
    \begin{proof}
        For let $R=R_{1}\cap{R}_{2}$ and suppose $R$ is not a
        relation on $A$. Then there is an $x\in{R}$ such that
        $x\notin{A}\times{A}$. But if $x\in{R}$ then
        $x\in{R}_{1}$ and $x\in{R}_{2}$. But for all
        $x\in{R}_{1}$, $x\in{A}\times{A}$, since $R_{1}$ is a
        relation on $A$, a contradiction as
        $x\notin{A}\times{A}$. Therefore, $R$ is a relation on
        $A$.
    \end{proof}
    \begin{theorem}
        \label{thm:Set_Theory_Union_of_Relations_Is_Relation}
        If $A$ is a set and if $R_{1}$ and $R_{2}$ are relations
        on $A$, then $R_{1}\cup{R}_{2}$ is a relation on $A$.
    \end{theorem}
    \begin{proof}
        For let $R=R_{1}\cup{R}_{2}$ and suppose $R$ is not a
        relation on $A$. Then there is an $x\in{R}$ such that
        $x\notin{A}\times{A}$. But if $x\in{R}$ then
        $x\in{R}_{1}$ or $x\in{R}_{2}$. But for all $x\in{R}_{1}$
        and for all $x\in{R}_{2}$,
        $x\in{A}\times{A}$, since $R_{1}$ and $R_{2}$ are
        relations on $A$, a contradiction. Therefore, etc.
    \end{proof}
    \begin{theorem}
        If $A$ is a set and $R$ is a relation on $A$, then there
        is a relation $\mathcal{U}$ on $A$ such that
        $R\cap\mathcal{U}=R$.
    \end{theorem}
    \begin{proof}
        For let $\mathcal{U}={A}\times{A}$. Then by
        Thm.~\ref{thm:Cartesian_Product_Is_Relation}, $A\times{A}$ is
        a relation on $A$. But since $R$ is a relation,
        $R\subseteq{A}\times{A}$. But then
        $R\cap\mathcal{U}=R$. Therefore, etc.
    \end{proof}
    \begin{theorem}
        If $A$ is a set and $R$ is a relation on $A$, then there
        is a relation $\mathcal{U}$ on $A$ such that
        $R\cup\mathcal{U}=R$
    \end{theorem}
    \begin{proof}
        For let $\mathcal{U}=\emptyset$. Then by
        Thm.~\ref{thm:Empty_Set_Is_Relation},
        $\mathcal{U}$ is a relation. But if $R$ is a set, then
        $R\cup\emptyset=R$. Thus, $R\cup\mathcal{U}=R$.
        Therefore, etc.
    \end{proof}
    Since a general relation is simply a subset of $A\times{A}$,
    there's not much structure on them, and thus there's not a lot
    that can be said about them. We can add more constraints to
    certain relations to get the more familiar properties
    we're used to.
    \begin{fdefinition}{Reflexive Relations}{Reflexive_Relations}
        A reflexive relation on a set $A$ is a
        relation $R$ on $A$ such that for all $a\in{A}$
        it is true that $aRa$.
    \end{fdefinition}
    A reflexive relation on $A$ is simply any subset of
    $A\times{A}$ that contains the entire \textit{diagonal}. That,
    all of the pairs $(a,a)$. A reflexive relation can contain more
    than this, however. The only strict requirement is that
    $aRa$ for all $a\in{A}$.
    \begin{theorem}
        If $A$ is a set, and if $R_{1}$ and $R_{2}$ are reflexive
        relations on $A$, then $R_{1}\cap{R}_{2}$ is a reflexive
        relation on $A$.
    \end{theorem}
    \begin{proof}
        For let $R=R_{1}\cap{R}_{2}$. Then by
        Thm.~\ref{thm:Intersection_of_Relations_Is_Relation}, $R$ is a relation.
        Suppose $R$ is not reflexive.
        Then there is an $a\in{A}$ such that $(a,a)\notin{R}$. But
        if $a\in{A}$, then $(a,a)\in{R}_{1}$, since $R_{1}$ is
        reflexive. Similarly, $(a,a)\in{R}_{2}$ since $R_{2}$ is
        reflexive. But if $(a,a)\in{R}_{1}$ and $(a,a)\in{R}_{2}$,
        then $(a,a)\in{R}$ since $R=R_{1}\cap{R}_{2}$, a
        contradiction. Therefore, $R$ is reflexive.
    \end{proof}
    \begin{theorem}
        If $A$ is a set, if $R_{1}$ is a reflexive relation on
        $A$, and if $R_{2}$ is a relation on $A$, then
        $R_{1}\cup{R}_{2}$ is a reflexive relation on $A$.
    \end{theorem}
    \begin{proof}
        For let $R=R_{1}\cup{R}_{2}$. Since $R_{1}$ and $R_{2}$ are
        relations, by
        Thm.~\ref{thm:Set_Theory_Union_of_Relations_Is_Relation},
        $R$ is a relation. Suppose it is not reflexive.
        Then there is an $a\in{A}$ such that
        $(a,a)\notin{R}$. But if $a\in{A}$ then $(a,a)\in{R}_{1}$
        since $R_{1}$ is reflexive. But if $(a,a)\in{R}_{1}$ then
        $(a,a)\in{R}_{1}\cup{R}_{2}$, a contradiction.
        Therefore, etc.
    \end{proof}
    Given an arbitrary relation $R$ on a set $A$, it may not be
    true that $R$ is reflexive. It may often be useful to add in
    only the necessary points of $A$ that will make $R$
    reflexive. This is called the reflexive closure of $R$.
    \begin{fdefinition}{Reflexive Closure of a Relation}
                       {Reflexive_Closure_of_Relation}
        The reflexive closure of a relation $R$ on a set $A$
        is the set:
        \begin{equation}
            S=R\cup\{(a,a):a\in{A}\}
        \end{equation}
    \end{fdefinition}
    \begin{theorem}
        If $A$ is a set, $R$ is a relation on $A$, and if $S$ is the
        reflexive closure of $R$, then $S$ is a reflexive relation on $A$.
    \end{theorem}
    \begin{theorem}
        \label{thm:Set_Theory_Refl_Clos_Is_Smallest_Refl_With_R}
        If $A$ is a set, if $R$ is a relation on $A$, if
        $S$ is the reflexive closure of $R$, and if $T$ is a
        reflexive relation on $A$ such that $R\subseteq{T}$, then
        $S\subseteq{T}$.
    \end{theorem}
    \begin{proof}
        For if $x\in{S}$, then either $x\in{R}$ or there is an
        $a\in{A}$ such that $x=(a,a)$. But if $x\in{R}$, then
        $x\in{T}$ since $R\subseteq{T}$. If $x\notin{R}$ then
        there is an $a\in{A}$ such that $x=(a,a)$. But $T$ is
        reflexive, and therefore $(a,a)\in{T}$. But then
        $x\in{T}$. Therefore, $S\subseteq{T}$.
    \end{proof}
    Thm.~\ref{thm:Set_Theory_Refl_Clos_Is_Smallest_Refl_With_R}
    says that the reflexive closure of a relation $R$ is, in a sense,
    the \textit{smallest} relation that is reflexive and contains
    $R$ as a subset.
    \begin{theorem}
        If $A$ is a set, $R_{1}$ and $R_{2}$ are relations on $A$,
        and if $S_{1}$ and $S_{2}$ are the reflexive closures of
        $R_{1}$ and $R_{2}$, respectively, then the reflexive closure
        of $R_{1}\cap{R}_{2}$ is:
        \begin{equation}
            S=S_{1}\cap{S}_{2}
        \end{equation}
    \end{theorem}
    \begin{proof}
        By the definition of reflexive closure, we have:
        \begin{align}
            S_{1}&=R_{1}\cup\{(a,a):a\in{A}\}
            \tag{Def.~\ref{def:Reflexive_Closure_of_Relation}}\\
            S_{1}&=R_{2}\cup\{(a,a):a\in{A}\}
            \tag{Def.~\ref{def:Reflexive_Closure_of_Relation}}\\
            \nonumber
            S_{1}\cap{S}_{2}&=
            (R_{1}\cup\{(a,a):a\in{A}\})
            \cap(R_{2}\cup\{(a,a):a\in{A}\})\\
            &=(R_{1}\cap{R}_{2})
            \cup\{(a,a):a\in{A}\}
            \tag{Distributive Law}
        \end{align}
        But by the definition of the transitive closure of
        $R_{1}\cap{R}_{2}$:
        \begin{equation}
            S=(R_{1}\cap{R}_{2})\cup\{(a,a):a\in{A}\}
            \tag{Def.~\ref{def:Reflexive_Closure_of_Relation}}
        \end{equation}
        Therefore, etc.
    \end{proof}
    \begin{fdefinition}{Symmetric Relation}{Symmetric_Relation}
        A symmetric relation on a set $A$ is a
        relation $R$ on $A$ such that for all $a,b\in{A}$
        such that $aRb$, it is true that $bRa$.
    \end{fdefinition}
    \begin{theorem}
        If $A$ is a set, if $S_{1}$ and $S_{2}$ are symmetric relations
        on $A$, then $S_{1}\cap{S}_{2}$ is a symmetric relation on $A$.
    \end{theorem}
    \begin{proof}
        For since $S_{1}$ and $S_{2}$ are relations, $S_{1}\cap{S}_{2}$ is a
        relation (Thm.~\ref{thm:Intersection_of_Relations_Is_Relation}). Suppose
        it is not symmetric. Then there is an $(x,y)\in{S}_{1}\cap{S}_{2}$ such
        that $(y,x)\notin{S}_{1}\cap{S}_{2}$. But if
        $(x,y)\in{S}_{1}\cap{S}_{2}$, then $(x,y)\in{S}_{1}$ and
        $(x,y)\in{S}_{2}$ (Def.~\ref{def:Intersection_of_Two_Sets}). But $S_{1}$
        is symmetric and if $(x,y)\in{S}_{1}$, then $(y,x)\in{S}_{1}$
        (Def.~\ref{def:Symmetric_Relation}). Similarly $(y,x)\in{S}_{2}$, and
        therefore $(y,x)\in{S}_{1}\cap{S}_{2}$
        (Def.~\ref{def:Intersection_of_Two_Sets})), a contradiction. Therefore,
        $S_{1}\cap{S}_{2}$ is symmetric.
    \end{proof}
    \begin{fdefinition}{Transitive Relation}{Transitive_Relation}
        A transitive relation on a set $A$ is a relation $R$ on $A$
        such that for all $a,b,c\in{A}$ such that $aRb$ and $bRc$,
        is it true that $aRc$.
    \end{fdefinition}
    \begin{theorem}
        \label{thm:Entire_Cartesian_is_Transitive}%
        If $A$ is a set, then $A\times{A}$ is a transitive relation on $A$.
    \end{theorem}
    \begin{proof}
        For suppose not. Then there exists $a,b,c\in{A}$ such that
        $(a,b)\in{A}\times{A}$ and $(b,c)\in{A}\times{A}$, yet
        $(a,c)\in{A}\times{A}$. But if $a\in{A}$ and $c\in{A}$, then
        $(a,c)\in{A}\times{A}$ (Def.~\ref{def:Cartesian_Product_of_Two_Sets}), a
        contradiction. Therefore $A\times{A}$ is a transitive relation on $A$.
    \end{proof}
    Using the Cartesian product definition of a relation, we can visualize the
    requirement imposed on transitive relations in the diagram below
    (Fig.~\ref{fig:Transitive_Relation_Diagram}).
    \begin{figure}[H]
        \centering
        \captionsetup{type=figure}
        \begin{tikzpicture}
    \foreach\x in {0,1,2,3,4,5,6,7,8,9}{
        \foreach\y in {0,1,2,3,4,5,6,7,8,9}{
            \draw[fill=black] (\x,\y) circle (0.1);
        }
        \node at (\x, -1) {$\x$};
        \node at (-1, \x) {$\x$};
    }

    \draw[fill=red,draw=red,opacity=0.4] ( 1.7, -0.3) rectangle (2.3, 9.3);
    \draw[fill=red,draw=red,opacity=0.4] (-0.3,  5.7) rectangle (9.3, 6.3);

    \draw[draw=blue,fill=cyan,opacity=0.5] (2, 4) circle (0.2);
    \draw[draw=blue,fill=cyan,opacity=0.5] (4, 6) circle (0.2);
    \draw[draw=blue,fill=cyan,opacity=0.5] (2, 6) circle (0.2);
\end{tikzpicture}
        \caption{Diagram for a Transitive Relation}
        \label{fig:Transitive_Relation_Diagram}
    \end{figure}
    Given a point $(a,b)$ that is in the relation and another point $(b,c)$, for
    the relation to be transitive requires $(a,c)$ to be contained in it. That
    is, if we take the first coordinate from the first element and the second
    coordinate from the second element and then combine them to form a new
    ordered pair, this element must also be in the relation.
    \begin{theorem}
        If $A$ is a set, if $T_{1}$ and $T_{2}$ are transitive relations on $A$,
        and if $R=T_{1}\cap{T}_{2}$, then $R$ is a transitive relation.
    \end{theorem}
    \begin{proof}
        For since $T_{1}$ and $T_{2}$ are relations, $T_{1}\cap{T}_{2}$ is a
        relation (Thm.~\ref{thm:Intersection_of_Relations_Is_Relation}). Suppose
        it is not transitive. Then there are $(x,y),(y,z)\in{R}$ such that
        $(x,z)\notin{R}$ (Def.~\ref{def:Transitive_Relation}). But if
        $(x,y),(y,z)\in{R}$, then by the definition of intersection,
        $(x,y),(y,z)\in{T}_{1}$ and $(x,y),(y,z)\in{T}_{1}$
        (Def.~\ref{def:Intersection_of_Two_Sets}). But $T_{1}$ is transitive, and
        thus if $xT_{1}y$ and $yT_{1}z$, then $xT_{1}z$. But similarly $T_{2}$
        is transitive, and therefore $xT_{2}z$. But then $(x,z)\in{T}_{1}$ and
        $(x,z)\in{T}_{2}$, and thus $(x,z)\in{T}_{1}\cap{T}_{2}$, a
        contradiction. Therefore, $R$ is transitive.
    \end{proof}
    \begin{example}
        The requirement that both relations $T_{1}$ and $T_{2}$ are transitive
        cannot be weakened. For consider the relations $S$ and $T$ on
        $\mathbb{Z}_{3}$ defined by:
        \par\hfill\par
        \begin{subequations}
            \begin{minipage}[b]{0.49\textwidth}
                \centering
                \begin{equation}
                    S=\big\{\,(0,1),\,(1,2)\,\}
                \end{equation}
            \end{minipage}
            \hfill
            \begin{minipage}[b]{0.49\textwidth}
                \centering
                \begin{equation}
                    T=\big\{\,(0,1),\,(1,2),\,(0,1)\,\}
                \end{equation}
            \end{minipage}
        \end{subequations}
        \par\vspace{2.5ex}
        Then $T$ is transitive and $S$ is not. Moreover $S\subseteq{T}$, and
        hence $S\cap{T}=S$ (Thm.~\ref{thm:Intersection_of_Subset}), and
        therefore the intersection is not transitive. This example is
        demonstrated in
        Fig.~\subref{fig:Trans_Intersect_Non_Trans_May_Not_Be_Trans}. The
        opposite is possible, and to construct an example we need only find a
        transitve relation $T$ and a non-transitive relation $S$ such that
        $T\subseteq{S}$. Define:
        \par\hfill\par
        \begin{subequations}
            \begin{minipage}[b]{0.49\textwidth}
                \centering
                \begin{equation}
                    T=\big\{\,(0,0)\,\}
                \end{equation}
            \end{minipage}
            \hfill
            \begin{minipage}[b]{0.49\textwidth}
                \centering
                \begin{equation}
                    S=\big\{\,(0,0),\,(0,1),\,(1,2)\,\}
                \end{equation}
            \end{minipage}
        \end{subequations}
        \par\vspace{2.5ex}
        Then $T$ is transitive, $S$ is not, and $S\cap{T}=T$
        (See \subref{fig:Trans_Int_Trans_May_Not_Be_Trans}).
    \end{example}
    \begin{figure}[H]
        \centering
        \captionsetup{type=figure}
        \begin{subfigure}[b]{0.49\textwidth}
            \centering
            \begin{tikzpicture}
    \foreach\x in {0,1,2}{
        \foreach\y in {0,1,2}{
            \draw[fill=black] (\x,\y) circle (0.4mm);
        }
    }
    \node at (-1, 2) {$a$};
    \node at (-1, 1) {$b$};
    \node at (-1, 0) {$c$};
    \node at (2, 3) {$a$};
    \node at (1, 3) {$b$};
    \node at (0, 3) {$c$};
    \draw[draw=blue,fill=none] (0,1) circle (0.3);
    \draw[draw=blue,fill=none] (1,0) circle (0.3);
    \draw[draw=blue,fill=none] (2,2) circle (0.3);
    \draw[draw=red,fill=none] (-0.4243, 0.5757) rectangle (0.4243, 1.4243);
    \draw[draw=red,fill=none] ( 0.5757,-0.4243) rectangle (1.4243, 0.4243);
\end{tikzpicture}
            \subcaption{The intersection is not transitive}
            \label{fig:Trans_Intersect_Non_Trans_May_Not_Be_Trans}
        \end{subfigure}
        \hfill
        \begin{subfigure}[b]{0.49\textwidth}
            \centering
            \begin{tikzpicture}
    \foreach\x in {0,1,2}{
        \foreach\y in {0,1,2}{
            \draw[fill=black] (\x,\y) circle (0.4mm);
        }
    }
    \node at (-1, 2) {$a$};
    \node at (-1, 1) {$b$};
    \node at (-1, 0) {$c$};
    \node at (2, 3) {$a$};
    \node at (1, 3) {$b$};
    \node at (0, 3) {$c$};
    \draw[draw=blue,fill=none] (0,2) circle (0.3);
    \draw[draw=red,fill=none] (-0.4243, 1.5557) rectangle (0.4243, 2.4243);
    \draw[draw=red,fill=none] (-0.4243, 0.5757) rectangle (0.4243, 1.4243);
    \draw[draw=red,fill=none] ( 0.5757,-0.4243) rectangle (1.4243, 0.4243);
\end{tikzpicture}
            \subcaption{The intersection is transitive}
            \label{fig:Trans_Int_Trans_May_Not_Be_Trans}
        \end{subfigure}
        \label{fig:Intersection_of_Transitive_and_Non_Transitive_Relations}
        \caption{The Intersection of Transitive and Non-Transitive Relations}
    \end{figure}
    We can strengthened our claim that the intersection of two transitive
    relations is again transitive and show that any arbitrary intersection will
    again be transitive.
    \begin{theorem}
        \label{thm:Intersection_of_Transitive_is_Transitive}%
        If $A$ is a set, if $\mathcal{P}(A\times{A})$ denotes the power set of
        $A\times{A}$, if $\mathcal{O}\subseteq\mathcal{P}(A\times{A})$ is such
        that for all $\mathcal{U}\in\mathcal{O}$ it is true that $\mathcal{U}$
        is a transitive relation on $A$, if $\mathcal{T}$ is defined by:
        \begin{equation}
            \mathcal{T}=\bigcap_{\mathcal{U}\in\mathcal{O}}\mathcal{U}
        \end{equation}
        Then $\mathcal{T}$ is a transitive relation on $A$.
    \end{theorem}
    \begin{proof}
        For suppose not. Then there exists $a,b,c\in{A}$ such that
        $(a,b)\in\mathcal{T}$ and $(b,c)\in\mathcal{T}$, yet
        $(a,c)\notin\mathcal{T}$. But if $(a,b)\in\mathcal{T}$, then for all
        $\mathcal{U}\in\mathcal{O}$ it is true that $(a,b)\in\mathcal{U}$
        (Def.~\ref{def:Intersection_Over_a_Collection}). Similarly, for all
        $\mathcal{U}\in\mathcal{O}$ it is true that $(b,c)\in\mathcal{U}$.
        But by hypothesis, for all $\mathcal{U}\in\mathcal{O}$ it is true that
        $\mathcal{U}$ is a transitive relation and thus if $(a,b)\in\mathcal{U}$
        and $(b,c)\in\mathcal{U}$, then it is true that $(a,c)\in\mathcal{U}$
        (Def.~\ref{def:Transitive_Relation}). But then for all
        $\mathcal{U}\in\mathcal{O}$ it is true that $(a,c)\in\mathcal{U}$, and
        therefore $(a,c)\in\mathcal{T}$
        (Def.~\ref{def:Intersection_Over_a_Collection}), a contradiction.
        Therefore, $\mathcal{T}$ is a transitive relation on $A$.
    \end{proof}
    This allows us to define the transitive closure of any relation $R$ on a set
    $A$. It is, in a sense, the \textit{smallest} transitive relation that
    contains $R$.
    \begin{theorem}
        If $A$ is a set and if $R$ is a relation on $A$, then there exists a
        transitive relation $\mathcal{T}$ on $A$ such that
        $R\subseteq\mathcal{T}$ and such that for transitive relations $T$ on
        $A$ such that $R\subseteq{T}$ it is true that $\mathcal{T}\subseteq{T}$.
    \end{theorem}
    \begin{proof}
        For let $P$ be the proposition \textit{True if} $S$
        \textit{is a transitive relation on} $A$ \textit{such that}
        $R\subseteq{S}$, \textit{false otherwise}. Then by the axiom schema of
        specification (Ax.~\ref{ax:Axiom_Schema_of_Specification}) there exists
        a set:
        \begin{equation}
            \mathcal{O}=\big\{\,S\in\mathcal{P}(A\times{A})\;|\;P(S)\,\big\}
        \end{equation}
        But then for all $S\in\mathcal{O}$, $P(S)$ is true and therefore
        $R\subseteq{S}$ and $S$ is transitive. Moreover, $\mathcal{O}$ is
        non-empty since by Thm.~\ref{thm:Entire_Cartesian_is_Transitive},
        $A\times{A}$ is a transitive relation. Define $\mathcal{T}$ by:
        \begin{equation}
            \mathcal{T}=\bigcap_{S\in\mathcal{O}}S
        \end{equation}
        Then by Thm.~\ref{thm:Intersection_of_Transitive_is_Transitive},
        $\mathcal{T}$ is a transitive relation. Moreover, suppose $S$ is a
        transitive relation such that $R\subseteq{S}$. But if $S$ is a relation
        on $A$, then $S\subseteq{A}\times{A}$
        (Def.~\ref{def:Relation_on_a_Set}) and therefore
        $S\in\mathcal{P}(A\times{A})$ (Def.~\ref{def:Power_Set}). But if $S$ is
        a transitive relation and if $R\subseteq{S}$, then $P(S)$ is true, and
        therefore $S\in\mathcal{P}$. Thus, $\mathcal{T}\subseteq{S}$.
    \end{proof}
    \begin{fdefinition}{Transitive Closure}{Transitive_Closure}
        The transitive closure of a relation $R$ on a set
        $A$ is the the set $R^{t}\subseteq{A}\times{A}$ defined by:
        \begin{equation}
            R^{t}
        \end{equation}
    \end{fdefinition}
    \begin{fdefinition}{Asymmetric Relation}{Assymetric_Relation}
        An asymmetric relation on a set $A$ is a relation $R$
        on $A$ such that for all $a,b\in{A}$ such that $aRb$
        it is true that $(b,a)\notin{R}$.
    \end{fdefinition}
    \begin{fdefinition}{Total Relation}{Total_Relation}
        A total relation on a set $A$ is a relation $R$ on $A$ such
        that for all $a,b\in{A}$ it is true that either
        $aRb$ or $bRa$, or both.
    \end{fdefinition}
    The notion of equality can be defined as a relation
    with the following properties:
    \begin{enumerate}
        \item Equality is Reflexive: $a=a$ for all $a\in{A}$.
        \item Equality is Symmetric: $a=b$ if and only if $b=a$.
        \item Equality is Transitive: If $a=b$ and $b=c$, then $a=c$.
        \item The relation is uniquely defined by the set
              $\{(a,a)\in A\times A:a\in A\}$.
    \end{enumerate}
    That is, equality can be seen as the \textit{diagonal} in the
    Cartesian product $A\times{A}$.
    \begin{fdefinition}{Antisymmetric Relation}
        An antisymmetric relation on a set $A$ is a relation $R$ on $A$
        such that for all $a,b\in{A}$ such that $aRb$ and $bRa$, it
        is true that $a=b$.
    \end{fdefinition}
    \begin{fdefinition}{Equivalence Relation}{Equivalence_Relation}
        An equivalence relation on a set $A$ is a relation $R$ on $A$ such that
        $R$ is reflexive, symmetric, and transitive.
    \end{fdefinition}
    Equivalence relations attempt to model equality. They are fundamental in
    mathematics as they allow us to define \textit{equivalence classes}, which
    are used to define quotients. There are many examples such as quotient
    topologies, quotient groups, quotient rings, and quotient modules, all of
    which will be discussed later.
    \begin{fdefinition}{Equivalence Class}{Equivalence_Class}
        The equivalence class of an element $x$ in a set $A$ by an
        equivalence relation $R$ is the set:
        \begin{equation*}
            [x]=\{\,y\in{A}\;|\;xRy\,\}
        \end{equation*}
    \end{fdefinition}
    It's important to note that the term class here is different from the notion
    of a collection of sets. And equivalence class of an element $x$ in a set
    $A$ under an equivalene relation $R$ will indeed be a set in $ZFC$.
    \begin{theorem}
        \label{thm:Equivalence_Classes_Disjoint_or_Equal}%
        If $A$ is a set, if $R$ is an equivalence relation on $A$, and if
        $x,y\in{A}$, then either $[x]=[y]$ or $[x]\cap[y]=\emptyset$.
    \end{theorem}
    \begin{proof}
        For suppose not and suppose $[x]\ne[y]$ and that
        $[x]\cap[y]\ne\emptyset$. That is, suppose:
        \begin{equation*}
            \neg([x]=[y])\land\neg([x]\cap[y]=\emptyset)
        \end{equation*}
        If $[x]\cap[y]$ is non-empty then there is a
        $z\in{A}$ such that $z\in[x]$ and $z\in[y]$
        (Def.~\ref{def:Non_Empty_Set}). But if $z\in[x]$, then $xRz$
        (Def.~\ref{def:Equivalence_Class}). But also
        $z\in[y]$ and therefore $yRz$. But $R$ is an equivalence relation and
        is therefore symmetric (Def.~\ref{def:Equivalence_Relation}) and thus
        if $yRz$ then $zRy$ (Def.~\ref{def:Symmetric_Relation}). But an
        equivalence relation is also transitive, and thus if $xRz$ and $zRy$,
        then $xRy$ (Def.~\ref{def:Transitive_Relation}). But if $[x]\ne[y]$ then
        either $[x]\nsubseteq[y]$ or $[y]\nsubseteq[x]$. Suppose
        $[x]\nsubseteq[y]$ and let $a\in[x]$ be such that $a\notin[y]$. But
        if $a\in[x]$ then $xRa$ (Def.~\ref{def:Equivalence_Class}). But since
        equivalence relations are symmetric, if $xRa$, then $aRx$. But it was
        proven that $xRy$ and since equivalence relations are transitive, if
        $aRx$ and $xRy$, then $aRy$. But again if $aRy$, then $yRa$ and
        therefore $a\in[y]$, a contradiction. Therefore $[x]\subseteq[y]$.
        Similarly, $[y]\subseteq[x]$ and therefore $[x]=[y]$, a contradiction.
        By the law of the excluded middle, the negation is true:
        \begin{equation*}
            \neg\big(\neg([x]=[y])\land\neg([x]\cap[y]=\emptyset)\big)
            =([x]=[y])\lor([x]\cap[y]=\emptyset)
        \end{equation*}
        Thus, either $[x]=[y]$ or $[x]\cap[y]=\emptyset$.
    \end{proof}
    \begin{fdefinition}{Quotient Set}{Quotient_Set}
        The quotient set of a set $A$ by an equivalence relation $R$ on $A$ is
        the set:
        \begin{equation*}
            A/R=\{\,[x]\in\mathcal{P}(A)\;|\;x\in{A}\,\}
        \end{equation*}
        Where $[x]$ is the equivalence class of $x$ under $R$.
    \end{fdefinition}
    \begin{example}
        The definition of the quotient set comes naturally when one considers
        functions between sets. Suppose $A$ and $B$ are sets, and suppose
        $f:A\rightarrow{B}$ is a function. In general, it may not be true that
        $f(a_{1})=f(a_{2})$ implies that $a_{1}=a_{2}$, and so we wish to find a
        subset of $A$ with this property. The quotient set does this. Let
        $R$ be the relation:
        \begin{equation}
            R=\{\,(a,b)\in{A}^{2}\;|\;f(a)=f(b)\,\}
        \end{equation}
        If we form the quotient set $A/R$ and consider the projective mapping
        $\pi:A\rightarrow{A}/R$ that sends $a\in{A}$ to its equivalence class.
        That is, $\pi(a)=[a]$. We then seek a function
        $\tilde{f}:A/R\rightarrow{B}$ such that $\tilde{f}\circ{\pi}=f$.
        That is, we wish to make the diagram below \textit{commute}.
        \begin{figure}[H]
            \centering
            \begin{tikzpicture}[%
    >=latex,
    every path/.style={->},
    line width=0.2mm,
    line cap=round
]
    \node (A) at (0.0,  0.0) {$A$};
    \node (AR) at (0.0, -2.0) {$A/R$};
    \node (B) at  (2.0,  0.0) {$B$};
    \path (A) edge node [above]        {$f$}         (B);
    \path (AR) edge node [below right] {$\tilde{f}$} (B);
    \path (A) edge node [left]         {$\pi$}       (AR);
\end{tikzpicture}
            \label{fig:Comm_Diagram_Quotient_Set}
            \caption{Commutative Diagram for the Quotient Set}
        \end{figure}
        So we need to map $[x]$ to $f(x)$. That is, $\tilde{f}([x])=f(x)$. For
        this problem to be well posed requires that the equivalence class that
        make up the elements of $A/R$ come from equivalence relations. That is,
        that the relation $R$ is transitive, symmetric, and reflexive.
    \end{example}
    \begin{theorem}
        If $A$ is a set and if $R$ is an equivalence relation on $A$, then
        $A/R$ is a partition of $A$.
    \end{theorem}
    \begin{proof}
        For by Thm.~\ref{thm:Equivalence_Classes_Disjoint_or_Equal}, if
        $\mathcal{U},\mathcal{V}\in{A}/R$, then either
        $\mathcal{U}=\mathcal{V}$ or $\mathcal{U}\cap\mathcal{V}=\emptyset$.
        But also, for all $x\in{A}$, there is a $\mathcal{U}\in{A}/R$ such that
        $x\in\mathcal{U}$ since $x\in[x]$ and $[x]\in{R}/A$. Therefore,
        $A/R$ is a partition of $A$.
    \end{proof}
    \chapter{Function Theory}
        \label{chapt:Function_Theory}%
        Functions serve as a basic tool for studying mathematics, so much so
        that it is often taken as fundamental and no definition is given. We've
        adopted the definition that a function\index{Function} from a set $A$ to
        a set $B$, $f:A\rightarrow{B}$, is a subset of the Cartesian product
        $A\times{B}$ with a few properties (see Def.~\ref{def:Function}). We now
        take the time to examine the implications of this definition.
        \section{Definitions}
    \begin{fdefinition}{Fields}{Fields}
        A field is a commutative ring $(\mathbb{F},+,\cdot\,)$ such that, for
        all $a\in\mathbb{F}$ such that $a$ is not the unital element of
        $(\mathbb{F},+)$, it is true that $a$ is an invertible element of
        $(\mathbb{F},\cdot\,)$.
    \end{fdefinition}
    \begin{fdefinition}{Subfield}{Subfield}
        A subfield of a field $(F,+,\cdot)$ is a set $K\subset F$, such that
        $(K,+,\cdot)$ is a field.
    \end{fdefinition}
    Given an element $a\in\mathbb{F}$, if $b$ is such that
    $a+b=0$ then we write $b=\minus{a}$. Subtraction of two elements
    $a$ and $c$, denoted $a-c$, is defined as $a+(\minus{c})$. The
    structure $(\mathbb{F},+)$ forms an Abelian group. From this we have
    that the identity is unique, as are additive inverses.
    It is common in the definition of a field to require that
    $0\ne{1}$. This is because if $0=1$ then we have $\mathbb{F}=\{0\}$.
    This comes from the following.
    \begin{ltheorem}{Multiplication by Zero}{Multiplication_by_Zero}
        If $(\mathbb{F},\,+,\,\cdot\,)$ is a field, and if $a\in\mathbb{F}$,
        then $a\cdot{0}=0$.
    \end{ltheorem}
    \begin{proof}
        For we have:
        \begin{equation}
            0=a\cdot(0)-a\cdot(0)=a\cdot(0-0)=a\cdot{0}
        \end{equation}
        This simply combines the distributive law with the additive
        property of zero, completing the proof.
    \end{proof}
    \begin{theorem}
        If $(\mathbb{F},\,+,\,\cdot\,)$ is a field, and if $0=1$, then
        $\mathbb{F}=\{0\}$.
    \end{theorem}
    \begin{proof}
        For suppose not, and let $a\in\mathbb{F}$ be such that $a\ne{0}$.
        But then, by Thm.~\ref{thm:Multiplication_by_Zero}:
        \begin{equation}
            a=a\cdot{1}=a\cdot{0}=0
        \end{equation}
        And thus $a=0$, a contradiction. Therefore,
        $\mathbb{F}$ is trivial.
    \end{proof}
    It is thus common to either call such a field a trivial field, or
    to require that $0\ne{1}$.
    \begin{lexample}{Examples of Fields}{Examples_of_Fields}
        There are several fields that should be familiar to the reader.
        If we let $\mathbb{R}$ denote the real numbers and $+$ and $\cdot$
        be the usual notations of addition and multiplication, then
        $(\mathbb{R},\,+,\,\cdot\,)$ is a field. Similarly, letting
        $\mathbb{Q}$ denote the rational numbers and $\mathbb{C}$ denote
        the complex numbers, $(\mathbb{Q},\,+,\,\cdot\,)$ is a field, as
        is $(\mathbb{C},\,+,\,\cdot\,)$. There are finite fields as well.
        Let $\mathbb{F}_{2}=\{0,\,1\}$ and define multiplication and
        addition as follows:
        \par\hfill\par
        \begin{table}[H]
            \centering
            \captionsetup{type=table}
            \parbox{.45\linewidth}{%
                \centering
                \begin{tabular}{c|cc}
                    $+$&0&1\\
                    \hline
                    0&0&1\\
                    1&1&0
                \end{tabular}
            }
            \parbox{.45\linewidth}{%
                \centering
                \begin{tabular}{c|cc}
                    $\cdot$&0&1\\
                    \hline
                    0&0&0\\
                    1&0&1
                \end{tabular}
            }
            \caption{The Arithmetic of $\mathbb{F}_{2}$}
        \end{table}
        $(\mathbb{F}_{2},\,+,\,\cdot)$ forms a field. Finally, if
        $p\in\mathbb{N}$ is prime, and if $+$ and $\cdot$ are addition
        and multiplication mod $p$, respectively, then
        $(\mathbb{Z}_{p},\,+,\,\cdot\,)$ is a field.
    \end{lexample}
    \begin{fdefinition}{Vector Space}{Vector_Space}
        A vector space over a field $(\mathbb{F},\,+,\,\cdot\,)$ is a
        set $V$ and a function
        $\boldsymbol{\cdot}:\mathbb{F}\times{V}\rightarrow{V}$ and
        a binary operation $\boldsymbol{+}$ on $V$, usuall called
        scalar multiplication and vector addition, respectively, 
        such that for all $\mathbf{x},\mathbf{y},\mathbf{z}\in{V}$,
        and all $a,b\in\mathbf{F}$, the following is true:
        \begin{enumerate}
            \item $\mathbf{x}\boldsymbol{+}%
                   (\mathbf{y}\boldsymbol{+}\mathbf{z})=%
                   (\mathbf{x}\boldsymbol{+}\mathbf{y})%
                   \boldsymbol{+}\mathbf{z}$
                  \hfill[Associative of Vector Addition]
            \item $\mathbf{x}\boldsymbol{+}\mathbf{y}=%
                   \mathbf{y}\boldsymbol{+}\mathbf{x}$
                  \hfill[Commutativity of Vector Addition]
            \item There is a $\mathbf{0}\in{V}$ such that
                  $\mathbf{0}\boldsymbol{+}\mathbf{x}=\mathbf{x}$
                  \hfill[Existence of Zero Vector]
            \item For all $\mathbf{x}$ there is a $\mathbf{y}$ such that
                  $\mathbf{x}\boldsymbol{+}\mathbf{y}=\mathbf{0}$
                  \hfill[Additive Inverses]
            \item $(a\cdot{b})\boldsymbol{\cdot}\mathbf{x}=%
                    a\boldsymbol{\cdot}(b\boldsymbol{\cdot}\mathbf{x})$
                  \hfill[Compatibility of Multiplication]
            \item $(a+b)\boldsymbol{\cdot}\mathbf{x}=%
                   (a\boldsymbol{\cdot}\mathbf{x})\boldsymbol{+}%
                   (b\boldsymbol{\cdot}\mathbf{x})$
                  \hfill[Distributive Law for Field Addition]
            \item $a\boldsymbol{\cdot}(\mathbf{x}\boldsymbol{+}\mathbf{y})=%
                   (a\boldsymbol{\cdot}\mathbf{x})\boldsymbol{+}%
                   (a\boldsymbol{\cdot}\mathbf{y})$
                  \hfill[Distributive Law for Vector Addition]
        \end{enumerate}
    \end{fdefinition}
    It is quite common not to distinguish between scalar multiplication
    $\boldsymbol{\cdot}$ and field multiplication $\cdot$, which may cause
    confusion. It is also common to drop the use of a symbol altogether and
    simply representation multiplication by concatenation of the the
    two variables, for example $a\mathbf{x}$ or $ab$, which represents
    scalar multiplication and field multiplication, respectively.
    \begin{example}
        If we let $\mathbb{F}=\mathbb{R}$ and let
        $V=\mathbb{R}^{n}$, where addition, multiplication, scalar
        multiplication, and vector addition are defined in their usual
        manner, then this forms a vector space. Similarly, the space
        $C([a,b])$ of continuous functions forms a vector space over
        $\mathbb{R}$, as does $L^{2}(\mathbb{R})$, the space of
        square integrable functions.
    \end{example}
    \begin{fdefinition}{Bilinear Operations}{Bilinear_Operations}
        A bilinear operation on a vector space
        $(V,\,\boldsymbol{+},\,\boldsymbol{\cdot}\,)$ over a field
        $(\mathbf{F},\,+,\,\cdot\,)$ is a function
        $[\,]:V\times{V}\rightarrow{V}$ such that, for all
        $\mathbf{x},\mathbf{y},\mathbf{z}\in{V}$, and for all
        $a,b\in\mathbf{F}$, the following is true:
        \begin{enumerate}
            \item $[\mathbf{x}\boldsymbol{+}\mathbf{y}, \mathbf{z}]=%
                   [\mathbf{x},\mathbf{z}]\boldsymbol{+}%
                   [\mathbf{y},\mathbf{z}]$
                  \hfill[Right Distributive Law]
            \item $[\mathbf{x},\mathbf{y}\boldsymbol{+}\mathbf{z}]=%
                   [\mathbf{x},\mathbf{y}]\boldsymbol{+}%
                   [\mathbf{x},\mathbf{z}]$
                  \hfill[Left Distributive Law]
            \item $[a\boldsymbol{\cdot}\mathbf{x},%
                    b\boldsymbol{\cdot}\mathbf{y}]=%
                   (a\cdot{b})\boldsymbol{\cdot}[\mathbf{x},\mathbf{y}]$
                  \hfill[Compatibility with Scalars]
        \end{enumerate}
    \end{fdefinition}
    \begin{lexample}{Examples of Bilinear Operations}
                    {Examples_of_Bilinear_Operation}
        The quintessential example of a bilinear operation is the
        cross product that one encounters in a multivariable calculus
        course. That is, for any three vectors
        $\mathbf{x},\mathbf{y},\mathbf{z}$, we have:
        \begin{equation}
            \mathbf{x}\times(\mathbf{y}+\mathbf{z})=
            \mathbf{x}\times\mathbf{y}+\mathbf{x}\times\mathbf{z}
        \end{equation}
        Similarly for right sided multiplication. The compatibility of
        the cross product with scalar multiplication is also true:
        \begin{equation}
            (a\mathbf{x})\times(b\mathbf{y})=ab(\mathbf{x}\times\mathbf{y})
        \end{equation}
        This serves somewhat as a motivating example for bilinear
        operations. If we think of the field of invertible matrices,
        then multiplication forms a bilinear operation as well, with
        scalar multiplication being the usual entry wise operation that
        is done on matrices. Lastly, if $\langle\,\rangle$ is an inner
        product on $\mathbb{R}$ or $\mathbb{C}$, then this is a bilinear
        operation, the vector space being the underlying field itself.
    \end{lexample}
    \begin{fdefinition}{Algebra over a Field}{Algebra_over_a_Field}
        An algebra of a field $(\mathbf{F},\,+,\,\cdot\,)$ is a
        vector space $(\mathbf{V},\,\boldsymbol{+},\,\boldsymbol{\cdot}\,)$
        and a bilinear operation $[\,]:V\times{V}\rightarrow{V}$.
    \end{fdefinition}
    \begin{fdefinition}{Associative Algebra over a Field}
                       {Associative_Algebra_over_a_Field}
        An associative algebra over a field $(\mathbb{F},\,+,\,\cdot\,)$
        is an algebra $(V,[\,])$ over $\mathbb{F}$ such that, for all
        $r\in\mathbb{F}$ and for all $\mathbf{x},\mathbf{y}\in{V}$,
        the following is true:
        \begin{equation}
            r[\mathbf{x},\,\mathbf{y}]=[r\mathbf{x},\,\mathbf{y}]
                                      =[\mathbf{x},\,r\mathbf{y}]
        \end{equation}
    \end{fdefinition}
    \begin{fdefinition}{Derivation on an Algebra}{Derivation_on_an_Algebra}
        A derivation on an algebra $(V,\,[\,])$ is a function
        $D:V\rightarrow{V}$ such that for all $\mathbf{x},\mathbf{y}\in{V}$,
        the following (Liebniz's Rule) is true:
        \begin{equation}
            D([\mathbf{x},\mathbf{y}])
            =[\mathbf{x},D(\mathbf{y})]+[D(\mathbf{x}),\mathbf{y}]
        \end{equation}
    \end{fdefinition}
    \begin{theorem}
        In a field, $0$ and $1$ are unique.
    \end{theorem}
    \begin{proof}
        For suppose not, and let $0'$ and $1'$ be other identities.
        Then $1'=1'\cdot 1 = 1$ and $0'=0'+0=0$.
    \end{proof}
    \begin{theorem}
        For any field $\langle{F},+,\cdot\rangle$ and $a\in{F}$, $a\cdot{0}=0$.
    \end{theorem}
    \begin{proof}
        For:
        \begin{equation}
            0=a\cdot{0}+(\minus{a}\cdot{0})
             =a\cdot(0+0)+(\minus{a}\cdot{0})
             =a\cdot{0}+a\cdot{0}+(\minus{a}\cdot{0})
             =a\cdot 0
        \end{equation}
        Thus, $a\cdot{0}=0$.
    \end{proof}
    If $1=0$, then $a=a\cdot{1}=a\cdot{0}=0$, and thus every element is
    zero. A very boring field.
    \begin{theorem}
        In a field $\langle F, +,\cdot \rangle$, if $0\ne 1$, then $0$ has no
        inverse.
    \end{theorem}
    \begin{proof}
        For let $a$ be such an inverse. Then $a\cdot{0}=1$. But for any element
        of $F$, $a\cdot{0}=0$. But $0\ne{1}$, a contradiction.
    \end{proof}
    \begin{theorem}
        If $a+b=0$, then $b=(\minus{1})\cdot{a}$ where $(\minus{1})$ is the
        solution to $1+(\minus{1})=0$.
    \end{theorem}
    \begin{proof}
        $a+(\minus{1})a=a(1+(\minus{1}))=a\cdot{0}=0$. From uniqueness,
        $b=(\minus{1})a$. We may thus write additive inverses as $\minus{a}$.
    \end{proof}
    \begin{definition}
        Given two fields $(F,+,\cdot)$ and $(F',+',\times)$, a bijection
        function $f:F\rightarrow{F}'$ is said to be a field isomorphism if and
        only if for allelements $a,b\in{F}$, $f(a+b)=f(a)+'f(b)$, and
        $f(a\cdot{b})=f(a)\times{f}(b)$
    \end{definition}
    \begin{definition}
        $(F,+,\cdot)$ and $(F',+',\times)$, are said to be isomorphic if and
        only if they have an isomorphism.
    \end{definition}
    \begin{theorem}
        Given an ismorphism between two fields $(F,+,\cdot)$ and
        $(F', +',\times)$, $f(1)=1'$ and $f(0)=0'$.
    \end{theorem}
    \begin{proof}
        For let $x\in{F}$. Then $f(x)=f(x\cdot 1)=f(x)\times{f}(1)$, and
        $f(x)=f(x+0)=f(x)+'f(0)$. Therefore, etc.
    \end{proof}
    \begin{theorem}
        In a field $(F,+,\cdot)$, $(a+b)^{2}=a^{2}+2ab+b^{2}$
        ($2$ being the solution to $1+1$).
    \end{theorem}
    \begin{proof}
        For:
        \begin{align}
            (a+b)^{2}&=(a+b)(a+b)\\
                     &=a(a+b)+b(a+b)\\
                     &=a^{2}+ab+ba+b^{2}\\
                     &=a^{2}+ab(1+1)+b^{2}\\
                     &=a^{2}+2ab+b^{2}
        \end{align}
    \end{proof}
        \section{Binary Operations}
    Binary operations are the standard tools that one uses when they develope
    arithmetic. As such, the most familiar examples of binary operations are
    those of addition, multiplication, and subtraction with real numbers.
    On the other hand, division is \textit{not} a binary operation on the real
    numbers since division by zero is undefined. To make this explicit we need
    to give a rigorous definition to binary operations. We can do this with the
    language of functions\index{Function} and by using the Cartesian product
    \index{Cartesian Product} of a set $A$ with itself.
    \begin{fdefinition}{Binary Operation}{Binary_Operation}
        A \gls{binary operation} on a set $A$ is a function
        $*:A\times{A}\rightarrow{A}$.
    \end{fdefinition}
    \begin{example}
        Let $\mathbb{R}$ be the set of real numbers and $+$ denote the addition
        of two real numbers. Then $+$ is a binary operation on $\mathbb{R}$.
        Similarly, if $\cdot$ denotes the multiplication of two real numbers,
        than it two is a binary operation on $\mathbb{R}$. For division, $\div$,
        we are lacking the requirement that \textit{for all}
        $(a,b)\in\mathbb{R}^{2}$ there is a unique $c\in\mathbb{R}$ such that
        $a\div{b}=c$, since if $b=0$ our expression is undefined. That is, this
        is not a function from $\mathbb{R}^{2}$ to $\mathbb{R}$. If we consider
        all of the non-zero elements, then division is a binary operation. That
        is, division is a binary operation on $\mathbb{R}\setminus\{0\}$.
    \end{example}
    \begin{lexample}{Binary Operation on the Set of Functions}
                    {Binary_Operation_on_the_Set_of_Functions}
        If $A$ is a set, and if $\mathcal{F}(A,A)$ denotes the set of all
        functions $f:A\rightarrow{A}$, and if $\circ$ denotes function
        composition, then $\circ$ is a binary operation on $\mathcal{F}(A,A)$.
        That is, for any two functions $f,g\in\mathcal{F}(A,A)$, the composition
        $g\circ{f}:A\rightarrow{A}$ is again an element of $\mathcal{F}(A,A)$
    \end{lexample}
    Just like functions, there are three important conditions that a binary
    operation must satisfy. Given any ordered pair $(a,b)\in{A}^{2}$, it must
    be true that $*(a,b)$ is defined. This comes from the definition of a
    function on a set (Def.~\ref{def:Function}). Next, the image of $(a,b)$ must
    be unique. That is, if $*(a,b)=c$ and $*(a,b)=d$, then $c=d$. Note that this
    is not the same as requiring that $*(a,b)=*(b,a)$, and in general this is
    not true. Such binary operations are called
    \textit{commutative}\index{Commutative Operation}. Lastly, for any
    $(a,b)\in{A}^{2}$, $*(a,b)$ must be an element of $A$. That is,
    $*(a,b)\in{A}$. All of these requirements come from the definition of a
    function, so in a sense it is redundant to repeat these. In practice one
    defines a binary operation by a formula $\varphi$, and it then becomes
    necessary to show that this formula satisfies these properties before we can
    rightly call it a binary operation.
    \begin{example}
        Let $A=\mathbb{Z}_{2}$ and consider all of the binary operations on
        $\mathbb{Z}_{2}$. We can count these by constructing tables:
        \begin{table}[H]
            \centering
            \begin{tabular}{c|c}
                $(x,y)$&$*(x,y)$\\
                \hline
                $(0,0)$&0\\
                $(0,1)$&0\\
                $(1,0)$&1\\
                $(1,1)$&0
            \end{tabular}
            \label{tab:Binary_Operation_on_Z_2}
            \caption{Simple Binary Operation on $\mathbb{Z}_{2}$}
        \end{table}
        This is one such binary operation, there are 15 others. To see this,
        recall that the number of functions from a set $A$ to a set $B$, where
        both $A$ and $B$ are finite sets with $m$ and $n$ elements,
        respectively, is $n^{m}$. Since $\mathbb{Z}_{2}$ has 2 elements, and
        since a binary operation is a function
        $*\mathbb{Z}_{2}\times\mathbb{Z}_{2}\rightarrow\mathbb{Z}_{2}$, the
        total number of binary operations is $2^{(2^{2})}=2^{4}=16$. In general,
        if $A$ has $n$ elements, and if $B$ is the set of all binary operations
        on $A$, then:
        \begin{equation}
            \textrm{Card}(B)=n^{(n^2)}
        \end{equation}
    \end{example}
    \begin{example}
        Let's consider some formula that take in numbers and return numbers, and
        see if they can define operations on various sets. Suppose we have:
        \begin{equation}
            a*b=\{\,r\in\mathbb{R}\;|\;r^{2}=|ab|\,\}
        \end{equation}
        Where $|ab|$ denotes the absolute value of $a$ times $b$. If we take the
        positive square root we can write this as $a*b=\sqrt{|ab|}$. If we
        consider this formula on the rational numbers $\mathbb{Q}$, does it
        define a function? One might recall that $\sqrt{2}$ is not a rational
        number. That is, it is \textit{irrational}. Thus $1*2$ is not a rational
        number, and so $*$ is not a binary operation on $\mathbb{Q}$. It is a
        binary operation on $\mathbb{R}$, however. Suppose we change the formula
        to state:
        \begin{equation}
            a*b=\{\,r\in\mathbb{R}\:|\;r^{2}-ab=0\,\}
        \end{equation}
        and where we consider this formula to take inputs from $\mathbb{R}$.
        This is not a binary operation since it is poorly defined. That is,
        should $1*1=1$, or should $1*1=\minus{1}$? The formula is ambgious and
        thus $*$ is not a binary operation.
    \end{example}
    \begin{example}
        If we consider subtraction on the integers $\mathbb{Z}$, this is a
        binary operation. The operation is well defined and returns an integer
        for all integer inputs. If instead we consider subtraction on
        $\mathbb{N}$, this is \textit{not} a binary operation since it may take
        in non-negative integers and return a negative integer. For example,
        $1-2=\minus{1}$, and $\minus{1}\notin\mathbb{N}$. A simple fix for this
        is considering again the absolute value function. If we define
        $n*m=|n-m|$, then $*$ is indeed a binary operation on $\mathbb{N}$.
    \end{example}
    \begin{fnotation}{Binary Operation}{Binary_Operation}
        If $A$ is a set and if $*:A\times{A}\rightarrow{A}$ is a binary
        operation on $A$, for any ordered pair $(a,b)\in{A}^{2}$, the image
        of $*(a,b)$ is denoted $a*b$.
    \end{fnotation}
    It is occasionally useful to think of binary operations purely as functions,
    and so we will use function notation at these times. For the most part we
    will stick with notation defined in Not.~\ref{not:Binary_Operation}. There
    are several types of binary operations worth studying, and several key
    properties that these operations can have. One of the most fundamental is
    the existence of a \textit{unital} element, also known as an identity.
    \begin{fdefinition}{Left Unital Element}{Left_Unital_Element}
        A left unital element in a \gls{set} $A$ under a \gls{binary operation}
        $*$ on $A$ is an element $e_{L}\in{A}$ such that, for all $a\in{A}$ it
        is true that $e_{L}*a=a$.
    \end{fdefinition}
    \begin{example}
        From the definition of a left unital element
        (Def.~\ref{def:Left_Unital_Element}) it would seem natural to define a
        right unital element. The importance is to note that left and right
        unital elements need not be equal. Indeed, if $A$ is a set and $*$ is
        a binary operation, given a left identity $e_{L}$ and a right identity
        $e_{R}$ it will be true that $e_{R}=e_{L}$ and thus all left and right
        unital elements will be the same
        (see Thm.~\ref{thm:left_and_right_identity_implies_identity}). Thus to
        find counterexamples to the claim that the existence of a left unital
        element implies the existence of a right unital element we need to think
        of strange operations. Let $A=\mathbb{R}$ and let $*$ be defined by
        $a*b=b$ for all $a,b\in\mathbb{R}$. Then every element of $\mathbb{R}$
        is a left unital element. Moreover, none of the element of $\mathbb{R}$
        are right unital elements.
    \end{example}
    \begin{fdefinition}{Right Unital Element}{Right_Unital_Element}
        A right unital element of a \gls{set} $A$ under a \gls{binary operation}
        $*$ is an element $e_{R}$ such that for all $a\in{A}$ it is true that
        $a*e_{R}=a$.
    \end{fdefinition}
    \begin{example}
        Consider $\mathbb{R}$ with the operation $*$ defined by $a*b=a+b+1$.
        This operation has a right unital element, $\minus{1}$. For:
        \begin{equation}
            a*(\minus{1})=a+(\minus{1})+1=a+0=a
        \end{equation}
        And this is true for all $a\in\mathbb{R}$, so $\minus{1}$ is a right
        unital element. It turns out this is also a left unital element, and
        hence a unital element, and this can be proven if addition is known to
        be a \textit{commutative} operation.
    \end{example}
    \begin{theorem}
        \label{thm:left_and_right_identity_implies_identity}%
        If $A$ is a set, if $*$ is a binary operation on $A$, if $e_{L}$ is a
        left unital element of $A$, and if $e_{R}$ is a right unital element of
        $a$, then $e_{L}=e_{R}$.
    \end{theorem}
    \begin{proof}
        For:
        \begin{equation}
            e_{L}=e_{L}*e_{R}=e_{R}
        \end{equation}
        And thus $e_{L}=e_{R}$.
    \end{proof}
    \begin{example}
        Consider a non-empty set $A$ and the set of all functions from $A$ to
        itself, $\mathcal{F}(A,A)$. Let $\circ$ denote the binary operation of
        function composition. Then $\mathcal{F}(A,A)$ has a right identity under
        $\circ$, and a left identity. For
        the identity function $\textrm{id}_{A}$ acts as a right identity:
        \begin{equation}
            (f\circ\textrm{id}_{A})(x)
            =f\big(\textrm{id}_{A}(x)\big)
            =f(x)
        \end{equation}
        And thus $\textrm{id}_{A}$ is a right identity. By
        Thm.~\ref{thm:left_and_right_identity_implies_identity}, any left
        identity must also be a right identity, and so the likely candidate to
        check is $\textrm{id}_{A}$. And indeed we have:
        \begin{equation}
            (\textrm{id}_{A}\circ{f})(x)
            =\textrm{id}_{A}\big(f(x)\big)
            =f(x)
        \end{equation}
        And thus $\textrm{id}_{A}$ is a left identity as well.
    \end{example}
    \begin{fdefinition}{Unital Element}{Unital_Element}
        A \gls{unital element} of a \gls{set} $A$ under a \gls{binary operation}
        $*$ is an element $e\in{A}$ that is both a right unital element and a
        left unital element.
    \end{fdefinition}
    \begin{example}
        Let $\mathbb{R}$ be the set of real numbers and let $+$ be the usual
        notion of addition. Then 0 is a unital element of $\mathbb{R}$ with
        respect to this operation. That is, for any real number $x$ we have
        $x+0=0+x=x$. For multiplication the unital element is 1. This is because
        $1\cdot{x}=x\cdot{1}=x$. Subtraction has a right unital element, and
        again it is 0 since $x-0=x$, but no left identity. To see this, suppose
        $e-x=x$ for all $x$. Applying some algebra we have that $e=2x$, meaning
        there is no constant $e\in\mathbb{R}$ such that for all $x$, $e-x=x$.
        Since subtraction has no left unital element, it has no unital element
        either.
    \end{example}
    \begin{theorem}
        \label{thm:Unital_Elements_are_Unique}%
        If $A$ is a set, if $*$ is a binary operation on $A$, and if $e$ and
        $e'$ are unital elements of $A$, then $e=e'$
    \end{theorem}
    \begin{proof}
        For:
        \begin{equation}
            e=e*e'=e'
        \end{equation}
        And thus by transitivity, $e=e'$.
    \end{proof}
    The next thing to discuss is that of inverses. There are five types, but in
    practice only one of these is discussed.
    \begin{fdefinition}{Weak Right Inverse}{Weak_Right_Inverse}
        A weak right inverse of an element $a$ in a \gls{set} $A$ under a
        \gls{binary operation} $*$ on $A$ is an element $b\in{A}$ such that
        $a*b$ is a right unital element.
    \end{fdefinition}
    This definition will not recieve much use until we discuss
    groups\index{Group}. A group is a set with a binary operation $*$ that has
    a unital element, inverse elements, and is associatied (to be defined soon).
    As it turns out these conditions are stronger than necessary and it suffices
    to check that there are weak right inverses and a right unital element. The
    next thing to define is right inverses.
    \begin{fdefinition}{Right Inverse}{Right_Inverse}
        A right inverse of an element $a$ in a \gls{set} $A$ under a
        \gls{binary operation} is an element $b\in{A}$ such that $a*b$ is a
        \gls{unital element}.
    \end{fdefinition}
    Here, we've simply strengthened the requirement that $a*b$ not only be a
    right unital element, but also a left unital element as well. A right
    inverse is therefore necessarily a weak right inverse.
    \begin{fdefinition}{Weakly Left Invertible}{Weakly_Left_Invertible}
        A weakly left invertible element of a \gls{set} $A$ under a
        \gls{binary operation} $*$ is an element $a\in{A}$ such that there
        exists a $b\in{A}$ such that $b*a$ is a left unital element.
    \end{fdefinition}
    \begin{fdefinition}{Left Invertible Element}{Left_Inverse}
        A left invertible element of a \gls{set} $A$ under a
        \gls{binary operation} is an element $a\in{A}$ such that there exists a
        $b\in{A}$ such that $b*a$ is a \gls{unital element}.
    \end{fdefinition}
    \begin{fdefinition}{Invertible Element}{Invertible_Element}
        An invertible element of a a \gls{set} $A$ under a
        \gls{binary operation} is an element $a\in{A}$ that is both
        left invertible and right invertible.
    \end{fdefinition}
    \begin{fdefinition}{Commutative Operation}{Commutative_Operation}
        A \gls{commutative operation} on a \gls{set} $A$ is a
        \gls{binary operation} $*$ such that for all $(a,b)\in{A}^{2}$ it is
        true that $a*b=b*a$.
    \end{fdefinition}
    \begin{fdefinition}{Associative Operation}{Associative_Operation}
        A \gls{associative operation} on a \gls{set} $A$ is a
        \gls{binary operation} $*$ such that, for all $a,b,c\in{A}$ it is true
        that $a*(b*c)=(a*b)*c$.
    \end{fdefinition}
    \begin{example}
        Consider a finite set $A$ and consider the set of all functions from
        $\mathbb{Z}_{n}$ to $A$. That is, $\mathcal{F}_{n}(\mathbb{Z}_{n},A)$.
        Define $A[x]$ by:
        \begin{equation}
            \mathcal{F}=\bigcup_{n\in\mathbb{N}}\mathcal{F}_{n}
        \end{equation}
        That is, the set of all finite sequences in $A$. We can form an
        associative operation on this set by defining the concatenation
        operation. Given $f,g\in\mathcal{F}$, suppose
        $f\in\mathcal{F}(\mathbb{Z}_{m},A)$ and
        $g\in\mathcal{F}(\mathbb{Z}_{n},A)$. We define
        $f*g\in\mathcal{F}(\mathbb{Z}_{m+n},A)$ as follows:
        \begin{equation}
            (f*g)(k)=
            \begin{cases}
                f(k),&k\in\mathbb{Z}_{m}\\
                g(k-m),&k\in\mathbb{Z}_{m+n}\textrm{ and }k\geq{m}
            \end{cases}
        \end{equation}
        That is, given two sequences $f_{0},f_{1},\dots,f_{m-1}$ and
        $g_{0},g_{1},\dots,g_{n-1}$, we concatenate them to form the sequence
        $f_{0},\dots,f_{m-1},g_{0},\dots,g_{n-1}$. This operation is associative
        since if $f,g,h\in\mathcal{F}$, then:
        \begin{subequations}
            \begin{align}
                f*(g*h)&=(f_{0},f_{1},\dots,f_{m-1})
                    *(g_{0},g_{1},\dots,g_{m-1},h_{0},h_{1},\dots,h_{r-1})\\
                &=f_{0},f_{1},\dots,f_{m-1},
                    g_{0},g_{1},\dots,g_{m-1},h_{0},h_{1},\dots,h_{r-1}\\
                &=(f_{0},f_{1},\dots,f_{m-1},
                    g_{0},g_{1},\dots,g_{m-1})*(h_{0},h_{1},\dots,h_{r-1})\\
                &=(f*g)*h
            \end{align}
        \end{subequations}
        If $A$ has more than one point than $*$ is not commutative. For let
        $f,g:\mathbb{Z}_{1}\rightarrow{A}$ be defined by $f(0)=a$ and $g(0)=b$,
        respectively. Then $f*g=a,b$ but $g*f=b,a$, and thus $f*g\ne{g}*f$.
        There is, however, an identity. Consider a $\mathbb{Z}_{0}$, which is
        the empty set. Any function from $\mathbb{Z}_{0}$ to $A$ is therefore
        the \textit{empty sequence}. If we concatenate $f$ with the empty
        sequence we get back $f$, and this then acts as our unital element.
    \end{example}


        \section{Boolean Algebras}
    We now attempt to make set theory more algebraic. We wish to model
    as an object the triple $(\mathcal{P}(X),\cup,\cap)$, where
    $\mathcal{P}(X)$ is the \gls{power set} of some set, and $\cup$ and
    $\cap$ and union and intersection, respectively. These can be seen as
    binary operations on $\mathcal{P}(X)$. We take a few of the properties
    of this structure and state them as the definition for our new object:
    \textit{Boolean Algebras}.
    \begin{fdefinition}{Complement in a Boolean Algebra}
                       {Complement}
        A complement of a \gls{set} $A$ with respect to two
        \glspl{binary operation} $*$ and $\circ$ is an element
        $a^{\minus{1}}\in{A}$ such that:
        \begin{equation*}
            a*a^{\minus{1}}=a^{\minus{1}}*a=e_{\circ}
            \quad\quad
            a\circ{a}^{\minus{1}}=a^{\minus{1}}\circ{a}=e_{*}
        \end{equation*}
        Where $e_{\circ}$ and $e_{*}$ are the \glspl{unital element} of
        $\circ$ and $*$, respectively.
    \end{fdefinition}
    \begin{fdefinition}{Boolean Algebras}{Boolean_Algebra}
        A Boolean algebra is a set $A$ with two
        \glslink{commutative operation}{commutative} \glspl{binary operation}
        $\circ$ and $*$ on $A$ with \glspl{unital element} $e_{*}$ and
        $e_{\circ}$, respectively, such that:
        \begin{itemize}
            \item[1.)]  $\circ$ \glslink{distributive operation}{distributes}
                        over $*$ and $*$ distributes over $\circ$.
            \item[2.)]  For all $a\in{A}$ there is a complement of $a$.
        \end{itemize}
    \end{fdefinition}
    The second property is known as the complement property and it is very
    different from the notion of inverses. An inverse of an element $a$ with
    respective to an operation $\cdot$ is such that $a\cdot{b}$ is a unital
    element with respect to the operation $\cdot$. A complement produces a
    unital element with respect the the \textit{other} operation. That is,
    $a*a^{\minus{1}}$ is a unital element of $\circ$, and not $*$. Similarly,
    $a\circ{a}^{\minus{1}}$ is a unital element of $*$ and not $\circ$.
    \begin{theorem}
        If $(A,\circ,*)$ is a Boolean algebra and if $b\in{X}$ is a
        unital element of $\circ$, then $b=e_{\circ}$.
    \end{theorem}
    \begin{proof}
        For unital elements are unique
        (Thm.~\ref{thm:Unital_Elements_are_Unique}), and therefore
        $b=e_{\circ}$.
    \end{proof}
    \begin{theorem}
        If $(X,\circ,*)$ is a Boolean algebra and if $b\in{X}$ is a
        unital element of $*$, then $b=e_{*}$.
    \end{theorem}
    \begin{proof}
        For unital elements are unique
        (Thm.~\ref{thm:Unital_Elements_are_Unique}), and therefore $b=e_{*}$.
    \end{proof}
    \begin{theorem}
        \label{thm:Bool_Alg_Boundary_of_Circ}%
        If $(X,\circ,*)$ is a Boolean algebra, if $e_{*}$ is the unital
        element of $*$, and if $a\in{X}$, then $a\circ{e}_{*}=e_{*}$.
    \end{theorem}
    \begin{proof}
        For if $a\in{X}$ then there is an $a^{\minus{1}}\in{X}$ such that
        $a\circ{a}^{\minus{1}}=e_{*}$ (Def.~\ref{def:Boolean_Algebra}).
        But then:
        \par\vspace{-2.5ex}
        \begin{minipage}[t]{0.51\textwidth}
            \centering
            \begin{align}
                e_{*}&=a\circ{a}^{\minus{1}}
                \tag{Complement}\\
                &=a\circ(a^{\minus{1}}*e_{*})
                \tag{Identity}\\
                &=(a\circ{a}^{\minus{1}})*(a\circ{e}_{*})
                \tag{Distributivity}
            \end{align}
        \end{minipage}
        \hfill
        \begin{minipage}[t]{0.47\textwidth}
            \centering
            \begin{align}
                &=e_{*}*(a\circ{e}_{*})
                \tag{Complement}\\
                &=a\circ{e}_{*}
                \tag{Identity}
            \end{align}
        \end{minipage}
        \par\vspace{2.5ex}
        And therefore $e_{*}=a\circ{e}_{*}$.
    \end{proof}
    This theorem is equivalent to the notion that a Boolean algebra is a
    bounded lattice\index{Bounded Lattice} and the $e_{*}$ is a boundary. The
    theorem holds for $\circ$ as well.
    \begin{theorem}
        \label{thm:Bool_Alg_Boundary_of_Star}%
        If $(X,\circ,*)$ is a Boolean algebra, if $e_{*}$ is the unital
        element of $*$, and if $a\in{X}$, then $a*{e}_{\circ}=e_{\circ}$.
    \end{theorem}
    \begin{proof}
        For if $a\in{X}$ then there is an $a^{\minus{1}}\in{X}$ such that
        $a*{a}^{\minus{1}}=e_{\circ}$ (Def.~\ref{def:Boolean_Algebra}).
        But then:
        \par\vspace{-2.5ex}
        \begin{minipage}[t]{0.51\textwidth}
            \centering
            \begin{align}
                e_{\circ}&=a*{a}^{\minus{1}}
                \tag{Complement}\\
                &=a*(a^{\minus{1}}\circ{e}_{\circ})
                \tag{Identity}\\
                &=(a*a^{\minus{1}})\circ(a*e_{*})
                \tag{Distributivity}
            \end{align}
        \end{minipage}
        \hfill
        \begin{minipage}[t]{0.47\textwidth}
            \centering
            \begin{align}
                &=e_{\circ}\circ(a*e_{\circ})
                \tag{Complement}\\
                &=a*e_{\circ}
                \tag{Identity}
            \end{align}
        \end{minipage}
        \par\vspace{2.5ex}
        And therefore $e_{\circ}=a*e_{\circ}$.
    \end{proof}
    \begin{theorem}
        If $(A,\circ,*)$ is a Boolean algebra, if $e_{\circ}$ and $e_{*}$
        are the unital elements of $\circ$ and $*$, respectively, then
        $e_{\circ}$ is the complement of $e_{*}$ and $e_{*}$ is the
        complement of $e_{\circ}$.
    \end{theorem}
    \begin{proof}
        From identity:
        \par\vspace{-2.5ex}
        \begin{subequations}
            \begin{minipage}[b]{0.49\textwidth}
                \centering
                \begin{equation}
                    e_{\circ}\circ{e}_{*}=e_{*}\circ{e}_{\circ}=e_{*}
                    \tag{Identity}
                \end{equation}
            \end{minipage}
            \hfill
            \begin{minipage}[b]{0.49\textwidth}
                \centering
                \begin{equation}
                    e_{\circ}*{e}_{*}=e_{*}*e_{\circ}=e_{\circ}
                    \tag{Identity}
                \end{equation}
            \end{minipage}
        \end{subequations}
        \par\vspace{2.5ex}
        Thus, $e_{*}$ is a complement of $e_{\circ}$ and $e_{\circ}$ is a
        complement of $e_{*}$ (Def.~\ref{def:Complement}).
    \end{proof}
    Every element of a Boolean algebra is idempotent with respect to both
    operations.
    \begin{theorem}
        \label{thm:Bool_Alg_Idempotent_of_Star}%
        If $(A,\circ,*)$ is a Boolean algebra and if $a\in{A}$, then $a*a=a$.
    \end{theorem}
    \begin{proof}
        For:
        \par\vspace{-2.5ex}
        \begin{subequations}
            \begin{minipage}[t]{0.49\textwidth}
                \centering
                \begin{align}
                    a&=a*e_{*}
                    \tag{Identity}\\
                    &=a*(a\circ{a}^{\minus{1}})
                    \tag{Complement}\\
                    &=(a*a)\circ(a*a^{\minus{1}})
                    \tag{Distributivity}
                \end{align}
            \end{minipage}
            \hfill
            \begin{minipage}[t]{0.49\textwidth}
                \centering
                \begin{align}
                    &=(a*a)\circ{e}_{\circ}
                    \tag{Complement}\\
                    &=a*a\tag{Identity}
                \end{align}
            \end{minipage}
        \end{subequations}
        \par\vspace{2.5ex}
        And therefore $a=a*a$.
    \end{proof}
    \begin{theorem}
        \label{thm:Bool_Alg_Idempotent_of_Circ}%
        If $(A,\circ,*)$ is a Boolean algebra and if $a\in{A}$, then
        $a\circ{a}=a$.
    \end{theorem}
    \begin{proof}
        For:
        \par\vspace{-2.5ex}
        \begin{subequations}
            \begin{minipage}[t]{0.49\textwidth}
                \centering
                \begin{align}
                    a&=a\circ{e}_{\circ}
                    \tag{Identity}\\
                    &=a\circ(a*a^{\minus{1}})
                    \tag{Complement}\\
                    &=(a\circ{a})*(a\circ{a}^{\minus{1}})
                    \tag{Distributivity}
                \end{align}
            \end{minipage}
            \hfill
            \begin{minipage}[t]{0.49\textwidth}
                \centering
                \begin{align}
                    &=(a\circ{a})*{e}_{*}
                    \tag{Complement}\\
                    &=a\circ{a}\tag{Identity}
                \end{align}
            \end{minipage}
        \end{subequations}
        \par\vspace{2.5ex}
        And therefore $a=a\circ{a}$.
    \end{proof}
    \begin{theorem}
        \label{thm:Bool_Alg_aob_equal_a_acb_equal_a_implies_a_equal_b}%
        If $(A,\circ,*)$ is a Boolean algebra, if $a,b\in{A}$, if $a\circ{b}=a$,
        and if $a*b=a$, then $b=a$.
    \end{theorem}
    \begin{proof}
        For:
        \par
        \begin{minipage}[b]{0.49\textwidth}
            \centering
            \begin{align}
                b&=b*e_{*}\tag{Identity}\\
                 &=b*(a\circ a^{-1})\tag{Complement}\\
                 &=(b*a)\circ(b* a^{-1})\tag{Distributivity}\\
                 &=a\circ (b* a^{-1})\tag{Hypothesis}
            \end{align}
        \end{minipage}
        \hfill
        \begin{minipage}[b]{0.49\textwidth}
            \centering
            \begin{align}
                &=(a\circ b)*(a\circ a^{-1})\tag{Distributivity}\\
                &=(a\circ{b})*e_{*}\tag{Complement}\\
                &=a\circ{b}\tag{Identity}\\
                &=a\tag{Hypothesis}
            \end{align}
        \end{minipage}
        \par\vspace{2.5ex}
        And therefore $a=b$.
    \end{proof}
    \begin{theorem}
        If $(A,\circ,*)$ is a Boolean algebra, if $a\in{A}$ is such that
        $a=a^{\minus{1}}$, then $a=e_{\circ}=e_{*}$.
    \end{theorem}
    \begin{proof}
        For let $a\in{A}$ and let $a=a^{\minus{1}}$. Then by
        Thm.~\ref{thm:Bool_Alg_Idempotent_of_Star}:
        \begin{equation}
            a=a*a=a*a^{-1}=e_{\circ}
        \end{equation}
        Similarly, $a=e_{*}$.
    \end{proof}
    \begin{theorem}
        If $(A,\circ,*)$ is a Boolean algebra, if $a\in{A}$, and if $b,c\in{A}$
        are complements of $a$, then $b=c$.
    \end{theorem}
    \begin{proof}
        For:
        \par
        \begin{minipage}[t]{0.49\textwidth}
            \centering
            \begin{align}
                b&=b*e
                \tag{Identity}\\
                &=b*(a\circ{c})
                \tag{Complement}\\
                &=(b*a)\circ(b*c)
                \tag{Distributivity}\\
                &=e_{\circ}\circ(b*c)
                \tag{Complement}\\
                &=(c*a)\circ(b*c)
                \tag{Complement}
            \end{align}
        \end{minipage}
        \hfill
        \begin{minipage}[t]{0.49\textwidth}
            \centering
            \begin{align}
                &=(c*a)\circ(c*b)
                \tag{Commutativity}\\
                &=c\circ(a*b)
                \tag{Distributivity}\\
                &=c\circ{e}_{\circ}
                \tag{Complement}\\
                &=c
                \tag{Identity}
            \end{align}
        \end{minipage}
        \par\vspace{2.5ex}
        Therefore, $b=c$.
    \end{proof}
    \begin{theorem}
        If $(A,\circ,*)$ is a Boolean algebra and if $a\in{A}$,
        then $(a^{\minus{1}})^{\minus{1}}=a$.
    \end{theorem}
    \begin{proof}
        For:
        \begin{align}
            a&=a*e_{*}
            \tag{Identity}\\
            &=a*\big(a^{\minus{1}}\circ(a^{\minus{1}})^{\minus{1}})
            \tag{Complement}\\
            &=(a\circ{a}^{\minus{1}})*
                \big(a\circ(a^{\minus{1}})^{\minus{1}}\big)
            \tag{Distributivity}\\
            &=e_{*}*\big(a\circ(a^{\minus{1}})^{\minus{1}}\big)
            \tag{Complement}\\
            &=a\circ(a^{\minus{1}})^{\minus{1}}
            \tag{Identity}
        \end{align}
        And similarly $a*(a^{\minus{1}})^{\minus{1}}=a$. But if $a*b=a$ and
        $a\circ{b}=a$, then $a=b$
        (Thm.~\ref{thm:Bool_Alg_aob_equal_a_acb_equal_a_implies_a_equal_b}).
        Therefore, $a=(a^{\minus{1}})^{\minus{1}}$.
    \end{proof}
    \begin{ltheorem}{Absorption Laws}{Absorption_Law}
        If $(A,\circ,*)$ is a Boolean algebra, if $a\in{A}$ and if $b\in{A}$,
        then $a*(a\circ{b})=a$ and $a\circ(a*{b})=a$.
    \end{ltheorem}
    \begin{proof}
        For:
        \par
        \begin{minipage}[t]{0.59\textwidth}
            \centering
            \begin{align}
                a*(a\circ{b})&=(a*e_{*})\circ(a*b)
                \tag{Identity}\\
                &=a*(e_{*}\circ{b})
                \tag{Distributivity}
            \end{align}
        \end{minipage}
        \hfill
        \begin{minipage}[t]{0.39\textwidth}
            \centering
            \begin{align}
                &=a*e_{*}
                \tag{Thm.~\ref{thm:Bool_Alg_Boundary_of_Circ}}\\
                &=a
                \tag{Identity}
            \end{align}
        \end{minipage}
        \par\vspace{2.5ex}
        And therefore $a*(a\circ{b})=a$. Similarly:
        \par
        \begin{minipage}[t]{0.59\textwidth}
            \centering
            \begin{align}
                a\circ(a*b)&=(a\circ{e}_{\circ})*(a\circ{b})
                \tag{Identity}\\
                &=a\circ(e_{\circ}*b)
                \tag{Distributivity}
            \end{align}
        \end{minipage}
        \hfill
        \begin{minipage}[t]{0.39\textwidth}
            \begin{align}
                &=a\circ{e}_{\circ}
                \tag{Thm.~\ref{thm:Bool_Alg_Boundary_of_Star}}\\
                &=a
                \tag{Identity}
            \end{align}
        \end{minipage}
        \par\vspace{2.5ex}
        And therefore $a\circ(a*b)=a$.
    \end{proof}
    We can weaken the hypothesis of
    Thm.~\ref{thm:Bool_Alg_aob_equal_a_acb_equal_a_implies_a_equal_b} to obtain
    a more general result.
    \begin{theorem}
        \label{thm:Bool_Alg_aob_equal_acb_implies_a_equal_b}
        If $(A,\circ,*)$ is a Boolean algebra, if $a,b\in{A}$, and if
        $a*b=a\circ{b}$, then $a=b$.
    \end{theorem}
    \begin{proof}
        For:
        \begin{align}
            a&=a*e_{*}
            \tag{Identity}\\
            &=a*(b\circ{b}^{\minus{1}})
            \tag{Complement}\\
            &=(a*b)\circ(a*b^{\minus{1}})
            \tag{Distributivity}\\
            &=(a\circ{b})\circ(a*b^{\minus{1}})
            \tag{Hypothesis}\\
            &=((a\circ{b})\circ{a})*
                \big((a\circ{b})\circ{b}^{\minus{1}}\big)
            \tag{Distributivity}\\
            &=\big((a\circ{b})\circ{a}\big)*
                \big(a\circ(b\circ{b}^{\minus{1}})\big)
            \tag{Associativity}\\
            &=\big((a\circ{b})\circ{a}\big)*(a\circ{e}_{*})
            \tag{Complement}\\
            &=\big((a\circ{b})\circ{a}\big)*e_{*}
            \tag{Thm.~\ref{thm:Bool_Alg_Boundary_of_Circ}}\\
            &=(a\circ{b})\circ{a}
            \tag{Identity}\\
            &=(a\circ{a})\circ{b}
            \tag{Associativity and Commutativity}\\
            &=a\circ{b}
            \tag{Thm~\ref{thm:Bool_Alg_Idempotent_of_Circ}}
        \end{align}
        Thus $a=a\circ{b}$. But $a\circ{b}=a*{b}$, and so $a=a*{b}$. By
        Thm.~\ref{thm:Bool_Alg_aob_equal_a_acb_equal_a_implies_a_equal_b},
        $a=b$.
    \end{proof}
            \begin{definition} For $a\in S$, an inverse, or normal inverse, of the First Operation is an element $b\in S$ such that $a\circ b=e_{\circ}$. An inverse of the Second Operation is similarly defined. The normal inverses are denoted $a^{*}$ and $a^{\circ}$.
            \end{definition}
            \begin{theorem} If $a\in S$ has a normal inverse for either operation, than it is unique.
            \end{theorem}
            \begin{proof} For suppose not. Let $a\in S$ have a normal inverse for the First Operation. That is, there is an $a^{\circ}\in S$ such that $a\circ a^{\circ}=e_{\circ}$ and let $a'^{\circ}$ be a second normal inverse not equal to the first. But then $a^{\circ}=a^{\circ}\circ e_{\circ}=a^{\circ}\circ (a\circ a'^{\circ})$ and from associativity we have $a^{\circ}=(a^{\circ}\circ a)\circ a'^{\circ}=a'^{\circ}$. Thus, the normal inverse is unique. Similarly if there is an inverse for the Second Operation
            \end{proof}
            \begin{theorem} If $a\in S$ has a normal inverse, say $a'$, for one operation, then $a^{-1}=a'^{-1}$.
            \end{theorem}
            \begin{proof} For let $a\in S$ have a normal inverse $a'$ for the First Operation. That is, $a\circ a' = e_{\circ}$. But $a' \circ a'^{-1}=e_{*}$, and from theorem 1.3 $a\circ e_{*}=e_{*}$. So $a\circ (a' \circ a'^{-1})=e_{*}$. And from theorem 1.4, $a\circ a=a$, so we have $(a\circ a)\circ (a'\circ a'^{-1}=a\circ (a\circ a')\circ a'^{-1}=a\circ a'^{-1}=e_{*}$. But $a\circ a^{-1}=e_{\circ}$. And pseudo-inverses are unique. Thus, $a^{-1}=a'^{-1}$. 
            \end{proof}
            \begin{theorem} The identities have normal inverses for their respective operations.
            \end{theorem}
            \begin{proof} As normal inverses are unique, it suffices to find inverses for both identities. But $e_{\circ}\circ e_{\circ}=e_{\circ}$, so $e_{\circ}$ is its own inverse for the First Operation. Similarly, $e_{*}*e_{*}=e_{*}$.
            \end{proof}
            \begin{theorem} \textbf{(The Not-A-Field Theorem)} Only the identities have normal inverses.
            \end{theorem}
            \begin{proof} For suppose not. Suppose $a\in S,\ a\ne e_{\circ},\ a\ne e_{*}$ and a has an inverse for the First Operation. That is $\exists a^{\circ}\in S|\ a\circ a^{\circ}=e_{\circ}$. But by theorem 1.4, $a\circ a^{\circ}=(a\circ a)\circ a^{\circ}$. By associativity, we have $e_{\circ}=a\circ a^{\circ} = a\circ (a\circ a^{\circ})=a\circ e_{\circ}=a$. Thus, $a=e_{\circ}$. But by hypothesis, $a\ne e_{\circ}$. Thus, there is no inverse for $a$. Similarly, a has no inverse for the Second Operation.
            \end{proof}
            \begin{theorem}
            There exist pseudo-fields with only one element.
            \end{theorem}
            \begin{proof}
            For let $e_{\circ} = e_{*}$, and let no other elements be in the set. 
            \end{proof}
            \begin{theorem}
            A pseud-field has one element if and only if $e_{\circ} = e_{*}$.
            \end{theorem}
            \begin{proof}
            For suppose there is another element $a \ne e_{\circ}$. But then $a \circ e_{\circ} = a$, but also $a \circ e_{\circ} = a \circ e_{*} = e_{*}$. So $a = e_{*}$. If there is only one element, then clearly $e_{\circ} = e_{*}$ as otherwise there would be two elements.
            \end{proof}
            \begin{definition} A generating set on a pseudo-field is a subset $g_S \subset S$ such that every element of $S$ can be written as a finite combination of elements in $g_S$ using $\circ$ or $*$.
            \end{definition}
            \begin{theorem}
            The number of elements in a finite pseudo-field is a power of 2.
            \end{theorem}
            \begin{proof}
            Consider the set of all generators $g_S$ on $S$. Clearly for all such generators, $1\leq |g_S|\leq |S|$. Let $G$ be the smallest generator, such that $|G| \leq |g_S|$ for any other given generator. 
            \end{proof}
        \section{Sequences and Matrices}
    Matrices\index{Matrix} are the fundamental object studied in
    linear algebra\index{Linear Algebra}, and are used in the study of general
    algebra as well. To discuss the more interesting properties requires some
    notion of arithmetic that we do not yet posses. In particular, matrices are
    most interesting when there is an underlying \textit{ring}\index{Ring}
    structure. For now we simply introduce the set theoretic definition of a
    matrix, relate this to the familiar \textit{grid of numbers} definition, and
    provide examples.
    \begin{fdefinition}{Matrix}
        An $n\times{m}$, $n,m\in\mathbb{N}$, matrix on a set $X$ is a function
        $A:\mathbb{Z}_{n}\times\mathbb{Z}_{m}\rightarrow{X}$.
    \end{fdefinition}
    \chapter{The Real Numbers}
        Now that the notion of function has been developed, and most of ZFC laid
        out, we take the time to discuss the next axiom on the list:
        \textit{the axiom of infinity}\index{Axiom!of Infinity}. This axiom,
        while controversial, posits the existence of an infinite set and allows
        us to construct the natural numbers and the real numbers in the
        framework of set theory.
    \chapter{Cardinality}
        \section{Cardinality}
    As mentioned before, there is a way to discuss the size of sets in a manner
    that allows one to be precise when stating \textit{the set A is larger than}
    \textit{the set B}. If two sets are small enough, we can simply count out
    the number of elements contained in each and compare sets this way. For an
    infinite set it doesn't make sense to talk about the \textit{number} of
    elements, but we can still specify what it means two sets to have the same
    size. Sets $A$ and $B$ are equivalent if there exists a bijection
    $f:A\rightarrow{B}$. We then say that $A$ and $B$ have the same cardinality,
    and we denote this by $\textrm{Card}(A)=\textrm{Card}(B)$. A finite set is a
    set $A$ such that there is a bijection between $A$ and $\mathbb{Z}_{n}$, for
    some $n\in\mathbb{N}$. We can then view the elements of $A$ as
    $A=\{a_{1},\hdots,a_{n}\}$. A countable set is a set $A$ such that there is
    a bijection between $A$ and $\mathbb{N}$. These notions were first developed
    by Georg Cantor\index{Cantor, Georg}, and one natural question would be to
    ask if $\textrm{Card}(\mathbb{N})=\textrm{Card}(\mathbb{N})$? What about
    $\textrm{Card}(\mathbb{N})$ and $\textrm{Card}(\mathbb{R})$? This chapter
    aims to answer these questions, and develop the cardinal numbers along the
    way.
    \subsection{Equivalent Sets}
        \begin{fdefinition}{Equivalent Sets}{Equivalent_Sets}
            Equivalent sets are \glspl{set} $A$ and $B$ such that there exists a
            \gls{bijective function} $f:A\rightarrow{B}$.
        \end{fdefinition}
        The notion of equivalent sets defines an equivalence relation on sets.
        That is, the notion is reflexive, symmetric, and transitive.
        \begin{theorem}
            If $A$ is a set, then $A$ is equivalent to $A$.
        \end{theorem}
        \begin{proof}
            For the identity mapping $\textrm{id}_{A}:A\rightarrow{A}$ is a
            bijective function.
        \end{proof}
        \begin{theorem}
            If $A$ and $B$ are sets and if $A$ is equivalent to $B$, then $B$ is
            equivalent to $A$.
        \end{theorem}
        \begin{proof}
            For if $A$ is equivalent to $B$, then there is a bijection
            $f:A\rightarrow{B}$. But if $f$ is a bijection, then the inverse
            function $f^{-1}:B\rightarrow{A}$ is well-defined and is a
            bijection. Thus $B$ is equivalent to $A$.
        \end{proof}
        \begin{theorem}
            If $A$, $B$, and $C$ are sets, if $A$ is equivalent to $B$, and if
            $B$ is equivalent to $C$, then $A$ is equivalent to $C$.
        \end{theorem}
        \begin{proof}
            For if $A$ is equivalent to $B$, then there is a bijection
            $f:A\rightarrow{B}$. But if $B$ is equivalent ot $C$, then there is a
            bijection $g:B\rightarrow{C}$. But then $g\circ{f}:A\rightarrow{C}$ is a
            bijection, and thus $A$ and $C$ are equivalent.
        \end{proof}
        A bijection is a function that is both injective and surjective. Thus, two
        equivalent sets can be put into a one-to-one correspondence and can be said
        to have the same size. We then say that $A$ and $B$ have the same
        cardinality. The notation is written as $|A|=|B|$ or $\Card(A)=\Card(B)$.
        Cardinality splits sets into one of three categories.
        \begin{ldefinition}{Finite Sets}{Finite_Sets}
            A finite set is a set $A$ such that there exists
            an $n\in\mathbb{N}$ such that there is a
            bijection $f:\mathbb{Z}_{n}\rightarrow{A}$, or
            such that $A=\emptyset$.
        \end{ldefinition}
        Sets that are not finite are called infinite. There
        are two types of infinite sets. Let $\mathbb{N}$
        denote the set of positive integers, or
        \textit{natural} numbers.
        \begin{ldefinition}{Countably Infinite Sets}
              {Countably_Infinite}
            A countably infinite set is a set $A$ such that
            is a bijection $f:\mathbb{N}\rightarrow{A}$.
        \end{ldefinition}
        Combining the notions of finite sets and countably
        infinite sets, we get the notion of
        \textit{countable} sets.
        \begin{ldefinition}{Countable Sets}
              {Countable_Sets}
            A countable set is a set $A$ such that $A$ is
            either finite or countably infinite.
        \end{ldefinition}
        Countable sets are also called \textit{listable}.
        This is because if $A$ is a countably infinite set,
        and if $a:\mathbb{N}\rightarrow{A}$ is a bijection,
        we can write $A$ as:
        \begin{equation}
            A=\{\;a_{n}\,:\,n\in\mathbb{N}\;\}
            =\{\,a_{1},\,a_{2},\,\dots,\,a_{k},\,\dots\,\}
        \end{equation}
        If $A$ is finite, and if
        $a:\mathbb{Z}_{n}\rightarrow{A}$ is a
        bijection, then we can write:
        \begin{equation}
            A=\{\;a_{n}\,:\,n\in\mathbb{Z}_{n}\;\}
             =\{\,a_{1},\,\dots,\,a_{n}\,\}
        \end{equation}
        Recall that functions $a:\mathbb{N}\rightarrow{A}$
        are called \textit{sequences}, and the image of
        $n\in\mathbb{N}$ is written $a_{n}$, rather than
        $a(n)$.
        \begin{example}
            The set of all positive even integers is
            countable. For let $\mathbb{N}_{e}$ be the
            set of all even integers and define
            $f:\mathbb{N}\rightarrow\mathbb{N}_{e}$ be
            $f(n)=2n$ for all $n\in\mathbb{N}$. This is
            a bijection, and thus $\mathbb{N}_{e}$ is
            countable. The set of all odd positive integers
            is countable, as shown by letting
            $f(n)=2n-1$. Even though the set of even
            integers may seem ``smaller,'' than the set of
            all integers, they are equivalent. The set of
            all integers $\mathbb{Z}$ is also countable.
            For let $f:\mathbb{N}\rightarrow\mathbb{Z}$
            be defined as:
            \begin{equation}
                f(n)=
                \begin{cases}
                    \frac{1}{2}(n-1),&n\textrm{ odd}\\
                    -\frac{n}{2},&n\textrm{ even}
                \end{cases}
            \end{equation}
        \end{example}
        Any set that is infinite (Not finite) contains a
        countable subset. Thus, $\mathbb{N}$ can be
        considered as the \textit{smallest} infinite set.
        \begin{theorem}
            If $A$ is an infinite set, then there exists
            $S\subseteq{A}$ such that $S$ is countable.
        \end{theorem}
        \begin{proof}
            For as $A$ is infinite, for all $n\in\mathbb{N}$
            there exists a set $B\subseteq{A}$ such that
            $|B|=n$. For all $n\in\mathbb{N}$,
            define the following:
            \begin{equation}
                \mathcal{S}_{n}=\{B\subseteq{A}:|B|=n\}
            \end{equation}
            Let $\mathcal{S}$ be defined as:
            \begin{equation}
                \mathcal{S}=\{\mathcal{S}_{n}:n\in\mathbb{N}\}
            \end{equation}
            Then $\mathcal{S}$ is countable, for
            $a:\mathbb{N}\rightarrow\mathcal{S}$ defined
            by $a_{n}=\mathcal{S}_{n}$ is a bijection.
            By the axiom of choice, there is a function:
            \begin{equation}
                \alpha:\mathcal{S}\rightarrow
                \bigcup_{n=1}^{\infty}\mathcal{S}_{n}
            \end{equation}
            Such that, for all $x\in\mathcal{S}$,
            $\alpha(x)\in{x}$. But then, for all
            $x\in\mathcal{S}$, $\alpha(x)$ is a subset
            of $A$. But for all $x\in\mathcal{S}$, there
            is an $n\in\mathbb{N}$ such that
            $a_{n}=x$. Thus, let $S$ be the following:
            \begin{equation}
                S=\bigcup_{n=1}^{\infty}\alpha(a_{n})
            \end{equation}
        \end{proof}
        In the absence of the requirement that $a\cap{b}=\emptyset$ for all
        pairs in $\mathcal{U}$, we still have that the union is, at most,
        countable. The mapping we found would be a \textit{surjection}, rather
        than a bijection. The union is then either finite or countable. The
        Cantor-Schr\"{o}der-Bernstein Theorem can often be used to help identify
        the size of a set. This says that if $A$ and $B$ are sets such that
        there exists a surjective function $f:A\rightarrow{B}$ and a surjective
        function $g:B\rightarrow{A}$, then there is a bijective function
        $h:A\rightarrow{B}$. The requirement that $f$ and $g$ both be surjective
        can be replaced with the requirement that they both be injective. This
        is similar to saying that if $\Card(A)\leq\Card(B)$ and
        $\Card(B)\leq\Card(A)$, then $\Card(A)=\Card(B)$. Here, $\Card(A)$
        denotes the \textit{cardinality} of the set $A$.
        \begin{lexample}{Countable Sets}{Countable_Sets}
            There are many commonly discussed sets that are
            countably infinite. $\mathbb{N}$ is a trivial
            such example, but also $\mathbb{N}_{e}$ and
            $\mathbb{N}_{o}$, the sets of even and odd
            positive integers, respectively. For consider as
            bijections the following functions:
            \par
            \begin{subequations}
                \begin{minipage}[b]{0.49\textwidth}
                    \centering
                    \begin{equation}
                        f_{e}(n)=2n
                    \end{equation}
                \end{minipage}
                \hfill
                \begin{minipage}[b]{0.49\textwidth}
                    \centering
                    \begin{equation}
                        f_{0}(n)=2n-1
                    \end{equation}
                \end{minipage}
                \par\vspace{2.5ex}
                The set of all integers, $\mathbb{Z}$ is also
                countable, as shown in
                Fig.~\ref{fig:Bijection_N_and_Z}.
                One bijection is:
                \begin{equation}
                    f(n)=
                    \begin{cases}
                        \frac{n}{2},&n\mod{2}=0\\
                        \frac{1-n}{2},&n\mod{2}=1
                    \end{cases}
                \end{equation}
            \end{subequations}
            Any subset of $\mathbb{Z}$ is countable,
            and this is true of any countable set.
        \end{lexample}
        \begin{figure}[H]
            \centering
            \captionsetup{type=figure}
            \documentclass[crop,class=article]{standalone}
%----------------------------Preamble-------------------------------%
\usepackage{tikz}
\usepackage{amssymb}
\usetikzlibrary{arrows.meta}
%--------------------------Main Document----------------------------%
\begin{document}
    \begin{tikzpicture}[%
        >=latex
    ]
        \draw[<->, thick] (-5, 0) to (5, 0) node[below] {$\mathbb{Z}$};
        \foreach\x in {-4, -3, -2, -1, 0, 1, 2, 3, 4}{%
            \draw (\x, -0.1) to (\x, 0.1);
            \node at (\x, -0.4) {\x};
        }
        \draw[->] (0, 0.2) arc(180:0:0.5 and 0.4);
        \draw[->] (1, -0.6) arc(0:-180:1 and 0.6);
        \draw[->] (-1, 0.2) arc(180:0:1.5 and 0.7);
        \draw[->] (2, -0.6) arc(0:-180:2 and 0.8);
        \draw[->] (-2, 0.2) arc(180:0:2.5 and 0.9);
        \draw[->] (3, -0.6) arc(0:-180:3 and 1);
        \draw[->] (-3, 0.2) arc(180:0:3.5 and 1.1);
        \draw[->] (4, -0.6) arc(0:-180:4 and 1.2);
    \end{tikzpicture}
\end{document}
            \caption{Diagram of a Bijection Between
                     $\mathbb{N}$ and $\mathbb{Z}$.}
            \label{fig:Bijection_N_and_Z}
        \end{figure}
        One of the standard results about countable sets is
        that their subsets are also countable. This theorem
        relies, in a very subtle way, the use of the axiom
        of choice. There are a few stepping stones to get
        there. We will accept the various
        Cantor-Schr\"{o}eder-Bernstein theorems, which say
        the following:
        \begin{ltheorem}
              {First Cantor-Schr\"{o}eder-Bernstein Theorem}
              {First_Cantor_Schroeder_Bernstein}
            If $A$ and $B$ are sets such that there is an injective
            function $f:A\rightarrow{B}$ and an injective function
            $g:B\rightarrow{A}$, then there is a bijective function
            $h:A\rightarrow{B}$.
        \end{ltheorem}
        \begin{ltheorem}
              {Second Cantor-Schr\"{o}eder-Bernstein Theorem}
              {Second_Cantor_Schroeder_Bernstein}
            If $A$ and $B$ are sets such that there is a surjective
            function $f:A\rightarrow{B}$ and a surjective function
            $g:B\rightarrow{A}$, then there is a bijective function
            $h:A\rightarrow{B}$.
        \end{ltheorem}
        \par\hfill\par
        Using cardinalities, this says that if
        $\Card(A)\leq\Card(B)$ and $\Card(B)\leq\Card(A)$, then
        $\Card(A)=\Card(B)$. With this notation it becomes more
        intuitive. We will use this to prove that various sets are
        countable. Many sets that appear to be larger than $\mathbb{N}$
        can shown to to be the same size as $\mathbb{N}$ by finding
        a simple injective function, without finding an explicit
        bijection.
        \begin{ltheorem}
              {Third Cantor-Schr\"{o}eder-Bernstein Theorem}
              {Third_Cantor_Schroeder_Bernstein}
            If $A$, $B$, and $C$ are sets such that
            $A\subseteq{B}\subseteq{C}$, and if $A$ and $C$ are equivalent
            sets, then $B$ and $C$ are equivalent sets.
        \end{ltheorem}
        \par\hfill\par
        This says that if $\Card(A)\leq\Card(B)\leq\Card(C)$,
        and if $\Card(A)=\Card(C)$, then $\Card(B)=\Card(C)$.
        \begin{theorem}
            \label{thm:Measure_Theory_NxN_Is_Countable}
            $\mathbb{N}\times\mathbb{N}$ is countably infinite.
        \end{theorem}
        \begin{proof}
            There is a trivial injection
            $f:\mathbb{N}\rightarrow\mathbb{N}\times\mathbb{N}$
            defined by:
            \begin{equation}
                f(n)=(n,0)
            \end{equation}
            There is also an injection
            $g:\mathbb{N}\times\mathbb{N}\rightarrow\mathbb{N}$
            defined by:
            \begin{equation}
                g(n.m)=2^{n}3^{m}
            \end{equation}
            Since 2 and 3 are co-prime, if
            $g(n_{1},m_{1})=g(n_{2},m_{2})$, then
            $(n_{1},m_{1})=(n_{2},m_{2})$. Thus, $g$ is an injection.
            By the Cantor-Schr\"{o}eder-Bernstein Theorem, there is a
            bijection $h:\mathbb{N}\rightarrow\mathbb{N}\times\mathbb{N}$.
        \end{proof}
        One can intuitively see that the set of all positive
        rational numbers $\mathbb{Q}^{+}$ is countable by examining
        the zig-zag pattern shown in
        Fig.~\ref{fig:Bijection_N_and_Q_Plus}.
        Thm.~\ref{thm:Measure_Theory_NxN_Is_Countable} also
        shows this in a more rigorous way that. We can create
        a one-to-one correspondence with
        $\mathbb{N}\times\mathbb{N}$ by mapping
        $pq^{\minus{1}}\mapsto(p,q)$. Thus $\mathbb{Q}^{+}$
        and $\mathbb{N}\times\mathbb{N}$ are equivalent sets.
        But $\mathbb{N}\times\mathbb{N}$ and $\mathbb{N}$
        are equivalent sets, and therefore $\mathbb{Q}^{+}$
        is countable.
        Thm.~\ref{thm:Measure_Theory_NxN_Is_Countable} can also be used
        to show that the countable union of countable sets is also
        countable.
        \begin{ltheorem}{Equivalence of Countable Sets}
              {Countable_iff_exists_inj_to_N}
            A set $A$ is countable if and only if there is an injective
            function $f:A\rightarrow\mathbb{N}$.
        \end{ltheorem}
        Thm.~\ref{thm:Countable_iff_exists_inj_to_N} seems
        intuitively obvious, the injective function is
        simply the listing function. For a finite set, this
        is precisely how one constructs such an injection.
        For an infinite set $A$, this is equivalent to
        showing that any infinite subset of $\mathbb{N}$ is
        equivalent to $\mathbb{N}$. The standard proof
        using \textit{induction}, but actually has the axiom
        of choice underlying it.
        \begin{theorem}
            If $\mathcal{A}$ is a countably infinite set
            such that, for all $A\in\mathcal{A}$, $A$ is
            a non-empty countable set, then the set:
            \begin{equation}
                S=\bigcup_{A\in\mathcal{A}}A
            \end{equation}
            Is a countable set.
        \end{theorem}
        \begin{proof}
            If $\mathcal{F}$ is finite, then we are done. Suppose not.
            Let $A:\mathbb{N}\rightarrow\mathcal{A}$ be a bijection,
            and define:
            \begin{equation}
                S=\bigcup_{n\in\mathbb{N}}A_{n}
            \end{equation}
            Also, let:
            \begin{equation}
                \mathcal{F}_{n}
                =\{f:A_{n}\rightarrow\mathbb{N}:
                    f\textrm{ is injective}\}
            \end{equation}
            Since, for all $n\in\mathbb{N}$, $A_{n}$ is
            non-empty and countable, $\mathcal{F}_{n}$
            is non-empty. Let:
            \begin{equation}
                \mathcal{F}
                =\bigcup_{n\in\mathbb{N}}\mathcal{F}_{n}
            \end{equation}
            Thus, by the axiom of choice, there is a function
            $F:\mathbb{N}\rightarrow\mathcal{F}$ such that, for all
            $n\in\mathbb{N}$, $F_{n}\in\mathcal{F}_{n}$. For
            $x\in{S}$, let:
            \begin{equation}
                \varphi_{x}
                =\inf\{n\in\mathbb{N}:x\in{A}_{n}\}
            \end{equation}
            By the well-ordering of $\mathbb{N}$, for all
            $x\in{S}$, $\varphi_{x}$ is well defined. Let
            $\phi:S\rightarrow\mathbb{N}\times\mathbb{N}$
            be defined by:
            \begin{equation}
                \phi(x)
                =\big(\varphi_{x},F_{\varphi_{x}}(x)\big)
            \end{equation}
            Then $\phi$ is an injection. For if
            $\big(\varphi_{x},F_{\varphi_{x}}(x)\big)=%
             \big(\varphi_{y},F_{\varphi_{x}}(y)\big)$, then
            $\varphi_{x}=\varphi_{y}$, and thus
            $F_{\varphi(x)}(x)=F_{\varphi(x)}(y)$. But
            $F_{\varphi_{x}}$ is an injection, and
            thus $x=y$. Therefore $\phi$ is an injection.
            But $\mathbb{N}\times\mathbb{N}$ and $\mathbb{N}$
            are equivalent sets, and thus there's an
            injection $f:\mathbb{N}\times\mathbb{N}$. And
            the composition of injective functions is again
            injective, and thus
            $\phi\circ{f}:S\rightarrow\mathbb{N}$ is an
            injective function. But by
            Thm.~\ref{thm:Countable_iff_exists_inj_to_N},
            if there is an injective function
            $f:S\rightarrow\mathbb{N}$, then $S$ is
            countable. Therefore, etc.
        \end{proof}
        \begin{theorem}
            If $X$ is infinite, then there exists a
            countably infinite set $A\subseteq{X}$.
        \end{theorem}
        \begin{proof}
            If $A$ is finite, then we are done. Suppose not.
            For $n\in\mathbb{N}$, let:
            \begin{equation}
                A_{n}
                =\{g:\mathbb{Z}_{n}\rightarrow{A}:f\textrm{ is inective}\}
            \end{equation}
            Also, define:
            \begin{equation}
                \mathcal{F}=\bigcup_{n\in\mathbb{N}}A_{n}
            \end{equation}
            But by the axiom of choice, there is a function
            $f:\mathbb{N}\rightarrow\mathcal{F}$ such that
            $f_{n}\in{A}_{n}$. But then, for all
            $n\in\mathbb{N}$, the range of $f_{n}$ is finite.
            \begin{equation}
                A=\bigcup_{n\in\mathbb{N}}f_{n}
                    \Big(\mathbb{Z}_{n}\Big)
            \end{equation}
            Then $A\subseteq{X}$ is countably infinite.
        \end{proof}
        The set of rational numbers, $\mathbb{Q}$, is also
        countable. We may intuitively think of $\mathbb{N}$
        as being smaller than $\mathbb{Q}$, since there are
        simple \textit{surjections} that can be constructed
        from $\mathbb{Q}$ to $\mathbb{N}$. There is also a
        surjection from $\mathbb{N}$ onto $\mathbb{Q}^{+}$,
        as is shown in Fig.~\ref{fig:Bijection_N_and_Q_Plus}.
        To construct such a surjection, write out all of the
        positive rational numbers in a grid so that $a_{nm}$
        is the number $n/m$. Then zig-zag along the diagonals
        to construct the function. Thus there is a surjection
        $f:\mathbb{Q}^{+}\rightarrow\mathbb{N}$ and a surjection
        $g:\mathbb{N}\rightarrow\mathbb{Q}^{+}$. The
        Cantor-Schr\"{o}eder-Bernstein theorem says that if there is
        surjection from $A$ to $B$ and a surjection from $B$ to $A$, then
        there is a bijection between $A$ and $B$. Therefore there is a
        bijection between $\mathbb{N}$ and $\mathbb{Q}^{+}$, and
        $\mathbb{Q}^{+}$ is countable.
        \begin{figure}[H]
            \centering
            \captionsetup{type=figure}
            \resizebox{0.7\textwidth}{!}{%
                \documentclass[crop,class=article]{standalone}
%----------------------------Preamble-------------------------------%
\usepackage{tikz}
\usetikzlibrary{arrows.meta}
%--------------------------Main Document----------------------------%
\begin{document}
    \begin{tikzpicture}[%
        >=latex
    ]
        \foreach\y in {1, 2, 3, 4, 5, 6}{%
            \foreach\x in {1, 2, 3, 4, 5, 6}{%
                \node (\x\y) at (\x, 7-\y) {$\frac{\x}{\y}$};
            }
        }
        \foreach\x in {1, 2, 3, 4, 5, 6}{%
            \node at (7, \x) {$\cdots$};
            \node at (\x, 0) {$\vdots$};
        }
        \node at (7, 0) {$\ddots$};
        \draw[->] (11) to (12);
        \draw[->] (12) to (21);
        \draw[->] (21) to (31);
        \draw[->] (31) to (22);
        \draw[->] (22) to (13);
        \draw[->] (13) to (14);
        \draw[->] (14) to (23);
        \draw[->] (23) to (32);
        \draw[->] (32) to (41);
        \draw[->] (41) to (51);
        \draw[->] (51) to (42);
        \draw[->] (42) to (33);
        \draw[->] (33) to (24);
        \draw[->] (24) to (15);
        \draw[->] (15) to (16);
        \draw[->] (16) to (25);
        \draw[->] (25) to (34);
        \draw[->] (34) to (43);
        \draw[->] (43) to (52);
        \draw[->] (52) to (61);
        \draw[->] (61) to (62);
        \draw[->] (62) to (53);
        \draw[->] (53) to (44);
        \draw[->] (44) to (35);
        \draw[->] (35) to (26);
        \draw[->] (36) to (45);
        \draw[->] (45) to (54);
        \draw[->] (54) to (63);
        \draw[->] (64) to (55);
        \draw[->] (55) to (46);
        \draw[->] (56) to (65);
    \end{tikzpicture}
\end{document}
            }
            \caption{Diagram of a Surjection from
                     $\mathbb{N}$ onto $\mathbb{Q}^{+}$.}
            \label{fig:Bijection_N_and_Q_Plus}
        \end{figure}
        We can modify Fig.~\ref{fig:Bijection_N_and_Q_Plus}
        slightly to create a surjection between $\mathbb{N}$
        and $\mathbb{Q}$, see
        Fig.~\ref{fig:Bijection_N_and_Q}.
        It is important to note that this bijection will not
        preserve the order of the rational numbers. The
        bijection will have to jump around back and forth.
        Any such bijection will be forced to do this, as the
        rationals are everywhere dense on $\mathbb{R}$. Any
        monotonic sequence of $\mathbb{Q}$ cannot possibly
        be a bijection.
        \begin{theorem}
            If $A$ is a countably infinite set and
            $B\subseteq{A}$, then $B$ is countable.
        \end{theorem}
        \begin{proof}
            As $A$ is countably infinite, there is a bijection
            $a:\mathbb{N}\rightarrow{A}$. Define:
            \begin{equation}
                K=\{n\in\mathbb{N}:a_{n}\in{B}\}
            \end{equation}
            As $B\subseteq{A}$,
            this set contains a subsequence of points in
            $\mathbb{N}$ that get mapped into $B$. If $K$ is finite,
            then $B$ is finite, and if not then $K$ is countably
            infinite, and thus $B$ is countably infinite.
        \end{proof}
        \begin{figure}[H]
            \centering
            \captionsetup{type=figure}
            \resizebox{\textwidth}{!}{%
                \documentclass[crop,class=article]{standalone}
%----------------------------Preamble-------------------------------%
\usepackage{tikz}
\usepackage{amsmath}
\usetikzlibrary{arrows.meta}
%--------------------------Main Document----------------------------%
\begin{document}
    \begin{tikzpicture}[%
        >=latex
    ]
        \foreach\y in {1, 2, 3, 4}{%
            \foreach\x in {-4, -3, -2, -1, 0, 1, 2, 3, 4}{%
                \node (\x\y) at (\x, 7-\y) {$\frac{\x}{\y}$};
            }
        }
        \foreach\x in {-4, -3, -2, -1, 0, 1, 2, 3, 4}{%
            \node at (\x, 2) {$\vdots$};
        }
        \foreach\y in {3, 4, 5, 6}{%
            \node at (5, \y) {$\cdots$};
            \node at (-5, \y) {$\cdots$};
        }
        \node at (5, 2) {$\ddots$};
        \node at (-5, 2) {$\reflectbox{\ensuremath{\ddots}}$};
        \draw[->] (01) to (11);
        \draw[->] (11) to (12);
        \draw[->] (12) to (02);
        \draw[->] (02) to (-12);
        \draw[->] (-12) to (-11);
        \draw[->] (-1, 6.3) to (-1, 6.6)
                            to (2, 6.6)
                            to (2, 6.3);
        \draw[->] (21) to (22);
        \draw[->] (22) to (23);
        \draw[->] (23) to (13);
        \draw[->] (13) to (03);
        \draw[->] (03) to (-13);
        \draw[->] (-13) to (-23);
        \draw[->] (-23) to (-22);
        \draw[->] (-22) to (-21);
        \draw[->] (-2, 6.3) to (-2, 6.8)
                            to (3, 6.8)
                            to (3, 6.3);
        \draw[->] (31) to (32);
        \draw[->] (32) to (33);
        \draw[->] (33) to (34);
        \draw[->] (34) to (24);
        \draw[->] (24) to (14);
        \draw[->] (14) to (04);
        \draw[->] (04) to (-14);
        \draw[->] (-14) to (-24);
        \draw[->] (-24) to (-34);
        \draw[->] (-34) to (-33);
        \draw[->] (-33) to (-32);
        \draw[->] (-32) to (-31);
        \draw[->] (-3, 6.3) to (-3, 7)
                            to (4, 7)
                            to (4, 6.3);
        \draw[->] (41) to (42);
        \draw[->] (42) to (43);
        \draw[->] (43) to (44);
        \draw[->] (44) to (4, 2.3);
        \draw[->] (-4, 2.3) to (-44);
        \draw[->] (-44) to (-43);
        \draw[->] (-43) to (-42);
        \draw[->] (-42) to (-41);
    \end{tikzpicture}
\end{document}
            }
            \caption{Diagram of a Surjection from
                     $\mathbb{N}$ onto $\mathbb{Q}$.}
            \label{fig:Bijection_N_and_Q}
        \end{figure}
        \begin{theorem}
            If $A$ is an infinite set, then there exists a
            countable subset $B\subseteq{A}$.
        \end{theorem}
        \begin{proof}
            If $A$ is infinite then there is an
            $a_{1}\in{A}$. But, as $A$ is infinite,
            $A\setminus\{a_{1}\}$ is infinite, and there
            is an $a_{2}\in{A}\setminus\{a_{1}\}$. Continuing
            we obtain a sequence of distinct elements in $A$.
            Let $B=\{a_{n}:n\in\mathbb{N}\}$. Then
            $B\subseteq{A}$, and $B$ is countable.
        \end{proof}
        \begin{lexample}
            Suppose we have a collection of disjoint intervals
            of $\mathbb{R}$. This collection is either finite
            or countable. For in every interval, choose a
            rational number $q_{n}$. Let
            $A=\{q_{1},q_{2},\hdots\}$. Then
            $A\subseteq\mathbb{Q}$, and thus $A$ is either
            finite or countable. But this is also an enumeration
            of the intervals in the collection, and thus the
            collection is either finite or countable.
        \end{lexample}
        Given a countable collection of sets
        $A=\{\mathcal{A}_{1},\mathcal{A}_{2},\hdots\}$ such
        that, for all $n\in\mathbb{N}$, $\mathcal{A}_{n}$ is
        also a countable set, then the union is countable. That is:
        \begin{equation}
            B=\bigcup_{n=1}^{\infty}\mathcal{A}_{n}
        \end{equation}
        is a countable set. The proof of this is a mimicry of
        the proof of the countability of $\mathbb{Q}$. Not
        every set is either finite or countable. The real numbers,
        $\mathbb{R}$, is an example of an \textit{uncountable}
        set. First, some notes on the power set of a set.
        This is a bijection between the open unit interval $(0,1)$ and
        the closed unit interval $[0,1]$.
        \begin{equation}
            f(x)=
            \begin{cases}
                \frac{1}{2},&x=0\\
                \frac{1}{2^{n+2}},&x=\frac{1}{2^{n}}\\
                x,&\textrm{Otherwise}
            \end{cases}
        \end{equation}
        A graph of this is shown in
        Fig.~\ref{fig:Measure_Theory_Bijection_Closed_I_to_Open}.
        \begin{figure}[H]
            \centering
            \captionsetup{type=figure}
            \documentclass[crop,class=article]{standalone}
%----------------------------Preamble-------------------------------%
\usepackage{tikz}                       % Drawing/graphing tools.
\usetikzlibrary{arrows.meta}            % Latex and Stealth arrows.
%--------------------------Main Document----------------------------%
\begin{document}
    \begin{tikzpicture}[>=Latex, scale=2]
        \draw[->] (-0.15in, 0) to (1.1in, 0) node[above] {$x$};
        \draw[->] (0, -0.15in) to (0, 1.1in) node[right] {$y$};
        \draw (0, 0) to (1in, 1in);
        \draw[fill=black, draw=black] (0, 0.5in) circle (0.3mm);
        \foreach\x in{1in, 0.5in, 0.25in, 0.125in, 0.0625in, 0.03125in}{
            \draw[fill=white, draw=black] (\x, \x) circle (0.3mm);
            \draw[fill=black, draw=black] (\x, 0.25*\x) circle (0.3mm);
        }
    \end{tikzpicture}
\end{document}
            \caption{Bijection from $[0,1]$ to $(0,1)$.}
            \label{fig:Measure_Theory_Bijection_Closed_I_to_Open}
        \end{figure}
        The power set of any set is strictly larger than the
        original set. If $\Omega$ is finite with $n$ elements, it
        can be shown that $\mathcal{P}(\Omega)$ has $2^{n}$
        eleents. For infinite sets, there is a trivial surjection
        from $\mathcal{P}(\Omega)$ onto $\Omega$: for any element
        $x$, let $f(\{x\})=x$. This shows that
        $\Card(\Omega)\leq\Card(\mathcal{P}(\Omega))$. We now show
        that the inequality is strict.
        \begin{theorem}
            If $\Omega$ is a set, then there is no bijection
            $f:\Omega\rightarrow\mathcal{P}(\Omega)$
        \end{theorem}
        \begin{proof}
            For suppose not, and let
            $f:\Omega\rightarrow\mathcal{P}(\Omega)$ be such a
            bijection. Define:
            \begin{equation}
                A=\{x\in\Omega:x\in{f}(x)\}
            \end{equation}
            Then $A\subseteq\Omega$, and thus
            $A\in\mathcal{P}(\Omega)$. But then the complement of
            $A$ is also an element of $\mathcal{P}(\Omega)$. But
            $f$ is a bijection and thus there is an $x\in\Omega$
            such that $f(x)=A^{C}$. If $x\in{f}(x)$, then
            $x\in{A}$, a contradiction as $f(x)=A^{C}$, and thus
            $x\in{A}^{C}$ as well. Therefore $x\notin{f}(x)$. But
            then $x\in{A}^{C}$. But, from the definition of $A$,
            since $x\in{A}^{C}$ and $f(x)=A^{C}$, $x\in{f}(x)$
            and thus $x\in{A}$, a contradiction. Thus there is no
            $x$ such that $f(x)=A^{C}$. Therefore, $f$ is not a
            bijection.
        \end{proof}
        From this we conclude that $\mathcal{P}(\mathbb{N})$
        is an uncountable infinite set. But since $\mathbb{R}$
        and $\mathcal{P}(\mathbb{N})$ have the same cardinality,
        $\mathbb{R}$ is also uncountable.
        If a set $A$ has the same cardinality as $\mathbb{R}$,
        we say that $A$ has the cardinality of the continuum.
        \begin{lexample}
            There is a bijection between the open unit
            square $(0,1)\times(0,1)$ and the open unit interval
            $(0,1)$. For an element $(x,y)\in(0,1)\times(0,1)$,
            let $z\in(0,1)$ be defined as
            $z=0.x_{1}y_{1}x_{2}y_{2}x_{3}y_{3}\dots$ This is
            a bijection, for all $(x,y)$ in the square there is
            a corresponding $z\in(0,1)$, and for all
            $z\in(0,1)$ there is a corresponding element of
            $(0,1)\times(0,1)$. We can say that $(x,y)$ can
            be coded into $z$, and $z$ can be decoded into
            $(x,y)$. Hence, $(0,1)\times(0,1)$ has the cardinality
            of the continuum. By stereographic projection and induction
            we obtain:
            \par\hfill\par
            \begin{subequations}
                \begin{minipage}[b]{0.49\textwidth}
                    \begin{equation}
                        \Card(\mathbb{R}^{2})=\Card(\mathbb{R})
                    \end{equation}
                \end{minipage}
                \hfill
                \begin{minipage}[b]{0.49\textwidth}
                    \begin{equation}
                        \Card(\mathbb{R}^{n})=\Card(\mathbb{R})
                    \end{equation}
                \end{minipage}
                \par
            \end{subequations}
        \end{lexample}
        \begin{lexample}
            Consider the set of all real-valued sequences. We've seen
            that any real number can be represented as a function
            $f:\mathbb{N}\rightarrow\{0,1\}$. A real-valued sequence
            is a function $a:\mathbb{N}\rightarrow\mathbb{R}$, and
            thus the set of real-valued sequences can be seen as the
            set of functions whose domain is $\mathbb{N}$ and whose
            range is the set of all functions
            $f:\mathbb{N}\rightarrow\{0,1\}$. So given a sequence
            $a$, the image of $a_{n}$, for $n\in\mathbb{N}$, is a
            function $f_{n}:\mathbb{N}\rightarrow\{0,1\}$. Therefore
            the set of all real-valued sequences can be represented
            as the set of all functions
            $g:\mathbb{N}\times\mathbb{N}\rightarrow\{0,1\}$, where
            $g(n,m)=f_{n}(m)$. But $\mathbb{N}\times\mathbb{N}$ is
            countable, and thus the set of all functions of the form
            $g:\mathbb{N}\times\mathbb{N}\rightarrow\{0,1\}$ has the
            same cardinality as the set of all functions of the form
            $f:\mathbb{N}\rightarrow\{0,1\}$. But this has the
            cardinality of the continuum. Therefore, the set of all
            real-valued sequences has the cardinality of the continuum.
        \end{lexample}
        \begin{theorem}
            If $A$ is an infinite set, then there exists $S\subseteq{A}$ such that
            $S$ is countable.
        \end{theorem}
        \begin{proof}
            For as $A$ is infinite, for all $n\in\mathbb{N}$
            there exists a set $B\subseteq{A}$ such that
            $|B|=n$. For all $n\in\mathbb{N}$,
            define the following:
            \begin{equation}
                \mathcal{S}_{n}=\{B\subseteq{A}:|B|=n\}
            \end{equation}
            Let $\mathcal{S}$ be defined as:
            \begin{equation}
                \mathcal{S}=\{\mathcal{S}_{n}:n\in\mathbb{N}\}
            \end{equation}
            Then $\mathcal{S}$ is countable, for
            $a:\mathbb{N}\rightarrow\mathcal{S}$ defined
            by $a_{n}=\mathcal{S}_{n}$ is a bijection.
            By the axiom of choice, there is a function:
            \begin{equation}
                \alpha:\mathcal{S}\rightarrow
                \bigcup_{n=1}^{\infty}\mathcal{S}_{n}
            \end{equation}
            Such that, for all $x\in\mathcal{S}$,
            $\alpha(x)\in{x}$. But then, for all
            $x\in\mathcal{S}$, $\alpha(x)$ is a subset
            of $A$. But for all $x\in\mathcal{S}$, there
            is an $n\in\mathbb{N}$ such that
            $a_{n}=x$. Thus, let $S$ be the following:
            \begin{equation}
                S=\bigcup_{n=1}^{\infty}\alpha(a_{n})
            \end{equation}
        \end{proof}
        \begin{table}[H]
            \captionsetup{type=table}
            \centering
            \begin{tabular}{ccccc}
                $u_{11}$&$u_{12}$&$u_{13}$
                &$u_{14}$&$\hdots$\\
                $u_{21}$&$u_{22}$&$u_{23}$
                &$u_{24}$&$\hdots$\\
                $u_{31}$&$u_{32}$&$u_{33}$
                &$u_{34}$&$\hdots$\\
                $u_{41}$&$u_{42}$&$u_{43}$
                &$u_{44}$&$\hdots$\\
                $\vdots$&$\vdots$&$\vdots$
                &$\vdots$&$\ddots$
            \end{tabular}
            \caption{Construction of a Bijection on the
                     Countable Union of Countably Infinite
                     Sets.}
            \label{table:Countable_Union_of_Countable}
        \end{table}
        Where $u_{nm}$ is the $m^{th}$ element of
        $\mathcal{U}_{n}$.
        Using the \textit{diagonal argument},
        we obtain:
        In the absence of the requirement that
        $a\cap{b}=\emptyset$ for all pairs in $\mathcal{U}$,
        we still have that the union is, at most, countable.
        The mapping we found would be a
        \textit{surjection}, rather than a bijection.
        The union is then either finite or countable. The
        Cantor-Schr\"{o}der-Bernstein Theorem can often be
        used to help identify the size of a set. This says
        that if $A$ and $B$ are sets such that there exists
        a surjective function $f:A\rightarrow{B}$ and a
        surjective function $g:B\rightarrow{A}$, then there
        is a bijective function $h:A\rightarrow{B}$. The
        requirement that $f$ and $g$ both be surjective
        can be replaced with the requirement that they both
        be injective. This is similar to saying that if
        $\Card(A)\leq\Card(B)$ and $\Card(B)\leq\Card(A)$,
        then $\Card(A)=\Card(B)$. Here, $\Card(A)$ denotes
        the \textit{cardinality} of the set $A$.
        \begin{theorem}
            Equivalence has the following properties:
            \begin{enumerate}
                \item   $A\sim A$ for any set $A$.
                \item   If $A\sim B$, then $B\sim A$.
                \item   If $A\sim B$ and $B\sim C$, then $A\sim C$.
            \end{enumerate}
        \end{theorem}
        \begin{proof}
        In order,
        \begin{enumerate}
        \item   For let $f$ be the identity mapping. That is, for all
                $x\in A$, $f(x)=x$. This is bijective and thus $A\sim A$.
        \item   If $A\sim B$, there is a bijective function $f:A\rightarrow B$.
                Then $f^{-1}:B\rightarrow A$ is bijective, and $B\sim A$.
        \item   Let $f:A\rightarrow B$ and $g:B\rightarrow C$ be bijections.
                Then $g\circ f:A\rightarrow C$ is a bijection,
                and thus $A\sim C$.
        \end{enumerate}
        \end{proof}
        \begin{theorem}
            If $A\sim{C}$ and $B\sim{D}$, where $A,B$ and $C,D$ are disjoint,
            then $A\cup{B}\sim{C}\cup{D}$.
        \end{theorem}
        \begin{proof}
            Let $f:A\rightarrow C$ and $g:B\rightarrow D$ be isomorphisms.
            Let $h:A\cup{B}\rightarrow{C}\cup{D}$ be defined by:
            \begin{equation}
                h(x)=
                \begin{cases}
                    f(x),&x\in{A}\\
                    g(x),&x\in{B}
                \end{cases}
            \end{equation}
            As $A$ and $B$ are disjoint, this is indeed a function and it is
            bijective as $C$ and $D$ are disjoint. Therefore, etc.
        \end{proof}
        \begin{definition}
        A set $A$ is a said to be finite if and only if there is some $n\in \mathbb{N}$ such
        that there is a bijection $f:\mathbb{Z}_n \rightarrow A$.
        \end{definition}
        \begin{definition}
        If $A$ is a set that is equivalent to $\mathbb{Z}_n$ for some $n\in \mathbb{N}$,
        then the cardinality of $A$, denoted $|A|$, is $n$.
        \end{definition}
        \begin{theorem}
        For two finite sets $A$ and $B$, $A\sim B$ if and only if $|A|=|B|$.
        \end{theorem}
        \begin{proof}
        $[|A|=|B|=n]\Rightarrow[A\sim \mathbb{Z}_n]\land[B\sim \mathbb{Z}_n]\Rightarrow [A\sim B]$.
        $[A\sim B]\Rightarrow [\exists \underset{Bijective}{f:A\rightarrow B}]\Rightarrow [f(A) = B]\Rightarrow [|A|=|B|]$.
        \end{proof}
        \begin{definition}
        A set $A$ is said to be infinite if and only if there is a proper subset $B\underset{Proper}\subset A$ such that $B\sim A$.
        \end{definition}
        \begin{theorem}
        Infinite sets are not finite.
        \end{theorem}
        \begin{proof}
        Suppose not. Let $A$ be an infinite set and suppose there is an $n\in \mathbb{N}$ such
        that $A\sim \mathbb{Z}_n$. But as $A$ is an infinite set, there is a proper subset $B$
        such that $B\sim A$. But then $B\sim \mathbb{Z}_n$. But as $B$ is a proper subset,
        there is at least one point in $A$ not contained in $B$. But then $|B|<n$, a contradiction. Thus $A$ is not finite.
        \end{proof}
        \begin{theorem}
            If $A$ is an infinite set, then for every $n\in \mathbb{N}$
            there is a subset $B\subset A$ such that $B\sim \mathbb{Z}_n$.
        \end{theorem}
        \begin{proof}
        Suppose not. Then there is a least $n\in \mathbb{N}:B\subset A\Rightarrow |B|<n$.
        But then $A$ has at most $n$ elements, a contradiction.
        \end{proof}
        \begin{definition}
            A set $A$ is called countable if and only if $A\sim \mathbb{N}$.
        \end{definition}
        \begin{theorem}
        A set $A$ is infinite if and only if it contains a proper subset $B$ such that $B\sim \mathbb{N}$.
        \end{theorem}
        \begin{proof}
        If $A$ has a proper subset $B$ such that $B\sim \mathbb{N}$, then $A$ is not finite and is thus infinite.
        If $A$ is infinite, then for all $n\in \mathbb{N}$ there is a set $A_n\subset A$ such that
        $A_n \sim \mathbb{Z}_n$. Let $B = \{a_n: a_n \in A_n, a_n \notin A_{n-1}\}$.
        Note that $a_{n} = a_{m}$ if and only if $m= n$. Let $f:\mathbb{N} \rightarrow B$
        be defined by $n\mapsto a_n$. This is bijective, and thus $B\sim \mathbb{N}$.
        \end{proof}
        This shows that $\mathbb{N}$ is, in a sense, the "Smallest,"
        infinite set. $|\mathbb{N}|$ is denoted $\aleph_0$.
        \begin{definition}
        A set is called uncountable if and only if it is infinite and not countable.
        \end{definition}
        \begin{theorem}
            If $B\subset A$, $f:A\rightarrow B$ is injective, then there
            is a bijection $g:A\rightarrow B$
        \end{theorem}
        \begin{proof}
            Let $Y = A\setminus B$, and inductively define
            $f^{k+1}(Y)=f(f^{k}(Y))$. Let
            $X=Y\cup(\cup_{k=0}^{\infty}f^{k}(Y))$. As  $Y\cap{B}=\emptyset$,
            then $f(Y)\cap Y= \emptyset$. As $f$ is an injection,
            $f(f(Y))\cap f(Y)=\emptyset$,
            and similarly $f(f(Y))\cap Y = \emptyset$. Inductively,
            $f^{n}(Y)\cap f^{m}(Y)=\emptyset$,
        for $n\ne m$. It then also follows that $f(X) = \cup_{k=1}^{\infty} f^{k}(Y)$.
        Thus $A\setminus X = [B\cup Y]\setminus [Y\cup f(X)] = B\setminus f(X)$.
        Let $g(x) = \begin{cases} f(x), & x\in X \\ x, & x \in B\setminus f(X)\end{cases}$.
        This is a bijections from $A$ to $B$.
        \end{proof}
        \begin{theorem}[Cantor-Schr\"{o}der-Bernstein Theorem]
        If $A_1 \subset A$, $B_1 \subset B$, and $A\sim B_1$, $B \sim A_1$, then $A\sim B$.
        \end{theorem}
        \begin{proof}
        Let $f:A\rightarrow B_1$ and $g:B\rightarrow A_1$ be bijections.
        Then $(g\circ f):A\rightarrow A_1$ is an injection from $A$ into $A_1$.
        Thus, there is a bijection $h:A\rightarrow A_1$. Thus, $A\sim A_1 \sim B\Rightarrow A\sim B$.
        \end{proof}
        \begin{theorem}
            $\mathbb{N}\times \mathbb{N}$ is countable.
        \end{theorem}
        \begin{proof}
        For $f:\mathbb{N} \rightarrow \mathbb{N}\times \mathbb{N}$ defined by $f(n) = (0,n)$
        shows there is a subset $N_1$ of $\mathbb{N} \times \mathbb{N}$ such that
        $\mathbb{N}\sim N_1$. And $g:\mathbb{N}\times \mathbb{N} \rightarrow \mathbb{N}$
        defined by $g(n,m) =n+2^{n+m}$ shows that there is a subset $M_1 \subset \mathbb{N}$
        such that $\mathbb{N} \times \mathbb{N} \sim M_1$. By the Cantor-Schr\"{o}der-Bernstein Theorem,
        $\mathbb{N} \sim \mathbb{N}\times \mathbb{N}$.
        \end{proof}
        \begin{theorem}
            If $A$ is infinite and $f:A\rightarrow\mathbb{N}$ is injective,
            then $A$ is countable.
        \end{theorem}
        \begin{proof}
        As $A$ is infinite and $A\sim f(A)$, $f(A)$ is infinite.
        But as $f(A)\subset \mathbb{N}$ and $f(A)$ is infinite,
        $f(A)\sim \mathbb{N}$. Thus, $A\sim \mathbb{N}$.
        \end{proof}
        \begin{definition}
            If $A$ and $B$ are sets, we say that $|A|<|B|$ if there is an
            injective function $f:A\rightarrow B$, yet no bijection.
        \end{definition}
        \begin{theorem}[Cantor's Theorem]
            For a set $M$, $|M|<|\mathcal{P}(M)|$.
        \end{theorem}
        \begin{proof}
        For let $M$ be a set with cardinality $|M|$. Let $U_m \subset M$ such that $U_m \sim M$.
        Such a set exists, for example, the singletons of $\mathcal{P}(M)$. Thus, $M$ is split into
        two distinct sets $Class\ I=\{x\in M: \textrm{There is a subset } X\subset U_m\textrm{ such that }x\in X\}$,
        and $Class\ II=M-Class\ I$. Let $L = Class\ II$. $L\subset M$, and thus $L\in \mathcal{P}(M)$. However,
        $L \notin U_m$ for if it were, then the element $m_1$ paired with it in $M$ is of Class II
        (For it cannot be of Class I as $m_1$ would not appear in $L$). If $m_1$ were in Class II,
        then by definition $m_1 \notin L$. But as $m_1 \in L$, we see that $L\notin U_m$. Thus,
        $|U_m| <|\mathcal{P}(M)|$, and therefore $|M|<|\mathcal{P}(M)|$.
        \end{proof}
        \begin{theorem}
            The set $R=\{x\in \mathbb{R}:0<x<1\}$ is equivalent to
            $\mathcal{P}(\mathbb{N})$.
        \end{theorem}
        \begin{proof}
            For every real number has a binary representation (Proof of this
            is omitted). That is, for every real number $r$, there is a sequence
            $a:\mathbb{N}\rightarrow\{0,\,1\}$ such that:
            \begin{equation}
                r=\sum_{n=0}^{\infty}\frac{a_{n}}{2^{n}}
            \end{equation}
        As $0<x<1$, this sum is just $\sum_{n=1}^{\infty} \frac{a_n}{2^n}$.
        Let $f:\mathcal{P}(\mathbb{N})\rightarrow R$ be defined by the
        following: If $N\subset \mathcal{P}(\mathbb{N})$ and $n\in N$, then
        $a_{n}=1$, other wise $n=0$. Then every real number is matched to a
        subset of $\mathcal{P}(\mathbb{N})$, moreover this is done bijectively.
        Thus, $\mathcal{P}(\mathbb{N})\sim R$.
        \end{proof}
        \begin{theorem}
            $\mathbb{R}\sim\mathcal{P}(\mathbb{N})$.
        \end{theorem}
        \begin{proof}
            It suffices to show that $(0,1)\sim\mathbb{R}$.
            Let $f:(0,1)\rightarrow\mathbb{R}$ be defined by:
            \begin{equation}
                f(x)=
                \begin{cases}
                    \frac{x(1-x)}{2x-1},&x\ne\frac{1}{2}\\
                    0,&x=\frac{1}{2}
                \end{cases}
            \end{equation}
        \end{proof}
        \begin{theorem}
            The following are true:
            \begin{enumerate}
                \item $\Card(A)=0$ if and only if $A=\emptyset$.
                \item If ${A}\sim{\mathbb{Z}_{n}}$, then $\Card(A)=n$.
            \end{enumerate}
        \end{theorem}
        \begin{definition}
            A finite cardinal number is a cardinal
            number of a finite set.
        \end{definition}
        \begin{definition}
            The standard ordering on the finite cardinal
            number is $0<1<\hdots<n<n+1<\hdots$
        \end{definition}
        Thus, if $A$ and $B$ are finite sets, then we write $\Card(A)<\Card(B)$ if
        $A$ is equivalent to a subset of $B$ but not equivalent to $B$. We take this
        notion and generalize to all sets. For $A$ and $B$, we write
        $\Card(A)<\Card(B)$ if $A$ is equivalent to a subset of $B$ but is not
        equivalent to $B$. This is the same as saying $A$ is equivalent to a subset
        of $B$, but $B$ is not equivalent to a subset of $A$. We write that
        $\Card(A)\leq\Card(B)$ is $A$ is equivalent to a subset of $B$.
        \begin{theorem}
            The following are true:
            \begin{enumerate}
                \item If $\Card(A)\leq\Card(B)$ and
                      $\Card(B)\leq\Card(A)$, then
                      $\Card(A)\leq\Card(C)$.
                \item If $\Card(A)\leq\Card(B)$, then
                      $\Card(A)+\Card(C)\leq\Card(B)+\Card(C)$
            \end{enumerate}
        \end{theorem}
        \begin{theorem}
            If ${A}\subset{B}\subset{C}$, and
            $\Card(A)=\Card(C)$, then $\Card(B)=\Card(C)$.
        \end{theorem}
        \begin{theorem}
            If $f:{X}\rightarrow{Y}$ is a function,
            then $\Card(f(X))\leq\Card(X)$.
        \end{theorem}
        \begin{proof}
            Note that $f^{-1}(\{y\})$ creates a set of mutually disjoint
            subsets of $X$. By the axiom of choice there is a function
            $F:{f(X)}\rightarrow{X}$ such that for all ${y}\in{f(X)}$,
            ${F(y)}\in{f^{-1}(\{y\})}$. But since these sets are disjoint,
            $F$ is injective. Thus, $f(X)$ is equivalent to a subset of $X$.
            Therefore, $\Card(f(X))\leq\Card(X)$.
        \end{proof}
        The Schr\"{o}der-Bernstein theorem can be restated equivalently as
        ``If $A$ is equivalent to a subset of $B$ and $B$ is equivalent to a
        subset of $A$, then $A$ is equivalent to $B$.'' Addition and
        multiplication of finite cardinals follows directly from the standard
        arithmetic for the natural numbers. For cardinals of infinite sets,
        the arithmetic becomes a little more complicated.
        \begin{definition}
            The sum of two cardinal numbers is the cardinality of the union of two
            disjoint sets $A$ and $B$. That is, if ${A}\cap{B}=\emptyset$, then
            $\Card(A)+\Card(B)=\Card({A}\cup{B})$.
        \end{definition}
        \begin{theorem}
            If $a$ and $b$ are distinct cardinal numbers, then there exists sets $A$
            and $B$ such that ${A}\cap{B}=\emptyset$, $\textrm{Card}(A)=a$, and
            $\textrm{Card}(B)=b$.
        \end{theorem}
        \begin{theorem}
            If $A,B,C,$ and $D$ are sets such that $\Card(A)=\Card(C)$,
            $\Card(B)=\Card(D)$, and if ${A}\cap{B}=\emptyset$ and
            ${C}\cap{D}=\emptyset$, then
            $\Card({A}\cup{B})=\Card({C}\cup{D})$.
        \end{theorem}
        \begin{theorem}
            If $x,y,$ and $z$ are cardinal numbers, then
            $x+y=y+x$ and $x+(y+z)=(x+y)+z$.
        \end{theorem}
        The carinality of the set of natural numbers is denoted $\aleph_{0}$.
        That is, $\Card(\mathbb{N})=\aleph_{0}$
        \begin{example}
            Find the cardinal sum of $2$ and $5$. Let $N_{2}=\{1,2\}$ and
            $N_{5}=\{3,4,5,6,7\}$. Then $N_{2}$ and $N_{5}$ are disjoint,
            $\Card(N_{2})=2$ and $\Card(N_{5})=5$. Therefore
            $2+5=\Card(N_{2}\cup{N_{5}})$. But ${N_{2}}\cup{N_{5}}$ is just
            $\mathbb{Z}_{7}$, and $\Card(\mathbb{Z}_{7})=7$. Thus, $2+5=7$.
        \end{example}
        \begin{theorem}
            If $n$ and $m$ are finite cardinalities, then the cardinal sum of $n$
            and $m$ is the integer $n+m$, where $+$ is the usual arithmetic
            addition.
        \end{theorem}
        \begin{example}
            Compute the cardinal sum $\aleph_{0}+\aleph_{0}$. Let $\mathbb{N}_{e}$
            be the set of even natural numbers, and let $\mathbb{N}_{o}$ be the set
            of odd natural numbers. Then $\Card(\mathbb{N}_{e})=\aleph_{0}$,
            $\Card(\mathbb{N}_{o})=\aleph_{0}$, and
            ${\mathbb{N}_{o}}\cap{\mathbb{N}_{e}}=\emptyset$. Thus:
            \begin{equation}
                \aleph_{0}+\aleph_{0}=\Card({\mathbb{N}_{o}}\cup{\mathbb{N}_{e}})
            \end{equation}
            But ${\mathbb{N}_{o}}\cup{\mathbb{N}_{e}}=\mathbb{N}$ and
            $\Card(\mathbb{N})=\aleph_{0}$. Therefore,
            $\aleph_{0}+\aleph_{0}=\aleph_{0}$.
        \end{example}
        \begin{example}
            Find $n+\aleph_{0}$, where $n\in\mathbb{N}$. We have that
            $\Card(\mathbb{Z}_{n}z)=n$ and
            $\Card(\mathbb{N}\setminus\mathbb{Z}_{n})=\aleph_{0}$
            But then
            $n+\aleph_{0}=\Card(\mathbb{Z}_{n}\cup%
             \mathbb{N}\setminus\mathbb{Z}_{n})=\Card(\mathbb{N})=\aleph_{0}$.
            Therefore, $n+\aleph_{0}=\aleph_{0}$.
        \end{example}
        \begin{definition}
            The cardinality of the continuum, denoted $\mathfrak{c}$, is the
            cardinality of the set of real numbers. That is,
            $\mathfrak{c}=\Card(\mathbb{R})$.
        \end{definition}
        \begin{theorem}
            $\mathfrak{c}+\aleph_{0}=\mathfrak{c}$.
        \end{theorem}
        \begin{proof}
            We have $\Card((0,1))=\mathfrak{c}$ and $\Card(\mathbb{N})=\aleph_{0}$.
            But $(0,1)\cap\mathbb{N}=\emptyset$, and thus
            $\aleph_{0}+\mathfrak{c}=\Card((0,1)\cup\mathbb{N})$.
            But $\mathbb{R}\sim(0,1)$ and $\mathbb{N}\cup(0,1)\subset\mathbb{R}$. By
            the Schr\"{o}der-Bernstein theorem, $\mathbb{N}\cup(0,1)\sim\mathbb{R}$.
            Therefore, etc.
        \end{proof}
        \begin{definition}
            The product of two cardinal numbers $a$ and $b$ is the cardinality of
            the cartesian product of two set $A$ and $B$ such that $\Card(A)=a$ and
            $\Card(B)=b$. That is, ${a}\times{b}=\Card({A}\times{B})$.
        \end{definition}
        \begin{theorem}
            The following are true of cardinal numbers:
            \begin{enumerate}
                \item $xy=yx$
                \item $x(yz)=(xy)z$
                \item $x(y+z)=xy+xz$
            \end{enumerate}
        \end{theorem}
        \begin{proof}[Proof of Part 3]
            Let $A,B,$ and $C$ be disjoint.
            Then
            ${A}\times{({B}\cup{C})}%
             =({A}\times{B})\cup({A}\times{C})$, and thus
            $\Card({A}\times{({B}\cup{C})})%
             =\Card(({A}\times{B})\cup({A}\times{C}))$.
            But ${A}\times{B}$ and ${A}\times{C}$ are disjoint.
            Thus we have
            $\Card(({A}\times{B})\cup({A}\times{C}))%
             =\Card({A}\times{B})+\Card({A}\times{C})$.
            Therefore, etc.
        \end{proof}
        \begin{theorem}
            If $\Card(T)=x$ and $F:{T}\rightarrow{\mathcal{P}(T)}$ is a
            set-valued mapping such that for all ${t}\in{T}$ we have that
            $\Card(F(t))=y$ and for all ${t}\ne{t}$,
            ${F(t)}\cap{F(t')}=\emptyset$, then $\Card(\cup_{t=1}^{N}F(t))=xy$
        \end{theorem}
        \begin{example}
            Let $f:{\mathbb{N}^{2}}\rightarrow{\mathbb{N}}$
            be defined by $f(n,m)=2^{n}3^{m}$.
            Then $f$ is injective, since $2$ and $3$
            are coprime. Therefore,
            $\aleph_{0}\times\aleph_{0}=\aleph_{0}$.
        \end{example}
        \begin{example}
            Show that $\mathbb{R}^{2}\sim\mathbb{R}$.
            Let $f:\mathbb{R}^{2}\rightarrow\mathbb{R}$
            be the rather bizarre function defined by the image
            $f(x_{0}.x_{1}x_{2}\hdots,y_{0}.y_{1}y_{2}\hdots)%
             =x_{0}y_{0}.x_{0}y_{0}x_{1}y_{1}\hdots$ Then
            $f$ is inective. But the mapping
            $g:\mathbb{R}\rightarrow\mathbb{R}^{2}$
            defined by $g(x)=(x,0)$ is also injective.
            By Schr\"{o}der-Bernstein,
            $\mathbb{R}^{2}\sim\mathbb{R}$.
        \end{example}
        \begin{definition}
           Order isomorphic set are two sets $A$ and $B$
           with well orders $<_{A}$ and $<_{B}$ such that
           there exists a bijection $f:{B}A\rightarrow{B}$
           such that for all $a_{1},a_{2}\in{A}$ such that
           $a_{1}<_{A}a_{2}$, $f(a_{1})<_{B}f(a_{2})$.
        \end{definition}
        \begin{theorem}
           Order-Isomorphism is an equivalence relation.
        \end{theorem}
        To every well ordered set, an ordinal number is
        assigned, denoted $\Ord(A,<_{A})$. Conversely,
        for every ordinal number there is a set with a
        well order corresponding to it. Two ordinal numbers
        are equal if and only if the well-ordered sets
        corresponding to them are order isomorphic.
        That is,
        $\Ord(A,<_{A})=\Ord(B,<_{B})$ if and only if
        $(A,<_{A})$ and $(B,<_{B})$ are order isomorphic.
        \begin{theorem}
           If $(A,<_{A})$ and $(B,<_{B})$ are well ordered
           sets, and if $\Card(A)=\Card(B)$, then
           $(A,<_{A})$ and $(B,<_{B})$ are order
           isomorphic.
        \end{theorem}
        The ordinal number of the empty set is $0$. The
        ordinal number of a finite set of $n$ elements with
        a well ordering is denoted $n\in\mathbb{N}$.
        The ordinal for the natural numbers $\mathbb{N}$
        with their usual well-ordering is denoted $\omega$.
        A given well-ordered set has only one cardinal number,
        but it is possible for it to have two ordinal numbers.
        \begin{definition}
            An ordinal number $\alpha$ is less than or equal to an ordinal
            number $\beta$ if there are well-ordered sets $(A,<_{A})$ and
            $(B,<_{B})$ such that $\alpha=\Ord((A,<_{A}))$ and
            $\beta=\Ord(B,<_{B})$, and $(A,<_{B})$ is order isomorphic to
            subset of $(B,<_{B})$.
        \end{definition}
        \begin{theorem}
            The only order isomorphism from a well ordered set $(A,<_{A})$ to
            itself is the identity isomorphism.
        \end{theorem}
        \begin{theorem}
            If $\alpha$ and $\beta$ are ordinal numbers and
            ${\alpha}\leq{\beta}$ and ${\beta}\leq{\alpha}$,
            then $\alpha=\beta$.
        \end{theorem}
        \begin{theorem}
            If $\alpha$ and $\beta$ are ordinal numbers, either
            ${\alpha}\leq{\beta}$, or ${\beta}\leq{\alpha}$.
        \end{theorem}
        \begin{theorem}
            If $\alpha$ and $\beta$ are ordinal numbers, either
            $\alpha<\beta$, $\beta<\alpha$, or $\alpha=\beta$.
        \end{theorem}
        \begin{definition}
            The total ordering relation of a well-ordered set $(A,<_{A})$
            with respect
           to a well-ordered set $(B,<_{B})$ is the ordering
           on the set ${A}\cup{B}$ defined as: For all
           $a_{1},a_{2}\in{A}$ such that $a_{1}<_{A}a_{2}$,
           $a_{1}<_{*}a_{2}$, for all $b_{1},b_{2}\in{B}$
           such that $b_{1}<_{B}b_{2}$, $b_{1}<_{*}b_{2}$,
           and for all ${a}\in{A}$ and ${b}\in{B}$,
           ${a}<_{*}{b}$.
        \end{definition}
        \begin{theorem}
           The total ordering relation $<_{*}$ on the set
           ${A}\cup{B}$ is a well-ordering.
        \end{theorem}
        \begin{definition}
            The ordinal sum of two ordinal numbers $\Ord((A,<_{A}))$ and
            $\Ord((B,<_{B}))$, where $A$ and $B$ are disjoint, is the ordinal
            number $\Ord(({A}\cup{B},<_{*}))$.
        \end{definition}
        \begin{theorem}
           The following are true of ordinal numbers:
           \begin{enumerate}
                \item $\alpha<\beta\Rightarrow\alpha+\gamma<\beta+\gamma$
                \item $(\alpha+\beta)+\gamma=\alpha+(\beta+\gamma)$
                \item $\alpha+\beta=\alpha+\gamma\Rightarrow\beta=\gamma$
           \end{enumerate}
        \end{theorem}
        \begin{definition}
            The lexicographic ordering on the cartesianproduct of well
            ordered set $(A,<_{A})$ and $(B,<_{B})$ is the ordering on
            ${A}\times{B}$ defined by: If ${a}<_{A}{x}$, then
            $(a,b)<_{*}(x,y)$ for all $b,y\in{B}$, and if $a=x$ and
            $b<_{B}y$, then $(a,b)<_{*}(x,y)$.
        \end{definition}
        \begin{theorem}
            If $(A,<_{A})$ and $(B,<_{B})$ are well ordered sets, then the
            lexicographic ordering on ${A}\times{B}$ is a well ordering.
        \end{theorem}
        \begin{definition}
            The ordinal product of two ordinal numbers
            $\Ord((A,<_{A}))$ and $\Ord((B,<_{B}))$,
            is $\Ord(({A}\times{B},<_{*}))$
        \end{definition}
        \begin{theorem}
            The following are true of ordinal numbers:
            \begin{enumerate}
                \item $\alpha(\beta\gamma)=(\alpha\beta)\gamma$
                \item $\alpha(\beta+\gamma)=\alpha\beta+\alpha\gamma$
            \end{enumerate}
        \end{theorem}
        \begin{definition}
           Relatively prime integers are integers
           $a,b\in\mathbb{N}$ such that $\gcd(a,b)=1$.
        \end{definition}
        \begin{theorem}
           If $p$ is prime and $a\in\mathbb{N}$ is
           such that $p$ does not divide $a$, then $a$ and $p$
           are relatively prime.
        \end{theorem}
        \begin{theorem}
           There are infinitely many prime numbers.
        \end{theorem}
        \begin{theorem}
           If $a\in\mathbb{N}$, $a>1$, then either
           $a$ is a prime number, or $a$ is the product
           of finitely many primes.
        \end{theorem}
        \begin{theorem}
           If $a\in\mathbb{N}$, $a>1$, and if $a$ is not
           prime, then the prime expansion of $a$ is
           unique.
        \end{theorem}
        \begin{definition}
           A diophantine equation is an equation whose
           solutions are required to be integers.
        \end{definition}
        \begin{definition}
           A linear diophantine equation in two variables
           $x$ and $y$ is an equation
           $ax+by=c$, where $a,b,c\in\mathbb{Z}$.
        \end{definition}
        \begin{theorem}
           If $a,b,c\in\mathbb{Z}$ $d=\gcd(a,b)$, and if $d$ does not
           divide $c$, then $ax+by=c$ has no integral solutions.
        \end{theorem}
        \begin{theorem}
           If $a,b,c\in\mathbb{Z}$ $d=\gcd(a,b)$, and if $d$ divides $c$,
           then $ax+by=c$ has infinitely many solutions.
        \end{theorem}
    \renewcommand{\PATH}{\OLDPATH}
\endgroup