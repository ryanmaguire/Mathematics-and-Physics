\begingroup
    \ifcsname\PATH\endcsname
        \newcommand{\PATH}{books/Algebra/Ring_Theory}
        \newcommand{\OLDPATH}{\PATH}
    \else
        \newcommand{\OLDPATH}{\PATH}
        \renewcommand{\PATH}{books/Algebra/Ring_Theory}
    \fi
    \chapter{Rings}
        We now add more structure by considering a set with two operations.
        Everything thus far (semi-groups, quasi-groups, monoids, groups) has
        had only one operation associated to it, but in the most fundamental
        forms of arithmetic there are two. The only structure we've encountered
        with two operations so far has been Boolean algebras
        (see Book~\ref{book:Foundations}), but as well will see when we study
        topology, there is essentialy only one type of Boolean algebra and thus
        this study is, in a sense, complete. If we are going to axiomitize some
        algebraic structure it is then wise to avoid all of the properties of
        a Boolean algebra, and so instead we try to model the arithmetic of the
        real numbers. The most fundamental properties can be stated quite
        succintly: $(\mathbb{R},+)$ is an Abelian group and
        $(\mathbb{R},\cdot\,)$ is a monoid. We cannot just leave it there,
        however, since we've no way of knowing how $+$ and $\cdot$ play
        together. As presented, we have two potentially unrelated binary
        operations and thus we cannot procede any further. To complete our
        structure, we add the
        \glslink{distributive operation}{distributive property}.
        \section{Definitions}
    \begin{fdefinition}{Fields}{Fields}
        A field is a commutative ring $(\mathbb{F},+,\cdot\,)$ such that, for
        all $a\in\mathbb{F}$ such that $a$ is not the unital element of
        $(\mathbb{F},+)$, it is true that $a$ is an invertible element of
        $(\mathbb{F},\cdot\,)$.
    \end{fdefinition}
    \begin{fdefinition}{Subfield}{Subfield}
        A subfield of a field $(F,+,\cdot)$ is a set $K\subset F$, such that
        $(K,+,\cdot)$ is a field.
    \end{fdefinition}
    Given an element $a\in\mathbb{F}$, if $b$ is such that
    $a+b=0$ then we write $b=\minus{a}$. Subtraction of two elements
    $a$ and $c$, denoted $a-c$, is defined as $a+(\minus{c})$. The
    structure $(\mathbb{F},+)$ forms an Abelian group. From this we have
    that the identity is unique, as are additive inverses.
    It is common in the definition of a field to require that
    $0\ne{1}$. This is because if $0=1$ then we have $\mathbb{F}=\{0\}$.
    This comes from the following.
    \begin{ltheorem}{Multiplication by Zero}{Multiplication_by_Zero}
        If $(\mathbb{F},\,+,\,\cdot\,)$ is a field, and if $a\in\mathbb{F}$,
        then $a\cdot{0}=0$.
    \end{ltheorem}
    \begin{proof}
        For we have:
        \begin{equation}
            0=a\cdot(0)-a\cdot(0)=a\cdot(0-0)=a\cdot{0}
        \end{equation}
        This simply combines the distributive law with the additive
        property of zero, completing the proof.
    \end{proof}
    \begin{theorem}
        If $(\mathbb{F},\,+,\,\cdot\,)$ is a field, and if $0=1$, then
        $\mathbb{F}=\{0\}$.
    \end{theorem}
    \begin{proof}
        For suppose not, and let $a\in\mathbb{F}$ be such that $a\ne{0}$.
        But then, by Thm.~\ref{thm:Multiplication_by_Zero}:
        \begin{equation}
            a=a\cdot{1}=a\cdot{0}=0
        \end{equation}
        And thus $a=0$, a contradiction. Therefore,
        $\mathbb{F}$ is trivial.
    \end{proof}
    It is thus common to either call such a field a trivial field, or
    to require that $0\ne{1}$.
    \begin{lexample}{Examples of Fields}{Examples_of_Fields}
        There are several fields that should be familiar to the reader.
        If we let $\mathbb{R}$ denote the real numbers and $+$ and $\cdot$
        be the usual notations of addition and multiplication, then
        $(\mathbb{R},\,+,\,\cdot\,)$ is a field. Similarly, letting
        $\mathbb{Q}$ denote the rational numbers and $\mathbb{C}$ denote
        the complex numbers, $(\mathbb{Q},\,+,\,\cdot\,)$ is a field, as
        is $(\mathbb{C},\,+,\,\cdot\,)$. There are finite fields as well.
        Let $\mathbb{F}_{2}=\{0,\,1\}$ and define multiplication and
        addition as follows:
        \par\hfill\par
        \begin{table}[H]
            \centering
            \captionsetup{type=table}
            \parbox{.45\linewidth}{%
                \centering
                \begin{tabular}{c|cc}
                    $+$&0&1\\
                    \hline
                    0&0&1\\
                    1&1&0
                \end{tabular}
            }
            \parbox{.45\linewidth}{%
                \centering
                \begin{tabular}{c|cc}
                    $\cdot$&0&1\\
                    \hline
                    0&0&0\\
                    1&0&1
                \end{tabular}
            }
            \caption{The Arithmetic of $\mathbb{F}_{2}$}
        \end{table}
        $(\mathbb{F}_{2},\,+,\,\cdot)$ forms a field. Finally, if
        $p\in\mathbb{N}$ is prime, and if $+$ and $\cdot$ are addition
        and multiplication mod $p$, respectively, then
        $(\mathbb{Z}_{p},\,+,\,\cdot\,)$ is a field.
    \end{lexample}
    \begin{fdefinition}{Vector Space}{Vector_Space}
        A vector space over a field $(\mathbb{F},\,+,\,\cdot\,)$ is a
        set $V$ and a function
        $\boldsymbol{\cdot}:\mathbb{F}\times{V}\rightarrow{V}$ and
        a binary operation $\boldsymbol{+}$ on $V$, usuall called
        scalar multiplication and vector addition, respectively, 
        such that for all $\mathbf{x},\mathbf{y},\mathbf{z}\in{V}$,
        and all $a,b\in\mathbf{F}$, the following is true:
        \begin{enumerate}
            \item $\mathbf{x}\boldsymbol{+}%
                   (\mathbf{y}\boldsymbol{+}\mathbf{z})=%
                   (\mathbf{x}\boldsymbol{+}\mathbf{y})%
                   \boldsymbol{+}\mathbf{z}$
                  \hfill[Associative of Vector Addition]
            \item $\mathbf{x}\boldsymbol{+}\mathbf{y}=%
                   \mathbf{y}\boldsymbol{+}\mathbf{x}$
                  \hfill[Commutativity of Vector Addition]
            \item There is a $\mathbf{0}\in{V}$ such that
                  $\mathbf{0}\boldsymbol{+}\mathbf{x}=\mathbf{x}$
                  \hfill[Existence of Zero Vector]
            \item For all $\mathbf{x}$ there is a $\mathbf{y}$ such that
                  $\mathbf{x}\boldsymbol{+}\mathbf{y}=\mathbf{0}$
                  \hfill[Additive Inverses]
            \item $(a\cdot{b})\boldsymbol{\cdot}\mathbf{x}=%
                    a\boldsymbol{\cdot}(b\boldsymbol{\cdot}\mathbf{x})$
                  \hfill[Compatibility of Multiplication]
            \item $(a+b)\boldsymbol{\cdot}\mathbf{x}=%
                   (a\boldsymbol{\cdot}\mathbf{x})\boldsymbol{+}%
                   (b\boldsymbol{\cdot}\mathbf{x})$
                  \hfill[Distributive Law for Field Addition]
            \item $a\boldsymbol{\cdot}(\mathbf{x}\boldsymbol{+}\mathbf{y})=%
                   (a\boldsymbol{\cdot}\mathbf{x})\boldsymbol{+}%
                   (a\boldsymbol{\cdot}\mathbf{y})$
                  \hfill[Distributive Law for Vector Addition]
        \end{enumerate}
    \end{fdefinition}
    It is quite common not to distinguish between scalar multiplication
    $\boldsymbol{\cdot}$ and field multiplication $\cdot$, which may cause
    confusion. It is also common to drop the use of a symbol altogether and
    simply representation multiplication by concatenation of the the
    two variables, for example $a\mathbf{x}$ or $ab$, which represents
    scalar multiplication and field multiplication, respectively.
    \begin{example}
        If we let $\mathbb{F}=\mathbb{R}$ and let
        $V=\mathbb{R}^{n}$, where addition, multiplication, scalar
        multiplication, and vector addition are defined in their usual
        manner, then this forms a vector space. Similarly, the space
        $C([a,b])$ of continuous functions forms a vector space over
        $\mathbb{R}$, as does $L^{2}(\mathbb{R})$, the space of
        square integrable functions.
    \end{example}
    \begin{fdefinition}{Bilinear Operations}{Bilinear_Operations}
        A bilinear operation on a vector space
        $(V,\,\boldsymbol{+},\,\boldsymbol{\cdot}\,)$ over a field
        $(\mathbf{F},\,+,\,\cdot\,)$ is a function
        $[\,]:V\times{V}\rightarrow{V}$ such that, for all
        $\mathbf{x},\mathbf{y},\mathbf{z}\in{V}$, and for all
        $a,b\in\mathbf{F}$, the following is true:
        \begin{enumerate}
            \item $[\mathbf{x}\boldsymbol{+}\mathbf{y}, \mathbf{z}]=%
                   [\mathbf{x},\mathbf{z}]\boldsymbol{+}%
                   [\mathbf{y},\mathbf{z}]$
                  \hfill[Right Distributive Law]
            \item $[\mathbf{x},\mathbf{y}\boldsymbol{+}\mathbf{z}]=%
                   [\mathbf{x},\mathbf{y}]\boldsymbol{+}%
                   [\mathbf{x},\mathbf{z}]$
                  \hfill[Left Distributive Law]
            \item $[a\boldsymbol{\cdot}\mathbf{x},%
                    b\boldsymbol{\cdot}\mathbf{y}]=%
                   (a\cdot{b})\boldsymbol{\cdot}[\mathbf{x},\mathbf{y}]$
                  \hfill[Compatibility with Scalars]
        \end{enumerate}
    \end{fdefinition}
    \begin{lexample}{Examples of Bilinear Operations}
                    {Examples_of_Bilinear_Operation}
        The quintessential example of a bilinear operation is the
        cross product that one encounters in a multivariable calculus
        course. That is, for any three vectors
        $\mathbf{x},\mathbf{y},\mathbf{z}$, we have:
        \begin{equation}
            \mathbf{x}\times(\mathbf{y}+\mathbf{z})=
            \mathbf{x}\times\mathbf{y}+\mathbf{x}\times\mathbf{z}
        \end{equation}
        Similarly for right sided multiplication. The compatibility of
        the cross product with scalar multiplication is also true:
        \begin{equation}
            (a\mathbf{x})\times(b\mathbf{y})=ab(\mathbf{x}\times\mathbf{y})
        \end{equation}
        This serves somewhat as a motivating example for bilinear
        operations. If we think of the field of invertible matrices,
        then multiplication forms a bilinear operation as well, with
        scalar multiplication being the usual entry wise operation that
        is done on matrices. Lastly, if $\langle\,\rangle$ is an inner
        product on $\mathbb{R}$ or $\mathbb{C}$, then this is a bilinear
        operation, the vector space being the underlying field itself.
    \end{lexample}
    \begin{fdefinition}{Algebra over a Field}{Algebra_over_a_Field}
        An algebra of a field $(\mathbf{F},\,+,\,\cdot\,)$ is a
        vector space $(\mathbf{V},\,\boldsymbol{+},\,\boldsymbol{\cdot}\,)$
        and a bilinear operation $[\,]:V\times{V}\rightarrow{V}$.
    \end{fdefinition}
    \begin{fdefinition}{Associative Algebra over a Field}
                       {Associative_Algebra_over_a_Field}
        An associative algebra over a field $(\mathbb{F},\,+,\,\cdot\,)$
        is an algebra $(V,[\,])$ over $\mathbb{F}$ such that, for all
        $r\in\mathbb{F}$ and for all $\mathbf{x},\mathbf{y}\in{V}$,
        the following is true:
        \begin{equation}
            r[\mathbf{x},\,\mathbf{y}]=[r\mathbf{x},\,\mathbf{y}]
                                      =[\mathbf{x},\,r\mathbf{y}]
        \end{equation}
    \end{fdefinition}
    \begin{fdefinition}{Derivation on an Algebra}{Derivation_on_an_Algebra}
        A derivation on an algebra $(V,\,[\,])$ is a function
        $D:V\rightarrow{V}$ such that for all $\mathbf{x},\mathbf{y}\in{V}$,
        the following (Liebniz's Rule) is true:
        \begin{equation}
            D([\mathbf{x},\mathbf{y}])
            =[\mathbf{x},D(\mathbf{y})]+[D(\mathbf{x}),\mathbf{y}]
        \end{equation}
    \end{fdefinition}
    \begin{theorem}
        In a field, $0$ and $1$ are unique.
    \end{theorem}
    \begin{proof}
        For suppose not, and let $0'$ and $1'$ be other identities.
        Then $1'=1'\cdot 1 = 1$ and $0'=0'+0=0$.
    \end{proof}
    \begin{theorem}
        For any field $\langle{F},+,\cdot\rangle$ and $a\in{F}$, $a\cdot{0}=0$.
    \end{theorem}
    \begin{proof}
        For:
        \begin{equation}
            0=a\cdot{0}+(\minus{a}\cdot{0})
             =a\cdot(0+0)+(\minus{a}\cdot{0})
             =a\cdot{0}+a\cdot{0}+(\minus{a}\cdot{0})
             =a\cdot 0
        \end{equation}
        Thus, $a\cdot{0}=0$.
    \end{proof}
    If $1=0$, then $a=a\cdot{1}=a\cdot{0}=0$, and thus every element is
    zero. A very boring field.
    \begin{theorem}
        In a field $\langle F, +,\cdot \rangle$, if $0\ne 1$, then $0$ has no
        inverse.
    \end{theorem}
    \begin{proof}
        For let $a$ be such an inverse. Then $a\cdot{0}=1$. But for any element
        of $F$, $a\cdot{0}=0$. But $0\ne{1}$, a contradiction.
    \end{proof}
    \begin{theorem}
        If $a+b=0$, then $b=(\minus{1})\cdot{a}$ where $(\minus{1})$ is the
        solution to $1+(\minus{1})=0$.
    \end{theorem}
    \begin{proof}
        $a+(\minus{1})a=a(1+(\minus{1}))=a\cdot{0}=0$. From uniqueness,
        $b=(\minus{1})a$. We may thus write additive inverses as $\minus{a}$.
    \end{proof}
    \begin{definition}
        Given two fields $(F,+,\cdot)$ and $(F',+',\times)$, a bijection
        function $f:F\rightarrow{F}'$ is said to be a field isomorphism if and
        only if for allelements $a,b\in{F}$, $f(a+b)=f(a)+'f(b)$, and
        $f(a\cdot{b})=f(a)\times{f}(b)$
    \end{definition}
    \begin{definition}
        $(F,+,\cdot)$ and $(F',+',\times)$, are said to be isomorphic if and
        only if they have an isomorphism.
    \end{definition}
    \begin{theorem}
        Given an ismorphism between two fields $(F,+,\cdot)$ and
        $(F', +',\times)$, $f(1)=1'$ and $f(0)=0'$.
    \end{theorem}
    \begin{proof}
        For let $x\in{F}$. Then $f(x)=f(x\cdot 1)=f(x)\times{f}(1)$, and
        $f(x)=f(x+0)=f(x)+'f(0)$. Therefore, etc.
    \end{proof}
    \begin{theorem}
        In a field $(F,+,\cdot)$, $(a+b)^{2}=a^{2}+2ab+b^{2}$
        ($2$ being the solution to $1+1$).
    \end{theorem}
    \begin{proof}
        For:
        \begin{align}
            (a+b)^{2}&=(a+b)(a+b)\\
                     &=a(a+b)+b(a+b)\\
                     &=a^{2}+ab+ba+b^{2}\\
                     &=a^{2}+ab(1+1)+b^{2}\\
                     &=a^{2}+2ab+b^{2}
        \end{align}
    \end{proof}
        \section{Ring Morphisms}
    \begin{fdefinition}{Ring Homomorphism}{Ring_Homomorphism}
        A \gls{ring homomorphism}\index{Ring Homomorphism} from a \gls{ring}
        $(R_{1},\,+,\,\cdot\,)$ to a ring $(R_{2},\,+',\,*\,)$ is a
        \gls{function} $f:R_{1}\rightarrow{R}_{2}$ such that, for all
        $x,y\in{R}_{1}$, the following are true:
        \begin{align}
            f(x+y)&=f(x)+'f(y)
            \tag{Preservation of Addition}\\
            f(x\cdot{y})&=f(x)*f(y)
            \tag{Preservation of Multiplication}\\
            f(1_{R_{1}})&=1_{R_{2}}
            \tag{Preservation of Identities}
        \end{align}
        Where $1_{R_{1}}$ is the unital element of $R_{1}$ and
        $1_{R_{2}}$ is the unital element of $R_{2}$.
    \end{fdefinition}
    There's a special name for a homomorphism from a ring $(R,\,+,\,\cdot\,)$
    to itself.
    \begin{fdefinition}{Ring Endomorphisms}{Ring_Endomorphisms}
        A \gls{ring endomorphism}\index{Ring Endomorphism} on a \gls{ring}
        $(R,\,+,\,\cdot\,)$ is a \gls{ring homomorphism} from
        $(R,\,+,\,\cdot\,)$ to itself. That is, a ring homomorphism
        $f:R\rightarrow{R}$.
    \end{fdefinition}
    \begin{fnotation}{Set of Ring Endomorphisms}{Ring_Endomorphisms}
        The set of ring endomorphisms on a ring $\mathcal{R}=(R,\,+,\,\cdot)$
        is denoted $\textrm{End}(\mathcal{R})$.
    \end{fnotation}
    \chapter{Fields}
        \section{Definitions}
    \begin{fdefinition}{Fields}{Fields}
        A field is a commutative ring $(\mathbb{F},+,\cdot\,)$ such that, for
        all $a\in\mathbb{F}$ such that $a$ is not the unital element of
        $(\mathbb{F},+)$, it is true that $a$ is an invertible element of
        $(\mathbb{F},\cdot\,)$.
    \end{fdefinition}
    \begin{fdefinition}{Subfield}{Subfield}
        A subfield of a field $(F,+,\cdot)$ is a set $K\subset F$, such that
        $(K,+,\cdot)$ is a field.
    \end{fdefinition}
    Given an element $a\in\mathbb{F}$, if $b$ is such that
    $a+b=0$ then we write $b=\minus{a}$. Subtraction of two elements
    $a$ and $c$, denoted $a-c$, is defined as $a+(\minus{c})$. The
    structure $(\mathbb{F},+)$ forms an Abelian group. From this we have
    that the identity is unique, as are additive inverses.
    It is common in the definition of a field to require that
    $0\ne{1}$. This is because if $0=1$ then we have $\mathbb{F}=\{0\}$.
    This comes from the following.
    \begin{ltheorem}{Multiplication by Zero}{Multiplication_by_Zero}
        If $(\mathbb{F},\,+,\,\cdot\,)$ is a field, and if $a\in\mathbb{F}$,
        then $a\cdot{0}=0$.
    \end{ltheorem}
    \begin{proof}
        For we have:
        \begin{equation}
            0=a\cdot(0)-a\cdot(0)=a\cdot(0-0)=a\cdot{0}
        \end{equation}
        This simply combines the distributive law with the additive
        property of zero, completing the proof.
    \end{proof}
    \begin{theorem}
        If $(\mathbb{F},\,+,\,\cdot\,)$ is a field, and if $0=1$, then
        $\mathbb{F}=\{0\}$.
    \end{theorem}
    \begin{proof}
        For suppose not, and let $a\in\mathbb{F}$ be such that $a\ne{0}$.
        But then, by Thm.~\ref{thm:Multiplication_by_Zero}:
        \begin{equation}
            a=a\cdot{1}=a\cdot{0}=0
        \end{equation}
        And thus $a=0$, a contradiction. Therefore,
        $\mathbb{F}$ is trivial.
    \end{proof}
    It is thus common to either call such a field a trivial field, or
    to require that $0\ne{1}$.
    \begin{lexample}{Examples of Fields}{Examples_of_Fields}
        There are several fields that should be familiar to the reader.
        If we let $\mathbb{R}$ denote the real numbers and $+$ and $\cdot$
        be the usual notations of addition and multiplication, then
        $(\mathbb{R},\,+,\,\cdot\,)$ is a field. Similarly, letting
        $\mathbb{Q}$ denote the rational numbers and $\mathbb{C}$ denote
        the complex numbers, $(\mathbb{Q},\,+,\,\cdot\,)$ is a field, as
        is $(\mathbb{C},\,+,\,\cdot\,)$. There are finite fields as well.
        Let $\mathbb{F}_{2}=\{0,\,1\}$ and define multiplication and
        addition as follows:
        \par\hfill\par
        \begin{table}[H]
            \centering
            \captionsetup{type=table}
            \parbox{.45\linewidth}{%
                \centering
                \begin{tabular}{c|cc}
                    $+$&0&1\\
                    \hline
                    0&0&1\\
                    1&1&0
                \end{tabular}
            }
            \parbox{.45\linewidth}{%
                \centering
                \begin{tabular}{c|cc}
                    $\cdot$&0&1\\
                    \hline
                    0&0&0\\
                    1&0&1
                \end{tabular}
            }
            \caption{The Arithmetic of $\mathbb{F}_{2}$}
        \end{table}
        $(\mathbb{F}_{2},\,+,\,\cdot)$ forms a field. Finally, if
        $p\in\mathbb{N}$ is prime, and if $+$ and $\cdot$ are addition
        and multiplication mod $p$, respectively, then
        $(\mathbb{Z}_{p},\,+,\,\cdot\,)$ is a field.
    \end{lexample}
    \begin{fdefinition}{Vector Space}{Vector_Space}
        A vector space over a field $(\mathbb{F},\,+,\,\cdot\,)$ is a
        set $V$ and a function
        $\boldsymbol{\cdot}:\mathbb{F}\times{V}\rightarrow{V}$ and
        a binary operation $\boldsymbol{+}$ on $V$, usuall called
        scalar multiplication and vector addition, respectively, 
        such that for all $\mathbf{x},\mathbf{y},\mathbf{z}\in{V}$,
        and all $a,b\in\mathbf{F}$, the following is true:
        \begin{enumerate}
            \item $\mathbf{x}\boldsymbol{+}%
                   (\mathbf{y}\boldsymbol{+}\mathbf{z})=%
                   (\mathbf{x}\boldsymbol{+}\mathbf{y})%
                   \boldsymbol{+}\mathbf{z}$
                  \hfill[Associative of Vector Addition]
            \item $\mathbf{x}\boldsymbol{+}\mathbf{y}=%
                   \mathbf{y}\boldsymbol{+}\mathbf{x}$
                  \hfill[Commutativity of Vector Addition]
            \item There is a $\mathbf{0}\in{V}$ such that
                  $\mathbf{0}\boldsymbol{+}\mathbf{x}=\mathbf{x}$
                  \hfill[Existence of Zero Vector]
            \item For all $\mathbf{x}$ there is a $\mathbf{y}$ such that
                  $\mathbf{x}\boldsymbol{+}\mathbf{y}=\mathbf{0}$
                  \hfill[Additive Inverses]
            \item $(a\cdot{b})\boldsymbol{\cdot}\mathbf{x}=%
                    a\boldsymbol{\cdot}(b\boldsymbol{\cdot}\mathbf{x})$
                  \hfill[Compatibility of Multiplication]
            \item $(a+b)\boldsymbol{\cdot}\mathbf{x}=%
                   (a\boldsymbol{\cdot}\mathbf{x})\boldsymbol{+}%
                   (b\boldsymbol{\cdot}\mathbf{x})$
                  \hfill[Distributive Law for Field Addition]
            \item $a\boldsymbol{\cdot}(\mathbf{x}\boldsymbol{+}\mathbf{y})=%
                   (a\boldsymbol{\cdot}\mathbf{x})\boldsymbol{+}%
                   (a\boldsymbol{\cdot}\mathbf{y})$
                  \hfill[Distributive Law for Vector Addition]
        \end{enumerate}
    \end{fdefinition}
    It is quite common not to distinguish between scalar multiplication
    $\boldsymbol{\cdot}$ and field multiplication $\cdot$, which may cause
    confusion. It is also common to drop the use of a symbol altogether and
    simply representation multiplication by concatenation of the the
    two variables, for example $a\mathbf{x}$ or $ab$, which represents
    scalar multiplication and field multiplication, respectively.
    \begin{example}
        If we let $\mathbb{F}=\mathbb{R}$ and let
        $V=\mathbb{R}^{n}$, where addition, multiplication, scalar
        multiplication, and vector addition are defined in their usual
        manner, then this forms a vector space. Similarly, the space
        $C([a,b])$ of continuous functions forms a vector space over
        $\mathbb{R}$, as does $L^{2}(\mathbb{R})$, the space of
        square integrable functions.
    \end{example}
    \begin{fdefinition}{Bilinear Operations}{Bilinear_Operations}
        A bilinear operation on a vector space
        $(V,\,\boldsymbol{+},\,\boldsymbol{\cdot}\,)$ over a field
        $(\mathbf{F},\,+,\,\cdot\,)$ is a function
        $[\,]:V\times{V}\rightarrow{V}$ such that, for all
        $\mathbf{x},\mathbf{y},\mathbf{z}\in{V}$, and for all
        $a,b\in\mathbf{F}$, the following is true:
        \begin{enumerate}
            \item $[\mathbf{x}\boldsymbol{+}\mathbf{y}, \mathbf{z}]=%
                   [\mathbf{x},\mathbf{z}]\boldsymbol{+}%
                   [\mathbf{y},\mathbf{z}]$
                  \hfill[Right Distributive Law]
            \item $[\mathbf{x},\mathbf{y}\boldsymbol{+}\mathbf{z}]=%
                   [\mathbf{x},\mathbf{y}]\boldsymbol{+}%
                   [\mathbf{x},\mathbf{z}]$
                  \hfill[Left Distributive Law]
            \item $[a\boldsymbol{\cdot}\mathbf{x},%
                    b\boldsymbol{\cdot}\mathbf{y}]=%
                   (a\cdot{b})\boldsymbol{\cdot}[\mathbf{x},\mathbf{y}]$
                  \hfill[Compatibility with Scalars]
        \end{enumerate}
    \end{fdefinition}
    \begin{lexample}{Examples of Bilinear Operations}
                    {Examples_of_Bilinear_Operation}
        The quintessential example of a bilinear operation is the
        cross product that one encounters in a multivariable calculus
        course. That is, for any three vectors
        $\mathbf{x},\mathbf{y},\mathbf{z}$, we have:
        \begin{equation}
            \mathbf{x}\times(\mathbf{y}+\mathbf{z})=
            \mathbf{x}\times\mathbf{y}+\mathbf{x}\times\mathbf{z}
        \end{equation}
        Similarly for right sided multiplication. The compatibility of
        the cross product with scalar multiplication is also true:
        \begin{equation}
            (a\mathbf{x})\times(b\mathbf{y})=ab(\mathbf{x}\times\mathbf{y})
        \end{equation}
        This serves somewhat as a motivating example for bilinear
        operations. If we think of the field of invertible matrices,
        then multiplication forms a bilinear operation as well, with
        scalar multiplication being the usual entry wise operation that
        is done on matrices. Lastly, if $\langle\,\rangle$ is an inner
        product on $\mathbb{R}$ or $\mathbb{C}$, then this is a bilinear
        operation, the vector space being the underlying field itself.
    \end{lexample}
    \begin{fdefinition}{Algebra over a Field}{Algebra_over_a_Field}
        An algebra of a field $(\mathbf{F},\,+,\,\cdot\,)$ is a
        vector space $(\mathbf{V},\,\boldsymbol{+},\,\boldsymbol{\cdot}\,)$
        and a bilinear operation $[\,]:V\times{V}\rightarrow{V}$.
    \end{fdefinition}
    \begin{fdefinition}{Associative Algebra over a Field}
                       {Associative_Algebra_over_a_Field}
        An associative algebra over a field $(\mathbb{F},\,+,\,\cdot\,)$
        is an algebra $(V,[\,])$ over $\mathbb{F}$ such that, for all
        $r\in\mathbb{F}$ and for all $\mathbf{x},\mathbf{y}\in{V}$,
        the following is true:
        \begin{equation}
            r[\mathbf{x},\,\mathbf{y}]=[r\mathbf{x},\,\mathbf{y}]
                                      =[\mathbf{x},\,r\mathbf{y}]
        \end{equation}
    \end{fdefinition}
    \begin{fdefinition}{Derivation on an Algebra}{Derivation_on_an_Algebra}
        A derivation on an algebra $(V,\,[\,])$ is a function
        $D:V\rightarrow{V}$ such that for all $\mathbf{x},\mathbf{y}\in{V}$,
        the following (Liebniz's Rule) is true:
        \begin{equation}
            D([\mathbf{x},\mathbf{y}])
            =[\mathbf{x},D(\mathbf{y})]+[D(\mathbf{x}),\mathbf{y}]
        \end{equation}
    \end{fdefinition}
    \begin{theorem}
        In a field, $0$ and $1$ are unique.
    \end{theorem}
    \begin{proof}
        For suppose not, and let $0'$ and $1'$ be other identities.
        Then $1'=1'\cdot 1 = 1$ and $0'=0'+0=0$.
    \end{proof}
    \begin{theorem}
        For any field $\langle{F},+,\cdot\rangle$ and $a\in{F}$, $a\cdot{0}=0$.
    \end{theorem}
    \begin{proof}
        For:
        \begin{equation}
            0=a\cdot{0}+(\minus{a}\cdot{0})
             =a\cdot(0+0)+(\minus{a}\cdot{0})
             =a\cdot{0}+a\cdot{0}+(\minus{a}\cdot{0})
             =a\cdot 0
        \end{equation}
        Thus, $a\cdot{0}=0$.
    \end{proof}
    If $1=0$, then $a=a\cdot{1}=a\cdot{0}=0$, and thus every element is
    zero. A very boring field.
    \begin{theorem}
        In a field $\langle F, +,\cdot \rangle$, if $0\ne 1$, then $0$ has no
        inverse.
    \end{theorem}
    \begin{proof}
        For let $a$ be such an inverse. Then $a\cdot{0}=1$. But for any element
        of $F$, $a\cdot{0}=0$. But $0\ne{1}$, a contradiction.
    \end{proof}
    \begin{theorem}
        If $a+b=0$, then $b=(\minus{1})\cdot{a}$ where $(\minus{1})$ is the
        solution to $1+(\minus{1})=0$.
    \end{theorem}
    \begin{proof}
        $a+(\minus{1})a=a(1+(\minus{1}))=a\cdot{0}=0$. From uniqueness,
        $b=(\minus{1})a$. We may thus write additive inverses as $\minus{a}$.
    \end{proof}
    \begin{definition}
        Given two fields $(F,+,\cdot)$ and $(F',+',\times)$, a bijection
        function $f:F\rightarrow{F}'$ is said to be a field isomorphism if and
        only if for allelements $a,b\in{F}$, $f(a+b)=f(a)+'f(b)$, and
        $f(a\cdot{b})=f(a)\times{f}(b)$
    \end{definition}
    \begin{definition}
        $(F,+,\cdot)$ and $(F',+',\times)$, are said to be isomorphic if and
        only if they have an isomorphism.
    \end{definition}
    \begin{theorem}
        Given an ismorphism between two fields $(F,+,\cdot)$ and
        $(F', +',\times)$, $f(1)=1'$ and $f(0)=0'$.
    \end{theorem}
    \begin{proof}
        For let $x\in{F}$. Then $f(x)=f(x\cdot 1)=f(x)\times{f}(1)$, and
        $f(x)=f(x+0)=f(x)+'f(0)$. Therefore, etc.
    \end{proof}
    \begin{theorem}
        In a field $(F,+,\cdot)$, $(a+b)^{2}=a^{2}+2ab+b^{2}$
        ($2$ being the solution to $1+1$).
    \end{theorem}
    \begin{proof}
        For:
        \begin{align}
            (a+b)^{2}&=(a+b)(a+b)\\
                     &=a(a+b)+b(a+b)\\
                     &=a^{2}+ab+ba+b^{2}\\
                     &=a^{2}+ab(1+1)+b^{2}\\
                     &=a^{2}+2ab+b^{2}
        \end{align}
    \end{proof}
    \renewcommand{\PATH}{\OLDPATH}
\endgroup