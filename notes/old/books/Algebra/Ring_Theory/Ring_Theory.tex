\begingroup
    \ifcsname\PATH\endcsname
        \newcommand{\PATH}{books/Algebra/Ring_Theory}
        \newcommand{\OLDPATH}{\PATH}
    \else
        \newcommand{\OLDPATH}{\PATH}
        \renewcommand{\PATH}{books/Algebra/Ring_Theory}
    \fi
    \chapter{Rings}
        We now add more structure by considering a set with two operations.
        Everything thus far (semi-groups, quasi-groups, monoids, groups) has
        had only one operation associated to it, but in the most fundamental
        forms of arithmetic there are two. The only structure we've encountered
        with two operations so far has been Boolean algebras
        (see Book~\ref{book:Foundations}), but as well will see when we study
        topology, there is essentialy only one type of Boolean algebra and thus
        this study is, in a sense, complete. If we are going to axiomitize some
        algebraic structure it is then wise to avoid all of the properties of
        a Boolean algebra, and so instead we try to model the arithmetic of the
        real numbers. The most fundamental properties can be stated quite
        succintly: $(\mathbb{R},+)$ is an Abelian group and
        $(\mathbb{R},\cdot\,)$ is a monoid. We cannot just leave it there,
        however, since we've no way of knowing how $+$ and $\cdot$ play
        together. As presented, we have two potentially unrelated binary
        operations and thus we cannot procede any further. To complete our
        structure, we add the
        \glslink{distributive operation}{distributive property}.
        \section{Definitions}
    Given a function $f:X\rightarrow{Y}$, and any non-empty subset
    $S\subseteq{X}$, the image $f(S)$ is non-empty. This is not true for the
    pre-image of a function. For let $f:\mathbb{R}\rightarrow\mathbb{R}$ be
    defined by $f(x)=1$ for all $x\in\mathbb{R}$. Then, for any subset
    $S\subset\mathbb{R}$
    such that $1\notin{S}$, we have that $f^{\minus{1}}(S)=\emptyset$.
    There are many examples of functions, but certain ones are easier
    to study than others. We give some of these special functions names.
    \begin{ldefinition}{Injective Functions}{Injective_Function}
        An \gls{injective function} is a function
        $f:X\rightarrow{Y}$ such that, for all
        $x,y\in{X}$ such that $x\ne{y}$, it is true that
        $f(x)\ne{f}(y)$.
    \end{ldefinition}
    That is, an injective function is a function
    $f:X\rightarrow{Y}$ such that $f(x_{1})=f(x_{2})$
    if and only if $x_{1}=x_{2}$. Such functions are also
    called \textit{one-to-one}.
    \begin{lexample}{}{Natural_Log_Is_Injective}
        Consider the natural logarithm
        $\ln:\mathbb{R}^{+}\rightarrow\mathbb{R}$. This is an injective
        function. For let $x,y\in\mathbb{R}^{+}$ be such that
        $x\ne{y}$. Suppose $\ln(x)=\ln(y)$. But then:
        \begin{equation}
            \ln(x)-\ln(y)=\ln\Big(\frac{x}{y}\Big)=0
        \end{equation}
        Recall the definition of the natural logarithm:
        \begin{equation}
            \ln(t)=\int_{1}^{t}\frac{1}{x}\diff{x}
        \end{equation}
        But then $\ln(t)=0$ if and only if $t=1$. Thus $x=y$, a
        contradiction. Therefore $\ln$ is an injective function. Not
        every function is injective, for define
        $f:\mathbb{R}\rightarrow\mathbb{R}$ by $f(x)=x^{2}$. Then, for
        all $x\in\mathbb{R}^{+}$, $f(\minus{x})=f(x)$, and thus $f$
        cannot be an injective function.
    \end{lexample}
    One might think that most functions are not injective,
    and indeed for the \textit{finite} case, this is true.
    For let $A$ and $B$ be finite sets with $n$ and $m$
    elements, respectively. If $m<n$, there can't be
    any injective function. Consider the case when $n=m$.
    Then we are simply counting the number of ways to
    permute the elements of $A$. This is $n!$. On the
    other hand, the total number of functions is
    $n^{n}$. Thus, the ratio of the number of injective
    functions to the number of functions is
    $n!/n^{n}$, and this decays to zero rapidly as
    $n$ get's large. Finally, if $m>n$, then the total
    number of injective functions is
    $n!\binom{m}{n}$, where $\binom{m}{n}$ is the
    binomial coefficient. The total number of functions
    is $n^{m}$. The ratio is thus:
    \begin{equation}
        \frac{n!\binom{m}{n}}{n^{m}}=\frac{n!\frac{m!}{n!(m-n)!}}{n^{m}}
                                    =\frac{m!}{(m-n)!n^{m}}
    \end{equation}
    And again, this decays rapidly to zero and $n$ and $m$
    get large. Later, when we define infinite sets
    and the notion of Cardinality, we'll show that this
    trend continues. That is, in a sense, \textit{most}
    functions from a set $A$ to a sufficiently large set
    $B$ are not injective. Next, we define
    \textit{surjective} functions.
    \begin{ldefinition}{Surjective Functions}{Surjective_Function}
        A \gls{surjective function} is a function
        $f:X\rightarrow{Y}$ such that $f(X)=Y$.
        That is, for all $y\in{Y}$, there is an
        $x\in{X}$ such that $f(x)=y$.
    \end{ldefinition}
    That is, every point $y\in{Y}$ gets mapped to by
    at least one point in $X$. It may also be true that
    many points in $X$ map to the same point in $Y$.
    The notions of surjective functions and injective
    functions are distinct, and neither implies the
    other. Surjective functions are also called
    \textit{onto}.
    \begin{ldefinition}{Bijective Functions}{Bijective_Function}
        A \gls{bijective function} is a function
        that is both injective and surjective.
    \end{ldefinition}
    \begin{theorem}
        \label{thm:Image_of_Empty_Set_Is_Empty}%
        If $A$ and $B$ are sets, and if $f:A\rightarrow{B}$
        is a function, then:
        \begin{equation}
            f(\emptyset)=\emptyset
        \end{equation}
    \end{theorem}
    \begin{theorem}
        If $A$ and $B$ are sets, and if $f:A\rightarrow{B}$
        is a function, then:
        \begin{equation}
            f^{-1}(\emptyset)=\emptyset
        \end{equation}
    \end{theorem}
    \begin{theorem}
        If $X$ and $Y$ are sets, if $A\subseteq{X}$, and if
        $f:X\rightarrow{Y}$ is a function such that
        $f(A)=\emptyset$, then $A=\emptyset$.
    \end{theorem}
    \begin{proof}
        For suppose not. If $A\ne\emptyset$, then there is an $x\in{A}$.
        But then $f(x)\in{f}(A)$, a contradiction as $f(A)=\emptyset$.
    \end{proof}
    \begin{theorem}
        If $X$ and $Y$ are sets, if $B$ is a subset of $Y$,
        and if $f:X\rightarrow{Y}$ is a function, then:
        \begin{equation}
            f\big(f^{-1}(B)\big)\subseteq{B}
        \end{equation}
    \end{theorem}
    \begin{proof}
        For if $y\in{f(f^{-1}(B))}$, then there is an
        $x\in{f^{-1}(B)}$ such that $y=f(x)$. But if
        $x\in{f^{-1}(B)}$, then $f(x)\in{B}$. Thus,
        $y\in{B}$. Therefore, etc.
    \end{proof}
    \begin{theorem}
        If $X$ and $Y$ are non-empty sets and if there exists
        $y_{1},y_{2}\in{Y}$ such that $y_{1}\ne{y}_{2}$, then
        there is a function $f:X\rightarrow{Y}$ and a
        $B\subseteq{Y}$ such that:
        \begin{equation}
            f\big(f^{-1}(B)\big)\ne{B}
        \end{equation}
    \end{theorem}
    \begin{proof}
        \begin{subequations}
            For if $X$ and $Y$ are non-empty, let $f:X\rightarrow{Y}$
            be defined by:
            \begin{equation}
                f=\{(x,y_{1}):x\in{X}\}
            \end{equation}
            Then $f$ is a function, since $f\subseteq{X}\times{Y}$
            as $y_{1}\in{Y}$. Moreover, for all $x\in{X}$ there is a
            unique $y\in{Y}$ such that $(x,y)\in{f}$. Thus, $f$ is a
            function from $X$ to $Y$. However since for all
            $x\in{X}$, $f(x)=y_{1}$, we have that:
            \begin{equation}
                f^{-1}(\{y_{2}\})=\emptyset
            \end{equation}
            For suppose $x\in{f}^{-1}(\{y_{2}\})$.
            Then $f(x)=y_{2}$, but for all $x\in{X}$, $f(x)=y_{1}$,
            and $y_{1}\ne{y}_{2}$. Thus
            $f^{-1}(\{y_{2}\})=\emptyset$. But by
            Thm.~\ref{thm:Image_of_Empty_Set_Is_Empty},
            $f(\emptyset)=\emptyset$. Therefore:
            \begin{equation}
                f\big(f^{-1}(\{y_{2}\})\big)=\emptyset
            \end{equation}
            But $\{y_{2}\}\ne\emptyset$ and
            $\{y_{2}\}\subseteq{Y}$. Therefore, etc.
        \end{subequations}
    \end{proof}
    \begin{theorem}
        If $X$ and $Y$ are sets, if $A$ is a subset of $X$,
        and if $f:X\rightarrow{Y}$ is a function, then:
        \begin{equation}
            A\subseteq{f^{-1}}\big(f(A)\big)
        \end{equation}
    \end{theorem}
    \begin{proof}
        For if $x\in{A}$, then there is a $y\in{f}(A)$ such that
        $f(x)=y$. But then $x\in{f^{-1}(f(A))}$. Therefore, etc.
    \end{proof}
    \begin{theorem}
        If $X$ and $Y$ are sets, if $A_{1}$ and $A_{2}$ are
        subsets of $X$ such that $A_{1}\subseteq{A}_{2}$,
        and if $f:X\rightarrow{Y}$ is a function, then:
        \begin{equation}
            f(A_{1})\subseteq{f}(A_{2})
        \end{equation}
    \end{theorem}
    \begin{proof}
        For if $y\in{f}(A_{1})$, then there is an $x\in{A}_{1}$
        such that $f(x)=y$. But $A_{1}\subseteq{A}_{2}$, and
        therefore $x\in{A}_{2}$. But if $x\in{A}_{2}$, then
        $f(x)\in{f}(A_{2})$. Thus, $y\in{f}(A_{2})$. Therefore, etc.
    \end{proof}
    \begin{theorem}
        If $X$ and $Y$ are sets, if $B_{1}$ and $B_{2}$ are subsets of
        $Y$ such that $B_{1}\subseteq{B}_{2}$, and if $f:X\rightarrow{Y}$
        is a function, then:
        \begin{equation}
            f^{-1}(B_{1})\subseteq{f^{-1}}(B_{2})
        \end{equation}
    \end{theorem}
    \begin{proof}
        For if $x\in{f}^{-1}(B_{1})$, then there is a
        $y\in{B}_{1}$ such that $f(x)=y$. But
        $B_{1}\subseteq{B}_{2}$, and therefore $y\in{B}_{2}$.
        Thus, $x\in{f}^{-1}(B_{2})$. Therefore, etc.
    \end{proof}
    \begin{theorem}
    If $f:A\rightarrow B$, $A_1,A_2\subset A$, then $f(A_1 \cup A_2) = f(A_1)\cup f(A_2)$.
    \end{theorem}
    \begin{proof}
    $[y\in f(A_1\cup A_2)]\Rightarrow [\exists x\in A_1 \cup A_2:y=f(x)]\Rightarrow [y \in f(A_1)\cup f(A_2)]$. $[y\in f(A_1)\cup f(A_2)]\Rightarrow \big[[\exists x\in A_1] \lor[\exists x\in A_2]: y=f(x)\big]\Rightarrow [x\in A_1\cup A_2]\Rightarrow [f(x)\in f(A_1\cup A_2)]$
    \end{proof}
    \begin{theorem}
        If $f:A\rightarrow B$, $A_{1},A_{}2\subset A$, then
        $f(A_{1}\cap{A}_{2})\subset{f}(A_{1})\cap{f}(A_{2})$.
    \end{theorem}
    \begin{proof}
        $[y\in f(A_1 \cap A_2)]\Rightarrow [\exists x\in A_1 \cap A_2:y=f(x)]\Rightarrow [x\in A_1 \land x \in A_2] \Rightarrow[y \in f(A_1)\cap f(A_2)]$.
    \end{proof}
    \begin{theorem}
        If $A$ and $B$ are sets, $f:A\rightarrow{B}$ is a function,
        and $B_{1},B_{2}\subseteq{B}$, then:
        \begin{equation}
            f^{-1}(B_{1}\cup{B}_{2})=f^{-1}(B_{1})\cup{f}^{-1}(B_{2})
        \end{equation}
    \end{theorem}
    \begin{proof}
        For if $x\in{B}_{1}\cup{B}_{2}$, then
        $f(x)\in{B}_{1}\cup{B}_{2}$. but then either
        $f(x)\in{B}_{1}$ or $f(x)\in{B}_{2}$, and therefore
        $x\in{f}^{\minus{1}}(B_1)\cup{f}^{\minus{1}}(B_2)$. But if
        $x\in{f}^{\minus{1}}(B_{1})\cup{f}^{\minus{1}}(B_2)$, then
        $f(x)\in{B}_{1}$ or $f(x)\in{B}_{2}$. Therefore
        $f(x)\in{B}_{1}\cup{B}_{2}$. Thus, $x\in{f}^{-1}(B_1\cup{B}_2)$.
    \end{proof}
    \begin{theorem}
        If $A$ and $B$ are sets, $f:A\rightarrow{B}$ is a function,
        and $B_{1},B_{2}\subseteq{B}$, then:
        \begin{equation}
            f^{-1}(B_{1}\cap{B}_{2})=f^{-1}(B_{1})\cap{f}^{-1}(B_{2})
        \end{equation}
    \end{theorem}
    \begin{proof}
        $[x\in f^{-1}(B_1\cap B_2)]\Rightarrow [f(x) \in B_1 \cap B_2]\Rightarrow [f(x)\in B_1\land f(x) \in B_2 ]\Rightarrow [x\in f^{-1}(B_1)\cap f^{-1}(B_2)]$. $[x\in f^{-1}(B_1)\cap f^{-1}(B_2)]\Rightarrow [x\in f^{-1}(B_1)\land x\in f^{-1}(B_2)]\Rightarrow [f(x) \in B_1\land f(x) \in B_2]\Rightarrow [f(x)\in B_1\cap B_2]\Rightarrow [x\in f^{-1}(B_1\cap B_2)]$.
    \end{proof}
    \begin{theorem}
    If $f:A\rightarrow B$, $B_1 \subset B$, then $f^{-1}(B\setminus B_1) = f^{-1}(B)\setminus f^{-1}(B_1)$.
    \end{theorem}
    \begin{proof}
    $[x\in f^{-1}(B\setminus B_1)]\Leftrightarrow [f(x)\notin B_1]\Leftrightarrow [x\in f^{-1}(B)\setminus f^{-1}(B_1)]$
    \end{proof}
    If $f:A\rightarrow B$, the image of $A$ under $f$
    is often called the range (A is often called the domain).
    \begin{ldefinition}{Permutations}{Permutations}
        A permutation on a set $A$ is a bijective function
        $f:A\rightarrow{A}$.
    \end{ldefinition}
    \begin{theorem}
    If $f:A\rightarrow B$ is bijective, then $f^{-1}$ is bijective.
    \end{theorem}
    \begin{proof}
    $[f^{-1}(y_1) = f^{-1}(y_2)]\Rightarrow [\exists x\in A:[f(x) = y_1]\land [f(x)=y_2]]\Rightarrow [y_1=y_2]$. By definition, $f^{-1}$ is surjective.
    \end{proof}
    \begin{definition}
    If $f:A\rightarrow B$ and $g:B\rightarrow C$, then $g\circ f:A\rightarrow C$ is defined by the image $g(f(x)), x\in A$. 
    \end{definition}
    \begin{theorem}
    If $f:A\rightarrow B$, $g:B\rightarrow C$, and $\mathcal{V}\subset C$, then $(g\circ g)^{-1}(\mathcal{V}) = f^{-1}(g^{-1}(\mathcal{V}))$.
    \end{theorem}
    \begin{proof}
    $[x\in (g\circ f)^{-1}(\mathcal{V})]\Leftrightarrow [g(f(x))\in \mathcal{V}] \Leftrightarrow [f(x)\in g^{-1}(\mathcal{V})]\Leftrightarrow [x\in f^{-1}(g^{-1}(\mathcal{V}))]$.
    \end{proof}
    \begin{theorem}
    If $f:A\rightarrow B$ is bijective, $g:B\rightarrow C$ is bijective, then $g\circ f$ is bijective.
    \end{theorem}
    \begin{proof}
    $\big[[f(A) = B]\land [g(B) = C]\big]\Rightarrow [g(f(A)) = g(B) = C]$. $[g(f(x_1))=g(f(x_2))]\Leftrightarrow [f(x_1)=f(x_2)]\Leftrightarrow [x_1=x_2]$.
    \end{proof}
    \begin{theorem}
    If $f:A\rightarrow B$ is bijective, $A_1\subset A$, and $f(A_1) = B$, then $A_1=A$.
    \end{theorem}
    \begin{proof}
    $\Big[\big[[A_1^c \ne \emptyset]\Rightarrow [f(A_1^c) \ne \emptyset]\big]\land[f(A_1)\cap f(A_1^c) = \emptyset]\Big]\Rightarrow [\exists y\in B:y\notin f(A_1)]$, a contradiction.
    \end{proof}
        \input{\PATH/Elementary_Ring_Theory/Ring_Morphisms.tex}
    \chapter{Fields}
        \section{Definitions}
    Given a function $f:X\rightarrow{Y}$, and any non-empty subset
    $S\subseteq{X}$, the image $f(S)$ is non-empty. This is not true for the
    pre-image of a function. For let $f:\mathbb{R}\rightarrow\mathbb{R}$ be
    defined by $f(x)=1$ for all $x\in\mathbb{R}$. Then, for any subset
    $S\subset\mathbb{R}$
    such that $1\notin{S}$, we have that $f^{\minus{1}}(S)=\emptyset$.
    There are many examples of functions, but certain ones are easier
    to study than others. We give some of these special functions names.
    \begin{ldefinition}{Injective Functions}{Injective_Function}
        An \gls{injective function} is a function
        $f:X\rightarrow{Y}$ such that, for all
        $x,y\in{X}$ such that $x\ne{y}$, it is true that
        $f(x)\ne{f}(y)$.
    \end{ldefinition}
    That is, an injective function is a function
    $f:X\rightarrow{Y}$ such that $f(x_{1})=f(x_{2})$
    if and only if $x_{1}=x_{2}$. Such functions are also
    called \textit{one-to-one}.
    \begin{lexample}{}{Natural_Log_Is_Injective}
        Consider the natural logarithm
        $\ln:\mathbb{R}^{+}\rightarrow\mathbb{R}$. This is an injective
        function. For let $x,y\in\mathbb{R}^{+}$ be such that
        $x\ne{y}$. Suppose $\ln(x)=\ln(y)$. But then:
        \begin{equation}
            \ln(x)-\ln(y)=\ln\Big(\frac{x}{y}\Big)=0
        \end{equation}
        Recall the definition of the natural logarithm:
        \begin{equation}
            \ln(t)=\int_{1}^{t}\frac{1}{x}\diff{x}
        \end{equation}
        But then $\ln(t)=0$ if and only if $t=1$. Thus $x=y$, a
        contradiction. Therefore $\ln$ is an injective function. Not
        every function is injective, for define
        $f:\mathbb{R}\rightarrow\mathbb{R}$ by $f(x)=x^{2}$. Then, for
        all $x\in\mathbb{R}^{+}$, $f(\minus{x})=f(x)$, and thus $f$
        cannot be an injective function.
    \end{lexample}
    One might think that most functions are not injective,
    and indeed for the \textit{finite} case, this is true.
    For let $A$ and $B$ be finite sets with $n$ and $m$
    elements, respectively. If $m<n$, there can't be
    any injective function. Consider the case when $n=m$.
    Then we are simply counting the number of ways to
    permute the elements of $A$. This is $n!$. On the
    other hand, the total number of functions is
    $n^{n}$. Thus, the ratio of the number of injective
    functions to the number of functions is
    $n!/n^{n}$, and this decays to zero rapidly as
    $n$ get's large. Finally, if $m>n$, then the total
    number of injective functions is
    $n!\binom{m}{n}$, where $\binom{m}{n}$ is the
    binomial coefficient. The total number of functions
    is $n^{m}$. The ratio is thus:
    \begin{equation}
        \frac{n!\binom{m}{n}}{n^{m}}=\frac{n!\frac{m!}{n!(m-n)!}}{n^{m}}
                                    =\frac{m!}{(m-n)!n^{m}}
    \end{equation}
    And again, this decays rapidly to zero and $n$ and $m$
    get large. Later, when we define infinite sets
    and the notion of Cardinality, we'll show that this
    trend continues. That is, in a sense, \textit{most}
    functions from a set $A$ to a sufficiently large set
    $B$ are not injective. Next, we define
    \textit{surjective} functions.
    \begin{ldefinition}{Surjective Functions}{Surjective_Function}
        A \gls{surjective function} is a function
        $f:X\rightarrow{Y}$ such that $f(X)=Y$.
        That is, for all $y\in{Y}$, there is an
        $x\in{X}$ such that $f(x)=y$.
    \end{ldefinition}
    That is, every point $y\in{Y}$ gets mapped to by
    at least one point in $X$. It may also be true that
    many points in $X$ map to the same point in $Y$.
    The notions of surjective functions and injective
    functions are distinct, and neither implies the
    other. Surjective functions are also called
    \textit{onto}.
    \begin{ldefinition}{Bijective Functions}{Bijective_Function}
        A \gls{bijective function} is a function
        that is both injective and surjective.
    \end{ldefinition}
    \begin{theorem}
        \label{thm:Image_of_Empty_Set_Is_Empty}%
        If $A$ and $B$ are sets, and if $f:A\rightarrow{B}$
        is a function, then:
        \begin{equation}
            f(\emptyset)=\emptyset
        \end{equation}
    \end{theorem}
    \begin{theorem}
        If $A$ and $B$ are sets, and if $f:A\rightarrow{B}$
        is a function, then:
        \begin{equation}
            f^{-1}(\emptyset)=\emptyset
        \end{equation}
    \end{theorem}
    \begin{theorem}
        If $X$ and $Y$ are sets, if $A\subseteq{X}$, and if
        $f:X\rightarrow{Y}$ is a function such that
        $f(A)=\emptyset$, then $A=\emptyset$.
    \end{theorem}
    \begin{proof}
        For suppose not. If $A\ne\emptyset$, then there is an $x\in{A}$.
        But then $f(x)\in{f}(A)$, a contradiction as $f(A)=\emptyset$.
    \end{proof}
    \begin{theorem}
        If $X$ and $Y$ are sets, if $B$ is a subset of $Y$,
        and if $f:X\rightarrow{Y}$ is a function, then:
        \begin{equation}
            f\big(f^{-1}(B)\big)\subseteq{B}
        \end{equation}
    \end{theorem}
    \begin{proof}
        For if $y\in{f(f^{-1}(B))}$, then there is an
        $x\in{f^{-1}(B)}$ such that $y=f(x)$. But if
        $x\in{f^{-1}(B)}$, then $f(x)\in{B}$. Thus,
        $y\in{B}$. Therefore, etc.
    \end{proof}
    \begin{theorem}
        If $X$ and $Y$ are non-empty sets and if there exists
        $y_{1},y_{2}\in{Y}$ such that $y_{1}\ne{y}_{2}$, then
        there is a function $f:X\rightarrow{Y}$ and a
        $B\subseteq{Y}$ such that:
        \begin{equation}
            f\big(f^{-1}(B)\big)\ne{B}
        \end{equation}
    \end{theorem}
    \begin{proof}
        \begin{subequations}
            For if $X$ and $Y$ are non-empty, let $f:X\rightarrow{Y}$
            be defined by:
            \begin{equation}
                f=\{(x,y_{1}):x\in{X}\}
            \end{equation}
            Then $f$ is a function, since $f\subseteq{X}\times{Y}$
            as $y_{1}\in{Y}$. Moreover, for all $x\in{X}$ there is a
            unique $y\in{Y}$ such that $(x,y)\in{f}$. Thus, $f$ is a
            function from $X$ to $Y$. However since for all
            $x\in{X}$, $f(x)=y_{1}$, we have that:
            \begin{equation}
                f^{-1}(\{y_{2}\})=\emptyset
            \end{equation}
            For suppose $x\in{f}^{-1}(\{y_{2}\})$.
            Then $f(x)=y_{2}$, but for all $x\in{X}$, $f(x)=y_{1}$,
            and $y_{1}\ne{y}_{2}$. Thus
            $f^{-1}(\{y_{2}\})=\emptyset$. But by
            Thm.~\ref{thm:Image_of_Empty_Set_Is_Empty},
            $f(\emptyset)=\emptyset$. Therefore:
            \begin{equation}
                f\big(f^{-1}(\{y_{2}\})\big)=\emptyset
            \end{equation}
            But $\{y_{2}\}\ne\emptyset$ and
            $\{y_{2}\}\subseteq{Y}$. Therefore, etc.
        \end{subequations}
    \end{proof}
    \begin{theorem}
        If $X$ and $Y$ are sets, if $A$ is a subset of $X$,
        and if $f:X\rightarrow{Y}$ is a function, then:
        \begin{equation}
            A\subseteq{f^{-1}}\big(f(A)\big)
        \end{equation}
    \end{theorem}
    \begin{proof}
        For if $x\in{A}$, then there is a $y\in{f}(A)$ such that
        $f(x)=y$. But then $x\in{f^{-1}(f(A))}$. Therefore, etc.
    \end{proof}
    \begin{theorem}
        If $X$ and $Y$ are sets, if $A_{1}$ and $A_{2}$ are
        subsets of $X$ such that $A_{1}\subseteq{A}_{2}$,
        and if $f:X\rightarrow{Y}$ is a function, then:
        \begin{equation}
            f(A_{1})\subseteq{f}(A_{2})
        \end{equation}
    \end{theorem}
    \begin{proof}
        For if $y\in{f}(A_{1})$, then there is an $x\in{A}_{1}$
        such that $f(x)=y$. But $A_{1}\subseteq{A}_{2}$, and
        therefore $x\in{A}_{2}$. But if $x\in{A}_{2}$, then
        $f(x)\in{f}(A_{2})$. Thus, $y\in{f}(A_{2})$. Therefore, etc.
    \end{proof}
    \begin{theorem}
        If $X$ and $Y$ are sets, if $B_{1}$ and $B_{2}$ are subsets of
        $Y$ such that $B_{1}\subseteq{B}_{2}$, and if $f:X\rightarrow{Y}$
        is a function, then:
        \begin{equation}
            f^{-1}(B_{1})\subseteq{f^{-1}}(B_{2})
        \end{equation}
    \end{theorem}
    \begin{proof}
        For if $x\in{f}^{-1}(B_{1})$, then there is a
        $y\in{B}_{1}$ such that $f(x)=y$. But
        $B_{1}\subseteq{B}_{2}$, and therefore $y\in{B}_{2}$.
        Thus, $x\in{f}^{-1}(B_{2})$. Therefore, etc.
    \end{proof}
    \begin{theorem}
    If $f:A\rightarrow B$, $A_1,A_2\subset A$, then $f(A_1 \cup A_2) = f(A_1)\cup f(A_2)$.
    \end{theorem}
    \begin{proof}
    $[y\in f(A_1\cup A_2)]\Rightarrow [\exists x\in A_1 \cup A_2:y=f(x)]\Rightarrow [y \in f(A_1)\cup f(A_2)]$. $[y\in f(A_1)\cup f(A_2)]\Rightarrow \big[[\exists x\in A_1] \lor[\exists x\in A_2]: y=f(x)\big]\Rightarrow [x\in A_1\cup A_2]\Rightarrow [f(x)\in f(A_1\cup A_2)]$
    \end{proof}
    \begin{theorem}
        If $f:A\rightarrow B$, $A_{1},A_{}2\subset A$, then
        $f(A_{1}\cap{A}_{2})\subset{f}(A_{1})\cap{f}(A_{2})$.
    \end{theorem}
    \begin{proof}
        $[y\in f(A_1 \cap A_2)]\Rightarrow [\exists x\in A_1 \cap A_2:y=f(x)]\Rightarrow [x\in A_1 \land x \in A_2] \Rightarrow[y \in f(A_1)\cap f(A_2)]$.
    \end{proof}
    \begin{theorem}
        If $A$ and $B$ are sets, $f:A\rightarrow{B}$ is a function,
        and $B_{1},B_{2}\subseteq{B}$, then:
        \begin{equation}
            f^{-1}(B_{1}\cup{B}_{2})=f^{-1}(B_{1})\cup{f}^{-1}(B_{2})
        \end{equation}
    \end{theorem}
    \begin{proof}
        For if $x\in{B}_{1}\cup{B}_{2}$, then
        $f(x)\in{B}_{1}\cup{B}_{2}$. but then either
        $f(x)\in{B}_{1}$ or $f(x)\in{B}_{2}$, and therefore
        $x\in{f}^{\minus{1}}(B_1)\cup{f}^{\minus{1}}(B_2)$. But if
        $x\in{f}^{\minus{1}}(B_{1})\cup{f}^{\minus{1}}(B_2)$, then
        $f(x)\in{B}_{1}$ or $f(x)\in{B}_{2}$. Therefore
        $f(x)\in{B}_{1}\cup{B}_{2}$. Thus, $x\in{f}^{-1}(B_1\cup{B}_2)$.
    \end{proof}
    \begin{theorem}
        If $A$ and $B$ are sets, $f:A\rightarrow{B}$ is a function,
        and $B_{1},B_{2}\subseteq{B}$, then:
        \begin{equation}
            f^{-1}(B_{1}\cap{B}_{2})=f^{-1}(B_{1})\cap{f}^{-1}(B_{2})
        \end{equation}
    \end{theorem}
    \begin{proof}
        $[x\in f^{-1}(B_1\cap B_2)]\Rightarrow [f(x) \in B_1 \cap B_2]\Rightarrow [f(x)\in B_1\land f(x) \in B_2 ]\Rightarrow [x\in f^{-1}(B_1)\cap f^{-1}(B_2)]$. $[x\in f^{-1}(B_1)\cap f^{-1}(B_2)]\Rightarrow [x\in f^{-1}(B_1)\land x\in f^{-1}(B_2)]\Rightarrow [f(x) \in B_1\land f(x) \in B_2]\Rightarrow [f(x)\in B_1\cap B_2]\Rightarrow [x\in f^{-1}(B_1\cap B_2)]$.
    \end{proof}
    \begin{theorem}
    If $f:A\rightarrow B$, $B_1 \subset B$, then $f^{-1}(B\setminus B_1) = f^{-1}(B)\setminus f^{-1}(B_1)$.
    \end{theorem}
    \begin{proof}
    $[x\in f^{-1}(B\setminus B_1)]\Leftrightarrow [f(x)\notin B_1]\Leftrightarrow [x\in f^{-1}(B)\setminus f^{-1}(B_1)]$
    \end{proof}
    If $f:A\rightarrow B$, the image of $A$ under $f$
    is often called the range (A is often called the domain).
    \begin{ldefinition}{Permutations}{Permutations}
        A permutation on a set $A$ is a bijective function
        $f:A\rightarrow{A}$.
    \end{ldefinition}
    \begin{theorem}
    If $f:A\rightarrow B$ is bijective, then $f^{-1}$ is bijective.
    \end{theorem}
    \begin{proof}
    $[f^{-1}(y_1) = f^{-1}(y_2)]\Rightarrow [\exists x\in A:[f(x) = y_1]\land [f(x)=y_2]]\Rightarrow [y_1=y_2]$. By definition, $f^{-1}$ is surjective.
    \end{proof}
    \begin{definition}
    If $f:A\rightarrow B$ and $g:B\rightarrow C$, then $g\circ f:A\rightarrow C$ is defined by the image $g(f(x)), x\in A$. 
    \end{definition}
    \begin{theorem}
    If $f:A\rightarrow B$, $g:B\rightarrow C$, and $\mathcal{V}\subset C$, then $(g\circ g)^{-1}(\mathcal{V}) = f^{-1}(g^{-1}(\mathcal{V}))$.
    \end{theorem}
    \begin{proof}
    $[x\in (g\circ f)^{-1}(\mathcal{V})]\Leftrightarrow [g(f(x))\in \mathcal{V}] \Leftrightarrow [f(x)\in g^{-1}(\mathcal{V})]\Leftrightarrow [x\in f^{-1}(g^{-1}(\mathcal{V}))]$.
    \end{proof}
    \begin{theorem}
    If $f:A\rightarrow B$ is bijective, $g:B\rightarrow C$ is bijective, then $g\circ f$ is bijective.
    \end{theorem}
    \begin{proof}
    $\big[[f(A) = B]\land [g(B) = C]\big]\Rightarrow [g(f(A)) = g(B) = C]$. $[g(f(x_1))=g(f(x_2))]\Leftrightarrow [f(x_1)=f(x_2)]\Leftrightarrow [x_1=x_2]$.
    \end{proof}
    \begin{theorem}
    If $f:A\rightarrow B$ is bijective, $A_1\subset A$, and $f(A_1) = B$, then $A_1=A$.
    \end{theorem}
    \begin{proof}
    $\Big[\big[[A_1^c \ne \emptyset]\Rightarrow [f(A_1^c) \ne \emptyset]\big]\land[f(A_1)\cap f(A_1^c) = \emptyset]\Big]\Rightarrow [\exists y\in B:y\notin f(A_1)]$, a contradiction.
    \end{proof}
    \renewcommand{\PATH}{\OLDPATH}
\endgroup