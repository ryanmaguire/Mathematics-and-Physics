%------------------------------------------------------------------------------%
\documentclass{article}                                                        %
%------------------------------Preamble----------------------------------------%
\makeatletter                                                                  %
    \def\input@path{{../../}}                                                  %
\makeatother                                                                   %
%---------------------------Packages----------------------------%
\usepackage{geometry}
\geometry{b5paper, margin=1.0in}
\usepackage[T1]{fontenc}
\usepackage{graphicx, float}            % Graphics/Images.
\usepackage{natbib}                     % For bibliographies.
\bibliographystyle{agsm}                % Bibliography style.
\usepackage[french, english]{babel}     % Language typesetting.
\usepackage[dvipsnames]{xcolor}         % Color names.
\usepackage{listings}                   % Verbatim-Like Tools.
\usepackage{mathtools, esint, mathrsfs} % amsmath and integrals.
\usepackage{amsthm, amsfonts, amssymb}  % Fonts and theorems.
\usepackage{tcolorbox}                  % Frames around theorems.
\usepackage{upgreek}                    % Non-Italic Greek.
\usepackage{fmtcount, etoolbox}         % For the \book{} command.
\usepackage[newparttoc]{titlesec}       % Formatting chapter, etc.
\usepackage{titletoc}                   % Allows \book in toc.
\usepackage[nottoc]{tocbibind}          % Bibliography in toc.
\usepackage[titles]{tocloft}            % ToC formatting.
\usepackage{pgfplots, tikz}             % Drawing/graphing tools.
\usepackage{imakeidx}                   % Used for index.
\usetikzlibrary{
    calc,                   % Calculating right angles and more.
    angles,                 % Drawing angles within triangles.
    arrows.meta,            % Latex and Stealth arrows.
    quotes,                 % Adding labels to angles.
    positioning,            % Relative positioning of nodes.
    decorations.markings,   % Adding arrows in the middle of a line.
    patterns,
    arrows
}                                       % Libraries for tikz.
\pgfplotsset{compat=1.9}                % Version of pgfplots.
\usepackage[font=scriptsize,
            labelformat=simple,
            labelsep=colon]{subcaption} % Subfigure captions.
\usepackage[font={scriptsize},
            hypcap=true,
            labelsep=colon]{caption}    % Figure captions.
\usepackage[pdftex,
            pdfauthor={Ryan Maguire},
            pdftitle={Mathematics and Physics},
            pdfsubject={Mathematics, Physics, Science},
            pdfkeywords={Mathematics, Physics, Computer Science, Biology},
            pdfproducer={LaTeX},
            pdfcreator={pdflatex}]{hyperref}
\hypersetup{
    colorlinks=true,
    linkcolor=blue,
    filecolor=magenta,
    urlcolor=Cerulean,
    citecolor=SkyBlue
}                           % Colors for hyperref.
\usepackage[toc,acronym,nogroupskip,nopostdot]{glossaries}
\usepackage{glossary-mcols}
%------------------------Theorem Styles-------------------------%
\theoremstyle{plain}
\newtheorem{theorem}{Theorem}[section]

% Define theorem style for default spacing and normal font.
\newtheoremstyle{normal}
    {\topsep}               % Amount of space above the theorem.
    {\topsep}               % Amount of space below the theorem.
    {}                      % Font used for body of theorem.
    {}                      % Measure of space to indent.
    {\bfseries}             % Font of the header of the theorem.
    {}                      % Punctuation between head and body.
    {.5em}                  % Space after theorem head.
    {}

% Italic header environment.
\newtheoremstyle{thmit}{\topsep}{\topsep}{}{}{\itshape}{}{0.5em}{}

% Define environments with italic headers.
\theoremstyle{thmit}
\newtheorem*{solution}{Solution}

% Define default environments.
\theoremstyle{normal}
\newtheorem{example}{Example}[section]
\newtheorem{definition}{Definition}[section]
\newtheorem{problem}{Problem}[section]

% Define framed environment.
\tcbuselibrary{most}
\newtcbtheorem[use counter*=theorem]{ftheorem}{Theorem}{%
    before=\par\vspace{2ex},
    boxsep=0.5\topsep,
    after=\par\vspace{2ex},
    colback=green!5,
    colframe=green!35!black,
    fonttitle=\bfseries\upshape%
}{thm}

\newtcbtheorem[auto counter, number within=section]{faxiom}{Axiom}{%
    before=\par\vspace{2ex},
    boxsep=0.5\topsep,
    after=\par\vspace{2ex},
    colback=Apricot!5,
    colframe=Apricot!35!black,
    fonttitle=\bfseries\upshape%
}{ax}

\newtcbtheorem[use counter*=definition]{fdefinition}{Definition}{%
    before=\par\vspace{2ex},
    boxsep=0.5\topsep,
    after=\par\vspace{2ex},
    colback=blue!5!white,
    colframe=blue!75!black,
    fonttitle=\bfseries\upshape%
}{def}

\newtcbtheorem[use counter*=example]{fexample}{Example}{%
    before=\par\vspace{2ex},
    boxsep=0.5\topsep,
    after=\par\vspace{2ex},
    colback=red!5!white,
    colframe=red!75!black,
    fonttitle=\bfseries\upshape%
}{ex}

\newtcbtheorem[auto counter, number within=section]{fnotation}{Notation}{%
    before=\par\vspace{2ex},
    boxsep=0.5\topsep,
    after=\par\vspace{2ex},
    colback=SeaGreen!5!white,
    colframe=SeaGreen!75!black,
    fonttitle=\bfseries\upshape%
}{not}

\newtcbtheorem[use counter*=remark]{fremark}{Remark}{%
    fonttitle=\bfseries\upshape,
    colback=Goldenrod!5!white,
    colframe=Goldenrod!75!black}{ex}

\newenvironment{bproof}{\textit{Proof.}}{\hfill$\square$}
\tcolorboxenvironment{bproof}{%
    blanker,
    breakable,
    left=3mm,
    before skip=5pt,
    after skip=10pt,
    borderline west={0.6mm}{0pt}{green!80!black}
}

\AtEndEnvironment{lexample}{$\hfill\textcolor{red}{\blacksquare}$}
\newtcbtheorem[use counter*=example]{lexample}{Example}{%
    empty,
    title={Example~\theexample},
    boxed title style={%
        empty,
        size=minimal,
        toprule=2pt,
        top=0.5\topsep,
    },
    coltitle=red,
    fonttitle=\bfseries,
    parbox=false,
    boxsep=0pt,
    before=\par\vspace{2ex},
    left=0pt,
    right=0pt,
    top=3ex,
    bottom=1ex,
    before=\par\vspace{2ex},
    after=\par\vspace{2ex},
    breakable,
    pad at break*=0mm,
    vfill before first,
    overlay unbroken={%
        \draw[red, line width=2pt]
            ([yshift=-1.2ex]title.south-|frame.west) to
            ([yshift=-1.2ex]title.south-|frame.east);
        },
    overlay first={%
        \draw[red, line width=2pt]
            ([yshift=-1.2ex]title.south-|frame.west) to
            ([yshift=-1.2ex]title.south-|frame.east);
    },
}{ex}

\AtEndEnvironment{ldefinition}{$\hfill\textcolor{Blue}{\blacksquare}$}
\newtcbtheorem[use counter*=definition]{ldefinition}{Definition}{%
    empty,
    title={Definition~\thedefinition:~{#1}},
    boxed title style={%
        empty,
        size=minimal,
        toprule=2pt,
        top=0.5\topsep,
    },
    coltitle=Blue,
    fonttitle=\bfseries,
    parbox=false,
    boxsep=0pt,
    before=\par\vspace{2ex},
    left=0pt,
    right=0pt,
    top=3ex,
    bottom=0pt,
    before=\par\vspace{2ex},
    after=\par\vspace{1ex},
    breakable,
    pad at break*=0mm,
    vfill before first,
    overlay unbroken={%
        \draw[Blue, line width=2pt]
            ([yshift=-1.2ex]title.south-|frame.west) to
            ([yshift=-1.2ex]title.south-|frame.east);
        },
    overlay first={%
        \draw[Blue, line width=2pt]
            ([yshift=-1.2ex]title.south-|frame.west) to
            ([yshift=-1.2ex]title.south-|frame.east);
    },
}{def}

\AtEndEnvironment{ltheorem}{$\hfill\textcolor{Green}{\blacksquare}$}
\newtcbtheorem[use counter*=theorem]{ltheorem}{Theorem}{%
    empty,
    title={Theorem~\thetheorem:~{#1}},
    boxed title style={%
        empty,
        size=minimal,
        toprule=2pt,
        top=0.5\topsep,
    },
    coltitle=Green,
    fonttitle=\bfseries,
    parbox=false,
    boxsep=0pt,
    before=\par\vspace{2ex},
    left=0pt,
    right=0pt,
    top=3ex,
    bottom=-1.5ex,
    breakable,
    pad at break*=0mm,
    vfill before first,
    overlay unbroken={%
        \draw[Green, line width=2pt]
            ([yshift=-1.2ex]title.south-|frame.west) to
            ([yshift=-1.2ex]title.south-|frame.east);},
    overlay first={%
        \draw[Green, line width=2pt]
            ([yshift=-1.2ex]title.south-|frame.west) to
            ([yshift=-1.2ex]title.south-|frame.east);
    }
}{thm}

%--------------------Declared Math Operators--------------------%
\DeclareMathOperator{\adjoint}{adj}         % Adjoint.
\DeclareMathOperator{\Card}{Card}           % Cardinality.
\DeclareMathOperator{\curl}{curl}           % Curl.
\DeclareMathOperator{\diam}{diam}           % Diameter.
\DeclareMathOperator{\dist}{dist}           % Distance.
\DeclareMathOperator{\Div}{div}             % Divergence.
\DeclareMathOperator{\Erf}{Erf}             % Error Function.
\DeclareMathOperator{\Erfc}{Erfc}           % Complementary Error Function.
\DeclareMathOperator{\Ext}{Ext}             % Exterior.
\DeclareMathOperator{\GCD}{GCD}             % Greatest common denominator.
\DeclareMathOperator{\grad}{grad}           % Gradient
\DeclareMathOperator{\Ima}{Im}              % Image.
\DeclareMathOperator{\Int}{Int}             % Interior.
\DeclareMathOperator{\LC}{LC}               % Leading coefficient.
\DeclareMathOperator{\LCM}{LCM}             % Least common multiple.
\DeclareMathOperator{\LM}{LM}               % Leading monomial.
\DeclareMathOperator{\LT}{LT}               % Leading term.
\DeclareMathOperator{\Mod}{mod}             % Modulus.
\DeclareMathOperator{\Mon}{Mon}             % Monomial.
\DeclareMathOperator{\multideg}{mutlideg}   % Multi-Degree (Graphs).
\DeclareMathOperator{\nul}{nul}             % Null space of operator.
\DeclareMathOperator{\Ord}{Ord}             % Ordinal of ordered set.
\DeclareMathOperator{\Prin}{Prin}           % Principal value.
\DeclareMathOperator{\proj}{proj}           % Projection.
\DeclareMathOperator{\Refl}{Refl}           % Reflection operator.
\DeclareMathOperator{\rk}{rk}               % Rank of operator.
\DeclareMathOperator{\sgn}{sgn}             % Sign of a number.
\DeclareMathOperator{\sinc}{sinc}           % Sinc function.
\DeclareMathOperator{\Span}{Span}           % Span of a set.
\DeclareMathOperator{\Spec}{Spec}           % Spectrum.
\DeclareMathOperator{\supp}{supp}           % Support
\DeclareMathOperator{\Tr}{Tr}               % Trace of matrix.
%--------------------Declared Math Symbols--------------------%
\DeclareMathSymbol{\minus}{\mathbin}{AMSa}{"39} % Unary minus sign.
%------------------------New Commands---------------------------%
\DeclarePairedDelimiter\norm{\lVert}{\rVert}
\DeclarePairedDelimiter\ceil{\lceil}{\rceil}
\DeclarePairedDelimiter\floor{\lfloor}{\rfloor}
\newcommand*\diff{\mathop{}\!\mathrm{d}}
\newcommand*\Diff[1]{\mathop{}\!\mathrm{d^#1}}
\renewcommand*{\glstextformat}[1]{\textcolor{RoyalBlue}{#1}}
\renewcommand{\glsnamefont}[1]{\textbf{#1}}
\renewcommand\labelitemii{$\circ$}
\renewcommand\thesubfigure{%
    \arabic{chapter}.\arabic{figure}.\arabic{subfigure}}
\addto\captionsenglish{\renewcommand{\figurename}{Fig.}}
\numberwithin{equation}{section}

\renewcommand{\vector}[1]{\boldsymbol{\mathrm{#1}}}

\newcommand{\uvector}[1]{\boldsymbol{\hat{\mathrm{#1}}}}
\newcommand{\topspace}[2][]{(#2,\tau_{#1})}
\newcommand{\measurespace}[2][]{(#2,\varSigma_{#1},\mu_{#1})}
\newcommand{\measurablespace}[2][]{(#2,\varSigma_{#1})}
\newcommand{\manifold}[2][]{(#2,\tau_{#1},\mathcal{A}_{#1})}
\newcommand{\tanspace}[2]{T_{#1}{#2}}
\newcommand{\cotanspace}[2]{T_{#1}^{*}{#2}}
\newcommand{\Ckspace}[3][\mathbb{R}]{C^{#2}(#3,#1)}
\newcommand{\funcspace}[2][\mathbb{R}]{\mathcal{F}(#2,#1)}
\newcommand{\smoothvecf}[1]{\mathfrak{X}(#1)}
\newcommand{\smoothonef}[1]{\mathfrak{X}^{*}(#1)}
\newcommand{\bracket}[2]{[#1,#2]}

%------------------------Book Command---------------------------%
\makeatletter
\renewcommand\@pnumwidth{1cm}
\newcounter{book}
\renewcommand\thebook{\@Roman\c@book}
\newcommand\book{%
    \if@openright
        \cleardoublepage
    \else
        \clearpage
    \fi
    \thispagestyle{plain}%
    \if@twocolumn
        \onecolumn
        \@tempswatrue
    \else
        \@tempswafalse
    \fi
    \null\vfil
    \secdef\@book\@sbook
}
\def\@book[#1]#2{%
    \refstepcounter{book}
    \addcontentsline{toc}{book}{\bookname\ \thebook:\hspace{1em}#1}
    \markboth{}{}
    {\centering
     \interlinepenalty\@M
     \normalfont
     \huge\bfseries\bookname\nobreakspace\thebook
     \par
     \vskip 20\p@
     \Huge\bfseries#2\par}%
    \@endbook}
\def\@sbook#1{%
    {\centering
     \interlinepenalty \@M
     \normalfont
     \Huge\bfseries#1\par}%
    \@endbook}
\def\@endbook{
    \vfil\newpage
        \if@twoside
            \if@openright
                \null
                \thispagestyle{empty}%
                \newpage
            \fi
        \fi
        \if@tempswa
            \twocolumn
        \fi
}
\newcommand*\l@book[2]{%
    \ifnum\c@tocdepth >-3\relax
        \addpenalty{-\@highpenalty}%
        \addvspace{2.25em\@plus\p@}%
        \setlength\@tempdima{3em}%
        \begingroup
            \parindent\z@\rightskip\@pnumwidth
            \parfillskip -\@pnumwidth
            {
                \leavevmode
                \Large\bfseries#1\hfill\hb@xt@\@pnumwidth{\hss#2}
            }
            \par
            \nobreak
            \global\@nobreaktrue
            \everypar{\global\@nobreakfalse\everypar{}}%
        \endgroup
    \fi}
\newcommand\bookname{Book}
\renewcommand{\thebook}{\texorpdfstring{\Numberstring{book}}{book}}
\providecommand*{\toclevel@book}{-2}
\makeatother
\titleformat{\part}[display]
    {\Large\bfseries}
    {\partname\nobreakspace\thepart}
    {0mm}
    {\Huge\bfseries}
\titlecontents{part}[0pt]
    {\large\bfseries}
    {\partname\ \thecontentslabel: \quad}
    {}
    {\hfill\contentspage}
\titlecontents{chapter}[0pt]
    {\bfseries}
    {\chaptername\ \thecontentslabel:\quad}
    {}
    {\hfill\contentspage}
\newglossarystyle{longpara}{%
    \setglossarystyle{long}%
    \renewenvironment{theglossary}{%
        \begin{longtable}[l]{{p{0.25\hsize}p{0.65\hsize}}}
    }{\end{longtable}}%
    \renewcommand{\glossentry}[2]{%
        \glstarget{##1}{\glossentryname{##1}}%
        &\glossentrydesc{##1}{~##2.}
        \tabularnewline%
        \tabularnewline
    }%
}
\newglossary[not-glg]{notation}{not-gls}{not-glo}{Notation}
\newcommand*{\newnotation}[4][]{%
    \newglossaryentry{#2}{type=notation, name={\textbf{#3}, },
                          text={#4}, description={#4},#1}%
}
%--------------------------LENGTHS------------------------------%
% Spacings for the Table of Contents.
\addtolength{\cftsecnumwidth}{1ex}
\addtolength{\cftsubsecindent}{1ex}
\addtolength{\cftsubsecnumwidth}{1ex}
\addtolength{\cftfignumwidth}{1ex}
\addtolength{\cfttabnumwidth}{1ex}

% Indent and paragraph spacing.
\setlength{\parindent}{0em}
\setlength{\parskip}{0em}                                                           %
%----------------------------Main Document-------------------------------------%
\begin{document}
    \title{Algebra}
    \author{Ryan Maguire}
    \date{\vspace{-5ex}}
    \maketitle
    \setcounter{section}{1}
    \begin{problem}
        Divide $f(x)=x^{4}-1$ by $g(x)=3x^{2}+3x$ in terms of quotient with
        remainder. Find the GCD and write $GCD=af+bg$ with $a,b\in\mathbb{Q}[x]$
        using Bezout's identity.
    \end{problem}
    \begin{solution}
        We note dividing by $x^{3}$ is useless, since $x^{3}g(x)$ has higher
        degree than $f$, and so we move on to $x^{2}$ achieving:
        \begin{equation}
            x^{4}-1=\big(3x^{2}+3x\big)\Big(\frac{x^{2}}{3}\Big)
                -(x^{3}+1)
        \end{equation}
        For the remainder we now look to dividing by a linear term, getting:
        \begin{equation}
            x^{4}-1=(3x^{2}+3x)\Big(\frac{x^{2}}{3}-\frac{x}{3}\Big)+(x^{2}-1)
        \end{equation}
        And finially, the constant term:
        \begin{equation}
            x^{4}-1=(3x^{2}+3)\Big(\frac{x^{2}}{3}-\frac{x}{3}+\frac{1}{3}\Big)
                -(x+1)
        \end{equation}
        So the quotient is $x^{2}/3-x/3+1/3$ and the remainder is $\minus(x+1)$.
        From this we get that the GCD is $x+1$. We can also see this since:
        \begin{equation}
            x^{4}-1=(x^{2}+1)(x^{2}-1)=(x^{2}+1)(x-1)(x+1)
        \end{equation}
        And moreover:
        \begin{equation}
            3x^{2}+3x=(3x)(x+1)
        \end{equation}
        Using Bezout's identity, we can simply rearrange:
        \begin{equation}
            x+1=(x^{3}+3)\Big(\frac{x^{2}}{3}-\frac{x}{3}+\frac{1}{3}\Big)
                -(x^{4}-1)
        \end{equation}
    \end{solution}
    \begin{problem}
        Factor some polynomials, if possible.
    \end{problem}
    \begin{solution}
        $x^{4}+1$ is reducible over $\mathbb{R}$ since it is a polynomial of
        degree higher than 2. We have:
        \begin{equation}
            x^{4}+1=(x^{2}+\sqrt{2}x+1)(x^{2}-\sqrt{2}x+1)
        \end{equation}
        Since the roots in $\mathbb{R}$ are irrational, it is irreducible over
        $\mathbb{Q}$. For $x^{7}+11x^{3}-33x+22\in\mathbb{Q}[x]$, we try a few
        prime and then finally set on $p=5$, obtaining the new polynomial:
        \begin{equation}
            f(x)=x^{7}+x^{3}+2x+2=x(x^{6}+x^{2}+2)+2
        \end{equation}
        We plug in 0, 1, 2, $\minus{1}$, and $\minus{2}$ and see that this is
        never zero. Since $f$ is irreducible over $\mathbb{Z}_{5}$, and since
        $f$ is monic, it is thus irreducible over $\mathbb{Q}$. For
        $x^{4}+x^{3}+x^{2}+x+1$ we apply the sliding trick from before:
        \begin{equation}
            f(x+1)=\sum_{k=0}^{4}(x+1)^{k}
            =\sum_{k=0}^{4}\sum_{j=0}^{k}\binom{k}{j}x^{j}
        \end{equation}
        Simplifying this out, we obtain:
        \begin{equation}
            x^{4}+5x^{3}+10x^{2}+10x+5
        \end{equation}
        The prime 5 instantly jumps out, and we notice that $5^{2}=25$ does not
        divide 5, and hence by Eisenstein's criterion, this is irreducible.
        Lastly, for $x^{7}-7x^{2}+3x+3$ we try a few primes, note that nothing
        works, and eventually convince ourselves that maybe it is reducible.
        Then we notice that 1 is a root since
        $1-7+3+3=0$, and so $x-1$ is a factor. Hence, the polynomial is
        reducible.
    \end{solution}
    \begin{problem}
        Determine all monic irreducible polynomials over $\mathbb{F}_{3}$ of
        degree less than or equal to 4.
    \end{problem}
    \begin{solution}
        We solve this by looking at all \textit{reducible} polynomials.
        We start with constants, and since it must be monic the polynomial is
        simply $f(x)=1$. Next, for linear polynomials, we have
        $x+a$ for some element $a\in\mathbb{Z}_{3}$. We note that
        $x+1$ has 2 as a root, $x+2$ has 1 as a root, and $x$ has 0 as a root.
        But we knew everything would have a root since this is a linear
        polynomial and $\mathbb{Z}_{3}$ is a field. For squares we look at
        $x^{2}+ax+b$. If 0 is a root, $b$ must be 0. If $1$ is a root,
        $a=\minus{b}$, and so we have $x^{2}+ax-a$. If $2$ is a root, then
        $1+2a+b=0$. In other words, $b=\minus(1+2a)=2-2a$. All other quadratics
        will be irreducible. For cubics we have
        $x^{3}+ax^{2}+bx+c$. If 0 is a root, $c=0$. If 1 is a root, we get the
        formula $a+b+c=0$. Lastly, if 2 is a root, then $\minus{1}+2a+2b+c=0$,
        which we can rearrange to obtain $c-a-b=1$. Lastly, the
        quartics, if $x^{4}+ax^{3}+bx^{2}+cx+d$ has 0 as a root, then $d=0$.
        If 1 is a root, then $1+a+b+c+d=0$, or $a+b+c+d=2$. Finally, if 2 is a
        root, then $1-a+b-c+d=0$. In general, for the $n^{th}$ degree polynomial
        we have zero as a root if and only if the constant term is zero, 1 as a
        root if and only if:
        \begin{equation}
            1+\sum_{k=0}^{n-1}a_{k}=0
        \end{equation}
        and 2 as a root if and only if:
        \begin{equation}
            (\minus{1})^{n}+\sum_{k=0}^{n-1}(\minus{1})^{k}a_{k}=0
        \end{equation}
    \end{solution}
    \begin{problem}
        Prove that $f,g\in\mathbb{Z}[x]$ are relatively prime in $\mathbb{Q}[x]$
        if and only if $(f,g)\subseteq\mathbb{Z}[x]$ contains a non-zero
        integer.
    \end{problem}
    \begin{solution}
        Going one way, if $f$ and $g$ are relatively prime in $\mathbb{Q}$, then
        their GCD is 1. Hence, by Bezout's identity, there are polynomials
        $a,b\in\mathbb{Q}[x]$ such that $af+bg=1$. If we clear all of the
        denominators we get $Naf+Nbg=N$, and $Na,Nb\in\mathbb{Z}[x]$. But then
        $N\in(f,g)\subseteq\mathbb{Z}[x]$ is a non-zero integer. Going the other
        way, if $N\in(f,g)$ is non-zero then there are polynomials
        $a,b\in\mathbb{Z}$ such that $af+bg=N$. But then
        $(a/N)f+(b/N)g=1$, and $a/N,b/N\in\mathbb{Q}[x]$. Hence, $f$ and $g$ are
        relatively prime in $\mathbb{Q}[x]$.
    \end{solution}
    \begin{problem}
        Some problems with the Vendermonde matrix.
    \end{problem}
    \begin{solution}
        Find the determinate. We proceed by induction. In the base case, we have
        the following:
        \begin{equation}
            \textrm{det}\Big(
                \begin{bmatrix}
                    1&x_{1}\\
                    1&x_{2}
                \end{bmatrix}
            \Big)=
            x_{2}-x_{1}
        \end{equation}
        And so we are done trivially. Suppose the result holds for
        $n\in\mathbb{N}$. Let $A$ be the $n+1$ Vandermonde matrix. We clear out
        the right-most column:
        \begin{align}
            A_{1}&=
            \begin{bmatrix}
                1&x_{1}&x_{1}^{2}&\dots&x_{1}^{n-1}&x_{1}^{n}\\
                1&x_{2}&x_{2}^{2}&\dots&x_{2}^{n-1}&x_{2}^{n}\\
                \vdots&\vdots&\vdots&\ddots&\vdots&\vdots\\
                1&x_{n+1}&x_{n+1}^{2}&\dots&x_{n+1}^{n-1}&x_{n+1}^{n}
            \end{bmatrix}\\
            A_{2}&=
            \begin{bmatrix}
                1&x_{1}&x_{1}^{2}&\dots&x_{1}^{n-1}&0\\
                1&x_{2}&x_{2}^{2}&\dots&x_{2}^{n-1}&x_{2}^{n}-x_{1}x_{2}^{n-1}\\
                \vdots&\vdots&\vdots&\ddots&\vdots&\vdots\\
                1&x_{n+1}&x_{n+1}^{2}&\dots&x_{n+1}^{n-1}
                    &x_{n+1}^{n}-x_{1}x_{n+1}^{n-1}
            \end{bmatrix}\\
            &\vdots\\
            A_{n}&=
            \begin{bmatrix}
                1&0&\dots&0\\
                1&x_{2}-x_{1}&\dots&x_{2}^{n}-x_{1}x_{2}^{n-1}\\
                \vdots&\vdots&\ddots&\vdots\\
                1&x_{n+1}-x_{1}&\dots&x_{n+1}^{n}-x_{1}x_{n+1}^{n-1}
            \end{bmatrix}
        \end{align}
        The determinant along the top row is now simply the determinant of the
        minor matrix obtained from deleting the first row and first column.
        Next we need to use multilinearity. However, we can factor out an
        $x_{j}-x_{1}$ from every element of this matrix, obtaining:
        \begin{equation}
            A_{n}=
            \begin{bmatrix}
                1&0&\dots&0\\
                1&x_{2}-x_{1}&\dots&x_{2}^{n-1}(x_{2}-x_{1})\\
                \vdots&\vdots&\ddots&\vdots\\
                1&x_{n+1}-x_{1}&\dots&x_{n+1}^{n-1}(x_{n+1}-x_{1})
            \end{bmatrix}
        \end{equation}
        Invoking multilinearity, we have:
        \begin{equation}
            \textrm{det}(A_{n})=
            \prod_{k=1}^{n+1}(x_{k}-x_{1})
            \textrm{det}\Bigg(
                \begin{bmatrix}
                    1&x_{2}&x_{2}^{2}&\dots&x_{2}^{n-1}&x_{2}^{n-1}\\
                    1&x_{3}&x_{3}^{2}&\dots&x_{3}^{n-1}&x_{3}^{n-1}\\
                    \vdots&\vdots&\vdots&\ddots&\vdots&\vdots\\
                    1&x_{n+1}&x_{n+1}^{2}&\dots&x_{n+1}^{n-1}&x_{n}^{n-1}
                \end{bmatrix}
            \Bigg)
        \end{equation}
        But this is the $n\times{n}$ Vandermonde matrix, and by the inductive
        hypothesis we can compute this determinant, it is simply the
        hypothesized product. Combining this with
        $\prod_{k=1}^{n+1}(x_{k}-x_{1})$ completes the proof. If $f$ is a
        polynomial of (finite) degree less than the cardinality of the field $F$
        that evaluates to zero for all $a\in{F}$, then $f$ is the zero
        polynomial. For set up the matrix:
        \begin{equation}
            A=
            \begin{bmatrix}
                a_{0}&a_{1}x_{1}&\dots&a_{n-1}x_{1}^{n-1}\\
                a_{0}&a_{1}x_{2}&\dots&a_{n-1}x_{2}^{n-1}\\
                \vdots&\vdots&\ddots&\vdots\\
                a_{0}&a_{1}x_{n}&\dots&a_{n-1}x_{n}^{n-1}
            \end{bmatrix}
        \end{equation}
        Where $a_{k}$ are the coefficients of the polynomial $f$, and
        $x_{k}$ are $n$ distinct points in $F$. The determinant, by
        multilinearity, is simply the product of all of the $a_{k}$ times the
        determinant of the Vandermonde matrix. But since all of the $x_{k}$
        are distinct, the determinant is non-zero. However, the rows of $A$ all
        sum to zero by hypothesis, and hence $\textrm{det}(A)=0$. But since the
        determinant of the Vandermonde matrix is non-zero, the product of the
        $a_{k}$'s must be zero. Since we are in a field, one of the $a_{k}$'s is
        zero. By induction (create a new polynomial with $n-1$ coefficients),
        all of the $a_{k}$ are zero, and hence $f$ is the zero polynomial.
        Thus, if $f,g\in{F}[x]$, where $F$ is infinite, and if
        $a\mapsto{f}(a)$ and $a\mapsto{g}(a)$ are the same mappings, then
        $f-g$ is the zero mapping. But since $F$ is infinite,
        $\textrm{max}\{\textrm{deg}(f),\textrm{deg}(g)\}<|F|$, and hence
        $f-g$ is a polynomial of degree less than the cardinality of $F$ which
        evaluates to zero for all points $a\in{F}$. Hence, it is the zero
        polynomial, and thus $f=g$. That is, $f\rightarrow(a\mapsto{f}(a)))$ is
        injective.
        \par\hfill\par
        If $p$ is prime, then $x^{p}-x=x(x^{p-1}-1)$ and by Fermat's little
        theorem, for all non-zero $a\in\mathbb{F}_{p}$ (since $p$ does not
        divide $a$) we have that $x^{p-1}-1=0$ in $\mathbb{F}_{p}$. For the case
        for $a=0$, $x^{p}-x=0$ trivially. Hence, $x^{p}-x$ is zero for every
        element of $\mathbb{F}_{p}$. If $f\in\mathbb{F}_{p}[x]$ evaluates to
        zero for all $a\in\mathbb{F}_{p}$, then either $f$ is the zero
        polynomial or the degree is greater than $n$. In the latter case
        $x^{p}-x$ can divide this with remainder, the remainder being a
        polynomial of degree strictly less than $p$. Thus:
        \begin{equation}
            f(x)=(x^{p}-x)q(x)+r(x)
        \end{equation}
        But $f(x)=0$ for all $x$, and $x^{p}-x=0$ for all $x$, and hence
        $r(x)=0$ for all $x$. But $r$ is a polynomial of degree less than $n$,
        and hence is the zero polynomial. That is, $x^{p}-x$ divides $f$, so
        $x^{p}-x$ generates the kernel.
    \end{solution}
    \begin{problem}
        Some stuff about symmetric polynomials.
    \end{problem}
    \begin{solution}
        Since we take upon axiom that $\sigma$ distributes over addition and
        multiplication, given $f,g\in{S}$,
        $\sigma(f+g)=\sigma\cdot{f}+\sigma\cdot{g}=f+g$, and thus $f+g$ is fixed
        as well. Moreover, $\sigma(fg)=(\sigma\cdot{f})(\sigma\cdot{g})=fg$,
        so $fg$ is fixed as well. Lastly, 0 and 1 are elements, and if $f\in{S}$
        and so is $\minus{f}$. Hence, $S$ is a subring.
        \par\hfill\par
        Each $e_{k}$ is a symmetric polynomial. For we have:
        \begin{equation}
            \sigma\cdot\sum\prod{x}_{k}=
            \sum\sigma\cdot\prod{x}_{k}=
            \sum\prod\sigma\cdot{x}_{k}=
            \sum\prod{x}_{j}
        \end{equation}
        Where $j$ is where $\sigma$ permutes $k$ too. Since $R$ is a commutative
        ring, and since every combination will occur, we may rearrange this too
        obtain the original $e_{k}$. Thus, $\sigma\cdot{e}_{k}=e_{k}$.
        \par\hfill\par
        Prove that:
        \begin{equation}
            \prod_{k=1}^{n}(x-x_{k})=\sum_{k=0}^{n}(\minus{1})^{n-k}e_{n-k}x^{k}
        \end{equation}
        By induction. The base case of $n=1$ simply says that
        $\minus(x-x_{1})=x_{1}-x$, which is true. Suppose it is true for $n$.
        Thus:
        \begin{align}
            \prod_{k=1}^{n+1}(x-x_{k})
                &=(x-x_{n+1})\prod_{k=1}^{n}(x-x_{k})\\
                &=(x-x_{n+1})\sum_{k=0}^{n}(\minus{1})^{n-k}e_{n-k}x^{k}\\
                &=\sum_{k=0}^{n}(\minus{1})^{n-k}e_{n-k}x^{k+1}
                    -\sum_{k=0}^{n}(\minus{1})^{n-k}x_{n+1}e_{n-k}x^{k}\\
                &=\sum_{k=0}^{n}(\minus{1})^{n+1-k}x_{n+1}e_{n-k}x^{k}
                    -\sum_{k=0}^{n}(\minus{1})^{n+1-k}e_{n-k}x^{k+1}\\
                &=\sum_{k=0}^{n}(\minus{1})^{n+1-k}e_{n+1-k}x^{k}
                    -\sum_{k=0}^{n}(\minus{1})^{n+1-k}e_{n-k}x^{k+1}\\
                &=\sum_{k=0}^{n}(\minus{1})^{n+1-k}e_{n+1-k}x^{k}
                    +\sum_{k=1}^{n+1}(\minus{1})^{n+1-k}e_{n+1-k}x^{k}\\
                &=\sum_{k=1}^{n+1}(\minus{1})^{n+1-k}e_{n+1-k}x^{k}
        \end{align}
        Lastly, prove the Newton identities for $k\geq{n}$.
    \end{solution}
    \begin{problem}
        More Newton identities!
    \end{problem}
    \begin{solution}
        We can prove by induction. The base case is true by hypothesis.
        For any $n\geq{3}$, we have:
        \begin{equation}
            x^{n}+y^{n}+z^{n}=p_{n}
            =\sum_{k=1}^{n-1}(\minus{1})^{k}e_{k}p_{n-k}
        \end{equation}
        Where $p_{0}=1$. But since each of the $p_{k}$ are rational, and since
        each $e_{k}$ can be written as a polynomial of the $p_{k}$ with
        integer coefficients, it follows that $p_{n}$ is the sum of rational
        numbers, and is hence rational. Computing $p_{4}$ directly, we have:
        \begin{align}
            x^{4}+y^{4}+z^{4}
            &=p_{3}e_{1}-p_{2}e_{2}+p_{1}e_{3}-e_{4}\\
            &=3e_{1}-2e_{2}+e_{3}\\
            &=3(x+y+z)-2(xy+xz+yz)+xyz\\
            &=3-2(xy+xz+yz)+xyz\\
            &=3-(x+y+z)^{2}+x^{2}+y^{2}+z^{2}+xyz\\
            &=3-2+1+xyz\\
            &=2+xyz
        \end{align}
        For suppose all three are rational. Then:
        \begin{equation}
            x^{2}=2-\frac{p^{2}}{q^{2}}-\frac{n^{2}}{m^{2}}
                =\frac{2q^{2}m^{2}-p^{2}m^{2}-n^{2}q^{2}}{q^{2}m^{2}}
                =\Big(1-\frac{p}{q}-\frac{n}{m}\Big)^{2}
        \end{equation}
        Simplifying, we conclude:
        \begin{equation}
            \frac{2q^{2}m^{2}-p^{2}m^{2}-n^{2}q^{2}}{q^{2}m^{2}}
            =\frac{p^{2}m^{2}-2pqm^{2}-2nmq^{2}-p^{2}m^{2}-q^{2}n^{2}}
                  {q^{2}m^{2}}
        \end{equation}
        So:
        \begin{equation}
            2q^{2}m^{2}-p^{2}m^{2}-n^{2}q^{2}
                =p^{2}m^{2}-2pqm^{2}-2nmq^{2}-p^{2}m^{2}-q^{2}n^{2}
        \end{equation}
        Grouping terms we can conclude that $\sqrt{2}$ is irrational, a
        contradiction. Hence one of them must be irrational. Suppose two of them
        are rational. Then $2-x^{2}-y^{2}$ is the difference of rational numbers
        and is hence rational, but if $z$ is irrational and $z^{2}$ is rational,
        then $z=\sqrt{2}r$, where $r$ is rational. But invoking the last
        equation, $z^{3}$ is rational. But $z^{2}=\sqrt{2}(2r^{3})$, which is
        irrational, a contradiction. Hence 2 must be irrational. Suppose one is
        rational. Then $1-x=y+z$, which is rational, and hence $y=z+r$, where
        $r$ is rational. But then:
        \begin{equation}
            x^{2}+y^{2}+z^{2}=x^{2}+y^{2}+(y-r)^{2}=2
        \end{equation}
        Since $r$ and $x$ are rational, $2-x^{2}-r^{2}$ is rational.
        But then $z^{2}+(z-r)^{2}$ is rational.
    \end{solution}
\end{document}