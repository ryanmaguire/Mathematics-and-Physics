%------------------------------------------------------------------------------%
\documentclass{article}                                                        %
%------------------------------Preamble----------------------------------------%
\makeatletter                                                                  %
    \def\input@path{{../../}}                                                  %
\makeatother                                                                   %
%---------------------------Packages----------------------------%
\usepackage{geometry}
\geometry{b5paper, margin=1.0in}
\usepackage[T1]{fontenc}
\usepackage{graphicx, float}            % Graphics/Images.
\usepackage{natbib}                     % For bibliographies.
\bibliographystyle{agsm}                % Bibliography style.
\usepackage[french, english]{babel}     % Language typesetting.
\usepackage[dvipsnames]{xcolor}         % Color names.
\usepackage{listings}                   % Verbatim-Like Tools.
\usepackage{mathtools, esint, mathrsfs} % amsmath and integrals.
\usepackage{amsthm, amsfonts, amssymb}  % Fonts and theorems.
\usepackage{tcolorbox}                  % Frames around theorems.
\usepackage{upgreek}                    % Non-Italic Greek.
\usepackage{fmtcount, etoolbox}         % For the \book{} command.
\usepackage[newparttoc]{titlesec}       % Formatting chapter, etc.
\usepackage{titletoc}                   % Allows \book in toc.
\usepackage[nottoc]{tocbibind}          % Bibliography in toc.
\usepackage[titles]{tocloft}            % ToC formatting.
\usepackage{pgfplots, tikz}             % Drawing/graphing tools.
\usepackage{imakeidx}                   % Used for index.
\usetikzlibrary{
    calc,                   % Calculating right angles and more.
    angles,                 % Drawing angles within triangles.
    arrows.meta,            % Latex and Stealth arrows.
    quotes,                 % Adding labels to angles.
    positioning,            % Relative positioning of nodes.
    decorations.markings,   % Adding arrows in the middle of a line.
    patterns,
    arrows
}                                       % Libraries for tikz.
\pgfplotsset{compat=1.9}                % Version of pgfplots.
\usepackage[font=scriptsize,
            labelformat=simple,
            labelsep=colon]{subcaption} % Subfigure captions.
\usepackage[font={scriptsize},
            hypcap=true,
            labelsep=colon]{caption}    % Figure captions.
\usepackage[pdftex,
            pdfauthor={Ryan Maguire},
            pdftitle={Mathematics and Physics},
            pdfsubject={Mathematics, Physics, Science},
            pdfkeywords={Mathematics, Physics, Computer Science, Biology},
            pdfproducer={LaTeX},
            pdfcreator={pdflatex}]{hyperref}
\hypersetup{
    colorlinks=true,
    linkcolor=blue,
    filecolor=magenta,
    urlcolor=Cerulean,
    citecolor=SkyBlue
}                           % Colors for hyperref.
\usepackage[toc,acronym,nogroupskip,nopostdot]{glossaries}
\usepackage{glossary-mcols}
%------------------------Theorem Styles-------------------------%
\theoremstyle{plain}
\newtheorem{theorem}{Theorem}[section]

% Define theorem style for default spacing and normal font.
\newtheoremstyle{normal}
    {\topsep}               % Amount of space above the theorem.
    {\topsep}               % Amount of space below the theorem.
    {}                      % Font used for body of theorem.
    {}                      % Measure of space to indent.
    {\bfseries}             % Font of the header of the theorem.
    {}                      % Punctuation between head and body.
    {.5em}                  % Space after theorem head.
    {}

% Italic header environment.
\newtheoremstyle{thmit}{\topsep}{\topsep}{}{}{\itshape}{}{0.5em}{}

% Define environments with italic headers.
\theoremstyle{thmit}
\newtheorem*{solution}{Solution}

% Define default environments.
\theoremstyle{normal}
\newtheorem{example}{Example}[section]
\newtheorem{definition}{Definition}[section]
\newtheorem{problem}{Problem}[section]

% Define framed environment.
\tcbuselibrary{most}
\newtcbtheorem[use counter*=theorem]{ftheorem}{Theorem}{%
    before=\par\vspace{2ex},
    boxsep=0.5\topsep,
    after=\par\vspace{2ex},
    colback=green!5,
    colframe=green!35!black,
    fonttitle=\bfseries\upshape%
}{thm}

\newtcbtheorem[auto counter, number within=section]{faxiom}{Axiom}{%
    before=\par\vspace{2ex},
    boxsep=0.5\topsep,
    after=\par\vspace{2ex},
    colback=Apricot!5,
    colframe=Apricot!35!black,
    fonttitle=\bfseries\upshape%
}{ax}

\newtcbtheorem[use counter*=definition]{fdefinition}{Definition}{%
    before=\par\vspace{2ex},
    boxsep=0.5\topsep,
    after=\par\vspace{2ex},
    colback=blue!5!white,
    colframe=blue!75!black,
    fonttitle=\bfseries\upshape%
}{def}

\newtcbtheorem[use counter*=example]{fexample}{Example}{%
    before=\par\vspace{2ex},
    boxsep=0.5\topsep,
    after=\par\vspace{2ex},
    colback=red!5!white,
    colframe=red!75!black,
    fonttitle=\bfseries\upshape%
}{ex}

\newtcbtheorem[auto counter, number within=section]{fnotation}{Notation}{%
    before=\par\vspace{2ex},
    boxsep=0.5\topsep,
    after=\par\vspace{2ex},
    colback=SeaGreen!5!white,
    colframe=SeaGreen!75!black,
    fonttitle=\bfseries\upshape%
}{not}

\newtcbtheorem[use counter*=remark]{fremark}{Remark}{%
    fonttitle=\bfseries\upshape,
    colback=Goldenrod!5!white,
    colframe=Goldenrod!75!black}{ex}

\newenvironment{bproof}{\textit{Proof.}}{\hfill$\square$}
\tcolorboxenvironment{bproof}{%
    blanker,
    breakable,
    left=3mm,
    before skip=5pt,
    after skip=10pt,
    borderline west={0.6mm}{0pt}{green!80!black}
}

\AtEndEnvironment{lexample}{$\hfill\textcolor{red}{\blacksquare}$}
\newtcbtheorem[use counter*=example]{lexample}{Example}{%
    empty,
    title={Example~\theexample},
    boxed title style={%
        empty,
        size=minimal,
        toprule=2pt,
        top=0.5\topsep,
    },
    coltitle=red,
    fonttitle=\bfseries,
    parbox=false,
    boxsep=0pt,
    before=\par\vspace{2ex},
    left=0pt,
    right=0pt,
    top=3ex,
    bottom=1ex,
    before=\par\vspace{2ex},
    after=\par\vspace{2ex},
    breakable,
    pad at break*=0mm,
    vfill before first,
    overlay unbroken={%
        \draw[red, line width=2pt]
            ([yshift=-1.2ex]title.south-|frame.west) to
            ([yshift=-1.2ex]title.south-|frame.east);
        },
    overlay first={%
        \draw[red, line width=2pt]
            ([yshift=-1.2ex]title.south-|frame.west) to
            ([yshift=-1.2ex]title.south-|frame.east);
    },
}{ex}

\AtEndEnvironment{ldefinition}{$\hfill\textcolor{Blue}{\blacksquare}$}
\newtcbtheorem[use counter*=definition]{ldefinition}{Definition}{%
    empty,
    title={Definition~\thedefinition:~{#1}},
    boxed title style={%
        empty,
        size=minimal,
        toprule=2pt,
        top=0.5\topsep,
    },
    coltitle=Blue,
    fonttitle=\bfseries,
    parbox=false,
    boxsep=0pt,
    before=\par\vspace{2ex},
    left=0pt,
    right=0pt,
    top=3ex,
    bottom=0pt,
    before=\par\vspace{2ex},
    after=\par\vspace{1ex},
    breakable,
    pad at break*=0mm,
    vfill before first,
    overlay unbroken={%
        \draw[Blue, line width=2pt]
            ([yshift=-1.2ex]title.south-|frame.west) to
            ([yshift=-1.2ex]title.south-|frame.east);
        },
    overlay first={%
        \draw[Blue, line width=2pt]
            ([yshift=-1.2ex]title.south-|frame.west) to
            ([yshift=-1.2ex]title.south-|frame.east);
    },
}{def}

\AtEndEnvironment{ltheorem}{$\hfill\textcolor{Green}{\blacksquare}$}
\newtcbtheorem[use counter*=theorem]{ltheorem}{Theorem}{%
    empty,
    title={Theorem~\thetheorem:~{#1}},
    boxed title style={%
        empty,
        size=minimal,
        toprule=2pt,
        top=0.5\topsep,
    },
    coltitle=Green,
    fonttitle=\bfseries,
    parbox=false,
    boxsep=0pt,
    before=\par\vspace{2ex},
    left=0pt,
    right=0pt,
    top=3ex,
    bottom=-1.5ex,
    breakable,
    pad at break*=0mm,
    vfill before first,
    overlay unbroken={%
        \draw[Green, line width=2pt]
            ([yshift=-1.2ex]title.south-|frame.west) to
            ([yshift=-1.2ex]title.south-|frame.east);},
    overlay first={%
        \draw[Green, line width=2pt]
            ([yshift=-1.2ex]title.south-|frame.west) to
            ([yshift=-1.2ex]title.south-|frame.east);
    }
}{thm}

%--------------------Declared Math Operators--------------------%
\DeclareMathOperator{\adjoint}{adj}         % Adjoint.
\DeclareMathOperator{\Card}{Card}           % Cardinality.
\DeclareMathOperator{\curl}{curl}           % Curl.
\DeclareMathOperator{\diam}{diam}           % Diameter.
\DeclareMathOperator{\dist}{dist}           % Distance.
\DeclareMathOperator{\Div}{div}             % Divergence.
\DeclareMathOperator{\Erf}{Erf}             % Error Function.
\DeclareMathOperator{\Erfc}{Erfc}           % Complementary Error Function.
\DeclareMathOperator{\Ext}{Ext}             % Exterior.
\DeclareMathOperator{\GCD}{GCD}             % Greatest common denominator.
\DeclareMathOperator{\grad}{grad}           % Gradient
\DeclareMathOperator{\Ima}{Im}              % Image.
\DeclareMathOperator{\Int}{Int}             % Interior.
\DeclareMathOperator{\LC}{LC}               % Leading coefficient.
\DeclareMathOperator{\LCM}{LCM}             % Least common multiple.
\DeclareMathOperator{\LM}{LM}               % Leading monomial.
\DeclareMathOperator{\LT}{LT}               % Leading term.
\DeclareMathOperator{\Mod}{mod}             % Modulus.
\DeclareMathOperator{\Mon}{Mon}             % Monomial.
\DeclareMathOperator{\multideg}{mutlideg}   % Multi-Degree (Graphs).
\DeclareMathOperator{\nul}{nul}             % Null space of operator.
\DeclareMathOperator{\Ord}{Ord}             % Ordinal of ordered set.
\DeclareMathOperator{\Prin}{Prin}           % Principal value.
\DeclareMathOperator{\proj}{proj}           % Projection.
\DeclareMathOperator{\Refl}{Refl}           % Reflection operator.
\DeclareMathOperator{\rk}{rk}               % Rank of operator.
\DeclareMathOperator{\sgn}{sgn}             % Sign of a number.
\DeclareMathOperator{\sinc}{sinc}           % Sinc function.
\DeclareMathOperator{\Span}{Span}           % Span of a set.
\DeclareMathOperator{\Spec}{Spec}           % Spectrum.
\DeclareMathOperator{\supp}{supp}           % Support
\DeclareMathOperator{\Tr}{Tr}               % Trace of matrix.
%--------------------Declared Math Symbols--------------------%
\DeclareMathSymbol{\minus}{\mathbin}{AMSa}{"39} % Unary minus sign.
%------------------------New Commands---------------------------%
\DeclarePairedDelimiter\norm{\lVert}{\rVert}
\DeclarePairedDelimiter\ceil{\lceil}{\rceil}
\DeclarePairedDelimiter\floor{\lfloor}{\rfloor}
\newcommand*\diff{\mathop{}\!\mathrm{d}}
\newcommand*\Diff[1]{\mathop{}\!\mathrm{d^#1}}
\renewcommand*{\glstextformat}[1]{\textcolor{RoyalBlue}{#1}}
\renewcommand{\glsnamefont}[1]{\textbf{#1}}
\renewcommand\labelitemii{$\circ$}
\renewcommand\thesubfigure{%
    \arabic{chapter}.\arabic{figure}.\arabic{subfigure}}
\addto\captionsenglish{\renewcommand{\figurename}{Fig.}}
\numberwithin{equation}{section}

\renewcommand{\vector}[1]{\boldsymbol{\mathrm{#1}}}

\newcommand{\uvector}[1]{\boldsymbol{\hat{\mathrm{#1}}}}
\newcommand{\topspace}[2][]{(#2,\tau_{#1})}
\newcommand{\measurespace}[2][]{(#2,\varSigma_{#1},\mu_{#1})}
\newcommand{\measurablespace}[2][]{(#2,\varSigma_{#1})}
\newcommand{\manifold}[2][]{(#2,\tau_{#1},\mathcal{A}_{#1})}
\newcommand{\tanspace}[2]{T_{#1}{#2}}
\newcommand{\cotanspace}[2]{T_{#1}^{*}{#2}}
\newcommand{\Ckspace}[3][\mathbb{R}]{C^{#2}(#3,#1)}
\newcommand{\funcspace}[2][\mathbb{R}]{\mathcal{F}(#2,#1)}
\newcommand{\smoothvecf}[1]{\mathfrak{X}(#1)}
\newcommand{\smoothonef}[1]{\mathfrak{X}^{*}(#1)}
\newcommand{\bracket}[2]{[#1,#2]}

%------------------------Book Command---------------------------%
\makeatletter
\renewcommand\@pnumwidth{1cm}
\newcounter{book}
\renewcommand\thebook{\@Roman\c@book}
\newcommand\book{%
    \if@openright
        \cleardoublepage
    \else
        \clearpage
    \fi
    \thispagestyle{plain}%
    \if@twocolumn
        \onecolumn
        \@tempswatrue
    \else
        \@tempswafalse
    \fi
    \null\vfil
    \secdef\@book\@sbook
}
\def\@book[#1]#2{%
    \refstepcounter{book}
    \addcontentsline{toc}{book}{\bookname\ \thebook:\hspace{1em}#1}
    \markboth{}{}
    {\centering
     \interlinepenalty\@M
     \normalfont
     \huge\bfseries\bookname\nobreakspace\thebook
     \par
     \vskip 20\p@
     \Huge\bfseries#2\par}%
    \@endbook}
\def\@sbook#1{%
    {\centering
     \interlinepenalty \@M
     \normalfont
     \Huge\bfseries#1\par}%
    \@endbook}
\def\@endbook{
    \vfil\newpage
        \if@twoside
            \if@openright
                \null
                \thispagestyle{empty}%
                \newpage
            \fi
        \fi
        \if@tempswa
            \twocolumn
        \fi
}
\newcommand*\l@book[2]{%
    \ifnum\c@tocdepth >-3\relax
        \addpenalty{-\@highpenalty}%
        \addvspace{2.25em\@plus\p@}%
        \setlength\@tempdima{3em}%
        \begingroup
            \parindent\z@\rightskip\@pnumwidth
            \parfillskip -\@pnumwidth
            {
                \leavevmode
                \Large\bfseries#1\hfill\hb@xt@\@pnumwidth{\hss#2}
            }
            \par
            \nobreak
            \global\@nobreaktrue
            \everypar{\global\@nobreakfalse\everypar{}}%
        \endgroup
    \fi}
\newcommand\bookname{Book}
\renewcommand{\thebook}{\texorpdfstring{\Numberstring{book}}{book}}
\providecommand*{\toclevel@book}{-2}
\makeatother
\titleformat{\part}[display]
    {\Large\bfseries}
    {\partname\nobreakspace\thepart}
    {0mm}
    {\Huge\bfseries}
\titlecontents{part}[0pt]
    {\large\bfseries}
    {\partname\ \thecontentslabel: \quad}
    {}
    {\hfill\contentspage}
\titlecontents{chapter}[0pt]
    {\bfseries}
    {\chaptername\ \thecontentslabel:\quad}
    {}
    {\hfill\contentspage}
\newglossarystyle{longpara}{%
    \setglossarystyle{long}%
    \renewenvironment{theglossary}{%
        \begin{longtable}[l]{{p{0.25\hsize}p{0.65\hsize}}}
    }{\end{longtable}}%
    \renewcommand{\glossentry}[2]{%
        \glstarget{##1}{\glossentryname{##1}}%
        &\glossentrydesc{##1}{~##2.}
        \tabularnewline%
        \tabularnewline
    }%
}
\newglossary[not-glg]{notation}{not-gls}{not-glo}{Notation}
\newcommand*{\newnotation}[4][]{%
    \newglossaryentry{#2}{type=notation, name={\textbf{#3}, },
                          text={#4}, description={#4},#1}%
}
%--------------------------LENGTHS------------------------------%
% Spacings for the Table of Contents.
\addtolength{\cftsecnumwidth}{1ex}
\addtolength{\cftsubsecindent}{1ex}
\addtolength{\cftsubsecnumwidth}{1ex}
\addtolength{\cftfignumwidth}{1ex}
\addtolength{\cfttabnumwidth}{1ex}

% Indent and paragraph spacing.
\setlength{\parindent}{0em}
\setlength{\parskip}{0em}                                                           %
%----------------------------Main Document-------------------------------------%
\begin{document}
    \title{Differential Topology}
    \author{Ryan Maguire}
    \date{\vspace{-5ex}}
    \maketitle
    \section{Preliminary Stuff}
        The following were presented as exercises in the appendix of Lee's
        textbook and can justify the use of sequences without invoking more
        complicated results pertaining to metrizability.
        \begin{definition}
            A sequentially continuous function from a topological space
            $(X,\tau_{X})$ to a topological space $(Y,\tau_{Y})$ is a
            function $f:X\rightarrow{Y}$ such that for every convergent sequence
            $a:\mathbb{N}\rightarrow{X}$, it is true that:
            \begin{equation*}
                \lim_{n\rightarrow\infty}f(a_{n})
                =f(\lim_{n\rightarrow\infty}a_{n})
            \end{equation*}
        \end{definition}
        \begin{definition}
            A sequential topological space is a topological space $(X,\tau)$
            such that for every topological space $(Y,\tau_{Y})$ and for every
            sequentially continuous function $f:X\rightarrow{Y}$, it is true
            that $f$ is continuous.
        \end{definition}
        Continuity implies sequential continuity, the converse need not always
        hold. If $(X,\tau)$ is first countable, the result is true.
        \begin{theorem}
            If $(X,\tau)$ is a first countable topological space, if
            $\mathcal{U}\subseteq{X}$, then $\mathcal{U}$ is open if and only if
            for every sequence $a:\mathbb{N}\rightarrow{X}$ such that there is
            an $x\in\mathcal{U}$ such that $a_{n}\rightarrow{x}$, then there
            exists an $N\in\mathbb{N}$ such that for all $n>N$, it is true that
            $a_{n}\in\mathcal{U}$.
        \end{theorem}
        \begin{proof}
            One direction is the definition of convergence. In the other, since
            $(X,\tau)$ is first countable, for all $x\in\mathcal{U}$ there is a
            countable neighborhood basis $\mathcal{B}_{x}$. Let
            $\mathcal{V}_{x}:\mathbb{N}\rightarrow\mathcal{B}_{x}$ be a
            bijection. Then there exists an $N\in\mathbb{N}$ such that
            $\mathcal{V}_{x,N}\subseteq\mathcal{U}$. For suppose not. Let
            $B_{n}$ be defined by:
            \begin{equation}
                B_{n}=\bigcap_{k\in\mathbb{Z}_{n}}\mathcal{V}_{x,k}
            \end{equation}
            Since $B_{n}$ is the intersection of finitely many open sets, it is
            open. Moreover it is non-empty since $x\in{B}_{n}$ for all $n$.
            But then $B_{n}$ is an open neighborhood about $x$, and since
            $\mathcal{B}_{x}$ is a neighborhood basis there is an element
            $\mathcal{V}_{x,N}$ such that $\mathcal{V}_{x,N}\subseteq{B}_{n}$.
            But by hypothesis, for any such set there is a
            $y_{n}\in\mathcal{V}_{x,N}$ such that $y_{n}\notin\mathcal{U}$.
            Thus, for all $n$ there is a $y_{n}\in{B}_{n}$ such that
            $y_{n}\notin\mathcal{U}$. Then $y_{n}\rightarrow{x}$ since for any
            open subset about $x$ there is an $N\in\mathbb{N}$ such that
            $\mathcal{V}_{x,N}$ sits inside this open set. But for all $n>N$,
            $y_{n}\notin\mathcal{U}$. Thus $y_{n}$ is a sequence that
            converges to $x$, but is never contained inside $\mathcal{U}$,
            a contradiction. Thus, for all $x\in\mathcal{U}$ there is an open
            subset $\mathcal{V}_{x,N}$ such that $x\in\mathcal{V}_{x,N}$ and
            $\mathcal{V}_{x,N}\subseteq\mathcal{U}$. But then $\mathcal{U}$ is
            simply the union over all of these open sets, and is thus open.
        \end{proof}
        \begin{theorem}
            If $(X,\tau)$ is a first countable topological space, then it is
            a sequential space.
        \end{theorem}
        \begin{proof}
            For let $(Y,\tau_{Y})$ be a topological space, and let
            $f:X\rightarrow{Y}$ be a sequentially continuous function. Let
            $\mathcal{V}\in\tau_{Y}$ be an open subset of $Y$. If
            $f^{\minus{1}}[\mathcal{V}]=\emptyset$, we are done. If not, let
            $x\in{f}^{\minus{1}}[\mathcal{V}]$ and let
            $a:\mathbb{N}\rightarrow{X}$ be a sequence such that
            $a_{n}\rightarrow{x}$. But $f$ is sequentially continuous, and thus
            $f(a_{n})\rightarrow{f}(x)$. But since $\mathcal{V}$ is open, there
            is an $N\in\mathbb{N}$ such that for all $n>N$ it is true that
            $f(a_{n})\in\mathcal{V}$. But then for all $n>N$ it is true that
            $a_{n}\in{f}^{\minus{1}}[\mathcal{V}]$. Thus every sequence that
            converges to a point in $f^{\minus{1}}[\mathcal{V}]$ is eventually
            contained in $f^{\minus{1}}[\mathcal{V}]$, and thus by the
            previous theorem this set is open. Thus the pre-image of open is
            open, and hence $f$ is continuous.
        \end{proof}
    \section{Homework I: Part A}
        \begin{problem}
            Show that $\mathbb{RP}^{n}$ is Hausdorff and second countable.
        \end{problem}
        \begin{solution}
            $\mathbb{RP}^{n}$ can be seen as the quotient space of $S^{n}$ under
            the continuous action of $\mathbb{Z}_{2}^{\times}$ on $S^{n}$
            defined by $1\cdot\vector{x}=\vector{x}$ and
            $\minus{1}\cdot\vector{x}=\minus\vector{x}$. Since
            $\mathbb{Z}_{2}^{\times}$ is a compact topological group (it's
            finite), and since $S^{n}$ is Hausdorff and second countable (since
            it's a subspace of $\mathbb{R}^{n}$), then
            $S^{n}/\mathbb{Z}_{2}^{\times}$ is Hausdorff and second countable.
            \par\hfill\par
            Alternatively, Let $[\mathbf{x}]$ and $[\mathbf{y}]$ be distinct
            elements in the quotient space $\mathbb{RP}^{n}$. Let $\varepsilon$
            be defined by:
            \begin{equation}
                \varepsilon=\frac{1}{4}\cos^{\minus{1}}\Big(
                    \frac{\langle\vector{x}|\vector{y}\rangle}
                        {\norm{\vector{x}}\norm{\vector{y}}}
                \Big)
            \end{equation}
            Consider the open cones $\Lambda_{1}$ and $\Lambda_{2}$ about
            the lines passing thought $\vector{x}$ and the origin, and
            $\vector{y}$ and the origin, respectively, such that each line in
            $\Lambda_{1}$ has an angle of less than $\varepsilon$ with the line
            containing the origin and $\vector{x}$, and similarly for
            $\vector{y}$ and $\Lambda_{2}$. From the definition of
            $\varepsilon$, these cones intersect only at the origin. Hence, in
            $\mathbb{R}^{n+1}\setminus\{0\}$ they are disjoint. Moreover, they
            are saturated, i.e.
            $\pi^{\minus{1}}\big(\pi(\Lambda_{i})\big)=\Lambda_{i}$. Thus the
            forward image is an open subset in the quotient space. But
            $\pi(\Lambda_{1})$ contains $[\vector{x}]$ and $\pi(\Lambda_{2})$
            contains $[\vector{y}]$, and moreover these projections are
            disjoint. Thus $[\vector{x}]$ and $[\vector{y}]$ can be separeted
            by open disjoint sets, and hence $\mathbb{RP}^{n}$ is Hausdorff.
            \par\hfill\par
            One final way, using the open sets $\mathcal{U}_{i}$ constructed in
            Lee's text, if $n>2$, then for any $[\vector{x}]$ and $[\vector{y}]$
            there is a $\mathcal{U}_{i}$ containing both of these points. Since the
            $\mathcal{U}_{i}$ are homeomorphic to an open subset of Euclidean space,
            they are Hausdorff, and hence $[\vector{x}]$ and $[\vector{y}]$ can be
            separeted by disjoint open sets. For the case of $n=2$ the only problem
            we run in to is when $[\vector{x}]$ represents the $x$ axis and
            $[\vector{y}]$ represents the $y$ axis since $[\vector{x}]$ does not lie
            in $\mathcal{U}_{x}$ and $[\vector{y}]$ is not contained in
            $\mathcal{U}_{y}$. Thus there is no $\mathcal{U}_{i}$ in this
            construction that contains both $[\vector{x}]$ and $[\vector{y}]$. Form
            a new set $\tilde{\mathcal{U}}$ defined by the complement of the line
            $y=x$. By identical arguments the projection $\mathcal{U}$ will be
            homeomorphic to an open subset of $\mathbb{R}^{2}$, however this set
            will contain $[\vector{x}]$ and $[\vector{y}]$, and hence all distinct
            points can be separeted. Thus, $\mathbb{RP}^{2}$ is Hausdorff.
            \par\hfill\par
            For second countability, we've shown that $\mathbb{RP}^{n}$ can be
            covered by $n+1$ open subset $\mathcal{U}_{i}$, each of which is
            homeomorphic to an open subset of $\mathbb{R}^{n}$. Thus each
            $\mathcal{U}_{i}$ is second countable, since second countability is
            preserved by homeomorphisms. Let $\mathcal{B}$ be the union of these
            $n+1$ bases. Let $\mathcal{V}$ be an open subset of $\mathbb{RP}^{n}$.
            Since $\mathcal{V}$ and each $\mathcal{U}_{i}$ is open,
            $\mathcal{V}\cap\mathcal{U}_{i}$ is open. But this intersection is a
            subset of $\mathcal{U}_{i}$, and hence there is a subset
            $\Delta_{i}\subseteq\mathcal{B}$ such that:
            \begin{equation}
                \mathcal{V}\cap\mathcal{U}_{i}=
                    \bigcup_{\mathcal{O}\in\Delta}\mathcal{O}
            \end{equation}
            But then:
            \begin{equation}
                \mathcal{V}=
                \bigcup_{i\in\mathbb{Z}_{n+1}}\big(
                    \mathcal{V}\cap\mathcal{U}_{i}
                \big)
                =\bigcup_{i\in\mathbb{Z}_{n+1}}\Big(
                    \bigcup_{\mathcal{O}\in\Delta_{i}}\mathcal{O}\Big)
            \end{equation}
            Thus every open subset can be written as the union of elements of
            $\mathcal{B}$. Since $\mathcal{B}$ is the finite union of countable
            sets, it is therefore countable. Hence, $\mathcal{B}$ is a countable
            basis.
        \end{solution}
        \begin{problem}
            Show that a locally Euclidean Hausdorff topological space that is
            paracompact and contains countable many connected components is
            second countable. Show the converse as well.
        \end{problem}
        \begin{solution}
            That a topological manifold is paracompact is proved in the text. Let
            $\mathcal{B}$ be a countable basis. Then since a locally Euclidean space
            is locally path connected, then connected components of the space are
            open. But then each connected component must contain an element of
            $\mathcal{B}$ inside it, and thus the number of connected components
            must be countable. Going the other way, suppose $(X,\tau)$ is locally
            Euclidean, Hausdorff, paracompact, and contains countably many
            connected components. It suffices to show that any single connected
            component has a countable basis, since then we will have a basis for the
            whole space which is the countable union of countable sets, and will
            hence be countable. Since $(X,\tau)$ is locally Euclidean, there is a
            cover of precompact coordindate balls $\{\mathcal{U}_{\alpha}\}$. But
            since $(X,\tau)$ is paracompact, there is a locally finite refinement
            $\mathcal{O}$. But then the closure of each element of this refinement
            is a subset of the closure of one of the original elements, each of
            which is compact, and since closued subsets of compact sets are compact,
            the closure of every element of $\mathcal{O}$ is compact. Now we need
            connectednes. Since locally path connected spaces that are connected are
            necessarily path connected, we have that $(X,\tau)$ is a path connected
            topological space. Let $\mathcal{V}_{1},\mathcal{V}_{2}\in\mathcal{O}$.
            Then there is an integer $n\in\mathbb{N}$ and a sequence
            $\mathcal{D}:\mathbb{Z}_{n}\rightarrow\mathcal{O}$ such that
            $\mathcal{D}_{0}=\mathcal{V}_{1}$, $\mathcal{D}_{n-1}=\mathcal{V}_{2}$,
            and for all $k\in\mathbb{Z}_{n}$, the intersection of consecutive
            elements is non-empty. That is,
            $\mathcal{D}_{k}\cap\mathcal{D}_{k+1}\ne\emptyset$. For let
            $x_{1}\in\mathcal{V}_{1}$ and $x_{2}\in\mathcal{V}_{2}$. But since the
            space is path connected, there is a path $\gamma:[0,1]\rightarrow{X}$
            connecting $x_{1}$ and $x_{2}$. But a path is a continuous function, and
            the continuous image of a compact set is compact. But $\mathcal{O}$ is a
            cover of all of $X$, and hence is a cover of the image of $\gamma$. But
            then there is a finite subcover. Order this finite subcover so that
            $x_{1}$ lies in the first element and $x_{2}$ lies in the second
            element. But by the well ordering principal, there is necessarily a
            least integer $n$ such that there exists a sequence
            $\mathcal{D}:\mathbb{Z}_{n}\rightarrow{X}$ with the desired property.
            Thus (subtly invoking choice) there is a function
            $f:\mathcal{O}\rightarrow\mathbb{N}$ such that for all
            $\mathcal{V}\in\mathcal{O}$, $f(\mathcal{V})$ is the least integer
            number of sets in $\mathcal{O}$ that connect $\mathcal{V}_{1}$ to
            $\mathcal{V}$. By paracompactness, for every finite subset of
            $\mathcal{N}\subseteq\mathbb{N}$, the pre-image
            $f^{\minus{1}}[\mathcal{N}]$ is necessarily finite. For, by a theorem in
            the text, if $\mathcal{O}$ is a locally finite open cover, then the
            closure of all of the elements is again a locally finite cover. But the
            closure of all such elements is compact. From this,
            $f^{\minus{1}}[\mathcal{N}]$ is finite. But the union over
            $f^{\minus{1}}[\mathbb{Z}_{n}]$ is a cover of $X$, since every element
            of $\mathcal{O}$ is contained in one of these pre-images, and hence
            the refinement $\mathcal{O}$ is countable. But each element of
            $\mathcal{O}$ is homeomorphic to an open subset of $\mathbb{R}^{n}$, and
            is hence second countable. But the countable union of open second
            countable subsets is again second countable, and hence $(X,\tau)$ is
            second countable.
        \end{solution}
        \begin{problem}
            Stereographic projection.
        \end{problem}
        \begin{solution}
            For let $\vector{X}$ be a point in $\mathbb{R}^{n}$ and draw the line
            from the north pole to $\vector{X}$:
            \begin{equation}
                \Gamma(t)=(0,\dots,0,1)(1-t)+\vector{X}t
            \end{equation}
            We need to solve the value when $\norm{\Gamma(t)}_{2}=1$. This give:
            \begin{equation}
                t^{2}+(1-t)^{2}\norm{\vector{X}}_{2}=1
            \end{equation}
            Solving, we have:
            \begin{equation}
                t=\frac{\norm{\vector{X}}_{2}^{2}-1}{\norm{\vector{X}}_{2}^{2}+1}
            \end{equation}
            Substituting this value of $t$ into $\Gamma(t)$ gives us when
            $\Gamma$ crosses the sphere $S^{n}$. This point is then:
            \begin{equation}
                \frac{(2X_{1},\dots,2X_{n},\norm{\mathbf{X}}_{2}^{2})}
                    {\norm{\vector{X}}_{2}^{2}+1}
            \end{equation}
            This gives us the second equation in the problem. In the reverse
            direction, if $\vector{s}\in\mathbb{S}^{n}$, let $\gamma$ be the line
            through the north pole and $\vector{s}$:
            \begin{equation}
                \gamma(t)=(0,\dots,0,1)(1-t)+\vector{s}t
            \end{equation}
            We now need to solve for $t$ when $x_{n+1}=0$. We have:
            \begin{equation}
                (1-t)+s_{n+1}t=0
            \end{equation}
            And thus $t=1/(1=s_{n+1})$. Using this, we obtain:
            \begin{equation}
                \frac{(s_{1},\dots,s_{n},0)}{1-s_{n+1}}
            \end{equation}
            This shows that the $\sigma$ map is bijective, and that these formulas
            give the inverse functions. The transition function between the two is:
            \begin{equation}
                \tilde{\sigma}\circ\sigma^{\minus{1}}(x_{1},\dots,x_{n})=
                \tilde{\sigma}\Big(
                    \frac{(2x_{1},\dots,2x_{n},\norm{\vector{x}}_{2}^{2}-1)}
                        {\norm{\vector{x}}_{2}^{2}+1}
                \Big)
                =\frac{(2x_{1},\dots,2x_{n})}
                    {(1-x_{n})(1+\norm{\vector{x}}_{2}^{2})}
            \end{equation}
            Which is smooth. To show that this is compatible with the orthographic
            projection, we have:
            \begin{equation}
                \pi^{k}\circ\sigma^{\minus{1}}(x_{1},\dots,x_{n})
                =\frac{(2x_{1},\dots,2x_{k-1},2x_{k+1},\dots,2x_{n})}
                    {\norm{\vector{x}}_{2}^{2}+1}
            \end{equation}
            Which is smooth, and similarly in the other direction. Hence, both
            atlases produce the same smooth structure.
        \end{solution}
        \begin{problem}
            Show that $\mathbb{CP}^{n}$ is a $2n$ dimensional manifold.
        \end{problem}
        \begin{solution}
            This is similar to the problem before with the real projective plane,
            but now linear subspaces are essentially copies of the complex plane
            through the origin of $\mathbb{C}^{n+1}$. Similar to how
            $\mathbb{RP}^{n}$ could be described by identifying antipodal points,
            here we identify points by circular cuts through the sphere in
            $\mathbb{C}^{n+1}$. This is simply the action by the group of rotations
            on the circle, which is also compact, and hence the quotient will
            preserve Hausdorffness, second countability, and the locally Euclidean
            property. Thus, $\mathbb{CP}^{n}$ is a manifold.
        \end{solution}
        \begin{problem}
            Show that the product of smooth manifolds, together with a smooth
            manifold with boundary, is again a smooth manifold with boundary.
        \end{problem}
        \begin{solution}
            For since the product of smooth manifolds is again a smooth manifold,
            it suffices to show that $M\times{B}$ is smooth with boundary, where
            $M$ is a smooth manifold and $B$ is a smooth manifold with boundary.
        \end{solution}
    \section{Homework I: Part B}
        \begin{problem}
            Show that the action of $\textrm{Aut}(V)$ on $G$ by
            $(T,v)\mapsto{T}(v)$ is a continuous group action.
        \end{problem}
        \begin{solution}
            It is a group action since $(\textrm{Id}_{V},v)=\textrm{Id}_{V}v=v$
            and from the associativity of matrix multiplication, $(\cdot,\cdot)$
            preserves associativity. It is also continuous. For since
            both $V$ and $\textrm{Aut}(V)$ are homeomorphic to some
            $\mathbb{R}^{n}$ and an open subset of $\mathbb{R}^{n^{2}}$, they
            are second countable and hence first countable. Moreover, the
            product space is. As proved before, such spaces are sequential
            spaces. Let ordered pair $(T_{n},v_{n})$ be such that
            $(T_{n},v_{n})\rightarrow(T,v)$. Then:
            \begin{equation}
                \norm{T_{n}v_{n}-Tv}=
                \norm{T_{n}v_{n}-Tv_{n}+Tv_{n}-T_{n}v_{n}}
                \leq\norm{T_{n}v_{n}-Tv_{n}}+\norm{Tv_{n}-T_{nv_{n}}}
            \end{equation}
            And both of the converge to zero, hence $T_{n}v_{n}\rightarrow{T}v$.
            Since sequetially continuous functions are continuous in sequential
            spaces, the function $(T,v)\mapsto{T}v$ is continuous.
        \end{solution}
        \begin{problem}
            Let $G$ be a topological Lie group, $V$ a finite dimensional real
            vector space, and $\rho$ a real representation of $G$ on $V$. Show
            that $\Theta:G\times{V}\rightarrow{G}$ by
            $\Theta(g,v)=\rho(g)v$ is a continuous group action such that
            $\Theta(g,\cdot)\in\textrm{Aut}(V)$ if and only if $\rho$ is a
            real representation.
        \end{problem}
        \begin{solution}
            Again, all spaces here are first countable, so we may use sequences.
            Let $g_{n}\rightarrow{g}$, and let $\rho$ be a real representation.
            Then $\Theta(g_{n},v)=\rho(g_{n})v$. But $\rho$ is continuous, so
            $\rho(g_{n})v\rightarrow\rho(g)v$, and hence $\Theta$ is continuous.
            Moreover, fiving $g$, $\Theta(g,\cdot)=\rho(g)\in\textrm{Aut}(V)$.
            Going the other way, if $\Theta$ is a continuous group action and
            $\Theta(g,\cdot)\in\textrm{Aut}(V)$, then $\rho$ is a continuous
            group homomorphism. It is continuous, for if $g_{n}\rightarrow{g}$,
            then for all $v\in{V}$, $\Theta(g_{n},v)\rightarrow\Theta(g,v)$
            since $\Theta$ is continuous. But this implies
            $\rho(g_{n})v\rightarrow\rho(g)v$, and so
            $(\rho(g_{n})-\rho(g))v\rightarrow{0}$ for all $v\in{V}$, and hence
            $\rho(g_{n})-\rho(g)$ tends to the zero operator. Thus,
            $\rho(g_{n})\rightarrow\rho(g)$ and $\rho$ is continuous. It is a
            group homomorphism since $\Theta$ is a group action, and thus:
            \begin{equation}
                \rho(g*h)=\Theta(g*h,\cdot)
                =\Theta(g,\Theta(h,\cdot))=\rho(g)\rho(h)
            \end{equation}
        \end{solution}
\end{document}