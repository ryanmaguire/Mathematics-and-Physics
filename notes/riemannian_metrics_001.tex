%-----------------------------------LICENSE------------------------------------%
%   This file is part of Mathematics-and-Physics.                              %
%                                                                              %
%   Mathematics-and-Physics is free software: you can redistribute it and/or   %
%   modify it under the terms of the GNU General Public License as             %
%   published by the Free Software Foundation, either version 3 of the         %
%   License, or (at your option) any later version.                            %
%                                                                              %
%   Mathematics-and-Physics is distributed in the hope that it will be useful, %
%   but WITHOUT ANY WARRANTY; without even the implied warranty of             %
%   MERCHANTABILITY or FITNESS FOR A PARTICULAR PURPOSE.  See the              %
%   GNU General Public License for more details.                               %
%                                                                              %
%   You should have received a copy of the GNU General Public License along    %
%   with Mathematics-and-Physics.  If not, see <https://www.gnu.org/licenses/>.%
%------------------------------------------------------------------------------%
\documentclass{article}
\usepackage{amsmath}                            % Needed for align.
\usepackage{amssymb}                            % Needed for mathbb.
\usepackage{amsthm}                             % For the theorem environment.
\usepackage{mathtools}

\newtheoremstyle{normal}
    {\topsep}               % Amount of space above the theorem.
    {\topsep}               % Amount of space below the theorem.
    {}                      % Font used for body of theorem.
    {}                      % Measure of space to indent.
    {\bfseries}             % Font of the header of the theorem.
    {}                      % Punctuation between head and body.
    {.5em}                  % Space after theorem head.
    {}

\theoremstyle{normal}
\newtheorem{definition}{Definition}
\newtheorem{problem}{Problem}

\title{Riemannian Geometry HW II}
\author{Ryan Maguire}
\date{Spring 2022}

% No indent and no paragraph skip.
\setlength{\parindent}{0em}
\setlength{\parskip}{0em}

\begin{document}
    \maketitle
    \begin{problem}
        Show that the antipodal map on $\mathbb{S}^{2}$ is an isometry.
        Use this to induce a metric on $\mathbb{RP}^{2}$ that makes the
        quotient map a local isometry.
    \end{problem}
    \begin{proof}[Solution]
        There is a natural correspondence with tangent vectors
        $u,v\in{T}_{\mathbf{x}}\mathbb{S}^{2}$ for some
        $\mathbf{x}\in\mathbb{S}^{2}$ with vectors in
        $\mathbf{v}\in\mathbb{R}^{3}$ such that
        $\mathbf{x}\cdot\mathbf{v}=0$. Given two tangent vectors
        $u,v\in{T}_{\mathbf{x}}\mathbb{S}^{2}$ and the corresponding
        tangent vectors $\mathbf{v},\mathbf{u}\in\mathbb{R}^{3}$ we have:
        \begin{equation}
            \mathbf{x}\cdot\mathbf{y}
            =(-\mathbf{x})\cdot(-\mathbf{y})
            =\textrm{d}f_{\mathbf{x}}(v)\cdot\textrm{d}f_{\mathbf{x}}(u)
        \end{equation}
        where $f:\mathbb{S}^{2}\rightarrow\mathbb{S}^{2}$ is the antipodal
        mapping. Thus, $f$ is an isometry. The quotient map
        $q:\mathbb{S}^{2}\rightarrow\mathbb{RP}^{2}$ is a local embedding since
        the group action of $\{-1,1\}$ on $\mathbb{S}^{2}$ is properly
        discontinuous, meaning each $[\mathbf{x}]\in\mathbb{RP}^{2}$ has an
        open neighborhood $\tilde{\mathcal{U}}$ such that
        $q^{-1}[\tilde{\mathcal{U}}]$ consists of an even cover of open sets
        $\mathcal{U}_{\alpha}$, each of which being the domain of a chart.
        Hence the quotient map induces a Riemannian metric on $\mathbb{RP}^{2}$
        by pull-back, and the induced metric is a local isometry by definition.
    \end{proof}
    \begin{problem}
        Induce a metric on $\mathbb{T}^{n}$ via the natural projection of
        $\mathbb{R}^{n}$ by the group action of integer translations of
        $\mathbb{Z}^{n}$. Show that this induced metric is isometric to the
        flat torus.
    \end{problem}
    \begin{proof}[Solution]
        Similar to the previous problem, the group action of
        $\mathbb{Z}^{n}$ on $\mathbb{R}^{n}$ is properly discontinuous
        (take the ball of radius $1/4$ centered about a given point). This
        induces a local embedding of $\mathbb{R}^{n}$ into $\mathbb{T}^{n}$.
        The metric induced by this local embedding is thus a local isometry
        by definition. It is isomorphic to the product metric on
        $\mathbb{T}^{n}$ since the identification
        $\pi(x_{1},\dots,x_{n})=(e^{ix_{n}},\dots,e^{ix_{n}})$ is, in each
        coordinate, the canonical representation of $\mathbb{S}^{1}$ in
        $\mathbb{C}=\mathbb{R}^{2}$, from which the product metric is induced
        by inclusion. The identity mapping being an isometry, hence the induced
        metric from the quotient is isometry to the product metric.
    \end{proof}
    \begin{problem}
        Obtain an isometric immersion of the flat torus
        $\mathbb{T}^{n}$ into $\mathbb{R}^{2n}$.
    \end{problem}
    \begin{proof}[Solution]
        Points on the torus can be identified with points in the complex
        product space $\mathbb{C}^{n}$ via
        $(x_{1},\dots,x_{n})\mapsto(e^{ix_{1}},\dots,e^{ix_{n}})$. This is
        the same as the mapping into $\mathbb{R}^{2n}$ given by:
        \begin{equation}
            (x_{1},\dots,x_{n})\mapsto
            \big(\cos(x_{1}),\sin(x_{1}),\dots,\cos(x_{n}),\sin(x_{n})\big)
        \end{equation}
        This function is smooth, being a trigonometric function in each
        component, and is a smooth embedding of $\mathbb{T}^{n}$ into
        $\mathbb{R}^{2n}$, and is hence an immersion. By the previous problem
        the induced metric is isometric to the flat torus, meaning this function
        is indeed an isometric embedding, hence an isometric immersion.
    \end{proof}
    \begin{problem}
        Show that the group of isometries of $\mathbb{S}^{n}$ are just the
        orthogonal linear maps on $\mathbb{R}^{n+1}$ restricted to
        $\mathbb{S}^{n}$.
    \end{problem}
    \begin{proof}[Solution]
        One direction is essentially by definition since the restriction of
        an orthogonal linear function to $\mathbb{S}^{n}$ is a map from
        $\mathbb{S}^{n}$ to $\mathbb{S}^{n}$ and such functions preserve the
        dot product on $\mathbb{R}^{n+1}$, hence preserving the round metric
        on $\mathbb{S}^{n}$. Given an isometry $f$ on $\mathbb{S}^{n}$, extend
        $f$ to $\mathbb{R}^{n+1}$ via:
        \begin{equation}
            F(\mathbf{x})=
            \begin{cases}
                f\Big(\frac{\mathbf{x}}{||\mathbf{x}||}\Big)||\mathbf{x}||
                    &\mathbf{x}\ne{0}\\
                0&\mathbf{x}=0
            \end{cases}
        \end{equation}
        It suffices to show $F$ is a linear orthogonal function on
        $\mathbb{R}^{n+1}$. Since $f$ is an isometry we have:
        \begin{equation}
            F(\mathbf{x})\cdot{F}(\mathbf{y})=
                ||\mathbf{x}||\;||\mathbf{y}||
                f\Big(\frac{\mathbf{x}}{||\mathbf{x}||}\Big)\cdot
                f\Big(\frac{\mathbf{y}}{||\mathbf{y}||}\Big)
            =||\mathbf{x}||\;||\mathbf{y}||
                \frac{\mathbf{x}}{||\mathbf{x}||}\cdot
                \frac{\mathbf{y}}{||\mathbf{y}||}
            =\mathbf{x}\cdot\mathbf{y}
        \end{equation}
        For all $\mathbf{x},\mathbf{y}\ne{0}$, and since $F$ maps
        $\mathbf{0}$ to $\mathbf{0}$, we have that $F$ is an isometry of
        $\mathbb{R}^{n+1}$. Moreover, $F$ is an isometry of $\mathbb{R}^{n+1}$
        that fixes the origin. Hence $F$ is a linear orthogonal map. That is,
        all isometries of $\mathbb{R}^{n}$ are obtained as affine transformations,
        rotations, reflections, plus translations. Since the origin is fixed
        there is no translation part, leaving an orthogonal linear map.
    \end{proof}
    \begin{problem}
        Show that $M$ \textit{is locally isometric to} $N$ is not a symmetric
        relation.
    \end{problem}
    \begin{proof}[Solution]
        A local isometry is a local diffeomorphism, by definition. The
        real line has a local isometry into $\mathbb{S}^{1}$ given by
        $f(x)=e^{ix}$. There is no local isometry from $\mathbb{S}^{1}$ to
        $\mathbb{R}$. Any such function is continuous, and since
        $\mathbb{S}^{1}$ is compact and connected, the image of $\mathbb{S}^{1}$
        under such a function is a closed bounded interval $[a,b]$. But then at
        a point $p\in\mathbb{S}^{1}$ with $f(p)=a$ or $f(p)=b$ the differential
        must be zero (otherwise $a$ wouldn't be the min or $b$ wouldn't be the
        max). But local diffeomorphisms induce linear isomorphisms via the
        differential pushforward, so the differential can't be the zero map.
        So there is no local isometry of $\mathbb{S}^{1}$ into $\mathbb{R}$.
    \end{proof}
    \begin{problem}
        Show that parallel transport induces an isometry of tangent spaces.
    \end{problem}
    \begin{proof}[Solution]
        Assume the connection is compatible with the metric (I think this is
        needed, but the problem doesn't specify). Parallel transport of
        $X$ along $\gamma$ means $\nabla_{\dot{\gamma}(t)}X=0$ for all $t$.
        Given two tangent vectors $u,v\in{T}_{p}M$, $p=\gamma(0)$, with
        resulting vector fields $Y_{t}$ and $Z_{t}$ obtained by parallel
        transport along $\gamma$ for time $t$, we have:
        \begin{equation}
            \frac{\textrm{d}}{\textrm{d}t}g_{\gamma(t)}\big(
                Y_{t},Z_{t}
            \big)=g_{\gamma(t)}\big(\nabla_{\dot{\gamma}(t)}Y_{t},Z_{t}\big)+
                g_{\gamma(t)}\big(Y_{t},\nabla_{\dot{\gamma(t)}}Z_{t}\big)
        \end{equation}
        But $Y_{t}$ and $Z_{t}$ are parallel along $\gamma$, so
        $\nabla_{\dot{\gamma}(t)}Y_{t}=0$ and
        $\nabla_{\dot{\gamma}(t)}Z_{t}=0$, meaning:
        \begin{equation}
            \frac{\textrm{d}}{\textrm{d}t}\Big(
                g_{\gamma(t)}(Y_{t},Z_{t})\Big)=0
        \end{equation}
        and hence $g_{\gamma(t)}(Y_{t},Z_{t})$ is a constant. Hence the dot
        product at the end of the parallel translation is the same as the dot
        product at the beginning, hence this is an isometry between the tangent
        spaces.
    \end{proof}
    \begin{problem}
        Show by example that the parallel translation of a vector depends on
        the curve.
    \end{problem}
    \begin{proof}[Solution]
        Take any non-zero tangent vector at the north pole $N\in\mathbb{S}^{2}$.
        Transverse via great-circle (a geodesic) to the equator. Travel along
        the equator by an azimuthal angle of $\pi/2$. Then travel back up the
        meridian to the equator (again via great circle). The resulting vector
        is orthogonal to the starting tangent vector. Now take that same vector
        at the north pole, travel along the entirety of a great circle, to the
        south pole and back. The resulting tangent vector is the same as the
        one you started with.
    \end{proof}
    \begin{problem}
        do Carmo Chapt. 2 Problem 4 (Lots of words, didn't type it out).
    \end{problem}
    \begin{proof}[Solution]
        We have that for such a vector field $V$ the following:
        \begin{align}
            \frac{\textrm{d}}{\textrm{d}t}\Big(V\cdot{V}\Big)
            &=\frac{\textrm{d}V}{\textrm{d}t}\cdot{V}+
                V\cdot\frac{\textrm{d}V}{\textrm{d}t}\\
            &=2V\cdot\frac{\textrm{d}V}{\textrm{d}t}\\
            &=\nabla_{\dot{\gamma}}V\cdot{V}+
                V\cdot\nabla_{\dot{\gamma}}V\\
            &=2V\cdot\nabla_{\dot{\gamma}}V
        \end{align}
        where we have used symmetry and compatibility with the metric.
        Hence we have:
        \begin{equation}
            V\cdot\frac{\textrm{d}V}{\textrm{d}t}=
            V\cdot\nabla_{\dot{\gamma}}V
        \end{equation}
        The left side is zero if and only if $V$ is perpendicular to
        $\textrm{d}V/\textrm{d}t$ and the right hand side is zero if and only
        if $V$ is parallel along $\gamma$. Hence $V$ is parallel along $\gamma$
        if and only if $V$ is perpendicular to $\textrm{d}V/\textrm{d}t$.
        Applying this to the sphere with great circles, constant speed great
        circles have acceleration vectors that point radially outward,
        orthogonal to the sphere. The velocity vector of a curve in the sphere
        is tangential to it, meaning for great circles the acceleration and
        velocity vectors are perpendicular. By the previous result, the
        velocity vector field is a parallel vector field along great circles.
    \end{proof}
    \begin{problem}
        Let $M$ be a Riemannian manifold, $p\in{M}$, $\gamma:[0,1]\rightarrow{M}$
        a constant curve, $\gamma(t)=p$. Let $V$ be a vector field along
        $\gamma$, which is just a curve $V:[0,1]\rightarrow{T}_{p}M$. Show that
        $DV/dt=dV/dt$.
    \end{problem}
    \begin{proof}[Solution]
        We have, in local coordinates $(\mathcal{U},\varphi)$ with
        $p\in\mathcal{U}$, that:
        \begin{equation}
            \frac{\textrm{D}V}{\textrm{d}t}=
                \sum_{n=0}^{N-1}\frac{\textrm{d}v_{n}}{\textrm{d}t}
                    \frac{\partial}{\partial\varphi_{n}}+
                \sum_{m=0}^{N-1}\sum_{n=0}^{N-1}
                    \frac{\textrm{d}\varphi_{n}(t)}{\textrm{d}t}v_{m}(t)
                    \nabla_{\partial\varphi_{n}}\partial\varphi_{m}
        \end{equation}
        But since $\gamma$ is a constant curve, $\varphi_{n}(t)$ is constant
        meaning it's derivative is zero. The resulting function is:
        \begin{equation}
            \frac{\textrm{D}V}{\textrm{d}t}=
                \sum_{n=0}^{N-1}\frac{\textrm{d}v_{n}}{\textrm{d}t}
                    \frac{\partial}{\partial\varphi_{n}}
        \end{equation}
        But this is just $\frac{dV}{dt}$ by the chain rule.
    \end{proof}
\end{document}
