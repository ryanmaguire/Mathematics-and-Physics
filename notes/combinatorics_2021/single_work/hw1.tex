\documentclass{article}
\usepackage[utf8]{inputenc}
\usepackage{amsmath}
\usepackage{listings}
\usepackage{xcolor}

\title{HW I}
\author{Ryan Maguire}
\date{\today}


\begin{document}
    \maketitle
    \section*{Problem 1}
        I asked Ben Adenbaum for clarification on what \textit{combinatorial proof} meant
        and asked about what compositions are, as well as consulted Wikipedia.
        \par\hfill\par\noindent
        First, we compute the size of each set and then just compare the formulas.
        \par\hfill\par\noindent
        The number of compositions of a positive integer $n$ can be found by writing
        $n=1+1+\cdots+1$ where there are $n$ 1's. There are then $n-1$ + symbols. At each
        plus sign we can choose to insert a parenthesis or not, dividing the 1's to get
        new numbers. For example, $6=1+1+1+1+1+1=(1+1)+(1)+(1+1+1)=2+1+3$, which
        corresponds to the example $(2,1,3)$ outlined in the problem. Since there are
        $n-1$ plus signs, we have $2^{n-1}$ total possibilities. So there are
        $2^{n-1}$ compositions of a positive number $n$.
        \par\hfill\par\noindent
        The size of $\{1,\dots,n\}$ is, by the very definition of cardinality, $n$.
        \par\hfill\par\noindent
        The number of permutations of $\{1,2,\dots,n\}$ is $n!$ as can be shown by
        induction. Supposing it holds for $\{1,2,\dots,n\}$, given the set $\{1,\dots,n+1\}$,
        there are $n+1$ ways to choose the first element of our permutation, and then by
        induction $n!$ ways for the rest, yielding $(n+1)\cdot{n}!=(n+1)!$ permutations.
        \par\hfill\par\noindent
        The set of prime numbers smaller than $n$ can be injectively embedded into
        $\{1,2,\dots,n\}$ by the identity mapping. This injection is never surjective
        since $1$ and $n$ will not be in the set of prime numbers \textit{less} than $n$.
        \par\hfill\par\noindent
        Lastly, the power set of $\{1,2,\dots,n\}$ can be shown to have cardinality $2^{n}$
        by induction. Supposing it true for $\{1,2,\dots,n\}$, given the set
        $\{1,2,\dots,n,n+1\}$, for any subset either $n+1$ lies in the subset of it does not.
        Ignoring it, we obtain a subset of $\{1,2,\dots,n\}$. By induction there are
        $2^{n}$ such sets. Hence there are $2^{n}$ subsets of $\{1,2,\dots,n,n+1\}$
        containing $n+1$ and $2^{n}$ not containing it, and the intersection of this
        division is empty. Hence there are $2^{n}+2^{n}=2^{n+1}$ subsets of
        $\{1,2,\dots,n,n+1\}$.
        \par\hfill\par\noindent
        Now, we compare. Letting $P_{n}$ denote the number of primes less than $n$, we have
        $P_{n}<n$. We also trivially have $2^{n-1}<2^{n}$. We can see that $2^{n-1}>n$ for all
        $n>2$ by looking at the derivatives of $2^{x-1}$ and $x$. One is increasing, and the
        other is constant. For $n=1$ and $n=2$, $2^{n-1}=n$. So, for $n>2$ we so for have
        $P_{n}<n<2^{n-1}<2^{n}$. Lastly, $n!$. We can show that this will eventually be bigger
        than $2^{n}$ since:
        \begin{subequations}
            \begin{align}
                n!&=n\cdot(n-1)\cdots{4}\cdot{3}\cdot{2}\\
                  &>4\cdot{4}\cdots{4}\cdot{3}\cdot{2}\\
                  &>2\cdot{2}\cdots{2}\cdot{2}\cdot{2}\cdot{2}\\
                  &=2^{n}
            \end{align}
        \end{subequations}
        Where we split one of the fours into $2\cdot{2}$ to get the extra 2 at the end.
        So, for large enough $n$ we have:
        \begin{equation}
            P_{n}<n<2^{n-1}<2^{n}<n!
        \end{equation}
    \section*{Problem 2}
        Big $O$ notation indicates an upper bound, not a lower one.
        It would make sense to say \textit{The running time of the algorithm is at most}
        $O(n^{3})$. The $\Omega$ notation gives a lower bound. As mentioned in office hour,
        $O(n^{3})$ doesn't rule out the algorithm being $O(n^{2})$, for example.
    \section*{Problem 3}
        Suppose $f(N)=N\log_{2}(N)+O(N)$. Then:
        \begin{equation}
            \frac{f(N)}{N\log_{2}(N)}=\frac{N\log_{2}(N)+O(N)}{N\log_{2}(N)}
            =1+\frac{O(N)}{N\log_{2}(N)}
        \end{equation}
        But any element of $O(N)$ will be dominated, eventually, by $N\log_{2}(N)$ since 
        $N/(N\log_{2}(N))=1/\log_{2}(N)\Rightarrow{0}$. Hence $f(N)/(N\log_{2}(N))$
        tends to a constant, and is therefore bounded both above and below, so
        $f(N)=\Theta(N\log_{2}(N))$.
    \section*{Problem 4}
        My understanding is that this is a far more general version of the Birthday
        problem. We want the probability that 2 keys have the same value, 3 keys, 4 keys,
        etc., and the compute the expectation value of this. I was unable to get a
        general formula, unfortunately, and resorted to a program. The value for
        the probability that there are 2 keys with the same value is exactly the
        birthday problem. On the next page is the code used to simulate the problem
        and the output it generated.
        \newpage
        \begin{lstlisting}[
            language=C,
            gobble=12,
            basicstyle=\footnotesize\ttfamily,
            keywordstyle=\bfseries\color{green!40!black},
            commentstyle=\itshape\color{purple!40!black},
            identifierstyle=\color{blue}
        ]
            #include <stdlib.h>
            #include <stdio.h>
            
            int main(void)
            {
                /*  Variables for indexing.                         */
                unsigned int l, m, n;
                
                /*  Number of keys in the file.                     */
                unsigned int N     = 1E4;
                
                /*  Maximum key in the file.                        */
                unsigned int max   = 1E6;
                
                /*  Number of simulations we'll run.                */
                unsigned int iters = 100;
                
                /*  Random array of keys.                           */
                unsigned int *x = malloc(sizeof(*x) * N);
                
                /*  Variables for computing the expected value.     */
                double average = 0.0;
                unsigned int counter, old;
            
                /*  Loop over iters number of simulations.          */
                for (l = 0; l < iters; ++l)
                {
                    /*  Reset the counter to zero and randomize x.  */
                    counter = 0;
                    old = 0;
                    for (n = 0; n < N; ++n)
                        x[n] = (rand() % max);
            
                    /*  Check how many entries of x equal m.        */
                    for (m = 0; m < max; ++m)
                    {
                        for (n = 0; n < N; ++n)
                        {
                            if (x[n] == m)
                                ++counter;
                        }
                        /*  If counter = old + 1, no duplicates.    */
                        if (counter - old == 1)
                            counter = old;
                        else
                            old = counter;
                    }
                    average += counter;
                }
                average /= iters;
                printf("Average Value: %f\n", average);
                free(x);
                return 0;
            }
        \end{lstlisting}
        Fixed typo from previous rendition. 10 simulations yield "Average Value: 104.100000"
        and 100 gave "Average Value: 98.290000" which seems better.
\end{document}
