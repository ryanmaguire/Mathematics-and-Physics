\documentclass{article}
\usepackage[utf8]{inputenc}
\usepackage{amsmath}
\usepackage{listings}
\usepackage{xcolor}

\title{HW II}
\author{Ryan Maguire}
\date{\today}

\begin{document}
    \maketitle
    \section*{Problem 1}
        Actual code is attached. The pseudo-code goes something like:
        \begin{enumerate}
            \item If length of list is 1, return list.
            \item If length of list is 2, sort list. This is to avoid floor(length/3) being zero.
            \item Otherwise, split list in two sublists of lengths floor(length/3) and one of that is
                  length - $2\cdot\text{floor(length/3)}$, label LA, LB, and LC, respectively.
            \item Set LA to mergesort(LA), LB to mergesort(LB), and LC to mergesort LC.
            \item Return threewaymerge(LA, LB, LC).
        \end{enumerate}
        threewaymerge looks as follows:
        \begin{enumerate}
            \item Append ``infinity'' to the end of LA, LB, LC.
            \item Set k, indA, indB, and indC to zero.
            \item while ( k $<$ (size(LA) + size(LB) + size(LC)) do 
            \begin{enumerate}
                \item if (LA[indA] $\leq$ LB[indB] and LA[indA] $\leq$ LC[indC]) set out[k] = LA[indA]. Increment indA.
                \item else if (LB[indB] $\leq$ LC[indC]) set out[k] = LB[indB]. Increment indB.
                \item else set out[k] = LC[indC]. Increment indC.
                \item increment k.
            \end{enumerate}
            \item Return out.
        \end{enumerate}
        We can see that this will also be $O\big(n\ln(n)\big)$. We make the hypothesis that for $n=3^{m}$ we have
        $T(m)=T(m/3)+T(m/3)+T(m/3)+\Theta(n)$ where $\Theta(n)$ comes from the merging procedure. Following the
        exact steps from class, we end up with $3^{m}\Theta(1)+m\Theta(n)$ which we then simplify using $m=\log_{3}(n)$,
        and then recognizing $\Theta\big(\log_{3}(n)\big)=\Theta\big(\ln(n)\big)$, we have
        $T(n)\in\Theta\big(n\ln(n)\big)$. This can be seen in the attached plots as well where a least squares-fit to the
        sorting problem for arrays of size 1 to 8000 was performed, minimizing the error in:
        \begin{equation}
            \sqrt{\sum_{n=0}^{N-1}\Big(\text{data}[n]-an\ln(n)\Big)^2}
        \end{equation}
        Yielding $a\approx{0.048}$.
    \section*{Problem 2}
        Going in a slightly different order, we can find a closed form pretty nicely using the
        binomial theorem. For $2^{n}$, counting the number of 1's that occur in all of the numbers
        before $2^{n}$ is just counting the number of 1 that can fit into $n$ slots.
        There are $\binom{n}{1}$ ways to have one 1, $\binom{n}{2}$ ways to have 2, and so forth.
        The total number is the weighted sum:
        \begin{equation}
            p_{2^{n}}=\sum_{k=1}^{n}k\binom{n}{k}
        \end{equation}
        The binomial theorem tells us how to simplify this, and we get the closed form:
        \begin{equation}
            p_{2^{n}}=n2^{n-1}
        \end{equation}
        Which agrees with all of the values in the table. This can also give us a
        recurrence:
        \begin{subequations}
            \begin{align}
                p_{2^{n+1}}&=(n+1)2^{n}\\
                    &=n2^{n}+2^{n}\\
                    &=2(n2^{n-1})+2^{n}\\
                    &=2p_{2^{n}}+2^{n}
            \end{align}
        \end{subequations}
        Asympotically, we know the function is strictly increasing, so:
        \begin{equation}
            p_{2^{\text{floor}(\log_{2}(n))}}\leq{p}_{n}\leq{p}_{2^{\text{ceil}(\log_{2}(n))}}
        \end{equation}
        Which gives us:
        \begin{equation}
            m2^{m-1}\leq{p}_{n}\leq(m+1)2^{m}
        \end{equation}
        Using $m\approx\log_{2}(n)$, we have $p_{n}\in{O}(n\log_{2}(n))$.
        We can also try to make sense of this intuitively. $\log_{2}(n)$ is telling us
        how many binary digits there are, and $n/2$ is the average of the values between $1$ and $n$.
        So $(n/2)\cdot\log_{2}(n)$ is roughly how many ones occur in the numbers less than $n$.
        But $(n/2)\cdot\log_{2}(n)$ is in $O(n\cdot\log_{2}(n))$.
    \section*{Problem 3}
        The fastest way to find a single book is to just search through the entire list top to bottom.
        This linear search has a best case of $O(1)$ (the book is on top) and a worst case of $O(n)$.
        The average case is, similar to the max algorithm, $\ln(n)+\gamma$, $\gamma$ being the
        Euler-Mascheroni constant, which we can also approximate with $H_{n}$, the $n^{th}$ harmonic
        number.
        \par\hfill\par
        If we are getting queries daily, we should sort first, alphabetically. Perform a merge-sort
        giving us a time-complexity of $n\ln(n)$. When we receive a query we may then perform a
        binary search. Start at the middle, if the requested book is less than this middle book,
        go to the one-fourth point, if greater than go to the three-fourths point, and if the
        middle book is the one we want, we're done. This has worst case scenario of $\log_{2}(n)$
        meaning our overall worst-case scenario for the $\sqrt{n}$ days is $n\ln(n)+\sqrt{n}\ln(n)$.
        Compare this with a linear sort every day, we would have $n\sqrt{n}=n^{3/2}$, which is worse.
\end{document}