\documentclass{article}
\usepackage[utf8]{inputenc}
\usepackage{amsmath, amssymb}
\usepackage{listings}
\usepackage{xcolor}

\title{HW IV}
\author{Ryan Maguire}
\date{\today}

\begin{document}
    \maketitle
    \section*{Problem 1}
        No idea, moving on (I truly don't get this symbolic method yet. Sorry about that.
        I'll try and bug you in office hour next week).
    \section*{Problem 2}
        I forget how we did it in class, so hopefully this is not a repeat.
        My idea was the get a recurrence that is easy enough to work with, and try
        to extract the sum from this. Letting $a_{n}$ denote the average number of
        cycles in a permutation of length $n$, I claim:
        \begin{equation}
            a_{n}=a_{n-1}+\frac{1}{n}
        \end{equation}
        Given a permutation of $\mathbb{Z}_{n}$, there are $(n-1)!$ ways to introduce
        the extra element of $\mathbb{Z}_{n}$ to a permutation of $\mathbb{Z}_{n-1}$
        since the permutations mapping the greatest element of $\mathbb{Z}_{n}$ to
        itself do not introduce a new cycle. There are $n!$ permutations total, so on
        average we expect $(n-1)!/n!$ more cycles for a permutation of $\mathbb{Z}_{n}$
        than $\mathbb{Z}_{n-1}$, which gives us $a_{n}=a_{n-1}+1/n$. Going down this
        chain, we then see that:
        \begin{equation}
            a_{n}=a_{1}+\frac{1}{2}+\frac{1}{3}+\dots+\frac{1}{n}
        \end{equation}
        Since we known $a_{1}=1$, we obtain:
        \begin{equation}
            a_{n}=\sum_{k\in\mathbb{Z}_{n}}\frac{1}{n+1}
        \end{equation}
        (My $\mathbb{Z}_{n}$ starts at zero and ends at $n-1$, sorry). By definition,
        this is the $n^{th}$ harmonic number, $H_{n}$.
    \section*{Problem 3}
        We just need to expand the binomial distribution and look for asymptotic behavior.
        We have:
        \begin{equation}
            \binom{n}{k}p^{k}(1-p)^{n-k}
                =\frac{n(n-1)\cdots(n-k+1)}{k!}\Big(\frac{\lambda}{n}\Big)^{k}
                    \Big(1-\frac{\lambda}{n}\Big)^{n-k}
        \end{equation}
        We note that the top product is just a polynomial in $n$ of degree $k$.
        We write:
        \begin{equation}
            \binom{n}{k}p^{k}(1-p)^{n-k}
                =\frac{n^{k}+\mathcal{O}(n^{k-1})}{k!}\frac{\lambda^{k}}{n^{k}}
                    \Big(1-\frac{\lambda}{n}\Big)^{n-k}
        \end{equation}
        Now we use the $n^{k}$ on the bottom to write:
        \begin{equation}
            \binom{n}{k}p^{k}(1-p)^{n-k}
                =\frac{\lambda^{k}}{k!}\Big(1-\frac{\lambda}{n}\Big)^{n-k}
                    \Big(1+\mathcal{O}\big(\frac{1}{n}\big)\Big)
        \end{equation}
        For large $n$ we neglect the $\mathcal{O}(1/n)$ term, and note that, per the
        original definition put forward by Jacob Bernoulli when he was studying the
        mathematics of banking, we have:
        \begin{equation}
            \exp(x)=\lim_{n\rightarrow\infty}\Big(1+\frac{x}{n}\Big)^{n}
        \end{equation}
        Combining this with our expression, we have:
        \begin{equation}
            \binom{n}{k}p^{k}(1-p)^{n-k}\sim\frac{\lambda^{k}}{k!}\exp(-\lambda)
        \end{equation}
        Or, written more suggestively:
        \begin{equation}
            \binom{n}{k}p^{k}(1-p)^{n-k}\sim
                \frac{\lambda^{k}e^{-\lambda}}{k!}
        \end{equation}
        Which is the Poisson distribution.
    \section*{Problem 4}
        Note: Saw this trick in a numberphile video a few years back.
        \par\hfill\par\noindent
        The first thing to note is that $9801=99^{2}$, so $1/9801=1/99^{2}$. We can find
        the decimal expansion of $1/99$ since 99 is just 1 less than 100. We have:
        \begin{equation}
            \frac{1}{99}=\sum_{n=1}^{\infty}\frac{1}{100^{n}}=0.010101\dots
        \end{equation}
        Looking at the series for $x^{n}$ for $n\geq{1}$, we have:
        \begin{subequations}
            \begin{align}
                \sum_{n=1}^{\infty}x^{n}
                &=-1+\sum_{n=0}^{\infty}x^{n}\\
                &=-1+\frac{1}{1-x},\quad|x|<1\\
                &=\frac{x}{1-x}
            \end{align}
        \end{subequations}
        We want to compute the square of this:
        \begin{equation}
            \Big(\sum_{n=1}^{\infty}x^{n}\Big)^{2}=\frac{x^{2}}{(1-x)^{2}}
        \end{equation}
        And we can confirm plugging in $x=1/100$ gives us $1/9801$. So we need a nicer formula
        for this square of the sum. This is where Cauchy's product formula comes in handy. We
        have:
        \begin{subequations}
            \begin{align}
                \Big(\sum_{n=1}^{\infty}x^{n}\Big)^{2}
                    &=\Big(\sum_{n=0}^{\infty}x^{n+1}\Big)
                        \Big(\sum_{n=0}^{\infty}x^{n+1}\Big)\\
                    &=x^{2}\sum_{n=0}^{\infty}\Big(\sum_{k=0}^{n}1\Big)x^{n}\\
                    &=x^{2}\sum_{n=0}^{\infty}nx^{n}
            \end{align}
        \end{subequations}
        Plugging in $x=0.01$ we get $0.00xxx$ from the $x^{2}$ on the front, but we
        can then read off the decimals from the summand. It's just the integer $n$.
        So we have:
        \begin{equation}
            \frac{1}{9801}=0.00010203040506070809101112131415161718\dots
        \end{equation}
\end{document}