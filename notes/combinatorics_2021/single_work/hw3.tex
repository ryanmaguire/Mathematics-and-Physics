\documentclass{article}
\usepackage[utf8]{inputenc}
\usepackage{amsmath}
\usepackage{listings}
\usepackage{xcolor}

\title{HW III}
\author{Ryan Maguire}
\date{\today}

\begin{document}
    \maketitle
    \section*{Problem 1}
        We first take the log of the recurrence relation to
        get:
        \begin{subequations}
            \begin{align}
                \ln(a_{n})&=\ln\big(\sqrt{a_{n-1}a_{n-2}}\big)\\
                    &=\frac{1}{2}\ln(a_{n-1}a_{n-2})\\
                    &=\frac{1}{2}\ln(a_{n-1})+
                        \frac{1}{2}\ln(a_{n-2})
            \end{align}
        \end{subequations}
        Labeling $F_{n}=\ln(a_{n})$ (since this looks like a
        Fibonacci sequence), we obtain:
        \begin{equation}
            F_{n}=\frac{1}{2}F_{n-1}+\frac{1}{2}F_{n-2}
        \end{equation}
        If we've studied difference equations, we know how to
        handle this, but if not we can use differential equations
        to guess how to solve this. We propose a solution of the
        form $F_{n}=a^{n}$ for some constant $a$. We get:
        \begin{subequations}
            \begin{align}
                a^{n+2}&=\frac{1}{2}a^{n+1}+\frac{1}{2}a^{n}\\
                \Rightarrow
                a^{2}-\frac{1}{2}a-\frac{1}{2}&=0
            \end{align}
        \end{subequations}
        This quadratic has two solutions, $a=1$ and
        $a=-\frac{1}{2}$ so the solution is a linear combination
        of these two. We get:
        \begin{equation}
            F_{n}=c_{0}+c_{1}\big(-\frac{1}{2}\big)^{n}
        \end{equation}
        We have two initial conditions, $F_{0}=\ln(1)=0$ and
        $F_{1}=\ln(2)$. Using this we have:
        \begin{subequations}
            \begin{align}
                c_{0}+c_{1}&=0\\
                c_{0}-\frac{1}{2}c_{1}&=1
            \end{align}
        \end{subequations}
        This gives us $c_{0}=2\ln(2)/3$ and
        $c_{1}=-2\ln(2)/3$. The sequence we wanted was the
        exponential of this:
        \begin{equation}
            a_{n}=\exp\Big[
                \frac{2\ln(2)}{3}\Big(
                    1-\big(-\frac{1}{2}\big)^{n}
                \Big)
            \Big]
        \end{equation}
        We can determine the limit with this, invoking
        the definition $x^{y}=\exp(y\ln(x))$, we see that
        this tends to $2^{2/3}$.
        \par\hfill\par
        The attached code shows that within 50 iterations we
        approach the limit to within the precision of my machine
        (16 significant digits).
    \section*{Problem 2}
        Lots of help from Ben on this one, and I didn't quite get
        it.
        \par\hfill\par
        $\exp(x)$ is the exponential generating function for the
        constant sequence $a_{n}=1$. Since we can't have a
        zero partition, we need to remove the zeroth term, meaning
        our exponential generating function will look like:
        \begin{equation}
            f(x)=\sum_{n=0}^{\infty}\frac{(\exp(x)-1)}{n!}
                =\exp\big(\exp(x)-1\big)
        \end{equation}
        One way I think works, and gives the same answer, is to
        just try to find a formula for counting the number of
        partitions, which gives us a recurrence relation:
        \begin{equation}
            P_{n}=\sum_{k=0}^{n}\binom{n}{k}P_{k}
        \end{equation}
        If we differentiate $f(x)$ we get:
        \begin{subequations}
            \begin{align}
                f(x)&=\sum_{n=0}^{\infty}\frac{P_{n}}{n!}x^{n}\\
                \Rightarrow
                \dot{f}(x)&=\sum_{n=0}^{\infty}\Big(
                        \sum_{k=0}^{n}\binom{n}{k}P_{k}
                    \Big)\frac{x^{n-1}}{(n-1)!}
            \end{align}
        \end{subequations}
        But we can note that this is just a Cauchy product:
        \begin{equation}
            \sum_{n=0}^{\infty}\sum_{m=0}^{\infty}
                a_{m}b_{n}
            =\sum_{k=0}^{\infty}c_{k}
        \end{equation}
        Where $c_{k}=\sum_{k=0}^{n}a_{k}b_{n-k}$. So we have:
        \begin{equation}
            \dot{f}(x)=
                \sum_{n=0}^{\infty}\frac{x^{n}}{n!}
                \sum_{m=0}^{\infty}\frac{P_{m}}{m!}x^{m}
                =\exp(x)f(x)
        \end{equation}
        So now we have a first order linear differential
        equation, which we can rewrite as:
        \begin{equation}
            \dot{y}-\exp(x)y=0
        \end{equation}
        We do the standard trick with
        $\mu(x)=\exp(\int\exp(x))=\exp(\exp(x))$ giving us:
        \begin{equation}
            f(x)=C\exp\big(\exp(x)\big)
        \end{equation}
        The initial condition is $f(0)=1$, so $C=1/e$, hence:
        \begin{equation}
            f(x)=\exp\big(\exp(x)-1\big)
        \end{equation}
        For part b, ignoring singletons is removing both the
        zero term and the coefficient of $x$, which gives us:
        \begin{equation}
            g(x)=\exp\big(\exp(x)-x-1\big)
        \end{equation}
    \section*{Problem 3}
        Using an inclusion-exclusion idea where we count how
        many permutations have at least 1 fixed point, then how
        many have two, removing duplicates, and proceeding, we
        can get:
        \begin{equation}
            a_{n}=n!\sum_{k=0}^{n}\frac{(-1)^{k}}{k!}
        \end{equation}
        To compute the exponential generating function, we have:
        \begin{equation}
            f(x)=\sum_{n=0}^{\infty}\frac{a_{n}}{n!}x^{n}
            =\sum_{n=0}^{\infty}\sum_{k=0}^{n}
                \frac{(-1)^{k}}{k!}x^{n}
        \end{equation}
        Trying to get this into the form of a Cauchy product,
        we have:
        \begin{equation}
            f(x)=\sum_{n=0}^{\infty}\sum_{k=0}^{n}
                \frac{(-1)^{n-k}}{(n-k)!}x^{k}x^{n-k}
                =\sum_{k=0}^{\infty}\frac{(-1)^{k}}{k!}x^{k}
                    \sum_{n=0}^{\infty}x^{n}
                    =\frac{\exp(-x)}{(1-x)}
        \end{equation}
        The proportion amongsts all permutations is:
        \begin{equation}
            \frac{n!\sum_{k=0}^{n}\frac{(-1)^{k}}{k!}}{n!}
            =\sum_{k=0}^{n}\frac{(-1)^{k}}{k!}
        \end{equation}
        The limit as $n$ tends to infinity is just $\exp(-1)$,
        or $1/e$. For the last part, we are looking for a
        derangement to occur. The probability will be the
        $30^{th}$ partial sum, but since the convergence is so
        fast we can approximate this as $1/e$.
\end{document}