\documentclass{article}
\usepackage[utf8]{inputenc}
\usepackage{amsmath, amssymb}
\usepackage{listings}
\usepackage{xcolor}

\title{HW VII}
\author{Ryan Maguire}
\date{\today}

\begin{document}
    \maketitle
    \section*{Problem 1}
        $0,1,1,1,4,3,3,2,8$.
        \par\hfill\par\noindent
        Going right to left, there are $n$ possible numbers that could
        be greater than the last, $n-1$ greater than the second last,
        and so on, yielding $n!$, which is the same size as the number of
        permutations of $\mathbb{Z}_{n}$. If we cut this off at $k$, we'd have
        $1\cdot{2}\cdots\cdot{k}\cdot{k}\cdots{k}$. We can use this to get the
        expected maximum. The number of ways to have the maximum be 2 is the
        number of ways to have at most 2 minus the number of ways to have at
        most 1. Similarly, the number of ways to have maximum $k$ is the number
        of ways to have maximum at most $k$ minus the number of ways to have
        maximum at most $k-1$. To get the expectation  value, we weight this
        by $k$, yielding:
        \begin{equation}
            \sum_{k=0}^{n}\Big(\frac{kk!k^{n-k}}{n!}-
                \frac{k(k-1)^{n+1-k}}{n!}\Big)
        \end{equation}
        Shifting the sum for the term on the right, and noting the $k=0$ part
        is redundant, yields:
        \begin{equation}
            \sum_{k=1}^{n}\frac{k!k^{n+1-k}}{n!}
            -\sum_{k=0}^{n-1}\frac{(k+1)k^{n-k}}{n!}
        \end{equation}
        This telescopes, giving us roughly:
        \begin{equation}
            \sum_{k=1}^{n-1}\frac{k!k^{n-k}}{n!}
        \end{equation}
        Which, since the $k=n$ term is 1, we get similar asymptotics yielding
        $\sqrt{\pi{n}/2}$.
    \section*{Problem 2}
        A pass corresponds precisely to an inversion, so the expected number is
        the same as the expected maximum of the inversion table. There is
        always one extra pass at the end, but asymptotically this is the same.
\end{document}