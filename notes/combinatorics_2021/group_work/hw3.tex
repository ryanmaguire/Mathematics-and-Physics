\documentclass{article}
\usepackage{mathtools}
\usepackage{amsthm, amsfonts, amssymb,amsmath}
\usepackage{float,enumitem}
\usepackage{tikz}
\usepackage{tikz-cd}
\usepackage{xcolor}
\usetikzlibrary{
    arrows.meta,            % Latex and Stealth arrows.
    decorations.markings    % Adding arrows to the middle of line segments.
}

%%%%%-------  OPERATORS------------%%%%%%
\def\sgn{\operatorname{sgn}}

%%%%%%%------COLORS------%%%%%
\def\tco{\textcolor{orange}}
\def\tcm{\textcolor{magenta}}
\def\tcg{\textcolor{green}}
\def\tclg{\textcolor{lightgray}}
\def\tcc{\textcolor{cyan}}
\def\tcp{\textcolor{purple}}

% Define theorem style for default spacing and normal font.
\newtheoremstyle{normal}
    {\topsep}               % Amount of space above the theorem.
    {\topsep}               % Amount of space below the theorem.
    {}                      % Font used for body of theorem.
    {}                      % Measure of space to indent.
    {\bfseries}             % Font of the header of the theorem.
    {}                      % Punctuation between head and body.
    {.5em}                  % Space after theorem head.
    {}

\theoremstyle{normal}
\newtheorem{problem}{Problem}
\newtheorem{definition}{Definition}
\newtheorem{notation}{Notation}

% Italic header environment.
\newtheoremstyle{thmit}{\topsep}{\topsep}{}{}{\itshape}{}{0.5em}{}

% Define environments with italic headers.
\theoremstyle{thmit}
\newtheorem*{solution}{Solution}

\title{Team Taussky\\ Homework 3}
\author{Lizzie Buchanan, Richard Haburcak,\\
        Matt Jones, Ryan Maguire}
\date{October 2020}

\begin{document}
\maketitle

\section{Problems}

Let $\chi^\lambda$ denote the irreducible character of belonging to the irreducible representation
$S^\lambda$.
\begin{problem}
    Use the Murnaghan-Nakayama Rule to prove the following:
    \begin{enumerate}[label = (\alph*)]
        \item $\chi^{\lambda'}(\sigma) = \sgn(\sigma)\chi^\lambda(\sigma)$
        \item If $\sigma$ has cycle type $n$ (ie $\sigma$ is an n-cycle) then:
            \begin{equation}
                \chi^\lambda (\sigma)=
                    \begin{cases}
                        (-1)^{\ell},&\ell=(n-\ell)1^{\ell}\\
                        0,&\textrm{Otherwise}
                    \end{cases}
            \end{equation}
        \item for $\chi$ a character of $S^n$, let $\chi\downarrow_{S_{n-1}}^{S_n}$ denote the
              restriction of $\chi$ from $S_n$ to $S_{n-1}$ = $\{\sigma \in S_n : n\sigma = n\}$. Show
              that $\chi\downarrow_{S_{n-1}}^{S_n} = \sum_\mu \chi^\mu$ where the sum on the right is
              over $\mu \vdash n-1$ obtained by removing a corner square from $\lambda$.
        \item Use part (c) to conclude
        \begin{equation}
            \chi^{n-\ell,\ell}\downarrow^{S_n}_{S_{n-1}} = \begin{cases}
            \chi^{n-1} & \ell = 0 \\
            \chi^{n-\ell-1,\ell} + \chi^{n-\ell,\ell-1} & 0<\ell<\frac{n}{2}\\
            \chi^{\frac{n}{2},\frac{n}{2}-1} & n \textrm{ is even and } \ell=\frac{n}{2}
            \end{cases}
        \end{equation}
    \end{enumerate}
\end{problem}
\begin{solution}
\begin{enumerate}[label = (\alph*)]
    \item
        Let $S=(s_1, \dots, s_l)$ be a border strip tableau of shape $\lambda$ and content
        $\sigma$. Let $S'$ be the result of swapping rows and columns of $S$. Let $s_i'$ be the image
        of each $s_i$. Observe that $s_l'$ is a border strip of $S'$, and then by induction on $l$,
        $S'=(s_1', \dots, s_l')$ is a border strip tableau of shape $\lambda'$ and content $\sigma$.
        Thus BST for $\lambda'$ and $\lambda$ are in one to one correspondence, and we can write
        (using the MNR)
        \begin{equation}
            \chi^{\lambda'}(\sigma)=\sum_S\prod_{i\in I_S} (-1)^{h(s_i')-1}
        \end{equation}

        Let $(\sigma_1, \dots, \sigma_l)$ be the cycle type of $\sigma$. Strip $s_i$ is a hook consisting
        of $\sigma_i$ boxes. So the height of $s_i'$ is
        $\textcolor{magenta}{h(s_i')}=\textcolor{magenta}{\sigma_i-h(s_i)+1}$, and we get
        (for fixed $S$) that 
        \begin{subequations}
            \begin{align}
                \prod_{i}(-1)^{h(s_i')-1}
                &=\prod_{i} (-1)^{\tcm{\sigma_i-h(s_i)+1}-1}\\
                &=\prod_{i}(-1)^{\sigma_i+1}(-1)^{h(s_i)-1}\\
                &=\sgn(\sigma)\prod_{i} (-1)^{h(s_i)-1}
            \end{align}
        \end{subequations}
        Thus
        \begin{subequations}
            \begin{align}
                \chi^{\lambda'}(\sigma)
                    &=\sum_S\sgn(\sigma)\prod_{i\in I_S} (-1)^{h(s_i')-1}\\
                    &=\sgn(\sigma)\chi^{\lambda}(\sigma)
            \end{align}
        \end{subequations}
    \item First, because $\sigma = (n)$, we get that $\lambda$ must be a hook, or else there are no possible border strip tableauxs and the character is zero. Thus, let $\lambda = (n-\ell)1^\ell$.
    
    Because $\sigma = (n)$, there is only one possible BST, so $chi^\lambda(\sigma) = -1^{h(s_1)-1}$ where $h(s_1)$ is the height of border strip $s_1$ which is the entire tableaux. And since the tableaux has $\ell + 1$ rows, we get $\chi^\lambda(\sigma)=-1^\ell$
    
    \item By the Murnaghan-Nakayama Rule, $\chi^\lambda = \sum_S \sigma(S)$. However, when considering the restriction onto $S_{n-1}$, we are fixing the $n$th element, and all border strip tableaux $S$ that don't end with a single 1-cycle are thrown out because they don't fix the last element. These 1-cycles are single boxes in the tableaux and therefore must be corners. Thus, the original sum decomposes into a sum of sums, with each larger sum being the sum over all border strip tableauxs with a given shape $\mu$, determined by removing a single corner box from the larger tableaux $\lambda$. These inner sums become $\chi^\mu$ by the Murnaghan-Nakayama Rule, and we are done. 
    
    \item This follows quickly from part (c). 
    
    If $\ell = 0$, there is only one row, and therefore only one corner square to remove. This leaves a single row with n-1 squares.
    
    if $0<\ell<\frac{n}{2}$, there are two corner squares that could be removed, from the top or bottom row. Removing the top corner square gives us a tableaux of shape $n-\ell-1, \ell$, and removing the bottom corner square gives a tableaux of shape $n-\ell, \ell-1$. 
    
    if $n$ is even and $\ell = \frac{n}{2}$, the tableaux is a rectangle and the only corner square is on the bottom right. Removing it takes one away from the bottom row, and the shape is $\frac{n}{2}, \frac{n}{2}-1$. 
    
    In each case, knowing the possible shapes from removing a corner gives us the final result via part (c).
    
\end{enumerate}
\end{solution}
\begin{problem}
    Prove the following:
    \begin{equation}
        B_{n,k}=\bigoplus_{\ell=0}^{k}\chi^{n-\ell,\ell}
    \end{equation}
    \begin{enumerate}
        \item
            Show that:
            \begin{equation}
                \beta\downarrow_{S_{n-1}}^{S_{n}}\simeq\beta_{n-1,k}\oplus\beta_{n-1,k-1}
            \end{equation}
        \item
            Show that, if $\lambda$ has at least 3 parts, then:
            \begin{equation}
                \langle{\beta_{n,k},\chi^{\lambda}}\rangle=0
            \end{equation}
        \item
            Show that:
            \begin{equation}
                \beta_{n,k}\downarrow_{S_{n-1}}^{S_{n}}
                =\chi^{n-1-k,k}+2\sum_{\ell=0}^{k-1}\chi^{n-1-\ell,\ell}
            \end{equation}
        \item
            Writing:
            \begin{equation}
                \beta_{n,k}=\sum_{\ell}^{\lfloor{n/2\rfloor}}a_{n-\ell,\ell}\chi^{n-\ell,\ell}
            \end{equation}
            Show:
            \begin{equation}
                a_{n-\ell,\ell}=
                \begin{cases}
                    0,&\ell>0\\
                    1,&\ell\leq{k}
                \end{cases}
            \end{equation}
        \item For $n$ even, and $k=n/2$, show that:
            \begin{equation}
                B_{n,k}\downarrow_{S_{n-1}}^{S_{n}}\simeq{B}_{n-1,k-1}\oplus{B}_{n-1,k-1}
            \end{equation}
        \item Conclude that:
            \begin{equation}
                a_{n-\ell,\ell}+a_{n-\ell+1,\ell-1}=2
            \end{equation}
        \item
            Show that $a_{n,0}=1$.
        \item
            Conclude that:
            \begin{equation}
                \beta_{n,n/2}=\sum_{\ell}^{n/2}\chi^{n-\ell,\ell}
            \end{equation}
    \end{enumerate}
\end{problem}
\begin{solution}
    In order:
    \begin{enumerate}
        \item
            Given an element of $B\downarrow_{S_{n-1}}^{S_{n}}$ either $n$ is in the index set or not. If
            It is, there are $k-1$ other elements from $\{1,2,\dots,n-1\}$, so this corresponds to
            $B_{n-1,k-1}$. Otherwise, if $k$ is not in it, then we are just considering $k$ elements of
            $\{1,2,\dots,n-1\}$. Now the intersection of these is trivial, and their sum spans, so we have
            $B\downarrow_{S_{n-1}}^{S_{n}}\simeq{B}_{n-1,k}\oplus{B}_{n-1,k-1}$.
        \item
            With the exception of the $1-1-1$ shape, if we have a $\lambda$ with at least three parts,
            then there will be one corner that we can remove that leaves us with a shape containing
            three parts. By the equation in problem 1, part c, we have:
            \begin{equation}
                \chi^{\lambda}\downarrow_{S_{n-1}}^{S_{n}}=\sum_{\mu}\chi^{\mu}
            \end{equation}
            But by the induction hypothesis, $B_{n,k}$ is the direct sum of shapes with 2 parts. So by
            the orthogonality relations, $\langle{\beta_{n,k},\chi^{\lambda}}\rangle=0$.
        \item
            By the induction hypothesis, we have:
            \begin{subequations}
                \begin{align}
                    \beta_{n,k}\downarrow_{S_{n-1}}^{S_{n}}
                    &=\beta_{n-1,k}\oplus\beta_{n-1,k-1}\\
                    &=\Big(\bigoplus_{\ell=0}^{k}\chi^{n-1-\ell,\ell}\Big)
                        \oplus\Big(\bigoplus_{\ell=0}^{k-1}\chi^{n-1-\ell,\ell}\Big)
                \end{align}
            \end{subequations}
            We collect the lone $\ell=k$ term from the left side, and note we have two of everything
            for all other $\ell$, giving us the result.
        \item
            By problem 1, part d, we have for $\ell>k$ that:
            \begin{equation}
                \chi^{n-\ell,\ell}\downarrow_{S_{n-1}}^{S_{n}}
                =\chi^{n-\ell-1,\ell}+\chi^{n-\ell,\ell-1}
            \end{equation}
            But from the sum, there are no parts of this form for when $k>\ell$, so we conclude the
            the coefficient must be zero. Now, to get the $\chi^{n-k-1,k}$ term we must have the
            $\ell=k$ term, and we must have only one of those. But then to get the rest of the sum, this
            also fixes what the rest of the coefficients must be, and that is 1 for $\ell<k$.
        \item
            We do the same this as in part $a$, but now we note the for $k=n/2$ that, since
            $\binom{n}{k}=\binom{n}{n-k}$ and $n/2-1=(n-1)-n/2=(n-1)-k$ we can see that
            $B_{n-1,k-1}$ and $B_{n-1,k}$ will be the same size, i.e. we can get an isomorphism between
            them and hence can revert back to part a of this problem.
        \item
            Combining 1 part d with the induction hypothesis, we have:
            \begin{subequations}
                \begin{align}
                    \beta_{n,k}\downarrow_{S_{n-1}}^{S_{n}}
                            &=2\beta_{n-1,n/2-1}\\
                            &=2\sum_{\ell=0}^{k-1}\chi^{n-1-\ell,\ell}\\
                            &=\sum_{\ell=0}^{k-1}(a_{n-\ell,\ell}+a_{n-\ell+1,\ell-1})\chi^{n-\ell-1,\ell}
                \end{align}
            \end{subequations}
            And from this we conclude $a_{n-\ell,\ell}+a_{n-\ell+1,\ell-1}=2$.
        \item
            We note that $a_{n,0}$ is the coefficient of the trivial representation. There is one
            invariant subspace, and that is the space spanned by the sum of all of the basis elements,
            and so $a_{n,0}=1$.
        \item
            By combining part g and f of this problem, we have that all of the $a_{n-\ell,\ell}$ are
            1 which gives us the sum.
    \end{enumerate}
\end{solution}
\begin{problem}
    Let $\Gamma_G(\lambda)=\sum_{j=1}^{n-1}\dim(H_{n-3}^{(j)}(G))\lambda^j$. 
    \newline
    We are going to show that
    \begin{equation}
        \Gamma_G(\lambda)=(-1)^n\big(\chi_G(-\lambda)-(-\lambda)^n\big)\tag{*}
    \end{equation}
    which is equivalent to proving that 
    \begin{equation}
        \dim (H_{n-3}^{(j)}(G))=c_j
    \end{equation}
    (where the $c_j$ are from the equation
    $\chi_G(\lambda)=\lambda^n - c_{n-1}\lambda^{n-1} + c_{n-2}\lambda^{n-2}-\dots$ from HW 1).
    Later we proceed by induction, for now deal with base case: graph $E$ with one edge going
    from $u$ to $v$. 
    \begin{enumerate}[label = (\alph*)]
        \item
            Let $\beta = (B_1, \dots, B_{n-1})$ be the element of $\triangle_{n-3}(E)$ where
            $B_{n-1}=\{u,v\}$ and $B_1, \dots, B_{n-2}$ are singletons containing the remaining
            elements of $\{1,\dots, n\} - \{u,v\}$. Show that
            \begin{equation}
                d_{n-3}\Big(e_{n-1}^{(n-1)}\cdot(B_1,\dots,B_{n-1})\Big)=0
            \end{equation}
        \item Conclude that $(*)$ holds for $E$. 
    \end{enumerate}
\end{problem}
\begin{solution}
\begin{enumerate}[label = (\alph*)]
    \item 
    From class notes,
    $e_{n-1}^{(n-1)}=\epsilon_{n-1}:=\frac{1}{(n-1)!}\sum_{\sigma\in S_{n-1}}\sgn(\sigma)\sigma$. 

So, by linearity of $d$, we know that 
\begin{align*}
    d_{n-3}(e_{n-1}^{(n-1)}(B_1, \dots, B_{n-1}))
    &=\frac{1}{(n-1)!}\sum_{\sigma\in S_{n-1}}\sgn(\sigma)d_{n-3}\sigma(B)\\
    &=\frac{1}{(n-1)!}\sum_{\sigma\in S_{n-1}}\sgn(\sigma)
    \sum_{i=1}^{n-2}(-1)^i(B_{\sigma 1},\dots, B_{\sigma i}\cup B_{\sigma {i+1}},\dots,B_{\sigma(n-1)})
\end{align*}

(The fraction at the start will not affect us, so we ignore it from now on. )
For any $\sigma \in S_{n-1}$, the term $ (B_{\sigma 1}, \dots, B_{\sigma i} \cup B_{\sigma {i+1}}, \dots, B_{\sigma(n-1)})$ appears exactly twice on the RHS: once with coefficient $\sgn(\sigma)(-1)^i$, and once with coefficient $\sgn(\sigma (i, i+1)) (-1)^i$. As $\sgn(\sigma (i, i+1))=-\sgn(\sigma)$, we see that these coefficients will cancel out.

Since $\sigma$ was arbitrary, all terms have coefficient $0$ in the final sum, so the whole thing is $0$ and we are done. 

\item As noted in the problem set, it suffices to prove $\dim H_{n-3}^{(j)}=c_j.$ For our graph $E$, we have that $\chi_E(\lambda) = \lambda^n - \lambda^{n-1}$, so $c_{n-1}=1$ and all other $c_j=0$. 

For our graph $E$, we have that $a_E=2$. So 
by Johnsson's Theorem, $\dim H_{n-3}(E)= a_E -1 = 1$. %As the coloring complex splits, we know that $\dim (H_{n-3}(G))= \sum_{j} \dim H_{n-3}^{(j)}(G)$. 

In part $(a)$, we found a generator for $\ker(d_{n-3})$ that is not a boundary, so in fact we found a generator of $H_{n-3}$. And as this generator is an element of  $\triangle_{n-3}^{(n-1)}(E)$, and the coloring splits, it is a generator of $H_{n-3}^{n-1}$, and so $\dim H_{n-3}^{n-1} \geq 1$.
So, we have that 
\[1 \leq \dim H_{n-3}^{n-1} \leq \sum_{j} \dim H_{n-3}^{(j)}(G) =  \dim (H_{n-3}(G)) = 1,\]
so $\dim H_{n-3}^{n-1} = 1$ and all other terms are $0$, giving us $\dim H_{n-3}^{(j)}=c_j.$

\tco{(problem statements say that $\triangle_r^{(j)}(G) = \tcg{e_r}^{(j)} \triangle_r (G)$, but i don't think that that makes sense, i think it should be $\triangle_r^{(j)}(G) = \tcg{e_{r+2}}^{(j)} \triangle_r (G)$)}

\end{enumerate}

\end{solution}

\begin{problem}
Let $T$ be a graph for which $(\ast)$ holds. Show that \[ \sum_{j=1}^{n-1} \sum_{r} (-1)^{n-3-r} \dim \left( \Delta_{r}^{(j)}(T) \right) \lambda^j = (-1)^n \left( \chi_T(-\lambda)- (-\lambda)^n \right) \]
\end{problem}

\begin{solution}
We start on the LHS and work out way to the RHS:

\begin{align*}
    \sum_{j=1}^{n-1} \sum_{r} (-1)^{n-3-r} \dim \left( \Delta_{r}^{(j)}(T) \right) \lambda^j &= \sum_{j=1}^{n-1} \sum_{r} (-1)^{n-3-r} \dim \left( \operatorname{H}_{r}^{(j)}(T) \right) \lambda^j \text{ using the Euler characteristic }\\
    &=\sum_r \sum_{j=1}^{n-1}(-1)^{n-3-r}\dim\left(\operatorname{H}_r^{(j)} (T)\right) \lambda^j\\
    &= \Gamma_T (\lambda)+ \sum_{r\ne n-3} (-1)^{n-3-r}\dim\left(\operatorname{H}_r (T) \right) (\lambda^1 + \cdots + \lambda^{n-1}) \text{ by (3) }\\
    &= \Gamma_T (\lambda) \text{ by Johnsson's theorem }\\ &= (-1)^n \left( \chi_T(-\lambda)- (-\lambda)^n \right) \text{ by $(\ast)$ }
\end{align*}
\end{solution}

\begin{problem}

\begin{enumerate}[label = (\alph*)]
    \item Apply problems $4$(d) and $4$(e) from HW\#2 to show that \[\dim\left(\Delta_r^{(j)}(G) \right) = \dim\left(\Delta_r^{(j)}(G\smallsetminus e) \right)-\dim\left( \Delta_r^{(j)}(G/e) \right) +\dim\left( \Delta_r^{(j)} (E_e) \right) \]
    where $E_e$ is the graph obtained from $G$ by deleting all edges except $e$. 
    
    \begin{solution}
    Problem $4$(d) says \[\Delta_r(G) \oplus \Delta_r(G/e)\cong \Delta_r(G-e)\oplus\Delta_r(E).\] Hence as problem $4$(e) says that this isomprphism is as $S_n$-modules, the complex splits via the ${\cdot}^{(j)}$ grading as well, we have \[\dim \left(\Delta_r^{(j)}(G) \right) + \dim\left( \Delta_r^{(j)}(G/e) \right) = \dim\left( \Delta_r^{(j)}(G\smallsetminus e)\right) +\dim\left(\Delta_r^{(j)}(E)\right),\] whereby the conclusion follows.
    
    
    
    
    
    
    \end{solution}
    
    \item Combine the Euler characteristic equation (with $T=G$) and induction to conclude that $(\ast)$ holds for $G$. 
    
    \begin{solution}
    The Euler characteristic equation says that we can replace the $\dim\left( \Delta_r(G)\right)$ with $\dim\left( \operatorname{H}_r(G)\right)$ (with or without the ${\cdot}^{(j)}$-grading) when we sum over $r$ and have the appropriate $(-1)$ factors, i.e., when we take the Euler characteristic before or after taking homology. Thus as $(\ast)$ holds for everything on the left by induction, we see it holds on the right, and so we compute:
    
    \begin{align*}
        \Gamma_G(\lambda) &= \sum_{j=1}^{n-1} \dim (H_{n-3}^{(j)}(G))\lambda^j\\
        &=\sum_{j=1}^{n-1} \left(\dim\left(\Delta_{n-3}^{(j)}(G\smallsetminus e) \right)-\dim\left(\Delta_{n-3}^{(j)}(G/e) \right) +\dim\left( \Delta_{n-3}^{(j)} (E_e) \right)\right)\lambda^j \text{ by Euler and (a) }\\
        &= \Gamma_{G\smallsetminus e} - \Gamma_{G/e}(\text{ but with an extra $-1$ caused by indexing differences } ) +\Gamma_{E_e} \text{ by induction }\\
        &= (-1)^n \left( \chi_{G\smallsetminus e}(-\lambda)- (-\lambda)^n\right) - \left((-1)^n \left( \chi_{G/e}(-\lambda)-(-\lambda)^{n-1} \right) \right) + (-1)^n \left(\chi_{E_e}(-\lambda)-(-\lambda)^n \right)\\
        &= (-1)^n \left(\chi_{G\smallsetminus e}(-\lambda) - \chi_{G/e}(-\lambda)- (-\lambda)^n + (-\lambda)^{n-1} +(-\lambda)^n - (-\lambda)^{n-1}- (-\lambda)^n \right) \text{ by (3) }\\
        &= (-1)^n \left(\chi_G (-\lambda) - (-\lambda)^n \right) \text{ by HW\#1}
    \end{align*}
    
    Thus $(\ast)$ holds for $G$, as desired.
       
    
    
    
    
    
    
    
    
    
    \end{solution}

\end{enumerate}
\end{problem}

\end{document}