\documentclass{article}
\usepackage{mathtools, esint, mathrsfs} % amsmath and integrals.
\usepackage{amsthm, amsfonts, amssymb}  % Fonts and theorems.
\usepackage{hyperref}                   % Hyperlinks.

% Colors for hyperref.
\hypersetup{
    colorlinks=true,
    linkcolor=blue,
    filecolor=magenta,
    urlcolor=Cerulean,
    citecolor=SkyBlue
}

\title{Algebraic Topology}
\author{Ryan Maguire}
\date{Fall 2021}
\setlength{\parindent}{0em}
\setlength{\parskip}{0em}

\newtheoremstyle{normal}
    {\topsep}               % Amount of space above the theorem.
    {\topsep}               % Amount of space below the theorem.
    {}                      % Font used for body of theorem.
    {}                      % Measure of space to indent.
    {\bfseries}             % Font of the header of the theorem.
    {}                      % Punctuation between head and body.
    {.5em}                  % Space after theorem head.
    {}

\theoremstyle{plain}
\newtheorem{theorem}{Theorem}[section]
\theoremstyle{normal}
\newtheorem{definition}{Definition}[section]
\newtheorem{example}{Example}[section]

% TODO
%   PL Structures
%   transversality
%   surgery
%   Morse theory
%   intersection theory
%   cobordism
%   bundles
%   characteristic classes
%   geometric topology
%   Smale, sphere inversion

\begin{document}
    \maketitle
    These notes come from my personal study of algebraic topology. Many of
    the concepts come from Allen Hatcher's Algebraic Topology.
    Any errors in these notes are my own.
    \tableofcontents
    \section{Homotopy}
        In point-set topology one talks about a
        \textit{homeomorphism} between topological spaces which is a
        continuous function $f:X\rightarrow{Y}$ that is bijective and such that
        the inverse function $f^{-1}:Y\rightarrow{X}$ is also continuous. Two
        topological spaces can be considered the same if there exists a
        homeomorphism, but often is it useful to have a weaker notion of
        \textit{sameness} between topological spaces. This is the notion of
        \textit{homotopy equivalence}. We need a few definitions before we
        can fully describe the idea.
        \begin{definition}
            \label{def:homotopy}%
            A homotopy from a topological space $(X,\tau_{X})$ to a topological
            space $(Y,\tau_{Y})$ is a continuous function
            $H:X\times[0,1]\rightarrow{Y}$ where $[0,1]$ is the closed unit
            interval, and with respect to the product topology
            on $X\times[0,1]$.
        \end{definition}
        A homotopy can be thought of as a continuous deformation of one
        function to another. We start with a continuous function
        $f_{0}:X\rightarrow{Y}$ and continuously deform it into a new function
        $f_{1}:X\rightarrow{Y}$. If the homotopy
        $H:X\times[0,1]\rightarrow{Y}$ is such that $H(x,0)=f_{0}(x)$ and
        $H(x,1)=f_{1}(x)$, then $H$ can be throught of as the function that
        continuously deforms $f_{0}$ into $f_{1}$.
        \begin{example}
            Let $f,g:\mathbb{R}\rightarrow\mathbb{R}$ be continuous functions
            and $H:\mathbb{R}\times[0,1]\rightarrow\mathbb{R}$ be the
            function:
            \begin{equation}
                H(x,t)=(1-t)f(x)+tg(x)
            \end{equation}
            $H$ is continuous since it is the product and sum of continuous
            functions, but moreover $H(x,0)=f(x)$ and $H(x,1)=g(x)$. That is,
            $H$ is a homotopy that drags $f$ to $g$. Visually, every point
            $f(x)$ gets dragged to $g(x)$ via the straight line connecting the
            two. Because of this, it is called the
            \textit{straight-line homotopy}.
        \end{example}
        \begin{definition}
            \label{def:homotopic}%
            Homotopic functions from a topological space $(X,\tau_{X})$ to a
            topological space $(Y,\tau_{Y})$ are continuous functions
            $f,g:X\rightarrow{Y}$ such that there exists a homotopy
            $H:X\times[0,1]\rightarrow{Y}$ where for all $x\in{X}$ it is
            true that $H(x,0)=f(x)$ and $H(x,1)=g(x)$.
        \end{definition}
        \begin{example}
            If $f,g:\mathbb{R}^{n}\rightarrow\mathbb{R}^{m}$ are continuous
            functions, and if
            $H:\mathbb{R}^{n}\times[0,1]\rightarrow\mathbb{R}^{m}$ is the
            function:
            \begin{equation}
                H(\mathbf{x},t)=(1-t)f(\mathbf{x})+tg(\mathbf{x})
            \end{equation}
            Then $H$ is a homotopy between $f$ and $g$. $H$ is continuous, and
            we have $H(\mathbf{x},0)=f(\mathbf{x})$ and
            $H(\mathbf{x},1)=g(\mathbf{x})$. That is, $f$ and $g$ are homotopic.
        \end{example}
        The previous example has to do with the fact that $\mathbb{R}^{m}$ is
        \textit{contractible}, a term we'll define later. This is a feature
        of topological vector spaces over $\mathbb{R}$.
        \begin{theorem}
            If $V$ is a topological vector space over $\mathbb{R}$, if
            $(X,\tau)$ is a topological space, and if
            $f,g:X\rightarrow{V}$ are continuous functions, then $f$ and $g$
            are homotopic.
        \end{theorem}
        \begin{proof}
            Let $H:X\times[0,1]\rightarrow{V}$ be the function:
            \begin{equation}
                H(x,t)=(1-t)\cdot{f}(x)+t\cdot{g}(x)
            \end{equation}
            Since $V$ is a topological vector space,
            $(1-t)\cdot{f}(x)$ is continuous, and similarly for $t\cdot{g}(x)$.
            Since addition is continuous in a topological vector space,
            $H(x,t)$ is continuous. Thus $H$ is a homotopy by definition
            (Def.~\ref{def:homotopy}). But $H(x,0)=f(x)$ and $H(x,1)=g(x)$.
            Hence, $f$ and $g$ are homotopic (Def.~\ref{def:homotopic}).
        \end{proof}
        If we wish to be as general as possible, we could write the following.
        \begin{theorem}
            If $K$ is a topological field with a path between $0_{K}$ and
            $1_{K}$, if $V$ is a topological vector space over $K$, if
            $(X,\tau)$ is a topological space, and if $f,g:X\rightarrow{V}$
            are continuous functions, then $f$ and $g$ are homotopic.
        \end{theorem}
        \begin{proof}
            By hypothesis there is a path $\alpha:[0,1]\rightarrow{K}$ such
            that $\alpha(0)=0_{K}$ and $\alpha(1)=1_{K}$. Let
            $H:X\times[0,1]\rightarrow{V}$ be the function:
            \begin{equation}
                H(x,t)=(1_{K}-\alpha(t))f(x)+\alpha(t)g(x)
            \end{equation}
            Since $K$ is a topological field and $\alpha$ is continuous,
            $1_{K}-\alpha$ is continuous. But since $V$ is a topological
            vector space, $(1_{K}-\alpha(t))f(x)$ is continuous. Similarly for
            $\alpha(t)g(x)$. But then $H$ is a continuous function, and is
            therefore a homotopy (Def.~\ref{def:homotopy}). But
            $H(x,0)=f(x)$ and $H(x,1)=g(x)$. Thus, $f$ and $g$ are homotopic
            (Def.~\ref{def:homotopic}).
        \end{proof}
        Application of the previous theorem shows that for any topological
        space $(X,\tau)$ and for any complex vector space $V$, all continuous
        functions $f,g:X\rightarrow{V}$ are homotopic. Because of this,
        vector spaces are in a sense \textit{boring} as far as homotopy is
        concerned.
        \par\hfill\par
        Returning to more examples of homotopies, let's consider again
        $\mathbb{R}^{n}$. Moreover, let's consider a homotopy between
        $\mathbb{R}^{n}$ and itself.
        \begin{example}
            Consider the function
            $H:\mathbb{R}^{n}\times[0,1]\rightarrow\mathbb{R}^{n}$ defined by:
            \begin{equation}
                H(\mathbf{x},t)=(1-t)\mathbf{x}
            \end{equation}
            At time $t=0$ we have the identity map since
            $H(\mathbf{x},0)=\mathbf{x}$ and at time $t=1$ we have the zero
            map since $H(\mathbf{x},1)=\mathbf{0}$. In a sense, this map
            \textit{contracts} the entire Euclidean space $\mathbb{R}^{n}$ down
            to a point. 
        \end{example}
        Homotopies from a topological space to itself give rise to several
        other notions, one of the more commonly discussed is that of
        \textit{deformation retractions}. First, a definition.
        \begin{definition}
            A retract of a topological space $(X,\tau)$ onto a subspace
            $(A,\tau|_{A})$, with $A\subseteq{X}$ and $\tau|_{A}$ the subspace
            topology, is a continuous function $f:X\rightarrow{X}$ such that
            $f[X]=A$ and for all $x\in{A}$ it is true that $f(x)=x$.
        \end{definition}
        That is, a retract is a continuous function that maps entirely onto
        a specific subspace while leaving that subspace fixed.
        \begin{example}
            If $(X,\tau)$ is a topological space and if $x_{0}\in{X}$, then
            the function $f:X\rightarrow\{x_{0}\}$ defined by
            $f(x)=x_{0}$ is a retract. Since constant functions are always
            continuous, and since $f(x_{0})=x_{0}$ by definition, we have that
            $f$ is a retract of $X$ to $\{x_{0}\}$.
        \end{example}
        This shows that any topology space can always retract to a single
        point, but can any topological space \textit{continuously deform}
        onto a single point? Intuitively the answer should be \textit{no},
        since if we were to continuously deform one point to another this would
        give us a \textit{path} between the two points and so
        path connectedness seems to be a requirement. To make intuition
        rigorous requires a definition.
        \begin{definition}
            A deformation retraction from a topological space $(X,\tau)$ to a
            subspace $(A,\tau|_{A})$ is a homotopy
            $H:X\times[0,1]\rightarrow{X}$ such that for all $x\in{X}$ we
            have that $H(x,0)=x$ and $H(x,1)$ is a retract of $X$ onto $A$.
        \end{definition}
        We are now in the position to define \textit{contractible}.
        \begin{definition}
            A contractible topological space is a topological space $(X,\tau)$
            such that there exists $x_{0}\in{X}$ such that there is a
            deformation retraction of $X$ onto $\{x_{0}\}$.
        \end{definition}
        With this, we can prove our intuition is correct. That for a space to
        \textit{continuously deform} down to a point, it must at least be
        path connected.
        \begin{theorem}
            If $(X,\tau)$ is contractible, then it is path connected.
        \end{theorem}
        \begin{proof}
            If $(X,\tau)$ is contractible, then there is a point $x_{0}\in{X}$
            and a deformation retraction $H:X\times[0,1]\rightarrow\{x_{0}\}$.
            Let $x,y\in{X}$ and defined $\alpha:[0,1]\rightarrow{X}$ by:
            \begin{equation}
                \alpha(t)=
                \begin{cases}
                    H(x,2t)&0\leq{t}\leq\frac{1}{2}\\
                    H(y,2-2t)&\frac{1}{2}\leq{t}\leq{1}
                \end{cases}
            \end{equation}
            $\alpha$ is continuous by the gluing lemma, but also
            $\alpha(0)=H(x,0)=x$ and $\alpha(1)=H(y,0)=y$, so $\alpha$ is a
            path from $x$ to $y$. Hence, $(X,\tau)$ is path connected.
        \end{proof}
        Recapping, every topological space has a \textit{retract} to a
        point, but to have a \textit{deformation retraction} to a point
        requires, at the very least, the space to be path connected. This is
        not sufficient. The circle $\mathbb{S}^{1}$ does not have a
        deformation retraction to a point. We can't continuously deform the
        circle to a point since there's a \textit{hole} in it that we would
        need to cut. Proving this in a rigorous setting is difficult and one
        of the historical aims of algebraic topology, and one of its successes.
\end{document}
