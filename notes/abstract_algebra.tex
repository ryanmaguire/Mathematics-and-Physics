%-----------------------------------LICENSE------------------------------------%
%   This file is part of Mathematics-and-Physics.                              %
%                                                                              %
%   Mathematics-and-Physics is free software: you can redistribute it and/or   %
%   modify it under the terms of the GNU General Public License as             %
%   published by the Free Software Foundation, either version 3 of the         %
%   License, or (at your option) any later version.                            %
%                                                                              %
%   Mathematics-and-Physics is distributed in the hope that it will be useful, %
%   but WITHOUT ANY WARRANTY; without even the implied warranty of             %
%   MERCHANTABILITY or FITNESS FOR A PARTICULAR PURPOSE.  See the              %
%   GNU General Public License for more details.                               %
%                                                                              %
%   You should have received a copy of the GNU General Public License along    %
%   with Mathematics-and-Physics.  If not, see <https://www.gnu.org/licenses/>.%
%------------------------------------------------------------------------------%

\documentclass{article}
\usepackage{mathtools, esint, mathrsfs} % amsmath and integrals.
\usepackage{amsthm, amsfonts, amssymb}  % Fonts and theorems.
\usepackage{hyperref}                   % Hyperlinks.

% Colors for hyperref.
\hypersetup{
    colorlinks=true,
    linkcolor=blue,
    filecolor=magenta,
    urlcolor=Cerulean,
    citecolor=SkyBlue
}

\title{Abstract Algebra}
\author{Ryan Maguire}
\date{\today}
\setlength{\parindent}{0em}
\setlength{\parskip}{0em}

\newtheoremstyle{normal}
    {\topsep}               % Amount of space above the theorem.
    {\topsep}               % Amount of space below the theorem.
    {}                      % Font used for body of theorem.
    {}                      % Measure of space to indent.
    {\bfseries}             % Font of the header of the theorem.
    {}                      % Punctuation between head and body.
    {.5em}                  % Space after theorem head.
    {}

\theoremstyle{plain}
\newtheorem{theorem}{Theorem}[section]
\theoremstyle{normal}
\newtheorem{definition}{Definition}[section]
\newtheorem{example}{Example}[section]
\newtheorem{notation}{Notation}[section]

\begin{document}
    \maketitle
    These notes come from my personal study of algebra. Many of
    the concepts come from Dummit and Foote's Abstract Algebra.
    Any errors in these notes are my own.
    \tableofcontents
    \section{Preliminaries}
        The basic ideas of set theory will not be discussed in detail, but
        should be familiar. These concepts are as follows.
        \begin{definition}
            \label{def:set}%
            A \textit{set} $A$ is a collection of objects called the
            \textit{elements} of the set. If $x$ is an element of $A$, this is
            denoted $x\in{A}$. If not, this is written as $x\notin{A}$.
        \end{definition}
        There are two major operations we can do with sets.
        \begin{definition}
            \label{def:union}%
            The \textit{union} of two sets $A$ and $B$ is the set $A\cup{B}$
            defined by:
            \begin{equation}
                A\cup{B}=\{\,x\;|\;x\in{A}\textrm{ or }x\in{B}\,\}
            \end{equation}
        \end{definition}
        The word \textit{or} is the inclusive or. So $x\in{A}\cup{B}$ means
        either $x\in{A}$ or $x\in{B}$ or both.
        For example, if $A=\{1,2,3\}$ and $B=\{3,4,5\}$, then
        $A\cup{B}=\{1,2,3,4,5\}$. Sets do not consider the possibility of
        repetition of elements so the sets $\{1,1\}$ and $\{1\}$ are identical.
        \begin{definition}
            \label{def:intersection}%
            The \textit{intersection} of two sets $A$ and $B$ is the set
            $A\cap{B}$ defined by:
            \begin{equation}
                A\cap{B}=\{\,x\;|\;x\in{A}\textrm{ and }x\in{B}\,\}
            \end{equation}
        \end{definition}
        If $A=\{1,2,3\}$ and $B=\{3,4,5\}$, then $A\cap{B}=\{3\}$.
        \begin{definition}
            \label{def:cardinality}%
            The \textit{cardinality} of a set $A$ is the size of the set,
            denoted $\textrm{Card}(A)$.
        \end{definition}
        A rigorous treatement of cardinality requires a discussion of cardinal
        numbers, but that is beyond our topics. For finite sets, the
        cardinality is simply the number of distinct elements in the set.
        So if $A=\{1,2,3\}$ and $B=\{a,b,c,d\}$, then
        $\textrm{Card}(A)=3$ and $\textrm{Card}(B)=4$.
        \par\hfill\par
        The axioms of Zermelo-Fraenkel Set Theory tell us that if $A$ and $B$
        are sets, there is a set $A\times{B}$ defined by the set of all
        \textit{ordered pairs}. This is the called the
        \textit{Cartesian product} of $A$ and $B$.
        \begin{definition}
            \label{def:cartesian_product}%
            The \textit{Cartesian product} of sets $A$ and $B$ is the set
            $A\times{B}$ defined by:
            \begin{equation}
                A\times{B}=\{\,(a,b)\;|\;a\in{A},\,b\in{B}\,\}
            \end{equation}
        \end{definition}
        If $A=\{1,2\}$ and $B=\{a,b\}$, then $A\times{B}$ is the set
        $\{(1,a),(1,b),(2,a),(2,b)\}$.
        There's some (near) universally agreed upon notation in mathematics for
        certain sets.
        \begin{notation}
            \label{not:natural_numbers}%
            The set of all natural number 0, 1, 2, and so on is denoted
            $\mathbb{N}$. That is:
            \begin{equation}
                \mathbb{N}=\{\,0,\,1,\,2,\,3,\,4,\,\dots\,\}
            \end{equation}
        \end{notation}
        \begin{notation}
            \label{not:integers}%
            The set of all integers (positive, negative, and zero) is denoted
            $\mathbb{Z}$. That is:
            \begin{equation}
                \mathbb{Z}=\{\,\dots,\,-2,\,-1,\,0,\,1,\,2,\,\dots\,\}
            \end{equation}
        \end{notation}
        Thet letter Z isn't random. It comes from the German word for number:
        \textit{Zahlen}.
        \begin{notation}
            \label{not:rationals}%
            The set of all rational numbers, ratios of integers, is denoted
            $\mathbb{Q}$. That is:
            \begin{equation}
                \mathbb{Q}=\{\,\frac{p}{q}\;|\;p,q\in\mathbb{Z}\textrm{ and }
                    q\ne{0}\,\}
            \end{equation}
        \end{notation}
        I'd like to believe Q stands for \textit{quotient}, but I've never
        directly looked into this.
        \begin{notation}
            \label{not:reals}%
            The set of all real numbers is denoted $\mathbb{R}$:
            \begin{equation}
                \mathbb{R}=\{\,r\;|\;r\textrm{ is a real number}\,\}
            \end{equation}
        \end{notation}
        The definition of $\mathbb{R}$ seems circular. It is very
        difficult to precisely state what a real number is. Alas, this is a
        problem for set theorists and real analysis.
        \begin{notation}
            The set of \textit{complex number} of the form $z=x+iy$, where
            $x,y\in\mathbb{R}$ and $i$ is the \textit{imaginary unit}, is
            denoted $\mathbb{C}$. That is:
            \begin{equation}
                \mathbb{C}=\{\,x+iy\;|\;x,y\in\mathbb{R}\,\}
            \end{equation}
        \end{notation}
        Topologically speaking, $\mathbb{C}$ is the same as the cartesian
        product $\mathbb{R}\times\mathbb{R}=\mathbb{R}^{2}$. It is a
        \textit{plane}. Algebraically it has more structure, and is one of the
        more fundamental objects in abstract algebra.
        \begin{definition}
            A \textit{function} from a set $A$ to a set $B$, denoted
            $f:A\rightarrow{B}$, is a subset
            $f\subseteq{A}\times{B}$ such that for all $a\in{A}$ there is a
            unique $b\in{B}$ such that $(a,b)\in{f}$.
        \end{definition}
        Intuitively, a function is a \textit{rule} that assigns to each element
        in $A$ a unique element in $B$. This is usually given by some formula.
        If $a\in{A}$ corresponds to $b\in{B}$, we often use the notation
        $b=f(a)$. The words \textit{map}, \textit{operator},
        \textit{transformation}, etc., are all synomymns for functions that are
        usually restricted to particular areas of mathematics. I'm of the
        opinion that this creates an unnecessary confusion on the reader,
        having to memorize where which synomymn applies, and prefer to refer
        to functions as just that: \textit{functions}. I'm the minority in this
        belief, and the reader should acquaint themselves with these terms if
        they wish to discuss mathematics with others. For this document, the
        word function will suffice.
        \begin{example}
            Defining functions by formulas can be a tricky thing. We have to
            check if the formula is \textit{well-defined}. For example, suppose
            we have the following formula which takes in a rational number
            $a/b\in\mathbb{Q}$ and returns an integer in $\mathbb{Z}$:
            \begin{equation}
                f\big(\frac{a}{b}\big)=a
            \end{equation}
            This looks like its a function, I used the letter $f$, gave it
            some parenthesis, and a nice equation for it, but it's not.
            If I give the same rational number, but in a different form, it
            yields a different answer. For example,
            $\frac{1}{2}=\frac{2}{4}$, however $f(\frac{1}{2})=1$ and
            $f(\frac{2}{4})=2$. Two different values for the same input.
            Hence, $f$ is not a function.
        \end{example}
        \begin{example}
            Suppose we have the following formula that takes in a rational
            number and returns a rational:
            \begin{equation}
                f\big(\frac{a}{b}\big)=\frac{a^{2}}{b^{2}}
            \end{equation}
            Is this a function? It does indeed take in any rational number,
            and does return a rational number. The question is whether or not
            the function is well-defined. If two rational numbers $a/b$ and
            $c/d$ are equal, then $c/d = ka/kb$ for some non-zero integer $k$.
            But then:
            \begin{equation}
                f\big(\frac{c}{d}\big)=f\big(\frac{ka}{kb}\big)
                    =\frac{(ka)^{2}}{(kb)^{2}}
                    =\frac{a^{2}}{b^{2}}
                    =f\big(\frac{a}{b}\big)
            \end{equation}
            So the formula is well-defined, and defines a function
            $f:\mathbb{Q}\rightarrow\mathbb{Q}$.
        \end{example}
        \begin{example}
            A more challenging example from Dummit and Foote's book. Suppose
            we have the following formula that takes in a real number and
            returns a positive integer between 0 and 9. We write the real
            number $x$ in decimal and return the first digit to the right of
            the decimal point. So for $\pi=3.14159...$ we return 1. For
            $\sqrt{2}=1.414...$ we return 4. Is this a well-defined function?
            The answer is no since the same number can have two different
            decimal expansions. For example, $1=1.0$. However,
            $1=0.999...$ This second equality usually irks undergraduates, but
            it is true. The is no \textit{last decimal place} where the
            number differs. The real numbers do not have such a notion. There
            are number systems that do such as the hyper-reals, but these
            systems are fundamentally different than the real numbers. Here's
            a short algebraic proof that $1.0=0.999...$ First label
            $0.999...=x$. Then:
            \begin{align}
                10x&=10\cdot0.999...\\
                    &=9.999...\\
                    &=9+0.999...\\
                    &=9+x
            \end{align}
            So we have $10x=9+x$, which implies $9x=9$, and therefore $x=1$.
            Because $1.0=0.999...$, the formula $f$ is ambiguous for $1$.
            Should $f(1)=0$ or should $f(1)=9$? $f$ does not define a function.
        \end{example}
        The sets $A$ and $B$ are called the \textit{domain} and
        \textit{codomain}, respectively. More synomymns, in topology the
        codomain is often called the \textit{target}. Again, I will stick with
        codomain and avoid synomymns. The codomain should not be confused with
        the \textit{range} of a function. The range is a subset of the codomain.
        Namely, it is the set of all elements $b\in{B}$ that have a
        corresponding element $a\in{A}$ with $f(a)=b$.
        \begin{example}
            Letting $f:\mathbb{R}\rightarrow\mathbb{R}$ be defined by
            $f(x)=x^{2}$, the range of $f$ is the set of all
            \textit{non-negative} real numbers. There is no real number that
            squares to $-1$, and hence $f(x)=x^{2}=-1$ has no real solution.
            So $-1$ is \textit{not} in the range. It is in the codomain,
            however, by the definition of $f$
            (that is, $f:\mathbb{R}\rightarrow\mathbb{R}$ is defined as a
            function from $\mathbb{R}$ to $\mathbb{R}$).
        \end{example}
        \begin{definition}
            The image of a subset $\mathcal{U}\subseteq{A}$ of a function
            $f:A\rightarrow{B}$ between two sets $A$ and $B$ is the set
            $f[\mathcal{U}]$ defined by:
            \begin{equation}
                f[\mathcal{U}]=\{\,b\in{B}\;|\;
                    b=f(a)\,\textrm{for some }a\in\mathcal{U}\,\}
            \end{equation}
        \end{definition}
        The range of a function is precisely the image of the domain. That is,
        if $f:A\rightarrow{B}$ is a function, the range is $f[A]$.
        \begin{definition}
            The pre-image of a subset $\mathcal{V}\subseteq{B}$ of a function
            $f:A\rightarrow{B}$ between two sets $A$ and $B$ is the function
            $f^{-1}[\mathcal{V}]$ defined by:
            \begin{equation}
                f^{-1}[\mathcal{V}]=\{\,a\in{A}\;|\;
                    f(a)\in\mathcal{V}\,\}
            \end{equation}
        \end{definition}
        The notations $f[\mathcal{U}]$ and $f^{-1}[\mathcal{V}]$ are not
        universal. Many texts write $f(\mathcal{U})$ and
        $f^{-1}(\mathcal{V})$. This creates ambiguity as to whether or not I
        mean the inverse function $f^{-1}$ or the pre-image. A similar
        ambiguity exists for $f(\mathcal{U})$. Indeed, suppose
        $C=\{1,2\}$ and $A=\{1,2,\{1,2\}\}$. Then both $C\in{A}$ and
        $C\subseteq{A}$ (strange, I know). Given any other set $B$ and any
        function $f:A\rightarrow{B}$, what should $f(C)$ mean? The value of
        $f$ at the element $C\in{A}$, or the image of the subset
        $C\subseteq{A}$? I once stumbled across the notation
        $f[C]$ for subsets in some text whose name I've forgotten and thought
        it was genius. I have adopted it since.
        \begin{definition}
            The fiber of an element $b\in{B}$ under a function
            $f:A\rightarrow{B}$ between two sets $A$ and $B$ is the pre-image
            $f^{-1}[\{b\}]$.
        \end{definition}
        The fiber of an element $b\in{B}$ can be empty, and it can also have
        many elements. For example, $f:\mathbb{R}\rightarrow\mathbb{R}$ defined
        by $f(x)=x^2$, the fiber of $-1$ is empty since $f(x)=-1$ has no
        solution. On the other hand, the fiber of $1$ is the set $\{1,-1\}$
        since $1^{2}=1$ and $(-1)^{2}=1$.
        \begin{definition}
            The composition of two functions $f:A\rightarrow{B}$ and
            $g:B\rightarrow{C}$ is the function $g\circ{f}:A\rightarrow{C}$
            defined by:
            \begin{equation}
                (g\circ{f})(x)=g\big(f(x)\big)
            \end{equation}
        \end{definition}
        Note that the domain of $g$ needs to be the codomain of $f$ for this
        to be well-defined. The identity function on a set $A$ is the function
        $\textrm{id}_{A}:A\rightarrow{A}$ such that $\textrm{id}_{A}(a)=a$ for
        all $a\in{A}$.
        \begin{definition}
            The inverse of a function $f:A\rightarrow{B}$ from a set $A$ to
            a set $B$ is a function $g:B\rightarrow{A}$ such that
            $f\circ{g}=\textrm{id}_{B}$ and
            $g\circ{f}=\textrm{id}_{A}$.
        \end{definition}
        Weaker notions exist, such as left and right inverses.
        \begin{definition}
            A left inverse of a function $f:A\rightarrow{B}$ from a set $A$ to
            a set $B$ is a function $g:B\rightarrow{A}$ such that
            $g\circ{f}=\textrm{id}_{A}$.
        \end{definition}
        \begin{definition}
            An injective function from a set $A$ to a set $B$ is a function
            $f:A\rightarrow{B}$ such that for all $a_{0},a_{1}\in{A}$ with
            $f(a_{0})=f(a_{1})$, it is true that $a_{0}=a_{1}$.
        \end{definition}
        \begin{example}
            The function $f:\mathbb{R}\rightarrow\mathbb{R}$ defined by
            $f(x)=x^{3}$ is injective. If $a^{3}=b^{3}$, then $a=b$. The
            function $g:\mathbb{R}\rightarrow\mathbb{R}$ defined by
            $g(x)=x^{2}$ is not injective. $(-1)^{2}=1^{2}$, but
            $-1\ne{1}$.
        \end{example}
        \begin{theorem}
            If $A$ and $B$ are non-empty sets, and if $f:A\rightarrow{B}$ is a
            function, then $f$ has a left inverse $g:B\rightarrow{A}$ if and
            only if $f$ is injective.
        \end{theorem}
        \begin{proof}
            If $f$ is injective, then for all $a\in{A}$ there is a unique
            $b\in{B}$ such that $f(a)=b$. Since $A$ is non-empty, there exists
            an element $x\in{A}$. Define $g:B\rightarrow{A}$ as follows:
            \begin{equation}
                g(b)=
                \begin{cases}
                    a&b\in{f[A]}\textrm{ and }f(a)=b\\
                    x&\textrm{otherwise}
                \end{cases}
            \end{equation}
            Since $f$ is injective, $g$ is well-defined. Moreover, for all
            $a\in{A}$, $(g\circ{f})(a)=g(f(a))=g(b)=a$, so
            $g\circ{f}=\textrm{id}_{A}$. Next, if $f:A\rightarrow{A}$ has a
            a left-inverse $g:B\rightarrow{A}$, then suppose $a_{1},a_{2}\in{A}$
            and $f(a_{1})=f(a_{2})$. Then
            $g(f(a_{1}))=g(f(a_{2}))$. But $g$ is a left inverse so
            $a_{1}=g(f(a_{1}))=g(f(a_{2}))=a_{2}$ and therefore $a_{1}=a_{2}$.
            That is, $f$ is injective.
        \end{proof}
        \begin{definition}
            A right inverse of a function $f:A\rightarrow{B}$ from a set $A$ to
            a set $B$ is a function $g:B\rightarrow{A}$ such that
            $f\circ{g}=\textrm{id}_{B}$.
        \end{definition}
        \begin{definition}
            A surjective function from a set $A$ to a set $B$ is a function
            $f:A\rightarrow{B}$ such that for all $b\in{B}$ there is an
            $a\in{A}$ such that $f(a)=b$.
        \end{definition}
        \begin{theorem}
            If $A$ and $B$ are sets, and if $f:A\rightarrow{B}$ is a function,
            then $f$ has a right inverse $g:A\rightarrow{B}$ if and only if
            $f$ is surjective.
        \end{theorem}
        \begin{proof}
            If $f:A\rightarrow{B}$ is surjective, then for all $b\in{B}$ there
            is an $a\in{A}$ such that $f(a)=b$. Given a $b\in{A}$, label such
            a value $a\in{A}$ as $a_{b}$. Define $g$ to be:
            \begin{equation}
                g(b)=a_{b}
            \end{equation}
            Then $(f\circ{g})(b)=f(g(b))=f(a_{b})=b$ and thus $g$ is a right
            inverse. Next, if $f$ has a right inverse, let $g:B\rightarrow{A}$
            be such a right inverse, and let $b\in{B}$. Let $a=g(b)$. Since
            $g$ is a right inverse, $f(a)=f(g(b))=(f\circ{g})(b)=b$. That is,
            $f$ is surjective.
        \end{proof}
        \begin{definition}
            A bijective function from a set $A$ to a set $B$ is a function
            $f:A\rightarrow{B}$ such that $f$ is injective and surjective.
        \end{definition}
        \begin{theorem}
            If $A$ and $B$ are sets, if $f:A\rightarrow{B}$ is a function,
            if $g_{R}:B\rightarrow{A}$ is a right inverse of $f$, and if
            $g_{L}:B\rightarrow{A}$ is a left inverse of $f$, then
            $g_{R}=g_{L}$.
        \end{theorem}
        \begin{proof}
            If $f$ has a left inverse, then $f$ is injective. If $f$ has a
            right inverse, then $f$ is surjective. Therefore $f$ is bijective.
            If $b\in{B}$, there is an $a\in{A}$ such that $f(a)=b$. But then
            $g_{L}(b)=g_{L}(f(a))=(g_{L}\circ{f})(a)$. But since
            $g_{L}$ is a left inverse, $(g_{L}\circ{f})(a)=a$. Thus,
            $g_{L}(b)=a$. But since $g_{R}$ is a right inverse,
            $(f\circ{g}_{R})(b)=b$ and therefore $f(g_{R}(b))=b$. But
            $f(a)=b$ and $f$ is injective. Therefore $g_{R}(b)=a$. But
            $g_{L}(b)=a$ and $b\in{B}$ is arbitrary. Therefore, $g_{L}=g_{R}$.
        \end{proof}
        \begin{theorem}
            If $A$ and $B$ are sets, and if $f:A\rightarrow{B}$ is a function,
            then $f$ is invertible if and only if it is bijective.
        \end{theorem}
        \begin{proof}
            If $f$ is invertible, then it has a left inverse and a right
            inverse. By the previous theorem, $f$ is therefore injective and
            surjective, and is therefore bijective. Next, if $f$ is bijective,
            then it is injective and surjective. By the previous theorems, $f$
            has a left inverse and a right inverse. But if $f$ has a right
            inverse $g_{R}$ and a left inverse $g_{L}$, then $g_{R}=g_{L}$.
            Thus, $f$ is invertible with inverse $g=g_{L}=g_{R}$.
        \end{proof}
        \begin{definition}
            A permutation on a set $A$ is a bijective function
            $f:A\rightarrow{A}$.
        \end{definition}
\end{document}
