\documentclass{article}
\usepackage{mathtools, esint, mathrsfs} % amsmath and integrals.
\usepackage{amsthm, amsfonts, amssymb}  % Fonts and theorems.
\usepackage{hyperref}                   % Hyperlinks.

% Colors for hyperref.
\hypersetup{
    colorlinks=true,
    linkcolor=blue,
    filecolor=magenta,
    urlcolor=Cerulean,
    citecolor=SkyBlue
}

\title{Abstract Algebra}
\author{Ryan Maguire}
\date{\today}
\setlength{\parindent}{0em}
\setlength{\parskip}{0em}

\newtheoremstyle{normal}
    {\topsep}               % Amount of space above the theorem.
    {\topsep}               % Amount of space below the theorem.
    {}                      % Font used for body of theorem.
    {}                      % Measure of space to indent.
    {\bfseries}             % Font of the header of the theorem.
    {}                      % Punctuation between head and body.
    {.5em}                  % Space after theorem head.
    {}

\theoremstyle{plain}
\newtheorem{theorem}{Theorem}[section]
\theoremstyle{normal}
\newtheorem{definition}{Definition}[section]
\newtheorem{example}{Example}[section]
\newtheorem{notation}{Notation}[section]

% TODO
%   PL Structures
%   transversality
%   surgery
%   Morse theory
%   intersection theory
%   cobordism
%   bundles
%   characteristic classes
%   geometric topology
%   Smale, sphere inversion

\begin{document}
    \maketitle
    These notes come from my personal study of algebra. Many of
    the concepts come from Dummit and Foote's Abstract Algebra.
    Any errors in these notes are my own.
    \tableofcontents
    \section{Preliminaries}
        The basic ideas of set theory will not be discussed in detail, but
        should be familiar. These concepts are as follows.
        \begin{definition}
            \label{def:set}%
            A \textit{set} $A$ is a collection of objects called the
            \textit{elements} of the set. If $x$ is an element of $A$, this is
            denoted $x\in{A}$. If not, this is written as $x\notin{A}$.
        \end{definition}
        There are two major operations we can do with sets.
        \begin{definition}
            \label{def:union}%
            The \textit{union} of two sets $A$ and $B$ is the set $A\cup{B}$
            defined by:
            \begin{equation}
                A\cup{B}=\{\,x\;|\;x\in{A}\textrm{ or }x\in{B}\,\}
            \end{equation}
        \end{definition}
        The word \textit{or} is the inclusive or. So $x\in{A}\cup{B}$ means
        either $x\in{A}$ or $x\in{B}$ or both.
        For example, if $A=\{1,2,3\}$ and $B=\{3,4,5\}$, then
        $A\cup{B}=\{1,2,3,4,5\}$. Sets do not consider the possibility of
        repetition of elements so the sets $\{1,1\}$ and $\{1\}$ are identical.
        \begin{definition}
            \label{def:intersection}%
            The \textit{intersection} of two sets $A$ and $B$ is the set
            $A\cap{B}$ defined by:
            \begin{equation}
                A\cap{B}=\{\,x\;|\;x\in{A}\textrm{ and }x\in{B}\,\}
            \end{equation}
        \end{definition}
        If $A=\{1,2,3\}$ and $B=\{3,4,5\}$, then $A\cap{B}=\{3\}$.
        \begin{definition}
            \label{def:cardinality}%
            The \textit{cardinality} of a set $A$ is the size of the set,
            denoted $\textrm{Card}(A)$.
        \end{definition}
        A rigorous treatement of cardinality requires a discussion of cardinal
        numbers, but that is beyond our topics. For finite sets, the
        cardinality is simply the number of distinct elements in the set.
        So if $A=\{1,2,3\}$ and $B=\{a,b,c,d\}$, then
        $\textrm{Card}(A)=3$ and $\textrm{Card}(B)=4$.
        \par\hfill\par
        The axioms of Zermelo-Fraenkel Set Theory tell us that if $A$ and $B$
        are sets, there is a set $A\times{B}$ defined by the set of all
        \textit{ordered pairs}. This is the called the
        \textit{Cartesian product} of $A$ and $B$.
        \begin{definition}
            \label{def:cartesian_product}%
            The \textit{Cartesian product} of sets $A$ and $B$ is the set
            $A\times{B}$ defined by:
            \begin{equation}
                A\times{B}=\{\,(a,b)\;|\;a\in{A},\,b\in{B}\,\}
            \end{equation}
        \end{definition}
        If $A=\{1,2\}$ and $B=\{a,b\}$, then $A\times{B}$ is the set
        $\{(1,a),(1,b),(2,a),(2,b)\}$.
        There's some (near) universally agreed upon notation in mathematics for
        certain sets.
        \begin{notation}
            \label{not:natural_numbers}%
            The set of all natural number 0, 1, 2, and so on is denoted
            $\mathbb{N}$. That is:
            \begin{equation}
                \mathbb{N}=\{\,0,\,1,\,2,\,3,\,4,\,\dots\,\}
            \end{equation}
        \end{notation}
        \begin{notation}
            \label{not:integers}%
            The set of all integers (positive, negative, and zero) is denoted
            $\mathbb{Z}$. That is:
            \begin{equation}
                \mathbb{Z}=\{\,\dots,\,-2,\,-1,\,0,\,1,\,2,\,\dots\,\}
            \end{equation}
        \end{notation}
        Thet letter Z isn't random. It comes from the German word for number:
        \textit{Zahlen}.
        \begin{notation}
            \label{not:rationals}%
            The set of all rational numbers, ratios of integers, is denoted
            $\mathbb{Q}$. That is:
            \begin{equation}
                \mathbb{Q}=\{\,\frac{p}{q}\;|\;p,q\in\mathbb{Z}\textrm{ and }
                    q\ne{0}\,\}
            \end{equation}
        \end{notation}
        I'd like to believe Q stands for \textit{quotient}, but I've never
        directly looked into this.
        \begin{notation}
            \label{not:reals}%
            The set of all real numbers is denoted $\mathbb{R}$:
            \begin{equation}
                \mathbb{R}=\{\,r\;|\;r\textrm{ is a real number}\,\}
            \end{equation}
        \end{notation}
        The definition of $\mathbb{R}$ seems circular. It is very
        difficult to precisely state what a real number is. Alas, this is a
        problem for set theorists and real analysis.
        \begin{notation}
            The set of \textit{complex number} of the form $z=x+iy$, where
            $x,y\in\mathbb{R}$ and $i$ is the \textit{imaginary unit}, is
            denoted $\mathbb{C}$. That is:
            \begin{equation}
                \mathbb{C}=\{\,x+iy\;|\;x,y\in\mathbb{R}\,\}
            \end{equation}
        \end{notation}
        Topologically speaking, $\mathbb{C}$ is the same as the cartesian
        product $\mathbb{R}\times\mathbb{R}=\mathbb{R}^{2}$. It is a
        \textit{plane}. Algebraically it has more structure, and is one of the
        more fundamental objects in abstract algebra.
\end{document}
