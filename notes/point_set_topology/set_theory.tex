%-----------------------------------LICENSE------------------------------------%
%   This file is part of Mathematics-and-Physics.                              %
%                                                                              %
%   Mathematics-and-Physics is free software: you can redistribute it and/or   %
%   modify it under the terms of the GNU General Public License as             %
%   published by the Free Software Foundation, either version 3 of the         %
%   License, or (at your option) any later version.                            %
%                                                                              %
%   Mathematics-and-Physics is distributed in the hope that it will be useful, %
%   but WITHOUT ANY WARRANTY; without even the implied warranty of             %
%   MERCHANTABILITY or FITNESS FOR A PARTICULAR PURPOSE.  See the              %
%   GNU General Public License for more details.                               %
%                                                                              %
%   You should have received a copy of the GNU General Public License along    %
%   with Mathematics-and-Physics.  If not, see <https://www.gnu.org/licenses/>.%
%------------------------------------------------------------------------------%
\chapter{Set Theory}
    We now begin our work-in-earnest on point-set topology. Topology is often
    a students first or second course in mathematics that requires mathematical
    foundations for proof. To accomplish this requires set theory. This is not
    a lengthy treatise of mathematical logic, but rather will explore some of
    the central concepts, provide definitions, examples, and important theorems.
    \section{Sets and Logic}
        The main objects in mathematics are \textit{sets}. This development came
        about in the 1800's with figures like Georg Cantor\index{Cantor, Georg},
        Augustus De Morgan\index{De Morgan, Augustus}, and Bernard
        Bolzano\index{Bolzano, Bernard} making the first strides in the theory.
        The early history is intuitive, but vague. This led to Russell's
        paradox\index{Paradox!Russell's Paradox}\index{Russell's Paradox} which
        ultimately showed stricter \textit{axioms} of set theory were needed.
        Let's look to the primary source for a definition, posited by Georg
        Cantor (1845-1918 C.E.). He wrote \cite{Cantor1895}:
        \begin{center}
            \textit{A set is a gathering together into a whole of definite}
            \textit{distinct objects of our perception or of our thought, which}
            \textit{are called the elements of the set.}
            \par\hfill
            \textit{Beitr\"{a}ge zur Begr\"{u}ndung der Transfiniten}
            \textit{Mengenlehre}%
            \footnote{%
                English:
                \textit{Contributions in Support of Transfinite Set Theory}.%
            }
            \par\hfill
            \textit{Georg Cantor, 1985 C.E.}
        \end{center}
        Felix Hausdorff\index{Hausdorff, Felix} (1868-1942 C.E.) posits
        \cite[p.~11]{HausdorffSetTheory}:
        \begin{center}
            \textit{A set is formed by the grouping together of single objects}
            \textit{into a whole. A set is a plurality thought of as a unit.}
            \par\hfill
            \textit{Mengenlehre}%
            \footnote{%
                English:
                \textit{Set Theory}.%
            }
            \par\hfill
            \textit{Felix Hausdorff, 1927 C.E.}
        \end{center}
        Both are beautifully phrased, but circular since the terms
        \textit{gathering}, \textit{objects}, and \textit{grouping} are not
        defined. This form of circularity was addressed by Alfred
        Tarski\index{Tarski, Alfred} (1901 - 1983 C.E.) in his 1946 book
        \textit{Introduction to Logic and the Methodology of the Deductive}
        \textit{Science}. He expresses the need for primitive
        undefined notions that we take for granted and use freely. For us, the
        word \textit{set} is a primitive.
        \begin{definition}[\textbf{Set}]
            A \gls{set} is a collection of objects called elements.
        \end{definition}
        I have not defined \textit{collection} nor \textit{object}. To
        understand what a set is requires examples.
        \par\hfill\par\hrule
        \begin{example}
            If a set has a finite number of
            elements it is common notation to list the elements, separated by
            commas, and enclosed in curly braces $\{\}$. It is also common to
            denote sets by capital Latin letters like $A$, $B$, and $C$, and
            the elements by lower case letters like $a$, $b$, and $c$. Let us
            say that $A$ is the set of the numbers 0, 1, and 2.
            This may be written:
            \begin{equation}
                A=\{\,1,\,2,\,3\,\}
            \end{equation}
            The order is immaterial. We may also write:
            \begin{equation}
                A=\{\,2,\,1,\,3\,\}
            \end{equation}
            The number of times a single element occurs is also irrelevant. We
            also have:
            \begin{equation}
                A=\{\,1,\,1,\,2,\,3\,\}
            \end{equation}
            All of these are equivalent representations for the set $A$.
        \end{example}
        \hrule
        \begin{example}
            Sets are not restricted to numbers. We can have sets of almost
            anything. For example, we can consider the first few letters of
            the alphabet:
            \begin{equation}
                B=\{\,a,\,b,\,c\,\}
            \end{equation}
            So long as the symbols $a$, $b$, and $c$ have been defined, $B$ is
            a valid set. Repetition and order have no meaning here. We can write
            the following:
            \begin{equation}
                B=\{\,a,\,c,\,b\,\}=\{\,a,\,a,\,b,\,c\,\}
            \end{equation}
            These are also equivalent representations of the set $B$.
        \end{example}
        \hrule\par\hfill\par
        What defines a set are the elements it contains. The following is the
        notation for containment.
        \begin{notation}[\textbf{Containment}]
            If $A$ is a set, and $a$ is an element of $A$ we write
            $a\in{A}$. If $a$ is not an element of $a$ we write $a\notin{A}$.
        \end{notation}
        The expression $a\in{A}$ reads aloud as $a$ \textit{is in} $A$ or
        $a$ \textit{is contained in} $A$ or $a$ \textit{is an element of} $A$.
        Similarly, $a\notin{A}$ reads $a$ \textit{is not in} $A$. Containment is
        another one of our primitives that is best described with example.
        \par\hfill\par\hrule
        \begin{example}
            Let $A=\{\,1,\,2,\,3\,\}$. Then we have $1\in{A}$ since 1 is an
            element of the set $A$. Similarly $2\in{A}$ and $3\in{A}$. However,
            $4\notin{A}$ since 4 is not contained in $A$. Similarly
            $5\notin{A}$ and $\pi\notin{A}$.
        \end{example}
        \hrule
        \begin{example}
            Using our second example of letters, if $B=\{\,a,\,b,\,c\,\}$, then
            $a\in{B}$, $b\in{B}$, and $c\in{B}$, but $d\notin{B}$. Similarly
            $1\notin{B}$ and $7\notin{B}$.
        \end{example}
        \hrule\par\hfill\par
        There is a special set called the \textit{empty set}, denoted
        $\emptyset$, with the property that for all $x$ it is true that
        $x\notin\emptyset$. The empty set is occasionally written as
        $\emptyset=\{\;\}$. It is the set that contains no elements.
        \par\hfill\par
        Containment allows us to define what equality is. This is one of the
        fundamental \textit{axioms} of Zermelo-Fraenkel Set Theory (ZFC), the
        \textit{axiom of extensionality}. It states that two sets are equal
        precisely when they consist of the same elements. It is useful to
        discuss some of the elements of logic before presenting the statement.
        \subsection{Predicates, Propositions, and Axioms}
            Propositions and predicates will be two more of our primitive
            notions which we will vaguely define, but mostly rely on intuition.
            \begin{definition}[\textbf{Predicate}]
                \label{def:Predicate}%
                A \gls{predicate} $P$ on a \gls{set} $A$ is a sentence such
                that for all $x\in{A}$ one may state that $P(x)$ is either
                true or false.\index{Predicate}\index{Predicate!Definition}
            \end{definition}
            This definition involves a \textit{quantifier}, the statement
            \textit{for all}. There are two logical quantifiers commonly used
            in mathematics, \textit{for all} and \textit{there exists}. These
            are given the symbols $\forall$ and $\exists$, respectively, but
            for the most part I'll stick with the English language. This is
            clearer to me.
            \par\hfill\par
            Predicates are the main tool used in set theory for defining and
            building new sets via the \textit{axiom schema of specification}%
            \index{Axiom!Schema of Specification}. To see if our definition
            matches other standards, let's consult a dictionary and a textbook.
            Merriam-Webster writes that a predicate is
            \textit{something that is affirmed or denied of the subject in a}
            \textit{proposition in logic} \cite{MerriamWebsterPredicateDef}.
            This is a linguistic definition for the use of the word predicate
            in English, whereas mathematicians often use overly convoluted
            formulas to define things. Looking to Cunningham's
            \textit{A Logical Introduction to Proof} we have that
            \textit{A predicate is just a statement proclaiming that}
            \textit{certain variables satisfy a property}
            \cite{Cunningham2010}. He then uses the example \textit{x is tall}
            where $x$ is taken to be a variable. Thus our adopted definition is
            not at all different from the linguistical or mathematical ones,
            and it truly is a primitive notion.
            \par\hfill\par\hrule
            \begin{example}
                Let $A$ be the set $A=\{\,1,\,2,\,3\,\}$ and let $P$ be the
                predicate \textit{x is greater than 2}. To be precise, $P$ is a
                predicate on the set of all positive integers 1, 2, 3, 4, etc.
                The set of values in $A$ which satisfy $P$ is $B=\{\,3\,\}$. If
                we let $Q$ represent \textit{x is an even integer}, then the
                set of elements in $A$ satisfying $Q$ is $C=\{\,2\,\}$.
            \end{example}
            \hrule
            \begin{example}
                Let $A$ be any set and let $P$ be the sentence
                \textit{x is not equal to x}. There are no elements which
                satisfy this criterion and hence the sentence is always false
                for any input $x$. In other words, the set of elements which
                satisfies $P$ is the empty set $\emptyset$. A sentence that is
                always false is called a
                \textit{contradiction}.\index{Contradiction}
            \end{example}
            \hrule
            \begin{example}
                The sentence \textit{x is red, and x is not red} is another
                example of a contradiction. To the best of my knowledge it is
                impossible for any object to be both red and not red, and hence
                if $A$ is any set, then the set of elements in $A$ satisfying
                this sentence will always be the empty set $\emptyset$.
            \end{example}
            \hrule
            \begin{example}
                Let $A$ be any set and let $P$ be the sentence
                \textit{x is equal to x}. This sentence is always true,%
                \footnote{
                    In the C programming language, for implementations supporting
                    the IEEE 754 floating point format, a NaN (Not-a-Number) has
                    the unique property that $x\ne{x}$. In mathematics, however,
                    $x=x$ is always true.
                }
                regardless
                of $x$, and hence the set of all elements of $A$ satisfying $P$ is
                simply the set $A$. A sentence that is always true is called a
                \textit{tautology}.\index{Tautology}
            \end{example}
            \hrule\par\hfill\par
            This is the primary means of forming sets in mathematics. We take a set
            we have proven exists and then apply some sentence to it and collect all
            elements of the set satisfying our statement. To do this requires
            justification, and this is the \textit{axiom schema of specification}.
            For example, if we have $A=\{1,2,3\}$ and $P$ is the predicate on $A$
            claiming \textit{x is greater than 2}, then we form a new set $B$ by
            collecting the elements of $A$ which satisfy $P$. We denote this using
            the so-called \textit{set-builder notation}\index{Set-Builder Notation}%
            \index{Set!Set-Builder Notation}:
            \begin{equation}
                B=\{\,x\in{A}\;|\;P(x)\,\}
            \end{equation}
            In more relaxed settings one uses the \textit{axiom of unrestricted}
            \textit{comprehension}\index{Axiom!of Unrestricted Comprehension} which
            allows one to form a set by any sentence. That is, we need not first
            consider some set $A$ and then apply our sentence $P$ to the elements of
            $A$ and extract the elements $x\in{A}$ which satisfy $P$, instead we
            ask for the \textit{set of all things satisfying P}. We write:
            \begin{equation}
                B=\{\,x\;|\;P(x)\,\}
            \end{equation}
            Unfortunately this leads to contradiction and must be avoided.
            Nevertheless it is common in textbooks on measure theory or analysis,
            where convoluted sets need to be constructed, to appeal to unrestricted
            comprehension regardless of the consequences. It is fortunate enough
            that almost all of these constructions can be reformulated in a manner
            consistent with ZFC.
            \par\hfill\par
            Moving on to propositions, we first express the Aristotelian
            definition put forward by Aristotle\index{Aristotle} (384-322 B.C.E.).
            \begin{equation}
                \text{A proposition is a sentence which affirms or denies a }
                \text{predicate.}
            \end{equation}
            Examples of propositions are found in the so-called
            \textit{Socrates syllogism}\index{Socrates!Syllogism of}.
            \begin{example}
                We wish to conclude the ancient greek philosopher
                Socrates\index{Socrates} was mortal. We start with the following
                proposition: \textit{All men are mortal}. We assert this sentence is
                true and may be used later in the proofs of other claims. Second,
                we state \textit{Socrates is a man}. This is another proposition
                that we accept as true. To conclude Socrates is mortal we need some
                \textit{rule of inference} that allows us to tie these two
                propositions together. One such rule, known as
                \textit{modus ponens}\index{Modus Ponens}%
                \index{Axiom!of Modus Ponens}, does what we need. It states
                that if $P$ and $Q$ are propositions, if $P$ implies $Q$, and if $P$
                is true, then $Q$ is true. Using this, since all men are mortal, and
                since Socrates is a man, we conclude that Socrates is mortal.
            \end{example}
            An easier proof that Socrates is mortal goes as follows: He's dead.
            Nevertheless, we are starting to see what is needed to construct valid
            proofs. We formalize the Aristotelian view slightly, adopting the
            following definition:%
            \footnote{%
                Though recognizing that this is again another primitive.
            }
            \begin{definition}
                \label{def:Proposition}%
                A \gls{proposition} is a \gls{predicate} on a \gls{set} $A$
                evaluated at a particular element $x\in{A}$, which is then affirmed
                or denied to be true.%
                \index{Proposition}\index{Proposition!Definition}
            \end{definition}
            A \textit{theorem}\index{Theorem}\index{Theorem!Definition} is just a
            proposition that is \textit{mathematical} in nature. Not very precise,
            and many authors choose to label their theorems as propositions.
            Indeed, in the English translation of Euclid's Elements, all theorems
            are called propositions.
            \begin{example}
                Let's use the previous example of Socrates to motivate what we mean.
                Let $P$ be the predicate \textit{x is a man}.%
                \footnote{%
                    To be precise, it is a predicate on the set of all humans who
                    have ever lived.
                }
                If we input Socrates we obtain $P(\text{Socrates})=\text{True}$. If
                we input Hatshepsut, we get $P(\text{Hatshepsut})=\text{False}$.
                This is how we distinguish a predicate from a proposition. $P$ is
                a predicate, whereas $P(\text{Hatshepsut})=\text{False}$ is
                proposition.
            \end{example}
        In Euclid's \textit{Elements} (\textit{c.} 300 B.C.E.) he begins with
        five claims but offers no proof. He states they are true by intuition,
        or are \textit{obviously} true. These are not remarkable claims, by
        any means:
        \begin{enumerate}
            \item Given two points, it is possible to construct a line segment
                between them.
            \item Given a line segment, it is possible to extend it in either
                direction indefinitely.
            \item Given a point and a length, it is possible to construct a
                circle with the prescribed point as its center and the length
                as its radius.
            \item All right angles are equal to one another.
            \item Given a straight line and a point not on that line, there is
                a unique line passing through the point that is parallel to the
                line.
        \end{enumerate}
        This fifth postulate is actually do to John Playfair (1748 - 1819 C.E.)
        and is called \textit{Playfair's Axiom}. Given the first four it is
        equivalent to Euclid's fifth postulate, which is much wordier. In modern
        times these \textit{postulates} are called \textit{axioms}.
        \par\hfill\par
        Before stating what an axiom is we should briefly discuss the use of the
        terms \textit{if}, \textit{then}, \textit{if and only if}, and
        \textit{proposition} in mathematics.
        \begin{definition}[\textbf{Axiom}]
            An axiom is a proposition that is asserted to be true without proof.
        \end{definition}
        Any mathematical system starts with axioms. It is very hard to prove
        \textit{something} with nothing. A good system has intuitive axioms that
        are easy believe. After all, if you come up with an axiom it is your
        job to convince that mathematical community that the claim is
        \textit{obvious}.
        \par\hfill\par
        \begin{axiom}[\textbf{Axiom of Extensionality}]
            If $A$ and $B$ are sets, then $A=B$ if and only if for all
            $a$ it is true that $a\in{A}$ if and only if $a\in{B}$.
        \end{axiom}
        This defines the \textit{equality} symbol $=$ for sets. Let's see it in
        action.
        \par\hfill\par\hrule
        \begin{example}
            Let $A=\{\,1\,2\,3\,\}$ and $B=\{\,3,\,2,\,1\}$. For any $a\in{A}$
            either $a=1$, $a=2$, or $a=3$, by the definition of $A$. But
            $3\in{B}$, $2\in{B}$, and $1\in{B}$. Hence if $a\in{A}$, then
            $a\in{B}$. Similarly, for any $b\in{B}$, either $b=3$, $b=2$, or
            $b=1$. But $3\in{A}$, $2\in{A}$, and $1\in{A}$. Thus if $b\in{B}$,
            then $b\in{A}$. That is, for all $a$ it is true that $a\in{A}$ if
            and only if $a\in{B}$. Therefore, $A=B$.
        \end{example}
        \hrule
        \begin{example}
            Let us again consider $B=\{\,a,\,b,\,c\}$ and
            $A=\{\,a,\,a,\,b,\,c\,\}$. If $x\in{B}$, then either
            $x=a$, $x=b$, or $x=c$. But $a\in{A}$, $b\in{A}$, and $c\in{A}$.
            Hence if $x\in{B}$, then $x\in{A}$. Similarly, if $x\in{A}$ then
            either $x=a$, $x=b$, or $x=c$. Once again we conclude that if
            $x\in{A}$, then $x\in{B}$. Hence for all $x$ it is true that
            $x\in{A}$ if and only if $x\in{B}$. That is, $A=B$.
        \end{example}
        \hrule\par\hfill\par
        Set theory is a very abstract branch of mathematics and you may find
        yourself asking \textit{what's the point}? In the times of Euler and
        Gauss set theory did not exist. Many mathematical arguments relied on
        intuition alone. This led to a crisis in the mathematical world in
        1901 when Bertrand Russell (1872 - 1970 C.E.) discovered his paradox.
        Collections of objects were described by sentences, much like in
        English. For example, \textit{the collection of all cities on Earth},
        or \textit{the set of all students on a given college campus}. For the
        most part this is allowable, but certain sentences give rise to
        paradoxes. Bertrand Russell considered the following.
        \textrm{Is it possible for a set to contain itself?}
