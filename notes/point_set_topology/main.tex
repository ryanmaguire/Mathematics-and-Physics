%-----------------------------------LICENSE------------------------------------%
%   This file is part of Mathematics-and-Physics.                              %
%                                                                              %
%   Mathematics-and-Physics is free software: you can redistribute it and/or   %
%   modify it under the terms of the GNU General Public License as             %
%   published by the Free Software Foundation, either version 3 of the         %
%   License, or (at your option) any later version.                            %
%                                                                              %
%   Mathematics-and-Physics is distributed in the hope that it will be useful, %
%   but WITHOUT ANY WARRANTY; without even the implied warranty of             %
%   MERCHANTABILITY or FITNESS FOR A PARTICULAR PURPOSE.  See the              %
%   GNU General Public License for more details.                               %
%                                                                              %
%   You should have received a copy of the GNU General Public License along    %
%   with Mathematics-and-Physics.  If not, see <https://www.gnu.org/licenses/>.%
%----------------------------------Preamble------------------------------------%
\documentclass{book}
%-----------------------------------Packages-----------------------------------%
\usepackage{geometry}
\geometry{a5paper, margin=14mm}
\usepackage[T1]{fontenc}
\usepackage{graphics, float}            % Graphics/Images.
\usepackage{babel}                      % Language typesetting.
\usepackage[dvipsnames]{xcolor}         % Color names.
\usepackage{mathtools, esint, mathrsfs} % amsmath and integrals.
\usepackage{amsthm, amsfonts, amssymb}  % Fonts and theorems.
\usepackage[newparttoc]{titlesec}       % Formatting chapter, etc.
\usepackage[nottoc]{tocbibind}          % Bibliography in toc.
\usepackage[titles]{tocloft}            % ToC formatting.
\usepackage[hang,multiple]{footmisc}
\usepackage{imakeidx}                   % Used for index.
\usepackage{xr}
\usepackage[font=scriptsize,
            labelformat=simple,
            labelsep=colon]{subcaption} % Subfigure captions.
\usepackage[font={scriptsize},
            hypcap=true,
            labelsep=colon]{caption}    % Figure captions.
\usepackage[pdftex,
            pdfauthor={Ryan Maguire},
            pdftitle={Mathematics and Physics},
            pdfsubject={Mathematics, Physics, Science},
            pdfkeywords={Mathematics, Physics, Computer Science, Biology},
            pdfproducer={LaTeX},
            pdfcreator={pdflatex}]{hyperref}
\hypersetup{
    colorlinks=true,
    linkcolor=blue,
    filecolor=magenta,
    urlcolor=Cerulean,
    citecolor=SkyBlue
}                           % Colors for hyperref.
\usepackage[toc,acronym,nogroupskip,nopostdot]{glossaries}
\usepackage{glossary-mcols}
\usepackage{fancyhdr}
\pagestyle{fancy}
\fancyhf{}
\pagestyle{fancy}
\fancyhf{}
\fancyhead[LE]{\leftmark}
\fancyhead[LO]{\rightmark}
\fancyhead[RE, RO]{\thepage}

%------------------------Theorem Styles-------------------------%
\theoremstyle{plain}
\newtheorem{theorem}{Theorem}[section]

% Define theorem style for default spacing and normal font.
\newtheoremstyle{normal}
    {\topsep}               % Amount of space above the theorem.
    {\topsep}               % Amount of space below the theorem.
    {}                      % Font used for body of theorem.
    {}                      % Measure of space to indent.
    {\bfseries}             % Font of the header of the theorem.
    {}                      % Punctuation between head and body.
    {.5em}                  % Space after theorem head.
    {}

% Define default environments.
\theoremstyle{normal}
\newtheorem{examplex}{Example}[section]
\newtheorem{definitionx}{Definition}[section]
\newtheorem{axiomx}{Axiom}[section]
\newtheorem{notationx}{Notation}[section]

\newenvironment{example}{%
    \pushQED{\qed}\renewcommand{\qedsymbol}{$\blacksquare$}\examplex%
}{%
    \popQED\endexamplex%
}

\newenvironment{definition}{%
    \pushQED{\qed}\renewcommand{\qedsymbol}{$\blacksquare$}\definitionx%
}{%
    \popQED\enddefinitionx%
}

\newenvironment{axiom}{%
    \pushQED{\qed}\renewcommand{\qedsymbol}{$\blacksquare$}\axiomx%
}{%
    \popQED\endaxiomx%
}

\newenvironment{notation}{%
    \pushQED{\qed}\renewcommand{\qedsymbol}{$\blacksquare$}\notationx%
}{%
    \popQED\endnotationx%
}

%--------------------Declared Math Symbols--------------------%
\DeclareMathSymbol{\minus}{\mathbin}{AMSa}{"39} % Unary minus sign.
%------------------------New Commands---------------------------%
\definecolor{CoverPage}{RGB}{204,212,255}

\renewcommand*{\glstextformat}[1]{\textcolor{RoyalBlue}{#1}}
\renewcommand{\glsnamefont}[1]{\textbf{#1}}
\renewcommand\thesubfigure{%
    \arabic{chapter}.\arabic{figure}.\arabic{subfigure}}
\addto\captionsenglish{\renewcommand{\figurename}{Fig.}}
\numberwithin{equation}{section}

\newglossarystyle{longpara}{%
    \setglossarystyle{long}%
    \renewenvironment{theglossary}{%
        \begin{longtable}[l]{{p{0.25\hsize}p{0.65\hsize}}}
    }{\end{longtable}}%
    \renewcommand{\glossentry}[2]{%
        \glstarget{##1}{\glossentryname{##1}}%
        &\glossentrydesc{##1}{~##2.}
        \tabularnewline%
        \tabularnewline
    }%
}
\newglossary[not-glg]{notation}{not-gls}{not-glo}{Notation}
\newcommand*{\newnotation}[4][]{%
    \newglossaryentry{#2}{%
        type=notation,
        name={\textbf{#3}, },
        text={#4},
        description={#4},
        #1%
    }%
}
%----------------------------------Lengths-------------------------------------%
% Spacings for the Table of Contents.
\addtolength{\cftsecnumwidth}{1ex}
\addtolength{\cftsubsecindent}{1ex}
\addtolength{\cftsubsecnumwidth}{1ex}
\addtolength{\cftfignumwidth}{1ex}
\addtolength{\cfttabnumwidth}{1ex}

% Indent and paragraph spacing.
\setlength{\parindent}{0em}
\setlength{\parskip}{0em}
%----------------------------------GLOSSARY------------------------------------%
\makenoidxglossaries
\loadglsentries{glossary}
\loadglsentries{notation}
\makeindex[intoc]
%-----------------------------------MAIN---------------------------------------%
\begin{document}
    \pagenumbering{gobble}
    \pagecolor{CoverPage}
    \begin{titlepage}
        \centering
        \LARGE{\bfseries{Point-Set Topology}}
        \par\vspace{3.5cm}
        \includegraphics{../../images/CoverPage.pdf}
        \par\vspace{4cm}
        \Large{\itshape{Ryan Maguire}}
        \par\vspace{1.5ex}
        \normalsize{\today}
    \end{titlepage}
    \nopagecolor
    \pagenumbering{roman}
    \tableofcontents
    \listoffigures
    \clearpage
    \pagenumbering{arabic}
    %-----------------------------------LICENSE------------------------------------%
%   This file is part of Mathematics-and-Physics.                              %
%                                                                              %
%   Mathematics-and-Physics is free software: you can redistribute it and/or   %
%   modify it under the terms of the GNU General Public License as             %
%   published by the Free Software Foundation, either version 3 of the         %
%   License, or (at your option) any later version.                            %
%                                                                              %
%   Mathematics-and-Physics is distributed in the hope that it will be useful, %
%   but WITHOUT ANY WARRANTY; without even the implied warranty of             %
%   MERCHANTABILITY or FITNESS FOR A PARTICULAR PURPOSE.  See the              %
%   GNU General Public License for more details.                               %
%                                                                              %
%   You should have received a copy of the GNU General Public License along    %
%   with Mathematics-and-Physics.  If not, see <https://www.gnu.org/licenses/>.%
%------------------------------------------------------------------------------%
\chapter{Motivation}
    Topology can be a strange topic to those unfamiliar with its history.
    Intuitively it is the study of raw uncooked dough. If you have a lump of
    dough then you can stretch it and squeeze it without really changing the
    dough. But if you poke your finger through it, you now have a lump of dough
    with a hole in it. To be more precise, topology is the study of continuously
    deforming things. That is, stretching and squeezing is fine, but tearing,
    cutting, closing holes, puncturing holes, etc., is not allowed. A question
    of the $20^{\textrm{th}}$ and $21^{\textrm{st}}$ century asks one to
    classify various \textit{topological spaces} under such constraints.
    \par\hfill\par
    For a much simpler question, consider the sphere and the torus in
    Fig.~\ref{fig:sphere_and_torus}. The sphere $\mathbb{S}^{2}$ is the set of
    all points in $\mathbb{R}^{3}$ that are of unit length. It is not a
    \textit{solid} sphere like the Earth, we only consider the shell of it.
    Similarly for the torus. It is not the entirety of a donut, just the outter
    shell. With this it is hoped that it is clear that these are
    \textit{two dimensional} objects. On the sphere you may walk north-south or
    east-west (no \textit{digging} down is allowed, and no jumping). Similarly
    on the torus at any point there are two independent directions you may
    move. Are they different?
    \begin{figure}[H]
        \centering
        \includegraphics{../../images/torus_next_to_sphere.pdf}
        \caption{A Sphere and a Torus}
        \label{fig:sphere_and_torus}
    \end{figure}
    You may think that in order to go from the sphere to the torus you need to
    introduce a hole, which is not allowed, and then conclude that the torus
    and the sphere are different objects. This thinking is correct,
    but how can one \textit{prove} with mathematical rigor that these are
    distinct things? How does one even properly phrase the question? The
    framework that is needed is \textit{topology}, the study of
    \textit{topological spaces}. At the very end of this book we will prove that
    the torus and the sphere are different topological spaces.
    \par\hfill\par
    Topology was not initiated with the concern for the classification of
    surfaces. This work came in the $19^{\textrm{th}}$ century. What is likely
    the first topological problem was solved by Leonhard Euler
    (1707 - 1783 C.E.) in 1736 C.E. \cite{LeonhardEulerBridgesOfKonigsberg}. In
    the city of K\"{o}nigsberg there were 7 bridges connecting 4 plots of land
    (See Fig.~\ref{fig:bridges_of_konigsberg_001}). Examine the figure for a
    moment and try to trace your fingers over all seven bridges while crossing
    each bridge exactly once. That is, try to walk around K\"{o}nigsberg and
    cross each bridge once and only once. Is this possible?
    \begin{figure}
        \centering
        \includegraphics{../../images/bridges_of_konigsberg_001.pdf}
        \caption{The Seven Bridges of K\"{o}nigsberg}
        \label{fig:bridges_of_konigsberg_001}
    \end{figure}
    Leonhard Euler proposed the following. Disregard the buildings, the grass,
    the water, and the entire city. The only thing relevant to the problem are
    the bridges and how they connect the land. Replace each of the four plots
    of land with dots and replace the seven bridges with curves or
    line segments connecting the appropriate dots. The result is
    Fig.~\ref{fig:bridges_of_konigsberg_002}. This drawing is what is known as
    a \textit{graph}, which is a collection of dots called
    \textit{vertices} and curves connecting the dots called
    \textit{edges}. The seven bridges of K\"{o}nigsberg problem asks one to
    travel along all of the edges of the graph in
    Fig.~\ref{fig:bridges_of_konigsberg_002} without repeating.
    \par\hfill\par
    \begin{figure}
        \centering
        \includegraphics{../../images/bridges_of_konigsberg_002.pdf}
        \caption{The Seven Bridges of K\"{o}nigsberg as a Graph}
        \label{fig:bridges_of_konigsberg_002}
    \end{figure}
    The problem is impossible and the original solution is from Euler. We prove
    this by \textit{contradiction}. We'll suppose it is possible, discover an
    impossibility, and conclude there is no solution. So, suppose
    it is possible and that you've found some path through the graph that
    accomplishes the task. Euler noted that every time you enter a vertex from
    an edge, you must leave the vertex through a different edge,
    otherwise you'll have traversed the same edge twice. This is true for every
    vertex except for the start and end vertices along your path. This means for
    all but at most two vertices the number of edges connected to a vertex is
    even. The number of edges connected to a vertex is the \textit{degree} of
    the vertex, so what we've shown is that that every vertex in the graph,
    with possibly two exceptions, has even degree. If we examine
    Fig.~\ref{fig:bridges_of_konigsberg_003} we see that all four vertices
    have an odd number of edges connected to them, which is a contradiction.
    This shows there is no solution.
    \begin{figure}[H]
        \centering
        \includegraphics{../../images/bridges_of_konigsberg_003.pdf}
        \caption{The Degrees of the Seven Bridges of K\"{o}nigsberg Graph}
        \label{fig:bridges_of_konigsberg_003}
    \end{figure}
    The proof is not quite topological in nature, but the simplification is.
    The edges in Fig.~\ref{fig:bridges_of_konigsberg_002} do not have to be
    drawn like that. You may move them around as you wish as long as the same
    vertices remain connected. You may also adjust the locations of the
    vertices. This notion of \textit{continuously deforming} the space we're
    working with is the heart of topology.
    \par\hfill\par
    \begin{figure}
        \centering
        \begin{subfigure}[b]{0.49\textwidth}
            \includegraphics{../../images/convex_region_001.pdf}
            \subcaption{A Convex Set}
        \end{subfigure}
        \hfill
        \begin{subfigure}[b]{0.49\textwidth}
            \includegraphics{../../images/non_convex_region_001.pdf}
            \subcaption{A Non-Convex Set}
        \end{subfigure}
        \caption{Convex and Non-Convex Sets}
        \label{fig:convex_and_non_convex}
    \end{figure}
    Twenty years after Euler's solution to the Seven Bridges of K\"{o}nigsberg
    problem he discovered one of the most celebrated results of topology. He
    discovered the existence of the \textit{Euler characteristic}. Consider a
    \textit{convex polyhedron}. A polyhedron is called convex if for every
    two points on it the line segment joining them lies entirely on the inside.
    A similar definition holds for arbitrary subsets of $\mathbb{R}^{n}$.
    Fig~\ref{fig:convex_and_non_convex} depicts a convex and a non-convex
    subset of the plane.
    \par\hfill\par
    Now, consider a convex polyhedron. Take a tetrahedron as a simple
    example (Fig.~\ref{fig:tetrahedron_001}). A polyhedron is a collection of
    points, line segments, and faces. For the tetrahedron there are 4 vertices,
    6 edges, and 4 faces. Think of vertices as the
    \textit{zero dimensional} parts of the tetrahedron, edges as the
    \textit{one dimensional} components, and faces are \textit{two dimensional}.
    \begin{figure}[H]
        \centering
        \includegraphics{../../images/tetrahedron_001.pdf}
        \caption{A Tetrahedron}
        \label{fig:tetrahedron_001}
    \end{figure}
    Euler noticed that if you take the \textit{alternating sum} over the number
    of components in each dimension you always get 2. That is, the number of
    vertices minus the number of edges plus the number of faces is 2. For the
    tetrahedron we get $4-6+4=2$, as claimed. The other platonic solids are
    verified, for example the cube is 8 vertices, 12 edges, and 6 faces.
    $8-12+6=2$. In Euler's 1758 paper he proves that for \textit{any} convex
    polyhedron this formula is true \cite{LeonhardEulerPolyhedraFormula}. This
    is sometimes called \textit{Euler's Polyhedral Formula}. Letting $V$ be the
    number of vertices, $E$ the number of edges, and $F$ the number of faces,
    we have:
    \begin{equation}
        V-E+F=2
    \end{equation}
    What makes this topological is that the only thing that matters is that the
    polyhedron can be continuously deformed into a sphere, the convexity is not
    actually needed. If we had a polyhedron that was topologically the same as
    a torus we'd get a different answer.
    Fig.~\ref{fig:polyhedral_torus_001} depicts a polyhedral torus where the
    faces are slightly transparent so that we may easily count $V$, $E$, and
    $F$. I count $V=16$, $E=32$, and $F=16$. The alternating sum is then:
    \begin{equation}
        V-E+F=16-32+16=0
    \end{equation}
    This number is different from the tetrahedron. The reason being that the
    torus cannot be continuously deformed into a sphere, as alluded to earlier.
    The value of $V-E+F$ is called the \textit{Euler characteristic}. The
    actual number depends only on the \textit{topology} of the polyhedra.
    \begin{figure}
        \centering
        \includegraphics{../../images/polyhedral_torus_001.pdf}
        \caption{A Polyhedral Torus}
        \label{fig:polyhedral_torus_001}
    \end{figure}
    \par\hfill\par
    Moving on, one of the direct motivations of topology was work done in the
    $19^{\textrm{th}}$ century on complex analysis. For simplicity we will
    think only of curves in the plane $\mathbb{R}^{2}$. If you have a
    differentiable curve $\gamma:[0,1]\rightarrow\mathbb{R}^{2}$ and a
    continuous function
    $\mathbf{F}:\mathbb{R}^{2}\rightarrow\mathbb{R}^{2}$, you can compute the
    \textit{line integral} as follows:
    \begin{equation}
        \int_{\gamma}\mathbf{F}(\mathbf{r})\cdot\textrm{d}\mathbf{r}
            =\int_{0}^{1}\mathbf{F}\big(\gamma(t)\big)\cdot
                \dot{\gamma}(t)\;\textrm{d}t
    \end{equation}
    Where $\dot{\gamma}(t)$ denotes the derivative of $\gamma$ at $t$.
    The right hand side of the equation is now of the form of something seen
    in an integral calculus course. Line integrals in complex analysis are
    defined similarly. If we have a differentiable curve
    $\gamma:[0,1]\rightarrow\mathbb{C}$ and a differentiable function
    $f:\mathbb{C}\rightarrow\mathbb{C}$, we may define:
    \begin{equation}
        \int_{\gamma}f(z)\;\textrm{d}z
            =\int_{0}^{1}f\big(\gamma(t)\big)\dot{\gamma}(t)\;\textrm{d}t
    \end{equation}
    In the case that the curve $\gamma$ starts where it ends,
    $\gamma(0)=\gamma(1)$, mathematicians like to decorate the integral symbol
    with a circle. We write:
    \begin{equation}
        \oint_{\gamma}f(z)\;\textrm{d}z=
            \int_{0}^{1}f\big(\gamma(t)\big)\dot{\gamma}(t)\;\textrm{d}t
    \end{equation}
    The circle simply emphasizes that $\gamma$ is a \textit{closed} curve.
    Cauchy's integral formula states that if $\gamma$ is a closed curve that
    does not touch the origin (i.e. $\gamma(t)\ne{0}$ for all $t$), then:
    \begin{equation}
        f(0)=\frac{w}{2\pi{i}}\oint_{\gamma}\frac{f(z)}{z}\;\textrm{d}z
    \end{equation}
    where $w$ is the \textit{winding number}, the number of times the curve
    $\gamma$ wraps around the origin counter-clockwise
    (See Fig.~\ref{fig:winding_number_001}). The only part of the curve
    $\gamma$ that matters is the winding number. The winding number is a value
    that is preserved under \textit{continuous deformation}. Try to convince
    yourself of this. Take a look at the far-right curve in
    Fig~\ref{fig:winding_number_001}, the one with winding number 0. The curve
    does not enclose the origin. We are allowed to move the curve in any way
    we please, but it can \textit{never} pass through the origin
    (In the formula we are dividing by $z$. Passing through the origin equates
    to division by zero). It seems impossible that we would be able to move this
    curve in a way that it then encircles the origin. It is indeed impossible,
    but again, how does one prove this?
    \begin{figure}
        \centering
        \includegraphics{../../images/winding_number_001.pdf}
        \caption{Winding Number of Closed Curves}
        \label{fig:winding_number_001}
    \end{figure}
    \par\hfill\par
    Lastly, one of the strong motivating factors of point-set topology was
    the geometry of the $19^{\textrm{th}}$ and early
    $20^{\textrm{th}}$ centuries. Mathematicians like
    Bernhard Riemann (1826 - 1866 C.E.) and Carl Friedrich Gauss
    (1777 - 1855 C.E.) began pondering generalizations of smooth surfaces,
    like spheres and torii, in higher dimensions. Many of the ideas were based
    on intuition. Spaces that locally \textit{look like} Euclidean spaces
    $\mathbb{R}^{n}$. The sphere is like this. If you stand on the Earth and
    only look at your immediate surroundings, it's easy to convince yourself
    that \textit{locally} you're standing on a flat plane. All smooth surfaces
    in $\mathbb{R}^{3}$ are like this. The developments of the
    $19^{\textrm{th}}$ century sought a framework for higher dimensional
    analogues. At the turn of the $20^{\textrm{th}}$ century, the physicist
    Albert Einstein (1879 - 1955 C.E.) used these ideas to formulate his
    general theory of relativity. That the universe can be modelled as a
    four dimensional \textit{spacetime} satisfying certain properties. The
    foundation of all of this is the idea of a \textit{topological manifold},
    which will be explored in later chapters.

    %-----------------------------------LICENSE------------------------------------%
%   This file is part of Mathematics-and-Physics.                              %
%                                                                              %
%   Mathematics-and-Physics is free software: you can redistribute it and/or   %
%   modify it under the terms of the GNU General Public License as             %
%   published by the Free Software Foundation, either version 3 of the         %
%   License, or (at your option) any later version.                            %
%                                                                              %
%   Mathematics-and-Physics is distributed in the hope that it will be useful, %
%   but WITHOUT ANY WARRANTY; without even the implied warranty of             %
%   MERCHANTABILITY or FITNESS FOR A PARTICULAR PURPOSE.  See the              %
%   GNU General Public License for more details.                               %
%                                                                              %
%   You should have received a copy of the GNU General Public License along    %
%   with Mathematics-and-Physics.  If not, see <https://www.gnu.org/licenses/>.%
%------------------------------------------------------------------------------%
\chapter{Set Theory}
    We now begin our work-in-earnest on point-set topology. Topology is often
    a students first or second course in mathematics that requires mathematical
    foundations for proof. To accomplish this requires set theory. This is not
    a lengthy treatise of mathematical logic, but rather will explore some of
    the central concepts, provide definitions, examples, and important theorems.
    \section{Sets and Logic}
        The main objects in mathematics are \textit{sets}. This development came
        about in the 1800's with figures like Georg Cantor\index{Cantor, Georg},
        Augustus De Morgan\index{De Morgan, Augustus}, and Bernard
        Bolzano\index{Bolzano, Bernard} making the first strides in the theory.
        The early history is intuitive, but vague. This led to Russell's
        paradox\index{Paradox!Russell's Paradox}\index{Russell's Paradox} which
        ultimately showed stricter \textit{axioms} of set theory were needed.
        Let's look to the primary source for a definition, posited by Georg
        Cantor (1845 - 1918 C.E.). He wrote \cite{Cantor1895}:
        \begin{center}
            \textit{A set is a gathering together into a whole of definite}
            \textit{distinct objects of our perception or of our thought, which}
            \textit{are called the elements of the set.}
            \par\hfill
            \textit{Beitr\"{a}ge zur Begr\"{u}ndung der Transfiniten}
            \textit{Mengenlehre}%
            \footnote{%
                English:
                \textit{Contributions in Support of Transfinite Set Theory}.%
            }
            \par\hfill
            \textit{Georg Cantor, 1985 C.E.}
        \end{center}
        Felix Hausdorff\index{Hausdorff, Felix} (1868 - 1942 C.E.) posits
        \cite[p.~11]{HausdorffSetTheory}:
        \begin{center}
            \textit{A set is formed by the grouping together of single objects}
            \textit{into a whole. A set is a plurality thought of as a unit.}
            \par\hfill
            \textit{Mengenlehre}%
            \footnote{%
                English:
                \textit{Set Theory}.%
            }
            \par\hfill
            \textit{Felix Hausdorff, 1927 C.E.}
        \end{center}
        Both are beautifully phrased, but circular since the terms
        \textit{gathering}, \textit{objects}, and \textit{grouping} are not
        defined. This form of circularity was addressed by Alfred
        Tarski\index{Tarski, Alfred} (1901 - 1983 C.E.) in his 1946 book
        \textit{Introduction to Logic and the Methodology of the Deductive}
        \textit{Science}. He expresses the need for primitive
        undefined notions that we take for granted and use freely. For us, the
        word \textit{set} is a primitive.
        \begin{definition}[\textbf{Set}]
            A \gls{set} is a collection of objects called elements.
        \end{definition}
        I have not defined \textit{collection} nor \textit{object}. To
        understand what a set is requires examples.
        \par\hfill\par\hrule
        \begin{example}
            If a set has a finite number of
            elements it is common notation to list the elements, separated by
            commas, and enclosed in curly braces $\{\}$. It is also common to
            denote sets by capital Latin letters like $A$, $B$, and $C$, and
            the elements by lower case letters like $a$, $b$, and $c$. Let us
            say that $A$ is the set of the numbers 1, 2, and 3.
            This may be written:
            \begin{equation}
                A=\{\,1,\,2,\,3\,\}
            \end{equation}
            The order is immaterial. We may also write:
            \begin{equation}
                A=\{\,2,\,1,\,3\,\}
            \end{equation}
            The number of times a single element occurs is also irrelevant. We
            also have:
            \begin{equation}
                A=\{\,1,\,1,\,2,\,3\,\}
            \end{equation}
            All of these are equivalent representations for the set $A$.
        \end{example}
        \hrule
        \begin{example}
            Sets are not restricted to numbers. We can have sets of almost
            anything. For example, we can consider the first few letters of
            the alphabet:
            \begin{equation}
                B=\{\,a,\,b,\,c\,\}
            \end{equation}
            So long as the symbols $a$, $b$, and $c$ have been defined, $B$ is
            a valid set. Repetition and order have no meaning here. We can write
            the following:
            \begin{equation}
                B=\{\,a,\,c,\,b\,\}=\{\,a,\,a,\,b,\,c\,\}
            \end{equation}
            These are also equivalent representations of the set $B$.
        \end{example}
        \hrule\par\hfill\par
        What defines a set are the elements it contains. Order and repetition of
        and element are meaningless as far as a set is concerned.
        \begin{notation}[\textbf{Containment}]
            The symbol \gls{containmentsymb} denotes containment.
            If $A$ is a set, and $a$ is an element of $A$ we write
            $a\in{A}$. If $a$ is not an element of $a$ we write $a\notin{A}$.
        \end{notation}
        The expression $a\in{A}$ reads aloud as $a$ \textit{is in} $A$ or
        $a$ \textit{is contained in} $A$ or $a$ \textit{is an element of} $A$.
        Similarly, $a\notin{A}$ reads $a$ \textit{is not in} $A$. Containment is
        another one of our primitives that is best described with example.
        \par\hfill\par\hrule
        \begin{example}
            Let $A=\{\,1,\,2,\,3\,\}$. Then we have $1\in{A}$ since 1 is an
            element of the set $A$. Similarly $2\in{A}$ and $3\in{A}$. However,
            $4\notin{A}$ since 4 is not contained in $A$. Similarly
            $5\notin{A}$ and $\pi\notin{A}$.
        \end{example}
        \hrule
        \begin{example}
            Using our second example of letters, if $B=\{\,a,\,b,\,c\,\}$, then
            $a\in{B}$, $b\in{B}$, and $c\in{B}$, but $d\notin{B}$. Similarly
            $1\notin{B}$ and $7\notin{B}$.
        \end{example}
        \hrule\par\hfill\par
        There is a special set called the \textit{empty set}, denoted
        $\emptyset$, with the property that for all $x$ it is true that
        $x\notin\emptyset$. The empty set is occasionally written as
        $\emptyset=\{\;\}$. It is the set that contains no elements.
        \par\hfill\par
        Containment allows us to define what equality is. This is one of the
        fundamental \textit{axioms} of Zermelo-Fraenkel Set Theory (ZFC), the
        \textit{axiom of extensionality}. It states that two sets are equal
        precisely when they consist of the same elements. It is useful to
        discuss some of the elements of logic before presenting the statement.
        \subsection{Predicates, Propositions, and Axioms}
            Propositions and predicates will be two more of our primitive
            notions which we will vaguely define, but mostly rely on intuition.
            \begin{definition}[\textbf{Predicate}]
                \label{def:Predicate}%
                A \gls{predicate} $P$ on a \gls{set} $A$ is a sentence such
                that for all $x\in{A}$ one may state that $P(x)$ is either
                true or false.\index{Predicate}\index{Predicate!Definition}
            \end{definition}
            This definition involves a \textit{quantifier}, the statement
            \textit{for all}. There are two logical quantifiers commonly used
            in mathematics, \textit{for all} and \textit{there exists}. These
            are given the symbols $\forall$ and $\exists$, respectively, but
            for the most part I'll stick with the English language. This is
            clearer to me. These quantifiers take practice to get used to. Let's
            try some examples.
            \par\hfill\par\hrule
            \begin{example}
                Let $\mathbb{N}$ denote the set of \textit{natural numbers}.
                These are the non-negative whole numbers:
                \begin{equation}
                    \mathbb{N}=\{\,0,\,1,\,2,\,3,\,\dots\,\}
                \end{equation}
                The notation $\dots$ usually indicates that the set is
                \textit{infinite} and that the rest of the elements follow a
                pattern. The pattern I am trying to suggest continues as
                \begin{equation}
                    \mathbb{N}=\{\,0,\,1,\,2,\,3,\,4,\,5,\,6,\,7,\,\dots\,\}
                \end{equation}
                This is \textit{not rigorous}, but used frequently as it helps
                shorten notation. Let $\mathbb{R}$ denote the set of real
                numbers. This includes things like $\pi$, $e$, $\sqrt{2}$, and
                all of the integers. Let's considered the following.
                \begin{center}
                    \textit{for all real numbers} $x$
                    \textit{it is true that} $x$ \textit{is a natural number}.
                \end{center}
                This claim is true. We can use our quantifier notation to write
                this as follows:
                \begin{equation}
                    \forall_{x}(x\in\mathbb{R}\Rightarrow{x}\in\mathbb{N})
                \end{equation}
                This reads \textit{for all} $x$, \textit{if} $x$
                \textit{is in} $\mathbb{R}$, \textit{then} $x$
                \textit{is in} $\mathbb{N}$. The symbol $\Rightarrow$ means
                \textit{implies}. I will stick with ordinary English, but these
                are the symbols commonly used in logic and set theory.
            \end{example}
            \hrule
            \begin{example}
                Let $\mathbb{N}$ and $\mathbb{R}$ denote the natural and real
                numbers, respectively. Consider the following claim.
                \begin{center}
                    \textit{There exists a real number} $x$
                    \textit{such that} $x$ \textit{is not a natural number}.
                \end{center}
                We may rewrite this as follows.
                \begin{equation}
                    \exists_{x}(x\in\mathbb{R}\land{x}\notin\mathbb{N})
                \end{equation}
                This is true, the fraction $\frac{1}{2}$ serves as an example.
                This reads \textit{there exists} $x$ \textit{such that}
                $x$ \textit{is in} $\mathbb{R}$ \textit{and} $x$
                \textit{is not in} $\mathbb{N}$. The symbol $\land$ means
                \textit{and}. The words \textit{and} and \textit{or} have a
                precise meaning in logic and mathematics that we will explore
                in the next subsection.
            \end{example}
            \hrule\par\hfill\par
            Predicates are the main tool used in set theory for defining and
            building new sets via the \textit{axiom schema of specification}%
            \index{Axiom!Schema of Specification}. To see if our definition
            matches other standards, let's consult a dictionary and a textbook.
            Merriam-Webster writes that a predicate is
            \textit{something that is affirmed or denied of the subject in a}
            \textit{proposition in logic} \cite{MerriamWebsterPredicateDef}.
            This is a linguistic definition for the use of the word predicate
            in English, whereas mathematicians often use overly convoluted
            formulas to define things. Looking to Cunningham's
            \textit{A Logical Introduction to Proof} we have that
            \textit{A predicate is just a statement proclaiming that}
            \textit{certain variables satisfy a property}
            \cite{Cunningham2010}. He then uses the example \textit{x is tall}
            where $x$ is taken to be a variable. Thus our adopted definition is
            not at all different from the linguistical or mathematical ones,
            and it truly is a primitive notion.
            \par\hfill\par\hrule
            \begin{example}
                Let $A$ be the set $A=\{\,1,\,2,\,3\,\}$ and let $P$ be the
                predicate \textit{x is greater than 2}. To be precise, $P$ is a
                predicate on the set of all positive integers 1, 2, 3, 4, etc.
                The set of values in $A$ which satisfy $P$ is $B=\{\,3\,\}$. If
                we let $Q$ represent \textit{x is an even integer}, then the
                set of elements in $A$ satisfying $Q$ is $C=\{\,2\,\}$.
            \end{example}
            \hrule
            \begin{example}
                Let $A$ be any set and let $P$ be the sentence
                \textit{x is not equal to x}. There are no elements which
                satisfy this criterion and hence the sentence is always false
                for any input $x$. In other words, the set of elements which
                satisfies $P$ is the empty set $\emptyset$. A sentence that is
                always false is called a
                \textit{contradiction}.\index{Contradiction}
            \end{example}
            \hrule
            \begin{example}
                The sentence \textit{x is red, and x is not red} is another
                example of a contradiction. To the best of my knowledge it is
                impossible for any object to be both red and not red, and hence
                if $A$ is any set, then the set of elements in $A$ satisfying
                this sentence will always be the empty set $\emptyset$.
            \end{example}
            \hrule
            \begin{example}
                Let $A$ be any set and let $P$ be the sentence
                \textit{x is equal to x}. This sentence is always true,%
                \footnote{
                    In the C programming language, for implementations
                    supporting the IEEE 754 floating point format, a NaN
                    (Not-a-Number) has the unique property that $x\ne{x}$.
                    In mathematics, however, $x=x$ is always true.
                }
                regardless of $x$, and hence the set of all elements of $A$
                satisfying $P$ is simply the set $A$. A sentence that is always
                true is called a \textit{tautology}.\index{Tautology}
            \end{example}
            \hrule\par\hfill\par
            This is the primary means of forming sets in mathematics. We take a
            set we have proven exists and then apply some sentence to it and
            collect all elements of the set satisfying our statement. To do
            this requires justification, and this is the
            \textit{axiom schema of specification}. For example, if we have
            $A=\{1,\,2,\,3\}$ and $P$ is the predicate on $A$ claiming
            \textit{x is greater than 2}, then we form a new set $B$ by
            collecting the elements of $A$ which satisfy $P$. We denote this
            using the so-called
            \textit{set-builder notation}\index{Set-Builder Notation}%
            \index{Set!Set-Builder Notation}:
            \begin{equation}
                B=\{\,x\in{A}\;|\;P(x)\,\}
            \end{equation}
            In more relaxed settings one uses the \textit{axiom of unrestricted}
            \textit{comprehension}\index{Axiom!of Unrestricted Comprehension} which
            allows one to form a set by any sentence. That is, we need not first
            consider some set $A$ and then apply our sentence $P$ to the elements of
            $A$ and extract the elements $x\in{A}$ which satisfy $P$, instead we
            ask for the \textit{set of all things satisfying P}. We write:
            \begin{equation}
                B=\{\,x\;|\;P(x)\,\}
            \end{equation}
            Unfortunately this leads to contradiction and must be avoided.
            Nevertheless it is common in textbooks on measure theory or
            analysis, where convoluted sets need to be constructed, to appeal
            to unrestricted comprehension regardless of the consequences. It is
            fortunate enough that almost all of these constructions can be
            reformulated in a manner consisted with the rules of set theory.
            \par\hfill\par
            Moving on to propositions, we first express the Aristotelian
            definition put forward by Aristotle\index{Aristotle}
            (384 - 322 B.C.E.).
            \begin{equation}
                \text{A proposition is a sentence which affirms or denies a }
                \text{predicate.}
            \end{equation}
            Examples of propositions are found in the so-called
            \textit{Socrates syllogism}\index{Socrates!Syllogism of}.
            \par\hfill\par\hrule
            \begin{example}
                We wish to conclude the ancient greek philosopher
                Socrates\index{Socrates} was mortal. We start with the
                following proposition: \textit{All men are mortal}. We assert
                this sentence is true and may be used later in the proofs of
                other claims. Second, we state \textit{Socrates is a man}.
                This is another proposition that we accept as true. To conclude
                Socrates is mortal we need some \textit{rule of inference} that
                allows us to tie these two propositions together. One such
                rule, known as \textit{modus ponens}\index{Modus Ponens}%
                \index{Axiom!of Modus Ponens}, does what we need. It states
                that if $P$ and $Q$ are propositions, if $P$ implies $Q$, and
                if $P$ is true, then $Q$ is true. Using this, since all men are
                mortal, and since Socrates is a man, we conclude that Socrates
                is mortal.
            \end{example}
            \hrule\par\hfill\par
            We formalize the Aristotelian view slightly, adopting the
            following definition:%
            \footnote{%
                Though recognizing that this is again another primitive.
            }
            \begin{definition}[\textbf{Proposition}]
                \label{def:Proposition}%
                A \gls{proposition} is a \gls{predicate} on a \gls{set} $A$
                evaluated at a particular element $x\in{A}$, which is then
                affirmed or denied to be true.%
                \index{Proposition}\index{Proposition!Definition}
            \end{definition}
            A \textit{theorem}\index{Theorem}\index{Theorem!Definition} is just
            a proposition that is \textit{mathematical} in nature. Not very
            precise, and many authors choose to label their theorems as
            propositions. Indeed, in the English translation of Euclid's
            \textit{Elements}, one of the most important textbooks in the
            history of mathematics and science, all theorems are called
            propositions.
            \begin{example}
                Let's use the previous example of Socrates to motivate what we
                mean. Let $P$ be the predicate \textit{x is a man}.%
                \footnote{%
                    To be precise, it is a predicate on the set of all humans
                    who have ever lived.
                }
                If we input Socrates we obtain $P(\text{Socrates})=\text{True}$.
                If we input Hatshepsut, we get
                $P(\text{Hatshepsut})=\text{False}$. This is how we distinguish
                a predicate from a proposition. $P$ is a predicate, whereas
                $P(\text{Hatshepsut})=\text{False}$ is proposition.
            \end{example}
            The goal of this excursion into logic is explain what
            \textit{axioms} are. We now have the proper vocabularly to achieve
            this goal, but first some context.
            In Euclid's \textit{Elements} (\textit{c.} 300 B.C.E.) he begins
            with five claims but offers no proof. He states they are true by
            intuition, or are \textit{obviously} true. These are not remarkable
            claims, by any means:
            \begin{enumerate}
                \item Given two points, it is possible to construct a line
                    segment between them.
                \item Given a line segment, it is possible to extend it in
                    either direction indefinitely.
                \item Given a point and a length, it is possible to construct a
                    circle with the prescribed point as its center and the
                    length as its radius.
                \item All right angles are equal to one another.
                \item Given a straight line and a point not on that line,
                    there is a unique line passing through the point that is
                    parallel to the line.
            \end{enumerate}
            This fifth postulate is actually do to John Playfair
            (1748 - 1819 C.E.) and is called \textit{Playfair's Axiom}. Given
            the first four it is equivalent to Euclid's fifth postulate, which
            is much wordier. In modern times these \textit{postulates} are
            called \textit{axioms}. Euclid claimed all five were true, but ask
            yourself, is the fifth one true? If you and a friend stand on the
            equator of the earth, perhaps 10 meters apart, and walk
            \textit{north}, you will both be walking in \textit{straight lines}.
            After a lot of travel your paths will intersect at the north pole,
            even though you were travelling along supposedly \textit{parallel}
            lines. In \textit{spherical}, the geometry of the Earth, the fifth
            axiom is false. Our use of the word \textit{axiom} will be slightly
            stronger. We want an axiom to be something that is assumed to
            always be true regardless of the situation.
            \begin{definition}[\textbf{Axiom}]
                An axiom is a proposition that is asserted to be true
                without proof.
            \end{definition}
            Any mathematical system starts with axioms. It is very hard to prove
            \textit{something} with nothing. A good system has intuitive axioms
            that are easy believe. After all, if you come up with an axiom it
            is your job to convince that mathematical community that the claim
            is \textit{obvious}. The axioms that we'll use, the definitions
            that will be stated, and the theorems that will be proved will make
            heavy use of the logical words \textit{and, or}, and
            \textit{if-then}. Let's take some time acquainting ourselves with
            these.
        \subsection{Logical Connectives}
            The word \textit{and} is common enough in English, but in logic and
            mathematics it has a very precise meaning. Consider two predicates
            $P$ and $Q$ which act on the same set $A$. For example, let $A$ be
            the set of all animal species, $P(x)$ be the sentence
            \textit{x lives in the antarctic}, and $Q(x)$ the sentence
            \textit{x is a flightless bird}. What does it mean for
            $P(x)$ \textit{and} $Q(x)$ to be true, simultaneously? Well, let's
            consider what it \textit{doesn't} mean. $P(x)$ and $Q(x)$ is
            certainly not true if both $P(x)$ and $Q(x)$ are false. For example,
            if $x$ is an elephant, then $x$ is \textit{not} an animal that lives
            in the arctic, and $x$ is \textit{not} a flightless bird, so the
            sentence
            \begin{center}
                \textit{an elephant is an animal that lives in the arctic}
                \textit{and is a flightless bird}
            \end{center}is certainly false. What if $P$ is true and $Q$ is
            false? For example, if $x$ is a sea lion?
            The sentence
            \begin{center}
                \textit{a sea lion is an animal that lives in the}
                \textit{arctic and is a flightless bird}
            \end{center}
            is false since sea lions are not flightless birds. What about
            $P$ being false and $Q$ being true? For example, an ostrich.
            The sentence
            \begin{center}
                \textit{an ostrich is an animal that lives in the}
                \textit{arctic and is a flightless bird}
            \end{center}
            Ostriches are flightless birds, but they do not live in the
            antarctic. We conclude that this third sentence is false. Finally,
            what about penguins? The sentence
            \begin{center}
                \textit{a penguin is an animal that lives in the}
                \textit{arctic and is a flightless bird}
            \end{center}
            is indeed true. We see that $P(x)$ \textit{and} $Q(x)$ should only
            be regarded as true when both $P(x)$ is true and $Q(x)$ is true.
            All other cases are false. This is exemplified by table. The word
            \textit{and} is replaced by the symbol $\land$ in logic. The
            expression $P\land{Q}$ reads $P$ \textit{and} $Q$.
            Tab.~\ref{tab:truth_table_conjunction} shows the
            \textit{truth table} for the logical and, which is also called
            \textit{conjunction}.
            \begin{table}[H]
                \centering
                \begin{tabular}{c | c | c}
                    $P$&$Q$&$P\land{Q}$\\
                    \hline
                    False&False&False\\
                    \hline
                    False&True&False\\
                    \hline
                    True&False&False\\
                    \hline
                    True&True&True
                \end{tabular}
                \caption{Truth Table for Conjunction (Logical And)}
                \label{tab:truth_table_conjunction}
            \end{table}
            The word \textit{or} is also called
            \textit{logical disjunction}. It is reserved the symbol $\lor$. Like
            \textit{and} it combines two predicates into a new one. Let us
            consider the previous predicates
            $P(x)$ \textit{x is an animal that lives in the antarctic} and
            $Q(X)$ \textit{x is a flightless bird}. An elephant is neither a
            flightless bird nor an animal the lives in the arctic, so we
            conclude
            \begin{center}
                \textit{an elephant is an animal that lives in the antarctic}
                \textit{or is a flightless bird}
            \end{center}
            is false. Sea lions do live in the antarctic, so even though they
            are not flightless birds we can conclude that
            \begin{center}
                \textit{a sea lion is an animal that lives in the antarctic}
                \textit{or is a flightless bird}
            \end{center}
            is true. Ostriches are flightless birds that do not live in the
            antarctic. We may truthfully say that
            \begin{center}
                \textit{an ostrich is an animal that lives in the antarctic}
                \textit{or is a flightless bird}
            \end{center}
            Lastly, penguins are both flightless birds, and live in the
            antarctic, so there is no issue in claiming that
            \begin{center}
                \textit{a penguin is an animal that lives in the antarctic}
                \textit{or is a flightless bird}
            \end{center}
            The only time $P(x)$ \textit{or} $Q(x)$ is false is when
            \textit{both} sentences are false
            (Tab.~\ref{tab:truth_table_disjunction}).
            \begin{table}[H]
                \centering
                \begin{tabular}{c | c | c}
                    $P$&$Q$&$P\lor{Q}$\\
                    \hline
                    False&False&False\\
                    \hline
                    False&True&True\\
                    \hline
                    True&False&True\\
                    \hline
                    True&True&True
                \end{tabular}
                \caption{Truth Table for Disjunction (Logical Or)}
                \label{tab:truth_table_disjunction}
            \end{table}
            Students are usually fine with adopting the logical use of the
            words \textit{or} and \textit{and} since they mimic their standard
            use in English. What is more apt to cause confusion is the use of
            \textit{if} and \textit{then}. Let's try to see what it should mean
            with an example. Consider the sentence
            \begin{center}
                \textit{if I am late to work, then I will be fired}
            \end{center}
            This contains two predicates. $P(x)$ is \textit{x is late for work}
            and $Q(x)$ is \textit{x will be fired} (acting on the set of
            employees at a company). Like the previous truth tables there are
            four combinations of true and false for $P$ and $Q$. Let's work
            through them.
            \par\hfill\par
            \textit{I was not late to work, and I was not fire.}. Is the
            statement \textit{if I am late to work, then I will be fired}
            true or false? The situation that triggers the firing didn't happen
            so we certainly cannot conclude that the statement is false. We
            thus say that in this scenario the sentence is true.
            \par\hfill\par
            \textit{I was not late to work, and I was fired}. Is the statement
            false? Of course not, there are plenty of reasons for being fired.
            Perhaps I have a sailor's tongue or sloppy handwriting. In this
            situation we conclude that the sentence is true.
            \par\hfill\par
            \textit{I was late to work, and I was not fired}. In this event, my
            boss is very nice, however the sentence is false. I was late to work
            and yet I was not fired.
            \par\hfill\par
            \textit{I was late to work, and I was fired}. This is perhaps the
            easiest to grasp since it's the scenario one intuitively thinks
            about when hearing \textit{if-then} sentences. In this case the
            sentence is true.
            \par\hfill\par
            The technical name for \textit{if-then} is \textit{implication}. We
            have seen the symbol for this before, it is a large arror
            $\Rightarrow$. The truth table for implication is given in
            Tab.~\ref{tab:truth_table_implication}.
            \begin{table}
                \centering
                \begin{tabular}{c | c | c}
                    $P$&$Q$&$P\Rightarrow{Q}$\\
                    \hline
                    False&False&True\\
                    \hline
                    False&True&True\\
                    \hline
                    True&False&False\\
                    \hline
                    True&True&True
                \end{tabular}
                \caption{Truth Table for Implication}
                \label{tab:truth_table_implication}
            \end{table}
            \par\hfill\par
            The next logical connective to discuss is
            \textit{logical equivalence}. It is a combination of implication
            and conjunction (logical and) and it is given the symbol
            $\Leftrightarrow$. The expression $P\Leftrightarrow{Q}$ is read
            \textit{P if and only if Q} and it means $P$ and $Q$ are essentially
            the same sentence. $P\Leftrightarrow{Q}$ means
            $P\Rightarrow{Q}\land{Q}\Rightarrow{P}$. That is,
            $P$ implies $Q$ and $Q$ implies $P$. The truth table can be obtained
            by studying the truth tables for conjunction $(\land)$ and
            implication $(\Rightarrow$). It is given in
            Tab.~\ref{fig:truth_table_equivalence}. Before reading the table,
            try to concoct it yourself. Write the table for $P\Rightarrow{Q}$,
            then the table for $Q\Rightarrow{P}$, and then the table for
            $(P\Rightarrow{Q})\land(Q\Rightarrow{P})$. This final table is the
            truth table of equivalence.
            \begin{table}[H]
                \centering
                \begin{tabular}{c | c | c | c | c}
                    $P$&$Q$&$P\Rightarrow{Q}$&
                        $Q\Rightarrow{P}$&$P\Leftrightarrow{Q}$\\
                    \hline
                    False&False&True&True&True\\
                    \hline
                    False&True&True&False&False\\
                    \hline
                    True&False&False&True&False\\
                    \hline
                    True&True&True&True&True
                \end{tabular}
                \caption{Truth Table for Equivalence}
                \label{tab:truth_table_equivalence}
            \end{table}
            The phrase \textit{if and only if} is usually too long to write on
            a blackboard over and over, so mathematicians adopt the use of the
            abbreviation \textit{iff}. If you see \textit{P iff Q}, this is
            shorthand for \textit{P if and only Q}, which is equivalent to
            $P\Leftrightarrow{Q}$.
            \par\hfill\par
            There are two more common connectives in logic, \textit{negation}
            and \textit{exclusive or} (commonly written \textit{xor}). Negation
            we will use heavily, exclusive or we will not. For the
            computer-science enthusiast, the exclusive or is well-used in many
            applications. Since we wish to study topology, let us direct our
            attention to negation. The symbol for negation is $\neg$, the
            expression $\neg{P}$ reads aloud as \textit{not P}. It means what it
            sounds like, \textit{the opposite of P}. Let us look to examples.
            \par\hfill\par\hrule
            \begin{example}
                Given an integer $n\in\mathbb{N}$ either it is even or odd,
                essentially by definition. The word \textit{even} means
                \textit{is divisible by 2}, the word \textit{odd} means
                \textit{not even}. If we let $P(n)$ be \textit{n is even} and
                $Q(n)$ be \textit{n is odd}, then $Q$ can equally be defined as
                $Q(n)=\neg{P}(n)$. That is, $Q(n)$ is \textit{n is not even}.
            \end{example}
            \hrule\par\hfill\par
            \begin{table}
                \centering
                \begin{tabular}{c | c }
                    $P$&$\neg{P}$\\
                    \hline
                    False&True\\
                    \hline
                    True&False
                \end{tabular}
                \caption{Truth Table for Negation}
                \label{tab:truth_table_negation}
            \end{table}
            The truth table for negation is given in
            Tab.~\ref{tab:truth_table_negation}. It is important to see how
            negation interacts with the other connectives. Many theorems
            involve the negation of statements meaning a solid understanding of
            how these concepts interplay is required.
            \par\hfill\par
            Consider how negation and conjunction (logical \textit{and}) might
            interact. Ponder for a moment the previous sentences
            $P(x)$ \textit{x is an animal that lives in the antarctic} and
            $Q(x)$ \textit{x is a flightless bird}. What does
            \textit{not} $P$ and $Q$ mean? That is, what does
            $\neg(P\land{Q})$ describe? The only case where $P\land{Q}$ is true
            is when both $P$ is true and $Q$ is true. This was when we had a
            penguin for $x$. In all other cases $P\land{Q}$ was false because
            at least \textit{one} of the sentences $P(x)$ and $Q(x)$ was false
            (consider the elephant, sea lion, and ostrich). This means the only
            time $\neg(P\land{Q})$ is false is when both $P$ is true and $Q$ is
            true. That is, $\neg(P\land{Q})$ is true precisely when either $P$
            is false \textit{or} $Q$ is false. This is the
            \textit{De Morgan Law of Conjunction}.
            \begin{equation}
                \neg(P\land{Q})=(\neg{P})\lor(\neg{Q})
            \end{equation}
            The negation symbol seems to \textit{distribute} over the
            conjunction symbol, converting it into a disjunction.
            \par\hfill\par
            Next, how does disjunction interact with negation? The only way
            $P\lor{Q}$ can be false is if \textit{both} $P$ is false and $Q$ is
            false. So the only case when $\neg(P\lor{Q})$ is true is when both
            $P$ is false and $Q$ is false. This is the
            \textit{De Morgan Law of Disjunction}.
            \begin{equation}
                \neg(P\lor{Q})=(\neg{P})\land(\neg{Q})
            \end{equation}
            Notice the duality between De Morgan's Law of Disjunction and
            De Morgan's Law of Conjunction. This similarity plays a big role
            in set theory, lattice theory, and mathematical logic.
            \par\hfill\par
            Let's conclude our discussion of logic with the
            \textit{principle of explosion}. It states that if there is a
            proposition $P$ that is both true \textit{and} false, then every
            proposition is true. For let $Q$ be any proposition. Since $P$ is
            true, $P\lor{Q}$ is true (see the truth table for \textit{or}). But
            $P$ is also false, so if $P\lor{Q}$ is true, then $Q$ must be true.
            Hence, $Q$ is true. This creates an extremely boring world where
            every proposition is both true and false, clearly something one
            would wish to avoid. The reason for this chapter is that the earlier
            methods of mathematics made it possible to produce a proposition
            $P$ that is both true and false, rendering the rest of mathematics
            useless. We will explore this more in the next subsection.
        \subsection{The Axioms of Set Theory}
            We are now in a position to state the axioms of set theory. There
            are 9 of them, but we will only really consider 5 of them. The
            others are mostly the concern of set theorists and logicians.
            \begin{axiom}[\textbf{Axiom of Extensionality}]
                If $A$ and $B$ are sets, then $A=B$ if and only if for all
                $a$ it is true that $a\in{A}$ if and only if $a\in{B}$.
            \end{axiom}
            This defines the \textit{equality} symbol $=$ for sets. We may
            cryptically write this as:
            \begin{equation}
                \forall_{A}\forall_{B}\Big(
                    \forall_{x}\big(x\in{A}\Leftrightarrow{x}\in{B}\big)
                        \Leftrightarrow\big(A=B)\Big)
            \end{equation}
            Let's see the axiom in action.
            \par\hfill\par\hrule
            \begin{example}
                Let $A=\{\,1,\,2,\,3\,\}$ and $B=\{\,3,\,2,\,1\}$. For any
                $a\in{A}$ either $a=1$, $a=2$, or $a=3$, by the definition of
                $A$. But $3\in{B}$, $2\in{B}$, and $1\in{B}$. Hence if
                $a\in{A}$, then $a\in{B}$. Similarly, for any $b\in{B}$,
                either $b=3$, $b=2$, or $b=1$. But $3\in{A}$, $2\in{A}$,
                and $1\in{A}$. Thus if $b\in{B}$, then $b\in{A}$. That is, for
                all $a$ it is true that $a\in{A}$ if and only if $a\in{B}$.
                Therefore, $A=B$.
            \end{example}
            \hrule
            \begin{example}
                Let us again consider $B=\{\,a,\,b,\,c\}$ and
                $A=\{\,a,\,a,\,b,\,c\,\}$. If $x\in{B}$, then either
                $x=a$, $x=b$, or $x=c$. But $a\in{A}$, $b\in{A}$, and $c\in{A}$.
                Hence if $x\in{B}$, then $x\in{A}$. Similarly, if $x\in{A}$ then
                either $x=a$, $x=b$, or $x=c$. Once again we conclude that if
                $x\in{A}$, then $x\in{B}$. Hence for all $x$ it is true that
                $x\in{A}$ if and only if $x\in{B}$. That is, $A=B$.
            \end{example}
            \hrule\par\hfill\par
            Equality of sets can also be described in terms of
            \textit{subsets}.
            \begin{definition}[\textbf{Subset}]
                A subset of a set $B$ is a set $A$ such that for all
                $x\in{A}$ it is true that $x\in{B}$. This is denoted
                $A\subseteq{B}$.
            \end{definition}
            \hrule
            \begin{example}
                Let $B=\{\,1,\,2,\,3\,\}$ and $A=\{\,1,\,2\,\}$. For every
                element $x\in{A}$ we also have $x\in{B}$. That is,
                $A$ is a subset of $B$ and we write $A\subseteq{B}$.
            \end{example}
            \hrule
            \begin{example}
                Consider the set of natural numbers $\mathbb{N}$ and the set of
                real numbers $\mathbb{R}$. Every natural number is also a
                real number. That is, $\mathbb{N}\subseteq\mathbb{R}$.
            \end{example}
            \hrule\par\hfill\par
            The notion of subset can be visualized by blobs in the plane.
            If we have two blobs in the plane, set $A$ and set $B$, and $A$ is
            a subset of $B$ if every point in $A$ is also contained in $B$.
            See Fig.~\ref{fig:subsets_in_plane}.
            \begin{figure}
                \centering
                \includegraphics{../../images/subset_001.pdf}
                \caption{Subsets in the Plane}
                \label{fig:subsets_in_plane}
            \end{figure}
            \par\hfill\par
            As stated the notion of subset can also be used to define equality.
            We've already defined equality by the axiom of extensionality,
            meaning our next statement is a \textit{theorem}, not a definition,
            and so it needs a proof.
            \begin{theorem}
                If $A$ and $B$ are sets, then $A=B$ if and only if
                $A\subseteq{B}$ and $B\subseteq{A}$.
            \end{theorem}
            \begin{proof}
                Since this is an \textit{if and only if} theorem there are two
                things to prove. First we prove that if $A=B$, then
                $A\subseteq{B}$ and $B\subseteq{A}$. By the axiom of
                extensionality, since $A=B$,
                for all $x\in{A}$ it is true that $x\in{B}$, and hence
                $A\subseteq{B}$. Similarly, for all $x\in{B}$ it is true that
                $x\in{A}$, and hence $B\subseteq{A}$. Now we must prove the
                other direction. That is, if $A\subseteq{B}$ and
                $B\subseteq{A}$, then $A=B$. Since $A\subseteq{B}$ and
                $B\subseteq{A}$, for all $x$, $x\in{A}$ if and only if
                $x\in{B}$. By the axiom of extensionality, $A=B$.
            \end{proof}
            The notion of subset is \textit{reflexive}, meaning for any set
            $A$, it is true that $A\subseteq{A}$. Let's prove this.
            \begin{theorem}
                If $A$ is a set, then $A\subseteq{A}$.
            \end{theorem}
            \begin{proof}
                For if $A$ is a set, then for all $x\in{A}$ it is true that
                $x\in{A}$, hence $A\subseteq{A}$.
            \end{proof}
            Equality is also reflexive, for all $x$ it is true that $x=x$.
            Subsets that aren't equal are given a name.
            \begin{definition}[\textbf{Proper Subset}]
                A proper subset of a set $B$ is a set $A$ such that
                $A\subseteq{B}$ and $A\ne{B}$. That is, $A\subseteq{B}$ and
                there exists $x\in{B}$ such that $x\notin{A}$. This is denoted
                $A\subsetneq{B}$.
            \end{definition}
            The next axiom describes how mathematicians usually build sets.
            \begin{axiom}[\textbf{Axiom Schema of Specification}]
                If $A$ is a set, and if $P$ is a predicate on $A$, then there
                is a set $B$ such that for all $x$ it is true that $x\in{B}$ if
                and only if $x\in{A}$ and $P(x)$ is true. This set is denoted:
                \begin{equation}
                    B=\{\,x\in{A}\;|\;P(x)\,\}
                \end{equation}
            \end{axiom}
            \hrule
            \begin{example}
                Consider the predicate $P(n)$ on $\mathbb{N}$ defined by
                \textit{n is a perfect square}. By the axiom schema of
                specification we can define
                \begin{equation}
                    B=\{\,n\in\mathbb{N}\;|\;P(n)\,\}
                \end{equation}
                This is the \textit{set of all perfect squares}. We can describe
                the set as follows:
                \begin{equation}
                    B=\{\,0,\,1,\,4,\,9,\,16,\,25,\,\dots\,\}
                \end{equation}
                This equation is \textit{not} rigorous, but the axiom schema of
                specification says this set does exist, and hence there's no
                real harm in writing $B$ this way.
            \end{example}
            \hrule\par\hfill\par
            Set theory is a very abstract branch of mathematics and you may find
            yourself asking \textit{what's the point}? In the times of Euler and
            Gauss set theory did not exist. Many mathematical arguments relied
            on intuition alone. This led to a crisis in the mathematical world
            in 1901 when Bertrand Russell (1872 - 1970 C.E.) discovered his
            paradox. Collections of objects were described by sentences, much
            like in English. For example,
            \textit{the collection of all cities on Earth},
            or \textit{the set of all students on a given college campus}. For
            the most part this is allowable, but certain sentences give rise to
            paradoxes. Bertrand Russell considered the following.
            \textrm{Is it possible for a set to contain itself?} Does the
            sentence \textit{the set of all sets} mean anything? The
            (now abandoned) axiom of \textit{unrestricted comprehension} said
            that a set can be formed by any sentence. In particular, you can
            write $P(x)$ to be \textit{x is a set}, and then consider
            \begin{equation}
                A=\{\,x\;|\;P(x)\,\}=\{\,x\;|\;x\textrm{ is a set.}\,\}
            \end{equation}
            Then $A$ would be the set of all sets. Since $A$ is a set,
            $A\in{A}$. Quite bizarre, indeed. Bertrand Russell then proposed
            the following. Let $Q(x)$ be the sentence
            \textit{x is set such that} $x\notin{x}$ and construct the new
            set $B$ defined by:
            \begin{equation}
                B=\{\,x\in{A}\;|\;Q(x)\,\}=\{\,x\in{A}\;|\;x\notin{x}\,\}
            \end{equation}
            He then asked \textit{is B an element of B}? Since $B$ is the
            set of all set that do not contain themselves, by definition
            $B\notin{B}$. But if $B\notin{B}$, then by definition $B\in{B}$.
            This is a serious contradiction and if allowed to exists, then by
            the principle of explosion \textit{every} proposition is true.
            Quite boring. The axiom schema of specification does not allow you
            to build a set by \textit{any} sentence. You must apply the
            sentence to some other set that you already know exists and then
            extract the appropriate elements. The axiom of regularity, which
            we will not describe, can be used to prove there is no
            \textit{set of all sets}, avoiding Russell's paradox.
            \par\hfill\par
            The \textit{axiom of union} is used so commonly that many authors
            forget to mention that it is an axiom. It states that there is a
            set called the \textit{union} of other sets.
            \begin{axiom}[\textbf{Axiom of Union}]
                If $\mathcal{O}$ is a set such that for all $A\in\mathcal{O}$ it
                is true that $A$ is a set, then there is a set
                $\bigcup\mathcal{O}$ called the \textit{union} over
                $\mathcal{O}$ such that for all $x$ it is true that
                $x\in\bigcup\mathcal{O}$ if and only if there is an
                $A\in\mathcal{U}$ such that $x\in{A}$.
            \end{axiom}
            The language of this axiom can be confusing at first. We are
            considering \textit{sets of sets} and trying to collect together
            all of the elements of these sets.
            \par\hfill\par\hrule
            \begin{example}
                As an example, let $\mathcal{O}$ be the set consisting of
                the sets $\{\,0\,\}$, $\{\,1\,\}$, $\{\,2\,\}$, and so on.
                That is:
                \begin{equation}
                    \mathcal{O}=
                    \big\{\,\{\,0\,\},\,\{\,1\,\},\,\{\,2\,\},\,\dots\,\big\}
                \end{equation}
                This is different from the set
                $\mathbb{N}=\{\,1,\,2,\,3,\,\dots\,\}$. The set $\mathbb{N}$
                consists of natural numbers whereas $\mathcal{O}$ consists of
                \textit{sets} of natural numbers. Consider the union set
                $\bigcup\mathcal{O}$. Let's show the following is true:
                \begin{equation}
                    \bigcup\mathcal{O}=\mathbb{N}
                \end{equation}
                The union set $\bigcup\mathcal{O}$ is the set of all elements
                $x$ such that there is an $A\in\mathcal{O}$ with $x\in{A}$.
                We then have that $1\in\bigcup\mathcal{O}$ since
                $1\in\{\,1\,\}$ and $\{\,1\,\}\in\mathcal{O}$. Similarly
                $2\in\bigcup\mathcal{O}$ since $2\in\{\,2\,\}$ and
                $\{\,2\,\}\in\mathcal{O}$. For every natural number
                $n$ the set $\{\,n\,\}$ is an element of $\mathcal{O}$ by
                definition, and hence $n\in\bigcup\mathcal{O}$. We therefore
                see the equality $\bigcup\mathcal{O}=\mathbb{N}$.
            \end{example}
            \hrule
            \begin{example}
                Given two real numbers $a$ and $b$, the set $(a,\,b)$ is defined
                by:
                \begin{equation}
                    (a,\,b)=\{\,x\in\mathbb{R}\;|\;a<x\textrm{ and }x<b\,\}
                \end{equation}
                So the set $(1,\,2)$ consists of all real numbers between 1 and
                2, exluding 1 and excluding 2. Also if $b<a$, the set
                $(a,\,b)$ is empty since it is impossible for
                $x$ to be greater than $a$ but less than $b$.
                Let $\mathcal{O}$ be the set:
                \begin{equation}
                    \mathcal{O}=\{\,(a,\,b)\;|\;
                        a\textrm{ and }b\textrm{ are rational.}\,\}
                \end{equation}
                A rational number is a number $x=p/q$ where $p$ and $q$ are
                integers and $q$ is not zero. That set of all rational numbers
                is denoted $\mathbb{Q}$ (for \textit{quotient}). Using this
                notation we can equally write $\mathcal{O}$ by:
                \begin{equation}
                    \mathcal{O}=\{\,(a,\,b)\;|\;a,b\in\mathbb{Q}\,\}
                \end{equation}
                $\mathcal{O}$ is a \textit{set of sets}. It is a set of subsets
                of the real line. Let us consider the union set
                $\bigcup\mathcal{O}$ and show that
                \begin{equation}
                    \bigcup\mathcal{O}=\mathbb{R}
                \end{equation}
                Given any real number $r$ there are integers $m$ and $n$
                such that $m<r$ and $r<n$. For example if $r=\pi$, take
                $m=3$ and $n=4$. Integers are, in particular, rational numbers
                since we can write $m=m/1$ and $n=n/1$. By definition
                $r\in(m,\,n)$ and $(n,\,m)\in\mathcal{O}$ and therefore
                $r\in\bigcup\mathcal{O}$. Since $r$ is an arbitrary real number
                we can see that $\bigcup\mathcal{O}=\mathbb{R}$.
            \end{example}
            \hrule\par\hfill\par
            This is used a lot in topology and analysis. Often the set
            $\mathcal{O}$ is a set consisting of two other sets
            $\mathcal{O}=\{\,A,\,B\,\}$. The union $\bigcup\mathcal{O}$ is then
            rewritten as $A\cup{B}$.
            \begin{definition}[\textbf{Union of Two Sets}]
                The union of two sets $A$ and $B$ is the set $A\cup{B}$ defined
                by:
                \begin{equation}
                    A\cup{B}=\{\,x\;|\;x\in{A}\textrm{ or }x\in{B}\,\}
                \end{equation}
            \end{definition}
            Again, appealing to our newfound mathematical use of the word
            \textit{or}, $x\in{A}$ or $x\in{B}$ allows for $x\in{A}$, it allows
            for $x\in{B}$, and it allows for $x$ to be in both.
            \par\hfill\par\hrule
            \begin{example}
                Let $A=\{\,1,\,2,\,3\,\}$ and $B=\{\,a,\,b,\,c\,\}$. The
                union set is
                \begin{equation}
                    A\cup{B}=\{\,1,\,2,\,3,\,a,\,b,\,c\,\}
                \end{equation}
                This is the set of all elements in either $A$ or $B$ or both.
            \end{example}
            \hrule
            \begin{example}
                Let $A=\{\,1,\,2,\,3\,\}$ and $B=\{\,3,\,4,\,5\,\}$. The union
                set is
                \begin{equation}
                    A\cup{B}=\{\,1,\,2,\,3,\,4,\,5\,\}
                \end{equation}
                It does not matter that $3$ occurs both in $A$ and $B$. Sets
                have no notion of repetition so it would be redundant to
                include $3$ twice.
            \end{example}
            \hrule\par\hfill\par
            The existence of the intersection of a collection of sets does
            not need an axiom.
            \begin{theorem}
                If $\mathcal{O}$ is a set such that for all $A\in\mathcal{O}$
                it is true that $A$ is a set, then there is a set
                $\bigcap\mathcal{O}$ such that for all $x$ it is true that
                $x\in\bigcap\mathcal{O}$ if and only if $x\in{A}$ for all
                $A\in\mathcal{O}$.
            \end{theorem}
            \begin{proof}
                Consider the sentence \textit{x is in A for all A in}
                $\mathcal{O}$. We can apply this sentence to the union set
                $\bigcup\mathcal{O}$. Define $\bigcap\mathcal{O}$ by this
                sentence using the axiom schema of specification:
                \begin{equation}
                    \bigcap\mathcal{O}=\{\,x\in\bigcup\mathcal{O}\;|\;
                        x\textrm{ is in }A
                        \textit{ for all }A\in\mathcal{O}\,\}
                \end{equation}
                Then for all $x$, $x\in\bigcap\mathcal{O}$ if and only if
                $x\in{A}$ for all $A\in\mathcal{O}$.
            \end{proof}
            This set is called the \textit{intersection} over $\mathcal{O}$.
            Like unions it is often applied to a two element set
            $\mathcal{O}=\{\,A,\,B\,\}$ but it is very common in topology,
            analysis, and set theory to consider the more general type of
            intersection. Indeed, several concepts in topology such as
            \textit{closure} and \textit{generated topology} require an
            understanding of the general notion of intersection. Let's spell
            this out with examples.
            \par\hfill\par\hrule
            \begin{example}
                Define Given $x\in\mathbb{R}$ and $r>0$, define $B_{r}(x)$ to
                be:
                \begin{equation}
                    B_{r}(x)=\{\,y\in\mathbb{R}\;|\;|y-x|<r\,\}
                \end{equation}
                The absolute value function acts as a
                \textit{distance} function, and so $B_{r}(x)$ is the set of
                all real numbers $y$ that are closer than $r$ away from $x$.
                Note that, for any real number $r>0$, we have that
                $x\in{B}_{r}(x)$ since $x$ is \textit{zero} away from itself.
                Define $\mathcal{O}$ by
                \begin{equation}
                    \mathcal{O}=\{\,B_{r}(x)\;|\;r\in\mathbb{R}\textrm{ and }
                        r>0\,\}
                \end{equation}
                Since $x\in{B}_{r}(x)$ for all $r>0$, we can see that
                $x\in\bigcap\mathcal{O}$. Is there anything else in this
                intersection? Suppore $y\ne{x}$ is contained in the
                intersection. Since $y\ne{x}$, $|x-y|$ is positive. Denote this
                as $r=|x-y|$. Consider the set $B_{\frac{r}{2}}(x)$. Since
                $|x-y|=r$, and $r>\frac{r}{2}$, we have that
                $y\notin{B}_{\frac{r}{2}}(x)$. But
                $B_{\frac{r}{2}}(x)\in\mathcal{O}$ by definition and thus it
                is not true that $y\in\bigcap\mathcal{O}$. We conclude that
                $\bigcap\mathcal{O}=\{\,x\,\}$.
            \end{example}
            \hrule
            \begin{example}
                For each $N\in\mathbb{N}$ let $A_{N}\subseteq\mathbb{N}$ be
                defined by:
                \begin{equation}
                    A_{N}=\{\,n\in\mathbb{N}\;|\;n>N\,\}
                \end{equation}
                As a side note, the fact that we are allowed to define
                \textit{infinitely} many such sets $A_{N}$ (one for each
                integer) is possible in set theory. We'll simply take this
                fact for granted. Now, let $\mathcal{O}$ be the collection
                of all of these sets:
                \begin{equation}
                    \mathcal{O}=\{\,A_{N}\;|\;N\in\mathbb{N}\,\}
                \end{equation}
                For each $N\in\mathbb{N}$, the set $A_{N}$ is infinite.
                How big is the intersection? That is, what is the set
                $\bigcap\mathcal{O}$? Suppose there is some
                $n\in\bigcap\mathcal{O}$. Then for all $N\in\mathbb{N}$,
                $n\in{A}_{N}$, by the definition of intersection.
                Let $N=n+1$. Then $A_{N}$ is the set of all integers
                greater than $n+1$, and hence $n\notin{A}_{N}$, a
                contradiction. There are therefore no integers in the
                set $\bigcap\mathcal{O}$. The intersection is empty.
            \end{example}
            \hrule\par\hfill\par
            \begin{definition}[\textbf{Intersection of Two Sets}]
                The intersection of two sets $A$ and $B$ is the set
                $A\cap{B}$ defined by:
                \begin{equation}
                    A\cap{B}=\{\,x\;|\;x\in{A}\textrm{ and }x\in{B}\,\}
                \end{equation}
            \end{definition}
            Are use of the word \textit{and} here means that
            $x\in{A}\cap{B}$ if and only if $x\in{A}$ and $x\in{B}$
            simultaneously.
            \par\hfill\par\hrule
            \begin{example}
                Let $A=\{\,1,\,2,\,3\,\}$ and $B=\{\,a,\,b,\,c\,\}$. The
                intersection is:
                \begin{equation}
                    A\cap{B}=\emptyset
                \end{equation}
                This is because $A$ and $B$ have no elements in common.
            \end{example}
            \hrule
            \begin{example}
                Let $A=\{\,1,\,2,\,3\,\}$ and $B=\{3,\,4,\,5\,\}$. The
                intersection is:
                \begin{equation}
                    A\cap{B}=\{\,3\,\}
                \end{equation}
                This is true since the only element common to both $A$ and
                $B$ is $3$, so $3$ is the only member of the intersection.
            \end{example}
            \hrule\par\hfill\par
            There are three more axioms we will use, and they all have names:
            \textit{infinity}, \textit{power set}, and \textit{choice}.
            The axiom of choice is perhaps the most difficult to understand,
            so it's discussion is delayed until the end of this chapter. The
            axiom of infinity is perhaps controversial, but we adopt it anyways.
            \begin{axiom}[\textbf{The Axiom of Infinity}]
                The set $\mathbb{N}$ consisting of all natural numbers exists.
            \end{axiom}
            This is called the axiom of infinity since it says there is a set
            with infinitely many elements. The controversy stems from the
            following question:
            \textit{do you believe there exist infinite sets?} Topology is
            painfully boring without infinite sets (any finite Hausdorff
            topological space is a discrete space. This sentence will be made
            clear in later chapters), so for this book we will
            say that infinite sets exist. The next axiom is the axiom of the
            power set.
            \begin{axiom}[\textbf{The Axiom of the Power Set}]
                If $X$ is a set, then there exists a set
                $\mathcal{P}(X)$ called the \textit{power set} of $A$ such that
                for all $A$ it is true that $A\in\mathcal{P}(X)$ if and only if
                $A\subseteq{X}$.
            \end{axiom}
            There are a few other axioms (replacement, pairing, and regularity)
            that are worthy of study, but will not be needed for us. Let us
            divert our attention of power sets.
            \par\hfill\par\hrule
            \begin{example}
                Let $A=\emptyset$. What is the power set of $A$? For any set
                $X$ we know that $X\subseteq{X}$, so
                $\emptyset\in\mathcal{P}(A)$. There can be no other subsets of
                $A$ since the empty set contains no elements. We arrive at:
                \begin{equation}
                    \mathcal{P}(\emptyset)=\{\,\emptyset\,\}
                \end{equation}
                Note the $\mathcal{P}(\emptyset)$ is \textit{not} the empty set.
                It is the set that contains the empty set.
            \end{example}
            \hrule\par\hfill\par
            For some, this example is the start of the construction of the
            natural numbers, the set guaranteed to exist by the axiom of
            infinity. We label $0=\emptyset$. We proceed and denote
            $1=\{\,\emptyset\,\}$. Next we write
            $2=\{\emptyset,\,\{\,\emptyset\,\}\,\}$. Perhaps more suggestively,
            we are writing:
            \begin{align}
                0&=\emptyset\\
                1&=\{\,0\,\}\\
                2&=\{\,0,\,1\,\}\\
                3&=\{\,0,\,1,\,2\,\}
            \end{align}
            The \textit{integer} $n$ is really the set of all integers from $0$
            to $n-1$. This is useful for two reasons. We can write
            $m<n$ if and only if $m\in{n}$, allowing us to define
            \textit{order} via set theory. It also allows us to define
            \textit{size}. Something with size $n$ is just something with as
            many elements as the integer $n$. The notion of ordering and size
            will play a big role in topology. We'll return to these ideas at
            the end of this chapter.
            \par\hfill\par\hrule
            \begin{example}
                Let $A=\{\,1,\,2,\,3\,\}$. The power set consists of $8$
                elements:
                \begin{equation}
                    \mathcal{P}(A)=\big\{\,\emptyset,\,\{\,1\,\},\,
                        \{\,2\,\},\,\{\,3\,\},\,
                        \{\,1,\,2\,\},\,\{\,1,\,3\,\},\,\{\,2,\,3\,\},\,
                        \{\,1,\,2,\,3\,\}
                    \big\}
                \end{equation}
            \end{example}
            \hrule\par\hfill\par
            We've seen that a set with zero elements has a power set with 1
            element, and that a set with 3 elements has a power set with 8
            elements. The pattern is that a set with $n$ elements has
            $2^{n}$ elements. This can be proved via the principle of
            \textit{mathematical induction}. For those unfamiliar with the
            idea, it is worth trying to prove this claim yourself.
        \subsection{The Algebra of Sets}
            There are four operations in set theory that give the study a very
            algebraic nature. We've seen two of them: intersection and union.
            The other two are \textit{set difference} and
            \textit{symmetric difference}. Both of these operations act like
            subtraction, but for sets.
    \section{Functions}

    \newpage
    \bibliographystyle{plain}
    \bibliography{bib.bib}
    \newpage

    % Print glossary and notation pages.
    \printnoidxglossary[type=notation, style=mcolindex, sort=def]
    \clearpage
    \printnoidxglossary[style=longpara]

    % Print the index.
    \clearpage
    \printindex
\end{document}
