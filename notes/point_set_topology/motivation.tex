%-----------------------------------LICENSE------------------------------------%
%   This file is part of Mathematics-and-Physics.                              %
%                                                                              %
%   Mathematics-and-Physics is free software: you can redistribute it and/or   %
%   modify it under the terms of the GNU General Public License as             %
%   published by the Free Software Foundation, either version 3 of the         %
%   License, or (at your option) any later version.                            %
%                                                                              %
%   Mathematics-and-Physics is distributed in the hope that it will be useful, %
%   but WITHOUT ANY WARRANTY; without even the implied warranty of             %
%   MERCHANTABILITY or FITNESS FOR A PARTICULAR PURPOSE.  See the              %
%   GNU General Public License for more details.                               %
%                                                                              %
%   You should have received a copy of the GNU General Public License along    %
%   with Mathematics-and-Physics.  If not, see <https://www.gnu.org/licenses/>.%
%------------------------------------------------------------------------------%
\chapter{Motivation}
    Topology can be a strange topic to those unfamiliar with its history.
    Intuitively it is the study of raw uncooked dough. If you have a lump of
    dough then you can stretch it and squeeze it without really changing the
    dough. But if you poke your finger through it, you now have a lump of dough
    with a hole in it. To be more precise, topology is the study of continuously
    deforming things. That is, stretching and squeezing is fine, but tearing,
    cutting, closing holes, puncturing holes, etc., is not allowed. A question
    of the $20^{\textrm{th}}$ and $21^{\textrm{st}}$ century asks one to
    classify various \textit{topological spaces} under such constraints.
    \par\hfill\par
    For a much simpler question, consider the sphere and the torus in
    Fig.~\ref{fig:sphere_and_torus}. The sphere $\mathbb{S}^{2}$ is the set of
    all points in $\mathbb{R}^{3}$ that are of unit length. It is not a
    \textit{solid} sphere like the Earth, we only consider the shell of it.
    Similarly for the torus. It is not the entirety of a donut, just the outter
    shell. With this it is hoped that it is clear that these are
    \textit{2 dimensional} objects. On the sphere you may walk north-south or
    east-west (no \textit{digging} down is allowed, and no jumping). Similarly
    on the torus at any point there are two directions you may move. Now,
    are they different?
    \begin{figure}[H]
        \centering
        \includegraphics{../../images/torus_next_to_sphere.pdf}
        \caption{A Sphere and a Torus}
        \label{fig:sphere_and_torus}
    \end{figure}
    You may think that in order to go from the sphere to the torus you need to
    introduce a hole, which is not allowed, and then conclude that the torus
    and the sphere are different objects. This thinking is correct,
    but how can one \textit{prove} with mathematical rigor that these are
    distinct things? How does one even properly phrase the question? The
    framework that is needed is \textit{topology}, the study of
    \textit{topological spaces}. At the very end of this book we will prove that
    the torus and the sphere are different topological spaces.
    \par\hfill\par
    Topology was not initiated with the concern for the classification of
    surfaces. This work came in the $19^{\textrm{th}}$ century. What is likely
    the first topological problem was solved by Leonhard Euler
    (1707 - 1783 C.E.) in 1736 C.E. \cite{LeonhardEulerBridgesOfKonigsberg}. In
    the city of K\"{o}nigsberg there were 7 bridges connecting 4 plots of land
    (See Fig.~\ref{fig:bridges_of_konigsberg_001}). Examine the figure for a
    moment and try to trace your fingers over all seven bridges while crossing
    each bridge exactly once. That is, try to walk around K\"{o}nigsberg and
    cross each bridge once and only once. Is this possible?
    \begin{figure}
        \centering
        \includegraphics{../../images/bridges_of_konigsberg_001.pdf}
        \caption{The Seven Bridges of K\"{o}nigsberg}
        \label{fig:bridges_of_konigsberg_001}
    \end{figure}
    Leonhard Euler proposed the following. Disregard the buildings, the grass,
    the water, and the entire city. The only thing relevant to the problem are
    the bridges and how they connect the land. Replace each of the four plots
    of land with dots and replace the seven bridges with curves or
    line segments connecting the appropriate dots. The result is
    Fig.~\ref{fig:bridges_of_konigsberg_002}. This drawing is what is known as
    a \textit{graph}, which is a collection of dots called
    \textit{vertices} and curves connecting the dots called
    \textit{edges}. The seven bridges of K\"{o}nigsberg problem asks one to
    travel along all of the edges of the graph in
    Fig.~\ref{fig:bridges_of_konigsberg_002} without repeating.
    \par\hfill\par
    \begin{figure}
        \centering
        \includegraphics{../../images/bridges_of_konigsberg_002.pdf}
        \caption{The Seven Bridges of K\"{o}nigsberg as a Graph}
        \label{fig:bridges_of_konigsberg_002}
    \end{figure}
    The problem is impossible and the original solution is from Euler. We prove
    this by \textit{contradiction}. We'll suppose it is possible, discover an
    impossibility, and conclude there is no solution. So, suppose
    it is possible and that you've found some path through the graph that
    accomplishes the task. Euler noted that every time you enter a vertex from
    an edge, you must leave the vertex through a different edge,
    otherwise you'll have traversed the same edge twice. This is true for every
    vertex except for the start and end vertices along your path. This means for
    all but at most two vertices the number of edges connected to a vertex is
    even. The number of edges connected to a vertex is the \textit{degree} of
    the vertex, so what we've shown is that that every vertex in the graph,
    with possibly two exceptions, has even degree. If we examine
    Fig.~\ref{fig:bridges_of_konigsberg_003} we see that there are three
    vertices with an odd number of edges connected to them, and
    one vertex with an even number, which is a contradiction. This shows it is
    impossible to cross all bridges once and only once.
    \begin{figure}[H]
        \centering
        \includegraphics{../../images/bridges_of_konigsberg_003.pdf}
        \caption{The Degrees of the Seven Bridges of K\"{o}nigsberg Graph}
        \label{fig:bridges_of_konigsberg_003}
    \end{figure}
    The proof is not quite topological in nature, but the simplification is.
    The edges in Fig.~\ref{fig:bridges_of_konigsberg_002} do not have to be
    drawn like that. You may move them around as you wish as long as the same
    vertices remain connected. You may also adjust the locations of the
    vertices. This notion of \textit{continuously deforming} the space we're
    working with is the heart of topology.
    \par\hfill\par
    \begin{figure}
        \centering
        \begin{subfigure}[b]{0.49\textwidth}
            \includegraphics{../../images/convex_region_001.pdf}
            \subcaption{A Convex Set}
        \end{subfigure}
        \hfill
        \begin{subfigure}[b]{0.49\textwidth}
            \includegraphics{../../images/non_convex_region_001.pdf}
            \subcaption{A Non-Convex Set}
        \end{subfigure}
        \caption{Convex and Non-Convex Sets}
        \label{fig:convex_and_non_convex}
    \end{figure}
    Twenty years after Euler's solution to the Seven Bridges of K\"{o}nigsberg
    problem he discovered one of the most celebrated results of topology. He
    discovered the existence of the \textit{Euler characteristic}. Consider a
    \textit{convex polyhedra}. A polyhedra is called convex if for every
    two points on it the line segment joining them lies entirely on the inside.
    A similar definition holds for arbitrary subsets of $\mathbb{R}^{n}$.
    Fig~\ref{fig:convex_and_non_convex} depicts a convex and a non-convex
    subset of the plane.
    \par\hfill\par
    Now, consider a convex polyhedra. Take a tetrahedron as a simple
    example (Fig.~\ref{fig:tetrahedron_001}). A polyhedra is a collection of
    points, line segments, and faces. For the tetrahedron there are 4 vertices,
    6 edges, and 4 faces. Think of vertices as the
    \textit{zero-dimensional} parts of the tetrahedron, edges as the
    \textit{one-dimensional} components, and faces are \textit{two-dimensional}.
    \begin{figure}[H]
        \centering
        \includegraphics{../../images/tetrahedron_001.pdf}
        \caption{A Tetrahedron}
        \label{fig:tetrahedron_001}
    \end{figure}
    Euler noticed that if you take the \textit{alternating sum} over the number
    of components in each dimension you always get 2. That is, the number of
    vertices minus the number of edges plus the number of faces is 2. For the
    tetrahedron we get $4-6+4=2$, as claimed. The other platonic solids are
    verified, for example the cube is 8 vertices, 12 edges, and 6 faces.
    $8-12+6=2$. In Euler's 1758 paper he proves that for \textit{any} convex
    polyhedra this formula is true \cite{LeonhardEulerPolyhedraFormula}. This
    is sometimes called \textit{Euler's Polyhedral Formula}. Letting $V$ be the
    number of vertices, $E$ the number of edges, and $F$ the number of faces,
    we have:
    \begin{equation}
        V-E+F=2
    \end{equation}
    What makes this topological is that the only thing that matters is that the
    polyhedron can be continuously deformed into a sphere, the convexity is not
    needed. If we had a polyhedron that was topologically the same as a torus
    we'd get a different answer.
    Fig.~\ref{fig:polyhedral_torus_001} depicts a polyhedral torus where the
    faces are slightly transparent so that we may easily count $E$, $V$, and
    $F$. Examing we say $V=16$, $E=32$, and $F=16$. The alternating sum is
    then:
    \begin{equation}
        V-E+F=16-32+16=0
    \end{equation}
    This number is different from the tetrahedron. The reason being that the
    torus cannot be continuously deformed into a sphere. The value of
    $V-E+F$ is called the \textit{Euler characteristic}. The actual number
    depends only on the \textit{topology} of the polyhedra.
    \begin{figure}
        \centering
        \includegraphics{../../images/polyhedral_torus_001.pdf}
        \caption{A Polyhedral Torus}
        \label{fig:polyhedral_torus_001}
    \end{figure}
    \par\hfill\par
    Moving on, one of the direct motivations of topology was work done in the
    $19^{\textrm{th}}$ century on complex analysis. For simplicity we will
    think only of curves in the plane $\mathbb{R}^{2}$. If you have a
    differentiable curve $\gamma:[0,1]\rightarrow\mathbb{R}^{2}$ and a
    continuous function
    $\mathbf{F}:\mathbb{R}^{2}\rightarrow\mathbb{R}^{2}$, you can compute the
    \textit{line integral} as follows:
    \begin{equation}
        \int_{\gamma}\mathbf{F}(\mathbf{r})\cdot\textrm{d}\mathbf{r}
            =\int_{0}^{1}\mathbf{F}\big(\gamma(t)\big)\cdot
                \dot{\gamma}(t)\;\textrm{d}t
    \end{equation}
    Where $\dot{\gamma}(t)$ denotes the derivative of $\gamma$ at $t$.
    The right hand side of the equation is now of the form of something seen
    in an integral calculus course. Line integrals in complex analysis are
    defined similarly. If we have a differentiable curve
    $\gamma:[0,1]\rightarrow\mathbb{C}$ and a differentiable function
    $f:\mathbb{C}\rightarrow\mathbb{C}$, we may define:
    \begin{equation}
        \int_{\gamma}f(z)\;\textrm{d}z
            =\int_{0}^{1}f\big(\gamma(t)\big)\dot{\gamma}(t)\;\textrm{d}t
    \end{equation}
    In the case that the curve $\gamma$ starts where it ends,
    $\gamma(0)=\gamma(1)$, mathematicians like to decorate the integral symbol
    with a circle. We write:
    \begin{equation}
        \oint_{\gamma}f(z)\;\textrm{d}z=
            \int_{0}^{1}f\big(\gamma(t)\big)\dot{\gamma}(t)\;\textrm{d}t
    \end{equation}
    The circle simply emphasizes that $\gamma$ is a \textit{closed} curve.
    Cauchy's integral formula states that if $\gamma$ is a closed curve that
    does not touch the origin (i.e. $\gamma(t)\ne{0}$ for all $t$), then:
    \begin{equation}
        f(0)=\frac{w}{2\pi{i}}\oint_{\gamma}\frac{f(z)}{z}\;\textrm{d}z
    \end{equation}
    where $w$ is the \textit{winding number}, the number of times the curve
    $\gamma$ wraps around the origin counter-clockwise
    (See Fig.~\ref{fig:winding_number_001}). The only part of the curve
    $\gamma$ that matters is the winding number. The winding number is a value
    that is preserved under \textit{continuous deformation}. Try to convince
    yourself of this. Take a look at the far-right curve in
    Fig~\ref{fig:winding_number_001}, the one with winding number 0. The curve
    does not enclose the origin. We are allowed to move the curve in any way
    we please, but it can \textit{never} pass through the origin
    (In the formula we are dividing by $z$. Passing through the origin equates
    to division by zero). It seems impossible that we would be able to move this
    curve in a way that it then encircles the origin. It is indeed, but again,
    how does one prove this?
    \begin{figure}
        \centering
        \includegraphics{../../images/winding_number_001.pdf}
        \caption{Winding Number of Closed Curves}
        \label{fig:winding_number_001}
    \end{figure}
    \par\hfill\par
    Lastly, one of the strong motivating factors of point-set topology was
    the geometry of the $19^{\textrm{th}}$ and early
    $20^{\textrm{th}}$ centuries. Mathematicians like
    Bernhard Riemann (1826 - 1866 C.E.) and Carl Friedrich Gauss
    (1777 - 1855 C.E.) began pondering generalizations of smooth surfaces,
    like spheres and torii, in higher dimensions. Many of the ideas were based
    on intuition. Spaces that locally \textit{look like} Euclidean spaces
    $\mathbb{R}^{n}$. The sphere is like this. If you stand on the Earth and
    only look at your immediate surroundings, it's easy to convince yourself
    that \textit{locally} you're standing on a flat plane. All smooth surfaces
    in $\mathbb{R}^{3}$ are like this. The developments of the
    $19^{\textrm{th}}$ century sought a framework for higher dimensional
    analogues. At the turn of the $20^{\textrm{th}}$ century, the physicist
    Albert Einstein (1879 - 1955 C.E.) used these ideas to formulate his
    general theory of relativity. That the universe can be modelled as a
    four dimensional \textit{spacetime} satisfying certain properties. The
    foundation of all of this is the idea of a \textit{topological manifold},
    which will be explored in later chapters.
