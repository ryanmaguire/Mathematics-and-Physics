\documentclass{article}
\usepackage{mathtools, esint, mathrsfs} % amsmath and integrals.
\usepackage{amsthm, amsfonts, amssymb}  % Fonts and theorems.
\usepackage{hyperref}                   % Hyperlinks.

% Colors for hyperref.
\hypersetup{
    colorlinks=true,
    linkcolor=blue,
    filecolor=magenta,
    urlcolor=Cerulean,
    citecolor=SkyBlue
}

\title{Computational Topology}
\author{Ryan Maguire}
\date{Fall 2021}
\setlength{\parindent}{0em}
\setlength{\parskip}{0em}

\newtheoremstyle{normal}
    {\topsep}               % Amount of space above the theorem.
    {\topsep}               % Amount of space below the theorem.
    {}                      % Font used for body of theorem.
    {}                      % Measure of space to indent.
    {\bfseries}             % Font of the header of the theorem.
    {}                      % Punctuation between head and body.
    {.5em}                  % Space after theorem head.
    {}

\theoremstyle{plain}
\newtheorem{theorem}{Theorem}[section]
\theoremstyle{normal}
\newtheorem{example}{Example}[section]

% TODO
%   PL Structures
%   transversality
%   surgery
%   Morse theory
%   intersection theory
%   cobordism
%   bundles
%   characteristic classes
%   geometric topology
%   Smale, sphere inversion

\begin{document}
    \maketitle
    These notes come from my personal study of topological manifolds. Many of
    the concepts come from John M. Lee's Introduction to Topological
    Manifolds. Any errors in these notes are my own.
    \tableofcontents
    \section{Intuition}
        Topological manifolds are generalizations of the idea of curves and
        surfaces, allowing them to be \textit{higher dimensional}. The
        dimension of a manifold is, roughly speaking, the number of parameters
        needed in specifying a point. We try to model manifolds after Euclidean
        space $\mathbb{R}^{n}$ and say that manifolds are topological spaces
        that look \textit{locally} like $\mathbb{R}^{n}$, though this phrasing
        lacks some rigor. That said, some of easiest examples of manifolds are
        curves in $\mathbb{R}^{2}$ or $\mathbb{R}^{3}$, and surfaces in
        $\mathbb{R}^{3}$. Curves are examples of 1 dimensional manifolds, and
        surfaces are examples of 2 dimensional manifolds. Even easier is the
        real line $\mathbb{R}$ and the Euclidean plane $\mathbb{R}^{2}$, which
        are both manifolds as well (They're not just \textit{locally} like
        $\mathbb{R}^{n}$ for some $n$, the \textit{are}
        $\mathbb{R}^{n}$ for some $n$!). In general,
        $\mathbb{R}^{n}$ is an $n$ dimensional manifold, but these are some of
        the more boring examples.
        \par\hfill\par
        1 dimensional manifolds can be specified by a single variable. For
        example, if $\alpha:\mathbb{R}\rightarrow\mathbb{R}^{n}$ is a
        \textit{nice} continuous curve (nice meaning there are no points in
        $\mathbb{R}^{n}$ that are mapped to by two or more points in
        $\mathbb{R}$ by $\alpha$, including limits at $\pm\infty$), then this
        curve is a manifold and we can represent the points by a single
        variable. That is, $\alpha(t)$ specifies all of the points in the
        manifold as $t$ varies over $\mathbb{R}$. 2 dimensional manifolds like
        the plane and sphere can be specified by two parameters. Points in the
        plane can be determined by the Cartesian coordinates $(x,y)$, and
        points on the sphere can be specified by their azimuth and zenith
        angles $(\phi,\theta)$.
        \par\hfill\par
        Higher dimensional manifolds are hard to visualize, but not so hard to
        describe. The $n$ dimensional sphere, which lives as a subset of
        $\mathbb{R}^{n+1}$, can be described as the set of all points
        satisfying the following equation:
        \begin{equation}
            \sum_{k=0}^{n-1}x_{k}^{2}=1
        \end{equation}
        For $n=2$ we get $x^{2}+y^{2}=1$, which describes the unit circle, and
        for $n=3$ we get $x^{2}+y^{2}+z^{2}=1$, which describes the unit
        sphere. The next dimension up,
        $x_{0}^{2}+x_{1}^{2}+x_{2}^{2}+x_{3}^{2}=1$, describes a 3 dimensional
        object. If we fix $x_{0}$, $x_{1}$, and $x_{2}$, then the final
        variable $x_{3}$ is determined by these and it must be:
        \begin{equation}
            x_{4}=\sqrt{1-x_{1}^{2}-x_{2}^{2}-x_{3}^{2}}
        \end{equation}
        So we have 3 \textit{degrees of freedom} for our 3 dimensional
        manifold $\mathbb{S}^{3}$.
\end{document}
