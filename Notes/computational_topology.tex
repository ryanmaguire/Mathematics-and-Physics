\documentclass{article}
\usepackage{mathtools, esint, mathrsfs} % amsmath and integrals.
\usepackage{amsthm, amsfonts, amssymb}  % Fonts and theorems.
\usepackage{hyperref}                   % Hyperlinks.

% Colors for hyperref.
\hypersetup{
    colorlinks=true,
    linkcolor=blue,
    filecolor=magenta,
    urlcolor=Cerulean,
    citecolor=SkyBlue
}

\title{Computational Topology}
\author{Ryan Maguire}
\date{Fall 2021}
\setlength{\parindent}{0em}
\setlength{\parskip}{0em}

\newtheoremstyle{normal}
    {\topsep}               % Amount of space above the theorem.
    {\topsep}               % Amount of space below the theorem.
    {}                      % Font used for body of theorem.
    {}                      % Measure of space to indent.
    {\bfseries}             % Font of the header of the theorem.
    {}                      % Punctuation between head and body.
    {.5em}                  % Space after theorem head.
    {}

\theoremstyle{plain}
\newtheorem{theorem}{Theorem}[section]
\theoremstyle{normal}
\newtheorem{example}{Example}[section]

\begin{document}
    \maketitle
    These notes come from the homework, lectures, and project work for
    the computational topology course offered by the CS department at
    Dartmouth College during the Fall 2021 semester. Any errors contained
    in these notes are my own.
    \tableofcontents
    \section{Week 1}
        \subsection{Fun with Groups}
            A group is an ordered pair $(G,*)$ where $G$ is a set and $*$ is a
            \textit{binary operation}, a function $*:G\times{G}\rightarrow{G}$
            such that:
            \begin{enumerate}
                \item For all $a,b,c\in{G}$ it is true that $a*(b*c)=(a*b)*c$
                      (Associativty).
                \item There exists an element $e\in{G}$ such that, for all
                      $a\in{G}$ it is true that $a*e=e*a=a$
                      (Existence of Identity).
                \item For all $a\in{G}$ there exists $b\in{G}$ such that
                      $a*b=b*a=e$. (Existence of Inverses).
            \end{enumerate}
            The given $a,b\in{G}$ such that $a*b=e$, the element $b$ is often
            labeled $b=a^{-1}$.
            \begin{example}
                Letting $\mathbb{R}$ denote the real numbers,
                $\mathbb{R}^{+}$ denote the set of all positive real numbers,
                and $+$ and $\cdot$ denoting addition and multiplication of
                real numbers, respectively, $(\mathbb{R},+)$ and
                $(\mathbb{R}^{+},\cdot)$ are groups. This is because addition
                and multiplication are both associative:
                \begin{align}
                    a+(b+c)&=(a+b)+c\\
                    a\cdot(b\cdot{c})&=(a\cdot{b})\cdot{c}
                \end{align}
                $(\mathbb{R},+)$ has 0 as the identity since $a+0=0+a=a$ for
                any real number $a$. $(\mathbb{R}^{+},\cdot)$ has 1 as the
                identity since $a\cdot{1}=1\cdot{a}=a$. Lastly, inverses always
                exist. Given $a\in\mathbb{R}$, the negative of $a$, denoted
                $-a$, acts as the additive inverse since $a+(-a)=0$. Given
                $a\in\mathbb{R}^{+}$, the reciprocal $1/a$ is the
                multiplicative inverse since $a\cdot(1/a)=1$. Since all three
                properties are satisfied, both $(\mathbb{R},+)$ and
                $(\mathbb{R}^{+},\cdot)$ are groups.
            \end{example}
            \begin{example}
                \label{ex:sym_diff_is_group}%
                If $A$ is a set, if $G=\mathcal{P}(A)$ is the power set of $A$,
                and if $\ominus$ is the \textit{symmetric difference} of sets,
                then $(G,\ominus)$ is a group. The symmetric difference of two
                sets $\mathcal{U}$ and $\mathcal{V}$ is the set all of elements
                that are in either $\mathcal{U}$ or $\mathcal{V}$, but not
                both. In set theory notation we can write:
                \begin{equation}
                    \mathcal{U}\ominus\mathcal{V}=
                        (\mathcal{U}\cup\mathcal{V})\setminus
                        (\mathcal{U}\cap\mathcal{V})
                \end{equation}
                The identity element is the empty set $\emptyset=\{\}$ since:
                \begin{equation}
                    \mathcal{U}\ominus\emptyset
                        =(\mathcal{U}\cup\emptyset)\setminus
                            (\mathcal{U}\cap\emptyset)
                        =\mathcal{U}\setminus\emptyset
                        =\mathcal{U}
                \end{equation}
                And similarly for $\emptyset\ominus\mathcal{U}$.
                It is associative, but this requires a little bit of work.
                The inverse of $\mathcal{U}\in{G}$ is $\mathcal{U}$ since:
                \begin{align}
                    \mathcal{U}\ominus\mathcal{U}
                    &=(\mathcal{U}\cup\mathcal{U})\setminus
                      (\mathcal{U}\cap\mathcal{U})\\
                    &=\mathcal{U}\setminus\mathcal{U}\\
                    &=\emptyset
                \end{align}
                Since this structure satisfies the three properties,
                $(\mathcal{P}(A),\ominus)$ is a group.
            \end{example}
            A group homomorphism between two groups $(G,*)$ and $(H,\circ)$ is
            a function $f:G\rightarrow{H}$ such that, for all $a,b\in{G}$, it
            is true that:
            \begin{equation}
                f(a*b)=f(a)\circ{f}(b)
            \end{equation}
            That is, $f$ \textit{respects} the group operations.
            \begin{example}
                If $G=\mathbb{R}$, the set of all real numbers,
                $H=\mathbb{R}^{+}$, the set of all positive real numbers, and
                if $f:G\rightarrow{H}$ is the function $f(x)=\exp(x)$, then
                $f$ is a homomorphism with respect to the group structures
                $(\mathbb{R},+)$ and $(\mathbb{R}^{+},\cdot)$, where
                $+$ and $\cdot$ are the familiar notions of addition and
                multiplication, respectively. This is because of one of the
                fundamental identities for exponentiation:
                \begin{equation}
                    \exp(x+y)=\exp(x)\cdot\exp(y)
                \end{equation}
                This is also written as $e^{x+y}=e^{x}e^{y}$ and the
                multiplication symbol is dropped.
            \end{example}
            The kernel of a homomorphism $f:G\rightarrow{H}$ between two
            groups $(G,*)$ and $(H,\circ)$ is the set all all elements in
            $G$ which map to the identity element of $H$. In set theory
            notation, this is the fiber of the identity $e_{H}$, denoted
            $f^{-1}[\{e_{H}\}]$, where $e_{H}$ is the identity of $H$. We can
            also use set builder notation as follows:
            \begin{equation}
                \ker(f)=\{\,x\in{G}\,|\,f(x)=e_{H}\,\}
            \end{equation}
            A graph is an ordered pair $(V,E)$, where $V$ is a set, and
            $E\subset\mathcal{P}(V)$, the power set of $V$, with the property
            that for all $e\in{E}$, there are exactly two distinct elements
            $v_{0},v_{1}\in{e}$. The set $V$ is called the \textit{vertex set},
            and the set $E$ is called the \textit{edge set}. An element
            $e\in{E}$ tells us two vertices are connected by an edge, hence
            the requirement that $e=\{v_{0},v_{1}\}$ is a subset of $V$ with
            exactly two distinct elements.
            \par\hfill\par
            If $(V,E)$ is a graph, there are two groups we can associate to the
            graph. First is the \textit{vertex space}, which is the set of
            all subsets of vertices of the graph. In set theory notation, this
            is the set $\mathcal{P}(V)$, the power set of $V$. This is made a
            group with the symmetric difference operation $\ominus$, as we saw
            in Ex,~\ref{ex:sym_diff_is_group}. Similarly, there is the set of
            all subgraphs and this is called the \textit{edge space}, and it
            too becomes a group with $\ominus$
            \begin{theorem}
                If $(V,E)$ is a graph, if $(\mathcal{P}(V),\ominus)$ is the
                vertex space, if $(\mathcal{P}(E),\ominus)$ is the edge space,
                and if $f:\mathcal{P}(E)\rightarrow\mathcal{P}(V)$ is the
                function defined by $f(\mathcal{W})=\mathcal{U}$ where
                $\mathcal{W}\in\mathcal{P}(E)$ is a subset of edges, and
                $\mathcal{U}$ is the set of all elements of $V$ that have
                odd degree in the graph $(V,\mathcal{W})$, then $f$ is a
                group homomorphism.
            \end{theorem}
            \begin{proof}
                We must prove that, for two collections of edges
                $\mathcal{W}_{0}$ and $\mathcal{W}_{1}$, it is true that:
                \begin{equation}
                    f(\mathcal{W}_{0}\ominus\mathcal{W}_{1})
                    =f(\mathcal{W}_{0})\ominus{f}(\mathcal{W}_{1})
                \end{equation}
                Let $\mathcal{U}_{0}$ be the set of all vertices with odd
                degree in the graph $(V,\mathcal{W}_{0}\ominus\mathcal{W}_{1})$
                and $\mathcal{U}_{1}$ be the set
                $f(\mathcal{W}_{0})\ominus{f}(\mathcal{W}_{1})$. If
                $v\in\mathcal{U}_{0}$, then $v$ has odd degree in the subgraph
                $\mathcal{W}_{0}\ominus\mathcal{W}_{1}$ by definition.
            \end{proof}
\end{document}
