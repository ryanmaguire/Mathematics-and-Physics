\documentclass{article}
\usepackage{mathtools, esint, mathrsfs} % amsmath and integrals.
\usepackage{amsthm, amsfonts, amssymb}  % Fonts and theorems.
\usepackage{hyperref}                   % Hyperlinks.

% Colors for hyperref.
\hypersetup{
    colorlinks=true,
    linkcolor=blue,
    filecolor=magenta,
    urlcolor=Cerulean,
    citecolor=SkyBlue
}

\title{Computational Topology}
\author{Ryan Maguire}
\date{Fall 2021}
\setlength{\parindent}{0em}
\setlength{\parskip}{0em}

\newtheoremstyle{normal}
    {\topsep}               % Amount of space above the theorem.
    {\topsep}               % Amount of space below the theorem.
    {}                      % Font used for body of theorem.
    {}                      % Measure of space to indent.
    {\bfseries}             % Font of the header of the theorem.
    {}                      % Punctuation between head and body.
    {.5em}                  % Space after theorem head.
    {}

\theoremstyle{plain}
\newtheorem{theorem}{Theorem}[section]
\theoremstyle{normal}
\newtheorem{example}{Example}[section]

% TODO
%   PL Structures
%   transversality
%   surgery
%   Morse theory
%   intersection theory
%   cobordism
%   bundles
%   characteristic classes
%   geometric topology

\begin{document}
    \maketitle
    These notes come from my personal study of topological manifolds. Many of
    the concepts come from John M. Lee's Introduction to Topological
    Manifolds. Any errors in these notes are my own.
    \tableofcontents
    \section{Intuition}
        Topological manifolds are generalizations of the idea of curves and
        surfaces, allowing them to be \textit{higher dimensional}. The
        dimension of a manifold is, roughly speaking, the number of parameters
        needed in specifying a point. We try to model manifolds after the
        standard Euclidean space $\mathbb{R}^{n}$ and say that manifolds are
        topological spaces that look \textit{locally} like $\mathbb{R}^{n}$,
        though this phrasing lacks some rigor. That said, some of easiest
        examples of manifolds are curves in $\mathbb{R}^{2}$ or
        $\mathbb{R}^{3}$, and surfaces in $\mathbb{R}^{3}$. Curves are examples
        of 1 dimensional manifolds, and surfaces are examples of 2 dimensional
        manifolds. Even easier is the real line $\mathbb{R}$ and the
        Euclidean plane $\mathbb{R}^{2}$, which are both manifolds as well
        (They're not just \textit{locally} like $\mathbb{R}^{n}$ for some
        $n$, the \textit{are} $\mathbb{R}^{n}$ for some $n$!). In general,
        $\mathbb{R}^{n}$ is an $n$ dimensional manifold, but these are some of
        the more boring examples.
\end{document}