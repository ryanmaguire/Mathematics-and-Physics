\documentclass[crop=false,class=article,oneside]{standalone}
%----------------------------Preamble-------------------------------%
%---------------------------Packages----------------------------%
\usepackage{geometry}
\geometry{b5paper, margin=1.0in}
\usepackage[T1]{fontenc}
\usepackage{graphicx, float}            % Graphics/Images.
\usepackage{natbib}                     % For bibliographies.
\bibliographystyle{agsm}                % Bibliography style.
\usepackage[french, english]{babel}     % Language typesetting.
\usepackage[dvipsnames]{xcolor}         % Color names.
\usepackage{listings}                   % Verbatim-Like Tools.
\usepackage{mathtools, esint, mathrsfs} % amsmath and integrals.
\usepackage{amsthm, amsfonts, amssymb}  % Fonts and theorems.
\usepackage{tcolorbox}                  % Frames around theorems.
\usepackage{upgreek}                    % Non-Italic Greek.
\usepackage{fmtcount, etoolbox}         % For the \book{} command.
\usepackage[newparttoc]{titlesec}       % Formatting chapter, etc.
\usepackage{titletoc}                   % Allows \book in toc.
\usepackage[nottoc]{tocbibind}          % Bibliography in toc.
\usepackage[titles]{tocloft}            % ToC formatting.
\usepackage{pgfplots, tikz}             % Drawing/graphing tools.
\usepackage{imakeidx}                   % Used for index.
\usetikzlibrary{
    calc,                   % Calculating right angles and more.
    angles,                 % Drawing angles within triangles.
    arrows.meta,            % Latex and Stealth arrows.
    quotes,                 % Adding labels to angles.
    positioning,            % Relative positioning of nodes.
    decorations.markings,   % Adding arrows in the middle of a line.
    patterns,
    arrows
}                                       % Libraries for tikz.
\pgfplotsset{compat=1.9}                % Version of pgfplots.
\usepackage[font=scriptsize,
            labelformat=simple,
            labelsep=colon]{subcaption} % Subfigure captions.
\usepackage[font={scriptsize},
            hypcap=true,
            labelsep=colon]{caption}    % Figure captions.
\usepackage[pdftex,
            pdfauthor={Ryan Maguire},
            pdftitle={Mathematics and Physics},
            pdfsubject={Mathematics, Physics, Science},
            pdfkeywords={Mathematics, Physics, Computer Science, Biology},
            pdfproducer={LaTeX},
            pdfcreator={pdflatex}]{hyperref}
\hypersetup{
    colorlinks=true,
    linkcolor=blue,
    filecolor=magenta,
    urlcolor=Cerulean,
    citecolor=SkyBlue
}                           % Colors for hyperref.
\usepackage[toc,acronym,nogroupskip,nopostdot]{glossaries}
\usepackage{glossary-mcols}
%------------------------Theorem Styles-------------------------%
\theoremstyle{plain}
\newtheorem{theorem}{Theorem}[section]

% Define theorem style for default spacing and normal font.
\newtheoremstyle{normal}
    {\topsep}               % Amount of space above the theorem.
    {\topsep}               % Amount of space below the theorem.
    {}                      % Font used for body of theorem.
    {}                      % Measure of space to indent.
    {\bfseries}             % Font of the header of the theorem.
    {}                      % Punctuation between head and body.
    {.5em}                  % Space after theorem head.
    {}

% Italic header environment.
\newtheoremstyle{thmit}{\topsep}{\topsep}{}{}{\itshape}{}{0.5em}{}

% Define environments with italic headers.
\theoremstyle{thmit}
\newtheorem*{solution}{Solution}

% Define default environments.
\theoremstyle{normal}
\newtheorem{example}{Example}[section]
\newtheorem{definition}{Definition}[section]
\newtheorem{problem}{Problem}[section]

% Define framed environment.
\tcbuselibrary{most}
\newtcbtheorem[use counter*=theorem]{ftheorem}{Theorem}{%
    before=\par\vspace{2ex},
    boxsep=0.5\topsep,
    after=\par\vspace{2ex},
    colback=green!5,
    colframe=green!35!black,
    fonttitle=\bfseries\upshape%
}{thm}

\newtcbtheorem[auto counter, number within=section]{faxiom}{Axiom}{%
    before=\par\vspace{2ex},
    boxsep=0.5\topsep,
    after=\par\vspace{2ex},
    colback=Apricot!5,
    colframe=Apricot!35!black,
    fonttitle=\bfseries\upshape%
}{ax}

\newtcbtheorem[use counter*=definition]{fdefinition}{Definition}{%
    before=\par\vspace{2ex},
    boxsep=0.5\topsep,
    after=\par\vspace{2ex},
    colback=blue!5!white,
    colframe=blue!75!black,
    fonttitle=\bfseries\upshape%
}{def}

\newtcbtheorem[use counter*=example]{fexample}{Example}{%
    before=\par\vspace{2ex},
    boxsep=0.5\topsep,
    after=\par\vspace{2ex},
    colback=red!5!white,
    colframe=red!75!black,
    fonttitle=\bfseries\upshape%
}{ex}

\newtcbtheorem[auto counter, number within=section]{fnotation}{Notation}{%
    before=\par\vspace{2ex},
    boxsep=0.5\topsep,
    after=\par\vspace{2ex},
    colback=SeaGreen!5!white,
    colframe=SeaGreen!75!black,
    fonttitle=\bfseries\upshape%
}{not}

\newtcbtheorem[use counter*=remark]{fremark}{Remark}{%
    fonttitle=\bfseries\upshape,
    colback=Goldenrod!5!white,
    colframe=Goldenrod!75!black}{ex}

\newenvironment{bproof}{\textit{Proof.}}{\hfill$\square$}
\tcolorboxenvironment{bproof}{%
    blanker,
    breakable,
    left=3mm,
    before skip=5pt,
    after skip=10pt,
    borderline west={0.6mm}{0pt}{green!80!black}
}

\AtEndEnvironment{lexample}{$\hfill\textcolor{red}{\blacksquare}$}
\newtcbtheorem[use counter*=example]{lexample}{Example}{%
    empty,
    title={Example~\theexample},
    boxed title style={%
        empty,
        size=minimal,
        toprule=2pt,
        top=0.5\topsep,
    },
    coltitle=red,
    fonttitle=\bfseries,
    parbox=false,
    boxsep=0pt,
    before=\par\vspace{2ex},
    left=0pt,
    right=0pt,
    top=3ex,
    bottom=1ex,
    before=\par\vspace{2ex},
    after=\par\vspace{2ex},
    breakable,
    pad at break*=0mm,
    vfill before first,
    overlay unbroken={%
        \draw[red, line width=2pt]
            ([yshift=-1.2ex]title.south-|frame.west) to
            ([yshift=-1.2ex]title.south-|frame.east);
        },
    overlay first={%
        \draw[red, line width=2pt]
            ([yshift=-1.2ex]title.south-|frame.west) to
            ([yshift=-1.2ex]title.south-|frame.east);
    },
}{ex}

\AtEndEnvironment{ldefinition}{$\hfill\textcolor{Blue}{\blacksquare}$}
\newtcbtheorem[use counter*=definition]{ldefinition}{Definition}{%
    empty,
    title={Definition~\thedefinition:~{#1}},
    boxed title style={%
        empty,
        size=minimal,
        toprule=2pt,
        top=0.5\topsep,
    },
    coltitle=Blue,
    fonttitle=\bfseries,
    parbox=false,
    boxsep=0pt,
    before=\par\vspace{2ex},
    left=0pt,
    right=0pt,
    top=3ex,
    bottom=0pt,
    before=\par\vspace{2ex},
    after=\par\vspace{1ex},
    breakable,
    pad at break*=0mm,
    vfill before first,
    overlay unbroken={%
        \draw[Blue, line width=2pt]
            ([yshift=-1.2ex]title.south-|frame.west) to
            ([yshift=-1.2ex]title.south-|frame.east);
        },
    overlay first={%
        \draw[Blue, line width=2pt]
            ([yshift=-1.2ex]title.south-|frame.west) to
            ([yshift=-1.2ex]title.south-|frame.east);
    },
}{def}

\AtEndEnvironment{ltheorem}{$\hfill\textcolor{Green}{\blacksquare}$}
\newtcbtheorem[use counter*=theorem]{ltheorem}{Theorem}{%
    empty,
    title={Theorem~\thetheorem:~{#1}},
    boxed title style={%
        empty,
        size=minimal,
        toprule=2pt,
        top=0.5\topsep,
    },
    coltitle=Green,
    fonttitle=\bfseries,
    parbox=false,
    boxsep=0pt,
    before=\par\vspace{2ex},
    left=0pt,
    right=0pt,
    top=3ex,
    bottom=-1.5ex,
    breakable,
    pad at break*=0mm,
    vfill before first,
    overlay unbroken={%
        \draw[Green, line width=2pt]
            ([yshift=-1.2ex]title.south-|frame.west) to
            ([yshift=-1.2ex]title.south-|frame.east);},
    overlay first={%
        \draw[Green, line width=2pt]
            ([yshift=-1.2ex]title.south-|frame.west) to
            ([yshift=-1.2ex]title.south-|frame.east);
    }
}{thm}

%--------------------Declared Math Operators--------------------%
\DeclareMathOperator{\adjoint}{adj}         % Adjoint.
\DeclareMathOperator{\Card}{Card}           % Cardinality.
\DeclareMathOperator{\curl}{curl}           % Curl.
\DeclareMathOperator{\diam}{diam}           % Diameter.
\DeclareMathOperator{\dist}{dist}           % Distance.
\DeclareMathOperator{\Div}{div}             % Divergence.
\DeclareMathOperator{\Erf}{Erf}             % Error Function.
\DeclareMathOperator{\Erfc}{Erfc}           % Complementary Error Function.
\DeclareMathOperator{\Ext}{Ext}             % Exterior.
\DeclareMathOperator{\GCD}{GCD}             % Greatest common denominator.
\DeclareMathOperator{\grad}{grad}           % Gradient
\DeclareMathOperator{\Ima}{Im}              % Image.
\DeclareMathOperator{\Int}{Int}             % Interior.
\DeclareMathOperator{\LC}{LC}               % Leading coefficient.
\DeclareMathOperator{\LCM}{LCM}             % Least common multiple.
\DeclareMathOperator{\LM}{LM}               % Leading monomial.
\DeclareMathOperator{\LT}{LT}               % Leading term.
\DeclareMathOperator{\Mod}{mod}             % Modulus.
\DeclareMathOperator{\Mon}{Mon}             % Monomial.
\DeclareMathOperator{\multideg}{mutlideg}   % Multi-Degree (Graphs).
\DeclareMathOperator{\nul}{nul}             % Null space of operator.
\DeclareMathOperator{\Ord}{Ord}             % Ordinal of ordered set.
\DeclareMathOperator{\Prin}{Prin}           % Principal value.
\DeclareMathOperator{\proj}{proj}           % Projection.
\DeclareMathOperator{\Refl}{Refl}           % Reflection operator.
\DeclareMathOperator{\rk}{rk}               % Rank of operator.
\DeclareMathOperator{\sgn}{sgn}             % Sign of a number.
\DeclareMathOperator{\sinc}{sinc}           % Sinc function.
\DeclareMathOperator{\Span}{Span}           % Span of a set.
\DeclareMathOperator{\Spec}{Spec}           % Spectrum.
\DeclareMathOperator{\supp}{supp}           % Support
\DeclareMathOperator{\Tr}{Tr}               % Trace of matrix.
%--------------------Declared Math Symbols--------------------%
\DeclareMathSymbol{\minus}{\mathbin}{AMSa}{"39} % Unary minus sign.
%------------------------New Commands---------------------------%
\DeclarePairedDelimiter\norm{\lVert}{\rVert}
\DeclarePairedDelimiter\ceil{\lceil}{\rceil}
\DeclarePairedDelimiter\floor{\lfloor}{\rfloor}
\newcommand*\diff{\mathop{}\!\mathrm{d}}
\newcommand*\Diff[1]{\mathop{}\!\mathrm{d^#1}}
\renewcommand*{\glstextformat}[1]{\textcolor{RoyalBlue}{#1}}
\renewcommand{\glsnamefont}[1]{\textbf{#1}}
\renewcommand\labelitemii{$\circ$}
\renewcommand\thesubfigure{%
    \arabic{chapter}.\arabic{figure}.\arabic{subfigure}}
\addto\captionsenglish{\renewcommand{\figurename}{Fig.}}
\numberwithin{equation}{section}

\renewcommand{\vector}[1]{\boldsymbol{\mathrm{#1}}}

\newcommand{\uvector}[1]{\boldsymbol{\hat{\mathrm{#1}}}}
\newcommand{\topspace}[2][]{(#2,\tau_{#1})}
\newcommand{\measurespace}[2][]{(#2,\varSigma_{#1},\mu_{#1})}
\newcommand{\measurablespace}[2][]{(#2,\varSigma_{#1})}
\newcommand{\manifold}[2][]{(#2,\tau_{#1},\mathcal{A}_{#1})}
\newcommand{\tanspace}[2]{T_{#1}{#2}}
\newcommand{\cotanspace}[2]{T_{#1}^{*}{#2}}
\newcommand{\Ckspace}[3][\mathbb{R}]{C^{#2}(#3,#1)}
\newcommand{\funcspace}[2][\mathbb{R}]{\mathcal{F}(#2,#1)}
\newcommand{\smoothvecf}[1]{\mathfrak{X}(#1)}
\newcommand{\smoothonef}[1]{\mathfrak{X}^{*}(#1)}
\newcommand{\bracket}[2]{[#1,#2]}

%------------------------Book Command---------------------------%
\makeatletter
\renewcommand\@pnumwidth{1cm}
\newcounter{book}
\renewcommand\thebook{\@Roman\c@book}
\newcommand\book{%
    \if@openright
        \cleardoublepage
    \else
        \clearpage
    \fi
    \thispagestyle{plain}%
    \if@twocolumn
        \onecolumn
        \@tempswatrue
    \else
        \@tempswafalse
    \fi
    \null\vfil
    \secdef\@book\@sbook
}
\def\@book[#1]#2{%
    \refstepcounter{book}
    \addcontentsline{toc}{book}{\bookname\ \thebook:\hspace{1em}#1}
    \markboth{}{}
    {\centering
     \interlinepenalty\@M
     \normalfont
     \huge\bfseries\bookname\nobreakspace\thebook
     \par
     \vskip 20\p@
     \Huge\bfseries#2\par}%
    \@endbook}
\def\@sbook#1{%
    {\centering
     \interlinepenalty \@M
     \normalfont
     \Huge\bfseries#1\par}%
    \@endbook}
\def\@endbook{
    \vfil\newpage
        \if@twoside
            \if@openright
                \null
                \thispagestyle{empty}%
                \newpage
            \fi
        \fi
        \if@tempswa
            \twocolumn
        \fi
}
\newcommand*\l@book[2]{%
    \ifnum\c@tocdepth >-3\relax
        \addpenalty{-\@highpenalty}%
        \addvspace{2.25em\@plus\p@}%
        \setlength\@tempdima{3em}%
        \begingroup
            \parindent\z@\rightskip\@pnumwidth
            \parfillskip -\@pnumwidth
            {
                \leavevmode
                \Large\bfseries#1\hfill\hb@xt@\@pnumwidth{\hss#2}
            }
            \par
            \nobreak
            \global\@nobreaktrue
            \everypar{\global\@nobreakfalse\everypar{}}%
        \endgroup
    \fi}
\newcommand\bookname{Book}
\renewcommand{\thebook}{\texorpdfstring{\Numberstring{book}}{book}}
\providecommand*{\toclevel@book}{-2}
\makeatother
\titleformat{\part}[display]
    {\Large\bfseries}
    {\partname\nobreakspace\thepart}
    {0mm}
    {\Huge\bfseries}
\titlecontents{part}[0pt]
    {\large\bfseries}
    {\partname\ \thecontentslabel: \quad}
    {}
    {\hfill\contentspage}
\titlecontents{chapter}[0pt]
    {\bfseries}
    {\chaptername\ \thecontentslabel:\quad}
    {}
    {\hfill\contentspage}
\newglossarystyle{longpara}{%
    \setglossarystyle{long}%
    \renewenvironment{theglossary}{%
        \begin{longtable}[l]{{p{0.25\hsize}p{0.65\hsize}}}
    }{\end{longtable}}%
    \renewcommand{\glossentry}[2]{%
        \glstarget{##1}{\glossentryname{##1}}%
        &\glossentrydesc{##1}{~##2.}
        \tabularnewline%
        \tabularnewline
    }%
}
\newglossary[not-glg]{notation}{not-gls}{not-glo}{Notation}
\newcommand*{\newnotation}[4][]{%
    \newglossaryentry{#2}{type=notation, name={\textbf{#3}, },
                          text={#4}, description={#4},#1}%
}
%--------------------------LENGTHS------------------------------%
% Spacings for the Table of Contents.
\addtolength{\cftsecnumwidth}{1ex}
\addtolength{\cftsubsecindent}{1ex}
\addtolength{\cftsubsecnumwidth}{1ex}
\addtolength{\cftfignumwidth}{1ex}
\addtolength{\cfttabnumwidth}{1ex}

% Indent and paragraph spacing.
\setlength{\parindent}{0em}
\setlength{\parskip}{0em}
%--------------------------Main Document----------------------------%
\begin{document}
    \ifx\ifsub\undefined
        \section*{Preliminaries}
        \setcounter{section}{1}
    \fi
    \subsection{Equivalence and Transfinite Cardinal Numbers}
        \begin{definition}
        $A$ and $B$ are called equivalent if and only if there is a bijective function $f:A\rightarrow B$. We write $A\sim B$.
        \end{definition}
        \begin{theorem}
        Equivalence has the following properties:
        \begin{enumerate}
        \item $A\sim A$ for any set $A$.
        \item If $A\sim B$, then $B\sim A$.
        \item If $A\sim B$ and $B\sim C$, then $A\sim C$.
        \end{enumerate}
        \end{theorem}
        \begin{proof}
        In order,
        \begin{enumerate}
        \item For let $f$ be the identity mapping. That is, for all $x\in A$, $f(x) = x$. This is bijective and thus $A\sim A$.
        \item If $A\sim B$, there is a bijective function $f:A\rightarrow B$. Then $f^{-1}:B\rightarrow A$ is bijective, and $B\sim A$.
        \item Let $f:A\rightarrow B$ and $g:B\rightarrow C$ be bijections. Then $g\circ f:A\rightarrow C$ is a bijection, and thus $A\sim C$.
        \end{enumerate}
        \end{proof}
        \begin{theorem}
        If $A\sim C$ and $B\sim D$, where $A,B$ and $C,D$ are disjoint, then $A\cup B \sim C\cup D$
        \end{theorem}
        \begin{proof}
        Let $f:A\rightarrow C$ and $g:B\rightarrow D$ be isomorphisms. Let $h:A\cup B \rightarrow C\cup D$ be defined by $h(x) = \begin{cases} f(x), & x\in A\\ g(x), & x\in B\end{cases}$. As $A$ and $B$ are disjoint, this is indeed a function and it is bijective as $C$ and $D$ are disjoint. Therefore, etc.
        \end{proof}
        \begin{definition}
        The set $\mathbb{Z}_n$ is defined for all $n\in \mathbb{N}$ as $\{k\in \mathbb{N}: k\leq n\}$.
        \end{definition}
        \begin{definition}
        A set $A$ is a said to be finite if and only if there is some $n\in \mathbb{N}$ such that there is a bijection $f:\mathbb{Z}_n \rightarrow A$.
        \end{definition}
        \begin{definition}
        If $A$ is a set that is equivalent to $\mathbb{Z}_n$ for some $n\in \mathbb{N}$, then the cardinality of $A$, denoted $|A|$, is $n$.
        \end{definition}
        \begin{theorem}
        For two finite sets $A$ and $B$, $A\sim B$ if and only if $|A|=|B|$.
        \end{theorem}
        \begin{proof}
        $[|A|=|B|=n]\Rightarrow[A\sim \mathbb{Z}_n]\land[B\sim \mathbb{Z}_n]\Rightarrow [A\sim B]$. $[A\sim B]\Rightarrow [\exists \underset{Bijective}{f:A\rightarrow B}]\Rightarrow [f(A) = B]\Rightarrow [|A|=|B|]$.
        \end{proof}
        \begin{definition}
        A set $A$ is said to be infinite if and only if there is a proper subset $B\underset{Proper}\subset A$ such that $B\sim A$.
        \end{definition}
        \begin{theorem}
        Infinite sets are not finite.
        \end{theorem}
        \begin{proof}
        Suppose not. Let $A$ be an infinite set and suppose there is an $n\in \mathbb{N}$ such that $A\sim \mathbb{Z}_n$. But as $A$ is an infinite set, there is a proper subset $B$ such that $B\sim A$. But then $B\sim \mathbb{Z}_n$. But as $B$ is a proper subset, there is at least one point in $A$ not contained in $B$. But then $|B|<n$, a contradiction. Thus $A$ is not finite.
        \end{proof}
        \begin{corollary}
        If $A$ is an infinite set, then for every $n\in \mathbb{N}$ there is a subset $B\subset A$ such that $B\sim \mathbb{Z}_n$.
        \end{corollary}
        \begin{proof}
        Suppose not. Then there is a least $n\in \mathbb{N}:B\subset A\Rightarrow |B|<n$. But then $A$ has at most $n$ elements, a contradiction.
        \end{proof}
        \begin{definition}
        A set $A$ is called countable if and only if $A\sim \mathbb{N}$.
        \end{definition}
        \begin{theorem}
        A set $A$ is infinite if and only if it contains a proper subset $B$ such that $B\sim \mathbb{N}$.
        \end{theorem}
        \begin{proof}
        If $A$ has a proper subset $B$ such that $B\sim \mathbb{N}$, then $A$ is not finite and is thus infinite. If $A$ is infinite, then for all $n\in \mathbb{N}$ there is a set $A_n\subset A$ such that $A_n \sim \mathbb{Z}_n$. Let $B = \{a_n: a_n \in A_n, a_n \notin A_{n-1}\}$. Note that $a_{n} = a_{m}$ if and only if $m= n$. Let $f:\mathbb{N} \rightarrow B$ be defined by $n\mapsto a_n$. This is bijective, and thus $B\sim \mathbb{N}$.
        \end{proof}
        \begin{remark}
        This shows that $\mathbb{N}$ is, in a sense, the "Smallest," infinite set. $|\mathbb{N}|$ is denoted $\aleph_0$.
        \end{remark}
        \begin{definition}
        A set is called uncountable if and only if it is infinite and not countable.
        \end{definition}
        \begin{lemma}
        If $B\subset A$, $f:A\rightarrow B$ is injective, then there is a bijection $g:A\rightarrow B$
        \end{lemma}
        \begin{proof}
        Let $Y = A\setminus B$, and inductively define $f^{k+1}(Y) = f(f^{k}(Y))$. Let $X = Y\cup (\cup_{k=0}^{\infty} f^{k}(Y))$. As $Y\cap B = \emptyset$, then $f(Y)\cap Y= \emptyset$. As $f$ is an injection, $f(f(Y))\cap f(Y)=\emptyset$, and similarly $f(f(Y))\cap Y = \emptyset$. Inductively, $f^{n}(Y)\cap f^{m}(Y) = \emptyset$, for $n\ne m$. It then also follows that $f(X) = \cup_{k=1}^{\infty} f^{k}(Y)$. Thus $A\setminus X = [B\cup Y]\setminus [Y\cup f(X)] = B\setminus f(X)$. Let $g(x) = \begin{cases} f(x), & x\in X \\ x, & x \in B\setminus f(X)\end{cases}$. This is a bijections from $A$ to $B$.
        \end{proof}
        \begin{theorem}[Cantor-Schr\"{o}der-Bernstein Theorem]
        If $A_1 \subset A$, $B_1 \subset B$, and $A\sim B_1$, $B \sim A_1$, then $A\sim B$.
        \end{theorem}
        \begin{proof}
        Let $f:A\rightarrow B_1$ and $g:B\rightarrow A_1$ be bijections.Then $(g\circ f):A\rightarrow A_1$ is an injection from $A$ into $A_1$. Thus, there is a bijection $h:A\rightarrow A_1$. Thus, $A\sim A_1 \sim B\Rightarrow A\sim B$.
        \end{proof}
        \begin{theorem}
        $\mathbb{N}\times \mathbb{N}$ is countable.
        \end{theorem}
        \begin{proof}
        For $f:\mathbb{N} \rightarrow \mathbb{N}\times \mathbb{N}$ defined by $f(n) = (0,n)$ shows there is a subset $N_1$ of $\mathbb{N} \times \mathbb{N}$ such that $\mathbb{N}\sim N_1$. And $g:\mathbb{N}\times \mathbb{N} \rightarrow \mathbb{N}$ defined by $g(n,m) =n+2^{n+m}$ shows that there is a subset $M_1 \subset \mathbb{N}$ such that $\mathbb{N} \times \mathbb{N} \sim M_1$. By the Cantor-Schr\"{o}der-Bernstein Theorem, $\mathbb{N} \sim \mathbb{N}\times \mathbb{N}$.
        \end{proof}
        \begin{lemma}
        If $A$ is infinite and $f:A\rightarrow \mathbb{N}$ is injective, then $A$ is countable.
        \end{lemma}
        \begin{proof}
        As $A$ is infinite and $A\sim f(A)$, $f(A)$ is infinite. But as $f(A)\subset \mathbb{N}$ and $f(A)$ is infinite, $f(A)\sim \mathbb{N}$. Thus, $A\sim \mathbb{N}$. 
        \end{proof}
        \begin{theorem}
        $\mathbb{Q}$ is countable.
        \end{theorem}
        \begin{proof}
        For each $x\in \mathbb{Q}$, $x\ne 0$, let $p_x,q_x\in\mathbb{Z}:q_x>0$ be the unique integers such that $x = \frac{p_x}{q_x}$ and $g.c.d.(|p_x|,|q_x|)=1$. Define $f:\mathbb{Q}\rightarrow \mathbb{N}\times \mathbb{N}$ as $f(x) = \begin{cases}(p_x,q_x), & x\ne 0 \\ 0, & x=0\end{cases}$. This shows there is a subset $N_1$ of $\mathbb{N}\times \mathbb{N}$ such that $\mathbb{Q}\sim N_1$. But $N_1$ is an infinite subset of $\mathbb{N}\times\mathbb{N}$ as $\mathbb{Q}$ is infinite and $f$ is injective. Define $g$ as $n+2^{n+m}:(n,m)\in N_1$. This is an injective function into $\mathbb{N}$, and thus $N_1 \sim \mathbb{N}$. Therefore $\mathbb{Q}\sim \mathbb{N}$.
        \end{proof}
        \begin{definition}
        If $A$ and $B$ are sets, we say that $|A|<|B|$ if there is an injective function $f:A\rightarrow B$, yet no bijection.
        \end{definition}
        \begin{theorem}[Cantor's Theorem]
        For a set $M$, $|M|<|\mathcal{P}(M)|$.
        \end{theorem}
        \begin{proof}
        For let $M$ be a set with cardinality $|M|$. Let $U_m \subset M$ such that $U_m \sim M$. Such a set exists, for example, the singletons of $\mathcal{P}(M)$. Thus, $M$ is split into two distinct sets $Class\ I=\{x\in M: \textrm{There is a subset } X\subset U_m\textrm{ such that }x\in X\}$, and $Class\ II=M-Class\ I$. Let $L = Class\ II$. $L\subset M$, and thus $L\in \mathcal{P}(M)$. However, $L \notin U_m$ for if it were, then the element $m_1$ paired with it in $M$ is of Class II (For it cannot be of Class I as $m_1$ would not appear in $L$). If $m_1$ were in Class II, then by definition $m_1 \notin L$. But as $m_1 \in L$, we see that $L\notin U_m$. Thus, $|U_m| <|\mathcal{P}(M)|$, and therefore $|M|<|\mathcal{P}(M)|$.
        \end{proof}
        \begin{theorem}
        The set $R=\{x\in \mathbb{R}:0<x<1\}$ is equivalent to $\mathcal{P}(\mathbb{N})$.
        \end{theorem}
        \begin{proof}
        For every real number has a binary representation (Proof of this is omitted). That is, for every real number $r$, $ r = \sum_{n=-\infty}^{\infty} \frac{a_n}{2^n}$, where $a_n = 0$ or $1$. As $0<x<1$, this sum is just $\sum_{n=1}^{\infty} \frac{a_n}{2^n}$. Let $f:\mathcal{P}(\mathbb{N})\rightarrow R$ be defined by the following: If $N\subset \mathcal{P}(\mathbb{N})$ and $n\in N$, then $a_n = 1$, other wise $n=0$. Then every real number is matched to a subset of $\mathcal{P}(\mathbb{N})$, moreover this is done bijectively. Thus, $\mathcal{P}(\mathbb{N})\sim R$.
        \end{proof}
        \begin{theorem}
        $\mathbb{R} \sim \mathcal{P}(\mathbb{N})$.
        \end{theorem}
        \begin{proof}
        It suffices to show that $R\sim \mathbb{R}$, where $R$ is from the previous theorem. Let $f(x) = \begin{cases} \frac{x(1-x)}{2x-1}, & x \ne \frac{1}{2} \\ 0, & x = \frac{1}{2}\end{cases}$.
        \end{proof}
        \begin{remark}
        There is something called the continuum hypothesis which states that there is no set $S$ such that $|\mathbb{N}| < |S| < |\mathbb{R}|$. If this is accepted, we may write $|\mathbb{R}| = \aleph_1 = 2^{\aleph_0}$. Much like the axiom of choice, this is independent of the rest of set theory. Its acceptance or negation is possible without contradiction. We will not need to worry about this.
        \end{remark}
\end{document}