\documentclass[crop=false,class=article,oneside]{standalone}
%----------------------------Preamble-------------------------------%
%---------------------------Packages----------------------------%
\usepackage{geometry}
\geometry{b5paper, margin=1.0in}
\usepackage[T1]{fontenc}
\usepackage{graphicx, float}            % Graphics/Images.
\usepackage{natbib}                     % For bibliographies.
\bibliographystyle{agsm}                % Bibliography style.
\usepackage[french, english]{babel}     % Language typesetting.
\usepackage[dvipsnames]{xcolor}         % Color names.
\usepackage{listings}                   % Verbatim-Like Tools.
\usepackage{mathtools, esint, mathrsfs} % amsmath and integrals.
\usepackage{amsthm, amsfonts, amssymb}  % Fonts and theorems.
\usepackage{tcolorbox}                  % Frames around theorems.
\usepackage{upgreek}                    % Non-Italic Greek.
\usepackage{fmtcount, etoolbox}         % For the \book{} command.
\usepackage[newparttoc]{titlesec}       % Formatting chapter, etc.
\usepackage{titletoc}                   % Allows \book in toc.
\usepackage[nottoc]{tocbibind}          % Bibliography in toc.
\usepackage[titles]{tocloft}            % ToC formatting.
\usepackage{pgfplots, tikz}             % Drawing/graphing tools.
\usepackage{imakeidx}                   % Used for index.
\usetikzlibrary{
    calc,                   % Calculating right angles and more.
    angles,                 % Drawing angles within triangles.
    arrows.meta,            % Latex and Stealth arrows.
    quotes,                 % Adding labels to angles.
    positioning,            % Relative positioning of nodes.
    decorations.markings,   % Adding arrows in the middle of a line.
    patterns,
    arrows
}                                       % Libraries for tikz.
\pgfplotsset{compat=1.9}                % Version of pgfplots.
\usepackage[font=scriptsize,
            labelformat=simple,
            labelsep=colon]{subcaption} % Subfigure captions.
\usepackage[font={scriptsize},
            hypcap=true,
            labelsep=colon]{caption}    % Figure captions.
\usepackage[pdftex,
            pdfauthor={Ryan Maguire},
            pdftitle={Mathematics and Physics},
            pdfsubject={Mathematics, Physics, Science},
            pdfkeywords={Mathematics, Physics, Computer Science, Biology},
            pdfproducer={LaTeX},
            pdfcreator={pdflatex}]{hyperref}
\hypersetup{
    colorlinks=true,
    linkcolor=blue,
    filecolor=magenta,
    urlcolor=Cerulean,
    citecolor=SkyBlue
}                           % Colors for hyperref.
\usepackage[toc,acronym,nogroupskip,nopostdot]{glossaries}
\usepackage{glossary-mcols}
%------------------------Theorem Styles-------------------------%
\theoremstyle{plain}
\newtheorem{theorem}{Theorem}[section]

% Define theorem style for default spacing and normal font.
\newtheoremstyle{normal}
    {\topsep}               % Amount of space above the theorem.
    {\topsep}               % Amount of space below the theorem.
    {}                      % Font used for body of theorem.
    {}                      % Measure of space to indent.
    {\bfseries}             % Font of the header of the theorem.
    {}                      % Punctuation between head and body.
    {.5em}                  % Space after theorem head.
    {}

% Italic header environment.
\newtheoremstyle{thmit}{\topsep}{\topsep}{}{}{\itshape}{}{0.5em}{}

% Define environments with italic headers.
\theoremstyle{thmit}
\newtheorem*{solution}{Solution}

% Define default environments.
\theoremstyle{normal}
\newtheorem{example}{Example}[section]
\newtheorem{definition}{Definition}[section]
\newtheorem{problem}{Problem}[section]

% Define framed environment.
\tcbuselibrary{most}
\newtcbtheorem[use counter*=theorem]{ftheorem}{Theorem}{%
    before=\par\vspace{2ex},
    boxsep=0.5\topsep,
    after=\par\vspace{2ex},
    colback=green!5,
    colframe=green!35!black,
    fonttitle=\bfseries\upshape%
}{thm}

\newtcbtheorem[auto counter, number within=section]{faxiom}{Axiom}{%
    before=\par\vspace{2ex},
    boxsep=0.5\topsep,
    after=\par\vspace{2ex},
    colback=Apricot!5,
    colframe=Apricot!35!black,
    fonttitle=\bfseries\upshape%
}{ax}

\newtcbtheorem[use counter*=definition]{fdefinition}{Definition}{%
    before=\par\vspace{2ex},
    boxsep=0.5\topsep,
    after=\par\vspace{2ex},
    colback=blue!5!white,
    colframe=blue!75!black,
    fonttitle=\bfseries\upshape%
}{def}

\newtcbtheorem[use counter*=example]{fexample}{Example}{%
    before=\par\vspace{2ex},
    boxsep=0.5\topsep,
    after=\par\vspace{2ex},
    colback=red!5!white,
    colframe=red!75!black,
    fonttitle=\bfseries\upshape%
}{ex}

\newtcbtheorem[auto counter, number within=section]{fnotation}{Notation}{%
    before=\par\vspace{2ex},
    boxsep=0.5\topsep,
    after=\par\vspace{2ex},
    colback=SeaGreen!5!white,
    colframe=SeaGreen!75!black,
    fonttitle=\bfseries\upshape%
}{not}

\newtcbtheorem[use counter*=remark]{fremark}{Remark}{%
    fonttitle=\bfseries\upshape,
    colback=Goldenrod!5!white,
    colframe=Goldenrod!75!black}{ex}

\newenvironment{bproof}{\textit{Proof.}}{\hfill$\square$}
\tcolorboxenvironment{bproof}{%
    blanker,
    breakable,
    left=3mm,
    before skip=5pt,
    after skip=10pt,
    borderline west={0.6mm}{0pt}{green!80!black}
}

\AtEndEnvironment{lexample}{$\hfill\textcolor{red}{\blacksquare}$}
\newtcbtheorem[use counter*=example]{lexample}{Example}{%
    empty,
    title={Example~\theexample},
    boxed title style={%
        empty,
        size=minimal,
        toprule=2pt,
        top=0.5\topsep,
    },
    coltitle=red,
    fonttitle=\bfseries,
    parbox=false,
    boxsep=0pt,
    before=\par\vspace{2ex},
    left=0pt,
    right=0pt,
    top=3ex,
    bottom=1ex,
    before=\par\vspace{2ex},
    after=\par\vspace{2ex},
    breakable,
    pad at break*=0mm,
    vfill before first,
    overlay unbroken={%
        \draw[red, line width=2pt]
            ([yshift=-1.2ex]title.south-|frame.west) to
            ([yshift=-1.2ex]title.south-|frame.east);
        },
    overlay first={%
        \draw[red, line width=2pt]
            ([yshift=-1.2ex]title.south-|frame.west) to
            ([yshift=-1.2ex]title.south-|frame.east);
    },
}{ex}

\AtEndEnvironment{ldefinition}{$\hfill\textcolor{Blue}{\blacksquare}$}
\newtcbtheorem[use counter*=definition]{ldefinition}{Definition}{%
    empty,
    title={Definition~\thedefinition:~{#1}},
    boxed title style={%
        empty,
        size=minimal,
        toprule=2pt,
        top=0.5\topsep,
    },
    coltitle=Blue,
    fonttitle=\bfseries,
    parbox=false,
    boxsep=0pt,
    before=\par\vspace{2ex},
    left=0pt,
    right=0pt,
    top=3ex,
    bottom=0pt,
    before=\par\vspace{2ex},
    after=\par\vspace{1ex},
    breakable,
    pad at break*=0mm,
    vfill before first,
    overlay unbroken={%
        \draw[Blue, line width=2pt]
            ([yshift=-1.2ex]title.south-|frame.west) to
            ([yshift=-1.2ex]title.south-|frame.east);
        },
    overlay first={%
        \draw[Blue, line width=2pt]
            ([yshift=-1.2ex]title.south-|frame.west) to
            ([yshift=-1.2ex]title.south-|frame.east);
    },
}{def}

\AtEndEnvironment{ltheorem}{$\hfill\textcolor{Green}{\blacksquare}$}
\newtcbtheorem[use counter*=theorem]{ltheorem}{Theorem}{%
    empty,
    title={Theorem~\thetheorem:~{#1}},
    boxed title style={%
        empty,
        size=minimal,
        toprule=2pt,
        top=0.5\topsep,
    },
    coltitle=Green,
    fonttitle=\bfseries,
    parbox=false,
    boxsep=0pt,
    before=\par\vspace{2ex},
    left=0pt,
    right=0pt,
    top=3ex,
    bottom=-1.5ex,
    breakable,
    pad at break*=0mm,
    vfill before first,
    overlay unbroken={%
        \draw[Green, line width=2pt]
            ([yshift=-1.2ex]title.south-|frame.west) to
            ([yshift=-1.2ex]title.south-|frame.east);},
    overlay first={%
        \draw[Green, line width=2pt]
            ([yshift=-1.2ex]title.south-|frame.west) to
            ([yshift=-1.2ex]title.south-|frame.east);
    }
}{thm}

%--------------------Declared Math Operators--------------------%
\DeclareMathOperator{\adjoint}{adj}         % Adjoint.
\DeclareMathOperator{\Card}{Card}           % Cardinality.
\DeclareMathOperator{\curl}{curl}           % Curl.
\DeclareMathOperator{\diam}{diam}           % Diameter.
\DeclareMathOperator{\dist}{dist}           % Distance.
\DeclareMathOperator{\Div}{div}             % Divergence.
\DeclareMathOperator{\Erf}{Erf}             % Error Function.
\DeclareMathOperator{\Erfc}{Erfc}           % Complementary Error Function.
\DeclareMathOperator{\Ext}{Ext}             % Exterior.
\DeclareMathOperator{\GCD}{GCD}             % Greatest common denominator.
\DeclareMathOperator{\grad}{grad}           % Gradient
\DeclareMathOperator{\Ima}{Im}              % Image.
\DeclareMathOperator{\Int}{Int}             % Interior.
\DeclareMathOperator{\LC}{LC}               % Leading coefficient.
\DeclareMathOperator{\LCM}{LCM}             % Least common multiple.
\DeclareMathOperator{\LM}{LM}               % Leading monomial.
\DeclareMathOperator{\LT}{LT}               % Leading term.
\DeclareMathOperator{\Mod}{mod}             % Modulus.
\DeclareMathOperator{\Mon}{Mon}             % Monomial.
\DeclareMathOperator{\multideg}{mutlideg}   % Multi-Degree (Graphs).
\DeclareMathOperator{\nul}{nul}             % Null space of operator.
\DeclareMathOperator{\Ord}{Ord}             % Ordinal of ordered set.
\DeclareMathOperator{\Prin}{Prin}           % Principal value.
\DeclareMathOperator{\proj}{proj}           % Projection.
\DeclareMathOperator{\Refl}{Refl}           % Reflection operator.
\DeclareMathOperator{\rk}{rk}               % Rank of operator.
\DeclareMathOperator{\sgn}{sgn}             % Sign of a number.
\DeclareMathOperator{\sinc}{sinc}           % Sinc function.
\DeclareMathOperator{\Span}{Span}           % Span of a set.
\DeclareMathOperator{\Spec}{Spec}           % Spectrum.
\DeclareMathOperator{\supp}{supp}           % Support
\DeclareMathOperator{\Tr}{Tr}               % Trace of matrix.
%--------------------Declared Math Symbols--------------------%
\DeclareMathSymbol{\minus}{\mathbin}{AMSa}{"39} % Unary minus sign.
%------------------------New Commands---------------------------%
\DeclarePairedDelimiter\norm{\lVert}{\rVert}
\DeclarePairedDelimiter\ceil{\lceil}{\rceil}
\DeclarePairedDelimiter\floor{\lfloor}{\rfloor}
\newcommand*\diff{\mathop{}\!\mathrm{d}}
\newcommand*\Diff[1]{\mathop{}\!\mathrm{d^#1}}
\renewcommand*{\glstextformat}[1]{\textcolor{RoyalBlue}{#1}}
\renewcommand{\glsnamefont}[1]{\textbf{#1}}
\renewcommand\labelitemii{$\circ$}
\renewcommand\thesubfigure{%
    \arabic{chapter}.\arabic{figure}.\arabic{subfigure}}
\addto\captionsenglish{\renewcommand{\figurename}{Fig.}}
\numberwithin{equation}{section}

\renewcommand{\vector}[1]{\boldsymbol{\mathrm{#1}}}

\newcommand{\uvector}[1]{\boldsymbol{\hat{\mathrm{#1}}}}
\newcommand{\topspace}[2][]{(#2,\tau_{#1})}
\newcommand{\measurespace}[2][]{(#2,\varSigma_{#1},\mu_{#1})}
\newcommand{\measurablespace}[2][]{(#2,\varSigma_{#1})}
\newcommand{\manifold}[2][]{(#2,\tau_{#1},\mathcal{A}_{#1})}
\newcommand{\tanspace}[2]{T_{#1}{#2}}
\newcommand{\cotanspace}[2]{T_{#1}^{*}{#2}}
\newcommand{\Ckspace}[3][\mathbb{R}]{C^{#2}(#3,#1)}
\newcommand{\funcspace}[2][\mathbb{R}]{\mathcal{F}(#2,#1)}
\newcommand{\smoothvecf}[1]{\mathfrak{X}(#1)}
\newcommand{\smoothonef}[1]{\mathfrak{X}^{*}(#1)}
\newcommand{\bracket}[2]{[#1,#2]}

%------------------------Book Command---------------------------%
\makeatletter
\renewcommand\@pnumwidth{1cm}
\newcounter{book}
\renewcommand\thebook{\@Roman\c@book}
\newcommand\book{%
    \if@openright
        \cleardoublepage
    \else
        \clearpage
    \fi
    \thispagestyle{plain}%
    \if@twocolumn
        \onecolumn
        \@tempswatrue
    \else
        \@tempswafalse
    \fi
    \null\vfil
    \secdef\@book\@sbook
}
\def\@book[#1]#2{%
    \refstepcounter{book}
    \addcontentsline{toc}{book}{\bookname\ \thebook:\hspace{1em}#1}
    \markboth{}{}
    {\centering
     \interlinepenalty\@M
     \normalfont
     \huge\bfseries\bookname\nobreakspace\thebook
     \par
     \vskip 20\p@
     \Huge\bfseries#2\par}%
    \@endbook}
\def\@sbook#1{%
    {\centering
     \interlinepenalty \@M
     \normalfont
     \Huge\bfseries#1\par}%
    \@endbook}
\def\@endbook{
    \vfil\newpage
        \if@twoside
            \if@openright
                \null
                \thispagestyle{empty}%
                \newpage
            \fi
        \fi
        \if@tempswa
            \twocolumn
        \fi
}
\newcommand*\l@book[2]{%
    \ifnum\c@tocdepth >-3\relax
        \addpenalty{-\@highpenalty}%
        \addvspace{2.25em\@plus\p@}%
        \setlength\@tempdima{3em}%
        \begingroup
            \parindent\z@\rightskip\@pnumwidth
            \parfillskip -\@pnumwidth
            {
                \leavevmode
                \Large\bfseries#1\hfill\hb@xt@\@pnumwidth{\hss#2}
            }
            \par
            \nobreak
            \global\@nobreaktrue
            \everypar{\global\@nobreakfalse\everypar{}}%
        \endgroup
    \fi}
\newcommand\bookname{Book}
\renewcommand{\thebook}{\texorpdfstring{\Numberstring{book}}{book}}
\providecommand*{\toclevel@book}{-2}
\makeatother
\titleformat{\part}[display]
    {\Large\bfseries}
    {\partname\nobreakspace\thepart}
    {0mm}
    {\Huge\bfseries}
\titlecontents{part}[0pt]
    {\large\bfseries}
    {\partname\ \thecontentslabel: \quad}
    {}
    {\hfill\contentspage}
\titlecontents{chapter}[0pt]
    {\bfseries}
    {\chaptername\ \thecontentslabel:\quad}
    {}
    {\hfill\contentspage}
\newglossarystyle{longpara}{%
    \setglossarystyle{long}%
    \renewenvironment{theglossary}{%
        \begin{longtable}[l]{{p{0.25\hsize}p{0.65\hsize}}}
    }{\end{longtable}}%
    \renewcommand{\glossentry}[2]{%
        \glstarget{##1}{\glossentryname{##1}}%
        &\glossentrydesc{##1}{~##2.}
        \tabularnewline%
        \tabularnewline
    }%
}
\newglossary[not-glg]{notation}{not-gls}{not-glo}{Notation}
\newcommand*{\newnotation}[4][]{%
    \newglossaryentry{#2}{type=notation, name={\textbf{#3}, },
                          text={#4}, description={#4},#1}%
}
%--------------------------LENGTHS------------------------------%
% Spacings for the Table of Contents.
\addtolength{\cftsecnumwidth}{1ex}
\addtolength{\cftsubsecindent}{1ex}
\addtolength{\cftsubsecnumwidth}{1ex}
\addtolength{\cftfignumwidth}{1ex}
\addtolength{\cfttabnumwidth}{1ex}

% Indent and paragraph spacing.
\setlength{\parindent}{0em}
\setlength{\parskip}{0em}
%--------------------------Main Document----------------------------%
\begin{document}
    \ifx\ifworkmasterswork\undefined
        \section*{Preliminaries}
        \setcounter{section}{1}
    \fi
    \subsection{Vector Spaces and Euclidean Spaces}
        \begin{definition}
        A vector space is a set $V$ of vectors and a field $K$ of scalars with the following properties: For all $a,b\in K$, $\mathbf{u,v,w}\in V$:
        \begin{enumerate}
            \item $a\mathbf{v} \in V$. \hfill [Closure of Scalar Multiplication]
            \item $\mathbf{v}+\mathbf{u} \in V$. \hfill [Closure of Vector Addition]
            \item $\mathbf{u}+(\mathbf{v}+\mathbf{w}) = (\mathbf{u}+\mathbf{v})+\mathbf{w}$ \hfill [Vector Addition is Associative]
            \item $\mathbf{u}+\mathbf{v}=\mathbf{v}+\mathbf{u}$ \hfill [Vector Addition is Commutative]
            \item There exists a $\mathbf{0}\in V$ such that $\mathbf{0}+\mathbf{v}=\mathbf{v}$ for all $\mathbf{v}\in V$. \hfill [Existence of Zero Vector]
            \item $a(b\mathbf{v}) = (ab)\mathbf{v}$. \hfill [Associativity of Scalar Multiplication]
            \item $1 \mathbf{v} = \mathbf{v}$. \hfill[Multiplication by Scalar Identity]
            \item $a(\mathbf{v}+\mathbf{u}) = a\mathbf{v}+a\mathbf{u}$. \hfill [Scalar Multiplication Distributes of Vector Addition]
            \item $(a+b)\mathbf{v}= a\mathbf{v}+b\mathbf{v}$. \hfill [Scalar Multiplication Distributes over Field Addition]
        \end{enumerate}
        \end{definition}
        \begin{theorem}
        $\mathbf{0}$ is unique.
        \end{theorem}
        \begin{proof}
        For $\mathbf{0}'=\mathbf{0}'+\mathbf{0}=\mathbf{0}$.
        \end{proof}
        \begin{theorem}
        $0\mathbf{v} = \mathbf{0}$.
        \end{theorem}
        \begin{proof}
        For $\mathbf{v}+0\mathbf{v} = (1+0)\mathbf{v} = 1\mathbf{v} = \mathbf{v}$. As $\mathbf{0}$ is unique, $0\mathbf{v}=\mathbf{0}$.
        \end{proof}
        \begin{theorem}
        For all $\mathbf{v}\in V$, there exists a $\mathbf{u}$ such that $\mathbf{v}+\mathbf{u}=0$. That is, additive inverses exist.
        \end{theorem}
        \begin{proof}
        For let $\mathbf{u} = (-1)\mathbf{v}$. Then $\mathbf{v}+\mathbf{u} = \mathbf{v}+(-1)\mathbf{v} = (1+(-1))\mathbf{v} = 0\mathbf{v} = \mathbf{0}$.
        \end{proof}
        \begin{theorem}
        Inverses are unique.
        \end{theorem}
        \begin{proof}
        For $-\mathbf{v}'=-\mathbf{v}'+\mathbf{0}=-\mathbf{v}'+\mathbf{v}-\mathbf{v}=- \mathbf{v}$
        \end{proof}
        \begin{definition}
        A subspace of a vector space $V$ over a field $K$ is a vector space $W$ with the following properties:
        \begin{enumerate}
            \item $\mathbf{0} \in W$
            \item If $\mathbf{u,v}\in W$, then $\mathbf{u}+\mathbf{v} \in W$
            \item For all $a\in K$ and $\mathbf{V} \in W$, $a\mathbf{v} \in W$
        \end{enumerate}
        \end{definition}
        \begin{definition}
        An affine subspace of a vector space $V$ over a field $K$ is a subset $\xi\subset V$ such that $\xi = \{v+w:w\in W\}$, where $v$ is a fixed vector in $V$, and $W$ is a fixed subspace of $V$. That is, they are translations of subspaces.
        \end{definition}
        \begin{definition}
        An inner product on a vector space $V$ over a subfield $K$ of $\mathbb{R}$ is a function $\langle , \rangle:V\times V\rightarrow \mathbb{R}$ with the following properties: For all $x,y,z \in V,$ and $\alpha \in K$,
        \begin{enumerate}
            \item $\langle x,y \rangle = \langle y,x \rangle$ \hfill [Symmetry]
            \item $\langle \alpha x, y \rangle = \alpha \langle x,y \rangle$ \hfill [Linearity]
            \item $\langle x+y,z \rangle = \langle x,z\rangle + \langle y,z \rangle$ \hfill [Linearity]
            \item  If $x\ne \mathbf{0}$, then $\langle x,x\rangle >0$ \hfill [Positiveness]
        \end{enumerate}
        \end{definition}
        \begin{definition}
        An inner product space is a vector space with an inner product.
        \end{definition}
        \begin{theorem}
        $\langle x,y+z\rangle=\langle x,y\rangle+\langle x,z\rangle$
        \end{theorem}
        \begin{proof}
        For $\langle x,y+z\rangle=\langle y+z,x\rangle=\langle y,x\rangle+\langle z,x\rangle=\langle x,y\rangle+\langle x,z\rangle$
        \end{proof}
        \begin{theorem}
        $\langle x,x \rangle = 0$ if and only if $x= \mathbf{0}$.
        \end{theorem}
        \begin{proof}
        For $\langle \mathbf{0}, \mathbf{0} \rangle = \langle 0\mathbf{0},\mathbf{0} \rangle = 0 \langle \mathbf{0},\mathbf{0}\rangle = 0$. Suppose $\langle x,x \rangle =0$ but $x\ne \mathbf{0}$. But then $\langle x,x \rangle >0$, a contradiction.
        \end{proof}
        \begin{corollary}
        For all $x\in V$, $\langle \mathbf{0},x \rangle = 0$
        \end{corollary}
        \begin{proof}
        For $\langle \mathbf{0}, x\rangle = \langle 0\mathbf{0},x \rangle = 0\langle \mathbf{0},x\rangle = 0$.
        \end{proof}
        \begin{theorem}[Cauchy-Bunyakovsky-Schwarz Inequality]
        In an inner product space $V$, $x,y\in V\Rightarrow \langle x,y \rangle^2 \leq \langle x,x \rangle \langle y,y \rangle$
        \end{theorem}
        \begin{proof}
        $[y=\mathbf{0}]\Rightarrow [\langle x,y\rangle = 0]$. Suppose $y\ne \mathbf{0}$, and let $\lambda = \frac{\langle x,y \rangle}{\langle y,y \rangle}$. Then $[0 \leq \langle x-\lambda y, x-\lambda y\rangle = \langle x,x \rangle - 2\lambda \langle x,y \rangle + \lambda^2 \langle y,y \rangle]\Rightarrow [0\leq \langle x,x \rangle - 2\frac{\langle x,y \rangle ^2 }{\langle y,y \rangle} + \frac{\langle x,y \rangle^2}{\langle y,y \rangle} = \langle x,x \rangle - \frac{\langle x,y \rangle^2}{\langle y,y \rangle}]\Rightarrow [\frac{\langle x,y \rangle ^2}{\langle y,y \rangle} \leq \langle x,x \rangle]\Rightarrow [\langle x,y \rangle^2 \leq \langle x,x \rangle \langle y,y \rangle]$.
        \end{proof}
        \begin{definition}
        A norm on a vector space $V$ over a subfield $K$ of $\mathbb{R}$ is a function $\norm{}:V\rightarrow \mathbb{R}$ with the following properties: For all $x \in V$ and $\alpha \in K$:
        \begin{enumerate}
        \item $\norm{\alpha x} = |\alpha| \norm{x}$ \hfill [Absolute Homogeneity]
        \item $\norm{x+y} \leq \norm{x}+\norm{y}$ \hfill [Triangle Inequality]
        \item $\norm{x} = 0$ if and only if $x = \mathbf{0}$. \hfill [Definiteness]
        \end{enumerate}
        \end{definition}
        \begin{definition}
        A normed space is a vector space with a norm.
        \end{definition}
        \begin{theorem}
        If $V$ is a normed space, then for all $x\in V$, $0\leq \norm{x}$
        \end{theorem}
        \begin{proof}
        For $0=\norm{0} = \norm{\frac{x-x}{2}} \leq \norm{\frac{x}{2}}+\norm{-\frac{x}{2}} = \frac{1}{2}\norm{x} + \frac{1}{2}\norm{x} = \norm{x}$.
        \end{proof}
        \begin{definition}
        If $V$ is an inner product space, then the induced norm is $\norm{x}=\sqrt{\langle x,x \rangle}$.
        \end{definition}
        \begin{theorem}
        The induced norm is a norm.
        \end{theorem}
        \begin{proof}
        In order,
        \begin{enumerate}
        \item $\norm{\alpha x} = \sqrt{\langle \alpha x, \alpha x \rangle} = \sqrt{\alpha^2 \langle x,x\rangle} = |\alpha| \sqrt{\langle x,x \rangle} = |\alpha| \norm{x}$
        \item $\norm{x+y}^2= \langle x,x \rangle + 2\langle x,y \rangle + \langle y,y \rangle = \norm{x}^2 + 2\langle x,y \rangle + \norm{y}^2 \leq \norm{x}^2 +2\norm{x}\norm{y} + \norm{y}^2 = (\norm{x}+\norm{y})^2\Rightarrow \norm{x+y}\leq \norm{x}+\norm{y}$
        \item If $x= \mathbf{0}$, then $\sqrt{\langle x,x \rangle} = \sqrt{0} = 0$. If $\sqrt{\langle x,x \rangle} = 0$ then $\langle x,x \rangle = 0$, and thus $x = \mathbf{0}$.
        \end{enumerate}
        \end{proof}
        \begin{theorem}[Cauchy-Schwarz Inequality]
        If $V$ is an inner product space, then for all $x,y \in V$, $|\langle x,y\rangle| \leq \norm{x} \norm{y}$.
        \end{theorem}
        \begin{proof}
        For $\langle x,y \rangle^2 \leq \langle x,x \rangle \langle y,y \rangle = \norm{x}^2 \norm{y}^2 = (\norm{x}\norm{y})^2$. Thus, $|\langle x,y \rangle| \leq \norm{x}\norm{y}$
        \end{proof}
        \begin{definition}
        Euclidean $n$-space is defined as $\mathbb{R}^n=\{(x_1,\hdots, x_n):x_1,\hdots, x_n \in \mathbb{R}\}$, and has the following arithmetic: For all $x,y\in \mathbb{R}^n$, $\alpha \in \mathbb{R}$,
        \begin{enumerate}
        \item $x+y = (x_1+y_1,\hdots, x_n+y_n)$
        \item $\alpha x = (\alpha x_1,\hdots, \alpha x_n)$.
        \end{enumerate}
        \end{definition}
        \begin{theorem}
        $\mathbb{R}^n$, with its usual arithmetic, is a vector space over $\mathbb{R}$.
        \end{theorem}
        \begin{proof}
        In order (Laboriously): Let $\alpha, \beta \in \mathbb{R}$, $x,y,z\in \mathbb{R}^n$,
        \begin{enumerate}
        \item $\alpha x = \alpha(x_1,\hdots,x_n) = (\alpha x_1,\hdots, \alpha x_n)$. As $\alpha x_i \in \mathbb{R}$, $\alpha x \in \mathbb{R}^n$.
        \item $x+y = (x_1+y_1,\hdots,x_n+y_n)$. As $x_i+y_i \in \mathbb{R}$, $x+y\in \mathbb{R}^n$.
        \item $x+(y+z) = (x_1+(y_1+z_z),\hdots, x_n+(y_n+z_n)) = ((x_1+y_1)+z_1,\hdots, (x_n+y_n)+z_n) = (x+y)+z$
        \item $x+y = (x_1+y_1,\hdots,x_n+y_n) = (y_1+x_1,\hdots, y_n+x_n)=y+x$
        \item $\mathbf{0}+x = (0+x_1,\hdots, 0+x_n) = (x_1,\hdots, x_n) = x$.
        \item $\alpha(\beta x) = \alpha(\beta x_1,\hdots, \beta x_n) = \alpha \beta (x_1,\hdots, x_n) = (\alpha \beta) x$
        \item $1 x = (x_1,\hdots, x_n) = x$.
        \item
            \begin{align*}
                \alpha(x+y) &= \alpha(x_1+y_1,\hdots, x_n+y_n) & &= (\alpha x_1, \hdots, \alpha x_n) + (\alpha y_1,\hdots, \alpha y_n)\\
                &= (\alpha(x_1+y_n),\hdots, \alpha(x_n+y_n)) & &= \alpha(x_1,\hdots, x_n)+\alpha(y_1,\hdots, y_n)\\
                &= (\alpha x_1+\alpha y_1,\hdots, \alpha x_n + \alpha y_n) & &= \alpha x + \alpha y
            \end{align*} 
        \item
            \begin{align*}
                (\alpha + \beta)x &= ((\alpha+\beta)x_1,\hdots, (\alpha+\beta)x_n) & &= \alpha (x_1, \hdots, x_n)+\beta (x_1, \hdots, x_n)\\
                &= (\alpha x_1 + \beta x_1,\hdots, \alpha x_n + \beta x_n) & &= \alpha x+\beta x\\
                &= (\alpha x_1,\hdots, \alpha x_n) + (\beta x_1,\hdots, \beta x_n)
            \end{align*}
        \end{enumerate}
        \end{proof}
        \begin{definition}
        The dot product is a function $\cdot:\mathbb{R}^n \times \mathbb{R}^n \rightarrow \mathbb{R}$ defined as $x\cdot y = \sum_{i=1}^{n} x_iy_i$
        \end{definition}
        \begin{theorem}
        The dot product is an inner product on $\mathbb{R}^n$.
        \end{theorem}
        \begin{proof}
        In order,
        \begin{enumerate}
        \item $x\cdot y = \sum_{i=1}{n} x_i y_i = \sum_{i=1}^{n} y_i x_i = y\cdot x$
        \item $\alpha x\cdot y = \sum_{i=1}^{n} \alpha x_i y_i = \alpha \sum_{i=1}^{n} x_i y_i = \alpha x_i \cdot y_i$
        \item $(x+y)\cdot z = \sum_{i=1}^{n} (x_i+y_i)z_i = \sum_{i=1}^{n} (x_iz_i +y_i z_i)=\sum_{i=1}^{n}x_i z_i+\sum_{i=1}^{n} y_i z_i = x\cdot z + y\cdot z$
        \item $x\cdot x = \sum_{i=1}^{n} x_i^2 \geq 0$. Indeed, if $x\cdot x = 0$, then $x_i = 0$ for all $i=1,\hdots, n$, and thus $x=\mathbf{0}$.
        \end{enumerate}
        \end{proof}
        \begin{remark}
        The induced norm is thus $\norm{x} = \sqrt{\sum_{i=1}^{n} x_i^2}$.
        \end{remark}
        \begin{definition}
        A linear combination of vectors $\mathbf{v_i}$ in a vector space $V$ over a field $K$ is a sum $\sum_{i=1}^{n} a_i \mathbf{v_i}$, $a_i \in K$.
        \end{definition}
        \begin{definition}
        If $V$ is a vector space over $K$, and if $\mathbf{v_1},\hdots, \mathbf{v_n}\in V$, then they are said to be linearly dependent if and only if there are scalars $a_1,\hdots, a_n$, not all equal to zero, such that $\sum_{i=1}^{n} a_i \mathbf{v}_i = 0$.
        \end{definition}
        \begin{definition}
        If $V$ is a vector space over $K$, and if $\mathbf{v_1},\hdots, \mathbf{v_n}\in V$, then they are said to be linearly independent if and only if they are not linearly dependent.
        \end{definition}
        \begin{definition}
        A set of a vectors $\mathbf{v_1},\hdots, \mathbf{v_n}$ is said to span a vector space $V$ if and only if every element $x\in V$ can be written as a linear combination $x=\sum_{i=1}^{n} a_i \mathbf{v_i}$ for some scalars $a_i \in K$.
        \end{definition}
        \begin{definition}
        A basis of a vector space $V$ of a field $K$ is a set of linearly independent vectors that span $V$.
        \end{definition}
        \begin{definition}
        A vector space $V$ is said to be finite if it has a basis of finitely many elements.
        \end{definition}
        \begin{definition}
        The dimension of a vector space $V$, denoted $\dim(V)$, is the cardinality of the smallest basis of $V$.
        \end{definition}
        \begin{theorem}[The Dimension Theorem]
        If $V$ is a vector space and $\dim(V)=n$, then every basis of $V$ has $n$ vectors.
        \end{theorem}
        \begin{definition}
        If $V$ is a normed space, then an affine combination of vectors is $\sum_{k=1}^{n} \lambda_k v_k$, where $\lambda_k \in K$, $v_k \in V$, and $\sum_{k=1}^{n} \lambda_k = 1$.
        \end{definition}
        \begin{definition}
        A set of $n$ vectors $\{v_k:k\in \mathbb{Z}_n\}$ is said to be affinely dependent if and only if there exists scalars $\lambda_k \in K$, not all equal to zero, such that $\sum_{k=1}^{n} \lambda_k v_k = \mathbf{0}$ and $\sum_{k=1}^{n} \lambda_k = 0$.
        \end{definition}
        \begin{definition}
        A set of vectors is said to be affinely independent if and only if there are not affinely dependent.
        \end{definition}
        \begin{theorem}
        A set $\{v_k:k\in \mathbb{Z}_n\}$ is affinely independent if and only if $\{v_k-v_1:k\in \mathbb{Z}_n, k>1\}$ is linearly independent.
        \end{theorem}
        \begin{proof}
        Suppose the latter set is linearly independent. Then $\sum_{k=1}^{n} \lambda_k(v_k-v_1) \ne 0$ if at least one $\lambda_k \ne 0$. Let $\lambda_k$ be any sequence such that $\sum_{k=1}^{n} \lambda_k = 0$, but not all $\lambda_k$ are zero. Then $\sum_{k=1}^{n} \lambda_k(v_k-v_1)\ne 0$. Let this sum be $c_{\lambda}$. Then $\sum_{k=1}^{n} \lambda_k v_k = \sum_{k=1}^{n} \lambda_k v_1 + c_\lambda = c_{\lambda}$. Thus, the set $v_k$ is affinely independent. Suppose the former set is affinely independent. Then $\sum_{k=1}^{n} \lambda_k v_k = \mathbf{0} \Rightarrow \sum_{k=1}^{n} \lambda \ne 0$. But then $\sum_{k=1}^{n}\lambda_k (v_k-v_1) = - \sum_{k=1}^{n} \lambda_k v_1 \ne 0$. Thus, the latter set is linearly independent.
        \end{proof}
        \begin{corollary}
        For any $n-dimensional$ vector space $V$, any set of affinely independent vectors has at most $n+1$ vectors.
        \end{corollary}
        \begin{proof}
        If $v_k$ are affinely independent, then $v_k-v_1$ is linearly independent. The latter has at most $n$ vectors. Thus, etc.
        \end{proof}
        \begin{corollary}
        If $v_k$ are affinely independent and $\sum_{k=1}^{n}\lambda_k v_k = \sum_{k=1}^{n} \sigma_k v_k$, then $\lambda_k = \sigma_k$ for all $k$.
        \end{corollary}
        \begin{proof}
        $[\sum_{k=1}^{n}(\lambda_k - \sigma_k)v_k = 0]\land[\sum_{k=1}{^n}(\lambda_k-\sigma_k) = 0]\Rightarrow [\lambda_k-\sigma_k = 0]$. Therefore, etc.
        \end{proof}
        \begin{definition}
        The Affine Hull of a set $S\subset V$ of some normed space $V$ is $\textrm{aff}(S) = \{\sum_{i=1}^{m}\lambda_i x_i: x_i \in S\land \sum_{i=1}^{m}\lambda_i =1\}$.
        \end{definition}
        \begin{definition}
        If $V$ is a normed space, a convex combination is $\sum_{i=1}^{n}|\lambda_i| v_i$, where $v_i\in V$, $\sum_{i=1}^{n}|\lambda_i| = 1$.
        \end{definition}
        \begin{definition}
        For a normed space $V$, the Convex Hull of $S\subset V$ is $\textrm{conv}(S)=\{\sum_{i=1}^{n}|\lambda_i| x_i:x_i\in S\land \sum_{i=1}^{n} |\lambda_i| = 1 \}$.
        \end{definition}
        \begin{definition}
        In an inner product space $V$, $v,w\in V$ are said to be orthogonal, $v\perp w$, if and only if $\langle v,w \rangle = 0$.
        \end{definition}
        \begin{notation}
        In an inner product space $V$, $W\subset V$, $x\in V$, and $y\in W \Rightarrow \langle x,y\rangle = 0$, we write $x\perp W$.
        \end{notation}
        \begin{notation}
        Similarly for orthogonal subsets $W$ and $U$ of $V$, we write $W\perp U$.
        \end{notation}
        \begin{definition}
        A line in $\mathbb{R}^n$ containing the points $x,y\in \mathbb{R}^n$ is the set $\{\lambda y + (1-\lambda)x: \lambda \in \mathbb{R}\}$.
        \end{definition}
        \begin{definition}
        A line segment in $\mathbb{R}^n$ that begins at $x$ and terminates at $y$ is the set $\{\lambda y + (1-\lambda)x: 0\leq \lambda \leq 1 \}$.
        \end{definition}
        \begin{definition}
        If $W\underset{Subspace}\subset\mathbb{R}^n$, $K \subset \mathbb{R}^n$, then the orthogonal projection is $K_{W}\equiv\{x\in W: \exists y\in K: y-x \perp W\}$.
        \end{definition}
        \begin{theorem}
        If $W$ is a subspace of $\mathbb{R}^n$, $K \subset \mathbb{R}^n$, and $x\in K_{W}$, then there is a line through $x$ and a point $y\in K$ such that, for any $\alpha, \beta$ contained on said line, $\beta-\alpha \perp W$.
        \end{theorem}
        \begin{proof}
        For let $x\in W$. Then there is a $y\in K$ such that, for all $z\in W$, $\langle y-x,z\rangle = 0$. Let $\Gamma$ be the line $\lambda y + (1-\lambda)x$. If $\alpha,\beta \in \Gamma$, there are values $\lambda_1$ and $\lambda_2$ such that $\alpha = \lambda_1 y+ (1-\lambda_1)x$ and $\beta = \lambda_2 y +(1-\lambda_2)x$. But then $\beta-\alpha = y(\lambda_2-\lambda_1)-x(\lambda_2-\lambda_1) = (\lambda_2-\lambda_1)(x-y)$. But then $\langle \beta - \alpha,z\rangle = \langle (\lambda_2-\lambda_1)(x-y),z\rangle = (\lambda_2-\lambda_1)\langle x-y,z \rangle = 0$.
        \end{proof}
        \begin{remark}
        From the Cauchy-Schwarz inequality, we have that $|\langle x,y \rangle| \leq \norm{x}\norm{y}$, and thus $|\frac{\langle x,y \rangle}{\norm{x}\norm{y}}| \leq 1$. We define the $angle\ \theta$ between two non-zero vectors $x,y\in \mathbb{R}^n$ as $\theta=\cos^{-1}\big(\frac{\langle x,y \rangle}{\norm{x}\norm{y}}\big)$. We omit rigorous definition of the cosine function.
        \end{remark}
        \begin{definition}
        If $\mathcal{U},\mathcal{V}\subset V$, then $\mathcal{U}+\mathcal{V} = \{x+y:x\in \mathcal{U},y\in \mathcal{V}\}$.
        \end{definition}
        \begin{definition}
        If $V$ is a vector space over $K$, $\mathcal{U}\subset V$, and $\alpha \in K$, then $\alpha \mathcal{U} = \{\alpha x:x\in \mathcal{U}\}$.
        \end{definition}
\end{document}