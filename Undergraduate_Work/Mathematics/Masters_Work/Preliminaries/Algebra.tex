\documentclass[crop=false,class=article,oneside]{standalone}
%----------------------------Preamble-------------------------------%
%---------------------------Packages----------------------------%
\usepackage{geometry}
\geometry{b5paper, margin=1.0in}
\usepackage[T1]{fontenc}
\usepackage{graphicx, float}            % Graphics/Images.
\usepackage{natbib}                     % For bibliographies.
\bibliographystyle{agsm}                % Bibliography style.
\usepackage[french, english]{babel}     % Language typesetting.
\usepackage[dvipsnames]{xcolor}         % Color names.
\usepackage{listings, lstlinebgrd}      % Verbatim-Like Tools.
\usepackage{mathtools, esint, mathrsfs} % amsmath and integrals.
\usepackage{amsthm, amsfonts}           % Fonts and theorems.
\usepackage{tabularx}
\usepackage{tcolorbox}                  % Frames around theorems.
\usepackage{upgreek}                    % Non-Italic Greek.
\usepackage{paracol}                    % Two-column styling.
\usepackage{wrapfig}                    % Wrap text around figure.
\usepackage{fmtcount, etoolbox}         % For the \book{} command.
\usepackage[newparttoc]{titlesec}       % Formatting chapter, etc.
\usepackage{titletoc}                   % Allows \book in toc.
\usepackage[nottoc]{tocbibind}          % Bibliography in toc.
\usepackage[titles]{tocloft}            % ToC formatting.
\usepackage{multicol, enumitem}         % Multi-column/enumerate.
\usepackage{import}                     % Import external files.
\usepackage{pgfplots, tikz}             % Drawing/graphing tools.
\usetikzlibrary{
    calc,                   % Calculating right angles and more.
    angles,                 % Drawing angles within triangles.
    arrows.meta,            % Latex and Stealth arrows.
    quotes,                 % Adding labels to angles.
    positioning,            % Relative positioning of nodes.
    decorations.markings,   % Adding arrows in the middle of a line.
    patterns,
    arrows,
    shapes,
    shapes.geometric,
    cd,
    hobby,
    babel
}                                       % Libraries for tikz.
\pgfplotsset{compat=1.9}                % Version of pgfplots.
\usepackage[font=scriptsize,
            labelformat=simple,
            labelsep=colon]{subcaption} % Subfigure captions.
\usepackage[font={scriptsize},
            hypcap=true,
            labelsep=colon]{caption}    % Figure captions.
\usepackage{hyperref}                   % Allows for hyperlinks.
\hypersetup{
    colorlinks=true,
    linkcolor=blue,
    filecolor=magenta,
    urlcolor=Cerulean,
    citecolor=SkyBlue
}                           % Colors for hyperref.
\usepackage[toc,acronym,nogroupskip]{glossaries} % Glossaries and acronyms.
\usepackage[subpreambles=false]{standalone}      % Complileable sub files.

% Various font stuff from kiwi.
% Use this for Times text and Computer Modern math
%\usepackage{times}

% Quite nice
%\usepackage[charter, greekfamily=, greekuppercase=italicized]{mathdesign}
%\usepackage[utopia, greekuppercase=italicized]{mathdesign}    % Math is narrower

% Use this for Times text and math
%\usepackage{newtxtext}
%\usepackage[libertine,cmintegrals]{newtxmath}
%\usepackage{fix-cm}

%\usepackage{txfontsb}
% or
%\usepackage{mathptmx}

%\usepackage[scaled=0.92]{helvet}
%\renewcommand{\rmdefault}{ptm}

%\usepackage{mathpazo}    % add possibly `sc` and `osf` options
%\usepackage{eulervm}

%\usepackage{fourier}
%\renewcommand{\rmdefault}{ptm}
%\usepackage{mathptm}

%\usepackage{fontspec}
%\setmainfont{lmodern}

%\usepackage[varg]{txfonts}
%\usepackage{fouriernc}
%\usepackage{mathpazo}

%\usepackage{bookman}
%\usepackage[scaled]{uarial}
%\usepackage[scaled]{helvet}
%\renewcommand*\familydefault{\sfdefault}
%\usepackage[math]{anttor}

%\newcommand\fgeorgia{\fontfamily{jvn}\selectfont}
%\newcommand\ftimes{\fontfamily{ptm}\selectfont}
%\newcommand\fhelvetica{\fontfamily{phv}\selectfont}
%\newcommand\fcourier{\fontfamily{pcr}\selectfont}
%\newcommand\fbookman{\fontfamily{pbk}\selectfont}
%\newcommand\fnewcentury{\fontfamily{pnc}\selectfont}
%\newcommand\fpalatino{\fontfamily{ppl}\selectfont}
%\newcommand\favantgarde{\fontfamily{pag}\selectfont}
%\newcommand\fnormal{\normalfont}
%\newcommand\fsize[1]{\ifnum#1>0\fontsize{#1}{#1}\selectfont\else\normalsize\fi}
%------------------------Theorem Styles-------------------------%
% Define theorem style for default spacing and normal font.
\newtheoremstyle{normal}
    {\topsep}               % Amount of space above the theorem.
    {\topsep}               % Amount of space below the theorem.
    {}                      % Font used for body of theorem.
    {}                      % Measure of space to indent.
    {\bfseries}             % Font of the header of the theorem.
    {}                      % Punctuation between head and body.
    {.5em}                  % Space after theorem head.
    {}

% Define theorem style for default spacing with italicized font.
\newtheoremstyle{normalit}{\topsep}{\topsep}
                {\itshape}{}{\bfseries}{}{.5em}{}

% Italic header environment.
\newtheoremstyle{thmit}{\topsep}{\topsep}{}{}{\itshape}{}{0.5em}{}

% Define italicized environments.
\theoremstyle{normalit}
\newtheorem{theorem}{Theorem}[section]
\newtheorem{lemma}{Lemma}[section]
\newtheorem{corollary}{Corollary}[section]
\newtheorem{proposition}{Proposition}[section]
\newtheorem*{theorem*}{Theorem}

% Define environments with italic headers.
\theoremstyle{thmit}
\newtheorem*{solution}{Solution}
\newtheorem*{fsolution}{Solution}

% Define default environments.
\theoremstyle{normal}
\newtheorem{example}{Example}[section]
\newtheorem{definition}{Definition}[section]
\newtheorem{problem}{Problem}[section]
\newtheorem{question}{Question}[section]
\newtheorem{remark}{Remark}[section]
\newtheorem{properties}{Properties}[section]
\newtheorem{notation}{Notation}[section]
\newtheorem{axiom}{Axiom}[section]
\newtheorem*{properties*}{Properties}
\newtheorem*{remark*}{Remark}
\newtheorem*{definition*}{Definition}
\theoremstyle{plain}

% Define framed environment.
\tcbuselibrary{most}
\newtcbtheorem[use counter*=theorem]{ftheorem}{Theorem}%
    {colback=green!5,colframe=green!35!black,
     fonttitle=\bfseries\upshape}{th}

\newtcbtheorem[use counter*=example]{fdefinition}{Definition}%
    {fonttitle=\bfseries\upshape,
     colback=blue!5!white,colframe=blue!75!black}{def}

\newtcbtheorem[use counter*=example]{fexample}{Example}%
    {fonttitle=\bfseries\upshape,
     colback=red!5!white,colframe=red!75!black}{ex}

\newtcbtheorem[use counter*=notation]{fnotation}{Notation}%
    {fonttitle=\bfseries\upshape,
     colback=SeaGreen!5!white,colframe=SeaGreen!75!black}{ex}

\newtcbtheorem[use counter*=corollary]{fcorollary}{Corollary}%
    {fonttitle=\bfseries\upshape,
     colback=Orchid!5!white,colframe=Orchid!75!black}{ex}

\newenvironment{bproof}{\textit{Proof.}}{\hfill$\square$}
\tcolorboxenvironment{bproof}{blanker,breakable,left=5mm,
                             before skip=10pt,after skip=10pt,
                             borderline west={1mm}{0pt}{red}}
\tcolorboxenvironment{fsolution}
    {enhanced jigsaw,colframe=cyan,interior hidden,breakable}

%--------------------Declared Math Operators--------------------%
\DeclareMathOperator{\Refl}{Refl}           % Reflection operator.
\DeclareMathOperator{\Span}{Span}           % Span of a set of vectors.
\DeclareMathOperator{\Card}{Card}           % Cardinality of set.
\DeclareMathOperator{\Ord}{Ord}             % Ordinal of ordered set.
\DeclareMathOperator{\Tr}{Tr}               % Trace of matrix.
\DeclareMathOperator{\adjoint}{adj}         % Adjoint of matrix.
\DeclareMathOperator{\rk}{rk}               % Rank of operator.
\DeclareMathOperator{\nul}{nul}             % Null space of operator.
\DeclareMathOperator{\sgn}{sgn}             % Sign of a number.
\DeclareMathOperator{\multideg}{mutlideg}   % Multi-Degree (Graphs).
\DeclareMathOperator{\GCD}{GCD}             % Greatest common denominator.
\DeclareMathOperator{\LM}{LM}               % Leading monomial
\DeclareMathOperator{\LC}{LC}               % Leading coefficient.
\DeclareMathOperator{\LT}{LT}               % Leading term.
\DeclareMathOperator{\LCM}{LCM}             % Least common multiple.
\DeclareMathOperator{\Mon}{Mon}             % Monomial.
\DeclareMathOperator{\Spec}{Spec}           % Spectrum.
\DeclareMathOperator{\proj}{proj}           % Projection.
\DeclareMathOperator{\comp}{comp}           % Component.
\DeclareMathOperator{\sinc}{sinc}           % Sinc function.
\DeclareMathOperator{\Ima}{Im}              % Image of operator.
\DeclareMathOperator{\Prin}{Prin}           % Principal value.
\DeclareMathOperator{\Mod}{mod}             % Modulus.
%------------------------New Commands---------------------------%
\DeclarePairedDelimiter\norm{\lVert}{\rVert}
\DeclarePairedDelimiter\ceil{\lceil}{\rceil}
\DeclarePairedDelimiter\floor{\lfloor}{\rfloor}
\newcommand*\diff{\mathop{}\!\mathrm{d}}
\newcommand*\Diff[1]{\mathop{}\!\mathrm{d^#1}}
\renewcommand{\mod}{\ \Mod}
\renewcommand*{\glstextformat}[1]{\textcolor{RoyalBlue}{#1}}
\renewcommand{\glsnamefont}[1]{\textbf{#1}}
\renewcommand\labelitemii{$\circ$}
\renewcommand\thesubfigure{\arabic{chapter}.\arabic{figure}}
\renewcommand\thesubfigure{%
    \arabic{chapter}.\arabic{figure}.\arabic{subfigure}}
\addto\captionsenglish{\renewcommand{\figurename}{Fig.}}
%------------------------Book Command---------------------------%
\makeatletter
\renewcommand\@pnumwidth{1cm}
\newcounter{book}
\renewcommand\thebook{\@Roman\c@book}
\newcommand\book{%
    \if@openright
        \cleardoublepage
    \else
        \clearpage
    \fi
    \thispagestyle{plain}%
    \if@twocolumn
        \onecolumn
        \@tempswatrue
    \else
        \@tempswafalse
    \fi
    \null\vfil
    \secdef\@book\@sbook
}
\def\@book[#1]#2{%
    \ifnum \c@secnumdepth >-3\relax
        \refstepcounter{book}%
        \addcontentsline{toc}{book}{
            \bookname\ \thebook:\hspace{1em}#1
        }
    \else
        \addcontentsline{toc}{book}{#1}%
    \fi
    \markboth{}{}%
    {\centering
     \interlinepenalty \@M
     \normalfont
     \ifnum \c@secnumdepth >-2\relax
       \huge\bfseries \bookname\nobreakspace\thebook
       \par
       \vskip 20\p@
     \fi
     \Huge \bfseries #2\par}%
    \@endbook}
\def\@sbook#1{%
    {\centering
     \interlinepenalty \@M
     \normalfont
     \Huge \bfseries #1\par}%
    \@endbook}
\def\@endbook{
    \vfil\newpage
        \if@twoside
            \if@openright
                \null
                \thispagestyle{empty}%
                \newpage
            \fi
        \fi
        \if@tempswa
            \twocolumn
        \fi
}
\newcommand*\l@book[2]{%
    \ifnum \c@tocdepth >-2\relax
        \addpenalty{-\@highpenalty}%
        \addvspace{2.25em \@plus\p@}%
        \setlength\@tempdima{3em}%
        \begingroup
            \parindent \z@ \rightskip \@pnumwidth
            \parfillskip -\@pnumwidth
            {
                \leavevmode
                \Large \bfseries #1\hfil \hb@xt@\@pnumwidth{
                    \hss #2
                }
            }
            \par
            \nobreak
            \global\@nobreaktrue
            \everypar{\global\@nobreakfalse\everypar{}}%
        \endgroup
    \fi}
\newcommand\bookname{Book}
\renewcommand{\thebook}{\texorpdfstring{\Numberstring{book}}{book}}
\providecommand*{\toclevel@book}{-2}
\makeatother
\titlecontents{chapter}[0pt]
    {\bfseries}
    {\chaptername\ \thecontentslabel:\quad}
    {}
    {\hfill\contentspage}
\titleformat{\part}[display]
    {\Large\bfseries}
    {\partname\nobreakspace\thepart}
    {0mm}
    {\Huge\bfseries}
    \titlecontents{part}[0pt]
    {\large\bfseries}
    {\partname\ \thecontentslabel: \quad}
    {}
    {\hfill\contentspage}
\newcommand{\MarkRightAngle}[4][.3cm]
    {\coordinate (tempa) at ($(#3)!#1!(#2)$);
     \coordinate (tempb) at ($(#3)!#1!(#4)$);
     \coordinate (tempc) at ($(tempa)!0.5!(tempb)$);%midpoint
     \draw (tempa) -- ($(#3)!2!(tempc)$) -- (tempb);}
%--------------------------LENGTHS------------------------------%
% Spacings for the Table of Contents.
\addtolength{\cftsecnumwidth}{1ex}
\addtolength{\cftsubsecindent}{1ex}
\addtolength{\cftsubsecnumwidth}{1ex}
\addtolength{\cftfignumwidth}{1ex}
\addtolength{\cfttabnumwidth}{1ex}

% Spacing for multi-column and enumerate environments.
\setlength{\multicolsep}{6pt}
\setlist[enumerate]{itemsep=0pt,topsep=3pt}

% Indent and paragraph spacing.
\setlength{\parindent}{0em}
\setlength{\parskip}{0em}
%--------------------------Main Document----------------------------%
\begin{document}
    \ifx\ifworkmasterswork\undefined
        \section*{Preliminaries}
        \setcounter{section}{1}
    \fi
    \subsection{Algebra}
        \begin{definition}
        If $A$ is a set, a relation $R$ on $A$ is a subset of $A\times A$. If $a,b\in A$ and $(a,b)\in R$, we write $aR b$.
        \end{definition}
        \begin{remark}
        For a relation $R$ it is not necessary true that $aRb$ implies $bRa$, nor is it necessarily true that $aRa$.
        \end{remark}
        \begin{definition}
        A relation $R$ on a set $A$ is said to be reflexive if and only if $\forall a\in A$, $aRa$.
        \end{definition}
        \begin{definition}
        A relation $R$ on a set $A$ is said to be symmetric if and only if $\forall a,b\in A$, $aRb\Leftrightarrow bRa$.
        \end{definition}
        \begin{definition}
        A relation $R$ on a set $A$ is said to be transitive if and only if $\forall a,b,c\in A$, $aRb \land bRc \Rightarrow aRc$.
        \end{definition}
        \begin{definition}
        A relation $R$ on a set $A$ is said to be asymmetric if and only if $\forall a,b\in A$, $(a,b)\in R\Rightarrow (b,a) \notin R$.
        \end{definition}
        \begin{definition}
        A relation on $R$ is said to be total if and only if $\forall a,b \in A$, either $aRb$, $bRa$, or both.
        \end{definition}
        \begin{definition}[Relation of Equality]
        Equality is a relation with the following properties:
        \begin{enumerate}
        \item Equality is Reflexive: $a=a$ for all $a\in A$.
        \item Equality is Symmetric: $a=b$ if and only if $b=a$.
        \item Equality is Transitive: If $a=b$ and $b=c$, then $a=c$.
        \item The relation is uniquely defined by the set $\{(a,a)\in A\times A:a\in A\}$.
        \end{enumerate}
        \end{definition}
        \begin{definition}
        A relation $R$ on a set $A$ is said to be antisymmetric if and only if $\forall a,b \in A$, $aRb\land bRa\Rightarrow a=b$.
        \end{definition}
        \begin{definition}
        A function $f:A\rightarrow B$ is a subset of $A\times B: \forall a\in A$, $\exists b\in B: (a,b)\in f$ and $[(a,b)\in f\land (a,c)\in f]\Leftrightarrow [b=c]$. The image of $a\in A$ is the unique $b\in B:(a,b)\in f$, denoted $a\mapsto b$ or $f(a)=b$. Functions are also called maps/mappings.
        \end{definition}
        \begin{definition}
        An indexing set is a set whose elements label some other set.
        \end{definition}
        \begin{example}
        If $A$ is an indexing set, then $\{\mathcal{U}_{\alpha}:\alpha \in A\}$ is a set of elements $\mathcal{U}_{\alpha}$, and there is one for each $\alpha \in A$.
        \end{example}
        \begin{axiom}
        If $X$ is a set of nonempty sets $\mathcal{U}_{\alpha}$, indexed over $A$, then $\exists f:X\rightarrow \underset{\alpha \in A}\cup \mathcal{U}_{\alpha}$ such that $f(\mathcal{U}_{\alpha}) \in \mathcal{U}_{\alpha}$, $\forall \alpha\in A$.
        \end{axiom}
        \begin{remark}
        This is called the axiom of choice. It is a blatantly obvious statement, however many of the results it gives are far from intuitive. For those interested, the axiom of choice is consistent with modern set theory (Called Zermelo-Fraenkel set theory, or ZF). It may thus be rejected or accepted without logical contradiction. We shall accept it.
        \end{remark}
        \begin{definition}
        If $f:A\rightarrow B$ and if $\mathcal{O}\subset A$, then the image of $\mathcal{O}$ under $f$ is the set $f(\mathcal{O}) = \{f(a)\in B:a\in \mathcal{O}\}$.
        \end{definition}
        \begin{definition}
        If $f:A\rightarrow B$ and $\mathscr{O}\subset B$, the preimage is the set $f^{-1}(\mathscr{O}) = \{x\in A:f(x)\in \mathscr{O}\}$.
        \end{definition}
        \begin{corollary}
        If $f:A\rightarrow B$, $f(\emptyset) = f^{-1}(\emptyset) = \emptyset$.
        \end{corollary}
        \begin{proof}
        $[y\in f(\emptyset)]\Rightarrow [\exists x\in \emptyset:f(x)=y]$. A contradiction. Therefore, etc.
        \end{proof}
        \begin{theorem}
        If $f:A\rightarrow B$, $B_1\subset B$, then $f(f^{-1}(B_1))\subset B_1$.
        \end{theorem}
        \begin{proof}
        $[f(x)\in f(f^{-1}(B_1))]\Rightarrow [x\in f^{-1}(B_1)]\Rightarrow [f(x)\in B_1]$.
        \end{proof}
        \begin{theorem}
        If $f:A\rightarrow B$, $A_1\subset A$, then $A_1\subset f^{-1}(f(A_1))$.
        \end{theorem}
        \begin{proof}
        $[x\in A_1]\Rightarrow [f(x) \in f(A_1)]\Rightarrow x\in f^{-1}(f(A_1))$.
        \end{proof}
        \begin{theorem}
        If $f:A\rightarrow B$, $A_1\subset A$, then $f(A_1) = \emptyset \Leftrightarrow A_1 = \emptyset$.
        \end{theorem}
        \begin{proof}
        If $A_1 = \emptyset$, we are done. If not, let $x\in A_1$. Then $f(x)\in f(A_1)$, and thus $f(A_1)\ne \emptyset$. Therefore, etc.
        \end{proof}
        \begin{corollary}
        If $f:A\rightarrow B$, $A_1\subset A_2\subset A$, then $f(A_1)\subset f(A_2)$.
        \end{corollary}
        \begin{proof}
        $[y\in f(A_1)]\Rightarrow[\exists x\in A_1:f(x)=y]\Rightarrow [x\in A_2] \Rightarrow [f(x)\in f(A_2)]$
        \end{proof}
        \begin{corollary}
        If $f:A\rightarrow B$, $B_1\subset B_2\subset B$, then $f^{-1}(B_1)\subset f^{-1}(B_2)$.
        \end{corollary}
        \begin{proof}
        $[x\in f^{-1}(B_1)] \Rightarrow [f(x) \in B_1] \Rightarrow [f(x) \in B_2]\Rightarrow [x\in f^{-1}(B_2)]$.
        \end{proof}
        \begin{theorem}
        If $f:A\rightarrow B$, $A_1,A_2\subset A$, then $f(A_1 \cup A_2) = f(A_1)\cup f(A_2)$.
        \end{theorem}
        \begin{proof}
        $[y\in f(A_1\cup A_2)]\Rightarrow [\exists x\in A_1 \cup A_2:y=f(x)]\Rightarrow [y \in f(A_1)\cup f(A_2)]$. $[y\in f(A_1)\cup f(A_2)]\Rightarrow \big[[\exists x\in A_1] \lor [\exists x\in A_2]: y=f(x)\big]\Rightarrow [x\in A_1\cup A_2]\Rightarrow [f(x)\in f(A_1\cup A_2)]$
        \end{proof}
        \begin{theorem}
        If $f:A\rightarrow B$, $A_1,A_2\subset A$, then $f(A_1\cap A_2)\subset f(A_1)\cap f(A_2)$.
        \end{theorem}
        \begin{proof}
        $[y\in f(A_1 \cap A_2)]\Rightarrow [\exists x\in A_1 \cap A_2:y=f(x)]\Rightarrow [x\in A_1 \land x \in A_2] \Rightarrow[y \in f(A_1)\cap f(A_2)]$.
        \end{proof}
        \begin{theorem}
        If $f:A\rightarrow B$, $B_1,B_2\subset B$, then $f^{-1}(B_1\cup B_2) = f^{-1}(B_1)\cup f^{-1}(B_2)$.
        \end{theorem}
        \begin{proof}
        $[x\in B_1\cup B_2]\Rightarrow [f(x)\in B_1\cup B_2]\Rightarrow [f(x)\in B_1\lor f(x)\in B_2]\Rightarrow [x\in f^{-1}(B_1)\cup f^{-1}(B_2)]$. $[x \in f^{-1}(B_1)\cup f^{-1}(B_2)]\Rightarrow [f(x)\in B_1\lor f(x) \in B_2]\Rightarrow [f(x) \in B_1\cup B_2]\Rightarrow [x\in f^{-1}(B_1\cup B_2)]$.
        \end{proof}
        \begin{theorem}
        If $f:A\rightarrow B$, $B_1,B_2\subset B$, then $f^{-1}(B_1\cap B_2) = f^{-1}(B_1)\cap f^{-1}(B_2)$.
        \end{theorem}
        \begin{proof}
        $[x\in f^{-1}(B_1\cap B_2)]\Rightarrow [f(x) \in B_1 \cap B_2]\Rightarrow [f(x)\in B_1\land f(x) \in B_2 ]\Rightarrow [x\in f^{-1}(B_1)\cap f^{-1}(B_2)]$. $[x\in f^{-1}(B_1)\cap f^{-1}(B_2)]\Rightarrow [x\in f^{-1}(B_1)\land x\in f^{-1}(B_2)]\Rightarrow [f(x) \in B_1\land f(x) \in B_2]\Rightarrow [f(x)\in B_1\cap B_2]\Rightarrow [x\in f^{-1}(B_1\cap B_2)]$.
        \end{proof}
        \begin{theorem}
        If $f:A\rightarrow B$, $B_1 \subset B$, then $f^{-1}(B\setminus B_1) = f^{-1}(B)\setminus f^{-1}(B_1)$.
        \end{theorem}
        \begin{proof}
        $[x\in f^{-1}(B\setminus B_1)]\Leftrightarrow [f(x)\notin B_1]\Leftrightarrow [x\in f^{-1}(B)\setminus f^{-1}(B_1)]$
        \end{proof}
        \begin{remark}
        If $f:A\rightarrow B$, the image of $A$ under $f$ is often called the range (A is often called the domain).
        \end{remark}
        \begin{definition}
        A function $f:A\rightarrow B$ is said to be injective if and only if $\big[[(a,b)\in f]\land[(a',b)\in f]\big]\Leftrightarrow [a=a']$.
        \end{definition}
        \begin{definition}
        If $f:A\rightarrow B$ is injective, then the inverse $f^{-1}:f(A)\rightarrow A$ is defined by $f^{-1}(y)=x:y=f(x)$.
        \end{definition}
        \begin{definition}
        A function $f:A\rightarrow B$ is said to be surjective if and only if $f(A) = B$.
        \end{definition}
        \begin{definition}
        A function is said to be bijective if and only if it is injective and surjective.
        \end{definition}
        \begin{theorem}
        If $f:A\rightarrow B$ is bijective, then $f^{-1}$ is bijective.
        \end{theorem}
        \begin{proof}
        $[f^{-1}(y_1) = f^{-1}(y_2)]\Rightarrow [\exists x\in A:[f(x) = y_1]\land [f(x)=y_2]]\Rightarrow [y_1=y_2]$. By definition, $f^{-1}$ is surjective.
        \end{proof}
        \begin{definition}
        If $f:A\rightarrow B$ and $g:B\rightarrow C$, then $g\circ f:A\rightarrow C$ is defined by the image $g(f(x)), x\in A$. 
        \end{definition}
        \begin{theorem}
        If $f:A\rightarrow B$, $g:B\rightarrow C$, and $\mathcal{V}\subset C$, then $(g\circ g)^{-1}(\mathcal{V}) = f^{-1}(g^{-1}(\mathcal{V}))$.
        \end{theorem}
        \begin{proof}
        $[x\in (g\circ f)^{-1}(\mathcal{V})]\Leftrightarrow [g(f(x))\in \mathcal{V}] \Leftrightarrow [f(x)\in g^{-1}(\mathcal{V})]\Leftrightarrow [x\in f^{-1}(g^{-1}(\mathcal{V}))]$.
        \end{proof}
        \begin{theorem}
        If $f:A\rightarrow B$ is bijective, $g:B\rightarrow C$ is bijective, then $g\circ f$ is bijective.
        \end{theorem}
        \begin{proof}
        $\big[[f(A) = B]\land [g(B) = C]\big]\Rightarrow [g(f(A)) = g(B) = C]$. $[g(f(x_1))=g(f(x_2))]\Leftrightarrow [f(x_1)=f(x_2)]\Leftrightarrow [x_1=x_2]$.
        \end{proof}
        \begin{theorem}
        If $f:A\rightarrow B$ is bijective, $A_1\subset A$, and $f(A_1) = B$, then $A_1=A$.
        \end{theorem}
        \begin{proof}
        $\Big[\big[[A_1^c \ne \emptyset]\Rightarrow [f(A_1^c) \ne \emptyset]\big]\land[f(A_1)\cap f(A_1^c) = \emptyset]\Big]\Rightarrow [\exists y\in B:y\notin f(A_1)]$, a contradiction.
        \end{proof}
        \begin{definition}
        If $A$ is a set, then a binary operation $*$ on the set $A$ is a function from $A\times A$ to $A$.
        \end{definition}
        \begin{definition}
        A binary operation $*$ is said to be associative if and only if $a*(b*c) = (a*b)*c$.
        \end{definition}
        \begin{definition}
        An element $e\in A$ is said to be an identity element if and only if for all $a\in A$, $e*a = a*e = a$.
        \end{definition}
        \begin{definition}
        An element $b\in A$ is said to be an inverse of $a$ if and only if $a*b=b*a = e$. We write $b=a^{-1}$
        \end{definition}
        \begin{definition}[Group]
        A group is a set $G$ with a binary operation $*$, denoted $\langle G,* \rangle$, with the following properties: 
        \begin{enumerate}
        \item There exists an identity element $e$.
        \item For every element $a\in A$, there is an inverse element.
        \item The binary operation $*$ is associative.
        \end{enumerate}
        \end{definition}
        \begin{remark}
        Note that it is not necessarily true that $a*b = b*a$. These are special groups (Abelian Groups).
        \end{remark}
        \begin{theorem}
        If $\langle G, * \rangle$ is a group and $e$ is the identity, then it is unique.
        \end{theorem}
        \begin{proof}
        For suppose $e'$ is an identity element not equal to $e$. But $e' = e'*e  = e$. A contradiction. Thus, $e$ is unique.
        \end{proof}
        \begin{theorem}
        If $\langle G, * \rangle$ is a group and $a\in G$, then $a^{-1}\in G$ is unique.
        \end{theorem}
        \begin{proof}
        $a'^{-1} = a'^{-1}*e = a'^{-1}(a*a^{-1}) = (a'^{-1}*a)*a^{-1} = e*a^{-1} = a^{-1}$. Thus, $a^{-1}$ is unique.
        \end{proof}
        \begin{corollary}
        The identity element of a group is its own inverse.
        \end{corollary}
        \begin{proof}
        For $e=e*e = e$. As inverses are unique, $e=e^{-1}$.
        \end{proof}
        \begin{theorem}
        If $\langle G,*\rangle$ is a group and $a,b\in G$, then $(a*b)^{-1} = b^{-1}*a^{-1}$.
        \end{theorem}
        \begin{proof}
        $(a*b)*(b^{-1}*a^{-1}) = a*(b*b^{-1})*a^{-1} = a*a^{-1} = e=b^{-1}*b=b^{-1}*(a^{-1}*a)*b=(b^{-1}*a^{-1})*(a*b)  $.
        \end{proof}
        \begin{theorem}
        If $\langle G,* \rangle$ is a group and $a\in G$, then $(a^{-1})^{-1} = a$.
        \end{theorem}
        \begin{proof}
        For $a^{-1}*(a^{-1})^{-1} = (a^{-1}* a)^{-1} = e$, and $(a^{-1})^{-1}*a^{-1} = (a*a^{-1})^{-1} = e$. From uniqueness, $(a^{-1})^{-1} = a$.
        \end{proof}
        \begin{definition}
        If $\langle G, * \rangle$ and $\langle G',\circ \rangle$ are groups and $f:G\rightarrow G'$ is a bijective function, then $f$ is said to be an isomorphism between $\langle G, * \rangle$ and $\langle G',\circ \rangle$ if and only if for all $a,b\in G$, $f(a*b) =f(a)\circ f(b)$.
        \end{definition}
        \begin{definition}
        $\langle G, *\rangle$ and $\langle G', \circ \rangle$ are said to be isomorphic if and only if there is an isomorphism between them.
        \end{definition}
        \begin{theorem}
        If $\langle G, * \rangle$ and $\langle G', \circ \rangle$ are isomorphic with identities $e_*$ and $e_{\circ}$ are the identities, then $f(e_*) = e_{\circ}$.
        \end{theorem}
        \begin{proof}
        $\forall a\in G,\ f(a)=f(a* e_*) = f(a)\circ f(e_*)$ as $f$ is an isomorphism. As identities are unique, $f(e_*) = e_{\circ}$.
        \end{proof}
        \begin{theorem}
        If $\langle G, * \rangle$ and $\langle G', \circ \rangle$ are isomorphic, with isomorphism $f$, and if $a\in G$, then $f(a^{-1}) = f(a)^{-1}$.
        \end{theorem}
        \begin{proof}
        For $e_{\circ}=f(e_*) = f(a*a^{-1}) = f(a^{-1}*a) = f(a)\circ f(a^{-1})=f(a^{-1})\circ f(a)$. As inverses are unique, $f(a^{-1})=f(a)^{-1}$.
        \end{proof}
        \begin{definition}
        A binary operation $*$ on a set $A$ is said to be commutative if and only for all $a,b\in A$, $a*b = b*a$.
        \end{definition}
        \begin{definition}
        A field is a set $F$ with two operations $+$ and $\cdot$, denoted $\langle F, +,\cdot \rangle$, with the following properties:
        \begin{enumerate}
        \item $a+b=b+a$ \hfill [Addition is Commutative]
        \item $a+(b+c)=(a+b)+c$ \hfill [Addition is Associative]
        \item $a\cdot b = b\cdot a$ \hfill [Multiplication is Commutative]
        \item $a\cdot (b\cdot c) = (a\cdot b)\cdot c$ \hfill [Multiplication is Associative]
        \item There is a $0\in F$ such that $0+a=a$ for all $a\in F$ \hfill [Existence of Additive Identity]
        \item There is a $1\in F$ such that $1\cdot a = a$ for all $a\in F$ \hfill [Existence of Multiplicative Identity]
        \item For each $a\in F$ there is a $b\in F$ such that $a+b = 0$. $b$ is denoted $-a$ \hfill [Existence of Additive Inverses]
        \item For each $a\in F$, $a\ne 0$ there is a $b\in F$ such that $a\cdot b = 1$. $b$ is denoted $a^{-1}$. \hfill [Existence of Multiplicative Inverses]
        \item $a\cdot(b+c) = a\cdot b + a\cdot c$ \hfill [Distributive Property]
        \end{enumerate}
        \end{definition}
        \begin{definition}
        A subfield of a field $\langle F,+,\cdot \rangle$ is a set $K\subset F$, such that $\langle K, +,\cdot \rangle$ is a field.
        \end{definition}
        \begin{theorem}
        In a field, $0$ and $1$ are unique.
        \end{theorem}
        \begin{proof}
        For suppose not, and let $0'$ and $1'$ be other identities. Then $1'=1'\cdot 1 = 1$ and $0'=0'+0=0$.
        \end{proof}
        \begin{theorem}
        For any field $\langle F,+,\cdot \rangle$, for any $a\in F$, $a\cdot 0 = 0$.
        \end{theorem}
        \begin{proof}
        For $0 = a\cdot 0 + (-a\cdot 0) = a\cdot(0+0) +(-a\cdot 0) = a\cdot 0 + a\cdot 0 + (-a\cdot 0) = a\cdot 0$. Thus, $a\cdot 0 = 0$.
        \end{proof}
        \begin{remark}
        If $1=0$, then $a=a\cdot 1 = a\cdot 0 = 0$, and thus every element is zero. A very boring field.
        \end{remark}
        \begin{corollary}
        In a field $\langle F, +,\cdot \rangle$, if $0\ne 1$, then $0$ has no inverse.
        \end{corollary}
        \begin{proof}
        For let $a$ be such an inverse. Then $a\cdot 0 = 1$. But for any element of $F$, $a \cdot 0 = 0$. But $0\ne 1$, a contradiction.
        \end{proof}
        \begin{theorem}
        If $a+b = 0$, then $b= (-1)\cdot a$ where $(-1)$ is the solution to $1+(-1)=0$.
        \end{theorem}
        \begin{proof}
        $a+(-1)a = a(1+(-1)) = a\cdot 0 = 0$. From uniqueness, $b=(-1)a$. We may thus write additive inverses as $-a$
        \end{proof}
        \begin{definition}
        Given two fields $\langle F,+,\cdot \rangle$ and $\langle F', +',\times \rangle$, a bijection function $f:F\rightarrow F'$ is said to be a field isomorphism if and only if for all elements $a,b\in F$, $f(a+b)=f(a)+'f(b)$, and $f(a\cdot b) = f(a)\times f(b)$
        \end{definition}
        \begin{definition}
        $\langle F,+,\cdot \rangle$ and $\langle F', +',\times \rangle$, are said to be isomorphic if and only if they have an isomorphism.
        \end{definition}
        \begin{theorem}
        Given an ismorphism between two fields $\langle F,+,\cdot \rangle$ and $\langle F', +',\times \rangle$, $f(1) = 1'$ and $f(0) = 0'$.
        \end{theorem}
        \begin{proof}
        For let $x\in F$. Then $f(x)=f(x\cdot 1) = f(x)\times f(1)$, and $f(x)=f(x+0) = f(x)+'f(0)$. Therefore, etc.
        \end{proof}
        \begin{theorem}
        In a field $\langle F,+,\cdot \rangle$, $(a+ b)^2 = a^2 + 2ab + b^2$ ($2$ being the solution to $1+1$).
        \end{theorem}
        \begin{proof}
        For $(a+b)^2 = (a+b)(a+b) = a(a+b)+b(a+b) = a^2 + ab + ba + b^2 = a^2 +ab(1+1)+b^2 = a^2 + 2ab + b^2$.
        \end{proof}
\end{document}