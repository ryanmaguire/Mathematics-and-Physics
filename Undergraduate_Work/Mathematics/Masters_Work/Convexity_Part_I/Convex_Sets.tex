\documentclass[crop=false,class=article,oneside]{standalone}
%----------------------------Preamble-------------------------------%
%---------------------------Packages----------------------------%
\usepackage{geometry}
\geometry{b5paper, margin=1.0in}
\usepackage[T1]{fontenc}
\usepackage{graphicx, float}            % Graphics/Images.
\usepackage{natbib}                     % For bibliographies.
\bibliographystyle{agsm}                % Bibliography style.
\usepackage[french, english]{babel}     % Language typesetting.
\usepackage[dvipsnames]{xcolor}         % Color names.
\usepackage{listings}                   % Verbatim-Like Tools.
\usepackage{mathtools, esint, mathrsfs} % amsmath and integrals.
\usepackage{amsthm, amsfonts, amssymb}  % Fonts and theorems.
\usepackage{tcolorbox}                  % Frames around theorems.
\usepackage{upgreek}                    % Non-Italic Greek.
\usepackage{fmtcount, etoolbox}         % For the \book{} command.
\usepackage[newparttoc]{titlesec}       % Formatting chapter, etc.
\usepackage{titletoc}                   % Allows \book in toc.
\usepackage[nottoc]{tocbibind}          % Bibliography in toc.
\usepackage[titles]{tocloft}            % ToC formatting.
\usepackage{pgfplots, tikz}             % Drawing/graphing tools.
\usepackage{imakeidx}                   % Used for index.
\usetikzlibrary{
    calc,                   % Calculating right angles and more.
    angles,                 % Drawing angles within triangles.
    arrows.meta,            % Latex and Stealth arrows.
    quotes,                 % Adding labels to angles.
    positioning,            % Relative positioning of nodes.
    decorations.markings,   % Adding arrows in the middle of a line.
    patterns,
    arrows
}                                       % Libraries for tikz.
\pgfplotsset{compat=1.9}                % Version of pgfplots.
\usepackage[font=scriptsize,
            labelformat=simple,
            labelsep=colon]{subcaption} % Subfigure captions.
\usepackage[font={scriptsize},
            hypcap=true,
            labelsep=colon]{caption}    % Figure captions.
\usepackage[pdftex,
            pdfauthor={Ryan Maguire},
            pdftitle={Mathematics and Physics},
            pdfsubject={Mathematics, Physics, Science},
            pdfkeywords={Mathematics, Physics, Computer Science, Biology},
            pdfproducer={LaTeX},
            pdfcreator={pdflatex}]{hyperref}
\hypersetup{
    colorlinks=true,
    linkcolor=blue,
    filecolor=magenta,
    urlcolor=Cerulean,
    citecolor=SkyBlue
}                           % Colors for hyperref.
\usepackage[toc,acronym,nogroupskip,nopostdot]{glossaries}
\usepackage{glossary-mcols}
%------------------------Theorem Styles-------------------------%
\theoremstyle{plain}
\newtheorem{theorem}{Theorem}[section]

% Define theorem style for default spacing and normal font.
\newtheoremstyle{normal}
    {\topsep}               % Amount of space above the theorem.
    {\topsep}               % Amount of space below the theorem.
    {}                      % Font used for body of theorem.
    {}                      % Measure of space to indent.
    {\bfseries}             % Font of the header of the theorem.
    {}                      % Punctuation between head and body.
    {.5em}                  % Space after theorem head.
    {}

% Italic header environment.
\newtheoremstyle{thmit}{\topsep}{\topsep}{}{}{\itshape}{}{0.5em}{}

% Define environments with italic headers.
\theoremstyle{thmit}
\newtheorem*{solution}{Solution}

% Define default environments.
\theoremstyle{normal}
\newtheorem{example}{Example}[section]
\newtheorem{definition}{Definition}[section]
\newtheorem{problem}{Problem}[section]

% Define framed environment.
\tcbuselibrary{most}
\newtcbtheorem[use counter*=theorem]{ftheorem}{Theorem}{%
    before=\par\vspace{2ex},
    boxsep=0.5\topsep,
    after=\par\vspace{2ex},
    colback=green!5,
    colframe=green!35!black,
    fonttitle=\bfseries\upshape%
}{thm}

\newtcbtheorem[auto counter, number within=section]{faxiom}{Axiom}{%
    before=\par\vspace{2ex},
    boxsep=0.5\topsep,
    after=\par\vspace{2ex},
    colback=Apricot!5,
    colframe=Apricot!35!black,
    fonttitle=\bfseries\upshape%
}{ax}

\newtcbtheorem[use counter*=definition]{fdefinition}{Definition}{%
    before=\par\vspace{2ex},
    boxsep=0.5\topsep,
    after=\par\vspace{2ex},
    colback=blue!5!white,
    colframe=blue!75!black,
    fonttitle=\bfseries\upshape%
}{def}

\newtcbtheorem[use counter*=example]{fexample}{Example}{%
    before=\par\vspace{2ex},
    boxsep=0.5\topsep,
    after=\par\vspace{2ex},
    colback=red!5!white,
    colframe=red!75!black,
    fonttitle=\bfseries\upshape%
}{ex}

\newtcbtheorem[auto counter, number within=section]{fnotation}{Notation}{%
    before=\par\vspace{2ex},
    boxsep=0.5\topsep,
    after=\par\vspace{2ex},
    colback=SeaGreen!5!white,
    colframe=SeaGreen!75!black,
    fonttitle=\bfseries\upshape%
}{not}

\newtcbtheorem[use counter*=remark]{fremark}{Remark}{%
    fonttitle=\bfseries\upshape,
    colback=Goldenrod!5!white,
    colframe=Goldenrod!75!black}{ex}

\newenvironment{bproof}{\textit{Proof.}}{\hfill$\square$}
\tcolorboxenvironment{bproof}{%
    blanker,
    breakable,
    left=3mm,
    before skip=5pt,
    after skip=10pt,
    borderline west={0.6mm}{0pt}{green!80!black}
}

\AtEndEnvironment{lexample}{$\hfill\textcolor{red}{\blacksquare}$}
\newtcbtheorem[use counter*=example]{lexample}{Example}{%
    empty,
    title={Example~\theexample},
    boxed title style={%
        empty,
        size=minimal,
        toprule=2pt,
        top=0.5\topsep,
    },
    coltitle=red,
    fonttitle=\bfseries,
    parbox=false,
    boxsep=0pt,
    before=\par\vspace{2ex},
    left=0pt,
    right=0pt,
    top=3ex,
    bottom=1ex,
    before=\par\vspace{2ex},
    after=\par\vspace{2ex},
    breakable,
    pad at break*=0mm,
    vfill before first,
    overlay unbroken={%
        \draw[red, line width=2pt]
            ([yshift=-1.2ex]title.south-|frame.west) to
            ([yshift=-1.2ex]title.south-|frame.east);
        },
    overlay first={%
        \draw[red, line width=2pt]
            ([yshift=-1.2ex]title.south-|frame.west) to
            ([yshift=-1.2ex]title.south-|frame.east);
    },
}{ex}

\AtEndEnvironment{ldefinition}{$\hfill\textcolor{Blue}{\blacksquare}$}
\newtcbtheorem[use counter*=definition]{ldefinition}{Definition}{%
    empty,
    title={Definition~\thedefinition:~{#1}},
    boxed title style={%
        empty,
        size=minimal,
        toprule=2pt,
        top=0.5\topsep,
    },
    coltitle=Blue,
    fonttitle=\bfseries,
    parbox=false,
    boxsep=0pt,
    before=\par\vspace{2ex},
    left=0pt,
    right=0pt,
    top=3ex,
    bottom=0pt,
    before=\par\vspace{2ex},
    after=\par\vspace{1ex},
    breakable,
    pad at break*=0mm,
    vfill before first,
    overlay unbroken={%
        \draw[Blue, line width=2pt]
            ([yshift=-1.2ex]title.south-|frame.west) to
            ([yshift=-1.2ex]title.south-|frame.east);
        },
    overlay first={%
        \draw[Blue, line width=2pt]
            ([yshift=-1.2ex]title.south-|frame.west) to
            ([yshift=-1.2ex]title.south-|frame.east);
    },
}{def}

\AtEndEnvironment{ltheorem}{$\hfill\textcolor{Green}{\blacksquare}$}
\newtcbtheorem[use counter*=theorem]{ltheorem}{Theorem}{%
    empty,
    title={Theorem~\thetheorem:~{#1}},
    boxed title style={%
        empty,
        size=minimal,
        toprule=2pt,
        top=0.5\topsep,
    },
    coltitle=Green,
    fonttitle=\bfseries,
    parbox=false,
    boxsep=0pt,
    before=\par\vspace{2ex},
    left=0pt,
    right=0pt,
    top=3ex,
    bottom=-1.5ex,
    breakable,
    pad at break*=0mm,
    vfill before first,
    overlay unbroken={%
        \draw[Green, line width=2pt]
            ([yshift=-1.2ex]title.south-|frame.west) to
            ([yshift=-1.2ex]title.south-|frame.east);},
    overlay first={%
        \draw[Green, line width=2pt]
            ([yshift=-1.2ex]title.south-|frame.west) to
            ([yshift=-1.2ex]title.south-|frame.east);
    }
}{thm}

%--------------------Declared Math Operators--------------------%
\DeclareMathOperator{\adjoint}{adj}         % Adjoint.
\DeclareMathOperator{\Card}{Card}           % Cardinality.
\DeclareMathOperator{\curl}{curl}           % Curl.
\DeclareMathOperator{\diam}{diam}           % Diameter.
\DeclareMathOperator{\dist}{dist}           % Distance.
\DeclareMathOperator{\Div}{div}             % Divergence.
\DeclareMathOperator{\Erf}{Erf}             % Error Function.
\DeclareMathOperator{\Erfc}{Erfc}           % Complementary Error Function.
\DeclareMathOperator{\Ext}{Ext}             % Exterior.
\DeclareMathOperator{\GCD}{GCD}             % Greatest common denominator.
\DeclareMathOperator{\grad}{grad}           % Gradient
\DeclareMathOperator{\Ima}{Im}              % Image.
\DeclareMathOperator{\Int}{Int}             % Interior.
\DeclareMathOperator{\LC}{LC}               % Leading coefficient.
\DeclareMathOperator{\LCM}{LCM}             % Least common multiple.
\DeclareMathOperator{\LM}{LM}               % Leading monomial.
\DeclareMathOperator{\LT}{LT}               % Leading term.
\DeclareMathOperator{\Mod}{mod}             % Modulus.
\DeclareMathOperator{\Mon}{Mon}             % Monomial.
\DeclareMathOperator{\multideg}{mutlideg}   % Multi-Degree (Graphs).
\DeclareMathOperator{\nul}{nul}             % Null space of operator.
\DeclareMathOperator{\Ord}{Ord}             % Ordinal of ordered set.
\DeclareMathOperator{\Prin}{Prin}           % Principal value.
\DeclareMathOperator{\proj}{proj}           % Projection.
\DeclareMathOperator{\Refl}{Refl}           % Reflection operator.
\DeclareMathOperator{\rk}{rk}               % Rank of operator.
\DeclareMathOperator{\sgn}{sgn}             % Sign of a number.
\DeclareMathOperator{\sinc}{sinc}           % Sinc function.
\DeclareMathOperator{\Span}{Span}           % Span of a set.
\DeclareMathOperator{\Spec}{Spec}           % Spectrum.
\DeclareMathOperator{\supp}{supp}           % Support
\DeclareMathOperator{\Tr}{Tr}               % Trace of matrix.
%--------------------Declared Math Symbols--------------------%
\DeclareMathSymbol{\minus}{\mathbin}{AMSa}{"39} % Unary minus sign.
%------------------------New Commands---------------------------%
\DeclarePairedDelimiter\norm{\lVert}{\rVert}
\DeclarePairedDelimiter\ceil{\lceil}{\rceil}
\DeclarePairedDelimiter\floor{\lfloor}{\rfloor}
\newcommand*\diff{\mathop{}\!\mathrm{d}}
\newcommand*\Diff[1]{\mathop{}\!\mathrm{d^#1}}
\renewcommand*{\glstextformat}[1]{\textcolor{RoyalBlue}{#1}}
\renewcommand{\glsnamefont}[1]{\textbf{#1}}
\renewcommand\labelitemii{$\circ$}
\renewcommand\thesubfigure{%
    \arabic{chapter}.\arabic{figure}.\arabic{subfigure}}
\addto\captionsenglish{\renewcommand{\figurename}{Fig.}}
\numberwithin{equation}{section}

\renewcommand{\vector}[1]{\boldsymbol{\mathrm{#1}}}

\newcommand{\uvector}[1]{\boldsymbol{\hat{\mathrm{#1}}}}
\newcommand{\topspace}[2][]{(#2,\tau_{#1})}
\newcommand{\measurespace}[2][]{(#2,\varSigma_{#1},\mu_{#1})}
\newcommand{\measurablespace}[2][]{(#2,\varSigma_{#1})}
\newcommand{\manifold}[2][]{(#2,\tau_{#1},\mathcal{A}_{#1})}
\newcommand{\tanspace}[2]{T_{#1}{#2}}
\newcommand{\cotanspace}[2]{T_{#1}^{*}{#2}}
\newcommand{\Ckspace}[3][\mathbb{R}]{C^{#2}(#3,#1)}
\newcommand{\funcspace}[2][\mathbb{R}]{\mathcal{F}(#2,#1)}
\newcommand{\smoothvecf}[1]{\mathfrak{X}(#1)}
\newcommand{\smoothonef}[1]{\mathfrak{X}^{*}(#1)}
\newcommand{\bracket}[2]{[#1,#2]}

%------------------------Book Command---------------------------%
\makeatletter
\renewcommand\@pnumwidth{1cm}
\newcounter{book}
\renewcommand\thebook{\@Roman\c@book}
\newcommand\book{%
    \if@openright
        \cleardoublepage
    \else
        \clearpage
    \fi
    \thispagestyle{plain}%
    \if@twocolumn
        \onecolumn
        \@tempswatrue
    \else
        \@tempswafalse
    \fi
    \null\vfil
    \secdef\@book\@sbook
}
\def\@book[#1]#2{%
    \refstepcounter{book}
    \addcontentsline{toc}{book}{\bookname\ \thebook:\hspace{1em}#1}
    \markboth{}{}
    {\centering
     \interlinepenalty\@M
     \normalfont
     \huge\bfseries\bookname\nobreakspace\thebook
     \par
     \vskip 20\p@
     \Huge\bfseries#2\par}%
    \@endbook}
\def\@sbook#1{%
    {\centering
     \interlinepenalty \@M
     \normalfont
     \Huge\bfseries#1\par}%
    \@endbook}
\def\@endbook{
    \vfil\newpage
        \if@twoside
            \if@openright
                \null
                \thispagestyle{empty}%
                \newpage
            \fi
        \fi
        \if@tempswa
            \twocolumn
        \fi
}
\newcommand*\l@book[2]{%
    \ifnum\c@tocdepth >-3\relax
        \addpenalty{-\@highpenalty}%
        \addvspace{2.25em\@plus\p@}%
        \setlength\@tempdima{3em}%
        \begingroup
            \parindent\z@\rightskip\@pnumwidth
            \parfillskip -\@pnumwidth
            {
                \leavevmode
                \Large\bfseries#1\hfill\hb@xt@\@pnumwidth{\hss#2}
            }
            \par
            \nobreak
            \global\@nobreaktrue
            \everypar{\global\@nobreakfalse\everypar{}}%
        \endgroup
    \fi}
\newcommand\bookname{Book}
\renewcommand{\thebook}{\texorpdfstring{\Numberstring{book}}{book}}
\providecommand*{\toclevel@book}{-2}
\makeatother
\titleformat{\part}[display]
    {\Large\bfseries}
    {\partname\nobreakspace\thepart}
    {0mm}
    {\Huge\bfseries}
\titlecontents{part}[0pt]
    {\large\bfseries}
    {\partname\ \thecontentslabel: \quad}
    {}
    {\hfill\contentspage}
\titlecontents{chapter}[0pt]
    {\bfseries}
    {\chaptername\ \thecontentslabel:\quad}
    {}
    {\hfill\contentspage}
\newglossarystyle{longpara}{%
    \setglossarystyle{long}%
    \renewenvironment{theglossary}{%
        \begin{longtable}[l]{{p{0.25\hsize}p{0.65\hsize}}}
    }{\end{longtable}}%
    \renewcommand{\glossentry}[2]{%
        \glstarget{##1}{\glossentryname{##1}}%
        &\glossentrydesc{##1}{~##2.}
        \tabularnewline%
        \tabularnewline
    }%
}
\newglossary[not-glg]{notation}{not-gls}{not-glo}{Notation}
\newcommand*{\newnotation}[4][]{%
    \newglossaryentry{#2}{type=notation, name={\textbf{#3}, },
                          text={#4}, description={#4},#1}%
}
%--------------------------LENGTHS------------------------------%
% Spacings for the Table of Contents.
\addtolength{\cftsecnumwidth}{1ex}
\addtolength{\cftsubsecindent}{1ex}
\addtolength{\cftsubsecnumwidth}{1ex}
\addtolength{\cftfignumwidth}{1ex}
\addtolength{\cfttabnumwidth}{1ex}

% Indent and paragraph spacing.
\setlength{\parindent}{0em}
\setlength{\parskip}{0em}
%--------------------------Main Document----------------------------%
\begin{document}
    \ifx\ifsub\undefined
        \section*{Convexity: Part I}
        \setcounter{section}{1}
    \fi
    \subsection{Convex Sets}
        \begin{definition}
        A subset $K$ of a Banach Space $X$ is convex if and only if $\big[[x,y\in K]\land  [\lambda \in [0,1]]\big]\Rightarrow[(1-\lambda)x+\lambda y\in K]$.
        \end{definition}
        \begin{theorem}
        Convex spaces are path-connected.
        \end{theorem}
        \begin{proof}
        Let $x,y\in V$ and define $f:[0,1]\rightarrow V$ by $f(\lambda) = (1-\lambda)x+\lambda y$. Therefore, etc.
        \end{proof}
        \begin{corollary}
        Convex sets are connected.
        \end{corollary}
        \begin{proof}
        As convex sets are path-connected, they are connected.
        \end{proof}
        \begin{theorem}
        Any Banach Space $X$ is convex.
        \end{theorem}
        \begin{proof}
        Let $x,y\in V$. As $V$ is a vector space, for all $\lambda \in \mathbb{R}$, $(1-\lambda)x+\lambda y\in V$. Therefore, etc.
        \end{proof}
        \begin{notation}
        The set of all convex subsets of $\mathbb{R}^n$ is denoted $\mathscr{K}_n$.
        \end{notation}
        \begin{theorem}
        If $X$ is a Banach Space and $A,B\subset X$ are convex sets, then $A\cap B$ is convex.
        \end{theorem}
        \begin{proof}
        $[x,y \in A\cap B] \Rightarrow \big[[\lambda \in [0,1]]\Rightarrow[ (1-\lambda)x+\lambda y \in A]\land [ (1-\lambda)x+\lambda y \in B]\big] \Rightarrow [\forall \lambda \in [0,1]:(1-\lambda)x+\lambda y \in A\cap B]$. 
        \end{proof}
        \begin{lemma}
        In a Banach Space $X$, $\forall \xi \in X$, $\forall r>0$, $B_{r}(\xi)$ is convex.
        \end{lemma}
        \begin{proof}
        $[x,y\in B_{r}(\xi)]\Rightarrow\big[[\lambda \in [0,1]]\Rightarrow [\norm{(1-\lambda)x+\lambda y-\xi}\leq \frac{(1-\lambda)}{2}\norm{x-\xi}+\frac{\lambda}{2} \norm{y-\xi}< r]\big]\Rightarrow [(1-\lambda)x+\lambda y \in B_{r}(\xi)]$.
        \end{proof}
        \begin{theorem}
        There exist convex sets $A$ and $B$ such that $A\cap B \ne \emptyset$, yet $A\cup B$ is not convex.
        \end{theorem}
        \begin{proof}
        For let $A = \{(x,y)\in \mathbb{R}^2: (x-1)^2+y^2\leq 1\}$, and $B = \{(x,y)\in \mathbb{R}^2:(x+1)^2+y^2\leq 1\}$. Both of these sets are discs, and thus convex, their intersection is the point $(0,0)$, however their union is not convex.
        \end{proof}
        \begin{theorem}
        If $K\subset X$ is convex and $\psi:X\rightarrow X$ is a linear transformation, then $\psi(K)$ is convex.
        \end{theorem}
        \begin{proof}
        $[X,Y\in \psi(K)]\Rightarrow [\exists x,y\in K:\psi(x)=X\land \psi(y)=Y]$. $\big[\lambda \in [0,1]\big]\Rightarrow [(1-\lambda)x+\lambda y\in K]\Rightarrow [\psi\big((1-\lambda)x+\lambda y\big)=(1-\lambda)\psi(x)+\lambda\psi(y) = (1-\lambda)X+\lambda Y \in \psi(K)]$
        \end{proof}
        \begin{theorem}
        There exist non-convex, compact subsets $K$ of a Banach Space $X: K_{\xi}$ is convex for every subspace $\xi$.
        \end{theorem}
        \begin{proof}
        Let $r>0, \mathcal{M} = \{B_{r}(\mathbf{0})\setminus B_{r/2}(\mathbf{0})\}$. This is not convex, but for any subspace $\xi$, $\mathcal{M}_{\xi} = B_{r}(\mathbf{0})\cap \xi$, which is convex.
        \end{proof}
        \begin{theorem}
        If $X$ is a Banach Space, $K\subset X$, and for every affine subspace $\xi$, $K\cap \xi$ is convex, then $K$ is convex.
        \end{theorem}
        \begin{proof}
        Let $x,y\in K$. Let $W$ be a subspace containing $y$, and let $\xi$ be the affine subspace $\{x-y+v:v\in W\}$. Then $\xi\cap K$ is convex, and thus $(1-\lambda)x+\lambda y \in \xi \cap K \Rightarrow (1-\lambda)x+\lambda y \in K$.
        \end{proof}
        \begin{theorem}
        If $X$ is a Banach Space, $K,L\subset X$ are convex, then $K+L$ is convex.
        \end{theorem}
        \begin{proof}
        For let $\chi$ and $\zeta$ be elements of $K+L$. Then there are points $x_1,x_2$ and $y_1,y_2$ such that $\chi=x_1+x_2$ and $\zeta = y_1+y_2$ with $x_1,y_1\in K$ and $x_2,y_2\in L$. If $\lambda \in [0,1]$, then $(1-\lambda)\chi + \lambda \zeta = (1-\lambda)(x_1+x_2)+\lambda(y_1+y_2) = (1-\lambda)x_1 + \lambda y_1 + (1-\lambda)x_2 + \lambda y_2$. But $K$ and $L$ are convex, and thus $(1-\lambda)x_1 + \lambda y_1 \in K$ and $(1-\lambda)x_2 + \lambda y_2 \in L$. But then $(1-\lambda)x_1 + \lambda y_1 + (1-\lambda)x_2 + \lambda y_2=(1-\lambda)\chi + \lambda \zeta\in K+L$. As $\lambda$ is arbitrary in $[0,1]$, $K+L$ is convex.
        \end{proof}
        \begin{theorem}
        If $V$ is a Banach Space, $K\subset V$ is convex, and $\alpha \in \mathbb{R}$, then $\alpha K$ is convex.
        \end{theorem}
        \begin{proof}
        For let $X,Y\in \alpha K$. Then $X = \alpha x$ and $Y = \alpha y$ for $x,y\in K$. Let $\lambda \in (0,1)$ be arbitrary. Then $(1-\lambda)X+\lambda Y =(1-\lambda)(\alpha x)+\lambda (\alpha y) = \alpha\big[(1-\lambda)x+\lambda y\big]$. As $K$ is convex, $(1-\lambda)x+\lambda y \in K$. Thus $\alpha\big[(1-\lambda)x+\lambda y\big] \in \alpha K$, and $\alpha K$ is convex.
        \end{proof}
        \begin{theorem}
        If $K$ and $L$ are compact and convex, and $K\cup L$ is convex, then $K\cap L \ne \emptyset$.
        \end{theorem}
        \begin{proof}
        For if not, then $K\cup L$ is the union of two closed, disjoint subsets, and is thus disconnected. A contradiction. 
        \end{proof}
        \begin{lemma}
        Any set is a subset of its convex hull.
        \end{lemma}
        \begin{proof}
        For $x\in K \Rightarrow 1\cdot x=x \in \textrm{conv}(K)$. Therefore, etc.
        \end{proof}
        \begin{lemma}
        If $x\in \textrm{conv}(K)$, then there are two points in $\textrm{conv}(K)$ such that $x = \lambda v_1 +(1- \lambda) v_2$, $0 \leq \lambda \leq 1$.
        \end{lemma}
        \begin{proof}
        Let $x=\sum_{k=1}^{n+1}|\lambda_k| x_k$. If $x=|\lambda_{n+1}|x_{n+1}$, we are done. If not, then let $v_2 = \frac{1}{1-|\lambda_{n+1}|}\sum_{k=1}^{n}|\lambda_k|x_k$. As $\frac{1}{1-|\lambda_{n+1}|}\sum_{k=1}^{n}|\lambda_k| = 1$, $v_1\in \textrm{conv}(K)$. Thus, $x = \lambda_{n+1}x_{n+1}+(1-\lambda_{n+1})v_2$. Let $x_{n+1}=v_1$ and $\lambda = \lambda_{n+1}$. Therefore, etc.
        \end{proof}
        \begin{theorem}
        The convex hull of a convex set is itself.
        \end{theorem}
        \begin{proof}
        $[x\in \textrm{conv}(K)]\Rightarrow [\exists v_1,v_2\in K\land \lambda\in[0,1]:x=\lambda v_1+(1-\lambda)v_2]\Rightarrow [x\in K]$. The last step is from convexity.
        \end{proof}
        \begin{theorem}
        If $K$ is compact and convex, $\lambda_n\in[0,1]$, $v_n\in K$, and $\sum_{k=1}^{n}\lambda_k \rightarrow 1$, then $\sum_{k=1}^{n}\lambda_k v_k$ converges in $K$.
        \end{theorem}
        \begin{proof}
        Let $x\in K$ be arbitrary, $\varepsilon>0$, $s_n = \sum_{k=1}^{n}\lambda_k v_k$, and $\Lambda_n = \sum_{k=1}^{n}\lambda_k$. Let $S_n = s_n + (1-\Lambda_n)x$. For each $n\in \mathbb{N}$, $S_n\in K$. It now suffices to show $S_n$ converges. As $K$ is compact, $\{\norm{v}:v\in K\}$ is bounded, let $M$ be a bound. As $\Lambda_n\rightarrow 1$, $\exists N\in \mathbb{N}:n,m>N\Rightarrow |\Lambda_n-\Lambda_m|<\frac{\varepsilon}{2M}$. But then for $n\geq m >N$, $d(S_n,S_m) = \norm{\sum_{k=m}^{n}\lambda_k v_k+x(\Lambda_n-\Lambda_m)} \leq \sum_{k=m}^{n}|\lambda_k| \norm{v_k}+\norm{x}|(\Lambda_n-\Lambda_m)|\leq 2M|\Lambda_{n}-\Lambda_{m}|<\varepsilon$. Thus, $S_n$ is Cauchy. As $K$ is compact, it is complete, and thus $S_n$ converges in $K$. But $d(S_n,s_n)\leq |1-\Lambda_n|M\rightarrow 0$. Thus, $s_n$ converges to the same limit. Therefore, etc.
        \end{proof}
        \begin{theorem}
        The convex hull of a compact set is compact.
        \end{theorem}
        \begin{proof}
        Yeah
        \end{proof}
        \begin{theorem}
        The convex hull of an open set is open.
        \end{theorem}
        \begin{proof}
        Let $x\in \textrm{conv}(K)$. Then $x=\sum_{k=1}^{n}|\lambda_k| x_k$, where $\sum_{k=1}^{n}|\lambda_k| = 1$. At least one $\lambda_k$ must be non-zero, suppose $\lambda_1$ is. Define the function $f:\textrm{K}\rightarrow \textrm{K}$ by $f(z) = \frac{1}{|\lambda_1|}(z-\sum_{k=2}^{n}|\lambda_k|v_k$. Then $f$ is continuous, injective, and thus $f^{-1}$ exists and is equal to $f^{-1}(z) = \lambda_1 z +\sum_{k=2}^{n}\lambda_k v_k$. Thus, $x\in f^{-1}(K)\subset\textrm{conv}(K)$. Therefore, etc. 
        \end{proof}
        \begin{theorem}
        There exist closed sets whose convex hull is open (And not closed).
        \end{theorem}
        \begin{proof}
        Let $K = \{(x,y)\in \mathbb{R}^2:\frac{1}{1+x^2}\leq y\}$. The complement is open, and thus this is closed. But $\textrm{conv}(K) = \{(x,y)\in \mathbb{R}^2:y>0\}$, which is open. As $\mathbb{R}^2$ is connected, this set is not closed.
        \end{proof}
        \begin{theorem}[Carath\'{e}odory's Theorem]
        If $S$ is an $n$ dimensional subset of a Banach Space, there there exists $n+1$ points $x_k$ in $S$ such that $\textrm{conv}(S) = \textrm{conv}(x_1,\hdots, x_{n+1})$.
        \end{theorem}
        \begin{proof}
        Let $y\in \textrm{conv}(S)$. If $y\notin S$, then there are points $x_1,\hdots, x_m$ and positive real numbers $\lambda_1,\hdots, \lambda_m$ such that $y=\sum_{k=1}^{m}\lambda_m x_m$ and $\sum_{k=1}^{m}\lambda_k = 1$. Let $m$ be the minimal number of points needed. Suppose the $x_i$ are affinely dependent. Then there are real numbers $\alpha_i$ such that $\sum_{k=1}^{m}\alpha_k x_k = \mathbf{0}$ and $\sum_{k=1}^{n}\alpha_k =1$. But then $y = y+t\mathbf{0}$ for all $t\in \mathbb{R}$, and thus $y = \sum_{k=1}^{m}\lambda_k x_k + t\sum_{k=1}^{m}\alpha_k x_k = \sum_{k=1}^{m}(\lambda_k + t\alpha_k)x_k$ and $\sum_{k=1}^{m}(\lambda_k+t\alpha_t) = 1$. Let $|t|$ be the smallest value such that $\lambda_j + t\alpha_j = 0$ for some $j$. Then, for all other values, $\lambda_k + t\alpha_k \in [0,1]$. Thus, $y$ is represented by a convex combination of $m-1$ points, a contradiction as $m$ is minimal. Thus, the $x_k$ are affinely independent. But then $m \leq n+1$. Thus, etc.
        \end{proof}
        \begin{definition}
        A set $P$ is said to be a convex polytope if and only if it is the convex hull of finitely many points.
        \end{definition}
        \begin{definition}
        A polytope in $\mathbb{R}^2$ is called a polygon.
        \end{definition}
        \begin{notation}
        The set of all polytopes in $\mathbb{R}^n$ is denoted $\mathscr{P}_n$.
        \end{notation}
        \begin{definition}
        A polytope $P$ is said to be a simplex if and only if then generating points are affinely independent.
        \end{definition}
        \begin{theorem}
        An $n$ dimensional simplex has $n+1$ vertices.
        \end{theorem}
        \begin{theorem}
        If $P$ is a polytope in some Banach Space $V$, and $\xi$ is a subspace, then $P_{\xi}$ is a polytope.
        \end{theorem}
        \begin{theorem}
        If $K$ is a compact set of finite dimension $n\geq 3$, and if for every $n-1$ dimensional subspace $\xi$, $K_{\xi}$ is a convex polytope, then $K$ is a convex polytope. (Page 23)
        \end{theorem}
        \begin{theorem}[Radon's Theorem]
        If $S$ Is a finite dimensional affinely dependent subset of a Banach Space $V$, then there are sets $S_1,S_2\in V$ such that $S_1\cap S_2 = \emptyset$, $S=S_1\cup S_2$, and $\textrm{conv}(S_1)\cap \textrm{conv}(S_2) \ne \emptyset$.
        \end{theorem}
        \begin{proof}
        As $S$ is affinely dependent, there are distinct points $x_1,\hdots,x_n$ and non-zero real numbers $\lambda_1,\hdots,\lambda_n$ such that $\sum_{k=1}^{n}\lambda_k x_k= \mathbf{0}$ and $\sum_{k=1}^{n}\lambda_k = 0$. As none of the values of $\lambda_k$ are zero, some most be positive and some negative. Let $\lambda_1,\hdots, \lambda_j$ be positive and $\lambda_{j+1},\hdots, \lambda_n$ are negative. Then $\sum_{k=1}^{j} \lambda_k = - \lambda_{k=j+1}^{n}\lambda_k$. Let this sum be $c$. Let $T_1 = \{x_1,\hdots, x_j\}$ and $T_2=\{x_{j+1},\hdots, x_{n}\}$. As the $x_k$ are distinct, $T_1\cap T_2 = \emptyset$. However, $\frac{1}{c}\sum_{k=1}^{j}\lambda_k x_k = \frac{-1}{c}\sum_{k=j+1}^{n} \lambda_k x_k \in\textrm{conv}(T_1)\cap \textrm{conv}(T_2)$. Let $S_1 = S\setminus T_2$, and $S_2 = T_2$. Therefore, etc.
        \end{proof}
        \begin{theorem}
        If $K$ is compact, convex, and a subset of a two dimensional Banach Space, then if $A = \{x\in K:B_{1}(x)\subset K\}$, then $\underset{x\in A}\cup B_{1}(x)$ is convex.
        \end{theorem}
        \begin{theorem}
        If $K$ is a compact, convex subset of a two dimensional Banach Space, then $\underset{x\in K}\cup B_{1}(x)$ is convex.
        \end{theorem}
\end{document}