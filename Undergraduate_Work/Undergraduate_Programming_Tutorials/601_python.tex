\documentclass[crop=false,class=book]{standalone}
%---------------PREAMBLE------------%
%---------------------------Packages----------------------------%
\usepackage{geometry}
\geometry{b5paper, margin=1.0in}
\usepackage[T1]{fontenc}
\usepackage{graphicx, float}            % Graphics/Images.
\usepackage{natbib}                     % For bibliographies.
\bibliographystyle{agsm}                % Bibliography style.
\usepackage[french, english]{babel}     % Language typesetting.
\usepackage[dvipsnames]{xcolor}         % Color names.
\usepackage{listings}                   % Verbatim-Like Tools.
\usepackage{mathtools, esint, mathrsfs} % amsmath and integrals.
\usepackage{amsthm, amsfonts, amssymb}  % Fonts and theorems.
\usepackage{tcolorbox}                  % Frames around theorems.
\usepackage{upgreek}                    % Non-Italic Greek.
\usepackage{fmtcount, etoolbox}         % For the \book{} command.
\usepackage[newparttoc]{titlesec}       % Formatting chapter, etc.
\usepackage{titletoc}                   % Allows \book in toc.
\usepackage[nottoc]{tocbibind}          % Bibliography in toc.
\usepackage[titles]{tocloft}            % ToC formatting.
\usepackage{pgfplots, tikz}             % Drawing/graphing tools.
\usepackage{imakeidx}                   % Used for index.
\usetikzlibrary{
    calc,                   % Calculating right angles and more.
    angles,                 % Drawing angles within triangles.
    arrows.meta,            % Latex and Stealth arrows.
    quotes,                 % Adding labels to angles.
    positioning,            % Relative positioning of nodes.
    decorations.markings,   % Adding arrows in the middle of a line.
    patterns,
    arrows
}                                       % Libraries for tikz.
\pgfplotsset{compat=1.9}                % Version of pgfplots.
\usepackage[font=scriptsize,
            labelformat=simple,
            labelsep=colon]{subcaption} % Subfigure captions.
\usepackage[font={scriptsize},
            hypcap=true,
            labelsep=colon]{caption}    % Figure captions.
\usepackage[pdftex,
            pdfauthor={Ryan Maguire},
            pdftitle={Mathematics and Physics},
            pdfsubject={Mathematics, Physics, Science},
            pdfkeywords={Mathematics, Physics, Computer Science, Biology},
            pdfproducer={LaTeX},
            pdfcreator={pdflatex}]{hyperref}
\hypersetup{
    colorlinks=true,
    linkcolor=blue,
    filecolor=magenta,
    urlcolor=Cerulean,
    citecolor=SkyBlue
}                           % Colors for hyperref.
\usepackage[toc,acronym,nogroupskip,nopostdot]{glossaries}
\usepackage{glossary-mcols}
%------------------------Theorem Styles-------------------------%
\theoremstyle{plain}
\newtheorem{theorem}{Theorem}[section]

% Define theorem style for default spacing and normal font.
\newtheoremstyle{normal}
    {\topsep}               % Amount of space above the theorem.
    {\topsep}               % Amount of space below the theorem.
    {}                      % Font used for body of theorem.
    {}                      % Measure of space to indent.
    {\bfseries}             % Font of the header of the theorem.
    {}                      % Punctuation between head and body.
    {.5em}                  % Space after theorem head.
    {}

% Italic header environment.
\newtheoremstyle{thmit}{\topsep}{\topsep}{}{}{\itshape}{}{0.5em}{}

% Define environments with italic headers.
\theoremstyle{thmit}
\newtheorem*{solution}{Solution}

% Define default environments.
\theoremstyle{normal}
\newtheorem{example}{Example}[section]
\newtheorem{definition}{Definition}[section]
\newtheorem{problem}{Problem}[section]

% Define framed environment.
\tcbuselibrary{most}
\newtcbtheorem[use counter*=theorem]{ftheorem}{Theorem}{%
    before=\par\vspace{2ex},
    boxsep=0.5\topsep,
    after=\par\vspace{2ex},
    colback=green!5,
    colframe=green!35!black,
    fonttitle=\bfseries\upshape%
}{thm}

\newtcbtheorem[auto counter, number within=section]{faxiom}{Axiom}{%
    before=\par\vspace{2ex},
    boxsep=0.5\topsep,
    after=\par\vspace{2ex},
    colback=Apricot!5,
    colframe=Apricot!35!black,
    fonttitle=\bfseries\upshape%
}{ax}

\newtcbtheorem[use counter*=definition]{fdefinition}{Definition}{%
    before=\par\vspace{2ex},
    boxsep=0.5\topsep,
    after=\par\vspace{2ex},
    colback=blue!5!white,
    colframe=blue!75!black,
    fonttitle=\bfseries\upshape%
}{def}

\newtcbtheorem[use counter*=example]{fexample}{Example}{%
    before=\par\vspace{2ex},
    boxsep=0.5\topsep,
    after=\par\vspace{2ex},
    colback=red!5!white,
    colframe=red!75!black,
    fonttitle=\bfseries\upshape%
}{ex}

\newtcbtheorem[auto counter, number within=section]{fnotation}{Notation}{%
    before=\par\vspace{2ex},
    boxsep=0.5\topsep,
    after=\par\vspace{2ex},
    colback=SeaGreen!5!white,
    colframe=SeaGreen!75!black,
    fonttitle=\bfseries\upshape%
}{not}

\newtcbtheorem[use counter*=remark]{fremark}{Remark}{%
    fonttitle=\bfseries\upshape,
    colback=Goldenrod!5!white,
    colframe=Goldenrod!75!black}{ex}

\newenvironment{bproof}{\textit{Proof.}}{\hfill$\square$}
\tcolorboxenvironment{bproof}{%
    blanker,
    breakable,
    left=3mm,
    before skip=5pt,
    after skip=10pt,
    borderline west={0.6mm}{0pt}{green!80!black}
}

\AtEndEnvironment{lexample}{$\hfill\textcolor{red}{\blacksquare}$}
\newtcbtheorem[use counter*=example]{lexample}{Example}{%
    empty,
    title={Example~\theexample},
    boxed title style={%
        empty,
        size=minimal,
        toprule=2pt,
        top=0.5\topsep,
    },
    coltitle=red,
    fonttitle=\bfseries,
    parbox=false,
    boxsep=0pt,
    before=\par\vspace{2ex},
    left=0pt,
    right=0pt,
    top=3ex,
    bottom=1ex,
    before=\par\vspace{2ex},
    after=\par\vspace{2ex},
    breakable,
    pad at break*=0mm,
    vfill before first,
    overlay unbroken={%
        \draw[red, line width=2pt]
            ([yshift=-1.2ex]title.south-|frame.west) to
            ([yshift=-1.2ex]title.south-|frame.east);
        },
    overlay first={%
        \draw[red, line width=2pt]
            ([yshift=-1.2ex]title.south-|frame.west) to
            ([yshift=-1.2ex]title.south-|frame.east);
    },
}{ex}

\AtEndEnvironment{ldefinition}{$\hfill\textcolor{Blue}{\blacksquare}$}
\newtcbtheorem[use counter*=definition]{ldefinition}{Definition}{%
    empty,
    title={Definition~\thedefinition:~{#1}},
    boxed title style={%
        empty,
        size=minimal,
        toprule=2pt,
        top=0.5\topsep,
    },
    coltitle=Blue,
    fonttitle=\bfseries,
    parbox=false,
    boxsep=0pt,
    before=\par\vspace{2ex},
    left=0pt,
    right=0pt,
    top=3ex,
    bottom=0pt,
    before=\par\vspace{2ex},
    after=\par\vspace{1ex},
    breakable,
    pad at break*=0mm,
    vfill before first,
    overlay unbroken={%
        \draw[Blue, line width=2pt]
            ([yshift=-1.2ex]title.south-|frame.west) to
            ([yshift=-1.2ex]title.south-|frame.east);
        },
    overlay first={%
        \draw[Blue, line width=2pt]
            ([yshift=-1.2ex]title.south-|frame.west) to
            ([yshift=-1.2ex]title.south-|frame.east);
    },
}{def}

\AtEndEnvironment{ltheorem}{$\hfill\textcolor{Green}{\blacksquare}$}
\newtcbtheorem[use counter*=theorem]{ltheorem}{Theorem}{%
    empty,
    title={Theorem~\thetheorem:~{#1}},
    boxed title style={%
        empty,
        size=minimal,
        toprule=2pt,
        top=0.5\topsep,
    },
    coltitle=Green,
    fonttitle=\bfseries,
    parbox=false,
    boxsep=0pt,
    before=\par\vspace{2ex},
    left=0pt,
    right=0pt,
    top=3ex,
    bottom=-1.5ex,
    breakable,
    pad at break*=0mm,
    vfill before first,
    overlay unbroken={%
        \draw[Green, line width=2pt]
            ([yshift=-1.2ex]title.south-|frame.west) to
            ([yshift=-1.2ex]title.south-|frame.east);},
    overlay first={%
        \draw[Green, line width=2pt]
            ([yshift=-1.2ex]title.south-|frame.west) to
            ([yshift=-1.2ex]title.south-|frame.east);
    }
}{thm}

%--------------------Declared Math Operators--------------------%
\DeclareMathOperator{\adjoint}{adj}         % Adjoint.
\DeclareMathOperator{\Card}{Card}           % Cardinality.
\DeclareMathOperator{\curl}{curl}           % Curl.
\DeclareMathOperator{\diam}{diam}           % Diameter.
\DeclareMathOperator{\dist}{dist}           % Distance.
\DeclareMathOperator{\Div}{div}             % Divergence.
\DeclareMathOperator{\Erf}{Erf}             % Error Function.
\DeclareMathOperator{\Erfc}{Erfc}           % Complementary Error Function.
\DeclareMathOperator{\Ext}{Ext}             % Exterior.
\DeclareMathOperator{\GCD}{GCD}             % Greatest common denominator.
\DeclareMathOperator{\grad}{grad}           % Gradient
\DeclareMathOperator{\Ima}{Im}              % Image.
\DeclareMathOperator{\Int}{Int}             % Interior.
\DeclareMathOperator{\LC}{LC}               % Leading coefficient.
\DeclareMathOperator{\LCM}{LCM}             % Least common multiple.
\DeclareMathOperator{\LM}{LM}               % Leading monomial.
\DeclareMathOperator{\LT}{LT}               % Leading term.
\DeclareMathOperator{\Mod}{mod}             % Modulus.
\DeclareMathOperator{\Mon}{Mon}             % Monomial.
\DeclareMathOperator{\multideg}{mutlideg}   % Multi-Degree (Graphs).
\DeclareMathOperator{\nul}{nul}             % Null space of operator.
\DeclareMathOperator{\Ord}{Ord}             % Ordinal of ordered set.
\DeclareMathOperator{\Prin}{Prin}           % Principal value.
\DeclareMathOperator{\proj}{proj}           % Projection.
\DeclareMathOperator{\Refl}{Refl}           % Reflection operator.
\DeclareMathOperator{\rk}{rk}               % Rank of operator.
\DeclareMathOperator{\sgn}{sgn}             % Sign of a number.
\DeclareMathOperator{\sinc}{sinc}           % Sinc function.
\DeclareMathOperator{\Span}{Span}           % Span of a set.
\DeclareMathOperator{\Spec}{Spec}           % Spectrum.
\DeclareMathOperator{\supp}{supp}           % Support
\DeclareMathOperator{\Tr}{Tr}               % Trace of matrix.
%--------------------Declared Math Symbols--------------------%
\DeclareMathSymbol{\minus}{\mathbin}{AMSa}{"39} % Unary minus sign.
%------------------------New Commands---------------------------%
\DeclarePairedDelimiter\norm{\lVert}{\rVert}
\DeclarePairedDelimiter\ceil{\lceil}{\rceil}
\DeclarePairedDelimiter\floor{\lfloor}{\rfloor}
\newcommand*\diff{\mathop{}\!\mathrm{d}}
\newcommand*\Diff[1]{\mathop{}\!\mathrm{d^#1}}
\renewcommand*{\glstextformat}[1]{\textcolor{RoyalBlue}{#1}}
\renewcommand{\glsnamefont}[1]{\textbf{#1}}
\renewcommand\labelitemii{$\circ$}
\renewcommand\thesubfigure{%
    \arabic{chapter}.\arabic{figure}.\arabic{subfigure}}
\addto\captionsenglish{\renewcommand{\figurename}{Fig.}}
\numberwithin{equation}{section}

\renewcommand{\vector}[1]{\boldsymbol{\mathrm{#1}}}

\newcommand{\uvector}[1]{\boldsymbol{\hat{\mathrm{#1}}}}
\newcommand{\topspace}[2][]{(#2,\tau_{#1})}
\newcommand{\measurespace}[2][]{(#2,\varSigma_{#1},\mu_{#1})}
\newcommand{\measurablespace}[2][]{(#2,\varSigma_{#1})}
\newcommand{\manifold}[2][]{(#2,\tau_{#1},\mathcal{A}_{#1})}
\newcommand{\tanspace}[2]{T_{#1}{#2}}
\newcommand{\cotanspace}[2]{T_{#1}^{*}{#2}}
\newcommand{\Ckspace}[3][\mathbb{R}]{C^{#2}(#3,#1)}
\newcommand{\funcspace}[2][\mathbb{R}]{\mathcal{F}(#2,#1)}
\newcommand{\smoothvecf}[1]{\mathfrak{X}(#1)}
\newcommand{\smoothonef}[1]{\mathfrak{X}^{*}(#1)}
\newcommand{\bracket}[2]{[#1,#2]}

%------------------------Book Command---------------------------%
\makeatletter
\renewcommand\@pnumwidth{1cm}
\newcounter{book}
\renewcommand\thebook{\@Roman\c@book}
\newcommand\book{%
    \if@openright
        \cleardoublepage
    \else
        \clearpage
    \fi
    \thispagestyle{plain}%
    \if@twocolumn
        \onecolumn
        \@tempswatrue
    \else
        \@tempswafalse
    \fi
    \null\vfil
    \secdef\@book\@sbook
}
\def\@book[#1]#2{%
    \refstepcounter{book}
    \addcontentsline{toc}{book}{\bookname\ \thebook:\hspace{1em}#1}
    \markboth{}{}
    {\centering
     \interlinepenalty\@M
     \normalfont
     \huge\bfseries\bookname\nobreakspace\thebook
     \par
     \vskip 20\p@
     \Huge\bfseries#2\par}%
    \@endbook}
\def\@sbook#1{%
    {\centering
     \interlinepenalty \@M
     \normalfont
     \Huge\bfseries#1\par}%
    \@endbook}
\def\@endbook{
    \vfil\newpage
        \if@twoside
            \if@openright
                \null
                \thispagestyle{empty}%
                \newpage
            \fi
        \fi
        \if@tempswa
            \twocolumn
        \fi
}
\newcommand*\l@book[2]{%
    \ifnum\c@tocdepth >-3\relax
        \addpenalty{-\@highpenalty}%
        \addvspace{2.25em\@plus\p@}%
        \setlength\@tempdima{3em}%
        \begingroup
            \parindent\z@\rightskip\@pnumwidth
            \parfillskip -\@pnumwidth
            {
                \leavevmode
                \Large\bfseries#1\hfill\hb@xt@\@pnumwidth{\hss#2}
            }
            \par
            \nobreak
            \global\@nobreaktrue
            \everypar{\global\@nobreakfalse\everypar{}}%
        \endgroup
    \fi}
\newcommand\bookname{Book}
\renewcommand{\thebook}{\texorpdfstring{\Numberstring{book}}{book}}
\providecommand*{\toclevel@book}{-2}
\makeatother
\titleformat{\part}[display]
    {\Large\bfseries}
    {\partname\nobreakspace\thepart}
    {0mm}
    {\Huge\bfseries}
\titlecontents{part}[0pt]
    {\large\bfseries}
    {\partname\ \thecontentslabel: \quad}
    {}
    {\hfill\contentspage}
\titlecontents{chapter}[0pt]
    {\bfseries}
    {\chaptername\ \thecontentslabel:\quad}
    {}
    {\hfill\contentspage}
\newglossarystyle{longpara}{%
    \setglossarystyle{long}%
    \renewenvironment{theglossary}{%
        \begin{longtable}[l]{{p{0.25\hsize}p{0.65\hsize}}}
    }{\end{longtable}}%
    \renewcommand{\glossentry}[2]{%
        \glstarget{##1}{\glossentryname{##1}}%
        &\glossentrydesc{##1}{~##2.}
        \tabularnewline%
        \tabularnewline
    }%
}
\newglossary[not-glg]{notation}{not-gls}{not-glo}{Notation}
\newcommand*{\newnotation}[4][]{%
    \newglossaryentry{#2}{type=notation, name={\textbf{#3}, },
                          text={#4}, description={#4},#1}%
}
%--------------------------LENGTHS------------------------------%
% Spacings for the Table of Contents.
\addtolength{\cftsecnumwidth}{1ex}
\addtolength{\cftsubsecindent}{1ex}
\addtolength{\cftsubsecnumwidth}{1ex}
\addtolength{\cftfignumwidth}{1ex}
\addtolength{\cfttabnumwidth}{1ex}

% Indent and paragraph spacing.
\setlength{\parindent}{0em}
\setlength{\parskip}{0em}
%---------------GLOSSARY------------%
\makeglossaries
\loadglsentries{../glossary}
\loadglsentries{../acronym}
%--------------Title Page-----------%
\begin{document}
\chapter{Python}
\section{Summary of Tutorials}
\section{Tutorials from LearnPython.org}
\subsection{Introduction}
Python uses indents for blocks, rather than braces. Tabs and spaces are supported, by standard indentation requires Python code to use four spaces. To print a string in Python 3, do:\newline
\begin{minipage}[t]{.48\textwidth}
\centering
\begin{lstlisting}[language=bash,basicstyle=\small\ttfamily,frame=single,caption=input]
1  print("This line will be printed.")
2
3  x = 1
4  if x == 1
5      # indendted four spaces
6      print("x is 1.")
\end{lstlisting}
\end{minipage}\hfill
\begin{minipage}[t]{.48\textwidth}
\centering
\begin{lstlisting}[language=bash,basicstyle=\small\ttfamily,frame=single,caption=output]
This line will be printed.
x is 1.
\end{lstlisting}
\end{minipage}
\subsubsection{Hello World}
And now, for the quintessential coding problem: Print "Hello, World!"\newline
\begin{minipage}[t]{.48\textwidth}
\centering
\begin{lstlisting}[language=bash,basicstyle=\small\ttfamily,frame=single,caption=input]
1  print("Hello, World!")
\end{lstlisting}
\end{minipage}\hfill
\begin{minipage}[t]{.48\textwidth}
\centering
\begin{lstlisting}[language=bash,basicstyle=\small\ttfamily,frame=single,caption=output]
Hello, World!
\end{lstlisting}
\end{minipage}
\subsection{Variables and Types}
Python is an object oriented language. Variables do not need to be declared before using them, nor do their types need to be declared. Every variable in Python is an object.
\subsubsection{Numbers}
Python supports three types of numbers: Integers, floating point numbers, and complex numbers. To define an integer, do the following:\newline
\begin{minipage}[t]{.48\textwidth}
\centering
\begin{lstlisting}[language=bash,basicstyle=\small\ttfamily,frame=single,caption=input]
1  myint = 7
2  print(myint)
\end{lstlisting}
\end{minipage}\hfill
\begin{minipage}[t]{.48\textwidth}
\centering
\begin{lstlisting}[language=bash,basicstyle=\small\ttfamily,frame=single,caption=output]
7
\end{lstlisting}
\end{minipage}\newline
To define a floating point number, use one of the following notations:\newline
\begin{minipage}[t]{.48\textwidth}
\centering
\begin{lstlisting}[language=bash,basicstyle=\small\ttfamily,frame=single,caption=input]
1  myfloat = 7.0
2  print(myfloat)
3  myfloat = float(7)
4  print(myfloat)
\end{lstlisting}
\end{minipage}\hfill
\begin{minipage}[t]{.48\textwidth}
\centering
\begin{lstlisting}[language=bash,basicstyle=\small\ttfamily,frame=single,caption=output]
7.0
7.0
\end{lstlisting}
\end{minipage}
\subsubsection{Strings}
Strings are defined with either " or ' quotations. Using " allows apostrophe's to be included in a string, whereas using ' means an apostrophe would terminate the string.\newline
\begin{minipage}[t]{.48\textwidth}
\centering
\begin{lstlisting}[language=bash,basicstyle=\small\ttfamily,frame=single,caption=input]
1  mystring = 'hello'
2  print(mystring)
3  mystring = "hello"
4  print(mystring)
5
6  mystring = "Don't worry"
7  print(mystring)
\end{lstlisting}
\end{minipage}\hfill
\begin{minipage}[t]{.48\textwidth}
\centering
\begin{lstlisting}[language=bash,basicstyle=\small\ttfamily,frame=single,caption=output]
hello
hello
Don't worry
\end{lstlisting}
\end{minipage}\newline
\begin{minipage}[t]{.48\textwidth}
\centering
\begin{lstlisting}[language=bash,basicstyle=\small\ttfamily,frame=single,caption=input]
1  mystring = 'Don't worry'
2  print(mystring)
\end{lstlisting}
\end{minipage}\hfill
\begin{minipage}[t]{.48\textwidth}
\centering
\begin{lstlisting}[language=bash,basicstyle=\small\ttfamily,frame=single,caption=output]
SyntaxError: invalid syntax
\end{lstlisting}
\end{minipage}
\subsubsection{Basic Operations on Numbers and Strings}
Strings can be defined to include things like carriage returns, backslashes and Unicode characters. Simple operations like addition can be performed on numbers and strings. Also, multiple variables can be assigned simultaneously on the same line.\newline
\begin{minipage}[t]{.48\textwidth}
\centering
\begin{lstlisting}[language=bash,basicstyle=\small\ttfamily,frame=single,caption=input]
1  one = 1
2  two = 2
3  three = one + two
4  print(three)
5
6  hello = "hello"
7  world = "world"
8  helloworld = hello + " " + world
9  print(helloworld)
10
11 a, b = 3, 4
12  print(a,b)
\end{lstlisting}
\end{minipage}\hfill
\begin{minipage}[t]{.48\textwidth}
\centering
\begin{lstlisting}[language=bash,basicstyle=\small\ttfamily,frame=single,caption=output]
3
hello world
3, 4
\end{lstlisting}
\end{minipage}\newline
Mixing strings and numbers results in an error.\newline
\begin{minipage}[t]{.48\textwidth}
\centering
\begin{lstlisting}[language=bash,basicstyle=\small\ttfamily,frame=single,caption=input]
1  # This will not work!
2  one = 1
3  two = 2
4  hello = "hello"
5  print(one + two + hello)
\end{lstlisting}
\end{minipage}\hfill
\begin{minipage}[t]{.48\textwidth}
\centering
\begin{lstlisting}[language=bash,basicstyle=\small\ttfamily,frame=single,caption=output]
Traceback (most recent call last):
  File "<stdin>", line 6, in <module>
    print(one + two + hello)
TypeError: unsupported operand type(s)
    for +: 'int' and 'str'
\end{lstlisting}
\end{minipage}
\newpage
\subsection{Lists}
Lists can contain any type of variable and as many variables as desired. They can be iterated over as well.\newline
\begin{minipage}[t]{.48\textwidth}
\centering
\begin{lstlisting}[language=bash,basicstyle=\small\ttfamily,frame=single,caption=input]
1  mylist = []
2  mylist.append(1)
3  mylist.append(2)
4  mylist.append(3)
5  print(mylist[0]) # prints 1
6  print(mylist[1]) # prints 2
7  print(mylist[2]) # prints 3
8
9  # prints out 1,2,3
10 for x in mylist:
11     print(x)
\end{lstlisting}
\end{minipage}\hfill
\begin{minipage}[t]{.48\textwidth}
\centering
\begin{lstlisting}[language=bash,basicstyle=\small\ttfamily,frame=single,caption=output]
1
2
3
1
2
3
\end{lstlisting}
\end{minipage}\newline
Accessing an index which does not exist in a list produces an error.\newline
\begin{minipage}[t]{.48\textwidth}
\centering
\begin{lstlisting}[language=bash,basicstyle=\small\ttfamily,frame=single,caption=input]
1  mylist = [1,2,3]
2  print(mylist[10])
\end{lstlisting}
\end{minipage}\hfill
\begin{minipage}[t]{.48\textwidth}
\centering
\begin{lstlisting}[language=bash,basicstyle=\small\ttfamily,frame=single,caption=output]
Traceback (most recent call last):
  File "<stdin>", line 2, in <module>
    print(mylist[10])
IndexError: list index out of range
\end{lstlisting}
\end{minipage}\newline
Both \%s and \% are borrowed from C. It acts as a placeholder in a string, so that other values can be added to replace this later. Verbatim from the Python Documentation:\newline
\textcolor{blue}{Python supports formatting values into strings. Although this can include very complicated expressions, the most basic usage is to insert values into a string with the \%s placeholder.}\newline
For example:\newline
\begin{minipage}[t]{.48\textwidth}
\centering
\begin{lstlisting}[language=bash,basicstyle=\small\ttfamily,frame=single,caption=input]
1  print("Hello %s, my name is\
%s" % ('john', 'mike'))
\end{lstlisting}
\end{minipage}\hfill
\begin{minipage}[t]{.48\textwidth}
\centering
\begin{lstlisting}[language=bash,basicstyle=\small\ttfamily,frame=single,caption=output]
Hello john, my name is mike
\end{lstlisting}
\end{minipage}\newline
The \textbackslash\ character is used to continue a line of code onto the next line. That is, the computer sees the two lines in the previous example as one long line. Here is a brief tutorial on how to append variables to different lists.
\begin{lstlisting}[language=bash,basicstyle=\small\ttfamily,frame=single,caption=input]
1  numbers = []
2  strings = []
3  names = ["John", "Eric", "Jessica"]
4  numbers.append(1)
5  numbers.append(2)
6  numbers.append(3)
7  strings.append("hello")
8  strings.append("world")
9
10 second_name = names[1]
12
13 print(numbers)
14 print(strings)
15 print("Second name on the names list is %s" % second_name)
\end{lstlisting}
\begin{lstlisting}[language=python,frame=single,basicstyle=\footnotesize,frame=single,caption=output]
[1, 2, 3]
['hello', 'world']
Second name on the names list is Eric
\end{lstlisting}
\subsection{Basic Operators}
\subsubsection{Arithmetic Operators}
The four basic operations can be used with numbers inside of Python. Addition $(+)$, subtraction $(-)$, multiplication $(*)$ and division $(/)$ are all supported. Python also follows the standard \gls{pemdas} order of operations. Another useful operator is the modulo operator $(\%)$. This returns the integer remainder of division. Exponents $(**)$ are also allowed.\newline
\begin{minipage}[t]{.48\textwidth}
\centering
\begin{lstlisting}[language=python,frame=single,basicstyle=\footnotesize,frame=single,caption=input]
1  number = 1 + 2 * 3 / 4.0
2  print(number)
3  remainder = 11 % 3
4  print(remainder)
5  squared = 7 ** 2
6  print(squared)
7  cubed = 2 ** 3
8  print(cubed)
\end{lstlisting}
\end{minipage}\hfill
\begin{minipage}[t]{.48\textwidth}
\centering
\begin{lstlisting}[language=python,frame=single,basicstyle=\footnotesize,frame=single,caption=output]
2.5
2
49
8
\end{lstlisting}
\end{minipage}
\subsubsection{String Operators}
String can be concatenated by using $+$ with two strings. Multiplying a string by an integer $n$ produces a string that is $n$ multiples of the original.\newline
\begin{minipage}[t]{.48\textwidth}
\centering
\begin{lstlisting}[language=python,frame=single,basicstyle=\footnotesize,frame=single,caption=input]
1  greetings = "hello" + " " + "world"
2  print(greetings)
3  lotsofhellos = "hello" * 7
4  print(lotsofhellos)
\end{lstlisting}
\end{minipage}\hfill
\begin{minipage}[t]{.48\textwidth}
\centering
\begin{lstlisting}[language=python,frame=single,basicstyle=\footnotesize,frame=single,caption=output]
1  hello world
2  hellohellohellohellohellohellohello
\end{lstlisting}
\end{minipage}
\subsubsection{Using Operators with Lists}
List can be concatenated with the $+$ operator. Similar to how $\%s$ is used as a placeholder for strings, $\%d$ is used as a placeholder for integer numbers.\newline
\begin{minipage}[t]{.48\textwidth}
\centering
\begin{lstlisting}[language=python,frame=single,basicstyle=\footnotesize,frame=single,caption=input]
1  even_nums = [2,4,6,8]
2  odd_nums = [1,3,5,7]
3  all_nums = odd_nums + even_nums
4  print(all_nums)
5
6  print([1,2,3] * 3)
\end{lstlisting}
\end{minipage}\hfill
\begin{minipage}[t]{.48\textwidth}
\centering
\begin{lstlisting}[language=bash,basicstyle=\small\ttfamily,frame=single,caption=output]
[1, 3, 5, 7, 2, 4, 6, 8]
[1, 2, 3, 1, 2, 3, 1, 2, 3]
\end{lstlisting}
\end{minipage}
\begin{minipage}[t]{0.48\textwidth}
\begin{lstlisting}[language=python,frame=single,basicstyle=\footnotesize,frame=single,caption=input]
1  name, number = 'Bob Guy', 42
3  print('%s %d' % (name, number))
4  x, y     = object(),object()
5  x_list   = [x] * 10
6  y_list   = [y] * 10
7  big_list = x_list + y_list
8  nx, ny   = len(x_list), len(y_list)
9  nz       = len(big_list)
10 print("x_list: %d objects" % nx)
11 print("y_list: %d objects" % ny)
12 print("big_list: %d objects" % z)
13 if x_list.count(x) == 10:
14     print("Almost there...")
15 if y_list.count(y) == 10:
16     print("Almost there...")
17  if big_list.count(x) == 10:
18      print("Nearly done...")
19  if big_list.count(y) == 10:
20      print("Great!")
\end{lstlisting}
\end{minipage}\hfill
\begin{minipage}[t]{0.48\textwidth}
\begin{lstlisting}[language=python,frame=single,basicstyle=\footnotesize,frame=single,caption=output]
Bob Guy 42
x_list contains 10 objects
y_list contains 10 objects
big_list contains 20 objects
Almost there...
Great!
\end{lstlisting}
\end{minipage}
\clearpage
\subsection{String Formatting}
Python uses C-style string formatting to create new formatted strings. The $\%$ operator (Which is also the modulo operator) is used to fomrat a set of variables enclosed in an n-tupyle, which is a fixed size list of $n$ items, together with a format string, which contains normal text and argument specifiers. Argument specifiers are the characters like $\%s$ and $\%d$ that we've already seen. Objectes that are not strings can be formatted as strings using the $\%s$ operator as well. Here are the basic argument specifiers:
\begin{itemize}
    \begin{multicols}{2}
        \item $\%s$ - String
        \item $\%d$ - Integers
        \item $\%f$ - Floating point numbers
        \item $\%.nf$ - Float with n decimals
        \item $\%x$ - Int in hex representation (lowercase)
        \item $\%X$ - Int in hex representation (uppercase)
    \end{multicols}
\end{itemize}
\begin{lstlisting}[language=python,frame=single,basicstyle=\footnotesize,frame=single,caption=input]
1  # This prints out "Hello, John!"
2  name = "John"
3  print("Hello, %s!" % name)
4
5  # This prints out "John is 23 years old."
6  name = "John"
7  age = 23
8  print("%s is %d years old."% (name, age))
9
10 # This prints out: A list: [1, 2, 3]
11 mylist = [1,2,3]
12 print("A list: %s" % mylist)
13
14 # Example exercise
15 data = ("John", "Doe", 53.44)
16 format_string = "Hello %s %s. Your current balance is $%s."
17 print(format_string % data)
\end{lstlisting}
\begin{lstlisting}[language=python,frame=single,basicstyle=\footnotesize,frame=single,caption=output]
Hello, John!
John is 23 years old.
A list: [1, 2, 3]
Hello John Doe. Your current balance is $53.44.
\end{lstlisting}

\subsection{Conditions}
Python uses Boolean variables to evaluate conditions. True or False are returned when an expression is compared or evaluated. Variable assignment is done using the equals sign $(=)$, but comparisons between two variables are done using double equal signs $(==)$. The 'not equals' operators is an exclamation point an an equals sign $(!=)$.
\subsubsection{Boolean Operators}
The ``and" and ``or" Boolean operators allowing building Boolean expressions. $A$ and $B$ only returns True if both $A$ and $B$ are true. $A$ or $B$ returns True if either of $A$ or $B$ are true. The ``in" operator allows one to check if a specified object exists within an iterable object container, such as a list. This is very similar to the symbol $\in$ that is used in mathematics.\newline
\begin{minipage}[t]{0.48\textwidth}
\begin{lstlisting}[language=python,frame=single,basicstyle=\footnotesize,frame=single,caption=input]
1  x = 2
2  print(x == 2) # prints out True
3  print(x == 3) # prints out False
4  print(x < 3)  # prints out True
5  name = "John"
6  age = 23
7  if name == "John" and age == 23:
8      print("You're John and you are 23")
9  if name == "John" or name == "Joe":
10     print("You're either John or Joe")
11 if name in ["John", "Bob"]:
12     print("You're either John or Bob")
\end{lstlisting}
\end{minipage}\hfill
\begin{minipage}[t]{0.48\textwidth}
\begin{lstlisting}[language=python,frame=single,basicstyle=\footnotesize,frame=single,caption=output]
True
False
True
You're John and you are 23
You're either John or Joe
You're either John or Bob
\end{lstlisting}
\end{minipage}
Python uses indendation to define code blocks. Standard indent is 4 spaces, although tabs and other spacings work as long as its consistent. Code blocks do not need any termination either.\newline
\begin{minipage}[t]{.48\textwidth}
\centering
\begin{lstlisting}[language=python,frame=single,basicstyle=\footnotesize,frame=single,caption=input]
1  x = 2
2  if x == 2:
3      print("x equals two!")
4  else:
5     print("x does not equal two.")
\end{lstlisting}
\end{minipage}\hfill
\begin{minipage}[t]{.48\textwidth}
\centering
\begin{lstlisting}[language=python,frame=single,basicstyle=\footnotesize,frame=single,caption=output]
x equals two!
\end{lstlisting}
\end{minipage}\newline
A statement is considered true of one of the following holds:
\begin{enumerate}
    \item The True Boolean variable is given or calculated using an expression.
    \item An object which is not considered empty is passed. Empty objects are those such as the following:
    \begin{enumerate}
        \begin{multicols}{2}
            \item An empty string: ` ' or `` "
            \item An empty list: [ ]
            \item The number zero: 0
            \item The false Boolean variable: False
        \end{multicols}
    \end{enumerate}
\end{enumerate}
\subsubsection{More Operators}
The ``is" operators is different from the equals operator $(==)$. Instead of comparing the values of the variables, the ``is" operator compares the instances themselves. Using the ``not" operator on a Boolean expression simply inverts it.
\newpage
\begin{minipage}[t]{.48\textwidth}
\centering
\begin{lstlisting}[language=python,frame=single,basicstyle=\footnotesize,frame=single,caption=input]
1  x = [1,2,3]
2  y = [1,2,3]
3  z = x
4  print(x == y) # Prints out True
5  print(x is y) # Prints out False
6  print(x is z) # Prints out True
7  print(not False)
8  print((not False) == (False))
\end{lstlisting}
\end{minipage}\hfill
\begin{minipage}[t]{.48\textwidth}
\centering
\begin{lstlisting}[language=bash,basicstyle=\small\ttfamily,frame=single,caption=output]
True
False
True
True
False
\end{lstlisting}
\end{minipage}
\subsection{Loops}
\subsubsection{For Loops}
For loops iterator over a sequence of numbers using the ``range" and ``xrange" functions. The In Python 3 there is only ``range," which is a direct descendant of the Python 2 ``xrange." While loops repeat so long as a certain Boolean condition is True. \newline
\begin{minipage}[t]{.48\textwidth}
\centering
\begin{lstlisting}[language=python,frame=single,basicstyle=\footnotesize,frame=single,caption=input]
1  # Prints out the numbers 0,1,2,3,4
2  for x in range(5):
3      print(x)
4
5  # Prints out 3,4,5
6  for x in range(3, 6):
7      print(x)
8
9  # Prints out 3,5,7
10 for x in range(3, 8, 2):
11     print(x)
12
13 # Prints out 0,1,2,3,4
14
15 count = 0
16 while count < 5:
17     print(count)
18     count += 1 # count = count + 1
\end{lstlisting}
\end{minipage}\hfill
\begin{minipage}[t]{.48\textwidth}
\centering
\begin{lstlisting}[language=python,frame=single,basicstyle=\footnotesize,frame=single,caption=output]
0
1
2
3
4
3
4
5
3
5
7
0
1
2
3
4
0
1
2
3
4
\end{lstlisting}
\end{minipage}\newline
The ``break" command is used to exit a for loop or a while loop. The ``continue" command is used to skip the current block and to return to the for loop or while loop.\newline
\begin{minipage}[t]{.48\textwidth}
\centering
\begin{lstlisting}[language=python,frame=single,basicstyle=\footnotesize,frame=single,caption=input]
1  count = 0
2  while True:
3      print(count)
4      count += 1
5      if count >= 5:
6          break
7
8  for x in range(10):
9     # Check if x is even
10     if x % 2 == 0:
11         continue # Odd numbers only
12     print(x)
\end{lstlisting}
\end{minipage}\hfill
\begin{minipage}[t]{.48\textwidth}
\centering
\begin{lstlisting}[language=python,frame=single,basicstyle=\footnotesize,frame=single,caption=output]
0
1
2
3
4
1
3
5
7
9
\end{lstlisting}
\end{minipage}
Unlike in C and C++, else can be used in for loops.\newline
\begin{minipage}[t]{.48\textwidth}
\centering
\begin{lstlisting}[language=python,frame=single,basicstyle=\footnotesize,frame=single,caption=input]
1  count=0
2  while(count<5):
3      print(count)
4      count +=1
5  else:
6      print("value: %d" %(count))
7
8  # Prints out 1,2,3,4
9  for i in range(1, 10):
10     if(i%5==0):
11         break
12     print(i)
13 else:
14     print("this is not printed")
\end{lstlisting}
\end{minipage}\hfill
\begin{minipage}[t]{.48\textwidth}
\centering
\begin{lstlisting}[language=python,frame=single,basicstyle=\footnotesize,frame=single,caption=output]
0
1
2
3
4
value: 5
1
2
3
4
\end{lstlisting}
\end{minipage}
\subsection{Functions}
Functions allow one to divide code into blocks, order code, and make it more readable and easier to reuse. They also provide a means of sharing code. Python makes use of blocks. Functions can recieve arguments, which are variables passed from the caller to the function. In addition, functions can return values to the caller using the ``return" keyword.
\begin{lstlisting}[language=python,frame=single,basicstyle=\footnotesize,frame=single,caption=Block Example]
1  block_head:
2      1st block line
3      2nd block line
4      ...
\end{lstlisting}
A block line is more Python code, including another block, and the block head contains keywords like ``if", ``or", ``for", and ``while," and also something like block\_name(argument1,...,argumentn). Functions in Python are defined using the block keyword ``def" followed with the function's name as the block's name. 
\begin{lstlisting}[language=python,frame=single,basicstyle=\footnotesize,frame=single,caption=input]
1  def my_function():
2      print("Hello From My Function!")
3  def my_function_with_args(username, greeting):
4      print("Hello, %s! I wish you %s"%(username, greeting))
5  def sum_two_numbers(a, b):
6      return a + b
7
8  # print(a simple greeting)
9  my_function()
10
11 #prints - "Hello, John Doe! I wish you a great year!"
12 my_function_with_args("John Doe", "a great year!")
13
14 # after this line x will hold the value 3!
15 x = sum_two_numbers(1,2)
16 print(x)
\end{lstlisting}
\begin{lstlisting}[language=python,frame=single,basicstyle=\footnotesize,frame=single,caption=output]
Hello From My Function!
Hello, John Doe! I wish you a great year!
3
\end{lstlisting}
\newpage
An example of many functions being used together:
\begin{lstlisting}[language=python,frame=single,basicstyle=\footnotesize,frame=single,caption=input]
1  def list_benefits():
2      return "More organized code",\
3      "More readable code",\ 
4      "Easier code reuse",\
5      "Allowing programmers to share their code"
6
7  def build_sentence(benefit):
8      return "%s is a benefit of functions!" % benefit
9
10 def name_the_benefits_of_functions():
11     list_of_benefits = list_benefits()
12     for benefit in list_of_benefits:
13         print(build_sentence(benefit))
14
15 name_the_benefits_of_functions()
\end{lstlisting}
\begin{lstlisting}[language=python,frame=single,basicstyle=\footnotesize,frame=single,caption=output]
More organized code is a benefit of functions!
More readable code is a benefit of functions!
Easier code reuse is a benefit of functions!
Allowing programmers to share their code is a benefit of functions!
\end{lstlisting}
\subsection{Classes and Objects}
Objects are a combination of variables and functions into a single entity. Objects get these things from classes, which are essentially templates for creating objects. When a variable holds a class, it holds all of the objects within it. You can create multiple different objects that are of the same class and have the same variables and functions defined, but each object will contain independent copies of the variables in the class. Here's an example of a class:\newline
\begin{minipage}[t]{.48\textwidth}
\begin{lstlisting}[language=python,frame=single,basicstyle=\footnotesize,frame=single,caption=input]
1  class MyClass:
2      variable = "blah"
3
4      def function(self):
5          print("Message Inside Class.")
6
7  myobjectx = MyClass()
8  myobjecty = MyClass()
9
10 myobjecty.variable = "yackity"
11
12 # Then print out both values
13 print(myobjectx.variable)
14 print(myobjecty.variable)
15
16 myobjectx.function()
\end{lstlisting}
\end{minipage}\hfill
\begin{minipage}[t]{.48\textwidth}
\begin{lstlisting}[language=python,frame=single,basicstyle=\footnotesize,frame=single,caption=output]
blah
yackity
Message Inside Class.
\end{lstlisting}
\end{minipage}
\newpage
\subsection{Dictionaries}
A dictionary is a data type that works with keys and values, rather than indexes. A key is any type of object, a string, a number, a list, etc.\newline
\begin{minipage}[t]{.48\textwidth}
\begin{lstlisting}[language=python,frame=single,basicstyle=\footnotesize,frame=single,caption=input]
1  phonebook = {}
2  phonebook["John"] = 7566
3  phonebook["Jack"] = 7264
4  phonebook["Jill"] = 2781
5  print(phonebook)
6
7  phonebook1 = {
8      "John" : 7566,
9      "Jack" : 7264,
10     "Jill" : 2781
11 }
12 print(phonebook1)
\end{lstlisting}
\end{minipage}\hfill
\begin{minipage}[t]{.48\textwidth}
\begin{lstlisting}[language=python,frame=single,basicstyle=\footnotesize,frame=single,caption=output]
{'Jill': 2781, 'Jack': 7264, 'John': 7566}
{'Jill': 2781, 'Jack': 7264, 'John': 7566}
\end{lstlisting}
\end{minipage}
Dictionaries can be iterated over as well. Dictionaries do not keep the order of the values stored in it, so iterating is a little different.
\begin{lstlisting}[language=python,frame=single,basicstyle=\footnotesize,frame=single,caption=output]
1  phonebook = {"John" : 938477566,"Jack" : 938377264,"Jill" : 947662781}
2  for name, number in phonebook.items():
3      print("Phone number of %s is %d" % (name, number))
\end{lstlisting}
\begin{lstlisting}[language=python,frame=single,basicstyle=\footnotesize,frame=single,caption=output]
Phone number of Jill is 947662781
Phone number of Jack is 938377264
Phone number of John is 938477566
\end{lstlisting}
There are two ways to remove items from a dictionary.\newline
\begin{minipage}[t]{.48\textwidth}
\begin{lstlisting}[language=python,frame=single,basicstyle=\footnotesize,frame=single,caption=output]
1  phonebook = {
2     "John" : 938477566,
3     "Jack" : 938377264,
4     "Jill" : 947662781
5  }
6
7  del phonebook["John"]
8  print(phonebook)
9
10 phonebook.pop("Jack")
11 print(phonebook)
\end{lstlisting}
\end{minipage}\hfill
\begin{minipage}[t]{.48\textwidth}
\begin{lstlisting}[language=python,frame=single,basicstyle=\footnotesize,caption=output]
{'Jill': 947662781, 'Jack': 938377264}
{'Jill': 947662781}
\end{lstlisting}
\end{minipage}\newline
\subsection{Modules and Packages}
A module is a piece of software that has a specific functionality. Each module is a different file which can be edited separately. Modules in Python are Python files with a .py extension. The name of the module is the name of the file. A Python module can have functions, classes, or variables defined and implemented. To import from other modules use the ``import" command.
\section{Miscellaneous Tests}
\subsection{Time Test for Squaring an Array}
Running MacOSX 10.13.4 High Sierra on an iMac with a 3.4 GHz intel quad-core i5 processor, it has been shown that, while running Python 3.6.3 and Numpy 1.14.1, the squaring is much slower than multiplying a variable by itself.\newline
\begin{minipage}[t]{.48\textwidth}
\begin{lstlisting}[language=python,frame=single,basicstyle=\footnotesize,caption=Python Code Located in `test.py']
import time

def square(x):
    t1       = time.time()
    for i in range(10000000): y = x**2
    t2       = time.time()
    t        = t2-t1
    return t

def square2(x):
    t1       = time.time()
    for i in range(10000000): y = x*x
    t2       = time.time()
    t        = t2-t1
    return t
\end{lstlisting}
\end{minipage}\hfill
\begin{minipage}[t]{.48\textwidth}
\begin{lstlisting}[language=python,frame=single,basicstyle=\footnotesize,caption=Inside iPython]
In [1]: import test

In [2]: test.square(10)
Out[2]: 2.5573959350585938

In [3]: test.square2(10)
Out[3]: 0.3849520683288574

In [4]: x = np.arange(100)

In [5]: test.square(x)
Out[5]: 6.9455132484436035

In [6]: test.square2(x)
Out[6]: 4.55057692527771
\end{lstlisting}
\end{minipage}\newline
A similar test for multiplication versus adding.\newline
\begin{minipage}[t]{.48\textwidth}
\centering
\begin{lstlisting}[language=python,frame=single,basicstyle=\footnotesize,caption=Contents of test.py]
import time

def mult(x):
    t1       = time.time()
    for i in range(10000000): y = 2*x
    t2       = time.time()
    t        = t2-t1
    return t

def add(x):
    t1       = time.time()
    for i in range(10000000): y = x+x
    t2       = time.time()
    t        = t2-t1
    return t
\end{lstlisting}
\end{minipage}\hfill
\begin{minipage}[t]{.48\textwidth}
\centering
\begin{lstlisting}[language=python,frame=single,basicstyle=\footnotesize,caption=Inside iPython]
In [1]: from test import *

In [2]: mult(100)
Out[2]: 0.44980788230895996

In [3]: add(100)
Out[3]: 0.3794879913330078

In [4]: mult(10000000)
Out[4]: 0.5141329765319824

In [5]: add(10000000)
Out[5]: 0.4880061149597168
\end{lstlisting}
\end{minipage}
\end{document}