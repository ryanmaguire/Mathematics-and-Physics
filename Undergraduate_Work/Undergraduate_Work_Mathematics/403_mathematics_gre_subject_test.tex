\documentclass[crop=false,class=book]{standalone}
\RequirePackage{etex}
\usepackage{geometry}
\geometry{a4paper, margin = 1.0in}
\usepackage[utf8]{inputenc}
\usepackage[english]{babel}
\usepackage[dvipsnames]{xcolor}
\usepackage{graphicx}
\usepackage{listings}
\usepackage{mathtools}
\usepackage{amsfonts,amsthm,esint,mathrsfs}
\usepackage{paracol}
\usepackage{wrapfig}
\usepackage[nottoc]{tocbibind}
\usepackage{natbib}
\usepackage[font={scriptsize,it}]{caption}
\usepackage{float}
\usepackage{multicol}
\setlength{\multicolsep}{6.0pt plus 2.0pt minus 1.5pt}
\usepackage[font={scriptsize}]{subcaption}
\usepackage{pgfplots,tikz,tkz-euclide}
\usetikzlibrary{calc,patterns,angles,quotes,arrows,arrows.meta,shapes,shapes.geometric,cd,hobby,positioning,decorations.markings}
\usetkzobj{all}
\pgfplotsset{compat=1.9}
\usepackage{hyperref}
\usepackage[toc,acronym,nogroupskip]{glossaries}
\hypersetup{colorlinks=true,linkcolor=blue,filecolor=magenta,urlcolor=Cerulean,citecolor=SkyBlue}
\usepackage{enumitem}
\usepackage{upgreek}
\graphicspath{{../images/}}
%-------------Tikz Presets----------%
\pgfdeclareradialshading{myring}{\pgfpointorigin}{color(0cm)=(transparent!0);color(5mm)=(pgftransparent!50);color(1cm)=(pgftransparent!100)}
\pgfdeclarefading{ringo}{\pgfuseshading{myring}}
%------------Theorem Styles---------%
\newtheoremstyle{mystyle}{0.1em}{0pt}{}{}{\bfseries}{}{.5em}{}
\theoremstyle{mystyle}
\newtheorem{theorem}{Theorem}[section]
\newtheorem{definition}{Definition}[section]
\newtheorem{lemma}{Lemma}[section]
\newtheorem{corollary}{Corollary}[section]
\newtheorem{proposition}{Proposition}[section]
\newtheorem{example}{Example}[section]
\newtheorem{problem}{Problem}[section]
\newtheorem{question}{Question}[section]
\newtheorem{remark}{Remark}[section]
\newtheorem{properties}{Properties}[section]
\newtheorem{notation}{Notation}[section]
\newtheorem{axiom}{Axiom}[section]
\newtheorem*{theorem*}{Theorem}
\newtheorem*{definition*}{Definition}
\newtheorem*{properties*}{Properties}
\newtheorem*{remark*}{Remark}
%--------Declared Math Operators----%
\DeclareMathOperator{\Refl}{Refl}
\DeclareMathOperator{\Span}{Span}
\DeclareMathOperator{\sgn}{sgn}
\DeclareMathOperator{\multideg}{mutlideg}
\DeclareMathOperator{\GCD}{GCD}
\DeclareMathOperator{\LC}{LC}
\DeclareMathOperator{\LT}{LT}
\DeclareMathOperator{\Tr}{Tr}
\DeclareMathOperator{\LM}{LM}
\DeclareMathOperator{\rk}{rk}
\DeclareMathOperator{\nul}{nul}
\DeclareMathOperator{\LCM}{LCM}
\DeclareMathOperator{\Mon}{Mon}
\DeclareMathOperator{\Spec}{Spec}
\DeclareMathOperator{\proj}{proj}
\DeclareMathOperator{\comp}{comp}
\DeclareMathOperator{\sinc}{sinc}
%-------------New Commands----------%
\DeclarePairedDelimiter\norm{\lVert}{\rVert}
\DeclarePairedDelimiter\ceil{\lceil}{\rceil}
\DeclarePairedDelimiter\floor{\lfloor}{\rfloor}
\renewcommand*{\glstextformat}[1]{\textcolor{RoyalBlue}{#1}}
\renewcommand{\glsnamefont}[1]{\textbf{#1}}
\renewcommand\labelitemii{$\circ$}
\renewcommand\thesubfigure{\arabic{chapter}.\arabic{figure}.\arabic{subfigure}}
%---------------GLOSSARY------------%
\makeglossaries
\loadglsentries{../glossary}
\loadglsentries{../acronym}
%--------------Title Page-----------%
\setlength{\parindent}{0em}
\setlength{\parskip}{0em}
\setenumerate{itemsep=0pt,topsep=2pt}
\pagenumbering{arabic}
\begin{document}
\chapter{GRE Mathematics Subject Test}
\section{Theorems for the GRE}
\begin{theorem}[Mean Value Theorem]
If $f:(a,b)\rightarrow \mathbb{R}$ is continuous and bounded, and if $x\in (a,b)$, then there is a $c\in(a,x)$ such that $\int_{a}^{x}f = (x-a)f(c)$.
\end{theorem}
\begin{theorem}[Generalized Fundamental Theorem of Calculus]
If $\mathcal{U}$ is a non-empty subset of $\mathbb{R}$, $a\in \mathcal{U}$, and if $f:\mathcal{U}\rightarrow \mathbb{R}$ is bounded and continuous, then $F:\mathcal{U}\rightarrow \mathbb{R}$ defined by $F(x) = \int_{\mathcal{U}\cap (a,x)}f$ is differentiable and $F'(x) = f(x)$
\end{theorem}
\begin{proof}
For let $x\in \mathcal{U}$. Let $\{x_n\}_{n=1}^{\infty}\subset \mathcal{U}$ be a sequence such that $x_n \rightarrow x$, $x\notin \{x_n\}_{n=1}^{\infty}$. As $\mathcal{U}$ is open and $x\in \mathcal{U}$, there is an $\varepsilon_1>0$ such that $B_{\varepsilon_1}(x)\subset \mathcal{U}$. But, as $x_n\rightarrow x$, there is an $N\in \mathbb{N}$ such that for all $n>N$, $x_n\in B_{\varepsilon_1}(x)$. But then for all $n>N$, $\int_{\mathcal{U}\cap(a,x)}f - \int_{\mathcal{U}\cap(a,x_n)}f = \int_{x_n}^{x}f$. But, as $f$ is continuous, by the mean value theorem for all $n>N$ there is a $c_n\in (x_n,x)$ such that $\int_{x_n}^{x}f = (x-x_n)f(c_n)$. But then $\big|\frac{\int_{x_n}^{x}f}{x-x_n} - f(x)| = |f(c_n)-f(x)|$. But $c_n \in (x_n,c)$, and $x_n \rightarrow x$, and therefore $c_n \rightarrow x$. But $f$ is continuous, and therefore $f(c_n)\rightarrow f(x)$. Therefore, by the definition of the derivative of $F$ at $x$, $F'(x) = f(x)$. 
\end{proof}
\section{GR0568}
\begin{problem}
What is the length of the curve with parametric equations $x=\cos(t)$, $y=\sin(t)$, for $0\leq t \leq \pi$.
\end{problem}
\begin{proof}[Solution]
\begin{definition}
If $\Gamma(t) = \big(x(t),y(t)\big)$, for $a\leq t \leq b$, and $\Gamma'(t) = \big(x'(t),y'(t)\big)$ exists for $a<t<b$, then the length of $\Gamma$ from $a$ to $b$ is:
\begin{equation}
    L = \int_{a}^{b}\sqrt{\bigg(\frac{dx}{dt}\bigg)^2+\bigg(\frac{dy}{dt}\bigg)^2}dt
\end{equation}
\end{definition}
In this problem we have $x(t) = \cos(t)$ and $y(t) = \sin(t)$, or $\Gamma(t) = \big(\cos(t),\sin(t)\big)$. This is the parametrization of the unit circle for $0\leq t \leq 2\pi$. Knowing this we see that the length of $\Gamma$ from $0$ to $\pi$ is simply half the circumference of the unit circle, which is $\pi$. We can also use the definition of length to obtain the following:
\begin{align*}
    x(t)&=\cos(t)\Rightarrow\frac{dx}{dt}=-\sin(t)\\
    y(t)&=\sin(t)\Rightarrow\frac{dy}{dt}=\cos(t)\\
    \bigg(\frac{dx}{dt}\bigg)^{2}+\bigg(\frac{dy}{dt}\bigg)^{2}&=\big(-\sin(t)\big)^{2}+\big(\cos(t)\big)^{2}\\
    &=\sin^{2}(t)+\cos^{2}(t)=1\\
    \Rightarrow\int_{0}^{\pi}\sqrt{\bigg(\frac{dx}{dt}\bigg)^{2}+\bigg(\frac{dy}{dt}\bigg)^{2}}dt&=\int_{0}^{\pi}dt=\pi
\end{align*}
The correct answer is $\pi$.
\end{proof}
\begin{problem}
What is the equation of the line tangent to $y=x+e^x$ at $x=0$?
\end{problem}
\begin{proof}[Solution]
\begin{definition}
The tangent line of a diffentiable function $y:\mathbb{R}\rightarrow \mathbb{R}$ at a point $x_0\in \mathbb{R}$ is $y_T(x) = y'(x_0)(x-x_0) + y(x_0)$ 
\end{definition}
The tangent line of $y$ at $x_0$ is a line that touches and lies tangent to $y$ at the point $x_0$. Using the definition we have:
\begin{align*}
    y(x)&=x+e^{x}\Rightarrow y(0)=1\\
    y'(x)&=1+e^{x}\Rightarrow y'(0)=2\\
    \Rightarrow y_T(x)&=2(x-0)+1\\
    &=2x+1
\end{align*}
The answer is $2x+1$
\end{proof}
\begin{problem}
If $V$ and $W$ are $2-$dimensional subspaces in $\mathbb{R}^4$, what are the possible dimensions of $V\cap W$.
\end{problem}
\begin{proof}[Solution]
\begin{definition*}
The dimension of a vector space is the cardinality of any basis of the space. 
\end{definition*}
\begin{remark*}
By the Dimension Theorem, all bases of a vector space have the same cardinality.
\end{remark*}
\begin{theorem*}
If $V$ is a vector space and $A,B\subset V$ are subspaces, then $A\cap B$ is a subspace and $\dim(A\cap B) \leq \min\{\dim(A),\dim(B)\}$
\end{theorem*}
From this theorem we have in our problem that $\dim\{V\cap W\} \leq 2$. We now must show that $V\cap W$ can have dimensions $0,1,$ or $2$. If $V = \{(x,y,0,0):x,y\in \mathbb{R}\}$ and $W = \{(0,0,z,w):z,w\in \mathbb{R}\}$, then $V\cap W = \{(0,0,0,0)\}$ which has dimension $0$. If $V = \{(x,y,0,0):x,y\in \mathbb{R}\}$ and $W = \{(0,y,z,0):y,z\in \mathbb{R}\}$, then $V\cap W = \{(0,y,0,0):y\in \mathbb{R}\}$ which has dimension $1$. Finally, if $V=W$ then $V\cap W = V$, which has dimension $2$. So, the only possibility are $0,1$, or $2$.
\end{proof}
\begin{problem}
Let $k$ be the number of solutions of $e^x+x-2 = 0$ in the interval $[0,1]$ and let $n$ be the number of solutions not in $[0,1]$ What are the values of $k$ and $n$?
\end{problem}
\begin{proof}[Solution]
\begin{theorem*}
If $f:\mathbb{R}\rightarrow \mathbb{R}$ is differentiable and $f'(x) >0$ for all $x$, then $f$ is strictly increasing.
\end{theorem*}
\begin{theorem*}
If $f:\mathbb{R}\rightarrow \mathbb{R}$ is continuous, $a<b$, and $f(a)<0<f(b)$, then there is a $c\in (a,b)$ such that $f(c) = 0$.
\end{theorem*}
Now, back to the problem. Let $y=e^x+x-2$. Then $y'(x) = e^x+1 > 1 > 0$, for all $x\in \mathbb{R}$. Therefore $y'(x)>0$ for all $x$, and thus $y$ is strictly increasing. Recall that $e\approx 2.71$. We have:
\begin{align*}
    y(0) &= e^0 + 0 - 2 = 1-2 = -1 <0 \\
    y(1) &= e^1+1-2 = e-1 \approx 1.71 > 0.
\end{align*}
So $y(0)<0<y(1)$, and therefore there is a $c\in (0,1)$ such that $y(c) = 0$. But $y$ is strictly increasing, and therefore for all $x>c$ we have $y(x)>y(c) = 0$, and for all $x<c$ we have $y(x)<y(c) = 0$. Thus $c$ is the only solution. The answer is $k=1$ and $n=0$.
\end{proof}
\begin{problem}
Suppose $b$ is a real number and $f(x) = 3x^2+bx+12$ has its vertex at $x=2$. What is $f(5)$?
\end{problem}
\begin{proof}[Solution]
If the vertex of $f$ is at $x=2$, then $f'(2) = 0$.
\begin{align*}
    f(x)&=2x^{2}+bx+12\\
    \Rightarrow f'(x)&=6x+b\\
    f'(2)&=0\\
    \Rightarrow 12 + b &= 0 \\
    b&=-12
\end{align*}
So $f(x) = 3x^2 -12 x + 12$. Thus $f(5) = 3\cdot 5^2 - 12\cdot 5 + 12 = 75 - 60 + 12 = 15 + 12 = 27$. The answer is $27$.
\end{proof}
\begin{problem}
Which of the following circles has the greatest number of points of intersections with the parabola $x^2 = y+4$:
\begin{enumerate}
\begin{multicols}{5}
    \item[A.)] $x^2+y^2 = 1$
    \item[B.)] $x^2+y^2 = 2$
    \item[C.)] $x^2+y^2 = 9$
    \item[D.)] $x^2+y^2 = 16$
    \item[E.)] $x^2+y^2=25$
\end{multicols}
\end{enumerate}
\end{problem}
\begin{proof}[Solution]
We wish to solve the system of equations:
\begin{align*}
    x^{2}+y^{2}&=r^{2}\\
    x^{2}-y&=4
\end{align*}
The solutions are of the form $(\pm \sqrt{r^2-y^2},y)$. For $(x,y)$ to be a solution the second equation requires that $y\geq -4$. Substituting $x$, we have $y^2+y+4 - r^2 = 0$. Using the quadratic formula we obtain $y = \frac{-1 \pm \sqrt{1 + 4(r^2-4)}}{2}$. For $y$ to be real we need $1+4(r^2-4) \geq 0$. Solving for $r^2$, we have $r^2 \geq \frac{15}{4}$. Thus there are no solutions for $r^2< \frac{15}{4}$. If $r^2 = \frac{15}{4}$, then there are two solutions: $(\pm \frac{\sqrt{3}}{2}, \frac{-1}{2})$. If $r^2 > \frac{15}{4}$, then we are also constrained by $y\geq -4$. This means 
\begin{equation*}
    \frac{-1\pm\sqrt{1+4(r^{2}-4)}}{2}\geq -4\Rightarrow -1\pm\sqrt{1+4(r^{2}-4)}\geq -8\Rightarrow 7\pm\sqrt{1+4(r^{2}-4)}\geq 0
\end{equation*}
 For $r^2 \geq \frac{15}{4}$, $7+\sqrt{1+4(r^2-4)}>0$ is always true. Now we consider $7-\sqrt{1+4(r^2-4)}:$
\begin{align*}
    7-\sqrt{1+4(r^{2}-4)}&\geq 0\Rightarrow 7\geq\sqrt{1+4(r^{2}-4)}\\
    \Rightarrow 49&\geq 1+4(r^{2}-4)\Rightarrow 48\geq 4(r^{2}-4)\\
    \Rightarrow 12&\geq r^{2}-4\Rightarrow 16\geq r^{2}
\end{align*}
Thus if $\frac{15}{4}<r^2<16$, we have two $y$ values, with two $x$ values corresponding to each for a total of $4$ solutions. If $r^2 = 16$, then we have three solutions: $(0,-4)$, $(\pm \sqrt{7},3)$. If $r^2>16$ then there is only one possible value for $y$, with two corresponding $x$ values. Thus, if $r^2>16$ there are only two solutions. The maximum number of solutions is attained when $\frac{15}{4} < r^2 <16$. The only value listed that has this property is $r^2 = 9$. The correct answer is C.
\end{proof}
\begin{problem}
Solve $\int_{-3}^{3}|x+1|dx$
\end{problem}
\begin{proof}[Solution]
\begin{definition*}
The absolute value of $x$ is $|x| = \begin{cases}x, & x\geq 0 \\ -x, & x<0\end{cases}$.
\end{definition*}
\begin{theorem*}
If $f$ is integrable on $(a,b)$, and if $c\in (a,b)$, then $\int_{a}^{b} f = \int_{a}^{c}f + \int_{c}^{b} f$
\end{theorem*}
Using this, we see that $|x+1| = \begin{cases} x+1, & x\geq -1 \\ -(x+1), & x < -1\end{cases}$. So from the theorem $\int_{-3}^{3}|x+1|dx = -\int_{-3}^{-1}(x+1)dx + \int_{-1}^{3}(x+1)dx$. Evaluating this, we get:
\begin{align*}
    -\bigg[\frac{x^2}{2}+x\bigg]_{-3}^{-1}+\bigg[\frac{x^2}{2}+x\bigg]_{-1}^{3}&=-\bigg[(\frac{1}{2}-1)-(\frac{9}{2}-3)\bigg]+\bigg[(\frac{9}{2}+3)-(\frac{1}{2}-1)\bigg]\\
    &=-\bigg[-\frac{1}{2}-\frac{3}{2}\bigg]+\bigg[\frac{15}{2}-(-\frac{1}{2})\bigg]\\
    &=2+8\\ 
    &=10
\end{align*}
The answer is $10$.
\end{proof}
\begin{problem}
What is the greatest area of a triangular region with one vertex at the center of a circle of radius $1$, and the other two on the circle?
\end{problem}
\begin{proof}[Solution]
\begin{theorem*}
Area is invariant under translation and rotation.
\end{theorem*}
We can use this theorem to place the center at $(0,0)$, one of the points at $(1,0)$, and then solve for the third point. The third point lies on the circle $x^2+y^2 = 1$, so we have $y = \sqrt{1-x^2}$. The area of the triangle is $\frac{1}{2}bh$. The base is the distance from $(0,0)$ to $(1,0)$, which is $1$, and the height is $\sqrt{1-x^2}$. So we have $A = \frac{1}{2}\sqrt{1-x^2}$. To find the extremum we take the derivative with respect to $x$ and set it to $0$.
\begin{align*}
    A&=\frac{1}{2}\sqrt{1-x^{2}}\\
    =\frac{dA}{dx}&=0\\
    \Rightarrow\frac{1}{2}\frac{-x}{\sqrt{1-x^{2}}}&=0\\
    \Rightarrow x&=0
\end{align*}
So the extremum occurs when $x=0$. The area is then $\frac{1}{2}\sqrt{1-0^2} = \frac{1}{2}$. The answer is $\frac{1}{2}$.
\end{proof}
\begin{problem}
Order the following equations from least to greatest:
\begin{align*}
    I&=1\\
    J&=\int_{0}^{1}\sqrt{1-x^{4}}dx\\
    K&=\int_{0}^{1}\sqrt{1+x^{4}}dx\\
    L&=\int_{0}^{1}\sqrt{1-x^{8}}dx
\end{align*}
\end{problem}
\begin{proof}[Solution]
\begin{theorem*}
If $f$ and $g$ are continuous on $(a,b)$ and $f>g$, then $\int_{a}^{b}f>\int_{a}^{b}g$
\end{theorem*}
All three integrals occur over the interval $(0,1)$. For $0 < x < 1$, we have the following:
\begin{equation*}
    0<x^{8}<x^{4}<x<1
\end{equation*}
Using this, we have have:
\begin{equation*}
    0<\sqrt{1-x^{4}}<\sqrt{1-x^{8}}<1<\sqrt{1+x^{4}}
\end{equation*}
Using the theorem, we have:
\begin{equation*}
    0<\int_{0}^{1}\sqrt{1-x^4}dx<\int_{0}^{1}\sqrt{1-x^8}dx<1<\int_{0}^{1}\sqrt{1+x^4}dx
\end{equation*}
So $J<L<1<K$
\end{proof}
\begin{problem}
Which is the best estimate of $\sqrt{1.5}(266)^{3/2}$?
\begin{enumerate}
    \begin{multicols}{5}
    \item[A.)] 1,0000
    \item[B.)] 2,700
    \item[C.)] 3,200
    \item[D.)] 4,100
    \item[E.)] 5,300
    \end{multicols}
\end{enumerate}
\end{problem}
\begin{proof}[Solution]
First note that $1.5 = \frac{3}{2}$ and $266 = 2\cdot 133$. So we have $\sqrt{1.5}(266)^{3/2} = \sqrt{\frac{3}{2}}\sqrt{266}\cdot 266 = \frac{\sqrt{3}}{\sqrt{2}}\cdot \sqrt{2}\cdot \sqrt{133}\cdot 266 = \sqrt{3}\cdot \sqrt{133}\cdot 266 = \sqrt{3\cdot 133}\cdot 266 = \sqrt{399}\cdot 266$. Now for the approximating. We note that $399 \approx 400$. So $\sqrt{399} \approx \sqrt{400} = 20$. So, with this approximation we have $20\cdot 266 = 5320$. The closest option is E.) $5,300$.
\end{proof}
\begin{problem}
If $A$ is a $2\times 2$ matrix who's columns and rows add up to a constant $k$, which of the following must be an eigenvector:
\begin{enumerate}
    \begin{multicols}{3}
    \item[I] $\begin{pmatrix}1 \\ 0 \end{pmatrix}$
    \item[II] $\begin{pmatrix} 0 \\ 1 \end{pmatrix}$
    \item[III] $\begin{pmatrix} 1 \\ 1 \end{pmatrix}$
    \end{multicols}
\end{enumerate}
\begin{enumerate}
    \begin{multicols}{5}
    \item[A.)] I only
    \item[B.)] II only
    \item[C.)] III only
    \item[D.)] I and II only
    \item[E.)] I, II, and III
    \end{multicols}
\end{enumerate}
\end{problem}
\begin{proof}[Solution]
We have $A = \begin{bmatrix} a & b \\ c & d\end{bmatrix}$ With the condition that:
\begin{align*}
    a+b&=k\\
    a+c&=k\\
    b+d&=k\\
    c+d&=k
\end{align*}
From this we can see that $a = d$ and $b=c = k=a$.
So, we have $A = \begin{bmatrix} a & k-a \\ k-a & a\end{bmatrix}$. We could solve directly for the eignvectors, or we could be more time efficient and check the possibilities we were given. $A\begin{bmatrix}1 \\ 0 \end{bmatrix} = \begin{bmatrix} a \\ k-a \end{bmatrix}$. $A\begin{bmatrix} 0 \\ 1 \end{bmatrix} = \begin{bmatrix} k-a \\ a \end{bmatrix}$. So I and II are only solutions if $a = k = 0$. $A\begin{bmatrix} 1 \\ 1 \end{bmatrix} = \begin{bmatrix} k \\ k\end{bmatrix} = k \begin{bmatrix} 1 \\ 1 \end{bmatrix}$. Thus we see that $\begin{bmatrix} 1 \\ 1 \end{bmatrix}$ is always an eigenvector. The answer is C.
\end{proof}
\begin{problem}
A total of $x$ fee of fencing is to form three sides of a level rectangular yard. What is the maximum possible area in terms of $x$?
\end{problem}
\begin{proof}[Solution]
Let $ax$ be the length of one side of the rectangular and $bx$ be the length of the adjacent side, so we have $ax+2bx = x$. From this we have $a+2b = 1$, or $a = 1-2b$. The area of the rectangle is the product of these lengths, so $A = ax\cdot bx = abx^2$. Substituting $a = 1-2b$, we have $A = (1-2b)bx^2$. To find the extremum we differentiate with respect to $b$ and set equal to $0$.
\begin{align}
    A&=(1-2b)bx^2\\
    \frac{dA}{db}&=0\\
    \Rightarrow (1-4b)x^{2}&=0\\
    \Rightarrow b&=\frac{1}{4}
\end{align}
So we have $b = \frac{1}{4}$, and $a = 1-2b = \frac{1}{2}$. The corresponding area is $abx^2 = \frac{1}{2}\cdot \frac{1}{4} x^2 = \frac{x^2}{8}$. The answer is $\frac{x^2}{8}$.
\end{proof}
\begin{problem}
What is the unit digit in the decimal expansion of $7^{25}$?
\end{problem}
\begin{proof}[Solution]
    Asking what is the unit digit of $7^{25}$ is equivalent to asking what is $7^{25} \mod 10$.
\begin{align*}
    7^{25} &= 7\cdot7^{24}\\
    &= 7\cdot(7^2)^{12}\\
    &= 7\cdot(49)^{12}\\
    &\equiv 7\cdot(9)^{12}\mod 10\\
    &= 7\cdot (9^2)^{6}\\
    &= 7 \cdot (81)^{6}\\
    &\equiv 7 \cdot (1)^6 \mod 10\\
    &= 7
\end{align*}
So, $7^{25} \cong 7 \mod 10$. The answer is $7$.
\end{proof}
\begin{problem}
Let $f:[-2,3]\rightarrow \mathbb{R}$ be continuous. Which of the following is NOT necessarily true?
\begin{enumerate}
    \item[A.)] $f$ is bounded.
    \item[B.)] $\int_{-2}^{3}f$ exists.
    \item[C.)] For each $c$ between $f(-2)$ and $f(3)$ there is an $x\in (-2,3)$ such that $f(x) = c$.
    \item[D.)] There is an $M$ in $f([-2,3])$ such that $\int_{-2}^{3}f = 5M$.
    \item $\underset{h\rightarrow 0}\lim \frac{f(h)-f(0)}{h}$ exists.
\end{enumerate}
\end{problem}
\begin{proof}[Solution]
\begin{theorem*}
A subset $\mathcal{C}\subset \mathbb{R}$ is compact if and only if it is closed and bounded.
\end{theorem*}
\begin{theorem*}
If $\mathcal{C}$ is compact and $f:\mathcal{C}\rightarrow \mathbb{R}$ is continuous, then f is bounded.
\end{theorem*}
\begin{theorem*}
If is continuous and bounded, then f is integrable.
\end{theorem*}
\begin{theorem*}
If $f:[a,b] \rightarrow \mathbb{R}$ is continuous, then for all $c$ between $f(a)$ and $f(b)$ there is an $x\in (a,b)$ such that $f(x) = c$.
\end{theorem*}
\begin{theorem*}
If $f:[a,b]\rightarrow \mathbb{R}$ is continuous then there is an $M\in f\big([a,b]\big)$ such that $\int_{a}^{b}f = M(b-a)$
\end{theorem*}
Since $[-2,3]$ is closed and bounded, it is compact, and if $f$ is continuous then it must be bounded so A is true. Since $f$ is continuous and bounded, it is integrable so B is true. C is simply the Intermediate Value Theorem and D is simply the Mean Value Theorem. By process of elimination, E is not necessarily true. Let us find a counterexample. Let $f=|x|$. Then $\underset{h\rightarrow 0^+}\lim \frac{|h|-|0|}{h} = 1$, but $\underset{h\rightarrow 0^{-}}\lim \frac{|h|-|0|}{h} = -1$. Thus the limit does not exist for this function. The answer is E.
\end{proof}
\begin{problem}
What is the volume of the solid formed by revolving about the $x-$axis the region in the first quadrant of the $xy-$plane bounded by the coordinate axes and the graph of $x = \frac{1}{\sqrt{1+x^2}}$?
\end{problem}
\begin{proof}[Solution]
Working by definition, we could simply use the formula $V = \int_{0}^{\infty}\pi r^2(x)dx$, where $r(x) = \frac{1}{\sqrt{1+x^2}}$. But memorizing formulas is usually not the best idea. Instead let's try to do this intuitively. Suppose we had a small cylinder centered at $x$. The volume of a cylinder is $\pi r^2 h$, so our tiny slab will be $dV = \pi r^2(x) dx$. Integrating over all of these infinitesimal discs gives us $V = \pi \int_{0}^{\infty}r^2(x)dx$. For our problem, $r(x) = \frac{1}{\sqrt{1+x^2}}$, so we have $V = \pi \int_{0}^{\infty} \frac{1}{1+x^2}dx = \pi \tan^{-1}(x)\big|_{0}^{\infty} = \pi \cdot \frac{\pi}{2} = \frac{\pi^2}{2}$. The answer is $\frac{\pi^2}{2}$.
\end{proof}
\begin{problem}
How many real zeros does the polynomial $3x^5+8x-7$ have?
\end{problem}
\begin{proof}[Solution]
\begin{theorem*}
If $f = \sum_{k=0}^{n} a_k x^k$, then the number of positive zeroes of $f(x)$ is equal to the number of sign changes of $f$ or is an even number less than that.
\end{theorem*}
\begin{theorem*}
If $f=\sum_{k=0}^{n} a_k x^k$, then the number of negative zeroes is equal to the number of sign changes of $f(-x)$ or equal to an even number less than that.
\end{theorem*}
We have $f(x) = 2x^5+8x - 7$. So, the set of coefficients is $\{-7,8,0,0,0,2\}$. There is only one sign change, from $-7$ to $8$, so there is one positive zero. $f(-x) = -2x^5 - 8x - 7$, which has the set of coefficients $\{-7,-8,0,0,0,0,-2\}$. This has no sign changes, so there are no negative zeroes. Also, $f(0) = -7$, so $0$ is not a root of $f$. Thus $f$ has $1$ zero. The answer is $1$.
\end{proof}
\begin{problem}
If $T$ is a linear transformation of $V = \mathbb{R}^{2\times 3}$ $\textbf{onto}$ $W = \mathbb{R}^4$, what is $\dim(\{v\in V:T(v) = 0\})$?
\end{problem}
\begin{proof}[Solution]
\begin{theorem*}
If $V,W$ are vector spaces and $T:V\rightarrow W$ is a linear transformation, then $\dim(\Im(T)) + \dim(\ker(T)) = \dim(V)$.
\end{theorem*}
This set is called the kernel of $T$, denoted $\ker(T)$. Now, as $T$ is onto, the image of $T$ is $V$, which has dimension $4$. Also, $\dim(V) = 6$. So we have $\dim(\ker(T)) + 4 = 6$, so $\dim(\ker(T)) = 2$. The answer is $2$.
\end{proof}
\begin{problem}
If $f,g:\mathbb{R}\rightarrow \mathbb{R}$ are twice differentiable, and if for all $x>0$, $f'(x)>g'(x)$, then which of the following must be true for all $x>0$?
\begin{enumerate}
    \begin{multicols}{3}
    \item[A.)] $f(x)>g(x)$
    \item[B.)] $f''(x)>g''(x)$
    \item[C.)] $f(x)-f(0)>g(x)-g(0)$
    \item[D.)] $f'(x)-f'(0)>g'(x)-g'(0)$
    \item[E.)] $f''(x) - f''(0)>g''(x)-g''(0)$
    \end{multicols}
\end{enumerate}
\end{problem}
\begin{proof}[Solution]
\begin{theorem*}
If $f:(a,b)\rightarrow \mathbb{R}$ is continuous and $x\in (a,b)$, then $f(x)-f(a) = \int_{a}^{x}f'$ 
\end{theorem*}
\begin{theorem*}
If $f,g$ are continuous and bounded on $(a,b)$, and if $f>g$, then $\int_{a}^{b}f > \int_{a}^{b}g$
\end{theorem*}
\begin{theorem*}
If $f,g$ are continuous and bounded on $(a,b)$, $f'>g'$, and if $x\in (a,b)$, then $f(x) - f(0) > g(x) - g(0)$.
\end{theorem*}
From this we have that C is true. There are counterexamples for the others:
\begin{enumerate}
    \item[A.)] Let $f = x$ and $g = 2$. Then $f'(x) = 1$ and $g'(x) = 0$, but $f(x)<g(x)$ for $x\in [0,2)$.
    \item[B.)] Let $f = x$, $g(x) = \int_{0}^{x} \frac{1}{2}\sin(t^2)dt$. Then $f'(x) = 1$, $g'(x) = \frac{1}{2}\sin(x^2)$, $f''(x) = 0$, $g''(x) = x\cos(x^2)$.
    \item[D.)] Let $f(x) = x$, $g(x) = \frac{1}{2}x^2$. Then $f'(x) = 1$, $g'(x) = x$, $f(x)-f(0) = 0$, $g(x)-g(0) = x$.
    \item[E.)] Use the same example from B.
\end{enumerate}
\end{proof}
\begin{problem}
Where is the function $f(x) = \begin{cases} \frac{x}{2}, & x\in \mathbb{Q} \\ \frac{x}{3}, & x \notin \mathbb{Q}\end{cases}$ disccontinuous?
\end{problem}
\begin{proof}[Solution]
Let $x\in \mathbb{Q}\setminus \{0\}$ and let $x_n$ be any sequence of irrationals such that $x_n \rightarrow x$. Then $f(x_n) = \frac{x_n}{3} \rightarrow \frac{x}{3}$, but $f(x) = \frac{x}{2}$. Therefore $f$ is not continuous in $\mathbb{Q}\setminus \{0\}$. If $x\in \mathbb{R}\setminus \mathbb{Q}$, let $x_n$ be any sequence of rationals such that $x_n \rightarrow x$. Then $f(x_n) = \frac{x_n}{2} \rightarrow \frac{x}{2}$, but $f(x) = \frac{x}{3}$. So $f$ is discontinuous on $\mathbb{R}\setminus \mathbb{Q}$. If $x= 0$, let $\varepsilon>0$ and choose $\delta = \varepsilon$. Then for $|x|<\delta$ we have $|f(0) - f(x)| = |f(x)| < \frac{|x|}{2} < \frac{\varepsilon}{2}<\varepsilon$. Thus, $f$ is continuous at $0$. $f$ is discontinuous at all non-zero real numbers.
\end{proof}
\begin{problem}
Let $P_1 = \{2,3,5,7,11,\hdots\}$ and $P_n = \{2n,3n,5n,7n,11n,\hdots\}$. Which of the following is non-empty?
\begin{enumerate}
\begin{multicols}{5}
    \item[A.)] $P_1\cap P_{23}$
    \item[B.)] $P_7\cap P_{21}$
    \item[C.)] $P_{12}\cap P_{20}$
    \item[D.)] $P_{20}\cap P_{24}$
    \item[E.)] $P_{5}\cap P_{25}$
\end{multicols}
\end{enumerate}
\end{problem}
\begin{proof}[Solution]
\
\begin{enumerate}
    \item[A.)] If $x\in P_{1}\cap P_{23}$, $x = p = 23\cdot q$ for primes $p$ and $q$. But then $p = 23\cdot q$, a contradiction as $p$ is prime. So A is empty.
    \item[B.)] If $x\in P_{7}\cap P_{21}$, then $x= 7p = 21q$. So $7p = 7\cdot 3q$, and thus $p = 3q$, again a contradiction.
    \item[C.)] If $x\in P_{12}\cap P_{20}$, then $x = 12p = 20q$, so we have $4\cdot 3 p = 4\cdot 5 q$, and thus $3p = 5q$. So $p=5$ and $q = 3$. $60 \in P_{12}\cap P_{20}$.
    \item[D.)] If $x\in P_{20}\cap P_{24}$, then $20p = 24q$. So $4\cdot 5p = 4\cdot 6q$, and thus $5p = 6q$. The left-side is even, and thus the only possibility is $p=5$. But then $10 = 6p$, a contradiction as $p$ is prime.
    \item[E.)] If $x\in P_{5}\cap P_{25}$, then $x = 5p = 25q$, so $5p = 5\cdot 5q$, and therefore $p=5q$, a contradiction as $p$ is prime.
\end{enumerate}
C is the only non-empty set. The answer is C.
\end{proof}
\begin{problem}
Let $C(\mathbb{R})$ be the set of continuous function $f:\mathbb{R}\rightarrow \mathbb{R}$. Then $C(\mathbb{R})$ is a vector space with addition defined as $(f+g)(x) = f(x)+g(x)$ for all $f,g\in C(\mathbb{R})$, and scalar multiplication defined $(rf)(x) = r\cdot f(x)$, for all $x,r\in \mathbb{R}$. Which of the following are subspaces of $C(\mathbb{R})$?
\begin{enumerate}
    \item[I] $\{f:f''\textrm{ exists and }f'' - 2f'+3f = 0\}$
    \item[II] $\{g:g''\textrm{ exists and }g'' = 3g' \}$
    \item[III] $\{h:h''\textrm{ exists and }h'' = h+1\}$
\end{enumerate}
\begin{enumerate}
    \begin{multicols}{5}
    \item[A.)] I only
    \item[B.)] I and II only
    \item[C.)] I and III only
    \item[D.)] II and III only
    \item[E.)] I, II, and III
    \end{multicols}
\end{enumerate}
\end{problem}
\begin{proof}[Solution]
We have to check for closure under vector addition and scalar multiplication.
\begin{enumerate}
    \item[I] For addition, $(f_1+f_2)'' - 2(f_1+f_2)' +3(f_1+f_2) = (f''_1 - 2f'_1 +3f_1) + (f''_2 - 2f'_2 +3f_2) = 0+0 = 0$. For scalar multiplication $(rf)'' - 2(rf)' +3(rf) = r(f'' - 2f' + 3f) = r\cdot 0 = 0$. So I is closed under vector addition and scalar multiplication, and thus I is a subspace.
    \item For addition, $(g_1+g_2)'' = g''_1 +g''_2 = 3g_1 + 3g_2 = 3(g_1+g_2)$. For scalar multiplication, $(rg)'' =rg'' = r\cdot 3g = 3(rg)$. Thus II is closed under vector addition and scalar multiplication. II is a subspace.
    \item For addition, $(h_1+h_2)'' = h''_1 + h''_2 = h_1+1 + h_2+1 = (h_1+h_2) + 2 \ne (h_1+h_2)+1$. Therefore III is not closed under vector addition. III is not a subspace.
\end{enumerate}
We see that I and II are subspaces, but III is not. The answer is B.
\end{proof}
\begin{problem}
For what value of $b$ does the line $y_1=10x$ lie tangent to the curve $y_2=e^{bx}$?
\end{problem}
\begin{proof}[Solution]
If the two curves lie tangent at $x$, then $y'_1(x) = y'_2(x)$. Thus we have $10 = be^{bx}$. But also $y_1(x) = y_2(x)$. So $e^{bx} = 10x$. So we have $10 = be^{bx} = b\cdot (10x) = 10bx$. Therefore we have $bx = 10$, and $x = \frac{1}{b}$. But $10 = b e^{bx} = b e^{b\cdot \frac{1}{b}} = be$. Therefore $b = \frac{10}{e}$. The answer is $b = \frac{10}{e}$.
\end{proof}
\begin{problem}
Let $h(x)=\int_{0}{x^{2}}e^{x+t}dt$. What is $h'(1)$?
\end{problem}
\begin{proof}[Solution]
\begin{theorem*}
If $\alpha$ and $\beta$ are differentiable, $f$ is continuous, and if $F(x) = \int_{\alpha(x)}^{\beta(x)}f$, then $F'(x) = f\big(\beta(x)\big)\beta'(x) - f\big(\alpha(x)\big)\alpha'(x)$
\end{theorem*}
We have $h(x) = \int_{0}^{x^2}e^{x+t}dt = e^x \int_{0}^{x^2}e^t dt$. Using the product rule and the above theorem we have:
\begin{align*}
    h'(x) &= e^x \int_{0}^{x^2} e^t dt + e^{x} \big(e^{x^2}\frac{d}{dx}(x^2)\big) \\
    &= e^x\big(e^{x^2}-1\big) + 2xe^xe^{x^2} \\
    \Rightarrow h'(1) &= e(e-1) + 2e^2 = \\
    &= 3e^2 - e
\end{align*}
The answer is $3e^{2}-e$
\end{proof}
\begin{problem}
Let $\{a_n\}_{n=1}^{\infty}$ be recursively defined by $a_1 = 1$, and for all $n\in \mathbb{N}$, $a_{n+1} = \frac{n+2}{n}a_n$. What is $a_{30}$?
\end{problem}
\begin{proof}[Solution]
Let us try to find a closed form for this sequence. For $n>2$ we hav:
\begin{align*}
    a_{n+1} &= \frac{n+2}{n}a_n \\
    &= \frac{n+2}{n}\frac{n+1}{n-1}a_{n-1}\\
    &= \frac{n+2}{n}\frac{n+1}{n-1}\frac{n}{n-2}a_{n-2}
\end{align*}
The pattern seems to be $a_{n+1} = \frac{(n+2)!}{(n+1-k)!}\frac{(n-1-k)!}{n!}a_{n-k}$. When $k=n-1$, we have $a_{n} = \frac{(n+1)!}{2!(n-1)!}a_1$. We can confirm this by induction. The base case of $n=1$ is $\frac{2!}{2!0!}a_1 = a_1$. Suppose it is true for $n\in \mathbb{N}$. Then $a_{n+1} = \frac{n+2}{n} a_n = \frac{n+2}{n} \frac{(n+1)!}{2!(n-1)!}a_1 = \frac{(n+2)!}{2!n!}a_1$. This proves the induction step. Therefore, $a_n = \frac{(n+1)!}{2!n!}a_1$. So $a_{30} = \frac{31!}{2!29!} = \frac{31\cdot 30}{2} = 31\cdot 15$. The answer is $(15)(31)$.
\end{proof}
\section{GR1268}
\end{document}