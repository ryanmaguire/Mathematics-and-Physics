\documentclass[crop=false,class=book,oneside]{standalone}
%---------------PREAMBLE------------%
%---------------------------Packages----------------------------%
\usepackage{geometry}
\geometry{b5paper, margin=1.0in}
\usepackage[T1]{fontenc}
\usepackage{graphicx, float}            % Graphics/Images.
\usepackage{natbib}                     % For bibliographies.
\bibliographystyle{agsm}                % Bibliography style.
\usepackage[french, english]{babel}     % Language typesetting.
\usepackage[dvipsnames]{xcolor}         % Color names.
\usepackage{listings}                   % Verbatim-Like Tools.
\usepackage{mathtools, esint, mathrsfs} % amsmath and integrals.
\usepackage{amsthm, amsfonts, amssymb}  % Fonts and theorems.
\usepackage{tcolorbox}                  % Frames around theorems.
\usepackage{upgreek}                    % Non-Italic Greek.
\usepackage{fmtcount, etoolbox}         % For the \book{} command.
\usepackage[newparttoc]{titlesec}       % Formatting chapter, etc.
\usepackage{titletoc}                   % Allows \book in toc.
\usepackage[nottoc]{tocbibind}          % Bibliography in toc.
\usepackage[titles]{tocloft}            % ToC formatting.
\usepackage{pgfplots, tikz}             % Drawing/graphing tools.
\usepackage{imakeidx}                   % Used for index.
\usetikzlibrary{
    calc,                   % Calculating right angles and more.
    angles,                 % Drawing angles within triangles.
    arrows.meta,            % Latex and Stealth arrows.
    quotes,                 % Adding labels to angles.
    positioning,            % Relative positioning of nodes.
    decorations.markings,   % Adding arrows in the middle of a line.
    patterns,
    arrows
}                                       % Libraries for tikz.
\pgfplotsset{compat=1.9}                % Version of pgfplots.
\usepackage[font=scriptsize,
            labelformat=simple,
            labelsep=colon]{subcaption} % Subfigure captions.
\usepackage[font={scriptsize},
            hypcap=true,
            labelsep=colon]{caption}    % Figure captions.
\usepackage[pdftex,
            pdfauthor={Ryan Maguire},
            pdftitle={Mathematics and Physics},
            pdfsubject={Mathematics, Physics, Science},
            pdfkeywords={Mathematics, Physics, Computer Science, Biology},
            pdfproducer={LaTeX},
            pdfcreator={pdflatex}]{hyperref}
\hypersetup{
    colorlinks=true,
    linkcolor=blue,
    filecolor=magenta,
    urlcolor=Cerulean,
    citecolor=SkyBlue
}                           % Colors for hyperref.
\usepackage[toc,acronym,nogroupskip,nopostdot]{glossaries}
\usepackage{glossary-mcols}
%------------------------Theorem Styles-------------------------%
\theoremstyle{plain}
\newtheorem{theorem}{Theorem}[section]

% Define theorem style for default spacing and normal font.
\newtheoremstyle{normal}
    {\topsep}               % Amount of space above the theorem.
    {\topsep}               % Amount of space below the theorem.
    {}                      % Font used for body of theorem.
    {}                      % Measure of space to indent.
    {\bfseries}             % Font of the header of the theorem.
    {}                      % Punctuation between head and body.
    {.5em}                  % Space after theorem head.
    {}

% Italic header environment.
\newtheoremstyle{thmit}{\topsep}{\topsep}{}{}{\itshape}{}{0.5em}{}

% Define environments with italic headers.
\theoremstyle{thmit}
\newtheorem*{solution}{Solution}

% Define default environments.
\theoremstyle{normal}
\newtheorem{example}{Example}[section]
\newtheorem{definition}{Definition}[section]
\newtheorem{problem}{Problem}[section]

% Define framed environment.
\tcbuselibrary{most}
\newtcbtheorem[use counter*=theorem]{ftheorem}{Theorem}{%
    before=\par\vspace{2ex},
    boxsep=0.5\topsep,
    after=\par\vspace{2ex},
    colback=green!5,
    colframe=green!35!black,
    fonttitle=\bfseries\upshape%
}{thm}

\newtcbtheorem[auto counter, number within=section]{faxiom}{Axiom}{%
    before=\par\vspace{2ex},
    boxsep=0.5\topsep,
    after=\par\vspace{2ex},
    colback=Apricot!5,
    colframe=Apricot!35!black,
    fonttitle=\bfseries\upshape%
}{ax}

\newtcbtheorem[use counter*=definition]{fdefinition}{Definition}{%
    before=\par\vspace{2ex},
    boxsep=0.5\topsep,
    after=\par\vspace{2ex},
    colback=blue!5!white,
    colframe=blue!75!black,
    fonttitle=\bfseries\upshape%
}{def}

\newtcbtheorem[use counter*=example]{fexample}{Example}{%
    before=\par\vspace{2ex},
    boxsep=0.5\topsep,
    after=\par\vspace{2ex},
    colback=red!5!white,
    colframe=red!75!black,
    fonttitle=\bfseries\upshape%
}{ex}

\newtcbtheorem[auto counter, number within=section]{fnotation}{Notation}{%
    before=\par\vspace{2ex},
    boxsep=0.5\topsep,
    after=\par\vspace{2ex},
    colback=SeaGreen!5!white,
    colframe=SeaGreen!75!black,
    fonttitle=\bfseries\upshape%
}{not}

\newtcbtheorem[use counter*=remark]{fremark}{Remark}{%
    fonttitle=\bfseries\upshape,
    colback=Goldenrod!5!white,
    colframe=Goldenrod!75!black}{ex}

\newenvironment{bproof}{\textit{Proof.}}{\hfill$\square$}
\tcolorboxenvironment{bproof}{%
    blanker,
    breakable,
    left=3mm,
    before skip=5pt,
    after skip=10pt,
    borderline west={0.6mm}{0pt}{green!80!black}
}

\AtEndEnvironment{lexample}{$\hfill\textcolor{red}{\blacksquare}$}
\newtcbtheorem[use counter*=example]{lexample}{Example}{%
    empty,
    title={Example~\theexample},
    boxed title style={%
        empty,
        size=minimal,
        toprule=2pt,
        top=0.5\topsep,
    },
    coltitle=red,
    fonttitle=\bfseries,
    parbox=false,
    boxsep=0pt,
    before=\par\vspace{2ex},
    left=0pt,
    right=0pt,
    top=3ex,
    bottom=1ex,
    before=\par\vspace{2ex},
    after=\par\vspace{2ex},
    breakable,
    pad at break*=0mm,
    vfill before first,
    overlay unbroken={%
        \draw[red, line width=2pt]
            ([yshift=-1.2ex]title.south-|frame.west) to
            ([yshift=-1.2ex]title.south-|frame.east);
        },
    overlay first={%
        \draw[red, line width=2pt]
            ([yshift=-1.2ex]title.south-|frame.west) to
            ([yshift=-1.2ex]title.south-|frame.east);
    },
}{ex}

\AtEndEnvironment{ldefinition}{$\hfill\textcolor{Blue}{\blacksquare}$}
\newtcbtheorem[use counter*=definition]{ldefinition}{Definition}{%
    empty,
    title={Definition~\thedefinition:~{#1}},
    boxed title style={%
        empty,
        size=minimal,
        toprule=2pt,
        top=0.5\topsep,
    },
    coltitle=Blue,
    fonttitle=\bfseries,
    parbox=false,
    boxsep=0pt,
    before=\par\vspace{2ex},
    left=0pt,
    right=0pt,
    top=3ex,
    bottom=0pt,
    before=\par\vspace{2ex},
    after=\par\vspace{1ex},
    breakable,
    pad at break*=0mm,
    vfill before first,
    overlay unbroken={%
        \draw[Blue, line width=2pt]
            ([yshift=-1.2ex]title.south-|frame.west) to
            ([yshift=-1.2ex]title.south-|frame.east);
        },
    overlay first={%
        \draw[Blue, line width=2pt]
            ([yshift=-1.2ex]title.south-|frame.west) to
            ([yshift=-1.2ex]title.south-|frame.east);
    },
}{def}

\AtEndEnvironment{ltheorem}{$\hfill\textcolor{Green}{\blacksquare}$}
\newtcbtheorem[use counter*=theorem]{ltheorem}{Theorem}{%
    empty,
    title={Theorem~\thetheorem:~{#1}},
    boxed title style={%
        empty,
        size=minimal,
        toprule=2pt,
        top=0.5\topsep,
    },
    coltitle=Green,
    fonttitle=\bfseries,
    parbox=false,
    boxsep=0pt,
    before=\par\vspace{2ex},
    left=0pt,
    right=0pt,
    top=3ex,
    bottom=-1.5ex,
    breakable,
    pad at break*=0mm,
    vfill before first,
    overlay unbroken={%
        \draw[Green, line width=2pt]
            ([yshift=-1.2ex]title.south-|frame.west) to
            ([yshift=-1.2ex]title.south-|frame.east);},
    overlay first={%
        \draw[Green, line width=2pt]
            ([yshift=-1.2ex]title.south-|frame.west) to
            ([yshift=-1.2ex]title.south-|frame.east);
    }
}{thm}

%--------------------Declared Math Operators--------------------%
\DeclareMathOperator{\adjoint}{adj}         % Adjoint.
\DeclareMathOperator{\Card}{Card}           % Cardinality.
\DeclareMathOperator{\curl}{curl}           % Curl.
\DeclareMathOperator{\diam}{diam}           % Diameter.
\DeclareMathOperator{\dist}{dist}           % Distance.
\DeclareMathOperator{\Div}{div}             % Divergence.
\DeclareMathOperator{\Erf}{Erf}             % Error Function.
\DeclareMathOperator{\Erfc}{Erfc}           % Complementary Error Function.
\DeclareMathOperator{\Ext}{Ext}             % Exterior.
\DeclareMathOperator{\GCD}{GCD}             % Greatest common denominator.
\DeclareMathOperator{\grad}{grad}           % Gradient
\DeclareMathOperator{\Ima}{Im}              % Image.
\DeclareMathOperator{\Int}{Int}             % Interior.
\DeclareMathOperator{\LC}{LC}               % Leading coefficient.
\DeclareMathOperator{\LCM}{LCM}             % Least common multiple.
\DeclareMathOperator{\LM}{LM}               % Leading monomial.
\DeclareMathOperator{\LT}{LT}               % Leading term.
\DeclareMathOperator{\Mod}{mod}             % Modulus.
\DeclareMathOperator{\Mon}{Mon}             % Monomial.
\DeclareMathOperator{\multideg}{mutlideg}   % Multi-Degree (Graphs).
\DeclareMathOperator{\nul}{nul}             % Null space of operator.
\DeclareMathOperator{\Ord}{Ord}             % Ordinal of ordered set.
\DeclareMathOperator{\Prin}{Prin}           % Principal value.
\DeclareMathOperator{\proj}{proj}           % Projection.
\DeclareMathOperator{\Refl}{Refl}           % Reflection operator.
\DeclareMathOperator{\rk}{rk}               % Rank of operator.
\DeclareMathOperator{\sgn}{sgn}             % Sign of a number.
\DeclareMathOperator{\sinc}{sinc}           % Sinc function.
\DeclareMathOperator{\Span}{Span}           % Span of a set.
\DeclareMathOperator{\Spec}{Spec}           % Spectrum.
\DeclareMathOperator{\supp}{supp}           % Support
\DeclareMathOperator{\Tr}{Tr}               % Trace of matrix.
%--------------------Declared Math Symbols--------------------%
\DeclareMathSymbol{\minus}{\mathbin}{AMSa}{"39} % Unary minus sign.
%------------------------New Commands---------------------------%
\DeclarePairedDelimiter\norm{\lVert}{\rVert}
\DeclarePairedDelimiter\ceil{\lceil}{\rceil}
\DeclarePairedDelimiter\floor{\lfloor}{\rfloor}
\newcommand*\diff{\mathop{}\!\mathrm{d}}
\newcommand*\Diff[1]{\mathop{}\!\mathrm{d^#1}}
\renewcommand*{\glstextformat}[1]{\textcolor{RoyalBlue}{#1}}
\renewcommand{\glsnamefont}[1]{\textbf{#1}}
\renewcommand\labelitemii{$\circ$}
\renewcommand\thesubfigure{%
    \arabic{chapter}.\arabic{figure}.\arabic{subfigure}}
\addto\captionsenglish{\renewcommand{\figurename}{Fig.}}
\numberwithin{equation}{section}

\renewcommand{\vector}[1]{\boldsymbol{\mathrm{#1}}}

\newcommand{\uvector}[1]{\boldsymbol{\hat{\mathrm{#1}}}}
\newcommand{\topspace}[2][]{(#2,\tau_{#1})}
\newcommand{\measurespace}[2][]{(#2,\varSigma_{#1},\mu_{#1})}
\newcommand{\measurablespace}[2][]{(#2,\varSigma_{#1})}
\newcommand{\manifold}[2][]{(#2,\tau_{#1},\mathcal{A}_{#1})}
\newcommand{\tanspace}[2]{T_{#1}{#2}}
\newcommand{\cotanspace}[2]{T_{#1}^{*}{#2}}
\newcommand{\Ckspace}[3][\mathbb{R}]{C^{#2}(#3,#1)}
\newcommand{\funcspace}[2][\mathbb{R}]{\mathcal{F}(#2,#1)}
\newcommand{\smoothvecf}[1]{\mathfrak{X}(#1)}
\newcommand{\smoothonef}[1]{\mathfrak{X}^{*}(#1)}
\newcommand{\bracket}[2]{[#1,#2]}

%------------------------Book Command---------------------------%
\makeatletter
\renewcommand\@pnumwidth{1cm}
\newcounter{book}
\renewcommand\thebook{\@Roman\c@book}
\newcommand\book{%
    \if@openright
        \cleardoublepage
    \else
        \clearpage
    \fi
    \thispagestyle{plain}%
    \if@twocolumn
        \onecolumn
        \@tempswatrue
    \else
        \@tempswafalse
    \fi
    \null\vfil
    \secdef\@book\@sbook
}
\def\@book[#1]#2{%
    \refstepcounter{book}
    \addcontentsline{toc}{book}{\bookname\ \thebook:\hspace{1em}#1}
    \markboth{}{}
    {\centering
     \interlinepenalty\@M
     \normalfont
     \huge\bfseries\bookname\nobreakspace\thebook
     \par
     \vskip 20\p@
     \Huge\bfseries#2\par}%
    \@endbook}
\def\@sbook#1{%
    {\centering
     \interlinepenalty \@M
     \normalfont
     \Huge\bfseries#1\par}%
    \@endbook}
\def\@endbook{
    \vfil\newpage
        \if@twoside
            \if@openright
                \null
                \thispagestyle{empty}%
                \newpage
            \fi
        \fi
        \if@tempswa
            \twocolumn
        \fi
}
\newcommand*\l@book[2]{%
    \ifnum\c@tocdepth >-3\relax
        \addpenalty{-\@highpenalty}%
        \addvspace{2.25em\@plus\p@}%
        \setlength\@tempdima{3em}%
        \begingroup
            \parindent\z@\rightskip\@pnumwidth
            \parfillskip -\@pnumwidth
            {
                \leavevmode
                \Large\bfseries#1\hfill\hb@xt@\@pnumwidth{\hss#2}
            }
            \par
            \nobreak
            \global\@nobreaktrue
            \everypar{\global\@nobreakfalse\everypar{}}%
        \endgroup
    \fi}
\newcommand\bookname{Book}
\renewcommand{\thebook}{\texorpdfstring{\Numberstring{book}}{book}}
\providecommand*{\toclevel@book}{-2}
\makeatother
\titleformat{\part}[display]
    {\Large\bfseries}
    {\partname\nobreakspace\thepart}
    {0mm}
    {\Huge\bfseries}
\titlecontents{part}[0pt]
    {\large\bfseries}
    {\partname\ \thecontentslabel: \quad}
    {}
    {\hfill\contentspage}
\titlecontents{chapter}[0pt]
    {\bfseries}
    {\chaptername\ \thecontentslabel:\quad}
    {}
    {\hfill\contentspage}
\newglossarystyle{longpara}{%
    \setglossarystyle{long}%
    \renewenvironment{theglossary}{%
        \begin{longtable}[l]{{p{0.25\hsize}p{0.65\hsize}}}
    }{\end{longtable}}%
    \renewcommand{\glossentry}[2]{%
        \glstarget{##1}{\glossentryname{##1}}%
        &\glossentrydesc{##1}{~##2.}
        \tabularnewline%
        \tabularnewline
    }%
}
\newglossary[not-glg]{notation}{not-gls}{not-glo}{Notation}
\newcommand*{\newnotation}[4][]{%
    \newglossaryentry{#2}{type=notation, name={\textbf{#3}, },
                          text={#4}, description={#4},#1}%
}
%--------------------------LENGTHS------------------------------%
% Spacings for the Table of Contents.
\addtolength{\cftsecnumwidth}{1ex}
\addtolength{\cftsubsecindent}{1ex}
\addtolength{\cftsubsecnumwidth}{1ex}
\addtolength{\cftfignumwidth}{1ex}
\addtolength{\cfttabnumwidth}{1ex}

% Indent and paragraph spacing.
\setlength{\parindent}{0em}
\setlength{\parskip}{0em}
%---------------GLOSSARY------------%
\makeglossaries
\loadglsentries{../../glossary}
\loadglsentries{../../acronym}
%--------------Title Page-----------%
\title{Lecons sur la Theorie Mathematique de la Lutte pour la Vie}
\author{By: Vito Volterra\\ Translation by: Ryan Maguire \\ Supervised by: Enrique Gonzalez-Velasco}
\date{\vspace{-5ex}}
\begin{document}
\maketitle
\chapter*{Preface (Pr\'{e}face)}
\addcontentsline{toc}{chapter}{Preface}
\begin{paracol}{2}
After speaking with Mr. D'Ancona, who asked me if it was possible to find some mathematical way to study the variations  in the compositions of biological associations, I began my research into the subject at the end of 1925. I continued during the first months of the following year, and simultaneously drafted my memoire on biological fluctuations, where I have detailed my methods and given the laws of biology that arise. It was published by the Academy of Lincei in the same year.\par\hfill\par
\hfill\par
\switchcolumn
\foreignlanguage{french}{
\begin{sloppypar}
A la suite de conversations avec M. D'Ancona, qui me demandait s'il \'{e}tait possible de trouver quelque voie math\'{e}matique pour \'{e}tudier les variations dans la composition des associations biologiques, j'ai commenc\'{e} mes recherches sur ce sujet \`{a} la fin de 1925. Je les ai poursuivies durant les premiers mois de l'ann\'{e}e suivante, et en m\^{e}me temps j'ai r\'{e}dige mon m\'{e}moire sur les fluctuations biologiques, o\`{u} j'ai expos\'{e} mes m\'{e}thods et donn\'{e} les lois biologiques qui en d\'{e}coulent. Il fut publi\'{e} par l'Acad\'{e}mie des Lincei au cours de la m\^{e}me ann\'{e}e.
\end{sloppypar}}
\par\hfill\par
\switchcolumn
That edition quickly went out of print and a new one appeared, with some modifications and additions, in the memoire of the Italian Seabed Topography Committee. In particular, they have added a piece concerning the case where heredity intervenes.
\par\hfill\par
\hfill\par
\switchcolumn
\foreignlanguage{french}{
Cette \'{e}dition ayant \'{e}t\'{e} rapidement \'{e}puis\'{e}e, une nouvelle en parut, avec quelques modifications et additions, dans les m\'{e}moires du Comit\'{e} Talaxographique Italien. En particulier, s'y trouve ajout\'{e}e une partie concernatn le cas o\`{u} une partie concernant le cas o\`{u} intervient l'h\'{e}r\'{e}dit\'{e}.}
\par\hfill\par
\switchcolumn
Zoologists, as well as mathematicians, are interested in this research and abstracts have appeared in various journals. They are also mentioned in the recent work of Mr. Friederichs who reports that these applications can be used in agronomical zoology.\par\hfill\par
\switchcolumn
\foreignlanguage{french}{
\begin{sloppypar}
Les zoologistes ainsi que les math\'{e}maticiens s'int\'{e}ress\`{e}rent \'{a} ces recherches de sorte qu'il parut des r\'{e}sum\'{e}s dans diff\'{e}rentes revues. Elles sont aussi mentionn\'{e}es dans le r\'{e}cent ouvrage de M. Friederichs qui signale les applications qu'on peut en faire dans la zoologie agronomique.
\end{sloppypar}}
\par\hfill\par
\switchcolumn
During the winter of 1928-1929, Mr. Borel and the direction of the new Henri Poincare Institute gave me the grand honor of hosting some conferences. The topic I chose was the mathematical theory of biological fluctuations. The present work has the same title as the conference: The Mathematical Theory of the Struggle for Life.\par\hfill\par
\switchcolumn
\foreignlanguage{french}{
Au cours de l'hiver 1928-1929, M. Borel et la Direction du nouvel Institut Henri Poincar\'{e} me firent le grand honneur de me demander quelques conf\'{e}rences. Je choisis comme sujet la th\'{e}orie math\'{e}matique des fluctuations biologiques. Le pr\'{e}sent ouvrage a le titre m\^{e}me de ces conf\'{e}ferences: Th\'{e}orie math\'{e}matique de la lutte pour la Vie.}
\par\hfill\par
\switchcolumn
In effect, the domain of applications of this research includes all of the phenomena of life between individuals and populations, the gains of some being obtained thanks to the losses of others, gains and losses that can be evaluated numerically.\par\hfill\par
\switchcolumn
\foreignlanguage{french}{
En effet le domaine d'application de ces recherches comprend tous les ph\'{e}nom\`{e}nes de lutte entre les individus d'une collectivit\'{e}, les gains des uns \'{e}tant obtenus gr\^{a}ce aux pertes desautres, gains et pertes pouvant s'\'{e}valuer num\'{e}riquement.}
\par\hfill\par
\switchcolumn
This study rests on the integrals of certain differential equations and on integro-differentials, which need to be examined in great detail, that is to say there is a qualitative way, and often only a qualitative way.\par\hfill\par
\switchcolumn
\foreignlanguage{french}{
\begin{sloppypar}
Cette \'{e}tude repose sur celle des int\'{e}grales de certaines \'{e}quations diff\'{e}rentielles et int\'{e}gro-diff\'{e}rentielles, qu'il faut examiner tr\'{e}s en d\'{e}tail, soit d'une mani\`{e}re quantitative, soit, bien souvent, d'une mani\`{e}re seulement qualitative.
\end{sloppypar}}
\par\hfill\par
\switchcolumn
I want to give a tribute to the memory of Henri Poincare and his genius, and remind us of how he insisted, in certain portions of his classic works, on the role in which the qualitative study of differential equations and integrals can play in natural philosophy.\par\hfill\par
\switchcolumn
\foreignlanguage{french}{
Je tiens ici \'{a} rendre hommage \'{a} m\'{e}moire de Henri Poincar\'{e} et \`{a} son g\'{e}nie, en rappelant combien il a insist\'{e}, dans certains de ses travaux classiques, sur le r\'{o}le que peut jouer dans la philosophie naturelle l'\'{e}tude qualitative des inte\'{e}grales des \'{e}equations diff\'{e}renetielles.}
\par\hfill\par
\switchcolumn
The conferences which I gave at the Poincare Institute over the summer have been collected by Mr. Marcel Brelot, a former student of the Normal Superior College, and they appear now in this volume.\par\hfill\par
\switchcolumn
\foreignlanguage{french}{
Les conf\'{e}rences que j'ai faites \'{a} l'Institut Henri Poincar\'{e} ont \'{e}t\'{e} recueillies par M. Marcel Brelot, ancien \'{e}l\`{e}ve de l'Ecole Normale sup\'{e}rieure, et elles paraissent maintenant dans ce volume.}
\par\hfill\par
\switchcolumn
I must thank this young geometer for the zeal and care he put into his writing. On various points he has improved and simplified the demonstrations and even posited new solutions to some questions.\par\hfill\par
\hfill\par
\switchcolumn
\foreignlanguage{french}{
Je remercie ce jeune g\'{e}ometre du z\`{e}le et du soin qu'il a mis en les r\'{e}digeant. En divers points il a am\'{e}lior\'{e} et simplifi\'{e} les d\'{e}monstrations et m\^{e}me indiqu\'{e} des solutions nouvelles de quelques questions.}
\par\hfill\par
\switchcolumn
I must say a word on the mathematical notes that I have been requested to add. This work does not address solely the mathematician who will see the analytical developments, but also to the naturalists who will work on the laws of biology. They will not, perhaps, be aware of all of some of the chapters on analysis used in this book. Mr. Brelot has drafted for their intention two mathematicals notes on determinants, linear equations, and quadratic forms. It was difficult to choose the prerequisites without writing an entire course on analysis. That is why he has taken as base the elements that are taught in France in the course "General Mathematics," and limits itself to that which is needed to know more.\par\hfill\par
\hfill\par
\hfill\par
\switchcolumn
\foreignlanguage{french}{
\begin{sloppypar}
Je dois dire un mot sur les Notes math\'{e}matiques que je l'avais pri\'{e} d'ajouter. Cet ouvrage ne s'adresse pas aux seuls manth\'{e}maticiens qui y verront des d\'{e}veloppements analytiques, mais aussi aux naturalistes qui y trouveront des lois biologiques. Or ceux-ci ne seront peut-\^{e}tre pas tous au courant de certains chapitres de l'Analyse utilis\'{e}s dans ce volume. M. Brelot a r\'{e}dig\'{e} \`{a} leur intetion deux Notes math\'{e}matiques sur les d\'{e}terminants, les \'{e}quations lin\'{e}aires et les formes quadratiques. Il \'{e}tait difficile de choisir ce qu'il fallait exposer sans \'{e}crire tout un trait\'{e} d'Analyse. C'est pourquoi il a pris comme base les \'{e}l\'{e}ments qu'on enseigne en France dans les cours de "Math\'{e}matiques g\'{e}nerales," et s'est born\'{e} \`{a} d\'{e}velopper bri\`{e}vement ce qu'il \'{e}tait n\'{e}cessaire de savoir en plus.
\end{sloppypar}}
\par\hfill\par
\switchcolumn
Mrs. Elena Freda, who had effectively helped me in the publication of my previous memoir on this subject, had desired to examine the present and assist Mr. Brelot and myself in correcting tests and improving this text. Other than the part relative to the biological associations, a history and a bibliography is contained in the last chapter. I hold at their express all my recognition four their precious contest.\par\hfill\par
\switchcolumn
\foreignlanguage{french}{
Mme. Elena Freda, qui m'avait apport\'{e} une aide efficace pour la publication de mes premiers m\'{e}moires sur ce sujet, a bien voulu examiner le pr\'{e}sent Ouvrage et seconder M. Brelot et moi-m\^{e}me dans la correction des \'{e}preuves et l'am\'{e}lioration du texte. D'autre part relatifs aux associations biologiques, un historique et une bibliographie qui figurent dans le dernier chapitre. Je tiens \'{a} leur exprimer toute ma reconnaissance pour leur pr\'{e}cieux concours.}
\par\hfill\par
\switchcolumn
I hope that they will give place to new studies and new applications.\par\hfill\par
Vito Volterra\hfill\par
Saint Gervais-les-Bains\hfill\par
July, 1930\par
\switchcolumn
\begin{sloppypar}
J'esp\`{e}re qu'elles donneront lieu \'{a} de nouvelles \'{e}tudes et \`{a} de nouvelles applications.\par\hfill\par
Vito Volterra\hfill\par
Saint-Gervais-les-Bains\hfill\par
Juillet 1930
\end{sloppypar}
\end{paracol}
\chapter{Introduction}
\begin{paracol}{2}
\section{Purpose of the Book}
We have made great progress in the application of mathematics to biology. There is, in the first place, present research on the questions of physiology relative to the senses, on the circulation of blood, the movement of animals, which we also see in chapters on optics, acoustics, hydrodynamics, on the mechanics of solid bodies and who, therefore, do not give place to the constitution of new methods outside of the domaine of classical mathematical physics.
\switchcolumn
\section{But de l'Ouvrage}
\begin{sloppypar}
On a fait bien des application des math\'{e}matiques \`{a} la biologie. Il y a, en premier lieu, les recherches sur les questions physiologiques relatives aux sens, \`{a} la circulation du sang, au mouvement des animaux, que l'on peut regarder comme des chapitres de l'optique, de l'acoustique, de l'hydrodynamique, de la m\'{e}canique des corps solides et qui, par suite, n'ont pas donn\'{e} lieu \`{a} la constituion de m\'{e}thodes nouvelles en dehors du domaine de la physique math\'{e}matique classique.
\end{sloppypar}
\end{paracol}












































\end{document}