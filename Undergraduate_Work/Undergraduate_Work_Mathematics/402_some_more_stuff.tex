\documentclass[crop=false,class=book]{standalone}
%----------------------------Preamble-------------------------------%
%---------------------------Packages----------------------------%
\usepackage{geometry}
\geometry{b5paper, margin=1.0in}
\usepackage[T1]{fontenc}
\usepackage{graphicx, float}            % Graphics/Images.
\usepackage{natbib}                     % For bibliographies.
\bibliographystyle{agsm}                % Bibliography style.
\usepackage[french, english]{babel}     % Language typesetting.
\usepackage[dvipsnames]{xcolor}         % Color names.
\usepackage{listings}                   % Verbatim-Like Tools.
\usepackage{mathtools, esint, mathrsfs} % amsmath and integrals.
\usepackage{amsthm, amsfonts, amssymb}  % Fonts and theorems.
\usepackage{tcolorbox}                  % Frames around theorems.
\usepackage{upgreek}                    % Non-Italic Greek.
\usepackage{fmtcount, etoolbox}         % For the \book{} command.
\usepackage[newparttoc]{titlesec}       % Formatting chapter, etc.
\usepackage{titletoc}                   % Allows \book in toc.
\usepackage[nottoc]{tocbibind}          % Bibliography in toc.
\usepackage[titles]{tocloft}            % ToC formatting.
\usepackage{pgfplots, tikz}             % Drawing/graphing tools.
\usepackage{imakeidx}                   % Used for index.
\usetikzlibrary{
    calc,                   % Calculating right angles and more.
    angles,                 % Drawing angles within triangles.
    arrows.meta,            % Latex and Stealth arrows.
    quotes,                 % Adding labels to angles.
    positioning,            % Relative positioning of nodes.
    decorations.markings,   % Adding arrows in the middle of a line.
    patterns,
    arrows
}                                       % Libraries for tikz.
\pgfplotsset{compat=1.9}                % Version of pgfplots.
\usepackage[font=scriptsize,
            labelformat=simple,
            labelsep=colon]{subcaption} % Subfigure captions.
\usepackage[font={scriptsize},
            hypcap=true,
            labelsep=colon]{caption}    % Figure captions.
\usepackage[pdftex,
            pdfauthor={Ryan Maguire},
            pdftitle={Mathematics and Physics},
            pdfsubject={Mathematics, Physics, Science},
            pdfkeywords={Mathematics, Physics, Computer Science, Biology},
            pdfproducer={LaTeX},
            pdfcreator={pdflatex}]{hyperref}
\hypersetup{
    colorlinks=true,
    linkcolor=blue,
    filecolor=magenta,
    urlcolor=Cerulean,
    citecolor=SkyBlue
}                           % Colors for hyperref.
\usepackage[toc,acronym,nogroupskip,nopostdot]{glossaries}
\usepackage{glossary-mcols}
%------------------------Theorem Styles-------------------------%
\theoremstyle{plain}
\newtheorem{theorem}{Theorem}[section]

% Define theorem style for default spacing and normal font.
\newtheoremstyle{normal}
    {\topsep}               % Amount of space above the theorem.
    {\topsep}               % Amount of space below the theorem.
    {}                      % Font used for body of theorem.
    {}                      % Measure of space to indent.
    {\bfseries}             % Font of the header of the theorem.
    {}                      % Punctuation between head and body.
    {.5em}                  % Space after theorem head.
    {}

% Italic header environment.
\newtheoremstyle{thmit}{\topsep}{\topsep}{}{}{\itshape}{}{0.5em}{}

% Define environments with italic headers.
\theoremstyle{thmit}
\newtheorem*{solution}{Solution}

% Define default environments.
\theoremstyle{normal}
\newtheorem{example}{Example}[section]
\newtheorem{definition}{Definition}[section]
\newtheorem{problem}{Problem}[section]

% Define framed environment.
\tcbuselibrary{most}
\newtcbtheorem[use counter*=theorem]{ftheorem}{Theorem}{%
    before=\par\vspace{2ex},
    boxsep=0.5\topsep,
    after=\par\vspace{2ex},
    colback=green!5,
    colframe=green!35!black,
    fonttitle=\bfseries\upshape%
}{thm}

\newtcbtheorem[auto counter, number within=section]{faxiom}{Axiom}{%
    before=\par\vspace{2ex},
    boxsep=0.5\topsep,
    after=\par\vspace{2ex},
    colback=Apricot!5,
    colframe=Apricot!35!black,
    fonttitle=\bfseries\upshape%
}{ax}

\newtcbtheorem[use counter*=definition]{fdefinition}{Definition}{%
    before=\par\vspace{2ex},
    boxsep=0.5\topsep,
    after=\par\vspace{2ex},
    colback=blue!5!white,
    colframe=blue!75!black,
    fonttitle=\bfseries\upshape%
}{def}

\newtcbtheorem[use counter*=example]{fexample}{Example}{%
    before=\par\vspace{2ex},
    boxsep=0.5\topsep,
    after=\par\vspace{2ex},
    colback=red!5!white,
    colframe=red!75!black,
    fonttitle=\bfseries\upshape%
}{ex}

\newtcbtheorem[auto counter, number within=section]{fnotation}{Notation}{%
    before=\par\vspace{2ex},
    boxsep=0.5\topsep,
    after=\par\vspace{2ex},
    colback=SeaGreen!5!white,
    colframe=SeaGreen!75!black,
    fonttitle=\bfseries\upshape%
}{not}

\newtcbtheorem[use counter*=remark]{fremark}{Remark}{%
    fonttitle=\bfseries\upshape,
    colback=Goldenrod!5!white,
    colframe=Goldenrod!75!black}{ex}

\newenvironment{bproof}{\textit{Proof.}}{\hfill$\square$}
\tcolorboxenvironment{bproof}{%
    blanker,
    breakable,
    left=3mm,
    before skip=5pt,
    after skip=10pt,
    borderline west={0.6mm}{0pt}{green!80!black}
}

\AtEndEnvironment{lexample}{$\hfill\textcolor{red}{\blacksquare}$}
\newtcbtheorem[use counter*=example]{lexample}{Example}{%
    empty,
    title={Example~\theexample},
    boxed title style={%
        empty,
        size=minimal,
        toprule=2pt,
        top=0.5\topsep,
    },
    coltitle=red,
    fonttitle=\bfseries,
    parbox=false,
    boxsep=0pt,
    before=\par\vspace{2ex},
    left=0pt,
    right=0pt,
    top=3ex,
    bottom=1ex,
    before=\par\vspace{2ex},
    after=\par\vspace{2ex},
    breakable,
    pad at break*=0mm,
    vfill before first,
    overlay unbroken={%
        \draw[red, line width=2pt]
            ([yshift=-1.2ex]title.south-|frame.west) to
            ([yshift=-1.2ex]title.south-|frame.east);
        },
    overlay first={%
        \draw[red, line width=2pt]
            ([yshift=-1.2ex]title.south-|frame.west) to
            ([yshift=-1.2ex]title.south-|frame.east);
    },
}{ex}

\AtEndEnvironment{ldefinition}{$\hfill\textcolor{Blue}{\blacksquare}$}
\newtcbtheorem[use counter*=definition]{ldefinition}{Definition}{%
    empty,
    title={Definition~\thedefinition:~{#1}},
    boxed title style={%
        empty,
        size=minimal,
        toprule=2pt,
        top=0.5\topsep,
    },
    coltitle=Blue,
    fonttitle=\bfseries,
    parbox=false,
    boxsep=0pt,
    before=\par\vspace{2ex},
    left=0pt,
    right=0pt,
    top=3ex,
    bottom=0pt,
    before=\par\vspace{2ex},
    after=\par\vspace{1ex},
    breakable,
    pad at break*=0mm,
    vfill before first,
    overlay unbroken={%
        \draw[Blue, line width=2pt]
            ([yshift=-1.2ex]title.south-|frame.west) to
            ([yshift=-1.2ex]title.south-|frame.east);
        },
    overlay first={%
        \draw[Blue, line width=2pt]
            ([yshift=-1.2ex]title.south-|frame.west) to
            ([yshift=-1.2ex]title.south-|frame.east);
    },
}{def}

\AtEndEnvironment{ltheorem}{$\hfill\textcolor{Green}{\blacksquare}$}
\newtcbtheorem[use counter*=theorem]{ltheorem}{Theorem}{%
    empty,
    title={Theorem~\thetheorem:~{#1}},
    boxed title style={%
        empty,
        size=minimal,
        toprule=2pt,
        top=0.5\topsep,
    },
    coltitle=Green,
    fonttitle=\bfseries,
    parbox=false,
    boxsep=0pt,
    before=\par\vspace{2ex},
    left=0pt,
    right=0pt,
    top=3ex,
    bottom=-1.5ex,
    breakable,
    pad at break*=0mm,
    vfill before first,
    overlay unbroken={%
        \draw[Green, line width=2pt]
            ([yshift=-1.2ex]title.south-|frame.west) to
            ([yshift=-1.2ex]title.south-|frame.east);},
    overlay first={%
        \draw[Green, line width=2pt]
            ([yshift=-1.2ex]title.south-|frame.west) to
            ([yshift=-1.2ex]title.south-|frame.east);
    }
}{thm}

%--------------------Declared Math Operators--------------------%
\DeclareMathOperator{\adjoint}{adj}         % Adjoint.
\DeclareMathOperator{\Card}{Card}           % Cardinality.
\DeclareMathOperator{\curl}{curl}           % Curl.
\DeclareMathOperator{\diam}{diam}           % Diameter.
\DeclareMathOperator{\dist}{dist}           % Distance.
\DeclareMathOperator{\Div}{div}             % Divergence.
\DeclareMathOperator{\Erf}{Erf}             % Error Function.
\DeclareMathOperator{\Erfc}{Erfc}           % Complementary Error Function.
\DeclareMathOperator{\Ext}{Ext}             % Exterior.
\DeclareMathOperator{\GCD}{GCD}             % Greatest common denominator.
\DeclareMathOperator{\grad}{grad}           % Gradient
\DeclareMathOperator{\Ima}{Im}              % Image.
\DeclareMathOperator{\Int}{Int}             % Interior.
\DeclareMathOperator{\LC}{LC}               % Leading coefficient.
\DeclareMathOperator{\LCM}{LCM}             % Least common multiple.
\DeclareMathOperator{\LM}{LM}               % Leading monomial.
\DeclareMathOperator{\LT}{LT}               % Leading term.
\DeclareMathOperator{\Mod}{mod}             % Modulus.
\DeclareMathOperator{\Mon}{Mon}             % Monomial.
\DeclareMathOperator{\multideg}{mutlideg}   % Multi-Degree (Graphs).
\DeclareMathOperator{\nul}{nul}             % Null space of operator.
\DeclareMathOperator{\Ord}{Ord}             % Ordinal of ordered set.
\DeclareMathOperator{\Prin}{Prin}           % Principal value.
\DeclareMathOperator{\proj}{proj}           % Projection.
\DeclareMathOperator{\Refl}{Refl}           % Reflection operator.
\DeclareMathOperator{\rk}{rk}               % Rank of operator.
\DeclareMathOperator{\sgn}{sgn}             % Sign of a number.
\DeclareMathOperator{\sinc}{sinc}           % Sinc function.
\DeclareMathOperator{\Span}{Span}           % Span of a set.
\DeclareMathOperator{\Spec}{Spec}           % Spectrum.
\DeclareMathOperator{\supp}{supp}           % Support
\DeclareMathOperator{\Tr}{Tr}               % Trace of matrix.
%--------------------Declared Math Symbols--------------------%
\DeclareMathSymbol{\minus}{\mathbin}{AMSa}{"39} % Unary minus sign.
%------------------------New Commands---------------------------%
\DeclarePairedDelimiter\norm{\lVert}{\rVert}
\DeclarePairedDelimiter\ceil{\lceil}{\rceil}
\DeclarePairedDelimiter\floor{\lfloor}{\rfloor}
\newcommand*\diff{\mathop{}\!\mathrm{d}}
\newcommand*\Diff[1]{\mathop{}\!\mathrm{d^#1}}
\renewcommand*{\glstextformat}[1]{\textcolor{RoyalBlue}{#1}}
\renewcommand{\glsnamefont}[1]{\textbf{#1}}
\renewcommand\labelitemii{$\circ$}
\renewcommand\thesubfigure{%
    \arabic{chapter}.\arabic{figure}.\arabic{subfigure}}
\addto\captionsenglish{\renewcommand{\figurename}{Fig.}}
\numberwithin{equation}{section}

\renewcommand{\vector}[1]{\boldsymbol{\mathrm{#1}}}

\newcommand{\uvector}[1]{\boldsymbol{\hat{\mathrm{#1}}}}
\newcommand{\topspace}[2][]{(#2,\tau_{#1})}
\newcommand{\measurespace}[2][]{(#2,\varSigma_{#1},\mu_{#1})}
\newcommand{\measurablespace}[2][]{(#2,\varSigma_{#1})}
\newcommand{\manifold}[2][]{(#2,\tau_{#1},\mathcal{A}_{#1})}
\newcommand{\tanspace}[2]{T_{#1}{#2}}
\newcommand{\cotanspace}[2]{T_{#1}^{*}{#2}}
\newcommand{\Ckspace}[3][\mathbb{R}]{C^{#2}(#3,#1)}
\newcommand{\funcspace}[2][\mathbb{R}]{\mathcal{F}(#2,#1)}
\newcommand{\smoothvecf}[1]{\mathfrak{X}(#1)}
\newcommand{\smoothonef}[1]{\mathfrak{X}^{*}(#1)}
\newcommand{\bracket}[2]{[#1,#2]}

%------------------------Book Command---------------------------%
\makeatletter
\renewcommand\@pnumwidth{1cm}
\newcounter{book}
\renewcommand\thebook{\@Roman\c@book}
\newcommand\book{%
    \if@openright
        \cleardoublepage
    \else
        \clearpage
    \fi
    \thispagestyle{plain}%
    \if@twocolumn
        \onecolumn
        \@tempswatrue
    \else
        \@tempswafalse
    \fi
    \null\vfil
    \secdef\@book\@sbook
}
\def\@book[#1]#2{%
    \refstepcounter{book}
    \addcontentsline{toc}{book}{\bookname\ \thebook:\hspace{1em}#1}
    \markboth{}{}
    {\centering
     \interlinepenalty\@M
     \normalfont
     \huge\bfseries\bookname\nobreakspace\thebook
     \par
     \vskip 20\p@
     \Huge\bfseries#2\par}%
    \@endbook}
\def\@sbook#1{%
    {\centering
     \interlinepenalty \@M
     \normalfont
     \Huge\bfseries#1\par}%
    \@endbook}
\def\@endbook{
    \vfil\newpage
        \if@twoside
            \if@openright
                \null
                \thispagestyle{empty}%
                \newpage
            \fi
        \fi
        \if@tempswa
            \twocolumn
        \fi
}
\newcommand*\l@book[2]{%
    \ifnum\c@tocdepth >-3\relax
        \addpenalty{-\@highpenalty}%
        \addvspace{2.25em\@plus\p@}%
        \setlength\@tempdima{3em}%
        \begingroup
            \parindent\z@\rightskip\@pnumwidth
            \parfillskip -\@pnumwidth
            {
                \leavevmode
                \Large\bfseries#1\hfill\hb@xt@\@pnumwidth{\hss#2}
            }
            \par
            \nobreak
            \global\@nobreaktrue
            \everypar{\global\@nobreakfalse\everypar{}}%
        \endgroup
    \fi}
\newcommand\bookname{Book}
\renewcommand{\thebook}{\texorpdfstring{\Numberstring{book}}{book}}
\providecommand*{\toclevel@book}{-2}
\makeatother
\titleformat{\part}[display]
    {\Large\bfseries}
    {\partname\nobreakspace\thepart}
    {0mm}
    {\Huge\bfseries}
\titlecontents{part}[0pt]
    {\large\bfseries}
    {\partname\ \thecontentslabel: \quad}
    {}
    {\hfill\contentspage}
\titlecontents{chapter}[0pt]
    {\bfseries}
    {\chaptername\ \thecontentslabel:\quad}
    {}
    {\hfill\contentspage}
\newglossarystyle{longpara}{%
    \setglossarystyle{long}%
    \renewenvironment{theglossary}{%
        \begin{longtable}[l]{{p{0.25\hsize}p{0.65\hsize}}}
    }{\end{longtable}}%
    \renewcommand{\glossentry}[2]{%
        \glstarget{##1}{\glossentryname{##1}}%
        &\glossentrydesc{##1}{~##2.}
        \tabularnewline%
        \tabularnewline
    }%
}
\newglossary[not-glg]{notation}{not-gls}{not-glo}{Notation}
\newcommand*{\newnotation}[4][]{%
    \newglossaryentry{#2}{type=notation, name={\textbf{#3}, },
                          text={#4}, description={#4},#1}%
}
%--------------------------LENGTHS------------------------------%
% Spacings for the Table of Contents.
\addtolength{\cftsecnumwidth}{1ex}
\addtolength{\cftsubsecindent}{1ex}
\addtolength{\cftsubsecnumwidth}{1ex}
\addtolength{\cftfignumwidth}{1ex}
\addtolength{\cfttabnumwidth}{1ex}

% Indent and paragraph spacing.
\setlength{\parindent}{0em}
\setlength{\parskip}{0em}
%----------------------------GLOSSARY-------------------------------%
\makeglossaries
\loadglsentries{../../glossary}
\loadglsentries{../../acronym}
%---------------------------Title Page------------------------------%
\begin{document}
\chapter{Some More Stuff}
    \section{On the Quotients of Primes}
        \begin{theorem}
            The set of rational numbers $\frac{p}{q}$ where $p$
            and $q$ are prime is dense in $\mathbb{R}^{+}$.
        \end{theorem}
        \begin{proof}
            If $x=0$, from Euclid we have
            $\frac{1}{p_n}\rightarrow 0$,
            where $p_n$ is the $n^{th}$ prime. Let
            $x\in\mathbb{R}^{+}$ be given. From the Prime Number
            Theorem, $\frac{p_n}{n\ln(n)}\rightarrow 1$. Let
            $p_{\ceil{nx}}$ be the $\ceil{nx}^{th}$ prime. Then
            $\frac{p_{\ceil{nx}}}{p_n}\frac{n\ln(n)}{nx\ln(nx)}
            \rightarrow 1$. But
            $\frac{n\ln(x)}{nx\ln(nx)}\rightarrow \frac{1}{x}$.
            Therefore $\frac{p_{\ceil{nx}}}{p_n}\rightarrow x$.
        \end{proof}
    \section{On Uniform Convergence}
        \begin{definition}
            A sequence of functions $f_n$ is said to converge
            point-wise on a set $A$ to a function $f$, if
            $\forall\varepsilon>0$ and $\forall x\in A$, there is
            an $N\in\mathbb{N}$ such that $n>N \Rightarrow
            |f(x)-f_n(x)|<\varepsilon$.
        \end{definition}
        \begin{definition}
            A sequence of functions $f_n$ converge uniformly on a
            set $A$ to $f$ if and only if $\forall \varepsilon>0$
            $\exists N\in\mathbb{N}$ such that $\forall x \in A$
            and $n>N$, $|f(x) -f_n(x)|<\varepsilon$.
        \end{definition}
        \begin{definition}
            A sequence of functions $f_n$ are point-wise
            equicontinuous on a set $A$ if and only if
            $\forall_{\varepsilon>0}\forall_{x\in A}
            \exists_{\delta>0}\forall_{n\in\mathbb{N}}:
            |x-x_0|<\delta\Rightarrow|f_{n}(x)-f_{n}(x_0)|<
            \varepsilon$
        \end{definition}
        \begin{definition}
            A sequence of functions $f_n$ are uniformly
            equicontinuous on a set A if and only if
            $\forall_{\varepsilon>0}\exists_{\delta>0}
            \forall_{x\in A}\forall_{n\in\mathbb{N}}:
            |x-x_{0}|<\delta\Rightarrow|f_{n}(x)-f_{n}(x_{0})|<
            \varepsilon$.
        \end{definition}
        \begin{definition}
            A subset $A$ of the real line is open if
            $\forall_{x\in A}\exists_{r>0}:
            \forall_{y\in (x-r,x+r)},y\in A$.
        \end{definition}
        \begin{definition}
            An open cover $\Delta$ of a set $A\subset S$ is a set
            of open subsets $A_k\subset S$, such that
            $A\subset\cup_{k\in I}A_{k}$, where $I$ is some
            index, countable or uncountable.
        \end{definition}
        \begin{definition}
            A set $A$ is said to be compact if and only if for all open coverings $\Delta$ there is a finite sub-cover $\Delta_0\subset \Delta$, such that $A\subset \cup_{A_k \in \Delta_0} A_k$.
        \end{definition}
        \begin{theorem}
            [Heine-Borel Theorem]
            Any closed-bounded subset of the real line is compact. 
        \end{theorem}
        \begin{proof}
            For let $A$ be a closed and bounded subset of $\mathbb{R}$ with least upper bound $b$ and greatest lower bound $a$. Let $\Delta$ be an open covering, and let $X$ be the set of points $y\in A$ such that for all $s<y$ such that $s\in A$, there is a finite refinement of $\Delta$ which covers these points. $X$ is non-empty, as $a\in X$. The set $X$ is bounded, as for all points $y\in X$ we have that $a\leq y \leq b$. As bounded sets have a least upper bound, let $x$ be the least upper bound. Suppose $x<b$. As $x\in [a,b]$, there exists an element $A_k$ of $\Delta$ such that $x \in A_k$. But as $A_k$ is open and therefore there is an $r>0$ such that $y\in (x-r,x+r)\Rightarrow y \in A_k$. But then $y=x + \frac{r}{2} > x$ and $y\in A_k$. Therefore x is not the least upper bound as we have found an element in $X$ greater than $x$. Therefore $x\not<b$. And thus $x=b$.
        \end{proof}
        \begin{theorem}
            If a sequence of functions are point-wise equicontinuous on a closed and bounded set, then they are uniformly equicontinuous.
        \end{theorem}
        \begin{proof}
            For let $A$ be a closed bounded subset of $\mathbb{R}$, and let $f_n(x)$ be a sequence of point-wise equicontinuous functions on $A$. As the set is closed and bounded, it is compact by the Heine-Borel theorem. Let $\varepsilon>0$ be given. For $x\in A$, define the function $\delta(x) = \min\{\sup\{\delta>0:\ |x-x_0|<\delta,\ x_0\in A \Rightarrow |f_n(x)-f_n(x_0)|<\frac{\varepsilon}{2}, \forall n\in\mathbb{N}\},b-a\}$. Construct the open covering $\mathcal{U}$ as follows: $\mathcal{U} = \{(x-\delta(x),x+\delta(x)):\ x\in A\}$. This is an open covering, as every set in $\mathcal{U}$ is open, and for all $x\in A$, $x\in(x-\delta(x),x+\delta(x))\in\mathcal{U}$. But as $A$ is compact, there is a finite sub-cover. Let $a=x_0<x_1<...<x_{n-1}<x_n=b$ be the centers of the remaining open sets in the sub-cover. Further refine this sub-covering as follows: If $(x_j-\delta(x_j),x_j+\delta(x_j))\subset (x_k-\delta(x_k),x_k+\delta(x_k))$ for $j\ne k$, then remove it from the sub-cover as it is superfluous. We now have a set of points $a=a_0<z_1<...<z_{N-1}<z_N=b$ such that $A \subset \cup_{i=0}^{N} (z_i-\delta(z_i),z_i+\delta(z_i))$. Let $\delta = \min\{\delta(z_0),...,\delta(z_N),\delta(b), \frac{(a+\delta(a)) - (z_1-\delta(z_1))}{2}, ..., \frac{(z_{N-1} + \delta(z_{N-1})) - (b-\delta(b))}{2}\}$. That is, $\delta$ is the smallest of the $\delta(z_i)$, or half of the smallest intersection of two consecutive intervals. Let $x\in A$ be arbitrary. If $(x-\delta,x+\delta)$ is contained entirely in one of the $(z_i-\delta(z_i),z_i+\delta(z_i))$ sets, then we have that $|x-x_0|<\delta \Rightarrow |x-x_0| <\delta(z_i) \Rightarrow |f_n(x)-f_n(x_0)|<\frac{\varepsilon}{2}$ for all $n\in\mathbb{N}$. Suppose that $(x-\delta,x+\delta)$ is contained in two of the $(x-\delta(z_i),x+\delta(z_i))$ sets. Note, it cannot be in three or more as we have refined the sub-cover in such a manner as to prevent this. Let $y$ be the center of the intersection of these two sets. Then we have that for $|x-x_0|<\delta$, then $|f_n(x)-f_n(x_0)| = |f_n(x) - f_n(y) + f_n(y) - f_n(x_0)| \leq |f_n(x) - f_n(y)| + |f_n(y) - f_n(x_0)|$. But $|x-y|$ and $|x_0-y|$ are less than $\frac{(z_i + \delta(z_i))-(z_{i+1}-\delta(z_{i+1}))}{2}$ apart, and therefore $|f_n(x) - f_n(y)|<\frac{\varepsilon}{2}$, and $|f_n(y) - f_n(x_0)| < \frac{\varepsilon}{2}$. Therefore, $|f_n(x) - f_n(x_0)|<\varepsilon$. And as $x$ is arbitrary, $f_n(x)$ is uniformly equicontinuous.
        \end{proof}
        \begin{theorem}
            If a sequence of point-wise equicontinuous functions converge, then the limit is point-wise continuous.
        \end{theorem}
        \begin{proof}
            For let $f_n:A\rightarrow \mathbb{R}$ be equicontinuous, $\varepsilon>0$ and $x\in A$ be given. Choose $\delta>0$ to satisfy the criterion of equicontinuity at $x$. Let $x_0$ be an arbitrary point in $(x-\delta,x+\delta)\cap A$. It suffices to show that $|f(x) - f(x_0)|<\varepsilon$. As $f_n \rightarrow f$ we have that $\exists N_1 \in\mathbb{N}$ such that $n>N_1\Rightarrow |f(x) - f_n(x)|<\varepsilon$. We also have that $\exists N_2 \in \mathbb{N}$ such that $n>N_2 \Rightarrow |f(x_0)-f_n(x_0)|<\varepsilon$. Let $N=\max\{N_1,N_2\}+1$. But we have that $|f(x) - f(x_0)| = |f(x) - f_N(x) + f_N(x)-f_N(x_0) + f_N(x_0) - f(x_0)|\leq |f(x) - f_n(x)| + |f_n(x)-f_n(x_0)| + |f_n(x_0) - f(x_0)| < 3\varepsilon$. $f$ is continuous.
        \end{proof}
        \begin{theorem}
            If $f_n \rightarrow f$ on a closed bounded subset of $\mathbb{R}$, and if $f_n$ is equicontinuous, then the convergence is uniform.
        \end{theorem}
        \begin{proof}
            Let $A$ be a closed bounded subset of $\mathbb{R}$, $f_n(x)$ a sequence of equicontinuous functions, and let $\varepsilon>0$ be given. As $f_n(x)$ is equicontinuous on a closed bounded set, it is uniformly equicontinuous. But the limit of equicontinuous functions is continuous. Let $\delta>0$ be such that, $\forall x\in A$, $\forall n\in\mathbb{N}$, $|x-x_0|<\delta, x_0\in A \Rightarrow |f_n(x)-f_n(x_0)|<\frac{\varepsilon}{3}$ and $|x-x_0|<\delta \Rightarrow |f(x)-f(x_0)|<\frac{\varepsilon}{3}$. Let $\mathcal{U} = \{(x-\frac{\delta}{2},x+\frac{\delta}{2}): x\in A\}$. This is an open cover of $A$ and thus there is a finite subcover. Let $x_0<x_1<\hdots<x_n$ be the centers of the finitely many sets $(x_k-\frac{\delta}{2},x_k+\frac{\delta}{2})$ that cover $A$. There is thus another finite sequence of positive integers, $N_0, N_1,... N_n$, such that $n>N_k \Rightarrow |f(x_k)-f_n(x_k)|<\frac{\varepsilon}{3}$, for $k=0,1,2,...,n$. Let $N= \max\{N_0, N_1, ..., N_n\}$.It suffices to show that, for any point $x_0 \in A$, for all $n>N$, $|f(x_0)-f_n(x_0)|<\varepsilon$. Let $x_0$ be arbitrary and let $x_k$ be the nearest point to $x_0$ in the above sequence (If there are two nearest points, pick your favorite). Then we have that, for $n>N$, $|f(x_0) - f_n(x_0)| = |f(x_0)-f(x_k)+f(x_k)-f_n(x_k)+f_n(x_k)-f_n(x_0)|\leq |f(x_k)-f(x_0)|+|f(x_k)-f_n(x_k)|+|f_n(x_k)-f_n(x_0)|<\varepsilon$. The convergence is uniform.
        \end{proof}
        \begin{theorem}
            [Integration of a Uniformly Convergent Sequence of Functions]
            If $f_n\rightarrow f$ uniformly on a closed bounded set $A$ with $g.u.b(A)=a$, then $\int_{a}^{x} f_n \rightarrow \int_{a}^{x} f$ uniformly on $A$.
        \end{theorem}
        \begin{proof}
            Let $\varepsilon >0$ be given, let $b=l.u.b.(A)$, and choose $N\in\mathbb{N}$ such that $n>N\Rightarrow |f(x)-f_n(x)|<\frac{\varepsilon}{b-a}$. Then we have $|\int_{a}^{x} f_n - \int_{a}^{x} f| = |\int_{a}^{x} (f_n-f)| \leq \int_{a}^{x} |f_n-f| < \int_{a}^{x} \frac{\varepsilon}{b-a}= \frac{\varepsilon}{b-a}(x-a) \leq \varepsilon$.
        \end{proof}
        \begin{theorem}
            [Differentiation of a Uniformly Convergent Sequence of Functions]
            If $f_n'\rightarrow g$ uniformly on a closed bounded set $A$, and if $f_n \rightarrow f$ on $A$, then $f'=g$.
        \end{theorem}
        \begin{proof}
            Let $a=g.u.b.(A)$ and $b=l.u.b.(A)$. We have that $f_n(x) - f_n(a) = \int_{a}^{x}f_n' \rightarrow \int_{a}^{x}g$ uniformly. But $f_n(x)-f_n(a) \rightarrow f(x) - f(a)$. Therefore $f'(x)=\frac{d}{dx}(f(x)-f(a)) = \frac{d}{dx}\int_{a}^{x} g = g(x)$. $f' = g$.
        \end{proof}
        \begin{theorem}
            [The Product of a Uniformly Convergence Sequence and a Bounded Function]
            If $f_n \rightarrow f$ uniformly, and if $g$ is a bounded function, then $f_n g \rightarrow fg$ uniformly.
        \end{theorem}
        \begin{proof}
            For let $\varepsilon>0$ and $x$ be given, and let $g$ be a bounded function with bound $M$, and choose $N\in\mathbb{N}$ such that $n>N \Rightarrow |f(x)-f_n(x)|<\frac{\varepsilon}{M}$. Then we have that $|f(x)g(x)-f_n(x)g(x)| = |g(x)||f(x)-f_n(x)| < M|f(x)-f_n(x)| <\varepsilon$.
        \end{proof}
        \begin{theorem}
            If $f$ is continuous on a compact set $A$, then it is uniformly continuous.
        \end{theorem}
        \begin{proof}
            For let $\varepsilon>0$ be given, let $a=g.u.b.(A)$, $b=l.u.b.(A)$, and for $x\in A$ define $\delta(x) = \min\{\sup\{\delta>0: |x-x_0|<\delta,x_0\in A\Rightarrow |f(x)-f(x_0)|<\frac{\varepsilon}{2}\},b-a\}$. Let $\Delta = \{(x-\delta(x),x+\delta(x)):x\in A\}$. Then $\Delta$ is an open cover of $A$ and therefore there is an open subscover. Let $x_k$ be the centers of the finitely many sets $(x_k-\delta(x_k),x+\delta(x_k))$ that cover $A$. Further refine this by removing overlaps. That is, if $(x_i-\delta(x_i),x_i+\delta(x_i))\subset (x_j-\delta(x_j),x_j+\delta(x_k))$ for $i\ne j$, then remove it for it is superfluous. We thus obtain a new sequence $z_1,\hdots, z_N$ such that the intervals $(z_k-\delta(z_k),z_k+\delta(z_k))$ cover $A$. Define $\delta = \min\{\delta(z_1),\hdots,\delta(z_N), \frac{(z_0+\delta(z_0))-(z_1-\delta(z_1))}{2},\hdots,(\frac{z_{N-1}+\delta(z_{N-1}))-(z_{N}-\delta(z_{N})}{2}\}$. Let $x,x_0\in A$ such that $|x-x_0|<\delta$. Let $x_k$ be the closest point in the sequence to $x$ (If there are two such points, pick your favorite). Then $|f(x)-f(x_0)|=|f(x)-f(x_k)+f(x_k)-f(x_0)|\leq |f(x)-f(x_k)|+|f(x_k)-f(x_0)|<\varepsilon$
        \end{proof}
        \begin{remark}
            The proof of this is a mimicry of the proof that equicontinuity on a compact set implies uniform equicontinuity.
        \end{remark}
        \begin{definition}
            A set $A$ is called sequentially compact if given a sequence $x_n\in A$, there is a convergent subsequence $x_{n_k}$.
        \end{definition}
        \begin{theorem}
            Compact sets of $\mathbb{R}$ are sequentially compact.
        \end{theorem}
        \begin{proof}
            Let $A$ be a compact set in $\mathbb{R}$, and let $x_n$ be a sequence in $A$. A point $x\in A$ is the limit of a subsequence of $x_n$ if for every $\varepsilon>0$ there are infinitely many of the $x_n$ such that $|x-x_n|<\varepsilon$. Suppose there is no such point. That is, for each $x\in A$ only finitely many of the $x_n$ lie within sufficiently small $\varepsilon-$neighborhoods. Let $\varepsilon(x) = \sup\{\varepsilon>0:\textrm{Only finitely many }x_n \textrm{ lie within } \varepsilon \textrm{ of } x\}$. Define $E=\{(x-\varepsilon(x)<x+\varepsilon(x)):x\in A\}$. This is an open cover of $A$, and therefore there is a finite subcover. Thus, at least one of the finitely many intervals $(x-\varepsilon(x),x+\varepsilon(x))$ must contain infinitely many of the $x_n$, a contradiction. Thus there is a convergent subsequence.
        \end{proof}
        \begin{theorem}
            Continuous functions on compact sets are bounded.
        \end{theorem}
        \begin{proof}
            For suppose not. Let $f:A\rightarrow \mathbb{R}$ be a continuous function on a compact set $A$, and let $x_n$ be a sequence of points in $A$ such that $f(x_n)>n$. Such a sequence must exist as $f$ is not bounded. As $A$ is compact, there must a point $x\in A$ such that some subsequence $x_{n_k}$ that converges to $x$. Let $\varepsilon >0$. Then, as $f$ is continuous, there is a $\delta>0$ such that $|x-x_0|<\delta,\ x_0\in A\Rightarrow |f(x)-f(x_0)|<\varepsilon$. But then for all points $x_{n_k}$ such that $|x-x_{n_k}|<\delta$, $-\varepsilon<f(x_{n_k})-f(x)<\varepsilon \Rightarrow f(x)-\varepsilon < f(x_{n_k})<f(x)+\varepsilon$. A contradiction as $f(x_{n_k})$ is unbounded. Thus, $f$ is bounded.
        \end{proof}
        \begin{corollary}
            Continuous functions on compact sets attain their maximum and minimum.
        \end{corollary}
        \begin{proof}
            For let $f:A\rightarrow \mathbb{R}$ be a continuous function on a compact set $A$. Let $f(A) = \{y\in \mathbb{R}:\exists x\in A|\ f(x)=y\}$. (This is called the image of $A$ under $f$). As $f$ is continuous, it is bounded, and thus the set $f(A)$ is bounded. But bounded sets have an l.u.b. and a g.u.b. Therefore, etc.
        \end{proof}
        \begin{lemma}
            [Uniform Limit Theorem]
            If $f_n\rightarrow f$ uniformly, and if the $f_n$ are continuous, then $f$ is continuous.
        \end{lemma}
        \begin{proof}
            For let $\varepsilon>0$ be given and let $x\in A$. Let $N\in \mathbb{N}$ such that $n>N$ implies $|f(\chi)-f_n(\chi)|<\frac{\varepsilon}{3}$ for all $\chi\in A$. Let $\delta>0$ be chosen such that $|x-x_0|<\delta, x_0\in A\Rightarrow |f_N(x)-f_N(x_0)|<\frac{\varepsilon}{3}$. Then  $|f(x)-f(x_0)|=|f(x)-f_N(x)+f_N(x)-f_N(x_0)+f_N(x_0)-f(x_0)|\leq |f(x)-f_N(x_0)|+|f_N(x)-f_N(x_0)|+|f(x_0)-f_N(x_0)|<\varepsilon$.
        \end{proof}
        \begin{theorem}
            If $f_ng\rightarrow fg$ uniformly on a compact set $A$, and if $g$ is continuous and positive, then $f_n\rightarrow f$ uniformly.
        \end{theorem}
        \begin{proof}
            As $g$ is positive on a compact set, its minimum is also positive and is attained on $A$. Let $x_{min}\in A$ be such a minimum of $g$. Let $\varepsilon>0$ be given and let $N\in \mathbb{N}$ be such that for $n>N$, $|f_ng-fg|<\varepsilon\cdot g(x_{min})$. Then, $|f_ng-fg|=|g||f_n-f|\leq |g(x_{min})||f_n-f|<\varepsilon \cdot g(x_{min})\Rightarrow |f_n-f|<\varepsilon$.
        \end{proof}
        \begin{lemma}
            If $f_n'$ is uniformly bounded, then $f_n$ is equicontinuous.
        \end{lemma}
        \begin{proof}
            For let $M$ be such a bound for $f_n'$ and let $\varepsilon>0$ be given. Choose $\delta = \frac{\varepsilon}{M}$. Then for $x,x_0\in A$ and $|x-x_0|<\delta$, $|\int_{x_0}^{x}f_n'| =|f_n(x)-f_n(x_0)| \leq \int_{x_0}^{x}|f_n'| \leq (x-x_0)M < \varepsilon$.
        \end{proof}
        \begin{theorem}
            If $f_n'$ is uniformly bounded, and if $f_n \rightarrow f$ on a closed and bounded subset of $\mathbb{R}$, then the convergence is uniform.
        \end{theorem}
        \begin{proof}
            From the previous lemma, $f_n$ is equicontinuous. But a sequence of equicontinuous functions on a compact set is uniformly equicontinuous. And a sequence of uniformly equicontinuous functions that converge does so uniformly. Therefore, etc.
        \end{proof}
        \begin{theorem}
            If $f_n \rightarrow f$, $f_n'\rightarrow g$ and if $f_n''-f_n'$ is uniformly bounded on a closed bounded set, then the convergences are uniform and $f' = g$.
        \end{theorem}
        \begin{proof}
            Let $A$ be the closed bounded set under consideration. First note that as $f''_n - f'_n$ is uniformly bounded, $f_n'-f_n$ is equicontinuous. But as $f_n'$ and $f_n$ converge to $g$ and $f$, respectively, then $f_n'-f_n$ converges to $g-f$ uniformly. Let $M$ be a bounded for $f_n''-f_n'$. Let $a$ be the greatest lower bound and $b$ be the least upper bound of $A$. We then have that $-Me^{-a}\leq e^{-x}[f_n''(x)-f_n'(x)]=\frac{d}{dx}[e^{-x}f_n'(x)] < Me^{-a}$. That is, $\frac{d}{dx}[e^{-x}f_n'(x)]$ is uniformly bounded, and therefore $e^{-x}f_n'(x)$ is equicontinuous. But equicontinuity on a compact set implies uniform equicontinuity. As $f_n'\rightarrow g$, and $e^{-x}$ is bounded on $A$, $e^{-x}f_n'\rightarrow e^{-x}g$. But a convergent uniformly equicontinuous sequence of functions converges uniformly. Thus, $e^{-x}f_n'(x) \rightarrow e^{-x}g(x)$ uniformly, and therefore, as $e^{-x}$ is continuous and positive on $A$, $f_n'(x)\rightarrow g(x)$ uniformly. But also $f_n'-f_n \rightarrow g-f$ uniformly, and therefore $f_n \rightarrow f$ uniformly. Thus, $f'=g$.
        \end{proof}
        \begin{corollary}
            If $f_n' - f_n$ is uniformly bounded and if $f_n \rightarrow f$ on a closed and bounded set $A$, then the convergence is uniform.
        \end{corollary}
        \begin{proof}
            Using the inequality from the previous theorem, let $M$ be a bound for $f_n'-f_n$ and let $a$ be the least upper bound of $A$. Then $-Me^{-a}\leq \frac{d}{dx}[e^{-x}f_n] \leq Me^{-a}$. Thus $e^{-x}f_n$ is uniformly equicontinuous and therefore $e^{-x}f_n\rightarrow e^{-x}f$ uniformly, and thus $f_n\rightarrow f$ uniformly.
        \end{proof}
        \begin{corollary}
            If $f_n^{(N+1)}-f_n^{(N)}$ is bounded and a compact set, and if $f_n^{(k)}\rightarrow f_k$ for $k=0,1,\hdots, N$, then the convergence is uniform and $f_{k}' = f_{k+1}$ for $k=0,1,\hdots,N-1$.
        \end{corollary}
        \begin{proof}
            A simple induction and application of the previous theorem proves this.
        \end{proof}
    \section{On Analyticity}
        We deal with functions on intervals for simplicity.
        \begin{definition}
            A real-valued function $f$ is said to be smooth, denoted $f\in C^{\infty}$ if, for all $k$, $\frac{d^k}{dx^k}f(x) \equiv f^{(k)}(x)$ exists.
        \end{definition}
    \begin{theorem}[Taylor's Theorem]
    If $f\in C^{\infty}$, on some interval $[a,b]$, and if $x_0\in (a,b)$, then $f(x) - \sum_{k=0}^{n} f^{(k)}(x_0)\frac{(x-x_0)^k}{k!} = \int_{x_0}^{x} f^{(n+1)}(t)\frac{(x-t)^n}{n!}dt$
    \end{theorem}
    \begin{proof}
    We prove by induction. The base case says $f(x)-f(x_0) = \int_{x_0}^{x} f'(t)dt$, which is true. Suppose it holds for some $n\in \mathbb{N}$. Then $f(x)-\sum_{k=0}^{n+1} f^{(k)}(x_0)\frac{(x-x_0)^k}{k!} = f(x)-\sum_{k=0}^{n} f^{(k)}(x_0)\frac{(x-x_0)^k}{k!} - f^{(n+1)}(x)\frac{(x-x_0)^{n+1}}{(n+1)!} = \int_{x_0}^{x} f^{(n+1)}(t)\frac{(x-t)^n}{n!}dt - f^{(n+1)}(x)\frac{(x-x_0)^{n+1}}{(n+1)!}$. But $\int_{x_0}^{x} f^{(n+1)}(t)\frac{(x-t)^n}{n!}dt =  \int_{x_0}^{x} f^{(n+2)}(t) \frac{(x-t)^{n+1}}{(n+1)!} dt + f^{(n+1)}(x)\frac{(x-x_0)^{n+1}}{(n+1)!}$, from integration by parts. Thus, $f(x)-\sum_{k=0}^{n+1} f^{(k)}(x_0)\frac{(x-x_0)^k}{k!}= \int_{x_0}^{x} f^{(n+2)}(t) \frac{(x-t)^{n+1}}{(n+1)!} dt$
    \end{proof}
    \begin{lemma}
    If $f\in C^{\infty}$ and $f^{(n)}(x)\rightarrow 0$ (Point-wise) on $[a,b]$, and if $F(x) \equiv f(x)-\sum_{k=0}^{\infty} f^{(k)}(x_0)\frac{(x-x_0)^{k}}{k!}$, where $x_0\in [a,b]$ is fixed, then $\int_{x_0}^{x} F^{(n+1)}(t)\frac{(x-t)^{n}}{n!}dt$ converges. 
    \end{lemma}
    \begin{proof}
    For let $x,x_0\in [a,b]$ fixed. We will show that $\int_{x_0}^{x} F^{(n+1)}(t)\frac{(x-t)^{n}}{n!}dt$ is Cauchy. Let $\varepsilon>0$, $N_0 = 1$, and let $n>m>N_0$ be arbitrary. We have that $F(x) = \bigg(f(x)-\sum_{k=0}^{N} f^{(k)}(x_0)\frac{(x-x_0)^{k}}{k!}\bigg)-\bigg(g(x)-\sum_{k=0}^{N} f^{(k)}(x_0)\frac{(x-x_0)^{k}}{k!}\bigg)$, where $N\in \mathbb{N}$ is arbitrary. From Taylor's Theorem we thus have $F(x) = \int_{x_0}^{x}F^{N+1}(t)\frac{(x-t)^N}{N!}dt$. Then $|\int_{x_0}^{x}F^{n+1}(t)\frac{(x-t)^n}{n!}dt-\int_{x_0}^{x}F^{m+1}(t)\frac{(x-t)^m}{m!}dt| = |F(x)-F(x)|= 0 <\varepsilon$. 
    \end{proof}
    \begin{theorem}
    If $f\in C^{\infty}$ and $f^{(n)}(x)\rightarrow 0$ (Point-wise) on some interval $[a,b]$, then $f^{(n)}(x)$ is uniformly bounded.
    \end{theorem}
    \begin{proof}
    For let $x_0\in (a,b)$ be arbitrary. As $f^{(n)}(x_0)\rightarrow 0$, $\sum_{k=0}^{\infty} f^{(k)}(x_0)\frac{(x-x_0)^{k}}{k!}$ converges everywhere. Let $g(x)\equiv \sum_{k=0}^{\infty} f^{(k)}(x_0)\frac{(x-x_0)^{k}}{k!}$. Define $F(x) = f(x)-g(x)$. Then:
    \begin{align*}
        F^{(n)}(x) &= f^{(n)}(x)-g^{(n)}(x)\\
        &= \bigg(f^{(n)}(x)-\sum_{k=n}^{N} f^{(k)}(x_0)\frac{(x-x_0)^{k}}{k!}\bigg)-\bigg(g^{(n)}(x)-\sum_{k=n}^{N} f^{(k)}(x_0)\frac{(x-x_0)^{k}}{k!}\bigg)    
    \end{align*}
    From Taylor's theorem, this is equal to:
    \begin{align*}
        \int_{x_0}^{x} f^{(N+n+1)}(t)\frac{(x-t)^{N+n}}{(N+n)!}dt &- \int_{x_0}^{x} g^{(N+n+1)}(t)\frac{(x-t)^{N+n}}{(N+n)!}dt\\
        &= \int_{x_0}^{x} F^{(N+n+1)}(t)\frac{(x-t)^{N+n}}{(N+n)!}dt    
    \end{align*}
    That is, for all $N>n$, $F^{(n)}(x) = \int_{x_0}^{x} F^{(N+n+1)}(t)\frac{(x-t)^{N+n}}{(N+n)!}dt$. But for all $x_1 \in (a,b)$:
    \begin{equation*}
        F^{(n)}(x)-\sum_{k=n}^{N} F^{(k)}(x_1)\frac{(x-x_1)^k}{k!} = \int_{x_1}^{x} F^{(N+n+1)}(t)\frac{(x-t)^{N+n}}{(N+n)!}dt    
    \end{equation*}
    Now, suppose $f^{(n)}(x)$ is not uniformly bounded. $g^{(n)}(x)$ is uniformly bounded by its definition, and thus $F^{(n)}(x)$ is not uniformly bounded. Let ${k_n}$ be a subsequence of $n$ such that $F^{(k_n)}(x_{k_n})>n$. Such a sequence exists as $F^{(n)}(x)$ is not uniformly bounded. As $[a,b]$ is closed and bounded, it is compact. Thus $x_{k_n}$ has a convergent subsquence $\varphi(x_{k_n})$ (We use this notation so as to avoid writing $x_{k_{m_n}}$). Let $x_1$ be the limit of this subsequence. As $F^{(n)}(x_1)\rightarrow 0$, $\sum_{k=n}^{N} F^{(k)}(x_1)\frac{(x-x_1)^k}{k!}$ converges. Let $M$ be a bound for $F^{(k)}(x_1)$. Such a bound exists as this sequence converges. As $F^{(n)}(x) = \int_{x_0}^{x} F^{(N+n+1)}(t)\frac{(x-t)^{N+n}}{(N+n)!}dt$, we have that:
    \begin{equation*}
        \sum_{k=n}^{N} F^{(k)}(x_1)\frac{(x-x_1)^k}{k!} = -\int_{x_0}^{x_1} F^{(N+n+1)}(t)\frac{(x-t)^{N+n}}{(N+n)!}dt    
    \end{equation*}
    Thus, for all $n$ and $N$:
    \begin{equation*}
        |\int_{x_0}^{x_1} F^{(N+n+1)}(t)\frac{(x-t)^{N+n}}{(N+n)!}dt|\leq Me^{b-a}
    \end{equation*}
    Thus, we have that:
    \begin{align*}
        |F^{(n)}(x)| &= |\int_{x_0}^{x} F^{(N+n+1)}(t)\frac{(x-t)^{N+n}}{(N+n)!}dt|\\ &= |\int_{x_0}^{x_1} F^{(N+n+1)}(t)\frac{(x-t)^{N+n}}{(N+n)!}dt+\int_{x_1}^{x} F^{(N+n+1)}(t)\frac{(x-t)^{N+n}}{(N+n)!}dt|\\
        &\leq Me^{b-a}+|\int_{x_1}^{x} F^{(N+n+1)}(t)\frac{(x-t)^{N+n}}{(N+n)!}dt|
    \end{align*}
    But as $N$ is arbitrary, we may take it to be large enough to make the latter term close to a fixed finite value for each point. Thus $F^{(n)}(\varphi(x_{k_n}))\not\rightarrow \infty$ and therefore $F^{(n)}(x)$ is not unbounded, and is therefore uniformly bounded. Thus $f^{(n)}(x)$ is uniformly bounded.
    \end{proof}
    \begin{definition}
    An analytic function about a point $x_0$ is a function $f:\mathcal{U}\rightarrow\mathbb{R}$ such that $f(x) = \sum_{n=0}^{\infty} f^{n}(x_0) \frac{(x-x_0)^{n}}{n!}$ for all $x\in\mathcal{U}$.
    \end{definition}
    \begin{theorem}[Lagrange's Remainder Theorem]
    A function $f(x)$ is analytic if and only if $\int_{x_0}^{x}f^{n+1}(t)\frac{(x-t)^n}{n!}dt\rightarrow 0$.
    \end{theorem}
    \begin{proof}
    For if $f(x)$ is analytic, then $f(x)-\sum_{k=0}^{n} f^{(k)}(x_0)\frac{(x-x_0)^n}{n!} = \int_{x_0}^{x}f^{n+1}(t)\frac{(x-t)^n}{n!}dt \rightarrow 0$. If $\int_{x_0}^{x}f^{n+1}(t)\frac{(x-t)^n}{n!}dt\rightarrow 0$, then $f(x)-\sum_{k=0}^{n}f^{(k)}\frac{(x-x_0)^{k}}{k!}\rightarrow 0$, and thus $f(x)$ is analytic.
    \end{proof}
    \begin{lemma}
    If $f\in C^{\infty}$ and $f^{(n)}$ is uniformly bounded, then it is analytic.
    \end{lemma}
    \begin{proof}
    For $|\int_{x_0}^{x}f^{n+1}(t)\frac{(x-t)^n}{n!}dt|\leq \int_{x_0}^{x}|f^{n+1}(t)||\frac{(x-t)^n}{n!}|dt$. As $f^{(n)}(x)$ is uniformly bounded, and for all $x$ $\frac{(x-x_0)^n}{n!} \rightarrow 0$, we have that $\int_{x_0}^{x}f^{n+1}(t)\frac{(x-t)^n}{n!}dt\rightarrow 0$.
    \end{proof}
    \begin{corollary}
    If $f^{(n)}(x)\rightarrow 0$, then $f$ is analytic.
    \end{corollary}
    \begin{proof}
    For $f^{(n)}(x)$ is thus uniformly bounded, and therefore analytic.
    \end{proof}
    \section{On Infinite Order O.D.E.'s}
    \begin{definition}
    An infinite order O.D.E. is a differential equation with no largest order of derivative.
    \end{definition}
    \begin{remark}
    An infinite order O.D.E. then necessarily has an infinite number of terms.
    \end{remark}
    \begin{definition}
    A linear infinite order O.D.E. is a differential equation of the form $\sum_{n=0}^{\infty} a_n(x) \frac{d^n f}{dx^n} = F(x)$.
    \end{definition}
    \begin{remark}
    Unlike normal differential equation of order $n\in \mathbb{N}$, infinite order differential equations have the problem of convergence. That is, $\sum_{n=0}^{\infty} a_n(x) \frac{d^n f}{dx^n} = F(x)$ may have a different solution set if point-wise convergence is considered rather than uniform.
    \end{remark}
    We now consider the main topic of the paper.
    \begin{proposition}
    Consider the following differential equation on some interval $(a,b)$:
    \begin{equation}
    \nonumber \sum_{n=0}^{\infty} \frac{d^n f}{dx^n} = 0
    \end{equation}
    Be the convergence uniform or point-wise, the only solution is $f(x)=0$
    \end{proposition}
    We will prove this via the tools we have developed in the previous sections. First, some preliminary results.
    \begin{theorem}
    If, for some open set $A$, $f:A\rightarrow \mathbb{R}$ is continuous and positive at some point $x_0$, then there exists and open interval $(a,b)$ that contains $x_0$ such that $f(x)>0$ on this interval.
    \end{theorem}
    \begin{proof}
    For let $A$ be open, let $f:A\rightarrow \mathbb{R}$ be continuous, and let $x_0\in A$ be such that $f(x_0)>0$. Let $\varepsilon = f(x_0)>0$. As $f$ is continuous, there is a $\delta>0$ such that $|x-x_0|<\delta$ and $x\in A$ implies $|f(x_0)-f(x)|<\varepsilon = f(x_0)$. As $A$ is open and $x_0\in A$ there is an $r>0$ such that $(x_0-r,x_0+r)\in A$. Then $(x_0-r,x_0+r)\cap (x_0-\delta,x_0+\delta)$ is an open interval in $A$ such that $0<f(x)<2f(x_0)$.
    \end{proof}
    \begin{theorem}[The Fundamental Theorem of the Calculus of Variations]
    If $f$ is a continuous function on $(a,b)$, and if for all $\alpha,\beta\in (a,b)$ $\int_{\alpha}^{\beta}f = 0$, then $f=0$.
    \end{theorem}
    \begin{proof}
    For suppose not. Let $f$ be positive at some point $x$. Then, as $f$ is continuous, there is a $\delta>0$ such that for all $x_0\in (x-\delta,x+\delta)\cap(a,b)$, $f(x_0)>0$. But then the integral on this subinterval is positive, a contradiction. Thus $f=0$.
    \end{proof}
    \begin{theorem}[Cauchy Criterion]
    If $\sum a_n$ converges, then $a_n \rightarrow 0$.
    \end{theorem}
    \begin{proof}
    For let $s_n$ be the $n^{th}$ partial sum. As convergent sequence are Cauchy sequences, $s_{n+1}-s_n \rightarrow 0\Rightarrow a_{n+1}\rightarrow 0$.
    \end{proof}
    \begin{theorem}
    If $\sum_{n=0}^{N} \frac{d^{n}f}{dx^n} \rightarrow 0$ uniformly on some interval $(a,b)$, then $f=0$.
    \end{theorem}
    \begin{proof}
    For any $\alpha, \beta\in (a,b)$, $\int_{\alpha}^{\beta} \sum_{n=0}^{N} \frac{d^{n}f}{dx^n} \rightarrow \int_{\alpha}^{\beta} 0 = 0$. Thus, $\int_{\alpha}^{\beta} f + \sum_{n=0}^{N-1} \frac{d^n f}{dx^n}\bigg|_{\alpha}^{\beta} \rightarrow 0$. As the latter term tends to $0$, $\int_{\alpha}^{\beta} f = 0$. As $\alpha$ and $\beta$ are arbitrary, $f=0$.
    \end{proof}
    \begin{theorem}
    If $\sum_{n=0}^{N} \frac{d^n f}{dx^n} \rightarrow 0$ point-wise on some interval $(a,b)$, then $f=0$.
    \end{theorem}
    \begin{proof}
    Suppose not. Let $x\in (a,b)$ be such that $f(x)\ne 0$. Consider the interval $[\frac{a+x}{2},\frac{x+b}{2}]=[\alpha,\beta]$ and let $S_N =\sum_{n=0}^{N} \frac{d^n f}{dx^n}$. Note that $S_N' = S_{N+1}-f$. So $S_N' - S_N = f^{(n+1)}-f$, and thus $|S_N'-S_N| = |f^{(n+1)}-f|$. As $\sum_{n=0}^{N} \frac{d^n f}{dx^n}$ converges, $\frac{d^n f}{dx^n} \rightarrow 0$. But then $f^{(n)}(x)$ is uniformly bounded on $[\alpha,\beta]$. Let $M_1$ be such a bound. As $f$ is continuous on $[\alpha,\beta]$ it is bounded. Let $M_2$ be such a bound. Let $M=M_1+M_2$. Then $|S_N'-S_N| = |f-f^{(N+1)}|\leq M$. That is, $|S_N'-S_N|$ is uniformly bounded. Therefore $S_N$ converges uniformly to zero. But if the convergence is uniform, then $f=0$. A contradiction. Thus $f$ is not nonzero anywhere, and therefore $f=0$.
    \end{proof}
    \begin{remark}
    $a$ and $b$ need not be finite. The theorem holds on all of $\mathbb{R}$. 
    \end{remark}
    \section{Other Results}
    \begin{theorem}
    A sum of $K$ continuous functions is continuous. 
    \end{theorem}
    \begin{proof}
    For let $f_n$, $n=1,2,\hdots,K$ be continuous, let $x$ be a point in their domains, and let $\varepsilon>0$ be given. Then, there is a $\delta_n$ such that $|x-x_0|<\delta_n$, with $x_0$ also in the domain, implies $|f_n(x)-f_n(x_0)|<\frac{\varepsilon}{K}$. Let $\delta = \min\{\delta_1,\hdots,\delta_n\}$. Then $|\sum_{n=1}^{K}[f_n(x)-f_n(x_0)]| \leq \sum_{n=1}^{K}|f_n(x)-f_n(x_0)| < \sum_{n=1}^{K} \frac{\varepsilon}{K} = \varepsilon$.
    \end{proof}
    \section{An Uninteresting Algebraic Structure}
    \subsubsection{Properties}
    We define a Pseudo-Field to be a set equipped with two operations $<S,\circ, *>$ satisfying the following axioms.
    $\forall a,b,c \in S$
    \begin{enumerate}
        \item $a\circ b = b\circ a$ \hfill Commutativity of the First Operation
        \item $a\circ (b\circ c)=(a \circ b)\circ c$ \hfill Associativity of the First Operation
        \item $a*b = b*a$ \hfill Commutativity of the Second Operation
        \item $a*(b*c) = (a*b)*c$ \hfill Associativity of the Second Operation
        \item $a*(b\circ c)=(a\circ b)*(a\circ c)$ \hfill The Second Operation Distributes over the First Operation
        \item $a\circ (b*c) = (a\circ b)*(a\circ c)$ \hfill The First Operation Distributes over the Second Operation
        \item $\exists e_{\circ}\in S|\ e_{\circ}\circ a = a$ \hfill Identity of the First Operation
        \item $\exists e_{*} \in S|\ e_{*}*a = a$ \hfill Identity of the Second Operation
        \item For all $a\in S$ there is an $a^{-1}\in S$ called the Pseudo-Inverse such that:
        \begin{enumerate}
            \item $a*a^{-1} = e_{\circ}$
            \item $a\circ a^{-1}=e_{*}$
        \end{enumerate}
    \end{enumerate}
    \begin{theorem} The identities are unique
    \end{theorem}
    \begin{proof} For suppose not. Suppose $e_{\circ}$ and $e_{\circ}'$ are identities not equal to each other. But then $e_{\circ}=e_{\circ}\circ e_{\circ}'=e_{\circ}'$. So the two are not unequal, and thus the identity is unique. Similarly for $e_{*}$.
    \end{proof}
    \begin{theorem} $e_{\circ}$ and $e_{*}$ are pseudo-inverses of each other.
    \end{theorem}
    \begin{proof} From identity, $e_{\circ}\circ e_{*}=e_{*}$ and $e_{*}*e_{\circ}=e_{\circ}$
    \end{proof}
    \begin{theorem} For any $a\in S$, $a*e_{\circ}=e_{\circ}$ and $a\circ e_{*}=e_{*}$
    \end{theorem}
    \begin{proof} By the definition of pseudo-inverses, we have $a*e_{\circ}=a*(a^{-1}*a)$, and from associativity and commutativity $a*(a^{-1}*a)=(a*a)*a^{-1}$. But from identity, we have $(a*a)*a^{-1}=[(a*a)\circ e_{\circ}]*a^{-1}=[(a*a^{-1})\circ (a*a^{-1})]*a^{-1}$. From the distributive property, $[(a*a^{-1})\circ (a*a^{-1})]*a^{-1}=[a*(a\circ a^{-1})]*a^{-1}=(a*e_{*})*a^{-1}=a*a^{-1}=e_{\circ}$. Similarly for $a\circ e_{*}=e_{*}$
    \end{proof}
    \begin{theorem} For any $a\in S$, $a*a = a\circ a = a$.
    \end{theorem}
    \begin{proof} Let $a\in S$. Then $a=a*e_{*}=a*(a\circ a^{-1})=(a*a)\circ(a*a^{-1})=(a*a)\circ e_{\circ}=a*a$. Similarly, $a=a\circ a$.
    \end{proof}
    \begin{theorem} If $a\circ b = a*b = a$, then $b=a$. 
    \end{theorem}
    \begin{proof}
    For $b = b*(a\circ a^{-1}) = (b*a)\circ(b* a^{-1})= a\circ (b* a^{-1}) = (a\circ b)*(a\circ a^{-1}) = a$.
    \end{proof}
    \begin{theorem} The pseudo-inverses are unique.
    \end{theorem}
    \begin{proof} For suppose not. Suppose $a^{-1}$ and $a'^{-1}$ are both pseudo-inverses for some $a\in S$ not equal to each other.  Then $a*a^{-1}=a* a'^{-1}=e_{\circ}$. And $a\circ a^{-1}=a\circ a'^{-1}=e_{*}$. So then $a^{-1}=a^{-1}*(a\circ a'^{-1})=(a^{-1}*a)\circ (a^{-1}*a'^{-1})$ from the distributive property. Thus, from the property of pseudo-inverses and identity $a^{-1}=e_{\circ}\circ (a^{-1}*a'^{-1})=a^{-1}*a'^{-1}$. Similarly, $a'^{-1}=a'^{-1}*(a\circ a^{-1})=(a'^{-1}*a)\circ (a'^{-1}*a^{-1})=a'^{-1}*a^{-1}$. But it was just proven that $a^{-1}=a'^{-1}*a^{-1}$. So $a^{-1}=a'^{-1}$. The pseudo-inverse is unique.
    \end{proof}
    \begin{theorem} If for some $a\in S$, if $a=a^{-1}$, then $a=e_{\circ}=e_{*}$
    \end{theorem}
    \begin{proof} For let $a\in S$ and let $a=a^{-1}$. Then $a=a*a=a*a^{-1}$ from theorem 1.4. So $a=a*a^{-1}=e_{\circ}$. Similarly, $a=a\circ a^{-1} = e_{*}$
    \end{proof}
    \begin{theorem} For $a\in S$, $(a^{-1})^{-1} =a$.
    \end{theorem}
    \begin{proof} For we have $a = a\circ (a^{-1}* (a^{-1})^{-1}) = (a\circ a^{-1})*(a\circ (a^{-1})^{-1}) =a \circ (a^{-1})^{-1}$. Similarly, $a = a* (a^{-1})^{-1}$. But if $a = a\circ (a^{-1})^{-1} = a*(a^{-1})^{-1}$, then $a = (a^{-1})^{-1}$.
    \end{proof}
    \begin{definition} For $a\in S$, an inverse, or normal inverse, of the First Operation is an element $b\in S$ such that $a\circ b=e_{\circ}$. An inverse of the Second Operation is similarly defined. The normal inverses are denoted $a^{*}$ and $a^{\circ}$.
    \end{definition}
    \begin{theorem} If $a\in S$ has a normal inverse for either operation, than it is unique.
    \end{theorem}
    \begin{proof} For suppose not. Let $a\in S$ have a normal inverse for the First Operation. That is, there is an $a^{\circ}\in S$ such that $a\circ a^{\circ}=e_{\circ}$ and let $a'^{\circ}$ be a second normal inverse not equal to the first. But then $a^{\circ}=a^{\circ}\circ e_{\circ}=a^{\circ}\circ (a\circ a'^{\circ})$ and from associativity we have $a^{\circ}=(a^{\circ}\circ a)\circ a'^{\circ}=a'^{\circ}$. Thus, the normal inverse is unique. Similarly if there is an inverse for the Second Operation
    \end{proof}
    \begin{theorem} If $a\in S$ has a normal inverse, say $a'$, for one operation, then $a^{-1}=a'^{-1}$.
    \end{theorem}
    \begin{proof} For let $a\in S$ have a normal inverse $a'$ for the First Operation. That is, $a\circ a' = e_{\circ}$. But $a' \circ a'^{-1}=e_{*}$, and from theorem 1.3 $a\circ e_{*}=e_{*}$. So $a\circ (a' \circ a'^{-1})=e_{*}$. And from theorem 1.4, $a\circ a=a$, so we have $(a\circ a)\circ (a'\circ a'^{-1}=a\circ (a\circ a')\circ a'^{-1}=a\circ a'^{-1}=e_{*}$. But $a\circ a^{-1}=e_{\circ}$. And pseudo-inverses are unique. Thus, $a^{-1}=a'^{-1}$. 
    \end{proof}
    \begin{theorem} The identities have normal inverses for their respective operations.
    \end{theorem}
    \begin{proof} As normal inverses are unique, it suffices to find inverses for both identities. But $e_{\circ}\circ e_{\circ}=e_{\circ}$, so $e_{\circ}$ is its own inverse for the First Operation. Similarly, $e_{*}*e_{*}=e_{*}$.
    \end{proof}
    \begin{theorem} \textbf{(The Not-A-Field Theorem)} Only the identities have normal inverses.
    \end{theorem}
    \begin{proof} For suppose not. Suppose $a\in S,\ a\ne e_{\circ},\ a\ne e_{*}$ and a has an inverse for the First Operation. That is $\exists a^{\circ}\in S|\ a\circ a^{\circ}=e_{\circ}$. But by theorem 1.4, $a\circ a^{\circ}=(a\circ a)\circ a^{\circ}$. By associativity, we have $e_{\circ}=a\circ a^{\circ} = a\circ (a\circ a^{\circ})=a\circ e_{\circ}=a$. Thus, $a=e_{\circ}$. But by hypothesis, $a\ne e_{\circ}$. Thus, there is no inverse for $a$. Similarly, a has no inverse for the Second Operation.
    \end{proof}
    \begin{theorem}
    There exist pseudo-fields with only one element.
    \end{theorem}
    \begin{proof}
    For let $e_{\circ} = e_{*}$, and let no other elements be in the set. 
    \end{proof}
    \begin{theorem}
    A pseud-field has one element if and only if $e_{\circ} = e_{*}$.
    \end{theorem}
    \begin{proof}
    For suppose there is another element $a \ne e_{\circ}$. But then $a \circ e_{\circ} = a$, but also $a \circ e_{\circ} = a \circ e_{*} = e_{*}$. So $a = e_{*}$. If there is only one element, then clearly $e_{\circ} = e_{*}$ as otherwise there would be two elements.
    \end{proof}
    \begin{definition} A generating set on a pseudo-field is a subset $g_S \subset S$ such that every element of $S$ can be written as a finite combination of elements in $g_S$ using $\circ$ or $*$.
    \end{definition}
    \begin{theorem}
    The number of elements in a finite pseudo-field is a power of 2.
    \end{theorem}
    \begin{proof}
    Consider the set of all generators $g_S$ on $S$. Clearly for all such generators, $1\leq |g_S|\leq |S|$. Let $G$ be the smallest generator, such that $|G| \leq |g_S|$ for any other given generator. 
    \end{proof}
    \section{An Almost Group}
    \begin{definition}
    A group is a set $G$ with an operation $*$ satisfying the following:
    \begin{enumerate}
        \item $a*(b*c) = (a*b)*c$ for all $a,b,c\in G$
        \item There is an $e\in G$ such that $a*e=e*a = a$ for all $a\in G$
        \item For all $a\in G$ there is an $a^{-1}\in G$ such that $a*a^{-1}=a^{-1}*a = e$
    \end{enumerate}
    \end{definition}
    \begin{theorem}
    The identity of a group is unique.
    \end{theorem}
    \begin{proof}
    Suppose not, and let $e'$ be a different identity. But $e' = e'*e = e$. Thus $e$ is unique.
    \end{proof}
    \begin{definition}
    A quasigroup is a group but the operation need not be associative.
    \end{definition}
    \begin{definition}
    An Abelian Quasigroup is a quasigroup with a commutative operation.
    \end{definition}
    An interesting thing to note is that $e$ is an identity for $all$ elements of $G$. There are, however, groups with elements $a,b$ such that $a*b = b*a = a$, and yet $b\ne e$. They key difference is that $a*b$ does not necessarily equal $a$ for $all$ $a\in G$. 
    \begin{theorem}
    There exist abelian quasigroups $\langle G,*\rangle$ with elements $a,b\in G$ such that $a*b = b*a = a$, yet $b\ne e$.
    \end{theorem}
    \begin{proof}
    In a pathological construction, let $G=\mathbb{R}$. Consider the following operation:
    $x* y = \begin{cases} (x+y)^2, & x,y\ne 0 \\ x, & y=0,x\ne 0 \\ y, & x=0,y\ne 0 \\ 0, & x,y=0 \end{cases}$.
    The identity is zero. For $0*0 = 0$, and if $x\ne 0$, then $x*0 = 0*x = x$. The inverse is $-x$. For if $x\ne 0$, then $x*(-x) = (x-x)=0$. The operation is not associative, for $x*(y*z) = (x+(y+z)^2)^2 \ne ((x+y)^2+z)^2$, in general. For take $x=2$, $y=1$, and $z=1$. Then $x*(y*z) = 36$, but $(x*y)*z = 100$. It is, however, commutative. For if $x,y \ne 0$, then $x*y = (x+y)^2 = (y+x)^2 = y*x$. The case of either element being zero is identity, and thus commutative. Let $x=4$ and $y=-2$. Then $x*y = (4-2)^2 = 4=x$, $y*x = (-2+4)^2 = 4 = x$. Also, $4*(-6) = (-6)*4 = (4-2)^2 = (-2)^2 = 4$. Thus, $4$ has three "Identities," that is $0,-2,-6$. $4$ is the only element, for let $x \ne 0$. Then $y = x-\sqrt{x}$ and $y=-x-\sqrt{x}$ are also "Identities," for $x$. Thus, with the exception of $0$ and $1$, every positive element has three "Identites." Note that $-2$ is only an "Identity," for the elements $4$ and $1$. Thus, for any other elements $x*(-2) \ne -2$. Thus, $-2$ is not a true identity.
    \end{proof}
    \section{On Sequences}
    \subsubsection{Some Fun Stuff}
    \begin{theorem}
    Given an enumeration $\{x_n\}_{n=1}^{\infty}$ of the rationals $\mathbb{Q}\cap [0,1]$, for all $\varepsilon>0$ there is a $k\in \mathbb{N}$ such that $|x_{k+1}-x_k|<\varepsilon$.
    \end{theorem}
    \begin{proof}
    For let $x_n$ be such an enumeration. Then, for all $n\in \mathbb{N}$, $0 \leq x_n \leq 1$.
    \end{proof}
    \begin{definition}
    The Fibonacci Numbers are formed by the sequence $F_{n+2}=F_{n+1}+F_{n}$, with $F_0=F_1 = 1$.
    \end{definition}
    \begin{definition}
    Two positive integers are said to be coprime if they share no common factors.
    \end{definition}
    \begin{theorem}
    Any two consecutive Fibonacci numbers are coprime.
    \end{theorem}
    \begin{proof}
    We have that $F_0=F_1 = 1$ and thus $F_2 = 2$, and also $F_3 = 3$. Suppose there is some integer $N\in \mathbb{N}$ such that $F_{N+2}$ and $F_{N+1}$ are not coprime. Then there is a least integer $n\in \mathbb{N}$ such that $F_{n+2}$ and $F_{n+1}$ are not corpime. That is, there are integers $a,b,c\in \mathbb{N}$ such that $F_{n+2} = ab$ and $F_{n+1} = ac$ where $b>c$. But then $F_{n} = F_{n+2} - F_{n+1} = a(b-c)$. Let $\alpha = b-c \in \mathbb{N}$. Then $F_n$ and $F_{n+1}$ are also not coprime. But this is impossible as $n$ is the least integer such that $F_{n+2}$ and $F_{n+1}$ are coprime, and $n-1<n$, a contradiction. Therefore there is no $N$ such that $F_{N+2}$ and $F_{n+1}$ are coprime. Consecutive Fibonacci numbers are coprime. 
    \end{proof}
    \begin{theorem}
    For all $N\in \mathbb{N}$, $\sum_{n=1}^{N} n\cdot n! = (N+1)!-1$.
    \end{theorem}
    \begin{proof}
    For $n\cdot n! = n\cdot n! + n! - n! = n!(n+1) - n!=(n+1)!-n!$. Thus, $\sum_{n=1}^{N} n\cdot n! = \sum_{n=1}^{N} (n+1)! -n! = (N+1)!-1$, as this is a telescoping series.
    \end{proof}
    \begin{theorem}
    If $f(x)$ is an increasing function on $[1,N+1]$, then $\sum_{n=2}^{N+1} f(n) \leq \int_{1}^{N+1} f(x) \leq \sum_{n=1}^{N} f(n)$.
    \end{theorem}
    \begin{proof}
    For $x\in [n,n+1]$, $f(n+1)\leq f(x)\leq f(n)$, as $f$ is decreasing. Thus $\int_{n}^{n+1} f(n+1)dx \leq \int_{n}^{n+1} f(x) dx \leq \int_{n}^{n+1} f(n)dx \Rightarrow f(n+1) \leq \int_{n}^{n+1}f(x)dx \leq f(n)$. Summing over this, we obtain $\sum_{n=1}^{N} f(n+1) \leq \int_{1}^{N+1} f(x) dx \leq \sum_{n=1}^{N} f(n)$. Finally, applying a shift of index to the leftmost term, $\sum_{n=2}^{N+1} \leq \int_{1}^{N+1}f(x)dx \leq \sum_{n=1}^{N} f(n)$. 
    \end{proof}
    \begin{corollary}
    If $f$ is decreasing, then $\int_{1}^{n+1} f(x)dx \leq \sum_{k=1}^{n+1} f(k) \leq \int_{1}^{n+1} f(x)dx + f(1)$
    \end{corollary}
    \begin{proof}
    For $\int_{1}^{n+1}f(x) dx \leq \sum_{k=1}^{n}f(k)\leq \sum_{k=1}^{n+1}$. But $\sum_{k=2}^{N+1} f(k) \leq \int_{1}^{n+1}f(x)dx$ so $\sum_{k=1}^{n+1}f(k) \leq \int_{1}^{n+1}f(x)dx +f(1)$. Combining these together gives the result.
    \end{proof}
    \begin{theorem}
    $\underset{n\rightarrow \infty}\lim \sum_{k=1}^{n} \frac{1}{n+k} = \ln(2)$.
    \end{theorem}
    \begin{proof}
    From the previous theorem, $\int_{1}^{n} \frac{1}{n+x} dx \leq \sum_{k=1}^{n} \frac{1}{n+k} \leq \frac{1}{n+1} + \int_{1}^{n} \frac{1}{n+x}dx$, and thus $\ln(n+x)\big|_{1}^{n+1} \leq \sum_{k=1}^{n} \frac{1}{n+k}\leq \frac{1}{n+1}+\ln(n+x)\big|_{1}^{n+1}\Rightarrow \ln(\frac{2n+1}{n+1})\leq \sum_{k=1}^{n} \frac{1}{n+k} \leq \ln(\frac{2n+1}{n+1})+\frac{1}{n+1}$. As $\frac{2n+1}{n+1}\rightarrow 2$ and as $\ln(x)$ is continuous, $\ln(\frac{2n+1}{n+1})\rightarrow \ln(2)$. But also $\frac{1}{n+1}\rightarrow 0$. Thus, by the squeeze theorem, $\sum_{k=1}^{n} \frac{1}{n+k} \rightarrow \ln(2)$.
    \end{proof}
    \begin{corollary}
    $\sum_{k=1}^{n}\frac{1}{\sqrt{k}}< 2\sqrt{n}$.
    \end{corollary}
    \begin{proof}
    From the theorem we have that $\sum_{k=1}^{n} \frac{1}{\sqrt{k}} \leq \int_{1}^{n}\frac{1}{\sqrt{x}}dx + 1 < \int_{1}^{n} \frac{1}{\sqrt{x}}dx +2 = 2\sqrt{n}-2+2 = 2\sqrt{n}$.
    \end{proof}
    \begin{lemma}
    If $x\mod 1 < \frac{1}{2}$, then $2\floor{x} = \floor{2x}$.
    \end{lemma}
    \begin{proof}
    Let $0\leq x \mod 1 \leq 0.5$. Then $0 \leq x-\floor{x}<0.5 \Rightarrow 2x-2\floor{x} <1$ and thus $2\floor{x} \leq \floor{2x} \leq 1+2\floor{x}$. But then we have that $0 \leq \floor{2x}-2\floor{x} <1$. But this is the difference of two integers, and is thus an integer. But there are no integers between $0$ and $1$, and therefore $\floor{2x}-2\floor{x} = 0$. Thus, $\floor{2x}=2\floor{x}$.
    \end{proof}
    \subsubsection{A Peculiar Family of Sequences and their Averages}
    Consider the sequence $1,2,1,1,3,1,1,1,4,1,1,1,1,5,\hdots, n,\hdots (n\ 1's)\hdots, n+1$ and also the generalization $1^k, 2^k,\hdots (2^k\ 1's)\hdots, 3^k, \hdots (3^k\ 1's)\hdots, n^k, \hdots (n^k\ 1's)\hdots, (n+1)^k$
    \begin{lemma}
    If $a_n, b_n$ are sequences, $a_n\rightarrow A$ and $a_n-b_n\rightarrow 0$, then $b_n \rightarrow A$.
    \end{lemma}
    \begin{proof}
    For $|A-b_n| \leq |A-a_n|+|a_n-b_n| \rightarrow 0$, thus $|A-b_n|\rightarrow 0$ and therefore $b_n \rightarrow A$.
    \end{proof}
    \begin{lemma}
    Let $a_n$ be a sequence and $f,g$ be strictly increasing integer valued functions such that for all $m<f(n)$, $a_{f(n)}>a_m$ and for all $m>g(n)$, $a_{g(n)}<a_m$. If $a_{f(n)}\rightarrow A$ and $a_{f(n)}-a_{g(n)}\rightarrow 0$, then $a_n \rightarrow A$.
    \end{lemma}
    \begin{proof}
    Let $\varepsilon>0$ be given. We have that $a_{g(n)}\rightarrow A$ as well from the previous lemma. Thus, there is an $N_1 \in \mathbb{N}$ such that for all $n>N_1$, $|A-a_{g(n)}|<\varepsilon$. Thus, for $n>N_1$, $A-\varepsilon < a_{g(n)}<A+\varepsilon$. But for all integers $n>g(N_1)$, $a_n >a_{g(N_1)}$, and thus $A-\varepsilon < a_n$ for all $n>g(N_1)$. As $a_{f(n)}\rightarrow A$, there is an $N_2$ such that for all $n>N_2$, $|A-a_{f(n)}|<\varepsilon$. Thus, for $n>N_2$, $A-\varepsilon < a_{f(n)}<A+\varepsilon$. As $f$ is a monotonically increasing function on the integers, $f(n)\geq n$. Thus, $a_{f(n)}>a_n$ for all $n$. But then for $n>\max\{g(N_1),N_2\}$, $A-\varepsilon < a_{g(n)} < a_n < a_{f(n)}<A-\varepsilon$. Thus, $a_n \rightarrow A$.
    \end{proof}
    \begin{lemma}
    If $f$ and $g$ are continuous functions defined on $\mathbb{R}^+$, and if $\underset{x\rightarrow \infty}\lim f(x) = \underset{x\rightarrow \infty}\lim g(x)=A$, and if $S = \{(x,y):x\in \mathbb{R}^+,\min\{f(x),g(x)\}\leq y \leq \max\{f(x),g(x)\}\}$, and if $a_n$ is any sequence such that $(n,a_n)\in S$ for all $n\in \mathbb{N}$, then $a_n \rightarrow A$.
    \end{lemma}
    \begin{proof}
    As $(n,a_n)\in S$:
    \begin{align*}
        \min\{f(n),g(n)\} &\leq a_n \leq \max\{f(n),g(n)\}\\
        \Rightarrow 0 &\leq a_n - \min\{f(n),g(n)\} \leq \max\{(f(n),g(n)\}-\min\{f(n),g(n)\}    
    \end{align*}
    But $\max\{f(n),g(n)\}-\min\{f(n),g(n)\} \rightarrow 0$, and thus $a_n - \min\{f(n),g(n)\} \rightarrow 0$. From the lemma, $a_n \rightarrow A$.
    \end{proof}
    \begin{lemma}
    If $P(x)$ and $Q(x)$ are polynomials of degree $n$, with leading coefficients $a_n$ and $b_n$, respectively, then $\underset{x\rightarrow \infty}\lim \frac{P(x)}{Q(x)} = \frac{a_n}{b_n}$.
    \end{lemma}
    \begin{proof}
    From repeated application of L'H\^{o}pital's Rule:
    \begin{equation*}
        \underset{x\rightarrow \infty}\lim \frac{P(x)}{Q(x)} = \underset{x\rightarrow \infty}\lim \frac{a_n x^n + \hdots + a_0}{b_n x^n + \hdots + b_0} = \underset{x\rightarrow \infty} \lim\frac{n! a_n}{n! b_n} = \frac{a_n}{b_n}
    \end{equation*}
    \end{proof}
    \begin{theorem}
    The average of the family of sequences we were considering is $2$. That is, let $a_n(k)$ be the $n^{th}$ term in the sequence $1^k, 2^k, \hdots (2^k\ 1's)\hdots,3^k,\hdots$, then the average $\frac{\sum_{n=1}^{N} a_n(k)}{N}$ converges to $2$ for all $k\geq 1$.
    \end{theorem}
    \section{A Class of Differentiability}
        \begin{definition}
            A function $f:(a,\infty)\rightarrow \mathbb{R}$, $a>0$, is said to be Kiwi Continuous if $f(x)-xf'(x)$ is bounded.
        \end{definition}
        \begin{remark}
            A function is Kiwi Continuous if the set of $y-$intercepts of the tangent lines of $f(x)$ is bounded.
        \end{remark}
        \begin{theorem}
            If $f:[a,\infty)\rightarrow \mathbb{R}$ is Kiwi Continuous, then $f'$ is bounded.
        \end{theorem}
        \begin{proof}
            By the definition, $-m \leq f(x)-xf'(x)\leq m$. Therefore $-\frac{m}{x^2} \leq \frac{f(x)}{x^2}- \frac{f'(x)}{x} \leq \frac{m}{x^2}$. But $\frac{f(x)}{x^2} - \frac{f'(x)}{x} = -\frac{d}{dx}\big(\frac{f(x)}{x}\big)$. So $-\frac{m}{x^2} \leq \frac{d}{dx}\big(\frac{f(x)}{x}\big) \leq \frac{m}{x^2}$. Let $x_0 \in (a,\infty)$. Then $-\int_{x_0}^x \frac{m}{\tau^2}d\tau = -\big[-\frac{m}{x}+ \frac{m}{x_0}\big] = \frac{m}{x}- \frac{m}{x_0} \leq \int_{x_0}^{x}\frac{d}{d\tau}\big(\frac{f(x)}{x}\big)d\tau = \frac{f(x)}{x} - \frac{f(x_0)}{x_0} \leq \int_{x_0}^{x} \frac{m}{\tau^2}d\tau = \frac{m}{x_0} - \frac{m}{x}$. So $\big|\frac{f(x)}{x}\big| \leq m|\frac{1}{x} - \frac{1}{x_0}| \leq m|\frac{2}{a}|$. Therefore $|f(x)| \leq 2\frac{m}{a}x$. But $|f(x) - xf'(x)| \leq m$. Thus $|f(x)-xf'(x)| \geq |f(x)| - x|f'(x)|$, and therefore $|f'(x)|  \leq \frac{m+|f(x)|}{x} \leq \frac{m+ \frac{2m}{a}x}{x} \leq \frac{m}{a} + \frac{2m}{a} = \frac{3m}{a}$. Therefore, $|f'(x)|$ is bounded.
        \end{proof}
    \section{Degenerate Fredholm Equations of the First Kind}
        \begin{definition}
            A Fredholm Equation of the first kind is an equation
            of the form:
            \begin{equation*}
                f(x)=\int_{a}^{b}g(x_{0})K(x,x_{0})dx_{0}
            \end{equation*}
        \end{definition}
        \begin{definition}
            A degenerate Fredholm of the First Kind is an
            equation of the form:
                \begin{equation*}
                    f(x)=\int_{a}^{b}g(x_{0})K_{1}(x)
                         K_{2}(x_{0})dx_{0}
                \end{equation*}
        \end{definition}
        \begin{theorem}
            If $f(x)=\int_{a}^{b}g(x_{0})K_{1}(x)K_{2}(x_{0})
            dx_{0}$, $f$ and $K_1$ are non-zero, and if $K_{2}$ is continuous and non-zero at some point $\xi\in(a,b)$,
            then there exists two solutions $g_{1}(x_{0})$ and
            $g_{2}(x_{0})$.
        \end{theorem}
        \begin{proof}
            If $f$ and $K_{1}$ are non-zero, then:
            \begin{equation*}
                f(x)=\int_{a}^{b}g(x_{0})K_{1}(x)
                     K_{2}(x_{0})dx_{0}
                    =K_{1}(x)\int_{a}^{b}g(x_{0})
                     K_{2}(x_{0})dx_{0}\Rightarrow
                \frac{f(x)}{K_{1}(x)}=
                \int_{a}^{b}g(x_{0})K_{2}(x_{0})dx_{0}
            \end{equation*}
            But $\int_{a}^{b}g(x_{0})K_{2}(x_{0})dx_{0}$ is a
            number $c\in\mathbb{R}$. Since $K_{2}$ is continuous
            and positive at a point $\xi\in(a,b)$, there is an
            $\varepsilon>0$ such that
            $\forall_{x\in B_{\varepsilon}(\xi)}$,
            $K_{2}(x)>\frac{K_{2}(\xi)}{2}$.
            Let $G_{r}(x)$ be defined as follows:
            \begin{equation}
                G_{r}(x)=\begin{cases}
                    0,&x\notin(\xi-\epsilon,\xi)\\
                    \frac{2r}{\varepsilon}(x-(\xi-\varepsilon)),
                    &x\in(\xi-\epsilon,\xi-\frac{\epsilon}{2})\\
                    \frac{2r}{\varepsilon}(x-\xi),&
                    x\in(\xi-\frac{\varepsilon}{2},\xi)
                \end{cases}
            \end{equation}
            Let $F(r)=\int_{a}^{b}G_{r}(x)K_{2}(x)dx$.
            Then we have:
            \begin{equation*}
                F(r)=\int_{a}^{b}G_{r}(x)K_{1}(x)dx
                    =\int_{\xi-\epsilon}^{\xi}G_{r}(x)K_{1}(x)dx
                \geq \frac{K_{1}(\xi)}{2}
                     \int_{\xi-\varepsilon}^{\xi}G_{r}(x)dx
                    =\frac{K_{1}(\xi)}{2}\frac{\varepsilon r}{2}
            \end{equation*}
            Therefore, $F(r)\rightarrow \infty$ as
            $r\rightarrow\infty$. Furthermore, $F(0) = 0$.
            Suppose $c>0$. Let $M=\{r\in\mathbb{R}:c<F(r)\}$.
            $M$ is bounded below, for $0$ is such a bound. Then
            there exists a Greatest Lower Bound $\alpha$. From
            the continuity of $F$,
            $\underset{r\rightarrow\alpha}{\lim}F(r)=F(\alpha)$,
            and $F(\alpha)=c$. Therefore $G_{\alpha}(x)$ is a
            function such that:
            \begin{equation*}
                \int_{a}^{b}G_{\alpha}(x)K_{1}(x)dx=c
            \end{equation*}
            For the second function, repeat the argument on the
            interval $(\xi,\xi+\varepsilon)$
        \end{proof}
        \begin{theorem}
            There infinitely many solutions to degenerate
            Fredholm Equations of the First Kind.
        \end{theorem}
        \begin{proof}
            By the previous theorem, there are at least two. Let
            $g_{1}$ and $g_{2}$ be such solutions. Then, for all
            $\lambda\in\mathbb{R}$, define $G_{\lambda}$ by:
            $G_{\lambda}(x)=\lambda g_{1}(x)+(1-\lambda)g_{2}(x)$
            For all $\lambda\in\mathbb{R}$, $G_{\lambda}$ is a
            solution.
        \end{proof}
\end{document}