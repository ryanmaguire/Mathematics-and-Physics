%---------------------------Packages----------------------------%
\usepackage{geometry}
\geometry{a4paper, margin = 1.0in}
\usepackage[T1]{fontenc}
\usepackage{graphicx, float}            % Graphics/Images.
\graphicspath{{images/}}                % Path to Image Folder.
\usepackage{natbib}                     % For bibliographies.
\bibliographystyle{abbrvnat}            % Bibliography style.
\usepackage[francais, english]{babel}   % Language typesetting.
\usepackage[dvipsnames]{xcolor}         % Color names.
\usepackage{listings, lstlinebgrd}      % Verbatim-Like Tools.
\usepackage{mathtools, esint, mathrsfs} % amsmath and integrals.
\usepackage{amsthm, amsfonts}           % Fonts and theorems.
\usepackage{upgreek}                    % Non-Italic Greek.
\usepackage{paracol}                    % Two-column styling.
\usepackage{wrapfig}                    % Wrap text around figure.
\usepackage{fmtcount, etoolbox}         % For the \book{} command.
\usepackage[newparttoc]{titlesec}       % Formatting chapter, etc.
\usepackage{titletoc}                   % Allows \book in toc.
\usepackage[nottoc]{tocbibind}          % Bibliography in toc.
\usepackage[titles]{tocloft}            % ToC formatting.
\usepackage{multicol, enumitem}         % Multi-column/enumerate.
\usepackage{import}                     % Import external files.
\usepackage{pgfplots, tikz}             % Drawing/graphing tools.
\usetikzlibrary{
    calc,
    patterns,
    angles,
    quotes,
    arrows,
    arrows.meta,            % Latex,Stealth, and Triangle arrows.
    shapes,
    shapes.geometric,
    cd,
    hobby,
    positioning,            % Relative positioning of nodes.
    decorations.markings,   % Adding arrows in middle of a line.
    babel
}                                       % Libraries for tikz.
\pgfplotsset{compat=1.9}                % Version of pgfplots.
\usepackage[%
    font=scriptsize,
    labelformat=simple,
    labelsep=colon%
]{subcaption}                           % Subfigure captions.
\usepackage[
    font={scriptsize},
    hypcap=true,
    labelsep=colon
]{caption}                              % Figure captions.
\usepackage{hyperref}                   % Allows for hyperlinks.
\hypersetup{
    colorlinks=true,
    linkcolor=blue,
    filecolor=magenta,
    urlcolor=Cerulean,
    citecolor=SkyBlue
}                                       % Colors for hyperref.
\usepackage[
    toc,
    acronym,
    nogroupskip
]{glossaries}                           % Glossaries and acronyms.
\usepackage[
    subpreambles=false
]{standalone}                           % Complileable sub files.
%-------------------------Tikz Presets--------------------------%
% Define ring shading (Used for atmospheres around planets).
\pgfdeclareradialshading{myring}
    {\pgfpointorigin}
    {
        color(0cm)=(transparent!0);
        color(5mm)=(pgftransparent!50);
        color(1cm)=(pgftransparent!100)
    }
% Define shading name for myring shading.
\pgfdeclarefading{ringo}
    {\pgfuseshading{myring}}
%------------------------Theorem Styles-------------------------%
% Define theorem style for default spacing and normal font.
\newtheoremstyle{normal}
    {\topsep}               % Amount of space above the theorem.
    {\topsep}               % Amount of space below the theorem.
    {}                      % Font used for body of theorem.
    {}                      % Measure of space to indent.
    {\bfseries}             % Font of the header of the theorem.
    {}                      % Punctuation between head and body.
    {.5em}                  % Space after theorem head.
    {}
% Define theorem style for default spacing with italicized font.
\newtheoremstyle{normalit}
    {\topsep}               % Amount of space above the theorem.
    {\topsep}               % Amount of space below the theorem.
    {\itshape}              % Font used for body of theorem.
    {}                      % Measure of space to indent.
    {\bfseries}             % Font of the header of the theorem.
    {}                      % Punctuation between head and body.
    {.5em}                  % Space after theorem head.
    {}              
% Define theorem spacing for zero top and normal bottom spacing.
\newtheoremstyle{notop}
    {0em}                   % Amount of space above the theorem.
    {\topsep}               % Amount of space below the theorem.
    {}                      % Font used for body of theorem.
    {}                      % Measure of space to indent.
    {\bfseries}             % Font of the header of the theorem.
    {}                      % Punctuation between head and body.
    {.5em}                  % Space after theorem head.
    {}
% Define italicized version of notop theorem style.
\newtheoremstyle{notopit}
    {0em}                   % Amount of space above the theorem.
    {\topsep}               % Amount of space below the theorem.
    {\itshape}              % Font used for body of theorem.
    {}                      % Measure of space to indent.
    {\bfseries}             % Font of the header of the theorem.
    {}                      % Punctuation between head and body.
    {.5em}                  % Space after theorem head.
    {}
% Define theorem spacing for zero bottom and normal top spacing.
\newtheoremstyle{nobot}
    {\topsep}               % Amount of space above the theorem.
    {0em}                   % Amount of space below the theorem.
    {}                      % Font used for body of theorem.
    {}                      % Measure of space to indent.
    {\bfseries}             % Font of the header of the theorem.
    {}                      % Punctuation between head and body.
    {.5em}                  % Space after theorem head.
    {}
% Define italicized version of nobot theorem style.
\newtheoremstyle{nobotit}
    {\topsep}               % Amount of space above the theorem.
    {0em}                   % Amount of space below the theorem.
    {\itshape}              % Font used for body of theorem.
    {}                      % Measure of space to indent.
    {\bfseries}             % Font of the header of the theorem.
    {}                      % Punctuation between head and body.
    {.5em}                  % Space after theorem head.
    {}
% Normal-like environment with break between head and body.
\newtheoremstyle{break}
    {\topsep}               % Amount of space above the theorem.
    {\topsep}               % Amount of space below the theorem.
    {\itshape}              % Font used for body of theorem.
    {}                      % Measure of space to indent.
    {\bfseries}             % Font of the header of the theorem.
    {}                      % Punctuation between head and body.
    {\newline}              % Space after theorem head.
    {}
% Set theoremstyle to normal for various environments.
% Define numbered environments.
\theoremstyle{normalit}
\newtheorem{theorem}{Theorem}[section]
\newtheorem{lemma}{Lemma}[section]
\newtheorem{corollary}{Corollary}[section]
\newtheorem{proposition}{Proposition}[section]
\newtheorem*{theorem*}{Theorem}
\theoremstyle{normal}
\newtheorem{definition}{Definition}[section]
\newtheorem{problem}{Problem}[section]
\newtheorem{question}{Question}[section]
\newtheorem{remark}{Remark}[section]
\newtheorem{properties}{Properties}[section]
\newtheorem{notation}{Notation}[section]
\newtheorem{axiom}{Axiom}[section]
\newtheorem{example}{Example}[section]
% Define unnumbered environments.
\newtheorem*{definition*}{Definition}
\newtheorem*{properties*}{Properties}
\newtheorem*{remark*}{Remark}
%--------------------Declared Math Operators--------------------%
% Define reflection operator (Linear Algebra).
\DeclareMathOperator{\Refl}{Refl}
% Define span operator (Linear Algebra).
\DeclareMathOperator{\Span}{Span}
% Define cardinality operator (Set Theory).
\DeclareMathOperator{\Card}{Card}
% Define ordinal operator (Set Theory).
\DeclareMathOperator{\Ord}{Ord}
% Define Trace operator (Linear Algebra).
\DeclareMathOperator{\Tr}{Tr}
% Define Rank operator (Linear Algebra).
\DeclareMathOperator{\adjoint}{adj}
% Define Adjoint operator (Linear Algebra).
\DeclareMathOperator{\rk}{rk}
% Define Nul operator (Linear Algebra).
\DeclareMathOperator{\nul}{nul}
% Define sign operator (x/|x|, or 0. Common in Calculus).
\DeclareMathOperator{\sgn}{sgn}
% Define multidegree operator (Graph Theory).
\DeclareMathOperator{\multideg}{mutlideg}
% Define Greatest Common Denominator operator (Number Theory).
\DeclareMathOperator{\GCD}{GCD}
% Define Leading Monomial operator (Algebraic Geometry).
\DeclareMathOperator{\LM}{LM}
% Define Leading Coefficient operator (Algebraic Geometry).
\DeclareMathOperator{\LC}{LC}
% Define Leading Term operator (Algebraic Geometry).
\DeclareMathOperator{\LT}{LT}
% Define Least Common Multiple operator (Algebraic Geometry).
\DeclareMathOperator{\LCM}{LCM}
% Define Monomial operator (Algebraic Geometry)
\DeclareMathOperator{\Mon}{Mon}
% Define Spectrum operator (Algebraic Geometry).
\DeclareMathOperator{\Spec}{Spec}
% Define Projection operator (Calculus III, Linear Algebra).
\DeclareMathOperator{\proj}{proj}
% Define Component operator (Calculus III, Linear Algebra).
\DeclareMathOperator{\comp}{comp}
% Define Sinc operator (sin(x)/x, common in Calculus).
\DeclareMathOperator{\sinc}{sinc}
% Define Image operator. Linear algebra and topology.
\DeclareMathOperator{\Ima}{Im}
% Define Principal Operator. Topology/Algebra.
\DeclareMathOperator{\Prin}{Prin}
%------------------------New Commands---------------------------%
\DeclarePairedDelimiter\norm{\lVert}{\rVert}
\DeclarePairedDelimiter\ceil{\lceil}{\rceil}
\DeclarePairedDelimiter\floor{\lfloor}{\rfloor}
\newcommand*\diff{\mathop{}\!\mathrm{d}}
\newcommand*\Diff[1]{\mathop{}\!\mathrm{d^#1}}
\renewcommand*{\glstextformat}[1]{\textcolor{RoyalBlue}{#1}}
\renewcommand{\glsnamefont}[1]{\textbf{#1}}
\renewcommand\labelitemii{$\circ$}
\renewcommand\thesubfigure{%
    \arabic{chapter}.\arabic{figure}%
}
\renewcommand\thesubfigure{%
    \arabic{chapter}.\arabic{figure}.\arabic{subfigure}%
}
\addto\captionsenglish{\renewcommand{\figurename}{Fig.}}
\makeatletter
\renewcommand\@pnumwidth{1cm}
\newcounter{book}
\renewcommand\thebook{\@Roman\c@book}
\newcommand\book{%
    \if@openright
        \cleardoublepage
    \else
        \clearpage
    \fi
    \thispagestyle{plain}%
    \if@twocolumn
        \onecolumn
        \@tempswatrue
    \else
        \@tempswafalse
    \fi
    \null\vfil
    \secdef\@book\@sbook
}
\def\@book[#1]#2{%
    \ifnum \c@secnumdepth >-3\relax
        \refstepcounter{book}%
        \addcontentsline{toc}{book}{
            \bookname\ \thebook:\hspace{1em}#1
        }
    \else
        \addcontentsline{toc}{book}{#1}%
    \fi
    \markboth{}{}%
    {\centering
     \interlinepenalty \@M
     \normalfont
     \ifnum \c@secnumdepth >-2\relax
       \huge\bfseries \bookname\nobreakspace\thebook
       \par
       \vskip 20\p@
     \fi
     \Huge \bfseries #2\par}%
    \@endbook}
\def\@sbook#1{%
    {\centering
     \interlinepenalty \@M
     \normalfont
     \Huge \bfseries #1\par}%
    \@endbook}
\def\@endbook{
    \vfil\newpage
        \if@twoside
            \if@openright
                \null
                \thispagestyle{empty}%
                \newpage
            \fi
        \fi
        \if@tempswa
            \twocolumn
        \fi
}
\newcommand*\l@book[2]{%
    \ifnum \c@tocdepth >-2\relax
        \addpenalty{-\@highpenalty}%
        \addvspace{2.25em \@plus\p@}%
        \setlength\@tempdima{3em}%
        \begingroup
            \parindent \z@ \rightskip \@pnumwidth
            \parfillskip -\@pnumwidth
            {
                \leavevmode
                \Large \bfseries #1\hfil \hb@xt@\@pnumwidth{
                    \hss #2
                }
            }
            \par
            \nobreak
            \global\@nobreaktrue
            \everypar{\global\@nobreakfalse\everypar{}}%
        \endgroup
    \fi}
\newcommand\bookname{Book}
\renewcommand{\thebook}
    {
        \texorpdfstring{\Numberstring{book}}{book}
    }
\providecommand*{\toclevel@book}{-2}
\makeatother
\titlecontents{chapter}% <section-type>
    [0pt]% <left>
    {\bfseries}% <above-code>
    {
        \chaptername\ \thecontentslabel:\quad
    }% <numbered-entry-format>
    {}% <numberless-entry-format>
    {\hfill\contentspage}% <filler-page-format>
\titleformat{\part}[display]
    {\Large\bfseries}
    {\partname\nobreakspace\thepart}
    {0mm}
    {\Huge\bfseries}
    \titlecontents{part}[0pt]
    {\large\bfseries}
    {\partname\ \thecontentslabel: \quad}
    {}
    {\hfill\contentspage}
\newcommand{\MarkRightAngle}
    [4]
    [.3cm]% #1=size (optional), #2-#4 three points: \angle #2#3#4
    {
        \coordinate (tempa) at ($(#3)!#1!(#2)$);
        \coordinate (tempb) at ($(#3)!#1!(#4)$);
        \coordinate (tempc) at ($(tempa)!0.5!(tempb)$);%midpoint
        \draw (tempa) -- ($(#3)!2!(tempc)$) -- (tempb);
    }
%--------------------------LENGTHS------------------------------%
% Spacings for the Table of Contents.
\addtolength{\cftsecnumwidth}{1ex}
\addtolength{\cftsubsecindent}{1ex}
\addtolength{\cftsubsecnumwidth}{1ex}
\addtolength{\cftfignumwidth}{1ex}
\addtolength{\cfttabnumwidth}{1ex}
% Spacing for multi-column and enumerate environments.
\setlength{\multicolsep}{6pt}
\setlist[enumerate]{itemsep=0pt,topsep=3pt}
% Indent and paragraph spacing.
\setlength{\parindent}{0em}
\setlength{\parskip}{0em}