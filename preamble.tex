\usepackage{geometry}
\geometry{a4paper, margin = 1.0in}
\usepackage[utf8]{inputenc}
\usepackage[T1]{fontenc}
\usepackage{natbib}             %To use bibliographies.
\bibliographystyle{abbrvnat}    %Bibliography style.
\usepackage[francais,english]{babel}%Proper typesetting of French and English
\usepackage[dvipsnames]{xcolor} %Color names.
\usepackage{graphicx}           %Graphics/Images.
\usepackage{listings,lstlinebgrd}   %Verbatim-Like Tools.
\usepackage{mathtools}%amsmath and more.
\usepackage{amsfonts,amsthm,esint,mathrsfs} %Fonts, theorems, definitions, integrals.
\usepackage{paracol}            %Allows two columns style.
\usepackage{wrapfig}            %Wrap text around figure.
\usepackage{fmtcount,etoolbox}  %Used for the \book{}.
\usepackage[newparttoc]{titlesec}   %Formatting chapter, etc.
\usepackage{titletoc}           %Allows \book to be in toc.
\usepackage[nottoc]{tocbibind}  %Places the biblio in toc.
\usepackage[titles]{tocloft}    %Allows for toc formatting.
\usepackage[font={scriptsize,it}]{caption}  %Fig captions.
\usepackage{float}  %Allows for manual placement of figures.
\usepackage{multicol}           %Multi-column enumerate, etc.
\usepackage[font={scriptsize}]{subcaption}  %Subcaptions for subfigures.
\usepackage{pgfplots,tikz}      %Drawing/graphing tools.
\usetikzlibrary{
    calc,patterns,angles,quotes,
    arrows,arrows.meta,shapes,shapes.geometric,
    cd,hobby,positioning,decorations.markings,babel
    }                           %Libraries for tikz.
\pgfplotsset{compat=1.9}        %Version of pgfplots.
\usepackage{hyperref}           %Allows for hyperlinks.
\hypersetup{
    colorlinks=true,
    linkcolor=blue,
    filecolor=magenta,
    urlcolor=Cerulean,
    citecolor=SkyBlue
    }                           %Color schemes for hyperref.
\usepackage[toc,acronym,nogroupskip]{glossaries}    %Glossaries and acronyms.   
\usepackage{enumitem}           %\begin{enumerate} environment
\usepackage{upgreek}            %non-italic greek letters.
\usepackage[subpreambles=false]{standalone} %Allows sub-tex files to be compilable on their own.
\usepackage{import}             %Allows main.tex to call all sub-tex files for compilation.
\graphicspath{{images/}}        %Location of images folder.
%-------------Tikz Presets----------%
\pgfdeclareradialshading{myring}
{\pgfpointorigin}
{color(0cm)=(transparent!0);color(5mm)=(pgftransparent!50);color(1cm)=(pgftransparent!100)}
\pgfdeclarefading{ringo}
{\pgfuseshading{myring}}
%------------Theorem Styles---------%
\newtheoremstyle{normal}
{\topsep}
{\topsep}
{}
{}
{\bfseries}
{}
{.5em}
{}
\newtheoremstyle{normalit}{\topsep}{\topsep}{\itshape}{}
{\bfseries}{}{.5em}{}
\newtheoremstyle{notop}{0em}{\topsep}{}{}{\bfseries}{}{.5em}{}
\newtheoremstyle{notopit}{0em}{\topsep}{\itshape}{}
{\bfseries}{}{.5em}{}
\newtheoremstyle{nobot}{\topsep}{0em}{}{}{\bfseries}{}{.5em}{}
\newtheoremstyle{nobotit}{\topsep}{0em}{\itshape}{}
{\bfseries}{}{.5em}{}
\newtheoremstyle{break}{\topsep}{\topsep}{\itshape}{}
{\bfseries}{}{\newline}{}
\theoremstyle{normalit}
\newtheorem{theorem}{Theorem}[section]
\newtheorem*{theorem*}{Theorem}
\newtheorem{lemma}{Lemma}[section]
\newtheorem{corollary}{Corollary}[section]
\newtheorem{proposition}{Proposition}[section]
\theoremstyle{normal}
\newtheorem{definition}{Definition}[section]
\newtheorem{problem}{Problem}[section]
\newtheorem{question}{Question}[section]
\newtheorem{remark}{Remark}[section]
\newtheorem{properties}{Properties}[section]
\newtheorem{notation}{Notation}[section]
\newtheorem{axiom}{Axiom}[section]
\newtheorem*{definition*}{Definition}
\newtheorem*{properties*}{Properties}
\newtheorem*{remark*}{Remark}
\newtheorem{example}{Example}[section]
%--------Declared Math Operators----%
\DeclareMathOperator{\Refl}{Refl}
\DeclareMathOperator{\Span}{Span}
\DeclareMathOperator{\sgn}{sgn}
\DeclareMathOperator{\multideg}{mutlideg}
\DeclareMathOperator{\GCD}{GCD}
\DeclareMathOperator{\LC}{LC}
\DeclareMathOperator{\LT}{LT}
\DeclareMathOperator{\Tr}{Tr}
\DeclareMathOperator{\LM}{LM}
\DeclareMathOperator{\rk}{rk}
\DeclareMathOperator{\nul}{nul}
\DeclareMathOperator{\LCM}{LCM}
\DeclareMathOperator{\Mon}{Mon}
\DeclareMathOperator{\Spec}{Spec}
\DeclareMathOperator{\proj}{proj}
\DeclareMathOperator{\comp}{comp}
\DeclareMathOperator{\sinc}{sinc}
%-------------New Commands----------%
\DeclarePairedDelimiter\norm{\lVert}{\rVert}
\DeclarePairedDelimiter\ceil{\lceil}{\rceil}
\DeclarePairedDelimiter\floor{\lfloor}{\rfloor}
\renewcommand*{\glstextformat}[1]{\textcolor{RoyalBlue}{#1}}
\renewcommand{\glsnamefont}[1]{\textbf{#1}}
\renewcommand\labelitemii{$\circ$}
\renewcommand\thesubfigure{\arabic{chapter}.\arabic{figure}.\arabic{subfigure}}
\makeatletter
\renewcommand\@pnumwidth{1cm}
\newcounter {book}
\renewcommand \thebook {\@Roman\c@book}
\newcommand\book{%
  \if@openright
    \cleardoublepage
  \else
    \clearpage
  \fi
  \thispagestyle{plain}%
  \if@twocolumn
    \onecolumn
    \@tempswatrue
  \else
    \@tempswafalse
  \fi
  \null\vfil
  \secdef\@book\@sbook}
\def\@book[#1]#2{%
    \ifnum \c@secnumdepth >-3\relax
      \refstepcounter{book}%
      \addcontentsline{toc}{book}{\bookname\ \thebook:\hspace{1em}#1}%
    \else
      \addcontentsline{toc}{book}{#1}%
    \fi
    \markboth{}{}%
    {\centering
     \interlinepenalty \@M
     \normalfont
     \ifnum \c@secnumdepth >-2\relax
       \huge\bfseries \bookname\nobreakspace\thebook
       \par
       \vskip 20\p@
     \fi
     \Huge \bfseries #2\par}%
    \@endbook}
\def\@sbook#1{%
    {\centering
     \interlinepenalty \@M
     \normalfont
     \Huge \bfseries #1\par}%
    \@endbook}
\def\@endbook{\vfil\newpage
              \if@twoside
               \if@openright
                \null
                \thispagestyle{empty}%
                \newpage
               \fi
              \fi
              \if@tempswa
                \twocolumn
              \fi}
\newcommand*\l@book[2]{%
  \ifnum \c@tocdepth >-2\relax
    \addpenalty{-\@highpenalty}%
    \addvspace{2.25em \@plus\p@}%
    \setlength\@tempdima{3em}%
    \begingroup
      \parindent \z@ \rightskip \@pnumwidth
      \parfillskip -\@pnumwidth
      {\leavevmode
       \Large \bfseries #1\hfil \hb@xt@\@pnumwidth{\hss #2}}\par
       \nobreak
         \global\@nobreaktrue
         \everypar{\global\@nobreakfalse\everypar{}}%
    \endgroup
  \fi}
\newcommand\bookname{Book}
\renewcommand{\thebook}{\texorpdfstring{\Numberstring{book}}{book}}
\providecommand*{\toclevel@book}{-2}
\makeatother

\titlecontents{chapter}% <section-type>
  [0pt]% <left>
  {\bfseries}% <above-code>
  {\chaptername\ \thecontentslabel:\quad}% <numbered-entry-format>
  {}% <numberless-entry-format>
  {\hfill\contentspage}% <filler-page-format>

\titleformat{\part}[display]
{\Large\bfseries}
{\partname\nobreakspace\thepart}
{0mm}
{\Huge\bfseries}
\titlecontents{part}[0pt]
{\large\bfseries}
{\partname\ \thecontentslabel: \quad}
{}
{\hfill\contentspage}

\newcommand{\MarkRightAngle}[4][.3cm]% #1=size (optional), #2-#4 three points: \angle #2#3#4
{\coordinate (tempa) at ($(#3)!#1!(#2)$);
\coordinate (tempb) at ($(#3)!#1!(#4)$);
\coordinate (tempc) at ($(tempa)!0.5!(tempb)$);%midpoint
\draw (tempa) -- ($(#3)!2!(tempc)$) -- (tempb);}
%---------------LENGTHS-------------%
\addtolength{\cftsecnumwidth}{1ex}
\addtolength{\cftsubsecindent}{1ex}
\addtolength{\cftsubsecnumwidth}{1ex}
\addtolength{\cftfignumwidth}{1ex}
\addtolength{\cfttabnumwidth}{1ex}
\setlength{\multicolsep}{6pt}
\setlist[enumerate]{itemsep=0pt,topsep=3pt}
\setlength{\parindent}{0em}
\setlength{\parskip}{0em}