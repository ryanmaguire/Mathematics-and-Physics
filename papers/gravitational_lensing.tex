\documentclass[12pt,reqno,a4paper]{amsart}
\usepackage{amssymb,amscd}
\usepackage{hyperref}

\begin{document}
    \def\ss{{X}}
    \renewcommand{\Im}{\mathop{\mathrm{Im}}\nolimits}
    \renewcommand{\Re}{\mathop{\mathrm{Re}}\nolimits}
    \newcommand{\alk}{\mathop{\mathrm{alk}}\nolimits}
    \newcommand{\lk}{\mathop{\mathrm{lk}}\nolimits}

    \newtheorem{mainthm}{Theorem}
    \renewcommand{\themainthm}{{\Alph{mainthm}}}
    \newtheorem{thm}{Theorem}[subsection]


    \theoremstyle{definition}
    \newtheorem{exm}[thm]{Example}
    \newtheorem{rem}[thm]{Remark}
    \newtheorem{df}[thm]{Definition}
    \renewcommand{\thesubsection}{\arabic{subsection}}


    \title[Affine linking number estimates]{%
        Affine linking number estimates for the
        number of times an observer sees a star%
    }
    \author{Vladimir Chernov and Ryan Maguire}
    \address{%
        6188 Kemeny Hall, Department of Mathematics,
        Dartmouth College, Hanover, NH 03755, USA%
    }
    \email{Vladimir.Chernov@dartmouth.edu}
    \address{%
        6188 Kemeny Hall, Department of Mathematics,
        Dartmouth College, Hanover, NH 03755, USA%
    }
    \email{Ryan.Maguire@dartmouth.edu}
    \begin{abstract}
        Affine linking numbers are the generalization of linking numbers to
        the case of nonzero homologous linked submanifolds. They were
        introduced by Rudyak and the first author who used them to study
        causality in globally hyperbolic spacetimes.

        In this paper we use affine linking numbers to estimate the number
        of times an observer sees light from a star, that is how many copies
        of the star do they see on the sky due to gravitational lensing.
    \end{abstract}
    \maketitle
    We work in the $C^{\infty}$-category and the word {\it smooth\/}
    means $C^{\infty}.$ All the manifolds, maps etc.~are
    assumed to be smooth unless the opposite is explicitly stated. The
    manifolds are assumed to be oriented.
    \subsection{Lorentz geometry: conventions and definitions\/}
        In this section we introduce the basic conepts of Lorentz geometry
        following the paper Nemirovski and the first
        author~\cite{ChernovNemirovski}.
        \par\hfill\par
        ``Let $(\ss^{m+1}, g)$ be an $(m+1)$-dimensional Lorentz manifold and
        $p\in \ss$. A nonzero ${\bf v}\in T_p\ss$ is called {\it timelike,
        nonspacelike, null, or spacelike\/} if $g({\bf v}, {\bf v})$ is
        respectively negative, non-positive, zero or positive. A piecewise
        smooth curve is timelike if all of its velocity vectors are timelike.
        Nonspacelike and null curves are defined similarly. Since $(\ss, g)$
        has a unique Levi-Cevita connection, see for
        example~\cite[page 22]{BeemEhrlichEasley}, we can talk about spacelike,
        timelike and null geodesics. A submanifold $M\subset \ss$ is
        {\it spacelike} if $g$ restricted to $TM$ is a Riemann metric.
        \par\hfill\par
        All nonspacelike vectors in $T_p\ss$ form a cone consisting of two
        hemicones, and the continuous with respect to $p\in \ss$ choice of
        one of the two hemicones is called the {\it time orientation\/} of
        $(\ss, g)$. The vectors from the chosen hemicones are called
        \textit{future pointing}.
        A time oriented Lorentz manifold is called a {\it spacetime\/} and
        its points are called events."
        \par\hfill\par
        For $x$ in a spacetime $(\ss, g)$ its {\it causal future\/}
        $J^+(x)\subset \ss$ is the set of all $y\in \ss$ that can be reached
        by a future pointing nonspacelike curve from $x.$ The causal past
        $J^-(x)$ and the chronological past $I^-(x)$ of the event $x\in \ss$
        are defined similarly. % and they are
        \par\hfill\par
        Two events $x,y$ are said to be {\it causally related\/} if
        $x\in J^+(y)$ or $y\in J^+(x).$
        \par\hfill\par
        ``A spacetime is said to be {\it globally hyperbolic\/} if
        $J^+(x)\cap J^-(y)$ is compact for every $x,y\in \ss$ and if it is
        {\it causal,\/} i.e.~it has no closed nonspacelike curves. The
        classical definition of a global hyperbolicty requires $(\ss,g)$ to
        be strongly causal rather than just causal, see~\cite{HawkingEllis},
        but Bernal and Sanchez~\cite[Theorem 3.2]{BernalSanchezCausal}
        proved that that two definitions are equivalent.''
        The compactness condition is
        known as {\em absence of naked singularities.}
        \par\hfill\par
        ``A {\it Cauchy surface\/} in $(\ss, g)$ is a subset such that every
        inextendible nonspacelike curve $\gamma(t)$ intersects it at exactly
        one value of $t.$ A classical result is that $(\ss, g)$ is globally
        hyperbolic if and only if it has a Cauchy surface,
        see~\cite[pages 211-212]{HawkingEllis}. Geroch~\cite{Geroch} proved
        that every globally hyperbolic $(\ss, g)$ is homemorphic to a product
        of $\mathbb{R}$ and a Cauchy surface. Bernal and
        Sanchez~\cite[Theorem 1]{BernalSanchez},
        ~\cite[Theorem 1.1]{BernalSanchezMetricSplitting},
        ~\cite[Theorem 1.2]{BernalSanchezFurther} proved that every globally
        hyperbolic $(\ss^{m+1}, g)$ has a smooth spacelike Cauchy surface
        $M^m$ and that moreover for every smooth spacelike Cauchy surface
        $M$ there is a diffeomorphism  $h:M\times \mathbb{R}\to \ss$ such that
        \begin{description}
            \item[a]
                $h(M\times t)$ is a smooth spacelike Cauchy surface for all $t$,
            \item[b]
                $h(x\times \mathbb{R})$ is a future pointing timelike
                curve for all $x\in M$, and finally
            \item[c]
                $h(M\times 0)=M$ with $h|_{M\times 0}:M\to M$
                being the identity map.''
        \end{description}

        % {\bf Convention:\/} in this paper $(\ss^{m+1}, g), m\geq 2$ i%s a
        % globally hyperbolic spacetime with a smooth spacelike Cauchy %
        % surface $M^m$ and $h:M\times \mathbb{R}\to \ss$
        % is the diffeomorphism as above.
        Globally hyperbolic spacetimes form the most important class of
        spacetimes and one of the versions of the famous Strong Cosmic
        Censorship Conjecture of R.~Penrose says that all physically relevant
        spacetimes are globally hyperbolic~\cite{Penrose}. What happens inside
        of the black holes is not relevant for us living outside of them so
        the conjecture says that if you cut out the black holes along the
        event horizons then what is left is globally hyperbolic.
        \par\hfill\par
        ``Let $\mathfrak N$ be the space of all future pointing null-geodesics
        in $(\ss, g)$ considered up to affine orientation preserving
        reparameterization. A null geodesic $\gamma(t)$ intersects a Cauchy
        surface $M$ at a time $\overline{t}.$
        We decompose $T_{\gamma(\overline{t})}\ss$
        into a direct sum of $T_{\gamma(\overline{t})}M$ and its $g$-orthogonal
        compliment. Since $M$ is spacelike and $\gamma$ is a null geodesic,
        the $T_{\gamma(\overline{t})}M$-component of
        $\gamma'(\overline{t})\neq {\bf 0}$
        and it gives a point in the spherical tangent bundle $STM$ of $M.$
        This identifies $\mathfrak N$ with $STM$. Since $M$ is spacelike we
        identify $TM=T^*M$ and $\mathfrak N=STM=ST^*M.$
        Low~\cite{LowLegendrian} showed that if $M'$ is another spacelike
        Cauchy surface then $ST^*M'\to \mathfrak N\to ST^*M$ is a
        contactomorphism and thus his identification $\mathfrak N=ST^*M$
        gives $\mathfrak N$ a natural contact structure.
        Low's work~\cite{Low0, LowLegendrian} deals with
        $3+1$-dimensional $\ss$, but the fact and his proof hold in all
        dimensions, see also Natario and Tod~\cite[pages 252-253]{NatarioTod}.''
        \par\hfill\par
        The {\em sky} $S_p\mathfrak N$ of an event $p\in \ss$ is the
        Legendrian sphere of all light rays passing through $p\in \ss$.
        A sky $S_p$ is isotopic to the sphere fiber of
        $\mathfrak N=ST^*M\to M$ and twe skies are called
        {\em unlinked \/}if they are isotopic to the link consisting
        of the pair of fibers over two distinct points.
        \par\hfill\par
    \subsection{Affine linking numbers and their application }
        The classical linking number $\lk$ of two zero homologous submanifolds
        $P_1^{p_1}, P_2^{p_2}$ in $Q^{p_1+p_2+1}$ was introduced by Gauss and
        it is the signed number of intersection points of $P_1$ and a compact
        oriented submanifold bounded by $P_2.$ Clearly under the passage
        through a double point of a link under its homotopy it changes by
        $\Delta_{\lk}$ which is the sign of the orientation frame given by
        concatenating the orientation frame of $P_1,$ of $ P_2$ and adding
        the vector going from the preimage of the double point on $P_1$ to
        the preimage on $P_2$ when you resolve the double point.
        \par\hfill\par
        Affine linking numbers $\alk$ are defined as follows:
        for the pair of sphere fibers of $ST^*M$ the $\alk$ is zero and it
        increases by one under every positive passage through a transverse
        double point of the link. Rudyak and the first
        author~\cite{ChernovRudyak} showed that the affine linking number of
        a pair of skies in $ST^*M=\mathfrak N$ is a well defined $\mathbb{Z}$-valued
        invariant provided that $M$ is {\bf not} an odd-dimensional
        rational homology sphere. In the last case $\alk$ takes values in
        $\mathbb{Z}/\Im \deg$ where $\deg:\pi_m(M)\to \mathbb{Z}$ is the degree homomorphism.
        \par\hfill\par
        For a pair of events $p_1, p_2$ in the future of $q$ connected by
        a generic timelike curve $\gamma$ and an event $q$ Rudyak and the
        first author proved that $\alk(S_{p_1}, S_q)-\alk(S_{p_2}, S_2)$
        equals to the sum of the signs of the intersection points
        $\gamma\cdot\exp(N_q)$~\cite{ChernovRudyak}. Every timelike curve can
        be made generic by a $C^{\infty}$-small perturbation so that it
        intersects the exponential of the null cone $\exp(N_q)$ of $q$ at the
        points where the exponential is immersed and there are no double points
        of it and the intersection is transvers at all points.
        \par\hfill\par
        Affine linking numbers can be used to estimate the number of
        times an observer sees the same star on the night sky, for
        example this happens in the famous Einstein Crosses of which about
        50 examples were found so far.
        \begin{thm}\label{theorem1}
            Assume $(\ss, g)$ is globally hyperbolic and the Cauchy surface of
            it is {\bf not} an odd dimensional rational homology sphere. Let
            $\gamma:(-\epsilon, \epsilon)\to \ss$ be a small timelike curve
            that passes through $p$ at time moment $0.$ Assume that
            $\bigl(\alk(S_{\gamma(\epsilon)},S_q)-\alk(S_{\gamma(-\epsilon)},S_q)\bigr)=N\in\mathbb{Z}$
            for all small $\epsilon$ then the observer at $p$ sees light from
            $q$ coming from at least $N$ different directions.
        \end{thm}
        \begin{proof}
            This follows since the number of times an observer at $p$ sees
            light at $q$ is the difference of the limits of this number of
            times for an observer at $\gamma(\epsilon)$ and for an observer
            at $\gamma(-\epsilon)$. The last quantities are related to $\alk$
            as it is explained in the beginning of this section.
        \end{proof}
        As it was explained in~\cite{ChernovRudyak} if all the timelike
        sectional curvatures of $(\ss, g)$ are non-negative then the
        exponential of the null cone $N_q$ of $q$ is an immersion hence the
        terms in the computation of the intersection number of $\exp(N_q)$ and
        $\gamma$ do not cancel and hence the inputs into $\alk$ can not cancel
        as well. So we get the following Theorem.
        \begin{thm}
            \label{theorem2}%
            Assume $(\ss, g)$ is globally hyperbolic and the Cauchy
            surface of it is {\bf not} an odd dimensional rational homology
            sphere. Assume moreover that all the timelike sectional curvatures
            of $(\ss,g)$ are non-negative. Let
            $\gamma:(-\epsilon, \epsilon)\to \ss$ be a small timelike curve
            that passes through $p$ at time moment $0.$ Assume that
            $\bigl ( \alk(S_{\gamma(\epsilon)}, S_q)-\alk(S_{\gamma(-\epsilon)}, S_q)\bigr)=N$
            for all small $\epsilon$ then the observer at $p$ sees light
            from $q$ coming from exactly $N=|N|$ different directions.
            \qed
        \end{thm}
        \begin{rem}
            Since the exponential of the null cone of $q$ is an immersion,
            all the intersection points of it with $\gamma$ are positive.
        \end{rem}
        \begin{rem}
            An example of such a spacetime is any of the warped Lorentz
            spacetime products of a complete Riemann manifold of non-positive
            sectional curvature and $\mathbb{R}$ with any positive warping function.
        \end{rem}
        \begin{rem}
            Note that these Theorems also allow one to estimate the
            difference between the value of the affine linking number $\alk$
            invariant based on how many times you see a copy of the same star.
            In the classical Einstein Cross that gave name to the phenomenon
            there are 4 copies of the same star on the night sky. So the
            difference between the $\alk$ invariant values is at least four.
        \end{rem}
\begin{thebibliography}{99}
\bibitem{BeemEhrlichEasley}
J.~K.~Beem, P.~E.~Ehrlich, K.~L.~Easley: {\em Global Lorentzian
geometry.\/} Second edition. Monographs and Textbooks in Pure and
Applied Mathematics, {\bf 202\/} Marcel Dekker, Inc., New York (1996)
\bibitem{BernalSanchezCausal}
A.~Bernal and M.~Sanchez, {\it Globally hyperbolic spacetimes can be defined as ``causal'' instead of
``strongly causal''},  Class.~Quant.~Grav.~24 (2007) 745-750
\bibitem{BernalSanchez}
A.~Bernal, M.~Sanchez: {\em On smooth Cauchy hypersurfaces and
Geroch's splitting theorem, \/} Comm.~Math.~Phys.~{\bf 243\/}
(2003), no.~3, 461--470
\bibitem{BernalSanchezMetricSplitting}
A.~Bernal, M.~Sanchez: {\em Smoothness of time functions and the
metric splitting of globally hyperbolic space-times.\/} Comm.~Math.~Phys.~{\bf 257} (2005), no.~1, 43--50.
\bibitem{BernalSanchezFurther}
A.~Bernal and M.~Sanchez: {\em Further results on the smoothability
of Cauchy hypersurfaces and Cauchy time functions.\/} Lett.~Math.~Phys.~{\bf 77} (2006), no.~2, 183--197.
\bibitem{ChernovNemirovski}
V.~Chernov, S.~Nemirovski: {\em Legendrian links, causality and the Low conjecture,\/} Geom. Funct. Anal. 19 (2010), 1320--1333
\bibitem{ChernovRudyak}
V.~Chernov (Tchernov) and Yu.~Rudyak, {\it Linking and Causality in Globally Hyperbolic Space-times,\/}
Communications in Mathematical Physics, {\bf 279}, no 2, (2008) pages 309-354
\bibitem{Geroch}
R.~P.~Geroch: {\em Domain of dependence\/}, J.~Math.~Phys., {\bf 11}
(1970) pp.~437--449
\bibitem{HawkingEllis}
S.~W.~Hawking and G.~F.~R.~Ellis, {\em The large scale structure of
space-time.\/} Cambridge Monographs on Mathematical Physics, No.~1,
Cambridge University Press, London-New York, (1973)
\bibitem{Low0}
R.~J.~Low: {\em Causal relations and spaces of null geodesics.\/}
PhD Thesis, Oxford University (1988)
%\bibitem{Low1}
%R.~J.~Low: {\em Twistor linking and causal relations.\/} Classical
%Quantum Gravity {\bf 7} (1990), no.~2, 177--187.
%\bibitem{Low3}
%R.~J.~Low: {\em Twistor linking and causal relations in exterior
%Schwarzschild space.\/} Classical Quantum Gravity {\bf 11} (1994),
%no.~2, 453--456.
\bibitem{LowLegendrian}
R.~J.~Low: {\em Stable singularities of wave-fronts in general relativity.\/} J.~Math.~Phys.~{\bf 39} (1998), no.~6, 3332--3335
%\bibitem{LowNullgeodesics}
%R.~J.~Low: {\em The space of null geodesics.\/} Proceedings of the
%Third World Congress of Nonlinear Analysts, Part 5 (Catania, 2000).
%Nonlinear Anal. 47 (2001), no. 5, 3005--3017
%\bibitem{LowRefocussing}
%R.~J.~Low: {\em The space of null geodesics {\rm(}and a new causal
%boundary{\rm )}.\/} Lecture Notes in Physics {\bf 692}, Springer, Belin
%Heidelberg New York (2006), 35--50
\bibitem{NatarioTod}
J.~Natario and P.~Tod: {\em Linking, Legendrian linking and
causality.\/} Proc.~London Math.~Soc.~(3) {\bf 88} (2004), no.~1,
251--272.
\bibitem{Penrose}
R.~Penrose, {\it The question of cosmic censorship},
Black holes and relativistic stars (Chicago, IL, 1996), pp.~103--122,
Univ. Chicago Press, Chicago, IL, (1998)
\end{thebibliography}
\end{document}
