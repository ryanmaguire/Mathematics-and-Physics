%-----------------------------------LICENSE------------------------------------%
%   This file is part of Mathematics-and-Physics.                              %
%                                                                              %
%   Mathematics-and-Physics is free software: you can redistribute it and/or   %
%   modify it under the terms of the GNU General Public License as             %
%   published by the Free Software Foundation, either version 3 of the         %
%   License, or (at your option) any later version.                            %
%                                                                              %
%   Mathematics-and-Physics is distributed in the hope that it will be useful, %
%   but WITHOUT ANY WARRANTY; without even the implied warranty of             %
%   MERCHANTABILITY or FITNESS FOR A PARTICULAR PURPOSE.  See the              %
%   GNU General Public License for more details.                               %
%                                                                              %
%   You should have received a copy of the GNU General Public License along    %
%   with Mathematics-and-Physics.  If not, see <https://www.gnu.org/licenses/>.%
%------------------------------------------------------------------------------%
%       Author: Ryan Maguire                                                   %
%       Date:   2022/03/14                                                     %
%------------------------------------------------------------------------------%
\documentclass{article}

% Needed for mathbb font style.
\usepackage{amssymb}

% Tools for creating the theorem environment.
\usepackage{amsthm}
\usepackage{amsmath}

% Used for pictures.
\usepackage{graphics}
\usepackage{float}

% Hyperlinks for labels.
\usepackage{hyperref}

% Display style for hyperlinks.
\hypersetup{
    colorlinks=true,
    linkcolor=blue
}

% Create a theorem environment, similar to most textbooks.
\theoremstyle{plain}
\newtheorem{theorem}{Theorem}
\newtheorem{conjecture}{Conjecture}
\newtheorem{question}{Question}

% No indent and no paragraph skip.
\setlength{\parindent}{0em}
\setlength{\parskip}{0em}

% Title page information.
\title{Conjectures on the Khovanov Homology of Legendrian and Transversely Simple Knots}
\author{Vladimir Chernov\hspace{2em}Ryan Maguire}
\date{March 2023}

\begin{document}
    \maketitle
    \tableofcontents
    \begin{abstract}
        \noindent
        A theorem of Kronheimer and Mrowka states that Khovanov homology is
        able to detect the unknot \cite{kronheimermrowka2010}.
        That is, if a knot has the Khovanov homology of the unknot, then it is
        equivalent to it. Similar results hold for the trefoils
        \cite{BaldwinSivekKhovanovTrefoils} and the figure-eight knot
        \cite{BaldwinDowlinKhovanovFigureEight}. These are the
        simplest of the Legendrian simple knots. It is conjectured that
        Khovanov homology is able to distinguish Legendrian and Transversely simple
        knots. Using the torus and twist knots, numerical evidence is provided
        for all knots up to 19 crossings.
        \par\hfill\par
        \textit{AMS Classification: Primary 57K18, Secondary 53D12, 53D10}
    \end{abstract}
    \section{Legendrian and Transverse Knots and Links}
        A knot is a smooth embedding of the circle $\mathbb{S}^{1}$ into
        $\mathbb{R}^{3}$. A link is a smooth embedding of $N\in\mathbb{N}$
        disjoint circles in $\mathbb{R}^{3}$. We may impose
        extra structure by considering the standard \textit{contact structure}
        of $\mathbb{R}^{3}$. This is an assignment
        of a plane to every point in $\mathbb{R}^{3}$ such that there is no
        surface $M\subset\mathbb{R}^{3}$ where for every $p\in{M}$ the tangent
        plane $T_{p}M$ is given by the plane of the contact structure. In
        $\mathbb{R}^{3}$ this is described by the one-form
        $\textrm{d}z-y\,\textrm{d}x$, the plane at $(x,y,z)$ being spanned by
        the vectors $\partial_{x}+y\partial_{z}$ and $\partial_{y}$. The
        hyperplane distribution is shown in Fig.~\ref{fig:darboux_form_001}.
        \begin{figure}
            \centering
            \resizebox{\textwidth}{!}{%
                \includegraphics{../images/darboux_form_001.pdf}
            }
            \caption{Hyperplane Distribution for $\textrm{d}z-y\,\textrm{d}x$}
            \label{fig:darboux_form_001}
        \end{figure}
        While there are no everywhere tangent surfaces, it is possible for a
        curve to be everywhere tangent to this distribution of planes.
        A \textit{Legendrian link} is a link that is everywhere tangent to
        the contact structure. A gentle introduction to the subject can be found
        in \cite{JoshuaMSabloffWhatIsLegendrianKnot}.
        \par\hfill\par
        Two Legendrian links are considered to be Legendrian equivalent if
        there is an isotopy $H:L\times[0,1]\rightarrow\mathbb{R}^{3}$ between
        them, $L=\sqcup_{k=0}^{N-1}\mathbb{S}^{1}$, such that for all
        $t\in[0,1]$ the link $H_{t}$ is Legendrian. It is possible for two links
        to be topologically equivalent but not Legendrian equivalent. In the other
        direction, any link can be made Legendrian by an appropriate isotopy
        (See the introduction of \cite{VeraVertessiTransNonSimpleKnots}). The
        two classical invariants of Legendrian links are the Thurston-Bennequin
        and rotation numbers.
        A link type is said to be \textit{Legendrian simple} if
        any two Legendrian embeddings of it with the same Thurston-Bennequin
        and rotation numbers are Legendrian equivalent. That is, the classical
        invariants uniquely classify all Legendrian representations of the knot.
        Certain knot types are known to be Legendrian simple such as the unknot
        \cite{EliashbergFraserClassificationTopTrivialLegKnots}, torus knots,
        and the figure eight knot \cite{EtnyreHondaContactTopologyI}.
        Not every knot is Legendrian simple, the $m_{3}$ twist knot (also
        known as the $5_{2}$ knot) being an example discovered by Chekanov
        \cite{ChekanovDifAlgOfLegLinks}. More examples can be found in the
        Legendrian knot atlas \cite{LegendrianKnotAtlas}.
        \par\hfill\par
        Transverse links are links that are everywhere transverse to the
        contact structure. That is, at every point $p$ on the link the velocity
        vector and the contact hyperplane at $p$ span $\mathbb{R}^{3}$. Any Legendrian
        link can be made transverse by a small perturbation in the direction
        normal to the given plane in the contact structure.
        Two transverse links are transversely equivalent if there is an
        isotopy $H:L\times[0,1]\rightarrow\mathbb{R}^{3}$ such that $H_{t}$ is a
        transverse link for all $t\in[0,1]$. The Bennequin number of a
        transverse knot is defined by the \textit{algebraic crossing number}
        $e(K)$ and the \textit{braid index} $n(K)$. It is:
        \begin{equation}
            \beta(K)=e(K)-n(K)
        \end{equation}
        It is not an invariant of topological knots, but is an invariant under
        transverse equivalence. This is essentially the self-linking number of
        the corresponding topological knot with respect to the natural framing
        coming from the trivialization of the contact structure.
        Similar to Legendrian simple, we define a knot
        (or link) type to be transversely simply if all of it's transverse
        representations are uniquely determined by their Bennequin number
        (See \cite{BirmanWrinkleTransversallySimpleKnots}) and by whether its
        velocity vectors point into the half space where the contact structure
        is positive or not. A paper by Etynre, Ng, and Vertesi
        \cite{EtnyreEtAlLegendrianAndTransverseTwistKnots}
        classifies when twist knots are transversely simple. In
        particular, infinitely many such knots are transversely simple giving
        us a family of knots to test conjectures with. It is also true that
        infinite families of non-Legendrian simple (and hence
        non-transversally simple) knots exist,
        see \cite{vfoldvari2019legnonsimple}.
        \par\hfill\par
        There are several common ways of representing topological knots, the
        three used in our computations are extended Gauss code, planar diagram
        code (PD code), and Dowker-Thistlewaite code (DT code). Given a knot
        diagram with $N$ crossings, the (unsigned) Gauss code is a string with $2N$
        characters (extended Gauss code is $3N$),
        PD code is a string that is $4N$ long, and DT code is $N$
        characters long. Each has its benefits. Extended Gauss code
        can distinguish mirrors, DT code cannot. PD code is the easiest to
        reconstruct the knot diagram, and DT code is the shortest. Because of
        this we will present our examples in DT code. To obtain the DT code of
        a knot diagram, place your finger on the knot and \textit{walk} along
        the diagram, labelling the crossings. When you get back to your starting
        point each crossing will have two numbers associated with it. It is
        not difficult to see that each crossing will have exactly one odd number
        and one even number. For each even number, if that number was associated
        with an \textit{over} crossing (that is, your finger ran over the
        crossing as you were labelling it), place a minus sign in front. Write
        out the pairs of integers as $(1,a_{1})$, $(3,a_{2})$, $\dots$,
        $(2n-1,a_{n})$. The DT code is the list $a_{1},a_{2},\dots,a_{n}$.
        See \cite{KatlasDTCode} for several examples.
        \par\hfill\par
        It is possible to go from DT code to unsigned
        Gauss code (i.e. the usual Gauss code, and not the extended Gauss code)
        and back. For certain computations, like the Alexander polynomial which
        is mirror insensitive, DT code is easiest since it is the shortest.
        For things like the Jones polynomial and Khovanov homology, invariants
        that distinguish mirrors, extended Gauss code is a must.
        \par\hfill\par
        A note on notation. The tables on the following pages identify knots using their DT
        codes. The \texttt{regina} library \cite{regina} uses a clever trick to efficiently
        express a knots DT code for all knots of up to 26 crossings, and we'll
        adopt this notation for the remainder of the paper. The DT code of an
        $n$ crossing knot is a string of length $n$ of signed even integers up
        to $2n$. Since all entries are even, we can divide by two and obtain
        a string of length $n$ of signed integers up to $n$. So long as $n$ is
        not larger than 26 we can replace integers with letters. We then use
        lower case letters for positive integers, and upper case for negative.
        For example, the trefoil $T(2,\,3)$ becomes:
        \begin{equation}
            4,\,6,\,2
            \Rightarrow
            2,\,3,\,1
            \Rightarrow
            \texttt{bca}
        \end{equation}
        The knot $10_{132}$, which is the 10 crossing knot in
        Tab.~\ref{table:matching_twist_knots} with the same Jones polynomial
        as $T(2,\,5)$, becomes:
        \begin{align}
            8,\,6,\,18,\,2,\,-12,\,-16,\,20,\,-10,\,4,\,14
            &\Rightarrow
            4,\,3,\,9,\,1,\,-6,\,-8,\,10,\,-5,\,2,\,7\\
            &\Rightarrow
            \texttt{dciaFHjEbg}
        \end{align}
    \section{Khovanov Homology}
        The Khovanov homology of a link is a powerful, if computationally
        expensive%
        \footnote{%
            The na\"{i}ve algorithm is exponential in the number of
            crossings. Improvements by Bar-Natan \cite{BarNatanFastKhoHo}
            have sped up computations but no polynomial-time algorithm is
            known at the time of this writing.
        },
        invariant first introduced by Mikhail Khovanov
        \cite{Khovanov1999CatJonesPoly} (See also \cite{barnatan2002khovanov}
        for an excellent introduction). It is closely related to the Jones
        polynomial, but able to distinguish many more knots and links. The
        homology groups $KH^{r}(L)$ of a link (or knot) $L$ are the direct sum
        of homogeneous components $KH_{\ell}^{r}(L)$ and the
        \textit{Khovanov Polynomial} (See \cite{KatlasKhoHo}) is given by:
        \begin{equation}
            Kh(L)(q,t)=
            \sum_{r,\ell}t^{r}q^{\ell}\textrm{dim}\big(KH_{\ell}^{r}(L)\big)
        \end{equation}
        The Jones polynomial of $L$ is recovered via:
        \begin{equation}
            J(L)(q)=Kh(L)(q,-1)
        \end{equation}
        Khovanov homology is not a perfect invariant
        \cite{Watson2007KnotsWI}. That is, there are
        distinct knots with the same Khovanov homology, but it is a powerful
        invariant and is capable of detecting certain knot types.
        \begin{theorem}[Kronheimer and Mrowka, 2001]
            If a knot $K$ has the same Khovanov homology as the unknot, then $K$
            is equivalent to the unknot.
        \end{theorem}
        The unknotting problem asks one to determine if a given knot diagram is
        equivalent to the unknot. Khovanov homology is a powerful enough tool
        to accomplish this task. The Khovanov polynomial is a generalization of
        the Jones polynomial and it has been conjectured that if a
        knot has the same Jones polynomial as the unknot, then that knot is
        equivalent to the unknot. At the time of this writing it has not been
        proven, but there is evidence for and against the claim.
        Thistlewaite found links with the same Jones polynomial as the unlink
        \cite{Thistlethwaite2001LINKSWT}, and there is a 3-crossing virtual
        knot that has the same Jones polynomial as the unknot. For the claim,
        all knots of up to 24 crossings are either the unknot, or have a
        Jones polynomial different from the unknot
        \cite{VerificationUnknotJonesConjUpTo24}.
        \begin{theorem}[Baldwin and Sivek, 2022]
            If a knot $K$ has the same Khovanov homology as either of the
            trefoils, then $K$ is equivalent to one of them.
        \end{theorem}
        \begin{theorem}[Baldwin, Dowlin, Levine, Lidman, and Sazdanovic, 2021]
            If a knot $K$ has the same Khovanov homology as the figure-eight
            knot, then $K$ is equivalent to it.
        \end{theorem}
        See \cite{BaldwinSivekKhovanovTrefoils} and
        \cite{BaldwinDowlinKhovanovFigureEight}, respectively.
    \section{Conjectures on Khovanov Homology}
        Khovanov homology is capable of detecting the
        unknot, trefoils, and figure-eight knot. The Khovanov homology with
        coefficients in $\mathbb{Z}/2\mathbb{Z}$ is also capable of detecting
        the cinquefoil knot \cite{BaldwinYingSivekCinquefoilKhovanov},
        which is the $T(5,2)$ torus knot. The Jones
        polynomial, on the other hand, is not capable of detecting the
        $T(5,2)$ torus knot since the $10_{132}$ knot yields the same
        polynomial. These are the easiest of the Legendrian simple knots
        leading us to the following.
        \begin{conjecture}
            If a link type $L$ is Legendrian simple, then the Khovanov homology
            of $L$ distinguish it. That is, if $\tilde{L}$
            is another link with the same Khovanov homology, then $\tilde{L}$ is
            equivalent to $L$.
        \end{conjecture}
        Numerical evidence has been tallied for torus knots
        against all knots of up to 19 crossings. There are many
        torus knots that have the same Jones polynomial as a non-torus knot
        ($T(2,5)$ matches 10 and 17 crossing knots, $T(2,7)$
        matches a 12 crossing knot, and $T(2,11)$ matches a 14 crossing knot)
        so we cannot generalize the Jones unknot conjecture. Nevertheless, in
        all cases the Khovanov homologies were different
        (see Numerical Results section).
        \par\hfill\par
        The computations were done as follows. There are libraries in Python,
        Sage, and C++ for working with knot polynomials. In particular, we used
        Regina \cite{regina}, SnapPy \cite{SnapPy}, the Sage knot library
        \cite{sage}, and our own ever-growing C library.
        \par\hfill\par
        We first gather the torus knots that could potentially have the same
        Jones polynomial as some other knot based on the bounds of the degree.
        The Jones polynomials of torus knots were computed using the formula
        \cite{jonesfordummyvjones}:
        \begin{equation}
            \label{eqn:jones_poly_torus}%
            J(T(m,n))(q)=q^{(m-1)(n-1)/2}
                \frac{1-q^{m+1}-q^{n+1}+q^{m+n}}{1-q^{2}}
        \end{equation}
        The Kauffman bracket, of which the Jones polynomial is a simple
        normalization \cite{barnatan2002khovanov},
        of a knot $K$ with $N$ crossings has the following formula:
        \begin{equation}
            \langle{K}\rangle=\sum_{n=0}^{2^{N}-1}(-q)^{w(n)}(q+q^{-1})^{c(n)}
        \end{equation}
        where $w(n)$ is the Hamming weight of $n$, the number of 1's in the
        binary representation, and $c(n)$ is the number of disjoint cycles
        that result in the complete smoothing of $K$ corresponding to
        $0\leq{n}\leq{2}^{N}-1$. With this the degree of the bracket polynomial
        is bounded by $3n$ ($w(n)$ is bounded by $n$, and $c(n)$ is bounded
        by $2n$).
        \par\hfill\par
        Since we are looking at knots up to $19$ crossings, we collect the
        co-prime pairs $(m,\,n)$ with $1<m<n$ such that the degree
        is less than $3\cdot{19}+1=58$. Since the Jones polynomial of a mirror
        can be computed by substituting $q\mapsto{q}^{-1}$, we need not look at
        negative values. The Khovanov polynomial makes a similar change,
        $(q,\,t)\mapsto(q^{-1},\,t^{-1})$.
        \par\hfill\par
        Using any of the aforementioned libraries, the Jones polynomial of all
        knots up to 19 crossings were computed and compared against this table
        of torus knot Jones polynomials (Eqn.~\ref{eqn:jones_poly_torus}).
        If a match was found the regina library was used to determine if the
        knots were actually identical. That is, if the knot whose Jones
        polynomial was being compared against the torus knots was indeed a
        torus knot itself. If the knots were distinct, this knot was saved in a
        text file for later examination. At the end of the computation 4
        non-torus knots had the same Jones polynomial as a torus knot
        (See Tab.~\ref{table:matching_torus_knots}).
        Since the Khovanov polynomial contains the Jones polynomial in it
        (recall $J(L)(q)=Kh(L)(q,-1)$) the only possible non-torus knots with
        the same Khovanov homology as a torus knot were these 4.
        \par\hfill\par
        Using the Java library JavaKh\footnote{%
            Thanks must be paid to Nikolay Pultsin who made edits to
            JavaKh-v2 so that it may run on a GNU/Linux machine using
            OpenJDK 17.
        }
        we found that these four knots with the same Jones polynomials as some
        torus knot all had different Khovanov homologies. Thus, we have the
        following claim:
        \begin{theorem}
            If a knot $K$ has less than or equal to 19 crossings and has the
            Khovanov homology of a torus knot $T$,
            then $K$ is equivalent to $T$.
        \end{theorem}
        A similar search through the twist knots yielded more results.
        The Jones polynomials of the twist knots are known, with the formula:
        \begin{equation}
            J(m_{n})(q)=
            \begin{cases}
                (1+q^{-2}+q^{-n}+q^{-n-3})/(1+q),&n\textrm{ odd}\\
                (1+q-q^{3-n}+q^{-n})/(1+q),&n\textrm{ even}
            \end{cases}
        \end{equation}
        A search through all knots up to 19 crossings against twist knots
        provided many matches for the Jones polynomial
        (See Tab.~\ref{table:matching_twist_knots}), but
        none for Khovanov homology. Infinitely many of the twists knots are
        transversally simple, making them a good candidate to test the following
        conjecture on.
        \begin{conjecture}
            If a link type $L$ is transversally simple, then the Khovanov
            homology of $L$ distinguish it. That is, if $\tilde{L}$
            is another link with the same Khovanov homology, then $\tilde{L}$ is
            equivalent to $L$.
        \end{conjecture}
        A similar search for Knot Floer Homology (KFH), a homology theory
        first introduced by Peter Ozsv\'{a}th and Zolt\'{a}n Szab\'{o}
        \cite{ozsvathszabo2004}, using the Alexander polynomial
        was performed, but a bug was found that caused knots with identical
        Alexander polynomials to give false negatives. This has been corrected,
        and when performing a new search there are several distinct knots
        with the same Knot Floer Homology as a Legendrian simple knot. Steven
        Sivek pointed
        out that the pretzel knots $P(-3,3,2n+1)$ all have the same KFH, meaning
        the $6_{1}$ twist knot matches the KFH of the $9_{46}$ knot in the
        Rolfsen table. This is what first hinted at a bug in our KFH code.
        Matt Hedden also informed us that the $T(4,3)$ knot and the
        $(2,3)$ cable of the trefoil also have matching KFH.
    \section{Numerical Results: Torus Knots}
        \begin{table}[H]
            \centering
            \resizebox{\textwidth}{!}{%
                \begin{tabular}{| l | l | l |}
                    \hline
                        Torus Knot&
                        Non-Torus Knot&
                        Jones Polynomial\\
                    \hline
                        $T(2,\,5)$&
                        \texttt{dciaFHjEbg}&
                        $-q^{14}+q^{12}-q^{10}+q^{8}+q^{4}$\\
                    \hline
                        $T(2,\,7)$&
                        \texttt{fJGkHlICEABd}&
                        $-q^{20}+q^{18}-q^{16}+q^{14}-q^{12}+q^{10}+q^{6}$\\
                    \hline
                        $T(2,\,11)$&
                        \texttt{gHlImJnKBDFAce}&
                        $-q^{32}+q^{30}-q^{28}+q^{26}-q^{24}+q^{22}-q^{20}+q^{18}-q^{16}+q^{14}+q^{10}$\\
                    \hline
                        $T(2,\,5)$&
                        \texttt{iNHlPJqCoKFmdABgE}&
                        $-q^{14}+q^{12}-q^{10}+q^{8}+q^{4}$\\
                    \hline
                \end{tabular}%
            }
            \caption{Knots whose Jones polynomial matches that of a Torus Knot}
            \label{table:matching_torus_knots}
        \end{table}
        In Tab.~\ref{table:matching_torus_knots}
        we have found four non-torus knots with the same
        Jones polynomial as some torus knot. In all four cases the Khovanov
        polynomials differ. The tables for these polynomials are given below.
        The coefficient of $q^{\ell}t^{r}$ is given by the corresponding entry.
        Empty equates to zero.
        \par\hfill\par
        The cinquefoil $T(2,\,5)$ has the same Jones polynomial as two
        non-torus knots. As shown below in
        Tabs.~\ref{table:t_2_5_kho}-\ref{table:iNHlPJqCoKFmdABgE_kho}, the
        Khovanov polynomials differ
        (This had already been known, see \cite{KatlasKhoHo}).
        \begin{table}[H]
            \centering
            \begin{tabular}{| c | c | c | c | c | c | c |}
                \hline
                $q\symbol{92}t$&$-5$&$-4$&$-3$&$-2$&$-1$&$0$\\
                \hline
                $-15$&1&&&&&\\
                \hline
                $-13$&&&&&&\\
                \hline
                $-11$&&1&1&&&\\
                \hline
                $-9$&&&&&&\\
                \hline
                $-7$&&&&1&&\\
                \hline
                $-5$&&&&&&1\\
                \hline
                $-3$&&&&&&1\\
                \hline
            \end{tabular}
            \caption{Khovanov Polynomial for $T(5,2)$}
            \label{table:t_2_5_kho}
        \end{table}
        \begin{table}[H]
            \centering
            \begin{tabular}{| c | c | c | c | c | c | c | c | c |}
                \hline
                $q\symbol{92}t$&$-7$&$-6$&$-5$&$-4$&$-3$&$-2$&$-1$&$0$\\
                \hline
                $-15$&1&&&&&&&\\
                \hline
                $-13$&&&&&&&&\\
                \hline
                $-11$&&1&1&&&&&\\
                \hline
                $-9$&&&&1&1&&&\\
                \hline
                $-7$&&&&1&&&&\\
                \hline
                $-5$&&&&&1&2&&\\
                \hline
                $-3$&&&&&&&&1\\
                \hline
                $-1$&&&&&&&1&1\\
                \hline
            \end{tabular}
            \caption{Khovanov Polynomial for \texttt{dciaFHjEbg}}
            \label{table:dciaFHjEbg_kho}
        \end{table}
        \begin{table}[H]
            \centering
            \begin{tabular}{| c | c | c | c | c | c | c | c | c | c | c |}
                \hline
                $q\symbol{92}t$&$-9$&$-8$&$-7$&$-6$&$-5$&$-4$&$-3$&$-2$&$-1$&$0$\\
                \hline
                $-15$&1&&&&&&&&&\\
                \hline
                $-13$&&&&&&&&&&\\
                \hline
                $-11$&&1&1&&&&&&&\\
                \hline
                $-9$&&&&1&1&&&&&\\
                \hline
                $-7$&&&&1&&&&&&\\
                \hline
                $-5$&&&&&1&2&&&&\\
                \hline
                $-3$&&&&&&&&1&&\\
                \hline
                $-1$&&&&&&&1&&&1\\
                \hline
                $1$&&&&&&&&&1&1\\
                \hline
            \end{tabular}
            \caption{Khovanov Polynomial for \texttt{iNHlPJqCoKFmdABgE}}
            \label{table:iNHlPJqCoKFmdABgE_kho}
        \end{table}
        The $T(7,2)$ knot, also the $7_{1}$ knot, and occasionally called the
        septafoil, has the same Jones polynomial as \texttt{fJGkHlICEABd}. The
        Khovanov polynomials are distinct
        (Tabs.~\ref{table:t_7_2_kho}-\ref{table:fJGkHlICEABd_kho}).
        \begin{table}[H]
            \centering
            \begin{tabular}{| c | c | c | c | c | c | c | c | c |}
                \hline
                $q\symbol{92}t$&$-7$&$-6$&$-5$&$-4$&$-3$&$-2$&$-1$&$0$\\
                \hline
                $-21$&1&&&&&&&\\
                \hline
                $-19$&&&&&&&&\\
                \hline
                $-17$&&1&1&&&&&\\
                \hline
                $-15$&&&&&&&&\\
                \hline
                $-13$&&&&1&1&&&\\
                \hline
                $-11$&&&&&&&&\\
                \hline
                $-9$&&&&&&1&&\\
                \hline
                $-7$&&&&&&&&1\\
                \hline
                $-5$&&&&&&&&1\\
                \hline
            \end{tabular}
            \caption{Khovanov Polynomial for $T(7,2)$}
            \label{table:t_7_2_kho}
        \end{table}
        \begin{table}[H]
            \centering
            \begin{tabular}{| c | c | c | c | c | c | c | c | c | c | c |}
                \hline
                $q\symbol{92}t$&$-9$&$-8$&$-7$&$-6$&$-5$&$-4$&$-3$&$-2$&$-1$&$0$\\
                \hline
                $-21$&1&&&&&&&&&\\
                \hline
                $-19$&&&&&&&&&&\\
                \hline
                $-17$&&1&1&&&&&&&\\
                \hline
                $-15$&&&&1&1&&&&&\\
                \hline
                $-13$&&&&1&1&&&&&\\
                \hline
                $-11$&&&&&1&2&1&&&\\
                \hline
                $-9$&&&&&&1&&&&\\
                \hline
                $-7$&&&&&&&1&2&&\\
                \hline
                $-5$&&&&&&&&&&1\\
                \hline
                $-3$&&&&&&&&&1&1\\
                \hline
            \end{tabular}
            \caption{Khovanov Polynomial for \texttt{fJGkHlICEABd}}
            \label{table:fJGkHlICEABd_kho}
        \end{table}
        Lastly, the $T(11,2)$ torus knot has the same Jones polynomial as the
        14 crossing knot \texttt{gHlImJnKBDFAce}. Once again the Khovanov
        polynomials differ.
        \begin{table}[H]
            \centering
            \begin{tabular}{| c | c | c | c | c | c | c | c | c | c | c | c | c |}
                \hline
                $q\symbol{92}t$&$0$&$1$&$2$&$3$&$4$&$5$&$6$&$7$&$8$&$9$&$10$&$11$\\
                \hline
                $9$&1&&&&&&&&&&&\\
                \hline
                $11$&1&&&&&&&&&&&\\
                \hline
                $13$&&&1&&&&&&&&&\\
                \hline
                $15$&&&&&&&&&&&&\\
                \hline
                $17$&&&&1&1&&&&&&&\\
                \hline
                $19$&&&&&&&&&&&&\\
                \hline
                $21$&&&&&&1&1&&&&&\\
                \hline
                $23$&&&&&&&&&&&&\\
                \hline
                $25$&&&&&&&&1&1&&&\\
                \hline
                $27$&&&&&&&&&&&&\\
                \hline
                $29$&&&&&&&&&&1&1&\\
                \hline
                $31$&&&&&&&&&&&&\\
                \hline
                $33$&&&&&&&&&&&&1\\
                \hline
            \end{tabular}
            \caption{Khovanov Polynomial for $T(11,\,2)$}
            \label{table:t_2_11_kho}
        \end{table}
        \begin{table}[H]
            \centering
            \begin{tabular}{| c | c | c | c | c | c | c | c | c | c | c | c | c | c | c |}
                \hline
                $q\symbol{92}t$&$0$&$1$&$2$&$3$&$4$&$5$&$6$&$7$&$8$&$9$&$10$&$11$&$12$&$13$\\
                \hline
                $9$&1&&&&&&&&&&&&&\\
                \hline
                $11$&1&&&&&&&&&&&&&\\
                \hline
                $13$&&&1&&&&&&&&&&&\\
                \hline
                $15$&&&&&1&1&&&&&&&&\\
                \hline
                $17$&&&&1&1&&&&&&&&&\\
                \hline
                $19$&&&&&&1&2&1&&&&&&\\
                \hline
                $21$&&&&&&&&&1&&&&&\\
                \hline
                $23$&&&&&&&&1&2&1&&&&\\
                \hline
                $25$&&&&&&&&&&1&1&&&\\
                \hline
                $27$&&&&&&&&&&1&1&&&\\
                \hline
                $29$&&&&&&&&&&&&1&1&\\
                \hline
                $31$&&&&&&&&&&&&&&\\
                \hline
                $33$&&&&&&&&&&&&&&1\\
                \hline
            \end{tabular}
            \caption{Khovanov Polynomial for \texttt{gHlImJnKBDFAce}}
            \label{table:t_gHlImJnKBDFAce_kho}
        \end{table}
    \section{Numerical Results: Twist Knots}
        \begin{table}[H]
            \centering
            \resizebox{\textwidth}{!}{%
                \begin{tabular}{| l | l | l |}
                    \hline
                        Twist Knot&
                        Non-Twist Knot&
                        Jones Polynomial\\
                    \hline
                        $m_{2}$&
                        \texttt{eikGbHJCaFd}&
                        $q^{4}-q^{2}+1-q^{-2}+q^{-4}$\\
                    \hline
                        $m_{3}$&
                        \texttt{dgikFHaEjbc}&
                        $-q^{12}+q^{10}-q^{8}+2q^{6}-q^{4}+q^{2}$\\
                    \hline
                        $m_{3}$&
                        \texttt{gfJKHlaIEBCD}&
                        $-q^{12}+q^{10}-q^{8}+2q^{6}-q^{4}+q^{2}$\\
                    \hline
                        $m_{3}$&
                        \texttt{hGJaMlCdEKBfI}&
                        $-q^{12}+q^{10}-q^{8}+2q^{6}-q^{4}+q^{2}$\\
                    \hline
                        $m_{5}$&
                        \texttt{bhDGijCkaef}&
                        $-q^{16}+q^{14}-q^{12}+2q^{10}-2q^{8}+2q^{6}-q^{4}+q^{2}$\\
                    \hline
                        $m_{6}$&
                        \texttt{cefIgbajkDh}&
                        $q^{12}-q^{10}+q^{8}-2q^{6}+2q^{4}-2q^{2}+2-q^{-2}+q^{-4}$\\
                    \hline
                        $m_{6}$&
                        \texttt{femIbaJKLCGHd}&
                        $q^{12}-q^{10}+q^{8}-2q^{6}+2q^{4}-2q^{2}+2-q^{-2}+q^{-4}$\\
                    \hline
                        $m_{6}$&
                        \texttt{jpIFNMrClqOhkEDabg}&
                        $q^{12}-q^{10}+q^{8}-2q^{6}+2q^{4}-2q^{2}+2-q^{-2}+q^{-4}$\\
                    \hline
                        $m_{7}$&
                        \texttt{cgjFHIaDEkb}&
                        $-q^{20}+q^{18}-q^{16}+2q^{14}-2q^{12}+2q^{10}-2q^{8}+2q^{6}-q^{4}+q^{2}$\\
                    \hline
                        $m_{8}$&
                        \texttt{knIHoBjCDQrMPaeLgF}&
                        $q^{16}-q^{14}+q^{12}-2q^{10}+2q^{8}-2q^{6}+2q^{4}-2q^{2}+2-q^{-2}+q^{-4}$\\
                    \hline
                        $m_{9}$&
                        jopIFMrDlqNhkEabcg&
                        $-q^{24}+q^{22}-q^{20}+2q^{18}-2q^{16}+2q^{14}-2q^{12}+2q^{10}-2q^{8}+2q^{6}-q^{4}+q^{2}$\\
                    \hline
                \end{tabular}%
            }
            \caption{Knots whose Jones polynomial matches that of a Twist Knot}
            \label{table:matching_twist_knots}
        \end{table}
        The twist knots have a few more matching non-twists knots for the
        Jones polynomial
        (Tab.~\ref{table:matching_twist_knots}). The first match is the
        figure-eight knot, $m_{2}$, which has the same Jones polynomial as
        K11n19 from the Hoste-Thistlewaite table.
        Again, this had already been known
        (See \cite{KatlasFigureEight} and \cite{KatlasK11n19}).
        The Khovanov polynomials are
        given in Tabs.~\ref{table:m_2_kho}-\ref{table:eikGbHJCaFd_kho}.
        \begin{table}[H]
            \centering
            \begin{tabular}{| c | c | c | c | c | c |}
                \hline
                $q\symbol{92}t$&$-2$&$-1$&$0$&$1$&$2$\\
                \hline
                $-5$&1&&&&\\
                \hline
                $-3$&&&&&\\
                \hline
                $-1$&&1&1&&\\
                \hline
                $1$&&&1&1&\\
                \hline
                $3$&&&&&\\
                \hline
                $5$&&&&&1\\
                \hline
            \end{tabular}
            \caption{Khovanov Polynomial for the Figure-Eight Knot}
            \label{table:m_2_kho}
        \end{table}
        \begin{table}[H]
            \centering
            \begin{tabular}{| c | c | c | c | c | c | c | c | c |}
                \hline
                $q\symbol{92}t$&$-4$&$-3$&$-2$&$-1$&$0$&$1$&$2$&$3$\\
                \hline
                $-5$&1&&&&&&&\\
                \hline
                $-3$&&&&&&&&\\
                \hline
                $-1$&&1&1&&&&&\\
                \hline
                $1$&&&&1&1&&&\\
                \hline
                $3$&&&&1&1&&&\\
                \hline
                $5$&&&&&1&1&1&\\
                \hline
                $7$&&&&&&&&\\
                \hline
                $9$&&&&&&&1&1\\
                \hline
            \end{tabular}
            \caption{Khovanov Polynomial for \texttt{eikGbHJCaFd}}
            \label{table:eikGbHJCaFd_kho}
        \end{table}
        The $m_{3}$ twist knot, which is $5_{2}$ on the Rolfsen table,
        has the same Jones polynomial as (at least) three other knots. In each
        case the Khovanov polynomial distinguishes it
        (Tabs.~\ref{table:m_3_kho}-\ref{table:hGJaMlCdEKBfI_kho}).
        \begin{table}[H]
            \centering
            \begin{tabular}{| c | c | c | c | c | c | c |}
                \hline
                $q\symbol{92}t$&$0$&$1$&$2$&$3$&$4$&$5$\\
                \hline
                $1$&1&&&&&\\
                \hline
                $3$&1&1&&&&\\
                \hline
                $5$&&&1&&&\\
                \hline
                $7$&&&1&&&\\
                \hline
                $9$&&&&1&1&\\
                \hline
                $11$&&&&&&\\
                \hline
                $13$&&&&&&1\\
                \hline
            \end{tabular}
            \caption{Khovanov Polynomial for the $5_{2}$ Knot}
            \label{table:m_3_kho}
        \end{table}
        \begin{table}[H]
            \centering
            \begin{tabular}{| c | c | c | c | c | c | c | c | c | c |}
                \hline
                $q\symbol{92}t$&$-2$&$-1$&$0$&$1$&$2$&$3$&$4$&$5$&$6$\\
                \hline
                $1$&1&&&&&&&&\\
                \hline
                $3$&&&&&&&&&\\
                \hline
                $5$&&1&2&&&&&&\\
                \hline
                $7$&&&1&1&1&&&&\\
                \hline
                $9$&&&&1&1&&&&\\
                \hline
                $11$&&&&&1&2&1&&\\
                \hline
                $13$&&&&&&1&1&1&\\
                \hline
                $15$&&&&&&&1&1&\\
                \hline
                $17$&&&&&&&&1&1\\
                \hline
            \end{tabular}
            \caption{Khovanov Polynomial for \texttt{dgikFHaEjbc}}
            \label{table:dgikFHaEjbc_kho}
        \end{table}
        \begin{table}[H]
            \centering
            \begin{tabular}{| c | c | c | c | c | c | c | c | c |}
                \hline
                $q\symbol{92}t$&$0$&$1$&$2$&$3$&$4$&$5$&$6$&$7$\\
                \hline
                $-1$&1&1&&&&&&\\
                \hline
                $1$&1&&&&&&&\\
                \hline
                $3$&&&1&1&&&&\\
                \hline
                $5$&&&&&1&&&\\
                \hline
                $7$&&&&&1&&&\\
                \hline
                $9$&&&&&&1&1&\\
                \hline
                $11$&&&&&&&&\\
                \hline
                $13$&&&&&&&&1\\
                \hline
            \end{tabular}
            \caption{Khovanov Polynomial for \texttt{gfJKHlaIEBCD}}
            \label{table:gfJKHlaIEBCD_kho}
        \end{table}
        \begin{table}[H]
            \centering
            \begin{tabular}{| c | c | c | c | c | c | c | c | c | c | c | c |}
                \hline
                $q\symbol{92}t$&$-2$&$-1$&$0$&$1$&$2$&$3$&$4$&$5$&$6$&$7$&$8$\\
                \hline
                $1$&1&&&&&&&&&&\\
                \hline
                $3$&&&&&&&&&&&\\
                \hline
                $5$&&1&2&&&&&&&&\\
                \hline
                $7$&&&1&1&1&&&&&&\\
                \hline
                $9$&&&&1&1&&&&&&\\
                \hline
                $11$&&&&&1&2&1&&&&\\
                \hline
                $13$&&&&&&1&1&1&&&\\
                \hline
                $15$&&&&&&&1&1&&&\\
                \hline
                $17$&&&&&&&&1&2&1&\\
                \hline
                $19$&&&&&&&&&&&\\
                \hline
                $21$&&&&&&&&&&1&1\\
                \hline
            \end{tabular}
            \caption{Khovanov Polynomial for \texttt{hGJaMlCdEKBfI}}
            \label{table:hGJaMlCdEKBfI_kho}
        \end{table}
        The $m_{5}$ twist knot, which is $7_{2}$ on the Rolfsen table, has the
        same Jones polynomial as \texttt{bhDGijCkaef}. The Khovanov polynomials
        are given in Tabs.~\ref{table:m_5_kho} and \ref{table:bhDGijCkaef_kho}.
        \begin{table}[H]
            \centering
            \begin{tabular}{| c | c | c | c | c | c | c | c | c |}
                \hline
                $q\symbol{92}t$&$0$&$1$&$2$&$3$&$4$&$5$&$6$&$7$\\
                \hline
                $1$&1&&&&&&&\\
                \hline
                $3$&1&1&&&&&&\\
                \hline
                $5$&&&1&&&&&\\
                \hline
                $7$&&&1&1&&&&\\
                \hline
                $9$&&&&1&1&&&\\
                \hline
                $11$&&&&&1&&&\\
                \hline
                $13$&&&&&&1&1&\\
                \hline
                $15$&&&&&&&&\\
                \hline
                $17$&&&&&&&&1\\
                \hline
            \end{tabular}
            \caption{Khovanov Polynomial for the $7_{2}$ Knot}
            \label{table:m_5_kho}
        \end{table}
        \begin{table}[H]
            \centering
            \begin{tabular}{| c | c | c | c | c | c | c | c | c |}
                \hline
                $q\symbol{92}t$&$-2$&$-1$&$0$&$1$&$2$&$3$&$4$&$5$\\
                \hline
                $1$&1&&&&&&&\\
                \hline
                $3$&&&&&&&&\\
                \hline
                $5$&&1&2&&&&&\\
                \hline
                $7$&&&1&&&&&\\
                \hline
                $9$&&&&1&1&&&\\
                \hline
                $11$&&&&&1&1&1&\\
                \hline
                $13$&&&&&&1&1&\\
                \hline
                $15$&&&&&&&1&1\\
                \hline
                $17$&&&&&&&&1\\
                \hline
            \end{tabular}
            \caption{Khovanov Polynomial for \texttt{bhDGijCkaef}}
            \label{table:bhDGijCkaef_kho}
        \end{table}
        Expanding our experiment from 17 crossings to 19 added a few new
        results, the first of which comes with
        $m_{6}$. An 18 crossing knot was found
        that has the same Jones polynomial, but the Khovanov polynomials
        differ. This is shown below
        (Tabs.~\ref{table:m_6_kho} and \ref{table:jpIFNMrClqOhkEDabg_kho}.).
        $m_{6}$ also shares its Jones polynomial with an 11 and a 13 crossing
        knot. The Khovanov polynomials are given in
        Tabs.~\ref{table:cefIgbajkDh_kho} and
        \ref{table:femIbaJKLCGHd_kho}.
        \begin{table}[H]
            \centering
            \begin{tabular}{| c | c | c | c | c | c | c | c | c | c |}
                \hline
                $q\symbol{92}t$&$-2$&$-1$&$0$&$1$&$2$&$3$&$4$&$5$&$6$\\
                \hline
                $-5$&1&&&&&&&&\\
                \hline
                $-3$&&&&&&&&&\\
                \hline
                $-1$&&1&2&&&&&&\\
                \hline
                $1$&&&1&1&&&&&\\
                \hline
                $3$&&&&1&1&&&&\\
                \hline
                $5$&&&&&1&1&&&\\
                \hline
                $7$&&&&&&1&&&\\
                \hline
                $9$&&&&&&&1&1&\\
                \hline
                $11$&&&&&&&&&\\
                \hline
                $13$&&&&&&&&&1\\
                \hline
            \end{tabular}
            \caption{Khovanov Polynomial for the $8_{1}$ Knot}
            \label{table:m_6_kho}
        \end{table}
        \begin{table}[H]
            \centering
            \begin{tabular}{| c | c | c | c | c | c | c | c | c | c |}
                \hline
                $q\symbol{92}t$&$-4$&$-3$&$-2$&$-1$&$0$&$1$&$2$&$3$&$4$\\
                \hline
                $-5$&1&&&&&&&&\\
                \hline
                $-3$&&&&&&&&&\\
                \hline
                $-1$&&1&2&&&&&&\\
                \hline
                $1$&&&&1&1&&&&\\
                \hline
                $3$&&&&2&2&&&&\\
                \hline
                $5$&&&&&1&2&1&&\\
                \hline
                $7$&&&&&&1&&&\\
                \hline
                $9$&&&&&&&2&2&\\
                \hline
                $11$&&&&&&&&&\\
                \hline
                $13$&&&&&&&&&1\\
                \hline
            \end{tabular}
            \caption{Khovanov Polynomial for \texttt{cefIgbajkDh}}
            \label{table:cefIgbajkDh_kho}
        \end{table}
        \begin{table}[H]
            \centering
            \begin{tabular}{| c | c | c | c | c | c | c | c | c | c |}
                \hline
                $q\symbol{92}t$&$-4$&$-3$&$-2$&$-1$&$0$&$1$&$2$&$3$&$4$\\
                \hline
                $-5$&1&&&&&&&&\\
                \hline
                $-3$&&&&&&&&&\\
                \hline
                $-1$&&1&2&&&&&&\\
                \hline
                $1$&&&&1&1&&&&\\
                \hline
                $3$&&&&2&2&&&&\\
                \hline
                $5$&&&&&1&2&1&&\\
                \hline
                $7$&&&&&&1&&&\\
                \hline
                $9$&&&&&&&2&2&\\
                \hline
                $11$&&&&&&&&&\\
                \hline
                $13$&&&&&&&&&1\\
                \hline
            \end{tabular}
            \caption{Khovanov Polynomial for \texttt{femIbaJKLCGHd}}
            \label{table:femIbaJKLCGHd_kho}
        \end{table}
        \begin{table}[H]
            \centering
            \begin{tabular}{| c | c | c | c | c | c | c | c | c | c | c | c | c | c |}
                \hline
                $q\symbol{92}t$&$-4$&$-3$&$-2$&$-1$&$0$&$1$&$2$&$3$&$4$&$5$&$6$&$7$&$8$\\
                \hline
                $-5$&1&&&&&&&&&&&&\\
                \hline
                $-3$&&&&&&&&&&&&&\\
                \hline
                $-1$&&1&2&&&&&&&&&&\\
                \hline
                $1$&&&&1&1&&&&&&&&\\
                \hline
                $3$&&&&2&2&&&&&&&&\\
                \hline
                $5$&&&&&2&4&2&&&&&&\\
                \hline
                $7$&&&&&&1&1&1&&&&&\\
                \hline
                $9$&&&&&&&3&4&1&&&&\\
                \hline
                $11$&&&&&&&&1&3&2&&&\\
                \hline
                $13$&&&&&&&&&2&1&&&\\
                \hline
                $15$&&&&&&&&&&2&3&1&\\
                \hline
                $17$&&&&&&&&&&&&&\\
                \hline
                $19$&&&&&&&&&&&&1&1\\
                \hline
            \end{tabular}
            \caption{Khovanov Polynomial for \texttt{jpIFNMrClqOhkEDabg}}
            \label{table:jpIFNMrClqOhkEDabg_kho}
        \end{table}
        Note the Khovanov polynomials of \texttt{cefIgbajkDh} and
        \texttt{femIbaJKLCGHd} are identical, showing this is not a perfect
        invariant.
        \par\hfill\par
        The knots $m_{7}$, $m_{8}$, and $m_{9}$ have the same Jones polynomial
        of at least one other knot. The Khovanov polynomials are given below.
        \begin{table}[H]
            \centering
            \begin{tabular}{| c | c | c | c | c | c | c | c | c | c | c |}
                \hline
                $q\symbol{92}t$&$0$&$1$&$2$&$3$&$4$&$5$&$6$&$7$&$8$&$9$\\
                \hline
                $1$&1&&&&&&&&&\\
                \hline
                $3$&1&1&&&&&&&&\\
                \hline
                $5$&&&1&&&&&&&\\
                \hline
                $7$&&&1&1&&&&&&\\
                \hline
                $9$&&&&1&1&&&&&\\
                \hline
                $11$&&&&&1&1&&&&\\
                \hline
                $13$&&&&&&1&1&&&\\
                \hline
                $15$&&&&&&&1&&&\\
                \hline
                $17$&&&&&&&&1&1&\\
                \hline
                $19$&&&&&&&&&&\\
                \hline
                $21$&&&&&&&&&&1\\
                \hline
            \end{tabular}
            \caption{Khovanov Polynomial for $m_{7}$}
            \label{table:m_7_kho}
        \end{table}
        \begin{table}[H]
            \centering
            \begin{tabular}{| c | c | c | c | c | c | c | c | c | c | c |}
                \hline
                $q\symbol{92}t$&$-2$&$-1$&$0$&$1$&$2$&$3$&$4$&$5$&$6$&$7$\\
                \hline
                $1$&1&&&&&&&&&\\
                \hline
                $3$&&&&&&&&&&\\
                \hline
                $5$&&1&2&&&&&&&\\
                \hline
                $7$&&&1&1&&&&&&\\
                \hline
                $9$&&&&1&1&&&&&\\
                \hline
                $11$&&&&&1&1&&&&\\
                \hline
                $13$&&&&&&1&1&&&\\
                \hline
                $15$&&&&&&&1&&&\\
                \hline
                $17$&&&&&&&&1&1&\\
                \hline
                $19$&&&&&&&&&&\\
                \hline
                $21$&&&&&&&&&&1\\
                \hline
            \end{tabular}
            \caption{Khovanov Polynomial for \texttt{cgjFHIaDEkb}}
            \label{table:cgjFHIaDEkb_kho}
        \end{table}
        \begin{table}[H]
            \centering
            \begin{tabular}{| c | c | c | c | c | c | c | c | c | c | c | c |}
                \hline
                $q\symbol{92}t$&$-2$&$-1$&$0$&$1$&$2$&$3$&$4$&$5$&$6$&$7$&$8$\\
                \hline
                $-5$&1&&&&&&&&&&\\
                \hline
                $-3$&&&&&&&&&&&\\
                \hline
                $-1$&&1&2&&&&&&&&\\
                \hline
                $1$&&&1&1&&&&&&&\\
                \hline
                $3$&&&&1&1&&&&&&\\
                \hline
                $5$&&&&&1&1&&&&&\\
                \hline
                $7$&&&&&&1&1&&&&\\
                \hline
                $9$&&&&&&&1&1&&&\\
                \hline
                $11$&&&&&&&&1&&&\\
                \hline
                $13$&&&&&&&&&1&1&\\
                \hline
                $15$&&&&&&&&&&&\\
                \hline
                $17$&&&&&&&&&&&1\\
                \hline
            \end{tabular}
            \caption{Khovanov Polynomial for $m_{8}$}
            \label{table:m_8_kho}
        \end{table}
        \begin{table}[H]
            \centering
            \begin{tabular}{| c | c | c | c | c | c | c | c | c | c | c | c | c | c |}
                \hline
                $q\symbol{92}t$&$-6$&$-5$&$-4$&$-3$&$-2$&$-1$&$0$&$1$&$2$&$3$&$4$&$5$&$6$\\
                \hline
                $-5$&1&&&&&&&&&&&&\\
                \hline
                $-3$&&&&&&&&&&&&&\\
                \hline
                $-1$&&1&2&&&&&&&&&&\\
                \hline
                $1$&&&&1&1&&&&&&&&\\
                \hline
                $3$&&&&2&1&&1&&&&&&\\
                \hline
                $5$&&&&&1&3&2&&&&&&\\
                \hline
                $7$&&&&&&1&&1&2&&&&\\
                \hline
                $9$&&&&&&&2&2&&&&&\\
                \hline
                $11$&&&&&&&&&1&3&1&&\\
                \hline
                $13$&&&&&&&&&1&&&1&\\
                \hline
                $15$&&&&&&&&&&&1&1&\\
                \hline
                $17$&&&&&&&&&&&&&1\\
                \hline
            \end{tabular}
            \caption{Khovanov Polynomial for \texttt{knIHoBjCDQrMPaeLgF}}
            \label{table:knIHoBjCDQrMPaeLgF_kho}
        \end{table}
        \begin{table}[H]
            \centering
            \begin{tabular}{| c | c | c | c | c | c | c | c | c | c | c | c | c |}
                \hline
                $q\symbol{92}t$&$0$&$1$&$2$&$3$&$4$&$5$&$6$&$7$&$8$&$9$&$10$&$11$\\
                \hline
                $1$&1&&&&&&&&&&&\\
                \hline
                $3$&1&1&&&&&&&&&&\\
                \hline
                $5$&&&1&&&&&&&&&\\
                \hline
                $7$&&&1&1&&&&&&&&\\
                \hline
                $9$&&&&1&1&&&&&&&\\
                \hline
                $11$&&&&&1&1&&&&&&\\
                \hline
                $13$&&&&&&1&1&&&&&\\
                \hline
                $15$&&&&&&&1&1&&&&\\
                \hline
                $17$&&&&&&&&1&1&&&\\
                \hline
                $19$&&&&&&&&&1&&&\\
                \hline
                $21$&&&&&&&&&&1&1&\\
                \hline
                $23$&&&&&&&&&&&&\\
                \hline
                $25$&&&&&&&&&&&&1\\
                \hline
            \end{tabular}
            \caption{Khovanov Polynomial for $m_{9}$}
            \label{table:m_9_kho}
        \end{table}
        \begin{table}[H]
            \centering
            \begin{tabular}{| c | c | c | c | c | c | c | c | c | c | c | c | c |}
                \hline
                $q\symbol{92}t$&$-2$&$-1$&$0$&$1$&$2$&$3$&$4$&$5$&$6$&$7$&$8$&$9$\\
                \hline
                $1$&1&&&&&&&&&&&\\
                \hline
                $3$&&&&&&&&&&&&\\
                \hline
                $5$&&1&2&&&&&&&&&\\
                \hline
                $7$&&&1&2&1&&&&&&&\\
                \hline
                $9$&&&&1&1&&&&&&&\\
                \hline
                $11$&&&&&2&3&1&&&&&\\
                \hline
                $13$&&&&&&1&2&1&&&&\\
                \hline
                $15$&&&&&&&2&2&&&&\\
                \hline
                $17$&&&&&&&&2&3&1&&\\
                \hline
                $19$&&&&&&&&&1&&&\\
                \hline
                $21$&&&&&&&&&&2&2&\\
                \hline
                $23$&&&&&&&&&&&&\\
                \hline
                $25$&&&&&&&&&&&&1\\
                \hline
            \end{tabular}
            \caption{Khovanov Polynomial for \texttt{jopIFMrDlqNhkEabcg}}
            \label{table:jopIFMrDlqNhkEabcg_kho}
        \end{table}
        We may now make the following claim.
        \begin{theorem}
            If a knot $K$ has less than or equal to 19 crossings and has the
            Khovanov homology of a twist knot $T$,
            then $K$ is equivalent to $T$.
        \end{theorem}
        An interesting thing to note is that not all twist knots are
        transversally, or Legendrian, simple, yet there are no knots up to
        19 crossings with the same Khovanov polynomial as a twist knot. We
        ask the following.
        \begin{question}
            Does Khovanov homology distinguish all twist knots?
        \end{question}
    \section{Acknowledgements}
        We thank Shadi Ali Ahmad, Ilya Kryukov and Jacob Swenberg for their
        observation and remarks in the early stage of this work. The numerical
        results of Samantha Allen and Jacob Swenberg about the Jones Polynomial
        of a connected sum of two Hopf links were very inspirational since they
        provided the connection to the work of Fan Ding and Hansjorg Geiges on
        the Legendrian simple links . We are very grateful to Nikolay Pultsin
        for his indispensable help with the Khovanov homology code available
        on the internet. We are thankful to Jim Hoste, Morwen Thistlethwaite
        and Jeff Weeks for sharing their database of DT codes of knots.
        We also thank Steven Sivek and Matthew Heddon
        for alarming us about the bug that
        existed in our KFH code.
    \newpage
    \bibliographystyle{plain}
    \bibliography{bib.bib}
    \newpage
    The source code used to generate this document, including figures,
    is free software and released under version 3 of the GNU General Public
    License.
    \par\hfill\par
    Vladimir Chernov
    \par
    6188 Kemeny Hall
    \par
    Mathematics Department, Dartmouth College
    \par
    Hanover, NH USA 03755
    \par
    Vladimir.Chernov@dartmouth.edu
    \par\hfill\par
    Ryan Maguire
    \par
    6188 Kemeny Hall
    \par
    Mathematics Department, Dartmouth College
    \par
    Hanover, NH USA 03755
    \par
    Ryan.J.Maguire.GR@dartmouth.edu
\end{document}
