%-----------------------------------LICENSE------------------------------------%
%   This file is part of Mathematics-and-Physics.                              %
%                                                                              %
%   Mathematics-and-Physics is free software: you can redistribute it and/or   %
%   modify it under the terms of the GNU General Public License as             %
%   published by the Free Software Foundation, either version 3 of the         %
%   License, or (at your option) any later version.                            %
%                                                                              %
%   Mathematics-and-Physics is distributed in the hope that it will be useful, %
%   but WITHOUT ANY WARRANTY; without even the implied warranty of             %
%   MERCHANTABILITY or FITNESS FOR A PARTICULAR PURPOSE.  See the              %
%   GNU General Public License for more details.                               %
%                                                                              %
%   You should have received a copy of the GNU General Public License along    %
%   with Mathematics-and-Physics.  If not, see <https://www.gnu.org/licenses/>.%
%------------------------------------------------------------------------------%
%       Author: Ryan Maguire                                                   %
%       Date:   2024/12/20                                                     %
%------------------------------------------------------------------------------%
\documentclass{article}
\usepackage{amsmath}
\usepackage{amsthm}
\usepackage{listings}
\usepackage{graphicx}
\usepackage{hyperref}
\hypersetup{colorlinks = true, linkcolor = blue}
\setlength{\parindent}{0em}
\setlength{\parskip}{0em}
\graphicspath{{../images/}}
\lstset{
    basicstyle = \small,
    basewidth = 4.5pt,
    linewidth = 360.0pt,
    showstringspaces = false
}
\theoremstyle{plain}
\newtheorem{theorem}{Theorem}

% Title page information.
\title{%
    Applications of Legendre Polynomials for
    Fresnel Inversion and Occultation Observations%
}
\author{%
    Ben Adenbaum\footnote{%
        Department of Mathematics,
        Florida Gulf Coast University, Fort Myers, FL 33965, USA
    }
    \and
    Ryan Maguire\footnote{%
        Department of Mathematics,
        MIT, Cambridge, MA 02139, USA
    }
}
\date{\today}
\begin{document}
    \maketitle
    \begin{abstract}
        We doing stuff.
    \end{abstract}
    \tableofcontents
    \section{The Cylindrical Fresnel Transform}
        Consider a plane wave of light incident on a cylindrically symmetric
        aperture that lies in the $z=0$ plane. Let $\mathbf{R}$ be the position
        vector of the observer, let $(\rho,\,\phi)$ denote the polar coordinates
        of the plane, and define $B$ to be the angle made between the plane and
        the line segment between the observer and the source of light.
        Diffraction effects are described by the Fresnel-Huygens principle
        given by the following integral equation:
        \begin{equation}
            \hat{T}(\rho_{0})
            =\frac{1-i}{2F}\int_{0}^{\infty}
                T(\rho)\exp\left(
                    i\psi\left(\rho,\,\phi_{s};\,\rho_{0},\,\phi_{0}\right)
                \right)\,\textrm{d}\rho
        \end{equation}
        where $F$ is the Fresnel scale,
        $(\rho_{0},\,\phi_{0})$ is the point of interest in the plane,
        $\psi$ is the Fresnel kernel, and $\phi_{s}$ is the stationary
        angle for $\psi$ (the solution to $\partial\psi/\partial\phi=0$).
        $\psi$ is a purely geometric quantity, given by:
        \begin{subequations}
            \begin{align}
                \xi
                &=\frac{\cos(B)}{D}\big(
                    \rho\cos(\phi)-\rho_{0}\cos(\phi_{0})
                \big)\\[1em]
                \eta
                &=\frac{%
                    \rho^{2}+\rho_{0}^{2}-2\rho\rho_{0}\cos(\phi-\phi_{0})%
                }{D^{2}}\\[1em]
                \psi
                &=kD\left(\sqrt{1-2\xi+\eta}+\xi-1\right)
            \end{align}
        \end{subequations}
        where $D=||\mathbf{R}-\mathbf{r}||$, with $\mathbf{r}$ being the
        position vector for $(\rho,\,\phi)$ in the plane. $\phi_{s}$ is
        approximated using Newton's method with starting guess
        $\phi_{s,\,1}=\phi_{0}$. The first iterate is then:
        \begin{equation}
            \phi_{s,\,1}
            =\phi_{0}-\frac{\psi^{\prime}_{0}}{\psi^{\prime\prime}_{0}}
        \end{equation}
        where $\psi^{\prime}_{0}$ and $\psi^{\prime\prime}_{0}$ denote the
        first and second partial derivatives of $\psi$ with respect to $\phi$
        evaluated at $\phi=\phi_{0}$. If $B=\frac{\pi}{2}$, $\phi_{s}=\phi_{0}$
        is exact, and if $B$ is not too small (say, greater than 30 degrees),
        the first iterate is sufficient for numerical purposes. Using a
        second order Taylor expansion yields the following approximation:
        \begin{subequations}
            \begin{align}
                \psi_{s}&
                \approx\psi_{0}+\psi^{\prime}_{0}(\phi_{s}-\phi_{0})
                +\frac{1}{2}\psi^{\prime\prime}_{0}(\phi_{s}-\phi_{0})^{2}\\
                &=\psi_{0}-\frac{1}{2}
                    \frac{\psi^{\prime\prime\,2}_{0}}{\psi^{\prime\prime}_{0}}
            \end{align}
        \end{subequations}
        where $\psi_{0}=\psi(\rho,\,\phi_{0};\,\rho_{0},\,\phi_{0})$.
        Defining the new variables $x$ and $\alpha$ via:
        \begin{subequations}
            \begin{align}
                \label{eqn:alpha_and_x}
                \alpha&=\cos(B)\cos(\phi_{0})\\
                x&=\frac{\rho-\rho_{0}}{D}
            \end{align}
        \end{subequations}
        then $\psi_{0}$ and $\psi^{\prime}_{0}$ have the following explicit
        formulas:
        \begin{subequations}
            \begin{align}
                \psi_{0}
                &=kD\left(
                    \sqrt{1-2\alpha{x}+x^{2}}+x-1
                \right)\\
                \psi^{\prime}_{0}
                &=kD\cos(B)\sin(\phi_{0})\frac{\rho}{D}\left(
                    \frac{1}{\sqrt{1-2\alpha{x}+x^{2}}}-1
                \right)
            \end{align}
        \end{subequations}
        Both of these expressions have expansions in terms of Legendre
        polynomials:
        \begin{subequations}
            \begin{align}
                \psi_{0}
                &=kD\sum_{n=0}^{\infty}
                    \frac{P_{n}(\alpha)-\alpha{P}_{n+1}(\alpha)}{n+2}x^{n+2}\\
                \psi^{\prime}_{0}
                &=kD\cos(B)\sin(\phi_{0})\frac{\rho}{D}\left(
                    \sum_{n=1}^{\infty}P_{n}(\alpha)x^{n}
                \right)^{2}
            \end{align}
        \end{subequations}
        $\psi^{\prime}_{0}$ can be further expanded using either a Cauchy
        product, or by introducing the Chebyshev polynomials of the second kind,
        $U_{n}(\alpha)$. We have:
        \begin{subequations}
            \begin{align}
                \psi^{\prime}_{0}
                &=kD\cos(B)\sin(\phi_{0})\frac{\rho}{D}
                    \sum_{n=0}^{\infty}\left(
                        \sum_{k=0}^{n}P_{n+1-k}(\alpha)P_{k+1}(\alpha)
                    \right)x^{n+2}\\
                    &=kD\cos(B)\sin(\phi_{0})\frac{\rho}{D}
                    \sum_{n=0}^{\infty}\left(
                        U_{n+2}(\alpha)-2P_{n+2}(\alpha)
                    \right)x^{n+2}
            \end{align}
        \end{subequations}
        $\psi$ can then be approximated using this. The Fresnel-Legendre
        polynomials are given by:
        \begin{equation}
            \label{eqn:fresnel_legendre_coeffs}
            L_{n}(\alpha,\,\beta)
            =\frac{P_{n}(\alpha)-\alpha{P}_{n+1}(\alpha)}{n+2}
                -\beta\sum_{n=0}^{n}P_{n+1-k}(\alpha)P_{k+1}(\alpha)
        \end{equation}
        where:
        \begin{equation}
            \beta=
            \frac{%
                \cos^{2}(B)\sin^{2}(\phi_{0})\rho^{2}
            }{%
                2(1-\cos^{2}(B)\sin^{2}(\phi_{0}))\rho_{0}^{2}
            }
        \end{equation}
        $\psi$ is then given by:
        \begin{equation}
            \psi_{s}\approx
            kD\sum_{n=0}^{\infty}L_{n}(\alpha,\,\beta)x^{n+2}
        \end{equation}
        We direct our attention to the Cauchy product term, labeling this
        $c_{n}(\alpha)$:
        \begin{equation}
            c_{n}(\alpha)=\sum_{k=0}^{n}P_{n+1-k}(\alpha)P_{k+1}(\alpha)
        \end{equation}
        Define $a_{n}$ to be the numerator of the leading coefficient of
        $c_{n}(\alpha)$ (after putting the coefficient in reduced form).
        We prove that this sequence is identical to OEIS A119951.
    \section{The Eta Triangle}
        The OEIS A119951 sequence is defined by:
        \begin{equation}
            \tilde{a}_{n}
            =\textrm{Numerator}\left(
                \sum_{k=0}^{n}\frac{C_{k+1}}{2^{2k}}
            \right)
        \end{equation}
        where $C_{k}$ are the Catalan numbers. We can expand this as follows:
        \begin{equation}
            \tilde{a}_{n}
            =\textrm{Numerator}\left(
                4-\frac{1}{2^{2n+1}}\binom{2n+4}{n+2}
            \right)
        \end{equation}
        Legendre polynomials and the Chebyshev polynomials of the second kind
        have well-known explicit formulas:
        \begin{subequations}
            \begin{align}
                U_{n}(\alpha)&=
                \sum_{k=0}^{\lfloor{n/2}\rfloor}
                    (-1)^{k}\binom{n-k}{k}(2x)^{n-2k}\\
                P_{n}(\alpha)&=
                \frac{1}{2^{n}}\sum_{k=0}^{\lfloor{n/2}\rfloor}
                    (-1)^{k}\binom{n}{k}\binom{2n-2k}{n}x^{n-2k}
            \end{align}
        \end{subequations}
        From this, we have:
        \begin{subequations}
            \begin{align}
                &\sum_{k=0}^{n}P_{n+1-k}(\alpha)P_{k+1}(\alpha)
                =U_{n+2}(\alpha)-2P_{n+2}(\alpha)\\
                &=\sum_{k=0}^{\lfloor{(n+2)/2}\rfloor}(-1)^{k}\alpha^{n+2-2k}
                \nonumber\\
                &\quad\quad\times\left(
                    \binom{n+2-k}{k}2^{n+2-2k}
                    -\frac{1}{2^{n+1}}\binom{n+2}{k}\binom{2n+4-2k}{n+2}
                \right)
            \end{align}
        \end{subequations}
        The leading term occurs when $k=0$, giving us:
        \begin{subequations}
            \begin{align}
                \textrm{LC}\left(
                    \sum_{k=0}^{n}P_{n+1-k}(\alpha)P_{k+1}(\alpha)
                \right)
                &=2^{n+2}-\frac{1}{2^{n+1}}\binom{2n+4}{n+2}\\
                &=2^{n}\left(
                    4-\frac{1}{2^{2n+1}}\binom{2n+4}{n+2}
                \right)
            \end{align}
        \end{subequations}
    \newpage
    \setcounter{secnumdepth}{0}
    \section{Appendix: Computing the sequence with \texttt{sympy}}
        The following script symbolically calculates the Fresnel approximation.
        It is written in Python using the \texttt{sympy} package.
        \begin{lstlisting}[language = Python]
import sympy
sympy.init_printing()

def chebyshev(Un, Un1, x):
    return 2*x*Un1 - Un

def legendre(Pn, Pn1, n, x):
    return ((2*n+3)*x*Pn1 - (n+1)*Pn) / (n + 2)

def get_value(poly):
    leading_coefficient = sympy.polys.polytools.LC(poly)
    numerator, _ = sympy.fraction(leading_coefficient)
    return numerator

alpha = sympy.symbols("alpha")
P = [1, alpha]
U = [1, 2*alpha]
c = []
a = []

for n in range(6):
    P.append(legendre(P[n], P[n + 1], n, alpha))
    U.append(chebyshev(U[n], U[n+1], alpha))
    c.append(sympy.simplify(U[n+2] - 2*P[n+2]))
    a.append(get_value(c[n]))

print(a)

def formula(n):
    n_factorial = sympy.factorial(n + 2)
    two_n_factorial = sympy.factorial(2*n + 4)
    power = 2**(n + 1)
    return (power - two_n_factorial / (2*power * n_factorial**2)) * sympy.gcd(n_factorial, 2*power)

def func(n):
    return (2**(2*n+3) - sympy.binomial(2*n+4, n+2)) / 2**(2*n+1)


        \end{lstlisting}
        Running this will print out \texttt{[1, 3, 29, 65, 281, 595]},
        in agreement with the first terms of OEIS A119951.
\end{document}
