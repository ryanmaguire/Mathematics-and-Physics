%-----------------------------------LICENSE------------------------------------%
%   This file is part of Mathematics-and-Physics.                              %
%                                                                              %
%   Mathematics-and-Physics is free software: you can redistribute it and/or   %
%   modify it under the terms of the GNU General Public License as             %
%   published by the Free Software Foundation, either version 3 of the         %
%   License, or (at your option) any later version.                            %
%                                                                              %
%   Mathematics-and-Physics is distributed in the hope that it will be useful, %
%   but WITHOUT ANY WARRANTY; without even the implied warranty of             %
%   MERCHANTABILITY or FITNESS FOR A PARTICULAR PURPOSE.  See the              %
%   GNU General Public License for more details.                               %
%                                                                              %
%   You should have received a copy of the GNU General Public License along    %
%   with Mathematics-and-Physics.  If not, see <https://www.gnu.org/licenses/>.%
%------------------------------------------------------------------------------%
%       Author: Ryan Maguire                                                   %
%       Date:   2024/12/02                                                     %
%------------------------------------------------------------------------------%
\documentclass{article}
\usepackage{amsmath}
\usepackage{amsthm}
\usepackage{amssymb}
\usepackage{listings}
\usepackage[framemethod=TikZ]{mdframed}
\usepackage{graphicx}
\usepackage{xcolor}
\usepackage{hyperref}
\hypersetup{colorlinks = true, linkcolor = blue}
\setlength{\parindent}{0em}
\setlength{\parskip}{0em}
\graphicspath{{../images/}}
\definecolor{lightgray}{rgb}{0.9, 0.9, 0.9}
\lstset{
    basicstyle = \small,
    basewidth = 6.0pt,
    linewidth = \textwidth,
    keywordstyle = \color{blue},
    captionpos = b,
    showstringspaces = false
}
\mdfsetup{%
    outerlinewidth = 1.25,%
    roundcorner = 6pt,%
    linecolor = blue!60!black,%
    backgroundcolor = black!10!white,%
    innertopmargin = -1,%
    innerbottommargin = -1,%
    innerleftmargin = 5
}

\lstdefinelanguage[x64]{Assembler}
{
    morekeywords = {
        mulsd, roundsd, xmm0, xmm1, add, addsd, sub, subsd,
        mov, movsd, movapd, quad, rsp, r
    }
}

\lstdefinelanguage[aarch64]{Assembler}
{
    morekeywords = {
        fmov, fmul, frintz, fmsub, ret
    }
}

\theoremstyle{plain}
\newtheorem{theorem}{Theorem}

% Title page information.
\title{%
    Fast and Accurate Evaluation of Trigonometric Functions
    with a Normalized Argument
}
\author{%
    Ryan Maguire\footnote{%
        Department of Mathematics,
        MIT, Cambridge, MA 02139, USA
    }
}
\date{\today}
\begin{document}
    \maketitle
    \begin{abstract}
        It is common in mathematics and physics to see expressions of the form
        $\sin(\pi{x})$ or $\cos(\pi{x})$ where $x$ is some unitless
        parameter. Simple examples include Fraunhofer diffraction, which uses
        the normalized $\textrm{sinc}$ function, and Fresnel diffraction, where
        the Fresnel cosine and Fresnel sine functions are employed.
        Algorithms such as the one described in
        \cite{NgArgumentReductionForPi} allow us to compute
        $\sin(x)$ and $\cos(x)$ accurately for all single or double precision
        floating point numbers, and modifications for extended and quadruple
        precisions have been implemented in \texttt{glibc}
        \cite{GLIBC}, \texttt{OpenLibm} \cite{OpenLibm}, and others
        (\cite{FreeBSDmsun},
        \cite{NetlibLibm},
        \cite{OpenBSDLibm}), but na\"{i}vely computing
        $\texttt{cos(M\_PI*x)}$ or $\texttt{sin(M\_PI*x)}$ can lead to
        serious precision loss. In this paper we describe a simple method for
        quickly evaluating trigonometric functions with argument
        $\pi{x}$, where $x$ is an IEEE-754 floating point number
        (32-bit single, 64-bit double, 80-bit extended, or 128-bit quadruple),
        or a 128-bit double-double.
    \end{abstract}
    \tableofcontents
    \section{The Algorithm}
        The standard mathematical library in the C programming language is
        required to provide $\cos$, $\sin$, and $\tan$ for inputs in radians.
        Some compilers for other languages provide these functions with inputs
        in degrees, usually naming the functions $\textrm{cosd}$,
        $\textrm{sind}$, and $\textrm{tand}$, respectively, but prior to the
        new \texttt{C23} standard, the functions $\textrm{cospi}$,
        $\textrm{sinpi}$, and $\textrm{tanpi}$ were often omitted. The
        \textit{normalized} variants are significantly easier to compute than
        there radian or degree counterparts.
        The trigonometric functions are periodic with period
        $2\pi$; the degree variants have period $360$, and the normalized ones
        have period $2$.
        Computations in radians thus requires
        computing $x\;\textrm{mod}\;\pi$, or some variant thereof, and this
        is a difficult challenge. The calculation in degrees is signficantly
        easier, but since $360$ is not a power of two the calculation
        $x\;\textrm{mod}\;360$ is more cumbersome than computing
        $x\;\textrm{mod}\;2$.\footnote{%
            No \texttt{C} compiler that we tested optimizes
            $\textrm{fmod}(x,\,360)$ in a way that is faster than doing out
            the computation explicitly. The generated assembly simply calls
            $\textrm{fmod}$, which often uses bit-tricks to perform the
            calculation. The same is true of $\textrm{fmod}(x,\,2)$.
        }
        \par\hfill\par
        Evaluation of $x\;\textrm{mod}\;2$ is very simple. With the
        standard library, one could do:
        \begin{mdframed}
            \begin{lstlisting}[language={C}]
#include <math.h>

double mod2(double x)
{
    return x - 2.0 * trunc(0.5 * x);
}
            \end{lstlisting}
        \end{mdframed}
        Since most instruction sets provide truncation operations, we could
        write this directly in assembly for a decent speed boost.
        \texttt{libtmpl} uses the following for \texttt{x86\_64} machines:
        \begin{mdframed}
            \begin{lstlisting}[language={[x64]Assembler}]
ASM_BEGIN(tmpl_Double_Mod_2)
    sub     $24, %rsp
    movsd   %xmm0, 8(%rsp)
    mulsd   .TMPL_DOUBLE_MOD2_HALF_AS_HEX(%rip), %xmm0
    roundsd $3, %xmm0, %xmm0
    movsd   8(%rsp), %xmm1
    add     $24, %rsp
    addsd   %xmm0, %xmm0
    subsd   %xmm0, %xmm1
    movapd  %xmm1, %xmm0
    ret

.TMPL_DOUBLE_MOD2_HALF_AS_HEX:
    .quad 0x3FE0000000000000
ASM_END(tmpl_Double_Mod_2)
            \end{lstlisting}
        \end{mdframed}
        \newpage
        The code for \texttt{aarch64} machines is shorter:
        \begin{mdframed}
            \begin{lstlisting}[language={[aarch64]Assembler}]
ASM_BEGIN(tmpl_Double_Mod_2)
    fmov    d31, 5.0e-1
    fmov    d30, 2.0e+0
    fmul    d31, d0, d31
    frintz  d31, d31
    fmsub   d0, d31, d30, d0
    ret
ASM_END(tmpl_Double_Mod_2)
            \end{lstlisting}
        \end{mdframed}
        The computation of $\cos(\pi{x})$, $\sin(\pi{x})$, and $\tan(\pi{x})$
        all follow the same scheme.
        \begin{enumerate}
            \item
                For \texttt{nan} or \texttt{inf}, return \texttt{nan}.
            \item
                If $|x|$ is very small, use the first term in the Taylor
                series expansion ($\pi{x}$ for $\sin$ and $\tan$, 1 for
                $\cos$). For slightly larger (say, less than $1/16$), a
                Maclaurin or Remez polynomial of small degree may be used.
            \item
                For small $|x|$, say $|x|<\frac{1}{2}$, a rational Remez
                approximation can be used, greatly speeding up the computation.
                Since $\sin(x)/x$, $\tan(x)/x$, and $\cos(x)$ are all even
                with respect to $x$, then degree $(2n,\,2m)$ rational Remez
                approximation requires $n+1$ non-zero terms in the numerator
                and $m+1$ non-zero terms in the denominator. We obtain a very
                accurate approximation using only a few terms.
            \item
                For all other inputs, reduce to $y=|x|\;\textrm{mod}\;2$.
                Write $y=r+\textrm{d}r$ where $r$ is an integer multiple of
                $2^{-7}$ and $|\textrm{d}r|<2^{-7}$. Compute using a lookup
                table, combined with the angle-sum formula.
        \end{enumerate}
        The bulk of the problem is the computation of $r$ and $\textrm{d}r$.
    \section{Sine of a Normalized Argument}
        We now give the full description for $\sin(\pi{x})$ assuming 64-bit
        floating point precision (1 bit for the sign, 11 bits for the exponent,
        and 52 bits for the mantissa).
        \par\hfill\par
        Let $\oplus$, $\ominus$, and $\otimes$ denote floating point addition,
        subtraction, and multiplication, respectively. We use \texttt{C} family
        syntax and write \texttt{>>} for bit-shifting to the right,
        $\veebar$ denotes bit-wise exclusive-or, and
        $\land$ represents bit-wise and.
        After setting $y=|x|\;\textrm{mod}\;2$, we compute:
        \begin{equation}
            s=y\oplus{2}^{45}
        \end{equation}
        Since $0\leq{y}<2$, and since floating point arithmetic rounds the
        result, the last eight bits of $s$ now store the values of the one's bit
        down to the $2^{-7}$ bit in $y$, correctly rounded. We create our
        lookup tables by precomputing $\cos(\pi{r}_{n})$ and
        $\sin(\pi{r}_{n})$ with $r_{n}=n\times2^{-7}$ for
        $n=0,\,1,\,\dots,\,127$. The value $n$ is then just the last 7 bits of
        $s$, read in binary as a whole number. Let $\textrm{LEB}$ denote the
        \textit{last eight bits} of a floating point number, and let
        $\textrm{SIGN}(x)$ be the sign bit of the floating point number $x$
        (zero if positive, one if negative).
        \begin{align}
            \ell&=\textrm{LEB}(s)\\
            n&=\ell\land\textrm{0x7F}\\
            \textrm{neg}&=(\ell\;\texttt{>>}\;7)\veebar\textrm{SIGN}(x)\\
            \textrm{d}r&=y\ominus(s\ominus2^{45})
        \end{align}
        $n$ is now the index for the lookup table, and $\textrm{neg}$ determines
        if we need to negate the output, but it is not obvious as to why.
        The last nine bits of $s$ determine a number $r$ such that
        $0\leq{r}\leq{2}$, and such that the least significant bit is not
        smaller than the $2^{-7}$ place. That is, since $0\leq{y}<2$, and since
        $s=y\oplus{2}^{45}=y\oplus{2}^{53-8}$, the last nine bits of $s$
        correspond to $y$ correctly rounded to the $2^{-7}$ bit. It is hence
        possible for $r=2$ to occur if, say, $y=2-2^{-8}$. In this case, the
        lowest eight bits are zero, and the ninth bit is one.
        \par\hfill\par
        Should the eighth bit (which is $\ell\;\texttt{>>}\;7$)
        be a zero, then either $0\leq{r}<1$, or $r=2$.
        The \textrm{neg} Boolean is $0\veebar\texttt{SIGN}(x)=\texttt{SIGN}(x)$,
        meaning we only negate the output if the sign bit of $x$ is set to 1.
        In the case $0\leq{r}<1$ this makes sense since the index for the
        lookup table corresponds to precisely the values of $r$.
        In the latter case the index is zero, $\textrm{d}r$ is negative,
        and we have:
        \begin{align}
            \sin(\pi{y})
            &=\sin\Big(\pi(r+\textrm{d}r)\Big)\\
            &=\sin(\pi{r}+\pi\textrm{d}r)\\
            &=\cos(\pi{r})\sin(\pi\textrm{d}r)
                +\sin(\pi{r})\cos(\pi\textrm{d}r)\\
            &\approx
            \textrm{cos\_table}[0]\otimes\sin(\pi\textrm{d}r)
                \oplus
                \textrm{sin\_table}[0]\otimes\cos(\pi\textrm{d}r)\\
            &=\sin(\pi\textrm{d}r)
        \end{align}
        We should again only negate this if $x$ is negative.
        \par\hfill\par
        If the eight bit is 1, then $1\leq{r}<2$. In this case, the index for
        the table is the integer $n$ such that $r=1+n\times{2}^{-7}$. Examining
        the lowest 7 bits of $s$ will again retrieve this value, and we may
        resume the computation using:
        \begin{equation}
            \sin(\pi{y})=-\sin\Big(\pi(y-1)\Big)
        \end{equation}
        We pick a minus sign in this reduction. Since double negatives cancle,
        we only need to negate the final output if either $x$ is negative,
        or the eigth bit is 1, but not if both occur simultaneously. This is
        the exclusive or, which justifies setting
        $\textrm{neg}=(\ell\;\texttt{>>}\;7)\veebar\textrm{SIGN}(x)$.
        \par\hfill\par
        Note that if the index been computed by truncating $y$,
        zeroing all of the bits smaller than $2^{-7}$,
        then the index would be $127$ and $\textrm{d}r$
        would be positive. The expression
        $%
            \textrm{costable}[127]\otimes\sin(\pi\textrm{d}r)%
            \oplus%
            \textrm{sintable}[0]\otimes\cos(\pi\textrm{d}r)%
        $
        rounds the result three times (once for each floating point operation),
        and would cause very poor relative error for values $y<2$ that are very
        close to $2$. The issue disappears entirely by using rounding since
        $\cos(0)=1$ and $\sin(0)=0$, meaning the only error occurs in the
        evaluation of $\sin(\pi\textrm{d}r)$. This is done with a Maclaurin
        series, meaning as $y\rightarrow{2}$ we maintain excellent relative
        error. Experiment shows the peak error is bounded by 2 ULP,
        and the root-mean-square error is less then
        half a ULP.
    \section{The Program}
        The function is so short that it can be included into this paper
        (with comments removed). We have removed the middle lines of the
        code that check if the input is \texttt{nan} or \texttt{inf},
        and also handle small inputs using Remez approximations.
        The full source (with comments) is available in \cite{MaguireLibtmpl}.
        \par\hfill\par
        We access the bits of a double by type punning. The struct used for
        this is defined as follows (assuming little endianness):
        \begin{mdframed}
            \begin{lstlisting}[language={C}]
typedef union tmpl_IEEE754_Double_Def {
    struct {
        unsigned int man3 : 16;
        unsigned int man2 : 16;
        unsigned int man1 : 16;
        unsigned int man0 : 4;
        unsigned int expo : 11;
        unsigned int sign : 1;
    } bits;
    double r;
} tmpl_IEEE754_Double;
            \end{lstlisting}
        \end{mdframed}
        The definition is quite portable since bit-fields are required to
        store at least as many bits as \texttt{unsigned int}, which the
        standard requires to be at least 16. By making the most significant
        part of the mantissa, \texttt{man0}, 4 bits and placing it next to the
        12-bit sign and exponent block, we eliminate the likelyhood that
        the struct will be padded in any way, resulting in valid type punning.
        Every single compiler tested (\texttt{GCC}, \texttt{clang},
        \texttt{TCC}, \texttt{PCC}, \texttt{aocc}, \texttt{icc}, \texttt{MSVC})
        handles this struct just fine.
        \par\hfill\par
        The setup is small, just a few declarations for tables and functions.
        In what follows, the symbolic constant \texttt{TMPL\_STATIC\_INLINE}
        expands to \texttt{static inline} on compilers supporting the
        \texttt{inline} keyword, and \texttt{static} otherwise. The
        definitions for the $\cos$ and $\sin$ Maclaurin series use
        5 non-zero terms evaluated with Horner's method.
        \begin{mdframed}
            \begin{lstlisting}[language={C}]
extern const double tmpl_double_cospi_table[128];
extern const double tmpl_double_sinpi_table[128];
extern double tmpl_Double_Mod_2(double x);

TMPL_STATIC_INLINE
double tmpl_Double_CosPi_Maclaurin(double x){...}

TMPL_STATIC_INLINE
double tmpl_Double_SinPi_Maclaurin(double x){...}
            \end{lstlisting}
        \end{mdframed}
        The $\textrm{sinpi}$ function is computed as follows.
        \begin{mdframed}
            \begin{lstlisting}[language={C}]
double tmpl_Double_SinPi(double x)
{
    double cos_pi_r, cos_pi_dr, sin_pi_r, sin_pi_dr;
    double dr, out;
    unsigned int lowest_eight_bits, index;
    unsigned int negate, sign;
    tmpl_IEEE754_Double w, shifted;
    const double shifter = 3.5184372088832E+13;

    w.r = x;

    /* Handle nan, inf, and small x, not shown here. */

    sign = w.bits.sign;
    w.bits.sign = 0x00U;

    w.r = tmpl_Double_Mod_2(w.r);
    shifted.r = w.r + shifter;

    lowest_eight_bits = (0x00FFU & shifted.bits.man3);
    negate = (lowest_eight_bits >> 7U) ^ sign;
    index = (lowest_eight_bits & 0x007FU);
    dr = w.r - (shifted.r - shifter);

    sin_pi_r = tmpl_double_sinpi_table[index];
    cos_pi_r = tmpl_double_cospi_table[index];
    sin_pi_dr = tmpl_Double_SinPi_Maclaurin(dr);
    cos_pi_dr = tmpl_Double_CosPi_Maclaurin(dr);
    out = cos_pi_r*sin_pi_dr + sin_pi_r*cos_pi_dr;

    if (negate)
        return -out;

    return out;
}
            \end{lstlisting}
        \end{mdframed}

    \section{Verifying Accuracy}
    \section{Relative Performance}
    \bibliographystyle{plain}
    \bibliography{bib.bib}
\end{document}
