%-----------------------------------LICENSE------------------------------------%
%   This file is part of Mathematics-and-Physics.                              %
%                                                                              %
%   Mathematics-and-Physics is free software: you can redistribute it and/or   %
%   modify it under the terms of the GNU General Public License as             %
%   published by the Free Software Foundation, either version 3 of the         %
%   License, or (at your option) any later version.                            %
%                                                                              %
%   Mathematics-and-Physics is distributed in the hope that it will be useful, %
%   but WITHOUT ANY WARRANTY; without even the implied warranty of             %
%   MERCHANTABILITY or FITNESS FOR A PARTICULAR PURPOSE.  See the              %
%   GNU General Public License for more details.                               %
%                                                                              %
%   You should have received a copy of the GNU General Public License along    %
%   with Mathematics-and-Physics.  If not, see <https://www.gnu.org/licenses/>.%
%------------------------------------------------------------------------------%
%       Author: Ryan Maguire                                                   %
%       Date:   2024/12/02                                                     %
%------------------------------------------------------------------------------%
\documentclass{article}
\usepackage{amsmath}
\usepackage{amsthm}
\usepackage{listings}
\usepackage{graphicx}
\usepackage{hyperref}
\hypersetup{colorlinks = true, linkcolor = blue}
\setlength{\parindent}{0em}
\setlength{\parskip}{0em}
\graphicspath{{../images/}}
\lstset{
    basicstyle = \small,
    basewidth = 4.5pt,
    linewidth = 360.0pt,
    showstringspaces = false
}
\theoremstyle{plain}
\newtheorem{theorem}{Theorem}

% Title page information.
\title{%
    Applications of Legendre Polynomials for
    Fresnel Inversion and Occultation Observations%
}
\author{%
    Ryan Maguire\footnote{%
        Department of Mathematics,
        MIT, Cambridge, MA 02139, USA
    }
    \and
    Richard French\footnote{
        Department of Physics and Astronomy,
        Wellesley College, Wellelsey, MA 02481, USA
    }
}
\date{\today}
\begin{document}
    \maketitle
    \begin{abstract}
        Radio occultations can be used to study the rings of Saturn and
        Uranus. The wavelength of the signal emitted from a space probe is
        large enough that significant diffraction effects occur,
        and the data must be processed numerically in order to construct a
        meaningful ring profile. The classic Fresnel approximation uses a
        quadratic polynomial for the Fresnel kernel, and in some real cases
        this is sufficient (i.e. Cassini's Rev 7, Voyager's fly-by of Uranus),
        but certain scenarios require higher order terms in order to perform
        accurate reconstructions. By following the derivation of the
        Fresnel approximation carefully, one can obtain a polynomial expansion
        for the Fresnel kernel in terms of Legendre polynomials, the degree
        of which can be arbitrarily high. Quartic and octic
        expansions are sufficient for the vast majority of the Cassini data
        sets, and these approximations require only a small amount of
        additional computational time when compared to the quadratic Fresnel
        method.
    \end{abstract}
    \tableofcontents
    \section{Fourier Optics in a Ring System}
        The geometry of a radio occultation observation is determined by the
        ring plane of the planet, the trajectory of the spacecraft, and the
        location of Earth. Let $(\rho,\,\phi)$ denote the cylindrical
        coordinates of the ring plane, let $\mathbf{R}$ be the position
        vector of the spacecraft with respect to the core of the planet, and
        let $B$ indicate the ring opening angle, which is the angle made
        by the ring plane and the line segment between the spacecraft and
        Earth. Since the distance between Earth and the other planets is
        several orders of magnitute greater than the length of $\mathbf{R}$,
        or even the diameter of the entire ring system, the angle $B$ may be
        treated as a constant.
        \par\hfill\par
        Given a plane wave incident on the rings, the diffracted complex
        transmittance $\hat{T}(\rho_{0},\,\phi_{0})$ at the point
        $(\rho_{0},\,\phi_{0})$ is given by:
        \begin{equation}
            \label{eqn:huygens_principal}
            \hat{T}(\rho_{0},\,\phi_{0})
            =\frac{\mu_{0}}{i\lambda}\int_{0}^{\infty}\int_{0}^{2\pi}
                T(\rho,\,\phi)
                \frac{%
                    \exp\left(
                        i\,\psi(\rho_{0},\,\phi_{0};\,\rho,\,\phi)
                    \right)
                }{%
                    ||\mathbf{R}-\mathbf{r}||_{2}
                }
                \rho\,\textrm{d}\rho\,\textrm{d}\phi
        \end{equation}
        where $\mathbf{r}$ is the position vector for the point
        $(\rho,\,\phi)$ in the ring plane, $T$ is the complex transmittance,
        and where $||\mathbf{R}-\mathbf{r}||_{2}$ is the standard
        Euclidean distance between this point and the spacecraft.
        For convenience, we label this distance $D$:
        \begin{equation}
            D=||\mathbf{R}-\mathbf{r}||_{2}
        \end{equation}
        The quantity $\psi$ is purely geometric, and the complex
        exponental $\exp(i\,\psi)$ is the \textit{Fresnel kernel} (the
        reader should note that the function $\psi$ is also called the Fresnel
        kernel by many authors).
        Following MTR86, $\psi$ is given in terms of auxiliary parameters
        $\hat{\xi}$ and $\hat{\eta}$. We write:
        \begin{align}
            \hat{\xi}
            &=\frac{\cos(B)}{D}\big(
                \rho_{0}\cos(\phi_{0})-\rho\cos(\phi)
            \big)\\[1em]
            \hat{\eta}
            &=\frac{%
                \rho^{2}+\rho_{0}^{2}-2\rho\rho_{0}\cos(\phi-\phi_{0})%
            }{D^{2}}\\[1em]
            \psi
            &=kD\left(
                \sqrt{1+2\hat{\xi}+\hat{\eta}}-\hat{\xi}-1
            \right)
        \end{align}
        Where $k$ is the wavenumber, $k=2\pi/\lambda$, with $\lambda$ being
        the wavelength of the incident plane wave.
        We define $\xi$ and $\eta$ using this, and rewrite $\psi$ accordingly:
        \begin{align}
            \xi&=-\hat{\xi}\\
            \eta&=\hat{\eta}\\
            \label{eqn:fresnel_kernel_psi}
            \psi&=kD\left(
                \sqrt{1-2\xi+\eta}+\xi-1
            \right)
        \end{align}
        The reason for the subtle change in the notation used by other
        authors is that it allows us to rewrite $\psi$ and its derivatives
        more naturally in terms of Legendre polynomials later on. Indeed,
        the generating function for the Legendre polynomials is given by:
        \begin{equation}
            \sum_{n=0}^{\infty}P_{n}(x)t^{n}
            =\frac{1}{\sqrt{1-2xt+t^{2}}}
        \end{equation}
        and one can already start to see some similarities between this
        expression and our new formula for $\psi$.
        \par\hfill\par
        By assuming perfectly symmetric rings, we may reduce
        Eqn.~\ref{eqn:huygens_principal} to the following:
        \begin{subequations}
            \begin{align}
                \hat{T}(\rho_{0})
                &=\frac{\mu_{0}}{i\lambda}\int_{0}^{\infty}\int_{0}^{2\pi}
                    T(\rho)
                    \frac{%
                        \exp\left(
                            i\,\psi(\rho_{0},\,\phi_{0};\,\rho,\,\phi)
                        \right)
                    }{%
                        ||\mathbf{R}-\mathbf{r}||_{2}
                    }
                    \rho\,\textrm{d}\rho\,\textrm{d}\phi\\
                &=\frac{\mu_{0}}{i\lambda}\int_{0}^{\infty}
                    \rho\,T(\rho)\left(
                        \int_{0}^{2\pi}
                        \frac{%
                            \exp\left(
                                i\,\psi(\rho_{0},\,\phi_{0};\,\rho,\,\phi)
                            \right)
                        }{%
                            ||\mathbf{R}-\mathbf{r}||_{2}
                        }
                        \textrm{d}\phi
                    \right)
                    \textrm{d}\rho
                \end{align}
        \end{subequations}
        Such an assumption is valid for Saturn's rings, and most of the rings
        of Uranus. Elliptical rings are not treated in this paper.
        \par\hfill\par
        The stationary phase approximation allows us to reduce this double
        integral to a single integral, eliminating the inner integration over
        the azimuthal component $\phi$. Let $\phi_{s}$ denote the stationary
        azimuthal angle, which is the solution to
        $\partial\psi/\partial\phi=0$ as a function of the other
        parameters. The reduction becomes:
        \begin{equation}
            \label{eqn:single_variable_huygens}
            \hat{T}(\rho_{0})
            =\frac{1-i}{2F}\int_{0}^{\infty}
                T(\rho)\exp\left(
                    i\,\psi(\rho_{0},\,\phi_{0};\,\rho,\,\phi)
                \right)
                \,\textrm{d}\rho
        \end{equation}
        where $F$ is the characteristic Fresnel scale:
        \begin{equation}
            \label{eqn:fresnel_scale}
            F=
            \sqrt{%
                \frac{\lambda{D}}{2}%
                \frac{1-\cos^{2}(B)\sin^{2}(\phi_{0})}{\sin^{2}(B)}%
            }
        \end{equation}
        Note that in factoring $F$ \textit{outside} of the integral for
        $\rho$, we are assuming the parameter varies very little with $\rho$.
        This is the case when $||\mathbf{R}||_{2}>>||\mathbf{r}||_{2}$, but
        for high resolution reconstructions this need not hold.
        Such an approximation is valid if we may
        treat $D$ as the following constant:
        \begin{equation}
            \label{eqn:d_approx}
            D\approx||\mathbf{R}-\mathbf{r}_{0}||_{2}
        \end{equation}
        The limits in the integral in Eqn.~\ref{eqn:single_variable_huygens}
        are changed dramatically when one considers real data. The true
        integral takes place over a range $|\rho-\rho_{0}|<W/2$, where $W$ is
        the window width, which is determined by the desired resolution.
        For most reasonable scenarios, we have $W<<||\mathbf{R}||_{2}$,
        meaning the approximation in Eqn.~\ref{eqn:d_approx} is valid.
        \par\hfill\par
        To obtain ring profiles we must solve this integral equation for $T$;
        $\hat{T}$ is the measured diffraction data.
        The geometric variables $\rho_{0}$, $\phi_{0}$, $B$, and $D$ are also
        measured quantities, and $\phi_{s}$ is computed numerically.
        The remaining variable $\rho$ is just a dummy variable of integration.
    \section{The Fresnel Approximation}
        Computing the stationary azimuth angle $\phi_{s}$ for $\psi$ can be
        done using Newton's method. Starting with
        $\phi_{s,\,0}=\phi_{0}$, we obtained an improved guess via:
        \begin{equation}
            \phi_{s,\,n+1}=\phi_{s,\,n}-
            \frac{\partial\psi/\partial\phi}
                 {\partial^{2}\psi/\partial\phi^{2}}
        \end{equation}
        We may expand $\psi$ in terms of the $\phi$ using the Taylor
        series:
        \begin{equation}
            \psi(\rho,\,\phi;\,\rho_{0},\,\phi_{0})
            =\sum_{n=0}^{\infty}
                \frac{\psi^{(n)}(\rho,\,\phi_{0};\,\rho_{0},\,\phi_{0})}{n!}
                (\phi-\phi_{0})^{n}
        \end{equation}
        where $\psi^{(n)}$ denotes the $n^{\textrm{th}}$ partial derivative of
        $\psi$ with respect to $\phi$. Truncating to the quadratic term yields
        a nice approximation:
        \begin{equation}
            \label{eqn:quadratic_taylor}
            \psi(\rho,\,\phi;\,\rho_{0},\,\phi_{0})
            \approx
            \psi_{0}+\psi^{\prime}_{0}(\phi-\phi_{0})
            +\frac{1}{2}\psi^{\prime\prime}_{0}(\phi-\phi_{0})^{2}
        \end{equation}
        where $\psi_{0}$, $\psi^{\prime}_{0}$, and $\psi^{\prime\prime}_{0}$
        denote the zeroth, first, and second partial derivatives of $\psi$
        evaluated at $(\rho,\,\phi_{0};\,\rho_{0},\,\phi_{0})$, respectively.
        That is, we substitute $\phi=\phi_{0}$ into
        Eqn.~\ref{eqn:fresnel_kernel_psi}, and similarly for the partial
        derivatives of $\psi$.
        If we assume that the first Newton iterate is sufficient for numerical
        purposes (i.e. $\phi_{s}=\phi_{s,\,1}$), we obtain:
        \begin{equation}
            \phi_{s}-\phi_{0}
            =\left(
                \phi_{0}-
                \frac{\psi^{\prime}_{0}}{\psi^{\prime\prime}_{0}}
            \right)-\phi_{0}
            =-\frac{\psi^{\prime}_{0}}{\psi^{\prime\prime}_{0}}
        \end{equation}
        Plugging this into the quadratic approximation
        (Eqn.~\ref{eqn:quadratic_taylor}) yields:
        \begin{subequations}
            \begin{align}
                \psi(\rho,\,\phi_{s};\,\rho_{0},\,\phi_{0})
                &\approx\psi_{0}+\psi^{\prime}_{0}(\phi_{s}-\phi_{0})
                    +\frac{1}{2}\psi^{\prime\prime}_{0}(\phi_{s}-\phi_{0})^{2}\\
                &=\psi_{0}+\psi^{\prime}_{s}\left(
                    -\frac{\psi^{\prime}_{0}}{\psi^{\prime\prime}_{0}}
                \right)+
                \frac{1}{2}\psi^{\prime\prime}_{s}\left(
                    -\frac{\psi^{\prime}_{0}}{\psi^{\prime\prime}_{0}}
                \right)^{2}\\
                &=\psi_{0}
                    -\frac{1}{2}
                    \frac{\psi^{\prime\,2}_{0}}{\psi^{\prime\prime}_{0}}
            \end{align}
        \end{subequations}
        The quadratic Fresnel approximation can be obtained once we
        evaluate these derivatives. Substituting $\phi=\phi_{0}$
        into $\xi$ and $\eta$ produces:
        \begin{align}
            \xi_{\phi=\phi_{0}}
            &=\cos(B)\cos(\phi_{0})\left(
                \frac{\rho-\rho_{0}}{D}
            \right)\\
            \eta_{\phi=\phi_{0}}
            &=\left(\frac{\rho-\rho_{0}}{D}\right)^{2}
        \end{align}
        Define the auxiliary parameters $\alpha$ and $x$ by:
        \begin{align}
            \label{eqn:alpha_and_x}
            \alpha&=\cos(B)\cos(\phi_{0})\\
            x&=\frac{\rho-\rho_{0}}{D}
        \end{align}
        The formulas for $\xi$, $\eta$, and $\psi_{0}$ then become:
        \begin{align}
            \xi_{\phi=\phi_{0}}
            &=\alpha{x}\\
            \eta_{\phi=\phi_{0}}
            &=x^{2}\\
            \psi_{\phi=\phi_{0}}
            &=kD\left(
                \sqrt{1-2\alpha{x}+x^{2}}
                +\alpha{x}-1
            \right)
        \end{align}
        The observant reader may note how strikingly similar our new formula is
        to the generating function for the Legendre polynomials. We will
        touch upon that in the next section.
        \par\hfill\par
        The partial derivatives for $\psi$ can be obtained from the partial
        derivatives of $\xi$ and $\eta$. For $\xi$ we have:
        \begin{align}
            \xi^{\prime}
            &=-\cos(B)\sin(\phi)\frac{\rho}{D}\\
            \xi^{\prime\prime}
            &=-\cos(B)\cos(\phi)\frac{\rho}{D}
        \end{align}
        and for $\eta$ the derivatives are:
        \begin{align}
            \eta^{\prime}
            &=\frac{2\rho\rho_{0}\sin(\phi-\phi_{0})}{D^{2}}\\
            \eta^{\prime\prime}
            &=\frac{2\rho\rho_{0}\cos(\phi-\phi_{0})}{D^{2}}
        \end{align}
        The first partial derivative of $\psi$ is then:
        \begin{align}
            \psi^{\prime}
            &=kD\left(
                \frac{-2\xi^{\prime}+\eta^{\prime}}{2\sqrt{1-2\xi+\eta}}
                +\xi^{\prime}
            \right)
        \end{align}
        Plugging in $\phi=\phi_{0}$ yields:
        \begin{equation}
            \psi^{\prime}_{0}
            =kD\cos(B)\sin(\phi_{0})\frac{\rho}{D}\left(
                \frac{1}{\sqrt{1-2\alpha{x}+x^{2}}}-1
            \right)
        \end{equation}
        The second partial derivative isn't quite as nice:
        \begin{equation}
            \psi^{\prime\prime}
            =kD\left(
                    \frac{-2\xi^{\prime\prime}+\eta^{\prime\prime}}
                          {2(1-2\xi+\eta)^{1/2}}-
                    \frac{(-2\xi^{\prime}+\eta^{\prime})^{2}}
                         {4(1-2\xi+\eta)^{3/2}}+
                    \xi^{\prime\prime}
            \right)
        \end{equation}
        Again substituting $\phi=\phi_{0}$, we obtain:
        \begin{equation}
            \psi^{\prime\prime}_{0}
            =kD\left(
                \frac{%
                    -\alpha\rho{D}+\rho\rho_{0}
                }{%
                    D^{2}(1-2\alpha{x}+x^{2})^{1/2}
                }
                -\frac{%
                    \cos^{2}(B)\sin^{2}(\phi_{0})\rho^{2}
                }{%
                    D^{2}(1-2\alpha{x}+x^{2})^{3/2}
                }
                -\frac{\alpha\rho}{D}
            \right)
        \end{equation}
        Considering the approximation
        $%
            \psi\approx%
            \psi_{0}-\frac{1}{2}\psi^{\prime\,2}_{0}/\psi^{\prime\prime}_{0}%
        $,
        if we compute the Taylor series for this expression in terms of
        $\rho$, centered about $\rho_{0}$, one finds the that constant and
        linear terms are zero, and quadratic factor is:
        \begin{equation}
            \psi_{F}=
            \frac{1}{2}
            \frac{kD\sin^{2}(B)}{1-\sin^{2}(\phi_{0})\cos^{2}(B)}
            \left(
                \frac{\rho-\rho_{0}}{D}
            \right)^{2}
        \end{equation}
        Rewriting this in terms of the Fresnel scale $F$
        (Eqn.\ref{eqn:fresnel_scale}) yields the Fresnel approximation:
        \begin{equation}
            \psi_{F}
            =\frac{\pi}{2}\left(
                \frac{\rho-\rho_{0}}{F}
            \right)^{2}
        \end{equation}
        Doing this calculation by hand quickly becomes in exercise in tedium.
        Using the \texttt{sympy} package in Python can greatly ease this burden.
        A short script for this can be found in the appendix.
        \par\hfill\par
        The quadratic Fresnel transform is then:
        \begin{equation}
            \hat{T}(\rho_{0})
            =\frac{1-i}{2F}\int_{0}^{\infty}
                T(\rho)\,\exp\left(
                    \frac{i\pi}{2}\left(\frac{\rho-\rho_{0}}{F}\right)^{2}
                \right)\,\textrm{d}\rho
        \end{equation}
        In practice, the limits of the integral are reduced to
        $\rho_{0}\pm{W}/2$, where $W$ is the window width given by the
        desired resolution. Because of this, we may assume
        $T(\rho)$ is zero for sufficiently large $\rho$.
        We may also extend the domain of $T$ to the negative real numbers, and
        define $T(\rho)=0$ whenever $\rho<0$. These two assumptions mean that
        $T$ is compactly supported. We now rewrite our transform as follows:
        \begin{equation}
            \hat{T}(\rho_{0})
            =\frac{1-i}{2F}\int_{-\infty}^{\infty}
                T(\rho)\,\exp\left(
                    \frac{i\pi}{2}\left(\frac{\rho-\rho_{0}}{F}\right)^{2}
                \right)\,\textrm{d}\rho
        \end{equation}
        This is the formula for the convolution of $T$ with the complex
        Fresnel kernel $\exp(i\psi_{F})$. That is:
        \begin{equation}
            \hat{T}(\rho_{0})
            =\frac{1-i}{2F}\big(T*\exp(i\psi_{F})\big)(\rho_{0})
        \end{equation}
        Since $T$ is compactly supported, and $\exp(i\psi_{F})$ is integrable,
        we may apply the convolution theorem. Taking the Fourier transform
        of both sides yields:
        \begin{subequations}
            \begin{align}
                \mathcal{F}_{\xi}(\hat{T})
                &=\mathcal{F}_{\xi}\left(
                    \frac{1-i}{2F}\big(T*\exp(i\psi_{F})\big)(\rho_{0})
                \right)\\
                &=\frac{1-i}{2F}\mathcal{F}_{\xi}\left(T\right)\cdot
                    \mathcal{F}_{\xi}\left(\exp(i\psi_{F})\right)
            \end{align}
        \end{subequations}
        We may solve for $T$ using the inverse Fourier transform. We have:
        \begin{equation}
            T(\rho)=\frac{2F}{1-i}\mathcal{F}^{-1}_{\rho}\left(
                \frac{%
                    \mathcal{F}_{\xi}(\hat{T})%
                }{%
                    \mathcal{F}_{\xi}(\exp(i\psi_{F}))
                }
            \right)
        \end{equation}
        Applying techniques from Fourier analysis, this reduces to:
        \begin{equation}
            T(\rho)=\frac{1+i}{2F}\int_{-\infty}^{\infty}
                \hat{T}(\rho_{0})\exp\left(
                    -\frac{i\pi}{2}\left(\frac{\rho-\rho_{0}}{F}\right)^{2}
                \right)\,\textrm{d}\rho_{0}
        \end{equation}
        Numerical evaluation of this expression can lead to accurate
        ring profiles when the angle $B$ is not too small, or when the
        desired resolution is not excessively fine. See
        [INSERT PLOT HERE].
        \par\hfill\par
        For a more general $\psi$ we define an approximate inverse using this
        formula. We write:
        \begin{equation}
            T(\rho)
            \approx\frac{1+i}{2F}\int_{-\infty}^{\infty}
                \hat{T}(\rho_{0})\exp\left(
                    -i\psi(\rho,\,\phi_{s}\;\rho_{0},\,\phi_{0})
                \right)\,\textrm{d}\rho_{0}
        \end{equation}
        where $\phi_{s}$ is again the stationary azimuthal angle.
    \section{Applications of Legendre Polynomials}
        To obtain the Fresnel approximation, we started with exacts formulas
        for $\psi_{0}$, $\psi^{\prime}_{0}$, and $\psi^{\prime\prime}_{0}$,
        and then proceeded to approximate $\psi$ via
        $%
            \psi\approx%
            \psi_{0}-\frac{1}{2}\psi^{\prime\,2}_{0}/\psi^{\prime\prime}_{0}%
        $.
        We then estimate this final expression using a quadratic
        Taylor series in $\rho$ centered about $\rho_{0}$. If we do some
        bookkeeping, we can obtain an expansion for this expression in terms of
        Legendre polynomials. These polynomials are defined as the
        solution obtained from the power series method for the Legendre
        differential equation:
        \begin{equation}
            (1-\alpha^{2})P^{\prime\prime}_{n}(\alpha)
            -2\alpha{P}^{\prime}_{n}(\alpha)
            +n(n+1)P_{n}(\alpha)=0
        \end{equation}
        The $n^{\textrm{th}}$ polynomial can be obtained from Bonnet's
        recursion formula:
        \begin{equation}
            P_{n+2}(\alpha)
            =\frac{(2n+3)\alpha{P}_{n+1}(\alpha)-(n+1)P_{n}(\alpha)}{n+2}
        \end{equation}
        starting with $P_{0}(\alpha)=1$ and $P_{1}(\alpha)=\alpha$.
        The key fact we need is that the Legendre polynomials produce
        the following generating function:
        \begin{equation}
            \sum_{n=0}^{\infty}P_{n}(\alpha)x^{n}
            =\frac{1}{\sqrt{1-2\alpha{x}+x^{2}}}
        \end{equation}
        This formula is valid for $|x|<1$. Since $x=(\rho-\rho_{0})/D$, and
        since $W<<D$, for all $x$ under consideration this equation is true.
        The Taylor series for $\psi_{0}$ in terms of $x$ centered about
        the origin can be readiliy computed from this:
        \begin{subequations}
            \begin{align}
                \psi_{0}
                &=kD\left(\sqrt{1-2\alpha{x}+x^{2}}+\alpha{x}-1\right)\\
                &=kD\left(
                    -1+\alpha{x}+
                    (1-2\alpha{x}+x^{2})\sum_{n=2}^{\infty}P_{n}(\alpha)x^{n}
                \right)\\
                &=kD\left(
                    \frac{1-\alpha^{2}}{2}x^{2}
                    +\frac{\alpha(1-\alpha^{2})}{3}x^{2}
                    +\frac{-4\alpha^{4}+6\alpha^{2}-1}{8}x^{4}
                    +\cdots
                \right)
            \end{align}
        \end{subequations}
        The partial derivative, $\psi^{\prime}_{0}$, also has a nice formula:
        \begin{subequations}
            \begin{align}
                \psi^{\prime}_{0}
                &=kD\cos(B)\sin(\phi_{0})\frac{\rho}{D}\left(
                    \frac{1}{\sqrt{1-2\alpha{x}+x^{2}}}-1
                \right)\\
                &=kD\cos(B)\sin(\phi_{0})\frac{\rho}{D}
                    \sum_{n=1}^{\infty}P_{n}(\alpha)x^{n}\\
                &=kD\cos(B)\sin(\phi_{0})\frac{\rho}{D}\left(
                    \alpha{x}
                    +\frac{3\alpha^{2}-1}{2}x^{2}
                    +\cdots
                \right)
            \end{align}
        \end{subequations}
        where we have made use of the fact that $P_{0}(\alpha)=1$ to simplify
        our sum. The coefficients in the expansion for $\psi_{0}$ can be
        expressed in terms of $\alpha$ and the Legendre polynomials.
        We now demonstrate this fact.
        \begin{subequations}
            \begin{align}
                \frac{\textrm{d}}{\textrm{d}x}\left(
                    \sqrt{1-2\alpha{x}+x^{2}}
                \right)
                &=\frac{x-\alpha}{\sqrt{1-2\alpha{x}+x^{2}}}\\
                &=(x-\alpha)\sum_{n=0}^{\infty}P_{n}(\alpha)x^{n}\\
                &=\sum_{n=0}^{\infty}P_{n}(\alpha)x^{n+1}
                    -\sum_{n=0}^{\infty}\alpha{P}_{n}(\alpha)x^{n}
            \end{align}
        \end{subequations}
        Integrating term-by-term, we obtain:
        \begin{subequations}
            \begin{align}
                \sqrt{1-2\alpha{x}+x^{2}}
                &=1+\sum_{n=0}^{\infty}P_{n}(\alpha)\frac{x^{n+2}}{n+2}
                    -\alpha\sum_{n=0}^{\infty}P_{n}(\alpha)\frac{x^{n+1}}{n+1}\\
                &=1-\alpha{x}+\sum_{n=0}^{\infty}
                    \frac{P_{n}(\alpha)-\alpha{P}_{n+1}(\alpha)}{n+2}x^{n+2}
            \end{align}
        \end{subequations}
        where we have again used the fact that $P_{0}(\alpha)=1$ to shift the
        index of the right-most sum by one, pulling out $-\alpha{x}$
        in the process. Shifting the constant and linear terms to
        the left-hand side, we obtain:
        \begin{equation}
            \sqrt{1-2\alpha{x}+x^{2}}+\alpha{x}-1
            =\sum_{n=0}^{\infty}
                \frac{P_{n}(\alpha)-\alpha{P}_{n+1}(\alpha)}{n+2}x^{n+2}
        \end{equation}
        The function $\psi_{0}$ then becomes:
        \begin{equation}
            \psi_{0}
            =kD\sum_{n=0}^{\infty}
                \frac{P_{n}(\alpha)-\alpha{P}_{n+1}(\alpha)}{n+2}x^{n+2}
            =kD\sum_{n=0}^{\infty}
                b_{n}(\alpha)x^{n+2}
        \end{equation}
        where $b_{n}(\alpha)$ is our new
        \textit{modified Legendre polynomial}:
        \begin{equation}
            b_{n}(\alpha)=\frac{P_{n}(\alpha)-\alpha{P}_{n+1}(\alpha)}{n+2}
        \end{equation}
        The problem is that $\phi_{0}$ does not sufficiently approximate the
        stationary azimuthal angle $\phi_{s}$. Because of this, using
        $\psi_{0}$ directly in our inversion formula will result in a
        very poor ring profile. We reintroduce the terms from the Newton
        iterate to account for this. That is, we consider the
        factor $-\psi^{\prime\,2}_{0}/\psi^{\prime\prime}_{0}$ in our
        expansion. The second partial
        derivative can be approximated fairly well by the constant term
        for window sizes that arise from reasonable resolutions
        (say, one kilometer). The constant approximation is given by the
        Taylor series in $\rho$ centered about $\rho_{0}$. We have:
        \begin{equation}
            \psi^{\prime\prime}_{0}
            \approx{kD}\left(1-\cos^{2}(B)\sin^{2}(\phi_{0})\right)\left(
                \frac{\rho_{0}}{D}
            \right)^{2}
        \end{equation}
        The expression
        $\psi_{0}-\frac{1}{2}\psi^{\prime\,2}_{0}/\psi^{\prime\prime}_{0}$
        then becomes:
        \begin{subequations}
            \begin{align}
                \psi
                &\approx\psi_{0}-
                    \frac{1}{2}\psi^{\prime\,2}_{0}/\psi^{\prime\prime}_{0}\\
                &\approx
                \psi_{0}-\frac{1}{2}
                \frac{\psi^{\prime\,2}_{0}}{%
                    kD\left(1-\cos^{2}(B)\sin^{2}(\phi_{0})\right)
                    \left(\frac{\rho_{0}}{D}\right)^{2}
                }\\
                &=\psi_{0}-\frac{1}{2}
                    \frac{%
                        \left(
                            kD\cos(B)\sin(\phi_{0})\frac{\rho}{D}
                        \right)^{2}
                    }{%
                        kD\left(1-\cos^{2}(B)\sin^{2}(\phi_{0})\right)
                        \left(\frac{\rho_{0}}{D}\right)^{2}
                    }\left(
                        \frac{1}{\sqrt{1-2\alpha{x}+x^{2}}}-1
                    \right)^{2}\\
                &=\psi_{0}-
                    \frac{%
                        kD\cos^{2}(B)\sin^{2}(\phi_{0})\rho^{2}%
                    }{%
                        2\left(
                            1-\cos^{2}(B)\sin^{2}(\phi_{0})
                        \right)\rho_{0}^{2}
                    }\left(
                        \frac{1}{\sqrt{1-2\alpha{x}+x^{2}}}-1
                    \right)^{2}\\
                &=\psi_{0}-
                    \frac{%
                        kD\cos^{2}(B)\sin^{2}(\phi_{0})\rho^{2}%
                    }{%
                        2\left(1-\cos^{2}(B)\sin^{2}(\phi_{0})\right)
                        \rho_{0}^{2}
                    }\left(
                        \sum_{n=1}^{\infty}P_{n}(\alpha)x^{n}
                    \right)^{2}\\
                &=kD\left(
                    \sum_{n=0}^{\infty}b_{n}(\alpha)x^{n+2}-
                    \frac{%
                        \cos^{2}(B)\sin^{2}(\phi_{0})\rho^{2}
                    }{%
                        2\left(1-\cos^{2}(B)\sin^{2}(\phi_{0})\right)
                        \rho_{0}^{2}
                    }\left(
                        \sum_{n=1}^{\infty}P_{n}(\alpha)x^{n}
                    \right)^{2}
                \right)
            \end{align}
        \end{subequations}
        where again we use $P_{0}(\alpha)=1$ to start the index
        of the right-most sum at $n=1$. The quartic, sextic, and octic
        truncations of this formula yield more
        accurate reconstructions than the classic Fresnel approximation. We
        compute partial sums by using the Cauchy product formula for the
        right-most sum that is being squared. We have:
        \begin{equation}
            \left(\sum_{n=1}^{\infty}P_{n}(\alpha)x^{n}\right)^{2}
            =\sum_{n=0}^{\infty}c_{n}(\alpha)x^{n+2}
        \end{equation}
        where $c_{n}$ is the sum of the cross-terms from the Cauchy product:
        \begin{equation}
            c_{n}(\alpha)=\sum_{n=0}^{n}P_{n+1-k}(\alpha)P_{k+1}(\alpha)
        \end{equation}
        The expansion for $\psi$ then becomes:
        \begin{equation}
            \label{eqn:fresnel_legendre_formula}
            \psi\approx
            kD\sum_{n=0}^{\infty}\left(
                b_{n}(\alpha)-
                \frac{%
                    \cos^{2}(B)\sin^{2}(\phi_{0})\rho^{2}
                }{%
                    2\left(1-\cos^{2}(B)\sin^{2}(\phi_{0})\right)
                    \rho_{0}^{2}
                }
                c_{n}(\alpha)
            \right)x^{n+2}
        \end{equation}
        We define the \textbf{Fresnel-Legendre} coefficients to be the
        summand of this formula:
        \begin{equation}
            \label{eqn:fresnel_legendre_coeffs}
            L_{n}(\alpha,\,\beta)
            =\frac{P_{n}(\alpha)-\alpha{P}_{n+1}(\alpha)}{n+2}
                -\beta\sum_{n=0}^{n}P_{n+1-k}(\alpha)P_{k+1}(\alpha)
        \end{equation}
        where:
        \begin{equation}
            \beta=
            \frac{%
                \cos^{2}(B)\sin^{2}(\phi_{0})\rho^{2}
            }{%
                2(1-\cos^{2}(B)\sin^{2}(\phi_{0}))\rho_{0}^{2}
            }
        \end{equation}
        Rewriting Eqn.~\ref{eqn:fresnel_legendre_formula} in terms of
        these new variables, we have:
        \begin{equation}
            \psi\approx{k}D\sum_{n=0}^{N}L_{n}(\alpha,\,\beta)x^{n+2}
        \end{equation}
        Using $N=2$ is sufficient for nearly all of the Cassini occultation
        observations with 1 kilometer resolutions. Substituting $N=4$
        allows one to explore higher resolutions. The Uranus data from the
        Voyager mission can also be processed using $N=2$.
    \section{Implementation and Results}
        Direct evaluation of the modified Legendre polynomial using Eqn BLAH
        requires the regular Legendre polynomials to have already been
        computed. We can skip this by inventing a new recursive relation
        for $b_{n+2}(x)$ in terms of $b_{n}(x)$ and $b_{n+1}(x)$, and then
        compute in an upward iterative fashion. We prove the following.
        \begin{theorem}
            \label{thm:modified_legendre_recursion}
            The modified Legendre polynomials satisfy the recursion relation:
            \begin{equation}
                b_{n+2}(x)=
                \frac{(2n+5)xb_{n+1}(x)-(n+1)b_{n}(x)}{n+4}
            \end{equation}
            starting with $b_{0}(x)=(1-x^{2})/2$ and $b_{1}(x)=x(1-x^{2})/2$.
        \end{theorem}
        \begin{proof}
            The initial conditions $b_{0}(x)$ and $b_{1}(x)$ can be obtained
            directly from the definition of $b_{n}(x)$, which is
            $(P_{n}(x)-xP_{n+1}(x))/(n+2)$, combined with the first few
            terms in the Legendre sequence: $P_{0}(x)=1$, $P_{1}(x)=x$, and
            $P_{2}(x)=(3x^{2}-1)/2$. The general recursion relation can be
            proven by expanding $b_{n+1}(x)$ and $b_{n}(x)$ in terms of the
            Legendre polynomials, and applying the Bonnet recursion formula.
            We have:
            \begin{equation}
                \begin{aligned}
                    &\frac{(2n+5)xb_{n+1}(x)-(n+1)b_{n}(x)}{n+4}\\
                    &\quad\quad=\frac{%
                        (2n+5)x\frac{P_{n+1}(x)-P_{n+2}(x)}{n+3}-
                        (n+1)\frac{P_{n}(x)-xP_{n+1}(x)}{n+2}
                    }{n+4}
                \end{aligned}
            \end{equation}
            Reversing the Bonnet formula and solving for $P_{n}(x)$ gives us:
            \begin{equation}
                P_{n}(x)=\frac{(2n+3)xP_{n+1}(x)-(n+2)P_{n+2}(x)}{n+1}
            \end{equation}
            If we substitute $P_{n}(x)$ into the previous equation we get an
            expression in terms of $P_{n+1}(x)$ and $P_{n+2}(x)$ alone.
            Again using the Bonnet recursion formula, but for $P_{n+3}(x)$,
            this simplifies to:
            \begin{equation}
                \frac{(2n+5)xb_{n+1}(x)-(n+1)b_{n}(x)}{n+4}
                =\frac{P_{n+2}(x)-xP_{n+3}(x)}{n+4}
            \end{equation}
            which is the defining equation for $b_{n+2}(x)$.
        \end{proof}
        This formula allows us to compute $b_{n}(x)$ in an upward iterative
        fashion without precomputing the Legendre polynomial. The real
        bottleneck in the computation of the Fresnel-Legendre polynomials is
        the evaluation of the Cauchy product, which runs in quadratic time.
        We introduce a new formulation that can be computed linearly. The
        functions $c_{n}(\alpha)$, by definition, satisify the following
        equation:
        \begin{equation}
            \sum_{n=0}^{\infty}c_{n}(\alpha)x^{n+2}
            =\left(
                \frac{1}{\sqrt{1-2\alpha{x}+x^{2}}}-1
            \right)^{2}
        \end{equation}
        Expanding the right-hand side yields:
        \begin{equation}
            \sum_{n=0}^{\infty}c_{n}(\alpha)x^{n+2}
            =\frac{1}{1-2\alpha{x}+x^{2}}
            -\frac{2}{\sqrt{1-2\alpha{x}+x^{2}}}
            +1
        \end{equation}
        We can see that the generating function for the Legendre polynomials
        appear once again, but this left-most expression is actually the
        generating function for the Chebyshev polynomials of the second kind.
        That is:
        \begin{equation}
            \sum_{n=0}^{\infty}U_{n}(\alpha)x^{n}
            =\frac{1}{1-2\alpha{x}+x^{2}}
        \end{equation}
        These polynomials also have a recursive formula:
        \begin{equation}
            U_{n+2}(\alpha)=2\alpha{U}_{n+1}(\alpha)-U_{n}(\alpha)
        \end{equation}
        starting with $U_{0}(\alpha)=1$ and $U_{1}(\alpha)=2x$. Thus, we have:
        \begin{equation}
            \sum_{n=0}^{\infty}c_{n}(\alpha)x^{n+2}
            =\sum_{n=0}^{\infty}\left(
                U_{n+2}(\alpha)-2P_{n+2}(\alpha)
            \right)x^{n+2}
        \end{equation}
        Comparing coefficients, we obtain the following simple expression:
        \begin{equation}
            c_{n}(\alpha)=U_{n+2}(\alpha)-2P_{n+2}(\alpha)
        \end{equation}
        The Fresnel-Legendre polynomials simplify to:
        \begin{equation}
            L_{n}(\alpha,\,\beta)
            =\frac{P_{n}(\alpha)-\alpha{P}_{n+1}(\alpha)}{n+2}
                -\beta\big(U_{n+2}(\alpha)-2P_{n+2}(\alpha)\big)
        \end{equation}
        The left part of this formula can be computed using the recursion
        relation in Thm.~\ref{thm:modified_legendre_recursion}, and the
        expression on the right may be computed using the recursion relations
        for $U_{n}(\alpha)$ and $P_{n}(\alpha)$. Doing this in an upward
        iterative fashion means the entire computation is linear in $n$
        (assuming arithmetic operations run in constant time).
        \par\hfill\par
        The integral is computed using a simple Riemann sum across the window.
        The variable $\beta$ depends on $\rho$, meaning it varies across the
        window of integration. As a practical matter, this can often be ignored
        since $\rho^{2}/\rho_{0}^{2}$ is nearly constant for reasonable
        resolutions. If one were to consider Saturn's Maxwell ringlet, we have
        $\rho\approx{8}\times{10}^{5}\,\textrm{km}$, whereas the window width
        may be somewhere between 30 km (Rev007) and 1000 km (Rev133).
        The ratio $(\rho_{0}+W/2)^{2}/\rho_{0}^{2}$ is close to 1, even for
        the most extreme cases. Still, more accurate results are obtained by
        allowing $\beta$ to vary with $\rho$, and the additional computational
        time introduced is negligible.
    \section{Limitations}
        The primary source of error in this method is the assumption that
        the first iteration of Newton's method yields a sufficient
        approximation for the stationary azimuthal angle. The case where the
        opening angle is $B=\pi/2$ is most ideal.
        The $\xi$ term becomes $0$ since $\cos(\pi/2)=0$, meaning the
        Fresnel kernel $\psi$ reduces to:
        \begin{equation}
            \psi=kD\left(
                \sqrt{1+\eta}-1
            \right)
        \end{equation}
        The partial derivative is:
        \begin{equation}
            \frac{\partial\psi}{\partial\phi}
            =\frac{\eta^{\prime}}{2\sqrt{1+\eta}}
            =\frac{\rho\rho_{0}\sin(\phi-\phi_{0})}{\sqrt{1+\eta}}
        \end{equation}
        The stationary azimuthal angle is thus \textbf{exactly}
        $\phi_{s}=\phi_{0}$. Using the fact that $\eta_{\phi=\phi_{0}}=x^{2}$,
        we then get the following exact formula for the stationary Fresnel
        kernel:
        \begin{equation}
            \label{eqn:stationary_fresnel_with_b_zero}
            \psi_{s}
            =kD\left(
                \sqrt{1+\eta_{0}}-1
            \right)
            =kD(\sqrt{1+x^{2}}-1)
        \end{equation}
        This expression can easily be computed by a Taylor series, but since
        square roots are not too prohibitive on modern processors, one may
        use Eqn.~\ref{eqn:stationary_fresnel_with_b_zero} directly.
        \par\hfill\par
        The formula in Eqn.~\ref{eqn:stationary_fresnel_with_b_zero} can be
        used for Voyager data with Uranus since $B$ is very close to begin a
        right angle, but for Saturn the opening never exceeds $30^{\circ}$.
        Disregarding the Newtonian correction leads to meaningless results, but
        the first iteration of Newton's method may also be insufficient. The
        issue with Saturn is that the angle $B$ can be arbitrarily close to
        zero (if $B=0$ the Huygens' principal is not well-formed and we do not
        attempt to perform Fresnel inversion). Indeed, Cassini's Rev133 had an
        opening angle of less than 2 degrees. This causes the first Newton
        iterate to be a poor approximation for the stationary azimuthal angle,
        meaning both the Fresnel approximation, and these new Fresnel-Legendre
        approximations, are inadequate. One must perform additional iterations
        of Newton's method to calculate $\phi_{s}$, and there is no simple
        formula for $\psi_{s}$ in terms of $\phi_{s,\,n}$ for $n\geq{2}$.
        \par\hfill\par
        Another cause for inaccuracy is the assumption that
        $D\approx||\mathbf{R}-\mathbf{r}_{0}||_{2}$ holds. As one requests
        finer and finer resolutions, the window width $W$ may become large
        enough that $||\mathbf{R}-\mathbf{r}||_{2}$ differs significantly from
        $||\mathbf{R}-\mathbf{r}_{0}||$. Such is the case with Rev133 where the
        small opening angle $B$ requires larger window widths in order to
        achieve one kilometer resolutions. For comparison, Rev007, where $B$ is
        close to $30^{\circ}$, requires window widths that are about 30 km for
        1 km reconstructions, whereas Rev133, with $B\approx{0}$, requires
        1,000 km windows.
    \setcounter{secnumdepth}{0}
    \section{Appendix: Fresnel Approximation via \texttt{sympy}}
        The following script symbolically calculates the Fresnel approximation.
        It is written in Python using the \texttt{sympy} package.
        \begin{lstlisting}[language = Python]
import sympy
sympy.init_printing()
r, r0, phi, phi0, B, D = sympy.symbols('r r0 phi phi0 B D')
xi = sympy.cos(B) * (r*sympy.cos(phi) - r0*sympy.cos(phi0)) / D
eta = (r**2 + r0**2 - 2*r*r0*sympy.cos(phi - phi0)) / D**2
psi = sympy.sqrt(1 - 2*xi + eta) + xi - 1
psi_prime = sympy.diff(psi, phi)
psi_2prime = sympy.diff(psi_prime, phi)
psi0 = psi.subs(phi, phi0)
psi_prime0 = psi_prime.subs(phi, phi0)
psi_2prime0 = psi_2prime.subs(phi, phi0)
psi_approx = psi0 - psi_prime0**2 / psi_2prime0 / 2
psi_series = sympy.series(psi_approx, x = r, x0 = r0, n = 3).removeO()
psi_simplify = sympy.simplify(psi_series)
print(psi_simplify)
        \end{lstlisting}
        The output is:
        \begin{equation}
            \texttt{%
                -(r - r0)**2*sin(B)**2/(2*D**2*(sin(phi0)**2*cos(B)**2 - 1))%
            }
        \end{equation}
        By multiplying this by $kD$, and rewriting this in terms of $F$,
        we obtain the Fresnel approximation.
\end{document}
