%-----------------------------------LICENSE------------------------------------%
%   This file is part of Mathematics-and-Physics.                              %
%                                                                              %
%   Mathematics-and-Physics is free software: you can redistribute it and/or   %
%   modify it under the terms of the GNU General Public License as             %
%   published by the Free Software Foundation, either version 3 of the         %
%   License, or (at your option) any later version.                            %
%                                                                              %
%   Mathematics-and-Physics is distributed in the hope that it will be useful, %
%   but WITHOUT ANY WARRANTY; without even the implied warranty of             %
%   MERCHANTABILITY or FITNESS FOR A PARTICULAR PURPOSE.  See the              %
%   GNU General Public License for more details.                               %
%                                                                              %
%   You should have received a copy of the GNU General Public License along    %
%   with Mathematics-and-Physics.  If not, see <https://www.gnu.org/licenses/>.%
%------------------------------------------------------------------------------%
%       Author: Ryan Maguire                                                   %
%       Date:   2024/12/02                                                     %
%------------------------------------------------------------------------------%
\documentclass{article}
\usepackage{amssymb}
\usepackage{amsmath}
\usepackage{amscd}
\usepackage{graphicx}
\usepackage{hyperref}
\hypersetup{colorlinks = true, linkcolor = blue}
\setlength{\parindent}{0em}
\setlength{\parskip}{0em}
\graphicspath{{../images/}}

% Title page information.
\title{%
    Applications of Legendre Polynomial for
    Fresnel Inversion and Occultation Observations%
}
\author{%
    Ryan Maguire\footnote{%
        Department of Mathematics,
        MIT, Cambridge, MA 02139, USA
    }
    \and
    Richard French\footnote{
        Department of Physics and Astronomy,
        Wellesley College, Wellelsey, MA 02481, USA
    }
}
\date{\today}
\begin{document}
    \maketitle
    \begin{abstract}
        Radio occultations can be used to study the rings of Saturn and
        Uranus. The wavelength of the signal emitted from a space probe is
        usually large enough that significant diffraction effects can occur,
        and the data must be processed numerically in order to construct a
        meaningful ring profile. The classic Fresnel approximation uses a
        quadratic polynomial for the Fresnel kernel, and in some real cases
        this is sufficient (i.e. Cassini's Rev 7, Voyager's fly-by of Uranus),
        but certain scenarios require higher order terms in order to perform
        accurate reconstructions. By following the derivation of the
        Fresnel approximation carefully, one can obtain a polynomial expansion
        for the Fresnel kernel in terms of Legendre polynomials, the degree
        of which can be arbitrarily high. Quartic and octic
        expansions are sufficient for the vast majority of the Cassini data
        sets, and these approximations require only a small amount of
        additional computational time when compared to the quadratic Fresnel
        method.
    \end{abstract}
    \tableofcontents
    \section{Fourier Optics in a Ring System}
        The geometry of a radio occultation observation is determined by the
        ring plane of the planet, the trajectory of the spacecraft, and the
        location of Earth. Let $(\rho,\,\phi)$ denote the cylindrical
        coordinates of the ring plane, let $\mathbf{R}$ be the position
        vector of the spacecraft with respect to the core of the plane, and
        let $B$ indicate the ring opening angle, which is the angle made
        by the ring plane and the line segment between the spacecraft and
        Earth. Since the distance between Earth and the other planets is
        several orders of magnitute greater than the length of $\mathbf{R}$,
        or even the diameter of the entire ring system, the angle $B$ may be
        treated as a constant.
        \par\hfill\par
        Given a plane wave incident on the rings, the diffracted complex
        transmittance at a point $\hat{T}(\rho_{0},\,\phi_{0})$ at the point
        $(\rho_{0},\,\phi_{0})$ is given by:
        \begin{equation}
            \label{eqn:huygens_principal}
            \hat{T}(\rho_{0},\,\phi_{0})
            =\frac{\mu_{0}}{i\lambda}\int_{0}^{\infty}\int_{0}^{2\pi}
                T(\rho,\,\phi)
                \frac{%
                    \exp\left(
                        i\,\psi(\rho_{0},\,\phi_{0};\,\rho,\,\phi)
                    \right)
                }{%
                    ||\mathbf{R}-\mathbf{r}||_{2}
                }
                \rho\,\textrm{d}\rho\,\textrm{d}\phi
        \end{equation}
        where $\mathbf{r}$ is the position vector for the point
        $(\rho,\,\phi)$ in the ring plane, $T$ is the complex transmittance,
        and where $||\mathbf{R}-\mathbf{r}||_{2}$ is the standard
        Euclidean distance between this point and the spacecraft.
        For convenience, we label this distance $D$:
        \begin{equation}
            D=||\mathbf{R}-\mathbf{r}||_{2}
        \end{equation}
        The quantity $\psi$ is purely geometric, and the complex
        exponention $\exp(i\,\psi)$ is the \textit{Fresnel kernel}.
        Following MTR86, $\psi$ is given in terms of auxiliary parameters
        $\hat{\xi}$ and $\hat{\eta}$. We write:
        \begin{align}
            \hat{\xi}
            &=\frac{\cos(B)}{D}\big(
                \rho_{0}\cos(\phi_{0})-\rho\cos(\phi)
            \big)\\[1em]
            \hat{\eta}
            &=\frac{%
                \rho^{2}+\rho_{0}^{2}-2\rho\rho_{0}\cos(\phi-\phi_{0})%
            }{D^{2}}\\[1em]
            \psi
            &=kD\left(
                \sqrt{1+2\hat{\xi}+\hat{\eta}}-\hat{\xi}-1
            \right)
        \end{align}
        Where $k$ is the wavenumber, $k=2\pi/\lambda$, with $\lambda$ being
        the wavelength of the incident plane wave.
        We define $\xi$ and $\eta$ using this, and rewrite $\psi$ accordingly:
        \begin{align}
            \xi&=-\hat{\xi}\\
            \eta&=\hat{\eta}\\
            \psi&=kD\left(
                \sqrt{1-2\xi+\eta}+\xi-1
            \right)
        \end{align}
        The reason for the subtle change in the notation used by other
        authors is that it allows us to rewrite $\psi$ and its derivatives
        more naturally in terms of Legendre polynomials later on. Indeed,
        the generating function for the Legendre polynomials is given by:
        \begin{equation}
            \sum_{n=0}^{\infty}P_{n}(x)t^{n}
            =\frac{1}{\sqrt{1-2xt+t^{2}}}
        \end{equation}
        and one can already start to see some similarities between this
        expression and our new formula for $\psi$.
        \par\hfill\par
        By assuming perfectly symmetric rings, we may reduce
        Eqn.~\ref{eqn:huygens_principal} to the following:
        \begin{subequations}
            \begin{align}
                \hat{T}(\rho_{0})
                &=\frac{\mu_{0}}{i\lambda}\int_{0}^{\infty}\int_{0}^{2\pi}
                    T(\rho)
                    \frac{%
                        \exp\left(
                            i\,\psi(\rho_{0},\,\phi_{0};\,\rho,\,\phi)
                        \right)
                    }{%
                        ||\mathbf{R}-\mathbf{r}||_{2}
                    }
                    \rho\,\textrm{d}\rho\,\textrm{d}\phi\\
                &=\frac{\mu_{0}}{i\lambda}\int_{0}^{\infty}
                    \rho\,T(\rho)\left[
                        \int_{0}^{2\pi}
                        \frac{%
                            \exp\left(
                                i\,\psi(\rho_{0},\,\phi_{0};\,\rho,\,\phi)
                            \right)
                        }{%
                            ||\mathbf{R}-\mathbf{r}||_{2}
                        }
                        \textrm{d}\phi
                    \right]
                    \textrm{d}\rho
                \end{align}
        \end{subequations}
        Such an assumption is valid for Saturn's rings, but the rings of
        Uranus are elliptical. We treat this special case in the appendix.
        \par\hfill\par
        The stationary phase approximation allows us to reduce this double
        integral to a single integral, eliminating the inner integration over
        the azimuthal component $\phi$. Let $\phi_{s}$ denote the stationary
        azimuthal angle, which is the solution to
        $\frac{\partial\psi}{\partial\phi}=0$ as a function of the other
        parameters. The reduction becomes:
        \begin{equation}
            \hat{T}(\rho_{0})
            =\frac{1-i}{2F}\int_{0}^{\infty}
                T(\rho)\exp\left(
                    i\,\psi(\rho_{0},\,\phi_{0};\,\rho,\,\phi)
                \right)
                \,\textrm{d}\rho
        \end{equation}
        where $F$ is the characteristic Fresnel scale:
        \begin{equation}
            F=\frac{\lambda{D}}{2}\frac{1-\cos^{2}(B)\sin^{2}(\phi_{0}}{\sin^{2}(B)}
        \end{equation}
        Note that in factoring $F$ \textit{outside} of the integral for
        $\rho$, we are assuming the parameter varies very little with $\rho$.
        This is the case when $||\mathbf{R}||_{2}>>||\mathbf{r}||_{2}$, but
        for high resolution reconstructions this need not hold. We
        handle this in the appendix.
        \par\hfill\par
        To obtain ring profiles we must solve this integral equation for $T$;
        $\hat{T}$ is the measured diffraction data.
        The geometric variables $\rho_{0}$, $\phi_{0}$, $B$, and $D$ are also
        measured quantities, and remaining two parameters $\rho$ and $\phi$ are
        just dummy variables of integration.
    \section{The Fresnel Approximation}
        Computing the stationary azimuth angle $\phi_{s}$ for $\psi$ can be
        done numerically using Newton's method. Starting with
        $\phi_{s,\,0}=\phi_{0}$, we obtained an improved solution via:
        \begin{equation}
            \phi_{s,\,n+1}=\phi_{s,\,n}-
            \frac{\partial\psi/\partial\phi}
                 {\partial^{2}\psi/\partial\phi^{2}}
        \end{equation}
        We may expand $\psi$ in terms of the $\phi_{s}$ using the Taylor
        series:
        \begin{equation}
            \psi(\rho,\,\phi;\,\rho_{0},\,\phi_{0})
            =\sum_{n=0}^{\infty}
                \frac{\psi^{(n)}(\rho,\,\phi_{0};\,\rho_{0},\,\phi_{0})}{n!}
                (\phi-\phi_{0})^{n}
        \end{equation}
        where $\psi^{(n)}$ denotes the $n^{\textrm{th}}$ partial derivative of
        $psi$ with respect to $\phi$. Truncating at the quadratic term yields
        a nice approximation:
        \begin{equation}
            \label{eqn:quadratic_taylor}
            \psi(\rho,\,\phi;\,\rho_{0},\,\phi_{0})
            \approx
            \psi_{0}+\psi^{\prime}_{0}(\phi-\phi_{0})
            +\frac{1}{2}\psi^{\prime\prime}_{0}(\phi-\phi_{0})^{2}
        \end{equation}
        where $\psi_{0}$, $\psi^{\prime}_{0}$, and $\psi^{\prime\prime}_{0}$
        denote the zeroth, first, and second partial derivatives of $\psi$
        evaluated at $(\rho,\,\phi_{0};\,\rho_{0},\,\phi_{0})$, respectively.
        If we assume that the first Newton iterate is sufficient for numerical
        purpose (i.e. $\phi_{s}=\phi_{s,\,1}$), we obtain:
        \begin{subequations}
            \begin{align}
                \phi_{s}-\phi_{0}
                &=\left(
                    \phi_{0}-
                    \frac{\psi^{\prime}_{0}}{\psi^{\prime\prime}_{0}}
                \right)-\phi_{0}\\[1em]
                &=-\frac{\psi^{\prime}_{0}}{\psi^{\prime\prime}_{0}}
            \end{align}
        \end{subequations}
        Plugging this into the quadratic approximation in
        Eqn.~\ref{eqn:quadratic_taylor} yields:
        \begin{align}
            \psi(\rho,\,\phi_{s};\,\rho_{0},\,\phi_{0})
            &=\psi_{0}+\psi^{\prime}_{0}(\phi_{s}-\phi_{0})
                +\frac{1}{2}\psi^{\prime\prime}_{0}(\phi_{s}-\phi_{0})^{2}\\
            &=\psi_{0}+\psi^{\prime}_{s}\left(
                -\frac{\psi^{\prime}_{0}}{\psi^{\prime\prime}_{0}}
            \right)+
            \frac{1}{2}\psi^{\prime\prime}_{s}\left(
                -\frac{\psi^{\prime}_{0}}{\psi^{\prime\prime}_{0}}
            \right)^{2}\\
            &=\psi_{0}
                -\frac{1}{2}
                \frac{\psi^{\prime\,2}_{0}}{\psi^{\prime\prime}_{0}}
        \end{align}
        The quadratic Fresnel approximation can be obtained once we
        evaluate these derivatives. First, we substitute $\phi=\phi_{0}$
        into $\xi$, $\eta$.
        \begin{align}
            \xi_{\phi=\phi_{0}}
            &=\cos(B)\cos(\phi_{0})\frac{\rho-\rho_{0}}{D}\\
            \eta_{\phi=\phi_{0}}
            &=\left(\frac{\rho-\rho_{0}}{D}\right)^{2}
        \end{align}
        Define the auxiliary parameters $\alpha$ and $x$ via:
        \begin{align}
            \alpha&=\cos(B)\cos(\phi_{0})\\
            x&=\frac{\rho-\rho_{0}}{D}
        \end{align}
        The formulas for $\xi$, $\eta$, and $\psi_{0}$ then become:
        \begin{align}
            \xi_{\phi=\phi_{0}}
            &=\alpha{x}\\
            \eta_{\phi=\phi_{0}}
            &=x^{2}\\
            \psi_{\phi=\phi_{0}}
            &=kD\left(
                \sqrt{1-2\alpha{x}+x^{2}}
                +\alpha{x}-1
            \right)
        \end{align}
        The observant reader may note how strikingly similar our formula is
        to the generating function for the Legendre polynomials. We will
        touch upon that in the next section.
        \par\hfill\par
        The partial derivatives for $\psi$ can be obtained from the partial
        derivatives of $\xi$ and $\eta$. We have:
        \begin{align}
            \xi^{\prime}
            &=-\cos(B)\sin(\phi)\frac{\rho}{D}\\
            \xi^{\prime\prime}
            &=-\cos(B)\cos(\phi)\frac{\rho}{D}\\
            \eta^{\prime}
            &=\frac{2\rho\rho_{0}\sin(\phi-\phi_{0})}{D^{2}}\\
            \eta^{\prime\prime}
            &=\frac{2\rho\rho_{0}\cos(\phi-\phi_{0})}{D^{2}}
        \end{align}
        The first partial derivative of $\psi$ is then:
        \begin{align}
            \psi^{\prime}
            &=kD\left(
                \frac{-2\xi^{\prime}+\eta^{\prime}}{2\sqrt{1-2\xi+\eta}}
                +\xi^{\prime}
            \right)
        \end{align}
        Plugging in $\phi=\phi_{0}$ yields:
        \begin{equation}
            \psi^{\prime}_{0}
            =kD\cos(B)\sin(\phi_{0})\frac{\rho}{D}\left(
                \frac{1}{\sqrt{1-2\alpha{x}+x^{2}}}-1
            \right)
        \end{equation}
        The second partial derivative isn't quite as nice:
        \begin{equation}
            \psi^{\prime\prime}
            =kD\left(
                    \frac{-2\xi^{\prime\prime}+\eta^{\prime\prime}}
                          {2(1-2\xi+\eta)^{1/2}}-
                    \frac{(-2\xi^{\prime}+\eta^{\prime})^{2}}
                         {4(1-2\xi+\eta)^{3/2}}+
                    \xi^{\prime\prime}
            \right)
        \end{equation}
        Again substituting $\phi=\phi_{0}$, we obtain:
        \begin{equation}
            \psi^{\prime\prime}_{0}
            =kD\left(
                \frac{%
                    -2\alpha\rho{D}+2\rho\rho_{0}
                }{%
                    2D^{2}(1-2\alpha{x}+x^{2})^{1/2}
                }
                -\frac{%
                    \cos^{2}(B)\sin^{2}(\phi_{0})\rho^{2}
                }{%
                    D^{2}(1-2\alpha{x}+x^{2})^{3/2}
                }
                -\frac{\alpha\rho}{D}
            \right)
        \end{equation}
    \section{Applications of Legendre Polynomial}
    \section{Implementation and Results}

\end{document}
