%-----------------------------------LICENSE------------------------------------%
%   This file is part of Mathematics-and-Physics.                              %
%                                                                              %
%   Mathematics-and-Physics is free software: you can redistribute it and/or   %
%   modify it under the terms of the GNU General Public License as             %
%   published by the Free Software Foundation, either version 3 of the         %
%   License, or (at your option) any later version.                            %
%                                                                              %
%   Mathematics-and-Physics is distributed in the hope that it will be useful, %
%   but WITHOUT ANY WARRANTY; without even the implied warranty of             %
%   MERCHANTABILITY or FITNESS FOR A PARTICULAR PURPOSE.  See the              %
%   GNU General Public License for more details.                               %
%                                                                              %
%   You should have received a copy of the GNU General Public License along    %
%   with Mathematics-and-Physics.  If not, see <https://www.gnu.org/licenses/>.%
%------------------------------------------------------------------------------%
%       Author: Ryan Maguire                                                   %
%       Date:   2022/03/14                                                     %
%------------------------------------------------------------------------------%
\documentclass{article}

\usepackage{geometry}
\geometry{margin=0.75in}

% Needed for mathbb font style.
\usepackage{amssymb}

% Tools for creating the theorem environment.
\usepackage{amsthm}
\usepackage{amsmath}

% Used for pictures.
\usepackage{graphics}
\usepackage{float}

% Hyperlinks for labels.
\usepackage{hyperref}

% Display style for hyperlinks.
\hypersetup{
    colorlinks=true,
    linkcolor=blue
}

% Create a theorem environment, similar to most textbooks.
\theoremstyle{plain}
\newtheorem{theorem}{Theorem}
\newtheorem{conjecture}{Conjecture}
\newtheorem{question}{Question}

% No indent and no paragraph skip.
\setlength{\parindent}{0em}
\setlength{\parskip}{0em}

% Title page information.
\title{New Data for a Conjecture on Legendrian and Transversally Simple Knots}
\author{Vladimir Chernov\hspace{2em}Ryan Maguire}
\date{February 2023}

\begin{document}
    \maketitle
    \tableofcontents
    \begin{abstract}
        \noindent
        In \cite{ChernovMaguireLegendrianConjecture} we conjectured that
        Khovanov homology is able to distinguish legendrian and transversally
        simple knots. Numerical data was provided using twist and torus knots,
        comparing the Khovanov homologies of these families with all knots up
        to 17 crossings. This experiment has been expanded to 19 crossings.
        \par\hfill\par
        \textit{AMS Classification: Primary 57K18, Secondary 53D12, 53D10}
    \end{abstract}
    \section{The Conjecture and Computations}
        In \cite{ChernovMaguireLegendrianConjecture} we proposed the following
        conjecture.
        \begin{conjecture}
            If a link type $L$ is Legendrian simple or transversally simple,
            then the Khovanov homology of $L$ distinguishes it. That is,
            if $\tilde{L}$ is another link with the same Khovanov homology,
            then $\tilde{L}$ is equivalent to $L$.
        \end{conjecture}
        Numerical evidence was tallied using the torus and twist knots to gather
        numerical evidence for this claim. The torus knots are known to be
        Legendrian simple, and Etnyre \textit{et al.} classifies when twist
        knots are transversally simple
        \cite{EtnyreEtAlLegendrianAndTransverseTwistKnots},
        giving us two infinite families to use.
        Knots having the same Jones polynomial as a torus or twist knot were
        saved for later examination since the Jones polynomial is
        computationally more efficient than Khovanov homology. For torus knots
        we were left with four matching knots
        (See Tab.~\ref{table:matching_torus_knots}). These are the same four
        knots that appear in our previous paper and nothing new occurs with
        18 and 19 crossings. For twist knots we have found new knots that have
        the same Jones polynomial (See Tab.~\ref{table:matching_twist_knots}).
        \par\hfill\par
        Computations for the Jones polynomial were performed using the
        \texttt{regina} library \cite{regina}. The Jones polynomial of a knot
        $K$ has an explicit formula given as follows:
        \begin{equation}
            J(K)=\sum_{n=0}^{2^{N}-1}(-q)^{w(n)}(q+q^{-1})^{c(n)}
        \end{equation}
        where $N$ is the number of crossings in $K$, $w(n)$ is the Hamming
        weight of $n$, and $c(n)$ is the number of cycles corresponding to
        the $n^{\textrm{th}}$ resolution of the crossings of $K$. $w(n)$ is
        bounded by $N$ and $c(n)$ is bounded by $2N$, meaning the degree of
        $J(K)$ is bounded above by $3N$ and below by $-N$. We collect all
        torus and twist knots having a Jones polynomial satisfying these bounds
        for $N=19$ and then pre-compute the Jones polynomials using known
        formulas (see \cite{ChernovMaguireLegendrianConjecture},
        Eqns. 4 and 5). Through brute force calculations, we find that there
        are four torus knots and eight twist knots that have the same Jones
        polynomial as another knot, but in all cases the Khovanov polynomials
        (and hence Khovanov homologies) differed. The polynomials are given
        explicitly in the next section.
        \begin{table}
            \centering
            \resizebox{\textwidth}{!}{%
                \begin{tabular}{| c | c | c |}
                    \hline
                        Torus Knot&
                        Non-Torus Knot&
                        Jones Polynomial\\
                    \hline
                        $(2,\,5)$&
                        $8,\,6,\,18,\,2,\,-12,\,-16,\,20,\,-10,\,4,\,14$&
                        $-q^{14}+q^{12}-q^{10}+q^{8}+q^{4}$\\
                    \hline
                        $(2,\,7)$&
                        $12,\,-20,\,-14,\,22,\,-16,\,24,\,-18,\,-6,\,-10,\,-2,\,-4,\,8$&
                        $-q^{20}+q^{18}-q^{16}+q^{14}-q^{12}+q^{10}+q^{6}$\\
                    \hline
                        $(2,\,11)$&
                        $14,\,-16,\,24,\,-18,\,26,\,-20,\,28,\,-22,\,-4,\,-8,\,-12,\,-2,\,6,\,10$&
                        $-q^{32}+q^{30}-q^{28}+q^{26}-q^{24}+q^{22}-q^{20}+q^{18}-q^{16}+q^{14}+q^{10}$\\
                    \hline
                        $(2,\,5)$&
                        $18,\,-28,\,-16,\,24,\,-32,\,-20,\,34,\,-6,\,30,\,-22,\,-12,\,26,\,8,\,-2,\,-4,\,14,\,-10$&
                        $-q^{14}+q^{12}-q^{10}+q^{8}+q^{4}$\\
                    \hline
                \end{tabular}%
            }
            \caption{Knots whose Jones polynomial matches that of a Torus Knot}
            \label{table:matching_torus_knots}
        \end{table}
        \begin{table}
            \centering
            \resizebox{\textwidth}{!}{%
                \begin{tabular}{| c | c | c |}
                    \hline
                        Twist Knot&
                        Non-Twist Knot&
                        Jones Polynomial\\
                    \hline
                        $m_{2}$&
                        $10,\,18,\,22,\,-14,\,4,\,-16,\,-20,\,-6,\,2,\,-12,\,8$&
                        $q^{4}-q^{2}+1-q^{-2}+q^{-4}$\\
                    \hline
                        $m_{3}$&
                        $8,\,14,\,18,\,22,\,-12,\,-16,\,2,\,-10,\,20,\,4,\,6$&
                        $-q^{12}+q^{10}-q^{8}+2q^{6}-q^{4}+q^{2}$\\
                    \hline
                        $m_{3}$&
                        $14,\,12,\,-20,\,-22,\,-16,\,24,\,2,\,-18,\,-10,\,-4,\,-6,\,-8$&
                        $-q^{12}+q^{10}-q^{8}+2q^{6}-q^{4}+q^{2}$\\
                    \hline
                        $m_{3}$&
                        $16,\,-14,\,-20,\,2,\,-26,\,24,\,-6,\,8,\,-10,\,-22,\,-4,\,12,\,-18$&
                        $-q^{12}+q^{10}-q^{8}+2q^{6}-q^{4}+q^{2}$\\
                    \hline
                        $m_{5}$&
                        $4,\,16,\,-8,\,-14,\,18,\,20,\,-6,\,22,\,2,\,10,\,12$&
                        $-q^{16}+q^{14}-q^{12}+2q^{10}-2q^{8}+2q^{6}-q^{4}+q^{2}$\\
                    \hline
                        $m_{6}$&
                        $6,\,10,\,12,\,-18,\,14,\,4,\,2,\,20,\,22,\,-8,\,16$&
                        $q^{12}-q^{10}+q^{8}-2q^{6}+2q^{4}-2q^{2}+2-q^{-2}+q^{-4}$\\
                    \hline
                        $m_{6}$&
                        $12,\,10,\,26,\,-18,\,4,\,2,\,-20,\,-22,\,-24,\,-6,\,-14,\,-16,\,8$&
                        $q^{12}-q^{10}+q^{8}-2q^{6}+2q^{4}-2q^{2}+2-q^{-2}+q^{-4}$\\
                    \hline
                        $m_{6}$&
                        $20,\,32,\,-18,\,-12,\,-28,\,-26,\,36,\,-6,\,24,\,34,\,-30,\,16,\,22,\,-10,\,-8,\,2,\,4,\,14$&
                        $q^{12}-q^{10}+q^{8}-2q^{6}+2q^{4}-2q^{2}+2-q^{-2}+q^{-4}$\\
                    \hline
                        $m_{7}$&
                        $6,\,14,\,20,\,-12,\,-16,\,-18,\,2,\,-8,\,-10,\,22,\,4$&
                        $-q^{20}+q^{18}-q^{16}+2q^{14}-2q^{12}+2q^{10}-2q^{8}+2q^{6}-q^{4}+q^{2}$\\
                    \hline
                        $m_{8}$&
                        $22,\,28,\,-18,\,-16,\,30,\,-4,\,20,\,-6,\,-8,\,-34,\,36,\,-26,\,-32,\,2,\,10,\,-24,\,14,\,-12$&
                        $q^{16}-q^{14}+q^{12}-2q^{10}+2q^{8}-2q^{6}+2q^{4}-2q^{2}+2-q^{-2}+q^{-4}$\\
                    \hline
                        $m_{9}$&
                        $20,\,30,\,32,\,-18,\,-12,\,-26,\,36,\,-8,\,24,\,34,\,-28,\,16,\,22,\,-10,\,2,\,4,\,6,\,14$&
                        $-q^{24}+q^{22}-q^{20}+2q^{18}-2q^{16}+2q^{14}-2q^{12}+2q^{10}-2q^{8}+2q^{6}-q^{4}+q^{2}$\\
                    \hline
                \end{tabular}%
            }
            \caption{Knots whose Jones polynomial matches that of a Twist Knot}
            \label{table:matching_twist_knots}
        \end{table}
    \section{Numerical Results}
        A note on notation. The above tables identify knots using their DT
        codes. The \texttt{regina} library uses a clever trick to efficiently
        express a knots DT code for all knots of up to 26 crossings, and we'll
        adopt this notation for the remainder of the paper. The DT code of an
        $n$ crossing knot is a string of length $n$ of signed even integers up
        to $2n$. Since all entries are even, we can divide by two and obtain
        a string of length $n$ of signed integers up to $n$. So long as $n$ is
        not larger than 26 we can replace integers with letters. We then use
        lower case letters for positive integers, and upper case for negative.
        For example, the trefoil $T(2,\,3)$ becomes:
        \begin{equation}
            4,\,6,\,2
            \Rightarrow
            2,\,3,\,1
            \Rightarrow
            \texttt{bca}
        \end{equation}
        The knot $10_{132}$, which is the 10 crossing knot in
        Tab.~\ref{table:matching_twist_knots} with the same Jones polynomial
        as $T(2,\,5)$, becomes:
        \begin{equation}
            8,\,6,\,18,\,2,\,-12,\,-16,\,20,\,-10,\,4,\,14
            \Rightarrow
            4,\,3,\,9,\,1,\,-6,\,-8,\,10,\,-5,\,2,\,7
            \Rightarrow
            \texttt{dciaFHjEbg}
        \end{equation}
        In the above tables we have found four non-torus knots with the same
        Jones polynomial as some torus knot. In all four cases the Khovanov
        polynomials differ. The tables for these polynomials are given below.
        The coefficient of $q^{\ell}t^{r}$ is given by the corresponding entry.
        \par\hfill\par
        The cinquefoil $T(2,\,5)$ has the same Jones polynomial as two
        non-torus knots. As shown below in
        Tabs.~\ref{table:t_2_5_kho}-\ref{table:iNHlPJqCoKFmdABgE_kho}, the
        Khovanov polynomials differ.
        \begin{table}[H]
            \centering
            \begin{tabular}{| c | c | c | c | c | c | c |}
                \hline
                $q\symbol{92}t$&$-5$&$-4$&$-3$&$-2$&$-1$&$0$\\
                \hline
                $-15$&1&&&&&\\
                \hline
                $-13$&&&&&&\\
                \hline
                $-11$&&1&1&&&\\
                \hline
                $-9$&&&&&&\\
                \hline
                $-7$&&&&1&&\\
                \hline
                $-5$&&&&&&1\\
                \hline
                $-3$&&&&&&1\\
                \hline
            \end{tabular}
            \caption{Khovanov Polynomial for $T(5,2)$}
            \label{table:t_2_5_kho}
        \end{table}
        \begin{table}[H]
            \centering
            \begin{tabular}{| c | c | c | c | c | c | c | c | c |}
                \hline
                $q\symbol{92}t$&$-7$&$-6$&$-5$&$-4$&$-3$&$-2$&$-1$&$0$\\
                \hline
                $-15$&1&&&&&&&\\
                \hline
                $-13$&&&&&&&&\\
                \hline
                $-11$&&1&1&&&&&\\
                \hline
                $-9$&&&&1&1&&&\\
                \hline
                $-7$&&&&1&&&&\\
                \hline
                $-5$&&&&&1&2&&\\
                \hline
                $-3$&&&&&&&&1\\
                \hline
                $-1$&&&&&&&1&1\\
                \hline
            \end{tabular}
            \caption{Khovanov Polynomial for \texttt{dciaFHjEbg}}
            \label{table:dciaFHjEbg_kho}
        \end{table}
        \begin{table}[H]
            \centering
            \begin{tabular}{| c | c | c | c | c | c | c | c | c | c | c |}
                \hline
                $q\symbol{92}t$&$-9$&$-8$&$-7$&$-6$&$-5$&$-4$&$-3$&$-2$&$-1$&$0$\\
                \hline
                $-15$&1&&&&&&&&&\\
                \hline
                $-13$&&&&&&&&&&\\
                \hline
                $-11$&&1&1&&&&&&&\\
                \hline
                $-9$&&&&1&1&&&&&\\
                \hline
                $-7$&&&&1&&&&&&\\
                \hline
                $-5$&&&&&1&2&&&&\\
                \hline
                $-3$&&&&&&&&1&&\\
                \hline
                $-1$&&&&&&&1&&&1\\
                \hline
                $1$&&&&&&&&&1&1\\
                \hline
            \end{tabular}
            \caption{Khovanov Polynomial for \texttt{iNHlPJqCoKFmdABgE}}
            \label{table:iNHlPJqCoKFmdABgE_kho}
        \end{table}
        The $T(7,2)$ knot, also the $7_{1}$ knot, and occasionally called the
        septafoil, has the same Jones polynomial as \texttt{fJGkHlICEABd}. The
        Khovanov polynomials are distinct
        (Tabs.~\ref{table:t_7_2_kho}-\ref{table:fJGkHlICEABd_kho}).
        \begin{table}[H]
            \centering
            \begin{tabular}{| c | c | c | c | c | c | c | c | c |}
                \hline
                $q\symbol{92}t$&$-7$&$-6$&$-5$&$-4$&$-3$&$-2$&$-1$&$0$\\
                \hline
                $-21$&1&&&&&&&\\
                \hline
                $-19$&&&&&&&&\\
                \hline
                $-17$&&1&1&&&&&\\
                \hline
                $-15$&&&&&&&&\\
                \hline
                $-13$&&&&1&1&&&\\
                \hline
                $-11$&&&&&&&&\\
                \hline
                $-9$&&&&&&1&&\\
                \hline
                $-7$&&&&&&&&1\\
                \hline
                $-5$&&&&&&&&1\\
                \hline
            \end{tabular}
            \caption{Khovanov Polynomial for $T(7,2)$}
            \label{table:t_7_2_kho}
        \end{table}
        \begin{table}[H]
            \centering
            \begin{tabular}{| c | c | c | c | c | c | c | c | c | c | c |}
                \hline
                $q\symbol{92}t$&$-9$&$-8$&$-7$&$-6$&$-5$&$-4$&$-3$&$-2$&$-1$&$0$\\
                \hline
                $-21$&1&&&&&&&&&\\
                \hline
                $-19$&&&&&&&&&&\\
                \hline
                $-17$&&1&1&&&&&&&\\
                \hline
                $-15$&&&&1&1&&&&&\\
                \hline
                $-13$&&&&1&1&&&&&\\
                \hline
                $-11$&&&&&1&2&1&&&\\
                \hline
                $-9$&&&&&&1&&&&\\
                \hline
                $-7$&&&&&&&1&2&&\\
                \hline
                $-5$&&&&&&&&&&1\\
                \hline
                $-3$&&&&&&&&&1&1\\
                \hline
            \end{tabular}
            \caption{Khovanov Polynomial for \texttt{fJGkHlICEABd}}
            \label{table:fJGkHlICEABd_kho}
        \end{table}
        Lastly, the $T(11,2)$ torus knot has the same Jones polynomial as the
        14 crossing knot \texttt{gHlImJnKBDFAce}. Once again the Khovanov
        polynomials differ.
        \begin{table}[H]
            \centering
            \begin{tabular}{| c | c | c | c | c | c | c | c | c | c | c | c | c |}
                \hline
                $q\symbol{92}t$&$0$&$1$&$2$&$3$&$4$&$5$&$6$&$7$&$8$&$9$&$10$&$11$\\
                \hline
                $9$&1&&&&&&&&&&&\\
                \hline
                $11$&1&&&&&&&&&&&\\
                \hline
                $13$&&&1&&&&&&&&&\\
                \hline
                $15$&&&&&&&&&&&&\\
                \hline
                $17$&&&&1&1&&&&&&&\\
                \hline
                $19$&&&&&&&&&&&&\\
                \hline
                $21$&&&&&&1&1&&&&&\\
                \hline
                $23$&&&&&&&&&&&&\\
                \hline
                $25$&&&&&&&&1&1&&&\\
                \hline
                $27$&&&&&&&&&&&&\\
                \hline
                $29$&&&&&&&&&&1&1&\\
                \hline
                $31$&&&&&&&&&&&&\\
                \hline
                $33$&&&&&&&&&&&&1\\
                \hline
            \end{tabular}
            \caption{Khovanov Polynomial for $T(11,\,2)$}
            \label{table:t_2_11_kho}
        \end{table}
        \begin{table}[H]
            \centering
            \begin{tabular}{| c | c | c | c | c | c | c | c | c | c | c | c | c | c | c |}
                \hline
                $q\symbol{92}t$&$0$&$1$&$2$&$3$&$4$&$5$&$6$&$7$&$8$&$9$&$10$&$11$&$12$&$13$\\
                \hline
                $9$&1&&&&&&&&&&&&&\\
                \hline
                $11$&1&&&&&&&&&&&&&\\
                \hline
                $13$&&&1&&&&&&&&&&&\\
                \hline
                $15$&&&&&1&1&&&&&&&&\\
                \hline
                $17$&&&&1&1&&&&&&&&&\\
                \hline
                $19$&&&&&&1&2&1&&&&&&\\
                \hline
                $21$&&&&&&&&&1&&&&&\\
                \hline
                $23$&&&&&&&&1&2&1&&&&\\
                \hline
                $25$&&&&&&&&&&1&1&&&\\
                \hline
                $27$&&&&&&&&&&1&1&&&\\
                \hline
                $29$&&&&&&&&&&&&1&1&\\
                \hline
                $31$&&&&&&&&&&&&&&\\
                \hline
                $33$&&&&&&&&&&&&&&1\\
                \hline
            \end{tabular}
            \caption{Khovanov Polynomial for \texttt{gHlImJnKBDFAce}}
            \label{table:t_gHlImJnKBDFAce_kho}
        \end{table}
        \par\hfill\par
        Unlike torus knots, with twist knots we found a few new matches after
        expanding our search to 19 crossings
        (Tab.~\ref{table:matching_twist_knots}). The first match is the
        figure-eight knot, $m_{2}$, which has the same Jones polynomial as
        K11n19 from the Hoste-Thistlewaite table. The Khovanov polynomials are
        given in Tabs.~\ref{table:m_2_kho}-\ref{table:eikGbHJCaFd_kho}.
        \begin{table}[H]
            \centering
            \begin{tabular}{| c | c | c | c | c | c |}
                \hline
                $q\symbol{92}t$&$-2$&$-1$&$0$&$1$&$2$\\
                \hline
                $-5$&1&&&&\\
                \hline
                $-3$&&&&&\\
                \hline
                $-1$&&1&1&&\\
                \hline
                $1$&&&1&1&\\
                \hline
                $3$&&&&&\\
                \hline
                $5$&&&&&1\\
                \hline
            \end{tabular}
            \caption{Khovanov Polynomial for the Figure-Eight Knot}
            \label{table:m_2_kho}
        \end{table}
        \begin{table}[H]
            \centering
            \begin{tabular}{| c | c | c | c | c | c | c | c | c |}
                \hline
                $q\symbol{92}t$&$-4$&$-3$&$-2$&$-1$&$0$&$1$&$2$&$3$\\
                \hline
                $-5$&1&&&&&&&\\
                \hline
                $-3$&&&&&&&&\\
                \hline
                $-1$&&1&1&&&&&\\
                \hline
                $1$&&&&1&1&&&\\
                \hline
                $3$&&&&1&1&&&\\
                \hline
                $5$&&&&&1&1&1&\\
                \hline
                $7$&&&&&&&&\\
                \hline
                $9$&&&&&&&1&1\\
                \hline
            \end{tabular}
            \caption{Khovanov Polynomial for \texttt{eikGbHJCaFd}}
            \label{table:eikGbHJCaFd_kho}
        \end{table}
        The $m_{3}$ twist knot, which is $5_{2}$ on the Rolfsen table,
        has the same Jones polynomial as (at least) three other knots. In each
        case the Khovanov polynomial distinguishes it
        (Tabs.~\ref{table:m_3_kho}-\ref{table:hGJaMlCdEKBfI_kho}).
        \begin{table}[H]
            \centering
            \begin{tabular}{| c | c | c | c | c | c | c |}
                \hline
                $q\symbol{92}t$&$0$&$1$&$2$&$3$&$4$&$5$\\
                \hline
                $1$&1&&&&&\\
                \hline
                $3$&1&1&&&&\\
                \hline
                $5$&&&1&&&\\
                \hline
                $7$&&&1&&&\\
                \hline
                $9$&&&&1&1&\\
                \hline
                $11$&&&&&&\\
                \hline
                $13$&&&&&&1\\
                \hline
            \end{tabular}
            \caption{Khovanov Polynomial for the $5_{2}$ Knot}
            \label{table:m_3_kho}
        \end{table}
        \begin{table}[H]
            \centering
            \begin{tabular}{| c | c | c | c | c | c | c | c | c | c |}
                \hline
                $q\symbol{92}t$&$-2$&$-1$&$0$&$1$&$2$&$3$&$4$&$5$&$6$\\
                \hline
                $1$&1&&&&&&&&\\
                \hline
                $3$&&&&&&&&&\\
                \hline
                $5$&&1&2&&&&&&\\
                \hline
                $7$&&&1&1&1&&&&\\
                \hline
                $9$&&&&1&1&&&&\\
                \hline
                $11$&&&&&1&2&1&&\\
                \hline
                $13$&&&&&&1&1&1&\\
                \hline
                $15$&&&&&&&1&1&\\
                \hline
                $17$&&&&&&&&1&1\\
                \hline
            \end{tabular}
            \caption{Khovanov Polynomial for \texttt{dgikFHaEjbc}}
            \label{table:dgikFHaEjbc_kho}
        \end{table}
        \begin{table}[H]
            \centering
            \begin{tabular}{| c | c | c | c | c | c | c | c | c |}
                \hline
                $q\symbol{92}t$&$0$&$1$&$2$&$3$&$4$&$5$&$6$&$7$\\
                \hline
                $-1$&1&1&&&&&&\\
                \hline
                $1$&1&&&&&&&\\
                \hline
                $3$&&&1&1&&&&\\
                \hline
                $5$&&&&&1&&&\\
                \hline
                $7$&&&&&1&&&\\
                \hline
                $9$&&&&&&1&1&\\
                \hline
                $11$&&&&&&&&\\
                \hline
                $13$&&&&&&&&1\\
                \hline
            \end{tabular}
            \caption{Khovanov Polynomial for \texttt{gfJKHlaIEBCD}}
            \label{table:gfJKHlaIEBCD_kho}
        \end{table}
        \begin{table}[H]
            \centering
            \begin{tabular}{| c | c | c | c | c | c | c | c | c | c | c | c |}
                \hline
                $q\symbol{92}t$&$-2$&$-1$&$0$&$1$&$2$&$3$&$4$&$5$&$6$&$7$&$8$\\
                \hline
                $1$&1&&&&&&&&&&\\
                \hline
                $3$&&&&&&&&&&&\\
                \hline
                $5$&&1&2&&&&&&&&\\
                \hline
                $7$&&&1&1&1&&&&&&\\
                \hline
                $9$&&&&1&1&&&&&&\\
                \hline
                $11$&&&&&1&2&1&&&&\\
                \hline
                $13$&&&&&&1&1&1&&&\\
                \hline
                $15$&&&&&&&1&1&&&\\
                \hline
                $17$&&&&&&&&1&2&1&\\
                \hline
                $19$&&&&&&&&&&&\\
                \hline
                $21$&&&&&&&&&&1&1\\
                \hline
            \end{tabular}
            \caption{Khovanov Polynomial for \texttt{hGJaMlCdEKBfI}}
            \label{table:hGJaMlCdEKBfI_kho}
        \end{table}
        The $m_{5}$ twist knot, which is $7_{2}$ on the Rolfsen table, has the
        same Jones polynomial as \texttt{bhDGijCkaef}. The Khovanov polynomials
        are given in Tabs.~\ref{table:m_5_kho} and \ref{table:bhDGijCkaef_kho}.
        \begin{table}[H]
            \centering
            \begin{tabular}{| c | c | c | c | c | c | c | c | c |}
                \hline
                $q\symbol{92}t$&$0$&$1$&$2$&$3$&$4$&$5$&$6$&$7$\\
                \hline
                $1$&1&&&&&&&\\
                \hline
                $3$&1&1&&&&&&\\
                \hline
                $5$&&&1&&&&&\\
                \hline
                $7$&&&1&1&&&&\\
                \hline
                $9$&&&&1&1&&&\\
                \hline
                $11$&&&&&1&&&\\
                \hline
                $13$&&&&&&1&1&\\
                \hline
                $15$&&&&&&&&\\
                \hline
                $17$&&&&&&&&1\\
                \hline
            \end{tabular}
            \caption{Khovanov Polynomial for the $7_{2}$ Knot}
            \label{table:m_5_kho}
        \end{table}
        \begin{table}[H]
            \centering
            \begin{tabular}{| c | c | c | c | c | c | c | c | c |}
                \hline
                $q\symbol{92}t$&$-2$&$-1$&$0$&$1$&$2$&$3$&$4$&$5$\\
                \hline
                $1$&1&&&&&&&\\
                \hline
                $3$&&&&&&&&\\
                \hline
                $5$&&1&2&&&&&\\
                \hline
                $7$&&&1&&&&&\\
                \hline
                $9$&&&&1&1&&&\\
                \hline
                $11$&&&&&1&1&1&\\
                \hline
                $13$&&&&&&1&1&\\
                \hline
                $15$&&&&&&&1&1\\
                \hline
                $17$&&&&&&&&1\\
                \hline
            \end{tabular}
            \caption{Khovanov Polynomial for \texttt{bhDGijCkaef}}
            \label{table:bhDGijCkaef_kho}
        \end{table}
        Our first new result comes with $m_{6}$. An 18 crossing knot was found
        that has the same Jones polynomial, but the Khovanov polynomials
        differ. This is shown below
        (Tabs.~\ref{table:m_6_kho} and \ref{table:jpIFNMrClqOhkEDabg_kho}.).
        $m_{6}$ also shares its Jones polynomial with an 11 and a 13 crossing
        knot. The Khovanov polynomials are given in
        Tabs.~\ref{table:cefIgbajkDh_kho} and
        \ref{table:femIbaJKLCGHd_kho}.
        \begin{table}[H]
            \centering
            \begin{tabular}{| c | c | c | c | c | c | c | c | c | c |}
                \hline
                $q\symbol{92}t$&$-2$&$-1$&$0$&$1$&$2$&$3$&$4$&$5$&$6$\\
                \hline
                $-5$&1&&&&&&&&\\
                \hline
                $-3$&&&&&&&&&\\
                \hline
                $-1$&&1&2&&&&&&\\
                \hline
                $1$&&&1&1&&&&&\\
                \hline
                $3$&&&&1&1&&&&\\
                \hline
                $5$&&&&&1&1&&&\\
                \hline
                $7$&&&&&&1&&&\\
                \hline
                $9$&&&&&&&1&1&\\
                \hline
                $11$&&&&&&&&&\\
                \hline
                $13$&&&&&&&&&1\\
                \hline
            \end{tabular}
            \caption{Khovanov Polynomial for the $8_{1}$ Knot}
            \label{table:m_6_kho}
        \end{table}
        \begin{table}[H]
            \centering
            \begin{tabular}{| c | c | c | c | c | c | c | c | c | c |}
                \hline
                $q\symbol{92}t$&$-4$&$-3$&$-2$&$-1$&$0$&$1$&$2$&$3$&$4$\\
                \hline
                $-5$&1&&&&&&&&\\
                \hline
                $-3$&&&&&&&&&\\
                \hline
                $-1$&&1&2&&&&&&\\
                \hline
                $1$&&&&1&1&&&&\\
                \hline
                $3$&&&&2&2&&&&\\
                \hline
                $5$&&&&&1&2&1&&\\
                \hline
                $7$&&&&&&1&&&\\
                \hline
                $9$&&&&&&&2&2&\\
                \hline
                $11$&&&&&&&&&\\
                \hline
                $13$&&&&&&&&&1\\
                \hline
            \end{tabular}
            \caption{Khovanov Polynomial for \texttt{cefIgbajkDh}}
            \label{table:cefIgbajkDh_kho}
        \end{table}
        \begin{table}[H]
            \centering
            \begin{tabular}{| c | c | c | c | c | c | c | c | c | c |}
                \hline
                $q\symbol{92}t$&$-4$&$-3$&$-2$&$-1$&$0$&$1$&$2$&$3$&$4$\\
                \hline
                $-5$&1&&&&&&&&\\
                \hline
                $-3$&&&&&&&&&\\
                \hline
                $-1$&&1&2&&&&&&\\
                \hline
                $1$&&&&1&1&&&&\\
                \hline
                $3$&&&&2&2&&&&\\
                \hline
                $5$&&&&&1&2&1&&\\
                \hline
                $7$&&&&&&1&&&\\
                \hline
                $9$&&&&&&&2&2&\\
                \hline
                $11$&&&&&&&&&\\
                \hline
                $13$&&&&&&&&&1\\
                \hline
            \end{tabular}
            \caption{Khovanov Polynomial for \texttt{femIbaJKLCGHd}}
            \label{table:femIbaJKLCGHd_kho}
        \end{table}
        \begin{table}[H]
            \centering
            \begin{tabular}{| c | c | c | c | c | c | c | c | c | c | c | c | c | c |}
                \hline
                $q\symbol{92}t$&$-4$&$-3$&$-2$&$-1$&$0$&$1$&$2$&$3$&$4$&$5$&$6$&$7$&$8$\\
                \hline
                $-5$&1&&&&&&&&&&&&\\
                \hline
                $-3$&&&&&&&&&&&&&\\
                \hline
                $-1$&&1&2&&&&&&&&&&\\
                \hline
                $1$&&&&1&1&&&&&&&&\\
                \hline
                $3$&&&&2&2&&&&&&&&\\
                \hline
                $5$&&&&&2&4&2&&&&&&\\
                \hline
                $7$&&&&&&1&1&1&&&&&\\
                \hline
                $9$&&&&&&&3&4&1&&&&\\
                \hline
                $11$&&&&&&&&1&3&2&&&\\
                \hline
                $13$&&&&&&&&&2&1&&&\\
                \hline
                $15$&&&&&&&&&&2&3&1&\\
                \hline
                $17$&&&&&&&&&&&&&\\
                \hline
                $19$&&&&&&&&&&&&1&1\\
                \hline
            \end{tabular}
            \caption{Khovanov Polynomial for \texttt{jpIFNMrClqOhkEDabg}}
            \label{table:jpIFNMrClqOhkEDabg_kho}
        \end{table}
        Note the Khovanov polynomials of \texttt{cefIgbajkDh} and
        \texttt{femIbaJKLCGHd} are identical, showing this is not a perfect
        invariant.
        \par\hfill\par
        The knots $m_{7}$, $m_{8}$, and $m_{9}$ have the same Jones polynomial
        of at least one other knot. The Khovanov polynomials are given below.
        \begin{table}[H]
            \centering
            \begin{tabular}{| c | c | c | c | c | c | c | c | c | c | c |}
                \hline
                $q\symbol{92}t$&$0$&$1$&$2$&$3$&$4$&$5$&$6$&$7$&$8$&$9$\\
                \hline
                $1$&1&&&&&&&&&\\
                \hline
                $3$&1&1&&&&&&&&\\
                \hline
                $5$&&&1&&&&&&&\\
                \hline
                $7$&&&1&1&&&&&&\\
                \hline
                $9$&&&&1&1&&&&&\\
                \hline
                $11$&&&&&1&1&&&&\\
                \hline
                $13$&&&&&&1&1&&&\\
                \hline
                $15$&&&&&&&1&&&\\
                \hline
                $17$&&&&&&&&1&1&\\
                \hline
                $19$&&&&&&&&&&\\
                \hline
                $21$&&&&&&&&&&1\\
                \hline
            \end{tabular}
            \caption{Khovanov Polynomial for $m_{7}$}
            \label{table:m_7_kho}
        \end{table}
        \begin{table}[H]
            \centering
            \begin{tabular}{| c | c | c | c | c | c | c | c | c | c | c |}
                \hline
                $q\symbol{92}t$&$-2$&$-1$&$0$&$1$&$2$&$3$&$4$&$5$&$6$&$7$\\
                \hline
                $1$&1&&&&&&&&&\\
                \hline
                $3$&&&&&&&&&&\\
                \hline
                $5$&&1&2&&&&&&&\\
                \hline
                $7$&&&1&1&&&&&&\\
                \hline
                $9$&&&&1&1&&&&&\\
                \hline
                $11$&&&&&1&1&&&&\\
                \hline
                $13$&&&&&&1&1&&&\\
                \hline
                $15$&&&&&&&1&&&\\
                \hline
                $17$&&&&&&&&1&1&\\
                \hline
                $19$&&&&&&&&&&\\
                \hline
                $21$&&&&&&&&&&1\\
                \hline
            \end{tabular}
            \caption{Khovanov Polynomial for \texttt{cgjFHIaDEkb}}
            \label{table:cgjFHIaDEkb_kho}
        \end{table}
        \begin{table}[H]
            \centering
            \begin{tabular}{| c | c | c | c | c | c | c | c | c | c | c | c |}
                \hline
                $q\symbol{92}t$&$-2$&$-1$&$0$&$1$&$2$&$3$&$4$&$5$&$6$&$7$&$8$\\
                \hline
                $-5$&1&&&&&&&&&&\\
                \hline
                $-3$&&&&&&&&&&&\\
                \hline
                $-1$&&1&2&&&&&&&&\\
                \hline
                $1$&&&1&1&&&&&&&\\
                \hline
                $3$&&&&1&1&&&&&&\\
                \hline
                $5$&&&&&1&1&&&&&\\
                \hline
                $7$&&&&&&1&1&&&&\\
                \hline
                $9$&&&&&&&1&1&&&\\
                \hline
                $11$&&&&&&&&1&&&\\
                \hline
                $13$&&&&&&&&&1&1&\\
                \hline
                $15$&&&&&&&&&&&\\
                \hline
                $17$&&&&&&&&&&&1\\
                \hline
            \end{tabular}
            \caption{Khovanov Polynomial for $m_{8}$}
            \label{table:m_8_kho}
        \end{table}
        \begin{table}[H]
            \centering
            \begin{tabular}{| c | c | c | c | c | c | c | c | c | c | c | c | c | c |}
                \hline
                $q\symbol{92}t$&$-6$&$-5$&$-4$&$-3$&$-2$&$-1$&$0$&$1$&$2$&$3$&$4$&$5$&$6$\\
                \hline
                $-5$&1&&&&&&&&&&&&\\
                \hline
                $-3$&&&&&&&&&&&&&\\
                \hline
                $-1$&&1&2&&&&&&&&&&\\
                \hline
                $1$&&&&1&1&&&&&&&&\\
                \hline
                $3$&&&&2&1&&1&&&&&&\\
                \hline
                $5$&&&&&1&3&2&&&&&&\\
                \hline
                $7$&&&&&&1&&1&2&&&&\\
                \hline
                $9$&&&&&&&2&2&&&&&\\
                \hline
                $11$&&&&&&&&&1&3&1&&\\
                \hline
                $13$&&&&&&&&&1&&&1&\\
                \hline
                $15$&&&&&&&&&&&1&1&\\
                \hline
                $17$&&&&&&&&&&&&&1\\
                \hline
            \end{tabular}
            \caption{Khovanov Polynomial for \texttt{knIHoBjCDQrMPaeLgF}}
            \label{table:knIHoBjCDQrMPaeLgF_kho}
        \end{table}
        \begin{table}[H]
            \centering
            \begin{tabular}{| c | c | c | c | c | c | c | c | c | c | c | c | c |}
                \hline
                $q\symbol{92}t$&$0$&$1$&$2$&$3$&$4$&$5$&$6$&$7$&$8$&$9$&$10$&$11$\\
                \hline
                $1$&1&&&&&&&&&&&\\
                \hline
                $3$&1&1&&&&&&&&&&\\
                \hline
                $5$&&&1&&&&&&&&&\\
                \hline
                $7$&&&1&1&&&&&&&&\\
                \hline
                $9$&&&&1&1&&&&&&&\\
                \hline
                $11$&&&&&1&1&&&&&&\\
                \hline
                $13$&&&&&&1&1&&&&&\\
                \hline
                $15$&&&&&&&1&1&&&&\\
                \hline
                $17$&&&&&&&&1&1&&&\\
                \hline
                $19$&&&&&&&&&1&&&\\
                \hline
                $21$&&&&&&&&&&1&1&\\
                \hline
                $23$&&&&&&&&&&&&\\
                \hline
                $25$&&&&&&&&&&&&1\\
                \hline
            \end{tabular}
            \caption{Khovanov Polynomial for $m_{9}$}
            \label{table:m_9_kho}
        \end{table}
        \begin{table}[H]
            \centering
            \begin{tabular}{| c | c | c | c | c | c | c | c | c | c | c | c | c |}
                \hline
                $q\symbol{92}t$&$-2$&$-1$&$0$&$1$&$2$&$3$&$4$&$5$&$6$&$7$&$8$&$9$\\
                \hline
                $1$&1&&&&&&&&&&&\\
                \hline
                $3$&&&&&&&&&&&&\\
                \hline
                $5$&&1&2&&&&&&&&&\\
                \hline
                $7$&&&1&2&1&&&&&&&\\
                \hline
                $9$&&&&1&1&&&&&&&\\
                \hline
                $11$&&&&&2&3&1&&&&&\\
                \hline
                $13$&&&&&&1&2&1&&&&\\
                \hline
                $15$&&&&&&&2&2&&&&\\
                \hline
                $17$&&&&&&&&2&3&1&&\\
                \hline
                $19$&&&&&&&&&1&&&\\
                \hline
                $21$&&&&&&&&&&2&2&\\
                \hline
                $23$&&&&&&&&&&&&\\
                \hline
                $25$&&&&&&&&&&&&1\\
                \hline
            \end{tabular}
            \caption{Khovanov Polynomial for \texttt{jopIFMrDlqNhkEabcg}}
            \label{table:jopIFMrDlqNhkEabcg_kho}
        \end{table}
    \section{Remarks}
        Given a knot $K$ and a torus or twist knot $T$, the Jones polynomials
        of $K$, $T$, and the \textit{mirrors} of $K$ and $T$ were computed and
        compared. This is obtained quickly by
        substituting $q\mapsto{q}^{-1}$. The Khovanov polynomial has a similar
        reflection formula, $(q,\,t)\mapsto(q^{-1},\,t^{-1})$. DT code is
        mirror agnostic, and the tables in the previous section give the
        Khovanov polynomials for the matching knot (whether that be the knot
        given by the corresponding DT code or its mirror).
        \par\hfill\par
        An interesting thing to note is that not all twist knots are
        transversally, or Legendrian, simple, yet there are no knots up to
        19 crossings with the same Khovanov polynomial as a twist knot. This
        leads to the following question.
        \begin{question}
            Does Khovanov homology distinguish all twist knots?
        \end{question}
    \section{Future Work}
        The Legendrian knot atlas \cite{LegendrianKnotAtlas} has a list of
        Legendrian knots that are neither torus nor twist knots. We will be
        expanding our search to include these knots, comparing them against
        all knots up to 19 crossings as well.
    \section{Acknowledgements}
        We thank Shadi Ali Ahmad, Ilya Kryukov and Jacob Swenberg for their
        observation and remarks in the early stage of this work. The numerical
        results of Samantha Allen and Jacob Swenberg about the Jones Polynomial
        of a connected sum of two Hopf links were very inspirational since they
        provided the connection to the work of Fan Ding and Hansjorg Geiges on
        the Legendrian simple links . We are very grateful to Nikolay Pultsin
        for his indispensable help with the Khovanov homology code available
        on the internet. We are thankful to Jim Hoste, Morwen Thistlethwaite
        and Jeff Weeks for sharing their database of DT codes of knots.
        We also thank Steven Sivek and Matthew Heddon
        for alarming us about the bug that
        existed in our KFH code.
    \newpage
    \bibliographystyle{plain}
    \bibliography{bib.bib}
    \newpage
    The source code used to generate this document, including figures,
    is free software and released under version 3 of the GNU General Public
    License.
    \par\hfill\par
    Vladimir Chernov
    \par
    6188 Kemeny Hall
    \par
    Mathematics Department, Dartmouth College
    \par
    Hanover, NH USA 03755
    \par
    Vladimir.Chernov@dartmouth.edu
    \par\hfill\par
    Ryan Maguire
    \par
    6188 Kemeny Hall
    \par
    Mathematics Department, Dartmouth College
    \par
    Hanover, NH USA 03755
    \par
    Ryan.J.Maguire.GR@dartmouth.edu
\end{document}
