%-----------------------------------LICENSE------------------------------------%
%   This file is part of Mathematics-and-Physics.                              %
%                                                                              %
%   Mathematics-and-Physics is free software: you can redistribute it and/or   %
%   modify it under the terms of the GNU General Public License as             %
%   published by the Free Software Foundation, either version 3 of the         %
%   License, or (at your option) any later version.                            %
%                                                                              %
%   Mathematics-and-Physics is distributed in the hope that it will be useful, %
%   but WITHOUT ANY WARRANTY; without even the implied warranty of             %
%   MERCHANTABILITY or FITNESS FOR A PARTICULAR PURPOSE.  See the              %
%   GNU General Public License for more details.                               %
%                                                                              %
%   You should have received a copy of the GNU General Public License along    %
%   with Mathematics-and-Physics.  If not, see <https://www.gnu.org/licenses/>.%
%------------------------------------------------------------------------------%
%       Author: Ryan Maguire                                                   %
%       Date:   2022/03/14                                                     %
%------------------------------------------------------------------------------%
\documentclass{article}

\usepackage{geometry}
\geometry{margin=0.75in}

% Needed for mathbb font style.
\usepackage{amssymb}

% Tools for creating the theorem environment.
\usepackage{amsthm}
\usepackage{amsmath}

% Used for pictures.
\usepackage{graphics}
\usepackage{float}

% Hyperlinks for labels.
\usepackage{hyperref}

% Display style for hyperlinks.
\hypersetup{
    colorlinks=true,
    linkcolor=blue
}

% Create a theorem environment, similar to most textbooks.
\theoremstyle{plain}
\newtheorem{theorem}{Theorem}
\newtheorem{conjecture}{Conjecture}

% No indent and no paragraph skip.
\setlength{\parindent}{0em}
\setlength{\parskip}{0em}

% Title page information.
\title{New Data for a Conjecture on Legendrian and Transversally Simple Knots}
\author{Vladimir Chernov\hspace{2em}Ryan Maguire}
\date{July 2022}

\begin{document}
    \maketitle
    \tableofcontents
    \begin{abstract}
        \noindent
        In [PREVIOUS PAPER] we conjectured that Khovanov homology is able to
        distinguish legendrian and transversally simple knots. Numerical data
        was provided using twist and torus knots, comparing the Khovanov
        homologies of these families with all knots up to 17 crossings. This
        experiment has been expanded to 19 crossings.
        \par\hfill\par
        \textit{AMS Classification: Primary 57K18, Secondary 53D12, 53D10}
    \end{abstract}
    \section{Conjectures on Khovanov Homology}
        In [PREVIOUS PAPER] we proposed the following conjecture.
        \begin{conjecture}
            If a link type $L$ is Legendrian simple, then the Khovanov homology
            of $L$ distinguish it. That is, if $\tilde{L}$
            is another link with the same Khovanov homology, then $\tilde{L}$ is
            equivalent to $L$.
        \end{conjecture}
        Numerical evidence was tallied using the torus and twist knots to gather
        numerical evidence for this claim. The torus knots are known to be
        Legendrian simple, and Etnyre \textit{et al.} classifies when twist
        knots are transversally simple, giving us two infinite families to use.
        Knots having the same Jones polynomial as a torus or twist knot were
        saved for later examination since the Jones polynomial is
        computationally more efficient than Khovanov homology. For torus knots
        we were left with four matching knots.
        \begin{table}
            \centering
            \resizebox{\textwidth}{!}{%
                \begin{tabular}{| c | c | c |}
                    \hline
                        Torus Knot&
                        Non-Torus Knot&
                        Jones Polynomial\\
                    \hline
                        $(2,\,5)$&
                        $8,\,6,\,18,\,2,\,-12,\,-16,\,20,\,-10,\,4,\,14$&
                        $-q^{14}+q^{12}-q^{10}+q^{8}+q^{4}$\\
                    \hline
                        $(2,\,7)$&
                        $12,\,-20,\,-14,\,22,\,-16,\,24,\,-18,\,-6,\,-10,\,-2,\,-4,\,8$&
                        $-q^{20}+q^{18}-q^{16}+q^{14}-q^{12}+q^{10}+q^{6}$\\
                    \hline
                        $(2,\,11)$&
                        $14,\,-16,\,24,\,-18,\,26,\,-20,\,28,\,-22,\,-4,\,-8,\,-12,\,-2,\,6,\,10$&
                        $-q^{32}+q^{30}-q^{28}+q^{26}-q^{24}+q^{22}-q^{20}+q^{18}-q^{16}+q^{14}+q^{10}$\\
                    \hline
                        $(2,\,5)$&
                        $18,\,-28,\,-16,\,24,\,-32,\,-20,\,34,\,-6,\,30,\,-22,\,-12,\,26,\,8,\,-2,\,-4,\,14,\,-10$&
                        $-q^{14}+q^{12}-q^{10}+q^{8}+q^{4}$\\
                    \hline
                \end{tabular}%
            }
            \caption{Knots whose Jones polynomial matches that of a Torus Knot}
        \end{table}

        The computations were carried out using the \texttt{regina} and
        \texttt{JavaKh-v2} libraries.
    \section{Numerical Results}
        In our search we found that four torus knots had the same Jones
        polynomial as a non-torus knot. The $T(2,5)$ knot, which is the
        cinquefoil, matches $10_{132}$. This result has already been known,
        and it has also been known for some time that Khovanov homology
        distinguishes these two (See \cite{KatlasKhoHo}). The resulting
        Khovanov polynomials are given in the following tables. The coefficient
        of $q^{\ell}t^{r}$ is given by the $(r,\ell)$ slot in the tables. Empty
        represents a zero coefficient. The DT code of $10_{132}$ is
        4, 8, $-12$, 2, $-16$, $-6$, $-20$, $-18$, $-10$, $-14$.
        \begin{table}[H]
            \centering
            \begin{tabular}{| c | c | c | c | c | c | c |}
                \hline
                $q/t$&$-5$&$-4$&$-3$&$-2$&$-1$&$0$\\
                \hline
                $-15$&1&&&&&\\
                \hline
                $-13$&&&&&&\\
                \hline
                $-11$&&1&1&&&\\
                \hline
                $-9$&&&&&&\\
                \hline
                $-7$&&&&1&&\\
                \hline
                $-5$&&&&&&1\\
                \hline
                $-3$&&&&&&1\\
                \hline
            \end{tabular}
            \caption{Khovanov Polynomial for $T(5,2)$}
        \end{table}
        \begin{table}[H]
            \centering
            \begin{tabular}{| c | c | c | c | c | c | c | c | c |}
                \hline
                $q/t$&$-7$&$-6$&$-5$&$-4$&$-3$&$-2$&$-1$&$0$\\
                \hline
                $-15$&1&&&&&&&\\
                \hline
                $-13$&&&&&&&&\\
                \hline
                $-11$&&1&1&&&&&\\
                \hline
                $-9$&&&&1&1&&&\\
                \hline
                $-7$&&&&1&&&&\\
                \hline
                $-5$&&&&&1&2&&\\
                \hline
                $-3$&&&&&&&&1\\
                \hline
                $-1$&&&&&&&1&1\\
                \hline
            \end{tabular}
            \caption{Khovanov Polynomial for $10_{132}$}
        \end{table}
        The cinquefoil also has the same Jones polynomial as a 17 crossing knot,
        but once again the Khovanov homologies were able to distringuish
        between them. The DT code of this 17 crossing knot is
        18, $-28$, $-16$, 24, $-32$, $-20$, 34, $-6$, 30, $-22$, $-12$, 26,
        8, $-2$, $-4$, 14, $-10$.
        \begin{table}[H]
            \centering
            \begin{tabular}{| c | c | c | c | c | c | c | c | c | c | c |}
                \hline
                $q/t$&$-9$&$-8$&$-7$&$-6$&$-5$&$-4$&$-3$&$-2$&$-1$&$0$\\
                \hline
                $-15$&1&&&&&&&&&\\
                \hline
                $-13$&&&&&&&&&&\\
                \hline
                $-11$&&1&1&&&&&&&\\
                \hline
                $-9$&&&&1&1&&&&&\\
                \hline
                $-7$&&&&1&&&&&&\\
                \hline
                $-5$&&&&&1&2&&&&\\
                \hline
                $-3$&&&&&&&&1&&\\
                \hline
                $-1$&&&&&&&1&&&1\\
                \hline
                $1$&&&&&&&&&1&1\\
                \hline
            \end{tabular}
            \caption{Khovanov Polynomial for the 17 Crossing Knot}
        \end{table}
        The $T(7,2)$ knot, also the $7_{1}$ knot, and occasionally called the
        septafoil, has the same Jones polynomial as a 12 crossing knot. The
        Khovanov polynomials are distinct. The DT of this 12 crossing knot is
        given by 6, 10, 14, $-18$, 2, $-20$, 4, 22, 24, $-8$, $-12$, 16.
        \begin{table}[H]
            \centering
            \begin{tabular}{| c | c | c | c | c | c | c | c | c |}
                \hline
                $q/t$&$-7$&$-6$&$-5$&$-4$&$-3$&$-2$&$-1$&$0$\\
                \hline
                $-21$&1&&&&&&&\\
                \hline
                $-19$&&&&&&&&\\
                \hline
                $-17$&&1&1&&&&&\\
                \hline
                $-15$&&&&&&&&\\
                \hline
                $-13$&&&&1&1&&&\\
                \hline
                $-11$&&&&&&&&\\
                \hline
                $-9$&&&&&&1&&\\
                \hline
                $-7$&&&&&&&&1\\
                \hline
                $-5$&&&&&&&&1\\
                \hline
            \end{tabular}
            \caption{Khovanov Polynomial for $T(7,2)$}
        \end{table}
        \begin{table}[H]
            \centering
            \begin{tabular}{| c | c | c | c | c | c | c | c | c | c | c |}
                \hline
                $q/t$&$-9$&$-8$&$-7$&$-6$&$-5$&$-4$&$-3$&$-2$&$-1$&$0$\\
                \hline
                $-21$&1&&&&&&&&&\\
                \hline
                $-19$&&&&&&&&&&\\
                \hline
                $-17$&&1&1&&&&&&&\\
                \hline
                $-15$&&&&1&1&&&&&\\
                \hline
                $-13$&&&&1&1&&&&&\\
                \hline
                $-11$&&&&&1&2&1&&&\\
                \hline
                $-9$&&&&&&1&&&&\\
                \hline
                $-7$&&&&&&&1&2&&\\
                \hline
                $-5$&&&&&&&&&&1\\
                \hline
                $-3$&&&&&&&&&1&1\\
                \hline
            \end{tabular}
            \caption{Khovanov Polynomial for the 12 Crossing Knot}
        \end{table}
        Lastly, the $T(11,2)$ torus knot has the same Jones polynomial as a
        14 crossing knot. The table for the Khovanov polynomial of this knot
        is quite large and has been omitted. Nevertheless, it differs from the
        Khovanov polynomial of $T(11,2)$. The DT code of this knot is
        8, $-12$, $-16$, $-20$, 22, $-2$, 24, $-4$, 26, $-6$, 28, 10, 14, 18.
        \par\hfill\par
        For twist knots there were 8 matches for the Jones polynomial. In all
        cases the Khovanov homologies were distinct. The following table shows
        the DT code of the matching knots.
        \begin{table}[H]
            \centering
            \begin{tabular}{| c | c |}
                \hline
                Twist Knot&Non-Twist Knot\\
                \hline
                $m_{2}$&\texttt{4 8 10 -16 2 -18 -20 -22 -6 -12 -14}\\
                \hline
                $m_{3}$&\texttt{4 8 -14 2 -16 -18 -20 -6 -10 -22 -12}\\
                \hline
                $m_{3}$&\texttt{4 10 -16 -18 2 20 22 24 -8 -6 12 14}\\
                \hline
                $m_{3}$&\texttt{4 12 16 -22 14 -20 2 8 24 26 -10 -6 18}\\
                \hline
                $m_{5}$&\texttt{4 10 12 16 18 2 -20 6 8 -22 -14}\\
                \hline
                $m_{6}$&\texttt{4 8 -14 2 -18 -20 -6 -22 -12 -10 -16}\\
                \hline
                $m_{6}$&\texttt{4 10 -16 -24 -18 2 -20 -22 -26 -12 -14 -8 -6}\\
                \hline
                $m_{7}$&\texttt{4 8 10 16 2 -18 -20 6 -22 -12 -14}\\
                \hline
            \end{tabular}
            \caption{Non-Twist Knots with the Same Jones Polynomial as a Twist Knot}
        \end{table}
        The $m_{2}$ twist knot is the figure eight knot. The matching 11
        crossing knot is the K11n19 knot from the
        Hoste-Thistlewaite table of 11 crossing knots. That the Jones polynomial
        of these knots match has been known, and this can be found on the
        knot atlas (See \cite{KatlasFigureEight} and \cite{KatlasK11n19}). Also
        found on the knot atlas is that Khovanov homology does indeed
        distinguish these knots.
        \par\hfill\par
        To conclude, for up to 17 crossings, Khovanov homologies
        have thus far been able to distinguish the Legendrian twist knots.
    \section{Acknowledgements}
        We thank Shadi Ali Ahmad, Ilya Kryukov and Jacob Swenberg for their
        observation and remarks in the early stage of this work. The numerical
        results of Samantha Allen and Jacob Swenberg about the Jones Polynomial
        of a connected sum of two Hopf links were very inspirational since they
        provided the connection to the work of Fan Ding and Hansjorg Geiges on
        the Legendrian simple links . We are very grateful to Nikolay Pultsin
        for his indispensable help with the Khovanov homology code available
        on the internet. We are thankful to Jim Hoste, Morwen Thistlethwaite
        and Jeff Weeks for sharing their database of DT codes of knots.
        We also thank Steven Sivek and Matthew Heddon
        for alarming us about the bug that
        existed in our KFH code.
    \newpage
    \bibliographystyle{plain}
    \bibliography{bib.bib}
    \newpage
    The source code used to generate this document, including figures,
    is free software and released under version 3 of the GNU General Public
    License.
    \par\hfill\par
    Vladimir Chernov
    \par
    6188 Kemeny Hall
    \par
    Mathematics Department, Dartmouth College
    \par
    Hanover, NH USA 03755
    \par
    Vladimir.Chernov@dartmouth.edu
    \par\hfill\par
    Ryan Maguire
    \par
    6188 Kemeny Hall
    \par
    Mathematics Department, Dartmouth College
    \par
    Hanover, NH USA 03755
    \par
    Ryan.J.Maguire.GR@dartmouth.edu
\end{document}
