%-----------------------------------LICENSE------------------------------------%
%   This file is part of Mathematics-and-Physics.                              %
%                                                                              %
%   Mathematics-and-Physics is free software: you can redistribute it and/or   %
%   modify it under the terms of the GNU General Public License as             %
%   published by the Free Software Foundation, either version 3 of the         %
%   License, or (at your option) any later version.                            %
%                                                                              %
%   Mathematics-and-Physics is distributed in the hope that it will be useful, %
%   but WITHOUT ANY WARRANTY; without even the implied warranty of             %
%   MERCHANTABILITY or FITNESS FOR A PARTICULAR PURPOSE.  See the              %
%   GNU General Public License for more details.                               %
%                                                                              %
%   You should have received a copy of the GNU General Public License along    %
%   with Mathematics-and-Physics.  If not, see <https://www.gnu.org/licenses/>.%
%------------------------------------------------------------------------------%
%       Author: Ryan Maguire                                                   %
%       Date:   2021/11/16                                                     %
%------------------------------------------------------------------------------%
\documentclass{article}

% Needed for figures.
\usepackage{graphicx}

% Needed for mathbb font style.
\usepackage{amssymb}

% Tools for creating the theorem environment.
\usepackage{amsthm}

% The align environment is here.
\usepackage{amsmath}

% Library of colors, used here for displaying code.
\usepackage{xcolor}

% Used for correct hyperlinks with figures.
\usepackage[font={scriptsize}, hypcap=true, labelsep=colon]{caption}

% Hyperlinks for labels.
\usepackage{hyperref}

% Used for writing algorithms.
\usepackage{algorithm}
\usepackage[noend]{algpseudocode}

\makeatletter
\def\BState{\State\hskip-\ALG@thistlm}
\makeatother

% Display style for hyperlinks.
\hypersetup{
    colorlinks=true,
    linkcolor=blue
}

% Create a theorem environment, similar to most textbooks.
\theoremstyle{plain}
\newtheorem{theorem}{Theorem}

% No indent and no paragraph skip.
\setlength{\parindent}{0em}
\setlength{\parskip}{0em}

% Title page information.
\title{A Survey of Algorithms for the Jones Polynomial}
\author{Ryan Maguire}
\date{\today}

\begin{document}
    \maketitle
    \begin{abstract}
        The Jones polynomial is one of the simpler knot invariants available
        to topologists. It is easy enough to describe, and powerful enough to
        distinguish a great many knots (thought it is not a perfect invariant).
        It is amazing, then, that all known algorithms for its computation are
        tremendously slow. In this survey different methods for computing the
        Jones' polynomial are described and implemented in the C programming
        language, to see which algorithm is best suited to the task of computing
        the invariant for \textit{every day} knots
        (say, less than 50 crossings).
    \end{abstract}
    \section{The Na\"{i}ve Algorithm}
        The simplest algorithm for the Jones polynomial simply uses the
        definition. The Kauffman bracket $\langle{L}\rangle$ of a link $L$ is
        defined as follows:
        \begin{align}
            \langle{\emptyset}\rangle&=1\\
            \langle{L\sqcup\mathbb{S}^{1}}\rangle&=(q+q^{-1})\langle{L}\rangle\\
            \langle{L}\rangle&=\langle{L_{n,0}}\rangle-q\langle{L_{n,1}}\rangle
        \end{align}
        Here, $L\sqcup\mathbb{S}^{1}$ is the disjoint union
        (the \textit{unlink}) of a circle $\mathbb{S}^{1}$ and a link $L$.
        $L_{n,0}$ and $L_{n,1}$ are the links obtained from the 0 and 1
        smoothing of the $n^{th}$ crossing in $L$. The list equation takes an
        $N$ crossing link and reduces it to two $N-1$ crossing links. Using
        the first two equations tells us how to compute the Kauffman bracket
        once we have the disjoint union of a bunch of circles.
\end{document}

