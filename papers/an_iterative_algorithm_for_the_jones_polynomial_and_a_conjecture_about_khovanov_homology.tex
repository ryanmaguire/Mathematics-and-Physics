%-----------------------------------LICENSE------------------------------------%
%   This file is part of Mathematics-and-Physics.                              %
%                                                                              %
%   Mathematics-and-Physics is free software: you can redistribute it and/or   %
%   modify it under the terms of the GNU General Public License as             %
%   published by the Free Software Foundation, either version 3 of the         %
%   License, or (at your option) any later version.                            %
%                                                                              %
%   Mathematics-and-Physics is distributed in the hope that it will be useful, %
%   but WITHOUT ANY WARRANTY; without even the implied warranty of             %
%   MERCHANTABILITY or FITNESS FOR A PARTICULAR PURPOSE.  See the              %
%   GNU General Public License for more details.                               %
%                                                                              %
%   You should have received a copy of the GNU General Public License along    %
%   with Mathematics-and-Physics.  If not, see <https://www.gnu.org/licenses/>.%
%------------------------------------------------------------------------------%
\documentclass{article}
\usepackage{graphicx}                           % Needed for figures.
\usepackage{amsmath}                            % Needed for align.
\usepackage{amssymb}                            % Needed for mathbb.
\usepackage{amsthm}                             % For the theorem environment.
\usepackage[nottoc]{tocbibind}                  % Bibliography in toc.
\usepackage{float}
\usepackage{xcolor}
\usepackage{listings}
\usepackage[font={scriptsize},
            hypcap=true,
            labelsep=colon]{caption}            % Figure captions.
\usepackage{hyperref}

\definecolor{background}{rgb}{0.9,0.9,0.9}

\lstdefinestyle{CStyle}{ 
    backgroundcolor=\color{background},
    commentstyle=\color{gray},
    keywordstyle=\color{blue},
    numberstyle=\tiny\color{red},
    stringstyle=\color{orange},
    basicstyle=\footnotesize,
    breakatwhitespace=false,         
    breaklines=true,                 
    captionpos=b,                    
    keepspaces=true,                 
    numbers=left,                    
    numbersep=5pt,                  
    showspaces=false,                
    showstringspaces=false,
    showtabs=false,                  
    tabsize=2,
    language=C
}

\hypersetup{
    colorlinks=true,
    linkcolor=blue
}

\newtheoremstyle{normal}
    {\topsep}               % Amount of space above the theorem.
    {\topsep}               % Amount of space below the theorem.
    {}                      % Font used for body of theorem.
    {}                      % Measure of space to indent.
    {\bfseries}             % Font of the header of the theorem.
    {}                      % Punctuation between head and body.
    {.5em}                  % Space after theorem head.
    {}

\theoremstyle{normal}
\newtheorem{definition}{Definition}

\theoremstyle{plain}
\newtheorem{theorem}{Theorem}

\title{%
    An Iterative Algorithm for the Jones' Polynomial and a
    Conjecture about Khovanov Homology%
}

\author{Ryan Maguire\hspace{2em}Vladimir Chernov\hspace{2em}Peter Doyle}
\date{Fall 2021}

% No indent and no paragraph skip.
\setlength{\parindent}{0em}
\setlength{\parskip}{0em}

\begin{document}
    \maketitle
    \tableofcontents
    \begin{abstract}
        An iterative algorithm for computing the Jones' polynomial of a knot
        is described and an analysis of the computational complexity is given.
        A generalized conjecture about the Jones' polynomial is shown to be
        false via counterexample, and a conjecture generalizing Kronheimer and
        Mrowka's result on the unknot is made, with supporting numerical
        evidence.
    \end{abstract}
    \section{Extended Gauss Code}
        Given a knot diagram, it is reasonable for a mathematician to wish to
        describe the knot with finite data in such a way that a computer can
        understand and perform computations. This is achieved via Gauss code.
        This is achieved by orienting the knot, labeling the crossings, and
        picking any point on the knot. You then walk along the knot, following
        the orientation, and keep track of the crossings you encounter. That is,
        we label the crossing number, and whether we're on the under strand or
        over strand with a $U$ and $O$, respectively. This is
        illustrated via example for the right handle trefoil knot in
        Fig.~\ref{fig:right_handed_trefoil_gauss_code}.
        \begin{figure}[H]
            \centering
            \includegraphics{../images/trefoil_knot_oriented_with_gauss_code.pdf}
            \caption{Gauss Code for the Right Handed Trefoil}
            \label{fig:right_handed_trefoil_gauss_code}
        \end{figure}
        Gauss code is not unique to a knot diagram since the code is dependent
        on the choice of starting point. This causes two different issues.
        First, when trying to determine if two knots are the same, we must
        check if one Gauss code is a cyclic permutation of another. Secondly,
        and more importantly, this version of Gauss code cannot distinguish
        between a knot and its mirror. If we take the left handed trefoil,
        give it a similar orientation as before, but choose a different
        starting point, we end up with
        Fig.~\ref{fig:left_handed_trefoil_gauss_code}. The resulting Gauss code
        is the same as the Gauss code we get for the right handed trefoil.
        The right handed and left handed trefoil knots are not equivalent
        (they have different Jones' polynomials), meaning Gauss code can't
        distinguish mirrors.
        \begin{figure}[H]
            \centering
            \includegraphics{../images/trefoil_knot_mirror_oriented_with_gauss_code.pdf}
            \caption{Gauss Code for the Left Handed Trefoil}
            \label{fig:left_handed_trefoil_gauss_code}
        \end{figure}
        The solution is to \textit{sign} the crossings. Given an oriented knot,
        we label a crossing positive or negative depending on which strand is
        the over strand and which one is the under strand. This is shown in
        Fig.~\ref{fig:crossing_signs}.
        \begin{figure}[H]
            \centering
            \includegraphics{../images/crossing_signs.pdf}
            \caption{Gauss Code for the Left Handed Trefoil}
            \label{fig:crossing_signs}
        \end{figure}
        By keeping track of the signs, in addition to whether or not we're on
        the under strand or over strand, we get the extended Gauss code. This
        is also called signed Gauss code.
        \begin{figure}[H]
            \centering
            \includegraphics{../images/trefoil_knot_oriented_with_extended_gauss_code.pdf}
            \caption{Extended Gauss Code for the Right Handed Trefoil}
            \label{fig:right_hand_trefoil_extended_gauss}
        \end{figure}
        Note the orientation does not matter. If we reverse the orientation,
        the signs are still preserved. Examine Fig.~\ref{fig:crossing_signs} to
        convince yourself of this.
        \par\hfill\par
        Computationally, Extended Gauss code is a finite sequence of ordered
        triples. The length of the sequence is $2n$ where $n$ is the number of
        crossings in the diagram, and the ordered triples are of the form
        $(t,k,s)$ where $t\in\{O,U\}$, $s\in\{+1,-1\}$, and
        $0\leq{k}\leq{n-1}$ ($t$ for \textit{type}, $s$ for \textit{sign}, and
        $k$ for indexing). One could represent this in the C programming
        language as follows.
        \begin{lstlisting}[style=CStyle, gobble=12]
            enum crossing_sign {negative_crossing, positive_crossing};

            enum crossing_type {under_crossing, over_crossing};

            struct knot {
                unsigned int number_of_crossings;
                enum crossing_sign *sign;
                enum crossing_type *type;
                unsigned int *crossing_number;
            };
        \end{lstlisting}
        Similar definitions could easily be given with classes using the
        Python programming language. Regardless of preference, we now have a
        means of representing a knot in a computer.
    \section{Resolving a Crossing}
        The Kauffman bracket polynomial is an invariant for
        \textit{framed links}. It is not a link invariant since the polynomial
        is not preserved by Reidemeister I moves (though this can be salvaged
        by a proper normalization, resulting in the Jones' polynomial). The
        definition we'll give in the next section is a mimicry of the
        description provided in \cite{barnatan2002khovanov}. It is recursive
        and requires us to \textit{resolve crossings}. Resolving a crossings
        equates to making it \textit{go away}. There are two ways to get rid of
        a crossing. Given an unoriented knot diagram, we rotate our heads until
        the over crossing travels from the top left to the bottom right. The
        0 and 1 resolution of the crossing are given in
        Fig.~\ref{fig:resolving_crossing}.
        \begin{figure}[H]
            \centering
            \includegraphics{../images/resolving_crossings.pdf}
            \caption{Resolving a Crossing}
            \label{fig:resolving_crossing}
        \end{figure}
    \newpage
    \bibliographystyle{annotate}
    \bibliography{../biblio.bib}
    \newpage
    The source code used to generate this document is free software and released
    under version 3 of the GNU General Public License.
\end{document}
