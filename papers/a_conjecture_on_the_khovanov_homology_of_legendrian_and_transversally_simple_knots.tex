%-----------------------------------LICENSE------------------------------------%
%   This file is part of Mathematics-and-Physics.                              %
%                                                                              %
%   Mathematics-and-Physics is free software: you can redistribute it and/or   %
%   modify it under the terms of the GNU General Public License as             %
%   published by the Free Software Foundation, either version 3 of the         %
%   License, or (at your option) any later version.                            %
%                                                                              %
%   Mathematics-and-Physics is distributed in the hope that it will be useful, %
%   but WITHOUT ANY WARRANTY; without even the implied warranty of             %
%   MERCHANTABILITY or FITNESS FOR A PARTICULAR PURPOSE.  See the              %
%   GNU General Public License for more details.                               %
%                                                                              %
%   You should have received a copy of the GNU General Public License along    %
%   with Mathematics-and-Physics.  If not, see <https://www.gnu.org/licenses/>.%
%------------------------------------------------------------------------------%
%       Author: Ryan Maguire                                                   %
%       Date:   2022/03/14                                                     %
%------------------------------------------------------------------------------%
\documentclass{article}

% Needed for mathbb font style.
\usepackage{amssymb}

% Tools for creating the theorem environment.
\usepackage{amsthm}

% Hyperlinks for labels.
\usepackage{hyperref}

% Display style for hyperlinks.
\hypersetup{
    colorlinks=true,
    linkcolor=blue
}

% Create a theorem environment, similar to most textbooks.
\theoremstyle{plain}
\newtheorem{theorem}{Theorem}

% No indent and no paragraph skip.
\setlength{\parindent}{0em}
\setlength{\parskip}{0em}

% Title page information.
\title{Conjectures on The Khovanov and Knot Floer Homologies of
       Legendrian and Transversally Simple Knots}
\author{Vladimir Chernov\hspace{2em}Ryan Maguire}
\date{March 2022}

\begin{document}
    \maketitle
    \tableofcontents
    \begin{abstract}
        \noindent
        A theorem of Kronheimer and Mrowka states that Khovanov homology is
        able to detect the unknot \cite{JoshuaMSabloffWhatIsLegendrianKnot}.
        That is, if a knot has the Khovanov homology of the unknot, then it is
        isomorphic to it. A similar result holds for Knot Floer
        homology. The unknot is the simplest of the torus knots, which is a
        class of knots known to be Legendrian simple. It is conjectured that
        Khovanov and Knot Floer homology are able to distinguish Legendrian
        and Transversally simple knots. Numerical evidence is provided for
        all knots up to and including 17 crossings. 
	\end{abstract}
	\section{Legendrian Knots}
	    A knot is a smooth embedding of the circle $\mathbb{S}^{1}$ into
	    $\mathbb{R}^{3}$. We may impose extra structure by considering the
	    standard \textit{contact structure} of $\mathbb{R}^{3}$ (an assignment
	    of a plane to every point in $\mathbb{R}^{3}$ such that there is no
	    surface $M\subset\mathbb{R}^{3}$ where for every $p\in{M}$ the tangent
	    plane $T_{p}M$ is given by the plane of the contact structure). A
	    \textit{Legendrian knot} is a knot that is everywhere tangent to the
	    contact structure. A gentle introduction to the subject can be found in
	    \cite{JoshuaMSabloffWhatIsLegendrianKnot}. Transverse knots are knots
	    that are everywhere transverse to the contact structure. Any Legendrian
	    knot can be made transverse by a small perturbation in the direction
	    normal to the given plane in the contact structure.
	    \par\hfill\par
	    Two Legendrian knots are considered to be Legendrian equivalent if
	    there is an isotopy
	    $H:\mathbb{S}^{1}\times[0,1]\rightarrow\mathbb{R}^{3}$ between them
	    such that for all $t\in[0,1]$ the knot $H_{t}$ is Legendrian. It is
	    possible for two knot embeddings to be topologically equivalent but not
	    Legendrian. In the other direction, any knot can be made Legendrian by
	    an appropriate isotopy (See the introduction of
	    \cite{VeraVertessiTransNonSimpleKnots}). The two classical invariants
	    of Legendrian knots are the Thurston-Bennequin number and rotation
	    numbers. A knot type is said to be \textit{Legendrian simple} if
	    any two Legendrian embeddings of it with the same Thurston-Bennequin
	    and rotation numbers are Legendrian equivalent. That is, the classical
	    invariants uniquely classify all Legendrian representations of the
	    knot.
	    \par\hfill\par
	    Certain knot types are known to be Legendrian simple such as the unknot
	    \cite{EliashbergFraserClassificationTopTrivialLegKnots}, torus knots,
	    and the figure eight knot \cite{EtnyreHondaContactTopologyI}. The torus
	    knots are of particular importance since it provides an infinite
	    family to test conjectures and computations with. On the other hand,
	    not every knot is Legendrian simple, the $m_{3}$ twist knot (also
	    known as the $5_{2}$ knot) being an example discovered by Chekanov
	    \cite{ChekanovDifAlgOfLegLinks}.
	\section{Khovanov Homology}
	    The Khovanov homology of a knot is a powerful, if computationally
	    expensive%
	    \footnote{The na\"{i}ve algorithm is exponential in time. Improvements
	    by Bar-Natan \cite{Khovanov1999CatJonesPoly} have sped up computations
	    but no polynomial-time algorithm is known at the time of this writing.},
	    invariant first described by Mikhail Khovanov
	    \cite{Khovanov1999CatJonesPoly} (See also \cite{barnatan2002khovanov}
	    for an excellent introduction). It is closely related to the Jones
	    polynomial of a knot but is able to distinguish many more knots.
    \newpage
    \bibliographystyle{plain}
    \bibliography{bib.bib}
    \newpage
\end{document}
