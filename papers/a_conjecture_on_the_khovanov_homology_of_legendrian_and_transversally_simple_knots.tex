%-----------------------------------LICENSE------------------------------------%
%   This file is part of Mathematics-and-Physics.                              %
%                                                                              %
%   Mathematics-and-Physics is free software: you can redistribute it and/or   %
%   modify it under the terms of the GNU General Public License as             %
%   published by the Free Software Foundation, either version 3 of the         %
%   License, or (at your option) any later version.                            %
%                                                                              %
%   Mathematics-and-Physics is distributed in the hope that it will be useful, %
%   but WITHOUT ANY WARRANTY; without even the implied warranty of             %
%   MERCHANTABILITY or FITNESS FOR A PARTICULAR PURPOSE.  See the              %
%   GNU General Public License for more details.                               %
%                                                                              %
%   You should have received a copy of the GNU General Public License along    %
%   with Mathematics-and-Physics.  If not, see <https://www.gnu.org/licenses/>.%
%------------------------------------------------------------------------------%
%       Author: Ryan Maguire                                                   %
%       Date:   2022/03/14                                                     %
%------------------------------------------------------------------------------%
\documentclass{article}

% Needed for mathbb font style.
\usepackage{amssymb}

% Tools for creating the theorem environment.
\usepackage{amsthm}
\usepackage{amsmath}

% Used for pictures.
\usepackage{graphics}

% Hyperlinks for labels.
\usepackage{hyperref}

% Display style for hyperlinks.
\hypersetup{
    colorlinks=true,
    linkcolor=blue
}

% Create a theorem environment, similar to most textbooks.
\theoremstyle{plain}
\newtheorem{theorem}{Theorem}
\newtheorem{conjecture}{Conjecture}

% No indent and no paragraph skip.
\setlength{\parindent}{0em}
\setlength{\parskip}{0em}

% Title page information.
\title{Conjectures on the Khovanov and Knot Floer Homologies of
       Legendrian and Transversally Simple Knots}
\author{Vladimir Chernov\hspace{2em}Ryan Maguire}
\date{April 2022}

\begin{document}
    \maketitle
    \tableofcontents
    \begin{abstract}
        \noindent
        A theorem of Kronheimer and Mrowka states that Khovanov homology is
        able to detect the unknot \cite{kronheimermrowka2010}.
        That is, if a knot has the Khovanov homology of the unknot, then it is
        equivalent to it. A similar result holds for Knot Floer
        homology. The unknot is the simplest of the torus knots, which is a
        class of knots known to be Legendrian simple. It is conjectured that
        Khovanov and Knot Floer homology are able to distinguish Legendrian
        and Transversally simple knots. Numerical evidence is provided for
        all knots up to and including 17 crossings. 
    \end{abstract}
    \section{Legendrian Knots and Links}
        A knot is a smooth embedding of the circle $\mathbb{S}^{1}$ into
        $\mathbb{R}^{3}$. A link is a smooth embedding of $N\in\mathbb{N}$
        disjoint circles in $\mathbb{R}^{3}$. We may impose
        extra structure by considering the standard \textit{contact structure}
        of $\mathbb{R}^{3}$. This is an assignment
        of a plane to every point in $\mathbb{R}^{3}$ such that there is no
        surface $M\subset\mathbb{R}^{3}$ where for every $p\in{M}$ the tangent
        plane $T_{p}M$ is given by the plane of the contact structure). In
        $\mathbb{R}^{3}$ this is described by the one-form
        $\textrm{d}z-y\,\textrm{d}x$, the plane at $(x,y,z)$ being spanned by
        the vectors $\partial_{x}+y\partial_{z}$ and $\partial_{y}$. The
        hyperplane distribution is shown in Fig.~\ref{fig:darboux_form_001}.
        \begin{figure}
            \centering
            \resizebox{\textwidth}{!}{%
                \includegraphics{../images/darboux_form_001.pdf}
            }
            \caption{Hyperplane Distribution for $\textrm{d}z-y\,\textrm{d}x$}
            \label{fig:darboux_form_001}
        \end{figure}
        While there are no everywhere tangent surfaces, it is possible for a
        curve to be everywhere tangent to this distribution of planes. Indeed,
        $\gamma(t)=(t,0,0)$ is such a curve.
        A \textit{Legendrian link} is a link that is everywhere tangent to
        the contact structure. A gentle introduction to the subject can be found
        in \cite{JoshuaMSabloffWhatIsLegendrianKnot}. Transverse links are links
        that are everywhere transverse to the contact structure. Any Legendrian
        link can be made transverse by a small perturbation in the direction
        normal to the given plane in the contact structure.
        \par\hfill\par
        The condition $\textrm{d}z-y\,\textrm{d}x$ tells us that for a
        Legendrian knot with parameterization $\gamma(t)=(x(t),y(t),z(y))$ we
        need, for all $t$, the following:
        \begin{equation}
            y(t)=\frac{\dot{z}(t)}{\dot{x}(t)}
        \end{equation}
        Since knots are closed loops, the $x$ component will be cyclic, meaning
        there are values $t_{0}$ such that $\dot{x}(t_{0})=0$. For the $y$
        value not to tend off to infinity it is then required that
        $\dot{z}(t_{0})=0$. The projection of $\gamma$ onto the $xz$ plane then
        has \textit{cusps} (See Fig.~\ref{fig:legendrian_unknot_cusps_001}).
        A parameterization of a Legendrian unknot can be obtained from this
        figure. The curve in the plane is $(\cos(t),\sin^{3}(t))$. Solving for
        $y(t)=\dot{z}(t)/\dot{x}(t)$ we get:
        \begin{equation}
            \gamma(t)=\Big(\cos(t),\,-3\cos(t)\sin(t),\,\sin^{3}(t)\Big)
        \end{equation}
        The validity of $\textrm{d}z-y\,\textrm{d}x$ can be computed directly.
        This embedding is shown in
        Fig.~\ref{fig:legendrian_unknot_001}.
        Fig~\ref{fig:legendrian_unknot_002} shows that it is indeed tangent to
        the hyperplane distribution.
        \begin{figure}
            \centering
            \resizebox{!}{0.4\textheight}{%
                \includegraphics{../images/legendrian_unknot_cusps_001.pdf}
            }
            \caption{Projection of a Legendrian Unknot}
            \label{fig:legendrian_unknot_cusps_001}
        \end{figure}
        \begin{figure}
            \centering
            \resizebox{!}{0.4\textheight}{%
                \includegraphics{../images/legendrian_unknot_001.pdf}
            }
            \caption{Legendrian Unknot in $\mathbb{R}^{3}$}
            \label{fig:legendrian_unknot_001}
        \end{figure}
        \begin{figure}
            \centering
            \resizebox{!}{0.4\textheight}{%
                \includegraphics{../images/legendrian_unknot_002.pdf}
            }
            \caption{Legendrian Unknot in $\mathbb{R}^{3}$ with Hyperplanes}
            \label{fig:legendrian_unknot_002}
        \end{figure}
        Two Legendrian links are considered to be Legendrian equivalent if
        there is an isotopy $H:L\times[0,1]\rightarrow\mathbb{R}^{3}$ between
        them, $L=\sqcup_{k=0}^{N-1}\mathbb{S}^{1}$, such that for all
        $t\in[0,1]$ the link $H_{t}$ is Legendrian. It is possible for two links
        to be topologically equivalent but not Legendrian. In the other
        direction, any link can be made Legendrian by an appropriate isotopy
        (See the introduction of \cite{VeraVertessiTransNonSimpleKnots}). The
        two classical invariants of Legendrian links are the Thurston-Bennequin
        and rotation numbers. A link type is said to be
        \textit{Legendrian simple} if any two Legendrian embeddings of it with
        the same Thurston-Bennequin and rotation numbers are Legendrian
        equivalent. That is, the classical invariants uniquely classify all
        Legendrian representations of the knot.
        \par\hfill\par
        Certain knot types are known to be Legendrian simple such as the unknot
        \cite{EliashbergFraserClassificationTopTrivialLegKnots}, torus knots,
        and the figure eight knot \cite{EtnyreHondaContactTopologyI}. The torus
        knots are of particular importance since it provides an infinite
        family to test conjectures and computations. On the other hand,
        not every knot is Legendrian simple, the $m_{3}$ twist knot (also
        known as the $5_{2}$ knot) being an example discovered by Chekanov
        \cite{ChekanovDifAlgOfLegLinks}.
    \section{Khovanov Homology}
        The Khovanov homology of a knot is a powerful, if computationally
        expensive%
        \footnote{The na\"{i}ve algorithm is exponential in the number of
        crossings. Improvements
        by Bar-Natan \cite{Khovanov1999CatJonesPoly} have sped up computations
        but no polynomial-time algorithm is known at the time of this writing.},
        invariant first described by Mikhail Khovanov
        \cite{Khovanov1999CatJonesPoly} (See also \cite{barnatan2002khovanov}
        for an excellent introduction). It is closely related to the Jones
        polynomial, but is able to distinguish many more knots. The homology
        groups $KH^{r}(L)$ of a link (or knot) $L$ are the direct sum of
        homogeneous components $KH_{\ell}^{r}(L)$ and the
        \textit{Khovanov Polynomial} (See \cite{KatlasKhoHo}) is given by:
        \begin{equation}
            Kh(L)(q,t)=
            \sum_{r,\ell}t^{r}q^{\ell}\textrm{dim}\big(KH_{\ell}^{r}(L)\big)
        \end{equation}
        The Jones polynomial of $L$ is recovered via:
        \begin{equation}
            J(L)(q)=Kh(q,-1)
        \end{equation}
        Khovanov homology is not a perfect invariant. That is, there are
        distinct knots with the same Khovanov homology, but it is a powerful
        invariant and is capable of detecting certain knot types.
        \begin{theorem}[Kronheimer and Mrowka]
            If a knot $K$ has the same Khovanov homology as the unknot, then $K$
            is equivalent to the unknot.
        \end{theorem}
        The unknotting problem asks one to determine if a given knot diagram is
        equivalent to the unknot. Khovanov homology is a powerful enough tool
        to accomplish this task. Since the Khovanov polynomial is a
        generalization of the Jones polynomial it has be conjectured that if a
        knot has the same Jones polynomial as the unknot, then that knot is
        equivalent to the unknot. At the time of this writing it has not been
        proven, but there is evidence for and against the claim. Morwen
        Thistlewaite found links with the same Jones polynomial as the unlink
        \cite{Thistlethwaite2001LINKSWT}, and there is a 3-crossing virtual
        knot that has the same Jones polynomial as the unknot. For the claim,
        all knots of up to 24 crossings are either the unknot, or have a
        Jones polynomial different from the unknot
        \cite{VerificationUnknotJonesConjUpTo24}.
    \section{Conjectures on Khovanov and Knot Floer Homology}
        Another homology theory that is deeply studied is Knot Floer Homology
        (abbreviated KFH). KFH is also capable of detecting the unknot. Just as
        Khovanov homology is the categorification of the Jones polynomial, so
        is Knot Floer Homology for the Alexander polynomial.
        The question we wish to ask in this paper is whether the unknot is
        a special case of a broader class of knots for which Khovanov homology
        (and Knot Floer homology) is capable of detecting.
        \begin{conjecture}
            If a link type $L$ is Legendrian simple, then the Khovanov and Knot
            Floer homologies of $L$ distinguish it. That is, if $\tilde{L}$
            is another link with the same Khovanov or Knot Floer homology,
            then $\tilde{L}$ is equivalent to $L$.
        \end{conjecture}
        Numerical evidence has been tallied for all Torus knots with up to 50
        crossings against all knots of up to 17 crossings. There are many
        torus knots that have the same Jones polynomial as a non-torus knot
        ($T(2,5)$ matches a 10 crossing knot and a 17 crossing knot, $T(2,7)$
        matches a 12 crossing knot, and $T(2,11)$ matches a 14 crossing knot)
        so we cannot generalize the Jones unknot conjecture. Nevertheless, in
        all cases the Khovanov homologies were different.
        \par\hfill\par
        The computation was done as follows. There are libraries for Python and
        Sage for working with knot polynomials. In particular, we used
        Regina \cite{regina}, SnapPy \cite{SnapPy}, the Sage knot library
        \cite{sage}, and our own ever-growing C library. The need for four
        different libraries was for the sake of sanity. One library alone is
        sufficient for the computation of the Jones polynomial but it never
        hurts to double check. The Jones polynomials of all torus knots up to
        50 crossings were computed using the formula:
        \begin{equation}
            J(T(m,n))(q)=q^{(m-1)(n-1)/2}
                \frac{1-q^{m+1}-q^{n+1}+q^{m+n}}{1-q^{2}}
        \end{equation}
        Using any of the aforementioned libraries, the Jones polynomial of all
        knots up to 17 crossings was computed and compared against the table of
        torus knot Jones polynomials we've computed. If a match was found,
        the regina Library was used to determine if the knots were actually
        identical. That is, if the knot whose Jones polynomial was being
        compared against the torus knots was indeed a torus knot itself. If the
        knots were distinct, this knot was saved in a text file for later
        examination. At the end of the computation 4 non-torus knots had the
        same Jones polynomial as a torus knot (the 4 mentioned previously).
        Since the Khovanov polynomial contains the Jones polynomial in it
        (recall $J(L)(q)=Kh(L)(q,-1)$) the only possible non-torus knots with
        the same Khovanov homology as a torus knot were these 4.
        \par\hfill\par
        Using the Java library JavaKh\footnote{%
            Thanks must be paid to Nikolay Pultsin who made edits to
            JavaKh-v2 so that it may run on a GNU/Linux machine using
            OpenJDK 17.
        }
        we found that these four knots with the same Jones polynomials as some
        torus knots all had different Khovanov homologies. Thus, we have the
        following claim:
        \begin{theorem}
            If a knot $K$ has less than or equal to 17 crossings and has the
            Khovanov homology of a torus knot $T$ with less than 50 crossings,
            then $K$ is equivalent to $T$.
        \end{theorem}
        A similar search for Knot Floer Homology has been performed for up to 15
        crossings. First, the Alexander polynomial was computed.
        In comparison to the Jones polynomial (which has no known
        \textit{classical} polynomial-time algorithm)\footnote{%
            A \textit{quantum} algorithm has been discovered
            \cite{JonesQuantumAlgorithm}.
        }, the na\"{i}ve algorithm for the Alexander polynomial is cubic in
        time (one need only compute the determinant of a particular matrix), but
        improvements have been made here as well. Needless to say, this greatly
        improves performance in the search for a match for Knot Floer homology.
        \begin{theorem}
            If a knot with less than or equal to 15 crossings has the same
            Knot Floer homology as a torus knot with less than or equal to 50
            crossings, then the knot is equivalent to the torus knot.
        \end{theorem}
        Lastly, a search through the twist knots yielded some more results.
        The Jones polynomials of the twist knots are known, with the formula:
        \begin{equation}
            J(m_{n})(q)=
            \begin{cases}
                (1+q^{-2}+q^{-n}+q^{-n-3})/(1+q),&n\textrm{ odd}\\
                (1+q-q^{3-n}+q^{-n})/(1+q),&n\textrm{ even}
            \end{cases}
        \end{equation}
        A search through all knots up to 17 crossings against all twist knots
        up to 40 crossings provided many matches for the Jones polynomial, but
        none for Khovanov homology.
    \section{Future Work}
        The process of looping through knots and comparing invariants is
        \textit{embarrassingly parallelizable} and yet no attempts at
        parallel processing has been attempted. We wish to check all crossings
        up to 20 crossings in the near future and multiprocessing may be the
        key to doing this in a sane amount of time. We also wish to check
        Knot Floer Homology up to 17 crossings as we did Khovanov homology.
        At present, only up to 15 crossings have been checked for the Knot
        Floer homology conjecture.
    \newpage
    \bibliographystyle{plain}
    \bibliography{bib.bib}
    \newpage
\end{document}
