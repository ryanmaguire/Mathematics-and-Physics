%-----------------------------------LICENSE------------------------------------%
%   This file is part of Mathematics-and-Physics.                              %
%                                                                              %
%   Mathematics-and-Physics is free software: you can redistribute it and/or   %
%   modify it under the terms of the GNU General Public License as             %
%   published by the Free Software Foundation, either version 3 of the         %
%   License, or (at your option) any later version.                            %
%                                                                              %
%   Mathematics-and-Physics is distributed in the hope that it will be useful, %
%   but WITHOUT ANY WARRANTY; without even the implied warranty of             %
%   MERCHANTABILITY or FITNESS FOR A PARTICULAR PURPOSE.  See the              %
%   GNU General Public License for more details.                               %
%                                                                              %
%   You should have received a copy of the GNU General Public License along    %
%   with Mathematics-and-Physics.  If not, see <https://www.gnu.org/licenses/>.%
%------------------------------------------------------------------------------%
%       Author: Ryan Maguire                                                   %
%       Date:   2022/03/14                                                     %
%------------------------------------------------------------------------------%
\documentclass{article}

% Needed for mathbb font style.
\usepackage{amssymb}

% Tools for creating the theorem environment.
\usepackage{amsthm}
\usepackage{amsmath}

% Hyperlinks for labels.
\usepackage{hyperref}

% Display style for hyperlinks.
\hypersetup{
    colorlinks=true,
    linkcolor=blue
}

% Create a theorem environment, similar to most textbooks.
\theoremstyle{plain}
\newtheorem{theorem}{Theorem}
\newtheorem{conjecture}{Conjecture}

% No indent and no paragraph skip.
\setlength{\parindent}{0em}
\setlength{\parskip}{0em}

% Title page information.
\title{Conjectures on The Khovanov and Knot Floer Homologies of
       Legendrian and Transversally Simple Knots}
\author{Vladimir Chernov\hspace{2em}Ryan Maguire}
\date{March 2022}

\begin{document}
    \maketitle
    \tableofcontents
    \begin{abstract}
        \noindent
        A theorem of Kronheimer and Mrowka states that Khovanov homology is
        able to detect the unknot \cite{kronheimermrowka2010}.
        That is, if a knot has the Khovanov homology of the unknot, then it is
        isomorphic to it. A similar result holds for Knot Floer
        homology. The unknot is the simplest of the torus knots, which is a
        class of knots known to be Legendrian simple. It is conjectured that
        Khovanov and Knot Floer homology are able to distinguish Legendrian
        and Transversally simple knots. Numerical evidence is provided for
        all knots up to and including 17 crossings. 
	\end{abstract}
	\section{Legendrian Knots}
	    A knot is a smooth embedding of the circle $\mathbb{S}^{1}$ into
	    $\mathbb{R}^{3}$. We may impose extra structure by considering the
	    standard \textit{contact structure} of $\mathbb{R}^{3}$ (an assignment
	    of a plane to every point in $\mathbb{R}^{3}$ such that there is no
	    surface $M\subset\mathbb{R}^{3}$ where for every $p\in{M}$ the tangent
	    plane $T_{p}M$ is given by the plane of the contact structure). A
	    \textit{Legendrian knot} is a knot that is everywhere tangent to the
	    contact structure. A gentle introduction to the subject can be found in
	    \cite{JoshuaMSabloffWhatIsLegendrianKnot}. Transverse knots are knots
	    that are everywhere transverse to the contact structure. Any Legendrian
	    knot can be made transverse by a small perturbation in the direction
	    normal to the given plane in the contact structure.
	    \par\hfill\par
	    Two Legendrian knots are considered to be Legendrian equivalent if
	    there is an isotopy
	    $H:\mathbb{S}^{1}\times[0,1]\rightarrow\mathbb{R}^{3}$ between them
	    such that for all $t\in[0,1]$ the knot $H_{t}$ is Legendrian. It is
	    possible for two knot embeddings to be topologically equivalent but not
	    Legendrian. In the other direction, any knot can be made Legendrian by
	    an appropriate isotopy (See the introduction of
	    \cite{VeraVertessiTransNonSimpleKnots}). The two classical invariants
	    of Legendrian knots are the Thurston-Bennequin number and rotation
	    numbers. A knot type is said to be \textit{Legendrian simple} if
	    any two Legendrian embeddings of it with the same Thurston-Bennequin
	    and rotation numbers are Legendrian equivalent. That is, the classical
	    invariants uniquely classify all Legendrian representations of the
	    knot.
	    \par\hfill\par
	    Certain knot types are known to be Legendrian simple such as the unknot
	    \cite{EliashbergFraserClassificationTopTrivialLegKnots}, torus knots,
	    and the figure eight knot \cite{EtnyreHondaContactTopologyI}. The torus
	    knots are of particular importance since it provides an infinite
	    family to test conjectures and computations. On the other hand,
	    not every knot is Legendrian simple, the $m_{3}$ twist knot (also
	    known as the $5_{2}$ knot) being an example discovered by Chekanov
	    \cite{ChekanovDifAlgOfLegLinks}.
	\section{Khovanov Homology}
	    The Khovanov homology of a knot is a powerful, if computationally
	    expensive%
	    \footnote{The na\"{i}ve algorithm is exponential in the number of
	    crossings. Improvements
	    by Bar-Natan \cite{Khovanov1999CatJonesPoly} have sped up computations
	    but no polynomial-time algorithm is known at the time of this writing.},
	    invariant first described by Mikhail Khovanov
	    \cite{Khovanov1999CatJonesPoly} (See also \cite{barnatan2002khovanov}
	    for an excellent introduction). It is closely related to the Jones
	    polynomial, but is able to distinguish many more knots. The homology
	    groups $KH^{r}(L)$ of a link (or knot) $L$ are the direct sum of
	    homogeneous components $KH_{\ell}^{r}(L)$ and the
	    \textit{Khovanov Polynomial} (See \cite{KatlasKhoHo}) is given by:
	    \begin{equation}
	        Kh(L)(q,t)=
	        \sum_{r,\ell}t^{r}q^{\ell}\textrm{dim}\big(KH_{\ell}^{r}(L)\big)
	    \end{equation}
	    The Jones polynomial of $L$ is recovered via:
	    \begin{equation}
	        J(L)(q)=Kh(q,-1)
	    \end{equation}
	    Khovanov homology is not a perfect invariant. That is, there are
	    distinct knots with the same Khovanov homology, but it is a powerful
	    invariant and is capable of detecting certain knot types.
	    \begin{theorem}[Kronheimer and Mrowka]
            If a knot has the same Khovanov homology as the unknot, then the
            knot is equivalent to the unknot.
	    \end{theorem}
	    The unknotting problem asks one to determine if a given knot diagram is
	    equivalent to the unknot. Khovanov homology is a powerful enough tool
	    to accomplish this task. Since the Khovanov polynomial is a
	    generalization of the Jones polynomial, the following has been
	    conjectured:
	    \begin{conjecture}
	        If a knot has the same Jones polynomial as the unknot, is the knot
	        equivalent to the unknot?
	    \end{conjecture}
	    This is a restatement of Kronheimer and Mrowka's result, but for the
	    Jones polynomial. At the time of this writing it has not been proven,
	    but there is evidence for and against the claim. Morwen Thistlewaite
	    has found links with the same Jones polynomial as the unlink
	    \cite{Thistlethwaite2001LINKSWT}, and there is a 3-crossing virtual
	    knot that has the same Jones polynomial as the unknot. For the claim,
	    all knots of up to 24 crossings are either the unknot, or have a
	    Jones polynomial different from the unknot
	    \cite{VerificationUnknotJonesConjUpTo24}.
	\section{Conjectures on Khovanov and Knot Floer Homology}
	    Another homology theory that is deeply study is Knot Floer Homology
	    (abbreviated KFH). KFH is also capable of detecting the unknot. Just as
	    Khovanov homology is the categorification of the Jones polynomial, so
	    is Knot Floer Homology for the Alexander polynomial.
	    The question we wish to ask in this paper is whether the unknot is
	    a special case of a broader class of knots for which Khovanov homology
	    (and Knot Floer homology) is capable of detecting. There are two
	    natural routes to take. The unknot is the simplest of the torus knots,
	    and it is the simplest of the twist knots. Torus knots being an
	    infinite family of Legendrian simple knots, we make the following
	    conjecture:
	    \begin{conjecture}
	        If a knot type is Legendrian simple, does Khovanov homology and
	        Knot Floer homology distinguish it from all other knot types?
	    \end{conjecture}
	    Numerical evidence has been tallied for all Torus knots with up to 50
	    crossings against all knots of up to 17 crossings. There are many
	    torus knots that have the same Jones polynomial as a non-torus knot
	    ($T(2,5)$ matches a 10 crossing knot and a 17 crossing knot, $T(2,7)$
	    matches a 12 crossing knot, and $T(2,11)$ matches a 14 crossing knot)
	    so we cannot generalize the Jones unknot conjecture. Nevertheless, in
	    all cases the Khovanov homologies were different.
	    \par\hfill\par
	    The computation was done as follows. There are libraries for Python and
	    Sage for working with knot polynomials. In particular, we used
	    Regina \cite{regina}, SnapPy \cite{SnapPy}, the Sage knot library
	    \cite{sage}, and our own ever-growing C library. The need for four
	    different libraries was for the sake of sanity. One library alone is
	    sufficient for the computation of the Jones polynomial but it never
	    hurts to double check. The Jones polynomials of all torus knots up to
	    50 crossings were computed using the formula:
	    \begin{equation}
	        J(T(m,n))(q)=q^{(m-1)(n-1)/2}
	            \frac{1-q^{m+1}-q^{n+1}+q^{m+n}}{1-q^{2}}
	    \end{equation}
	    Using any of the aforementioned libraries, the Jones polynomial of all
	    knots up to 17 crossings was computed and compared against the table of
	    torus knot Jones polynomials we've computed. If a match was found,
	    the regina Library was used to determine if the knots were actually
	    identical. That is, if the knot whose Jones polynomial was being
	    compared against the torus knots was indeed a torus knot itself. If the
	    knots were distinct, this knot was saved in a text file for later
	    examination. At the end of the computation 4 non-torus knots had the
	    same Jones polynomial as a torus knot (the 4 mentioned previously).
	    Since the Khovanov polynomial contains the Jones polynomial in it
	    (recall $J(L)(q)=Kh(L)(q,-1)$) the only possible non-torus knots with
	    the same Khovanov homology as a torus knot were these 4.
	    \par\hfill\par
        Using the Java library JavaKh\footnote{%
            Thanks must be paid to Nikolay Pultsin who made edits to
            JavaKh-v2 so that it may run on a GNU/Linux machine using
            OpenJDK 17.
        }
        we found that these four knots with the same Jones polynomials as some
        torus knots all had different Khovanov homologies. Thus, we have the
        following claim:
        \begin{theorem}
            If a knot has less than or equal to 17 crossings and has the
            Khovanov homology of a torus knot with less than 50 crossings, then
            the knot is equivalent to the torus knot.
        \end{theorem}
        A similar search for Knot Floer Homology has been performed for up to 15
        crossings. First, the Alexander polynomial was computed.
        In comparison to the Jones polynomial (which has no known
        \textit{classical} polynomial-time algorithm)\footnote{%
            A \textit{quantum} algorithm has been discovered
            \cite{JonesQuantumAlgorithm}.
        }, the na\"{i}ve algorithm for the Alexander polynomial is cubic in
        time (one need only compute the determinant of a particular matrix), but
        improvements have been made here as well. Needless to say, this greatly
        improves performance in the search for a match for Knot Floer homology.
        \begin{theorem}
            If a knot with less than or equal to 15 crossings has the same
            Knot Floer homology as a torus knot with less than or equal to 50
            crossings, then the knot is equivalent to the torus knot.
        \end{theorem}
        Lastly, a search through the twist knots yielded some more results.
        The Jones polynomials of the twist knots are known, with the formula:
        \begin{equation}
            J(m_{n})(q)=
            \begin{cases}
                (1+q^{-2}+q^{-n}+q^{-n-3})/(1+q),&n\textrm{ odd}\\
                (1+q-q^{3-n}+q^{-n})/(1+q),&n\textrm{ even}
            \end{cases}
        \end{equation}
        A search through all knots up to 17 crossings against all twist knots
        up to 40 crossings provided many matches for the Jones polynomial, but
        none for Khovanov homology.
    \section{Future Work}
        The process of looping through knots and comparing invariants is
        \textit{embarrassingly parallelizable} and yet no attempts at
        parallel processing has been attempted. We wish to check all crossings
        up to 20 crossings in the near future and multiprocessing may be the
        key to doing this in a sane amount of time. We also wish to check
        Knot Floer Homology up to 17 crossings as we did Khovanov homology.
        At present, only up to 15 crossings have been checked for the Knot
        Floer homology conjecture.
    \newpage
    \bibliographystyle{plain}
    \bibliography{bib.bib}
    \newpage
\end{document}
