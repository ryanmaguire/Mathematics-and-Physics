%-----------------------------------LICENSE------------------------------------%
%   This file is part of Mathematics-and-Physics.                              %
%                                                                              %
%   Mathematics-and-Physics is free software: you can redistribute it and/or   %
%   modify it under the terms of the GNU General Public License as             %
%   published by the Free Software Foundation, either version 3 of the         %
%   License, or (at your option) any later version.                            %
%                                                                              %
%   Mathematics-and-Physics is distributed in the hope that it will be useful, %
%   but WITHOUT ANY WARRANTY; without even the implied warranty of             %
%   MERCHANTABILITY or FITNESS FOR A PARTICULAR PURPOSE.  See the              %
%   GNU General Public License for more details.                               %
%                                                                              %
%   You should have received a copy of the GNU General Public License along    %
%   with Mathematics-and-Physics.  If not, see <https://www.gnu.org/licenses/>.%
%------------------------------------------------------------------------------%
%       Author: Ryan Maguire                                                   %
%       Date:   2021/11/16                                                     %
%------------------------------------------------------------------------------%
\documentclass{article}

% Needed for figures.
\usepackage{graphicx}

% Needed for the iint symbol.
\usepackage{amsmath}

% Location of all of the images for the paper.
\graphicspath{{../images/}}

% No indent and no paragraph skip.
\setlength{\parindent}{0em}
\setlength{\parskip}{0em}

% Title page information.
\title{Rayleigh-Sommerfeld Diffraction Modeling}
\author{Ryan Maguire}
\date{\today}

\begin{document}
    \maketitle
    We may model the diffraction pattern seen by a radio wave passing through a
    thin screen using the Rayleigh-Sommerfeld equation, which is obtained
    from the wave equation. Given a screen $S$ in the $xy$ plane and an observer
    $\mathbf{R}$, if we let $\mathbf{r}_{0}$ denote the point where the line of
    sight from the observer to the source of the incoming wave intersects the screen,
    then we may model the diffraction pattern $\hat{T}$ by:
    \begin{equation}
        \label{eqn:forward_model}
        \hat{T}(\mathbf{r}_{0})
        =
        \frac{\mu_{0}}{i\lambda}
        \iint_{S}
            T(\mathbf{r})
            \frac{%
                \exp\Big(
                    ik\big(
                        \hat{\mathbf{u}}\cdot(\mathbf{r}-\mathbf{R})+\|\mathbf{r}-\mathbf{R}\|
                    \big)
                \Big)
            }{%
                \|\mathbf{R}-\mathbf{r}\|
            }\,\textrm{d}\mathbf{r},
    \end{equation}
    where $\mathbf{r}$ is a dummy variable of integration,
    $\mu_{0}$ is the \textit{ubiquty factor}, and
    $k$ and $\lambda$ are the wavenumber and wavelength, respectively. The $\hat{\mathbf{u}}$
    vector is the normalized relative position vector of $\mathbf{R}$ with respect to
    $\mathbf{r}_{0}$. That is:
    \begin{equation}
        \hat{\mathbf{u}}
        =
        \frac{\mathbf{R}-\mathbf{r}_{0}}{\|\mathbf{R}-\mathbf{r}_{0}\|}
    \end{equation}
    The function $T$ is the \textit{complex transmittance}, which measures how much of
    the wave is allowed to pass through a given point in the screen.
    From this description we can see that $\mathbf{r}$ is the only quantity that varies
    across the integral; $\mathbf{R}$, $\mathbf{r}_{0}$, and $\hat{\mathbf{u}}$ are all held fixed.
    \par\hfill\par
    We may rewrite Eqn.~\ref{eqn:forward_model} in terms of the \textit{Fresnel kernel}, $\psi$,
    which is defined by
    \begin{equation}
        \psi(\mathbf{r},\,\mathbf{r}_{0},\,\mathbf{R})
        =
        k\big(
            \hat{\mathbf{u}}\cdot(\mathbf{r}-\mathbf{R})+\|\mathbf{r}-\mathbf{R}\|
        \big)
    \end{equation}
    Hence we may rewrite the forward model (Eqn.~\ref{eqn:forward_model}) as
    \begin{equation}
        \label{eqn:forward_model_with_psi}
        \hat{T}(\mathbf{r}_{0})
        =
        \frac{\mu_{0}}{i\lambda}
        \iint_{S}
            T(\mathbf{r})
            \frac{%
                \exp\left(
                    i\psi(\mathbf{r},\,\mathbf{r}_{0},\,\mathbf{R})
                \right)
            }{%
                \|\mathbf{R}-\mathbf{r}\|
            }\,\textrm{d}\mathbf{r}.
    \end{equation}
    If we assume a circularly symmetric transmittance function, $T$, then
    by applying the \textit{stationary phase approximation} we may reduce this
    double integral down to a single one without compromising the accuracy of the model.
    Working in polar coordinates, $\mathbf{r}=(r,\,\phi)$,
    $\mathbf{r}_{0}=(r_{0},\,\phi_{0})$, we have
    \begin{equation}
        \label{eqn:single_forward_model}
        \hat{T}(r_{0})
        \approx
        \frac{1-i}{2F}
        \int_{0}^{\infty}
            T(r)
            \exp\left(
                i\psi(\mathbf{r}_{s},\,\mathbf{r},\,\mathbf{R})
            \right)\,\textrm{d}r,
    \end{equation}
    where $\mathbf{r}_{s}=(r,\,\phi_{s})$, with $\phi_{s}$ being the
    \textit{stationary azimuth angle}, which is the solution to
    $\partial\psi/\partial\phi=0$.
    The geometry for this is given in Fig.~\ref{fig:fig}, where we let
    $D=\|\mathbf{R}-\mathbf{r}_{0}\|$, the distance from the observer to the
    point in the screen where the line of sight intersects.
    \begin{figure}
        \centering
        \resizebox{\textwidth}{!}{%
            \includegraphics{fresnel_diffraction_geometry_saturn_001}%
        }
        \caption{The Geometry of Diffraction}
        \label{fig:fig}
    \end{figure}
    If we apply the classic Fresnel approximation to
    $\psi(\mathbf{r}_{s},\,\mathbf{r}_{0},\,\mathbf{R})$ the we obtain the
    \textit{Fresnel transform} of $T$. Quite fortunately, this has an explicit
    inverse formula. We mimic this explicit inverse formula for the Fresnel
    approximation to produce an approximate inverse for
    Eqn.~\ref{eqn:single_forward_model}:
    \begin{equation}
        \label{eqn:single_inverse_model}
        T(r)
        \approx
        \frac{1+i}{2F}
        \int_{0}^{\infty}
            \hat{T}(r_{s})
            \exp\left(
                -i\psi(\mathbf{r}_{s},\,\mathbf{r},\,\mathbf{R})
            \right)\,\textrm{d}r_{0},
    \end{equation}
    For the inverse model we have $\mathbf{r}_{0}$ varying while
    $\mathbf{r}$ is fixed. In practice, the observer $\mathbf{R}$ moves as
    well. Since the point $\mathbf{r}_{0}$ is entirely dependent on the
    position vector $\mathbf{R}$ and the direction of the incoming wave,
    in the inverse model we must allow $\mathbf{R}$ to vary with $\mathbf{r}_{0}$
    as well.

\end{document}

