\section{Fresnel Inversion}
    \subsection{Diffraction Through a Square Well}
        \label{Subsec:Cassini_Math_Diffraction_Through_a_Square_well}
        We wish to solve for $\hat{T}(\rho_{0})$. We have that:
        \begin{equation}
            \hat{T}(\rho_{0})
            =\frac{1-i}{2F}
                \int_{-\infty}^{\infty}T(\rho)
                \exp\Big(i\frac{\pi}{2}
                    \big(\frac{\rho-\rho_{0}}{F}\big)^{2}\Big)
                \diff{\rho}
        \end{equation}
        For the square well of height $M$
        starting at $a$ and ending at $b$:
        \begin{equation}
            T(\rho)=
            \begin{cases}
                0,&\rho<a\\
                M,&a\leq\rho\leq{b}\\
                0,&b<\rho
            \end{cases}
        \end{equation}
        Let $H_{M}(\rho_0,F;a,b)$ denote the solution.
        So, we have:
        \begin{subequations}
            \begin{align}
                H_{M}(\rho_0,F;a,b)
                &=\frac{1-i}{2F}\int_{a}^{b}M
                    \exp\Big(
                        \frac{i\pi}{2}
                        \big(\frac{\rho-\rho_{0}}{F}\big)^{2}
                    \Big)\diff{\rho}\\
                &=\frac{M}{F}\frac{1-i}{2}
                    \int_{a}^{b}
                    \exp\Big(
                        \frac{i\pi}{2}
                        \big(\frac{\rho-\rho_{0}}{F}\big)^{2}
                    \Big)\diff{\rho}
            \end{align}
        \end{subequations}
        Now from Euler's formula,
        $\exp(i\theta)=\cos(\theta)+i\sin(\theta).$
        Using this we obtain:
        \begin{equation}
            \exp\Big(
                \frac{i\pi}{2}
                \big(\frac{\rho-\rho_{0}}{F}\big)^{2}
            \Big)
            =\cos\Big(
                \frac{\pi}{2}
                \big(\frac{\rho-\rho_{0}}{F}\big)^{2}
            \Big)+
            i\sin\Big(
                \frac{\pi}{2}
                \big(\frac{\rho-\rho_{0}}{F}\big)^{2}
            \Big)
        \end{equation}
        Recall the Fresnel Cosine Integral and
        Sine Integrals $\big(C(x),S(x)\big)$ are defined as:
        \par
        \begin{subequations}
            \begin{minipage}[b]{0.49\textwidth}
                \begin{equation}
                    S(x)=\int_{0}^{x}\sin(t^{2})\diff{t}
                \end{equation}
            \end{minipage}
            \hfill
            \begin{minipage}[b]{0.49\textwidth}
                \begin{equation}
                    C(x)=\int_{0}^{x}\cos(t^{2})\diff{t}
                \end{equation}
            \end{minipage}
        \end{subequations}
        Letting $u=\sqrt{\frac{\pi}{2}}x$, we obtain:
        \begin{subequations}
            \begin{align}
                \int_{a}^{b}\sin\Big(\frac{\pi}{2}x^{2}\Big)\diff{x}
                &=\int_{0}^{b}\sin\Big(\frac{\pi}{2}x^{2}\Big)\diff{x}-
                    \int_{0}^{a}\sin\Big(\frac{\pi}{2}x^{2}\Big)\diff{x}\\
                &=\sqrt{\frac{2}{\pi}}\Bigg[
                    \int_{0}^{\sqrt{\frac{\pi}{2}}b}\sin(u^{2})\diff{u}-
                    \int_{0}^{\sqrt{\frac{\pi}{2}}a}\sin(u^{2})\diff{u}
                \Bigg]\\
                &=\sqrt{\frac{2}{\pi}}\Big[
                    S\Big(\sqrt{\frac{\pi}{2}}b\Big)-
                    S\Big(\sqrt{\frac{\pi}{2}}a\Big)
                \Big]\\
                    \int_{a}^{b}\cos\Big(\frac{\pi}{2}x^{2}\Big)\diff{x}
                &=\int_{0}^{b}\cos\Big(\frac{\pi}{2}x^{2}\Big)\diff{x}-
                    \int_{0}^{a}\cos\Big(\frac{\pi}{2}x^{2}\Big)\diff{x}\\
                &=C(b)-C(a)
            \end{align}
        \end{subequations}
        We can now compute
        $H_{M}(\rho_0,F;a,b)$.
        Let $x=\frac{\rho-\rho_0}{F}$. Then $dx = \frac{d\rho}{F}$.
        We have:
        \begin{align*}
            H_{M}(\rho_0,F;a,b)
            &=\frac{M}{F}\frac{1-i}{2}
                \int_{a}^{b}
                \exp\Big(i\frac{\pi}{2}
                \big(\frac{\rho-\rho_0}{F}\big)^{2}\Big)
            \diff{\rho}\\
            &=\frac{M}{F}\frac{1-i}{2}
                \bigg[\int_{a}^{b}
                \cos\Big(\frac{\pi}{2}
                \big(\frac{\rho-\rho_0}{F}\big)^{2}\Big)
            \diff{\rho}+
            i\int_{a}^{b}
                \sin\Big(\frac{\pi}{2}
                \big(\frac{\rho-\rho_0}{F}\big)^{2}\Big)
                \diff{\rho}\bigg]\\
            &=M\frac{1-i}{2}\bigg[
                \int_{\frac{a-\rho_0}{F}}^{\frac{b-\rho_0}{F}}
                \cos\Big(\frac{\pi}{2}x^{2}\Big)\diff{x}+
            i\int_{\frac{a-\rho_0}{F}}^{\frac{b-\rho_0}{F}}
                \sin\Big(\frac{\pi}{2}x^{2}\Big)\diff{x}
            \bigg]\\
            &=M\frac{1-i}{2}
                \bigg[\bigg(C\Big(\frac{b-\rho_{0}}{F}\Big)-
                    C\Big(\frac{a-\rho_0}{F}\Big)\bigg)
                +i\bigg(S\Big(\frac{b-\rho_0}{F}\Big)-
                    S\Big(\frac{a-\rho_0}{F}\Big)\bigg)\bigg]
        \end{align*}
    \subsection{Diffraction Through an Inverted Square Well}
        This time we have:
        \begin{equation*}
            T(\rho)=
            \begin{cases}
                M,&\rho<a\\
                0,&a\leq\rho\leq{b}\\
                M,&b<\rho
            \end{cases}
        \end{equation*}
        So $T(\rho) = M - T_{Sq}(\rho)$,
        where $T_{Sq}(\rho)$ is the impulse from the standard
        square well (See
        \ref{Subsec:Cassini_Math_Diffraction_Through_a_Square_well}).
        So we wish to solve:
        \begin{equation*}
            \hat{T}(\rho_{0})=
            \frac{1-i}{2F}\int_{-\infty}^{\infty}
            \big(M-T_{Sq}(\rho)\big)
            \exp(i\frac{\pi}{2}\Big(\frac{\rho-\rho_{0}}{F}\Big)^{2})
            \diff{\rho}
        \end{equation*}
            Simplifying, we have:
            \begin{align*}
                \hat{T}(\rho_{0})
                &=\frac{1-i}{2F}\int_{-\infty}^{\infty}
                \big(M-T(\rho)\big)
                \exp\Big(i\frac{\pi}{2}
                    \big(\frac{\rho-\rho_0}{F}\big)^{2}\Big)
                \diff{\rho}\\
                &=\frac{M}{F}\frac{1-i}{2}
                \bigg(\int_{-\infty}^{\infty}
                \exp\Big(i\frac{\pi}{2}
                    \big(\frac{\rho-\rho_0}{F}\big)^2\Big)
                \diff{\rho}-
                \int_{a}^{b}
                \exp\Big(i\frac{\pi}{2}
                    \big(\frac{\rho-\rho_0}{F}\big)^2\Big)
                \diff{\rho}\Bigg)\\
            \end{align*}
            But from the previous derivation:
            \begin{equation*}
                \int_{a}^{b}
                \exp\Big(i\frac{\pi}{2}
                    \big(\frac{\rho-\rho_0}{F}\big)^{2}\Big)
                \diff{\rho}
                =F\bigg(C\Big(\frac{b-\rho_0}{F}\Big)-
                C\Big(\frac{a-\rho_0}{F}\Big)\bigg)+
                i\bigg(S\Big(\frac{b-\rho_0}{F}\Big)-
                S\Big(\frac{a-\rho_0}{F}\Big)\bigg)
            \end{equation*}
            And:
            \begin{equation*}
                \int_{-\infty}^{\infty}
                \exp\Big(i\frac{\pi}{2}
                    \big(\frac{\rho-\rho_0}{F}\big)^{2}\Big)
                \diff{\rho}
                =\underset{a\rightarrow \infty}{\lim}
                \int_{-a}^{a}
                \exp\Big(i\frac{\pi}{2}
                    \big(\frac{\rho-\rho_0}{F}\big)^{2}\Big)
                \diff{\rho}
                =F(1+i)
            \end{equation*}
            Piecing this together, and using the notation from before,
            we have:
            \begin{equation*}
            \hat{T}(\rho_0)=M-H_{M}(\rho_0,F;a,b)
            \end{equation*}
