\section{Fourier Analysis}
    Fourier analysis is deeply tied to the theory of integration, and as such
    it is worth while to develop a few of the basic elements of this field.
    Many of theorems we wish to use, such as the \textit{convolution theorem},
    \textit{Fubini's theorem}, and the \textit{Fourier inversion theorem},
    are often presented with hand-wavy ``proofs,'' that obscure some of the
    problems that arise when applying these results to real-world problems. We
    will not dive into the entirety of measure theory, but rather present the
    elementary definitions, provide examples, and move on to the
    more important theorems.
    \subsection{Basic Notions}
        We wish to define what it means for some function
        $f:\Omega\rightarrow\mathbb{R}$ to be \textit{integrable}, where
        $\Omega$ is some space. Any attempt at defining an integral will
        require one to start with approximations such as the following:
        \begin{equation}
            \int_{\Omega}f(x)\diff{x}\approx\sum_{n}f(x_{n})\mu(X_{n})
        \end{equation}
        Where $X_{n}$ is a bunch of sets that partition $\Omega$,
        $x_{n}\in{X}_{n}$ for all $n$, and $\mu(X_{n})$ is the \textit{size} or
        the \textit{width} of $X_{n}$. This is precisely what is done in a
        calculus course where the Riemann integral is defined. To make this
        concrete, we'll need to decide what sets $X_{n}$ are allowed to
        partition $\Omega$, and what are the properties of our
        \textit{measure} $\mu$. Throughout, $\emptyset$ is used to denote
        the empty set. This is the set with nothing in it. Our inclusion of
        this set in various definitions is for technical reasons that we won't
        often be concerned with.
        \begin{ldefinition}{$\sigma\textrm{-Algebras}$}{Sigma_Algebra}
            A $\sigma$-Algebra on a set $\Omega$ is a collection of subsets
            $\mathcal{A}$ of $\Omega$ such that:
            \begin{enumerate}
                \item It is true that $\emptyset\in\mathcal{A}$ and that
                      $\Omega\in\mathcal{A}$.
                \item For all $A\in\mathcal{A}$, it is true that the complement
                      of $A$ is in $\mathcal{A}$. That is,
                      $A^{C}\in\mathcal{A}$.
                \item For any sequence $A:\mathbb{N}\rightarrow\mathcal{A}$ of
                      sets in $\mathcal{A}$, so is their intersection:
                      \begin{equation}
                          \bigcap_{n=1}^{\infty}A_{n}\in\mathcal{A}
                      \end{equation}
            \end{enumerate}
            The elements of $\mathcal{A}$ are called the
            \textit{measurable subsets} of $\Omega$. Given a set $\Omega$, and
            a $\sigma\textrm{-Algebra}$ $\mathcal{A}$ on $\Omega$, we call
            the pair $(\Omega,\,\mathcal{A})$ a \textit{measure space}.
        \end{ldefinition}
        \begin{lexample}{}{Power_Set_and_Trivial_Sigma_Algebras}
            Given a set $\Omega$, there are two simple
            $\sigma\textrm{-Algebras}$ that one can define. Let
            $\mathcal{A}=\{\emptyset,\,\Omega\}$. This satisfies all three
            properties and is called the trivial $\sigma\textrm{-Algebra}$.
            Going in the other direction, if we let $\mathcal{A}$ be the set of
            \textit{all} subsets of $\Omega$ (Also known as the
            \textit{power set} of $\Omega$, denoted $\mathcal{P}(\Omega)$),
            then this also a $\sigma\textrm{-Algebra}$.
        \end{lexample}
        The motivation for defining measurable sets in such a way is to allow
        one to easily describe \textit{measures} and \textit{probabilities}
        later.
        \begin{ldefinition}{Borel $\sigma\textrm{-Algebra}$}{Borel_Sig_Alg}
            The Borel $\sigma\textrm{-Algebra}$ is the \textit{smallest}
            $\sigma\textrm{-Algebra}$ on $\mathbb{R}$, denoted $\mathcal{B}$,
            such that for all $a<b$, the interval $(a,b)$ is a measurable
            set. That is, $(a,b)\in\mathcal{B}$.
        \end{ldefinition}
        \begin{ldefinition}{Measures}{Measures}
            A measure on a measure space $(\Omega,\,\mathcal{A})$ is a
            function $\mu:\mathcal{A}\rightarrow\mathbb{R}$ such that:
            \begin{enumerate}
                \item For all $A\in\mathcal{A}$, $\mu(A)\geq{0}$.
                \item $\mu(\emptyset)=0$.
                \item Given a \textit{mutually disjoint} list of sets
                      $A_{1},\,A_{2},\,\dots$ that are contained in
                      $\mathcal{A}$, the following is true:
                      \begin{equation}
                          \mu\Big(\bigcup_{n=1}^{\infty}A_{n}\Big)=
                          \sum_{n=1}^{\infty}\mu(A_{n})
                      \end{equation}
            \end{enumerate}
            The triple $(\Omega,\,\mathcal{A},\,\mu)$ is called a
            \textit{measurable space}.
        \end{ldefinition}
        It is important to remember that we are trying to model \textit{size},
        and this is what motivates our definition of measure. The first rule
        says that the width, or length, or size of a set is non-negative,
        and the second rule states that the size of \textit{nothing} is simply
        zero. The last rule, which is called countable additivity, is
        very important but also intuitive. We may define the length of
        the set interval $(0,\,1)$ as $1$, and the length of
        $(a,\,b)$ to be $b-a$ (For $a<b$). What about the length of the
        \textit{union} of the intervals $(0,\,1)$ and $(3,\,4)$? Since they
        have no overlap, we may as well add the lengths of the two
        individual intervals and claim that the length of the whole is 2.
        This is precisely what countable additivity tells us.
    \subsection{Fourier Series}
    \subsection{The Fourier Transform}
        Suppose you are asked to compute $z=x/y$ for
        two non-zero real numbers $x$ and $y$. We could perform
        long-hand division, or transform the problem into
        subtraction by using the natural logarithm.
        \begin{equation}
            z=x/y\Rightarrow
            \ln(z)=\ln(x/y)\Rightarrow
            \ln(z)=\ln(x)-\ln(y)
        \end{equation}
        Provided that $\ln(x)$ and $\ln(y)$ are somehow known,
        one can compute the difference and then compute $z$
        by exponentiating the result. In a similar manner, the
        Fourier transform is often introduced as a tool for
        converting one problem into another.
        \begin{ldefinition}{Fourier Transform}{Fourier_Transform}
            The Fourier transform of a complex valued integrable function
            $f:\mathbb{R}\rightarrow\mathbb{C}$ is the function
            $\mathcal{F}_{\xi}(f):\mathbb{R}\rightarrow\mathbb{C}$ defined by:
            \begin{equation}
                F_{\xi}(f)=\int_{\minus\infty}^{\infty}f(t)
                    \exp(\minus{2}\pi{i}\xi{t})\diff{t}
            \end{equation}
            This is also called the \textit{spectrum} of $f$.
        \end{ldefinition}
        The requirement that $f$ be integrable is to avoid
        strange issues in mathematics. For the sake of
        physical application, one may assume every function
        is integrable. Mathematically this is far from true,
        but oh well. For the sake of Fourier Analysis, when
        we say integrable we mean Lebesgue integrable. This
        simply means that:
        \begin{equation}
            \int_{\minus\infty}^{\infty}|f(x)|\diff{x}<\infty
        \end{equation}
        \begin{lexample}{}{FT_Hat_Func}
            Consider the hat function:
            \begin{equation}
                f(t)=
                \begin{cases}
                    0,&|t|\leq{1}\\
                    1,&|t|>1
                \end{cases}
            \end{equation}
            We can compute the Fourier transform of the this
            explicitly:
            \begin{equation}
                \mathcal{F}_{\xi}(f)
                =\int_{\minus\infty}^{\infty}f(t)
                    \exp(\minus{2}\pi{i}\xi{t})\diff{t}
                =\int_{\minus{1}}^{1}
                    \exp(\minus{2}\pi{i}\xi{t})\diff{t}
            \end{equation}
            Here we invoke Euler's Theorem,
            Thm.~\ref{thm:Euler_Expo_Formula},
            and note that the integral has symmetric limits.
            But $\sin$ is an \textit{odd} function, and thus
            it's integral is zero, and $\cos$ is an even
            function. Thus, we are left with:
            \begin{equation}
                \mathcal{F}_{\xi}(f)=
                2\int_{0}^{1}\cos(2\pi{i}\xi{t})\diff{t}
                =\frac{\sin(2\pi\xi)}{\pi\xi}
            \end{equation}
            We can be even more general, defining:
            \begin{equation}
                f(x)=
                \begin{cases}
                    1,&a\leq{x}\leq{b}\\
                    0,&\textrm{Otherwise}
                \end{cases}
            \end{equation}
            The Fourier transform is then:
            \begin{equation}
                \mathcal{F}_{\xi}(f)=
                \frac{i\big(\exp(\minus{2}\pi{i}\xi{b})-
                        \exp(\minus{2}\pi{i}\xi{a}\big)}{2\pi\xi}
            \end{equation}
            Thus we see that the range of the Fourier transform
            generally lies in the complex plane. Only with
            sufficient symmetry does the problem collapse down
            to $\mathbb{R}$.
        \end{lexample}
        Recall that a complex number has a polar representation
        $z=r\exp(i\theta)$. Similarly, for a complex valued
        function we can write:
        \begin{equation}
            f(z)=R(r)\exp\big(i\Theta(\theta)\big)
        \end{equation}
        For the Fourier transform of a function, the function
        $R(r)$ is called the principle amplitude, and
        $\Theta(\theta)$ is the \textit{phase offset} from
        this amplitude. The Fourier transform of the hat
        function defined from $\minus{1}$ to $1$ is plotted
        in Fig.~\ref{fig:Diff_Theory_FT_of_Hat_Func}.
        \begin{figure}[H]
            \centering
            \captionsetup{type=figure}
            \begin{subfigure}[b]{0.49\textwidth}
                \centering
                \resizebox{\textwidth}{!}{%
                    %--------------------------------Dependencies----------------------------------%
%   amssymb                                                                    %
%   tikz                                                                       %
%       arrows.meta                                                            %
%   Unary minus sign.                                                          %
%       \DeclareMathSymbol{\minus}{\mathbin}{AMSa}{"39}                        %
%-------------------------------Main Document----------------------------------%
\begin{tikzpicture}[%
    >=Latex,
    line width=0.2mm,
    line cap=round,
    scale=1.7
]

    % Coordinates for the square well.
    \coordinate (P1) at (-2, 0);
    \coordinate (P2) at (-1, 0);
    \coordinate (P3) at (-1, 1);
    \coordinate (P4) at (1, 1);
    \coordinate (P5) at (1, 0);
    \coordinate (P6) at (2, 0);

    % Axes:
    \begin{scope}[thick]
        \draw[<->] (-2.2, 0) to (2.2, 0) node [above]       {$t$};
        \draw[->]  (0, -0.5) to (0, 2.3) node [below right] {$f$};
    \end{scope}

    % Axes labels:
    \begin{scope}[font=\large]
        \draw (1, 0.05)  to (1, -0.05)  node [below] {1};
        \draw (2, 0.05)  to (2, -0.05)  node [below] {2};
        \draw (-1, 0.05) to (-1, -0.05) node [below] {$\minus{1}$};
        \draw (-2, 0.05) to (-2, -0.05) node [below] {$\minus{2}$};
        \draw (0.05, 1)  to (-0.05, 1)  node [left]  {1};
        \draw (0.05, 2)  to (-0.05, 2)  node [left]  {2};
    \end{scope}

    % Draw the square well.
    \draw[red] (P1) to (P2) to (P3) to (P4) to (P5) to (P6);
\end{tikzpicture}

                }
                \subcaption{Time Domain.}
            \end{subfigure}
            \begin{subfigure}[b]{0.49\textwidth}
                \centering
                \resizebox{\textwidth}{!}{%
                    %--------------------------------Dependencies----------------------------------%
%   amssymb                                                                    %
%   tikz                                                                       %
%       arrows.meta                                                            %
%   Unary minus sign.                                                          %
%       \DeclareMathSymbol{\minus}{\mathbin}{AMSa}{"39}                        %
%-------------------------------Main Document----------------------------------%
\begin{tikzpicture}[%
    >=Latex,
    line width=0.2mm,
    line cap=round,
    scale=1.7
]

    % Axes:
    \begin{scope}[thick]
        \draw[<->] (-2.2, 0) to (2.2, 0) node [above] {$\xi$};
        \draw[->]  (0, -0.5) to (0, 2.3);
        \node at (0, 2.3) [below right] {$\mathcal{F}_{\xi}(f)$};
    \end{scope}

    % Axes labels:
    \begin{scope}[font=\large]
        \draw (1, 0.05)  to (1, -0.05)  node[below] {1};
        \draw (2, 0.05)  to (2, -0.05)  node[below] {2};
        \draw (-1, 0.05) to (-1, -0.05) node[below] {$\minus{1}$};
        \draw (-2, 0.05) to (-2, -0.05) node[below] {$\minus{2}$};
        \draw (0.05, 1)  to (-0.05, 1)  node[left] {1};
        \draw (0.05, 2)  to (-0.05, 2)  node[left] {2};
    \end{scope}

    % Draw the curve. Points have be precomputed.
    \draw[blue] (-2.0, -7.796343665038751e-17)
        to (-1.99, -0.010043639761204455)
        to (-1.98, -0.02014889258126266)
        to (-1.97, -0.03027681467956927)
        to (-1.96, -0.04038798452984946)
        to (-1.95, -0.050442648363254745)
        to (-1.94, -0.060400868281710686)
        to (-1.93, -0.07022267245465998)
        to (-1.92, -0.07986820686038558)
        to (-1.91, -0.0892978880230314)
        to (-1.9, -0.09847255618827785)
        to (-1.89, -0.10735362837445696)
        to (-1.88, -0.11590325073170454)
        to (-1.8699999999999999, -0.12408444963960635)
        to (-1.8599999999999999, -0.1318612809736909)
        to (-1.8499999999999999, -0.1391989769731038)
        to (-1.8399999999999999, -0.14606409014583852)
        to (-1.8299999999999998, -0.15242463365402076)
        to (-1.8199999999999998, -0.15825021762993058)
        to (-1.8099999999999998, -0.1635121808836764)
        to (-1.7999999999999998, -0.1681837174757017)
        to (-1.7899999999999998, -0.1722399976415592)
        to (-1.7799999999999998, -0.17565828257260543)
        to (-1.7699999999999998, -0.1784180325743938)
        to (-1.7599999999999998, -0.18050100814453648)
        to (-1.7499999999999998, -0.1818913635335947)
        to (-1.7399999999999998, -0.18257573237608282)
        to (-1.7299999999999998, -0.18254330500385949)
        to (-1.7199999999999998, -0.18178589708095205)
        to (-1.7099999999999997, -0.18029800922712902)
        to (-1.6999999999999997, -0.1780768773272133)
        to (-1.6899999999999997, -0.1751225132541148)
        to (-1.6799999999999997, -0.1714377357657579)
        to (-1.6699999999999997, -0.16702819136937563)
        to (-1.6599999999999997, -0.161902364980929)
        to (-1.6499999999999997, -0.15607158024257065)
        to (-1.6399999999999997, -0.1495499893969906)
        to (-1.6299999999999997, -0.14235455265402638)
        to (-1.6199999999999997, -0.13450500702197754)
        to (-1.6099999999999997, -0.1260238246134923)
        to (-1.5999999999999996, -0.11693616047357955)
        to (-1.5899999999999996, -0.10726979001508762)
        to (-1.5799999999999996, -0.09705503618477161)
        to (-1.5699999999999996, -0.08632468652069587)
        to (-1.5599999999999996, -0.07511390029904998)
        to (-1.5499999999999996, -0.06346010600538449)
        to (-1.5399999999999996, -0.051402889401625705)
        to (-1.5299999999999996, -0.038983872495915096)
        to (-1.5199999999999996, -0.02624658375717056)
        to (-1.5099999999999996, -0.013236319950196054)
        to (-1.4999999999999996, 8.318726944787771e-16)
        to (-1.4899999999999995, 0.01341398867436105)
        to (-1.4799999999999995, 0.02695595088574403)
        to (-1.4699999999999995, 0.04057505096513741)
        to (-1.4599999999999995, 0.054219486081168125)
        to (-1.4499999999999995, 0.06783666504023989)
        to (-1.4399999999999995, 0.08137339199063875)
        to (-1.4299999999999995, 0.09477605443181418)
        to (-1.4199999999999995, 0.10799081490981746)
        to (-1.4099999999999995, 0.12096380576169573)
        to (-1.3999999999999995, 0.13364132625552064)
        to (-1.3899999999999995, 0.14597004145879433)
        to (-1.3799999999999994, 0.1578971821562355)
        to (-1.3699999999999994, 0.1693707451285144)
        to (-1.3599999999999994, 0.180339693096372)
        to (-1.3499999999999994, 0.19075415362980946)
        to (-1.3399999999999994, 0.2005656163196591)
        to (-1.3299999999999994, 0.20972712750891623)
        to (-1.3199999999999994, 0.2181934818836926)
        to (-1.3099999999999994, 0.22592141022859147)
        to (-1.2999999999999994, 0.23286976265866408)
        to (-1.2899999999999994, 0.2389996866499156)
        to (-1.2799999999999994, 0.2442747992025297)
        to (-1.2699999999999994, 0.24866135248557267)
        to (-1.2599999999999993, 0.25212839232887646)
        to (-1.2499999999999993, 0.2546479089470327)
        to (-1.2399999999999993, 0.25619497930192275)
        to (-1.2299999999999993, 0.2567479005338837)
        to (-1.2199999999999993, 0.25628831391740775)
        to (-1.2099999999999993, 0.25480131882511614)
        to (-1.1999999999999993, 0.25227557621355207)
        to (-1.1899999999999993, 0.2487034011760116)
        to (-1.1799999999999993, 0.2440808441410788)
        to (-1.1699999999999993, 0.23840776033064714)
        to (-1.1599999999999993, 0.23168786712788086)
        to (-1.1499999999999992, 0.22392878904368801)
        to (-1.1399999999999992, 0.21514209000970533)
        to (-1.1299999999999992, 0.20534329276642718)
        to (-1.1199999999999992, 0.19455188515678848)
        to (-1.1099999999999992, 0.18279131317812802)
        to (-1.0999999999999992, 0.17008896068884227)
        to (-1.0899999999999992, 0.1564761157100812)
        to (-1.0799999999999992, 0.14198792330735047)
        to (-1.0699999999999992, 0.12666332508176809)
        to (-1.0599999999999992, 0.1105449853457708)
        to (-1.0499999999999992, 0.09367920410318571)
        to (-1.0399999999999991, 0.0761158169985607)
        to (-1.0299999999999991, 0.0579080824453877)
        to (-1.0199999999999991, 0.03911255618715535)
        to (-1.0099999999999991, 0.019788953588905898)
        to (-0.9999999999999991, -1.7742592667642644e-15)
        to (-0.9899999999999991, -0.020188730429089686)
        to (-0.9799999999999991, -0.04070898705194081)
        to (-0.9699999999999991, -0.06149002568943619)
        to (-0.9599999999999991, -0.08245880174844435)
        to (-0.9499999999999991, -0.10354017295615647)
        to (-0.9399999999999991, -0.12465711113459639)
        to (-0.929999999999999, -0.14573092240590899)
        to (-0.919999999999999, -0.16668147518689336)
        to (-0.909999999999999, -0.18742743530108955)
        to (-0.899999999999999, -0.20788650750858853)
        to (-0.889999999999999, -0.22797568272778077)
        to (-0.879999999999999, -0.24761149019955225)
        to (-0.869999999999999, -0.26671025382306346)
        to (-0.859999999999999, -0.2851883518733333)
        to (-0.849999999999999, -0.30296247929440406)
        to (-0.839999999999999, -0.31994991174802867)
        to (-0.829999999999999, -0.3360687705865772)
        to (-0.819999999999999, -0.3512382879103352)
        to (-0.8099999999999989, -0.3653790708635251)
        to (-0.7999999999999989, -0.37841336432032996)
        to (-0.7899999999999989, -0.3902653111118883)
        to (-0.7799999999999989, -0.4008612089477415)
        to (-0.7699999999999989, -0.4101297631904905)
        to (-0.7599999999999989, -0.4180023346505061)
        to (-0.7499999999999989, -0.4244131815783882)
        to (-0.7399999999999989, -0.4292996950464654)
        to (-0.7299999999999989, -0.43260262692695495)
        to (-0.7199999999999989, -0.4342663096933856)
        to (-0.7099999999999989, -0.43423886729350797)
        to (-0.6999999999999988, -0.4324724163660894)
        to (-0.6899999999999988, -0.42892325710065765)
        to (-0.6799999999999988, -0.4235520530683426)
        to (-0.6699999999999988, -0.4163239993833686)
        to (-0.6599999999999988, -0.4072089785883965)
        to (-0.6499999999999988, -0.39618170369267847)
        to (-0.6399999999999988, -0.38322184782978724)
        to (-0.6299999999999988, -0.3683141600413688)
        to (-0.6199999999999988, -0.3514485667348432)
        to (-0.6099999999999988, -0.33262025840610104)
        to (-0.5999999999999988, -0.31182976126287687)
        to (-0.5899999999999987, -0.289082993430488)
        to (-0.5799999999999987, -0.2643913054688588)
        to (-0.5699999999999987, -0.2377715049780549)
        to (-0.5599999999999987, -0.2092458651187796)
        to (-0.5499999999999987, -0.17884211692426247)
        to (-0.5399999999999987, -0.14659342533055988)
        to (-0.5299999999999987, -0.11253834890329932)
        to (-0.5199999999999987, -0.07672078329018768)
        to (-0.5099999999999987, -0.039189888479988604)
        to (-0.49999999999999867, 5.449566898677675e-15)
        to (-0.48999999999999866, 0.0407894757648972)
        to (-0.47999999999999865, 0.08311418189771459)
        to (-0.46999999999999864, 0.12690494663564683)
        to (-0.45999999999999863, 0.17208793408371229)
        to (-0.4499999999999986, 0.21858480957410997)
        to (-0.4399999999999986, 0.26631291924209455)
        to (-0.4299999999999986, 0.31518548334301455)
        to (-0.4199999999999986, 0.36511180279034033)
        to (-0.4099999999999986, 0.41599747835120177)
        to (-0.3999999999999986, 0.46774464189432696)
        to (-0.38999999999999857, 0.5202521990454517)
        to (-0.37999999999999856, 0.5734160825673873)
        to (-0.36999999999999855, 0.6271295157461259)
        to (-0.35999999999999854, 0.6812832850307441)
        to (-0.34999999999999853, 0.7357660211435566)
        to (-0.3399999999999985, 0.7904644878480745)
        to (-0.3299999999999985, 0.8452638775359408)
        to (-0.3199999999999985, 0.900048112770238)
        to (-0.3099999999999985, 0.9547001529014733)
        to (-0.2999999999999985, 1.0091023048542176)
        to (-0.2899999999999985, 1.063136537166873)
        to (-0.2799999999999985, 1.116684796354428)
        to (-0.26999999999999846, 1.169629324654367)
        to (-0.25999999999999845, 1.2218529782091776)
        to (-0.24999999999999845, 1.2732395447351708)
        to (-0.23999999999999844, 1.3236740597266081)
        to (-0.22999999999999843, 1.373043120246429)
        to (-0.21999999999999842, 1.4212351953601783)
        to (-0.2099999999999984, 1.4681409322780585)
        to (-0.1999999999999984, 1.513653457281321)
        to (-0.1899999999999984, 1.55766867052345)
        to (-0.17999999999999838, 1.600085533813748)
        to (-0.16999999999999837, 1.640806350510934)
        to (-0.15999999999999837, 1.679737036677147)
        to (-0.14999999999999836, 1.7167873826682853)
        to (-0.13999999999999835, 1.7518713043647554)
        to (-0.12999999999999834, 1.7849070832774179)
        to (-0.11999999999999833, 1.8158175947967063)
        to (-0.10999999999999832, 1.8445305238883971)
        to (-0.09999999999999831, 1.8709785675772823)
        to (-0.0899999999999983, 1.8950996235998903)
        to (-0.0799999999999983, 1.9168369646492533)
        to (-0.06999999999999829, 1.9361393976784786)
        to (-0.05999999999999828, 1.9529614077753144)
        to (-0.04999999999999827, 1.9672632861669341)
        to (-0.03999999999999826, 1.9790112419626187)
        to (-0.02999999999999825, 1.988177497291704)
        to (-0.01999999999999824, 1.9947403655450078)
        to (-0.009999999999998233, 1.9986843124796831)
        to (1.7763568394002505e-15, 2.0)
        to (0.010000000000001563, 1.9986843124796823)
        to (0.020000000000001794, 1.9947403655450062)
        to (0.030000000000002025, 1.988177497291701)
        to (0.04000000000000181, 1.979011241962615)
        to (0.0500000000000016, 1.9672632861669297)
        to (0.06000000000000183, 1.9529614077753088)
        to (0.07000000000000206, 1.936139397678472)
        to (0.08000000000000185, 1.916836964649246)
        to (0.09000000000000163, 1.8950996235998823)
        to (0.10000000000000187, 1.8709785675772732)
        to (0.1100000000000021, 1.8445305238883867)
        to (0.12000000000000188, 1.815817594796696)
        to (0.13000000000000167, 1.784907083277407)
        to (0.1400000000000019, 1.7518713043647431)
        to (0.15000000000000213, 1.7167873826682718)
        to (0.16000000000000192, 1.6797370366771334)
        to (0.1700000000000017, 1.640806350510921)
        to (0.18000000000000194, 1.6000855338137334)
        to (0.19000000000000217, 1.5576686705234337)
        to (0.20000000000000195, 1.5136534572813052)
        to (0.21000000000000174, 1.4681409322780432)
        to (0.22000000000000197, 1.4212351953601612)
        to (0.2300000000000022, 1.3730431202464106)
        to (0.240000000000002, 1.3236740597265904)
        to (0.2500000000000018, 1.2732395447351537)
        to (0.260000000000002, 1.2218529782091592)
        to (0.27000000000000224, 1.1696293246543472)
        to (0.280000000000002, 1.116684796354409)
        to (0.2900000000000018, 1.0631365371668549)
        to (0.30000000000000204, 1.0091023048541983)
        to (0.3100000000000023, 0.9547001529014526)
        to (0.32000000000000206, 0.9000481127702183)
        to (0.33000000000000185, 0.8452638775359221)
        to (0.3400000000000021, 0.7904644878480548)
        to (0.3500000000000023, 0.7357660211435357)
        to (0.3600000000000021, 0.6812832850307248)
        to (0.3700000000000019, 0.6271295157461079)
        to (0.3800000000000021, 0.5734160825673683)
        to (0.39000000000000234, 0.5202521990454319)
        to (0.40000000000000213, 0.4677446418943086)
        to (0.4100000000000019, 0.41599747835118467)
        to (0.42000000000000215, 0.3651118027903225)
        to (0.4300000000000024, 0.315185483342996)
        to (0.44000000000000217, 0.2663129192420775)
        to (0.45000000000000195, 0.2185848095740943)
        to (0.4600000000000022, 0.17208793408369577)
        to (0.4700000000000024, 0.12690494663562985)
        to (0.4800000000000022, 0.08311418189769908)
        to (0.490000000000002, 0.04078947576488339)
        to (0.5000000000000022, -8.686231685604598e-15)
        to (0.5100000000000025, -0.039189888480002974)
        to (0.5200000000000022, -0.07672078329020064)
        to (0.530000000000002, -0.11253834890331092)
        to (0.5400000000000023, -0.1465934253305716)
        to (0.5500000000000025, -0.17884211692427424)
        to (0.5600000000000023, -0.20924586511879004)
        to (0.5700000000000021, -0.2377715049780641)
        to (0.5800000000000023, -0.2643913054688679)
        to (0.5900000000000025, -0.2890829934304971)
        to (0.6000000000000023, -0.31182976126288475)
        to (0.6100000000000021, -0.33262025840610765)
        to (0.6200000000000023, -0.35144856673484964)
        to (0.6300000000000026, -0.3683141600413749)
        to (0.6400000000000023, -0.38322184782979213)
        to (0.6500000000000021, -0.3961817036926824)
        to (0.6600000000000024, -0.40720897858840005)
        to (0.6700000000000026, -0.4163239993833717)
        to (0.6800000000000024, -0.4235520530683448)
        to (0.6900000000000022, -0.42892325710065915)
        to (0.7000000000000024, -0.43247241636609035)
        to (0.7100000000000026, -0.43423886729350836)
        to (0.7200000000000024, -0.43426630969338526)
        to (0.7300000000000022, -0.4326026269269541)
        to (0.7400000000000024, -0.429299695046464)
        to (0.7500000000000027, -0.42441318157838603)
        to (0.7600000000000025, -0.4180023346505036)
        to (0.7700000000000022, -0.4101297631904876)
        to (0.7800000000000025, -0.40086120894773797)
        to (0.7900000000000027, -0.39026531111188395)
        to (0.8000000000000025, -0.37841336432032535)
        to (0.8100000000000023, -0.36537907086352067)
        to (0.8200000000000025, -0.35123828791032974)
        to (0.8300000000000027, -0.3360687705865712)
        to (0.8400000000000025, -0.3199499117480228)
        to (0.8500000000000023, -0.3029624792943984)
        to (0.8600000000000025, -0.2851883518733269)
        to (0.8700000000000028, -0.2667102538230566)
        to (0.8800000000000026, -0.2476114901995454)
        to (0.8900000000000023, -0.2279756827277743)
        to (0.9000000000000026, -0.20788650750858137)
        to (0.9100000000000028, -0.187427435301082)
        to (0.9200000000000026, -0.16668147518688597)
        to (0.9300000000000024, -0.14573092240590185)
        to (0.9400000000000026, -0.12465711113458892)
        to (0.9500000000000028, -0.10354017295614842)
        to (0.9600000000000026, -0.08245880174843691)
        to (0.9700000000000024, -0.06149002568942911)
        to (0.9800000000000026, -0.0407089870519335)
        to (0.9900000000000029, -0.02018873042908191)
        to (1.0000000000000027, 5.293640025376871e-15)
        to (1.0100000000000025, 0.01978895358891254)
        to (1.0200000000000027, 0.0391125561871621)
        to (1.030000000000003, 0.05790808244539477)
        to (1.0400000000000027, 0.07611581699856702)
        to (1.0500000000000025, 0.09367920410319154)
        to (1.0600000000000027, 0.11054498534577686)
        to (1.070000000000003, 0.1266633250817741)
        to (1.0800000000000027, 0.14198792330735593)
        to (1.0900000000000025, 0.15647611571008574)
        to (1.1000000000000028, 0.17008896068884694)
        to (1.110000000000003, 0.18279131317813266)
        to (1.1200000000000028, 0.19455188515679248)
        to (1.1300000000000026, 0.20534329276643054)
        to (1.1400000000000028, 0.2151420900097086)
        to (1.150000000000003, 0.22392878904369107)
        to (1.1600000000000028, 0.23168786712788342)
        to (1.1700000000000026, 0.23840776033064917)
        to (1.1800000000000028, 0.24408084414108064)
        to (1.190000000000003, 0.2487034011760131)
        to (1.2000000000000028, 0.2522755762135532)
        to (1.2100000000000026, 0.25480131882511686)
        to (1.2200000000000029, 0.2562883139174081)
        to (1.230000000000003, 0.25674790053388363)
        to (1.2400000000000029, 0.25619497930192237)
        to (1.2500000000000027, 0.254647908947032)
        to (1.260000000000003, 0.2521283923288754)
        to (1.2700000000000031, 0.24866135248557117)
        to (1.280000000000003, 0.24427479920252793)
        to (1.2900000000000027, 0.23899968664991386)
        to (1.300000000000003, 0.23286976265866188)
        to (1.3100000000000032, 0.22592141022858875)
        to (1.320000000000003, 0.21819348188368964)
        to (1.3300000000000027, 0.20972712750891326)
        to (1.340000000000003, 0.2005656163196559)
        to (1.3500000000000032, 0.19075415362980574)
        to (1.360000000000003, 0.18033969309636808)
        to (1.3700000000000028, 0.16937074512851058)
        to (1.380000000000003, 0.15789718215623152)
        to (1.3900000000000032, 0.14597004145878986)
        to (1.400000000000003, 0.13364132625551603)
        to (1.4100000000000028, 0.12096380576169138)
        to (1.420000000000003, 0.10799081490981302)
        to (1.4300000000000033, 0.09477605443180928)
        to (1.440000000000003, 0.0813733919906338)
        to (1.4500000000000028, 0.06783666504023528)
        to (1.460000000000003, 0.054219486081163504)
        to (1.4700000000000033, 0.04057505096513239)
        to (1.480000000000003, 0.02695595088573903)
        to (1.4900000000000029, 0.013413988674356473)
        to (1.500000000000003, -4.068537481405745e-15)
        to (1.5100000000000033, -0.013236319950200876)
        to (1.5200000000000031, -0.02624658375717493)
        to (1.530000000000003, -0.03898387249591937)
        to (1.5400000000000031, -0.05140288940163021)
        to (1.5500000000000034, -0.06346010600538886)
        to (1.5600000000000032, -0.07511390029905386)
        to (1.570000000000003, -0.08632468652069959)
        to (1.5800000000000032, -0.09705503618477546)
        to (1.5900000000000034, -0.10726979001509128)
        to (1.6000000000000032, -0.11693616047358275)
        to (1.610000000000003, -0.1260238246134953)
        to (1.6200000000000032, -0.1345050070219805)
        to (1.6300000000000034, -0.14235455265402935)
        to (1.6400000000000032, -0.14954998939699296)
        to (1.650000000000003, -0.15607158024257275)
        to (1.6600000000000033, -0.16190236498093102)
        to (1.6700000000000035, -0.16702819136937752)
        to (1.6800000000000033, -0.1714377357657593)
        to (1.690000000000003, -0.1751225132541159)
        to (1.7000000000000033, -0.17807687732721428)
        to (1.7100000000000035, -0.18029800922712974)
        to (1.7200000000000033, -0.18178589708095244)
        to (1.730000000000003, -0.18254330500385962)
        to (1.7400000000000033, -0.18257573237608268)
        to (1.7500000000000036, -0.1818913635335943)
        to (1.7600000000000033, -0.18050100814453585)
        to (1.7700000000000031, -0.17841803257439304)
        to (1.7800000000000034, -0.17565828257260438)
        to (1.7900000000000036, -0.1722399976415578)
        to (1.8000000000000034, -0.1681837174757001)
        to (1.8100000000000032, -0.16351218088367483)
        to (1.8200000000000034, -0.1582502176299287)
        to (1.8300000000000036, -0.1524246336540185)
        to (1.8400000000000034, -0.1460640901458361)
        to (1.8500000000000032, -0.13919897697310157)
        to (1.8600000000000034, -0.13186128097368835)
        to (1.8700000000000037, -0.12408444963960341)
        to (1.8800000000000034, -0.11590325073170148)
        to (1.8900000000000032, -0.10735362837445399)
        to (1.9000000000000035, -0.0984725561882748)
        to (1.9100000000000037, -0.08929788802302797)
        to (1.9200000000000035, -0.07986820686038207)
        to (1.9300000000000033, -0.07022267245465667)
        to (1.9400000000000035, -0.060400868281707334)
        to (1.9500000000000037, -0.05044264836325106)
        to (1.9600000000000035, -0.040387984529845745)
        to (1.9700000000000033, -0.030276814679565833)
        to (1.9800000000000035, -0.020148892581259226)
        to (1.9900000000000038, -0.010043639761200752);
\end{tikzpicture}

                }
                \subcaption{Frequency Domain.}
            \end{subfigure}
            \caption{Fourier Transform of the Hat Function.}
            \label{fig:Diff_Theory_FT_of_Hat_Func}
        \end{figure}
        \begin{lexample}{}{FT_Decaying_Expo}
            Let $f:\mathbb{R}\rightarrow\mathbb{R}$ be defined by:
            \begin{equation}
                f(t)=
                \begin{cases}
                    \beta\exp(\minus\alpha{t}),&t\geq{0}\\
                    0,&t<0
                \end{cases}
            \end{equation}
            Where $\alpha$ and $\beta$ are positive real numbers.
            Since $f(t)$ decays to zero rapidly as
            $t\rightarrow\infty$, we see that $f$ is a
            Lebesgue integrable function and has a Fourier
            transform. We can compute this using the standard
            methods obtained from a course on integral calculus.
            \begin{equation}
                \mathcal{F}_{\xi}(f)=\int_{\minus\infty}^{\infty}f(t)
                                        \exp(\minus{2}\pi{i}\xi{t})\diff{t}
                                    =\beta\int_{0}^{\infty}
                                        \exp(\minus\alpha{t})
                                        \exp(\minus{2}\pi{i}\xi{t})\diff{t}
            \end{equation}
            Using the product rule for exponents, we can
            reduce this to the following line integral:
            \begin{equation}
                \mathcal{F}_{\xi}(f)=
                \beta\int_{0}^{\infty}\exp\big(
                    \minus(\alpha+2\pi{i}\xi)t\big)\diff{t}
            \end{equation}
            Thus, we are integrating the exponential function
            along the line $z=\theta{t}$, where
            $\theta=\tan^{\minus{1}}(2\pi\xi/\alpha)$.
            Since $\alpha$ is positive, we can use the result
            from Jordan's Lemma to obtain the solution:
            \begin{equation}
                \mathcal{F}_{\xi}(f)
                =\frac{\beta}{\alpha+2\pi{i}\xi}
            \end{equation}
            There are two ways to view the Fourier
            transform: The Cartesian form and the
            polar form. As was stated before, the
            polar form represents the \textit{amplitude}
            and \textit{phase offset} of the Fourier
            transform, whereas the Cartesian form simply
            represents the real and imaginary parts. To
            compute the Cartesian form, we simply invoke
            Eqn.~\ref{eqn:Mult_Inv_of_Complex} for the inverse
            of a complex number, and obtain:
            \begin{equation}
                \mathcal{F}_{\xi}(f)
                =\beta\frac{\alpha-2\pi{i}\xi}
                    {\alpha^{2}+4\pi^{2}\xi^{2}}
            \end{equation}
            For the polar form, we invoke
            Thm.~\ref{thm:Polar_Form_Comp_Num},
            take the modulus of $\mathcal{F}_{\xi}(f)$
            and compute inverse tangents:
            \begin{equation}
                \mathcal{F}_{\xi}(f)=
                \frac{\beta}{\sqrt{\alpha^{2}+4\pi^{2}\xi^{2}}}
                \exp\Big[i\tan^{\minus{1}}
                    \Big(\frac{\minus{2}\pi\xi}{\alpha}\Big)
                \Big]
            \end{equation}
            The two are plotted below for the case of
            $\alpha=\beta=1$.
        \end{lexample}
        \begin{ldefinition}{Inverse Fourier Transform}
              {Diff_Theory_Inverse_Fourier_Transform}
            The inverse Fourier transform of a complex valued
            integrable function
            $f:\mathbb{R}\rightarrow\mathbb{C}$ is the function
            $\mathcal{F}_{t}^{\minus{1}}(f):%
                \mathbb{R}\rightarrow\mathbb{C}$ defined by:
            \begin{equation}
                \mathcal{F}_{t}^{\minus{1}}(f)=
                \int_{\minus{\infty}}^{\infty}
                    f(\xi)\exp(2\pi{i}\xi{t})\diff{\xi}
            \end{equation}
        \end{ldefinition}
        We now prove what is probably the most useful theorem
        in Fourier Analysis.
        \begin{theorem}
            If $f:\mathbb{R}\rightarrow\mathbb{R}$ is a
            continuous Lebesgue integrable function,
            and if its spectrum $F$ is also continuous
            and Lebesgue integrable, then:
            \begin{equation}
                f(t)=\int_{-\infty}^{\infty}F(\omega)
                \exp(2\pi{i}\omega{t})\diff{\omega}
            \end{equation}
        \end{theorem}
        A powerful application of this is
        Shannon's Sampling Theorem.
        \begin{theorem}[Shannon's Sampling Theorem]
            If $f(t)$ is a continuous
            Lebesgue integrable function such that
            its spectrum $F(\omega)$ is differentiable and zero
            outside the interval $[-W,W]$,
            then $f(t)$ is uniquely determined
            by the points $f(\frac{n}{2W})$, $n\in\mathbb{N}$.
        \end{theorem}
        \begin{proof}
            For let $F$ be the spectrum of $f$. That is:
            \begin{equation}
                f(t)=\int_{-\infty}^{\infty}F(\omega)
                \exp(-2\pi{i}\omega{t})\diff{\omega}
            \end{equation}
            But $F(\omega)=0$ for $|\omega|>W$. Thus we have:
            \begin{equation}
                f(t)=\int_{-W}^{W}F(\omega)
                \exp(-2\pi{i}\omega{t})\diff{\omega}
            \end{equation}
            Then for $n\in\mathbb{N}$, we have:
            \begin{equation}
                f\big(\frac{n}{2W}\big)=\int_{-W}^{W}F(\omega)
                \exp(-2\pi{i}\frac{n}{2W}\omega)\diff{\omega}
            \end{equation}
            But $F$ is differentiable, and thus it's Fourier
            series converges. That is:
            \begin{subequations}
                \begin{align}
                    F(\omega)
                    &=\sum_{n=\minus\infty}^{\infty}
                        \exp(2\pi{i}n\omega)
                    \int_{\minus{W}}^{W}F(\tau)
                    \exp(\minus{2}\pi{i}\frac{n}{2W}\tau)
                        \diff{\tau}\\
                    &=\sum_{n=-\infty}^{\infty}f
                      \big(\frac{n}{2W}\big)e^{2\pi{i}n\omega}
                \end{align}
            \end{subequations}
            Therefore $f(\frac{n}{2W})$, $n\in \mathbb{N}$
            uniquely determines $F(\omega)$. But the
            spectrum $F(\omega)$ uniquely determines
            $f(t)$. Therefore $f(t)$ is
            uniquely determined and:
            \begin{equation}
                f(t)=\sum_{n=-\infty}^{\infty}\int_{-W}^{W}
                f\big(\frac{n}{2W}\big)
                \exp(2\pi{i}\omega(n+t))\diff{\omega}
            \end{equation}
        \end{proof}
    \subsection{Convolutions}
    \subsection{Sampling}
