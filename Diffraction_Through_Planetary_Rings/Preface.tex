\chapter*{Preface}
\addcontentsline{toc}{chapter}{Preface}
    This book was written to provide a student of physics the material
    needed to independently study the radio occultation observations that
    have been made by both the Voyager and Cassini radio science missions.
    In developing the mathematics necessary for diffraction theory, it is
    foolishly assumed that the reader has a strong
    familiarity with differential and integral calculus, and is aware of the
    $\varepsilon-\delta$ definition of continuity. An attempt was made to
    make this the only prerequesite, and the first (And longest) chapter
    is entirely devoted to developing mathematics. This chapter covers
    complex, Fourier, and numerical analysis, as well as partial
    differential equations and bits of vector calculus. No effort was made
    to shy away from rigor, and proofs of almost all theorems are given.
    It is not the author's wish to present mathematics as a list of
    mysterious formulae with no background.
    \par\hfill\par
    Those familiar with the preliminaries may skip chapter 1, but this is
    discouraged. Many counter-intuitive results from the theory of
    integration and of distributions will be presented, and the
    derivations made in later chapters make constant reference to the
    theorems proved here. The second chapter derives the Fresnel-Huygens
    principle, taking the Maxwell Equations as the axioms of
    electromagnetism. No previous experience with electricity and magnetism
    is expected, but will make understanding the derivations easier.
    But indeed, in a course on electromagnetism one often takes
    \textit{Coulomb's Law} and the \textit{Biot-Savart Law} as axioms,
    on the basis of experimental data, and from there derives Maxwell's
    equations. It is perfectly valid to go on the other direction as well.
    \par\hfill\par
    The Third and Fourth chapters are dedicated to diffraction in the
    context of occultation observations of planetary rings. The geometries
    of such experiments are presented, and the numerical methods we'll
    develop are used to justify various approximations for
    reconstructing ring profiles from the diffraction limited data.
    \par\hfill\par
    Part II of this books pertains to applying the theory to the Cassini
    radio science mission, and analyzes the rings of Saturn. Various
    algorithms are presented, and routines in the C programming language
    are given. This part will also demonstrate the limitations of the
    approximations, and the need for careful analysis.
    \par\hfill\par
    \hfill
    Ryan Maguire
    \par\hfill
    Wellesley College
    \par\hfill
    August, 2019