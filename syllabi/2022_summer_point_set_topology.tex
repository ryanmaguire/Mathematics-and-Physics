%-----------------------------------LICENSE------------------------------------%
%   This file is part of Mathematics-and-Physics.                              %
%                                                                              %
%   Mathematics-and-Physics is free software: you can redistribute it and/or   %
%   modify it under the terms of the GNU General Public License as             %
%   published by the Free Software Foundation, either version 3 of the         %
%   License, or (at your option) any later version.                            %
%                                                                              %
%   Mathematics-and-Physics is distributed in the hope that it will be useful, %
%   but WITHOUT ANY WARRANTY; without even the implied warranty of             %
%   MERCHANTABILITY or FITNESS FOR A PARTICULAR PURPOSE.  See the              %
%   GNU General Public License for more details.                               %
%                                                                              %
%   You should have received a copy of the GNU General Public License along    %
%   with Mathematics-and-Physics.  If not, see <https://www.gnu.org/licenses/>.%
%------------------------------------------------------------------------------%
\documentclass{article}
\usepackage{xcolor}

% No indent and no paragraph skip.
\setlength{\parindent}{0em}
\setlength{\parskip}{0em}

\begin{document}
    \LARGE
    \textbf{Point-Set Topology}
    \hrule\par\hfill\par
    \normalsize
    \textbf{Math 54 (Summer 2022)}\\
    Ryan Maguire\\
    \color{gray}{Ryan.J.Maguire.GR@dartmouth.edu}
    \par\vspace{0.5cm}
    \color{black}
    Class Time: MWF 10:10 - 11:15, Kemeny 108\\
    Office Hours: MW 12:00 - 2:00, Kemeny 241
    \par\vspace{0.5cm}
    \textbf{Course Description}
    \par\hfill\par
    Topology is the study of \textit{generalized geometry}. In geometry one
    considers rigid motions that leave an object in the same shape such as
    translations, rotations, and reflections. Topology allows you to
    \textit{continuously deform} the object by stretching, twisting, squeezing,
    and bending. Because of this it is often said a topologist can't tell the
    difference between a coffee cup and a donut (they're both objects with
    one hole in it). Certain geometrical properties like distance, volume,
    and curvature are not preserved by the motions allowed in topology,
    meaning other \textit{topological invariants} are needed to discern
    different types of \textit{topological spaces}. In this course we will
    discuss what topological spaces are, continuous functions, and various
    topological properties such as connectedness and compactness.
    \par\hfill\par
    \textbf{Course Objectives}
    \par\hfill\par
    By the end of the course one should have a solid understanding as to what
    topological spaces are, how they are used in other branches of mathematics
    and physics, and know about some of the important topological invariants
    that are used in the field. The course is very pictorial, and it is hoped
    students will appreciate topology for both its practical applications and
    also its aesthetics.
    \par\hfill\par
    \textbf{Prerequesites}
    \par\hfill\par
    Logically, there are no mathematical prerequesites. The course will cover
    the basics of logic and set theory before diving in to the main topics.
    Historically, set theory and point-set topology were originally the same
    field so it is fitting to start here. Practically, a student needs some
    mathematical maturity. Understanding the need for proof, and having some
    motivation for topology is recommended. This can be found in a course on
    real analysis (Math 35).
    \par\hfill\par
    \textbf{Textbook}
    \par\hfill\par
    There is \textbf{no required textbook}. I will be handing out notes and a
    mini textbook of my own that will cover everything needed in the class.
    For those who become really fascinated with topology, the following are
    \textit{recommended}, but not required.
    \begin{itemize}
        \item \textit{Topology}, second edition by James Munkres
        \item \textit{Introduction to Topology}, second edition
            by Theodore W. Gamelin and Robert Everist Greene.
        \item \textit{General Topology} by Stephen Willard
        \item \textit{General Topology} by John L. Kelley
    \end{itemize}
    \par\hfill\par
    \textbf{Grading}
    \par\hfill\par
    Your grade will consist of 3 components:
    \begin{enumerate}
        \item Bi-weekly homework assignments (50\%)
        \item Take-Home Midterm Exam (20\%)
        \item Take-Home Final Exam (30\%)
    \end{enumerate}
    The homework will consist of about 10 problems for each set. For the most
    part they will not have the feel of a problem from a calculus course where
    you have a concept and you need to apply. Rather, you'll have to ponder
    over the ideas and perhaps \textit{create} something new.
    \par\hfill\par
    \textbf{Order of Topics}
    \par\hfill\par
    There are 10 topics that we'll cover. Coincidentally there are 10 weeks in
    the course. However, some topics will get longer amounts of time than
    others.
    \begin{enumerate}
        \item Motivation, sets, and logic.
            \begin{itemize}
                \item What \textit{or}, \textit{and}, and \textit{if-then} mean.
                \item The notion of unions, intersections, and complements.
                \item Functions and cardinality.
                \item The axiom of choice and well-ordering.
            \end{itemize}
        \item Metric spaces
            \begin{itemize}
                \item Definition of metric spaces, the metric function.
                \item Sequences, what it means to be open and closed.
                \item Continuity, three equivalent definitions.
            \end{itemize}
        \item Topologicaly spaces
            \begin{itemize}
                \item Definition of topological spaces.
                \item Definition of continuous functions.
                \item Definition of homotopy equivalent and homeomorphism.
                \item The Hausdorff property.
                \item Equivalent notions of topological spaces.
            \end{itemize}
        \item Bases
            \begin{itemize}
                \item Definition of bases, sub-bases.
                \item Sequential spaces, their relation to metric spaces.
                \item First and second countable spaces.
                \item Separable spaces, what these terms mean for metric spaces.
            \end{itemize}
        \item Creating new spaces from old.
            \begin{itemize}
                \item Subspaces.
                \item Product spaces.
                \item Quotient spaces.
            \end{itemize}
        \item Separation properties.
            \begin{itemize}
                \item Regular spaces, normal spaces, and more.
            \end{itemize}
        \item Connectedness
            \begin{itemize}
                \item Definition of connected, path connected.
                \item What it means to by simply path connected.
            \end{itemize}
        \item Compactness
            \begin{itemize}
                \item Definition, Tychonoff theorem.
                \item Weaker notions (paracompactness, locally compact).
            \end{itemize}
        \item Metrization
            \begin{itemize}
                \item Metrization theorems.
                \item Topological properties of metrizable spaces.
            \end{itemize}
        \item Manifolds
            \begin{itemize}
                \item Definitions.
                \item The Euler characteristic.
                \item The fundamental group.
            \end{itemize}
    \end{enumerate}
    \textbf{The Honor Principle}
    \par\hfill\par
    On written homework, you are encouraged to work together, and you may get
    help from others, but you must write up the answers yourself. If you
    are part of a group of students that produces an answer to a problem,
    you cannot then copy that group answer. You must write up the answer
    individually, in your own words.
    \par\hfill\par
    On exams, you may not give or receive help from anyone.
    The take-home portion of the exams in this course are open book and open
    notes. But no other sources are allowed. In particular accessing the
    internet to research answers is an Honor Code violation. During take home
    exams, it is perfectly ok to interact with classmates as long as you do not
    discuss the exam in any way. Seriously, you shouldn't even comment on
    length, difficulty, or anything else.
    \par\hfill\par
    \textbf{Disabilities}
    \par\hfill\par
    Students with disabilities who may need disability-related academic
    adjustments and services for this course are encouraged to see me privately
    as early in the term as possible. Students requiring disability-related
    academic adjustments and services must consult the Student Accessibility
    Services office (Carson Hall, Suite 125, 646-9900). Once SAS has authorized
    services, students must show the originally signed SAS Services and Consent
    Form and/or a letter on SAS letterhead to their professor. As a first step,
    if students have questions about whether they qualify to receive academic
    adjustments and services, they should contact the SAS office. All inquiries
    and discussions will remain confidential.
    \par\hfill\par
    \textbf{Stress and Mental Well-Being}
    \par\hfill\par
    The academic environment at Dartmouth is challenging, our terms are
    intensive, and classes are not the only demanding part of your life. There
    are a number of resources available to you on campus to support your
    wellness, including your undergraduate dean, Counseling and Human
    Development, and the Student Wellness Center.
    \par\hfill\par
    \textbf{Religious Observances}
    \par\hfill\par
    Some students may wish to take part in religious observances that occur
    during this academic term. If you have a religious observance that
    conflicts with your participation in the course, please meet with me before
    the end of the second week of the term to discuss appropriate
    accommodations.
\end{document}
