%--------------------------------Dependencies----------------------------------%
%   tikz                                                                       %
%       arrows.meta                                                            %
%-------------------------------Main Document----------------------------------%
\begin{tikzpicture}[scale=10, >=Latex, font=\large]

    % Draw the x and y axes.
    \begin{scope}[thick]
        \draw[->] (-0.1,  0.0) to (1.1, 0.0) node[above left] {$x$};
        \draw[->] ( 0.0, -0.1) to (0.0, 0.6) node[right]      {$f(x)$};
    \end{scope}

    % Label some points on the y-axis.
    \foreach\n in {2, 3, 4}{
        \draw (0.02, 1/\n) to (-0.02, 1/\n) node [left] {$\frac{1}{\n}$};
    }

    % Loop over x and y and plot the function.
    \foreach\X[evaluate=\X as \Ymax using {int(\X-1)}]in {25,24,...,2}{%
        \foreach\Y in {1,...,\Ymax}{%
            \ifnum\X<4
                \draw (\Y/\X, 0.02) to (\Y/\X, -0.02)
                    node[below] {$\frac{\Y}{\X}$};
            \else
                \draw[ultra thin] (\Y/\X,0.01) to (\Y/\X,-0.01);
            \fi
            \pgfmathtruncatemacro{\TST}{gcd(\X, \Y)}
            \ifnum\TST=1
                \fill ({\Y/\X}, 1/\X)  circle (0.2pt);
            \fi
        }
    }

    % Plot some points that fall on the x-axis.
    \foreach\X in {0, 1, ..., 80}{%
        \fill (\X/80, 0) circle (0.2pt);
    }
\end{tikzpicture}