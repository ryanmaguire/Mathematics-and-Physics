%--------------------------------Dependencies----------------------------------%
%   tikz                                                                       %
%       calc                                                                   %
%       arrows.meta                                                            %
%       decorations.markings                                                   %
%-------------------------------Main Document----------------------------------%
\begin{tikzpicture}[%
    >=Latex,
    line width=0.2mm,
    line cap=round,
    scale=1.7
]

    % Start a group to locally define variouble for color.
    \begingroup
        % Color for the ring.
        \newcommand{\rcolor}{white!100!blue}

        % Coordinates
        \coordinate (S)     at (4,1.6) {};
        \coordinate (E)     at (-4,1.6) {};
        \coordinate (O)     at (0,0);
        \coordinate (p1)    at (-1.5,2);
        \coordinate (p2)    at (1.5,2);
        \coordinate (p_in)  at (0.44, 2.133);
        \coordinate (p_out) at (-0.44, 2.144);
        \coordinate (P)     at (0,2.2);
        \coordinate (v_S)   at (4.2, 1);
        \coordinate (v_E)   at (-4, 1);
        \coordinate (dash)  at (-3.5,2.8);

        % Labels for various points.
        \node at (E) [left]        {\footnotesize{Earth}};
        \node at (S) [above right] {\footnotesize{Spacecraft}};

        % Draw outer ring to simulate the atmosphere.
        \draw[draw=none, inner color=blue, outer color=\rcolor, opacity=0.6]
            (0,0) circle (26mm) (0,0) circle (1.9);

        % Draw inner circle.
        \draw[inner color=white, outer color=gray] (0, 0) circle (1.9);

        % Draw path for the emitted signal through the atmosphere.
        \begin{scope}[line width=0.3mm,->]
            \draw[%
                postaction={decorate},
                decoration={%
                    markings,
                    mark=at position .29 with \arrow{Latex},
                    mark=at position .72 with \arrow{Latex}
                }
            ]   (S) to (p2) to[bend right=10] (p1) to (E);
            \path (O) edge node [right]       {$\textbf{p}_{in}$}  (p_in);
            \path (O) edge node [left]        {$\textbf{p}_{out}$} (p_out);
            \path (O) edge node [below left]  {$\mathbf{r}_{E}$}   (E);
            \path (O) edge node [below right] {$\mathbf{r}_{S}$}   (S);
            \path (S) edge node [below left]  {$\mathbf{v}_{S}$}   (v_S);
            \path (E) edge node [below left]  {$\mathbf{v}_{E}$}   (v_E);
        \end{scope}

        % More labels.
        \node (nin) at (2.9,2.1) {$\hat{\mathbf{n}}_{in}$};
        \node (nout) at (-2.7,2.2) {$\hat{\mathbf{n}}_{out}$};

        % Draw more lines.
        \begin{scope}[thick]
          \draw[shorten >=1cm,shorten <=0cm,->] (nin)  to +($(p2)-(S)$);
          \draw[shorten >=1cm,shorten <=0cm,->] (nout) to +($(E)-(p1)$);
          \draw[dashed]                         (p2)   to (dash);
          \draw                                 (O)    to (0, 2.15);
        \end{scope}

        % And some more labels.
        \node at (-2,2.3) {$\alpha$};
        \node at (-0.1,1.6) [right] {$\rho$};
    \endgroup
\end{tikzpicture}
