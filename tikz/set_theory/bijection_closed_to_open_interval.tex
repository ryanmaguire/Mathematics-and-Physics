%--------------------------------Dependencies----------------------------------%
%   tikz                                                                       %
%       arrows.meta                                                            %
%-------------------------------Main Document----------------------------------%
%-----------------------------------LICENSE------------------------------------%
%   This file is part of Mathematics-and-Physics.                              %
%                                                                              %
%   Mathematics-and-Physics is free software: you can redistribute it and/or   %
%   modify it it under the terms of the GNU General Public License as          %
%   published by the Free Software Foundation, either version 3 of the         %
%   License, or (at your option) any later version.                            %
%                                                                              %
%   Mathematics-and-Physics is distributed in the hope that it will be useful, %
%   but WITHOUT ANY WARRANTY; without even the implied warranty of             %
%   MERCHANTABILITY or FITNESS FOR A PARTICULAR PURPOSE.  See the              %
%   GNU General Public License for more details.                               %
%                                                                              %
%   You should have received a copy of the GNU General Public License along    %
%   with Mathematics-and-Physics.  If not, see <https://www.gnu.org/licenses/>.%
%------------------------------------------------------------------------------%

%   Use the standalone class for displaying the tikz image on a small PDF.
\documentclass[crop, tikz]{standalone}

%   Import the tikz package to use for the drawing.
\usepackage{tikz}

% Used for the arrows on the x and y axes.
\usetikzlibrary{arrows.meta}

%   Begin the document.
\begin{document}

    %   Draw the figure.
    \begin{tikzpicture}[%
        >=Latex,
        line width=0.2mm,
        line cap=round,
        scale=1.7
    ]

        % Coordinates for the various points.
        \coordinate (O)   at (0.0, 0.0);
        \coordinate (P)   at (4.0, 4.0);

        % Axes:
        \begin{scope}[thick, ->]
            \draw (-0.5,  0.0) to (4.4, 0.0) node [above] {$x$};
            \draw ( 0.0, -0.5) to (0.0, 4.4) node [right] {$y$};
        \end{scope}

        % Draw the main part of the function.
        \draw (O) to (P);

        % Draw a dot marking f(0).
        \draw[fill=black, draw=black] (0, 0.5in) circle (0.3mm);

        % Draw the rest of the function.
        \foreach\x in {4, 2, 1, 0.5, 0.25, 0.125} {
            \draw[fill=white, draw=black] (\x, \x) circle (0.3mm);
            \draw[fill=black, draw=black] (\x, 0.25*\x) circle (0.3mm);
        }
    \end{tikzpicture}
\end{document}
