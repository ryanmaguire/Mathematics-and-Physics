%--------------------------------Dependencies----------------------------------%
%   tikz                                                                       %
%       arrows.meta                                                            %
%-------------------------------Main Document----------------------------------%
\begin{tikzpicture}[>=Latex, line width=0.2mm]
    % Coordinates for the manifold X.
    \coordinate (X0) at (-5.0,  0.0);
    \coordinate (X1) at (-3.5, -2.5);
    \coordinate (X2) at ( 1.0, -2.0);
    \coordinate (X3) at ( 5.0,  0.0);
    \coordinate (X4) at ( 0.0,  1.0);

    % Coordinates for the subset U.
    \coordinate (U0) at (-4.0, -0.5);
    \coordinate (U1) at (-3.0, -2.0);
    \coordinate (U2) at ( 1.5, -0.5);
    \coordinate (U3) at (-0.6,  0.2);

    % Coordinates for the subset V.
    \coordinate (V0) at ( 4.0,  0.0);
    \coordinate (V1) at ( 3.0, -1.5);
    \coordinate (V2) at (-1.5, -0.5);
    \coordinate (V3) at ( 0.6,  0.2);

    % Draw the manifold X.
    \draw   (X0) to[out=-90, in=120]  (X1)
                 to[out=-60, in=-170] (X2)
                 to[out=10, in=-90]   (X3)
                 to[out=90, in=0]     (X4)
                 to[out=-180, in=90]  cycle;

    % Fill in U and V first and then outline with dashes.
    % This prevents the fill option from drawing over the outline.
    % Setting opacity makes the overlapping part mix colors as well.

    % Fill in the background of U blue.
    \draw[fill=blue, opacity=0.5, draw=none]
        (U0) to[out=-90, in=-180] (U1)
             to[out=0, in=-100]   (U2)
             to[out=80, in=0]     (U3)
             to[out=-180, in=90]  cycle;

    % Fill in the background of V red.
    \draw[fill=red, opacity=0.5, draw=none]
        (V0) to[out=-90, in=0]   (V1)
             to[out=180, in=-80] (V2)
             to[out=100, in=180] (V3)
             to[out=0, in=90]    cycle;

    % Draw dashed lines around U.
    \draw[densely dashed]
        (U0) to[out=-90, in=-180] (U1)
             to[out=0, in=-100]   (U2)
             to[out=80, in=0]     (U3)
             to[out=-180, in=90]  cycle;

    \draw[densely dashed]
        (V0) to[out=-90, in=0]   (V1)
             to[out=180, in=-80] (V2)
             to[out=100, in=180] (V3)
             to[out=0, in=90]    cycle;

    \begin{scope}[xshift=-5cm, yshift=3cm]

        % Coordinates for phi of U.
        \coordinate (P0) at (0.5, 0.5);
        \coordinate (P1) at (1.5, 0.2);
        \coordinate (P2) at (3.3, 0.8);
        \coordinate (P3) at (2.8, 2.1);
        \coordinate (P4) at (2.2, 3.6);
        \coordinate (P5) at (1.2, 2.8);

        % Coordinate for some midpoint inside U.
        \coordinate (PM) at (2.0, 1.5);

        \draw[->] (-0.5,  0.0) to ( 4.0,  0.0);
        \draw[->] ( 0.0, -0.5) to ( 0.0,  4.0);

        \draw[draw=none, fill=blue!20!white]
            (P0)    to[out=-30,  in=180]    (P1)
                    to[out=0,    in=-90]    (P2)
                    to[out=90,   in=-120]   (P3)
                    to[out=60,   in=30]     (P4)
                    to[out=-150, in=60]     (P5)
                    to[out=-120, in=150]    cycle;

        \draw[densely dashed]
            (P0)    to[out=-30,  in=180]    (P1)
                    to[out=0,    in=-90]    (P2)
                    to[out=90,   in=-120]   (P3)
                    to[out=60,   in=30]     (P4)
                    to[out=-150, in=60]     (P5)
                    to[out=-120, in=150]    cycle;

        \draw[densely dashed, fill=cyan]
            (P3)    to[out=180,  in=70]   (PM)
                    to[out=-110, in=180]  (P1)
                    to[out=0,    in=-90]  (P2)
                    to[out=90,   in=-120] cycle;

        \node at (2.00, 3.0) {$\phi(\mathcal{U})$};
        \node at (2.45, 0.8) {$\phi(\mathcal{U}\cap\mathcal{V})$};
        \node at (3.50, 3.5) {\large{$\mathbb{R}^{n}$}};
    \end{scope}

    \begin{scope}[xshift=2cm, yshift=3cm]

        % Coordinates for phi of U.
        \coordinate (Q0) at (3.5, 0.5);
        \coordinate (Q1) at (2.5, 0.2);
        \coordinate (Q2) at (0.5, 0.8);
        \coordinate (Q3) at (1.2, 2.1);
        \coordinate (Q4) at (1.8, 3.6);
        \coordinate (Q5) at (2.8, 2.8);

        % Coordinate for some midpoint inside U.
        \coordinate (QM) at (2.0, 1.5);

        \draw[->] (-0.5,  0.0) to ( 4.0,  0.0);
        \draw[->] ( 0.0, -0.5) to ( 0.0,  4.0);

        \draw[draw=none, fill=red!20!white]
            (Q0)    to[out=-150,    in=0]       (Q1)
                    to[out=-180,    in=-90]     (Q2)
                    to[out=90,      in=-120]    (Q3)
                    to[out=60,      in=150]     (Q4)
                    to[out=-30,     in=60]      (Q5)
                    to[out=-120,    in=30]      cycle;

        \draw[densely dashed]
            (Q0)    to[out=-150,    in=0]       (Q1)
                    to[out=-180,    in=-90]     (Q2)
                    to[out=90,      in=-120]    (Q3)
                    to[out=60,      in=150]     (Q4)
                    to[out=-30,     in=60]      (Q5)
                    to[out=-120,    in=30]      cycle;

        \draw[densely dashed, fill=red!50!white]
            (Q3)    to[out=-60,     in=70]      (QM)
                    to[out=-110,    in=0]       (Q1)
                    to[out=-180,    in=-90]     (Q2)
                    to[out=90,      in=-120]    cycle;

        \node at (2.00, 3.0) {$\xi(\mathcal{V})$};
        \node at (1.25, 0.8) {$\xi(\mathcal{U}\cap\mathcal{V})$};
        \node at (3.50, 3.5) {\large{$\mathbb{R}^{n}$}};
    \end{scope}

    \begin{scope}[line width=0.4mm, ->, font=\large]
        \draw (-2.0, 0.5) to[out=130, in=-100] node[left]  {$\phi$} (-3.0, 3);
        \draw ( 2.5, 0.7) to[out=50,  in=-80]  node[right] {$\xi$}  ( 3.5, 3);
        \draw (-1.5, 4.5) to[out=30, in=150]
            node[above] {$\xi\circ\phi^{\minus{1}}$} ( 1.5, 4.5);
        \draw ( 1.5, 3.5) to[out=-150, in=-30]
            node[below] {$\phi\circ\xi^{\minus{1}}$} (-1.5, 3.5);
    \end{scope}

    \node at (-4.0,  0.5) {$X$};
    \node at (-3.0, -1.5) {$\mathcal{U}$};
    \node at ( 3.0, -1.3) {$\mathcal{V}$};
    \node at ( 0.0, -0.5) {$\mathcal{U}\cap\mathcal{V}$};
\end{tikzpicture}