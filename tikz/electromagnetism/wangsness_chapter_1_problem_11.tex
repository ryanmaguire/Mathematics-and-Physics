%-----------------------------------LICENSE------------------------------------%
%   This file is part of Mathematics-and-Physics.                              %
%                                                                              %
%   Mathematics-and-Physics is free software: you can redistribute it and/or   %
%   modify it it under the terms of the GNU General Public License as          %
%   published by the Free Software Foundation, either version 3 of the         %
%   License, or (at your option) any later version.                            %
%                                                                              %
%   Mathematics-and-Physics is distributed in the hope that it will be useful, %
%   but WITHOUT ANY WARRANTY; without even the implied warranty of             %
%   MERCHANTABILITY or FITNESS FOR A PARTICULAR PURPOSE.  See the              %
%   GNU General Public License for more details.                               %
%                                                                              %
%   You should have received a copy of the GNU General Public License along    %
%   with Mathematics-and-Physics.  If not, see <https://www.gnu.org/licenses/>.%
%------------------------------------------------------------------------------%

%   Use the standalone class for displaying the tikz image on a small PDF.
\documentclass[crop, tikz]{standalone}

%   Import the tikz package to use for the drawing.
\usepackage{tikz}

%   The markings and arrows libraries will be used, so load those.
\usetikzlibrary{decorations.markings, arrows.meta}

%   Begin the document.
\begin{document}

    %   Draw the figure.
    \begin{tikzpicture}[%
        >=latex,%
        font=\footnotesize,%
        line width=0.3pt,%
        line cap=round,%
        ->-/.style={%   This style is for a line with an arrow in the center.
            decoration={%
                markings,%
                mark=at position .55 with \arrow{Stealth}%
            },%
            postaction={decorate}%
        }%
    ]
        %   Draw the x and y axes.
        \draw[->] (-0.1in, 0in) to (2.2in, 0in);
        \draw[->] (0in, -0.1in) to (0in, 2.1in);

        %   Mark the axes with a few numbers.
        \draw (1in, 0.03in) -- (1in, -0.03in) node[below] {1};
        \draw (2in, 0.03in) -- (2in, -0.03in) node[below] {2};
        \draw (0.03in, 1in) -- (-0.03in, 1in) node[left] {1};
        \draw (0.03in, 2in) -- (-0.03in, 2in) node[left] {2};

        %   Draw the curve y^2 = x.
        \draw[rotate=90, draw=blue, ->-] (0in, 0in) parabola (1.4141in, -2in);

        %   Draw perpendicular lines to indicate the position of the point.
        \draw[densely dashed] (2in, 0in) -- (2in, 1.4141in);
        \draw[densely dashed] (0in, 1.4141in) -- (2in, 1.4141in);

        %   Label the function.
        \node at (1in, 0.8in) {$y^{2}=x$};
    \end{tikzpicture}
\end{document}
