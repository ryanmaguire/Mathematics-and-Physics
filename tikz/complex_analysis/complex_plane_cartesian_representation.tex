%-----------------------------------LICENSE------------------------------------%
%   This file is part of Mathematics-and-Physics.                              %
%                                                                              %
%   Mathematics-and-Physics is free software: you can redistribute it and/or   %
%   modify it it under the terms of the GNU General Public License as          %
%   published by the Free Software Foundation, either version 3 of the         %
%   License, or (at your option) any later version.                            %
%                                                                              %
%   Mathematics-and-Physics is distributed in the hope that it will be useful, %
%   but WITHOUT ANY WARRANTY; without even the implied warranty of             %
%   MERCHANTABILITY or FITNESS FOR A PARTICULAR PURPOSE.  See the              %
%   GNU General Public License for more details.                               %
%                                                                              %
%   You should have received a copy of the GNU General Public License along    %
%   with Mathematics-and-Physics.  If not, see <https://www.gnu.org/licenses/>.%
%------------------------------------------------------------------------------%

% Use the standalone class for displaying the tikz image on a small PDF.
\documentclass[crop, tikz]{standalone}

% Import the tikz package to use for the drawing.
\usepackage{tikz}

% Needed for blackboard bold C.
\usepackage{amssymb}

% The arrows package is used for the LaTeX arrow.
\usetikzlibrary{arrows.meta}

% Begin the document.
\begin{document}

    % Draw the figure.
    \begin{tikzpicture}[%
        >=Latex,
        line width=0.2mm,
        line cap=round,
        scale=1.7
    ]

        % Coordinates for the various points.
        \coordinate (O)   at (0.0, 0.0);
        \coordinate (z)   at (2.3, 2.1);
        \coordinate (z_x) at (2.3, 0.0);
        \coordinate (z_y) at (0.0, 2.1);
        \coordinate (C)   at (3.0, 3.0);

        % Axes:
        \begin{scope}[thick]
            \draw[->] (-0.5,  0.0) to (4.4, 0.0) node [above] {$\Re\{z\}$};
            \draw[->] ( 0.0, -0.5) to (0.0, 4.4) node [right] {$\Im\{z\}$};
        \end{scope}

        % Axes labels:
        \foreach\n in {1,2,3,4}{%
            \draw (\n, 3pt) to (\n, -3pt) node [below] {$\n$};
            \draw (3pt, \n) to (-3pt, \n) node [left] {$\n{i}$};
        }

        % Draw a line from the origin to the point z.
        \draw (O) to (z);

        % Scope for dashed lines.
        \begin{scope}[densely dashed, thin]
            \draw (z_x) to (z);
            \draw (z_y) to (z);
        \end{scope}

        % Draw a point to denote z.
        \draw[fill=black] (2.3, 2.1) circle (0.4mm);

        % Nodes for labeling.
        \node at (C)          {\Large{$\mathbb{C}$}};
        \node at (z) [above]  {$z=(x,\,y)$};
    \end{tikzpicture}
\end{document}
