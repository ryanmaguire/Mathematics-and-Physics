%--------------------------------Dependencies----------------------------------%
%   amssymb                                                                    %
%   tikz                                                                       %
%       arrows.meta                                                            %
%   Int DeclareMathOperator                                                    %
%       \DeclareMathOperator{\Int}{Int}                                        %
%-------------------------------Main Document----------------------------------%
\begin{tikzpicture}[%
    >=Latex,
    line width=0.2mm,
    line cap=round,
    scale=2.5
]

    % Coordinates for the points that define the frame of the figure.
    \coordinate (P1) at (0.3, 0.3);
    \coordinate (P2) at (1.8, 0.3);
    \coordinate (P3) at (1.9, 0.8);
    \coordinate (P4) at (1.8, 1.5);
    \coordinate (P5) at (1.8, 1.8);
    \coordinate (P6) at (0.5, 1.8);
    \coordinate (P7) at (0.7, 1.0);

    % Axes:
    \begin{scope}[thick]
        \draw[->] (-0.2, 0) to (2, 0) node [above] {$\Re\{z\}$};
        \draw[->] (0, -0.2) to (0, 2) node [right] {$\Im\{z\}$};
    \end{scope}

    % Draw the simple region.
    \draw[blue,fill=cyan,opacity=0.7] (P1) to                  (P2)
                                           to [out=30,in=-80]  (P3)
                                           to [out=100,in=-45] (P4)
                                           to [out=135,in=-30] (P5)
                                           to                  (P6)
                                           to [out=-80,in=90]  (P7)
                                           to [out=-90,in=90]  cycle;

    % Nodes to label the Jordan curve and its interior.
    \node at (0.5,1)   {$\Gamma$};
    \node at (1.2,1.2) {$\interior[]{\Gamma}$};
\end{tikzpicture}
