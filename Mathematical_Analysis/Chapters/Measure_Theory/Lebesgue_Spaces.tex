\section{Lebesgue Spaces}
    \begin{ldefinition}{Lebesgue Number}
          {Funct_Analysis_Lebesgue_Number}
        A Lebesgue Number of an open cover
        $\mathcal{O}$ of a metric space $(X,d)$ is
        a non-zero number $d>0$ such that, for all
        $x\in{X}$, there exists
        a $\mathcal{U}\in\mathcal{O}$ such that:
        \begin{equation}
            B_{d}^{(X,d)}(x)\subseteq\mathcal{U}
        \end{equation}
    \end{ldefinition}
    \begin{lexample}
        Let $X=\mathbb{R}$, and let $d$ be the
        standard metric. Let
        $\mathcal{O}=\{\mathcal{U}_{i}:i=1,2,3\}$ where:
        \begin{equation}
            \mathcal{U}_{1}=(-\infty,1)
            \quad\quad
            \mathcal{U}_{2}=(0,2)
            \quad\quad
            \mathcal{U}_{3}=(1,\infty)
        \end{equation}
        Then $d=1/2$ is a Lebesgue number of this cover.
        Letting $X=(0,1)$ with the standard metric, for all
        $x\in{X}$ the is a $\delta_{x}>0$ such that:
        \begin{equation}
            B_{\delta_{x}}^{(X,d)}(x)
            \subseteq{X}
        \end{equation}
        And thus these open balls are a covering of the unit
        interval, but this covering has no Lebesgue number.
    \end{lexample}
    \begin{ltheorem}{Lebesgue Covering Lemma}
          {Funct_Analysis_Lebesgue_Covering_Lemma}
        If $(X,d)$ is a compact metric space, and if
        $\mathcal{O}$ is an open covering of $X$, then
        $\mathcal{O}$ has a Lebesgue number.
    \end{ltheorem}
    \begin{proof}
        Suppose not. Suppose $(X,d)$ is compact, and suppose
        that $\mathcal{O}$ is a covering of $X$ with no
        Lebesgue number. But then, for all $n\in\mathbb{N}$,
        there is an $a_{n}$ such that, for all
        $\mathcal{U}\in\mathcal{O}$:
        \begin{equation}
            B_{1/n}^{(X,d)}(a_{n})\not\subset\mathcal{U}
        \end{equation}
        But $X$ is compact, and thus there is a convergent
        subsequence such that $a_{k_{n}}\rightarrow{X}$.
        But then there is a $\mathcal{U}\in\mathcal{O}$ such
        that $x\in\mathcal{U}$. But $\mathcal{U}$ is open,
        and thus there is an $r>0$ such that:
        \begin{equation}
            B_{r}^{(X,d)}(x)\subseteq\mathcal{U}
        \end{equation}
        Let $N\in\mathbb{N}$ be such that, for all
        $k_{n}>N$, $d(x_{k_{n}},x)<r/2$. Let
        $n>N$ be such that $1/k_{n}<r/2$. But then:
        \begin{equation}
            B_{1/k_{n}}(a_{k_{n}})\subseteq\mathcal{U}
        \end{equation}
        A contradiction.
    \end{proof}
    \begin{theorem}
        If $(X.d)$ is a compact metric space, if $(Y,\rho)$
        is a metric space, and if $f:X\rightarrow{Y}$ is a
        continuous function, then $f$ is
        uniformly continuous.
    \end{theorem}
    \begin{proof}
        For let $\varepsilon>0$. since $f$ is
        continuous, for all $x\in{X}$ there is a
        $\delta_{x}$ such that, for all $y\in{X}$ such
        that $d(x,y)<\delta_{x}$, it is true that
        $\rho(f(x),f(y))<\varepsilon/2$. But then:
        \begin{equation}
            X\subseteq
                \bigcup_{x\in{X}}B_{\delta_{x}}^{(X,d)}(x)
        \end{equation}
        But $X$ is compact, and thus this covering has a
        Lebesgue number. Let $\delta$ be such a
        Lebesgue number. But then if $d(x,y)<\delta$,
        then there is a $z\in{X}$ such that
        $x,y\in{B}_{\delta_{z}}(z)$. But then:
        \begin{equation}
            \rho(f(x),f(x))\leq
            \rho(f(x),f(z))+\rho(f(z),f(y))
            <\varepsilon
        \end{equation}
    \end{proof}
    Let $(\Omega,\mathcal{A},\mu)$ be a measure space and let
    $f\geq{0}$ be measurable. From before we were able to define
    the integral of $f$ is $\mu$ is $\sigma\textrm{-finite}$. We
    approximate $f$ with an increasing sequence of simple functions
    that are also non-negative. The integral of $f$ is defined as
    the limit of the integrals of the approximating
    simple functions. That is, we define the integral to be:
    \begin{equation}
        \int_{\Omega}f\diff{\mu}=
        \underset{n\rightarrow\infty}{\lim}
        \int_{\Omega}f_{n}\diff{\mu}
    \end{equation}
    We have seen from a previous theorem that the value
    of the integral is independent of the approximating
    sequence. That is, for $f_{n}$ and $g_{n}$ are a
    sequence of simple functions that are monotonically
    increasing to $f$, then:
    \begin{equation}
        \underset{n\rightarrow\infty}{\lim}
        \int_{\Omega}f_{n}\diff{\mu}=
        \underset{n\rightarrow\infty}{\lim}
        \int_{\Omega}g_{n}\diff{\mu}
    \end{equation}
    We then proved the monotone convergence theorem.
    \begin{ltheorem}{Monotone Convergence Theorem}
        If $f_{n}$ is a sequence of positive measurable functions,
        not necessarily simple, and if $f_{n}$ is monotonically
        increasing, then:
        \begin{equation}
            \underset{n\rightarrow\infty}{\lim}
            \int_{\Omega}f_{n}\diff{\mu}
            =\int_{\Omega}
            \underset{n\rightarrow\infty}{\lim}f_{n}\diff{\mu}
        \end{equation}
    \end{ltheorem}
    Note that we are still only talking about non-negative measurable
    functions. We have yet to discuss functions that are possibly
    negative.
    \begin{ltheorem}{Fatou's Theorem}
        If $f_{n}$ is a sequence of non-negative measurable functions,
        then:
        \begin{equation}
            \int_{\Omega}
            \underset{n\rightarrow\infty}{\underline{\lim}}
            f_{n}\diff{\mu}
            \leq
            \underset{n\rightarrow\infty}{\underline{\lim}}
            \int_{\Omega}f_{n}\diff{\mu}
        \end{equation}
        Where $\underline{\lim}$ denotes the limit-inferior.
    \end{ltheorem}
    \begin{proof}
        For:
        \begin{equation}
            0\leq\inf_{k\geq{n}}f_{k}(\omega)
            \leq{f}_{k}(\omega)
        \end{equation}
        And therefore:
        \begin{equation}
            \int_{\Omega}\inf_{k\geq{n}}f_{k}\diff{\mu}
            \leq\int_{\Omega}f_{k}\diff{\mu}
        \end{equation}
        And therefore:
        \begin{equation}
            \int_{\Omega}\inf_{k\geq{n}}f_{k}\diff{\mu}
            \leq\inf_{k\geq{n}}\int_{\Omega}f_{k}\diff{\mu}
        \end{equation}
        But:
        \begin{equation}
            \underset{n\rightarrow\infty}{\lim}
            \int_{\Omega}\inf_{k\geq{n}}f_{k}\diff{\mu}
            \leq
            \underset{n\rightarrow\infty}{\lim}
            \inf_{k\geq{n}}\int_{\Omega}f_{k}\diff{\mu}
            =\underset{n\rightarrow\infty}{\underline{\lim}}
            \int_{\Omega}f_{n}\diff{\mu}
        \end{equation}
        Therefore, etc.
    \end{proof}
    \begin{theorem}
        If $f_{n}$ is a sequence of non-negative measurable functions,
        then the function $f$ defined by:
        \begin{equation}
            f=\underset{N\rightarrow\infty}{\lim}
            \sum_{n=0}^{N}f_{n}
        \end{equation}
        Is measurable.
    \end{theorem}
    \begin{theorem}
        If $f_{n}$ is a sequence of non-negative measurable functions
        and if $f$ is defined by:
        \begin{equation}
            f=\underset{N\rightarrow\infty}{\lim}
            \sum_{n=0}^{N}f_{n}
        \end{equation}
        Then:
        \begin{equation}
            \int_{\Omega}f\diff{\mu}=
            \underset{N\rightarrow\infty}{\lim}
            \sum_{n=0}^{N}\int_{\Omega}f_{n}\diff{\mu}
        \end{equation}
    \end{theorem}
    \begin{theorem}
        If $(\Omega,\mathcal{A},\mu)$ is a measure space,
        if $f$ is measurable and non-negative, and if
        $\nu:\mathcal{A}\rightarrow\mathbb{R}$ is defined by:
        \begin{equation}
            \nu(E)=\int_{E}f\diff{\mu}=
            \int_{\Omega}f_{E}\diff{\mu}
        \end{equation}
        Then $\nu$ is a measure on $\mathcal{A}$.
    \end{theorem}
    \begin{proof}
        For $\mu(\emptyset)=0$ by definition. Since $f$ is positive,
        for all $E\in\mathcal{A}$:
        \begin{equation}
            \nu(E)=\int_{\Omega}f_{E}\diff{\mu}\geq{0}
        \end{equation}
        And finally, if $E_{n}$ are pairwise disjoint then:
        \begin{equation}
            \nu\Big(\bigcup_{n=1}^{\infty}E_{n}\Big)=
            \int_{\bigcup_{n=1}^{\infty}E_{n}}f\diff{\mu}
            =\sum_{n=1}^{\infty}\int_{E_{n}}f\diff{\mu}
            =\sum_{n=1}^{\infty}\nu(E_{n})
        \end{equation}
        Therefore, etc.
    \end{proof}
    \begin{theorem}
        If $(\Omega,\mathcal{A},\mu)$ is a measure space,
        if $f$ is measurable and non-negative, if
        $\nu:\mathcal{A}\rightarrow\mathbb{R}$ is defined by:
        \begin{equation}
            \nu(E)=\int_{E}f\diff{\mu}=
            \int_{\Omega}f_{E}\diff{\mu}
        \end{equation}
        And if $E\in\mathcal{A}$ is such that $\mu(E)=0$, then
        $\nu(E)=0$.
    \end{theorem}
    \begin{ldefinition}{Absolute Continuity}
        An absolutely continuous measure $\nu$ with respect
        to a measure space $(\Omega,\mathcal{A},\mu)$ is a meausre
        $\nu$ on $\mathcal{A}$ such that for all $E\in\mathcal{A}$
        such that $\mu(E)=0$, it is true that $\nu(E)=0$. This is
        denoted $\nu<<\mu$.
    \end{ldefinition}
    \begin{ltheorem}{Radon-Nikodym Theorem}
        If $(\Omega,\mathcal{A},\mu)$ is a measure space and if
        $\nu$ is absolutely continuous with respect to
        $(\Omega,\mathcal{A},\nu)$. then there is a measurable
        non-negative function $f$ such that, for all $E\in\mathcal{A}$:
        \begin{equation}
            \nu(E)=\int_{E}f\diff{\mu}
        \end{equation}
    \end{ltheorem}
    The function $f$ in the previous theorem is often called the
    density of $\nu$ against $\mu$, or the
    Radon-Nikodym derivative of $\nu$ with respect to $\mu$. The
    function $f$ is unique $\mu$ almost everywhere.