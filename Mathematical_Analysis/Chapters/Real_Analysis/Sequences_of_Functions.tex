\section{Sequences of Functions}
    \begin{definition}
        A sequence of functions from a
        set $X$ to a set $Y$ is a function
        $F:\mathbb{N}\times{X}\rightarrow{Y}$.
        We often write the image of
        $(n,x)\in\mathbb{N}\times{X}$ as
        $F(n,x)=F_{n}(x)$.
    \end{definition}
    \begin{definition}
        A point-wise convergent
        sequence of real-valued functions on
        a subset $S\subseteq\mathbb{R}$
        is a function
        $F:\mathbb{N}\times{S}\rightarrow\mathbb{R}$
        such that there exists a function
        $f:S\rightarrow\mathbb{R}$ such that,
        for all $\varepsilon>0$ and for all
        $x\in{S}$, there exists an
        $N\in\mathbb{N}$ such that, for all
        $n>N$, $|F_{n}(x)-f(x)|<\varepsilon$.
        That is:
        \begin{equation}
            \label{eqn:FUNCTIONAL_ANALYSIS:POINTWISE_CONV_DEF}
            \forall_{\varepsilon>0}
            \forall_{x\in{S}}
            \exists_{N\in\mathbb{N}}:
            n>N\Rightarrow
            |f(x)-F_{n}(x)|<\varepsilon
        \end{equation}
    \end{definition}
    That is, a sequence $F$ converges point-wise
    to $f$ if, for all $x\in\mathbb{R}$,
    $F_{n}(x)\rightarrow{f(x)}$.
    Uniform continuity requires that all of the
    points of the domain converge to $f(x)$ at
    the same speed. That is, given any $\varepsilon>0$
    there is an $N\in\mathbb{N}$ that works for
    all points. Point-Wise convergence may not
    have this property.
    \begin{definition}
        A uniformly convergent
        sequence of real-valued functions on
        a subset $S\subseteq\mathbb{R}$
        is a function
        $F:\mathbb{N}\times{S}\rightarrow\mathbb{R}$
        such that there exists a function
        $f:S\rightarrow\mathbb{R}$ such that,
        for all $\varepsilon>0$ there exists
        an $N\in\mathbb{N}$ such that, for all
        $x\in{S}$ and for all
        $n>N$, $|F_{n}(x)-f(x)|<\varepsilon$.
        That is:
        \begin{equation}
            \label{eqn:FUNCTIONAL_ANALYSIS:UNIFORM_CONV_DEF}
            \forall_{\varepsilon>0}
            \exists_{N\in\mathbb{N}}
            \forall_{x\in{S}}:
            n>N\Rightarrow
            |f(x)-F_{n}(x)|<\varepsilon
        \end{equation}
    \end{definition}
    That is, $F_{n}\rightarrow{f}$
    point-wise if for all $x$,
    $|F_{n}(x)-f(x)|\rightarrow{0}$ and
    $F_{n}\rightarrow{f}$ uniformly if
    $\sup\{|F_{n}(x)-f(x)|\}\rightarrow{0}$.
    It is worthwhile spotting the very subtle
    difference between
    expressions~\ref{eqn:FUNCTIONAL_ANALYSIS:POINTWISE_CONV_DEF}
    and~\ref{eqn:FUNCTIONAL_ANALYSIS:UNIFORM_CONV_DEF}.
    \begin{definition}
        A limit function on a subset
        $S\subseteq\mathbb{R}$ of a convergent sequence
        of real-valued functions
        $F:\mathbb{N}\times{S}\rightarrow\mathbb{R}$
        is a function $f:S\rightarrow\mathbb{R}$ such that,
        for all $x\in{S}$, $F_{n}(x)\rightarrow{f(x)}$.
    \end{definition}
    \begin{theorem}
        If $S\subseteq\mathbb{R}$, if
        $F:\mathbb{N}\times{S}\rightarrow\mathbb{R}$
        is a convergent sequence of real-valued function,
        and if $f,g:S\rightarrow\mathbb{R}$ are limit
        functions of $F$, then $f=g$.
    \end{theorem}
    \begin{proof}
        For suppose not. Suppose there is an
        $x\in{S}$ such that $f(x)\ne{g(x)}$.
        But $F_{n}(x)\rightarrow{f(x)}$ and
        $F_{n}(x)\rightarrow{g(x)}$. From the
        uniqueness of limits, $f(x)=g(x)$, 
        a contradiction. Therefore, etc.
    \end{proof}
    \begin{example}
        Let
        $F:\mathbb{N}\times[0,1]\rightarrow\mathbb{R}$
        be defined by $F_{n}(x)=nx\exp(-nx)$.
        $F_{n}(x)\rightarrow{0}$ for all $x\in[0,1]$,
        and therefore $F$ converges point-wise to zero.
        Note that $F_{n}'(x)=(n-n^{2}x)\exp(-nx)$.
        This has a zero at $x=n^{-1}$, so
        $F_{n}(x)$ has a maximum of $e^{-1}$. But then
        $\sup|F_{n}(x)-f(x)|=\sup|F_{n}(x)|=e^{-1}$.
        So $F_{n}(x)$ does not converge
        \textit{uniformly} to $0$. The
        convergence is only \textit{point-wise}.
    \end{example}
    \begin{example}
        Let $F_{n}(x)=n^{2}x\exp(-nx)$. Then
        $F_{n}(x)\rightarrow{0}$ for all
        $x\geq{0}$. But $F_{n}(x)$ has
        a maximum of $ne^{-1}$ at $x=n^{-1}$.
        Thus $F_{n}(n^{-1})\rightarrow\infty$.
        It is possible for a sequence
        of functions to converge point-wise to zero
        and for there to be a sequence such that
        $F_{n}(x_{n})\rightarrow\infty$. Uniform
        convergence does not allow this.
    \end{example}
    \begin{definition}
        A sequence of continuous real-valued functions
        on a subset $S\subseteq\mathbb{R}$ is a
        function
        $F:\mathbb{N}\times{S}\rightarrow\mathbb{R}$
        such that, for all $n\in\mathbb{N}$, the function
        $g:S\rightarrow\mathbb{R}$ defined by
        $g(x)=F_{n}(x)$ for all $x\in{S}$,
        is continuous.
    \end{definition}
    \begin{theorem}
        If $S\subseteq\mathbb{R}$ and if
        $F:\mathbb{N}\times{S}\rightarrow\mathbb{R}$
        is a uniformly convergent sequence of
        real-valued continuous functions and if
        ${f}$ is the limit function of $F$, then
        $f$ is continuous.
    \end{theorem}
    \begin{proof}
        For let $x\in{S}$ and let
        $\varepsilon>0$. As $F$ converges
        uniformly to $f$, there is an
        $N_{0}\in\mathbb{N}$ such that, for all
        $n>N_{0}$ and for all $x_{0}\in{S}$,
        $|F_{n}(x_{0})-f(x_{0})|<\varepsilon/3$.
        Let $N=N_{0}+1$.
        But, for all $n\in\mathbb{N}$,
        $F_{n}$ is a continuous function, and
        thus there is a $\delta>0$ such that,
        for all $x_{0}\in{S}$ such that
        $|x-x_{0}|<\delta$,
        $|F_{N}(x)-F_{N}(x_{0})|<\varepsilon/3$.
        But then, from the triangle inequality,
        for all $x_{0}\in{S}$ such that
        $|x-x_{0}|<\delta$:
        \begin{align}
            \nonumber
            |f(x)-f(x_{0})|&\leq
            |f(x)-F_{N}(x)|
            +|F_{N}(x)-F_{N}(x_{0})|
            +|F_{N}(x_{0})-f(x_{0})|\\
            &<\frac{\varepsilon}{3}+
            \frac{\varepsilon}{3}+
            \frac{\varepsilon}{3}
            =\varepsilon
        \end{align}
    \end{proof}
    The word ``uniformly,'' is crucial.
    This theorem is not necessarily true of
    point-wise converging functions. Let $F$ be
    defined on $[0,1]$ as
    $F_{n}(x)=x^{n}$. Then $F$ converges to
    $0$ if $x\ne{1}$, and $1$ if $x=1$. Thus,
    the limit function is discontinuous. This is
    possible because the convergence is
    point-wise and not uniform.
    \begin{theorem}
            \label{thm:funct:Weak_Weierstrass_%
                   Approx_Theorem}
            If $f:[0,1]\rightarrow\mathbb{R}$
            is continuous,
            and if $f(0)=f(1)=0$,
            then there is a sequence of polynomials
            $F$ such that $F_{n}\rightarrow{f}$
            uniformly on $[0,1]$.
        \end{theorem}
    \begin{proof}
        \begin{subequations}
            Extend $f$ to be zero outside of $[0,1]$. Let
            $Q_{n}(x)=c_{n}(1-x^{2})^{n}$ on $[-1,1]$,
            and choose $c_{n}$ such that
            $\int_{-1}^{1}Q_{n}(x)\diff{x}=1$. So we have:
            \begin{align}
                c_{n}\int_{-1}^{1}(1-x^{2})^{n}\diff{x}
                &=2c_{n}\int_{0}^{1}(1-x^{2})^{n}\diff{x}\\
                &=2c_{n}\int_{0}^{1}(1-x)^{n}(1+x)^{n}\diff{x}\\
                &\geq{2c_{n}}\int_{0}^{1}(1-x)^{n}\diff{x}\\
                &=\frac{2}{n+1}c_{n}
            \end{align}
            From this we have that $c_{n}\leq{n+1}$. Let
            $f_{n}(x)=\int_{0}^{1}f(t)Q_{n}(x-t)\diff{x}$.
            Then $f_{n}(x)$ is a polynomial. Note that
            $f(t)Q_{n}(x-t)$ is roughly zero when $t$ differs
            from $x$ and $n$ is large enough. So we have:
            \begin{equation}
                f_{n}(x)=
                \int_{0}^{1}f(t)Q_{n}(x-t)\diff{t}
                \approx{f(x)}\int_{0}^{1}Q_{n}(x-t)\diff{t}
                =f(x)
            \end{equation}
            The remainder of the proof is to quantify this.
            Since $f$ is zero outside of $[0,1]$, if
            we let $s=t-x$, then:
            \begin{align}
                f_{n}(x)
                &=\int_{-x}^{1-x}f(s+x)Q_{n}(s)\diff{s}\\
                &=\int_{-1}^{1}f(s+x)Q_{n}(s)\diff{s}
            \end{align}
            Using this, we obtain:
            \begin{align}
                |f_{n}(x)-f(x)|
                &=\bigg|\int_{-1}^{1}f(x+t)Q_{n}(t)\diff{t}
                -\int_{-1}^{1}f(x)Q_{n}(t)\diff{t}\bigg|\\
                &\leq\int_{-1}^{1}|f(x+t)-f(x)|Q_{n}(t)\diff{t}
            \end{align}
            This comes for the fact
            that $\int_{-1}^{1}Q_{n}(t)=1$
            and from the integral version of the triangle
            inequaility.
            Suppose $\varepsilon>0$. Since $f$ is continuous
            on $[0,1]$, it is uniformly continuous. But
            if $f$ is uniformly continuous then there exists
            a $\delta>0$ such that
            $|f(x+t)-f(x)|<\frac{\varepsilon}{2}$ for all
            $t<\delta$. So we have:
            \begin{equation}
                \begin{split}
                |f_{n}(x)-f(x)|
                &\leq
                \int_{-1}^{-\delta}|f(x+t)-f(x)|
                Q_{n}(t)\diff{t}\\
                &+\int_{-\delta}^{\delta}
                |f(x+t)-f(x)|Q_{n}(t)\diff{t}\\
                &+\int_{\delta}^{1}
                |f(x+t)-f(x)|Q_{n}(t)\diff{t}
                \end{split}
            \end{equation}
            But $f$ is continuous on a closed and bounded
            set and therefore $f$ is bounded. Let $M$ be
            such a bound. Then $|f(x+t)-f(x)|\leq{2M}$.
            We have:
            \begin{equation}
                |f_{n}(x)-f(x)|\leq
                2M\int_{-1}^{-\delta}Q_{n}(t)\diff{t}
                +\frac{\varepsilon}{2}
                \int_{-\delta}^{\delta}Q_{n}(t)\diff{t}
                +2M\int_{\delta}^{1}Q_{n}(t)\diff{t}
            \end{equation}
            But for all $t\in[-1,-\delta]$,
            $Q_{n}(t)\leq{Q_{n}(-\delta)}$. Similarly for
            $t$ in $[\delta,1]$. Since $Q_{n}(t)$
            is an even function:
            \begin{equation}
                |f_{n}(x)-f(x)|\leq
                4MQ_{n}(\delta)+
                \frac{\varepsilon}{2}
                \int_{-1}^{1}Q_{n}(t)\diff{t}
                =4MQ_{n}(\delta)+\frac{\varepsilon}{2}
            \end{equation}
            But since $\delta>0$, $Q_{n}(\delta)\rightarrow0$.
            Therefore, there is an $N\in\mathbb{N}$ such that
            for all $n>N$,
            $|Q_{n}(\delta)|<\frac{\varepsilon}{8M}$.
            But then $4MQ_{n}(\delta)<\frac{\varepsilon}{2}$.
            Therefore, etc.
        \end{subequations}
    \end{proof}
    Another way to put this is that if $f$ is continuous
    on $[a,b]$ and if $\varepsilon>0$, then there is
    a polynomial $P$ such that for all $x\in[0,1]$,
    $|f(x)-P(x)|<\varepsilon$. There is a generalization
    of this and the set of functions need not be
    polynomials. The set needs to be closed
    under addition, multiplication, and scalar
    multiplication, it must separate points,
    and must not take every point to zero.
    This is the Stone-Weierstrass theorem.
    This shows that continuous functions on compact sets
    can be approximated arbitrarily well by polynomials.
    Furthermore, any continuous function on a compact
    set can be approximated arbitrarily well by
    polynomials with rational coefficients. To see this,
    let $f[0,1]\rightarrow\mathbb{R}$ be continuous,
    and let $\varepsilon>0$. Then there is a
    polynomial $P$ such that
    $\sup|P(x)-f(x)|<\varepsilon/2$.
    Suppose $P$ is of degree $n$.
    As $\mathbb{Q}$ is dense
    in $\mathbb{R}$, for each coefficient
    $c_{k}$ of $P$ there is a $d_{k}\in\mathbb{Q}$
    such that
    $|c_{k}-d_{k}|<\frac{\varepsilon}{2n}$.
    Let $Q(x)=\sum{d_{k}x^{k}}$. Then:
    \begin{equation}
            |P(x)-Q(x)|\leq
            \sum_{k=0}^{n}|c_{k}-d_{k}||x|^{n}
            <\frac{\varepsilon}{2}
        \end{equation}
    Thus, by the triangle inequality:
    $\sup|Q(x)-f(x)|<\varepsilon$. A set is called
    \textit{separable} if it
    contains a countable dense subset. $\mathbb{R}$
    is separable since $\mathbb{Q}$ is dense in
    $\mathbb{R}$, and $\mathbb{Q}$ is countable.
    The set of all continuous functions from
    $[0,1]$ to $\mathbb{R}$, which we label as
    $C(I,\mathbb{R})$, is also separable.
    Since any continuous function can be approximated
    arbitrarily well by a polynomial with rational
    coefficients, we can say the set of polynomials
    with rational coefficients is \textit{dense} in
    $C(I,\mathbb{R})$. But the set of polynomials with
    rational coefficients is countable. For all
    $N\in\mathbb{N}$, define $P_{N}$ as:
    \begin{equation}
            P_{N}=\Big\{\sum_{k=0}^{N}
            q_{k}x^{k}:q_{k}\in\mathbb{Q},
            q_{N}\ne{0}\Big\}
        \end{equation}
    This is the set of all rational polynomials
    of degree $N$. It is countable since there is
    a one-to-one correspondence with
    the set $\mathbb{Q}^{N}$, and $\mathbb{Q}^{N}$
    is countable for all $N\in\mathbb{N}$. But the
    set of all rational polynomials is simply the
    union over all $P_{N}$. And the countable union
    of countably many disjoint sets is countable.
    Therefore, the set of all polynomials with
    rational coefficients is countable. Thus
    $C(I,\mathbb{R})$ is \textit{separable}. We need
    to be careful when we say \textit{dense} and
    \textit{separable}, for we are implicitly speaking
    of some sort of notion of \textit{closeness} on the
    sets. This all comes from the notion of
    \textit{metrics} and \textit{metric spaces},
    and the more general \textit{topological space}.
    \begin{theorem}
        \label{thm:Funct:Weierstrass_%
               Approx_on_unit_interval}
        If $f:[0,1]\rightarrow\mathbb{R}$ is
        continuous, then there is a sequence
        of polynomials $F$ such that
        $F_{n}\rightarrow{f}$ uniformly on $[0,1]$.
    \end{theorem}
    \begin{proof}
        If $f:[0,1]\rightarrow\mathbb{R}$ be
        continuous, let
        $g(x)=xf(1)+(1-x)f(0)$. Then
        $h(x)=f(x)-g(x)$ is a continuous function
        such that $h(0)=h(1)=0$ and thus by
        Thm.~\ref{thm:funct:Weak_Weierstrass_%
                  Approx_Theorem}
        there is a sequence of polynomials
        $P_{n}(x)$ such that
        $P_{n}(x)\rightarrow{h(x)}$
        uniformly on $[a,b]$.
        But $g(x)$ is a polynomial and
        $f(x)=h(x)+g(x)$. Therefore
        $F_{n}(x)=P_{n}(x)+g(x)$ is a sequence
        of polynomials and
        $F_{n}(x)\rightarrow{f(x)}$
        uniformly on $[0,1]$.
    \end{proof}
    \begin{theorem}[Weierstrass Approximation Theorem]
        If $f:[a,b]\rightarrow\mathbb{R}$ is a
        continuous function, then there is a sequence
        of polynomials $P$ such that
        $P_{n}\rightarrow{f}$ uniformly.
    \end{theorem}
    \begin{proof}
        If $f:[a,b]\rightarrow\mathbb{R}$ is
        continuous, define
        $g:[0,1]\rightarrow\mathbb{R}$ by
        $g(x)=f(\frac{x-a}{b-a})$. Then, since
        the composition of continuous functions
        is continuous, $g$ is a continuou function
        on $[0,1]$. But by the Weierstrass
        approximation theorem there is a sequence
        of polynomials $P_{n}(x)$ such that
        $P_{n}(x)\rightarrow{g(x)}$. Let
        $F_{n}(x)=P_{n}(bx+(1-x)a)$. Then
        $F_{n}(x)$ is a sequence of polynomials
        on $[a,b]$, and $F_{n}(x)\rightarrow{f(x)}$.
    \end{proof}
    An application of this is in the uniform
    approximation of continuous periodic functions
    by Cosines.
\begin{theorem}
    If $f\in{C[0,\pi]}$ and $\varepsilon>0$,
    then there exists
    $a_{0},\hdots,a_{n}\in\mathbb{R}$ such that, for all
    $x\in[0,\pi]$:
    \begin{equation}
        |f(x)-\sum_{k=0}^{n}a_{k}\cos(kx)|
        <\varepsilon
    \end{equation}
\end{theorem}
\begin{proof}
    \begin{subequations}
        For $\cos(x)$ is a bijective function on
        the interval $[0,\pi]$. Thus we can consider the
        function $f(\cos^{-1}(x))$. But since $\cos(x)$ is
        continuous on $[0,\pi]$, $\cos^{-1}(x)$ is
        continuous on $[-1,1]$. And the composition of
        continuous functions is continuous. So
        $f(\cos^{-1}(x))$ is continuous. By the
        Weierstrass Approximation Theorem, there is a
        sequence of polynomials $P$ such that
        $P_{n}(x)\rightarrow{f(\cos^{-1}(x))}$. But then
        $P_{n}(\cos(x))\rightarrow{f(x)}$. But $P_{n}(x)$
        is a polynomial of the form
        $\sum_{k=0}^{n}a_{k}x^{k}$, and thus:
        \begin{equation}
            P_{n}(\cos(x))=\sum_{k=0}^{n}a_{k}\cos^{k}(x)    
        \end{equation}
        It now suffices to show that
        $\cos^{k}(x)=\sum_{m=0}^{N}c_{m}\cos(mx)$ for
        suitable $c_{m}$. We prove by induction.
        The base case is trivial. Suppose it holds
        for some $k\in\mathbb{N}$.
        Then:
        \begin{equation}
            \cos^{k+1}(x)=\cos(x)\cos^{k}(x)
            =\cos(x)\sum_{k=0}^{N}c_{k}\cos(kx)
        \end{equation}
        Note that
        $\cos(x)\cos(kx)%
         =\frac{1}{2}\cos((k-1)x)+\frac{1}{2}\cos((k+1)x)$.
        So we have:
        \begin{equation}
            \cos^{k+1}(x)
            =\frac{1}{2}\sum_{k=0}^{N}c_{k}
            \Big(
                \cos\big((k+1)x\big)+
                \cos\big((k-1)x\big)
            \Big)
        \end{equation}
        This completes the theorem.
    \end{subequations}
\end{proof}