\section{Order Theory}
    \subsection{Relations}
        \begin{ldefinition}{Relation on a Set}{Relation}
            A relation on a set $A$ is a subset $R$ of
            $A\times{A}$. That is, $R\subseteq{A}\times{A}$.
            For elements $(a,b)\in{R}$ we write $aRb$.
        \end{ldefinition}
        For a relation $R$ it is not necessary true that $aRb$
        implies $bRa$, nor is it necessarily true that $aRa$. These
        are called symmetric and reflexive relations, respectively.
        There are many basic properties that relations have, and we
        prove them now.
        \begin{theorem}
            \label{thm:Cartesian_Product_Is_Relation}%
            If $A$ is a set, then $A\times{A}$ is a relation on $A$.
        \end{theorem}
        \begin{proof}
            For if $A$ is a set, then
            $A\times{A}\subseteq{A}\times{A}$. Therefore, etc.
        \end{proof}
        \begin{theorem}
            \label{thm:Empty_Set_Is_Relation}%
            If $A$ is a set, and then $\emptyset$ is a relation
            on $A$.
        \end{theorem}
        \begin{proof}
            For if $A$ is a set, then
            $\emptyset\subseteq{A}\times{A}$. Therefore, etc.
        \end{proof}
        These provide the two most basic examples of relations on a
        set. The empty set is the relation that says no two elements
        are related. Indeed, even single elements are unrelated to
        themselves. The second, the entire Cartesian product
        $A\times{A}$, says that everything is related. These are the
        two extreme cases, but provide useful examples and
        counterexamples in various contexts. More useful is that the
        union and intersection of relations is also a relation. We
        prove this now.
        \begin{theorem}
            \label{thm:Intersection_of_Relations_Is_Relation}%
            If $A$ is a set and if $R_{1}$ and $R_{2}$ are relations
            on $A$, then $R_{1}\cap{R}_{2}$ is a relation on $A$.
        \end{theorem}
        \begin{proof}
            For let $R=R_{1}\cap{R}_{2}$ and suppose $R$ is not a
            relation on $A$. Then there is an $x\in{R}$ such that
            $x\notin{A}\times{A}$. But if $x\in{R}$ then
            $x\in{R}_{1}$ and $x\in{R}_{2}$. But for all
            $x\in{R}_{1}$, $x\in{A}\times{A}$, since $R_{1}$ is a
            relation on $A$, a contradiction as
            $x\notin{A}\times{A}$. Therefore, $R$ is a relation on
            $A$.
        \end{proof}
        \begin{theorem}
            \label{thm:Set_Theory_Union_of_Relations_Is_Relation}
            If $A$ is a set and if $R_{1}$ and $R_{2}$ are relations
            on $A$, then $R_{1}\cup{R}_{2}$ is a relation on $A$.
        \end{theorem}
        \begin{proof}
            For let $R=R_{1}\cup{R}_{2}$ and suppose $R$ is not a
            relation on $A$. Then there is an $x\in{R}$ such that
            $x\notin{A}\times{A}$. But if $x\in{R}$ then
            $x\in{R}_{1}$ or $x\in{R}_{2}$. But for all $x\in{R}_{1}$
            and for all $x\in{R}_{2}$,
            $x\in{A}\times{A}$, since $R_{1}$ and $R_{2}$ are
            relations on $A$, a contradiction. Therefore, etc.
        \end{proof}
        \begin{theorem}
            If $A$ is a set and $R$ is a relation on $A$, then there
            is a relation $\mathcal{U}$ on $A$ such that
            $R\cap\mathcal{U}=R$.
        \end{theorem}
        \begin{proof}
            For let $\mathcal{U}={A}\times{A}$. Then by
            Thm.~\ref{thm:Cartesian_Product_Is_Relation},
            $A\times{A}$ is a relation on $A$. But since $R$ is
            a relation, $R\subseteq{A}\times{A}$. But then
            $R\cap\mathcal{U}=R$. Therefore, etc.
        \end{proof}
        \begin{theorem}
            If $A$ is a set and $R$ is a relation on $A$, then there
            is a relation $\mathcal{U}$ on $A$ such that
            $R\cup\mathcal{U}=R$
        \end{theorem}
        \begin{proof}
            For let $\mathcal{U}=\emptyset$. Then by
            Thm.~\ref{thm:Empty_Set_Is_Relation},
            $\mathcal{U}$ is a relation. But if $R$ is a set, then
            $R\cup\emptyset=R$. Thus, $R\cup\mathcal{U}=R$.
            Therefore, etc.
        \end{proof}
        Since a general relation is simply a subset of $A\times{A}$,
        there's not much structure on them, and thus there's not a lot
        that can be said about them. We can add more constraints to
        certain relations to get the more familiar properties
        we're used to.
        \begin{ldefinition}{Reflexive Relations}
                           {Reflexive_Relations}
            A reflexive relation on a set $A$ is a
            relation $R$ on $A$ such that for all $a\in{A}$
            it is true that $aRa$.
        \end{ldefinition}
        A reflexive relation on $A$ is simply any subset of
        $A\times{A}$ that contains the entire \textit{diagonal}. That,
        all of the pairs $(a,a)$. A reflexive relation can contain more
        than this, however. The only strict requirement is that
        $aRa$ for all $a\in{A}$.
        \begin{theorem}
            If $A$ is a set, and if $R_{1}$ and $R_{2}$ are reflexive
            relations on $A$, then $R_{1}\cap{R}_{2}$ is a reflexive
            relation on $A$.
        \end{theorem}
        \begin{proof}
            For let $R=R_{1}\cap{R}_{2}$. Then by
            Thm.~\ref{thm:Intersection_of_Relations_Is_Relation},
            $R$ is a relation. Suppose $R$ is not reflexive.
            Then there is an $a\in{A}$ such that $(a,a)\notin{R}$.
            But if $a\in{A}$, then $(a,a)\in{R}_{1}$, since $R_{1}$
            is reflexive. Similarly, $(a,a)\in{R}_{2}$ since
            $R_{2}$ is reflexive. But if $(a,a)\in{R}_{1}$ and
            $(a,a)\in{R}_{2}$, then $(a,a)\in{R}$ since
            $R=R_{1}\cap{R}_{2}$, a contradiction. Therefore,
            $R$ is reflexive.
        \end{proof}
        \begin{theorem}
            If $A$ is a set, if $R_{1}$ is a reflexive relation on
            $A$, and if $R_{2}$ is a relation on $A$, then
            $R_{1}\cup{R}_{2}$ is a reflexive relation on $A$.
        \end{theorem}
        \begin{proof}
            For let $R=R_{1}\cup{R}_{2}$. Since $R_{1}$ and $R_{2}$ are
            relations, by
            Thm.~\ref{thm:Set_Theory_Union_of_Relations_Is_Relation},
            $R$ is a relation. Suppose it is not reflexive.
            Then there is an $a\in{A}$ such that
            $(a,a)\notin{R}$. But if $a\in{A}$ then $(a,a)\in{R}_{1}$
            since $R_{1}$ is reflexive. But if $(a,a)\in{R}_{1}$ then
            $(a,a)\in{R}_{1}\cup{R}_{2}$, a contradiction.
            Therefore, etc.
        \end{proof}
        Given an arbitrary relation $R$ on a set $A$, it may not be
        true that $R$ is reflexive. It may often be useful to add in
        only the necessary points of $A$ that will make $R$
        reflexive. This is called the reflexive closure of $R$.
        \begin{ldefinition}{Reflexive Closure of a Relation}
                           {Reflexive_Closure_of_Relation}
            The reflexive closure of a relation $R$ on a set $A$
            is the set:
            \begin{equation}
                S=R\cup\{(a,a):a\in{A}\}
            \end{equation}
        \end{ldefinition}
        \begin{theorem}
            If $A$ is a set, $R$ is a relation on $A$, and if $S$ is the
            reflexive closure of $R$, then $S$ is a reflexive relation on $A$.
        \end{theorem}
        \begin{theorem}
            \label{thm:Set_Theory_Refl_Clos_Is_Smallest_Refl_With_R}
            If $A$ is a set, if $R$ is a relation on $A$, if
            $S$ is the reflexive closure of $R$, and if $T$ is a
            reflexive relation on $A$ such that $R\subseteq{T}$, then
            $S\subseteq{T}$.
        \end{theorem}
        \begin{proof}
            For if $x\in{S}$, then either $x\in{R}$ or there is an
            $a\in{A}$ such that $x=(a,a)$. But if $x\in{R}$, then
            $x\in{T}$ since $R\subseteq{T}$. If $x\notin{R}$ then
            there is an $a\in{A}$ such that $x=(a,a)$. But $T$ is
            reflexive, and therefore $(a,a)\in{T}$. But then
            $x\in{T}$. Therefore, $S\subseteq{T}$.
        \end{proof}
        Thm.~\ref{thm:Set_Theory_Refl_Clos_Is_Smallest_Refl_With_R}
        says that the reflexive closure of a relation $R$ is, in a sense,
        the \textit{smallest} relation that is reflexive and contains
        $R$ as a subset.
        \begin{theorem}
            If $A$ is a set, $R_{1}$ and $R_{2}$ are relations on $A$,
            and if $S_{1}$ and $S_{2}$ are the reflexive closures of
            $R_{1}$ and $R_{2}$, respectively, then the reflexive closure
            of $R_{1}\cap{R}_{2}$ is:
            \begin{equation}
                S=S_{1}\cap{S}_{2}
            \end{equation}
        \end{theorem}
        \begin{proof}
            By the definition of reflexive closure, we have:
            \begin{align}
                S_{1}&=R_{1}\cup\{(a,a):a\in{A}\}
                \tag{Def.~\ref{def:Reflexive_Closure_of_Relation}}\\
                S_{1}&=R_{2}\cup\{(a,a):a\in{A}\}
                \tag{Def.~\ref{def:Reflexive_Closure_of_Relation}}\\
                \nonumber
                S_{1}\cap{S}_{2}&=
                (R_{1}\cup\{(a,a):a\in{A}\})
                \cap(R_{2}\cup\{(a,a):a\in{A}\})\\
                &=(R_{1}\cap{R}_{2})
                \cup\{(a,a):a\in{A}\}
                \tag{Distributive Law}
            \end{align}
            But by the definition of the transitive closure of
            $R_{1}\cap{R}_{2}$:
            \begin{equation}
                S=(R_{1}\cap{R}_{2})\cup\{(a,a):a\in{A}\}
                \tag{Def.~\ref{def:Reflexive_Closure_of_Relation}}
            \end{equation}
            Therefore, etc.
        \end{proof}
        \begin{ldefinition}{Symmetric Relations}
                           {Symmetric_Relations}
            A symmetric relation on a set $A$ is a
            relation $R$ on $A$ such that for all $a,b\in{A}$
            such that $aRb$, it is true that $bRa$.
        \end{ldefinition}
        \begin{ldefinition}{Transitive Relations}
                           {Transitive_Relations}
            A transitive relation on a set $A$ is a relation $R$ on $A$
            such that for all $a,b,c\in{A}$ such that $aRb$ and $bRc$,
            is it true that $aRc$.
        \end{ldefinition}
        \begin{ldefinition}{Transitive Closure}{Transitive_Closure}
            The transitive closure of a relation $R$ on a set
            $A$ is the the set $R^{t}\subseteq{A}\times{A}$ defined by:
            \begin{equation}
                R^{t}
            \end{equation}
        \end{ldefinition}
        \begin{ldefinition}{Asymmetric Relation}
                           {Asymmetric_Relation}
            An asymmetric relation on a set $A$ is a relation $R$
            on $A$ such that for all $a,b\in{A}$ such that $aRb$
            it is true that $(b,a)\notin{R}$.
        \end{ldefinition}
        \begin{ldefinition}{Total Relation}{Total_Relation}
            A total relation on a set $A$ is a relation $R$ on $A$ such
            that for all $a,b\in{A}$ it is true that either
            $aRb$ or $bRa$, or both.
        \end{ldefinition}
        The notion of equality can be defined as a relation
        with the following properties:
        \begin{enumerate}
            \item Equality is Reflexive: $a=a$ for all $a\in{A}$.
            \item Equality is Symmetric: $a=b$ if and only if $b=a$.
            \item Equality is Transitive: If $a=b$ and $b=c$, then $a=c$.
            \item The relation is uniquely defined by the set
                  $\{(a,a)\in A\times A:a\in A\}$.
        \end{enumerate}
        That is, equality can be seen as the \textit{diagonal} in the
        Cartesian product $A\times{A}$.
        \begin{ldefinition}{Antisymmetric Relation}
                           {Antisymmetric_Relation}
            An antisymmetric relation on a set $A$ is a relation $R$ on $A$
            such that for all $a,b\in{A}$ such that $aRb$ and $bRa$, it
            is true that $a=b$.
        \end{ldefinition}
    \subsection{Zorn's Lemma}
        A relation on a set $X$ is a subset
        $R\subseteq{X}\times{X}$. Given an element
        $(x,y)\in{R}$, we often write $xRy$ to denote this.
        Here we'll write $x\leq{y}$.
        \begin{ldefinition}{Ordered Sets}
              {Funct_Analysis_Ordered_Set}
            An ordered set is a set $X$ and a relation
            $\leq$ on $X$, denoted $(X,\leq)$ such that
            the following are true:
            \begin{enumerate}
                \item For all $x\in{X}$, $x\leq{x}$.
                \item For all $x,y\in{X}$ such that $x\leq{y}$ and
                      $y\leq{x}$, it is true that $x=y$.
                \item For all $x,y,z\in{X}$ such that $x\leq{y}$
                      and $y\leq{z}$, it is true that $x\leq{z}$.
            \end{enumerate}
        \end{ldefinition}
        \begin{ldefinition}{Majorants in Ordered Sets}{Majorant_in_Ord_Set}
            A majorant of a subset $Y\subseteq{X}$ of
            and ordered set $(X,\leq)$ is an element $x\in{X}$
            such that, for all $y\in{Y}$, it is true that
            $y\leq{x}$.
        \end{ldefinition}
        \begin{lexample}{}{Reverse_Containment_Example}
            Let $(X,d)$ be a metric space, and let
            $x_{0}\in{X}$. Define the following:
            \begin{equation}
                \mathscr{N}(x_{0})=
                \big\{\mathcal{V}\subseteq{X}:\mathcal{V}
                    \textrm{ is a neighborhood of $x_{0}$}\big\}
            \end{equation}
            We can order $\mathscr{N}$ by reverse containment.
            That is, We have the following relation:
            \begin{equation}
                \leq=\big\{(\mathcal{U},\mathcal{V})\in
                    \mathscr{N}(x_{0})\times\mathscr{N}(x_{0})
                    :\mathcal{V}\subseteq\mathcal{U}\big\}
            \end{equation}
            That is, we write $\mathcal{U}\leq\mathcal{V}$ if
            $\mathcal{V}$ is a subset of $\mathcal{U}$. Note
            that, for all $x_{0}$,
            $\mathscr{x_{0}}$ has a least element, or a minorant,
            but $X$ is such an element. But, if $\{x_{0}\}$ is
            not open, then there is no majorant.
        \end{lexample}
        \begin{ldefinition}{Totally Ordered Sets}
              {Funct_Analysis_Tot_Ord_Set}
            A totally ordered set is an ordered set
            $(X,\leq)$ such that, for all $x,y\in{X}$, either
            $x\leq{y}$ or $y\leq{x}$.
        \end{ldefinition}
        \begin{ldefinition}{Maximal Element}
              {Funct_Analysis_Maximal_Element}
            A maximal element of a subset $Y\subseteq{X}$ of
            a totally ordered set $(X,\leq)$ is an element $y$
            such that:
            \begin{equation}
                \{y'\in{Y}:y\leq{y}'\}=\{y\}
            \end{equation}
            Note that $y$ is not necessary a majorant for $Y$
            nor is $y$ necessarily unique.
        \end{ldefinition}
        \begin{ldefinition}{Inductively Ordered Sets}
              {Funct_Analysis_Induct_Ordered_Set}
            An inductively ordered set is an ordered set
            $)X,\leq)$ such that, for all totally ordered
            subsets $S\subseteq{X}$, there is a majorant
            $x\in{X}$ of $S$.
        \end{ldefinition}
        That is, there exists $x\in{X}$ such that, for all
        $y\in{S}$, $y\leq{x}$.
        \begin{lexample}{}{Trivial_Inductively_Ordered_Set}
            Let $X=\mathbb{R}$ and consider the set
            $\mathscr{N}(0)$. Then $\mathscr{N}(0)$ is
            not inductively ordered.
        \end{lexample}
        \begin{ltheorem}{Zorn's Lemma}{Zorns_Lemma}
            If $(X,\leq)$ is an inductively ordered set,
            then there is a maximal element $x\in{X}$.
        \end{ltheorem}