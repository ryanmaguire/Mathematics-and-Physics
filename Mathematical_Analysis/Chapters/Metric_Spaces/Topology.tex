\chapter{Topology}
    \section{Topological Spaces}
        Before moving into the heart of measure theory,
        we briefly discuss some basic notions from topological
        This will include the definition of a topological
        space, and some of the properties of continuous
        functions.
        \begin{ldefinition}{Topology}{Topology}
            A topology on a set $X$ is a subset
            $\tau\subseteq\mathcal{P}(X)$ such that
            $\emptyset\in\tau$, $X\in\tau$, and such that
            for any function $A:\mathbb{Z}_{n}\rightarrow\tau$,
            it is true that:
            \begin{equation}
                \bigcap_{k=1}^{n}A_{n}\in\tau
            \end{equation}
            And for any subset $\mathcal{O}\subseteq\tau$,
            it is true that:
            \begin{equation}
                \bigcup_{\mathcal{U}\in\mathcal{O}}
                    \mathcal{U}\in\tau
            \end{equation}
            That is, $\tau$ is closed to finite intersections
            and arbitrary unions.
        \end{ldefinition}
        \begin{ldefinition}{Topological Space}
              {Topological_Space}
            A topological space, denoted $(X,\tau)$, is a
            set $X$ and a topology $\tau$ on $X$.
        \end{ldefinition}
        Topological spaces generalize the notion of metric
        spaces, which are themselves a generalization of
        Euclidean spaces $\mathbb{R}^{n}$. A subset of
        $\mathcal{U}\subseteq\mathbb{R}^{n}$ is open if
        for all $\mathbf{x}\in\mathcal{U}$ there is an $r>0$
        such that:
        \begin{equation}
            B_{r}(\mathbf{x})=
            \{\mathbf{y}\in\mathbb{R}^{n}:
                \norm{\mathbf{x}-\mathbf{y}}_{2}<r\}
            \subseteq\mathcal{U}
        \end{equation}
        That is, one can find an open ball about a given point
        that sits inside of $\mathcal{U}$. In this context, open
        sets have the property that a finite intersection of
        open sets is open, and an arbitrary union of open sets is
        also open. Vacuously, one can show that the empty set is
        open, and $\mathbb{R}^{n}$ is open rather trivially.
        This gives the motivation for defining topological spaces
        in this way. It also motivates us to define what
        \textit{open} sets are in a general topological space.
        \begin{ldefinition}{Open Subsets}{Open_Subsets}
            An open subset of a topological space $(X,\tau)$
            is a set $\mathcal{U}\subseteq{X}$ such that
            $\mathcal{U}\in\tau$.
        \end{ldefinition}
    \section{Compactness}
        \begin{definition}
            A metric space $(X,d)$ is
            sequentially compact if every
            sequence in $X$ has a convergent
            subsequence.
        \end{definition}
        In topology there is a difference
        between sequential compactness
        and regular compactness, but in
        metric spaces they turn out
        to be the same.
        A subset of $S$ of $X$ is
        compact if every sequence in
        $S$ has a subsequence which converges.
        That is, $(S,d)$ is compact.
        \begin{theorem}
            A subset $S$ of a compact
            metric space $(X,d)$ is compact
            if and only if $S$ is closed.
        \end{theorem}
        \begin{proof}
            For let $x_{n}$ be a sequence
            in $S$. Then $x_{n}$ is a
            sequence in $X$ and thus there
            is a convergent subsequence
            $x_{k_{n}}$ with a limit $x$.
            But $x_{k_{n}}$ is in $S$ and
            $S$ is closed, and therefore
            $x$ is in $S$. Thus, $S$
            is compact. Conversely, if
            $S$ is compact, suppose it is
            not closed. Then there is a point
            $y\in{X}$ such that $y$ is a
            limit point of $S$ but not
            contained in $S$. Let
            $x_{n}$ be a sequence that
            converges to $y$. Then, as
            $S$ is compact, there is
            a convergent subsequence. But
            the limit of this subsequence
            is $y$, a contradiction as
            $y\notin{S}$. Therefore $S$
            is closed.
        \end{proof}
        \begin{theorem}
            If $(X,d)$ is a compact metric
            space, then
            $(X,d)$ is complete.
        \end{theorem}
        \begin{proof}
            If $x_{n}$ is Cauchy in $X$,
            then there is a convergent
            subsequence $x_{k_{n}}$
            in $X$. But if $x_{k_{n}}$
            converges to $x$, then
            $x_{n}$ converges to $x$ as
            well, as $x_{n}$ is Cauchy.
            Therefore, $(X,d)$ is complete.
        \end{proof}
        \begin{theorem}[Heine-Borel Theorem]
            A subset of
            $\mathbb{R}^{n}$ is
            compact if and only if
            it is closed and bounded.
        \end{theorem}
        \begin{example}
            The closed unit ball
            of $\ell^{p}$ is not compact,
            if $1\leq{p}\leq{\infty}$.
            Let $x_{n}(m)$ be the sequence
            (of sequences) such that
            $x_{n}(m)=1$ if $n=m$, and
            zero otherwise. Then
            $d_{p}(x_{n},x_{m})=2^{1/p}$,
            so $x_{n}$ has no subsequence
            which is Cauchy. But then there
            is no convergent subsequence
            either, and therefore
            $\ell^{p}$ is not compact.
        \end{example}
        \begin{example}
            The closed unit ball in
            $(C[0,1],d_{\infty})$ is
            not compact. For let
            $x_{n}(t)=t^{2^{n}}$. Then
            (Do some calculus) the maximum of
            $d(x_{n},x_{n+1})$ is always
            $1/4$. So this has no subsequence
            which is Cauchy, and thus no
            convergent subsequence exists.
        \end{example}
        \begin{definition}
            A metric space $X$ is totally
            bounded if for all
            $\varepsilon>0$ there is a finite
            number of points $x_{n}$ such
            that $B_{\varepsilon}(x_{n})$
            covers the entirety of $X$.
        \end{definition}
        \begin{theorem}
            A compact metric space is
            totally bounded.
        \end{theorem}
        \begin{proof}
            Suppose not. Then there is an
            $\varepsilon>0$ such that
            no finite collection
            $B_{\varepsilon}(x_{n})$
            is a covering of $X$. Let
            $x_{1}\in{X}$. Then
            $B_{\varepsilon}(x_{1})$ is not
            $X$. Thus there is an $x_{2}$
            such that
            $x_{2}\notin%
             B_{\varepsilon}(x_{1})$.
            But also
            $B_{\varepsilon}(x_{1})\cup%
             B_{\varepsilon}(x_{2})$ is
            not the entirety of $X$.
            Continuing we have that there
            is a sequence $x_{n}$ such that,
            for all $n\ne{m}$,
            $d(x_{n},x_{m})\geq{\varepsilon}$.
            So there is no convergent
            subsequence. But $X$ is
            compact, a contradiction.
            Therefore, etc.
        \end{proof}
        There are metric spaces that are
        bounded but not totally bounded.
        For let
        $X=\mathbb{R}$ and $d$ be the
        discrete metric. Then, for
        $\varepsilon=1/2$, the is no
        finite covering. Every point needs
        it's own ball, so the covering is
        uncountable.
        \begin{theorem}
            If $(X,d)$ is complete and
            totally bounded, then it
            is compact.
        \end{theorem}
        \begin{proof}
            Let $x_{n}$ be a sequence
            in $X$. Let $\varepsilon=1$. Then
            there are finitely many points
            $y_{k}$ such that
            $B_{\varepsilon}(y_{k})$ covers
            $X$. Then one of these
            balls has infinitely many of
            the $x_{n}$. Similarly, for
            $\varepsilon=\frac{1}{n}$, there
            is a finite number of points
            $y_{k}$ such that
            $B_{\frac{1}{n}}(y_{k})$ covers
            $X$. Thus there is a point with
            infinitely many of the $x_{n}$
            in it. So, we can find a
            subsequence such that, for
            $n,m>N$,
            $d(x_{k_{n}},x_{k_{m}})<%
             \frac{1}{N}$. But $(X,d)$ is
            complete, and therefore
            $x_{k_{n}}$ converges. Therefore
            $x_{n}$ has a convergent
            subsequence. Thus, $(X,d)$ is
            compact.
        \end{proof}
        \begin{theorem}
            Compact spaces are separable.
        \end{theorem}
        \begin{proof}
            If $X$ is compact, then
            it is totally bounded. But
            then, for $\varepsilon=1/n$
            there is a finite covering of
            $X$ with balls of radius
            $\varepsilon$. Then,
            taking all of the
            centers of all of the points
            for all $n$ (Countable union
            of finite points is countable),
            we obtain a countable dense
            subset.
        \end{proof}
        \begin{example}
            There are ``infinite dimension''
            sets that are also compact. Two
            in particular worth mentioning.
            The first is the hilbert Cube.
            It's a subset of $\ell^{2}$
            whose elements are such that
            $|x_{n}|<1/n$. That is, elements
            are sequences whose $n^{th}$
            elements are less than
            $1/n$. This is compact.
            Arzela-Ascoli. Peano.
        \end{example}
    \section{Continuous Functions}
        Consider a function $f:\mathbb{R}\rightarrow\mathbb{R}$.
        Such a function is continuous at a point
        $x_{0}\in\mathbb{R}$ if, for all $\varepsilon>0$ there exists
        a $\delta>0$ such that, for all $x\in\mathbb{R}$ such that
        $|x-x_{0}|>\delta$, it is true that
        $|f(x)-f(x_{0})|<\varepsilon$. We can write this
        equivalently by saying that for all
        $x\in(x_{0}-\delta,x_{0}+\delta)$, it is true that
        $f(x)\in(f(x_{0})-\varepsilon,f(x_{0})+\varepsilon)$.
        Thus:
        \begin{equation}
            f\Big((x_{0}-\delta,x_{0}+\delta)\Big)\subseteq
            \Big((f(x_{0}-\varepsilon,f(x_{0})+\varepsilon)\Big)
        \end{equation}
        And finally, from this we can write:
        \begin{equation}
            (x_{0}-\delta,x_{0}+\delta)\subseteq
            f^{-1}\Big(
                \big(f(x_{0})-\varepsilon,f(x_{0})+\varepsilon\big)
            \Big)
        \end{equation}
        This gives us the following theorem about
        continuous functions.
        \begin{theorem}
            A function $f:\mathbb{R}\rightarrow\mathbb{R}$ is
            continuous at every point $x\in\mathbb{R}$ if and only
            if for every open set $\mathcal{U}\subseteq\mathbb{R}$,
            the pre-image $f^{-1}(\mathcal{U})$ is an open subset
            of $\mathbb{R}$.
        \end{theorem}
        \begin{proof}
            Going one way, every open interval is an open set.
            Thus, if $f$ is such that for every open set
            $\mathcal{U}\subseteq\mathbb{R}$, the pre-image of
            $\mathcal{U}$ is also open, then the pre-image of every
            open interval is open. Thus, let $\varepsilon>0$ be
            given. Then
            $f^{-1}(f(x_{0}-\varepsilon,f(x_{0})+\varepsilon)$ is
            open. But as $x_{0}$ is contained in this set, and as
            this set is open, there is a $\delta>0$ such that:
            \begin{equation}
                (x_{0}-\delta,x_{0}+\delta)\subseteq
                f^{-1}\big(
                    f(x_{0}-\varepsilon,f(x_{0})+\varepsilon\big)
            \end{equation}
            Thus, for all $x$ such that
            $|x-x_{0}|<\delta$, $|f(x)-f(x_{0})|<\varepsilon$.
            Now for the other direction. Suppose
            $f:\mathbb{R}\rightarrow\mathbb{R}$ is continuous at
            every point $x\in\mathbb{R}$ and let
            $\mathcal{U}\subseteq\mathbb{R}$ be an open set, and
            let $\mathcal{V}=f^{-1}(\mathcal{U})$. Then, for all
            $x\in\mathcal{V}$, $f(x)\in\mathcal{U}$. But, as
            $\mathcal{U}$ is open, there is a $\varepsilon>0$
            such that
            $(f(x)-\varepsilon,f(x)+\varepsilon)\subseteq\mathcal{U}$.
            But, as $f$ is continuous, there is a $\delta>0$ such
            that, for for all $|x-x_{0}|<\delta$,
            $|f(x)-f(x_{0})|<\varepsilon$. But then
            $(x-\delta,x+\delta)\subseteq\mathcal{V}$, and thus
            $\mathcal{V}$ is open.
        \end{proof}
        From a topological point of view, we take this as the
        definition of a continuous function.
        \begin{ldefinition}
              {Continuous Functions Between Topological Spaces}
              {Cont_Function}
            A continuous function from a topological space
            $(X,\tau_{X})$ to a topological space
            $(Y,\tau_{Y})$ is a function $f:X\rightarrow{Y}$,
            denoted $f:(X,\tau_{X})\rightarrow(X,\tau_{Y})$, such
            that for all open subsets $\mathcal{U}\subseteq{Y}$,
            it is true that $f^{-1}(\mathcal{U})$ is an open
            subset of $X$.
        \end{ldefinition}
    \section{Basis}
        Let $(X,\tau)$ be a topological space. Recall that
        elements of $\tau$ are called open sets.
        \begin{ldefinition}{Basis of a Topology}
              {Funct_Analysis_Basis_of_Topology}
            A basis of a topological space $(X,\tau)$ is a
            subset $\beta\subseteq\tau$ such that, for all
            $\mathcal{U}\in\tau$ and $x\in\mathcal{U}$, there
            is a $V\in\beta$ such that
            $x\in{V}\subset\mathcal{U}$.
        \end{ldefinition}
        We say that a subset $V\subseteq{X}$ is a neighborhood
        of $x$ if there is an open set
        $\mathcal{U}\in\tau$ such that
        $x\in\mathcal{U}\subseteq{V}$. We write
        $\mathscr{N}(x)$ for the collection of neighborhoods
        of $x$.
        \begin{ldefinition}{Neighborhood Basis}
              {Funct_Analysis_Neighborhood_Basis}
            A neighborhood basis of a point $x$ in a topological
            space $(X,\tau)$ is a subset
            $\alpha\subseteq\mathscr{N}(x)$ such that
            for all $\mathcal{U}\in\mathscr{N}(x)$ there is
            a $V\in\alpha$ such that
            $x\in{V}\subseteq\mathcal{U}$.
        \end{ldefinition}
        \begin{lexample}{}{Metric_Space_Basis}
            In a metric space, the open balls form
            a basis for the metric topology. In $\mathbb{R}^{n}$
            every point has a neighborhood basis consisting of
            compact sets. Thus, we say that $\mathbb{R}^{n}$
            is locally compact. Thus it is useful to allow
            neighborhoods of point be be more than just open
            sets containing the point.
        \end{lexample}
        \begin{theorem}
            If $\alpha(x)$ is a neighborhood basis of
            $x$ consisting of open sets, and:
            \begin{equation}
                \beta=\bigcup_{x\in{X}}\alpha(x)
            \end{equation}
            Then $\beta$ is a basis for $(X,\tau)$.
        \end{theorem}
        \begin{theorem}
            If $(X,\tau)$ a topological space, then
            $\beta\subseteq\tau$ is a basis if and only if,
            for all $\mathcal{U}\in\tau$, we have:
            \begin{equation}
                \mathcal{U}=
                \bigcup_{\underset{V\subseteq\mathcal{U}}
                    {V\in\beta}}V
            \end{equation}
        \end{theorem}
        \begin{ldefinition}{Separable Topological Space}
              {Funct_Analysis_Sep_Top_Space}
            A separable topological space is a topological
            space $(X,\tau)$ such that there is a countable
            dense subset $\mathcal{D}\subseteq{X}$.
        \end{ldefinition}
        \begin{ldefinition}{First Countable Topological Space}
              {Funct_Analysis_First_Count_Top_Space}
            A first countable topological space is a
            topological space $(X,\tau)$ such that, for
            all $x\in{X}$, there is a countable neighborhood
            basis $\alpha(x)\subseteq\tau$.
        \end{ldefinition}
        \begin{ldefinition}{Second Countable Topological Space}
              {Funct_Analysis_Second_Count_Top_Space}
            A second countable topological space is a
            topological space $(X,\tau)$ such that there is
            a countable basis $\beta\subseteq\tau$.
        \end{ldefinition}
        \begin{lexample}{}{Discrete_Topology}
            Let $X$ be any set, and let $\tau=\mathcal{P}(X)$.
            This is called the discrete topology on $X$
            and is metrizable, for it is generated by the
            discrete metric on $X$. The topological space
            $(X,\tau)$ is not second countable if $X$ is an
            uncountable set.
        \end{lexample}
        \begin{theorem}
            If $(X,\tau)$ is a metrizable metric space, then
            $(X,\tau)$ is second countable if and only if
            $(X,\tau)$ is separable.
        \end{theorem}
        \begin{theorem}
            If $(X,\tau)$ is metrizable, then it is first
            countable.
        \end{theorem}
        \begin{theorem}
            If $X$ is a set and $S\subseteq\mathcal{P}(X)$,
            then there is a smallest topology, $\tau(S)$, such
            that $S\subseteq\tau(S)$.
        \end{theorem}
        \begin{proof}
            Since $\mathcal{P}(X)$ is a topology on $X$ such
            that $S\subseteq\mathcal{P}(X)$, the set of
            topologies on $X$ such that $S$ is contained in
            the topology is non-empty. Let $A$ denote this set,
            and define:
            \begin{equation}
                \tau=\bigcap_{\tau_{A}\in{A}}\tau_{A}
            \end{equation}
            Then $\tau$ is a topology, and $S\subseteq\tau$.
            Moreover, for any topology $\tau'$ such
            that $S\subseteq\tau'$, we have $\tau\subseteq\tau$.
            Thus, $\tau$ is the smallest topology.
        \end{proof}
        \begin{theorem}
            If $\beta\subseteq\mathcal{P}(X)$ has a cover of
            $X$, then $\beta$ is a basis for
            $\tau(\beta)$ if and only if, given $\mathcal{U}$,
            $\mathcal{V}$ in $\tau(\beta)$, and
            $x\in\mathcal{U}\cap\mathcal{V}$, then there is an
            $\omega\in\beta$ such that
            $x\in\omega\subseteq\mathcal{U}\cap\mathcal{V}$.
        \end{theorem}
        \begin{theorem}
            If $X$ is a topological space, if
            $\beta\subseteq\mathcal{P}(X)$ is a cover of $X$
            such that $\beta$ is closed under intersection,e
            then $\beta$ is a basis for the topological space
            $(X,\tau(\beta))$.
        \end{theorem}
        \begin{theorem}
            If $\rho\subseteq\mathcal{P}(X)$ is a cover of
            $X$, and if:
            \begin{equation}
                \beta=\Big\{\bigcap_{i=1}^{n}V_{i}:
                    n\in\mathbb{N},V_{i}\in\rho\Big\}
            \end{equation}
            Then $\beta$ is a basis for $\tau(\rho)$. We call
            $\rho$ a sub-basis for $\tau(\rho)$.
        \end{theorem}
        \begin{ldefinition}{Topology Induced by Functions}
              {Funct_Analysis_Top_Induced_by_Funcs}
            The topology induced on a set $X$ by a
            set $\mathcal{F}$ of functions $f$ from $X$ to
            a topological space $(Z_{F},\tau_{F})$ is the
            smallest topology on $X$ such that, for all
            $f\in\mathcal{F}$, $f$ is continuous.
        \end{ldefinition}
        \begin{theorem}
            If $X$ is a set, $\mathcal{F}$ is a set of functions
            from $X$ to a topological space $(Z_{F},\tau_{F})$,
            if $\tau$ is the topology induced by $\mathcal{F}$,
            then:
            \begin{equation}
                \beta=\big\{f^{\minus{1}}(V):
                    f\in\mathcal{F},V\in\tau_{F}\big\}
            \end{equation}
            Is a sub-basis for $\tau$.
        \end{theorem}
    \begin{definition}
        The open ball of radius $r>0$
        about a point $x$ in a metric space
        $(X,d)$ is the set
        $B_{r}(x)=\{y\in{X}:d(x,y)<r\}$
    \end{definition}
    The picture for this is a ``circle'' around the
    point $x$ or radius $r$. However, this circle
    can look very strange for weird metrics.
    \begin{example}
        If $X$ is a set and $d$ is the discrete metric,
        then $B_{r}(x)$ is either the point $x$
        (If $r\leq{1}$), or it is the entire set $X$.
    \end{example}
    \begin{example}
        With $X=\mathbb{R}$ and $d$ the standard metric
        $d(x,y)=|x-y|$, we have $B_{r}(x)$ is simply
        the open interval $(x-r,x+r)$.
    \end{example}
    \begin{example}
        \label{EXAMPLE:FUNCTIONAL:UNIT_BALLS_EXAMPLE}
        Let $X=\mathbb{R}^{2}$ and define
        $d_{p}(x,y)%
         =(|x_{1}-y_{1}|^{p}+|x_{2}-y_{2}|^{p})^{1/p}$.
        For $p=2$, an open ball is a circle around
        the point $(x,y)$ of radius $r$. For $p=1$,
        we have ``diamonds'' around the point $x$.
        And for $p=\infty$ we have a square
        around $x$.
        Let $X=\mathbb{R}^{2}$ and let $d$ be the metric
        such that you can only travel parallel to the
        $y$ axis, or along the $x$ axis.
        Consider the unit balls in $(X,d)$
        about the following points:
        \begin{align*}
            a.)\quad(0,0)&&
            b.)\quad(0,1)&&
            c.)\quad(0,\tfrac{1}{2})&&
            d.)\quad(\tfrac{1}{2},\tfrac{1}{2})&&
        \end{align*}
        If $\mathbf{x}_{1}=(x_{1},y_{1})$ and
        $\mathbf{x}_{2}=(x_{2},y_{2})$, then we have:
        \begin{equation*}
            d(\mathbf{x}_{1},\mathbf{x}_{2})=
            \begin{cases}
                |y_{2}-y_{1}|,&x_{1}=x_{2}\\
                |x_{2}-x_{1}|+|y_{1}|+|y_{2}|,
                &x_{1}\ne{x_{2}}
            \end{cases}
        \end{equation*}
    About the point $(0,0)$, the unit ball
    is simply points
    $(x,y)$ such that $|x|+|y|<1$. This is a ``diamond.''
    About $(0,1)$, first note that to get to any point
    whose $x$ coordinate is not $0$, you first must travel
    the entirety of the $y$ axis. Since this length is
    already $1$, you can't go left or right
    on the $x$ axis.
    The unit ball is the line segment on the $y$ axis
    between $(0,0)$ and $(0,2)$. For the third one, if
    the $x$ coordinate changes, we have
    $0.5+|y|+|x|<1$, which implies
    $|y|+|x|<0.5$. This is again a diamond, but a
    smaller one. If the $x$ coordinate does not
    change, we have $|y-0.5|<1$. This is another
    line segment. Repeat the same arguments for the
    fourth coordinate. The diagrams are show in
    Fig.~\ref{FUNCTIONAL:HOMEWORK:2:PROBLEM:4:FIGURES}.
    \begin{figure}[H]
        \centering
        \captionsetup{type=figure}
        %--------------------------------Dependencies----------------------------------%
%   tikz                                                                       %
%       arrows.meta                                                            %
%-------------------------------Main Document----------------------------------%
\begin{tikzpicture}[%
    >=Latex,
    line width=0.2mm,
    line cap=round,
    scale=1.5
]

    % Axes for the first ball.
    \begin{scope}[<->, thick]
        \draw (-1.8, 0.0) to (1.8, 0.0) node [above] {$x$};
        \draw (0.0, -1.5) to (0.0, 2.5) node [right] {$y$};
    \end{scope}

    % Draw the first ball.
    \draw[fill=cyan,opacity=0.8,densely dashed] (-1, 0) to (0, -1) to (1,0)
                                                        to (0,  1) to cycle;

    % Axes for the second ball..
    \begin{scope}[<->, thick, xshift=4cm]
        \draw[<->] (-1.8, 0.0) to (1.8, 0.0) node [above] {$x$};
        \draw[<->] (0.0, -1.5) to (0.0, 2.5) node [right] {$y$};
    \end{scope}

    % Draw the second ball.
    \draw[blue] (4,0) to (4,2);
    \draw[blue, fill=white, thick] (4.0, 2.0) circle (0.3mm);
    \draw[blue, fill=white, thick] (4.0, 0.0) circle (0.3mm);

    % Axes for the third ball.
    \begin{scope}[<->, thick, yshift=-4.5cm]
        \draw (-1.8,  0.0) to (1.8, 0.0) node [above] {$x$};
        \draw ( 0.0, -1.5) to (0.0, 2.5) node [right] {$y$};
    \end{scope}

    % Draw the third ball.
    \begin{scope}[yshift=-4.5cm]
        \draw[fill=cyan, opacity=0.8, densely dashed]
            (-0.5, 0.0) to (0.0, -0.5) to (0.5,  0.0) to (0,0.5) to cycle;
        \draw[blue] (0.0, 0.5) to (0,1.5);
        \draw[blue, fill=white, thick] (0.0, 1.5) circle (0.3mm);
    \end{scope}

    % Draw the fourth open ball.
    \begin{scope}[<->, thick, xshift=4cm, yshift=-4.5cm]
        \draw (-1.8,  0.0) to (1.8, 0.0) node [above] {$x$};
        \draw ( 0.0, -1.5) to (0.0, 2.5) node [right] {$y$};
    \end{scope}

    \begin{scope}[xshift=4cm, yshift=-4.5cm]]
        \draw[fill=cyan, opacity=0.8, densely dashed]
            (0.0, 0.0) to (0.5, 0.5) to (1.0, 0.0) to (0.5, -0.5) to cycle;
        \draw[blue] (0.5,0.5) to (0.5,1.5);
        \draw[blue, fill=white, thick] (0.5, 1.5) circle (0.3mm);
    \end{scope}

    % Labels.
    \node at (-1.0,  2.0) {$B_{1}^{(\mathbb{R}^{2},\,d)}\Big((0,\,0)\Big)$};
    \node at ( 3.0,  2.0) {$B_{1}^{(\mathbb{R}^{2},\,d)}\Big((0,\,1)\Big)$};
    \node at (-1.0, -2.5) {$B_{1}^{(\mathbb{R}^{2},\,d)}%
                            \Big(\big(0,\,\frac{1}{2}\big)\Big)$};
    \node at ( 3.0, -2.5) {$B_{1}^{(\mathbb{R}^{2},\,d)}%
                            \Big((\frac{1}{2},\,\frac{1}{2})\Big)$};
\end{tikzpicture}
        \caption[Open Unit Balls with a Strange Metric]
            {Open Unit Balls with the Metric from
             Ex.~\ref{EXAMPLE:FUNCTIONAL:UNIT_BALLS_EXAMPLE}.}
        \label{FUNCTIONAL:HOMEWORK:2:PROBLEM:4:FIGURES}
    \end{figure}
    \end{example}
    If you have a vector space and a norm on it,
    then the open balls about a point will have the
    property of convexity. Convexity is a vector space
    property, given two points the ``line'' between the
    two remains in the set. Metric spaces have no such
    notion. Since the balls of $\norm{}_{p}$ are not
    convex with $p<1$, we have that $\norm{}_{p}$ is
    a metric on $\mathbb{R}^{n}$
    if and only if $p\geq{1}$.
    \begin{definition}
        An open subset of a metric space
        $(X,d)$ is a set $S\subset{X}$ such that,
        for all $x\in{S}$, there is an
        $r>0$ such that
        $B_{r}(x)\subset{S}$.
    \end{definition}
    \begin{example}
        If $(X,d)$ is a metric space, then
        $X$ is open and $\emptyset$ is open
        (Vacuously true).
    \end{example}
    \begin{theorem}
        If $(X,d)$ is a metric space, $x\in{X}$,
        and $r>0$, then $B_{r}(x)$ is an open
        subset of $X$.
    \end{theorem}
    \begin{proof}
        If $z\in{B_{r}(x)}$, let $t=d(x,z)$.
        Then $0\leq{t}<r$. Let $r'=r-t$.
        But if $y\in{B_{r'}(z)}$, then
        $d(x,y)\leq{d(x,z)+d(y,z)}<t+r'=t+r-t=r$.
        Therefore $B_{r'}(z)\subset{B_{r}(x)}$.
    \end{proof}
    \begin{theorem}
        A finite intersection of open sets is open.
    \end{theorem}
    \begin{proof}
        If $\mathcal{U}_{1},\hdots,\mathcal{U}_{n}$
        are open and if
        $x\in\cap_{k=1}^{n}\mathcal{U}_{k}$, then there
        exists $r_{1},\hdots,r_{n}$ such that
        $B_{r_{i}}(x)\subset\mathcal{U}_{i}$. Let
        $r=\min\{r_{1},\hdots,r_{n}\}$. Then
        $B_{r}(x)\subset\cap_{k=1}^{n}\mathcal{U}_{i}$
    \end{proof}
    \begin{theorem}
        Arbitrary unions of open sets are open.
    \end{theorem}
    Infinite intersections need not be open.
    The proof above would fail since the
    $r_{i}$ can form a sequence tending to zero.
    But indeed, let $X=\mathbb{R}$ and let
    $d(x,y)=|x-y|$, and take
    $\mathcal{U}_{n}=(-\frac{1}{n},\frac{1}{n})$.
    Then all of the $\mathcal{U}_{n}$ are open,
    yet the intersection, which is the set $\{0\}$,
    is not open. All of this mumbo-jumbo creates
    the more general notion of a topological space.
    \begin{definition}
        A topological space is a set $X$ and a
        subset $\tau\subset\mathcal{P}(X)$ such that:
        \begin{enumerate}
            \item $\emptyset,X\in\tau$
            \item Finite intersections of sets in $\tau$
                  are also sets in $\tau$.
            \item Arbitrary unions of sets in $\tau$
                  are also sets in $\tau$.
        \end{enumerate}
    \end{definition}
    Here, $\mathcal{P}(X)$ denotes the \textit{power set}
    of $X$. This is the set of all subsets of $X$.
    The notion of a topological space generalizes the
    notion of a metric space. There is no notion of
    distance in such spaces, and things can be weird.
    There are topological spaces that have no metric
    associated with them.
    \begin{definition}
        An open subset of a topological space
        $(X,\tau)$ is a set $\mathcal{U}\in\tau$.
    \end{definition}
    \begin{definition}
        An interior point of a subset $S$ of a topological
        space $(X,\tau)$ is a point $x\in{S}$ such that
        there is an open subset $\mathcal{U}\subseteq{S}$
        such that $x\in\mathcal{U}$.
    \end{definition}
    \begin{definition}
        The interior of a subset $S$ of a topological
        space $(X,\tau)$, denoted $\Int(S)$, is the set
        of all interior points of $S$.
    \end{definition}
    \begin{theorem}
        If $S$ is an open subset of
        $(X,\tau)$, then $\Int(S)=S$.
    \end{theorem}
    \begin{definition}
        A function from a metric space
        $(X,d_{X})$ to a metric space $(Y,d_{Y})$
        continuous at $x\in{X}$ is a function
        $f:X\rightarrow{Y}$ such that
        for all $\varepsilon>0$ there is
        a $\delta>0$ such that for all
        $x_{0}\in{X}$ such that
        $d_{X}(x,x_{0})<\delta$, we have
        $d_{Y}(f(x),f(x_{0})<\varepsilon$
    \end{definition}
    \begin{theorem}
        If $(X,\tau)$ is a topological space and
        $S\subseteq{X}$, and if $\mathcal{O}$ is the set
        of all open sets $\mathcal{U}$ such that
        $\mathcal{U}\subseteq{S}$, then
        $\Int(S)=%
         \bigcup_{\mathcal{U}\in\mathcal{O}}\mathcal{U}$.
    \end{theorem}
    \begin{definition}
        A nowhere dense subset of a topological space
        $(X,\tau)$ is a subset $S\subseteq{X}$ such that
        $\Int(S)=\emptyset$.
    \end{definition}
    \begin{theorem}
        If $(X,d)$ is a metric space,
        $y\in{X}$, then
        $f:X\rightarrow\mathbb{R}$ defined by
        $f(x)=d(x,y)$ is uniformly continuous.
    \end{theorem}
    A surprising theorem, and the entire
    basis of the study of topology, goes as
    follows:
    \begin{theorem}
        If $(X,d_{x})$ and $(Y,d_{Y})$
        are metric spaces, then
        $f:X\rightarrow{Y}$ is continuous
        at $x\in{X}$ if and only if
        for all open subsets
        of $S\subset{Y}$ such that
        $f(x)\in{S}$, $f^{-1}(S)$ is an
        open subset of $X$.
    \end{theorem}
    This allows us to talk about continuous
    functions without a notion of metric.
    Thus, for topological spaces, this is
    the \textit{definition} of continuity.
    When the space we're discussing is a
    metric space, this theorem shows that the
    definition from topology and the defintition
    from real analysis are in fact equivalent.
    \begin{theorem}
        A function $f:X\rightarrow{Y}$ between
        metric spaces is continuous at a point
        $x\in{X}$ if and only if for all
        sequences $x_{n}$ such that
        $d_{X}(x,x_{n})\rightarrow{0}$, we have
        $d_{Y}(f(x),f(x_{n})\rightarrow{0}$.
    \end{theorem}
    We now have three different ways to talk
    about continuity. Topological spaces can be
    nastier, however. We saw in
    Thm.~\ref{thm:Funct:Limit_of_Metric_Sequence_Unique}
    that the limit of a convergent sequence in a
    metric space is unique.
    This is not true in a topological space and there
    are topological spaces with sequences
    which converge to every point in the
    space simultaneously. Indeed, it may be impossible
    to distinguish two points in a topological
    space. The ability to
    ``Separate,'' points is special.
    Hausdorff spaces can, but
    we'll save that for topology.
    \section{Closed Sets}
        \begin{definition}
            A limit point of a subset
            $S\subset{X}$ of a metric space
            $(X,d)$ is a point $a\in{X}$ such
            that there is a sequence
            $x:\mathbb{N}\rightarrow{S}$ such that
            $d(a,x_{n})\rightarrow{0}$.
        \end{definition}
        \begin{definition}
            A closed subset of a metric space $(X,d)$
            is a set $S$ such that for all $x\in{X}$ such
            that $x$ is a limit point of $S$, $x\in{S}$.
        \end{definition}
        This says that if $S$ is closed, and
        $x$ is a sequence in $S$ such
        that $x_{n}\rightarrow{a}$, then
        $a\in{S}$.
        \begin{example}
            In $\mathbb{R}$, with the standard
            metric, $(a,b)$ is open,
            $\mathbb{R}$ is open (and closed),
            $[a,b]$ is closed,
            $[a,\infty)$ is closed,
            $[a,b)$ is neither closed nor open.
        \end{example}
        \begin{example}
            If $X=(0,1)$, and
            $d(x,y)=|x-y|$, then
            $(0,1)$ is closed. This is because
            there is no sequence that converges
            to a point in the space whose limit
            is not in the space. There are no sequences
            in $X$ which converge to zero or one since,
            as far as $X$ is concerned,
            neither or these points exist.
        \end{example}
        \begin{theorem}
            If $(X,d)$ is a metric space,
            then a subset $S\subset{X}$ is open
            if and only if $X\setminus{S}$ is closed.
        \end{theorem}
        \begin{proof}
            Suppose $S$ is open, and let
            $x_{n}$ be a sequence in $S^{c}$.
            Suppose $x_{n}\rightarrow{x}$ and
            $x\in{S}$. But $S$ is open, and thus
            there is an $\varepsilon>0$ such that
            $B_{\varepsilon}(x)\subset{S}$.
            But $x_{n}\rightarrow{x}$, and thus
            this is an $N\in\mathbb{N}$ such that
            for all $n>N$, $d(x,x_{n})<\varepsilon$.
            But then for all $n>N$,
            $x_{n}\in{B_{\varepsilon}(x)}$. But
            $x_{n}\in{S^{c}}$, a contradiction.
            Therefore, $S^{c}$ is closed. On the
            other hand, if $S^{c}$ is closed
            and there is an $x\in{S}$ such that
            for all $r>0$,
            $B_{r}(x)\cap{S}\ne\emptyset$, then
            for all $n\in\mathbb{N}$ there is
            an $x_{n}\in{S^{c}}$ such that
            $d(x,x_{n})<\frac{1}{n}$. But then
            $x_{n}\rightarrow{x}$, and therefore
            $x\in{S^{c}}$. But $x\in{S}$,
            a contradiction. Thus, $S$ is open.
        \end{proof}
        In topology we take the definition of
        closed sets to be the compliment of open
        sets. This theorem shows that the
        topological definition is equivalent when we
        consider metric spaces.
        \begin{definition}
            The closure of a subset
            $S$ of a metric space
            $(X,d)$, denoted $\overline{S}$,
            is the set of all
            limit points of $S$.
        \end{definition}
        \begin{theorem}
            If $(X,d)$ is a metric space, if
            $S\subset{X}$, and if
            $\Delta$ is the set of all closed subsets
            $\mathcal{C}\subset{X}$ such that
            $S\subset\mathcal{C}$, then:
            $\overline{S}=
             \bigcap_{\mathcal{C}\in\Delta}
             \mathcal{C}$
        \end{theorem}
        Thus we may loosely say that
        the closure of a set $S$ is the
        ``Smallest,'' closed set that contains $S$.
        \begin{definition}
            The closed ball of radius $r>0$ about
            a point $x$ in a metric space
            $(X,d)$ is the set:
            \begin{equation*}
                \overline{B}_{r}(x)=
                \{y\in{X}:d(x,y)\leq{r}\}
            \end{equation*}
        \end{definition}
        There exists metric spaces $(X,d)$
        such that
        $\overline{B}_{r}(x)\ne\overline{B_{r}(x)}$.
        For take the discrete metric, $r=1$.
        Then the closure of $B_{1}(x)$ is simply
        the point $x$. However, the closed ball
        $\overline{B}_{1}(x)$ is the entire space.
        Metric spaces can be very weird like this.
        They have a property, that given a nested
        sequence of closed balls whose radius
        tends to zero, there is precisely one
        point that lies in the intersection. However,
        if the radius does not tend to zero it is
        possible that the intersection is empty.
        This is very counter-intuitive.
        \begin{definition}
            A dense subset of a metric space $(X,d)$
            is a set $S\subset{X}$ such that
            $\overline{S}=X$.
        \end{definition}
        A subset $S$ is dense in $X$ if every point
        in $X$ can be approximated arbitrarily well
        by points in $S$. For any point $a\in{X}$
        there is a sequence $x\in{S}$
        such that $x_{n}\rightarrow{a}$. The
        classic example is $\mathbb{Q}$ and
        $\mathbb{R}$. Every real number can be
        approximated arbitrary well by a rational
        number. To see this, just take the continued
        fraction of a real number and stop once
        the approximation is less than
        $\varepsilon$. When we say $\mathbb{Q}$ is
        dense in $\mathbb{R}$, we of course mean with
        respect to the standard metric on $\mathbb{R}$.
        $\mathbb{Q}$ is \textbf{not} dense in
        $\mathbb{R}$ with respect to the discrete metric.
        Indeed, if $d$ is the discrete metric on $X$,
        then $S\subset{X}$ is dense in $X$ if and only if
        $S=X$.
        \begin{example}
            $\mathbb{Q}$ is dense in $\mathbb{R}$
            with respect to $d_{p}$ for all
            $p\geq{1}$. This includes
            $d(x,y)=|x-y|$.
        \end{example}
        \begin{example}
            The set of polynomials on the interval
            $[a,b]$ are dense in the set of
            continuous functions on $[a,b]$ with
            respect to the $d_{\infty}$ metric.
            This comes from Weierstrass's Theorem.
        \end{example}
        \begin{example}
            The set of polynomials on $[a,b]$
            is dense in the set of continuous
            functions on $[a,b]$ with respect to
            the $d_{p}$ metric, for $p\geq{1}$. This
            is because:
            \begin{align*}
                d_{p}(P,x)&=
                \Big(
                    \int_{a}^{b}|P(t)-x(t)|^{p}\diff{t}
                \Big)^{1/p}
                &
                &=\Big(
                    d_{\infty}(P,x)^{p}\int_{a}^{b}\diff{t}
                \Big)^{1/p}\\
                &\leq\Big(\int_{a}^{b}
                    |\max\{P(t)-x(t)\}|^{p}\diff{t}
                \Big)^{1/p}
                &
                &=(b-a)^{1/p}d_{\infty}(P,x)
            \end{align*}
        \end{example}
        \begin{example}
            The continuous functions are not dense
            in the set of integrable functions,
            with respect to the supremum metric
            $d_{\infty}$. This is more or less
            because integrable functions can
            be discontinuous, or have jumps. This
            means, with respect to $d_{\infty}$,
            that no continuous functions could
            approximate such a discontinuous function
            arbitrary well.
        \end{example}
        \begin{definition}
            A separable metric space
            is a metric space $(X,d)$ with
            a countable dense subset $S$.
        \end{definition}
        \begin{example}
            $\mathbb{R}$ is separable, with
            the standard metric, since
            $\mathbb{Q}$ is countable and also
            dense in $\mathbb{R}$.
        \end{example}
        \begin{example}
            The set of continuous functions on
            $[a,b]$ is separable. For
            take the set of polynomials with
            rational coefficients. This can
            be seen as a countable union of
            countably many elements. For let
            $P_{N}$ be the set of polynomials
            of degree $N$ with rational
            coefficients. This is countable,
            and the set of all polynomials with
            rational coefficients is simply the
            union of $P_{N}$ over all $N$. This
            is dense in the set of polynomials,
            and the set of polynomials is dense
            in $C[a,b]$, and thus
            the set of polynomials with rational
            coefficients is dense in $C[a,b]$. Thus
            $C[a,b]$ is separable.
        \end{example}
        \begin{example}
            $\ell^{p}$ is separable with the
            $d_{p}$ metric, simply use elements
            with rational entries. That is,
            sequences of rational numbers.
        \end{example}
        \begin{example}
            $\ell^{p}$ with the $d_{\infty}$ metric
            is NOT separable. Consider the real
            numbers in $(0,1)$.
        \end{example}
        \begin{ldefinition}{Covers}
              {Funct_Analysis_Covers}
            A cover of a subset $\mathcal{E}\subseteq{X}$ of
            a set $X$ is a subset
            $\mathcal{O}\subseteq\mathcal{P}(X)$ such that:
            \begin{equation}
                \mathcal{E}\subseteq
                \bigcup_{\mathcal{U}\in\mathcal{O}}
                    \mathcal{U}
            \end{equation}
        \end{ldefinition}
        \begin{ldefinition}{Sub-Cover}
              {Funct_Analysis_Subcover}
            A sub-cover of a cover $\mathcal{O}$ of a subset
            $E\subseteq{X}$ of a set $X$ is a subset
            $\Delta\subseteq\mathcal{O}$ such that:
          \begin{equation}
                \mathcal{E}\subseteq
                \bigcup_{\mathcal{U}\in\Delta}
                    \mathcal{U}
            \end{equation}
        \end{ldefinition}
        \begin{ldefinition}{Open Covers}
              {Funct_Analysis_Open_Cover}
            An open cover of a metric space $(X,d)$ is a cover
            $\mathcal{O}\subseteq\mathcal{P}(X)$ of $X$ such
            that, for all $\mathcal{U}\in\mathcal{O}$,
            $\mathcal{U}$ is open.
        \end{ldefinition}
        \begin{ldefinition}{Compact Sets}
              {Funct_Analysis_Compact_Set}
            A compact metric space is a metric space $(X,d)$
            such that for any open cover $\mathcal{O}$ of
            $X$, there is a finite
            sub-cover $\Delta\subseteq\mathcal{O}$.
        \end{ldefinition}
        \begin{lexample}
            Let $X=[0,1)$ with the usual topology, and let:
            \begin{equation}
                \mathcal{U}_{x}=[0,x)
                \quad\quad
                x\in(0,1)
            \end{equation}
            Then $\mathcal{O}=\{\mathcal{U}_{x}:x\in(0,1)\}$
            is an open cover of $X$, but there is no finite
            sub-cover. For given any finite sub-cover,
            there is a greatest $x$ such that
            $\mathcal{U}_{x}$ is contained in the sub-cover.
            But then for all $y\in(x,1)$, $y$ is not in
            the sub-cover. As a trivial example, any
            finite metric space is compact.
        \end{lexample}
        \begin{theorem}
            If $K$ is a subspace of $X$, then $K$ is compact
            if and only if every open cover of $K$ has a
            finite sub-cover.
        \end{theorem}
        \begin{proof}
            For suppose $(K,d_{K})$ is compact, and let
            $\mathcal{O}$ be an open cover of $K$. Then:
            \begin{equation}
                \mathcal{O}_{K}=\{K\cup\mathcal{U}:
                    \mathcal{U}\in\mathcal{O}\}
            \end{equation}
            Is an open cover of $K$. But $K$ is compact,
            and thus there is a finite sub-cover
            $\Delta_{K}$. But then:
            \begin{equation}
                \Delta=\{\mathcal{U}\in\mathcal{U}:
                         \mathcal{U}\cap{K}\in\Delta_{K}\}
            \end{equation}
            And this is a finite sub-cover.
        \end{proof}
        \begin{ldefinition}{Finite Intersection Property}
              {Funct_Analysis_Finite_Intersect_Prop}
            A set with the finite intersection property
            in a metric space $(X,d)$ is a collection of sets
            $\mathscr{F}\subseteq\mathcal{P}(X)$ such that,
            for any sequence
            $F:\mathbb{Z}_{n}\rightarrow\mathscr{F}$,
            it is true that $\cap_{k=1}^{n}F_{k}\ne\emptyset$.
        \end{ldefinition}
        \begin{theorem}
            A metric space $(X,d)$ is compact if and only if
            every collection $\mathscr{F}$ of closed sets in
            $X$ with the
            finite intersection property is such that:
            \begin{equation}
                \bigcap_{\mathcal{C}\in\mathcal{F}}
                    \mathcal{C}\ne\emptyset
            \end{equation}
        \end{theorem}
        \begin{lexample}
            Let $F_{n}=[n,\infty)$, and let
            $\mathscr{F}=\{F_{n}:n\in\mathbb{N}\}$.
            Then $\mathscr{F}$ has the finite intersection
            property. However, the intersection over the
            entire set is empty, and hence $\mathbb{R}$
            (With the standard metric) is not compact.
        \end{lexample}
        \begin{ldefinition}{Totally Bounded Metric Space}
              {Funct_Analysis_Totally_Bounded_Met_Space}
            A totally bounded metric space is a metric
            space $(X,d)$ such that, for all $\varepsilon>0$,
            there exists an $n\in\mathbb{N}$ and a sequence
            $a:\mathbb{Z}_{n}\rightarrow{X}$ such that:
            \begin{equation}
                X=\cup_{k=1}^{n}B_{\varepsilon}^{(X,d)}(a_{k})
            \end{equation}
        \end{ldefinition}
        \begin{ldefinition}{$\varepsilon\textrm{-Nets}$}
              {Funct_Analysis_epsilon_Net}
            An $\varepsilon\textrm{-Net}$ of a subspace
            $(\mathcal{E},d_{\mathcal{E}})$ of a
            metric space $(X,d)$ is a finite collection:
            \begin{equation}
                E=\{B_{\varepsilon}^{(X,d)}(x_{k}):
                    k\in\mathbb{Z}_{n}\}
            \end{equation}
            Such that $E$ is an open cover of $\mathcal{E}$.
        \end{ldefinition}
        \begin{lexample}
            Let $X=\ell^{2}$ and let:
            \begin{equation}
                e_{n}(x)=
                \begin{cases}
                    1,&k=n\\
                    0,&k\ne{n}
                \end{cases}
            \end{equation}
            Then $e_{n}\in\ell^{2}$ and $\norm{e}_{2}=1$,
            but for all $n\ne{m}$,
            $\norm{e_{n}-e_{m}}_{2}=\sqrt{2}$. Let:
            \begin{equation}
                B_{1}=\{f\in\ell^{2}:\norm{f}_{2}\leq{1}\}
            \end{equation}
            Then $B_{1}$ is bounded, but if
            $\varepsilon=\sqrt{2}/2$ then no finite collection
            of $\varepsilon$ balls can cover $B_{1}$ since
            each ball can contain at most one of the
            $e_{n}$. Thus any cover is infinite.
        \end{lexample}
        \begin{theorem}
            A subset of $(\mathbb{R}^{n},\norm{\cdot}_{2})$
            is totally bounded if and only if it's bounded.
        \end{theorem}
        \begin{proof}
            Totally bounded implies bounded, so it suffices
            to show that if $\mathbb{R}^{n}$ is bounded then
            it is totally bounded. Let
            $\mathcal{E}\subseteq\mathbb{R}^{n}$ be bounded.
            Then there is an $r>0$ such that:
            \begin{equation}
                \mathcal{E}\subseteq[-r,r]^{n}
            \end{equation}
            Then, compactness, stuff like that.
        \end{proof}
        This works for any norm on $\mathbb{R}^{n}$, since all
        norm's on $\mathbb{R}^{n}$ are strongly equivalent.
        \begin{ldefinition}{Sequential Compactness}
              {Funct_Analysis_Seq_Compact}
            A sequentially compact metric space is a metric
            space $(X,d)$ such that, for all
            $a:\mathbb{N}\rightarrow{X}$, there is a convergent
            subsequence of $a$.
        \end{ldefinition}
        \begin{lexample}
            Let $X\subseteq\mathbb{R}$ be defined by:
            \begin{equation}
                X=\{\frac{1}{n}:n\in\mathbb{N}\}\cup\{0\}
            \end{equation}
            Then $X$ is sequentially compact, with respect
            to the subspace metric.
        \end{lexample}
        \begin{theorem}
            IF $(X,d)$ is a metric space, then the following are
            equivalent:
            \begin{enumerate}
                \item $X$ is compact.
                \item $X$ is complete and totally bounded.
                \item $X$ is sequentially compact.
            \end{enumerate}
        \end{theorem}
        \begin{proof}
            Suppose $(X,d)$ is not compact, and let
            $\mathcal{U}_{i}$ be an open cover with no finite
            subcover. If $X$ is totally bounded, then there is a
            finite covering of $1/2$ balls. But then at least
            one of these isn't covered by finitely many of the
            $\mathcal{U}_{i}$. Let $F_{1}$ be the closure of
            this. Then $F_{1}$ is totally bounded, and has
            a finite covering of $1/4$ balls. One of these
            must not be covered by finitely many of the
            $\mathcal{U}_{i}$. Let $F_{2}'$ be the closure
            of such a ball, and let $F_{2}=F_{1}\cap{F}_{2}'$.
            Then $F_{2}$ is closed, non-empty, and
            $\mathrm{diam}(F_{2})\leq{1/2}$. Continuing, we
            obtain a sequence of non-empty closed sets
            $F_{n}$ such that for all $n\in\mathbb{N}$,
            $F_{n+1}\subseteq{F}_{n}$ and
            $\mathrm{diam}(F_{n})<1/2^{n}$. Thus, if $X$ is complete,
            there is a unique $x$ that lies in the intersection
            of all of the $F_{n}$. But then there is a
            $\mathcal{U}_{i}$ such that $x\in\mathcal{U}_{i}$,
            and thus eventually $F_{n}\subset\mathcal{U}_{i}$,
            a contradiction. Thus, $X$ is compact. Now, suppose
            $X$ is compact and let $a:\mathbb{N}\rightarrow{X}$
            be a sequence in $X$. Let:
            \begin{equation}
                F_{n}=\overline{\{x_{k}:l\geq{n}\}}
                \quad\quad
                \mathscr{F}=\{F_{n}:n\in\mathbb{N}\}
            \end{equation}
            Then $\mathscr{F}$ has the finite intersection
            property. Since $X$ is compact, the intersection of
            the $F_{n}$ is non-empty. Let $x$ be contained in
            the intersection. Then:
            \begin{equation}
                B_{1}(x)\cap\{x_{k}:k\geq{1}\}\ne\emptyset
            \end{equation}
            Pick $n_{1}$ such that $x_{n_{1}}\in{B}_{1}(x)$.
            Then there is an $n_{2}>n_{1}$ such that
            $x_{n_{2}}\in{B}_{1/2}(x)$. Continuing, we obtain
            a subsequence $n_{k}$ such that
            $x_{k}\in{B}_{1/k}(x)$, and thus
            $x_{k}\rightarrow{x}$. Finally, we show that
            sequential compactness implies that $X$ is
            complete and totally bounded. For suppose $X$ is
            not totally bounded. Then there exists
            $\varepsilon>0$ such that $X$ has no finite
            covering of $\varepsilon$ balls. We can thus
            obtain a sequence $a:\mathbb{N}\rightarrow{X}$
            such that, for all $n\ne{m}$,
            $d(a_{n},a_{m})\geq\varepsilon$. But this
            has no convergence subsequence, for any convergent
            subsequence would be a Cauchy sequence. Moreover,
            $X$ is complete. For suppose not, and let
            $a:\mathbb{N}\rightarrow{X}$ be a Cauchy sequence
            and suppose it does not converge. But then there
            is no convergent subsequence, since Cauchy
            sequences with convergent subsequences converge.
            Thus, $X$ is complete.
        \end{proof}
        \begin{ltheorem}{Heine-Borel Theorem}
              {Funct_Analysis_Heine_Borel}
            A subset $\mathcal{E}\subseteq\mathbb{R}^{n}$ is
            compact with respect to the standard topology if and
            only if $\mathcal{E}$ is closed and bounded.
        \end{ltheorem}
        This theorem does not generalize to other spaces. For
        consider $\ell^{2}$ and the closed unit ball about the
        origin. This is closed and bounded, but it is not
        compact. This is simply because it is not totally
        bounded, nor is it sequentially compact.
        \begin{ltheorem}{Extreme Value Theorem}
              {Funct_Analysis_Extreme_Value_Theorem}
            If $(X,d)$ is a compact metric space and if
            $f:X\rightarrow\mathbb{R}$ is continuous, then
            $f$ attains it's maximum and minimum. In particular,
            if $f:X\rightarrow\mathbb{C}$ is continuous, then
            $f$ is bounded.
        \end{ltheorem}
        \begin{proof}
            Note that if $f:X\rightarrow\mathbb{C}$ is
            continuous, then $|f|:X\rightarrow\mathbb{R}$ is
            continuous, so we only need to prove the first
            statement. For if $X$ is compact, then $f(X)$ is
            compact, for $f$ is continuous. But then $f(X)$
            is closed and bounded. Let:
            \begin{equation}
                M=\underset{x\in{X}}\sup\{f(x)\}
            \end{equation}
            Then, since $f(X)$ is bounded, $M\in\mathbb{R}$.
            But then there is a sequence
            $a:\mathbb{N}\rightarrow{X}$ such that
            $f(a_{n})\rightarrow{M}$. But if $X$ is
            compact, then it is sequentially compact, and
            thus there is an $x\in{X}$ an a subsequence
            $a_{k}$ such that $a_{k_{n}}\rightarrow{x}$.
            But then $f(x)=M$. Similarly for the minimum value.
        \end{proof}