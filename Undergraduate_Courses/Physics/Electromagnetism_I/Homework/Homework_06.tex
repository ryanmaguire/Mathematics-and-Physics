\documentclass[crop=false,class=article,oneside]{standalone}
%----------------------------Preamble-------------------------------%
%---------------------------Packages----------------------------%
\usepackage{geometry}
\geometry{b5paper, margin=1.0in}
\usepackage[T1]{fontenc}
\usepackage{graphicx, float}            % Graphics/Images.
\usepackage{natbib}                     % For bibliographies.
\bibliographystyle{agsm}                % Bibliography style.
\usepackage[french, english]{babel}     % Language typesetting.
\usepackage[dvipsnames]{xcolor}         % Color names.
\usepackage{listings, lstlinebgrd}      % Verbatim-Like Tools.
\usepackage{mathtools, esint, mathrsfs} % amsmath and integrals.
\usepackage{amsthm, amsfonts}           % Fonts and theorems.
\usepackage{tabularx}
\usepackage{tcolorbox}                  % Frames around theorems.
\usepackage{upgreek}                    % Non-Italic Greek.
\usepackage{paracol}                    % Two-column styling.
\usepackage{wrapfig}                    % Wrap text around figure.
\usepackage{fmtcount, etoolbox}         % For the \book{} command.
\usepackage[newparttoc]{titlesec}       % Formatting chapter, etc.
\usepackage{titletoc}                   % Allows \book in toc.
\usepackage[nottoc]{tocbibind}          % Bibliography in toc.
\usepackage[titles]{tocloft}            % ToC formatting.
\usepackage{multicol, enumitem}         % Multi-column/enumerate.
\usepackage{import}                     % Import external files.
\usepackage{pgfplots, tikz}             % Drawing/graphing tools.
\usetikzlibrary{
    calc,                   % Calculating right angles and more.
    angles,                 % Drawing angles within triangles.
    arrows.meta,            % Latex and Stealth arrows.
    quotes,                 % Adding labels to angles.
    positioning,            % Relative positioning of nodes.
    decorations.markings,   % Adding arrows in the middle of a line.
    patterns,
    arrows,
    shapes,
    shapes.geometric,
    cd,
    hobby,
    babel
}                                       % Libraries for tikz.
\pgfplotsset{compat=1.9}                % Version of pgfplots.
\usepackage[font=scriptsize,
            labelformat=simple,
            labelsep=colon]{subcaption} % Subfigure captions.
\usepackage[font={scriptsize},
            hypcap=true,
            labelsep=colon]{caption}    % Figure captions.
\usepackage{hyperref}                   % Allows for hyperlinks.
\hypersetup{
    colorlinks=true,
    linkcolor=blue,
    filecolor=magenta,
    urlcolor=Cerulean,
    citecolor=SkyBlue
}                           % Colors for hyperref.
\usepackage[toc,acronym,nogroupskip]{glossaries} % Glossaries and acronyms.
\usepackage[subpreambles=false]{standalone}      % Complileable sub files.

% Various font stuff from kiwi.
% Use this for Times text and Computer Modern math
%\usepackage{times}

% Quite nice
%\usepackage[charter, greekfamily=, greekuppercase=italicized]{mathdesign}
%\usepackage[utopia, greekuppercase=italicized]{mathdesign}    % Math is narrower

% Use this for Times text and math
%\usepackage{newtxtext}
%\usepackage[libertine,cmintegrals]{newtxmath}
%\usepackage{fix-cm}

%\usepackage{txfontsb}
% or
%\usepackage{mathptmx}

%\usepackage[scaled=0.92]{helvet}
%\renewcommand{\rmdefault}{ptm}

%\usepackage{mathpazo}    % add possibly `sc` and `osf` options
%\usepackage{eulervm}

%\usepackage{fourier}
%\renewcommand{\rmdefault}{ptm}
%\usepackage{mathptm}

%\usepackage{fontspec}
%\setmainfont{lmodern}

%\usepackage[varg]{txfonts}
%\usepackage{fouriernc}
%\usepackage{mathpazo}

%\usepackage{bookman}
%\usepackage[scaled]{uarial}
%\usepackage[scaled]{helvet}
%\renewcommand*\familydefault{\sfdefault}
%\usepackage[math]{anttor}

%\newcommand\fgeorgia{\fontfamily{jvn}\selectfont}
%\newcommand\ftimes{\fontfamily{ptm}\selectfont}
%\newcommand\fhelvetica{\fontfamily{phv}\selectfont}
%\newcommand\fcourier{\fontfamily{pcr}\selectfont}
%\newcommand\fbookman{\fontfamily{pbk}\selectfont}
%\newcommand\fnewcentury{\fontfamily{pnc}\selectfont}
%\newcommand\fpalatino{\fontfamily{ppl}\selectfont}
%\newcommand\favantgarde{\fontfamily{pag}\selectfont}
%\newcommand\fnormal{\normalfont}
%\newcommand\fsize[1]{\ifnum#1>0\fontsize{#1}{#1}\selectfont\else\normalsize\fi}
%------------------------Theorem Styles-------------------------%
% Define theorem style for default spacing and normal font.
\newtheoremstyle{normal}
    {\topsep}               % Amount of space above the theorem.
    {\topsep}               % Amount of space below the theorem.
    {}                      % Font used for body of theorem.
    {}                      % Measure of space to indent.
    {\bfseries}             % Font of the header of the theorem.
    {}                      % Punctuation between head and body.
    {.5em}                  % Space after theorem head.
    {}

% Define theorem style for default spacing with italicized font.
\newtheoremstyle{normalit}{\topsep}{\topsep}
                {\itshape}{}{\bfseries}{}{.5em}{}

% Italic header environment.
\newtheoremstyle{thmit}{\topsep}{\topsep}{}{}{\itshape}{}{0.5em}{}

% Define italicized environments.
\theoremstyle{normalit}
\newtheorem{theorem}{Theorem}[section]
\newtheorem{lemma}{Lemma}[section]
\newtheorem{corollary}{Corollary}[section]
\newtheorem{proposition}{Proposition}[section]
\newtheorem*{theorem*}{Theorem}

% Define environments with italic headers.
\theoremstyle{thmit}
\newtheorem*{solution}{Solution}
\newtheorem*{fsolution}{Solution}

% Define default environments.
\theoremstyle{normal}
\newtheorem{example}{Example}[section]
\newtheorem{definition}{Definition}[section]
\newtheorem{problem}{Problem}[section]
\newtheorem{question}{Question}[section]
\newtheorem{remark}{Remark}[section]
\newtheorem{properties}{Properties}[section]
\newtheorem{notation}{Notation}[section]
\newtheorem{axiom}{Axiom}[section]
\newtheorem*{properties*}{Properties}
\newtheorem*{remark*}{Remark}
\newtheorem*{definition*}{Definition}
\theoremstyle{plain}

% Define framed environment.
\tcbuselibrary{most}
\newtcbtheorem[use counter*=theorem]{ftheorem}{Theorem}%
    {colback=green!5,colframe=green!35!black,
     fonttitle=\bfseries\upshape}{th}

\newtcbtheorem[use counter*=example]{fdefinition}{Definition}%
    {fonttitle=\bfseries\upshape,
     colback=blue!5!white,colframe=blue!75!black}{def}

\newtcbtheorem[use counter*=example]{fexample}{Example}%
    {fonttitle=\bfseries\upshape,
     colback=red!5!white,colframe=red!75!black}{ex}

\newtcbtheorem[use counter*=notation]{fnotation}{Notation}%
    {fonttitle=\bfseries\upshape,
     colback=SeaGreen!5!white,colframe=SeaGreen!75!black}{ex}

\newtcbtheorem[use counter*=corollary]{fcorollary}{Corollary}%
    {fonttitle=\bfseries\upshape,
     colback=Orchid!5!white,colframe=Orchid!75!black}{ex}

\newenvironment{bproof}{\textit{Proof.}}{\hfill$\square$}
\tcolorboxenvironment{bproof}{blanker,breakable,left=5mm,
                             before skip=10pt,after skip=10pt,
                             borderline west={1mm}{0pt}{red}}
\tcolorboxenvironment{fsolution}
    {enhanced jigsaw,colframe=cyan,interior hidden,breakable}

%--------------------Declared Math Operators--------------------%
\DeclareMathOperator{\Refl}{Refl}           % Reflection operator.
\DeclareMathOperator{\Span}{Span}           % Span of a set of vectors.
\DeclareMathOperator{\Card}{Card}           % Cardinality of set.
\DeclareMathOperator{\Ord}{Ord}             % Ordinal of ordered set.
\DeclareMathOperator{\Tr}{Tr}               % Trace of matrix.
\DeclareMathOperator{\adjoint}{adj}         % Adjoint of matrix.
\DeclareMathOperator{\rk}{rk}               % Rank of operator.
\DeclareMathOperator{\nul}{nul}             % Null space of operator.
\DeclareMathOperator{\sgn}{sgn}             % Sign of a number.
\DeclareMathOperator{\multideg}{mutlideg}   % Multi-Degree (Graphs).
\DeclareMathOperator{\GCD}{GCD}             % Greatest common denominator.
\DeclareMathOperator{\LM}{LM}               % Leading monomial
\DeclareMathOperator{\LC}{LC}               % Leading coefficient.
\DeclareMathOperator{\LT}{LT}               % Leading term.
\DeclareMathOperator{\LCM}{LCM}             % Least common multiple.
\DeclareMathOperator{\Mon}{Mon}             % Monomial.
\DeclareMathOperator{\Spec}{Spec}           % Spectrum.
\DeclareMathOperator{\proj}{proj}           % Projection.
\DeclareMathOperator{\comp}{comp}           % Component.
\DeclareMathOperator{\sinc}{sinc}           % Sinc function.
\DeclareMathOperator{\Ima}{Im}              % Image of operator.
\DeclareMathOperator{\Prin}{Prin}           % Principal value.
\DeclareMathOperator{\Mod}{mod}             % Modulus.
%------------------------New Commands---------------------------%
\DeclarePairedDelimiter\norm{\lVert}{\rVert}
\DeclarePairedDelimiter\ceil{\lceil}{\rceil}
\DeclarePairedDelimiter\floor{\lfloor}{\rfloor}
\newcommand*\diff{\mathop{}\!\mathrm{d}}
\newcommand*\Diff[1]{\mathop{}\!\mathrm{d^#1}}
\renewcommand{\mod}{\ \Mod}
\renewcommand*{\glstextformat}[1]{\textcolor{RoyalBlue}{#1}}
\renewcommand{\glsnamefont}[1]{\textbf{#1}}
\renewcommand\labelitemii{$\circ$}
\renewcommand\thesubfigure{\arabic{chapter}.\arabic{figure}}
\renewcommand\thesubfigure{%
    \arabic{chapter}.\arabic{figure}.\arabic{subfigure}}
\addto\captionsenglish{\renewcommand{\figurename}{Fig.}}
%------------------------Book Command---------------------------%
\makeatletter
\renewcommand\@pnumwidth{1cm}
\newcounter{book}
\renewcommand\thebook{\@Roman\c@book}
\newcommand\book{%
    \if@openright
        \cleardoublepage
    \else
        \clearpage
    \fi
    \thispagestyle{plain}%
    \if@twocolumn
        \onecolumn
        \@tempswatrue
    \else
        \@tempswafalse
    \fi
    \null\vfil
    \secdef\@book\@sbook
}
\def\@book[#1]#2{%
    \ifnum \c@secnumdepth >-3\relax
        \refstepcounter{book}%
        \addcontentsline{toc}{book}{
            \bookname\ \thebook:\hspace{1em}#1
        }
    \else
        \addcontentsline{toc}{book}{#1}%
    \fi
    \markboth{}{}%
    {\centering
     \interlinepenalty \@M
     \normalfont
     \ifnum \c@secnumdepth >-2\relax
       \huge\bfseries \bookname\nobreakspace\thebook
       \par
       \vskip 20\p@
     \fi
     \Huge \bfseries #2\par}%
    \@endbook}
\def\@sbook#1{%
    {\centering
     \interlinepenalty \@M
     \normalfont
     \Huge \bfseries #1\par}%
    \@endbook}
\def\@endbook{
    \vfil\newpage
        \if@twoside
            \if@openright
                \null
                \thispagestyle{empty}%
                \newpage
            \fi
        \fi
        \if@tempswa
            \twocolumn
        \fi
}
\newcommand*\l@book[2]{%
    \ifnum \c@tocdepth >-2\relax
        \addpenalty{-\@highpenalty}%
        \addvspace{2.25em \@plus\p@}%
        \setlength\@tempdima{3em}%
        \begingroup
            \parindent \z@ \rightskip \@pnumwidth
            \parfillskip -\@pnumwidth
            {
                \leavevmode
                \Large \bfseries #1\hfil \hb@xt@\@pnumwidth{
                    \hss #2
                }
            }
            \par
            \nobreak
            \global\@nobreaktrue
            \everypar{\global\@nobreakfalse\everypar{}}%
        \endgroup
    \fi}
\newcommand\bookname{Book}
\renewcommand{\thebook}{\texorpdfstring{\Numberstring{book}}{book}}
\providecommand*{\toclevel@book}{-2}
\makeatother
\titlecontents{chapter}[0pt]
    {\bfseries}
    {\chaptername\ \thecontentslabel:\quad}
    {}
    {\hfill\contentspage}
\titleformat{\part}[display]
    {\Large\bfseries}
    {\partname\nobreakspace\thepart}
    {0mm}
    {\Huge\bfseries}
    \titlecontents{part}[0pt]
    {\large\bfseries}
    {\partname\ \thecontentslabel: \quad}
    {}
    {\hfill\contentspage}
\newcommand{\MarkRightAngle}[4][.3cm]
    {\coordinate (tempa) at ($(#3)!#1!(#2)$);
     \coordinate (tempb) at ($(#3)!#1!(#4)$);
     \coordinate (tempc) at ($(tempa)!0.5!(tempb)$);%midpoint
     \draw (tempa) -- ($(#3)!2!(tempc)$) -- (tempb);}
%--------------------------LENGTHS------------------------------%
% Spacings for the Table of Contents.
\addtolength{\cftsecnumwidth}{1ex}
\addtolength{\cftsubsecindent}{1ex}
\addtolength{\cftsubsecnumwidth}{1ex}
\addtolength{\cftfignumwidth}{1ex}
\addtolength{\cfttabnumwidth}{1ex}

% Spacing for multi-column and enumerate environments.
\setlength{\multicolsep}{6pt}
\setlist[enumerate]{itemsep=0pt,topsep=3pt}

% Indent and paragraph spacing.
\setlength{\parindent}{0em}
\setlength{\parskip}{0em}
%--------------------------Main Document----------------------------%
\begin{document}
    \ifx\ifphysicscourseselectromagnetismI\undefined
        \section*{Electromagnetism I}
        \setcounter{section}{6}
        \renewcommand\thefigure{%
            \arabic{section}.\arabic{figure}%
        }
        \renewcommand\thesubfigure{%
            \arabic{section}.\arabic{figure}.\arabic{subfigure}%
        }
    \fi
    \subsection{Homework VI}
    \begin{problem}[Wangsness 5-1]
        Can
        $\mathbf{E}%
         =(yz-2x)\hat{\mathbf{x}}%
         +xz\hat{\mathbf{y}}%
         +xy\hat{\mathbf{z}}$
        be a possible electrostatic field? If so,
        find a possible potential function $\phi$.
    \end{problem}
    \begin{proof}[Solution]
        We have that:
        \begin{equation*}
            \nabla\times\mathbf{E}
            =(x-x)\hat{\mathbf{x}}
            -(y-y)\hat{\mathbf{y}}
            +(z-z)\hat{\mathbf{z}}
            =\mathbf{0}    
        \end{equation*}
        Therefore $\mathbf{E}$ is a possible electrostatic field.
        Indeed, writting $\mathbf{E}=-\nabla(\phi)$, we get:
        \begin{align*}
            -\frac{\partial\phi}{\partial{x}}
            &=yz-2x
            \Rightarrow\phi=-xyz-x^{2}+g(y,z)\\
            -\frac{\partial\phi}{\partial{y}}
            &=xz
            \Rightarrow\frac{\partial{g}}{\partial{y}}=0
            \Rightarrow{g(y,z)=g(z)}\\
            -\frac{\partial\phi}{\partial{z}}
            &=xy
            \Rightarrow\frac{\partial{g}}{\partial{z}}=0
            \Rightarrow{g=constant}
        \end{align*}
        The reason $g$ is a function of $y$ and $z$ is because
        we are taking partial derivatives, and thus the
        ``constants'' of integration can be functions of
        the other variables. Thus we must verify what $g(y,z)$
        is. Taking partial derivatives with respect to
        $y$ and $z$ revealed to us that $g$ is indeed just
        a constant. So, we have:
        \begin{equation*}
            \phi(x,y,z)=-xyz+x^{2}+C    
        \end{equation*}
        where $C$ is some constant. We may choose $C$ as we desire,
        so let $C=0$ to make things easy. The path integral
        from the origin to a point $(x,y,z)$ is independent
        of path and may be computed by using the fundamental
        theorem of gradients. This theorem says that,
        if $\phi$ is differentiable
        (All of its partial derivatives exists), then:
        \begin{equation*}
            \int_{P_{1}}^{P_{2}}
            \nabla(\phi)\cdot\boldsymbol{\diff{\ell}}
            =\phi(P_{2})-\phi(P_{1})
        \end{equation*}
        That is, the path integral of the gradient of
        $\phi$ is independent of the path. It only depends
        on the endpoints.
        This is \textit{multivariate} form of the
        Fundamental Theorem of Calculus. Using this, we have:
        \begin{align*}
            \int_{C}\mathbf{E}\cdot\boldsymbol{\diff{\ell}}
            =-\int _{(0,0,0)}^{(x,y,z)}
            \nabla(\phi)\cdot\boldsymbol{\diff{\ell}}
            =\phi(0,0,0)-\phi(x,y,z)
            =xyz-x^2
        \end{align*}
    \end{proof}
    \begin{problem}[Wangsness 5-3]
        \label{problem:EMAG_Wangsness_5_3}
        Give two point charges $q$ and $-q$ on the
        $z$ axis at $z=a$ and $z=-a$, respectively,
        Find $\phi$ everywhere. Show that the
        $xy$ plane is an equipotential surface.
    \end{problem}
    \begin{proof}[Solution]
        The potential is defined as:
        \begin{equation*}
            \phi=\sum_{k}\frac{q_{k}}{4\pi\epsilon{R_{k}}}
        \end{equation*}
        Using this, we obtain:
        \begin{equation*}
            \phi
            =\frac{1}{4\pi\epsilon_0}\bigg(
                \frac{q}{R_{+}}-\frac{q}{R_{-}}
            \bigg)
            =\frac{q}{4\pi\epsilon_0}
            \bigg(
                \frac{1}{\sqrt{x^2+y^2+(z-a)^2}}
                -\frac{1}{\sqrt{x^2+y^2+(z+a)^2}}
            \bigg)
        \end{equation*}
        Evaulating at $z=0$, we have:
        \begin{equation*}
            \phi_{z=0}
            =\frac{q}{4\pi\epsilon_0}\bigg(
                \frac{1}{\sqrt{x^2+y^2+a^2}}
                -\frac{1}{\sqrt{x^2+y^2+a^2}}
            \bigg)
            =0
        \end{equation*}
        Thus, the entire $xy$ plane is an
        equipotential surface with $\phi=0$.
    \end{proof}
    \begin{problem}[Wangsness 5-4]
        \label{problem:EMAG_Wangsness_5_4}
        Consider the charge distribution show in
        Fig.~\subref{fig:EMAG_Wangsness_5_4}.
        Find $\phi$ at the center of the square.
        Why can't you compute $\mathbf{E}$ at this
        point from your result?
    \end{problem}
    \begin{proof}[Solution]
    \begin{equation*}
        \phi=\frac{1}{4\pi\epsilon_{0}}
        \sum\frac{q_{i}}{R_{i}}
        =\frac{1}{4\pi \epsilon_0}\bigg(
            \frac{q}{\sqrt{a^2/4}}
            +\frac{2q}{\sqrt{a^2/2}}
            -\frac{4q}{\sqrt{a^2/2}}
            +\frac{3q}{\sqrt{a^2/2}}
        \bigg)
        =\frac{q}{\sqrt{2}\pi\epsilon_{0}a}    
    \end{equation*}
    To know $\mathbf{E}$ from $\phi$,
    $\phi$ must be known in some region about the point,
    not just at the point. To be precise in mathematical terms,
    we must know $\phi$ in some open set about the point in
    order to compute $\nabla(\phi)$. This is analogous to
    functions in calculus. Suppose $f$ is a function and
    $a$ is a real number, and suppose we know the value
    of $f(a)$. Can we determine what $f'(a)$ is?
    The answer is no, there is not enough information.
    If we know what $f(x)$ is in some interval $(a-\epsilon,a+\epsilon)$, then we can compute $f'(a)$.
    \end{proof}
    \begin{figure}[H]
        \centering
        \captionsetup{type=figure}
        \begin{subfigure}[b]{0.49\textwidth}
            \centering
            \subimport{../../../../tikz/}{Wangsness_5_3}
            \caption{Drawing for Wangsness 5-3}
            \label{fig:EMAG_Wangsness_5_3}
        \end{subfigure}
        \begin{subfigure}[b]{0.49\textwidth}
            \centering
            \subimport{../../../../tikz/}{Wangsness_5_4}
            \caption{Drawing for Wangsness 5-4}
            \label{fig:EMAG_Wangsness_5_4}
        \end{subfigure}
        \caption{%
            Drawings for problems
            \ref{problem:EMAG_Wangsness_5_3}
            and
            \ref{problem:EMAG_Wangsness_5_4}
        }
    \end{figure}
    \begin{problem}[Wangsness 5-10]
        Given a sphere with radius $a$ which has
        a charge density that varies by
        $\rho_{ch}(r)=Ar^{1/2}$ for $r<a$, where $A$
        is a constant, find $\phi$ at all points inside
        and outside of the sphere by using the path
        integral definition.
    \end{problem}
    \begin{proof}[Solution]
        The pontential difference between two points
        $P_{1}$ and $P_{2}$ is defined as:
        \begin{equation*}
            \Delta\phi=-\int_{P_{1}}^{P_{2}}
            \mathbf{E}\cdot\boldsymbol{\diff{\ell}}
        \end{equation*}
        We have that:
        \begin{equation*}
            \mathbf{E}
            =
            \begin{cases}
                \frac{2Aa^{7/2}}{7\epsilon_{0}r^{2}}
                \hat{\mathbf{r}},
                &r>a\\
                \frac{2Ar^{3/2}}{7\epsilon_{0}}
                \hat{\mathbf{r}},
                &r<a
            \end{cases}    
        \end{equation*}
        Now,
        $\boldsymbol{\diff{\ell}}%
         =-\diff{r}(-\hat{\mathbf{r}})%
         =\boldsymbol{\diff{r}}$.
        Let $\phi$ be $0$ at the origin.
        If $r_{0}<a$, we have:
        \begin{equation*}
            \phi
            =-\int_{0}^{r_{0}}
            \mathbf{E}\cdot\boldsymbol{\diff{\ell}}
            =-\frac{2A}{7\epsilon_{0}}
            \int_{0}^{r_{0}}r^{3/2}\diff{r}
            =-\frac{4Ar_{0}^{5/2}}{35\epsilon_{0}}
        \end{equation*}
        If $r_{0}>a$, then we have:
        \begin{equation*}
            \phi
            =-\int_{0}^{a}
            \mathbf{E}\cdot\boldsymbol{\diff{\ell}}
            -\int_{a}^{r_{0}}
            \mathbf{E}\cdot\boldsymbol{\diff{\ell}}
            =-\frac{4Aa^{5/2}}{35\epsilon_{0}}
            -\frac{2Aa^{7/2}}{7\epsilon_{0}}
            \int_{a}^{r_{0}}\frac{1}{r^{2}}\diff{r}
            =-\frac{4Aa^{5/2}}{35\epsilon_{0}}
            -\frac{2Aa^{7/2}}{7\epsilon_{0}}
            \Big(\frac{1}{r}-\frac{1}{a}\Big)
        \end{equation*}
    \end{proof}
    \begin{problem}[Wangsness 5-11]
        Given two concentric spheres with radii
        $a$ and $b$, with $a<b$, such that the region
        between them is filled with a charge of constant
        density $\rho_{ch}$, find $\phi$ at all points
        and express the answer in terms of $\rho_{ch}$.
    \end{problem}
        We have solved for the $\mathbf{E}$ field in
        a previous problem, and have that:
        \begin{equation*}
            \mathbf{E}
            =
            \begin{cases}
                \mathbf{0},
                &r<a\\
                \frac{Q}{4\pi \epsilon_0}
                \Big(\frac{r^3-a^3}{b^3-a^3}\Big)
                \hat{\mathbf{r}},
                &a\leq{r}\leq{b}\\
                \frac{Q}{4\pi\epsilon_{0}r^{2}}
                \hat{\mathbf{r}},
                &r>b
            \end{cases}    
        \end{equation*}
        We thus split the integral
        into three regions and compute:
        \begin{equation*}
            \Delta\phi
            =\int\mathbf{E}\cdot\boldsymbol{\diff{\ell}}
            =\int_{0}^{a}\mathbf{E}\cdot\mathbf{dr}
            +\int_{a}^{b}\mathbf{E}\cdot\mathbf{dr}
            +\int_{b}^{\infty}\mathbf{E}\cdot \mathbf{dr}    
            \Rightarrow\phi(\mathbf{r})
            =
            \begin{cases}
                \frac{\rho_{ch}(b^{3}-a^{3})}
                     {3\epsilon_{0}r},
                &r>b\\
                \frac{\rho_{ch}}{3\epsilon_{0}}\Big(
                    \frac{3}{2}b^{2}
                    -\frac{r^2}{2}
                    -\frac{a^3}{r}
                \Big),&a\leq{r}\leq{b}\\
                \frac{\rho_{ch}}{2\epsilon_0}(b^2-a^2),
                &r<a
            \end{cases}
        \end{equation*}
    \begin{problem}[Wangsness 5-14]
        Given a sphere of radius $a$ with constant
        surface charge density $\rho_{ch}$, but no
        volume charge density, find $\phi$ everywhere.
    \end{problem}
    \begin{proof}[Solution]
        $\phi$ may be defined as follows:
        \begin{equation*}
            \phi
            =\frac{1}{4\pi\epsilon_{0}}
            \iint_{\Sigma}\frac{\sigma_{ch}\diff{a}'}{R}
        \end{equation*}
        Here,
        $R=|\mathbf{r}-\mathbf{r}'|%
         =\sqrt{a^2+r^2-2ar\cos(\theta')}$,
        and
        $\diff{a}'=a^2\sin(\theta')\diff{\theta}'\diff{\phi}'$.
        So, we have
        \begin{equation*}
            \phi=\frac{\sigma_c}{4\pi\epsilon_0}
            \int_{0}^{2\pi}\int_{0}^{\pi}
            \frac{a^2\sin(\theta')\diff{\theta}'\diff{\phi}'}
                 {\sqrt{a^2+r^2-2ar\cos(\theta')}}
            =\frac{\sigma_c a^2}{2\epsilon_0}
            \int_{0}^{\pi}
            \frac{\sin(\theta')\diff{\theta}'}
                 {\sqrt{a^2-r^2-2ar\cos(\theta')}}
            =
            \begin{cases}
                \frac{a^{2}\sigma}{\epsilon_{0}r},
                &r>a\\
                \frac{a\sigma}{\epsilon_{0}},
                &r<a
            \end{cases}
        \end{equation*}
    \end{proof}
\end{document}