\documentclass[crop=false,class=article,oneside]{standalone}
%----------------------------Preamble-------------------------------%
%---------------------------Packages----------------------------%
\usepackage{geometry}
\geometry{b5paper, margin=1.0in}
\usepackage[T1]{fontenc}
\usepackage{graphicx, float}            % Graphics/Images.
\usepackage{natbib}                     % For bibliographies.
\bibliographystyle{agsm}                % Bibliography style.
\usepackage[french, english]{babel}     % Language typesetting.
\usepackage[dvipsnames]{xcolor}         % Color names.
\usepackage{listings}                   % Verbatim-Like Tools.
\usepackage{mathtools, esint, mathrsfs} % amsmath and integrals.
\usepackage{amsthm, amsfonts, amssymb}  % Fonts and theorems.
\usepackage{tcolorbox}                  % Frames around theorems.
\usepackage{upgreek}                    % Non-Italic Greek.
\usepackage{fmtcount, etoolbox}         % For the \book{} command.
\usepackage[newparttoc]{titlesec}       % Formatting chapter, etc.
\usepackage{titletoc}                   % Allows \book in toc.
\usepackage[nottoc]{tocbibind}          % Bibliography in toc.
\usepackage[titles]{tocloft}            % ToC formatting.
\usepackage{pgfplots, tikz}             % Drawing/graphing tools.
\usepackage{imakeidx}                   % Used for index.
\usetikzlibrary{
    calc,                   % Calculating right angles and more.
    angles,                 % Drawing angles within triangles.
    arrows.meta,            % Latex and Stealth arrows.
    quotes,                 % Adding labels to angles.
    positioning,            % Relative positioning of nodes.
    decorations.markings,   % Adding arrows in the middle of a line.
    patterns,
    arrows
}                                       % Libraries for tikz.
\pgfplotsset{compat=1.9}                % Version of pgfplots.
\usepackage[font=scriptsize,
            labelformat=simple,
            labelsep=colon]{subcaption} % Subfigure captions.
\usepackage[font={scriptsize},
            hypcap=true,
            labelsep=colon]{caption}    % Figure captions.
\usepackage[pdftex,
            pdfauthor={Ryan Maguire},
            pdftitle={Mathematics and Physics},
            pdfsubject={Mathematics, Physics, Science},
            pdfkeywords={Mathematics, Physics, Computer Science, Biology},
            pdfproducer={LaTeX},
            pdfcreator={pdflatex}]{hyperref}
\hypersetup{
    colorlinks=true,
    linkcolor=blue,
    filecolor=magenta,
    urlcolor=Cerulean,
    citecolor=SkyBlue
}                           % Colors for hyperref.
\usepackage[toc,acronym,nogroupskip,nopostdot]{glossaries}
\usepackage{glossary-mcols}
%------------------------Theorem Styles-------------------------%
\theoremstyle{plain}
\newtheorem{theorem}{Theorem}[section]

% Define theorem style for default spacing and normal font.
\newtheoremstyle{normal}
    {\topsep}               % Amount of space above the theorem.
    {\topsep}               % Amount of space below the theorem.
    {}                      % Font used for body of theorem.
    {}                      % Measure of space to indent.
    {\bfseries}             % Font of the header of the theorem.
    {}                      % Punctuation between head and body.
    {.5em}                  % Space after theorem head.
    {}

% Italic header environment.
\newtheoremstyle{thmit}{\topsep}{\topsep}{}{}{\itshape}{}{0.5em}{}

% Define environments with italic headers.
\theoremstyle{thmit}
\newtheorem*{solution}{Solution}

% Define default environments.
\theoremstyle{normal}
\newtheorem{example}{Example}[section]
\newtheorem{definition}{Definition}[section]
\newtheorem{problem}{Problem}[section]

% Define framed environment.
\tcbuselibrary{most}
\newtcbtheorem[use counter*=theorem]{ftheorem}{Theorem}{%
    before=\par\vspace{2ex},
    boxsep=0.5\topsep,
    after=\par\vspace{2ex},
    colback=green!5,
    colframe=green!35!black,
    fonttitle=\bfseries\upshape%
}{thm}

\newtcbtheorem[auto counter, number within=section]{faxiom}{Axiom}{%
    before=\par\vspace{2ex},
    boxsep=0.5\topsep,
    after=\par\vspace{2ex},
    colback=Apricot!5,
    colframe=Apricot!35!black,
    fonttitle=\bfseries\upshape%
}{ax}

\newtcbtheorem[use counter*=definition]{fdefinition}{Definition}{%
    before=\par\vspace{2ex},
    boxsep=0.5\topsep,
    after=\par\vspace{2ex},
    colback=blue!5!white,
    colframe=blue!75!black,
    fonttitle=\bfseries\upshape%
}{def}

\newtcbtheorem[use counter*=example]{fexample}{Example}{%
    before=\par\vspace{2ex},
    boxsep=0.5\topsep,
    after=\par\vspace{2ex},
    colback=red!5!white,
    colframe=red!75!black,
    fonttitle=\bfseries\upshape%
}{ex}

\newtcbtheorem[auto counter, number within=section]{fnotation}{Notation}{%
    before=\par\vspace{2ex},
    boxsep=0.5\topsep,
    after=\par\vspace{2ex},
    colback=SeaGreen!5!white,
    colframe=SeaGreen!75!black,
    fonttitle=\bfseries\upshape%
}{not}

\newtcbtheorem[use counter*=remark]{fremark}{Remark}{%
    fonttitle=\bfseries\upshape,
    colback=Goldenrod!5!white,
    colframe=Goldenrod!75!black}{ex}

\newenvironment{bproof}{\textit{Proof.}}{\hfill$\square$}
\tcolorboxenvironment{bproof}{%
    blanker,
    breakable,
    left=3mm,
    before skip=5pt,
    after skip=10pt,
    borderline west={0.6mm}{0pt}{green!80!black}
}

\AtEndEnvironment{lexample}{$\hfill\textcolor{red}{\blacksquare}$}
\newtcbtheorem[use counter*=example]{lexample}{Example}{%
    empty,
    title={Example~\theexample},
    boxed title style={%
        empty,
        size=minimal,
        toprule=2pt,
        top=0.5\topsep,
    },
    coltitle=red,
    fonttitle=\bfseries,
    parbox=false,
    boxsep=0pt,
    before=\par\vspace{2ex},
    left=0pt,
    right=0pt,
    top=3ex,
    bottom=1ex,
    before=\par\vspace{2ex},
    after=\par\vspace{2ex},
    breakable,
    pad at break*=0mm,
    vfill before first,
    overlay unbroken={%
        \draw[red, line width=2pt]
            ([yshift=-1.2ex]title.south-|frame.west) to
            ([yshift=-1.2ex]title.south-|frame.east);
        },
    overlay first={%
        \draw[red, line width=2pt]
            ([yshift=-1.2ex]title.south-|frame.west) to
            ([yshift=-1.2ex]title.south-|frame.east);
    },
}{ex}

\AtEndEnvironment{ldefinition}{$\hfill\textcolor{Blue}{\blacksquare}$}
\newtcbtheorem[use counter*=definition]{ldefinition}{Definition}{%
    empty,
    title={Definition~\thedefinition:~{#1}},
    boxed title style={%
        empty,
        size=minimal,
        toprule=2pt,
        top=0.5\topsep,
    },
    coltitle=Blue,
    fonttitle=\bfseries,
    parbox=false,
    boxsep=0pt,
    before=\par\vspace{2ex},
    left=0pt,
    right=0pt,
    top=3ex,
    bottom=0pt,
    before=\par\vspace{2ex},
    after=\par\vspace{1ex},
    breakable,
    pad at break*=0mm,
    vfill before first,
    overlay unbroken={%
        \draw[Blue, line width=2pt]
            ([yshift=-1.2ex]title.south-|frame.west) to
            ([yshift=-1.2ex]title.south-|frame.east);
        },
    overlay first={%
        \draw[Blue, line width=2pt]
            ([yshift=-1.2ex]title.south-|frame.west) to
            ([yshift=-1.2ex]title.south-|frame.east);
    },
}{def}

\AtEndEnvironment{ltheorem}{$\hfill\textcolor{Green}{\blacksquare}$}
\newtcbtheorem[use counter*=theorem]{ltheorem}{Theorem}{%
    empty,
    title={Theorem~\thetheorem:~{#1}},
    boxed title style={%
        empty,
        size=minimal,
        toprule=2pt,
        top=0.5\topsep,
    },
    coltitle=Green,
    fonttitle=\bfseries,
    parbox=false,
    boxsep=0pt,
    before=\par\vspace{2ex},
    left=0pt,
    right=0pt,
    top=3ex,
    bottom=-1.5ex,
    breakable,
    pad at break*=0mm,
    vfill before first,
    overlay unbroken={%
        \draw[Green, line width=2pt]
            ([yshift=-1.2ex]title.south-|frame.west) to
            ([yshift=-1.2ex]title.south-|frame.east);},
    overlay first={%
        \draw[Green, line width=2pt]
            ([yshift=-1.2ex]title.south-|frame.west) to
            ([yshift=-1.2ex]title.south-|frame.east);
    }
}{thm}

%--------------------Declared Math Operators--------------------%
\DeclareMathOperator{\adjoint}{adj}         % Adjoint.
\DeclareMathOperator{\Card}{Card}           % Cardinality.
\DeclareMathOperator{\curl}{curl}           % Curl.
\DeclareMathOperator{\diam}{diam}           % Diameter.
\DeclareMathOperator{\dist}{dist}           % Distance.
\DeclareMathOperator{\Div}{div}             % Divergence.
\DeclareMathOperator{\Erf}{Erf}             % Error Function.
\DeclareMathOperator{\Erfc}{Erfc}           % Complementary Error Function.
\DeclareMathOperator{\Ext}{Ext}             % Exterior.
\DeclareMathOperator{\GCD}{GCD}             % Greatest common denominator.
\DeclareMathOperator{\grad}{grad}           % Gradient
\DeclareMathOperator{\Ima}{Im}              % Image.
\DeclareMathOperator{\Int}{Int}             % Interior.
\DeclareMathOperator{\LC}{LC}               % Leading coefficient.
\DeclareMathOperator{\LCM}{LCM}             % Least common multiple.
\DeclareMathOperator{\LM}{LM}               % Leading monomial.
\DeclareMathOperator{\LT}{LT}               % Leading term.
\DeclareMathOperator{\Mod}{mod}             % Modulus.
\DeclareMathOperator{\Mon}{Mon}             % Monomial.
\DeclareMathOperator{\multideg}{mutlideg}   % Multi-Degree (Graphs).
\DeclareMathOperator{\nul}{nul}             % Null space of operator.
\DeclareMathOperator{\Ord}{Ord}             % Ordinal of ordered set.
\DeclareMathOperator{\Prin}{Prin}           % Principal value.
\DeclareMathOperator{\proj}{proj}           % Projection.
\DeclareMathOperator{\Refl}{Refl}           % Reflection operator.
\DeclareMathOperator{\rk}{rk}               % Rank of operator.
\DeclareMathOperator{\sgn}{sgn}             % Sign of a number.
\DeclareMathOperator{\sinc}{sinc}           % Sinc function.
\DeclareMathOperator{\Span}{Span}           % Span of a set.
\DeclareMathOperator{\Spec}{Spec}           % Spectrum.
\DeclareMathOperator{\supp}{supp}           % Support
\DeclareMathOperator{\Tr}{Tr}               % Trace of matrix.
%--------------------Declared Math Symbols--------------------%
\DeclareMathSymbol{\minus}{\mathbin}{AMSa}{"39} % Unary minus sign.
%------------------------New Commands---------------------------%
\DeclarePairedDelimiter\norm{\lVert}{\rVert}
\DeclarePairedDelimiter\ceil{\lceil}{\rceil}
\DeclarePairedDelimiter\floor{\lfloor}{\rfloor}
\newcommand*\diff{\mathop{}\!\mathrm{d}}
\newcommand*\Diff[1]{\mathop{}\!\mathrm{d^#1}}
\renewcommand*{\glstextformat}[1]{\textcolor{RoyalBlue}{#1}}
\renewcommand{\glsnamefont}[1]{\textbf{#1}}
\renewcommand\labelitemii{$\circ$}
\renewcommand\thesubfigure{%
    \arabic{chapter}.\arabic{figure}.\arabic{subfigure}}
\addto\captionsenglish{\renewcommand{\figurename}{Fig.}}
\numberwithin{equation}{section}

\renewcommand{\vector}[1]{\boldsymbol{\mathrm{#1}}}

\newcommand{\uvector}[1]{\boldsymbol{\hat{\mathrm{#1}}}}
\newcommand{\topspace}[2][]{(#2,\tau_{#1})}
\newcommand{\measurespace}[2][]{(#2,\varSigma_{#1},\mu_{#1})}
\newcommand{\measurablespace}[2][]{(#2,\varSigma_{#1})}
\newcommand{\manifold}[2][]{(#2,\tau_{#1},\mathcal{A}_{#1})}
\newcommand{\tanspace}[2]{T_{#1}{#2}}
\newcommand{\cotanspace}[2]{T_{#1}^{*}{#2}}
\newcommand{\Ckspace}[3][\mathbb{R}]{C^{#2}(#3,#1)}
\newcommand{\funcspace}[2][\mathbb{R}]{\mathcal{F}(#2,#1)}
\newcommand{\smoothvecf}[1]{\mathfrak{X}(#1)}
\newcommand{\smoothonef}[1]{\mathfrak{X}^{*}(#1)}
\newcommand{\bracket}[2]{[#1,#2]}

%------------------------Book Command---------------------------%
\makeatletter
\renewcommand\@pnumwidth{1cm}
\newcounter{book}
\renewcommand\thebook{\@Roman\c@book}
\newcommand\book{%
    \if@openright
        \cleardoublepage
    \else
        \clearpage
    \fi
    \thispagestyle{plain}%
    \if@twocolumn
        \onecolumn
        \@tempswatrue
    \else
        \@tempswafalse
    \fi
    \null\vfil
    \secdef\@book\@sbook
}
\def\@book[#1]#2{%
    \refstepcounter{book}
    \addcontentsline{toc}{book}{\bookname\ \thebook:\hspace{1em}#1}
    \markboth{}{}
    {\centering
     \interlinepenalty\@M
     \normalfont
     \huge\bfseries\bookname\nobreakspace\thebook
     \par
     \vskip 20\p@
     \Huge\bfseries#2\par}%
    \@endbook}
\def\@sbook#1{%
    {\centering
     \interlinepenalty \@M
     \normalfont
     \Huge\bfseries#1\par}%
    \@endbook}
\def\@endbook{
    \vfil\newpage
        \if@twoside
            \if@openright
                \null
                \thispagestyle{empty}%
                \newpage
            \fi
        \fi
        \if@tempswa
            \twocolumn
        \fi
}
\newcommand*\l@book[2]{%
    \ifnum\c@tocdepth >-3\relax
        \addpenalty{-\@highpenalty}%
        \addvspace{2.25em\@plus\p@}%
        \setlength\@tempdima{3em}%
        \begingroup
            \parindent\z@\rightskip\@pnumwidth
            \parfillskip -\@pnumwidth
            {
                \leavevmode
                \Large\bfseries#1\hfill\hb@xt@\@pnumwidth{\hss#2}
            }
            \par
            \nobreak
            \global\@nobreaktrue
            \everypar{\global\@nobreakfalse\everypar{}}%
        \endgroup
    \fi}
\newcommand\bookname{Book}
\renewcommand{\thebook}{\texorpdfstring{\Numberstring{book}}{book}}
\providecommand*{\toclevel@book}{-2}
\makeatother
\titleformat{\part}[display]
    {\Large\bfseries}
    {\partname\nobreakspace\thepart}
    {0mm}
    {\Huge\bfseries}
\titlecontents{part}[0pt]
    {\large\bfseries}
    {\partname\ \thecontentslabel: \quad}
    {}
    {\hfill\contentspage}
\titlecontents{chapter}[0pt]
    {\bfseries}
    {\chaptername\ \thecontentslabel:\quad}
    {}
    {\hfill\contentspage}
\newglossarystyle{longpara}{%
    \setglossarystyle{long}%
    \renewenvironment{theglossary}{%
        \begin{longtable}[l]{{p{0.25\hsize}p{0.65\hsize}}}
    }{\end{longtable}}%
    \renewcommand{\glossentry}[2]{%
        \glstarget{##1}{\glossentryname{##1}}%
        &\glossentrydesc{##1}{~##2.}
        \tabularnewline%
        \tabularnewline
    }%
}
\newglossary[not-glg]{notation}{not-gls}{not-glo}{Notation}
\newcommand*{\newnotation}[4][]{%
    \newglossaryentry{#2}{type=notation, name={\textbf{#3}, },
                          text={#4}, description={#4},#1}%
}
%--------------------------LENGTHS------------------------------%
% Spacings for the Table of Contents.
\addtolength{\cftsecnumwidth}{1ex}
\addtolength{\cftsubsecindent}{1ex}
\addtolength{\cftsubsecnumwidth}{1ex}
\addtolength{\cftfignumwidth}{1ex}
\addtolength{\cfttabnumwidth}{1ex}

% Indent and paragraph spacing.
\setlength{\parindent}{0em}
\setlength{\parskip}{0em}
%--------------------------Main Document----------------------------%
\begin{document}
    \ifx\ifphysicscourseselectromagnetismI\undefined
        \section*{Electromagnetism I}
        \setcounter{section}{3}
        \renewcommand\thefigure{%
            \arabic{section}.\arabic{figure}%
        }
        \renewcommand\thesubfigure{%
            \arabic{section}.\arabic{figure}.\arabic{subfigure}%
        }
    \fi    
    \subsection{Homework III}
        Wangsness Chapter 1 - Problems: 19, 20, 21, 22, 23, 24, 26
        \begin{problem}[Wangsness 1-19]
            Let
            $\mathbf{A}%
             =a\hat{\boldsymbol{\uprho}}%
             +b\hat{\boldsymbol{\upvarphi}}%
             +c\hat{\mathbf{z}}$,
            where $a,b,c$ are constants. Is $\mathbf{A}$ a
            constant vector? Find $\nabla\cdot\mathbf{A}$ and
            $\nabla\times\mathbf{A}$. Find the rectangular
            and spherical components of $\mathbf{A}$, expressing
            in terms of $x$, $y$, $z$ and
            $r$, $\theta$, $\varphi$, respectively.
        \end{problem}
        \begin{proof}[Solution]
            If $a$ or $b$ are non-zero, then $\mathbf{A}$
            is not a constant, for
            $\hat{\boldsymbol{\uprho}}$ and
            $\hat{\boldsymbol{\upvarphi}}$ are
            non-constant functions of $\varphi$. To compute
            $\nabla\cdot\mathbf{A}$, we use $\nabla$ in
            cylindrical coordinates and do:
            \begin{equation*}
                \nabla\cdot\mathbf{A}
                =\frac{\partial{A_{\rho}}}{\partial\rho}
                +\frac{A_{\rho}}{\rho}
                +\frac{1}{\rho}
                 \frac{\partial{A_{\phi}}}{\partial\varphi}
                +\frac{\partial{A_{z}}}{\partial{z}}
                =\frac{\partial{a}}{\partial\rho}
                +\frac{a}{\rho}
                +\frac{1}{\rho}
                 \frac{\partial{b}}{\partial\varphi}
                +\frac{\partial{c}}{\partial{z}}
                =\frac{a}{\rho}
            \end{equation*}
            For $\nabla\times\mathbf{A}$,
            we again use cylindrical coordinates and do:
            \begin{align*}
                \nabla\times\mathbf{A}
                =
                \begin{vmatrix}
                    \frac{1}{\rho}\hat{\boldsymbol{\uprho}}
                    &\hat{\boldsymbol{\upvarphi}}
                    &\frac{1}{\rho}\hat{\mathbf{z}}\\
                    \frac{\partial}{\partial\rho}
                    &\frac{\partial}{\partial\varphi}
                    &\frac{\partial}{\partial{z}}\\
                    A_{\rho}
                    &\rho{A_{\varphi}}
                    &A_{z}
                \end{vmatrix}
                &=\frac{\hat{\boldsymbol{\uprho}}}{\rho}\bigg(
                    \frac{\partial{A_{z}}}{\partial\varphi}
                    -\frac{\partial(\rho{A_{\varphi}})}
                          {\partial{z}}
                \bigg)
                +\hat{\boldsymbol{\upvarphi}}\bigg(
                    \frac{\partial{A_{\rho}}}{\partial{z}}
                    -\frac{\partial{A_{z}}}{\partial\rho}
                \bigg)
                +\frac{\hat{\mathbf{z}}}{\rho}\bigg(
                    \frac{\partial(\rho{A_{\varphi}})}
                         {\partial\rho}
                    -\frac{\partial{A_{\rho}}}
                          {\partial\varphi}
                \bigg)\\
                &=\hat{\boldsymbol{\uprho}}\bigg(
                    \frac{1}{\rho}
                    \frac{\partial{A_{z}}}{\partial\varphi}
                    -\frac{\partial{A_{\varphi}}}
                          {\partial{z}}
                \bigg)
                +\hat{\boldsymbol{\upvarphi}}\bigg(
                    \frac{\partial{A_{\rho}}}{\partial{z}}
                    -\frac{\partial{A_{z}}}{\partial\rho}
                \bigg)
                +\hat{\mathbf{z}}\bigg(
                    \frac{1}{\rho}
                    \frac{\partial(\rho{A_{\varphi}})}
                         {\partial\rho}
                    -\frac{1}{\rho}
                    \frac{\partial{A_{\rho}}}
                         {\partial\varphi}
                \bigg)\\
                &=\hat{\boldsymbol{\uprho}}(0-0)
                 +\hat{\boldsymbol{\upvarphi}}(0-0)
                 +\hat{\mathbf{z}}(\frac{A_{\phi}}{\rho}+0-0)
                 =\frac{b}{\rho}\hat{\mathbf{z}}
            \end{align*}
            Rectangular coordinates of $\mathbf{A}$:
            \begin{align*}
                \mathbf{A}
                &=a(\cos(\phi)\hat{\mathbf{x}}
                +\sin(\phi)\hat{\mathbf{y}})
                +b(-\sin(\phi)\hat{\mathbf{x}}
                +\cos(\phi)\hat{\mathbf{y}})
                +c\hat{\mathbf{z}}\\
                &=a\bigg(
                    \frac{x\hat{\mathbf{x}}
                    +y\hat{\mathbf{y}}}{\sqrt{x^{2}+y^{2}}}
                \bigg)
                +b\bigg(
                    \frac{-y\hat{\mathbf{x}}
                    +x\hat{\mathbf{y}}}{\sqrt{x^{2}+y^{2}}}
                \bigg)
                +c\hat{\mathbf{z}}
                =\frac{ax-by}{\sqrt{x^{2}+y^{2}}}\hat{\mathbf{x}}
                +\frac{ay+bx}{\sqrt{x^{2}+y^{2}}}\hat{\mathbf{y}}
                +c\hat{\mathbf{z}}
            \end{align*}
            For spherical coordinates:
            \begin{align*}
                \hat{\boldsymbol{\uprho}}
                &=\sin(\theta)\hat{\mathbf{r}}
                +\cos(\theta)\hat{\boldsymbol{\uptheta}}
                &\mathbf{A}
                &=a(\sin(\theta)\hat{\mathbf{r}}
                +\cos(\theta)\hat{\boldsymbol{\uptheta}})
                +b\hat{\boldsymbol{\upvarphi}}
                +c(\cos(\theta)\hat{\mathbf{r}}
                -\sin(\theta)\hat{\boldsymbol{\uptheta}})\\
                \hat{\mathbf{z}}
                &=\cos(\theta)\hat{\mathbf{r}}
                -\cos(\theta)\hat{\boldsymbol{\uptheta}}
                &
                &=(a\sin(\theta)
                +c\cos(\theta))\hat{\mathbf{r}}
                +b\hat{\boldsymbol{\upvarphi}}
                +(a\cos(\theta)-c\sin(\theta))
                \hat{\boldsymbol{\uptheta}}
            \end{align*}
        \end{proof}
        \begin{problem}[Wangsness 1-20]
            Let
            $\mathbf{A}%
             =a\hat{\mathbf{r}}%
             +b\hat{\boldsymbol{\uptheta}}%
             +c\hat{\boldsymbol{\upvarphi}}$.
            Is $\mathbf{A}$ a constant vector? Find
            $\nabla\cdot\mathbf{A}$ and
            $\nabla\times\mathbf{A}$. Find the rectangular
            and cylindrical components of $\mathbf{A}$,
            expressing in terms of $x$, $y$, $z$ and
            $\rho$, $\varphi$, $z$, respectively.
        \end{problem}
        \begin{proof}[Solution]
            If $a$ or $b$ or $c$ are non-zero,
            then $\mathbf{A}$ is not a constant
            vector, for $\hat{\mathbf{r}}$,
            $\hat{\boldsymbol{\uptheta}}$, and
            $\hat{\boldsymbol{\upvarphi}}$ are
            non-constant functions of
            $r$, $\theta$, $\varphi$.
            To compute $\nabla\cdot\mathbf{A}$,
            we use spherical coordinates and do:
            \begin{equation*}
                \nabla\cdot\mathbf{A}
                =\frac{1}{r^{2}}
                \frac{\partial(r^{2}A_{r})}{\partial{r}}
                +\frac{1}{r\sin(\theta)}
                \frac{\partial(\sin(\theta)A_{\theta})}
                     {\partial\theta}
                +\frac{1}{r\sin(\theta)}
                \frac{\partial A_{\varphi}}{\partial\varphi}
                =\frac{2a}{r}
                +\frac{b\cos(\theta)}{r\sin(\theta)}
            \end{equation*}
            For $\nabla\times\mathbf{A}$:
            \begin{align*}
                \nabla\times\mathbf{A}
                &=
                \begin{vmatrix}
                    \frac{1}{r^{2}\sin(\theta)}\hat{\mathbf{r}}
                    &\frac{1}{r\sin(\theta)}\hat{\boldsymbol{\uptheta}}
                    &\frac{1}{r}\hat{\boldsymbol{\upvarphi}}\\
                    \frac{\partial}{\partial{r}}
                    &\frac{\partial}{\partial\theta}
                    &\frac{\partial}{\partial\varphi}\\
                    A_{r}
                    &rA_{\theta}
                    &r\sin(\theta)A_{\varphi}
                \end{vmatrix}\\
                &=\frac{\hat{\mathbf{r}}}{r\sin(\theta)}
                \bigg(
                    \frac{\partial(\sin(\theta)A_{\varphi})}
                         {\partial\theta}
                    -\frac{\partial{A_{\theta}}}
                          {\partial \varphi}
                \bigg)
                +\frac{\hat{\boldsymbol{\uptheta}}}{r}
                \bigg(
                    \frac{1}{\sin(\theta)}
                    \frac{\partial{A_{r}}}{\partial\varphi}
                    -\frac{\partial(rA_{\varphi})}{\partial{r}}
                \bigg)
                +\frac{\hat{\boldsymbol{\upvarphi}}}{r}
                \bigg(
                    \frac{\partial(rA_{\theta})}{\partial{r}}
                    -\frac{\partial{A_{r}}}{\partial\theta}
                \bigg)\\
                &=\frac{\cos(\theta)}{r\sin(\theta)}\hat{\mathbf{r}}
                -\frac{c}{r}\hat{\boldsymbol{\uptheta}}
                +\frac{b}{r}\hat{\boldsymbol{\upvarphi}}
            \end{align*}
            In rectangular coordinates, we have:
            \begin{align*}
                \mathbf{A}=\hspace{0.5em}
                &a\big(\sin(\theta)\cos(\varphi)
                 \hat{\mathbf{x}}
                +\sin(\theta)\sin(\varphi)
                 \hat{\mathbf{y}}
                +\cos(\theta)\hat{\mathbf{z}}\big)+\\
                &b\big(
                    \cos(\theta)\cos(\varphi)
                    \hat{\mathbf{x}}
                    +\cos(\theta)\sin(\varphi)
                    \hat{\mathbf{y}}
                    -\sin(\theta)\hat{\mathbf{z}}
                \big)+\\
                &c\big(
                    -\sin(\varphi)\hat{\mathbf{x}}
                    +\cos(\varphi)\hat{\mathbf{y}}
                \big)\\
                =\hspace{0.5em}
                &\frac{a}{\sqrt{x^{2}+y^{2}+z^{2}}}
                \bigg(
                    x\hat{\mathbf{x}}
                    +y\hat{\mathbf{y}}
                    +z\hat{\mathbf{z}}
                \bigg)+\\
                &\frac{b}{\sqrt{x^{2}+y^{2}+z^{2}}}
                \bigg(
                    \frac{xz}{\sqrt{x^{2}+y^{2}}}
                    \hat{\mathbf{x}}
                    +\frac{yz}{\sqrt{x^{2}+y^{2}}}
                    \hat{\mathbf{y}}
                    -\sqrt{x^{2}+y^{2}}\hat{\mathbf{z}}
                \bigg)+\\
                &-\frac{y}{\sqrt{x^{2}+y^{2}}}\hat{\mathbf{x}}
                +\frac{x}{\sqrt{x^{2}+y^{2}}}\hat{\mathbf{y}}\\
                =\hspace{0.5em}
                &\bigg(
                    \frac{ax}{\sqrt{x^{2}+y^{2}+z^{2}}}
                    +\frac{bxz}
                          {\sqrt{x^{2}+y^{2}}
                           \sqrt{x^{2}+y^{2}+z^{2}}}
                    -\frac{y}{\sqrt{x^{2}+y^{2}}}
                \bigg)\hat{\mathbf{x}}+\\
                &\bigg(
                    \frac{ay}{\sqrt{x^{2}+y^{2}+z^{2}}}
                    +\frac{byz}
                          {\sqrt{x^{2}+y^{2}}
                           \sqrt{x^{2}+y^{2}+z^{2}}}
                        +\frac{x}{\sqrt{x^{2}+y^{2}}}
                \bigg)\hat{\mathbf{y}}+\\
                &\bigg(
                    \frac{az}{\sqrt{x^{2}+y^{2}+z^{2}}}
                    -\frac{b\sqrt{x^{2}+y^{2}}}
                          {\sqrt{x^{2}+y^{2}+z^{2}}}
                \bigg)
                \hat{\mathbf{z}}
            \end{align*}
            We can then use this to convert to cylindrical,
            recalling that $r^{2}=\rho^{2}+z^{2}$:
            \begin{equation*}
                \mathbf{A}
                =\bigg(
                    \frac{a\rho+bz}{\sqrt{\rho^{2}+z^{2}}}
                \bigg)
                \hat{\boldsymbol{\uprho}}
                +c\hat{\boldsymbol{\upvarphi}}
                +\bigg(
                    \frac{az-b\rho}{\sqrt{\rho^{2}+z^{2}}}
                \bigg)\hat{\mathbf{z}}
            \end{equation*}
        \end{proof}
        \begin{problem}[Wangsness 1-21]
            Find $\nabla\cdot\mathbf{r}$ for the position
            vector $\mathbf{r}$ expressed in rectangular,
            cylindrical, and spherical coordinates.
        \end{problem}
        \begin{proof}[Solution]
            In rectangular coordinates we have
            $\mathbf{r}%
             =x\hat{\mathbf{x}}%
             +y\hat{\mathbf{y}}%
             +z\hat{\mathbf{z}}$.
            So:
            \begin{equation*}
                \nabla\cdot\mathbf{r}
                =\frac{\partial x}{\partial x}
                +\frac{\partial y}{\partial y}
                +\frac{\partial z}{\partial z}
                =1+1+1
                =3
            \end{equation*}
            In cylindrical coordinates,
            $\mathbf{r}%
             =\rho\hat{\boldsymbol{\uprho}}%
             +z\hat{\mathbf{z}}$,
            So:
            \begin{equation*}
                \nabla\cdot\mathbf{r}
                =\frac{1}{\rho}
                \frac{\partial}{\partial\rho}
                \big(\rho^2\big)
                +\frac{1}{\rho}
                \frac{\partial}{\partial\phi}
                \big(0\big)
                +\frac{\partial z}{\partial z}
                =2+0+1
                =3
            \end{equation*}
            In spherical coordinates we have:
            \begin{equation*}
                \nabla\cdot\mathbf{r}
                =\frac{1}{r^{2}}
                \frac{\partial}{\partial{r}}
                \big(r^{2}r\big)
                =3
            \end{equation*}
        \end{proof}
        \begin{problem}[Wangsness 1-22]
        \label{problem:EMAG_wangsness_1_22}
            Let
            $\mathbf{A}%
             =a\rho\hat{\boldsymbol{\uprho}}%
             +b\hat{\boldsymbol{\upvarphi}}%
             +cz\hat{\mathbf{z}}$,
            where $a$, $b$, and $c$ are constants.
            Find $\oiint\mathbf{A}\cdot\boldsymbol{\diff{a}}$
            over the surface of a right circular cylinder of length
            $L$ and radius $\rho_{0}$ whose axis is along the
            positive $z$ axis and the origin is the center
            of the lower circular face
            (See Fig.~\subref{fig:EMAG_1_wangsness_1_22}).
            Find $\iiint\nabla\cdot\mathbf{A}\diff{\tau}$
            over the volume of the cylinder.
        \end{problem}
        \begin{proof}[Solution]
            We have that:
            \begin{equation*}
                \oint\mathbf{A}\cdot\boldsymbol{\diff{a}}
                =\int_{Top}\mathbf{A}\cdot\boldsymbol{\diff{a}}
                +\int_{Cylinder}\mathbf{A}\cdot\boldsymbol{\diff{a}}
                +\int_{Bottom}\mathbf{A}\cdot\boldsymbol{\diff{a}}
            \end{equation*}
            On the cylindrical surface, 
            $\diff{a}=\rho_{0}\diff{\phi}\diff{z}$,
            so the integral is:
            \begin{align*}
                \int_{0}^{L}\int_{0}^{2\pi}
                (a\rho_0\hat{\boldsymbol{\uprho}}
                 +b\hat{\boldsymbol{\upvarphi}}
                 +cz\hat{\mathbf{z}})
                \cdot\hat{\boldsymbol{\uprho}}\rho_{0}
                \diff{\varphi}\diff{z}
                &=\int_{0}^{L}\int_{0}^{2\pi}a\rho_{0}^{2}
                \diff{\phi}\diff{z}
                =a\rho_0^{2}(2\pi)L    
            \end{align*}
            For the top and bottom,
            $\diff{a}%
             =\pm\rho\diff{\rho}\diff{\phi}\hat{\mathbf{z}}$,
            respectively. On the bottom surface $z=0$ and thus
            the integral is zero. On the top we get:
            \begin{equation*}
                \int_{0}^{2\pi}\int_{0}^{\rho_0}
                cL\rho\diff{\rho}\diff{\phi}
                =\pi cL\rho_0^{2}
            \end{equation*}
            Thus:
            \begin{equation*}
                \oiint\mathbf{A}\cdot\boldsymbol{\diff{a}}
                =\pi{L}\rho_0^{2}(2a+c)
            \end{equation*}
            Computing the divergence, we get
            $\nabla\cdot\mathbf{A}=2a+c$. Therefore:
            \begin{equation*}
                \iiint_C\nabla\cdot\mathbf{A}\diff{\tau}
                =(2a+c)V=\pi{L}\rho_{0}^{2}(2a+c)
            \end{equation*}
        \end{proof}
        \begin{figure}[H]
            \centering
            \captionsetup{type=figure}
            \begin{subfigure}[b]{0.49\textwidth}
                \centering
                \captionsetup{type=figure}
                \subimport{../../../../tikz/}{Wangsness_1_22}
                \caption{Wangsness 1-22}
                \label{fig:EMAG_1_wangsness_1_22}
            \end{subfigure}
            \begin{subfigure}[b]{0.49\textwidth}
                \centering
                \captionsetup{type=figure}
                \subimport{../../../../tikz/}{Wangsness_1_23}
                \caption{Wangsness 1-23}
                \label{fig:EMAG_1_wangsness_1_23}
            \end{subfigure}
            \caption[Figures for Wangsness 1-22 and 1-23]{%
                Figures for problems \ref{problem:EMAG_wangsness_1_22}
                and \ref{problem:EMAG_wangsness_1_23}, respectively.
            }
        \end{figure}
        \begin{problem}[Wangsness 1-23]
            \label{problem:EMAG_wangsness_1_23}
            Let $\mathbf{A}=4\hat{\mathbf{r}}+3\hat{\boldsymbol{\uptheta}}
            -2\hat{\boldsymbol{\upvarphi}}$. Find the line integral
            around the closed path shown in
            Fig.~\subref{fig:EMAG_1_wangsness_1_23}.
            Find the surface integral of
            $\nabla \times \mathbf{A}$ over the enclosed area.
        \end{problem}
        \begin{proof}[Solution]
            We have:
            \begin{equation*}
                \oint_{\partial S}
                \mathbf{A}\cdot\boldsymbol{\diff{\ell}}
                =\sum_{i}\int_{\partial{S_{i}}}
                \mathbf{A}\cdot\boldsymbol{\diff{\ell}}
            \end{equation*}
            Along the first path $\varphi=0$,
            $\theta=\frac{\pi}{2}$, and
            $\boldsymbol{\diff{\ell}}%
             =\diff{r}\hat{\mathbf{r}}$.
            The integral is then $4r_{0}$.
            \begin{equation*}
                \int_{\partial S_{1}}
                \mathbf{A}\cdot\boldsymbol{\diff{\ell}}
                =\int_{0}^{r_{0}}
                (4\hat{\mathbf{r}}
                +3\hat{\boldsymbol{\uptheta}}
                -2\hat{\boldsymbol{\upvarphi}})
                \cdot(\hat{\mathbf{r}}\diff{r})
                =4\int_{0}^{r_{0}}\diff{r}
                =4r_{0}
            \end{equation*}
            Along the second path $r=r_{0}$,
            $\theta=\frac{\pi}{2}$, and 
            $\boldsymbol{\diff{\ell}}%
             =r_{0}\diff{\varphi}\hat{\boldsymbol{\upvarphi}}$.
            \begin{equation*}
                \int_{\partial S_{2}}
                \mathbf{A}\cdot\boldsymbol{\diff{\ell}}
                =\int_{0}^{\frac{\pi}{2}}(4\hat{\mathbf{r}}
                +3\hat{\boldsymbol{\uptheta}}
                -2\hat{\boldsymbol{\upvarphi}})
                \cdot(r_{0}d\varphi\hat{\boldsymbol{\upvarphi}})
                =-2\int_{0}^{\frac{\pi}{2}}r_{0}d\varphi
                =-\pi r_{0}
            \end{equation*}
            Along the final path, $\varphi=\frac{\pi}{2}$,
            $\theta=\frac{\pi}{2}$, and
            $\boldsymbol{\diff{\ell}}%
             =\diff{r}\hat{\mathbf{r}}$.
            \begin{equation*}
                \int_{\partial S_{3}}
                \mathbf{A}\cdot\boldsymbol{\diff{\ell}}
                =\int_{r_{0}}^{0}
                (4\hat{\mathbf{r}}
                 +3\hat{\boldsymbol{\uptheta}}
                 -2\hat{\boldsymbol{\upvarphi}})
                \cdot(\hat{\mathbf{r}}\diff{r})
                =4\int_{r_{0}}^{0}\diff{r}
                =-4r_{0}
            \end{equation*}
            Therefore:
            \begin{equation*}
                \oint_{\partial S}
                \mathbf{A}\cdot\boldsymbol{\diff{\ell}}
                =\sum_{i}\int_{\partial S_{i}}
                \mathbf{A}\cdot\boldsymbol{\diff{\ell}}
                =4r_{0}-\pi{r}_{0}-4r_{0}
                =-\pi r_{0}
            \end{equation*}
            The curl is:
            $\nabla\times\mathbf{A}%
             =\frac{-2\cot(\theta)}{r}\hat{\mathbf{r}}%
             +\frac{2}{r}\hat{\boldsymbol{\uptheta}}%
             +\frac{3}{r}\hat{\boldsymbol{\upvarphi}}$.
            For the plane,
            $\boldsymbol{\diff{a}}%
             =\hat{\mathbf{z}}r\diff{r}\diff{\varphi}$,
            $\theta=\frac{\pi}{2}$. So:
            \begin{align*}
                \iint\nabla\times
                \mathbf{A}\cdot\boldsymbol{\diff{a}}
                &=\int_{0}^{\frac{\pi}{2}}\int_{0}^{r_{0}}
                \bigg(
                    \frac{-2\cot(\theta)}{r}\hat{\mathbf{r}}
                    +\frac{2}{r}\hat{\boldsymbol{\uptheta}}
                    +\frac{3}{r}\hat{\boldsymbol{\upvarphi}}
                \bigg)
                \cdot\hat{\mathbf{z}}r\diff{r}\diff{\varphi}\\
                &=\int_{0}^{\frac{\pi}{2}}\int_{0}^{r_{0}}
                \bigg(
                    -2\cot(\theta)\cos(\theta)
                    -2\sin(\theta)
                \bigg)\diff{r}\diff{\varphi}
                =-2\int_{0}^{\frac{\pi}{2}}
                \int_{0}^{r_{0}}\diff{r}\diff{\varphi}
                =-\pi r_{0}
            \end{align*}
        \end{proof}
        \begin{problem}[Wangsness 1-24]
            Verify that
            $\nabla\times(u\mathbf{A})%
             =\nabla(u)\times\mathbf{A}%
             +u(\nabla\times\mathbf{A})$
        \end{problem}
        \begin{proof}[Solution]
            Let $\mathbf{A}=\langle{A_{x},A_{y},A_{z}}\rangle$
            and $u=u(x,y,z)$.
            Using the product rule, we get:
            \begin{align*}
                \nabla\times (u\mathbf{A})
                &=
                \begin{vmatrix}
                    \hat{\mathbf{x}}
                    &\hat{\mathbf{y}}
                    &\hat{\mathbf{z}}\\
                    \frac{\partial}{\partial{x}}
                    &\frac{\partial}{\partial{y}}
                    &\frac{\partial}{\partial{z}}\\
                    uA_{x}
                    &uA_{y}
                    &uA_{z}
                \end{vmatrix}\\
                &=\hat{\mathbf{x}}\big(
                    \frac{\partial{u}}{\partial{y}}A_{z}
                    +u\frac{\partial{A_{z}}}{\partial{y}}
                    -\frac{\partial{u}}{\partial{z}}A_{y}
                    -u\frac{\partial{A_{y}}}{\partial{z}}
                \big)+\\
                &\hspace{1.4em}\hat{\mathbf{y}}\big(
                    \frac{\partial{u}}{\partial{z}}A_{x}
                    +u\frac{\partial{A_{x}}}{\partial{z}}
                    -\frac{\partial{u}}{\partial{x}}A_{z}
                    -u\frac{\partial{A_{z}}}{\partial{x}}
                \big)+\\
                &\hspace{1.4em}
                \hat{\mathbf{z}}\big(
                    \frac{\partial{u}}{\partial{x}}A_{y}+
                    u\frac{\partial{A_{y}}}{\partial{x}}
                    -\frac{\partial{u}}{\partial{y}}A_{x}
                    -u\frac{\partial{A_{x}}}{\partial{y}}
                \big)
            \end{align*}
            But:
            \begin{equation*}
                \nabla(u)\times\mathbf{A}
                =\hat{\mathbf{x}}\big(
                    \frac{\partial u}{\partial y}
                    A_{z}-\frac{\partial u}{\partial z}A_{y}
                \big)
                +\hat{\mathbf{y}}\big(
                    \frac{\partial u}{\partial z}A_{x}
                    -\frac{\partial u}{\partial x}A_{z}
                \big)
                +\hat{\mathbf{z}}\big(
                    \frac{\partial u}{\partial x}A_{y}
                    -\frac{\partial u}{\partial y}A_{x}
                \big)
            \end{equation*}
            and
            \begin{equation*}
                u(\nabla\times\mathbf{A})
                =\hat{\mathbf{x}}\big(
                    u\frac{\partial{A_{z}}}{\partial{y}}
                    -u\frac{\partial{A_{y}}}{\partial{z}}
                \big)
                +\hat{\mathbf{y}}\big(
                    u\frac{\partial{A_{y}}}{\partial{z}}
                    -u\frac{\partial{A_{z}}}{\partial{y}}A_{x}
                \big)
                +\hat{\mathbf{z}}\big(
                    \frac{u\partial{A_{y}}}{\partial{x}}
                    -u\frac{\partial{A_{x}}}{\partial{y}}
                \big)
            \end{equation*}
            Summing these, we have
            $\nabla\times(u\mathbf{A})%
             =\nabla(u)\times\mathbf{A}%
             +u\nabla\times\mathbf{A}$
        \end{proof}
        \begin{problem}[Wangsness 1-26]
            Verify that
            $\oint_{S}u\boldsymbol{\diff{a}}%
             =\int_{V}\nabla(u)\diff{\tau}$
            and
            $\oint_{S}\mathbf{A}\times\boldsymbol{\diff{a}}%
             =-\int_{V}\nabla\times\mathbf{A}\diff{\tau}$
        \end{problem}
        \begin{proof}[Solution]
            Let $\mathbf{C}$ be an arbitrary constant vector. Then:
            \begin{equation*}
                \mathbf{C}\cdot\bigg(
                    \oiint_{S}u\boldsymbol{\diff{a}}
                    -\iiint_{V}\nabla(u)\diff{\tau}
                \bigg)
                =\oiint_{S}u\bigg(
                    \mathbf{C}\cdot\boldsymbol{\diff{a}}
                \bigg)
                -\iiint_{V}\bigg(
                    \mathbf{C}\cdot\nabla(u)
                \bigg)\diff{\tau}
            \end{equation*}
            But, as $\mathbf{C}$ is constant,
            $\nabla\cdot\mathbf{C}=0$, and thus we have:
            \begin{align*}
                \iiint_{V}\mathbf{C}\cdot\nabla(u)\diff{\tau}
                &=\iiint_{V}\nabla\cdot(u\mathbf{C})\diff{\tau}
                -\iiint_{V}u\nabla\cdot\mathbf{C}\diff{\tau}\\
                &=\iiint_{V}\nabla(u\mathbf{C})\diff{\tau}
            \end{align*}
            But from the divergence theorem:
            \begin{equation*}
                \iiint_{V}\nabla(u\mathbf{C})\diff{\tau}
                =\oiint_{S}u\mathbf{C}\cdot\boldsymbol{\diff{a}}
            \end{equation*}
            Thus,
            $\mathbf{C}\cdot\big(%
                 \oint_{S}u\boldsymbol{\diff{a}}%
                 -\int_{V}\nabla(u)\diff{\tau}%
            \big)=0$.
            As $\mathbf{C}$ is any arbitrary vector,
            $\oint_{S}u\boldsymbol{\diff{a}}%
             -\int_{V}\nabla(u)\diff{\tau}=0$
            and thus
            $\oint_{S}u\boldsymbol{\diff{a}}%
             =\int_{V}\nabla(u)\diff{\tau}$.
            It is a simple exercise in vector geometry
            to show that if $\mathbf{A}\cdot\mathbf{C}=0$
            for all vectors $\mathbf{C}$, then
            $\mathbf{A}=\mathbf{0}$. In an analogous manner:
            \begin{align*}
                \mathbf{C}\cdot\bigg(
                    \oiint_{S}\mathbf{A}\times\boldsymbol{\diff{a}}
                    +\iiint_{V}\nabla\times\mathbf{A}\diff{\tau}
                \bigg)
                =0\Rightarrow\oiint_{S}
                \mathbf{A}\times\boldsymbol{\diff{a}}
                =-\iiint_{V}\nabla\times\mathbf{A}\diff{\tau}
            \end{align*}
        \end{proof}
\end{document}