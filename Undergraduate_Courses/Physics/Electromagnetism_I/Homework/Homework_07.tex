\documentclass[crop=false,class=article,oneside]{standalone}
%----------------------------Preamble-------------------------------%
%---------------------------Packages----------------------------%
\usepackage{geometry}
\geometry{b5paper, margin=1.0in}
\usepackage[T1]{fontenc}
\usepackage{graphicx, float}            % Graphics/Images.
\usepackage{natbib}                     % For bibliographies.
\bibliographystyle{agsm}                % Bibliography style.
\usepackage[french, english]{babel}     % Language typesetting.
\usepackage[dvipsnames]{xcolor}         % Color names.
\usepackage{listings}                   % Verbatim-Like Tools.
\usepackage{mathtools, esint, mathrsfs} % amsmath and integrals.
\usepackage{amsthm, amsfonts, amssymb}  % Fonts and theorems.
\usepackage{tcolorbox}                  % Frames around theorems.
\usepackage{upgreek}                    % Non-Italic Greek.
\usepackage{fmtcount, etoolbox}         % For the \book{} command.
\usepackage[newparttoc]{titlesec}       % Formatting chapter, etc.
\usepackage{titletoc}                   % Allows \book in toc.
\usepackage[nottoc]{tocbibind}          % Bibliography in toc.
\usepackage[titles]{tocloft}            % ToC formatting.
\usepackage{pgfplots, tikz}             % Drawing/graphing tools.
\usepackage{imakeidx}                   % Used for index.
\usetikzlibrary{
    calc,                   % Calculating right angles and more.
    angles,                 % Drawing angles within triangles.
    arrows.meta,            % Latex and Stealth arrows.
    quotes,                 % Adding labels to angles.
    positioning,            % Relative positioning of nodes.
    decorations.markings,   % Adding arrows in the middle of a line.
    patterns,
    arrows
}                                       % Libraries for tikz.
\pgfplotsset{compat=1.9}                % Version of pgfplots.
\usepackage[font=scriptsize,
            labelformat=simple,
            labelsep=colon]{subcaption} % Subfigure captions.
\usepackage[font={scriptsize},
            hypcap=true,
            labelsep=colon]{caption}    % Figure captions.
\usepackage[pdftex,
            pdfauthor={Ryan Maguire},
            pdftitle={Mathematics and Physics},
            pdfsubject={Mathematics, Physics, Science},
            pdfkeywords={Mathematics, Physics, Computer Science, Biology},
            pdfproducer={LaTeX},
            pdfcreator={pdflatex}]{hyperref}
\hypersetup{
    colorlinks=true,
    linkcolor=blue,
    filecolor=magenta,
    urlcolor=Cerulean,
    citecolor=SkyBlue
}                           % Colors for hyperref.
\usepackage[toc,acronym,nogroupskip,nopostdot]{glossaries}
\usepackage{glossary-mcols}
%------------------------Theorem Styles-------------------------%
\theoremstyle{plain}
\newtheorem{theorem}{Theorem}[section]

% Define theorem style for default spacing and normal font.
\newtheoremstyle{normal}
    {\topsep}               % Amount of space above the theorem.
    {\topsep}               % Amount of space below the theorem.
    {}                      % Font used for body of theorem.
    {}                      % Measure of space to indent.
    {\bfseries}             % Font of the header of the theorem.
    {}                      % Punctuation between head and body.
    {.5em}                  % Space after theorem head.
    {}

% Italic header environment.
\newtheoremstyle{thmit}{\topsep}{\topsep}{}{}{\itshape}{}{0.5em}{}

% Define environments with italic headers.
\theoremstyle{thmit}
\newtheorem*{solution}{Solution}

% Define default environments.
\theoremstyle{normal}
\newtheorem{example}{Example}[section]
\newtheorem{definition}{Definition}[section]
\newtheorem{problem}{Problem}[section]

% Define framed environment.
\tcbuselibrary{most}
\newtcbtheorem[use counter*=theorem]{ftheorem}{Theorem}{%
    before=\par\vspace{2ex},
    boxsep=0.5\topsep,
    after=\par\vspace{2ex},
    colback=green!5,
    colframe=green!35!black,
    fonttitle=\bfseries\upshape%
}{thm}

\newtcbtheorem[auto counter, number within=section]{faxiom}{Axiom}{%
    before=\par\vspace{2ex},
    boxsep=0.5\topsep,
    after=\par\vspace{2ex},
    colback=Apricot!5,
    colframe=Apricot!35!black,
    fonttitle=\bfseries\upshape%
}{ax}

\newtcbtheorem[use counter*=definition]{fdefinition}{Definition}{%
    before=\par\vspace{2ex},
    boxsep=0.5\topsep,
    after=\par\vspace{2ex},
    colback=blue!5!white,
    colframe=blue!75!black,
    fonttitle=\bfseries\upshape%
}{def}

\newtcbtheorem[use counter*=example]{fexample}{Example}{%
    before=\par\vspace{2ex},
    boxsep=0.5\topsep,
    after=\par\vspace{2ex},
    colback=red!5!white,
    colframe=red!75!black,
    fonttitle=\bfseries\upshape%
}{ex}

\newtcbtheorem[auto counter, number within=section]{fnotation}{Notation}{%
    before=\par\vspace{2ex},
    boxsep=0.5\topsep,
    after=\par\vspace{2ex},
    colback=SeaGreen!5!white,
    colframe=SeaGreen!75!black,
    fonttitle=\bfseries\upshape%
}{not}

\newtcbtheorem[use counter*=remark]{fremark}{Remark}{%
    fonttitle=\bfseries\upshape,
    colback=Goldenrod!5!white,
    colframe=Goldenrod!75!black}{ex}

\newenvironment{bproof}{\textit{Proof.}}{\hfill$\square$}
\tcolorboxenvironment{bproof}{%
    blanker,
    breakable,
    left=3mm,
    before skip=5pt,
    after skip=10pt,
    borderline west={0.6mm}{0pt}{green!80!black}
}

\AtEndEnvironment{lexample}{$\hfill\textcolor{red}{\blacksquare}$}
\newtcbtheorem[use counter*=example]{lexample}{Example}{%
    empty,
    title={Example~\theexample},
    boxed title style={%
        empty,
        size=minimal,
        toprule=2pt,
        top=0.5\topsep,
    },
    coltitle=red,
    fonttitle=\bfseries,
    parbox=false,
    boxsep=0pt,
    before=\par\vspace{2ex},
    left=0pt,
    right=0pt,
    top=3ex,
    bottom=1ex,
    before=\par\vspace{2ex},
    after=\par\vspace{2ex},
    breakable,
    pad at break*=0mm,
    vfill before first,
    overlay unbroken={%
        \draw[red, line width=2pt]
            ([yshift=-1.2ex]title.south-|frame.west) to
            ([yshift=-1.2ex]title.south-|frame.east);
        },
    overlay first={%
        \draw[red, line width=2pt]
            ([yshift=-1.2ex]title.south-|frame.west) to
            ([yshift=-1.2ex]title.south-|frame.east);
    },
}{ex}

\AtEndEnvironment{ldefinition}{$\hfill\textcolor{Blue}{\blacksquare}$}
\newtcbtheorem[use counter*=definition]{ldefinition}{Definition}{%
    empty,
    title={Definition~\thedefinition:~{#1}},
    boxed title style={%
        empty,
        size=minimal,
        toprule=2pt,
        top=0.5\topsep,
    },
    coltitle=Blue,
    fonttitle=\bfseries,
    parbox=false,
    boxsep=0pt,
    before=\par\vspace{2ex},
    left=0pt,
    right=0pt,
    top=3ex,
    bottom=0pt,
    before=\par\vspace{2ex},
    after=\par\vspace{1ex},
    breakable,
    pad at break*=0mm,
    vfill before first,
    overlay unbroken={%
        \draw[Blue, line width=2pt]
            ([yshift=-1.2ex]title.south-|frame.west) to
            ([yshift=-1.2ex]title.south-|frame.east);
        },
    overlay first={%
        \draw[Blue, line width=2pt]
            ([yshift=-1.2ex]title.south-|frame.west) to
            ([yshift=-1.2ex]title.south-|frame.east);
    },
}{def}

\AtEndEnvironment{ltheorem}{$\hfill\textcolor{Green}{\blacksquare}$}
\newtcbtheorem[use counter*=theorem]{ltheorem}{Theorem}{%
    empty,
    title={Theorem~\thetheorem:~{#1}},
    boxed title style={%
        empty,
        size=minimal,
        toprule=2pt,
        top=0.5\topsep,
    },
    coltitle=Green,
    fonttitle=\bfseries,
    parbox=false,
    boxsep=0pt,
    before=\par\vspace{2ex},
    left=0pt,
    right=0pt,
    top=3ex,
    bottom=-1.5ex,
    breakable,
    pad at break*=0mm,
    vfill before first,
    overlay unbroken={%
        \draw[Green, line width=2pt]
            ([yshift=-1.2ex]title.south-|frame.west) to
            ([yshift=-1.2ex]title.south-|frame.east);},
    overlay first={%
        \draw[Green, line width=2pt]
            ([yshift=-1.2ex]title.south-|frame.west) to
            ([yshift=-1.2ex]title.south-|frame.east);
    }
}{thm}

%--------------------Declared Math Operators--------------------%
\DeclareMathOperator{\adjoint}{adj}         % Adjoint.
\DeclareMathOperator{\Card}{Card}           % Cardinality.
\DeclareMathOperator{\curl}{curl}           % Curl.
\DeclareMathOperator{\diam}{diam}           % Diameter.
\DeclareMathOperator{\dist}{dist}           % Distance.
\DeclareMathOperator{\Div}{div}             % Divergence.
\DeclareMathOperator{\Erf}{Erf}             % Error Function.
\DeclareMathOperator{\Erfc}{Erfc}           % Complementary Error Function.
\DeclareMathOperator{\Ext}{Ext}             % Exterior.
\DeclareMathOperator{\GCD}{GCD}             % Greatest common denominator.
\DeclareMathOperator{\grad}{grad}           % Gradient
\DeclareMathOperator{\Ima}{Im}              % Image.
\DeclareMathOperator{\Int}{Int}             % Interior.
\DeclareMathOperator{\LC}{LC}               % Leading coefficient.
\DeclareMathOperator{\LCM}{LCM}             % Least common multiple.
\DeclareMathOperator{\LM}{LM}               % Leading monomial.
\DeclareMathOperator{\LT}{LT}               % Leading term.
\DeclareMathOperator{\Mod}{mod}             % Modulus.
\DeclareMathOperator{\Mon}{Mon}             % Monomial.
\DeclareMathOperator{\multideg}{mutlideg}   % Multi-Degree (Graphs).
\DeclareMathOperator{\nul}{nul}             % Null space of operator.
\DeclareMathOperator{\Ord}{Ord}             % Ordinal of ordered set.
\DeclareMathOperator{\Prin}{Prin}           % Principal value.
\DeclareMathOperator{\proj}{proj}           % Projection.
\DeclareMathOperator{\Refl}{Refl}           % Reflection operator.
\DeclareMathOperator{\rk}{rk}               % Rank of operator.
\DeclareMathOperator{\sgn}{sgn}             % Sign of a number.
\DeclareMathOperator{\sinc}{sinc}           % Sinc function.
\DeclareMathOperator{\Span}{Span}           % Span of a set.
\DeclareMathOperator{\Spec}{Spec}           % Spectrum.
\DeclareMathOperator{\supp}{supp}           % Support
\DeclareMathOperator{\Tr}{Tr}               % Trace of matrix.
%--------------------Declared Math Symbols--------------------%
\DeclareMathSymbol{\minus}{\mathbin}{AMSa}{"39} % Unary minus sign.
%------------------------New Commands---------------------------%
\DeclarePairedDelimiter\norm{\lVert}{\rVert}
\DeclarePairedDelimiter\ceil{\lceil}{\rceil}
\DeclarePairedDelimiter\floor{\lfloor}{\rfloor}
\newcommand*\diff{\mathop{}\!\mathrm{d}}
\newcommand*\Diff[1]{\mathop{}\!\mathrm{d^#1}}
\renewcommand*{\glstextformat}[1]{\textcolor{RoyalBlue}{#1}}
\renewcommand{\glsnamefont}[1]{\textbf{#1}}
\renewcommand\labelitemii{$\circ$}
\renewcommand\thesubfigure{%
    \arabic{chapter}.\arabic{figure}.\arabic{subfigure}}
\addto\captionsenglish{\renewcommand{\figurename}{Fig.}}
\numberwithin{equation}{section}

\renewcommand{\vector}[1]{\boldsymbol{\mathrm{#1}}}

\newcommand{\uvector}[1]{\boldsymbol{\hat{\mathrm{#1}}}}
\newcommand{\topspace}[2][]{(#2,\tau_{#1})}
\newcommand{\measurespace}[2][]{(#2,\varSigma_{#1},\mu_{#1})}
\newcommand{\measurablespace}[2][]{(#2,\varSigma_{#1})}
\newcommand{\manifold}[2][]{(#2,\tau_{#1},\mathcal{A}_{#1})}
\newcommand{\tanspace}[2]{T_{#1}{#2}}
\newcommand{\cotanspace}[2]{T_{#1}^{*}{#2}}
\newcommand{\Ckspace}[3][\mathbb{R}]{C^{#2}(#3,#1)}
\newcommand{\funcspace}[2][\mathbb{R}]{\mathcal{F}(#2,#1)}
\newcommand{\smoothvecf}[1]{\mathfrak{X}(#1)}
\newcommand{\smoothonef}[1]{\mathfrak{X}^{*}(#1)}
\newcommand{\bracket}[2]{[#1,#2]}

%------------------------Book Command---------------------------%
\makeatletter
\renewcommand\@pnumwidth{1cm}
\newcounter{book}
\renewcommand\thebook{\@Roman\c@book}
\newcommand\book{%
    \if@openright
        \cleardoublepage
    \else
        \clearpage
    \fi
    \thispagestyle{plain}%
    \if@twocolumn
        \onecolumn
        \@tempswatrue
    \else
        \@tempswafalse
    \fi
    \null\vfil
    \secdef\@book\@sbook
}
\def\@book[#1]#2{%
    \refstepcounter{book}
    \addcontentsline{toc}{book}{\bookname\ \thebook:\hspace{1em}#1}
    \markboth{}{}
    {\centering
     \interlinepenalty\@M
     \normalfont
     \huge\bfseries\bookname\nobreakspace\thebook
     \par
     \vskip 20\p@
     \Huge\bfseries#2\par}%
    \@endbook}
\def\@sbook#1{%
    {\centering
     \interlinepenalty \@M
     \normalfont
     \Huge\bfseries#1\par}%
    \@endbook}
\def\@endbook{
    \vfil\newpage
        \if@twoside
            \if@openright
                \null
                \thispagestyle{empty}%
                \newpage
            \fi
        \fi
        \if@tempswa
            \twocolumn
        \fi
}
\newcommand*\l@book[2]{%
    \ifnum\c@tocdepth >-3\relax
        \addpenalty{-\@highpenalty}%
        \addvspace{2.25em\@plus\p@}%
        \setlength\@tempdima{3em}%
        \begingroup
            \parindent\z@\rightskip\@pnumwidth
            \parfillskip -\@pnumwidth
            {
                \leavevmode
                \Large\bfseries#1\hfill\hb@xt@\@pnumwidth{\hss#2}
            }
            \par
            \nobreak
            \global\@nobreaktrue
            \everypar{\global\@nobreakfalse\everypar{}}%
        \endgroup
    \fi}
\newcommand\bookname{Book}
\renewcommand{\thebook}{\texorpdfstring{\Numberstring{book}}{book}}
\providecommand*{\toclevel@book}{-2}
\makeatother
\titleformat{\part}[display]
    {\Large\bfseries}
    {\partname\nobreakspace\thepart}
    {0mm}
    {\Huge\bfseries}
\titlecontents{part}[0pt]
    {\large\bfseries}
    {\partname\ \thecontentslabel: \quad}
    {}
    {\hfill\contentspage}
\titlecontents{chapter}[0pt]
    {\bfseries}
    {\chaptername\ \thecontentslabel:\quad}
    {}
    {\hfill\contentspage}
\newglossarystyle{longpara}{%
    \setglossarystyle{long}%
    \renewenvironment{theglossary}{%
        \begin{longtable}[l]{{p{0.25\hsize}p{0.65\hsize}}}
    }{\end{longtable}}%
    \renewcommand{\glossentry}[2]{%
        \glstarget{##1}{\glossentryname{##1}}%
        &\glossentrydesc{##1}{~##2.}
        \tabularnewline%
        \tabularnewline
    }%
}
\newglossary[not-glg]{notation}{not-gls}{not-glo}{Notation}
\newcommand*{\newnotation}[4][]{%
    \newglossaryentry{#2}{type=notation, name={\textbf{#3}, },
                          text={#4}, description={#4},#1}%
}
%--------------------------LENGTHS------------------------------%
% Spacings for the Table of Contents.
\addtolength{\cftsecnumwidth}{1ex}
\addtolength{\cftsubsecindent}{1ex}
\addtolength{\cftsubsecnumwidth}{1ex}
\addtolength{\cftfignumwidth}{1ex}
\addtolength{\cfttabnumwidth}{1ex}

% Indent and paragraph spacing.
\setlength{\parindent}{0em}
\setlength{\parskip}{0em}
%--------------------------Main Document----------------------------%
\begin{document}
    \ifx\ifphysicscourseselectromagnetismI\undefined
        \section*{Electromagnetism I}
        \setcounter{section}{7}
        \renewcommand\thefigure{%
            \arabic{section}.\arabic{figure}%
        }
        \renewcommand\thesubfigure{%
            \arabic{section}.\arabic{figure}.\arabic{subfigure}%
        }
    \fi
    \subsection{Homework VII}
        \begin{problem}[Wangsness 6-6]
            Suppose there is a capacitor $C_{1}$ that is charged
            to a potential difference $\Delta\phi$
            between its plates, and another capacitor
            $C_{2}$ is uncharged. If one of the plates
            of $C_{1}$ is connected to $C_{2}$ by
            a conductor of negligible capacitance, and
            if the remaining plates are similarly connected,
            for the resultant equilibrium state, find the
            charge on each capacitor and the potential
            difference $\Delta\phi'$ between their respective
            plates.
        \end{problem}
        \begin{proof}[Solution]
            Before the connection we have
            $Q=C_{1}\Delta\phi$.
            After the connection, $Q_{1}=C_{1}\Delta\phi'$,
            $Q_{2}=C_{2}\Delta\phi'$, and from the
            conservation of charge we have $Q_{1}+Q_{2}=Q$.
            Putting this together, we obtain:
            \begin{equation*}
                Q_{1}+Q_{2}
                =(C_{1}+C_{2})\Delta\phi'
                =Q
                =C_{1}\Delta\phi
            \end{equation*}
            Thus:
            \begin{align*}
                \Delta\phi'
                &=\frac{C_{1}}{C_{1}+C_{2}}\Delta\phi
                &
                Q_{1}
                &=\frac{C_{1}^{2}}{C_{1}+C_{2}}\Delta\phi
                &
                Q_{2}
                &=\frac{C_{1}C_{2}}{C_{1}+C_{2}}\Delta\phi
            \end{align*}
        \end{proof}
        \begin{problem}[Wangsness 6-7]
            \label{Problem:EMAG_Wangsness_6_7}
            Suppose the plates of two capacitors
            $C_{1}$ and $C_{2}$ are connected by conductors
            of negligible capacitance in parallel as shown
            in Fig.~\subref{fig:EMAG_Parallel_Circuit}.
            If a potential difference is applied across the
            terminals of $\Delta\phi$, show that this
            combination is equivalent to a single capacitor
            of capacitance $C_{p}=C_{1}+C_{2}$. Similarly, do
            this for series
            (Fig.~\subref{fig:EMAG_Series_Circuit})
            and show that $1/C_{s}=1/C_{1}+1/C_{2}$.
        \end{problem}
        In parallel we have $Q_{1}=C_{1}\Delta\phi$
        and $Q_{2}=C_{2}\Delta\phi$,
        where $Q_{1}$ and $Q_{2}$ are the charges on the
        plates $C_{1}$ and $C_{2}$, respectively.
        The total charge is $Q_{1}+Q_{2}$. Thus we have:
        \begin{equation*}
            Q=Q_{1}+Q_{2}
             =C_{1}\Delta\phi+C_{2}\Delta\phi
             =(C_{1}+C_{2})\Delta\phi
             =C_{p}\Delta\phi
            \Rightarrow{C_{p}}=C_{1}+C_{2}
        \end{equation*}
        For series we have
        $\Delta\phi=\Delta\phi_{1}+\Delta\phi_{2}$,
        where $\Delta\phi_{1}$ and $\Delta\phi_{2}$
        are the potential differences across $C_{1}$ and $C_{2}$,
        respectively. If a charge $Q$ is on the left plate
        of $C_{1}$, then there is a charge $-Q$ on the right
        plate, and therefore there is a charge $Q$ on the
        left plate of $C_{2}$ as well. Thus, $Q=Q_{1}=Q_{2}$.
        So:
        \begin{equation*}
            \Delta\phi=\frac{Q_{1}}{C_{1}}+\frac{Q_{2}}{C_{2}}
            =\frac{Q}{C_{1}}+\frac{Q}{C_{2}}
            =Q\big(\frac{1}{C_{1}}+\frac{1}{C_{2}}\big)
            =\frac{Q}{C_{s}}
            \Rightarrow\frac{1}{C_{s}}
            =\frac{1}{C_{1}}+\frac{1}{C_{2}}
        \end{equation*}
        \begin{figure}
            \centering
            \captionsetup{type=figure}
            \begin{subfigure}[b]{0.49\textwidth}
                \centering
                \captionsetup{type=figure}
                \subimport{../../../../tikz/}
                          {Simple_Parallel_Circuit}
                \caption{Parallel Circuit.}
                \label{fig:EMAG_Parallel_Circuit}
            \end{subfigure}
            \begin{subfigure}[b]{0.49\textwidth}
                \centering
                \captionsetup{type=figure}
                \subimport{../../../../tikz/}
                          {Simple_Series_Circuit}
                \caption{Series Circuit.}
                \label{fig:EMAG_Series_Circuit}
            \end{subfigure}
            \caption{%
                Circuits for problem
                \ref{Problem:EMAG_Wangsness_6_7}
            }
            \label{fig:EMAG_Wangsness_6_7}
        \end{figure}
        \begin{problem}[Wangsness 6-9]
            Suppose that the potential difference between
            the plates of a spherical capacitor are kept
            constant at $\Delta\phi$. Show that the electric
            field at the surface of the inner sphere will be
            a minimum when $a=b/2$.
            Find this minimum value of $E$.
        \end{problem}
        \begin{proof}[Solution]
            The charge on a capacitor is
            $Q=C\Delta\phi$.
            The electric field for a
            spherical capacitor at $r=a$ is:
            \begin{equation*}
                E=\frac{Q}{4\pi\epsilon_{0}a^{2}}
            \end{equation*}
            For a spherical capacitor we have:
            \begin{equation*}
                C=\frac{4\pi\epsilon_{0}ab}{b-a}
                \Rightarrow
                E=\frac{b\Delta\phi}{a(b-a)}
            \end{equation*}
            To minimize this, we solve the following:
            \begin{equation*}
                \frac{\partial{E}}{\partial{a}}=0
                \Rightarrow
                \frac{b\Delta\phi}{a^{2}(b-a)^{2}}
                (b-2a)=0
                \Rightarrow
                a=\frac{b}{2}
            \end{equation*}
            To see that is is a minimum, we look at the
            second partial derivative:
            \begin{equation*}
                \frac{\partial^2 E}
                     {\partial a^2}\bigg|_{a=\frac{b}{2}}
                =\frac{32\Delta\phi}{b}>0
            \end{equation*}
            Therefore $a=b/2$ is a minimum. Evaluating
            $E$ at this point, we have
            $E=4\Delta\phi/b$
        \end{proof}
        \begin{problem}[Wangsness 6-10]
            \label{problem:EMAG_Wangsness_6_10}
            Given a capacitor made from two infinitely
            long conductors with coaxial cylindrical surfaces,
            like the ones shown in
            Fig.~\ref{fig:EMAG_Wangsness_6_10},
            show that the capacitance of a length $L$
            is given by $C=2\pi\epsilon_{0}L/\ln(b/a)$.
        \end{problem}
        \begin{proof}[Solution]
            We have that
            $E=\frac{\lambda}{2\pi \epsilon_0 \rho}$,
            where $\lambda$
            is the linear charge density,
            $\lambda=\frac{Q}{L}$.
            Thus,
            \begin{equation*}
                \Delta\phi=-\int_{b}^{a}
                \frac{\lambda}{2\pi\epsilon_0\rho}\diff{\rho}
                =\frac{\lambda}{2\pi\epsilon_{0}}
                \ln\Big(\frac{b}{a}\Big)    
            \end{equation*}
            Therefore:
            \begin{equation*}
                C=\frac{Q}{\Delta\phi}
                =\frac{2\pi\epsilon_{0}L}
                      {\ln\big(\frac{b}{a}\big)}
            \end{equation*}
        \end{proof}
        \begin{figure}[H]
            \centering
            \captionsetup{type=figure}
            \subimport{../../../../tikz/}{Wangsness_6_10}
            \caption{%
                Infinite Cylinders for problem
                \ref{problem:EMAG_Wangsness_6_10}
            }
            \label{fig:EMAG_Wangsness_6_10}
        \end{figure}
\end{document}