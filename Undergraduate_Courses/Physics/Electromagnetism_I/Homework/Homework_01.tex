\documentclass[crop=false,class=article,oneside]{standalone}
%----------------------------Preamble-------------------------------%
%---------------------------Packages----------------------------%
\usepackage{geometry}
\geometry{b5paper, margin=1.0in}
\usepackage[T1]{fontenc}
\usepackage{graphicx, float}            % Graphics/Images.
\usepackage{natbib}                     % For bibliographies.
\bibliographystyle{agsm}                % Bibliography style.
\usepackage[french, english]{babel}     % Language typesetting.
\usepackage[dvipsnames]{xcolor}         % Color names.
\usepackage{listings}                   % Verbatim-Like Tools.
\usepackage{mathtools, esint, mathrsfs} % amsmath and integrals.
\usepackage{amsthm, amsfonts, amssymb}  % Fonts and theorems.
\usepackage{tcolorbox}                  % Frames around theorems.
\usepackage{upgreek}                    % Non-Italic Greek.
\usepackage{fmtcount, etoolbox}         % For the \book{} command.
\usepackage[newparttoc]{titlesec}       % Formatting chapter, etc.
\usepackage{titletoc}                   % Allows \book in toc.
\usepackage[nottoc]{tocbibind}          % Bibliography in toc.
\usepackage[titles]{tocloft}            % ToC formatting.
\usepackage{pgfplots, tikz}             % Drawing/graphing tools.
\usepackage{imakeidx}                   % Used for index.
\usetikzlibrary{
    calc,                   % Calculating right angles and more.
    angles,                 % Drawing angles within triangles.
    arrows.meta,            % Latex and Stealth arrows.
    quotes,                 % Adding labels to angles.
    positioning,            % Relative positioning of nodes.
    decorations.markings,   % Adding arrows in the middle of a line.
    patterns,
    arrows
}                                       % Libraries for tikz.
\pgfplotsset{compat=1.9}                % Version of pgfplots.
\usepackage[font=scriptsize,
            labelformat=simple,
            labelsep=colon]{subcaption} % Subfigure captions.
\usepackage[font={scriptsize},
            hypcap=true,
            labelsep=colon]{caption}    % Figure captions.
\usepackage[pdftex,
            pdfauthor={Ryan Maguire},
            pdftitle={Mathematics and Physics},
            pdfsubject={Mathematics, Physics, Science},
            pdfkeywords={Mathematics, Physics, Computer Science, Biology},
            pdfproducer={LaTeX},
            pdfcreator={pdflatex}]{hyperref}
\hypersetup{
    colorlinks=true,
    linkcolor=blue,
    filecolor=magenta,
    urlcolor=Cerulean,
    citecolor=SkyBlue
}                           % Colors for hyperref.
\usepackage[toc,acronym,nogroupskip,nopostdot]{glossaries}
\usepackage{glossary-mcols}
%------------------------Theorem Styles-------------------------%
\theoremstyle{plain}
\newtheorem{theorem}{Theorem}[section]

% Define theorem style for default spacing and normal font.
\newtheoremstyle{normal}
    {\topsep}               % Amount of space above the theorem.
    {\topsep}               % Amount of space below the theorem.
    {}                      % Font used for body of theorem.
    {}                      % Measure of space to indent.
    {\bfseries}             % Font of the header of the theorem.
    {}                      % Punctuation between head and body.
    {.5em}                  % Space after theorem head.
    {}

% Italic header environment.
\newtheoremstyle{thmit}{\topsep}{\topsep}{}{}{\itshape}{}{0.5em}{}

% Define environments with italic headers.
\theoremstyle{thmit}
\newtheorem*{solution}{Solution}

% Define default environments.
\theoremstyle{normal}
\newtheorem{example}{Example}[section]
\newtheorem{definition}{Definition}[section]
\newtheorem{problem}{Problem}[section]

% Define framed environment.
\tcbuselibrary{most}
\newtcbtheorem[use counter*=theorem]{ftheorem}{Theorem}{%
    before=\par\vspace{2ex},
    boxsep=0.5\topsep,
    after=\par\vspace{2ex},
    colback=green!5,
    colframe=green!35!black,
    fonttitle=\bfseries\upshape%
}{thm}

\newtcbtheorem[auto counter, number within=section]{faxiom}{Axiom}{%
    before=\par\vspace{2ex},
    boxsep=0.5\topsep,
    after=\par\vspace{2ex},
    colback=Apricot!5,
    colframe=Apricot!35!black,
    fonttitle=\bfseries\upshape%
}{ax}

\newtcbtheorem[use counter*=definition]{fdefinition}{Definition}{%
    before=\par\vspace{2ex},
    boxsep=0.5\topsep,
    after=\par\vspace{2ex},
    colback=blue!5!white,
    colframe=blue!75!black,
    fonttitle=\bfseries\upshape%
}{def}

\newtcbtheorem[use counter*=example]{fexample}{Example}{%
    before=\par\vspace{2ex},
    boxsep=0.5\topsep,
    after=\par\vspace{2ex},
    colback=red!5!white,
    colframe=red!75!black,
    fonttitle=\bfseries\upshape%
}{ex}

\newtcbtheorem[auto counter, number within=section]{fnotation}{Notation}{%
    before=\par\vspace{2ex},
    boxsep=0.5\topsep,
    after=\par\vspace{2ex},
    colback=SeaGreen!5!white,
    colframe=SeaGreen!75!black,
    fonttitle=\bfseries\upshape%
}{not}

\newtcbtheorem[use counter*=remark]{fremark}{Remark}{%
    fonttitle=\bfseries\upshape,
    colback=Goldenrod!5!white,
    colframe=Goldenrod!75!black}{ex}

\newenvironment{bproof}{\textit{Proof.}}{\hfill$\square$}
\tcolorboxenvironment{bproof}{%
    blanker,
    breakable,
    left=3mm,
    before skip=5pt,
    after skip=10pt,
    borderline west={0.6mm}{0pt}{green!80!black}
}

\AtEndEnvironment{lexample}{$\hfill\textcolor{red}{\blacksquare}$}
\newtcbtheorem[use counter*=example]{lexample}{Example}{%
    empty,
    title={Example~\theexample},
    boxed title style={%
        empty,
        size=minimal,
        toprule=2pt,
        top=0.5\topsep,
    },
    coltitle=red,
    fonttitle=\bfseries,
    parbox=false,
    boxsep=0pt,
    before=\par\vspace{2ex},
    left=0pt,
    right=0pt,
    top=3ex,
    bottom=1ex,
    before=\par\vspace{2ex},
    after=\par\vspace{2ex},
    breakable,
    pad at break*=0mm,
    vfill before first,
    overlay unbroken={%
        \draw[red, line width=2pt]
            ([yshift=-1.2ex]title.south-|frame.west) to
            ([yshift=-1.2ex]title.south-|frame.east);
        },
    overlay first={%
        \draw[red, line width=2pt]
            ([yshift=-1.2ex]title.south-|frame.west) to
            ([yshift=-1.2ex]title.south-|frame.east);
    },
}{ex}

\AtEndEnvironment{ldefinition}{$\hfill\textcolor{Blue}{\blacksquare}$}
\newtcbtheorem[use counter*=definition]{ldefinition}{Definition}{%
    empty,
    title={Definition~\thedefinition:~{#1}},
    boxed title style={%
        empty,
        size=minimal,
        toprule=2pt,
        top=0.5\topsep,
    },
    coltitle=Blue,
    fonttitle=\bfseries,
    parbox=false,
    boxsep=0pt,
    before=\par\vspace{2ex},
    left=0pt,
    right=0pt,
    top=3ex,
    bottom=0pt,
    before=\par\vspace{2ex},
    after=\par\vspace{1ex},
    breakable,
    pad at break*=0mm,
    vfill before first,
    overlay unbroken={%
        \draw[Blue, line width=2pt]
            ([yshift=-1.2ex]title.south-|frame.west) to
            ([yshift=-1.2ex]title.south-|frame.east);
        },
    overlay first={%
        \draw[Blue, line width=2pt]
            ([yshift=-1.2ex]title.south-|frame.west) to
            ([yshift=-1.2ex]title.south-|frame.east);
    },
}{def}

\AtEndEnvironment{ltheorem}{$\hfill\textcolor{Green}{\blacksquare}$}
\newtcbtheorem[use counter*=theorem]{ltheorem}{Theorem}{%
    empty,
    title={Theorem~\thetheorem:~{#1}},
    boxed title style={%
        empty,
        size=minimal,
        toprule=2pt,
        top=0.5\topsep,
    },
    coltitle=Green,
    fonttitle=\bfseries,
    parbox=false,
    boxsep=0pt,
    before=\par\vspace{2ex},
    left=0pt,
    right=0pt,
    top=3ex,
    bottom=-1.5ex,
    breakable,
    pad at break*=0mm,
    vfill before first,
    overlay unbroken={%
        \draw[Green, line width=2pt]
            ([yshift=-1.2ex]title.south-|frame.west) to
            ([yshift=-1.2ex]title.south-|frame.east);},
    overlay first={%
        \draw[Green, line width=2pt]
            ([yshift=-1.2ex]title.south-|frame.west) to
            ([yshift=-1.2ex]title.south-|frame.east);
    }
}{thm}

%--------------------Declared Math Operators--------------------%
\DeclareMathOperator{\adjoint}{adj}         % Adjoint.
\DeclareMathOperator{\Card}{Card}           % Cardinality.
\DeclareMathOperator{\curl}{curl}           % Curl.
\DeclareMathOperator{\diam}{diam}           % Diameter.
\DeclareMathOperator{\dist}{dist}           % Distance.
\DeclareMathOperator{\Div}{div}             % Divergence.
\DeclareMathOperator{\Erf}{Erf}             % Error Function.
\DeclareMathOperator{\Erfc}{Erfc}           % Complementary Error Function.
\DeclareMathOperator{\Ext}{Ext}             % Exterior.
\DeclareMathOperator{\GCD}{GCD}             % Greatest common denominator.
\DeclareMathOperator{\grad}{grad}           % Gradient
\DeclareMathOperator{\Ima}{Im}              % Image.
\DeclareMathOperator{\Int}{Int}             % Interior.
\DeclareMathOperator{\LC}{LC}               % Leading coefficient.
\DeclareMathOperator{\LCM}{LCM}             % Least common multiple.
\DeclareMathOperator{\LM}{LM}               % Leading monomial.
\DeclareMathOperator{\LT}{LT}               % Leading term.
\DeclareMathOperator{\Mod}{mod}             % Modulus.
\DeclareMathOperator{\Mon}{Mon}             % Monomial.
\DeclareMathOperator{\multideg}{mutlideg}   % Multi-Degree (Graphs).
\DeclareMathOperator{\nul}{nul}             % Null space of operator.
\DeclareMathOperator{\Ord}{Ord}             % Ordinal of ordered set.
\DeclareMathOperator{\Prin}{Prin}           % Principal value.
\DeclareMathOperator{\proj}{proj}           % Projection.
\DeclareMathOperator{\Refl}{Refl}           % Reflection operator.
\DeclareMathOperator{\rk}{rk}               % Rank of operator.
\DeclareMathOperator{\sgn}{sgn}             % Sign of a number.
\DeclareMathOperator{\sinc}{sinc}           % Sinc function.
\DeclareMathOperator{\Span}{Span}           % Span of a set.
\DeclareMathOperator{\Spec}{Spec}           % Spectrum.
\DeclareMathOperator{\supp}{supp}           % Support
\DeclareMathOperator{\Tr}{Tr}               % Trace of matrix.
%--------------------Declared Math Symbols--------------------%
\DeclareMathSymbol{\minus}{\mathbin}{AMSa}{"39} % Unary minus sign.
%------------------------New Commands---------------------------%
\DeclarePairedDelimiter\norm{\lVert}{\rVert}
\DeclarePairedDelimiter\ceil{\lceil}{\rceil}
\DeclarePairedDelimiter\floor{\lfloor}{\rfloor}
\newcommand*\diff{\mathop{}\!\mathrm{d}}
\newcommand*\Diff[1]{\mathop{}\!\mathrm{d^#1}}
\renewcommand*{\glstextformat}[1]{\textcolor{RoyalBlue}{#1}}
\renewcommand{\glsnamefont}[1]{\textbf{#1}}
\renewcommand\labelitemii{$\circ$}
\renewcommand\thesubfigure{%
    \arabic{chapter}.\arabic{figure}.\arabic{subfigure}}
\addto\captionsenglish{\renewcommand{\figurename}{Fig.}}
\numberwithin{equation}{section}

\renewcommand{\vector}[1]{\boldsymbol{\mathrm{#1}}}

\newcommand{\uvector}[1]{\boldsymbol{\hat{\mathrm{#1}}}}
\newcommand{\topspace}[2][]{(#2,\tau_{#1})}
\newcommand{\measurespace}[2][]{(#2,\varSigma_{#1},\mu_{#1})}
\newcommand{\measurablespace}[2][]{(#2,\varSigma_{#1})}
\newcommand{\manifold}[2][]{(#2,\tau_{#1},\mathcal{A}_{#1})}
\newcommand{\tanspace}[2]{T_{#1}{#2}}
\newcommand{\cotanspace}[2]{T_{#1}^{*}{#2}}
\newcommand{\Ckspace}[3][\mathbb{R}]{C^{#2}(#3,#1)}
\newcommand{\funcspace}[2][\mathbb{R}]{\mathcal{F}(#2,#1)}
\newcommand{\smoothvecf}[1]{\mathfrak{X}(#1)}
\newcommand{\smoothonef}[1]{\mathfrak{X}^{*}(#1)}
\newcommand{\bracket}[2]{[#1,#2]}

%------------------------Book Command---------------------------%
\makeatletter
\renewcommand\@pnumwidth{1cm}
\newcounter{book}
\renewcommand\thebook{\@Roman\c@book}
\newcommand\book{%
    \if@openright
        \cleardoublepage
    \else
        \clearpage
    \fi
    \thispagestyle{plain}%
    \if@twocolumn
        \onecolumn
        \@tempswatrue
    \else
        \@tempswafalse
    \fi
    \null\vfil
    \secdef\@book\@sbook
}
\def\@book[#1]#2{%
    \refstepcounter{book}
    \addcontentsline{toc}{book}{\bookname\ \thebook:\hspace{1em}#1}
    \markboth{}{}
    {\centering
     \interlinepenalty\@M
     \normalfont
     \huge\bfseries\bookname\nobreakspace\thebook
     \par
     \vskip 20\p@
     \Huge\bfseries#2\par}%
    \@endbook}
\def\@sbook#1{%
    {\centering
     \interlinepenalty \@M
     \normalfont
     \Huge\bfseries#1\par}%
    \@endbook}
\def\@endbook{
    \vfil\newpage
        \if@twoside
            \if@openright
                \null
                \thispagestyle{empty}%
                \newpage
            \fi
        \fi
        \if@tempswa
            \twocolumn
        \fi
}
\newcommand*\l@book[2]{%
    \ifnum\c@tocdepth >-3\relax
        \addpenalty{-\@highpenalty}%
        \addvspace{2.25em\@plus\p@}%
        \setlength\@tempdima{3em}%
        \begingroup
            \parindent\z@\rightskip\@pnumwidth
            \parfillskip -\@pnumwidth
            {
                \leavevmode
                \Large\bfseries#1\hfill\hb@xt@\@pnumwidth{\hss#2}
            }
            \par
            \nobreak
            \global\@nobreaktrue
            \everypar{\global\@nobreakfalse\everypar{}}%
        \endgroup
    \fi}
\newcommand\bookname{Book}
\renewcommand{\thebook}{\texorpdfstring{\Numberstring{book}}{book}}
\providecommand*{\toclevel@book}{-2}
\makeatother
\titleformat{\part}[display]
    {\Large\bfseries}
    {\partname\nobreakspace\thepart}
    {0mm}
    {\Huge\bfseries}
\titlecontents{part}[0pt]
    {\large\bfseries}
    {\partname\ \thecontentslabel: \quad}
    {}
    {\hfill\contentspage}
\titlecontents{chapter}[0pt]
    {\bfseries}
    {\chaptername\ \thecontentslabel:\quad}
    {}
    {\hfill\contentspage}
\newglossarystyle{longpara}{%
    \setglossarystyle{long}%
    \renewenvironment{theglossary}{%
        \begin{longtable}[l]{{p{0.25\hsize}p{0.65\hsize}}}
    }{\end{longtable}}%
    \renewcommand{\glossentry}[2]{%
        \glstarget{##1}{\glossentryname{##1}}%
        &\glossentrydesc{##1}{~##2.}
        \tabularnewline%
        \tabularnewline
    }%
}
\newglossary[not-glg]{notation}{not-gls}{not-glo}{Notation}
\newcommand*{\newnotation}[4][]{%
    \newglossaryentry{#2}{type=notation, name={\textbf{#3}, },
                          text={#4}, description={#4},#1}%
}
%--------------------------LENGTHS------------------------------%
% Spacings for the Table of Contents.
\addtolength{\cftsecnumwidth}{1ex}
\addtolength{\cftsubsecindent}{1ex}
\addtolength{\cftsubsecnumwidth}{1ex}
\addtolength{\cftfignumwidth}{1ex}
\addtolength{\cfttabnumwidth}{1ex}

% Indent and paragraph spacing.
\setlength{\parindent}{0em}
\setlength{\parskip}{0em}
%--------------------------Main Document----------------------------%
\begin{document}
    \ifx\ifphysicscourseselectromagnetismI\undefined
        \section*{Electromagnetism I}
        \setcounter{section}{1}
    \fi
    \subsection{Homework I}
        Wangsness Chapter 1 - Problems: 2, 3, 4, 5, 8, 9
        \begin{problem}[Wangsness 1-2]
            Given
            $\mathbf{A}%
             =2\hat{\mathbf{x}}-3\hat{\mathbf{y}}%
             -4\hat{\mathbf{z}}$
            and
            $\mathbf{B}%
             =6\hat{\mathbf{x}}+5\hat{\mathbf{y}}%
             +\hat{\mathbf{z}}$,
            find the magnitudes and angles made with the $x$,
            $y$, and $z$ axes for
            $\mathbf{A}+\mathbf{B}$ and $\mathbf{A}-\mathbf{B}$.
        \end{problem}
        \begin{proof}[Solution]
            First, we need to find $\mathbf{A}+\mathbf{B}$
            and $\mathbf{A}-\mathbf{B}$:
            \begin{align*}
                \mathbf{A}+\mathbf{B}
                &
                =(2\hat{\mathbf{x}}
                -3\hat{\mathbf{y}}
                -4\hat{\mathbf{z}})
                +(6\hat{\mathbf{x}}
                +5\hat{\mathbf{y}}
                +\hat{\mathbf{z}})
                &
                \mathbf{A}-\mathbf{B}
                &
                =(2\hat{\mathbf{x}}
                -3\hat{\mathbf{y}}
                -4\hat{\mathbf{z}})
                -(6\hat{\mathbf{x}}
                +5\hat{\mathbf{y}}
                +\hat{\mathbf{z}})\\
                &
                =(2+6)\hat{\mathbf{x}}
                +(5-3)\hat{\mathbf{y}}
                +(1-4)\hat{\mathbf{z}}
                &
                &
                =(2-6)\hat{\mathbf{x}}
                -(3+5)\hat{\mathbf{y}}
                -(4+1)\hat{\mathbf{z}}\\
                &
                =8\hat{\mathbf{x}}
                +2\hat{\mathbf{y}}
                -3\hat{\mathbf{z}}
                &
                &=
                -4\hat{\mathbf{x}}
                -8\hat{\mathbf{y}}
                -5\hat{\mathbf{z}}
            \end{align*}
            The magnitude of a vector
            $\mathbf{A}%
             =a_{1}\hat{\mathbf{x}}_{1}%
             +\hdots+a_{N}\hat{\mathbf{x}}_{N}$,
            also called its \textit{norm}, is:
            \begin{equation*}
                \norm{\mathbf{A}}=\sqrt{\sum_{i=1}^{N}a_{i}^{2}}
            \end{equation*}
            Using this, we have:
            \begin{align*}
                \norm{\mathbf{A+B}}
                &=(8^{2}+2^{2}+3^{3})^{1/2}
                &
                \norm{\mathbf{A-B}}
                &=(4^{2}+8^{2}+5^{2})^{1/2}\\
                &=\sqrt{77}
                &
                &=\sqrt{105}
            \end{align*}
            The \textit{direction angle} between $\mathbf{A}$
            and the $\xi$ axis is:
            \begin{equation*}
                \alpha_{\xi}
                =\cos^{-1}\bigg(
                    \frac{\mathbf{A}
                    \cdot\hat{\boldsymbol{\upxi}}}
                    {\norm{\mathbf{A}}
                    \norm{\hat{\boldsymbol{\upxi}}}}
                \bigg)
                =\cos^{-1}\bigg(
                    \frac{\mathbf{A}
                    \cdot\hat{\boldsymbol{\upxi}}}
                    {\norm{\mathbf{A}}}
                \bigg)
            \end{equation*}
            The direction angles of $\mathbf{A}+\mathbf{B}$
            and $\mathbf{A}-\mathbf{B}$ for
            $\hat{\mathbf{x}}$, $\hat{\mathbf{y}}$,
            and $\hat{\mathbf{z}}$ are:
            \begin{align*}
                \alpha
                &=
                \cos^{-1}\bigg(
                    \frac{(\mathbf{A}+\mathbf{B})\cdot\hat{\mathbf{x}}}
                    {\norm{\mathbf{A+B}}}
                \bigg)
                &
                \beta
                &=\cos^{-1}\bigg(
                    \frac{(\mathbf{A}+\mathbf{B})\cdot \hat{\mathbf{y}}}
                    {\norm{\mathbf{A}+\mathbf{B}}}
                \bigg)
                &
                \gamma
                &=\cos^{-1}\bigg(
                    \frac{(\mathbf{A}+\mathbf{B})\cdot \hat{\mathbf{z}}}
                    {\norm{\mathbf{A}+\mathbf{B}}}
                \bigg)\\
                &=\cos^{-1}\bigg(\frac{8}{\sqrt{77}}\bigg)
                &
                &=\cos^{-1}\bigg(\frac{2}{\sqrt{77}}\bigg)&
                &=\cos^{-1}\bigg(\frac{-3}{\sqrt{77}}\bigg)\\
                &=24.3^{\circ}
                &
                &=76.8^{\circ}
                &
                &=110^{\circ}
            \end{align*}
            For $\mathbf{A}-\mathbf{B}$:
            \begin{align*}
                \alpha
                &=\cos^{-1}\bigg(
                    \frac{(\mathbf{A}-\mathbf{B})
                    \cdot\hat{\mathbf{x}}}
                    {\norm{\mathbf{A-B}}}
                \bigg)
                &
                \beta
                &=\cos^{-1}\bigg(
                \frac{(\mathbf{A}-\mathbf{B})
                \cdot\hat{\mathbf{y}}}
                    {\norm{\mathbf{A}-\mathbf{B}}}
                \bigg)
                &
                \gamma
                &=\cos^{-1}\bigg(
                    \frac{(\mathbf{A}-\mathbf{B})
                    \cdot\hat{\mathbf{z}}}
                    {\norm{\mathbf{A}-\mathbf{B}}}
                \bigg)\\
                &=
                \cos^{-1}\bigg(\frac{-4}{\sqrt{105}}\bigg)
                &
                &=\cos^{-1}\bigg(\frac{-8}{\sqrt{105}}\bigg)
                &
                &=\cos^{-1}\bigg(\frac{-5}{\sqrt{105}}\bigg)\\
                &=113^{\circ}
                &
                &=141.3^{\circ}
                &
                &=119.2^{\circ}
            \end{align*}
        \end{proof}
        \begin{problem}[Wangsness 1-3]
            Find the relative position vector $\mathbf{R}$
            of $\mathbf{P}=(2,-2,3)$
            with respect to $\mathbf{P}'=(-3,1,4)$.
            What are the direction angles of $\mathbf{R}$?
        \end{problem}
        \begin{proof}[Solution]
            The relative position vector of $\mathbf{B}$
            with respect to $\mathbf{A}$ is:
            \begin{equation*}
                \mathbf{R}_{\mathbf{A}\rightarrow\mathbf{B}}
                =\mathbf{B}-\mathbf{A}
            \end{equation*}
            Thus, we have:
            \begin{align*}
                \mathbf{R}&=\mathbf{P}-\mathbf{P}'\\
                &=
                (2\hat{\mathbf{x}}
                -2\hat{\mathbf{y}}
                +3\hat{\mathbf{z}})
                -
                (-3\hat{\mathbf{x}}
                +\hat{\mathbf{y}}
                +4\hat{\mathbf{z}})\\
                &
                =(2+3)\hat{\mathbf{x}}
                +(-2-1)\hat{\mathbf{y}}
                +(3-4)\hat{\mathbf{z}}\\
                &
                =5\hat{\mathbf{x}}
                -3\hat{\mathbf{y}}
                -\hat{\mathbf{z}}
            \end{align*}
            The direction angles for $\mathbf{R}$ are:
            \begin{align*}
                \alpha
                &=\cos^{-1}\bigg(
                    \frac{\mathbf{R}\cdot\hat{\mathbf{x}}}
                    {\norm{\mathbf{R}}}
                \bigg)
                &
                \beta
                &=\cos^{-1}\bigg(
                    \frac{\mathbf{R}\cdot\hat{\mathbf{y}}}
                    {\norm{\mathbf{R}}}
                \bigg)
                &
                \gamma
                &=\cos^{-1}\bigg(
                    \frac{\mathbf{R}\cdot\hat{\mathbf{z}}}
                    {\norm{\mathbf{R}}}
                \bigg)\\
                &=\cos^{-1}\bigg(\frac{5}{\sqrt{35}}\bigg)
                &
                &=\cos^{-1}\bigg(\frac{-3}{\sqrt{35}}\bigg)
                &
                &=\cos^{-1}\bigg(\frac{-1}{\sqrt{35}}\bigg)\\
                &=32.5^{\circ}
                &
                &=120^{\circ}
                &
                &=99.7^{\circ}
            \end{align*}
        \end{proof}
        \begin{problem}[Wangsness 1-4]
            Given
            $\mathbf{A}%
             =\hat{\mathbf{x}}+2\hat{\mathbf{y}}%
             +3\hat{\mathbf{z}}$
            and
            $\mathbf{B}%
             =4\hat{\mathbf{x}}-5\hat{\mathbf{y}}%
             +6\hat{\mathbf{z}}$,
            find the angle between them. Find the component of
            $\mathbf{A}$ in the direction of $\mathbf{B}$.
        \end{problem}
        \begin{proof}[Solution]
            The definition of the \textit{angle}
            between two vectors
            $\mathbf{A}$ and $\mathbf{B}$ is:
            \begin{equation*}
                \theta=
                \cos^{-1}\bigg(
                    \frac{\mathbf{A}\cdot\mathbf{B}}
                    {\norm{\mathbf{A}}\norm{\mathbf{B}}}
                \bigg)
            \end{equation*}
            We have that:
            \begin{align*}
                \mathbf{A}\cdot\mathbf{B}
                &=1\cdot4-2\cdot5+3\cdot6
                &
                \norm{\mathbf{A}}
                &=\sqrt{1^{2}+2^{2}+3^{2}}
                &
                \norm{\mathbf{B}}
                &=\sqrt{4^{2}+5^{2}+6^{2}}\\
                &=12
                &
                &=\sqrt{14}
                &
                &=\sqrt{77}\\
            \end{align*}
            Using this, we have:
            \begin{equation*}
                \theta=\cos^{-1}\bigg(
                    \frac{12}{\sqrt{14}{\sqrt{77}}}
                \bigg)
                =68.6^{\circ}
            \end{equation*}
            The \textit{component} of $\mathbf{A}$
            in the direction of
            $\mathbf{B}$ is defined as:
            \begin{equation*}
                \comp_{\mathbf{B}}(\mathbf{A})
                =\mathbf{A}\cdot\frac{\mathbf{B}}
                    {\norm{\mathbf{B}}}
            \end{equation*}
            Using this, we have:
            \begin{equation*}
                \comp_{\mathbf{B}}(\mathbf{A})
                =\mathbf{A}\cdot\frac{\mathbf{B}}
                    {\norm{\mathbf{B}}}
                =\frac{\mathbf{A}
                    \cdot\mathbf{B}}{\norm{\mathbf{B}}}
                =\frac{12}{\sqrt{77}}
                \approx{1.37}
            \end{equation*}
        \end{proof}
        \begin{problem}[Wangsness 1-5]
            Given 
            $\mathbf{A}=2\hat{\mathbf{x}}%
             +3\hat{\mathbf{y}}-4\hat{\mathbf{z}}$
            and
            $\mathbf{B}%
             =-6\hat{\mathbf{x}}-4\hat{\mathbf{y}}%
             +\hat{\mathbf{z}}$,
            find the component of
            $\mathbf{A}\times\mathbf{B}$
            along the direction of
            $\mathbf{C}%
             =\hat{\mathbf{x}}-\hat{\mathbf{y}}%
             +\hat{\mathbf{z}}$.
        \end{problem}
        \begin{proof}[Solution]
            The \textit{cross product} of $\mathbf{A}$
            with $\mathbf{B}$ is:
            \begin{equation*}
                \mathbf{A}\times\mathbf{B}
                =(A_{y}B_{z}-A_{z}B_{y})\hat{\mathbf{x}}
                +(A_{z}B_{x}-A_{x}B_{z})\hat{\mathbf{y}}
                +(A_{x}B_{y}-A_{y}B_{x})\hat{\mathbf{z}}
            \end{equation*}
            Note that
            $\mathbf{A}\times\mathbf{B}%
             =-\mathbf{B}\times\mathbf{A}$.
            A way to remember this formula is using matrices:
            \begin{equation*}
                \mathbf{A}\times\mathbf{B}
                =\det\Bigg(
                    \begin{bmatrix}
                        \hat{\mathbf{x}}
                        &\hat{\mathbf{y}}
                        &\hat{\mathbf{z}}\\
                        A_{x}&A_{y}&A_{z}\\
                        B_{x}&B_{y}&B_{z}
                    \end{bmatrix}
                \Bigg)
                =
                \begin{vmatrix}
                    \hat{\mathbf{x}}
                    &\hat{\mathbf{y}}
                    &\hat{\mathbf{z}}\\
                    A_{x}&A_{y}&A_{z}\\
                    B_{x}&B_{y}&B_{z}
                \end{vmatrix}
            \end{equation*}
            We have:
            \begin{align*}
                \mathbf{A}\times\mathbf{B}
                &=
                (2\hat{\mathbf{x}}
                +3\hat{\mathbf{y}}
                -4\hat{\mathbf{z}})
                \times
                (-6\hat{\mathbf{x}}
                -4\hat{\mathbf{y}}
                +\hat{\mathbf{z}})\\
                &=
                (3-16)\hat{\mathbf{x}}
                +(24-2)\hat{\mathbf{y}}
                +(-8+18)\hat{\mathbf{z}}\\
                &=
                -13\hat{\mathbf{x}}
                +22\hat{\mathbf{y}}
                +10\hat{\mathbf{z}}
            \end{align*}
            The component along the direction
            of $\mathbf{C}$ is:
            \begin{align*}
                \comp_{\mathbf{C}}(\mathbf{A}\times\mathbf{B})
                &=
                (\mathbf{A}\times\mathbf{B})
                \cdot
                \frac{\mathbf{C}}{\norm{\mathbf{C}}}
                &
                &=\frac{-13-22+10}{\sqrt{3}}\\
                &=
                \frac{(-13\hat{\mathbf{x}}
                +22\hat{\mathbf{y}}
                +10\hat{\mathbf{z}})
                \cdot
                (\hat{\mathbf{x}}
                -\hat{\mathbf{y}}
                +\hat{\mathbf{z}})}
                {\sqrt{1^2+(-1)^2+1^2}}
                &
                &=-\frac{25}{\sqrt{3}}
            \end{align*}
        \end{proof}
        \begin{problem}[Wangsness 1-8]
            Given a family of hyperbolas in the $xy$
            plane $u=xy$, find $\nabla(u)$.
            If
            $\mathbf{A}%
             =3\hat{\mathbf{x}}%
             +2\hat{\mathbf{y}}%
             +4\hat{\mathbf{z}}$,
            find the component of $\mathbf{A}$
            in the direction of $\nabla(u)$ at
            the point on the curve for which $u=3$ and $x=2$.
        \end{problem}
        \begin{proof}[Solution]
            $\nabla(u)$ is called the \textit{gradient} of $u$.
            In Cartesian coordinates this is defined as:
            \begin{equation*}
                \nabla(u)=\sum_{i=1}^{N}\frac{\partial u}
                    {\partial x_{i}}\hat{\mathbf{x}}_{i}
            \end{equation*}
            Where $\frac{\partial u}{\partial x_{i}}$
            is the partial derivative
            of $u$ with respect to the $i^{th}$ coordinate.
            We have $u=xy$, so:
            \begin{equation*}
                \nabla(u)
                =\frac{\partial(xy)}{\partial x}\hat{\mathbf{x}}
                +\frac{\partial(xy)}{\partial y}\hat{\mathbf{y}}
                =y\hat{\mathbf{x}}+x\hat{\mathbf{y}}
            \end{equation*}
            When $u=3$ and $x=2$, we have $y=\frac{3}{2}$.
            So the component of $\mathbf{A}$ along $\nabla(u)$
            when $u=3$ and $x=2$ is:
            \begin{align*}
                \comp_{\nabla(u)}(\mathbf{A})
                &=
                \mathbf{A}\cdot\frac{\nabla(u)}{\norm{\nabla(u)}}
                &
                &=\frac{9+8}{2\sqrt{\frac{9+16}{4}}}\\
                &=
                (3\hat{\mathbf{x}}
                +2\hat{\mathbf{y}}
                +4\hat{\mathbf{z}})
                \cdot
                \frac{%
                    \frac{3}{2}\hat{\mathbf{x}}%
                    +2\hat{\mathbf{y}}%
                }{%
                    \sqrt{(\frac{3}{2})^{2}+2^{2}}%
                }
                &
                &=\frac{17}{\sqrt{25}}\\
                &=\frac{\frac{9}{2}+4}{\sqrt{\frac{9}{4}+4}}
                &
                &=\frac{17}{5}\\
            \end{align*}
        \end{proof}
        \begin{problem}[Wangsness 1-9]
            An ellipsoid is define by
            $u%
             =\frac{x^{2}}{a^{2}}%
             +\frac{y^{2}}{b^{2}}%
             +\frac{z^{2}}{c^{2}}$.
            Find the unit vector normal to the
            surface of each point of an
            ellipsoid.
        \end{problem}
        \begin{proof}[Solution]
            The vector normal to a surface $u$
            is the gradient: $\nabla(u)$.
            The unit vector normal to a surface would then be
            $\frac{\nabla(u)}{\norm{\nabla(u)}}$. We have:
            \begin{equation*}
                \nabla(u)
                =\frac{\partial u}{\partial x}\hat{\mathbf{x}}
                +\frac{\partial u}{\partial y}\hat{\mathbf{y}}
                +\frac{\partial y}{\partial z}\hat{\mathbf{z}}
                =2\frac{x}{a^2}\hat{\mathbf{x}}
                +2\frac{y}{b^2}\hat{\mathbf{y}}
                +2\frac{z}{c^2}\hat{\mathbf{z}}
            \end{equation*}
            The norm of $\nabla(u)$ and the
            unit vector normal to the surface $u$ are:
            \begin{align*}
                \norm{\nabla(u)}
                &=
                \sqrt{\big(\nabla_{x}(u)\big)^{2}
                +\big(\nabla_{y}(u)\big)^{2}
                +\big(\nabla_{z}(u)\big)^{2}}
                &\hat{\mathbf{n}}
                &=\frac{\nabla(u)}{\norm{\nabla(u)}}\\
                &=
                \sqrt{\frac{4x^{2}}{a^{4}}
                +\frac{4y^{2}}{b^{4}}+\frac{4z^{2}}{c^{4}}}
                &
                &=\frac{2\frac{x}{a^{2}}\hat{\mathbf{x}}
                        +2\frac{y}{b^{2}}\hat{\mathbf{y}}
                        +2\frac{z}{c^{2}}\hat{\mathbf{z}}
                       }{
                            2\sqrt{\frac{x^{2}}{a^{4}}
                            +\frac{y^{2}}{b^{4}}
                            +\frac{z^{2}}{c^{4}}
                        }
                    }\\
                &=
                2\sqrt{\frac{x^{2}}{a^{4}}
                +\frac{y^{2}}{b^{4}}
                +\frac{z^{2}}{c^{4}}}
                &
                &=
                \frac{
                      \frac{x}{a^{2}}\hat{\mathbf{x}}
                      +\frac{y}{b^{2}}\hat{\mathbf{y}}
                      +\frac{z}{c^{2}}\hat{\mathbf{z}}}
                     {
                        \sqrt{
                            \frac{x^{2}}{a^{4}}
                            +\frac{y^{2}}{b^{4}}
                            +\frac{z^{2}}{c^{4}}
                        }
                    }
            \end{align*}
        \end{proof}
\end{document}