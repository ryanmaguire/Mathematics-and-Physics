\documentclass[crop=false,class=article,oneside]{standalone}
%----------------------------Preamble-------------------------------%
%---------------------------Packages----------------------------%
\usepackage{geometry}
\geometry{b5paper, margin=1.0in}
\usepackage[T1]{fontenc}
\usepackage{graphicx, float}            % Graphics/Images.
\usepackage{natbib}                     % For bibliographies.
\bibliographystyle{agsm}                % Bibliography style.
\usepackage[french, english]{babel}     % Language typesetting.
\usepackage[dvipsnames]{xcolor}         % Color names.
\usepackage{listings}                   % Verbatim-Like Tools.
\usepackage{mathtools, esint, mathrsfs} % amsmath and integrals.
\usepackage{amsthm, amsfonts, amssymb}  % Fonts and theorems.
\usepackage{tcolorbox}                  % Frames around theorems.
\usepackage{upgreek}                    % Non-Italic Greek.
\usepackage{fmtcount, etoolbox}         % For the \book{} command.
\usepackage[newparttoc]{titlesec}       % Formatting chapter, etc.
\usepackage{titletoc}                   % Allows \book in toc.
\usepackage[nottoc]{tocbibind}          % Bibliography in toc.
\usepackage[titles]{tocloft}            % ToC formatting.
\usepackage{pgfplots, tikz}             % Drawing/graphing tools.
\usepackage{imakeidx}                   % Used for index.
\usetikzlibrary{
    calc,                   % Calculating right angles and more.
    angles,                 % Drawing angles within triangles.
    arrows.meta,            % Latex and Stealth arrows.
    quotes,                 % Adding labels to angles.
    positioning,            % Relative positioning of nodes.
    decorations.markings,   % Adding arrows in the middle of a line.
    patterns,
    arrows
}                                       % Libraries for tikz.
\pgfplotsset{compat=1.9}                % Version of pgfplots.
\usepackage[font=scriptsize,
            labelformat=simple,
            labelsep=colon]{subcaption} % Subfigure captions.
\usepackage[font={scriptsize},
            hypcap=true,
            labelsep=colon]{caption}    % Figure captions.
\usepackage[pdftex,
            pdfauthor={Ryan Maguire},
            pdftitle={Mathematics and Physics},
            pdfsubject={Mathematics, Physics, Science},
            pdfkeywords={Mathematics, Physics, Computer Science, Biology},
            pdfproducer={LaTeX},
            pdfcreator={pdflatex}]{hyperref}
\hypersetup{
    colorlinks=true,
    linkcolor=blue,
    filecolor=magenta,
    urlcolor=Cerulean,
    citecolor=SkyBlue
}                           % Colors for hyperref.
\usepackage[toc,acronym,nogroupskip,nopostdot]{glossaries}
\usepackage{glossary-mcols}
%------------------------Theorem Styles-------------------------%
\theoremstyle{plain}
\newtheorem{theorem}{Theorem}[section]

% Define theorem style for default spacing and normal font.
\newtheoremstyle{normal}
    {\topsep}               % Amount of space above the theorem.
    {\topsep}               % Amount of space below the theorem.
    {}                      % Font used for body of theorem.
    {}                      % Measure of space to indent.
    {\bfseries}             % Font of the header of the theorem.
    {}                      % Punctuation between head and body.
    {.5em}                  % Space after theorem head.
    {}

% Italic header environment.
\newtheoremstyle{thmit}{\topsep}{\topsep}{}{}{\itshape}{}{0.5em}{}

% Define environments with italic headers.
\theoremstyle{thmit}
\newtheorem*{solution}{Solution}

% Define default environments.
\theoremstyle{normal}
\newtheorem{example}{Example}[section]
\newtheorem{definition}{Definition}[section]
\newtheorem{problem}{Problem}[section]

% Define framed environment.
\tcbuselibrary{most}
\newtcbtheorem[use counter*=theorem]{ftheorem}{Theorem}{%
    before=\par\vspace{2ex},
    boxsep=0.5\topsep,
    after=\par\vspace{2ex},
    colback=green!5,
    colframe=green!35!black,
    fonttitle=\bfseries\upshape%
}{thm}

\newtcbtheorem[auto counter, number within=section]{faxiom}{Axiom}{%
    before=\par\vspace{2ex},
    boxsep=0.5\topsep,
    after=\par\vspace{2ex},
    colback=Apricot!5,
    colframe=Apricot!35!black,
    fonttitle=\bfseries\upshape%
}{ax}

\newtcbtheorem[use counter*=definition]{fdefinition}{Definition}{%
    before=\par\vspace{2ex},
    boxsep=0.5\topsep,
    after=\par\vspace{2ex},
    colback=blue!5!white,
    colframe=blue!75!black,
    fonttitle=\bfseries\upshape%
}{def}

\newtcbtheorem[use counter*=example]{fexample}{Example}{%
    before=\par\vspace{2ex},
    boxsep=0.5\topsep,
    after=\par\vspace{2ex},
    colback=red!5!white,
    colframe=red!75!black,
    fonttitle=\bfseries\upshape%
}{ex}

\newtcbtheorem[auto counter, number within=section]{fnotation}{Notation}{%
    before=\par\vspace{2ex},
    boxsep=0.5\topsep,
    after=\par\vspace{2ex},
    colback=SeaGreen!5!white,
    colframe=SeaGreen!75!black,
    fonttitle=\bfseries\upshape%
}{not}

\newtcbtheorem[use counter*=remark]{fremark}{Remark}{%
    fonttitle=\bfseries\upshape,
    colback=Goldenrod!5!white,
    colframe=Goldenrod!75!black}{ex}

\newenvironment{bproof}{\textit{Proof.}}{\hfill$\square$}
\tcolorboxenvironment{bproof}{%
    blanker,
    breakable,
    left=3mm,
    before skip=5pt,
    after skip=10pt,
    borderline west={0.6mm}{0pt}{green!80!black}
}

\AtEndEnvironment{lexample}{$\hfill\textcolor{red}{\blacksquare}$}
\newtcbtheorem[use counter*=example]{lexample}{Example}{%
    empty,
    title={Example~\theexample},
    boxed title style={%
        empty,
        size=minimal,
        toprule=2pt,
        top=0.5\topsep,
    },
    coltitle=red,
    fonttitle=\bfseries,
    parbox=false,
    boxsep=0pt,
    before=\par\vspace{2ex},
    left=0pt,
    right=0pt,
    top=3ex,
    bottom=1ex,
    before=\par\vspace{2ex},
    after=\par\vspace{2ex},
    breakable,
    pad at break*=0mm,
    vfill before first,
    overlay unbroken={%
        \draw[red, line width=2pt]
            ([yshift=-1.2ex]title.south-|frame.west) to
            ([yshift=-1.2ex]title.south-|frame.east);
        },
    overlay first={%
        \draw[red, line width=2pt]
            ([yshift=-1.2ex]title.south-|frame.west) to
            ([yshift=-1.2ex]title.south-|frame.east);
    },
}{ex}

\AtEndEnvironment{ldefinition}{$\hfill\textcolor{Blue}{\blacksquare}$}
\newtcbtheorem[use counter*=definition]{ldefinition}{Definition}{%
    empty,
    title={Definition~\thedefinition:~{#1}},
    boxed title style={%
        empty,
        size=minimal,
        toprule=2pt,
        top=0.5\topsep,
    },
    coltitle=Blue,
    fonttitle=\bfseries,
    parbox=false,
    boxsep=0pt,
    before=\par\vspace{2ex},
    left=0pt,
    right=0pt,
    top=3ex,
    bottom=0pt,
    before=\par\vspace{2ex},
    after=\par\vspace{1ex},
    breakable,
    pad at break*=0mm,
    vfill before first,
    overlay unbroken={%
        \draw[Blue, line width=2pt]
            ([yshift=-1.2ex]title.south-|frame.west) to
            ([yshift=-1.2ex]title.south-|frame.east);
        },
    overlay first={%
        \draw[Blue, line width=2pt]
            ([yshift=-1.2ex]title.south-|frame.west) to
            ([yshift=-1.2ex]title.south-|frame.east);
    },
}{def}

\AtEndEnvironment{ltheorem}{$\hfill\textcolor{Green}{\blacksquare}$}
\newtcbtheorem[use counter*=theorem]{ltheorem}{Theorem}{%
    empty,
    title={Theorem~\thetheorem:~{#1}},
    boxed title style={%
        empty,
        size=minimal,
        toprule=2pt,
        top=0.5\topsep,
    },
    coltitle=Green,
    fonttitle=\bfseries,
    parbox=false,
    boxsep=0pt,
    before=\par\vspace{2ex},
    left=0pt,
    right=0pt,
    top=3ex,
    bottom=-1.5ex,
    breakable,
    pad at break*=0mm,
    vfill before first,
    overlay unbroken={%
        \draw[Green, line width=2pt]
            ([yshift=-1.2ex]title.south-|frame.west) to
            ([yshift=-1.2ex]title.south-|frame.east);},
    overlay first={%
        \draw[Green, line width=2pt]
            ([yshift=-1.2ex]title.south-|frame.west) to
            ([yshift=-1.2ex]title.south-|frame.east);
    }
}{thm}

%--------------------Declared Math Operators--------------------%
\DeclareMathOperator{\adjoint}{adj}         % Adjoint.
\DeclareMathOperator{\Card}{Card}           % Cardinality.
\DeclareMathOperator{\curl}{curl}           % Curl.
\DeclareMathOperator{\diam}{diam}           % Diameter.
\DeclareMathOperator{\dist}{dist}           % Distance.
\DeclareMathOperator{\Div}{div}             % Divergence.
\DeclareMathOperator{\Erf}{Erf}             % Error Function.
\DeclareMathOperator{\Erfc}{Erfc}           % Complementary Error Function.
\DeclareMathOperator{\Ext}{Ext}             % Exterior.
\DeclareMathOperator{\GCD}{GCD}             % Greatest common denominator.
\DeclareMathOperator{\grad}{grad}           % Gradient
\DeclareMathOperator{\Ima}{Im}              % Image.
\DeclareMathOperator{\Int}{Int}             % Interior.
\DeclareMathOperator{\LC}{LC}               % Leading coefficient.
\DeclareMathOperator{\LCM}{LCM}             % Least common multiple.
\DeclareMathOperator{\LM}{LM}               % Leading monomial.
\DeclareMathOperator{\LT}{LT}               % Leading term.
\DeclareMathOperator{\Mod}{mod}             % Modulus.
\DeclareMathOperator{\Mon}{Mon}             % Monomial.
\DeclareMathOperator{\multideg}{mutlideg}   % Multi-Degree (Graphs).
\DeclareMathOperator{\nul}{nul}             % Null space of operator.
\DeclareMathOperator{\Ord}{Ord}             % Ordinal of ordered set.
\DeclareMathOperator{\Prin}{Prin}           % Principal value.
\DeclareMathOperator{\proj}{proj}           % Projection.
\DeclareMathOperator{\Refl}{Refl}           % Reflection operator.
\DeclareMathOperator{\rk}{rk}               % Rank of operator.
\DeclareMathOperator{\sgn}{sgn}             % Sign of a number.
\DeclareMathOperator{\sinc}{sinc}           % Sinc function.
\DeclareMathOperator{\Span}{Span}           % Span of a set.
\DeclareMathOperator{\Spec}{Spec}           % Spectrum.
\DeclareMathOperator{\supp}{supp}           % Support
\DeclareMathOperator{\Tr}{Tr}               % Trace of matrix.
%--------------------Declared Math Symbols--------------------%
\DeclareMathSymbol{\minus}{\mathbin}{AMSa}{"39} % Unary minus sign.
%------------------------New Commands---------------------------%
\DeclarePairedDelimiter\norm{\lVert}{\rVert}
\DeclarePairedDelimiter\ceil{\lceil}{\rceil}
\DeclarePairedDelimiter\floor{\lfloor}{\rfloor}
\newcommand*\diff{\mathop{}\!\mathrm{d}}
\newcommand*\Diff[1]{\mathop{}\!\mathrm{d^#1}}
\renewcommand*{\glstextformat}[1]{\textcolor{RoyalBlue}{#1}}
\renewcommand{\glsnamefont}[1]{\textbf{#1}}
\renewcommand\labelitemii{$\circ$}
\renewcommand\thesubfigure{%
    \arabic{chapter}.\arabic{figure}.\arabic{subfigure}}
\addto\captionsenglish{\renewcommand{\figurename}{Fig.}}
\numberwithin{equation}{section}

\renewcommand{\vector}[1]{\boldsymbol{\mathrm{#1}}}

\newcommand{\uvector}[1]{\boldsymbol{\hat{\mathrm{#1}}}}
\newcommand{\topspace}[2][]{(#2,\tau_{#1})}
\newcommand{\measurespace}[2][]{(#2,\varSigma_{#1},\mu_{#1})}
\newcommand{\measurablespace}[2][]{(#2,\varSigma_{#1})}
\newcommand{\manifold}[2][]{(#2,\tau_{#1},\mathcal{A}_{#1})}
\newcommand{\tanspace}[2]{T_{#1}{#2}}
\newcommand{\cotanspace}[2]{T_{#1}^{*}{#2}}
\newcommand{\Ckspace}[3][\mathbb{R}]{C^{#2}(#3,#1)}
\newcommand{\funcspace}[2][\mathbb{R}]{\mathcal{F}(#2,#1)}
\newcommand{\smoothvecf}[1]{\mathfrak{X}(#1)}
\newcommand{\smoothonef}[1]{\mathfrak{X}^{*}(#1)}
\newcommand{\bracket}[2]{[#1,#2]}

%------------------------Book Command---------------------------%
\makeatletter
\renewcommand\@pnumwidth{1cm}
\newcounter{book}
\renewcommand\thebook{\@Roman\c@book}
\newcommand\book{%
    \if@openright
        \cleardoublepage
    \else
        \clearpage
    \fi
    \thispagestyle{plain}%
    \if@twocolumn
        \onecolumn
        \@tempswatrue
    \else
        \@tempswafalse
    \fi
    \null\vfil
    \secdef\@book\@sbook
}
\def\@book[#1]#2{%
    \refstepcounter{book}
    \addcontentsline{toc}{book}{\bookname\ \thebook:\hspace{1em}#1}
    \markboth{}{}
    {\centering
     \interlinepenalty\@M
     \normalfont
     \huge\bfseries\bookname\nobreakspace\thebook
     \par
     \vskip 20\p@
     \Huge\bfseries#2\par}%
    \@endbook}
\def\@sbook#1{%
    {\centering
     \interlinepenalty \@M
     \normalfont
     \Huge\bfseries#1\par}%
    \@endbook}
\def\@endbook{
    \vfil\newpage
        \if@twoside
            \if@openright
                \null
                \thispagestyle{empty}%
                \newpage
            \fi
        \fi
        \if@tempswa
            \twocolumn
        \fi
}
\newcommand*\l@book[2]{%
    \ifnum\c@tocdepth >-3\relax
        \addpenalty{-\@highpenalty}%
        \addvspace{2.25em\@plus\p@}%
        \setlength\@tempdima{3em}%
        \begingroup
            \parindent\z@\rightskip\@pnumwidth
            \parfillskip -\@pnumwidth
            {
                \leavevmode
                \Large\bfseries#1\hfill\hb@xt@\@pnumwidth{\hss#2}
            }
            \par
            \nobreak
            \global\@nobreaktrue
            \everypar{\global\@nobreakfalse\everypar{}}%
        \endgroup
    \fi}
\newcommand\bookname{Book}
\renewcommand{\thebook}{\texorpdfstring{\Numberstring{book}}{book}}
\providecommand*{\toclevel@book}{-2}
\makeatother
\titleformat{\part}[display]
    {\Large\bfseries}
    {\partname\nobreakspace\thepart}
    {0mm}
    {\Huge\bfseries}
\titlecontents{part}[0pt]
    {\large\bfseries}
    {\partname\ \thecontentslabel: \quad}
    {}
    {\hfill\contentspage}
\titlecontents{chapter}[0pt]
    {\bfseries}
    {\chaptername\ \thecontentslabel:\quad}
    {}
    {\hfill\contentspage}
\newglossarystyle{longpara}{%
    \setglossarystyle{long}%
    \renewenvironment{theglossary}{%
        \begin{longtable}[l]{{p{0.25\hsize}p{0.65\hsize}}}
    }{\end{longtable}}%
    \renewcommand{\glossentry}[2]{%
        \glstarget{##1}{\glossentryname{##1}}%
        &\glossentrydesc{##1}{~##2.}
        \tabularnewline%
        \tabularnewline
    }%
}
\newglossary[not-glg]{notation}{not-gls}{not-glo}{Notation}
\newcommand*{\newnotation}[4][]{%
    \newglossaryentry{#2}{type=notation, name={\textbf{#3}, },
                          text={#4}, description={#4},#1}%
}
%--------------------------LENGTHS------------------------------%
% Spacings for the Table of Contents.
\addtolength{\cftsecnumwidth}{1ex}
\addtolength{\cftsubsecindent}{1ex}
\addtolength{\cftsubsecnumwidth}{1ex}
\addtolength{\cftfignumwidth}{1ex}
\addtolength{\cfttabnumwidth}{1ex}

% Indent and paragraph spacing.
\setlength{\parindent}{0em}
\setlength{\parskip}{0em}
%----------------------------GLOSSARY-------------------------------%
\makeglossaries
\loadglsentries{../../../glossary}
\loadglsentries{../../../acronym}
%--------------------------Main Document----------------------------%
\begin{document}
    \ifx\ifphysicscourseselectromagnetismI\undefined
        \section*{Electromagnetism I}
        \setcounter{section}{5}
        \renewcommand\thesubfigure{%
            \arabic{section}.\arabic{figure}.\arabic{subfigure}%
        }
    \fi
    \subsection{Homework V}
    Wangsness Chapter 4 - Problems: 3, 5, 6, 7, 11, 12
    \begin{problem}[Wangsness 4-3]
        \label{problem:EMAG_wangsness_4_3}
        An infinitely long line is surrounded by an infinitely long cylinder of radius
        $\rho_{0}$ whose axis coincides with the line charge
        (See Fig.~\ref{fig:EMAG_1_Wangsness_4_3}). The surface of the 
        cylinder carries a charge of constant surface density $\sigma$. Find $\mathbf{E}$
        everywhere. What particular value of $\sigma$ will make $\mathbf{E}=\mathbf{0}$
        for all points outside of the charged cylinder?
    \end{problem}
    \begin{proof}[Solution]
        For $\rho<\rho_0$ choose a Gaussian cylinder concentric with the line. From
        Gauss' Law we have:
        \begin{equation*}
            \oiint \mathbf{E}\cdot \mathbf{da} = \frac{Q_{in}}{\epsilon_0}
            =E(2\pi\rho\ell)\Rightarrow
            \mathbf{E}=\frac{\lambda}{2\pi\epsilon_{0}\rho}\hat{\boldsymbol{\uprho}}
        \end{equation*}
        For $\rho>\rho_0$ choose a similar Gaussian cylinder. We get:
        \begin{equation*}
            \oiint\mathbf{E}\cdot\mathbf{da}=\frac{Q_{in}}{\epsilon_{0}}\Rightarrow
            \mathbf{E}=\frac{\lambda\ell+\sigma 2\pi\rho_{0}\ell}
            {2\pi\epsilon_{0}\rho\ell}\hat{\boldsymbol{\uprho}}
            =\frac{\lambda+2\pi\rho_{0}\sigma}{2\pi\epsilon_{0}\rho}
            \hat{\boldsymbol{\uprho}}
        \end{equation*}
        To make $\hat{\mathbf{E}}=\mathbf{0}$ for $\rho>\rho_0$ we need
        $\sigma=\frac{-\lambda}{2\pi\rho_0}$.
    \end{proof}
    \begin{figure}[H]
        \centering
        \begin{tikzpicture}[every path/.style={thick}]
            \draw[draw=black,thick=semithick]
                (-6,0) -- (6,0) node[above left] {$\lambda$};
            \draw[draw=blue]                (-4,0) circle (1);
            \draw[draw=blue]                (-4,1) -- (3,1);
            \draw[draw=blue]                (3,1) arc (90:-90:1);
            \draw[draw=blue]                (-4,-1) -- node[below] {$\sigma$} (3,-1);
            \draw[draw=red,densely dashed]  (-2,0) circle (0.75);
            \draw[draw=black]
                (-4,0) -- node[above right=0.02cm and 0.01cm] {$\rho_{0}$}(-4.71,0.71);
            \draw[draw=red,densely dashed]  (-2,0.75) -- node[below] {$\ell$} (2,0.75);
            \draw[draw=red,densely dashed]  (-2,-0.75) -- (2,-0.75);
            \draw[draw=red,densely dashed]  (2,0.75) arc (90:-90:0.75);
        \end{tikzpicture}
        \caption[Drawing for Wangsness 4-3]
        {Drawing for problem \ref{problem:EMAG_wangsness_4_3}}
        \label{fig:EMAG_1_Wangsness_4_3}
    \end{figure}
    \begin{problem}[Wangsness 4-5]
        \label{problem:EMAG_wangsness_4_5}
        A sphere of radius $a$ has a charge density that varies with distance $r$ from
        the center according to $\rho=Ar^{1/2}$, where $A$ is a constant. Find
        $\mathbf{E}$ everywhere.
    \end{problem}
    \begin{proof}[Solution]
        For $r>a$
        \begin{equation*}
            \oiint\mathbf{E}\cdot\mathbf{da}=\frac{Q_{in}}{\epsilon_{0}}\Rightarrow
            E(4\pi r^{2})=\iiint Ar'^{1/2}r'^{2}\sin(\theta')dr'd\theta'd\phi'
            =4\pi\frac{2}{7}\frac{a^{7/2}}{\epsilon_{0}}A\Rightarrow
            \mathbf{E}=\frac{2A a^{7/2}}{7\epsilon_{0}r^{2}}\hat{\mathbf{r}}
        \end{equation*}
        For $r<a$
        \begin{equation*}
            E(4\pi r^{2})=4\pi A\frac{2}{7}r^{7/2}\Rightarrow
            \mathbf{E}=\frac{2Ar^{3/2}}{7\epsilon_{0}}\hat{\mathbf{r}}
        \end{equation*}
    \end{proof}
    \begin{figure}[H]
        \centering
        \begin{tikzpicture}[every path/.style={thick}]
            \draw[draw=blue]                (-3.5,0) circle (2);
            \draw[draw=red,densely dashed]  (-3.5,0) circle (3);
            \draw[draw=red,densely dashed]  (3.5,0) circle (2);
            \draw[draw=blue]                (3.5,0) circle (3);
            \draw[draw=black] (-3.5,0) -- node[below left] {$a$} (-4.914,1.414);
            \draw[draw=black] (-3.5,0) -- node[below right] {$r$} (-1.378,2.121);
            \filldraw[fill=black] (-1.378,2.121) circle (0.5mm) node[above right] {$P$};
            \draw[draw=black] (3.5,0) -- node[below right] {$r$} (4.914,1.414);
            \draw[draw=black] (3.5,0) -- node[below left] {$a$} (1.378,2.121);
            \filldraw[fill=black] (4.914,1.414) circle (0.5mm) node[above right] {$P$};
        \end{tikzpicture}
        \caption[Drawing for Wangsness 4-3]
        {Drawing for problem \ref{problem:EMAG_wangsness_4_5}.
        Dashed red represents the Gaussian surface, and blue represents
        the charged sphere.}
        \label{fig:EMAG_1_Wangsness_4_5}
    \end{figure}
    \begin{problem}[Wangsness 4-6]
        Two concentric spheres have radii $a$ and $b$ with $b>a$. The between them
        $(a\leq r\leq b)$ is filled with a charge of constant density. The charge density
        is zero everywhere else. Find $\mathbf{E}$ everywhere and express it in terms of
        the total charge $Q$. What happens as $a\rightarrow 0$?
    \end{problem}
    \begin{proof}[Solution]
        From the symmetry of the problem we have, for $r<a$, that:
        \begin{equation*}
            \oiint\mathbf{E}\cdot\mathbf{da}=0\Rightarrow\mathbf{E}=\mathbf{0}
        \end{equation*}
        For $a\leq r\leq b$:
        \begin{equation*}
            \oiint\mathbf{E}\cdot\mathbf{da}=\frac{Q_{in}}{\epsilon_0}=
            \frac{\rho_{c}(\frac{4}{3}\pi)(r^{3}-a^{3})}{\epsilon_0}    
        \end{equation*}
        Where $\rho_{c}$ is the charge density
        $\rho_{c}=\frac{Q}{\frac{4}{3}\pi(b^{3}-a^{3})}$. Therefore:
        \begin{equation*}
            \mathbf{E}=\frac{Q}{4\pi\epsilon_{0}r^{2}}
            \bigg(\frac{r^{3}-a^{3}}{b^{3}-a^{3}}\bigg)\hat{\mathbf{r}}    
        \end{equation*}
        For $r\geq b$, this is similar to the uniform sphere problem:
        \begin{equation*}
            \oiint\mathbf{E}\cdot\mathbf{da}=\frac{Q_{in}}{\epsilon_{0}}\Rightarrow
            \mathbf{E}=\frac{Q}{4\pi\epsilon_{0}r^{2}}
        \end{equation*}
        In the limit as $a\rightarrow 0$ we have:
        \begin{equation*}
            \mathbf{E}=
            \begin{cases}
                \frac{Qr}{4\pi\epsilon_{0}b^{3}},&r\leq b\\
                \frac{Q}{4\pi\epsilon_{0}r^{2}},&r>b
            \end{cases}
        \end{equation*}
        This is expected, for in the limit as $a\rightarrow 0$ we obtain a uniformly
        charged sphere of radius $b$.
    \end{proof}
    \subsubsection{Wangsness 4-7}
    Choose as a Gaussian surface a cylinder of length $L$ and radius $\rho$ that is concentric with the infinite cylinder. For $\rho<a$, $\oiint \mathbf{E}\cdot \mathbf{da} = \frac{Q_{in}}{\epsilon_0}$. Now $\oiint \mathbf{E} \cdot \mathbf{da} = \iint_{Right\ Side} \mathbf{E}\cdot \mathbf{da} + \iint_{Left\ Side}\mathbf{E}\cdot \mathbf{da} + \iint_{Cylinder} \mathbf{E}\cdot \mathbf{da}$. From symmetry, we have that $\mathbf{E}$ and $\mathbf{da}$ are orthogonal along the left and right faces of the cylinder, leaving only the cylindrical body left to integrate over. We get $E(2\pi \rho L) = \frac{\rho_c \pi \rho^2 L}{\epsilon_0}$, where $\rho_c$ is the charge density. Combining this together, we obtain $\mathbf{E} = \frac{\rho_c \rho}{2\epsilon_0} \hat{\boldsymbol{\uprho}}$. For $\rho>a$, $\mathbf{E} = \frac{\rho_{c} a^2}{2\epsilon_0 \rho}\hat{\boldsymbol{\uprho}}$. We see that the electric field goes like $\frac{1}{\rho}$, which is consistent with the result obtained from $4-11$.
    \subsubsection{Wangsness 4-11}
    $\mathbf{E} = E_0 \big(\frac{\rho}{a}\big)^3 \hat{\boldsymbol{\uprho}}$ for $0 < \rho < a$, and $\mathbf{E} = 0$ otherwise. Thus, $\nabla \cdot \mathbf{E} = \frac{1}{\rho} \frac{\partial}{\partial \rho}\big(\rho E_{\rho}\big) + \frac{1}{\rho} \frac{\partial E_{\phi}}{\partial \phi} + \frac{\partial E_z}{\partial z}$ = $\frac{4E_0 \rho^2}{a^3}$. From Gauss' Law, $\nabla \cdot \mathbf{E} = \frac{\rho_c}{\epsilon_0}$. Thus, $\rho_c = \epsilon_0 \nabla \cdot \mathbf{E} = \frac{4\epsilon_0 E_0 \rho^2}{a^3}$ for $\rho<a$. For $\rho>a$, $\nabla \cdot \mathbf{E} = \nabla \cdot \mathbf{0} = 0$, and thus $\rho_c = 0$.
    \subsubsection{Wangsness 4-12}
    $\mathbf{E} = E_r \hat{\mathbf{r}}+E_{\theta} \hat{\boldsymbol{\uptheta}}$, where $E_r = \frac{2A\cos(\theta)}{r^3}$ and $E_{\theta} = \frac{A\sin(\theta)}{r^3}$. From Gauss' Law, $\nabla \cdot \mathbf{E} = \frac{\rho_c}{\epsilon_0}$, and thus $\rho_c = \epsilon_0 \nabla \cdot \mathbf{E}$. But $\nabla \cdot \mathbf{E} = 0$, and thus $\rho_c = 0$.
\end{document}