\documentclass[crop=false,class=article,oneside]{standalone}
%----------------------------Preamble-------------------------------%
%---------------------------Packages----------------------------%
\usepackage{geometry}
\geometry{b5paper, margin=1.0in}
\usepackage[T1]{fontenc}
\usepackage{graphicx, float}            % Graphics/Images.
\usepackage{natbib}                     % For bibliographies.
\bibliographystyle{agsm}                % Bibliography style.
\usepackage[french, english]{babel}     % Language typesetting.
\usepackage[dvipsnames]{xcolor}         % Color names.
\usepackage{listings}                   % Verbatim-Like Tools.
\usepackage{mathtools, esint, mathrsfs} % amsmath and integrals.
\usepackage{amsthm, amsfonts, amssymb}  % Fonts and theorems.
\usepackage{tcolorbox}                  % Frames around theorems.
\usepackage{upgreek}                    % Non-Italic Greek.
\usepackage{fmtcount, etoolbox}         % For the \book{} command.
\usepackage[newparttoc]{titlesec}       % Formatting chapter, etc.
\usepackage{titletoc}                   % Allows \book in toc.
\usepackage[nottoc]{tocbibind}          % Bibliography in toc.
\usepackage[titles]{tocloft}            % ToC formatting.
\usepackage{pgfplots, tikz}             % Drawing/graphing tools.
\usepackage{imakeidx}                   % Used for index.
\usetikzlibrary{
    calc,                   % Calculating right angles and more.
    angles,                 % Drawing angles within triangles.
    arrows.meta,            % Latex and Stealth arrows.
    quotes,                 % Adding labels to angles.
    positioning,            % Relative positioning of nodes.
    decorations.markings,   % Adding arrows in the middle of a line.
    patterns,
    arrows
}                                       % Libraries for tikz.
\pgfplotsset{compat=1.9}                % Version of pgfplots.
\usepackage[font=scriptsize,
            labelformat=simple,
            labelsep=colon]{subcaption} % Subfigure captions.
\usepackage[font={scriptsize},
            hypcap=true,
            labelsep=colon]{caption}    % Figure captions.
\usepackage[pdftex,
            pdfauthor={Ryan Maguire},
            pdftitle={Mathematics and Physics},
            pdfsubject={Mathematics, Physics, Science},
            pdfkeywords={Mathematics, Physics, Computer Science, Biology},
            pdfproducer={LaTeX},
            pdfcreator={pdflatex}]{hyperref}
\hypersetup{
    colorlinks=true,
    linkcolor=blue,
    filecolor=magenta,
    urlcolor=Cerulean,
    citecolor=SkyBlue
}                           % Colors for hyperref.
\usepackage[toc,acronym,nogroupskip,nopostdot]{glossaries}
\usepackage{glossary-mcols}
%------------------------Theorem Styles-------------------------%
\theoremstyle{plain}
\newtheorem{theorem}{Theorem}[section]

% Define theorem style for default spacing and normal font.
\newtheoremstyle{normal}
    {\topsep}               % Amount of space above the theorem.
    {\topsep}               % Amount of space below the theorem.
    {}                      % Font used for body of theorem.
    {}                      % Measure of space to indent.
    {\bfseries}             % Font of the header of the theorem.
    {}                      % Punctuation between head and body.
    {.5em}                  % Space after theorem head.
    {}

% Italic header environment.
\newtheoremstyle{thmit}{\topsep}{\topsep}{}{}{\itshape}{}{0.5em}{}

% Define environments with italic headers.
\theoremstyle{thmit}
\newtheorem*{solution}{Solution}

% Define default environments.
\theoremstyle{normal}
\newtheorem{example}{Example}[section]
\newtheorem{definition}{Definition}[section]
\newtheorem{problem}{Problem}[section]

% Define framed environment.
\tcbuselibrary{most}
\newtcbtheorem[use counter*=theorem]{ftheorem}{Theorem}{%
    before=\par\vspace{2ex},
    boxsep=0.5\topsep,
    after=\par\vspace{2ex},
    colback=green!5,
    colframe=green!35!black,
    fonttitle=\bfseries\upshape%
}{thm}

\newtcbtheorem[auto counter, number within=section]{faxiom}{Axiom}{%
    before=\par\vspace{2ex},
    boxsep=0.5\topsep,
    after=\par\vspace{2ex},
    colback=Apricot!5,
    colframe=Apricot!35!black,
    fonttitle=\bfseries\upshape%
}{ax}

\newtcbtheorem[use counter*=definition]{fdefinition}{Definition}{%
    before=\par\vspace{2ex},
    boxsep=0.5\topsep,
    after=\par\vspace{2ex},
    colback=blue!5!white,
    colframe=blue!75!black,
    fonttitle=\bfseries\upshape%
}{def}

\newtcbtheorem[use counter*=example]{fexample}{Example}{%
    before=\par\vspace{2ex},
    boxsep=0.5\topsep,
    after=\par\vspace{2ex},
    colback=red!5!white,
    colframe=red!75!black,
    fonttitle=\bfseries\upshape%
}{ex}

\newtcbtheorem[auto counter, number within=section]{fnotation}{Notation}{%
    before=\par\vspace{2ex},
    boxsep=0.5\topsep,
    after=\par\vspace{2ex},
    colback=SeaGreen!5!white,
    colframe=SeaGreen!75!black,
    fonttitle=\bfseries\upshape%
}{not}

\newtcbtheorem[use counter*=remark]{fremark}{Remark}{%
    fonttitle=\bfseries\upshape,
    colback=Goldenrod!5!white,
    colframe=Goldenrod!75!black}{ex}

\newenvironment{bproof}{\textit{Proof.}}{\hfill$\square$}
\tcolorboxenvironment{bproof}{%
    blanker,
    breakable,
    left=3mm,
    before skip=5pt,
    after skip=10pt,
    borderline west={0.6mm}{0pt}{green!80!black}
}

\AtEndEnvironment{lexample}{$\hfill\textcolor{red}{\blacksquare}$}
\newtcbtheorem[use counter*=example]{lexample}{Example}{%
    empty,
    title={Example~\theexample},
    boxed title style={%
        empty,
        size=minimal,
        toprule=2pt,
        top=0.5\topsep,
    },
    coltitle=red,
    fonttitle=\bfseries,
    parbox=false,
    boxsep=0pt,
    before=\par\vspace{2ex},
    left=0pt,
    right=0pt,
    top=3ex,
    bottom=1ex,
    before=\par\vspace{2ex},
    after=\par\vspace{2ex},
    breakable,
    pad at break*=0mm,
    vfill before first,
    overlay unbroken={%
        \draw[red, line width=2pt]
            ([yshift=-1.2ex]title.south-|frame.west) to
            ([yshift=-1.2ex]title.south-|frame.east);
        },
    overlay first={%
        \draw[red, line width=2pt]
            ([yshift=-1.2ex]title.south-|frame.west) to
            ([yshift=-1.2ex]title.south-|frame.east);
    },
}{ex}

\AtEndEnvironment{ldefinition}{$\hfill\textcolor{Blue}{\blacksquare}$}
\newtcbtheorem[use counter*=definition]{ldefinition}{Definition}{%
    empty,
    title={Definition~\thedefinition:~{#1}},
    boxed title style={%
        empty,
        size=minimal,
        toprule=2pt,
        top=0.5\topsep,
    },
    coltitle=Blue,
    fonttitle=\bfseries,
    parbox=false,
    boxsep=0pt,
    before=\par\vspace{2ex},
    left=0pt,
    right=0pt,
    top=3ex,
    bottom=0pt,
    before=\par\vspace{2ex},
    after=\par\vspace{1ex},
    breakable,
    pad at break*=0mm,
    vfill before first,
    overlay unbroken={%
        \draw[Blue, line width=2pt]
            ([yshift=-1.2ex]title.south-|frame.west) to
            ([yshift=-1.2ex]title.south-|frame.east);
        },
    overlay first={%
        \draw[Blue, line width=2pt]
            ([yshift=-1.2ex]title.south-|frame.west) to
            ([yshift=-1.2ex]title.south-|frame.east);
    },
}{def}

\AtEndEnvironment{ltheorem}{$\hfill\textcolor{Green}{\blacksquare}$}
\newtcbtheorem[use counter*=theorem]{ltheorem}{Theorem}{%
    empty,
    title={Theorem~\thetheorem:~{#1}},
    boxed title style={%
        empty,
        size=minimal,
        toprule=2pt,
        top=0.5\topsep,
    },
    coltitle=Green,
    fonttitle=\bfseries,
    parbox=false,
    boxsep=0pt,
    before=\par\vspace{2ex},
    left=0pt,
    right=0pt,
    top=3ex,
    bottom=-1.5ex,
    breakable,
    pad at break*=0mm,
    vfill before first,
    overlay unbroken={%
        \draw[Green, line width=2pt]
            ([yshift=-1.2ex]title.south-|frame.west) to
            ([yshift=-1.2ex]title.south-|frame.east);},
    overlay first={%
        \draw[Green, line width=2pt]
            ([yshift=-1.2ex]title.south-|frame.west) to
            ([yshift=-1.2ex]title.south-|frame.east);
    }
}{thm}

%--------------------Declared Math Operators--------------------%
\DeclareMathOperator{\adjoint}{adj}         % Adjoint.
\DeclareMathOperator{\Card}{Card}           % Cardinality.
\DeclareMathOperator{\curl}{curl}           % Curl.
\DeclareMathOperator{\diam}{diam}           % Diameter.
\DeclareMathOperator{\dist}{dist}           % Distance.
\DeclareMathOperator{\Div}{div}             % Divergence.
\DeclareMathOperator{\Erf}{Erf}             % Error Function.
\DeclareMathOperator{\Erfc}{Erfc}           % Complementary Error Function.
\DeclareMathOperator{\Ext}{Ext}             % Exterior.
\DeclareMathOperator{\GCD}{GCD}             % Greatest common denominator.
\DeclareMathOperator{\grad}{grad}           % Gradient
\DeclareMathOperator{\Ima}{Im}              % Image.
\DeclareMathOperator{\Int}{Int}             % Interior.
\DeclareMathOperator{\LC}{LC}               % Leading coefficient.
\DeclareMathOperator{\LCM}{LCM}             % Least common multiple.
\DeclareMathOperator{\LM}{LM}               % Leading monomial.
\DeclareMathOperator{\LT}{LT}               % Leading term.
\DeclareMathOperator{\Mod}{mod}             % Modulus.
\DeclareMathOperator{\Mon}{Mon}             % Monomial.
\DeclareMathOperator{\multideg}{mutlideg}   % Multi-Degree (Graphs).
\DeclareMathOperator{\nul}{nul}             % Null space of operator.
\DeclareMathOperator{\Ord}{Ord}             % Ordinal of ordered set.
\DeclareMathOperator{\Prin}{Prin}           % Principal value.
\DeclareMathOperator{\proj}{proj}           % Projection.
\DeclareMathOperator{\Refl}{Refl}           % Reflection operator.
\DeclareMathOperator{\rk}{rk}               % Rank of operator.
\DeclareMathOperator{\sgn}{sgn}             % Sign of a number.
\DeclareMathOperator{\sinc}{sinc}           % Sinc function.
\DeclareMathOperator{\Span}{Span}           % Span of a set.
\DeclareMathOperator{\Spec}{Spec}           % Spectrum.
\DeclareMathOperator{\supp}{supp}           % Support
\DeclareMathOperator{\Tr}{Tr}               % Trace of matrix.
%--------------------Declared Math Symbols--------------------%
\DeclareMathSymbol{\minus}{\mathbin}{AMSa}{"39} % Unary minus sign.
%------------------------New Commands---------------------------%
\DeclarePairedDelimiter\norm{\lVert}{\rVert}
\DeclarePairedDelimiter\ceil{\lceil}{\rceil}
\DeclarePairedDelimiter\floor{\lfloor}{\rfloor}
\newcommand*\diff{\mathop{}\!\mathrm{d}}
\newcommand*\Diff[1]{\mathop{}\!\mathrm{d^#1}}
\renewcommand*{\glstextformat}[1]{\textcolor{RoyalBlue}{#1}}
\renewcommand{\glsnamefont}[1]{\textbf{#1}}
\renewcommand\labelitemii{$\circ$}
\renewcommand\thesubfigure{%
    \arabic{chapter}.\arabic{figure}.\arabic{subfigure}}
\addto\captionsenglish{\renewcommand{\figurename}{Fig.}}
\numberwithin{equation}{section}

\renewcommand{\vector}[1]{\boldsymbol{\mathrm{#1}}}

\newcommand{\uvector}[1]{\boldsymbol{\hat{\mathrm{#1}}}}
\newcommand{\topspace}[2][]{(#2,\tau_{#1})}
\newcommand{\measurespace}[2][]{(#2,\varSigma_{#1},\mu_{#1})}
\newcommand{\measurablespace}[2][]{(#2,\varSigma_{#1})}
\newcommand{\manifold}[2][]{(#2,\tau_{#1},\mathcal{A}_{#1})}
\newcommand{\tanspace}[2]{T_{#1}{#2}}
\newcommand{\cotanspace}[2]{T_{#1}^{*}{#2}}
\newcommand{\Ckspace}[3][\mathbb{R}]{C^{#2}(#3,#1)}
\newcommand{\funcspace}[2][\mathbb{R}]{\mathcal{F}(#2,#1)}
\newcommand{\smoothvecf}[1]{\mathfrak{X}(#1)}
\newcommand{\smoothonef}[1]{\mathfrak{X}^{*}(#1)}
\newcommand{\bracket}[2]{[#1,#2]}

%------------------------Book Command---------------------------%
\makeatletter
\renewcommand\@pnumwidth{1cm}
\newcounter{book}
\renewcommand\thebook{\@Roman\c@book}
\newcommand\book{%
    \if@openright
        \cleardoublepage
    \else
        \clearpage
    \fi
    \thispagestyle{plain}%
    \if@twocolumn
        \onecolumn
        \@tempswatrue
    \else
        \@tempswafalse
    \fi
    \null\vfil
    \secdef\@book\@sbook
}
\def\@book[#1]#2{%
    \refstepcounter{book}
    \addcontentsline{toc}{book}{\bookname\ \thebook:\hspace{1em}#1}
    \markboth{}{}
    {\centering
     \interlinepenalty\@M
     \normalfont
     \huge\bfseries\bookname\nobreakspace\thebook
     \par
     \vskip 20\p@
     \Huge\bfseries#2\par}%
    \@endbook}
\def\@sbook#1{%
    {\centering
     \interlinepenalty \@M
     \normalfont
     \Huge\bfseries#1\par}%
    \@endbook}
\def\@endbook{
    \vfil\newpage
        \if@twoside
            \if@openright
                \null
                \thispagestyle{empty}%
                \newpage
            \fi
        \fi
        \if@tempswa
            \twocolumn
        \fi
}
\newcommand*\l@book[2]{%
    \ifnum\c@tocdepth >-3\relax
        \addpenalty{-\@highpenalty}%
        \addvspace{2.25em\@plus\p@}%
        \setlength\@tempdima{3em}%
        \begingroup
            \parindent\z@\rightskip\@pnumwidth
            \parfillskip -\@pnumwidth
            {
                \leavevmode
                \Large\bfseries#1\hfill\hb@xt@\@pnumwidth{\hss#2}
            }
            \par
            \nobreak
            \global\@nobreaktrue
            \everypar{\global\@nobreakfalse\everypar{}}%
        \endgroup
    \fi}
\newcommand\bookname{Book}
\renewcommand{\thebook}{\texorpdfstring{\Numberstring{book}}{book}}
\providecommand*{\toclevel@book}{-2}
\makeatother
\titleformat{\part}[display]
    {\Large\bfseries}
    {\partname\nobreakspace\thepart}
    {0mm}
    {\Huge\bfseries}
\titlecontents{part}[0pt]
    {\large\bfseries}
    {\partname\ \thecontentslabel: \quad}
    {}
    {\hfill\contentspage}
\titlecontents{chapter}[0pt]
    {\bfseries}
    {\chaptername\ \thecontentslabel:\quad}
    {}
    {\hfill\contentspage}
\newglossarystyle{longpara}{%
    \setglossarystyle{long}%
    \renewenvironment{theglossary}{%
        \begin{longtable}[l]{{p{0.25\hsize}p{0.65\hsize}}}
    }{\end{longtable}}%
    \renewcommand{\glossentry}[2]{%
        \glstarget{##1}{\glossentryname{##1}}%
        &\glossentrydesc{##1}{~##2.}
        \tabularnewline%
        \tabularnewline
    }%
}
\newglossary[not-glg]{notation}{not-gls}{not-glo}{Notation}
\newcommand*{\newnotation}[4][]{%
    \newglossaryentry{#2}{type=notation, name={\textbf{#3}, },
                          text={#4}, description={#4},#1}%
}
%--------------------------LENGTHS------------------------------%
% Spacings for the Table of Contents.
\addtolength{\cftsecnumwidth}{1ex}
\addtolength{\cftsubsecindent}{1ex}
\addtolength{\cftsubsecnumwidth}{1ex}
\addtolength{\cftfignumwidth}{1ex}
\addtolength{\cfttabnumwidth}{1ex}

% Indent and paragraph spacing.
\setlength{\parindent}{0em}
\setlength{\parskip}{0em}
%--------------------------Main Document----------------------------%
\begin{document}
    \ifx\ifphysicscourseselectromagnetismI\undefined
        \section*{Electromagnetism I}
        \setcounter{section}{2}
        \renewcommand\thefigure{%
            \arabic{section}.\arabic{figure}%
        }
        \renewcommand\thesubfigure{%
            \arabic{section}.\arabic{figure}.\arabic{subfigure}%
        }
    \fi    
    \subsection{Homework II}
        Wangsness Chapter 1 - Problems: 11, 12, 13, 14, 15
        \begin{problem}[Wangsness 1-11]
            \label{problem:EMAG_1_Wangsness_1_11}
            Calculate the path integral of
            $\mathbf{A}=x^{2}\hat{\mathbf{x}}
            +y^{2}\hat{\mathbf{y}}+z^{2}\hat{\mathbf{z}}$
            along the path shown in Fig.~\subref{%
                fig:EMAG_1_path_of_integration_for_wangsness_1_11%
            }
            by integrating over $y$.
        \end{problem}
        \begin{proof}[Solution]
            The \textit{path integral} of $\mathbf{A}$
            along a path $C$ is:
            \begin{equation*}
                \int_{C}\mathbf{A}\cdot\boldsymbol{\diff{\ell}}
                =\int_{C}\mathbf{A}\cdot\big(
                    \diff{x}\hat{\mathbf{x}}
                   +\diff{y}\hat{\mathbf{y}}
                   +\diff{z}\hat{\mathbf{z}}
                \big)
            \end{equation*}
            We have
            $\mathbf{A}%
             =x^{2}\hat{\mathbf{x}}%
             +y^{2}\hat{\mathbf{y}}%
             +z^{2}\hat{\mathbf{z}}$.
            Using this, we obtain:
            \begin{equation*}
                \int_{C}\mathbf{A}\cdot\boldsymbol{\diff{\ell}}
                =\int_{C}\big(
                    x^{2}\hat{\mathbf{x}}+y^{2}
                    \hat{\mathbf{y}}+z^{2}\hat{\mathbf{z}}
                \big)
                \cdot\big(
                     \diff{x}\hat{\mathbf{x}}
                    +\diff{y}\hat{\mathbf{y}}
                    +\diff{z}\hat{\mathbf{z}}
                \big)
                =\int_{c}\big(x^{2}\diff{x}+y^{2}\diff{y}\big)
            \end{equation*}
            Along the path of integration, we have $x=y^{2}$,
            and therefore $\diff{x}=2y\diff{y}$.
            Substituting this back in:
            \begin{align*}
                \int_{C}\mathbf{A}\cdot\boldsymbol{\diff{\ell}}
                &=\int_{C}\big(x^{2}\diff{x}+y^{2}\diff{y}\big)
                &
                &=\bigg[
                     \frac{1}{3}y^{6}
                    +\frac{1}{3}y^{3}
                \bigg]_{0}^{\sqrt{2}}\\
                &=\int_{0}^{\sqrt{2}}
                  \big((y^{2})^{2}(2y\diff{y})+y^{2}\diff{y}\big)
                &
                &=\frac{1}{3}\big(
                    (\sqrt{2})^{6}+(\sqrt{2})^{3}
                \big)\\
                &=\int_{0}^{\sqrt{2}}
                  \big(2y^{5}+y^{2}\big)\diff{y}
                &
                &=\frac{2}{3}\big(4+\sqrt{2}\big)\\
            \end{align*}
        \end{proof}
        \begin{figure}[H]
            \centering
            \captionsetup{type=figure}
            \begin{subfigure}[b]{0.49\textwidth}
                \centering
                \captionsetup{type=figure}
                \subimport{../../../../tikz/}{Wangsness_1_11}
                \caption{Path of Integration for Wangsness 1-11}
                \label{%
                    fig:EMAG_1_path_of_integration_%
                    for_wangsness_1_11%
                }
            \end{subfigure}
            \begin{subfigure}[b]{0.49\textwidth}
                \centering
                \captionsetup{type=figure}
                \subimport{../../../../tikz/}{Wangsness_1_12}
                \caption{Geometry for Wangsness 1-12}
                \label{fig:EMAG_1_geometry_for_wangsness_1_12}
            \end{subfigure}
            \caption[Figures for Wangsness 1-11 and 1-12]{%
                Figures for Problems
                \ref{problem:EMAG_1_Wangsness_1_11} and
                \ref{problem:EMAG_1_wangsness_1_12}, Respectively.
            }
            \label{fig:EMAG_1_figures_for_wangsness_1_11_and_1_12}
        \end{figure}
        \begin{problem}[Wangsness 1-12]
            \label{problem:EMAG_1_wangsness_1_12}
            Find the surface integral of $\mathbf{r}$ and the
            volume integral of $\nabla\cdot\mathbf{r}$
            for a sphere of radius $a_{0}$
            centered at the origin.
        \end{problem}
        \begin{proof}[Solution]
            The \textit{surface integral} of $\mathbf{A}$
            over a closed surface
            $\partial\Sigma$ is defined as:
            \begin{equation*}
                \oiint_{\partial\Sigma}
                \mathbf{A}\cdot\boldsymbol{\diff{a}}
                =\oiint_{\partial\Sigma}\mathbf{A}
                \cdot\hat{\boldsymbol{n}}\diff{a}
            \end{equation*}
            Where $\hat{\mathbf{n}}$ is the unit normal
            to the surface $\partial\Sigma$.
            For a sphere, we have:
            \begin{equation*}
                \hat{\mathbf{n}}
                =\frac{\nabla(u)}{\norm{\nabla(u)}}
                =\frac{2x\hat{\mathbf{x}}
                +2y\hat{\mathbf{y}}
                +2z\hat{\mathbf{z}}}
                {\sqrt{4x^{2}+4y^{2}+4z^{2}}}
                =\frac{x\hat{\mathbf{x}}
                +y\hat{\mathbf{y}}
                +z\hat{\mathbf{z}}}
                {\sqrt{x^{2}+y^{2}+z^{2}}}
            \end{equation*}
            Thus, we have:
            \begin{equation*}
                \oiint_{\partial\Sigma}
                \mathbf{r}\cdot\hat{\mathbf{n}}\diff{a}
                =\oiint_{\partial\Sigma}
                \bigg(
                     x\hat{\mathbf{x}}
                    +y\hat{\mathbf{y}}
                    +z\hat{\mathbf{z}}
                \bigg)\cdot\bigg(
                    \frac{%
                         x\hat{\mathbf{x}}
                        +y\hat{\mathbf{y}}
                        +z\hat{\mathbf{z}}}
                        {\sqrt{x^{2}+y^{2}+z^{2}}}
                    \bigg)\diff{a}
                =\oiint_{\partial\Sigma}
                \sqrt{x^{2}+y^{2}+z^{2}}\diff{a}
            \end{equation*}
            But recall that $x^{2}+y^{2}+z^{2}=a_{0}^{2}$,
            so we have:
            \begin{equation*}
                \oiint_{\partial\Sigma}\mathbf{r}
                \cdot\boldsymbol{\diff{a}}
                =a_{0}\oiint_{\partial\Sigma}\diff{a}\\
            \end{equation*}
            But $\oiint_{\partial\Sigma}\diff{a}$
            is just the surface area of $\partial\Sigma$.
            And the surface area of the sphere
            is $4\pi{a_{0}^{2}}$. So:
            \begin{equation*}
                \oiint_{\partial\Sigma}
                \mathbf{r}\cdot\boldsymbol{\diff{a}}
                =4\pi{a_{0}^{3}}
            \end{equation*}
            Using spherical coordinates is much easier.
            \begin{equation*}
                \oiint_{\partial\Sigma}
                \mathbf{r}\cdot\boldsymbol{\diff{a}}
                =\int_{0}^{2\pi}\int_{0}^{\pi}a_{0}
                \hat{\mathbf{r}}\cdot
                \hat{\mathbf{r}}a_{0}^{2}
                \sin(\theta)\diff{\theta}\diff{\varphi}
                =\int_{0}^{2\pi}\int_{0}^{\pi}a_{0}^{3}
                \sin(\theta)\diff{\theta}\diff{\varphi}
                =2\pi a_{0}^{3}\int_{0}^{\pi}
                \sin(\theta)\diff{\theta}=4\pi{a_{0}^{3}}
            \end{equation*}
            To compute the \textit{volume integral} of
            $\nabla\cdot\mathbf{r}$ within $\Sigma$,
            we compute $\nabla\cdot\mathbf{r}$
            and then integrate:
            \begin{align*}
                \nabla\cdot\mathbf{r}
                &=\frac{\partial{x}}{\partial{x}}
                +\frac{\partial{y}}{\partial{y}}
                +\frac{\partial{z}}{\partial{z}}=3\\
                \iiint_{\Sigma}
                \nabla\cdot\mathbf{r}\diff{\tau}
                &=\iiint_{\Sigma}3\diff{\tau}
                 =3\iiint_{\Sigma}\diff{\tau}
                 =3\frac{4}{3}\pi{a_{0}^{3}}
                 =4\pi{a_{0}^{3}}
            \end{align*}
        \end{proof} 
        \begin{problem}[Wangsness 1-13]
            \label{problem:EMAG_1_wangsness_1_13}
            Given the vector field
            $\mathbf{A}=%
             xy\hat{\mathbf{x}}%
             +yz\hat{\mathbf{y}}%
             +xz\hat{\mathbf{z}}$,
            evaluate the flux of $\mathbf{A}$
            through a parallelepiped of sides $a,b,c$
            shown in Fig.~\subref{%
                fig:EMAG_1_wangsness_1_13_%
                region_of_integration%
            }.
            Compute $\int\nabla\cdot\mathbf{A}\diff{\tau}$
            over the volume.
        \end{problem}
        \begin{proof}[Solution]
            There are six sides we must integrate over. Given
            $\mathbf{A}%
             =xy\hat{\mathbf{x}}%
             +yz\hat{\mathbf{y}}%
             +xz\hat{\mathbf{z}}$,
            we have:
            \begin{align*}
                \oiint_{\partial\Sigma}
                \mathbf{A}\cdot\boldsymbol{\diff{a}}
                &=\oiint_{\partial\Sigma}
                (xy\diff{y}\diff{z}
                +yz\diff{x}\diff{z}
                +xz\diff{x}\diff{z})\\
                &=\underset{\textrm{Front}}{\iint}xy\diff{y}\diff{z}
                -\underset{\textrm{Back}}{\iint}xy\diff{y}\diff{z}
                +\underset{\textrm{Right}}{\iint}yz\diff{x}\diff{z}
                -\underset{\textrm{Left}}{\iint}yz\diff{x}\diff{z}
                +\underset{\textrm{Top}}{\iint}xz\diff{x}\diff{y}
                -\underset{\textrm{Bottom}}{\iint}
                xz\diff{x}\diff{y}\\
                &=\int_{0}^{c}\int_{0}^{b}(a)y\diff{y}\diff{z}
                +\int_{0}^{c}\int_{0}^{a}(b)z\diff{x}\diff{z}
                +\int_{0}^{b}\int_{0}^{a}x(c)\diff{x}\diff{y}
                =\frac{abc}{2}(a+b+c)
            \end{align*}
            To compute
            $\iiint_{V}\nabla\cdot\mathbf{A}\diff{\tau}$,
            we have:
            $\nabla\cdot\mathbf{A}%
             =\frac{\partial(xy)}{\partial x}%
             +\frac{\partial(yz)}{\partial y}%
             +\frac{\partial(xz)}{\partial z}=x+y+z$.
            Thus:
            \begin{align*}
                \iiint_{\Sigma}\nabla\cdot\mathbf{A}\diff{\tau}
                &=\iiint_{\Sigma}(x+y+z)\diff{\tau}
                =\int_{0}^{c}\int_{0}^{b}\int_{0}^{a}
                (x+y+z)\diff{x}\diff{y}\diff{z}\\
                &=\int_{0}^{c}\int_{0}^{b}\int_{0}^{a}
                x\diff{x}\diff{y}\diff{z}
                +\int_{0}^{c}\int_{0}^{b}\int_{0}^{a}
                y\diff{x}\diff{y}\diff{z}
                +\int_{0}^{c}\int_{0}^{b}\int_{0}^{a}
                z\diff{x}\diff{y}\diff{z}\\
                &=\frac{a^{2}bc}{2}+\frac{ab^{2}c}{2}
                +\frac{abc^{2}}{2}=\frac{abc}{2}(a+b+c)
            \end{align*}
        \end{proof}
        \begin{figure}[H]
            \centering
            \captionsetup{type=figure}
            \begin{subfigure}[b]{0.49\textwidth}
                \centering
                \captionsetup{type=figure}
                \subimport{../../../../tikz/}{Wangsness_1_13}
            \caption{Wangsness 1-13}
            \label{fig:EMAG_1_wangsness_1_13_region_of_integration}
            \end{subfigure}
            \begin{subfigure}[b]{0.49\textwidth}
                \centering
                \captionsetup{type=figure}
                \subimport{../../../../tikz/}{Wangsness_1_14}
                \caption{Wangsness 1-14}
                \label{fig:EMAG_1_wangsness_1_14}
            \end{subfigure}
            \caption[Figures for Wangsness 1-13 and 1-14]
            {Figures for problems
            \ref{problem:EMAG_1_wangsness_1_13} and
            \ref{problem:EMAG_1_wangsness_1_14}, Respectively.}
        \end{figure}
        \begin{problem}[Wangsness 1-14]
            \label{problem:EMAG_1_wangsness_1_14}
            Given $\mathbf{A}=-y\hat{\mathbf{x}}+x\hat{\mathbf{y}}$,
            calculate the line integral
            $\oint\mathbf{A}\cdot\boldsymbol{\diff{\ell}}$
            over the closed path in the $xy$
            plane shown in
            Fig.~\subref{fig:EMAG_1_wangsness_1_14}.
        \end{problem}
        \begin{proof}[Solution]
            Given
            $\mathbf{A}=-y\hat{\mathbf{x}}+x\hat{\mathbf{y}}$,
            we have:
            \begin{align*}
                \oint_{\partial S}
                \mathbf{A}\cdot\boldsymbol{\diff{\ell}}
                &=\oint_{\partial{S}}\big(-y\hat{\mathbf{x}}
                +x\hat{\mathbf{y}}\big)
                \cdot\big(\diff{x}\hat{\mathbf{x}}
                +\diff{y}\hat{\mathbf{y}}\big)\\
                &=\underbrace{%
                    \int_{0}^{3}(-y\diff{x}+x\diff{y})
                }_{y=0,\ \diff{y}=0}
                +\underbrace{%
                    \int_{0}^{4}(-y\diff{x}+x\diff{y})
                }_{x=3,\ \diff{x}=0}
                +\underbrace{%
                    \int_{3}^{0}(-y\diff{x}+x\diff{y})
                }_{y=4,\ \diff{y}=0}
                +\underbrace{%
                    \int_{4}^{0}(-y\diff{x}+x\diff{y})
                }_{x=0,\ \diff{x}=0}\\
                &=0+12+12+0=24
            \end{align*}
            Next, we compute
            $\iint(\nabla\times\mathbf{A})%
             \cdot\boldsymbol{\diff{a}}$.
            We have
            $\nabla\times\mathbf{A}=%
             2\hat{\mathbf{z}}$.
            Thus:
            \begin{equation*}
                \iint_{S}\big(\nabla\times\mathbf{A}\big)
                \cdot\boldsymbol{\diff{a}}
                =\iint_{S}\big(2\hat{\mathbf{z}}\big)
                \cdot\big(\diff{y}\diff{z}\hat{\mathbf{x}}
                +\diff{x}\diff{z}\hat{\mathbf{y}}
                +\diff{x}\diff{y}\hat{\mathbf{z}}\big)
                =\int_{0}^{4}\int_{0}^{3}2\diff{y}\diff{x}
                =24
            \end{equation*}
        \end{proof}
        \begin{problem}[Wangsness 1-15]
            \label{problem:EMAG_1_wangsness_1_15}
            Given
            $\mathbf{A}%
             =x^{2}y\hat{\mathbf{x}}%
             +xy^{2}\hat{\mathbf{y}}%
             +a^{3}e^{-\beta{y}}\cos(\alpha{x})\hat{\mathbf{z}}$,
            compute
            $\oint\mathbf{A}\cdot\boldsymbol{\diff{\ell}}$
            along the path in Fig.~\ref{fig:EMAG_1_wangsness_1_15}.
            Compute
            $\iint(\nabla\times\mathbf{A})%
             \cdot\boldsymbol{\diff{a}}$
            over the same region.
        \end{problem}
        \begin{proof}[Solution]
            Along the entire contour, we have $z=0$ and $dz=0$.
            Thus, we have:
            \begin{equation*}
                \oint_{\partial S}
                \mathbf{A}\cdot\boldsymbol{\diff{\ell}}
                =\oint_{\partial S}
                \big(
                     x^{2}y\hat{\mathbf{x}}
                    +xy^{2}\hat{\mathbf{y}}
                    +a^{3}e^{\beta{y}}\cos(\alpha{x})\hat{\mathbf{z}}
                \big)
                \cdot\big(
                     \diff{x}\hat{\mathbf{x}}
                    +\diff{y}\hat{\mathbf{y}}
                \big)
                =\oint_{\partial S}
                \big(x^{2}y\diff{x}+xy^{2}\diff{y}\big)
            \end{equation*}
            Along the first path we have $x=0$ and $dx=0$.
            Along the second path, we
            have $y=\sqrt{2k}$ and thus $\diff{y}=0$.
            Along the third path we have $y^{2}=kx$,
            and therefore $\diff{x}=2y\diff{y}/k$. So:
            \begin{align*}
                \oint_{\partial S}
                \mathbf{A}\cdot\boldsymbol{\diff{\ell}}
                &=\int_{C_{1}}\mathbf{A}
                \cdot\boldsymbol{\diff{\ell}}
                +\int_{C_{2}}\mathbf{A}
                \cdot\boldsymbol{\diff{\ell}}
                +\int_{C_{3}}
                \mathbf{A}\cdot\boldsymbol{\diff{\ell}}\\
                &=\int_{0}^{\sqrt{2k}}(0)y^{2}\diff{y}
                 +\int_{0}^{2}x^{2}\sqrt{2k}\diff{x}
                 +\int_{\sqrt{2k}}^{0}\big(
                    \frac{y^{4}}{k^2}y\frac{2y}{k}\diff{y}
                    +\frac{y^{2}}{k}y^{2}\diff{y}
                \big)\\
                &=\frac{8}{3}\sqrt{2k}
                 +\int_{\sqrt{2k}}^{0}\big(2\frac{y^{6}}{k^{3}}
                 +\frac{y^{4}}{k}\big)\diff{y}
                =\frac{8}{3}\sqrt{2k}
                -\frac{16}{7}\sqrt{2k}
                -\frac{4k\sqrt{2k}}{5}
                =\sqrt{2k}\big(\frac{8}{21}-\frac{4}{5}k\big)
            \end{align*}
            Next we compute
            $\iint(\nabla\times\mathbf{A})%
             \cdot\boldsymbol{\diff{a}}$.
            Note that
            $\boldsymbol{\diff{a}}%
             =\hat{\mathbf{z}}\diff{x}\diff{y}$.
            The $\hat{\mathbf{z}}$
            component for $\nabla\times\mathbf{A}$ is
            $(y^{2}-x^{2})\hat{\mathbf{z}}$.
            We have:
            \begin{align*}
                \iint_{\Sigma}\big(\nabla\times\mathbf{A}\big)
                \cdot\boldsymbol{\diff{a}}
                &=\int_{0}^{2}\int_{\sqrt{kx}}^{\sqrt{2k}}
                \big(x^{2}-y^{2}\big)\diff{y}\diff{x}
                =\int_{0}^{2}\bigg[
                    x^{2}y-\frac{y^{3}}{3}
                \bigg]_{\sqrt{kx}}^{\sqrt{2k}}\diff{x}\\
                &=\int_{0}^{2}\bigg[
                    \bigg(
                        x^{2}\sqrt{2k}-\frac{2k\sqrt{2k}}{3}
                    \bigg)
                    -\bigg(
                        x^{2}\sqrt{kx}-\frac{kx\sqrt{kx}}{3}
                    \bigg)
                \bigg]dx\\
                &=\sqrt{2k}\int_{0}^{2}x^{2}\diff{x}
                 -\frac{2k}{3}\sqrt{2k}
                \int_{0}^{2}\diff{x}
                -\sqrt{k}\int_{0}^{2}x^{\frac{5}{2}}\diff{x}
                +\frac{k\sqrt{k}}{3}
                 \int_{0}^{2}x^{\frac{3}{2}}\diff{x}\\
                &=\frac{8}{3}\sqrt{2k}
                 -\frac{4k}{3}\sqrt{2k}-\frac{16}{7}\sqrt{2k}
                 +\frac{8k}{15}\sqrt{2k}
                 =\frac{8}{21}\sqrt{2k}-\frac{4}{5}k\sqrt{2k}
                 =\sqrt{2k}\big(\frac{8}{21}-\frac{4k}{5}\big)
            \end{align*}
        \end{proof}
        \begin{figure}[H]
            \centering
            \captionsetup{type=figure}
            \subimport{../../../../tikz/}{Wangsness_1_15}
            \caption[Figure for Wangsness 1-15]
            {Figure for problem \ref{problem:EMAG_1_wangsness_1_15}}
            \label{fig:EMAG_1_wangsness_1_15}
        \end{figure}
\end{document}