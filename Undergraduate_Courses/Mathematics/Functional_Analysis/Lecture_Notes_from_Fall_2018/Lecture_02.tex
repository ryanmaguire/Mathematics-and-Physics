\documentclass[crop=false,class=article,oneside]{standalone}
%----------------------------Preamble-------------------------------%
%---------------------------Packages----------------------------%
\usepackage{geometry}
\geometry{b5paper, margin=1.0in}
\usepackage[T1]{fontenc}
\usepackage{graphicx, float}            % Graphics/Images.
\usepackage{natbib}                     % For bibliographies.
\bibliographystyle{agsm}                % Bibliography style.
\usepackage[french, english]{babel}     % Language typesetting.
\usepackage[dvipsnames]{xcolor}         % Color names.
\usepackage{listings}                   % Verbatim-Like Tools.
\usepackage{mathtools, esint, mathrsfs} % amsmath and integrals.
\usepackage{amsthm, amsfonts, amssymb}  % Fonts and theorems.
\usepackage{tcolorbox}                  % Frames around theorems.
\usepackage{upgreek}                    % Non-Italic Greek.
\usepackage{fmtcount, etoolbox}         % For the \book{} command.
\usepackage[newparttoc]{titlesec}       % Formatting chapter, etc.
\usepackage{titletoc}                   % Allows \book in toc.
\usepackage[nottoc]{tocbibind}          % Bibliography in toc.
\usepackage[titles]{tocloft}            % ToC formatting.
\usepackage{pgfplots, tikz}             % Drawing/graphing tools.
\usepackage{imakeidx}                   % Used for index.
\usetikzlibrary{
    calc,                   % Calculating right angles and more.
    angles,                 % Drawing angles within triangles.
    arrows.meta,            % Latex and Stealth arrows.
    quotes,                 % Adding labels to angles.
    positioning,            % Relative positioning of nodes.
    decorations.markings,   % Adding arrows in the middle of a line.
    patterns,
    arrows
}                                       % Libraries for tikz.
\pgfplotsset{compat=1.9}                % Version of pgfplots.
\usepackage[font=scriptsize,
            labelformat=simple,
            labelsep=colon]{subcaption} % Subfigure captions.
\usepackage[font={scriptsize},
            hypcap=true,
            labelsep=colon]{caption}    % Figure captions.
\usepackage[pdftex,
            pdfauthor={Ryan Maguire},
            pdftitle={Mathematics and Physics},
            pdfsubject={Mathematics, Physics, Science},
            pdfkeywords={Mathematics, Physics, Computer Science, Biology},
            pdfproducer={LaTeX},
            pdfcreator={pdflatex}]{hyperref}
\hypersetup{
    colorlinks=true,
    linkcolor=blue,
    filecolor=magenta,
    urlcolor=Cerulean,
    citecolor=SkyBlue
}                           % Colors for hyperref.
\usepackage[toc,acronym,nogroupskip,nopostdot]{glossaries}
\usepackage{glossary-mcols}
%------------------------Theorem Styles-------------------------%
\theoremstyle{plain}
\newtheorem{theorem}{Theorem}[section]

% Define theorem style for default spacing and normal font.
\newtheoremstyle{normal}
    {\topsep}               % Amount of space above the theorem.
    {\topsep}               % Amount of space below the theorem.
    {}                      % Font used for body of theorem.
    {}                      % Measure of space to indent.
    {\bfseries}             % Font of the header of the theorem.
    {}                      % Punctuation between head and body.
    {.5em}                  % Space after theorem head.
    {}

% Italic header environment.
\newtheoremstyle{thmit}{\topsep}{\topsep}{}{}{\itshape}{}{0.5em}{}

% Define environments with italic headers.
\theoremstyle{thmit}
\newtheorem*{solution}{Solution}

% Define default environments.
\theoremstyle{normal}
\newtheorem{example}{Example}[section]
\newtheorem{definition}{Definition}[section]
\newtheorem{problem}{Problem}[section]

% Define framed environment.
\tcbuselibrary{most}
\newtcbtheorem[use counter*=theorem]{ftheorem}{Theorem}{%
    before=\par\vspace{2ex},
    boxsep=0.5\topsep,
    after=\par\vspace{2ex},
    colback=green!5,
    colframe=green!35!black,
    fonttitle=\bfseries\upshape%
}{thm}

\newtcbtheorem[auto counter, number within=section]{faxiom}{Axiom}{%
    before=\par\vspace{2ex},
    boxsep=0.5\topsep,
    after=\par\vspace{2ex},
    colback=Apricot!5,
    colframe=Apricot!35!black,
    fonttitle=\bfseries\upshape%
}{ax}

\newtcbtheorem[use counter*=definition]{fdefinition}{Definition}{%
    before=\par\vspace{2ex},
    boxsep=0.5\topsep,
    after=\par\vspace{2ex},
    colback=blue!5!white,
    colframe=blue!75!black,
    fonttitle=\bfseries\upshape%
}{def}

\newtcbtheorem[use counter*=example]{fexample}{Example}{%
    before=\par\vspace{2ex},
    boxsep=0.5\topsep,
    after=\par\vspace{2ex},
    colback=red!5!white,
    colframe=red!75!black,
    fonttitle=\bfseries\upshape%
}{ex}

\newtcbtheorem[auto counter, number within=section]{fnotation}{Notation}{%
    before=\par\vspace{2ex},
    boxsep=0.5\topsep,
    after=\par\vspace{2ex},
    colback=SeaGreen!5!white,
    colframe=SeaGreen!75!black,
    fonttitle=\bfseries\upshape%
}{not}

\newtcbtheorem[use counter*=remark]{fremark}{Remark}{%
    fonttitle=\bfseries\upshape,
    colback=Goldenrod!5!white,
    colframe=Goldenrod!75!black}{ex}

\newenvironment{bproof}{\textit{Proof.}}{\hfill$\square$}
\tcolorboxenvironment{bproof}{%
    blanker,
    breakable,
    left=3mm,
    before skip=5pt,
    after skip=10pt,
    borderline west={0.6mm}{0pt}{green!80!black}
}

\AtEndEnvironment{lexample}{$\hfill\textcolor{red}{\blacksquare}$}
\newtcbtheorem[use counter*=example]{lexample}{Example}{%
    empty,
    title={Example~\theexample},
    boxed title style={%
        empty,
        size=minimal,
        toprule=2pt,
        top=0.5\topsep,
    },
    coltitle=red,
    fonttitle=\bfseries,
    parbox=false,
    boxsep=0pt,
    before=\par\vspace{2ex},
    left=0pt,
    right=0pt,
    top=3ex,
    bottom=1ex,
    before=\par\vspace{2ex},
    after=\par\vspace{2ex},
    breakable,
    pad at break*=0mm,
    vfill before first,
    overlay unbroken={%
        \draw[red, line width=2pt]
            ([yshift=-1.2ex]title.south-|frame.west) to
            ([yshift=-1.2ex]title.south-|frame.east);
        },
    overlay first={%
        \draw[red, line width=2pt]
            ([yshift=-1.2ex]title.south-|frame.west) to
            ([yshift=-1.2ex]title.south-|frame.east);
    },
}{ex}

\AtEndEnvironment{ldefinition}{$\hfill\textcolor{Blue}{\blacksquare}$}
\newtcbtheorem[use counter*=definition]{ldefinition}{Definition}{%
    empty,
    title={Definition~\thedefinition:~{#1}},
    boxed title style={%
        empty,
        size=minimal,
        toprule=2pt,
        top=0.5\topsep,
    },
    coltitle=Blue,
    fonttitle=\bfseries,
    parbox=false,
    boxsep=0pt,
    before=\par\vspace{2ex},
    left=0pt,
    right=0pt,
    top=3ex,
    bottom=0pt,
    before=\par\vspace{2ex},
    after=\par\vspace{1ex},
    breakable,
    pad at break*=0mm,
    vfill before first,
    overlay unbroken={%
        \draw[Blue, line width=2pt]
            ([yshift=-1.2ex]title.south-|frame.west) to
            ([yshift=-1.2ex]title.south-|frame.east);
        },
    overlay first={%
        \draw[Blue, line width=2pt]
            ([yshift=-1.2ex]title.south-|frame.west) to
            ([yshift=-1.2ex]title.south-|frame.east);
    },
}{def}

\AtEndEnvironment{ltheorem}{$\hfill\textcolor{Green}{\blacksquare}$}
\newtcbtheorem[use counter*=theorem]{ltheorem}{Theorem}{%
    empty,
    title={Theorem~\thetheorem:~{#1}},
    boxed title style={%
        empty,
        size=minimal,
        toprule=2pt,
        top=0.5\topsep,
    },
    coltitle=Green,
    fonttitle=\bfseries,
    parbox=false,
    boxsep=0pt,
    before=\par\vspace{2ex},
    left=0pt,
    right=0pt,
    top=3ex,
    bottom=-1.5ex,
    breakable,
    pad at break*=0mm,
    vfill before first,
    overlay unbroken={%
        \draw[Green, line width=2pt]
            ([yshift=-1.2ex]title.south-|frame.west) to
            ([yshift=-1.2ex]title.south-|frame.east);},
    overlay first={%
        \draw[Green, line width=2pt]
            ([yshift=-1.2ex]title.south-|frame.west) to
            ([yshift=-1.2ex]title.south-|frame.east);
    }
}{thm}

%--------------------Declared Math Operators--------------------%
\DeclareMathOperator{\adjoint}{adj}         % Adjoint.
\DeclareMathOperator{\Card}{Card}           % Cardinality.
\DeclareMathOperator{\curl}{curl}           % Curl.
\DeclareMathOperator{\diam}{diam}           % Diameter.
\DeclareMathOperator{\dist}{dist}           % Distance.
\DeclareMathOperator{\Div}{div}             % Divergence.
\DeclareMathOperator{\Erf}{Erf}             % Error Function.
\DeclareMathOperator{\Erfc}{Erfc}           % Complementary Error Function.
\DeclareMathOperator{\Ext}{Ext}             % Exterior.
\DeclareMathOperator{\GCD}{GCD}             % Greatest common denominator.
\DeclareMathOperator{\grad}{grad}           % Gradient
\DeclareMathOperator{\Ima}{Im}              % Image.
\DeclareMathOperator{\Int}{Int}             % Interior.
\DeclareMathOperator{\LC}{LC}               % Leading coefficient.
\DeclareMathOperator{\LCM}{LCM}             % Least common multiple.
\DeclareMathOperator{\LM}{LM}               % Leading monomial.
\DeclareMathOperator{\LT}{LT}               % Leading term.
\DeclareMathOperator{\Mod}{mod}             % Modulus.
\DeclareMathOperator{\Mon}{Mon}             % Monomial.
\DeclareMathOperator{\multideg}{mutlideg}   % Multi-Degree (Graphs).
\DeclareMathOperator{\nul}{nul}             % Null space of operator.
\DeclareMathOperator{\Ord}{Ord}             % Ordinal of ordered set.
\DeclareMathOperator{\Prin}{Prin}           % Principal value.
\DeclareMathOperator{\proj}{proj}           % Projection.
\DeclareMathOperator{\Refl}{Refl}           % Reflection operator.
\DeclareMathOperator{\rk}{rk}               % Rank of operator.
\DeclareMathOperator{\sgn}{sgn}             % Sign of a number.
\DeclareMathOperator{\sinc}{sinc}           % Sinc function.
\DeclareMathOperator{\Span}{Span}           % Span of a set.
\DeclareMathOperator{\Spec}{Spec}           % Spectrum.
\DeclareMathOperator{\supp}{supp}           % Support
\DeclareMathOperator{\Tr}{Tr}               % Trace of matrix.
%--------------------Declared Math Symbols--------------------%
\DeclareMathSymbol{\minus}{\mathbin}{AMSa}{"39} % Unary minus sign.
%------------------------New Commands---------------------------%
\DeclarePairedDelimiter\norm{\lVert}{\rVert}
\DeclarePairedDelimiter\ceil{\lceil}{\rceil}
\DeclarePairedDelimiter\floor{\lfloor}{\rfloor}
\newcommand*\diff{\mathop{}\!\mathrm{d}}
\newcommand*\Diff[1]{\mathop{}\!\mathrm{d^#1}}
\renewcommand*{\glstextformat}[1]{\textcolor{RoyalBlue}{#1}}
\renewcommand{\glsnamefont}[1]{\textbf{#1}}
\renewcommand\labelitemii{$\circ$}
\renewcommand\thesubfigure{%
    \arabic{chapter}.\arabic{figure}.\arabic{subfigure}}
\addto\captionsenglish{\renewcommand{\figurename}{Fig.}}
\numberwithin{equation}{section}

\renewcommand{\vector}[1]{\boldsymbol{\mathrm{#1}}}

\newcommand{\uvector}[1]{\boldsymbol{\hat{\mathrm{#1}}}}
\newcommand{\topspace}[2][]{(#2,\tau_{#1})}
\newcommand{\measurespace}[2][]{(#2,\varSigma_{#1},\mu_{#1})}
\newcommand{\measurablespace}[2][]{(#2,\varSigma_{#1})}
\newcommand{\manifold}[2][]{(#2,\tau_{#1},\mathcal{A}_{#1})}
\newcommand{\tanspace}[2]{T_{#1}{#2}}
\newcommand{\cotanspace}[2]{T_{#1}^{*}{#2}}
\newcommand{\Ckspace}[3][\mathbb{R}]{C^{#2}(#3,#1)}
\newcommand{\funcspace}[2][\mathbb{R}]{\mathcal{F}(#2,#1)}
\newcommand{\smoothvecf}[1]{\mathfrak{X}(#1)}
\newcommand{\smoothonef}[1]{\mathfrak{X}^{*}(#1)}
\newcommand{\bracket}[2]{[#1,#2]}

%------------------------Book Command---------------------------%
\makeatletter
\renewcommand\@pnumwidth{1cm}
\newcounter{book}
\renewcommand\thebook{\@Roman\c@book}
\newcommand\book{%
    \if@openright
        \cleardoublepage
    \else
        \clearpage
    \fi
    \thispagestyle{plain}%
    \if@twocolumn
        \onecolumn
        \@tempswatrue
    \else
        \@tempswafalse
    \fi
    \null\vfil
    \secdef\@book\@sbook
}
\def\@book[#1]#2{%
    \refstepcounter{book}
    \addcontentsline{toc}{book}{\bookname\ \thebook:\hspace{1em}#1}
    \markboth{}{}
    {\centering
     \interlinepenalty\@M
     \normalfont
     \huge\bfseries\bookname\nobreakspace\thebook
     \par
     \vskip 20\p@
     \Huge\bfseries#2\par}%
    \@endbook}
\def\@sbook#1{%
    {\centering
     \interlinepenalty \@M
     \normalfont
     \Huge\bfseries#1\par}%
    \@endbook}
\def\@endbook{
    \vfil\newpage
        \if@twoside
            \if@openright
                \null
                \thispagestyle{empty}%
                \newpage
            \fi
        \fi
        \if@tempswa
            \twocolumn
        \fi
}
\newcommand*\l@book[2]{%
    \ifnum\c@tocdepth >-3\relax
        \addpenalty{-\@highpenalty}%
        \addvspace{2.25em\@plus\p@}%
        \setlength\@tempdima{3em}%
        \begingroup
            \parindent\z@\rightskip\@pnumwidth
            \parfillskip -\@pnumwidth
            {
                \leavevmode
                \Large\bfseries#1\hfill\hb@xt@\@pnumwidth{\hss#2}
            }
            \par
            \nobreak
            \global\@nobreaktrue
            \everypar{\global\@nobreakfalse\everypar{}}%
        \endgroup
    \fi}
\newcommand\bookname{Book}
\renewcommand{\thebook}{\texorpdfstring{\Numberstring{book}}{book}}
\providecommand*{\toclevel@book}{-2}
\makeatother
\titleformat{\part}[display]
    {\Large\bfseries}
    {\partname\nobreakspace\thepart}
    {0mm}
    {\Huge\bfseries}
\titlecontents{part}[0pt]
    {\large\bfseries}
    {\partname\ \thecontentslabel: \quad}
    {}
    {\hfill\contentspage}
\titlecontents{chapter}[0pt]
    {\bfseries}
    {\chaptername\ \thecontentslabel:\quad}
    {}
    {\hfill\contentspage}
\newglossarystyle{longpara}{%
    \setglossarystyle{long}%
    \renewenvironment{theglossary}{%
        \begin{longtable}[l]{{p{0.25\hsize}p{0.65\hsize}}}
    }{\end{longtable}}%
    \renewcommand{\glossentry}[2]{%
        \glstarget{##1}{\glossentryname{##1}}%
        &\glossentrydesc{##1}{~##2.}
        \tabularnewline%
        \tabularnewline
    }%
}
\newglossary[not-glg]{notation}{not-gls}{not-glo}{Notation}
\newcommand*{\newnotation}[4][]{%
    \newglossaryentry{#2}{type=notation, name={\textbf{#3}, },
                          text={#4}, description={#4},#1}%
}
%--------------------------LENGTHS------------------------------%
% Spacings for the Table of Contents.
\addtolength{\cftsecnumwidth}{1ex}
\addtolength{\cftsubsecindent}{1ex}
\addtolength{\cftsubsecnumwidth}{1ex}
\addtolength{\cftfignumwidth}{1ex}
\addtolength{\cfttabnumwidth}{1ex}

% Indent and paragraph spacing.
\setlength{\parindent}{0em}
\setlength{\parskip}{0em}
%--------------------------Main Document----------------------------%
\begin{document}
    \ifx\ifmathcoursesfunctional\undefined
        \section*{Functional Analysis}
        \setcounter{section}{1}
    \fi
    \subsection{Lecture 2: September 17, 2018}
        \subsubsection{More on Continuity}
            A function $f$ is continuous at a point $x$ if
            for all $\varepsilon>0$ there exists a $\delta>0$
            such that for all $x_{0}$ such that
            $|x-x_{0}|<\delta$, we have that
            $|f(x)-f(x_{0})|<\varepsilon$. This defines
            continuity at a point. Continuity on a set
            $S$ simply means that $f$ is continuous
            at every point on the set. Uniform continuity,
            however, is a set property. You don't have
            uniform continuity at a point, you have this
            on a set. A function $f$ is uniformly continuous
            if for all $\varepsilon>0$ there is a $\delta>0$
            such that for all $x$ and $x_{0}$ such that
            $|x-x_{0}|<\delta$, it is true that
            $|f(x)-f(x_{0})|<\varepsilon$. Note that in
            the definition of continuity, the point $x$
            is the first thing that is mentioned, whereas in
            uniform continuity we really don't care about it
            until the end of the definition.
            \begin{theorem}
                If $x_{n}$ and $y_{n}$ are sequences such
                in a set $S$ such that $x_{n}-y_{n}\rightarrow{0}$,
                and if $f:{S}\rightarrow\mathbb{R}$, then
                $f$ is uniformly continuous if and only if
                $f(x_{n})-f(y_{n})\rightarrow{0}$
            \end{theorem}
            \begin{theorem}
                If $f:[a,b]\rightarrow\mathbb{R}$ is continuous,
                then $f$ is uniformly continuous.
            \end{theorem}
            \begin{proof}
                Let $x_{n}$, $y_{n}\in[a,b]$ be sequences such
                that $x_{n}-y_{n}\rightarrow{0}$. By the
                Bolzano-Weierstrass theorem there is a
                convergent subsequence $x_{k_{n}}$. Let $x$
                be the limit. But
                $y_{k_{n}}=x_{k_{n}}-(y_{k_{n}}-x_{k_{n}})$,
                and therefore $y_{k_{n}}\rightarrow{x}$.
                Let $X_{n}=x_{k_{n}}$, and $Y_{n}=y_{k_{n}}$.
                Then
                $f(X_{n})-f(Y_{n})%
                 =(f(X_{n})-f(x))-(f(Y_{n}-f(x))$.
                But from continuity, $f(x_{n})\rightarrow{f(x)}$
                and $f(y_{n})\rightarrow{f(x)}$. Therefore, $f$
                is uniformly continuous.
            \end{proof}
            A few notes on pointwise convergence, recapping
            from last week. We say that $f_{n}\rightarrow{f}$
            point-wise on a set $S$ if for all $x\in{S}$,
            $f_{n}(x)-f(x)\rightarrow{0}$. We say that
            $f_{n}\rightarrow{f}$ uniformly on $S$ if
            $\underset{x\in{S}}{\sup}%
             |f_{n}(x)-f(x)|\rightarrow{0}$.
            \begin{example}
                Let $f_{n}(x)=nxe^{-nx}$ on $[0,1]$.
                Then for $x\in[0,1]$, $f_{n}(x)\rightarrow{0}$
                and therefore $f_{n}(x)\rightarrow{0}$ poinwise
                on $[0,1]$. Note that
                $f_{n}'(x)=(n-n^{2}x)e^{-nx}$. This has a
                zero at $x=\frac{1}{n}$, so we see that
                $f_{n}(x)$ has a maximum
                which evaluates to $e^{-1}$. But
                $\sup|f_{n}(x)-f(x)|=\sup|f_{n}(x)|=e^{-1}$.
                So $f_{n}(x)$ does not
                \textit{uniformly} converge to $0$. The
                convergence is only \textit{pointwise}.
            \end{example}
            \begin{theorem}[Weierstrass Approximation Theorem]
                If $f:[a,b]\rightarrow\mathbb{R}$ is continuous,
                then there is a sequence of polynomials
                $f_{n}$ such that $f_{n}\rightarrow{f}$
                uniformly on $[a,b]$.
            \end{theorem}
            \begin{proof}
                We'll assume, for now, that $[a,b]=[0,1]$
                and that $f(0)=f(1)=0$. Extend $f$ to be
                zero outside of $[0,1]$. Let
                $Q_{n}(x)=c_{n}(1-x^{2})^{n}$ on $[-1,1]$,
                and choose $c_{n}$ such that
                $\int_{-1}^{1}Q_{n}(x)dx=1$. So we have:
                \begin{align*}
                    c_{n}\int_{-1}^{1}(1-x^{2})^{n}dx
                    &=2c_{n}\int_{0}^{1}(1-x^{2})^{n}dx
                    &
                    &\geq{2c_{n}}\int_{0}^{1}(1-x)^{n}dx\\
                    &=2c_{n}\int_{0}^{1}(1-x)^{n}(1+x)^{n}dx
                    &
                    &=\frac{2}{n+1}c_{n}
                \end{align*}
            From this we have that $c_{n}\leq{n+1}$. Let
            $f_{n}(x)=\int_{0}^{1}f(t)Q_{n}(x-t)dx$.
            Then $f_{n}(x)$ is a polynomial. Note that
            $f(t)Q_{n}(x-t)$ is roughly zero when $t$ differs
            from $x$ and $n$ is large enough. So we have:
            \begin{equation*}
                f_{n}(x)=\int_{0}^{1}f(t)Q_{n}(x-t)dt
                \approx{f(x)}\int_{0}^{1}Q_{n}(x-t)dt=f(x)
            \end{equation*}
            The remainder of the proof is to quantify this.
            Since $f$ is zero outside of $[0,1]$, if
            we let $s=t-x$, then:
            \begin{align*}
                f_{n}(x)&=\int_{-x}^{1-x}f(s+x)Q_{n}(s)ds
                =\int_{-1}^{1}f(s+x)Q_{n}(s)ds\\
                \Rightarrow|f_{n}(x)-f(x)|
                &=\bigg|\int_{-1}^{1}f(x+t)Q_{n}(t)dt
                -\int_{-1}^{1}f(x)Q_{n}(t)dt\bigg|\\
                \Rightarrow|f_{n}(x)-f(x)|
                &\leq\int_{-1}^{1}|f(x+t)-f(x)|Q_{n}(t)dt
            \end{align*}
            This comes for the fact that$\int_{-1}^{1}Q_{n}(t)=1$
            and from the integral version of the triangle
            inequaility.
            Suppose $\varepsilon>0$. Since $f$ is continuous
            on $[0,1]$, it is uniformly continuous. But
            if $f$ is uniformly continuous then there exists
            a $\delta>0$ such that
            $|f(x+t)-f(x)|<\frac{\varepsilon}{2}$ for all
            $t<\delta$. So we have:
            \begin{equation*}
                |f_{n}(x)-f(x)|\leq
                \int_{-1}^{-\delta}|f(x+t)-f(x)|Q_{n}(t)dt
                +\int_{-\delta}^{\delta}|f(x+t)-f(x)|Q_{n}(t)dt
                +\int_{\delta}^{1}|f(x+t)-f(x)|Q_{n}(t)dt
            \end{equation*}
            But $f$ is continuous on a closed and bounded
            set and therefore $f$ is bounded. Let $M$ be
            such a bound. Then $|(f(x+t)-f(x)|\leq{2M}$.
            We have:
            \begin{equation*}
                |f_{n}(x)-f(x)|\leq
                2M\int_{-1}^{-\delta}Q_{n}(t)dt
                +\frac{\varepsilon}{2}
                \int_{-\delta}^{\delta}Q_{n}(t)dt
                +2M\int_{\delta}^{1}Q_{n}(t)dt
            \end{equation*}
            But for all $t\in[-1,-\delta]$,
            $Q_{n}(t)\leq{Q_{n}(-\delta)}$. Similarly for
            $t$ in $[\delta,1]$. Since $Q_{n}(t)$
            is an even function:
            \begin{equation*}
                |f_{n}(x)-f(x)|\leq
                4MQ_{n}(\delta)+
                \frac{\varepsilon}{2}\int_{-1}^{1}Q_{n}(t)dt
                =4MQ_{n}(\delta)+\frac{\varepsilon}{2}
            \end{equation*}
            But since $\delta>0$, $Q_{n}(\delta)\rightarrow0$.
            Therefore, there is an $N\in\mathbb{N}$ such that
            for all $n>N$,
            $|Q_{n}(\delta)|<\frac{\varepsilon}{8M}$.
            But then $4MQ_{n}(\delta)<\frac{\varepsilon}{2}$.
            Therefore, etc.
        \end{proof}
            Another way to put this is that if $f$ is continuous
            on $[a,b]$ and if $\varepsilon>0$, then there is
            a polynomial $P$ such that for all $x\in[0,1]$,
            $|f(x)-P(x)|<\varepsilon$. There is a generalization
            of this. The set of functions need not be
            polynomials. The set needs to be closed
            under addition, multiplication, and scalar
            multiplication, it must separate points,
            and must not take every point to zero.
            This is the Stone-Weierstrass theorem.
        \subsubsection{Functional Analysis}
            Functional analysis is concerned with normed spaces.
            This is a vector space $V$ with a function, called
            a norm, from $V$ to $[0,\infty)$. This is usually
            written $\norm{\mathbf{x}}$ for an element
            $\mathbf{x}\in{V}$. This norm must satisfy the
            folllowing for all $\mathbf{x}$, $\mathbf{y}\in{V}$:
            \begin{enumerate}
                \item $\norm{\mathbf{x}}=0$ if and only
                      if $\mathbf{x}=\mathbf{0}$
                      \hfill[Definiteness]
                \item $\norm{c\mathbf{x}}=|c|\norm{\mathbf{x}}$
                      for all $c\in\mathbb{R}$.
                      \hfill[Positiveness]
                \item $\norm{\mathbf{x}+\mathbf{y}}%
                       \leq\norm{\mathbf{x}}+\norm{\mathbf{y}}$
                      \hfill[Triangle Inequality]
            \end{enumerate}
            \begin{example}
                The following are normed spaces:
                \begin{enumerate}
                    \begin{multicols}{2}
                        \item $\mathbb{R}$ with $\norm{x}=|x|$
                        \item $\mathbb{R}^{n}$ with
                              $\norm{\mathbf{x}}%
                               =\sqrt{\sum_{k=1}^{n}x_{k}^{2}}$
                        \item $\mathbb{R}^{n}$ with
                              $\norm{\mathbf{x}}%
                               =(\sum_{k=1}^{n}x_{k}^{p})^{1/p}$
                        \item $\ell^{p}$: sequences $x_{n}$
                              such that
                              $\sum_{k=1}^{\infty}|x_{n}|^P%
                               <\infty$.
                        \item $\mathbb{R}^{n}$ with
                              $\norm{\mathbf{x}}%
                               =\max\{x_{1},\hdots,x_{n}\}$
                        \item $\ell^{\infty}$: Bounded sequences
                              with $\norm{v}=\sup\{v_{n}\}$
                    \end{multicols}
                \end{enumerate}
            \end{example}
            From the fact that $\ell^{\infty}$ is a normed space
            we have that the set of convergent sequences,
            again with the $\norm{}_{\infty}$ norm, is also
            a normed space. The set of null sequences, which
            is the set of sequences that converge to zero,
            is also a normed space. A stranger normed space
            is the set of all bounded continuous functions
            $f:S\rightarrow\infty$ with norm
            $\norm{f}=\sup\{|f(x)|\}$. Furthermore, the
            set of all integrable functions with
            bounded integrals, with norm
            $(\int_{S}|f|^{p})^{1/p}$. If you allow integral
            to mean Lebesgue Integrable, then this becomes
            a special space denoted $L^{p}(S)$. As a final
            example, the set of functions
            $f:[a,b]\rightarrow\mathbb{R}$ such that
            $(\int_{a}^{b}\sum_{k=0}^{n}|f^{(k)}|^{p})^{(1/p)}$
            exists, denoted $W^{n,p}([a,b])$ is called
            the Sobolev space.
        \subsubsection{Metric Spaces}
            A lot of the things we wish to
            prove don't rely on the fact that all of these
            spaces are vector spaces. Really, we only care about
            the properties that the norm on the space has.
            What matters is that there's a set and a notion
            of distance on the set. This abstraction is the
            fundamental concept of a metric space.
            \begin{definition}
                A metric space is a set $S$ and a function
                $d:{X}\times{X}\rightarrow[0,\infty)$ such that:
                \begin{enumerate}
                    \item For all $x$, $y\in{X}$, $d(x,y)=0$
                          if and only if $x=y$.
                          \hfill[Definitenes]
                    \item For all $x$, $y\in{X}$,
                          $d(x,y)=d(y,x)$
                          \hfill[Symmetry]
                    \item For all $x$, $y\in{X}$,
                          $d(x,z)\leq{d(x,y)+d(y,z)}$
                          \hfill[Triangile Inequality]
                \end{enumerate}
            \end{definition}
            \begin{theorem}
                If $(V,\norm{})$ is a vector space,
                and if $d:{V}\times{V}\rightarrow[0,\infty)$
                is defined by
                $d(\mathbf{x},\mathbf{y})%
                 =\norm{\mathbf{x}-\mathbf{y}}$,
                then $(V,d)$ is a metric space.
            \end{theorem}
            \begin{proof}
                In order:
                \begin{enumerate}
                    \item If $\norm{\mathbf{x}-\mathbf{y}}=0$,
                          then $\mathbf{x}=\mathbf{y}$.
                          Similarly,
                          $\norm{\mathbf{x}-\mathbf{x}}%
                           =\norm{\mathbf{0}}=0$.
                    \item $d(\mathbf{x},\mathbf{y})%
                           =\norm{\mathbf{x}-\mathbf{y}}%
                           =\norm{(-1)(\mathbf{y}-\mathbf{x})}%
                           =|-1|\norm{\mathbf{y}-\mathbf{x}}%
                           =\norm{\mathbf{y}-\mathbf{y}}%
                           =d(y,x)$
                    \item The triangle inequality follows
                          from the triangle inequality that
                          norms have.
                \end{enumerate}
            \end{proof}
            There are metric spaces that have nothing to do
            with vector spaces or norms. Metric spaces are
            a more abstract object. Every normed space
            has an associated metric space since there
            is the ``induced'' metric.
            \begin{example}
                Let $X$ be a set and let
                $d(x,y)=\begin{cases}%
                            0,&x=y\\%
                            1,&{x}\ne{y}%
                        \end{cases}$
                This is the discrete metric on $X$.
            \end{example}
            \begin{example}
                Let $X=\{a,b,c\}$, and
                $d(a,b)=1$, $d(b,c)=2$. What value
                must $d(a,c)$ have if $d$ is a metric on $X$?
                We need $d(a,c)\leq{d(a,b)+d(b,c)}=1+2=3$.
                Also $d(a,b)\leq{d(a,c)+d(c,b)}=d(a,c)+2$, and
                therefore $-1\leq{d(a,c)}$. Finally, we need
                $d(b,c)\leq{d(b,a)+d(a,c)}=1+d(a,c)$. Together
                we have $d(a,c)\leq{3}$ and $1\leq{d(a,c)}$. So
                Pick $d(a,c)=2$.
                This makes $(X,d)$ a metric space.
            \end{example}
            \begin{example}
                Let $X=\mathbb{R}$ and $d(x,y)=|x-y|$.
                Then $(X,d)$ is a metric space.
            \end{example}
            \begin{example}
                $\mathbb{R}$ with
                $d(x,y)=|f(x)-f(y)|$, where
                $f$ is injective, is a metric space.
            \end{example}
            \begin{example}
                $\mathbb{R}$ with
                $d(x,y)=|\tan^{-1}(x)-\tan^{-1}(y)|$ is a
                metric. Moreover, $d(x,y)<\pi$ for all
                $x,y\in\mathbb{R}$. Thus, we have found
                a metric that makes $\mathbb{R}$ a bounded
                set. As a fun fact, $x_{n}=n$ is a Cauchy
                sequence in this metric space, but
                this sequence does not converge to anything.
                Thus we've found a metric on
                $\mathbb{R}$ such that
                $(\mathbb{R},d)$ is not complete.
            \end{example}
        \subsubsection{Topology}
            \begin{definition}
                The open ball of radius $r>0$
                about a point $x$ in a metric space
                $(X,d)$ is the set
                $B_{r}(x)=\{y\in{X}:d(x,y)<r\}$
            \end{definition}
            The picture for this is a ``circle'' around the
            point $x$ or radius $r$. However, this circle
            can look very strange for weird metrics.
            \begin{example}
                If $X$ is a set and $d$ is the discrete metric,
                then $B_{r}(x)$ is either the point $x$
                (If $r\leq{1}$), or it is the entire set $X$.
            \end{example}
            \begin{example}
                With $X=\mathbb{R}$ and $d$ the standard metric
                $d(x,y)=|x-y|$, we have $B_{r}(x)$ is simply
                the open interval $(x-r,x+r)$.
            \end{example}
            \begin{example}
                Let $X=\mathbb{R}^{2}$ and define
                $d_{p}(x,y)%
                 =(|x_{1}-y_{1}|^{p}+|x_{2}-y_{2}|^{p})^{1/p}$.
                For $p=2$, an open ball is a circle around
                the point $(x,y)$ of radius $r$. For $p=1$,
                we have ``diamonds'' around the point $x$.
                And for $p=\infty$ we have a square
                around $x$.
            \end{example}
            If you have a vector space and a norm on it,
            then the open balls about a point will have the
            property of convexity. Convexity is a vector space
            property, given two points the ``line'' between the
            two remains in the set. Metric spaces have no such
            notion. Since the balls of $\norm{}_{p}$ are not
            convex with $p<1$, we have that $\norm{}_{p}$ is
            a metric on $\mathbb{R}^{n}$
            if and only if $p\geq{1}$.
            \begin{definition}
                We say that a subset of $X$ is open if
                for all $x\in{S}$ there is an $r>0$
                such that $B_{r}(x)\subset{S}$.
            \end{definition}
            \begin{example}
                If $(X,d)$ is a metric space, then
                $X$ is open, $\emptyset$ is open
                (Vacuously true), and if $x\in{X}$ and
                $r>0$, then $B_{r}(x)$ is open.
            \end{example}
            \begin{theorem}
                If $(X,d)$ is a metric space, $x\in{X}$,
                and $r>0$, then $B_{r}(x)$ is an open
                subset of $X$.
            \end{theorem}
            \begin{proof}
                If $z\in{B_{r}(x)}$, let $t=d(x,z)$.
                Then $0\leq{t}<r$. Let $r'=r-t$.
                But if $y\in{B_{r'}(z)}$, then
                $d(x,y)\leq{d(x,z)+d(y,z)}<t+r'=t+r-t=r$.
                Therefore $B_{r'}(z)\subset{B_{r}(x)}$.
            \end{proof}
            \begin{theorem}
                A finite intersection of open sets is open.
            \end{theorem}
            \begin{proof}
                If $\mathcal{U}_{1},\hdots,\mathcal{U}_{n}$
                are open and if
                $x\in\cap_{k=1}^{n}\mathcal{U}_{k}$, then there
                exists $r_{1},\hdots,r_{n}$ such that
                $B_{r_{i}}(x)\subset\mathcal{U}_{i}$. Let
                $r=\min\{r_{1},\hdots,r_{n}\}$. Then
                $B_{r}(x)\subset\cap_{k=1}^{n}\mathcal{U}_{i}$
            \end{proof}
            \begin{theorem}
                Arbitrary unions of open sets are open.
            \end{theorem}
            Infinite intersections need not be open.
            The proof above would fail since the
            $r_{i}$ can form a sequence tending to zero.
            But indeed, let $X=\mathbb{R}$ and let
            $d(x,y)=|x-y|$, and take
            $\mathcal{U}_{n}=(-\frac{1}{n},\frac{1}{n})$.
            Then all of the $\mathcal{U}_{n}$ are open,
            yet the intersection, which is the set $\{0\}$,
            is not open. All of this mumbo-jumbo creates
            the more general notion of a topological space.
            \begin{definition}
                A topological space is a set $X$ and a
                subset $\tau\subset\mathcal{P}(X)$ such that:
                \begin{enumerate}
                    \item $\emptyset,X\in\tau$
                    \item Finite intersections of sets in $\tau$
                          are also sets in $\tau$.
                    \item Arbitrary unions of sets in $\tau$
                          are also sets in $\tau$.
                \end{enumerate}
            \end{definition}
            \begin{definition}
                An open subset of a topological space
                $(X,\tau)$ is a set $\mathcal{U}\in\tau$.
            \end{definition}
            Here, $\mathcal{P}(X)$ denotes the \textit{power set}
            of $X$. This is the set of all subsets of $X$.
            The notion of a topological space generalizes the
            notion of a metric space. There is no notion of
            distance in such spaces, and things can be weird.
            There are topological spaces that have no metric
            associated with them.
\end{document}