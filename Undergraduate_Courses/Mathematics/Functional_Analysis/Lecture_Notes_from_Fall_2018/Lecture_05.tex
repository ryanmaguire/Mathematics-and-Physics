\documentclass[crop=false,class=article,oneside]{standalone}
%----------------------------Preamble-------------------------------%
%---------------------------Packages----------------------------%
\usepackage{geometry}
\geometry{b5paper, margin=1.0in}
\usepackage[T1]{fontenc}
\usepackage{graphicx, float}            % Graphics/Images.
\usepackage{natbib}                     % For bibliographies.
\bibliographystyle{agsm}                % Bibliography style.
\usepackage[french, english]{babel}     % Language typesetting.
\usepackage[dvipsnames]{xcolor}         % Color names.
\usepackage{listings, lstlinebgrd}      % Verbatim-Like Tools.
\usepackage{mathtools, esint, mathrsfs} % amsmath and integrals.
\usepackage{amsthm, amsfonts}           % Fonts and theorems.
\usepackage{tabularx}
\usepackage{tcolorbox}                  % Frames around theorems.
\usepackage{upgreek}                    % Non-Italic Greek.
\usepackage{paracol}                    % Two-column styling.
\usepackage{wrapfig}                    % Wrap text around figure.
\usepackage{fmtcount, etoolbox}         % For the \book{} command.
\usepackage[newparttoc]{titlesec}       % Formatting chapter, etc.
\usepackage{titletoc}                   % Allows \book in toc.
\usepackage[nottoc]{tocbibind}          % Bibliography in toc.
\usepackage[titles]{tocloft}            % ToC formatting.
\usepackage{multicol, enumitem}         % Multi-column/enumerate.
\usepackage{import}                     % Import external files.
\usepackage{pgfplots, tikz}             % Drawing/graphing tools.
\usetikzlibrary{
    calc,                   % Calculating right angles and more.
    angles,                 % Drawing angles within triangles.
    arrows.meta,            % Latex and Stealth arrows.
    quotes,                 % Adding labels to angles.
    positioning,            % Relative positioning of nodes.
    decorations.markings,   % Adding arrows in the middle of a line.
    patterns,
    arrows,
    shapes,
    shapes.geometric,
    cd,
    hobby,
    babel
}                                       % Libraries for tikz.
\pgfplotsset{compat=1.9}                % Version of pgfplots.
\usepackage[font=scriptsize,
            labelformat=simple,
            labelsep=colon]{subcaption} % Subfigure captions.
\usepackage[font={scriptsize},
            hypcap=true,
            labelsep=colon]{caption}    % Figure captions.
\usepackage{hyperref}                   % Allows for hyperlinks.
\hypersetup{
    colorlinks=true,
    linkcolor=blue,
    filecolor=magenta,
    urlcolor=Cerulean,
    citecolor=SkyBlue
}                           % Colors for hyperref.
\usepackage[toc,acronym,nogroupskip]{glossaries} % Glossaries and acronyms.
\usepackage[subpreambles=false]{standalone}      % Complileable sub files.

% Various font stuff from kiwi.
% Use this for Times text and Computer Modern math
%\usepackage{times}

% Quite nice
%\usepackage[charter, greekfamily=, greekuppercase=italicized]{mathdesign}
%\usepackage[utopia, greekuppercase=italicized]{mathdesign}    % Math is narrower

% Use this for Times text and math
%\usepackage{newtxtext}
%\usepackage[libertine,cmintegrals]{newtxmath}
%\usepackage{fix-cm}

%\usepackage{txfontsb}
% or
%\usepackage{mathptmx}

%\usepackage[scaled=0.92]{helvet}
%\renewcommand{\rmdefault}{ptm}

%\usepackage{mathpazo}    % add possibly `sc` and `osf` options
%\usepackage{eulervm}

%\usepackage{fourier}
%\renewcommand{\rmdefault}{ptm}
%\usepackage{mathptm}

%\usepackage{fontspec}
%\setmainfont{lmodern}

%\usepackage[varg]{txfonts}
%\usepackage{fouriernc}
%\usepackage{mathpazo}

%\usepackage{bookman}
%\usepackage[scaled]{uarial}
%\usepackage[scaled]{helvet}
%\renewcommand*\familydefault{\sfdefault}
%\usepackage[math]{anttor}

%\newcommand\fgeorgia{\fontfamily{jvn}\selectfont}
%\newcommand\ftimes{\fontfamily{ptm}\selectfont}
%\newcommand\fhelvetica{\fontfamily{phv}\selectfont}
%\newcommand\fcourier{\fontfamily{pcr}\selectfont}
%\newcommand\fbookman{\fontfamily{pbk}\selectfont}
%\newcommand\fnewcentury{\fontfamily{pnc}\selectfont}
%\newcommand\fpalatino{\fontfamily{ppl}\selectfont}
%\newcommand\favantgarde{\fontfamily{pag}\selectfont}
%\newcommand\fnormal{\normalfont}
%\newcommand\fsize[1]{\ifnum#1>0\fontsize{#1}{#1}\selectfont\else\normalsize\fi}
%------------------------Theorem Styles-------------------------%
% Define theorem style for default spacing and normal font.
\newtheoremstyle{normal}
    {\topsep}               % Amount of space above the theorem.
    {\topsep}               % Amount of space below the theorem.
    {}                      % Font used for body of theorem.
    {}                      % Measure of space to indent.
    {\bfseries}             % Font of the header of the theorem.
    {}                      % Punctuation between head and body.
    {.5em}                  % Space after theorem head.
    {}

% Define theorem style for default spacing with italicized font.
\newtheoremstyle{normalit}{\topsep}{\topsep}
                {\itshape}{}{\bfseries}{}{.5em}{}

% Italic header environment.
\newtheoremstyle{thmit}{\topsep}{\topsep}{}{}{\itshape}{}{0.5em}{}

% Define italicized environments.
\theoremstyle{normalit}
\newtheorem{theorem}{Theorem}[section]
\newtheorem{lemma}{Lemma}[section]
\newtheorem{corollary}{Corollary}[section]
\newtheorem{proposition}{Proposition}[section]
\newtheorem*{theorem*}{Theorem}

% Define environments with italic headers.
\theoremstyle{thmit}
\newtheorem*{solution}{Solution}
\newtheorem*{fsolution}{Solution}

% Define default environments.
\theoremstyle{normal}
\newtheorem{example}{Example}[section]
\newtheorem{definition}{Definition}[section]
\newtheorem{problem}{Problem}[section]
\newtheorem{question}{Question}[section]
\newtheorem{remark}{Remark}[section]
\newtheorem{properties}{Properties}[section]
\newtheorem{notation}{Notation}[section]
\newtheorem{axiom}{Axiom}[section]
\newtheorem*{properties*}{Properties}
\newtheorem*{remark*}{Remark}
\newtheorem*{definition*}{Definition}
\theoremstyle{plain}

% Define framed environment.
\tcbuselibrary{most}
\newtcbtheorem[use counter*=theorem]{ftheorem}{Theorem}%
    {colback=green!5,colframe=green!35!black,
     fonttitle=\bfseries\upshape}{th}

\newtcbtheorem[use counter*=example]{fdefinition}{Definition}%
    {fonttitle=\bfseries\upshape,
     colback=blue!5!white,colframe=blue!75!black}{def}

\newtcbtheorem[use counter*=example]{fexample}{Example}%
    {fonttitle=\bfseries\upshape,
     colback=red!5!white,colframe=red!75!black}{ex}

\newtcbtheorem[use counter*=notation]{fnotation}{Notation}%
    {fonttitle=\bfseries\upshape,
     colback=SeaGreen!5!white,colframe=SeaGreen!75!black}{ex}

\newtcbtheorem[use counter*=corollary]{fcorollary}{Corollary}%
    {fonttitle=\bfseries\upshape,
     colback=Orchid!5!white,colframe=Orchid!75!black}{ex}

\newenvironment{bproof}{\textit{Proof.}}{\hfill$\square$}
\tcolorboxenvironment{bproof}{blanker,breakable,left=5mm,
                             before skip=10pt,after skip=10pt,
                             borderline west={1mm}{0pt}{red}}
\tcolorboxenvironment{fsolution}
    {enhanced jigsaw,colframe=cyan,interior hidden,breakable}

%--------------------Declared Math Operators--------------------%
\DeclareMathOperator{\Refl}{Refl}           % Reflection operator.
\DeclareMathOperator{\Span}{Span}           % Span of a set of vectors.
\DeclareMathOperator{\Card}{Card}           % Cardinality of set.
\DeclareMathOperator{\Ord}{Ord}             % Ordinal of ordered set.
\DeclareMathOperator{\Tr}{Tr}               % Trace of matrix.
\DeclareMathOperator{\adjoint}{adj}         % Adjoint of matrix.
\DeclareMathOperator{\rk}{rk}               % Rank of operator.
\DeclareMathOperator{\nul}{nul}             % Null space of operator.
\DeclareMathOperator{\sgn}{sgn}             % Sign of a number.
\DeclareMathOperator{\multideg}{mutlideg}   % Multi-Degree (Graphs).
\DeclareMathOperator{\GCD}{GCD}             % Greatest common denominator.
\DeclareMathOperator{\LM}{LM}               % Leading monomial
\DeclareMathOperator{\LC}{LC}               % Leading coefficient.
\DeclareMathOperator{\LT}{LT}               % Leading term.
\DeclareMathOperator{\LCM}{LCM}             % Least common multiple.
\DeclareMathOperator{\Mon}{Mon}             % Monomial.
\DeclareMathOperator{\Spec}{Spec}           % Spectrum.
\DeclareMathOperator{\proj}{proj}           % Projection.
\DeclareMathOperator{\comp}{comp}           % Component.
\DeclareMathOperator{\sinc}{sinc}           % Sinc function.
\DeclareMathOperator{\Ima}{Im}              % Image of operator.
\DeclareMathOperator{\Prin}{Prin}           % Principal value.
\DeclareMathOperator{\Mod}{mod}             % Modulus.
%------------------------New Commands---------------------------%
\DeclarePairedDelimiter\norm{\lVert}{\rVert}
\DeclarePairedDelimiter\ceil{\lceil}{\rceil}
\DeclarePairedDelimiter\floor{\lfloor}{\rfloor}
\newcommand*\diff{\mathop{}\!\mathrm{d}}
\newcommand*\Diff[1]{\mathop{}\!\mathrm{d^#1}}
\renewcommand{\mod}{\ \Mod}
\renewcommand*{\glstextformat}[1]{\textcolor{RoyalBlue}{#1}}
\renewcommand{\glsnamefont}[1]{\textbf{#1}}
\renewcommand\labelitemii{$\circ$}
\renewcommand\thesubfigure{\arabic{chapter}.\arabic{figure}}
\renewcommand\thesubfigure{%
    \arabic{chapter}.\arabic{figure}.\arabic{subfigure}}
\addto\captionsenglish{\renewcommand{\figurename}{Fig.}}
%------------------------Book Command---------------------------%
\makeatletter
\renewcommand\@pnumwidth{1cm}
\newcounter{book}
\renewcommand\thebook{\@Roman\c@book}
\newcommand\book{%
    \if@openright
        \cleardoublepage
    \else
        \clearpage
    \fi
    \thispagestyle{plain}%
    \if@twocolumn
        \onecolumn
        \@tempswatrue
    \else
        \@tempswafalse
    \fi
    \null\vfil
    \secdef\@book\@sbook
}
\def\@book[#1]#2{%
    \ifnum \c@secnumdepth >-3\relax
        \refstepcounter{book}%
        \addcontentsline{toc}{book}{
            \bookname\ \thebook:\hspace{1em}#1
        }
    \else
        \addcontentsline{toc}{book}{#1}%
    \fi
    \markboth{}{}%
    {\centering
     \interlinepenalty \@M
     \normalfont
     \ifnum \c@secnumdepth >-2\relax
       \huge\bfseries \bookname\nobreakspace\thebook
       \par
       \vskip 20\p@
     \fi
     \Huge \bfseries #2\par}%
    \@endbook}
\def\@sbook#1{%
    {\centering
     \interlinepenalty \@M
     \normalfont
     \Huge \bfseries #1\par}%
    \@endbook}
\def\@endbook{
    \vfil\newpage
        \if@twoside
            \if@openright
                \null
                \thispagestyle{empty}%
                \newpage
            \fi
        \fi
        \if@tempswa
            \twocolumn
        \fi
}
\newcommand*\l@book[2]{%
    \ifnum \c@tocdepth >-2\relax
        \addpenalty{-\@highpenalty}%
        \addvspace{2.25em \@plus\p@}%
        \setlength\@tempdima{3em}%
        \begingroup
            \parindent \z@ \rightskip \@pnumwidth
            \parfillskip -\@pnumwidth
            {
                \leavevmode
                \Large \bfseries #1\hfil \hb@xt@\@pnumwidth{
                    \hss #2
                }
            }
            \par
            \nobreak
            \global\@nobreaktrue
            \everypar{\global\@nobreakfalse\everypar{}}%
        \endgroup
    \fi}
\newcommand\bookname{Book}
\renewcommand{\thebook}{\texorpdfstring{\Numberstring{book}}{book}}
\providecommand*{\toclevel@book}{-2}
\makeatother
\titlecontents{chapter}[0pt]
    {\bfseries}
    {\chaptername\ \thecontentslabel:\quad}
    {}
    {\hfill\contentspage}
\titleformat{\part}[display]
    {\Large\bfseries}
    {\partname\nobreakspace\thepart}
    {0mm}
    {\Huge\bfseries}
    \titlecontents{part}[0pt]
    {\large\bfseries}
    {\partname\ \thecontentslabel: \quad}
    {}
    {\hfill\contentspage}
\newcommand{\MarkRightAngle}[4][.3cm]
    {\coordinate (tempa) at ($(#3)!#1!(#2)$);
     \coordinate (tempb) at ($(#3)!#1!(#4)$);
     \coordinate (tempc) at ($(tempa)!0.5!(tempb)$);%midpoint
     \draw (tempa) -- ($(#3)!2!(tempc)$) -- (tempb);}
%--------------------------LENGTHS------------------------------%
% Spacings for the Table of Contents.
\addtolength{\cftsecnumwidth}{1ex}
\addtolength{\cftsubsecindent}{1ex}
\addtolength{\cftsubsecnumwidth}{1ex}
\addtolength{\cftfignumwidth}{1ex}
\addtolength{\cfttabnumwidth}{1ex}

% Spacing for multi-column and enumerate environments.
\setlength{\multicolsep}{6pt}
\setlist[enumerate]{itemsep=0pt,topsep=3pt}

% Indent and paragraph spacing.
\setlength{\parindent}{0em}
\setlength{\parskip}{0em}
%--------------------------Main Document----------------------------%
\begin{document}
    \ifx\ifmathcoursesfunctional\undefined
        \section*{Functional Analysis}
        \setcounter{section}{1}
    \fi
    \subsection{Lecture 5: October 11, 2018}
        \subsubsection{Banach Fixed Point Theorem}
            If $(X,d)$ is a complete metric space,
            and if $T:X\rightarrow{X}$ satisfies
            the property that, for all $x$ and $y$
            in $X$, $d(T(x),T(y))<kd(x,y)$ for
            some $k<1$, then $T$ has a unique
            point $x$, called a fixed point,
            such that $T(x)=x$.
            \begin{definition}
                A contraction of a metric
                space $(X,d)$ is a function
                $T:{X}\rightarrow{X}$ such that there
                exists a $k\in(0,1)$ such that
                for all $x,y\in{X}$,
                $d(T(x),T(y))<kd(x,y)$.
            \end{definition}
            \begin{definition}
                A fixed point of a function
                $f:X\rightarrow{X}$ is a point
                $x\in{X}$ such that
                $f(x)=x$.
            \end{definition}
            \begin{theorem}[%
                Banach's Fixed Point Theorem%
            ]
                If $(X,d)$ is a complete
                metric space and $T:X\rightarrow{X}$
                is a contraction, then there is
                a unique fixed point $x\in{X}$
                with respect to $T$.
            \end{theorem}
            \begin{definition}
                A Lipschitz continuous function is a
                function $f:[a,b]\rightarrow\mathbb{R}$
                such that there is an $L\in\mathbb{R}$
                such that
                $|f(x)-f(y)|<L|x-y|$ for all
                $x,y\in[a,b]$.
            \end{definition}
            This says that the slopes of the
            secant lines of the
            function are bounded. The square root
            function $y=\sqrt{x}$ is an example
            of a function that is not Lipschitz. The
            slopes of secant lines go to infinity
            as the points tend towards the origin.
            \begin{theorem}[Picard's Theorem]
                If $f:[a,b]\times\mathbb{R}%
                    \rightarrow\mathbb{R}$
                is Lipschitz continuous,
                Then there is a unique function
                $x:[a,b]\rightarrow\mathbb{R}$
                such that
                $\frac{dx}{dt}=f(t,x(t))$ and $x(a)=a$.
            \end{theorem}
            \begin{proof}
                We prove Picard by using the
                Banach Fixed Point Theorem. First
                we write the problem as an integral
                equation.
                If $\dot{x}=f(t,x(t))$, then:
                \begin{equation*}
                    x(t)
                    =\int_{a}^{t}\frac{dx}{dt}dt
                    =x_{0}+\int_{a}^{t}f(t,x(t))dt
                \end{equation*}
                Let $(X,d)$ be $C[a,b]$ with the
                supremum norm $d_{\infty}$. Then
                $(x,d)$ is a complete metric space.
                Let $T:{X}\rightarrow{X}$ be defined
                by:
                \begin{equation*}
                    Tx=x_{0}+\int_{a}^{t}f(t,x(t))dt
                \end{equation*}
                All we need to do is show that $T$ is
                a contraction. Applying the
                Banach Fixed Point theorem then
                shows that there is a unique
                fixed point of $T$, thus showing
                that there is a unique solution
                to our original initial value problem.
                If $x,y\in{X}$, then:
                \begin{align*}
                    d(Tx,Ty)
                    &=\sup\{|Tx(t)-Ty(t)|\}\\
                    &=\sup\{
                        (x_{0}+
                         \int_{a}^{t}f(t,x(t))dt)
                       -(x_{0}+
                         \int_{a}^{t}f(t,y(t))dt)
                    \}\\
                    =&\sup\{
                        \int_{a}^{t}f(t,x(t))dt)-
                        \int_{a}^{t}f(t,y(t))dt)
                    \}\\
                    &\leq\int_{a}^{t}|
                        f(t,x(t))-f(t,y(t))|dt
                \end{align*}
                But from the Lipschitz continuity
                of $f$, we have:
                \begin{align*}
                    d(Tx,Ty)&\leq
                    L\int_{a}^{t}|x(t)-y(t)|dt\\
                    &\leq{L}(t-a)d(x,y)\\
                    &\leq{L}(b-a)d(x,y)
                \end{align*}
                So $T$ is a contraction for
                $L(b-a)<1$. Usually we can
                extend this solution by taking
                $b$ as the initial condition and
                stepping forward one interval
                at a time. We'll take a different
                approach. We have that
                $d(Tx,Ty)\leq{L}(b-a)d(x,y)$. From
                this, we obtain:
                \begin{align*}
                    d(T^{2}x,T^{2}y)
                    &\leq{L}\int_{a}^{b}d(Tx,Ty)dt\\
                    &\leq{L}\int_{a}^{t}
                        L(t-a)d(x,t)dt\\
                    &=\frac{L^{2}}{2}(t-a)^{2}d(x,y)\\
                    &\leq
                    \frac{L^{2}}{2}(b-a)^{2}d(x,y)
                \end{align*}
                Applying induction, we have:
                \begin{equation*}
                    d(T^{n}x,T^{n}y)
                    \leq\frac{L^{n}}{n!}(b-a)^{n}
                \end{equation*}
                But this tends to zero, and thus
                there is an $N$ such that,
                for all $n>N$, $T^{n}$ is a
                contraction. But then, by the
                Banach Fixed Point Theorem, there
                is a unique point $x$ such that
                $T^{n}x=x$. But then
                $Tx=T^{n}(Tx)$, and thus
                $Tx$ is a fixed point of
                $T^{n}$. But the fixed point of
                $T^{n}$ is unique, and $x$ is a
                fixed point. Therefore
                $Tx=x$. Therefore, etc.
            \end{proof}
            Without Lipschitz continuous you may
            lose uniqueness, but you still have
            existence. This is Peano's theorem.
            An example is $\dot{x}=\sqrt{x}$
            with $x(0)=0$.
            This has solutions $x(t)=0$ and
            $(t)=t^{2}/4$. Now back to compactness.
            \subsubsection{Compactness}
                \begin{definition}
                    A metric space $(X,d)$ is
                    sequentially compact if every
                    sequence in $X$ has a convergent
                    subsequence.
                \end{definition}
                In topology there is a difference
                between sequential compactness
                and regular compactness, but in
                metric spaces they turn out
                to be the same.
                A subset of $S$ of $X$ is
                compact if every sequence in
                $S$ has a subsequence which converges.
                That is, $(S,d)$ is compact.
                \begin{theorem}
                    A subset $S$ of a compact
                    metric space $(X,d)$ is compact
                    if and only if $S$ is closed.
                \end{theorem}
                \begin{proof}
                    For let $x_{n}$ be a sequence
                    in $S$. Then $x_{n}$ is a
                    sequence in $X$ and thus there
                    is a convergent subsequence
                    $x_{k_{n}}$ with a limit $x$.
                    But $x_{k_{n}}$ is in $S$ and
                    $S$ is closed, and therefore
                    $x$ is in $S$. Thus, $S$
                    is compact. Conversely, if
                    $S$ is compact, suppose it is
                    not closed. Then there is a point
                    $y\in{X}$ such that $y$ is a
                    limit point of $S$ but not
                    contained in $S$. Let
                    $x_{n}$ be a sequence that
                    converges to $y$. Then, as
                    $S$ is compact, there is
                    a convergent subsequence. But
                    the limit of this subsequence
                    is $y$, a contradiction as
                    $y\notin{S}$. Therefore $S$
                    is closed.
                \end{proof}
                \begin{theorem}
                    If $(X,d)$ is a compact metric
                    space, then
                    $(X,d)$ is complete.
                \end{theorem}
                \begin{proof}
                    If $x_{n}$ is Cauchy in $X$,
                    then there is a convergent
                    subsequence $x_{k_{n}}$
                    in $X$. But if $x_{k_{n}}$
                    converges to $x$, then
                    $x_{n}$ converges to $x$ as
                    well, as $x_{n}$ is Cauchy.
                    Therefore, $(X,d)$ is complete.
                \end{proof}
                \begin{theorem}[Heine-Borel Theorem]
                    A subset of
                    $\mathbb{R}^{n}$ is
                    compact if and only if
                    it is closed and bounded.
                \end{theorem}
                \begin{example}
                    The closed unit ball
                    of $\ell^{p}$ is not compact,
                    if $1\leq{p}\leq{\infty}$.
                    Let $x_{n}(m)$ be the sequence
                    (of sequences) such that
                    $x_{n}(m)=1$ if $n=m$, and
                    zero otherwise. Then
                    $d_{p}(x_{n},x_{m})=2^{1/p}$,
                    so $x_{n}$ has no subsequence
                    which is Cauchy. But then there
                    is no convergent subsequence
                    either, and therefore
                    $\ell^{p}$ is not compact.
                \end{example}
                \begin{example}
                    The closed unit ball in
                    $(C[0,1],d_{\infty})$ is
                    not compact. For let
                    $x_{n}(t)=t^{2^{n}}$. Then
                    (Do some calculus) the maximum of
                    $d(x_{n},x_{n+1})$ is always
                    $1/4$. So this has no subsequence
                    which is Cauchy, and thus no
                    convergent subsequence exists.
                \end{example}
                \begin{definition}
                    A metric space $X$ is totally
                    bounded if for all
                    $\varepsilon>0$ there is a finite
                    number of points $x_{n}$ such
                    that $B_{\varepsilon}(x_{n})$
                    covers the entirety of $X$.
                \end{definition}
                \begin{theorem}
                    A compact metric space is
                    totally bounded.
                \end{theorem}
                \begin{proof}
                    Suppose not. Then there is an
                    $\varepsilon>0$ such that
                    no finite collection
                    $B_{\varepsilon}(x_{n})$
                    is a covering of $X$. Let
                    $x_{1}\in{X}$. Then
                    $B_{\varepsilon}(x_{1})$ is not
                    $X$. Thus there is an $x_{2}$
                    such that
                    $x_{2}\notin%
                     B_{\varepsilon}(x_{1})$.
                    But also
                    $B_{\varepsilon}(x_{1})\cup%
                     B_{\varepsilon}(x_{2})$ is
                    not the entirety of $X$.
                    Continuing we have that there
                    is a sequence $x_{n}$ such that,
                    for all $n\ne{m}$,
                    $d(x_{n},x_{m})\geq{\varepsilon}$.
                    So there is no convergent
                    subsequence. But $X$ is
                    compact, a contradiction.
                    Therefore, etc.
                \end{proof}
                There are metric spaces that are
                bounded but not totally bounded.
                For let
                $X=\mathbb{R}$ and $d$ be the
                discrete metric. Then, for
                $\varepsilon=1/2$, the is no
                finite covering. Every point needs
                it's own ball, so the covering is
                uncountable.
                \begin{theorem}
                    If $(X,d)$ is complete and
                    totally bounded, then it
                    is compact.
                \end{theorem}
                \begin{proof}
                    Let $x_{n}$ be a sequence
                    in $X$. Let $\varepsilon=1$. Then
                    there are finitely many points
                    $y_{k}$ such that
                    $B_{\varepsilon}(y_{k})$ covers
                    $X$. Then one of these
                    balls has infinitely many of
                    the $x_{n}$. Similarly, for
                    $\varepsilon=\frac{1}{n}$, there
                    is a finite number of points
                    $y_{k}$ such that
                    $B_{\frac{1}{n}}(y_{k})$ covers
                    $X$. Thus there is a point with
                    infinitely many of the $x_{n}$
                    in it. So, we can find a
                    subsequence such that, for
                    $n,m>N$,
                    $d(x_{k_{n}},x_{k_{m}})<%
                     \frac{1}{N}$. But $(X,d)$ is
                    complete, and therefore
                    $x_{k_{n}}$ converges. Therefore
                    $x_{n}$ has a convergent
                    subsequence. Thus, $(X,d)$ is
                    compact.
                \end{proof}
                \begin{theorem}
                    Compact spaces are separable.
                \end{theorem}
                \begin{proof}
                    If $X$ is compact, then
                    it is totally bounded. But
                    then, for $\varepsilon=1/n$
                    there is a finite covering of
                    $X$ with balls of radius
                    $\varepsilon$. Then,
                    taking all of the
                    centers of all of the points
                    for all $n$ (Countable union
                    of finite points is countable),
                    we obtain a countable dense
                    subset.
                \end{proof}
                \begin{example}
                    There are ``infinite dimension''
                    sets that are also compact. Two
                    in particular worth mentioning.
                    The first is the hilbert Cube.
                    It's a subset of $\ell^{2}$
                    whose elements are such that
                    $|x_{n}|<1/n$. That is, elements
                    are sequences whose $n^{th}$
                    elements are less than
                    $1/n$. This is compact.
                    Arzela-Ascoli. Peano.
                \end{example}
\end{document}