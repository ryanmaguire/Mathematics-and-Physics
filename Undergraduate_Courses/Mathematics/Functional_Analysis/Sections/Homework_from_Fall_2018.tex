\documentclass[crop=false,class=article,oneside]{standalone}
%----------------------------Preamble-------------------------------%
%---------------------------Packages----------------------------%
\usepackage{geometry}
\geometry{b5paper, margin=1.0in}
\usepackage[T1]{fontenc}
\usepackage{graphicx, float}            % Graphics/Images.
\usepackage{natbib}                     % For bibliographies.
\bibliographystyle{agsm}                % Bibliography style.
\usepackage[french, english]{babel}     % Language typesetting.
\usepackage[dvipsnames]{xcolor}         % Color names.
\usepackage{listings}                   % Verbatim-Like Tools.
\usepackage{mathtools, esint, mathrsfs} % amsmath and integrals.
\usepackage{amsthm, amsfonts, amssymb}  % Fonts and theorems.
\usepackage{tcolorbox}                  % Frames around theorems.
\usepackage{upgreek}                    % Non-Italic Greek.
\usepackage{fmtcount, etoolbox}         % For the \book{} command.
\usepackage[newparttoc]{titlesec}       % Formatting chapter, etc.
\usepackage{titletoc}                   % Allows \book in toc.
\usepackage[nottoc]{tocbibind}          % Bibliography in toc.
\usepackage[titles]{tocloft}            % ToC formatting.
\usepackage{pgfplots, tikz}             % Drawing/graphing tools.
\usepackage{imakeidx}                   % Used for index.
\usetikzlibrary{
    calc,                   % Calculating right angles and more.
    angles,                 % Drawing angles within triangles.
    arrows.meta,            % Latex and Stealth arrows.
    quotes,                 % Adding labels to angles.
    positioning,            % Relative positioning of nodes.
    decorations.markings,   % Adding arrows in the middle of a line.
    patterns,
    arrows
}                                       % Libraries for tikz.
\pgfplotsset{compat=1.9}                % Version of pgfplots.
\usepackage[font=scriptsize,
            labelformat=simple,
            labelsep=colon]{subcaption} % Subfigure captions.
\usepackage[font={scriptsize},
            hypcap=true,
            labelsep=colon]{caption}    % Figure captions.
\usepackage[pdftex,
            pdfauthor={Ryan Maguire},
            pdftitle={Mathematics and Physics},
            pdfsubject={Mathematics, Physics, Science},
            pdfkeywords={Mathematics, Physics, Computer Science, Biology},
            pdfproducer={LaTeX},
            pdfcreator={pdflatex}]{hyperref}
\hypersetup{
    colorlinks=true,
    linkcolor=blue,
    filecolor=magenta,
    urlcolor=Cerulean,
    citecolor=SkyBlue
}                           % Colors for hyperref.
\usepackage[toc,acronym,nogroupskip,nopostdot]{glossaries}
\usepackage{glossary-mcols}
%------------------------Theorem Styles-------------------------%
\theoremstyle{plain}
\newtheorem{theorem}{Theorem}[section]

% Define theorem style for default spacing and normal font.
\newtheoremstyle{normal}
    {\topsep}               % Amount of space above the theorem.
    {\topsep}               % Amount of space below the theorem.
    {}                      % Font used for body of theorem.
    {}                      % Measure of space to indent.
    {\bfseries}             % Font of the header of the theorem.
    {}                      % Punctuation between head and body.
    {.5em}                  % Space after theorem head.
    {}

% Italic header environment.
\newtheoremstyle{thmit}{\topsep}{\topsep}{}{}{\itshape}{}{0.5em}{}

% Define environments with italic headers.
\theoremstyle{thmit}
\newtheorem*{solution}{Solution}

% Define default environments.
\theoremstyle{normal}
\newtheorem{example}{Example}[section]
\newtheorem{definition}{Definition}[section]
\newtheorem{problem}{Problem}[section]

% Define framed environment.
\tcbuselibrary{most}
\newtcbtheorem[use counter*=theorem]{ftheorem}{Theorem}{%
    before=\par\vspace{2ex},
    boxsep=0.5\topsep,
    after=\par\vspace{2ex},
    colback=green!5,
    colframe=green!35!black,
    fonttitle=\bfseries\upshape%
}{thm}

\newtcbtheorem[auto counter, number within=section]{faxiom}{Axiom}{%
    before=\par\vspace{2ex},
    boxsep=0.5\topsep,
    after=\par\vspace{2ex},
    colback=Apricot!5,
    colframe=Apricot!35!black,
    fonttitle=\bfseries\upshape%
}{ax}

\newtcbtheorem[use counter*=definition]{fdefinition}{Definition}{%
    before=\par\vspace{2ex},
    boxsep=0.5\topsep,
    after=\par\vspace{2ex},
    colback=blue!5!white,
    colframe=blue!75!black,
    fonttitle=\bfseries\upshape%
}{def}

\newtcbtheorem[use counter*=example]{fexample}{Example}{%
    before=\par\vspace{2ex},
    boxsep=0.5\topsep,
    after=\par\vspace{2ex},
    colback=red!5!white,
    colframe=red!75!black,
    fonttitle=\bfseries\upshape%
}{ex}

\newtcbtheorem[auto counter, number within=section]{fnotation}{Notation}{%
    before=\par\vspace{2ex},
    boxsep=0.5\topsep,
    after=\par\vspace{2ex},
    colback=SeaGreen!5!white,
    colframe=SeaGreen!75!black,
    fonttitle=\bfseries\upshape%
}{not}

\newtcbtheorem[use counter*=remark]{fremark}{Remark}{%
    fonttitle=\bfseries\upshape,
    colback=Goldenrod!5!white,
    colframe=Goldenrod!75!black}{ex}

\newenvironment{bproof}{\textit{Proof.}}{\hfill$\square$}
\tcolorboxenvironment{bproof}{%
    blanker,
    breakable,
    left=3mm,
    before skip=5pt,
    after skip=10pt,
    borderline west={0.6mm}{0pt}{green!80!black}
}

\AtEndEnvironment{lexample}{$\hfill\textcolor{red}{\blacksquare}$}
\newtcbtheorem[use counter*=example]{lexample}{Example}{%
    empty,
    title={Example~\theexample},
    boxed title style={%
        empty,
        size=minimal,
        toprule=2pt,
        top=0.5\topsep,
    },
    coltitle=red,
    fonttitle=\bfseries,
    parbox=false,
    boxsep=0pt,
    before=\par\vspace{2ex},
    left=0pt,
    right=0pt,
    top=3ex,
    bottom=1ex,
    before=\par\vspace{2ex},
    after=\par\vspace{2ex},
    breakable,
    pad at break*=0mm,
    vfill before first,
    overlay unbroken={%
        \draw[red, line width=2pt]
            ([yshift=-1.2ex]title.south-|frame.west) to
            ([yshift=-1.2ex]title.south-|frame.east);
        },
    overlay first={%
        \draw[red, line width=2pt]
            ([yshift=-1.2ex]title.south-|frame.west) to
            ([yshift=-1.2ex]title.south-|frame.east);
    },
}{ex}

\AtEndEnvironment{ldefinition}{$\hfill\textcolor{Blue}{\blacksquare}$}
\newtcbtheorem[use counter*=definition]{ldefinition}{Definition}{%
    empty,
    title={Definition~\thedefinition:~{#1}},
    boxed title style={%
        empty,
        size=minimal,
        toprule=2pt,
        top=0.5\topsep,
    },
    coltitle=Blue,
    fonttitle=\bfseries,
    parbox=false,
    boxsep=0pt,
    before=\par\vspace{2ex},
    left=0pt,
    right=0pt,
    top=3ex,
    bottom=0pt,
    before=\par\vspace{2ex},
    after=\par\vspace{1ex},
    breakable,
    pad at break*=0mm,
    vfill before first,
    overlay unbroken={%
        \draw[Blue, line width=2pt]
            ([yshift=-1.2ex]title.south-|frame.west) to
            ([yshift=-1.2ex]title.south-|frame.east);
        },
    overlay first={%
        \draw[Blue, line width=2pt]
            ([yshift=-1.2ex]title.south-|frame.west) to
            ([yshift=-1.2ex]title.south-|frame.east);
    },
}{def}

\AtEndEnvironment{ltheorem}{$\hfill\textcolor{Green}{\blacksquare}$}
\newtcbtheorem[use counter*=theorem]{ltheorem}{Theorem}{%
    empty,
    title={Theorem~\thetheorem:~{#1}},
    boxed title style={%
        empty,
        size=minimal,
        toprule=2pt,
        top=0.5\topsep,
    },
    coltitle=Green,
    fonttitle=\bfseries,
    parbox=false,
    boxsep=0pt,
    before=\par\vspace{2ex},
    left=0pt,
    right=0pt,
    top=3ex,
    bottom=-1.5ex,
    breakable,
    pad at break*=0mm,
    vfill before first,
    overlay unbroken={%
        \draw[Green, line width=2pt]
            ([yshift=-1.2ex]title.south-|frame.west) to
            ([yshift=-1.2ex]title.south-|frame.east);},
    overlay first={%
        \draw[Green, line width=2pt]
            ([yshift=-1.2ex]title.south-|frame.west) to
            ([yshift=-1.2ex]title.south-|frame.east);
    }
}{thm}

%--------------------Declared Math Operators--------------------%
\DeclareMathOperator{\adjoint}{adj}         % Adjoint.
\DeclareMathOperator{\Card}{Card}           % Cardinality.
\DeclareMathOperator{\curl}{curl}           % Curl.
\DeclareMathOperator{\diam}{diam}           % Diameter.
\DeclareMathOperator{\dist}{dist}           % Distance.
\DeclareMathOperator{\Div}{div}             % Divergence.
\DeclareMathOperator{\Erf}{Erf}             % Error Function.
\DeclareMathOperator{\Erfc}{Erfc}           % Complementary Error Function.
\DeclareMathOperator{\Ext}{Ext}             % Exterior.
\DeclareMathOperator{\GCD}{GCD}             % Greatest common denominator.
\DeclareMathOperator{\grad}{grad}           % Gradient
\DeclareMathOperator{\Ima}{Im}              % Image.
\DeclareMathOperator{\Int}{Int}             % Interior.
\DeclareMathOperator{\LC}{LC}               % Leading coefficient.
\DeclareMathOperator{\LCM}{LCM}             % Least common multiple.
\DeclareMathOperator{\LM}{LM}               % Leading monomial.
\DeclareMathOperator{\LT}{LT}               % Leading term.
\DeclareMathOperator{\Mod}{mod}             % Modulus.
\DeclareMathOperator{\Mon}{Mon}             % Monomial.
\DeclareMathOperator{\multideg}{mutlideg}   % Multi-Degree (Graphs).
\DeclareMathOperator{\nul}{nul}             % Null space of operator.
\DeclareMathOperator{\Ord}{Ord}             % Ordinal of ordered set.
\DeclareMathOperator{\Prin}{Prin}           % Principal value.
\DeclareMathOperator{\proj}{proj}           % Projection.
\DeclareMathOperator{\Refl}{Refl}           % Reflection operator.
\DeclareMathOperator{\rk}{rk}               % Rank of operator.
\DeclareMathOperator{\sgn}{sgn}             % Sign of a number.
\DeclareMathOperator{\sinc}{sinc}           % Sinc function.
\DeclareMathOperator{\Span}{Span}           % Span of a set.
\DeclareMathOperator{\Spec}{Spec}           % Spectrum.
\DeclareMathOperator{\supp}{supp}           % Support
\DeclareMathOperator{\Tr}{Tr}               % Trace of matrix.
%--------------------Declared Math Symbols--------------------%
\DeclareMathSymbol{\minus}{\mathbin}{AMSa}{"39} % Unary minus sign.
%------------------------New Commands---------------------------%
\DeclarePairedDelimiter\norm{\lVert}{\rVert}
\DeclarePairedDelimiter\ceil{\lceil}{\rceil}
\DeclarePairedDelimiter\floor{\lfloor}{\rfloor}
\newcommand*\diff{\mathop{}\!\mathrm{d}}
\newcommand*\Diff[1]{\mathop{}\!\mathrm{d^#1}}
\renewcommand*{\glstextformat}[1]{\textcolor{RoyalBlue}{#1}}
\renewcommand{\glsnamefont}[1]{\textbf{#1}}
\renewcommand\labelitemii{$\circ$}
\renewcommand\thesubfigure{%
    \arabic{chapter}.\arabic{figure}.\arabic{subfigure}}
\addto\captionsenglish{\renewcommand{\figurename}{Fig.}}
\numberwithin{equation}{section}

\renewcommand{\vector}[1]{\boldsymbol{\mathrm{#1}}}

\newcommand{\uvector}[1]{\boldsymbol{\hat{\mathrm{#1}}}}
\newcommand{\topspace}[2][]{(#2,\tau_{#1})}
\newcommand{\measurespace}[2][]{(#2,\varSigma_{#1},\mu_{#1})}
\newcommand{\measurablespace}[2][]{(#2,\varSigma_{#1})}
\newcommand{\manifold}[2][]{(#2,\tau_{#1},\mathcal{A}_{#1})}
\newcommand{\tanspace}[2]{T_{#1}{#2}}
\newcommand{\cotanspace}[2]{T_{#1}^{*}{#2}}
\newcommand{\Ckspace}[3][\mathbb{R}]{C^{#2}(#3,#1)}
\newcommand{\funcspace}[2][\mathbb{R}]{\mathcal{F}(#2,#1)}
\newcommand{\smoothvecf}[1]{\mathfrak{X}(#1)}
\newcommand{\smoothonef}[1]{\mathfrak{X}^{*}(#1)}
\newcommand{\bracket}[2]{[#1,#2]}

%------------------------Book Command---------------------------%
\makeatletter
\renewcommand\@pnumwidth{1cm}
\newcounter{book}
\renewcommand\thebook{\@Roman\c@book}
\newcommand\book{%
    \if@openright
        \cleardoublepage
    \else
        \clearpage
    \fi
    \thispagestyle{plain}%
    \if@twocolumn
        \onecolumn
        \@tempswatrue
    \else
        \@tempswafalse
    \fi
    \null\vfil
    \secdef\@book\@sbook
}
\def\@book[#1]#2{%
    \refstepcounter{book}
    \addcontentsline{toc}{book}{\bookname\ \thebook:\hspace{1em}#1}
    \markboth{}{}
    {\centering
     \interlinepenalty\@M
     \normalfont
     \huge\bfseries\bookname\nobreakspace\thebook
     \par
     \vskip 20\p@
     \Huge\bfseries#2\par}%
    \@endbook}
\def\@sbook#1{%
    {\centering
     \interlinepenalty \@M
     \normalfont
     \Huge\bfseries#1\par}%
    \@endbook}
\def\@endbook{
    \vfil\newpage
        \if@twoside
            \if@openright
                \null
                \thispagestyle{empty}%
                \newpage
            \fi
        \fi
        \if@tempswa
            \twocolumn
        \fi
}
\newcommand*\l@book[2]{%
    \ifnum\c@tocdepth >-3\relax
        \addpenalty{-\@highpenalty}%
        \addvspace{2.25em\@plus\p@}%
        \setlength\@tempdima{3em}%
        \begingroup
            \parindent\z@\rightskip\@pnumwidth
            \parfillskip -\@pnumwidth
            {
                \leavevmode
                \Large\bfseries#1\hfill\hb@xt@\@pnumwidth{\hss#2}
            }
            \par
            \nobreak
            \global\@nobreaktrue
            \everypar{\global\@nobreakfalse\everypar{}}%
        \endgroup
    \fi}
\newcommand\bookname{Book}
\renewcommand{\thebook}{\texorpdfstring{\Numberstring{book}}{book}}
\providecommand*{\toclevel@book}{-2}
\makeatother
\titleformat{\part}[display]
    {\Large\bfseries}
    {\partname\nobreakspace\thepart}
    {0mm}
    {\Huge\bfseries}
\titlecontents{part}[0pt]
    {\large\bfseries}
    {\partname\ \thecontentslabel: \quad}
    {}
    {\hfill\contentspage}
\titlecontents{chapter}[0pt]
    {\bfseries}
    {\chaptername\ \thecontentslabel:\quad}
    {}
    {\hfill\contentspage}
\newglossarystyle{longpara}{%
    \setglossarystyle{long}%
    \renewenvironment{theglossary}{%
        \begin{longtable}[l]{{p{0.25\hsize}p{0.65\hsize}}}
    }{\end{longtable}}%
    \renewcommand{\glossentry}[2]{%
        \glstarget{##1}{\glossentryname{##1}}%
        &\glossentrydesc{##1}{~##2.}
        \tabularnewline%
        \tabularnewline
    }%
}
\newglossary[not-glg]{notation}{not-gls}{not-glo}{Notation}
\newcommand*{\newnotation}[4][]{%
    \newglossaryentry{#2}{type=notation, name={\textbf{#3}, },
                          text={#4}, description={#4},#1}%
}
%--------------------------LENGTHS------------------------------%
% Spacings for the Table of Contents.
\addtolength{\cftsecnumwidth}{1ex}
\addtolength{\cftsubsecindent}{1ex}
\addtolength{\cftsubsecnumwidth}{1ex}
\addtolength{\cftfignumwidth}{1ex}
\addtolength{\cfttabnumwidth}{1ex}

% Indent and paragraph spacing.
\setlength{\parindent}{0em}
\setlength{\parskip}{0em}
%--------------------------Main Document----------------------------%
\begin{document}
    \ifx\ifmathcoursesfunctional\undefined
        \section*{Functional Analysis}
        \setcounter{section}{3}
    \else
        \section{Homework from Fall 2018 (UML)}
    \fi
    \subsection{Homework 1}
        \begin{problem}
            Find a sequence $x_{n}$ in $\mathbb{R}$
            such that $x_{n+1}-x_{n}\rightarrow{0}$
            for which:
            \begin{enumerate}
                \begin{multicols}{2}
                    \item $x_{n}$ is unbounded.
                    \item $x_{n}$ is bounded but
                          not Cauchy.
                \end{multicols}
            \end{enumerate}
        \end{problem}
        \begin{proof}[Solution]
            In order:
            \begin{enumerate}
                \item Let $x_{n}=\sqrt{n}$. Then
                      $x_{n}$ is unbounded.
                      But from some algebra (And for $n>0$),
                      we have:
                      \begin{equation*}
                          x_{n+1}-x_{n}
                          =\sqrt{n+1}-\sqrt{n}
                          =\frac{n+1-n}{\sqrt{n+1}+\sqrt{n}}
                          \leq\frac{1}{2\sqrt{n}}
                          \rightarrow{0}
                      \end{equation*}
                \item Let $x_{n}=\sin(\sqrt{n})$. Then $|x_{n}|$
                      is bounded by $1$. But:
                      \begin{equation*}
                          x_{n+1}-x_{n}
                          =\sin\big(
                              \sqrt{n+1})-\sin(\sqrt{n}
                          \big)
                          =2\sin\bigg(
                              \frac{\sqrt{n+1}-\sqrt{n}}{2}
                          \bigg)
                          \cos\bigg(
                              \frac{\sqrt{n+1}-\sqrt{n}}{2}
                          \bigg)
                      \end{equation*}
                      But from the previous problem,
                      $\sqrt{n+1}-\sqrt{n}\rightarrow{0}$.
                      From the continuity of $\sin$,
                      we have that
                      $\sin(\sqrt{n+1}-\sqrt{n})\rightarrow{0}$.
                      Therefore, $x_{n+1}-x_{n}\rightarrow{0}$.
                      But $x_{n}$ is not Cauchy. For consider
                      the subsequence $k_{n}=n^{2}$. Then:
                      \begin{equation*}
                          x_{k_{n+1}}-x_{k_{n}}
                          =\sin(n+1)-\sin(n)
                          =2\sin\big(\frac{1}{2}\big)
                           \cos\big(n+\frac{1}{2}\big)
                      \end{equation*}
                      And this does not converge to zero.
            \end{enumerate}
        \end{proof}
        \begin{problem}
            Suppose $f:\mathbb{R}\rightarrow\mathbb{R}$ is
            continuous at $x$ and that $x_{n}\rightarrow{x}$.
            Prove the following:
            \begin{enumerate}
                \item If $x_{n}\geq{0}$ for all $n$,
                      then $x\geq{0}$
                \item $f(x_{n})\rightarrow{f(x)}$
                \item If $f(x_{n})\geq{0}$ for all $n$,
                      then $f(x)\geq{0}$
                \item If $f(x)>0$, then there is a
                      $\delta>0$ such that $f(y)>0$ for
                      all $|x-y|<\delta$
            \end{enumerate}
        \end{problem}
        \begin{proof}[Solution]
            In order:
            \begin{enumerate}
                \item Suppose not. Suppose $x<0$.
                      Let $\varepsilon=-\frac{x}{2}$. Then
                      $\varepsilon>0$ and therefore there is
                      an $N\in\mathbb{N}$ such that for all
                      $n>N$, $|x-x_{n}|<\varepsilon$. But
                      for all $n$, $x_{n}>0$, and therefore
                      $|x-x_{n}|\geq|x|$.
                      But $\varepsilon=\frac{|x|}{2}<|x|$,
                      a contradiction. Therefore, $x\geq{0}$.
                \item Let $\varepsilon>0$ be given. As
                      $f$ is continuous there is a $\delta>0$
                      such that for all $x_{0}$ such that
                      $|x-x_{0}|<\delta$,
                      $|f(x)-f(x_0)|<\varepsilon$. But as
                      $\delta>0$ and $x_{n}\rightarrow{x}$,
                      there is an $N\in\mathbb{N}$ such that
                      for all $n>N$, $|x-x_{n}|<\delta$.
                      But then for all $n>N$ we have that
                      $|f(x)-f(x_{n})|<\varepsilon$.
                      Therefore, etc.
                \item Let $y_{n}$ be the sequence
                      $f(x_{n})$ and let $y=f(x)$.
                      As $f$ is continuous,
                      from the previous problem we have that
                      $y_{n}\rightarrow{y}$. But from the
                      first part, if $y_{n}$ is a sequence
                      such that $y_{n}\geq{0}$ for all $n$,
                      and if $y_{n}\rightarrow{y}$, then
                      $y\geq{0}$. Therefore, etc.
                \item If $f(x)>0$ and $f$ is continuous, then
                      let $\varepsilon=\frac{f(x)}{2}$. Then
                      $\varepsilon>0$ and thus there is a
                      $\delta>0$ such that for all $x_{0}$
                      such that $|x-x_{0}|<\delta$, we have
                      that $|f(x)-f(x_{0})|<\varepsilon$.
                      But then
                      $f(x)-\varepsilon<f(x_{0})$,
                      and thus $0<\frac{f(x)}{2}<f(x_{0})$.
                      Therefore, $f(x_{0})>0$.
            \end{enumerate}
        \end{proof}
        \begin{problem}
            Prove there is a subsequence of $x_{n}=n$ for which
            both $\sin(x_{k_{n}})$ and $\cos(x_{k_{n}})$
            converge using:
            \begin{enumerate}
                \begin{multicols}{2}
                    \item Degrees.
                    \item Radians.
                \end{multicols}
            \end{enumerate}
            \textit{%
                Bonus: Can you make $\sin$ and $\cos$
                have the same limit?
            }
        \end{problem}
        \begin{proof}[Solution]
            In order:
            \begin{enumerate}
                \item Let $k_{n}=360+45n$. Then
                      $\sin(x_{k_n})=\cos(x_{k_{n}})%
                       =\frac{1}{\sqrt{2}}$
                \item Let $y_{n}=\sin(n)$. Then $y_{n}$ is bounded
                      by $1$. By the Bolzano-Weierstrass theorem,
                      there is a convergent subsequence $k_{n}$.
                      Let $z_{n}=\cos(x_{k_{n}})$. Then $z_{n}$
                      is bounded by $1$, and therefore by the
                      Bolzano-Weierstrass theorem there is a
                      convergent subsequence $k_{m_{n}}$. But
                      any subsequequence of a convergent sequence
                      converges to the same limit, and therefore
                      $\sin(x_{k_{m_{n}}})$ converges. Thus,
                      $\sin(x_{k_{m_{n}}})$ and
                      $\sin(x_{k_{m_{n}}})$ converge. To make
                      them converge to the same limit, we need
                      to know that the set
                      $\{n\mod\alpha:n\in\mathbb{N}\}$ is dense
                      in $(0,\alpha)$ when $\alpha$ is
                      irrational. Thus there is a
                      subsequence such that
                      $x_{k_{n}}\mod{2\pi}%
                       \rightarrow{\frac{\pi}{4}}$. Then
                      $\sin(x_{k_{n}})$ and
                      $\cos(x_{k_{n}})$ both converge to
                      $\frac{1}{\sqrt{2}}$.
            \end{enumerate}
        \end{proof}
    \subsection{Homework 2}
        \begin{problem}
            Prove $f:S\rightarrow\mathbb{R}$ is uniformly
            continuous if and only if for all sequences
            $x_{n}$, $y_{n}:\mathbb{N}\rightarrow{S}$
            such that $x_{n}-y_{n}\rightarrow{0}$, we have that
            $f(x_{n})-f(y_{n})\rightarrow{0}$.
        \end{problem}
        \begin{proof}[Solution]
            Let $\varepsilon>0$. If $f$ is uniformly continuous,
            then there is a $\delta>0$ such that for all
            $x$, $x_{0}\in{S}$ such that $|x-x_{0}|<\delta$,
            we have that $|f(x)-f(x_{0})|<\varepsilon$. But if
            $x_{n}-y_{n}\rightarrow{0}$, then there is an
            $N\in\mathbb{N}$ such that for all $n>N$,
            $|x_{n}-y_{n}|<\delta$. But then, for all $n>N$,
            $|f(x_{n})-f(y_{n})|<\varepsilon$. Thefefore,
            $f(x_{n})-f(y_{n})\rightarrow{0}$. Proving the
            converse, suppose not. If $f$ is not uniformly
            continuous, then there exists $\varepsilon>0$
            such that for all $\delta>0$ there exists
            $x$, $x_{0}\in{S}$ such that
            $|x-x_{0}|<\delta$ and yet
            $|f(x)-f(x_{0})|\geq{\varepsilon}$. Let
            $x_{n}$ and $y_{n}$ be points such that
            $|x_{n}-y_{n}|<\frac{1}{n}$ and yet
            $|f(x_{n})-f(y_{n})|\geq\varepsilon$. Then
            $x_{n}-y_{n}\rightarrow{0}$. But if
            $x_{n}-y_{n}\rightarrow{0}$, then
            $f(x_{n})-f(y_{n})\rightarrow{0}$. But for all
            $n$, $|f(x_{n})-f(y_{n})|\geq{\varepsilon}$,
            a contradiction. Therefore, etc.
        \end{proof}
        \begin{problem}
            In class we proved the Weierstrass Approximation
            theorem for elements of $C[0,1]$ which vanish
            at $x=0$ and $x=1$.
            \begin{enumerate}
                \item Extend the result to functions which
                      do not vanish at the endpoints.
                \item Extend the result to arbitrary
                      closed and bounded intervals $[a,b]$.
            \end{enumerate}
        \end{problem}
        \begin{proof}[Solution]
            In order:
            \begin{enumerate}
                \item Let $f:[0,1]\rightarrow\mathbb{R}$ be
                      continuous. Let
                      $g(x)=f(1)+(1-x)f(0)$. Then
                      $h(x)=f(x)-g(x)$ is a polynomial which
                      vanishes at $x=0$ and $x=1$, and thus
                      by the Weierstrass approximation theorem
                      there is a sequence of polynomials
                      $P_{n}(x)$ such that
                      $P_{n}(x)\rightarrow{h(x)}$.
                      But $g(x)$ is a polynomial and
                      $f(x)=h(x)+g(x)$. Therefore
                      $F_{n}(x)=P_{n}(x)+g(x)$ is a sequence
                      of polynomials and
                      $F_{n}(x)\rightarrow{f(x)}$.
                \item If $f:[a,b]\rightarrow\mathbb{R}$ is
                      continuous, define
                      $g:[0,1]\rightarrow\mathbb{R}$ by
                      $g(x)=f(\frac{x-a}{b-a})$. Then, since
                      the composition of continuous functions
                      is continuous, $g$ is a continuou function
                      on $[0,1]$. But by the Weierstrass
                      approximation theorem there is a sequence
                      of polynomials $P_{n}(x)$ such that
                      $P_{n}(x)\rightarrow{g(x)}$. Let
                      $F_{n}(x)=P_{n}(bx+(1-x)a)$. Then
                      $F_{n}(x)$ is a sequence of polynomials
                      on $[a,b]$, an    $F_{n}(x)\rightarrow{f(x)}$.
            \end{enumerate}
        \end{proof}
        \begin{problem}
            \
            \begin{enumerate}
                \item Let $(X,d)$ be a metric space. Show that
                      for all $x$, $y$, $z\in{X}$,
                      $|d(x,z)-d(y,z)|\leq{d(x,y)}$
                \item Let $X=\{a,b,c\}$ and suppose $d$ is
                      a metric such that $d(a,b)=1$ and
                      $d(b,c)=2$. What are the possible values
                      of $d(a,c)$?
                \item Let $f:\mathbb{R}\rightarrow\mathbb{R}$.
                      What property must $f$ have so that
                      $d(x,y)=|f(x)-f(y)|$ is a metric on
                      $\mathbb{R}$?
            \end{enumerate}
        \end{problem}
        \begin{proof}[Solution]
            In order:
            \begin{enumerate}
                \item Suppose $d(x,z)\geq{d(y,z)}$.
                      If $d(x,z)<d(y,z)$, the proof is
                      symmetric. Thus we have:
                      \begin{equation*}
                          |d(x,z)-d(y,z)|=d(x,z)-d(y,z)
                          \leq{(d(x,y)+d(y,z))-d(y,z)}
                          =d(x,y)
                      \end{equation*}
                      Therefore,
                      $|d(x,z)-d(y,z)|\leq{d(x,y)}$.
                \item Consider the following table:
                      \begin{table}[H]
                          \captionsetup{type=table}
                          \centering
                          \begin{tabular}{|c|c|c|c|}
                              \hline
                              $X$&a&b&c\\
                              \hline
                              a&0&1&?\\
                              \hline
                              b&1&0&2\\
                              \hline
                              c&?&2&0\\
                              \hline
                          \end{tabular}
                      \end{table}
                      This obeys everything except the triangle
                      inequality. We must pick $d(a,c)$
                      such that this is upheld.
                      So we need the following:
                      \begin{align*}
                          d(a,b)&\leq{d(a,c)+d(c,b)}&
                          d(a,c)&\leq{d(a,b)+d(b,c)}&
                          d(b,c)&\leq{d(b,a)+d(a,c)}\\
                          \Rightarrow{1}&\leq{2+d(a,c)}&
                          \Rightarrow{d(a,c)}&\leq{3}&
                          \Rightarrow{2}&\leq{1+d(a,c)}
                      \end{align*}
                      So we need $1\leq{d(a,c)}\leq{3}$.
                \item $f$ must be injective. Suppose $f$ is any
                      function. Then
                      $|f(x)-f(y)|\leq{|f(x)-f(z)|+|f(z)-f(y)|}$
                      simply from the triangle inequality. Also
                      $|f(x)-f(y)|%
                       =|(-1)(f(y)-f(x))|=|f(y)-f(x)|$.
                      The absolute value function is doing the
                      bulk of the work.
                      But finally we require that
                      $|f(x)-f(y)|=0$ if and only if
                      $x=y$. But $|f(x)-f(y)|=0$ if and only
                      if $f(x)=f(y)$. So we require that $f$
                      is injective. If $f$ is not injective,
                      then there exists $x_{1}$, $x_{2}$
                      such that
                      $x_{1}\ne{x_{2}}$ and yet
                      $f(x_{1})=f(x_{2})$. But then
                      $|f(x_{1})-f(x_{2})|=0$, contradicting the
                      fact that this is a metric. If $f$ is
                      injective, then this is a metric. Note
                      injective functions need not be
                      continuous, and can be very crazy.
            \end{enumerate}
        \end{proof}
        \begin{problem}
            \label{FUNCTIONAL:HOMEWORK:2:PROBLEM:4}
            Let $X=\mathbb{R}^{2}$ and let $d$ be the metric
            such that you can only travel parallel to the
            $y$ axis, or along the $x$ axis. Carefully draw
            the unit ball in $(X,d)$ about the following points:
            \begin{enumerate}
                \begin{multicols}{4}
                    \item $(0,0)$
                    \item $(0,1)$
                    \item $(0, 0.5)$
                    \item $(0.5,0.5)$
                \end{multicols}
            \end{enumerate}
        \end{problem}
        \begin{proof}[Solution]
            If $\mathbf{x}_{1}=(x_{1},y_{1})$ and
            $\mathbf{x}_{2}=(x_{2},y_{2})$, then we have:
            \begin{equation*}
                d(\mathbf{x}_{1},\mathbf{x}_{2})=
                \begin{cases}
                    |y_{2}-y_{1}|,&x_{1}=x_{2}\\
                    |x_{2}-x_{1}|+|y_{1}|+|y_{2}|,&x_{1}\ne{x_{2}}
                \end{cases}
            \end{equation*}
            About the point $(0,0)$, the unit ball is simply points
            $(x,y)$ such that $|x|+|y|<1$. This is a ``diamond.''
            About $(0,1)$, first note that to get to any point
            whose $x$ coordinate is not $0$, you first must travel
            the entirety of the $y$ axis. Since this length is
            already $1$, you can't go left or right on the $x$ axis.
            The unit ball is the line segment on the $y$ axis
            between $(0,0)$ and $(0,2)$. For the third one, if
            the $x$ coordinate changes, we have
            $0.5+|y|+|x|<1$, which implies
            $|y|+|x|<0.5$. This is again a diamond, but a
            smaller one. If the $x$ coordinate does not
            change, we have $|y-0.5|<1$. This is another
            line segment. Repeat the same arguments for the
            fourth coordinate. The diagrams are show in
            Fig.~\ref{FUNCTIONAL:HOMEWORK:2:PROBLEM:4:FIGURES}.
        \end{proof}
        \begin{figure}[H]
            \centering
            \captionsetup{type=figure}
            \subimport{../../../../tikz/}
                      {Functional_Analysis_Fall_2018_HW_2_Problem_4}
            \caption[Figures for Wangsness 1-11 and 1-12]{%
                Figures for Problem
                \ref{FUNCTIONAL:HOMEWORK:2:PROBLEM:4}.%
            }
            \label{FUNCTIONAL:HOMEWORK:2:PROBLEM:4:FIGURES}
        \end{figure}
        \begin{problem}
            Suppose $f\in{C[0,\pi]}$ and $\varepsilon>0$.
            Complete the steps to show that there exists
            $a_{0},\hdots,a_{n}\in\mathbb{R}$ such that:
            \begin{equation*}
                |f(x)-\sum_{k=0}^{n}a_{k}\cos(kx)|<\varepsilon\quad{x\in[0,\pi]}
            \end{equation*}
            \begin{enumerate}
                \item Show there is a polynomial $P$ with
                      $|f(x)-P(\cos(x))|<\varepsilon$
                      for all $x\in[0,\pi]$.
                \item Use induction to show
                      $\cos^{n}(x)=\sum_{k=0}^{n}c_{k}\cos(kx)$
                      for $c_{0},\hdots,c_{n}\in\mathbb{R}$.
            \end{enumerate}
        \end{problem}
        \begin{proof}[Solution]
            $\cos(x)$ is a bijective function when considered on
            the interval $[0,\pi]$. Thus we can consider the
            function $f(\cos^{-1}(x))$. But since $\cos(x)$ is
            continuous on $[0,\pi]$, $\cos^{-1}(x)$ is
            continuous on $[-1,1]$. And the composition of
            continuous functions is continuous. So
            $f(\cos^{-1}(x))$ is continuous. By the
            Weierstrass Approximation theorem, there is a
            sequence of polynomials $P_{n}(x)$ such that
            $P_{n}(x)\rightarrow{f(\cos^{-1}(x))}$. But then
            $P_{n}(\cos(x))\rightarrow{f(x)}$. But as $P_{n}(x)$
            is a polynomial, it is of the form
            $\sum_{k=0}^{n}a_{k}x^{k}$. Then
            $P_{n}(\cos(x))=\sum_{k=0}^{n}a_{k}\cos^{k}(x)$. It
            now suffices to show that
            $\cos^{k}(x)=\sum_{m=0}^{N}c_{m}\cos(mx)$ for
            suitable $c_{m}$. We prove by induction. The base case
            is trivial. Suppose it holds for some $k\in\mathbb{N}$.
            Then:
            \begin{equation*}
                \cos^{k+1}(x)=\cos(x)\cos^{k}(x)
                =\cos(x)\sum_{k=0}^{N}c_{k}\cos(kx)
            \end{equation*}
            Note that
            $\cos(x)\cos(kx)%
             =\frac{1}{2}\cos((k-1)x)+\frac{1}{2}\cos((k+1)x)$.
            So we have:
            \begin{equation*}
                \cos^{k+1}(x)
                =\frac{1}{2}\sum_{k=0}^{N}c_{k}
                \bigg(
                    \cos\Big((k+1)x\Big)+\cos\Big((k-1)x\Big)
                \bigg)
            \end{equation*}
            This completes the theorem.
        \end{proof}
        \begin{problem}
            Can $d(x,y)=f(x-y)$ be a metric on $\mathbb{R}$
            if $f$ is differentiable:
            \begin{enumerate}
                \begin{multicols}{3}
                    \item Everywhere?
                    \item At the origin?
                    \item Everywhere except the origin?
                \end{multicols}
            \end{enumerate}
        \end{problem}
        \begin{proof}[Solution]
            In order:
            \begin{enumerate}
                \item No, see part 3.
                \item This is possible, take $f(x)=|x|$.
                \item If $f(x-y)$ is a metric, then
                      $f$ can not be differentiable
                      at the origin. For suppose not.
                      Because $f$ is a metric, it
                      must be an even function. But
                      then $f'(0)=0$. But $f(x-y)$ is
                      a metric, and therefore it obeys
                      the triangle inequality. Therefore
                      $f(2x)\leq{f(x)+f(x)}=2f(x)$. But then
                      $f(1)\leq{2f(\frac{1}{2})}%
                       \leq{4f(\frac{1}{4})}\hdots$
                      Let $h(x)=\frac{f(x)}{x}$. Then, from
                      the previous statement, $h(2x)\leq{h(x)}$.
                      Then $h(1)\leq{h(\frac{1}{2})}\hdots$
                      But $h(\frac{1}{2^{n}})\rightarrow{h(0)}$,
                      since this is simply the derivative of
                      $f$ at $x=0$. Therefore
                      $h(1)\leq{f'(0)}$. But $h(1)>0$ since
                      $f$ is a metric, a contradiction.
                      Therefore, etc.
            \end{enumerate}
        \end{proof}
\end{document}