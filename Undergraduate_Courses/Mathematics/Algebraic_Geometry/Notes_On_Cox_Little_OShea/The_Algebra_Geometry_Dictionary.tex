\documentclass[crop=false,class=article,oneside]{standalone}
%----------------------------Preamble-------------------------------%
%---------------------------Packages----------------------------%
\usepackage{geometry}
\geometry{b5paper, margin=1.0in}
\usepackage[T1]{fontenc}
\usepackage{graphicx, float}            % Graphics/Images.
\usepackage{natbib}                     % For bibliographies.
\bibliographystyle{agsm}                % Bibliography style.
\usepackage[french, english]{babel}     % Language typesetting.
\usepackage[dvipsnames]{xcolor}         % Color names.
\usepackage{listings}                   % Verbatim-Like Tools.
\usepackage{mathtools, esint, mathrsfs} % amsmath and integrals.
\usepackage{amsthm, amsfonts, amssymb}  % Fonts and theorems.
\usepackage{tcolorbox}                  % Frames around theorems.
\usepackage{upgreek}                    % Non-Italic Greek.
\usepackage{fmtcount, etoolbox}         % For the \book{} command.
\usepackage[newparttoc]{titlesec}       % Formatting chapter, etc.
\usepackage{titletoc}                   % Allows \book in toc.
\usepackage[nottoc]{tocbibind}          % Bibliography in toc.
\usepackage[titles]{tocloft}            % ToC formatting.
\usepackage{pgfplots, tikz}             % Drawing/graphing tools.
\usepackage{imakeidx}                   % Used for index.
\usetikzlibrary{
    calc,                   % Calculating right angles and more.
    angles,                 % Drawing angles within triangles.
    arrows.meta,            % Latex and Stealth arrows.
    quotes,                 % Adding labels to angles.
    positioning,            % Relative positioning of nodes.
    decorations.markings,   % Adding arrows in the middle of a line.
    patterns,
    arrows
}                                       % Libraries for tikz.
\pgfplotsset{compat=1.9}                % Version of pgfplots.
\usepackage[font=scriptsize,
            labelformat=simple,
            labelsep=colon]{subcaption} % Subfigure captions.
\usepackage[font={scriptsize},
            hypcap=true,
            labelsep=colon]{caption}    % Figure captions.
\usepackage[pdftex,
            pdfauthor={Ryan Maguire},
            pdftitle={Mathematics and Physics},
            pdfsubject={Mathematics, Physics, Science},
            pdfkeywords={Mathematics, Physics, Computer Science, Biology},
            pdfproducer={LaTeX},
            pdfcreator={pdflatex}]{hyperref}
\hypersetup{
    colorlinks=true,
    linkcolor=blue,
    filecolor=magenta,
    urlcolor=Cerulean,
    citecolor=SkyBlue
}                           % Colors for hyperref.
\usepackage[toc,acronym,nogroupskip,nopostdot]{glossaries}
\usepackage{glossary-mcols}
%------------------------Theorem Styles-------------------------%
\theoremstyle{plain}
\newtheorem{theorem}{Theorem}[section]

% Define theorem style for default spacing and normal font.
\newtheoremstyle{normal}
    {\topsep}               % Amount of space above the theorem.
    {\topsep}               % Amount of space below the theorem.
    {}                      % Font used for body of theorem.
    {}                      % Measure of space to indent.
    {\bfseries}             % Font of the header of the theorem.
    {}                      % Punctuation between head and body.
    {.5em}                  % Space after theorem head.
    {}

% Italic header environment.
\newtheoremstyle{thmit}{\topsep}{\topsep}{}{}{\itshape}{}{0.5em}{}

% Define environments with italic headers.
\theoremstyle{thmit}
\newtheorem*{solution}{Solution}

% Define default environments.
\theoremstyle{normal}
\newtheorem{example}{Example}[section]
\newtheorem{definition}{Definition}[section]
\newtheorem{problem}{Problem}[section]

% Define framed environment.
\tcbuselibrary{most}
\newtcbtheorem[use counter*=theorem]{ftheorem}{Theorem}{%
    before=\par\vspace{2ex},
    boxsep=0.5\topsep,
    after=\par\vspace{2ex},
    colback=green!5,
    colframe=green!35!black,
    fonttitle=\bfseries\upshape%
}{thm}

\newtcbtheorem[auto counter, number within=section]{faxiom}{Axiom}{%
    before=\par\vspace{2ex},
    boxsep=0.5\topsep,
    after=\par\vspace{2ex},
    colback=Apricot!5,
    colframe=Apricot!35!black,
    fonttitle=\bfseries\upshape%
}{ax}

\newtcbtheorem[use counter*=definition]{fdefinition}{Definition}{%
    before=\par\vspace{2ex},
    boxsep=0.5\topsep,
    after=\par\vspace{2ex},
    colback=blue!5!white,
    colframe=blue!75!black,
    fonttitle=\bfseries\upshape%
}{def}

\newtcbtheorem[use counter*=example]{fexample}{Example}{%
    before=\par\vspace{2ex},
    boxsep=0.5\topsep,
    after=\par\vspace{2ex},
    colback=red!5!white,
    colframe=red!75!black,
    fonttitle=\bfseries\upshape%
}{ex}

\newtcbtheorem[auto counter, number within=section]{fnotation}{Notation}{%
    before=\par\vspace{2ex},
    boxsep=0.5\topsep,
    after=\par\vspace{2ex},
    colback=SeaGreen!5!white,
    colframe=SeaGreen!75!black,
    fonttitle=\bfseries\upshape%
}{not}

\newtcbtheorem[use counter*=remark]{fremark}{Remark}{%
    fonttitle=\bfseries\upshape,
    colback=Goldenrod!5!white,
    colframe=Goldenrod!75!black}{ex}

\newenvironment{bproof}{\textit{Proof.}}{\hfill$\square$}
\tcolorboxenvironment{bproof}{%
    blanker,
    breakable,
    left=3mm,
    before skip=5pt,
    after skip=10pt,
    borderline west={0.6mm}{0pt}{green!80!black}
}

\AtEndEnvironment{lexample}{$\hfill\textcolor{red}{\blacksquare}$}
\newtcbtheorem[use counter*=example]{lexample}{Example}{%
    empty,
    title={Example~\theexample},
    boxed title style={%
        empty,
        size=minimal,
        toprule=2pt,
        top=0.5\topsep,
    },
    coltitle=red,
    fonttitle=\bfseries,
    parbox=false,
    boxsep=0pt,
    before=\par\vspace{2ex},
    left=0pt,
    right=0pt,
    top=3ex,
    bottom=1ex,
    before=\par\vspace{2ex},
    after=\par\vspace{2ex},
    breakable,
    pad at break*=0mm,
    vfill before first,
    overlay unbroken={%
        \draw[red, line width=2pt]
            ([yshift=-1.2ex]title.south-|frame.west) to
            ([yshift=-1.2ex]title.south-|frame.east);
        },
    overlay first={%
        \draw[red, line width=2pt]
            ([yshift=-1.2ex]title.south-|frame.west) to
            ([yshift=-1.2ex]title.south-|frame.east);
    },
}{ex}

\AtEndEnvironment{ldefinition}{$\hfill\textcolor{Blue}{\blacksquare}$}
\newtcbtheorem[use counter*=definition]{ldefinition}{Definition}{%
    empty,
    title={Definition~\thedefinition:~{#1}},
    boxed title style={%
        empty,
        size=minimal,
        toprule=2pt,
        top=0.5\topsep,
    },
    coltitle=Blue,
    fonttitle=\bfseries,
    parbox=false,
    boxsep=0pt,
    before=\par\vspace{2ex},
    left=0pt,
    right=0pt,
    top=3ex,
    bottom=0pt,
    before=\par\vspace{2ex},
    after=\par\vspace{1ex},
    breakable,
    pad at break*=0mm,
    vfill before first,
    overlay unbroken={%
        \draw[Blue, line width=2pt]
            ([yshift=-1.2ex]title.south-|frame.west) to
            ([yshift=-1.2ex]title.south-|frame.east);
        },
    overlay first={%
        \draw[Blue, line width=2pt]
            ([yshift=-1.2ex]title.south-|frame.west) to
            ([yshift=-1.2ex]title.south-|frame.east);
    },
}{def}

\AtEndEnvironment{ltheorem}{$\hfill\textcolor{Green}{\blacksquare}$}
\newtcbtheorem[use counter*=theorem]{ltheorem}{Theorem}{%
    empty,
    title={Theorem~\thetheorem:~{#1}},
    boxed title style={%
        empty,
        size=minimal,
        toprule=2pt,
        top=0.5\topsep,
    },
    coltitle=Green,
    fonttitle=\bfseries,
    parbox=false,
    boxsep=0pt,
    before=\par\vspace{2ex},
    left=0pt,
    right=0pt,
    top=3ex,
    bottom=-1.5ex,
    breakable,
    pad at break*=0mm,
    vfill before first,
    overlay unbroken={%
        \draw[Green, line width=2pt]
            ([yshift=-1.2ex]title.south-|frame.west) to
            ([yshift=-1.2ex]title.south-|frame.east);},
    overlay first={%
        \draw[Green, line width=2pt]
            ([yshift=-1.2ex]title.south-|frame.west) to
            ([yshift=-1.2ex]title.south-|frame.east);
    }
}{thm}

%--------------------Declared Math Operators--------------------%
\DeclareMathOperator{\adjoint}{adj}         % Adjoint.
\DeclareMathOperator{\Card}{Card}           % Cardinality.
\DeclareMathOperator{\curl}{curl}           % Curl.
\DeclareMathOperator{\diam}{diam}           % Diameter.
\DeclareMathOperator{\dist}{dist}           % Distance.
\DeclareMathOperator{\Div}{div}             % Divergence.
\DeclareMathOperator{\Erf}{Erf}             % Error Function.
\DeclareMathOperator{\Erfc}{Erfc}           % Complementary Error Function.
\DeclareMathOperator{\Ext}{Ext}             % Exterior.
\DeclareMathOperator{\GCD}{GCD}             % Greatest common denominator.
\DeclareMathOperator{\grad}{grad}           % Gradient
\DeclareMathOperator{\Ima}{Im}              % Image.
\DeclareMathOperator{\Int}{Int}             % Interior.
\DeclareMathOperator{\LC}{LC}               % Leading coefficient.
\DeclareMathOperator{\LCM}{LCM}             % Least common multiple.
\DeclareMathOperator{\LM}{LM}               % Leading monomial.
\DeclareMathOperator{\LT}{LT}               % Leading term.
\DeclareMathOperator{\Mod}{mod}             % Modulus.
\DeclareMathOperator{\Mon}{Mon}             % Monomial.
\DeclareMathOperator{\multideg}{mutlideg}   % Multi-Degree (Graphs).
\DeclareMathOperator{\nul}{nul}             % Null space of operator.
\DeclareMathOperator{\Ord}{Ord}             % Ordinal of ordered set.
\DeclareMathOperator{\Prin}{Prin}           % Principal value.
\DeclareMathOperator{\proj}{proj}           % Projection.
\DeclareMathOperator{\Refl}{Refl}           % Reflection operator.
\DeclareMathOperator{\rk}{rk}               % Rank of operator.
\DeclareMathOperator{\sgn}{sgn}             % Sign of a number.
\DeclareMathOperator{\sinc}{sinc}           % Sinc function.
\DeclareMathOperator{\Span}{Span}           % Span of a set.
\DeclareMathOperator{\Spec}{Spec}           % Spectrum.
\DeclareMathOperator{\supp}{supp}           % Support
\DeclareMathOperator{\Tr}{Tr}               % Trace of matrix.
%--------------------Declared Math Symbols--------------------%
\DeclareMathSymbol{\minus}{\mathbin}{AMSa}{"39} % Unary minus sign.
%------------------------New Commands---------------------------%
\DeclarePairedDelimiter\norm{\lVert}{\rVert}
\DeclarePairedDelimiter\ceil{\lceil}{\rceil}
\DeclarePairedDelimiter\floor{\lfloor}{\rfloor}
\newcommand*\diff{\mathop{}\!\mathrm{d}}
\newcommand*\Diff[1]{\mathop{}\!\mathrm{d^#1}}
\renewcommand*{\glstextformat}[1]{\textcolor{RoyalBlue}{#1}}
\renewcommand{\glsnamefont}[1]{\textbf{#1}}
\renewcommand\labelitemii{$\circ$}
\renewcommand\thesubfigure{%
    \arabic{chapter}.\arabic{figure}.\arabic{subfigure}}
\addto\captionsenglish{\renewcommand{\figurename}{Fig.}}
\numberwithin{equation}{section}

\renewcommand{\vector}[1]{\boldsymbol{\mathrm{#1}}}

\newcommand{\uvector}[1]{\boldsymbol{\hat{\mathrm{#1}}}}
\newcommand{\topspace}[2][]{(#2,\tau_{#1})}
\newcommand{\measurespace}[2][]{(#2,\varSigma_{#1},\mu_{#1})}
\newcommand{\measurablespace}[2][]{(#2,\varSigma_{#1})}
\newcommand{\manifold}[2][]{(#2,\tau_{#1},\mathcal{A}_{#1})}
\newcommand{\tanspace}[2]{T_{#1}{#2}}
\newcommand{\cotanspace}[2]{T_{#1}^{*}{#2}}
\newcommand{\Ckspace}[3][\mathbb{R}]{C^{#2}(#3,#1)}
\newcommand{\funcspace}[2][\mathbb{R}]{\mathcal{F}(#2,#1)}
\newcommand{\smoothvecf}[1]{\mathfrak{X}(#1)}
\newcommand{\smoothonef}[1]{\mathfrak{X}^{*}(#1)}
\newcommand{\bracket}[2]{[#1,#2]}

%------------------------Book Command---------------------------%
\makeatletter
\renewcommand\@pnumwidth{1cm}
\newcounter{book}
\renewcommand\thebook{\@Roman\c@book}
\newcommand\book{%
    \if@openright
        \cleardoublepage
    \else
        \clearpage
    \fi
    \thispagestyle{plain}%
    \if@twocolumn
        \onecolumn
        \@tempswatrue
    \else
        \@tempswafalse
    \fi
    \null\vfil
    \secdef\@book\@sbook
}
\def\@book[#1]#2{%
    \refstepcounter{book}
    \addcontentsline{toc}{book}{\bookname\ \thebook:\hspace{1em}#1}
    \markboth{}{}
    {\centering
     \interlinepenalty\@M
     \normalfont
     \huge\bfseries\bookname\nobreakspace\thebook
     \par
     \vskip 20\p@
     \Huge\bfseries#2\par}%
    \@endbook}
\def\@sbook#1{%
    {\centering
     \interlinepenalty \@M
     \normalfont
     \Huge\bfseries#1\par}%
    \@endbook}
\def\@endbook{
    \vfil\newpage
        \if@twoside
            \if@openright
                \null
                \thispagestyle{empty}%
                \newpage
            \fi
        \fi
        \if@tempswa
            \twocolumn
        \fi
}
\newcommand*\l@book[2]{%
    \ifnum\c@tocdepth >-3\relax
        \addpenalty{-\@highpenalty}%
        \addvspace{2.25em\@plus\p@}%
        \setlength\@tempdima{3em}%
        \begingroup
            \parindent\z@\rightskip\@pnumwidth
            \parfillskip -\@pnumwidth
            {
                \leavevmode
                \Large\bfseries#1\hfill\hb@xt@\@pnumwidth{\hss#2}
            }
            \par
            \nobreak
            \global\@nobreaktrue
            \everypar{\global\@nobreakfalse\everypar{}}%
        \endgroup
    \fi}
\newcommand\bookname{Book}
\renewcommand{\thebook}{\texorpdfstring{\Numberstring{book}}{book}}
\providecommand*{\toclevel@book}{-2}
\makeatother
\titleformat{\part}[display]
    {\Large\bfseries}
    {\partname\nobreakspace\thepart}
    {0mm}
    {\Huge\bfseries}
\titlecontents{part}[0pt]
    {\large\bfseries}
    {\partname\ \thecontentslabel: \quad}
    {}
    {\hfill\contentspage}
\titlecontents{chapter}[0pt]
    {\bfseries}
    {\chaptername\ \thecontentslabel:\quad}
    {}
    {\hfill\contentspage}
\newglossarystyle{longpara}{%
    \setglossarystyle{long}%
    \renewenvironment{theglossary}{%
        \begin{longtable}[l]{{p{0.25\hsize}p{0.65\hsize}}}
    }{\end{longtable}}%
    \renewcommand{\glossentry}[2]{%
        \glstarget{##1}{\glossentryname{##1}}%
        &\glossentrydesc{##1}{~##2.}
        \tabularnewline%
        \tabularnewline
    }%
}
\newglossary[not-glg]{notation}{not-gls}{not-glo}{Notation}
\newcommand*{\newnotation}[4][]{%
    \newglossaryentry{#2}{type=notation, name={\textbf{#3}, },
                          text={#4}, description={#4},#1}%
}
%--------------------------LENGTHS------------------------------%
% Spacings for the Table of Contents.
\addtolength{\cftsecnumwidth}{1ex}
\addtolength{\cftsubsecindent}{1ex}
\addtolength{\cftsubsecnumwidth}{1ex}
\addtolength{\cftfignumwidth}{1ex}
\addtolength{\cfttabnumwidth}{1ex}

% Indent and paragraph spacing.
\setlength{\parindent}{0em}
\setlength{\parskip}{0em}
%--------------------------Main Document----------------------------%
\begin{document}
    \ifx\ifmathcoursesalgebraicgeometry\undefined
        \section*{Algebraic Geometry}
        \setcounter{section}{1}
    \fi
    \subsection{The Algebra-Geometry Dictionary}
        \subsubsection{Hilbert's Nullstellensatz}
            \begin{theorem}[The Weak Nullstellensatz Theorem]
                If $k$ is an algebraically closed field,
                $I\subset k[x_1,\hdots ,x_n]$ is an ideal,
                and $\mathbf{V}(I)=\emptyset$,
                then $I=k[x_1,\hdots ,x_n]$.
            \end{theorem}
            \begin{theorem}[Hilbert's Nullstellensatz]
                If $k$ is an algebraically closed,
                $f_{1},\hdots,f_{s}\in k[x_{1},\hdots,x_{n}]$,
                and if
                $f\in\textbf{I}\big(\mathbf{V}(f_1,\hdots,f_s)\big)$,
                then $\exists_{m\in\mathbb{N}}$ such that
                $f^m \in \langle f_1,\hdots, f_s \rangle$.
            \end{theorem}
        \subsubsection{Radical Ideals and the Ideal-Variety Correspondence}
            \begin{theorem}
                If $V$ is an affine variety, and if
                $f\in \textbf{I}(V)$, then $f^m\in \textbf{I}(V)$.
            \end{theorem}
            \begin{definition}
                An ideal $I$ is said to be radical $f^m \in I$
                implies $f\in I$ for some $m\geq 1$.
            \end{definition}
            \begin{theorem}
                If $V$ is an affine variety,
                then $\textbf{I}(V)$ is a radical ideal.
            \end{theorem}
            \begin{definition}
                The radical of an ideal
                $I\subset k[x_{1},\hdots,x_{n}]$ is the set
                $\sqrt{I}=\{f:f^{m}\in I,m\in\mathbb{N}\}$.
            \end{definition}
            \begin{theorem}
                If $I\subset k[x_1,\hdots ,x_n]$ is an ideal,
                then $\sqrt{I}$ is an ideal.
            \end{theorem}
            \begin{theorem}[The Strong Nullstellensatz]
                If $k$ is an algebraically closed,
                and $I\subset k[x_1,\hdots ,x_n]$ is an ideal,
                then $\textbf{I}(\mathbf{V}(I))=\sqrt{I}$.
            \end{theorem}
            \begin{theorem}[The Ideal-Variety Correspondence]
                If $k$ is a field, then the maps
                $\textrm{affine varieties}%
                 \overset{\textbf{I}}\rightarrow\textrm{ideals}$
                and
                $\textrm{ideals}%
                 \overset{\mathbf{V}}\rightarrow\textrm{affine varieties}$
                are inclusion reversing and for any
                afffine variety $V$,
                $\mathbf{V}\big(\textbf{I}(V)\big)=V$.
            \end{theorem}
            \begin{theorem}[Radical Membership Theorem]
                If $k$ is a field and
                $I=\langle f_1,\hdots,f_s\rangle\subset k[x_1,\hdots,x_n]$
                is an ideal, then $f\in \sqrt{I}$ if and only if
                the constant polynomial $1$ belongs to
                $\langle f_1,\hdots, f_s, 1-yf\rangle$.
            \end{theorem}
            \begin{theorem}
                If $f\in k[x_1,\hdots ,x_n]$, and
                $I=\langle f\rangle$, and if
                $f=f_1^{\alpha_1}\cdots f_s^{\alpha_s}$,
                then $\sqrt{I}=\langle f_1\cdots f_s\rangle$.
            \end{theorem}
            \begin{definition}
                The reduction of a polynomial
                $f\in k[x_1,\hdots ,x_n]$ is the polynomial
                $f_{red}$ such that
                $\langle f_{red}\rangle=\sqrt{\langle f\rangle}$.
            \end{definition}
            \begin{definition}
                A square free polynomial is a polynomial
                $f\in k[x_1,\hdots ,x_n]$ such that $f=f_{red}$.
            \end{definition}
            \begin{definition}
                If $f,g\in k[x_1,\hdots ,x_n]$, then
                $h\in k[x_1,\hdots ,x_n]$ is said to be the
                greatest common divisor of $f$ and $g$ if $f$
                divides $f$ and $g$, and if $p$ is any polynomial
                that divides $f$ and $g$, then $p$ divides $h$.
            \end{definition}
            \begin{theorem}
                If $k$ is a field such that $\mathbb{Q}\subset k$,
                and $I=\langle f\rangle$ for some
                $f\in k[x_1,\hdots ,x_n]$, then
                $\sqrt{I}=\langle f_{red}\rangle$,
                where
                $f_{red}=\frac{f}{GCD%
                     \big(%
                         f,%
                         \frac{\partial f}{\partial x_1},%
                         \frac{\partial f}{\partial x_2},%
                         \hdots,%
                         \frac{\partial f}{\partial x_n}%
                     \big)}$
            \end{theorem}
        \subsubsection{Sums, Products, and Intersections of Ideals}
            \begin{definition}
                If $I$ and $J$ are ideals of a the ring
                $k[x_1,\hdots ,x_n]$, then the sum of $I$ and $J$,
                denoted $I+J$, is the set
                $I+J=\{f+g: f\in I, g\in J\}$.
            \end{definition}
            \begin{theorem}
                If $I$ and $J$ are ideals in $k[x_1,\hdots ,x_n]$,
                then $I+J$ is also an ideal in $k[x_1,\hdots ,x_n]$.
            \end{theorem}
            \begin{theorem}
                If $I$ and $J$ are ideals in $k[x_1,\hdots ,x_n]$,
                then $I+J$ is the smallest ideal containing $I$ and $J$.
            \end{theorem}
            \begin{theorem}
                If $f_1,\hdots, f_r \in k[x_1,\hdots ,x_n]$,
                then
                $\langle f_1,\hdots, f_r\rangle%
                 =\sum_{k=1}^{r}\langle f_k\rangle$
            \end{theorem}
            \begin{theorem}
                If $I$ and $J$ are ideals in
                $k[x_1,\hdots ,x_n]$, then
                $\mathbf{V}(I+J)=\mathbf{V}(I)\cap\mathbf{V}(J)$.
            \end{theorem}
            \begin{definition}
                If $I$ and $J$ are two ideals in
                $k[x_1,\hdots ,x_n]$, then their product,
                denoted $I\cdot J$, is defined to be the ideal
                generated by all polynomials $f\cdot g$,
                where $f\in I$, and $g\in J$.
            \end{definition}
            \begin{theorem}
                If $I = \langle f_1,\hdots, f_r\rangle$ and
                $J = \langle g_1,\hdots, g_s\rangle$, then
                $I \cdot J$ is generated by the set of all
                products
                $\{f_ig_j:1\leq i\leq r, 1\leq j \leq s\}$
            \end{theorem}
            \begin{theorem}
                If $I,J\subset k[x_1,\hdots ,x_n]$
                are ideals, then
                $\mathbf{V}(I\cdot J)=\mathbf{V}(I)\cup\mathbf{V}(J)$.
            \end{theorem}
            \begin{definition}
                If $I,J\subset k[x_1,\hdots ,x_n]$ are ideals,
                then the intersection of $I$ and $J$,
                denoted $I\cap J$, is the set of polynomials
                in both $I$ and $J$.
            \end{definition}
            \begin{theorem}
                If $I,J\subset k[x_1,\hdots ,x_n]$ are ideals,
                then $I\cap J$ is an ideal.
            \end{theorem}
        \subsubsection{Zariski Closure and Quotients of Ideals}
            \begin{theorem}
                If $S\subset k^n$, then the affine variety
                $\mathbf{V}\big(\textbf{I}(S)\big)$ is
                the smallest affine variety that contains $S$.
            \end{theorem}
            \begin{definition}
                The Zariski Closure of a subset $S$,
                denoted $\overline{S}$, of an affine space
                is the smallest affine algebraic variety
                containing the set. 
            \end{definition}
            \begin{theorem}
                If $k$ is an algebraically closed field
                and $V=\mathbf{V}(f_1,\hdots, f_s)\subset k^n$,
                then $\mathbf{V}(I_{\ell})$ is the Zariski Closure
                of $\pi_{\ell}(V)$.
            \end{theorem}
            \begin{theorem}
                If $V$ and $W$ are varieties such that
                $V\subset W$,
                then $W=V\cup \overline{\big(W\setminus V\big)}$.
            \end{theorem}
            \begin{definition}
                If $I,J\subset k[x_1,\hdots ,x_n]$ are ideals,
                then $I:J$ is the set,
                $\{f\in k[x_1,\hdots ,x_n]: fg \in I\ \forall_{g\in J}\}$
                and is called the ideal quotient of $I$ by $J$.
            \end{definition}
            \begin{theorem}
                If $I,J\subset k[x_1,\hdots ,x_n]$ are ideals,
                then $I:J$ is an ideal.
            \end{theorem}
            \begin{theorem}
                If $I,J\subset k[x_1,\hdots ,x_n]$ are ideals,
                then
                $\overline{\mathbf{V}(I)\setminus%
                 \mathbf{V}(J)}\subset\mathbf{V}(I:J)$.
            \end{theorem}
            \begin{theorem}
                If $I,J\subset k^n$ are affine varieties,
                then $\textbf{I}(V):\textbf{I}(W)=\textbf{I}(V\setminus)$
            \end{theorem}
            \begin{theorem}
                If $I,J,K\subset k[x_1,\hdots ,x_n]$,
                then $I:k[x_1,\hdots ,x_n]=I$.
            \end{theorem}
            \begin{theorem}
                If $I,J,K \subset k[x_1,\hdots ,x_n]$ are ideals,
                then $I\cdot J\subset K$ if and only if $I\subset K:J$
            \end{theorem}
            \begin{theorem}
                If $I,J,K\subset k[x_1,\hdots ,x_n]$ are ideals,
                then $J\subset I$ if and only if
                $I:J=k[x_1,\hdots ,x_n]$
            \end{theorem}
            \begin{theorem}
                If $I$ is an ideal, $g\in k[x_1,\hdots ,x_n]$,
                and if $\{h_1,\hdots, h_p\}$ is a basis of the
                ideal $I\cap \langle g \rangle$, then
                $\{h_1/g,\hdots, h_p/g\}$ is a basis of
                $I:\langle g\rangle$.
            \end{theorem}
        \subsubsection{Irreducible Varieties and Prime Ideals}
            \begin{definition}
                An affine variety $V\subset k^n$ is irreducible
                if there are no affine varieties $V_1, V_2$,
                such that $V = V_1\cup V_2$, $V_1,V_2\ne \emptyset$,
                and $V_1 \ne V, V_2 \ne V$.
            \end{definition}
            \begin{definition}
                An ideal $I\subset k[x_1,\hdots ,x_n]$ is
                said to be prime if whenever
                $f,g\in k[x_1,\hdots ,x_n]$ and $fg\in I$,
                either $f\in I$ or $g\in I$.
            \end{definition}
            \begin{theorem}
                If $V\subset k^n$ is an affine variety,
                then $V$ is irreducible if and only if
                $\textbf{I}(V)$ is a prime ideal.
            \end{theorem}
            \begin{definition}
                An ideal $I\subset k[x_1,\hdots ,x_n]$ is
                said to be maximal if $I \ne k[x_1,\hdots ,x_n]$
                and any ideal $J$ containing $I$ is such that
                either $J=I$ or $J=k[x_1,\hdots ,x_n]$.
            \end{definition}
            \begin{definition}
                An ideal $I\subset k[x_1,\hdots,x_n]$
                is called proper if $I$ is not equal to
                $k[x_1,\hdots ,x_n]$.
            \end{definition}
            \begin{theorem}
                If $k$ is a field and
                $I=\langle x_1-a_1,\hdots,x_n-a_n\rangle$
                is and ideal where $a_1,\hdots, a_n \in k$,
                then $I$ is maximal.
            \end{theorem}
            \begin{theorem}
                If $k$ is a field, then any maximal
                ideal is also a prime ideal.
            \end{theorem}
            \begin{theorem}
                If $k$ is an algebraically closed field,
                then every maximal ideal of $k[x_1,\hdots ,x_n]$
                is of the form
                $\langle x_1-a_1,\hdots, x_n-a_n\rangle$
                for some $a_1,\hdots, a_n\in k$.
            \end{theorem}
            \begin{definition}
                A primary decomposition of an ideal $I$ is
                an expression of $I$ as an intersection of
                primary ideals $I=\cap_{i=1}^{r} Q_{i}$.
            \end{definition}
            \begin{definition}
                A primary decomposition of an ideal $I$
                is said to be minimal $\sqrt{Q_i}$ are all
                distinct and
                $\cap_{j\ne i}Q_j\not\subset Q_i$
            \end{definition}
            \begin{theorem}
                If $I,J$ are primary and
                $\sqrt{I}=\sqrt{J}$,
                then $I\cap J$ is primary.
            \end{theorem}
            \begin{theorem}[Lasker-Noether Theorem]
                Every ideal $I \subset k[x_1,\hdots ,x_n]$
                has a minimal primary decomposition.
            \end{theorem}
\end{document}