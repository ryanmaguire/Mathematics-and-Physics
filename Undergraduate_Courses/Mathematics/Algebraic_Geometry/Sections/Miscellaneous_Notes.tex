\documentclass[crop=false,class=article,oneside]{standalone}
%----------------------------Preamble-------------------------------%
%---------------------------Packages----------------------------%
\usepackage{geometry}
\geometry{b5paper, margin=1.0in}
\usepackage[T1]{fontenc}
\usepackage{graphicx, float}            % Graphics/Images.
\usepackage{natbib}                     % For bibliographies.
\bibliographystyle{agsm}                % Bibliography style.
\usepackage[french, english]{babel}     % Language typesetting.
\usepackage[dvipsnames]{xcolor}         % Color names.
\usepackage{listings, lstlinebgrd}      % Verbatim-Like Tools.
\usepackage{mathtools, esint, mathrsfs} % amsmath and integrals.
\usepackage{amsthm, amsfonts}           % Fonts and theorems.
\usepackage{tabularx}
\usepackage{tcolorbox}                  % Frames around theorems.
\usepackage{upgreek}                    % Non-Italic Greek.
\usepackage{paracol}                    % Two-column styling.
\usepackage{wrapfig}                    % Wrap text around figure.
\usepackage{fmtcount, etoolbox}         % For the \book{} command.
\usepackage[newparttoc]{titlesec}       % Formatting chapter, etc.
\usepackage{titletoc}                   % Allows \book in toc.
\usepackage[nottoc]{tocbibind}          % Bibliography in toc.
\usepackage[titles]{tocloft}            % ToC formatting.
\usepackage{multicol, enumitem}         % Multi-column/enumerate.
\usepackage{import}                     % Import external files.
\usepackage{pgfplots, tikz}             % Drawing/graphing tools.
\usetikzlibrary{
    calc,                   % Calculating right angles and more.
    angles,                 % Drawing angles within triangles.
    arrows.meta,            % Latex and Stealth arrows.
    quotes,                 % Adding labels to angles.
    positioning,            % Relative positioning of nodes.
    decorations.markings,   % Adding arrows in the middle of a line.
    patterns,
    arrows,
    shapes,
    shapes.geometric,
    cd,
    hobby,
    babel
}                                       % Libraries for tikz.
\pgfplotsset{compat=1.9}                % Version of pgfplots.
\usepackage[font=scriptsize,
            labelformat=simple,
            labelsep=colon]{subcaption} % Subfigure captions.
\usepackage[font={scriptsize},
            hypcap=true,
            labelsep=colon]{caption}    % Figure captions.
\usepackage{hyperref}                   % Allows for hyperlinks.
\hypersetup{
    colorlinks=true,
    linkcolor=blue,
    filecolor=magenta,
    urlcolor=Cerulean,
    citecolor=SkyBlue
}                           % Colors for hyperref.
\usepackage[toc,acronym,nogroupskip]{glossaries} % Glossaries and acronyms.
\usepackage[subpreambles=false]{standalone}      % Complileable sub files.

% Various font stuff from kiwi.
% Use this for Times text and Computer Modern math
%\usepackage{times}

% Quite nice
%\usepackage[charter, greekfamily=, greekuppercase=italicized]{mathdesign}
%\usepackage[utopia, greekuppercase=italicized]{mathdesign}    % Math is narrower

% Use this for Times text and math
%\usepackage{newtxtext}
%\usepackage[libertine,cmintegrals]{newtxmath}
%\usepackage{fix-cm}

%\usepackage{txfontsb}
% or
%\usepackage{mathptmx}

%\usepackage[scaled=0.92]{helvet}
%\renewcommand{\rmdefault}{ptm}

%\usepackage{mathpazo}    % add possibly `sc` and `osf` options
%\usepackage{eulervm}

%\usepackage{fourier}
%\renewcommand{\rmdefault}{ptm}
%\usepackage{mathptm}

%\usepackage{fontspec}
%\setmainfont{lmodern}

%\usepackage[varg]{txfonts}
%\usepackage{fouriernc}
%\usepackage{mathpazo}

%\usepackage{bookman}
%\usepackage[scaled]{uarial}
%\usepackage[scaled]{helvet}
%\renewcommand*\familydefault{\sfdefault}
%\usepackage[math]{anttor}

%\newcommand\fgeorgia{\fontfamily{jvn}\selectfont}
%\newcommand\ftimes{\fontfamily{ptm}\selectfont}
%\newcommand\fhelvetica{\fontfamily{phv}\selectfont}
%\newcommand\fcourier{\fontfamily{pcr}\selectfont}
%\newcommand\fbookman{\fontfamily{pbk}\selectfont}
%\newcommand\fnewcentury{\fontfamily{pnc}\selectfont}
%\newcommand\fpalatino{\fontfamily{ppl}\selectfont}
%\newcommand\favantgarde{\fontfamily{pag}\selectfont}
%\newcommand\fnormal{\normalfont}
%\newcommand\fsize[1]{\ifnum#1>0\fontsize{#1}{#1}\selectfont\else\normalsize\fi}
%------------------------Theorem Styles-------------------------%
% Define theorem style for default spacing and normal font.
\newtheoremstyle{normal}
    {\topsep}               % Amount of space above the theorem.
    {\topsep}               % Amount of space below the theorem.
    {}                      % Font used for body of theorem.
    {}                      % Measure of space to indent.
    {\bfseries}             % Font of the header of the theorem.
    {}                      % Punctuation between head and body.
    {.5em}                  % Space after theorem head.
    {}

% Define theorem style for default spacing with italicized font.
\newtheoremstyle{normalit}{\topsep}{\topsep}
                {\itshape}{}{\bfseries}{}{.5em}{}

% Italic header environment.
\newtheoremstyle{thmit}{\topsep}{\topsep}{}{}{\itshape}{}{0.5em}{}

% Define italicized environments.
\theoremstyle{normalit}
\newtheorem{theorem}{Theorem}[section]
\newtheorem{lemma}{Lemma}[section]
\newtheorem{corollary}{Corollary}[section]
\newtheorem{proposition}{Proposition}[section]
\newtheorem*{theorem*}{Theorem}

% Define environments with italic headers.
\theoremstyle{thmit}
\newtheorem*{solution}{Solution}
\newtheorem*{fsolution}{Solution}

% Define default environments.
\theoremstyle{normal}
\newtheorem{example}{Example}[section]
\newtheorem{definition}{Definition}[section]
\newtheorem{problem}{Problem}[section]
\newtheorem{question}{Question}[section]
\newtheorem{remark}{Remark}[section]
\newtheorem{properties}{Properties}[section]
\newtheorem{notation}{Notation}[section]
\newtheorem{axiom}{Axiom}[section]
\newtheorem*{properties*}{Properties}
\newtheorem*{remark*}{Remark}
\newtheorem*{definition*}{Definition}
\theoremstyle{plain}

% Define framed environment.
\tcbuselibrary{most}
\newtcbtheorem[use counter*=theorem]{ftheorem}{Theorem}%
    {colback=green!5,colframe=green!35!black,
     fonttitle=\bfseries\upshape}{th}

\newtcbtheorem[use counter*=example]{fdefinition}{Definition}%
    {fonttitle=\bfseries\upshape,
     colback=blue!5!white,colframe=blue!75!black}{def}

\newtcbtheorem[use counter*=example]{fexample}{Example}%
    {fonttitle=\bfseries\upshape,
     colback=red!5!white,colframe=red!75!black}{ex}

\newtcbtheorem[use counter*=notation]{fnotation}{Notation}%
    {fonttitle=\bfseries\upshape,
     colback=SeaGreen!5!white,colframe=SeaGreen!75!black}{ex}

\newtcbtheorem[use counter*=corollary]{fcorollary}{Corollary}%
    {fonttitle=\bfseries\upshape,
     colback=Orchid!5!white,colframe=Orchid!75!black}{ex}

\newenvironment{bproof}{\textit{Proof.}}{\hfill$\square$}
\tcolorboxenvironment{bproof}{blanker,breakable,left=5mm,
                             before skip=10pt,after skip=10pt,
                             borderline west={1mm}{0pt}{red}}
\tcolorboxenvironment{fsolution}
    {enhanced jigsaw,colframe=cyan,interior hidden,breakable}

%--------------------Declared Math Operators--------------------%
\DeclareMathOperator{\Refl}{Refl}           % Reflection operator.
\DeclareMathOperator{\Span}{Span}           % Span of a set of vectors.
\DeclareMathOperator{\Card}{Card}           % Cardinality of set.
\DeclareMathOperator{\Ord}{Ord}             % Ordinal of ordered set.
\DeclareMathOperator{\Tr}{Tr}               % Trace of matrix.
\DeclareMathOperator{\adjoint}{adj}         % Adjoint of matrix.
\DeclareMathOperator{\rk}{rk}               % Rank of operator.
\DeclareMathOperator{\nul}{nul}             % Null space of operator.
\DeclareMathOperator{\sgn}{sgn}             % Sign of a number.
\DeclareMathOperator{\multideg}{mutlideg}   % Multi-Degree (Graphs).
\DeclareMathOperator{\GCD}{GCD}             % Greatest common denominator.
\DeclareMathOperator{\LM}{LM}               % Leading monomial
\DeclareMathOperator{\LC}{LC}               % Leading coefficient.
\DeclareMathOperator{\LT}{LT}               % Leading term.
\DeclareMathOperator{\LCM}{LCM}             % Least common multiple.
\DeclareMathOperator{\Mon}{Mon}             % Monomial.
\DeclareMathOperator{\Spec}{Spec}           % Spectrum.
\DeclareMathOperator{\proj}{proj}           % Projection.
\DeclareMathOperator{\comp}{comp}           % Component.
\DeclareMathOperator{\sinc}{sinc}           % Sinc function.
\DeclareMathOperator{\Ima}{Im}              % Image of operator.
\DeclareMathOperator{\Prin}{Prin}           % Principal value.
\DeclareMathOperator{\Mod}{mod}             % Modulus.
%------------------------New Commands---------------------------%
\DeclarePairedDelimiter\norm{\lVert}{\rVert}
\DeclarePairedDelimiter\ceil{\lceil}{\rceil}
\DeclarePairedDelimiter\floor{\lfloor}{\rfloor}
\newcommand*\diff{\mathop{}\!\mathrm{d}}
\newcommand*\Diff[1]{\mathop{}\!\mathrm{d^#1}}
\renewcommand{\mod}{\ \Mod}
\renewcommand*{\glstextformat}[1]{\textcolor{RoyalBlue}{#1}}
\renewcommand{\glsnamefont}[1]{\textbf{#1}}
\renewcommand\labelitemii{$\circ$}
\renewcommand\thesubfigure{\arabic{chapter}.\arabic{figure}}
\renewcommand\thesubfigure{%
    \arabic{chapter}.\arabic{figure}.\arabic{subfigure}}
\addto\captionsenglish{\renewcommand{\figurename}{Fig.}}
%------------------------Book Command---------------------------%
\makeatletter
\renewcommand\@pnumwidth{1cm}
\newcounter{book}
\renewcommand\thebook{\@Roman\c@book}
\newcommand\book{%
    \if@openright
        \cleardoublepage
    \else
        \clearpage
    \fi
    \thispagestyle{plain}%
    \if@twocolumn
        \onecolumn
        \@tempswatrue
    \else
        \@tempswafalse
    \fi
    \null\vfil
    \secdef\@book\@sbook
}
\def\@book[#1]#2{%
    \ifnum \c@secnumdepth >-3\relax
        \refstepcounter{book}%
        \addcontentsline{toc}{book}{
            \bookname\ \thebook:\hspace{1em}#1
        }
    \else
        \addcontentsline{toc}{book}{#1}%
    \fi
    \markboth{}{}%
    {\centering
     \interlinepenalty \@M
     \normalfont
     \ifnum \c@secnumdepth >-2\relax
       \huge\bfseries \bookname\nobreakspace\thebook
       \par
       \vskip 20\p@
     \fi
     \Huge \bfseries #2\par}%
    \@endbook}
\def\@sbook#1{%
    {\centering
     \interlinepenalty \@M
     \normalfont
     \Huge \bfseries #1\par}%
    \@endbook}
\def\@endbook{
    \vfil\newpage
        \if@twoside
            \if@openright
                \null
                \thispagestyle{empty}%
                \newpage
            \fi
        \fi
        \if@tempswa
            \twocolumn
        \fi
}
\newcommand*\l@book[2]{%
    \ifnum \c@tocdepth >-2\relax
        \addpenalty{-\@highpenalty}%
        \addvspace{2.25em \@plus\p@}%
        \setlength\@tempdima{3em}%
        \begingroup
            \parindent \z@ \rightskip \@pnumwidth
            \parfillskip -\@pnumwidth
            {
                \leavevmode
                \Large \bfseries #1\hfil \hb@xt@\@pnumwidth{
                    \hss #2
                }
            }
            \par
            \nobreak
            \global\@nobreaktrue
            \everypar{\global\@nobreakfalse\everypar{}}%
        \endgroup
    \fi}
\newcommand\bookname{Book}
\renewcommand{\thebook}{\texorpdfstring{\Numberstring{book}}{book}}
\providecommand*{\toclevel@book}{-2}
\makeatother
\titlecontents{chapter}[0pt]
    {\bfseries}
    {\chaptername\ \thecontentslabel:\quad}
    {}
    {\hfill\contentspage}
\titleformat{\part}[display]
    {\Large\bfseries}
    {\partname\nobreakspace\thepart}
    {0mm}
    {\Huge\bfseries}
    \titlecontents{part}[0pt]
    {\large\bfseries}
    {\partname\ \thecontentslabel: \quad}
    {}
    {\hfill\contentspage}
\newcommand{\MarkRightAngle}[4][.3cm]
    {\coordinate (tempa) at ($(#3)!#1!(#2)$);
     \coordinate (tempb) at ($(#3)!#1!(#4)$);
     \coordinate (tempc) at ($(tempa)!0.5!(tempb)$);%midpoint
     \draw (tempa) -- ($(#3)!2!(tempc)$) -- (tempb);}
%--------------------------LENGTHS------------------------------%
% Spacings for the Table of Contents.
\addtolength{\cftsecnumwidth}{1ex}
\addtolength{\cftsubsecindent}{1ex}
\addtolength{\cftsubsecnumwidth}{1ex}
\addtolength{\cftfignumwidth}{1ex}
\addtolength{\cfttabnumwidth}{1ex}

% Spacing for multi-column and enumerate environments.
\setlength{\multicolsep}{6pt}
\setlist[enumerate]{itemsep=0pt,topsep=3pt}

% Indent and paragraph spacing.
\setlength{\parindent}{0em}
\setlength{\parskip}{0em}
%--------------------------Main Document----------------------------%
\begin{document}
    \ifx\ifmathcoursesalgebraicgeometry\undefined
        \section*{Algebraic Geometry}
        \setcounter{section}{1}
    \fi
    \subsection{Groebner Bases}
        \begin{definition}
            A ring is a set $R$ with two binary operations $+$
            and $\cdot$, called addition and multiplication,
            such that the following are true:
            \begin{enumerate}
                \begin{multicols}{3}
                    \item $(R,+)$ is an Abelian Group
                    \item $(a\cdot{b})\cdot{c}=a\cdot(b\cdot{c})$
                    \item $a\cdot(b+c)=(a\cdot b)+(a\cdot c)$
                \end{multicols}
            \end{enumerate}
        \end{definition}
        \begin{definition}
            A commutative ring is a ring $R$ such that
            $\forall_{a,b\in R},a\cdot{b}=b\cdot{a}$
        \end{definition}
        \begin{definition}
            A ring with identity is a ring $R$ such that
            $\exists_{1_{R}\in R}:\forall_{a\in R}, 1_{R}\cdot a=a\cdot 1_{R}=a$
        \end{definition}
        \begin{definition}
            A subring of a ring with identity $R$ is a set
            $S\subset R$ such that $1_{R}\in S$, and $S$ is
            closed under the ring operations.
        \end{definition}
        \begin{remark}
            Any field is a ring.
        \end{remark}
        \begin{definition}
            A monomial in variables $x_1,\hdots, x_n$ over a
            ring $R$ is a product
            $x^\alpha=\prod_{k=1}^{n} x_1^{\alpha_1}$,
            where $(\alpha_1,\hdots,\alpha_n)\in \mathbb{N}^n$.
        \end{definition}
        \begin{notation}
            The set of monomials in $n$ variables over
            $R$ is denoted $\Mon_{R}(x_1,\hdots, x_n)$
        \end{notation}
        \begin{definition}
            If $\alpha,\beta \in \mathbb{N}^n$ such that
            $\alpha_i \leq \beta_i$, then $x^{\alpha}$ is said
            to divide $x^\beta$, denoted $x^\alpha \vert x^\beta$,
            if $x^\beta = x^\alpha \cdot x^\gamma$ for some
            $\gamma\in\mathbb{N}^n$.
        \end{definition}
        \begin{definition}
            A term is a monomial multiplied by a coefficient in $R$.
        \end{definition}
        \begin{definition}
            A polnyomial over $R$ is a finite $R-$linear
            combination of monomials,
            $f=\sum_{\alpha} a_{\alpha}\cdot x^{\alpha}$.
        \end{definition}
        \begin{notation}
            The set of all polynomials in $n$ variables over
            a ring $R$ is denoted $R[x_1,\hdots, x_n]$.
        \end{notation}
        \begin{theorem}
            If $R$ is a commutative ring with identity,
            then $R[x_1,\hdots, x_n]$ is a commutative
            ring with identity.
        \end{theorem}
        \begin{definition}
            A polynomial $f\in R[x_1,\hdots, x_n]$ is
            called a constant polynomial if $f\in R$.
        \end{definition}
        \begin{definition}
            A field $k$ is a commutative ring with identity
            such that for all $a\in k$, $a\ne 0$, there is a
            $b\in k$ such that $a\cdot b=1$
        \end{definition}
        \begin{remark}
            We usually work with fields and consider
            polynomial rings of the form $k[x_1,\hdots ,x_n]$.
        \end{remark}
        \begin{definition}
            A total ordering on a set $A$ is a relation
            $>$ such that $\forall_{a,b\in A}$, precisely one
            of the following truee:
            \begin{enumerate}
                \begin{multicols}{3}
                    \item $a<b$
                    \item $a=b$
                    \item $b<a$
                \end{multicols}
            \end{enumerate}
        \end{definition}
        \begin{definition}
            A relation $\sim$ on a set $A$ is said to be
            transitive if for all $a,b,c\in A$, if $a\sim b$ and
            $b\sim c$, then $a\sim c$.
        \end{definition}
        \begin{definition}
            A well ordering on a set $A$ is a relation $<$
            such that for every subset $E\subset A$, there is an
            element $x\in E$ such that for all $y\in E$, $y\ne x$,
            we have $x<y$.
        \end{definition}
        \begin{remark}
            Equivalently, a well ordering on a set $A$
            is a relation $<$ such that for every monotonically
            decreasing sequence $\alpha_n$, there is an
            $N\in \mathbb{N}$ such that for all $n>N$,
            $\alpha_n = \alpha_N$. That is,
            decreasing sequences terminate.
        \end{remark}
        \begin{definition}
            A monomial ordering on $\mathbb{N}^n$ is a relation
            $>$ such that $>$ is total, transitive, well
            ordering. A well ordering on $k[x_1,\hdots ,x_n]$
            is a well ordering on
            $\alpha=(\alpha_1,\hdots,\alpha_n)\in\mathbb{N}^n$.
        \end{definition}
        \begin{definition}
            The lexicographic ordering on $\mathbb{N}^n$ is
            defined as
            $(\alpha_1,\hdots,\alpha_n)\underset{Lex}{>}%
             (\beta_1,\hdots,\beta_n)$
            if the left-most non-zero entry of
            $(\alpha_1-\beta_1,\hdots, \alpha_n-\beta_n)$
            is positive.
        \end{definition}
        \begin{theorem}
            The lexicographic ordering is a monomial ordering.
        \end{theorem}
        \begin{definition}
            The graded lexicographic ordering is defined as
            $(\alpha_1,\hdots,\alpha_n)\underset{GrLex}{>}%
             (\beta_1,\hdots, \beta_n)$
            if $|\alpha|>|\beta|$ or $|\alpha|=|\beta|$
            and $\alpha\underset{Lex}{>}\beta$.
        \end{definition}
        \begin{theorem}
            The graded lexicographic ordering is a monomial ordering.
        \end{theorem}
        \begin{theorem}[The Division Algorithm]
            If $f_1,\hdots, f_s\in k[x_1,\hdots ,x_n]$ are
            non-zero polynomials and if $>$ is a monomial ordering,
            then there are $r,q_1,\hdots, q_n\in k[x_1,\hdots ,x_n]$
            such that the following are true:
            \begin{enumerate}
                \item $f=q_{1}f_{1}+\hdots+q_{s}f_{s}+r$
                \item No term of $r$ is divisible by
                      any of $\LT(f_{1}),\hdots,\LT(f_{s})$.
                \item $\LT(f)=\max_{>}\{\LT(q_{i})%
                       \cdot\LT(f_i):q_i\ne{0}\}$
            \end{enumerate}
        \end{theorem}
        \begin{definition}
            An ideal
            $I=\langle{x}^{\alpha}:\alpha\in{A}\rangle%
              =\{\sum_{\alpha}h_{\alpha}x^\alpha,h_{\alpha}%
               \in k[x_1,\hdots ,x_n]\}$
            is called a monomial ideal.
        \end{definition}
        \begin{theorem}
            If $I=\langle{x}^\alpha:\alpha\in{A}\rangle$
            is a monomial ideal,
            $\beta\in\mathbb{N}^n$, then $x^\beta\in{I}$
            if and only if there is an $\alpha\in{A}$
            such that $x^{\alpha}$ divides $x^{\beta}$.
        \end{theorem}
        \begin{theorem}
            If $I$ is a monomial ideal,
            $f\in{k}[x_1,\hdots ,x_n]$,
            then the following are equivalent:
            \begin{enumerate}
                \item $f\in I$
                \item Every term of $f$ lies in $I$.
                \item $f$ is a $k-$linear combination of
                      monomials in $I$.
            \end{enumerate}
        \end{theorem}
        \begin{theorem}[Dickson's Lemma]
            Every monomial ideal of $k[x_{1},\hdots,x_{n}]$
            is finitely generated.
        \end{theorem}
        \begin{theorem}[Hilbert's Basis Theorem]
            Every ideal $I\subset{k}[x_{1},\hdots,x_{n}]$
            is finitely generated.
        \end{theorem}
        \begin{definition}
            If $>$ is a monomial ordering on $k[x_{1},\hdots,x_{n}]$,
            then a Groebner Basis of $I\subset k[x_{1},\hdots,x_{n}]$
            is a set $G=\{g_{1},\hdots,g_{s}\}$ such that
            $\langle\LT(I)\rangle%
             =\langle \LT(g_1),\hdots,\LT(g_s)\rangle$
        \end{definition}
        \begin{theorem}
            Every non-zero ideal
            $I\subset{k}[x_{1},\hdots,x_{n}]$
            has a Groebner Basis.
        \end{theorem}
    \subsection{Elimination Theory}
        \begin{definition}
            If $I\subset{k}[x_1,\hdots,x_{n}]$ is an ideal,
            then the $i^{th}$ elimination ideal of $I$,
            denoted $I_{i}$, is the set
            $I_{i}=I\cap{k}[x_{i+1},\hdots,x_{n}]$,
            where $1\leq{i}\leq{n}$, and $I_{0}=I$.
        \end{definition}
        \begin{theorem}[The Elimination Theorem]
            If $I\subset k[x_1,\hdots ,x_n]$ is an ideal and
            $G$ is a Groebner Basis of $I$ with respect to the
            lexicographic ordering, and $x_1>\hdots > x_n$, then
            for all $i=0,1,\hdots,n$, the set
            $G_{i}\cap{k}[x_1,\hdots,x_n]$ is a Groebner Basis of
            the $i^{th}$ elimination ideal $I_{i}$.
        \end{theorem}
        \begin{remark}
            Using the lexicographic ordering, and for some ideal
            $I=\langle{f_{1}},\hdots,f_{s}\rangle%
               \subset{k}[x_1,\hdots ,x_n]$,
            to compute all elimination ideals $I_{i}$:
            \begin{enumerate}
                \item Compute a Groebner Basis $G$ for $I$ with
                      respect to the lex order on $k[x_1,\hdots,x_n]$.
                \item For all $i$, the elements $g\in G$ with
                      $\LT(g)\in{k}[x_{i+1},\hdots,x_{n}]$ form a
                      Groebner basis $I_{i}$ with respect to
                      the lexicographic ordering on
                      $k[x_{i+1},\hdots,x_n]$.
            \end{enumerate}
        \end{remark}
        \begin{definition}
            A monomial order on
            $k[x_{1},\hdots,x_{n},y_{1},\hdots,y_{m}]$
            is an elimination order with respect to
            $x_{1},\hdots,x_{n}$ if the following holds for
            all $f\in{k}[x_{1},\hdots,x_{n},y_{1},\hdots,y_{m}]$:
            $L(f)\in{k}[y_{1},\hdots,y_{m}]%
             \Rightarrow{f}\in{k}[y_{1},\hdots,y_{m}]$
        \end{definition}
        \begin{theorem}[The Extension Theorem]
            If $k$ is an algebraically closed field,
            $I=\langle{f_{1}},\hdots,f_{s}\rangle$, $I_{1}$ is
            the first elimination ideal of $I$, and if
            $f_{i}=g_{i}(x_{2},\hdots,x_{n})x_{1}^{N_i}+r_{i}$,
            where $r_{i}$ contains only terms where the degree
            of $x_{1}$ is less than $N_{i}$, and if
            $(a_{2},\hdots,a_{n})\in{k}^{n-1}$ such that
            $(a_{2},\hdots,a_{n})\notin%
             \mathbf{V}(g_{1},\hdots,g_{s})$,
             then there is an $a_1 \in k$ such that
             $(a_{1},\hdots,a_{n})\in\mathbf{V}(I_1)$.
        \end{theorem}
        \begin{definition}
            The $k^{th}$ projection map on $k^{n}$ is
            $\pi_{k}:k^{n}\rightarrow k^{n-k}$ defined by
            $(a_{1},\hdots,a_{n})=(a_{k+1},\hdots,a_{n})$
        \end{definition}
        \begin{remark}
            If $I\subset{k}[x_{1},\hdots,x_{n}]$ is an ideal,
            $X=\mathbf{V}(I)$, and $f\in{I_{k}}$,
            then $f(X)=0$.
            Thus $f\big(\pi_{k}(X)\big)=0$, and therefore
            $\pi_{k}(X)\subset\mathbf{V}(I_k)$.
            Also $\pi_{k}(X)$ may NOT be Zariski closed.
        \end{remark}
        \begin{theorem}
            If $k$ is algebraically closed,
            then $\overline{\pi_k(X)}=\mathbf{V}(I_k)$.
        \end{theorem}
        \begin{theorem}
            If $k$ is an infinite field,
            $F:k^{m}\rightarrow{k^{n}}$ a function determined
            by some parametrization
            $x_{j}=f_{j}(t_{1},\hdots,t_{m})$, and if
            $I=\langle{x_{1}-f_{1}},\hdots,x_{n}-f_{n}\rangle$,
            then $\mathbf{V}(I_m)$ is the smallest algebraic
            set in $k^{n}$ containing $F(k^{m})$.
        \end{theorem}
        \begin{remark}
            $V(I_{m})$ is the Zariski closure of $F(k^{m})$.
        \end{remark}
    \subsection{\'{E}tale Cohomology}
        \subsubsection{Review of Schemes}
            \begin{remark}
                Limitations of Affine Varieties:
                \begin{enumerate}
                    \item One would like to construct spaces
                          by gluing together simpler pieces,
                          like in geometry and topology.
                    \item Difficult over non-algebraically
                          closed fields.
                    \item Keeping track of multiplicities.
                \end{enumerate}
            \end{remark}
            Grothendieck's Theory of Schemes gives solutions to
            these problems. Should $x^{2}+y^{2}=-1$ and
            $x^{2}+y^{2}=3$ be regarded as the same over
            $\mathbb{A}_{\mathbb{R}}^{2}$? They both have no
            solution. The answer is no. An isomorphism should
            be given by an invertible transformation. In general,
            the affine variety $X\subset\mathbb{A}_{R}^n$ is
            completely determined by the coordinate ring
            $S=\mathcal{O}(X)%
              =R[x_{1},\hdots,x_{n}]/(f_{1},\hdots,f_{N})$.
            Given a compact Hausdorff space $X$, let $C(X)$
            denote the set of continous complex valued functions.
            This is a commutative ring with identity. With the
            supremum norm, it becomes a unital $C^{*}-$algebra.
            \begin{theorem}
                The map $X\rightarrow\max\{C(X)\}$
                is a homeomorphism.
            \end{theorem}
            Given a continuous map of spaces $f:X\rightarrow Y$,
            we get a homomorphism $C(Y)\rightarrow C(X)$ given
            by $g\mapsto{g}\circ f$. Thus $C(X)$ can be
            regarded as a contravariant functor. 
            \begin{theorem}[Gelfand]
                The functor $X\mapsto C(X)$ induces an
                equivalence between the category of compact
                Hausdorff spaces and the opposite category of
                commutative unital $C^{*}-$algebras.
            \end{theorem}
            \begin{definition}
                The spectrum of $R$, denoted $\Spec(R)$,
                is the set of prime ideals of $R$.
            \end{definition}
            \begin{theorem}
                The Zariski topology on $\Spec(R)$
                contains open sets
                $D(f)=\{p\in\Spec(R):f\notin{p}\}$
            \end{theorem}
            A function $f:\mathbb{R}^{n}\rightarrow\mathbb{R}$
            is $C^{\infty}$ if and only if its restriction to
            the neighborhood of every point is $C^{\infty}$.
            That is, $f\in{C}^{\infty}(X)$ if and only if
            for any open cover
            $\{U_{i}\},f\big|_{U_{i}}\in{C}^{\infty}(U_i)$.
            \begin{definition}
                If $X$ is a topological space, a presheaf
                of sets $\mathcal{F}$ is a collection of
                sets $\mathcal{F}(U)$ for each open set
                $U\subset X$ together with maps
                $\rho_{UV}:\mathcal{F}(U)\rightarrow \mathcal{F}(V)$
                for each pair $U\subset V$ such that $\rho_{UU}=id$
                and $\rho_{WV}\circ\rho_{VU}=\rho_{WU}$
                whenever $U\subset{V}\subset{W}$.
            \end{definition}
            \begin{definition}
                A sheaf is a presheaf such that for any open
                cover $\{U_i\}$ of an open $U\subset{X}$ and
                section $f_{i}\in\mathcal{F}(U_{i})$ such that
                $F_{i}\big|_{U_{i}\cap I_{j}}%
                 =f_{j}\big|_{U_{i}\cap U_{j}}$,
                there is a unique $f\in\mathcal{F}(U)$
                such that $f_{i}=f\big|_{U_{i}}$.
            \end{definition}
            \begin{definition}
                A ringed space is a pair $(X,\mathcal{O}_{X})$,
                where $X$ is a topological space and
                $\mathcal{O}_{X}$ is a sheaf of commutative rings.
            \end{definition}
            The collection of presheaves of a topological space
            form a category, denoted $Sh(X)$. 
            \begin{definition}
                A scheme is a ringed space $(X,\mathcal{O}_X)$
                which is locally an affine space.
            \end{definition}
            \begin{theorem}
                A property of commutative rings extends
                to schemes if it is local.
            \end{theorem}
        \subsubsection{Differential Calculus of Schemes}
            \begin{definition}
                The tangent space of an affine variety
                $X=V(f_{1},\hdots,f_{N})\subset\mathbb{A}_{k}^{n}$,
                denoted $T_{X,p}$, is the set of points
                $v\in{k}^{n}$ such that
                $\sum\frac{\partial{f_{j}}}%
                          {\partial{x_{i}}}p)v_{i})%
                 =0$
            \end{definition}
            \begin{definition}
                A domain
                $R=k[x_{1},\hdots,x_{n}]/(f_{1},\hdots,f_{N})$
                or $\Spec(R)$ is smooth if and only if the rank of
                $\big(\frac{\partial{f_{j}}}{\partial{x_{i}}}(p)\big)$
                is $n=\dim(R)$ for all $p\in\max(R)$. 
            \end{definition}
            \begin{definition}
                An \'{e}tale $R$ algebra is smooth of relative
                dimension 0, where
                $\det(\frac{\partial{f_{i}}}{\partial{x_{j}}})$
                is a unit in $S$.
            \end{definition}
            \begin{theorem}
                If $k$ is a field, then an algebra over $k$
                is \'{e}tale if and only if it is a finite Cartesian
                product of separable field extensions. 
            \end{theorem}
            \begin{theorem}
                The tensor product of two
                \'{e}tale algebras is \'{e}tale.
            \end{theorem}
            \begin{theorem}
                If $S$ is \'{e}tale over $R$ and $T$ is
                \'{e}tale over $S$, then $T$ is \'{e}tale over $R$.
            \end{theorem}
            \begin{definition}
                If $R$ is a commutative ring and $S$ is an
                $R$ algebra and $M$ is an $S$ module,
                then an $R$ linear derivation from $S$ to $M$ is
                a map $\delta:S\rightarrow M$ such that
                $\delta(s_{1}+s_{2})=\delta(s_{1})+\delta(s_{2})$,
                $\delta(s_{1}s_{2})%
                 =s_{1}\delta(s_{2})+s_{2}\delta(s_{1})$,
                 and $\delta(r)=0$ for all $r\in{R}$.
            \end{definition}
            \begin{theorem}
                There exists an $S$ module $\Omega_{S/R}$
                with a universal $R$ linear derivation
                $d:S\rightarrow \Omega_{S/R}$.
            \end{theorem}
            \begin{theorem}
                If $M$ is a finitely generated module over a
                Noetherian ring $R$, then these are equivalent:
                \begin{enumerate}
                    \begin{multicols}{3}
                        \item $M$ is locally free.
                        \item $\forall_{p\in\Spec(R)},R_{p}\otimes M$
                              is free.
                        \item $M$ is projective.
                    \end{multicols}
                \end{enumerate}
            \end{theorem}
            \begin{definition}
                If $M$ is an $S-$module, then $M$ is called
                flat if $M\otimes{i}$ is injective for any $i$.
            \end{definition}
            \begin{theorem}
                If $S$ is an $R$ algebra and $M$ is an $S$ module,
                $f\in S$ is an element such that multiplication
                by $f$ is injective on $M\otimes k(m)$ for all
                $m\in\max(R)$, and if $M$ is flat over $R$,
                then $M/fM$ is flat over $R$.
            \end{theorem}
            \begin{theorem}
                A smooth algebra is flat.
            \end{theorem}
            \begin{theorem}
                If $R$ is a Noetherian ring, then a homomorphism
                $R\rightarrow S$ is \'{e}tale if and only if:
                \begin{enumerate}
                    \item $S$ is finitely generated as an algebra.
                    \item $S$ is flat as an $R$-module.
                    \item $\Omega_{S/R}=0$.
                \end{enumerate}
            \end{theorem}
            \begin{definition}
                A sheaf on a scheme is quasi-coherent if
                it is with respect to some affine open cover.
            \end{definition}
            \begin{theorem}
                If $f:X\rightarrow Y$ is a morphism, there
                exists a quasi-coherent sheaf $\Omega_{X/Y}$
                such that
                $\Omega_{X/Y}\big|_{\Spec(S_{ij})}=\Omega_{S_{ij}/R_i}$
                for open affine covers $\Spec(R_i)=U_{i}$.
            \end{theorem}
        \subsubsection{The Fundamental \'{E}tale Group}
            \begin{definition}
                A topological group is a topological space
                $(X,\tau)$ with a group structure $(X,*)$ such
                that $*:X\times X\rightarrow X$ is a continuous
                function with respect to the product topology.
            \end{definition}
            \begin{theorem}
                A topological space is profinite if and only
                if it is compact Hausdorff and totally disconnected.
            \end{theorem}
            \begin{definition}
                The topological fundamental group of a
                topological space $X$, denoted $\pi_1(X)$,
                is the group of homotopy classes of loops in $X$
                with a given base point.
            \end{definition}
            \begin{theorem}
                Any \'{e}tale morphism $Y\rightarrow X$ is
                a finite to one covering space of $X$ with
                the usual topology.
            \end{theorem}
            \begin{theorem}[Grothendieck's Theorem]
                If $X$ is a scheme of finite type of
                $\mathbb{C}$, then $\pi_{1}^{et}(X)$ is the
                profinite completion of $\pi_{1}{X}$.
            \end{theorem}
        \subsubsection{\'{E}tale Topology}
            \begin{remark}
                Given a topological space $(X,\tau)$,
                the topology $\tau$
                (That is, the collection of open sets) forms a
                partially ordered set with respect to set
                inclusion. There also exists a notion of
                open covering $U=\cup U_{i}$.
            \end{remark}
            \begin{definition}
                A Groethendieck Topology on a category $C$ with
                fibre products is a collection of families of
                morphisms $U_{i}\rightarrow U$ such that:
                \begin{enumerate}
                    \item The family consisting of a single
                          isomorphism $\{U\sim{U}\}$ is a covering.
                    \item If $\{U_{i}\rightarrow{U}\}$ and
                          $\{V_{ij}\rightarrow{U_{i}}\}$ are coverings,
                          then so is the composition
                          $\{V_{ij}\rightarrow{U}\}$.
                \end{enumerate}
            \end{definition}
            \begin{definition}
                A site is a category with a Grothendieck Topology.
            \end{definition}
    \subsection{The Zariski Topology}
        \subsubsection{The Zariski Topology}
            \begin{definition}
                A subset of $k^{n}$ is is closed in the
                Zariski Topology if it is an algebraic set.
                The Zariski Topology is formed by
                considering all such sets.
            \end{definition}
            \begin{definition}
                A topological space $X$ is called irreducible if
                for any closed subsets $X_1,X_2\subset X$ such that
                $X=X_{1}\cup{X_{2}}$, either $X=X_{1}$ or
                $X=X_{2}$. A topological space that is
                not irreducible is called reducible.
            \end{definition}
            \begin{definition}
                A subset $Y\subset X$ of a topological space
                is said to be irreducible if $Y$ is irreducible
                with respect to the inherited,
                or the induced topology.
            \end{definition}
            \begin{definition}
                A topological space $X$ is said to be disconnected
                if there are two non-empty closed subsets $X_1,X_2$
                such that $X=X_1\cup X_2$,
                and $X_1\cap X_2 = \emptyset$.
            \end{definition}
            \begin{theorem}
                If $X$ is disconnected, then it is reducible.
            \end{theorem}
            \begin{proof}
                For if $X$ is disconnected, there are two
                non-empty closed sets $X_1,X_2\subset X$ such
                that $X_1\cap X_2 = \emptyset$ and
                $X=X_1\cup X_2$. But if $X_1$ and $X_2$
                are non-empty and disjoint, then
                $X_1\ne X$ and $X_2 \ne X$.
                Therefore $X$ is reducible.
            \end{proof}
            \begin{definition}
                An algebraic affine variety is an
                irreducible closed subset of $k^n$.
            \end{definition}
            \begin{definition}
                An open subset of an affine variety
                is called a quasi-affine variety.
            \end{definition}
            \begin{definition}
                If $X\subset k^n$ is an algebraic set,
                then $k[x_1,\hdots ,x_n]/\mathbb{I}(X)$
                is called the coordinate ring of $X$.
            \end{definition}
            \begin{definition}
                A set $Y$ in a topological space $X$ is
                said to be dense in $X$ if for every
                non-empty open set $\mathcal{O}$,
                $\mathcal{O}\cap Y\ne \emptyset$.
            \end{definition}
            \begin{theorem}
                A topological space $X$ is irreducible if
                and only if every non-empty open set is dense.
            \end{theorem}
            \begin{definition}
                An irreducible component of $X$ is a
                maximal irreducible subset of $X$.
            \end{definition}
            \begin{theorem}
                If $X$ is a closed topological space,
                then any irreducible subset $Y\subset X$ is
                contained in a maximal component.
            \end{theorem}
            \begin{theorem}
                If $X$ is a topological space,
                then it is the union of irreducible components.
            \end{theorem}
            \begin{definition}
                A topological space $X$ is called Noetherian
                if every descending chain $X_n \subset X_{n+1}$
                of closed subsets stabilizes.
            \end{definition}
            \begin{theorem}
                If $X$ is a Noetherian Space,
                then every subset $Y\subset X$ can be
                written as a finite union of irreducible
                closed subsets.
            \end{theorem}
            \begin{theorem}
                Every algebraic set in $k^n$ can be expressed
                uniquely as a union of varieties.
            \end{theorem}
            \begin{theorem}
                If $R$ is an Noetherian ring,
                then $k[x_1,\hdots ,x_n]$ is Noetherian.
            \end{theorem}
            \begin{theorem}
                A ring $R$ is Noetherian if and only if every
                non-empty set of ideals in $R$ has a maximal element.
            \end{theorem}
            \begin{theorem}[Hilbert's Nullstellensats]
                If $k$ is an algebraically closed field,
                $I\subset R = k[x_1,\hdots ,x_n]$ is an ideal,
                and $f\in R$ is a polynomial which vanishes on
                $\mathbf{V}(I)$, then there is an $n\in \mathbb{N}$
                such that $f^{n}\in{I}$.
            \end{theorem}
            \begin{definition}
                The dimension of a topological space $X$ is the
                supremum of all $n\in \mathbb{N}$ such that
                there is a chain
                $Z_0\subset Z_1\subset\hdots\subset Z_n$
                of distinct irreducible closed
                subsets of $X$.
            \end{definition}
            \begin{theorem}
                If $k$ is a field, and $B$ is an integral domain
                which is finitely generated by a $k-$algebra,
                then the dimension of $B$ is equal to the
                transcendence degree of the quotient field $k(B)$
                of $B$ over $k$.
            \end{theorem}
            \begin{theorem}
                The dimension of $k^{n}$ is $n$.
            \end{theorem}
            \begin{theorem}
                If $Y$ is a quasi-affine variety,
                then $\dim(Y)=\dim(\overline{Y})$.
            \end{theorem}
        \subsubsection{Problems}
            \begin{problem}
                Let $f\in k[x]$ be a non-constant polynomial
                in one variable over a field $k$. $f$ is called
                irreducible if $f\notin k$ and if it is not
                the product of two polynomials of strictly smaller
                degree. Prove the following are equivalent:
                \begin{enumerate}
                    \item $k[x]/\langle f\rangle$ is a field.
                    \item $k[x]/\langle f\rangle$ is an
                          integral domain.
                    \item $f$ is irreducible.
                \end{enumerate}
            \end{problem}
            \begin{proof}[Solution]
                If $k[x]/\langle f\rangle$ is a field,
                then it is an integral domain. If $f$ is
                irreducible, then $\langle f\rangle$ is
                maximal and thus $k[x]/\langle f\rangle$ is
                a field. Finally, if $k[x]/\langle f\rangle$ is
                an integral domain, then $\langle f\rangle$ is
                prime. But if $\langle f\rangle$ is prime,
                then it is maximal. And if $\langle f\rangle$
                is maximal, then $f$ is irreducible. 
            \end{proof}
            \begin{problem}
                Show every prime ideal is radical.
            \end{problem}
            \begin{proof}[Solution]
                Let $I$ be a prime ideal. Then if $fg\in I$,
                either $f\in I$ or $g\in I$. Suppose $f^n \in I$
                for some $f\in R$. Then $f^{n-1}f \in R$.
                But then either $f^{n-1} \in I$ or $f\in I$.
                If $f\in I$, we are done. If not, by induction
                $f^{n-k} \in I$ and we obtain $f\in I$.
            \end{proof}
            \begin{problem}
                Show that any Noetherian
                Topological Space $X$ is compact.
            \end{problem}
            \begin{proof}[Solution]
                If $X$ is Noetherian, then every ascending
                chain terminates. Suppose $X$ is not compact.
                Then there is an open cover $\Delta$ with no
                finite subcover. Let $\mathcal{O}_1$ be a finite
                subcover. Then
                $\cup_{\mathcal{U}\in \mathcal{O}_1}\mathcal{U}$
                is not all of $X$, otherwise $X$ would be compact.
                Thus there is an open subcover $\mathcal{O}_2$
                such that $\mathcal{O}_1 \subset \mathcal{O}_2$.
                Inductively, we have a sequence
                $\mathcal{O}_n\subset \mathcal{O}_{n+1}$. Let
                $A_{n}=\cup_{k=1}^{n}\cup_{\mathcal{U}\in \mathcal{O}_k}\mathcal{U}$.
                Then $A_{n}\subset A_{n+1}$.
                But by the Noetherian property,
                this chain must stabilize.
                But then there is an $N\in \mathbb{N}$
                such that $\mathcal{O}_{N+1}=\mathcal{O}_N$,
                a contradiction as we said $X$ is not compact.
                Therefore, etc.
            \end{proof}
            \begin{remark}
                This proof subtly requires the axiom of
                choice in the construction of such
                $\mathcal{O}'s$.
            \end{remark}
    \subsection{Notes on Varieties}
        \subsubsection{Affine Varieties}
            Let $k$ denote an algebraically closed field.
            $\textbf{A}_{k}^n$ is the affine $k-$space in
            $n$ dimensions. An element $a=(a_1,\hdots, a_n)$
            is called a point, and $a_i$ is called a coordinate.
            \begin{definition}
                The zero set of a set of polynomials
                $T=\{f_{1},\hdots,f_{s}\}$ is the set
                $Z(T)%
                 =\{p\in\textbf{A}_{k}^{n}|f_{i}(p)=0,%
                    i=1,\hdots,s\}$.
            \end{definition}
            \begin{notation}
                The set of polynomials in $n$ variables
                over $\textbf{A}_{k}^{n}$ is denoted $A$.
            \end{notation}
            \begin{definition}
                A subset $Y\subset\textbf{A}_{k}^{n}$ is an
                algebraic set if there exists a subset
                $T\subset{A}$ such that $Z(T)=Y$.
            \end{definition}
            \begin{theorem}
                The union of two algebraic
                sets is algebraic.
            \end{theorem}
            \begin{theorem}
                The intersection of two algebraic
                sets is algebraic.
            \end{theorem}
            \begin{definition}
                The Zariski topology $\mathcal{Z}$ on
                $\textbf{A}_{k}^{n}$ is the set of compliments
                of algebraic sets. That is,
                algebraic sets are closed.
            \end{definition}
            \begin{definition}
                A non-empty subset $Y$ of a topological space
                $X$ is irreducible if it cannot be expressed
                as the union $Y={Y_{1}}\cup{Y_{2}}$ of
                two proper subsets, each on of which is
                closed in $Y$.
            \end{definition}
            \begin{definition}
                An affine algebraic variety is an irreducible
                subset of $\textbf{A}_{k}^{n}$ with respect
                to the induced topology.
            \end{definition}
            \begin{definition}
                An open subset of an affine variety
                is called a quasi-affine variety.
            \end{definition}
            \begin{notation}
                If $Y\subset\textbf{A}_{k}^{n}$,
                $I(Y)=\{f\in A:\forall_{p\in Y},f(p)=0\}$.
            \end{notation}
            \begin{theorem}
                \
                \begin{enumerate}
                    \item If $T_1\subset T_2$,
                          the $Z(T_2)\subset{Z}(T_1)$
                    \item If
                          $Y_{1}\subset{Y_{2}}\subset%
                          \textbf{A}_{k}^{n}$,
                          then $I(Y_{2})\subset{I}(Y_{1})$
                    \item $I(Y_{1}\cup{Y_{2}})%
                           =I(Y_{1})\cap{I}(Y_{2})$
                    \item If $a\subset A$,
                          then $I(Z(a))=\sqrt{a}$
                          (The radical of $a$)
                    \item If $Y\subset\textbf{A}_{k}^{n}$,
                          then $Z(I(Y))=\overline{Y}$
                          (The closure of $Y$)
                \end{enumerate}
            \end{theorem}
            \begin{theorem}[Hilbert's Nullstellensatz]
                If $k$ is an algebraically closed field,
                $a\subset{A}=k[x_{1},\hdots,x_{n}]$
                is an ideal, and if $f\in{A}$ is a polynomial
                which vanishes on $Z(a)$, then there is an
                $r\in\mathbb{N}$ such that $f^{r}\in{a}$.
            \end{theorem}
            \begin{definition}
                The affine coordinate ring of an affine
                algebraic set $Y\subset\textbf{A}_{k}^{n}$
                is $A/I(Y)$.
            \end{definition}
            \begin{definition}
                A topological space $X$ is called Noetherian
                if it satisfies the descending chain condition
                for closed subsets.
            \end{definition}
            \begin{theorem}
                A Noetherian Topological Space is compact.
            \end{theorem}
            \begin{definition}
                If $A$ is a ring, then height of a prime
                ideal $p$ is the supremum of all integers $n$
                such that there is a chain
                $p_{0}\subset\hdots\subset{p_{n}}=p$
                of distinct prime ideals.
            \end{definition}
            \begin{definition}
                The Krull dimension of a ring $A$ is the
                supremum of the height of all ideals.
            \end{definition}
            \begin{theorem}[Krull's Hauptidealsatz]
                If $A$ is a Noetherian Ring, and $f\in A$
                has neither a zero divisor nor a unit,
                then every minimal prime ideal $p$
                containing $f$ has height $1$.
            \end{theorem}
            \begin{theorem}
                The dimension of $\textbf{A}_{k}^{n}$ is $n$.
            \end{theorem}
        \subsubsection{Projective Varieties}
            \begin{definition}
                A subset $Y$ of $P^n$ is an algebraic
                set if there is a set $T$ of homogeneous
                elements of $S$ such that $Y=Z(T)$.
            \end{definition}
            \begin{definition}
                The Zariski Topology on $P^n$ is defined
                as the complements of algebraic sets.
                That is, algebraic sets are closed.
            \end{definition}
            \begin{definition}
                A projective algebraic variety is an
                irreducible algebraic set in $P^{n}$.
            \end{definition}
        \subsubsection{More Notes on Projective Varieties}
            \begin{definition}
                The projective $n-$space over $\mathbb{A}$,
                denoted $\mathbb{P}^{n}$, is the set of all
                one-dimensional linear subspaces of the vector
                space $\mathbb{A}^{n+1}$.
            \end{definition}
            \begin{remark}
                Equivalently, it is the set of all lines
                in $\mathbb{A}^{n+1}$ through the origin.
            \end{remark}
            \begin{definition}
                The projective $n$ space $\mathbb{P}^{n}$ over $k$
                is the set of all equivalence classes
                $\mathbb{A}^{n+1}/\{0\}$, where
                $(a_{1},\hdots,a_{n})\sim(b_{1},\hdots,b_{n})$
                if and only if there is a
                $\lambda\in\mathbb{A}\setminus\{0\}$
                such that $b_{i}=\lambda{a_{i}}$.
            \end{definition}
            \begin{remark}
                Elements of $\mathbb{P}^{n}$ are called points.
            \end{remark}
            \begin{definition}
                A homogenous polynomial of degree $d$
                is a polynomial $f$ such that
                $f(\lambda a_1,\hdots,\lambda a_n)%
                 =\lambda^d f(a_1,\hdots, a_n)$.
            \end{definition}
            \begin{theorem}
                If $I\subset k[x_1,\hdots ,x_n]$ is an ideal,
                then the following are equivalent:
                \begin{enumerate}
                    \item $I$ can be generated by
                          homogeneous polynomials.
                    \item For every $f\in I$, the degree
                          $d$ part of $f$ in contained in $I$
                \end{enumerate}
            \end{theorem}
            \begin{definition}
                If $I\subset k[x_1,\hdots ,x_n]$ is a
                homogeneous ideal, then
                $\mathbf{V}(I)%
                 =\{(a_1:\hdots:a_{n})\in\mathbb{P}^{n}:%
                 f(a_{1},\hdots,a_{n})=0,f\in I\}$.
            \end{definition}
            \begin{definition}
                An algebraic subset of $\mathbb{P}^{n}$ is a
                set of the form $\mathbf{V}(I)$.
                These are called projective algebraic sets.
            \end{definition}
            \begin{theorem}
                Every projective algebraic set can be
                written as the zero set of finitely many
                homogeneous polynomials of the same degree.
            \end{theorem}
            \begin{definition}
                The projective close of and algebraic set
                $X\subset\mathbb{A}^n$ is the Zariski closure
                in $\mathbb{P}^{n}$ under the mapping
                $\mathbb{A}^{n}\rightarrow\mathbb{P}^n$
                by $(x_{1},\hdots,x_{n})\mapsto(1:x_1,\hdots, x_n)$.
            \end{definition}
            \begin{theorem}
                If $f$ is the sum of forms $f=\sum_{d}f^{(d)}$,
                if $P\in \mathbb{P}^n$ and $f(x_1,\hdots, x_n)=0$
                for every choice of homogeneous coordinates,
                then for each $d$, $f^{(d)}(x_1,\hdots, x_n)=0$.
            \end{theorem}
            \begin{definition}
                If $F\in \mathbb{A}[x_1,\hdots, x_n]$ is homogeneous
                of degree $d$, then its de-homogenization is the
                polynomial $f(x_1,\hdots, x_n)=F(1,x_1,\hdots, x_n)$.
            \end{definition}
            \begin{theorem}
                Let $X\subset \mathbb{A}^n$ be an affine
                algebraic set, $\overline{X}$ the projective closure. Then
                $\mathbb{I}(\overline{X})\subset\mathbb{A}[x_1,\hdots,x_n]$
                is generated by the homogenization of all
                elements of $\mathbb{I}(X)$.
            \end{theorem}
            \begin{theorem}
                An algebraic set $X$ is irreducible
                if and only if the ideal $\mathbb{I}(X)$ is prime.
            \end{theorem}
            \begin{definition}
                An affine algebraic set $X\subset \mathbb{A}^{n+1}$
                is called a cone if it is not empty, and if for all
                $\lambda\in{k}$,
                $(x_1,\hdots, x_n)%
                 \in{X}\Rightarrow(\lambda{x_{1}},\hdots,\lambda{x_{n}})%
                 \in{X}$.
            \end{definition}
            \begin{theorem}[The Projective Nullstellensatz]
                \
                \begin{enumerate}
                    \item If $X_1\subset X_2$ are algebraic
                          set in $\mathbb{P}^{n}$,
                          then $I(X_{2})\subset{I}(X_{1})$.
                    \item For any algebraic set
                          $X\subset\mathbb{P}^{n}$, we have
                          $\mathbf{V}(I(X))=X$.
                    \item For any homogeneous ideal
                          $I\subset{k}[x_{1},\hdots,x_{n}]$ such
                          that $\mathbf{V}(I)\ne\emptyset$,
                          we have
                          $\mathbb{I}(\mathbf{V}(I))=\sqrt{I}$.
                \end{enumerate}
            \end{theorem}
\end{document}