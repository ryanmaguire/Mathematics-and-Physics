\documentclass[crop=false,class=article,oneside]{standalone}
%----------------------------Preamble-------------------------------%
%---------------------------Packages----------------------------%
\usepackage{geometry}
\geometry{b5paper, margin=1.0in}
\usepackage[T1]{fontenc}
\usepackage{graphicx, float}            % Graphics/Images.
\usepackage{natbib}                     % For bibliographies.
\bibliographystyle{agsm}                % Bibliography style.
\usepackage[french, english]{babel}     % Language typesetting.
\usepackage[dvipsnames]{xcolor}         % Color names.
\usepackage{listings, lstlinebgrd}      % Verbatim-Like Tools.
\usepackage{mathtools, esint, mathrsfs} % amsmath and integrals.
\usepackage{amsthm, amsfonts}           % Fonts and theorems.
\usepackage{tabularx}
\usepackage{tcolorbox}                  % Frames around theorems.
\usepackage{upgreek}                    % Non-Italic Greek.
\usepackage{paracol}                    % Two-column styling.
\usepackage{wrapfig}                    % Wrap text around figure.
\usepackage{fmtcount, etoolbox}         % For the \book{} command.
\usepackage[newparttoc]{titlesec}       % Formatting chapter, etc.
\usepackage{titletoc}                   % Allows \book in toc.
\usepackage[nottoc]{tocbibind}          % Bibliography in toc.
\usepackage[titles]{tocloft}            % ToC formatting.
\usepackage{multicol, enumitem}         % Multi-column/enumerate.
\usepackage{import}                     % Import external files.
\usepackage{pgfplots, tikz}             % Drawing/graphing tools.
\usetikzlibrary{
    calc,                   % Calculating right angles and more.
    angles,                 % Drawing angles within triangles.
    arrows.meta,            % Latex and Stealth arrows.
    quotes,                 % Adding labels to angles.
    positioning,            % Relative positioning of nodes.
    decorations.markings,   % Adding arrows in the middle of a line.
    patterns,
    arrows,
    shapes,
    shapes.geometric,
    cd,
    hobby,
    babel
}                                       % Libraries for tikz.
\pgfplotsset{compat=1.9}                % Version of pgfplots.
\usepackage[font=scriptsize,
            labelformat=simple,
            labelsep=colon]{subcaption} % Subfigure captions.
\usepackage[font={scriptsize},
            hypcap=true,
            labelsep=colon]{caption}    % Figure captions.
\usepackage{hyperref}                   % Allows for hyperlinks.
\hypersetup{
    colorlinks=true,
    linkcolor=blue,
    filecolor=magenta,
    urlcolor=Cerulean,
    citecolor=SkyBlue
}                           % Colors for hyperref.
\usepackage[toc,acronym,nogroupskip]{glossaries} % Glossaries and acronyms.
\usepackage[subpreambles=false]{standalone}      % Complileable sub files.

% Various font stuff from kiwi.
% Use this for Times text and Computer Modern math
%\usepackage{times}

% Quite nice
%\usepackage[charter, greekfamily=, greekuppercase=italicized]{mathdesign}
%\usepackage[utopia, greekuppercase=italicized]{mathdesign}    % Math is narrower

% Use this for Times text and math
%\usepackage{newtxtext}
%\usepackage[libertine,cmintegrals]{newtxmath}
%\usepackage{fix-cm}

%\usepackage{txfontsb}
% or
%\usepackage{mathptmx}

%\usepackage[scaled=0.92]{helvet}
%\renewcommand{\rmdefault}{ptm}

%\usepackage{mathpazo}    % add possibly `sc` and `osf` options
%\usepackage{eulervm}

%\usepackage{fourier}
%\renewcommand{\rmdefault}{ptm}
%\usepackage{mathptm}

%\usepackage{fontspec}
%\setmainfont{lmodern}

%\usepackage[varg]{txfonts}
%\usepackage{fouriernc}
%\usepackage{mathpazo}

%\usepackage{bookman}
%\usepackage[scaled]{uarial}
%\usepackage[scaled]{helvet}
%\renewcommand*\familydefault{\sfdefault}
%\usepackage[math]{anttor}

%\newcommand\fgeorgia{\fontfamily{jvn}\selectfont}
%\newcommand\ftimes{\fontfamily{ptm}\selectfont}
%\newcommand\fhelvetica{\fontfamily{phv}\selectfont}
%\newcommand\fcourier{\fontfamily{pcr}\selectfont}
%\newcommand\fbookman{\fontfamily{pbk}\selectfont}
%\newcommand\fnewcentury{\fontfamily{pnc}\selectfont}
%\newcommand\fpalatino{\fontfamily{ppl}\selectfont}
%\newcommand\favantgarde{\fontfamily{pag}\selectfont}
%\newcommand\fnormal{\normalfont}
%\newcommand\fsize[1]{\ifnum#1>0\fontsize{#1}{#1}\selectfont\else\normalsize\fi}
%------------------------Theorem Styles-------------------------%
% Define theorem style for default spacing and normal font.
\newtheoremstyle{normal}
    {\topsep}               % Amount of space above the theorem.
    {\topsep}               % Amount of space below the theorem.
    {}                      % Font used for body of theorem.
    {}                      % Measure of space to indent.
    {\bfseries}             % Font of the header of the theorem.
    {}                      % Punctuation between head and body.
    {.5em}                  % Space after theorem head.
    {}

% Define theorem style for default spacing with italicized font.
\newtheoremstyle{normalit}{\topsep}{\topsep}
                {\itshape}{}{\bfseries}{}{.5em}{}

% Italic header environment.
\newtheoremstyle{thmit}{\topsep}{\topsep}{}{}{\itshape}{}{0.5em}{}

% Define italicized environments.
\theoremstyle{normalit}
\newtheorem{theorem}{Theorem}[section]
\newtheorem{lemma}{Lemma}[section]
\newtheorem{corollary}{Corollary}[section]
\newtheorem{proposition}{Proposition}[section]
\newtheorem*{theorem*}{Theorem}

% Define environments with italic headers.
\theoremstyle{thmit}
\newtheorem*{solution}{Solution}
\newtheorem*{fsolution}{Solution}

% Define default environments.
\theoremstyle{normal}
\newtheorem{example}{Example}[section]
\newtheorem{definition}{Definition}[section]
\newtheorem{problem}{Problem}[section]
\newtheorem{question}{Question}[section]
\newtheorem{remark}{Remark}[section]
\newtheorem{properties}{Properties}[section]
\newtheorem{notation}{Notation}[section]
\newtheorem{axiom}{Axiom}[section]
\newtheorem*{properties*}{Properties}
\newtheorem*{remark*}{Remark}
\newtheorem*{definition*}{Definition}
\theoremstyle{plain}

% Define framed environment.
\tcbuselibrary{most}
\newtcbtheorem[use counter*=theorem]{ftheorem}{Theorem}%
    {colback=green!5,colframe=green!35!black,
     fonttitle=\bfseries\upshape}{th}

\newtcbtheorem[use counter*=example]{fdefinition}{Definition}%
    {fonttitle=\bfseries\upshape,
     colback=blue!5!white,colframe=blue!75!black}{def}

\newtcbtheorem[use counter*=example]{fexample}{Example}%
    {fonttitle=\bfseries\upshape,
     colback=red!5!white,colframe=red!75!black}{ex}

\newtcbtheorem[use counter*=notation]{fnotation}{Notation}%
    {fonttitle=\bfseries\upshape,
     colback=SeaGreen!5!white,colframe=SeaGreen!75!black}{ex}

\newtcbtheorem[use counter*=corollary]{fcorollary}{Corollary}%
    {fonttitle=\bfseries\upshape,
     colback=Orchid!5!white,colframe=Orchid!75!black}{ex}

\newenvironment{bproof}{\textit{Proof.}}{\hfill$\square$}
\tcolorboxenvironment{bproof}{blanker,breakable,left=5mm,
                             before skip=10pt,after skip=10pt,
                             borderline west={1mm}{0pt}{red}}
\tcolorboxenvironment{fsolution}
    {enhanced jigsaw,colframe=cyan,interior hidden,breakable}

%--------------------Declared Math Operators--------------------%
\DeclareMathOperator{\Refl}{Refl}           % Reflection operator.
\DeclareMathOperator{\Span}{Span}           % Span of a set of vectors.
\DeclareMathOperator{\Card}{Card}           % Cardinality of set.
\DeclareMathOperator{\Ord}{Ord}             % Ordinal of ordered set.
\DeclareMathOperator{\Tr}{Tr}               % Trace of matrix.
\DeclareMathOperator{\adjoint}{adj}         % Adjoint of matrix.
\DeclareMathOperator{\rk}{rk}               % Rank of operator.
\DeclareMathOperator{\nul}{nul}             % Null space of operator.
\DeclareMathOperator{\sgn}{sgn}             % Sign of a number.
\DeclareMathOperator{\multideg}{mutlideg}   % Multi-Degree (Graphs).
\DeclareMathOperator{\GCD}{GCD}             % Greatest common denominator.
\DeclareMathOperator{\LM}{LM}               % Leading monomial
\DeclareMathOperator{\LC}{LC}               % Leading coefficient.
\DeclareMathOperator{\LT}{LT}               % Leading term.
\DeclareMathOperator{\LCM}{LCM}             % Least common multiple.
\DeclareMathOperator{\Mon}{Mon}             % Monomial.
\DeclareMathOperator{\Spec}{Spec}           % Spectrum.
\DeclareMathOperator{\proj}{proj}           % Projection.
\DeclareMathOperator{\comp}{comp}           % Component.
\DeclareMathOperator{\sinc}{sinc}           % Sinc function.
\DeclareMathOperator{\Ima}{Im}              % Image of operator.
\DeclareMathOperator{\Prin}{Prin}           % Principal value.
\DeclareMathOperator{\Mod}{mod}             % Modulus.
%------------------------New Commands---------------------------%
\DeclarePairedDelimiter\norm{\lVert}{\rVert}
\DeclarePairedDelimiter\ceil{\lceil}{\rceil}
\DeclarePairedDelimiter\floor{\lfloor}{\rfloor}
\newcommand*\diff{\mathop{}\!\mathrm{d}}
\newcommand*\Diff[1]{\mathop{}\!\mathrm{d^#1}}
\renewcommand{\mod}{\ \Mod}
\renewcommand*{\glstextformat}[1]{\textcolor{RoyalBlue}{#1}}
\renewcommand{\glsnamefont}[1]{\textbf{#1}}
\renewcommand\labelitemii{$\circ$}
\renewcommand\thesubfigure{\arabic{chapter}.\arabic{figure}}
\renewcommand\thesubfigure{%
    \arabic{chapter}.\arabic{figure}.\arabic{subfigure}}
\addto\captionsenglish{\renewcommand{\figurename}{Fig.}}
%------------------------Book Command---------------------------%
\makeatletter
\renewcommand\@pnumwidth{1cm}
\newcounter{book}
\renewcommand\thebook{\@Roman\c@book}
\newcommand\book{%
    \if@openright
        \cleardoublepage
    \else
        \clearpage
    \fi
    \thispagestyle{plain}%
    \if@twocolumn
        \onecolumn
        \@tempswatrue
    \else
        \@tempswafalse
    \fi
    \null\vfil
    \secdef\@book\@sbook
}
\def\@book[#1]#2{%
    \ifnum \c@secnumdepth >-3\relax
        \refstepcounter{book}%
        \addcontentsline{toc}{book}{
            \bookname\ \thebook:\hspace{1em}#1
        }
    \else
        \addcontentsline{toc}{book}{#1}%
    \fi
    \markboth{}{}%
    {\centering
     \interlinepenalty \@M
     \normalfont
     \ifnum \c@secnumdepth >-2\relax
       \huge\bfseries \bookname\nobreakspace\thebook
       \par
       \vskip 20\p@
     \fi
     \Huge \bfseries #2\par}%
    \@endbook}
\def\@sbook#1{%
    {\centering
     \interlinepenalty \@M
     \normalfont
     \Huge \bfseries #1\par}%
    \@endbook}
\def\@endbook{
    \vfil\newpage
        \if@twoside
            \if@openright
                \null
                \thispagestyle{empty}%
                \newpage
            \fi
        \fi
        \if@tempswa
            \twocolumn
        \fi
}
\newcommand*\l@book[2]{%
    \ifnum \c@tocdepth >-2\relax
        \addpenalty{-\@highpenalty}%
        \addvspace{2.25em \@plus\p@}%
        \setlength\@tempdima{3em}%
        \begingroup
            \parindent \z@ \rightskip \@pnumwidth
            \parfillskip -\@pnumwidth
            {
                \leavevmode
                \Large \bfseries #1\hfil \hb@xt@\@pnumwidth{
                    \hss #2
                }
            }
            \par
            \nobreak
            \global\@nobreaktrue
            \everypar{\global\@nobreakfalse\everypar{}}%
        \endgroup
    \fi}
\newcommand\bookname{Book}
\renewcommand{\thebook}{\texorpdfstring{\Numberstring{book}}{book}}
\providecommand*{\toclevel@book}{-2}
\makeatother
\titlecontents{chapter}[0pt]
    {\bfseries}
    {\chaptername\ \thecontentslabel:\quad}
    {}
    {\hfill\contentspage}
\titleformat{\part}[display]
    {\Large\bfseries}
    {\partname\nobreakspace\thepart}
    {0mm}
    {\Huge\bfseries}
    \titlecontents{part}[0pt]
    {\large\bfseries}
    {\partname\ \thecontentslabel: \quad}
    {}
    {\hfill\contentspage}
\newcommand{\MarkRightAngle}[4][.3cm]
    {\coordinate (tempa) at ($(#3)!#1!(#2)$);
     \coordinate (tempb) at ($(#3)!#1!(#4)$);
     \coordinate (tempc) at ($(tempa)!0.5!(tempb)$);%midpoint
     \draw (tempa) -- ($(#3)!2!(tempc)$) -- (tempb);}
%--------------------------LENGTHS------------------------------%
% Spacings for the Table of Contents.
\addtolength{\cftsecnumwidth}{1ex}
\addtolength{\cftsubsecindent}{1ex}
\addtolength{\cftsubsecnumwidth}{1ex}
\addtolength{\cftfignumwidth}{1ex}
\addtolength{\cfttabnumwidth}{1ex}

% Spacing for multi-column and enumerate environments.
\setlength{\multicolsep}{6pt}
\setlist[enumerate]{itemsep=0pt,topsep=3pt}

% Indent and paragraph spacing.
\setlength{\parindent}{0em}
\setlength{\parskip}{0em}
%--------------------------Main Document----------------------------%
\begin{document}
    \ifx\ifmathcoursesalgebraicgeometry\undefined
        \section*{Algebraic Geometry}
        \setcounter{section}{1}
    \fi
    \subsection{Notes on Varieties}
        \subsubsection{Affine Varieties}
            Let $k$ denote an algebraically closed field.
            $\textbf{A}_{k}^n$ is the affine $k-$space in
            $n$ dimensions. An element $a=(a_1,\hdots, a_n)$
            is called a point, and $a_i$ is called a coordinate.
            \begin{definition}
                The zero set of a set of polynomials
                $T=\{f_{1},\hdots,f_{s}\}$ is the set
                $Z(T)%
                 =\{p\in\textbf{A}_{k}^{n}|f_{i}(p)=0,%
                    i=1,\hdots,s\}$.
            \end{definition}
            \begin{notation}
                The set of polynomials in $n$ variables
                over $\textbf{A}_{k}^{n}$ is denoted $A$.
            \end{notation}
            \begin{definition}
                A subset $Y\subset\textbf{A}_{k}^{n}$ is an
                algebraic set if there exists a subset
                $T\subset{A}$ such that $Z(T)=Y$.
            \end{definition}
            \begin{theorem}
                The union of two algebraic
                sets is algebraic.
            \end{theorem}
            \begin{theorem}
                The intersection of two algebraic
                sets is algebraic.
            \end{theorem}
            \begin{definition}
                The Zariski topology $\mathcal{Z}$ on
                $\textbf{A}_{k}^{n}$ is the set of compliments
                of algebraic sets. That is,
                algebraic sets are closed.
            \end{definition}
            \begin{definition}
                A non-empty subset $Y$ of a topological space
                $X$ is irreducible if it cannot be expressed
                as the union $Y={Y_{1}}\cup{Y_{2}}$ of
                two proper subsets, each on of which is
                closed in $Y$.
            \end{definition}
            \begin{definition}
                An affine algebraic variety is an irreducible
                subset of $\textbf{A}_{k}^{n}$ with respect
                to the induced topology.
            \end{definition}
            \begin{definition}
                An open subset of an affine variety
                is called a quasi-affine variety.
            \end{definition}
            \begin{notation}
                If $Y\subset\textbf{A}_{k}^{n}$,
                $I(Y)=\{f\in A:\forall_{p\in Y},f(p)=0\}$.
            \end{notation}
            \begin{theorem}
                \
                \begin{enumerate}
                    \item If $T_1\subset T_2$,
                          the $Z(T_2)\subset{Z}(T_1)$
                    \item If
                          $Y_{1}\subset{Y_{2}}\subset%
                          \textbf{A}_{k}^{n}$,
                          then $I(Y_{2})\subset{I}(Y_{1})$
                    \item $I(Y_{1}\cup{Y_{2}})%
                           =I(Y_{1})\cap{I}(Y_{2})$
                    \item If $a\subset A$,
                          then $I(Z(a))=\sqrt{a}$
                          (The radical of $a$)
                    \item If $Y\subset\textbf{A}_{k}^{n}$,
                          then $Z(I(Y))=\overline{Y}$
                          (The closure of $Y$)
                \end{enumerate}
            \end{theorem}
            \begin{theorem}[Hilbert's Nullstellensatz]
                If $k$ is an algebraically closed field,
                $a\subset{A}=k[x_{1},\hdots,x_{n}]$
                is an ideal, and if $f\in{A}$ is a polynomial
                which vanishes on $Z(a)$, then there is an
                $r\in\mathbb{N}$ such that $f^{r}\in{a}$.
            \end{theorem}
            \begin{definition}
                The affine coordinate ring of an affine
                algebraic set $Y\subset\textbf{A}_{k}^{n}$
                is $A/I(Y)$.
            \end{definition}
            \begin{definition}
                A topological space $X$ is called Noetherian
                if it satisfies the descending chain condition
                for closed subsets.
            \end{definition}
            \begin{theorem}
                A Noetherian Topological Space is compact.
            \end{theorem}
            \begin{definition}
                If $A$ is a ring, then height of a prime
                ideal $p$ is the supremum of all integers $n$
                such that there is a chain
                $p_{0}\subset\hdots\subset{p_{n}}=p$
                of distinct prime ideals.
            \end{definition}
            \begin{definition}
                The Krull dimension of a ring $A$ is the
                supremum of the height of all ideals.
            \end{definition}
            \begin{theorem}[Krull's Hauptidealsatz]
                If $A$ is a Noetherian Ring, and $f\in A$
                has neither a zero divisor nor a unit,
                then every minimal prime ideal $p$
                containing $f$ has height $1$.
            \end{theorem}
            \begin{theorem}
                The dimension of $\textbf{A}_{k}^{n}$ is $n$.
            \end{theorem}
        \subsubsection{Projective Varieties}
            \begin{definition}
                A subset $Y$ of $P^n$ is an algebraic
                set if there is a set $T$ of homogeneous
                elements of $S$ such that $Y=Z(T)$.
            \end{definition}
            \begin{definition}
                The Zariski Topology on $P^n$ is defined
                as the complements of algebraic sets.
                That is, algebraic sets are closed.
            \end{definition}
            \begin{definition}
                A projective algebraic variety is an
                irreducible algebraic set in $P^{n}$.
            \end{definition}
        \subsubsection{More Notes on Projective Varieties}
            \begin{definition}
                The projective $n-$space over $\mathbb{A}$,
                denoted $\mathbb{P}^{n}$, is the set of all
                one-dimensional linear subspaces of the vector
                space $\mathbb{A}^{n+1}$.
            \end{definition}
            \begin{remark}
                Equivalently, it is the set of all lines
                in $\mathbb{A}^{n+1}$ through the origin.
            \end{remark}
            \begin{definition}
                The projective $n$ space $\mathbb{P}^{n}$ over $k$
                is the set of all equivalence classes
                $\mathbb{A}^{n+1}/\{0\}$, where
                $(a_{1},\hdots,a_{n})\sim(b_{1},\hdots,b_{n})$
                if and only if there is a
                $\lambda\in\mathbb{A}\setminus\{0\}$
                such that $b_{i}=\lambda{a_{i}}$.
            \end{definition}
            \begin{remark}
                Elements of $\mathbb{P}^{n}$ are called points.
            \end{remark}
            \begin{definition}
                A homogenous polynomial of degree $d$
                is a polynomial $f$ such that
                $f(\lambda a_1,\hdots,\lambda a_n)%
                 =\lambda^d f(a_1,\hdots, a_n)$.
            \end{definition}
            \begin{theorem}
                If $I\subset k[x_1,\hdots ,x_n]$ is an ideal,
                then the following are equivalent:
                \begin{enumerate}
                    \item $I$ can be generated by
                          homogeneous polynomials.
                    \item For every $f\in I$, the degree
                          $d$ part of $f$ in contained in $I$
                \end{enumerate}
            \end{theorem}
            \begin{definition}
                If $I\subset k[x_1,\hdots ,x_n]$ is a
                homogeneous ideal, then
                $\mathbf{V}(I)%
                 =\{(a_1:\hdots:a_{n})\in\mathbb{P}^{n}:%
                 f(a_{1},\hdots,a_{n})=0,f\in I\}$.
            \end{definition}
            \begin{definition}
                An algebraic subset of $\mathbb{P}^{n}$ is a
                set of the form $\mathbf{V}(I)$.
                These are called projective algebraic sets.
            \end{definition}
            \begin{theorem}
                Every projective algebraic set can be
                written as the zero set of finitely many
                homogeneous polynomials of the same degree.
            \end{theorem}
            \begin{definition}
                The projective close of and algebraic set
                $X\subset\mathbb{A}^n$ is the Zariski closure
                in $\mathbb{P}^{n}$ under the mapping
                $\mathbb{A}^{n}\rightarrow\mathbb{P}^n$
                by $(x_{1},\hdots,x_{n})\mapsto(1:x_1,\hdots, x_n)$.
            \end{definition}
            \begin{theorem}
                If $f$ is the sum of forms $f=\sum_{d}f^{(d)}$,
                if $P\in \mathbb{P}^n$ and $f(x_1,\hdots, x_n)=0$
                for every choice of homogeneous coordinates,
                then for each $d$, $f^{(d)}(x_1,\hdots, x_n)=0$.
            \end{theorem}
            \begin{definition}
                If $F\in \mathbb{A}[x_1,\hdots, x_n]$ is homogeneous
                of degree $d$, then its de-homogenization is the
                polynomial $f(x_1,\hdots, x_n)=F(1,x_1,\hdots, x_n)$.
            \end{definition}
            \begin{theorem}
                Let $X\subset \mathbb{A}^n$ be an affine
                algebraic set, $\overline{X}$ the projective closure. Then
                $\mathbb{I}(\overline{X})\subset\mathbb{A}[x_1,\hdots,x_n]$
                is generated by the homogenization of all
                elements of $\mathbb{I}(X)$.
            \end{theorem}
            \begin{theorem}
                An algebraic set $X$ is irreducible
                if and only if the ideal $\mathbb{I}(X)$ is prime.
            \end{theorem}
            \begin{definition}
                An affine algebraic set $X\subset \mathbb{A}^{n+1}$
                is called a cone if it is not empty, and if for all
                $\lambda\in{k}$,
                $(x_1,\hdots, x_n)%
                 \in{X}\Rightarrow(\lambda{x_{1}},\hdots,\lambda{x_{n}})%
                 \in{X}$.
            \end{definition}
            \begin{theorem}[The Projective Nullstellensatz]
                \
                \begin{enumerate}
                    \item If $X_1\subset X_2$ are algebraic
                          set in $\mathbb{P}^{n}$,
                          then $I(X_{2})\subset{I}(X_{1})$.
                    \item For any algebraic set
                          $X\subset\mathbb{P}^{n}$, we have
                          $\mathbf{V}(I(X))=X$.
                    \item For any homogeneous ideal
                          $I\subset{k}[x_{1},\hdots,x_{n}]$ such
                          that $\mathbf{V}(I)\ne\emptyset$,
                          we have
                          $\mathbb{I}(\mathbf{V}(I))=\sqrt{I}$.
                \end{enumerate}
            \end{theorem}
\end{document}