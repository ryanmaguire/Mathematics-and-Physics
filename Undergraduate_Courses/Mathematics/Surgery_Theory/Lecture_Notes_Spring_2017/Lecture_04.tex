\documentclass[crop=false,class=article,oneside]{standalone}
%----------------------------Preamble-------------------------------%
%---------------------------Packages----------------------------%
\usepackage{geometry}
\geometry{b5paper, margin=1.0in}
\usepackage[T1]{fontenc}
\usepackage{graphicx, float}            % Graphics/Images.
\usepackage{natbib}                     % For bibliographies.
\bibliographystyle{agsm}                % Bibliography style.
\usepackage[french, english]{babel}     % Language typesetting.
\usepackage[dvipsnames]{xcolor}         % Color names.
\usepackage{listings}                   % Verbatim-Like Tools.
\usepackage{mathtools, esint, mathrsfs} % amsmath and integrals.
\usepackage{amsthm, amsfonts, amssymb}  % Fonts and theorems.
\usepackage{tcolorbox}                  % Frames around theorems.
\usepackage{upgreek}                    % Non-Italic Greek.
\usepackage{fmtcount, etoolbox}         % For the \book{} command.
\usepackage[newparttoc]{titlesec}       % Formatting chapter, etc.
\usepackage{titletoc}                   % Allows \book in toc.
\usepackage[nottoc]{tocbibind}          % Bibliography in toc.
\usepackage[titles]{tocloft}            % ToC formatting.
\usepackage{pgfplots, tikz}             % Drawing/graphing tools.
\usepackage{imakeidx}                   % Used for index.
\usetikzlibrary{
    calc,                   % Calculating right angles and more.
    angles,                 % Drawing angles within triangles.
    arrows.meta,            % Latex and Stealth arrows.
    quotes,                 % Adding labels to angles.
    positioning,            % Relative positioning of nodes.
    decorations.markings,   % Adding arrows in the middle of a line.
    patterns,
    arrows
}                                       % Libraries for tikz.
\pgfplotsset{compat=1.9}                % Version of pgfplots.
\usepackage[font=scriptsize,
            labelformat=simple,
            labelsep=colon]{subcaption} % Subfigure captions.
\usepackage[font={scriptsize},
            hypcap=true,
            labelsep=colon]{caption}    % Figure captions.
\usepackage[pdftex,
            pdfauthor={Ryan Maguire},
            pdftitle={Mathematics and Physics},
            pdfsubject={Mathematics, Physics, Science},
            pdfkeywords={Mathematics, Physics, Computer Science, Biology},
            pdfproducer={LaTeX},
            pdfcreator={pdflatex}]{hyperref}
\hypersetup{
    colorlinks=true,
    linkcolor=blue,
    filecolor=magenta,
    urlcolor=Cerulean,
    citecolor=SkyBlue
}                           % Colors for hyperref.
\usepackage[toc,acronym,nogroupskip,nopostdot]{glossaries}
\usepackage{glossary-mcols}
%------------------------Theorem Styles-------------------------%
\theoremstyle{plain}
\newtheorem{theorem}{Theorem}[section]

% Define theorem style for default spacing and normal font.
\newtheoremstyle{normal}
    {\topsep}               % Amount of space above the theorem.
    {\topsep}               % Amount of space below the theorem.
    {}                      % Font used for body of theorem.
    {}                      % Measure of space to indent.
    {\bfseries}             % Font of the header of the theorem.
    {}                      % Punctuation between head and body.
    {.5em}                  % Space after theorem head.
    {}

% Italic header environment.
\newtheoremstyle{thmit}{\topsep}{\topsep}{}{}{\itshape}{}{0.5em}{}

% Define environments with italic headers.
\theoremstyle{thmit}
\newtheorem*{solution}{Solution}

% Define default environments.
\theoremstyle{normal}
\newtheorem{example}{Example}[section]
\newtheorem{definition}{Definition}[section]
\newtheorem{problem}{Problem}[section]

% Define framed environment.
\tcbuselibrary{most}
\newtcbtheorem[use counter*=theorem]{ftheorem}{Theorem}{%
    before=\par\vspace{2ex},
    boxsep=0.5\topsep,
    after=\par\vspace{2ex},
    colback=green!5,
    colframe=green!35!black,
    fonttitle=\bfseries\upshape%
}{thm}

\newtcbtheorem[auto counter, number within=section]{faxiom}{Axiom}{%
    before=\par\vspace{2ex},
    boxsep=0.5\topsep,
    after=\par\vspace{2ex},
    colback=Apricot!5,
    colframe=Apricot!35!black,
    fonttitle=\bfseries\upshape%
}{ax}

\newtcbtheorem[use counter*=definition]{fdefinition}{Definition}{%
    before=\par\vspace{2ex},
    boxsep=0.5\topsep,
    after=\par\vspace{2ex},
    colback=blue!5!white,
    colframe=blue!75!black,
    fonttitle=\bfseries\upshape%
}{def}

\newtcbtheorem[use counter*=example]{fexample}{Example}{%
    before=\par\vspace{2ex},
    boxsep=0.5\topsep,
    after=\par\vspace{2ex},
    colback=red!5!white,
    colframe=red!75!black,
    fonttitle=\bfseries\upshape%
}{ex}

\newtcbtheorem[auto counter, number within=section]{fnotation}{Notation}{%
    before=\par\vspace{2ex},
    boxsep=0.5\topsep,
    after=\par\vspace{2ex},
    colback=SeaGreen!5!white,
    colframe=SeaGreen!75!black,
    fonttitle=\bfseries\upshape%
}{not}

\newtcbtheorem[use counter*=remark]{fremark}{Remark}{%
    fonttitle=\bfseries\upshape,
    colback=Goldenrod!5!white,
    colframe=Goldenrod!75!black}{ex}

\newenvironment{bproof}{\textit{Proof.}}{\hfill$\square$}
\tcolorboxenvironment{bproof}{%
    blanker,
    breakable,
    left=3mm,
    before skip=5pt,
    after skip=10pt,
    borderline west={0.6mm}{0pt}{green!80!black}
}

\AtEndEnvironment{lexample}{$\hfill\textcolor{red}{\blacksquare}$}
\newtcbtheorem[use counter*=example]{lexample}{Example}{%
    empty,
    title={Example~\theexample},
    boxed title style={%
        empty,
        size=minimal,
        toprule=2pt,
        top=0.5\topsep,
    },
    coltitle=red,
    fonttitle=\bfseries,
    parbox=false,
    boxsep=0pt,
    before=\par\vspace{2ex},
    left=0pt,
    right=0pt,
    top=3ex,
    bottom=1ex,
    before=\par\vspace{2ex},
    after=\par\vspace{2ex},
    breakable,
    pad at break*=0mm,
    vfill before first,
    overlay unbroken={%
        \draw[red, line width=2pt]
            ([yshift=-1.2ex]title.south-|frame.west) to
            ([yshift=-1.2ex]title.south-|frame.east);
        },
    overlay first={%
        \draw[red, line width=2pt]
            ([yshift=-1.2ex]title.south-|frame.west) to
            ([yshift=-1.2ex]title.south-|frame.east);
    },
}{ex}

\AtEndEnvironment{ldefinition}{$\hfill\textcolor{Blue}{\blacksquare}$}
\newtcbtheorem[use counter*=definition]{ldefinition}{Definition}{%
    empty,
    title={Definition~\thedefinition:~{#1}},
    boxed title style={%
        empty,
        size=minimal,
        toprule=2pt,
        top=0.5\topsep,
    },
    coltitle=Blue,
    fonttitle=\bfseries,
    parbox=false,
    boxsep=0pt,
    before=\par\vspace{2ex},
    left=0pt,
    right=0pt,
    top=3ex,
    bottom=0pt,
    before=\par\vspace{2ex},
    after=\par\vspace{1ex},
    breakable,
    pad at break*=0mm,
    vfill before first,
    overlay unbroken={%
        \draw[Blue, line width=2pt]
            ([yshift=-1.2ex]title.south-|frame.west) to
            ([yshift=-1.2ex]title.south-|frame.east);
        },
    overlay first={%
        \draw[Blue, line width=2pt]
            ([yshift=-1.2ex]title.south-|frame.west) to
            ([yshift=-1.2ex]title.south-|frame.east);
    },
}{def}

\AtEndEnvironment{ltheorem}{$\hfill\textcolor{Green}{\blacksquare}$}
\newtcbtheorem[use counter*=theorem]{ltheorem}{Theorem}{%
    empty,
    title={Theorem~\thetheorem:~{#1}},
    boxed title style={%
        empty,
        size=minimal,
        toprule=2pt,
        top=0.5\topsep,
    },
    coltitle=Green,
    fonttitle=\bfseries,
    parbox=false,
    boxsep=0pt,
    before=\par\vspace{2ex},
    left=0pt,
    right=0pt,
    top=3ex,
    bottom=-1.5ex,
    breakable,
    pad at break*=0mm,
    vfill before first,
    overlay unbroken={%
        \draw[Green, line width=2pt]
            ([yshift=-1.2ex]title.south-|frame.west) to
            ([yshift=-1.2ex]title.south-|frame.east);},
    overlay first={%
        \draw[Green, line width=2pt]
            ([yshift=-1.2ex]title.south-|frame.west) to
            ([yshift=-1.2ex]title.south-|frame.east);
    }
}{thm}

%--------------------Declared Math Operators--------------------%
\DeclareMathOperator{\adjoint}{adj}         % Adjoint.
\DeclareMathOperator{\Card}{Card}           % Cardinality.
\DeclareMathOperator{\curl}{curl}           % Curl.
\DeclareMathOperator{\diam}{diam}           % Diameter.
\DeclareMathOperator{\dist}{dist}           % Distance.
\DeclareMathOperator{\Div}{div}             % Divergence.
\DeclareMathOperator{\Erf}{Erf}             % Error Function.
\DeclareMathOperator{\Erfc}{Erfc}           % Complementary Error Function.
\DeclareMathOperator{\Ext}{Ext}             % Exterior.
\DeclareMathOperator{\GCD}{GCD}             % Greatest common denominator.
\DeclareMathOperator{\grad}{grad}           % Gradient
\DeclareMathOperator{\Ima}{Im}              % Image.
\DeclareMathOperator{\Int}{Int}             % Interior.
\DeclareMathOperator{\LC}{LC}               % Leading coefficient.
\DeclareMathOperator{\LCM}{LCM}             % Least common multiple.
\DeclareMathOperator{\LM}{LM}               % Leading monomial.
\DeclareMathOperator{\LT}{LT}               % Leading term.
\DeclareMathOperator{\Mod}{mod}             % Modulus.
\DeclareMathOperator{\Mon}{Mon}             % Monomial.
\DeclareMathOperator{\multideg}{mutlideg}   % Multi-Degree (Graphs).
\DeclareMathOperator{\nul}{nul}             % Null space of operator.
\DeclareMathOperator{\Ord}{Ord}             % Ordinal of ordered set.
\DeclareMathOperator{\Prin}{Prin}           % Principal value.
\DeclareMathOperator{\proj}{proj}           % Projection.
\DeclareMathOperator{\Refl}{Refl}           % Reflection operator.
\DeclareMathOperator{\rk}{rk}               % Rank of operator.
\DeclareMathOperator{\sgn}{sgn}             % Sign of a number.
\DeclareMathOperator{\sinc}{sinc}           % Sinc function.
\DeclareMathOperator{\Span}{Span}           % Span of a set.
\DeclareMathOperator{\Spec}{Spec}           % Spectrum.
\DeclareMathOperator{\supp}{supp}           % Support
\DeclareMathOperator{\Tr}{Tr}               % Trace of matrix.
%--------------------Declared Math Symbols--------------------%
\DeclareMathSymbol{\minus}{\mathbin}{AMSa}{"39} % Unary minus sign.
%------------------------New Commands---------------------------%
\DeclarePairedDelimiter\norm{\lVert}{\rVert}
\DeclarePairedDelimiter\ceil{\lceil}{\rceil}
\DeclarePairedDelimiter\floor{\lfloor}{\rfloor}
\newcommand*\diff{\mathop{}\!\mathrm{d}}
\newcommand*\Diff[1]{\mathop{}\!\mathrm{d^#1}}
\renewcommand*{\glstextformat}[1]{\textcolor{RoyalBlue}{#1}}
\renewcommand{\glsnamefont}[1]{\textbf{#1}}
\renewcommand\labelitemii{$\circ$}
\renewcommand\thesubfigure{%
    \arabic{chapter}.\arabic{figure}.\arabic{subfigure}}
\addto\captionsenglish{\renewcommand{\figurename}{Fig.}}
\numberwithin{equation}{section}

\renewcommand{\vector}[1]{\boldsymbol{\mathrm{#1}}}

\newcommand{\uvector}[1]{\boldsymbol{\hat{\mathrm{#1}}}}
\newcommand{\topspace}[2][]{(#2,\tau_{#1})}
\newcommand{\measurespace}[2][]{(#2,\varSigma_{#1},\mu_{#1})}
\newcommand{\measurablespace}[2][]{(#2,\varSigma_{#1})}
\newcommand{\manifold}[2][]{(#2,\tau_{#1},\mathcal{A}_{#1})}
\newcommand{\tanspace}[2]{T_{#1}{#2}}
\newcommand{\cotanspace}[2]{T_{#1}^{*}{#2}}
\newcommand{\Ckspace}[3][\mathbb{R}]{C^{#2}(#3,#1)}
\newcommand{\funcspace}[2][\mathbb{R}]{\mathcal{F}(#2,#1)}
\newcommand{\smoothvecf}[1]{\mathfrak{X}(#1)}
\newcommand{\smoothonef}[1]{\mathfrak{X}^{*}(#1)}
\newcommand{\bracket}[2]{[#1,#2]}

%------------------------Book Command---------------------------%
\makeatletter
\renewcommand\@pnumwidth{1cm}
\newcounter{book}
\renewcommand\thebook{\@Roman\c@book}
\newcommand\book{%
    \if@openright
        \cleardoublepage
    \else
        \clearpage
    \fi
    \thispagestyle{plain}%
    \if@twocolumn
        \onecolumn
        \@tempswatrue
    \else
        \@tempswafalse
    \fi
    \null\vfil
    \secdef\@book\@sbook
}
\def\@book[#1]#2{%
    \refstepcounter{book}
    \addcontentsline{toc}{book}{\bookname\ \thebook:\hspace{1em}#1}
    \markboth{}{}
    {\centering
     \interlinepenalty\@M
     \normalfont
     \huge\bfseries\bookname\nobreakspace\thebook
     \par
     \vskip 20\p@
     \Huge\bfseries#2\par}%
    \@endbook}
\def\@sbook#1{%
    {\centering
     \interlinepenalty \@M
     \normalfont
     \Huge\bfseries#1\par}%
    \@endbook}
\def\@endbook{
    \vfil\newpage
        \if@twoside
            \if@openright
                \null
                \thispagestyle{empty}%
                \newpage
            \fi
        \fi
        \if@tempswa
            \twocolumn
        \fi
}
\newcommand*\l@book[2]{%
    \ifnum\c@tocdepth >-3\relax
        \addpenalty{-\@highpenalty}%
        \addvspace{2.25em\@plus\p@}%
        \setlength\@tempdima{3em}%
        \begingroup
            \parindent\z@\rightskip\@pnumwidth
            \parfillskip -\@pnumwidth
            {
                \leavevmode
                \Large\bfseries#1\hfill\hb@xt@\@pnumwidth{\hss#2}
            }
            \par
            \nobreak
            \global\@nobreaktrue
            \everypar{\global\@nobreakfalse\everypar{}}%
        \endgroup
    \fi}
\newcommand\bookname{Book}
\renewcommand{\thebook}{\texorpdfstring{\Numberstring{book}}{book}}
\providecommand*{\toclevel@book}{-2}
\makeatother
\titleformat{\part}[display]
    {\Large\bfseries}
    {\partname\nobreakspace\thepart}
    {0mm}
    {\Huge\bfseries}
\titlecontents{part}[0pt]
    {\large\bfseries}
    {\partname\ \thecontentslabel: \quad}
    {}
    {\hfill\contentspage}
\titlecontents{chapter}[0pt]
    {\bfseries}
    {\chaptername\ \thecontentslabel:\quad}
    {}
    {\hfill\contentspage}
\newglossarystyle{longpara}{%
    \setglossarystyle{long}%
    \renewenvironment{theglossary}{%
        \begin{longtable}[l]{{p{0.25\hsize}p{0.65\hsize}}}
    }{\end{longtable}}%
    \renewcommand{\glossentry}[2]{%
        \glstarget{##1}{\glossentryname{##1}}%
        &\glossentrydesc{##1}{~##2.}
        \tabularnewline%
        \tabularnewline
    }%
}
\newglossary[not-glg]{notation}{not-gls}{not-glo}{Notation}
\newcommand*{\newnotation}[4][]{%
    \newglossaryentry{#2}{type=notation, name={\textbf{#3}, },
                          text={#4}, description={#4},#1}%
}
%--------------------------LENGTHS------------------------------%
% Spacings for the Table of Contents.
\addtolength{\cftsecnumwidth}{1ex}
\addtolength{\cftsubsecindent}{1ex}
\addtolength{\cftsubsecnumwidth}{1ex}
\addtolength{\cftfignumwidth}{1ex}
\addtolength{\cfttabnumwidth}{1ex}

% Indent and paragraph spacing.
\setlength{\parindent}{0em}
\setlength{\parskip}{0em}
%--------------------------Main Document----------------------------%
\begin{document}
    \ifx\ifmathcoursessurgery\undefined
        \section*{Surgery Theory}
        \setcounter{section}{1}
        \renewcommand\thesubfigure{%
            \arabic{section}.\arabic{figure}.\arabic{subfigure}%
        }
    \fi
    \subsection{Lecture 4: Principal G-Bundles}
        A brief discussion on complexes. A simplex
        is a generalization of the notation of a triangle.
        A triangle can be considered as the
        convex-hull of $3$ non-coplanar points.
        This is called a $2$-simplex. A $0$-simplex
        is thus a point, and a $1$-simplex is a line.
        This can be generalized to higher dimensions.
        A $3$-simplex is a tetrahedron,
        and an $n$-simplex is an $n$ dimensional triangle,
        defined on $n+1$ non-hyper-coplanar points.
        \begin{figure}[H]
            \centering
            \captionsetup{type=figure}
            \subimport{../../../../tikz/}{Simplices}
            \caption{Examples of Simplices.}
            \label{fig:surgery_theory_simplexes}
        \end{figure}
        A simplicial complex is a set of simplices
        $\mathcal{H}$ such that the face of any element
        of $\mathcal{H}$ is also contained in $\mathcal{H}$,
        and the intersection of two simplices
        $\sigma_{1},\sigma_{2}\in \mathcal{K}$ is a
        face of both $\sigma_{1}$ and $\sigma_{2}$.
        We return to studying surgery exact sequences
        for $n\geq 5$. Let $\mathcal{M}$ be an $n$
        dimensional manifold, and let $G = \pi_{1}(\mathcal{M})$.
        In our surgery exact sequence we still have this
        mysterious object $[\mathcal{M},G/Cat]$. Let Cat
        be either PL or Top. The generalized Poincare
        Conjecture says that, for $n\geq 5$,
        $S^{PL}(S^{n})=S^{Top}(S^{n})=\{S^{n}\}$.
        Let $\mathcal{M}=S^{n}$. Then
        $G=\pi_{1}(\mathcal{M})=\{e\}$.
        Then we have the following:
        \begin{figure}[H]
            \centering
            \captionsetup{type=figure}
            \subimport{../../../../tikz/}{Surgery_Exact_Sequence}
            \caption{Diagram for the Surgery
                     Exact Sequence of $S^{5}$.}
            \label{fig:surgery_theory_example_diagram_%
                   for_surgery_exact_sequence}
        \end{figure}
        So, $\pi_{5}(G/Cat)=\{e\}$. This gives us:
        \begin{align*}
            \cdots\rightarrow
            S^{Cat}(S^{6})\rightarrow
            [S^{6},G/Cat]
            &\rightarrow
            L_{6}(\mathbb{Z})\rightarrow\cdots\\
            \cdots
            &\rightarrow{0}\rightarrow\pi_{6}(G/Cat)\rightarrow
            \mathbb{Z}_{2}\rightarrow{0}
        \end{align*}
        So, we have $\pi_{6}(G/Cat)\cong\mathbb{Z}_{2}$.
        In general, $\pi_{n}(G/o)\cong L_{n}(\mathbb{Z})$.
        \begin{theorem}[Wall's Theorem]
            \begin{equation*}
                L_{n}(\mathbb{Z})=
                \begin{cases}
                    \mathbb{Z},&n\equiv{0}\mod{4}\\
                    0,&n\equiv{1}\mod{4}\\
                    \mathbb{Z}_{2},&n\equiv{2}\mod{4}\\
                    0,&n\equiv{3}\mod{4}
                \end{cases}
            \end{equation*}
        \end{theorem}
        All $L$ groups are periodic, and never have odd
        torsion. That is, there is never
        $\mathbb{Z}_{3},\mathbb{Z}_{5}$, etc. Wall groups
        are hard to compute. Whatever $G/Cat$ is, its
        homotopy groups for $n\geq 5$ are known.
        \subsubsection{Principle G-Bundles}
            A few things are needed:
            \begin{itemize}
                \item Map $p:E\rightarrow X$,
                      where $E$ is a total space
                      and $X$ is a base space.
                \item The inverse-image $E_{x}=p^{-1}(\{x\})$
                      is called the fiber over $x$.
                \item $G$ (Group) acts on each $E_{x}$
                      freely and transitively.
                \item $G$ has to act `continuously.'
                      Nearby points are taken to nearby points.
            \end{itemize}
            Then $p:E\rightarrow X$ is a $G-$principle bundle.
            \begin{remark}
                Freely means the only element that
                fixes everything is the identity.
            \end{remark}
            \begin{example}
                Take a sphere $S^{n}$ and a projection
                $p:S^{n} \rightarrow \mathbb{RP}^{n}$.
                $\mathbb{RP}^{n}$ is created by glueing
                antipotal points together.
                If $x\in \mathbb{RP}^{n}$, then $p^{-1}(\{x\})$
                consists of $2$ antipotal points in $S^{n}$.
                Now $\mathbb{Z}_{2}=\{0,1\}$
                can act on a sphere.
                $0$ maps $x\mapsto{x}$ and $1$ maps
                $x\mapsto{-x}$.
                Note that $1+1 = 0$, as in $\mathbb{Z}_{2}$.
                Given any point, you can get to another point
                in the fiber. This is trivial in this example
                as there are only two points in the fiber.
                Also only the identity maps a point back
                to itself. This action is free and transitive,
                so $p:S^{n}\rightarrow\mathbb{RP}^{n}$
                is a $\mathbb{Z}_{2}$-principle bundle.
            \end{example}
            \begin{remark}
                Let $M$ be a manifold (Or a space)
                with dimension $n$ and fundamental group $G$.
                A universal cover $\tilde{M}$ of $M$ includes
                a map $p:\tilde{M}\rightarrow M$ such that
                $\tilde{M}$ is simply connected of dimension
                $n$, i.e. $\pi_{1}(\tilde{M})=e$,
                and $\forall_{x\in M}$,
                $p^{-1}(x)$ is a collection of discrete points.
            \end{remark}
            \begin{remark}
                $\pi_{1}(\mathbb{RP}^{n})=\mathbb{Z}_{2}$
                and $S^{n}$ is a universal cover of
                $\mathbb{RP}^{n}$. This might come
                from a general theory.
            \end{remark}
            \begin{example}
                Take the circle $S^{1}$.
                $\pi_{1}(S^{1})=\mathbb{Z}$.
                There's a map $p(x)=e^{2\pi{ix}}$ of modulus 1.
                Note that $p^{-1}(0) = \mathbb{Z}$.
                So $p^{-1}(x)$ is just a shift of
                $\mathbb{Z}$ to $\mathbb{Z}+r$.
                Note that $\pi_{1}(\mathbb{R})=e$.
                So $\mathbb{R}$ is a universal cover of $S^{1}$.
            \end{example}
            \begin{example}
                We may think $\mathbb{R}^2$ is a universal
                cover of $S^{2}$, but $S^{2}$ is already
                simply connected. So $p$ is the
                identity map, and the universal cover of $S^{2}$
                is $S^{2}$. All universal covers are
                homotopy equivalent.
            \end{example}
            \begin{remark}
                Let $x\in\mathcal{M}$. Then, for all
                $z\in p^{-1}(\{x\})$, and for all
                $g\in \pi_{1}(M)$, there is an action
                $gx\in p^{-1}(x)$. This uses the homotopy
                lifting property.
            \end{remark}
                There is an action $G$ on $\tilde{\mathcal{M}}$
                which preserves the fiber
                (Takes every element of a fiber to the
                same fiber. It does not mix fibers),
                is transitive, and is free.
                The map
                $p:\tilde{\mathcal{M}}\rightarrow\mathcal{M}$
                is a $\pi_{1}(M)$ Principal Bundle.
            \subsubsection{Functors}
                Let $F$ be a functor
                $F:\textit{Space}\rightarrow\textit{Groups}$.
                So for all spaces $X$, we have a group $F(X)$.
                There are many such examples:
                \begin{itemize}
                    \begin{multicols}{4}
                        \item Cohomology
                        \item Homology
                        \item K-Theory
                        \item Other Stuff
                    \end{multicols}
                \end{itemize}
                \begin{remark}
                    Homology: Take $M$ and triangulate.
                    Take maps from the simplicial complex
                    of $M$ to $G$ (Group) (Certain conditions).
                    There's an equivalence relation on
                    these maps. That set after taking the
                    equivalence relations is the homology:
                    $H_{n}(M,G)$. $n$ describes the type
                    of simplicies. If $n>\dim(M)$,
                    then $H_{n}(G,M)=0$.
                    $H_{n}(M,G)=\{f:\Delta^{n}\rightarrow G\}$.
                \end{remark}
                \begin{remark}
                    Cohomology is the set
                    $H^{n}(M,G)=[H_{n}(M,G),G]$,
                    that is, the \textit{dual}.
                \end{remark}
                We want to talk about cohomology.
                Under special conditions there is
                something called the Brown Representation
                Theorem. Consider Cohomology $H^{n}(M,G)$,
                with coefficients in $G$. Cohomology
                is Homotopy invariant, that is if
                $M\cong{N}$, then $H^{n}(M,G)\cong H^{n}(N,G)$.
                The Brown-Representation Theorem says
                that there is a classifying space $BG$
                such that, for all spaces $M$, there is
                a one-to-one correspondence between
                $H^{n}(X,G)\leftrightarrow[X,BG]$.
                In general, if $F$ is a functor,
                then the Brown-Representation Theorem
                says that there is a classifying space
                $Y$ such that $F(X)$ has a one-to-one
                correspondence with the homotopy classes
                of maps, $[X,Y]$.
                $F(x)\leftrightarrow[X,Y]$.
                \begin{example}
                    The Eilenberg-MacLane Space $K(G,n)$
                    has the property that
                    $\forall_{j\ne{n}}$,
                    $\pi_{n}(K(G,n))=G$,
                    and $\pi_{j}(K(G,n))=0$.
                    $K(G,n)$ is the classifying
                    space for cohomology.
                \end{example}
                \begin{theorem}
                    $K(G,n)$ is the classifying
                    space for cohomology. That is,
                    up to homotopy,
                    $H^{n}(X,G)\leftrightarrow[X,K(G,n)]$.
                \end{theorem}
                Let $\textrm{Prin}_{G}(X)$ be the collection
                of $G-$principal bundles on $X$. With a
                certain equivalence relation, it turns out
                the $\textrm{Prin}_{G}(X)$ is a group. So
                $\textrm{Prin}_{G}:\textit{Spaces}\rightarrow%
                 \textit{Groups}$
                is a functor. The Brown-Representation
                Theorem implies that there is a classifying
                space $BG$
                $\textrm{Prin}_{G}\leftrightarrow[X,BG]$.
                \begin{theorem}
                    If $p:E\rightarrow X$ and $p':E'\rightarrow X$
                    are both bundles over $X$, then there exists
                    $p\oplus p':E\oplus E' \rightarrow X$
                    COME BACK TO LATER
                \end{theorem}
            \subsubsection{Grothendique Groupification of Semigroup}
                \begin{definition}
                A semi-group is a group without the
                requirement for invereses.
                \end{definition}
                \begin{example}
                    $\{0,1,2,\hdots\}$ is a semi-group
                    under addition.
                \end{example}
                Let $G$ be a semi-group. Constraint
                $G\times{G}/\sim$.
                $(a,b)\sim(c,d)$ if $a+d=b+c$.
                So, $(2,3)\sim(4,5)\sim(7,8)\sim(-1,0)\equiv -1$.
                The equivalence class of all of these things is
                called $-1$. We still have all of the positive
                integers, $(4,2)\sim(5,3) \sim(6,4) \equiv 2$.
                This process adds all of the negatives. This
                process, called Grothendique Construction on a
                Semi-group creates a group out of a semi-group.
                It is, in a way, the 'smallest' group containing
                the semi-group. The groupification of
                $\{0,1,2,3,\hdots\}$ will be $\mathbb{Z}$.
                \begin{example}
                    What are the vector bundles over a dot?
                    There is $\mathbb{R}^{0}$ (A dot),
                    $\mathbb{R}^{n},\hdots,\mathbb{R}^n,\hdots$
                    There is an operation on this set
                    $\{\mathbb{R}^{n}:n \geq 0\}$. This makes a
                    semi-group, and there is a Grothendique
                    Groupification
                    $G_{r}(\mathbb{Z}_{\geq 0},+)=\mathbb{Z}$
                \end{example}
                Suppose M is a monoid/semigroup.
                Not required to have an inverse but should
                have an identity. For example,
                $(\mathbb{Z}_{\geq 0},+)$ is a monoid.
                Has identity, but no inverse.
                Construct $M\times M=\{(a,b):a,b\in M\}$
                with the operation $(a,b)+(c,d) = (a+c,b+d)$.
                Think of $(a,b)$ as $a-b$. Note that in
                regular math $'3-1'='4-2'$, so we want $(3,1)$
                to equal $(4,2)$. We do this with the
                equivalence rlation $(a,b)\sim (c,d)$ if
                and only if $a+d=b+c$. Let $M\times M/\sim$
                be called $M_{G}$. 
                \begin{theorem}
                    $M_{G}$ is a group.
                \end{theorem}
                \begin{theorem}
                    There is an injection $i:M\rightarrow M_{G}$
                    with the following property:
                \begin{enumerate}
                    \item $i(a)\sim (a,0)\sim(a+1,1)\sim(a+2,2)\sim\hdots$
                \end{enumerate}
                \end{theorem}
            This construction is functorial, so if there are monoids $M,N$ with a semi-homomorphism $\phi:M\rightarrow N$ $(\phi(a*b)=\phi(a)*\phi(b))$ (Homomorphism for a semi-group), then there is a HM $\phi_{G}:M_{G}\rightarrow N_{G}$ so $G(Monoids,Semihomomorphism)\rightarrow (Groups,Homomorphism)$ is a functor.
            \subsubsection{Suspension}
            Let $X$ and $Y$ be disjoint topological spaces. The wedge product $X\vee Y$ is the one-point union of $X$ and $Y$. Take $X$, take $Y$, and glue one point together. 
            \begin{theorem}
            If $X$ and $Y$ are disjoint topological spaces, then $\pi_{1}(X\vee Y)=\pi_{1}(X)\oplus\pi_{1}(Y)$.
            \end{theorem}
            In the same context, the smash product of $X$ and $Y$ is $X\wedge Y=X\times Y/(X\vee Y)$. Picture $X=(0,1]$ in the $x$ axis and $Y=(0,1]$ in the $y$ axis. Then $X\times Y$ is a square in the $xy$ plane, and $X\vee Y$ is the $x$ and $y$ axes from $0$ to $1$. So $X\wedge Y$ takes all of the points on the two axes between $0$ and $1$ and smashes them down to the origin.
            \begin{example}
            $S^{1}\times [0,1]$ is the hollow cylinder, and $S^{1}\vee [0,1]$ is the boundary of the edge of the cylinder (The lid) and the line going down the cylinder parallel with the z-axis (the spine). So $X\wedge Y$ smashes down to a cone. This is then homeomorphic to $D^{2}$. 
            \end{example}
            \begin{example}
            The torus can be visualized by the diagram shown in \ref{fig:surgery_theory_plane_representation_of_a_torus}. So $T^{2}=S^{1}\times S^{1}$. Using the diagram, we can see that the smash product $S^{1}\wedge S^{1}$ is homotopy equivalent to a sphere.
            \end{example}
            \begin{definition}
            Let $X$ be a topological space. Then the suspension of $X$, denoted $\Sigma X$, is $S^{1}\wedge X$.
            \end{definition}
            So $\Sigma S^{n}=S^{n+1}$. The usefulness of smash has to do with $\pi_{k}(\Sigma X)=\pi_{k-1}(X)$. There's another thing called the Freudenthal suspension theorem.
            \subsubsection{Higher Homotopy}
            The fundamental group, which is the first homotopy group, is $\pi_{1}(X)$. Ingredients needed:
            \begin{enumerate}
                \item Topological Space $X$
                \item A basepoint $x_{0}$
            \end{enumerate}
            \begin{definition}
            The fundamental group of a topological space $X$ about a base point $x_{0}$ is the set $\pi_{1}(X)=[(S^{1},\star),(X,\star)]=Hom\big((S^{1},\star),(X,\star)\big)=\{f:S^{1}\rightarrow X:f(x)=\star\}/\textrm{Homotopy}$.
            \end{definition}
            $\pi_{1}(X)$ is a group using concatenation. Higher homotopy groups:
            \begin{definition}
            $\pi_{n}(X)=[(S^{n},\star),(X,\star)]$
            \end{definition}
            It turns out that $\pi_{n}(X)$ has a certain operation, for $n\geq 2$, such that it is an Abelian group. However, $\pi_{1}(X)$ need not be an Abelian group.
            \begin{example}
            The Klein bottle is an example of a space such that $\pi_{1}(X)$ is not an Abelian group.
            \end{example}
            The question becomes 'What are the Homotopy groups of sphere?' That is, what is $\pi_{m}(S^{n})$? Recall stereographic projection from before. Take $S^{n}$ and remove the north pole (The point $(0,0,1)$). This can be projected down to $\mathbb{R}^{2}$. This can be generalized to $n$ dimensions, and in general $S^{n}\setminus\{\textrm{North Pole}\}$ is homeomorphic to $\mathbb{R}^{n}$. Now $\mathbb{R}^{n}$ has $0$ homotopy groups because it is contractible (Can be smushed down to a point). If $m<n$, then we are mapping a small 'sphere' into a big 'sphere.
            \begin{theorem}
            If $n\ne m$, then there is no continuous function $f$ such that $f:S^{n}\rightarrow S^{m}$ is surjective.
            \end{theorem}
            Now, if $m<n$, then we map $S^{m}$ into $S^{n}$. But since there is no surjective continuous function we can remove a point from $S^{n}$, map it down to $\mathbb{R}^{n}$ and then contract. So, $\pi_{m}(S^{n})=0$ for all $m<n$. The next case is when $m=n$. There are three obvious maps: The constant map, the identity map, and the antipotal map. It can be shown that there are, for all $n$, countably many maps. So $\pi_{n}(S^{n})=\mathbb{Z}$. Another neat little fun fact is that $\pi_{3}(S^{2})=\mathbb{Z}$. (Related to Hopf fibration). Now, the suspension theorem says that $\pi_{n+1}(\Sigma X)=\pi_{n}(X)$. So $\mathbb{Z}=\pi_{3}(S^{2})=\pi_{4}(\Sigma S^4)=\pi_{4}(S^{3})=\pi_{5}(\Sigma S^{3})=\pi_{5}(S^{4})=\hdots$ so, if $m-n=1$, then $\pi_{m}(S^{n})=\mathbb{Z}$.
            \begin{theorem}
            $\pi_{3}(S^{2})=\mathbb{Z}$
            \end{theorem}
            \begin{theorem}
            If $m-n=1$, and $n\geq 2$, then $\pi_{m}(S^{n})=\mathbb{Z}$
            \end{theorem}
            These are examples of stability theorems, or stability results.
            \subsubsection{Fibrations}
            \begin{definition}
            A fibration is a map between topological spaces that has the homotopy lifting property for every space $X$.
            \end{definition}
            A fibration gives rise to a long exect sequence of homotopy groups
            \begin{align*}
                \pi_{3}(S^{1})\rightarrow\pi_{3}(S^{3})\rightarrow\pi_{3}(S^{2})\rightarrow\pi_{2}(S^{1})&\rightarrow\pi_{2}(S^{3})\rightarrow\pi_{2}(S^{1})\rightarrow\hdots\\
                &\hdots\rightarrow\pi_{2}(S^{3})\rightarrow\pi_{2}(S^{2})\rightarrow\pi_{1}(S^{1})\rightarrow\pi_{1}(S^{3})\rightarrow\pi_{1}(S^{2})
            \end{align*}
            We need to know that $\pi_{n}(S^{1})=0$ if $n\geq 2$. An element of $\pi_{n}(S^{1})$ is $f:S^{n}\rightarrow S^{1}$. Stanley owe's me an explanation.
            This becomes:
            \begin{align*}
                0\rightarrow\mathbb{Z}\rightarrow A\rightarrow 0\rightarrow 0\rightarrow\mathbb{Z}\rightarrow\mathbb{Z}\rightarrow 0\rightarrow 0
            \end{align*}
            \subsubsection{Classifying Space}
            If $G$ is a group (discrete or not), then there is a classifying space (Topological space) $BG$ (unique up to homotopy) such that :
            \begin{enumerate}
                \item $\pi_{1}(BG)=G$ and $\pi_{n}(BG)=0$ for all $n\geq 2$.
                \item There is a contractible space $EG$ that is a principle $G$ bundle with a $G$ action such that $BG\simeq EG/G$. $EG\rightarrow BG$.
                \item For all spaces $X$ with a continuous map $f:X\rightarrow BG$, there is a pullback diagram
                \item The correspondence $(X\rightarrow BG)\rightarrow (Y_{f}\downarrow X)$ has the property that, if $f\simeq g$, then $(Y_{f}\downarrow X)\overset{\textrm{Principle G-Bundle}}{=}(Y_{g}\downarrow X)$. This gives a map $[X,BG]\rightarrow Prin_{G}(X)$ which is a bijection.
            \end{enumerate}
            \begin{definition}
            Let $V$ be a finite dimensional vector space over $\mathbb{F}$. We say that $B:V\times V\rightarrow\mathbb{F}$ is symmetric bilinear if:
            \begin{enumerate}
                \item $B(v,v')=B(v',v)$
                \item $B(v+w,v')=B(v,v')+B(w,v')$
                \item $B(\lambda v,w)=\lambda B(v,w)$
            \end{enumerate}
            \end{definition}
            \begin{example}
            Let $A$ be a symmetric real $n\times n$ matrix. Then $B:\mathbb{R}^{n}\rightarrow\mathbb{R}^{n}\rightarrow \mathbb{R}$ given by $B(X,Y) = X^{T}AY$. This $B$ gives rise to a map $Q:V\rightarrow \mathbb{F}$ given by $Q(x)=B(x,x)$.
            \begin{equation*}
                Q(X)=\begin{bmatrix}x,y\end{bmatrix}\begin{bmatrix}2&0\\0&1\end{bmatrix}\begin{bmatrix}x&y\end{bmatrix}=2x^{2}+y
            \end{equation*}
            Which is a quadratic form.
            \end{example}
\end{document}