\documentclass[crop=false,class=article,oneside]{standalone}
%----------------------------Preamble-------------------------------%
%---------------------------Packages----------------------------%
\usepackage{geometry}
\geometry{b5paper, margin=1.0in}
\usepackage[T1]{fontenc}
\usepackage{graphicx, float}            % Graphics/Images.
\usepackage{natbib}                     % For bibliographies.
\bibliographystyle{agsm}                % Bibliography style.
\usepackage[french, english]{babel}     % Language typesetting.
\usepackage[dvipsnames]{xcolor}         % Color names.
\usepackage{listings, lstlinebgrd}      % Verbatim-Like Tools.
\usepackage{mathtools, esint, mathrsfs} % amsmath and integrals.
\usepackage{amsthm, amsfonts}           % Fonts and theorems.
\usepackage{tabularx}
\usepackage{tcolorbox}                  % Frames around theorems.
\usepackage{upgreek}                    % Non-Italic Greek.
\usepackage{paracol}                    % Two-column styling.
\usepackage{wrapfig}                    % Wrap text around figure.
\usepackage{fmtcount, etoolbox}         % For the \book{} command.
\usepackage[newparttoc]{titlesec}       % Formatting chapter, etc.
\usepackage{titletoc}                   % Allows \book in toc.
\usepackage[nottoc]{tocbibind}          % Bibliography in toc.
\usepackage[titles]{tocloft}            % ToC formatting.
\usepackage{multicol, enumitem}         % Multi-column/enumerate.
\usepackage{import}                     % Import external files.
\usepackage{pgfplots, tikz}             % Drawing/graphing tools.
\usetikzlibrary{
    calc,                   % Calculating right angles and more.
    angles,                 % Drawing angles within triangles.
    arrows.meta,            % Latex and Stealth arrows.
    quotes,                 % Adding labels to angles.
    positioning,            % Relative positioning of nodes.
    decorations.markings,   % Adding arrows in the middle of a line.
    patterns,
    arrows,
    shapes,
    shapes.geometric,
    cd,
    hobby,
    babel
}                                       % Libraries for tikz.
\pgfplotsset{compat=1.9}                % Version of pgfplots.
\usepackage[font=scriptsize,
            labelformat=simple,
            labelsep=colon]{subcaption} % Subfigure captions.
\usepackage[font={scriptsize},
            hypcap=true,
            labelsep=colon]{caption}    % Figure captions.
\usepackage{hyperref}                   % Allows for hyperlinks.
\hypersetup{
    colorlinks=true,
    linkcolor=blue,
    filecolor=magenta,
    urlcolor=Cerulean,
    citecolor=SkyBlue
}                           % Colors for hyperref.
\usepackage[toc,acronym,nogroupskip]{glossaries} % Glossaries and acronyms.
\usepackage[subpreambles=false]{standalone}      % Complileable sub files.

% Various font stuff from kiwi.
% Use this for Times text and Computer Modern math
%\usepackage{times}

% Quite nice
%\usepackage[charter, greekfamily=, greekuppercase=italicized]{mathdesign}
%\usepackage[utopia, greekuppercase=italicized]{mathdesign}    % Math is narrower

% Use this for Times text and math
%\usepackage{newtxtext}
%\usepackage[libertine,cmintegrals]{newtxmath}
%\usepackage{fix-cm}

%\usepackage{txfontsb}
% or
%\usepackage{mathptmx}

%\usepackage[scaled=0.92]{helvet}
%\renewcommand{\rmdefault}{ptm}

%\usepackage{mathpazo}    % add possibly `sc` and `osf` options
%\usepackage{eulervm}

%\usepackage{fourier}
%\renewcommand{\rmdefault}{ptm}
%\usepackage{mathptm}

%\usepackage{fontspec}
%\setmainfont{lmodern}

%\usepackage[varg]{txfonts}
%\usepackage{fouriernc}
%\usepackage{mathpazo}

%\usepackage{bookman}
%\usepackage[scaled]{uarial}
%\usepackage[scaled]{helvet}
%\renewcommand*\familydefault{\sfdefault}
%\usepackage[math]{anttor}

%\newcommand\fgeorgia{\fontfamily{jvn}\selectfont}
%\newcommand\ftimes{\fontfamily{ptm}\selectfont}
%\newcommand\fhelvetica{\fontfamily{phv}\selectfont}
%\newcommand\fcourier{\fontfamily{pcr}\selectfont}
%\newcommand\fbookman{\fontfamily{pbk}\selectfont}
%\newcommand\fnewcentury{\fontfamily{pnc}\selectfont}
%\newcommand\fpalatino{\fontfamily{ppl}\selectfont}
%\newcommand\favantgarde{\fontfamily{pag}\selectfont}
%\newcommand\fnormal{\normalfont}
%\newcommand\fsize[1]{\ifnum#1>0\fontsize{#1}{#1}\selectfont\else\normalsize\fi}
%------------------------Theorem Styles-------------------------%
% Define theorem style for default spacing and normal font.
\newtheoremstyle{normal}
    {\topsep}               % Amount of space above the theorem.
    {\topsep}               % Amount of space below the theorem.
    {}                      % Font used for body of theorem.
    {}                      % Measure of space to indent.
    {\bfseries}             % Font of the header of the theorem.
    {}                      % Punctuation between head and body.
    {.5em}                  % Space after theorem head.
    {}

% Define theorem style for default spacing with italicized font.
\newtheoremstyle{normalit}{\topsep}{\topsep}
                {\itshape}{}{\bfseries}{}{.5em}{}

% Italic header environment.
\newtheoremstyle{thmit}{\topsep}{\topsep}{}{}{\itshape}{}{0.5em}{}

% Define italicized environments.
\theoremstyle{normalit}
\newtheorem{theorem}{Theorem}[section]
\newtheorem{lemma}{Lemma}[section]
\newtheorem{corollary}{Corollary}[section]
\newtheorem{proposition}{Proposition}[section]
\newtheorem*{theorem*}{Theorem}

% Define environments with italic headers.
\theoremstyle{thmit}
\newtheorem*{solution}{Solution}
\newtheorem*{fsolution}{Solution}

% Define default environments.
\theoremstyle{normal}
\newtheorem{example}{Example}[section]
\newtheorem{definition}{Definition}[section]
\newtheorem{problem}{Problem}[section]
\newtheorem{question}{Question}[section]
\newtheorem{remark}{Remark}[section]
\newtheorem{properties}{Properties}[section]
\newtheorem{notation}{Notation}[section]
\newtheorem{axiom}{Axiom}[section]
\newtheorem*{properties*}{Properties}
\newtheorem*{remark*}{Remark}
\newtheorem*{definition*}{Definition}
\theoremstyle{plain}

% Define framed environment.
\tcbuselibrary{most}
\newtcbtheorem[use counter*=theorem]{ftheorem}{Theorem}%
    {colback=green!5,colframe=green!35!black,
     fonttitle=\bfseries\upshape}{th}

\newtcbtheorem[use counter*=example]{fdefinition}{Definition}%
    {fonttitle=\bfseries\upshape,
     colback=blue!5!white,colframe=blue!75!black}{def}

\newtcbtheorem[use counter*=example]{fexample}{Example}%
    {fonttitle=\bfseries\upshape,
     colback=red!5!white,colframe=red!75!black}{ex}

\newtcbtheorem[use counter*=notation]{fnotation}{Notation}%
    {fonttitle=\bfseries\upshape,
     colback=SeaGreen!5!white,colframe=SeaGreen!75!black}{ex}

\newtcbtheorem[use counter*=corollary]{fcorollary}{Corollary}%
    {fonttitle=\bfseries\upshape,
     colback=Orchid!5!white,colframe=Orchid!75!black}{ex}

\newenvironment{bproof}{\textit{Proof.}}{\hfill$\square$}
\tcolorboxenvironment{bproof}{blanker,breakable,left=5mm,
                             before skip=10pt,after skip=10pt,
                             borderline west={1mm}{0pt}{red}}
\tcolorboxenvironment{fsolution}
    {enhanced jigsaw,colframe=cyan,interior hidden,breakable}

%--------------------Declared Math Operators--------------------%
\DeclareMathOperator{\Refl}{Refl}           % Reflection operator.
\DeclareMathOperator{\Span}{Span}           % Span of a set of vectors.
\DeclareMathOperator{\Card}{Card}           % Cardinality of set.
\DeclareMathOperator{\Ord}{Ord}             % Ordinal of ordered set.
\DeclareMathOperator{\Tr}{Tr}               % Trace of matrix.
\DeclareMathOperator{\adjoint}{adj}         % Adjoint of matrix.
\DeclareMathOperator{\rk}{rk}               % Rank of operator.
\DeclareMathOperator{\nul}{nul}             % Null space of operator.
\DeclareMathOperator{\sgn}{sgn}             % Sign of a number.
\DeclareMathOperator{\multideg}{mutlideg}   % Multi-Degree (Graphs).
\DeclareMathOperator{\GCD}{GCD}             % Greatest common denominator.
\DeclareMathOperator{\LM}{LM}               % Leading monomial
\DeclareMathOperator{\LC}{LC}               % Leading coefficient.
\DeclareMathOperator{\LT}{LT}               % Leading term.
\DeclareMathOperator{\LCM}{LCM}             % Least common multiple.
\DeclareMathOperator{\Mon}{Mon}             % Monomial.
\DeclareMathOperator{\Spec}{Spec}           % Spectrum.
\DeclareMathOperator{\proj}{proj}           % Projection.
\DeclareMathOperator{\comp}{comp}           % Component.
\DeclareMathOperator{\sinc}{sinc}           % Sinc function.
\DeclareMathOperator{\Ima}{Im}              % Image of operator.
\DeclareMathOperator{\Prin}{Prin}           % Principal value.
\DeclareMathOperator{\Mod}{mod}             % Modulus.
%------------------------New Commands---------------------------%
\DeclarePairedDelimiter\norm{\lVert}{\rVert}
\DeclarePairedDelimiter\ceil{\lceil}{\rceil}
\DeclarePairedDelimiter\floor{\lfloor}{\rfloor}
\newcommand*\diff{\mathop{}\!\mathrm{d}}
\newcommand*\Diff[1]{\mathop{}\!\mathrm{d^#1}}
\renewcommand{\mod}{\ \Mod}
\renewcommand*{\glstextformat}[1]{\textcolor{RoyalBlue}{#1}}
\renewcommand{\glsnamefont}[1]{\textbf{#1}}
\renewcommand\labelitemii{$\circ$}
\renewcommand\thesubfigure{\arabic{chapter}.\arabic{figure}}
\renewcommand\thesubfigure{%
    \arabic{chapter}.\arabic{figure}.\arabic{subfigure}}
\addto\captionsenglish{\renewcommand{\figurename}{Fig.}}
%------------------------Book Command---------------------------%
\makeatletter
\renewcommand\@pnumwidth{1cm}
\newcounter{book}
\renewcommand\thebook{\@Roman\c@book}
\newcommand\book{%
    \if@openright
        \cleardoublepage
    \else
        \clearpage
    \fi
    \thispagestyle{plain}%
    \if@twocolumn
        \onecolumn
        \@tempswatrue
    \else
        \@tempswafalse
    \fi
    \null\vfil
    \secdef\@book\@sbook
}
\def\@book[#1]#2{%
    \ifnum \c@secnumdepth >-3\relax
        \refstepcounter{book}%
        \addcontentsline{toc}{book}{
            \bookname\ \thebook:\hspace{1em}#1
        }
    \else
        \addcontentsline{toc}{book}{#1}%
    \fi
    \markboth{}{}%
    {\centering
     \interlinepenalty \@M
     \normalfont
     \ifnum \c@secnumdepth >-2\relax
       \huge\bfseries \bookname\nobreakspace\thebook
       \par
       \vskip 20\p@
     \fi
     \Huge \bfseries #2\par}%
    \@endbook}
\def\@sbook#1{%
    {\centering
     \interlinepenalty \@M
     \normalfont
     \Huge \bfseries #1\par}%
    \@endbook}
\def\@endbook{
    \vfil\newpage
        \if@twoside
            \if@openright
                \null
                \thispagestyle{empty}%
                \newpage
            \fi
        \fi
        \if@tempswa
            \twocolumn
        \fi
}
\newcommand*\l@book[2]{%
    \ifnum \c@tocdepth >-2\relax
        \addpenalty{-\@highpenalty}%
        \addvspace{2.25em \@plus\p@}%
        \setlength\@tempdima{3em}%
        \begingroup
            \parindent \z@ \rightskip \@pnumwidth
            \parfillskip -\@pnumwidth
            {
                \leavevmode
                \Large \bfseries #1\hfil \hb@xt@\@pnumwidth{
                    \hss #2
                }
            }
            \par
            \nobreak
            \global\@nobreaktrue
            \everypar{\global\@nobreakfalse\everypar{}}%
        \endgroup
    \fi}
\newcommand\bookname{Book}
\renewcommand{\thebook}{\texorpdfstring{\Numberstring{book}}{book}}
\providecommand*{\toclevel@book}{-2}
\makeatother
\titlecontents{chapter}[0pt]
    {\bfseries}
    {\chaptername\ \thecontentslabel:\quad}
    {}
    {\hfill\contentspage}
\titleformat{\part}[display]
    {\Large\bfseries}
    {\partname\nobreakspace\thepart}
    {0mm}
    {\Huge\bfseries}
    \titlecontents{part}[0pt]
    {\large\bfseries}
    {\partname\ \thecontentslabel: \quad}
    {}
    {\hfill\contentspage}
\newcommand{\MarkRightAngle}[4][.3cm]
    {\coordinate (tempa) at ($(#3)!#1!(#2)$);
     \coordinate (tempb) at ($(#3)!#1!(#4)$);
     \coordinate (tempc) at ($(tempa)!0.5!(tempb)$);%midpoint
     \draw (tempa) -- ($(#3)!2!(tempc)$) -- (tempb);}
%--------------------------LENGTHS------------------------------%
% Spacings for the Table of Contents.
\addtolength{\cftsecnumwidth}{1ex}
\addtolength{\cftsubsecindent}{1ex}
\addtolength{\cftsubsecnumwidth}{1ex}
\addtolength{\cftfignumwidth}{1ex}
\addtolength{\cfttabnumwidth}{1ex}

% Spacing for multi-column and enumerate environments.
\setlength{\multicolsep}{6pt}
\setlist[enumerate]{itemsep=0pt,topsep=3pt}

% Indent and paragraph spacing.
\setlength{\parindent}{0em}
\setlength{\parskip}{0em}
%--------------------------Main Document----------------------------%
\begin{document}
    \ifx\ifmathcoursessurgery\undefined
        \section*{Surgery Theory}
        \setcounter{section}{1}
        \renewcommand\thefigure{%
            \arabic{section}.\arabic{figure}%
        }
        \renewcommand\thesubfigure{%
            \arabic{section}.\arabic{figure}.\arabic{subfigure}%
        }
    \fi
    \subsection{Lecture 3: Vector Bundles}
        \subsubsection{Group Rings}
            \begin{definition}
                If $G$ is a group and $R$ is a ring, then the
                group ring $RG$ is the collection of all
                finite linear combinations (Formal Sums):
                $r_{1}g_{1}+\hdots+r_{n}g_{n}$, where
                $r_{k}\in{R}$ and $g_{k}\in{G}$.
            \end{definition}
            \begin{example}
                If $G$ is a group, and
                $\mathbb{Z}G=\{\sum_{k=0}^{n}n_{k}g_{k}:%
                 n_{k}\in\mathbb{Z},g_{k}\in{G}\}$, then
                $\mathbb{Z}G$ is a group ring. This is a
                special group ring, denoted
                $\textrm{SP}_{\mathbb{Z}}(G)$.
            \end{example}
            \begin{theorem}
                If $R$ is a ring and $G$ is a group, then
                the group ring $RG$ is a ring.
            \end{theorem}
            From the previous lecture we saw that
            $L_{n+1}(\mathbb{Z}\pi_{1}(\mathcal{M}))$ is
            a group. But from the previous theorem, we have
            that $\mathbb{Z}\pi_{1}(\mathcal{M})$ is a ring.
            So, we may thing of the $L_{n}$ as a \textit{Functor}:
            $L_{n}:\textrm{Rings}\rightarrow\textrm{Groups}$.
            To recap the notation, $S(\mathcal{M})$ is the
            Surgery Structure Set on the manifold
            $\mathcal{M}$, and
            $L_{n}(\mathbb{Z}\pi_{1}(\mathcal{M}))$ is
            a Wall Group.
        \subsubsection{Matrices and Vector Bundles}
            The next monster we need to understand in the
            Surgery Exact Sequence is the
            $[\mathcal{M},G/o]$ that keep appearing.
            First, a quick recap on some notions in
            linear algebra.
            \begin{definition}
                An orthogonal matrix is an invertible
                square matrix $A$ such that
                $A^{T}=A^{-1}$
            \end{definition}
            Let $\mathcal{O}(n)$ be the group of
            $n\times{n}$ orthogonal matrices. There
            is a simple map then from
            $\mathcal{O}(n)$ to $\mathcal{O}(n+1)$,
            $\psi_{n}:%
             \mathcal{O}(n)\rightarrow\mathcal{O}(n+1)$,
            defined by:
            \begin{equation*}
                \psi_{n}(A)=
                \left[%
                    \begin{array}{c|c}
                        A&0\\
                        \hline
                        0&1
                    \end{array}
                \right]
            \end{equation*}
            We can also define a map
            $\varphi_{nm}:%
             \mathcal{O}(n)\times\mathcal{O}(m)%
             \rightarrow\mathcal{O}(n+m)$ defined
            by:
            \begin{equation*}
                \varphi_{nm}(A,B)=
                \left[%
                    \begin{array}{c|c}
                        A&0\\
                        \hline
                        0&B
                    \end{array}
                \right]
            \end{equation*}
            This is, in general, not a bijection.
            From this we can create a sequence:
            \begin{equation*}
                \mathcal{O}(1)
                \overset{\psi_{1}}{\longrightarrow}
                \mathcal{O}(2)
                \overset{\psi_{2}}{\longrightarrow}
                \mathcal{O}(3)
                \overset{\psi_{3}}{\longrightarrow}
                \mathcal{O}(4)
                \overset{\psi_{4}}{\longrightarrow}
                \cdots
                \mathcal{O}(n)
                \overset{\psi_{n}}{\longrightarrow}
                \cdots
            \end{equation*}
            We can then define $\mathcal{O}$ as the
            \textit{Direct Limit} of this sequence:
            \begin{equation*}
                \mathcal{O}
                =\underset{n\rightarrow\infty}{\lim}
                \mathcal{O}(n)
            \end{equation*}
            Now, let $\mathcal{M}$ be a manifold.
            An $n$ dimensional vector bundle is a map
            $P:E\rightarrow\mathcal{M}$ such that,
            for each point $x\in\mathcal{M}$,
            the \textit{fiber} of $x$, the
            pre-image $p^{-1}(x)$, is homeomorphic
            to $\mathbb{R}^{n}$.
            \begin{definition}
                The fiber of a point $y$ in a set $Y$
                under the map $f:X\rightarrow{Y}$ is the
                pre-image $f^{-1}(y)\subset{X}$.
            \end{definition}
            \begin{definition}
                A real $n$ dimensional vector bundle
                on a manifold
                $\mathcal{M}$ is a manifold
                $E$ and a continuous map
                $p:E\rightarrow\mathcal{M}$
                such that, for all
                $x\in\mathcal{M}$, the
                fiber of $x$ is homeomorphic
                to $\mathbb{R}^{n}$ and there
                exists an open set $\mathcal{U}$
                such that $x\in\mathcal{U}$ and
                $p^{-1}(\mathcal{U})$ is homeomorphic
                to $\mathcal{U}\times\mathbb{R}^{n}$
            \end{definition}
            The requirement that there is an open neighborhood
            $\mathcal{U}_{x}$ for all $x$ such that
            $p^{-1}(\mathcal{U})$ is homeomoprhic to
            $\mathcal{U}_{x}\times\mathbb{R}^{n}$ is called
            \textit{local triviality}. There is another notion called
            \textit{global triviality}.
            \begin{example}
                A classic example is a cylinder
                with a disk (Or the boundary of
                a cylinder with the circle).
                Given a point $(x,y,z)$ in the
                cylinder, collapse this (Or project it)
                down onto the $xy$ plane by the map
                $p(x,y,z)=(x,y)$. This is continuous,
                and is an example of a vector bundle:
                $(D^{1},D^{1}\times\mathbb{R},p)$.
                The pre-image, or fiber, of any point
                in $D^{1}$ is a line, which is
                certainly homeomorphic to
                $\mathbb{R}$. Again, taking any point $x$
                and looking at an open ball about that
                point that is entirely contained within
                $D^{1}$, the pre-image
                $p^{-1}(B_{r}(x))$ is another cylinder, which is
                homeomorphic to $\mathbb{R}^{3}$, which is itself
                homeomorphic to
                $B_{r}(x)\times\mathbb{R}^{1}$. The fibers of
                $x$ and $\mathcal{U}$ are shown in
                Fig.~\ref{fig:Surgery_Theory_Simply_%
                          Vector_Bundle_Cylinder_to_Disk}
            \end{example}
            \begin{figure}[H]
                \centering
                \captionsetup{type=figure}
                \subimport{../../../../tikz/}
                          {Vector_Bundle_Over_Circle}
                \caption{Example of a Vector Bundle:
                         $(D^{1},D^{1}\times\mathbb{R},p)$}
                \label{%
                    fig:Surgery_Theory_Simply_%
                    Vector_Bundle_Cylinder_to_Disk%
                }
            \end{figure}
            \begin{example}
                If $\mathcal{M}$ is a manifold,
                $E=\mathcal{M}\times\mathbb{R}^{n}$,
                and if $p:E\rightarrow\mathcal{M}$ is
                defined by
                $p(x,\mathbf{y})=x$ for all
                $(x,\mathbf{y})\in\mathcal{M}\times\mathbb{R}^{n}$,
                then $(E,\mathcal{M},p)$ is a vector bundle. This
                is called the trivial $n$ dimensional vector
                bundle of $\mathbb{R}^{n}$. The fibers
                of points $x\in\mathcal{M}$ are $\mathbb{R}^{n}$,
                which is homeomorphic to $\mathbb{R}^{n}$. Given
                any open set $\mathcal{U}$ containing $x$, the
                pre-image is $\mathcal{U}\times\mathbb{R}^{n}$.
            \end{example}
            \begin{example}
                The M\"{o}bius strip can be seen as
                a vector bundle $S^{1}\times[0,1]\rightarrow[0,1]$
                where the map $(x,t)\rightarrow{x}$ is by a
                ``twist.'' This is a non-oreientable bundle
                which is non-trivial. It has local triviality,
                but no global triviality.
            \end{example}
            \begin{figure}[H]
                \centering
                \captionsetup{type=figure}
                \subimport{../../../../tikz/}
                          {Mobius_Strip}
                \caption{M\"{o}bius Strip}
                \label{fig:Surgery_Theory_Mobius_%
                       Strip_Vector_Bundle}
            \end{figure}
            Returning to our discussion of orthogonal matrices,
            $\mathcal{O}(n)$ is a group under matrix multiplication.
            The matrix $I_{n}$ is orthogonal, and if $A$ is orthogonal,
            then $(A^{-1})^{T}=(A^{T})^{T}=A$. But $(A^{-1})^{-1}=A$,
            and thus $A^{-1}$ is orthogonal as well. Thus we have
            an identity, associativity, and closure of inverses.
            Therefore $\mathcal{O}(n)$ is a group. This group can
            act on the set of $n$ dimensional vectors in
            $\mathbb{R}^{n}$ by the map
            $(A,\mathbf{v})\rightarrow{A\mathbf{v}}$,
            for all $A\in\mathcal{O}(n),\mathbf{v}\in\mathbb{R}^{n}$.
            Thus, we have the $\mathcal{O}(n)$ acts over the fibers
            of an $n$ dimensional real vector bundle
            $(E,\mathcal{M},p)$. $\mathcal{O}(1)$ is the identity.
            That is, the ``Do nothing,'' action on a 1 dimensional
            vector bundle. $\mathcal{O}(2)$ can perform
            \textit{reflections} and \textit{rotations} on the
            fibers. Endowed with this action, any real
            vector bundle of dimension $n$ is an example of
            a \textit{principal $\mathcal{O}(n)$ bundle}.
            If $g\in\mathcal{O}(n)$ and $v\in{E}$,
            then $p(gv)=g(p(v)$.
        \subsubsection{Principal G-Bundles}
            If $G$ is a group, and if $X$ is a
            topogolical space, then there is a
            structure/notion of a
            \textit{principal G-Bundle} on $X$.
            That is, $X$ has some bundle over it
            (The space $E$ from our previous discussion),
            and $G$ acts on the fibers of $X$. This is
            denoted $\Prin_{G}(X)$.
            \par\hfill\par
            Construction by John Milnor, Classifying space.
            No idea why I wrote this...
            \par\hfill\par
            If $G$ is a group, there is a complex (space) $BG$
            such that we may form the set:
            \begin{equation*}
                [\mathcal{M},BG]
                =\{f:\mathcal{M}\rightarrow{BG}\}/Homotopy
            \end{equation*}
            That is, the set of continuous maps from $\mathcal{M}$ to
            $BG$ modded out by homotopy. Two maps are equivalent if they
            are homotopic.
            \begin{theorem}
                There is a continuous surjective function
                $f:\Prin_{G}(\mathcal{M})\rightarrow%
                 [\mathcal{M},BG]$.
            \end{theorem}
            Elaborating more on the $BG$,
            $B$ is a \textit{functor}
            $B:\textrm{Groups}\rightarrow\textrm{Spaces}$.
            If $G$ is a finitely presented group,
            then $\pi_{1}(BG)=G$, and more over,
            for all $n\geq{2}$,
            $\pi_{n}(BG)=0$. That is, $\pi_{n}(BG)$
            is the trivial group for all $n\geq{2}$.
            $\pi_{1}(X)$ ca be seen as the
            \textit{homotopy class} of $[S^{1},X]$.
            $\pi_{n}$ the homotopy class for
            $[S^{n},X]$.
            \begin{example}
                $\pi_{1}(B\mathbb{Z})=\mathbb{Z}$.
                For all $n\geq{2}$,
                $\pi_{n}(B\mathbb{Z})=0$.
                Thus $B\mathbb{Z}$ is homotopy
                equivalent to $S^{1}$.
            \end{example}
        \subsubsection{Covering Spaces}
            \begin{definition}
                A covering space of a
                topological space $X$
                is a space
                $E$ such that there exists
                a continuous surjection
                $p:E\rightarrow{X}$ such that
                for all $x\in{X}$, there is an
                open set $\mathcal{U}$ such that
                $x\in\mathcal{U}$ such that there
                exists a set of disjoint open sets
                $E_{r}\subset{E}$ where
                $p^{-1}(\mathcal{U})=\cup_{r}E_{r}$
                and for all $r$,
                $p$ is a homeomorphism between
                $E_{r}$ and $\mathcal{U}$.
            \end{definition}
            \begin{example}
                The first example is
                $S^{1}$ and $\mathbb{R}$.
                Define the map
                $p:\mathbb{R}\rightarrow{S^{1}}$
                by $p(x)=\exp(2\pi{i}x)$. This ``wraps,''
                the real line around the circle over and over again.
                Given a point $y\in{S^{1}}$, the pre-image, or fiber,
                of $y$ with respect to $p$ is
                $\{x+n:n\in\mathbb{Z}\}$ for some $x\in[0,1)$.
                Given a small enough neighborhood
                around $y$, the pre-image is of the form
                $\{x+n-\varepsilon,x+n+\varepsilon:n\in\mathbb{Z}\}$,
                which is a bunch of copies of $(0,1)$, or a bunch
                of copies of the neighborhood around $y$.
            \end{example}
            \begin{figure}[H]
                \centering
                \captionsetup{type=figure}
                \subimport{../../../../tikz/}
                          {Covering_Space_Real_Line_and_Circle}
                \caption{$\mathbb{R}$ is a Universal Covering of $S^{1}$.}
                \label{fig:Surgery_Theory_Reals_Cover_Circle}
            \end{figure}
            \begin{definition}
                A universal covering space of a topological space $X$
                is a covering space $E$ of $X$ such that
                $E$ is simply connected.
            \end{definition}
            That is, if $E$ is a covering space of $X$, then we say
            that $E$ is a universal covering space if
            $\pi_{1}(E)=0$. In the previous example we saw that
            $\mathbb{R}$ is a covering space of $S^{1}$. But
            $\mathbb{R}$ is simply connected. That is,
            $\pi_{1}(\mathbb{R})=0$. Therefore $\mathbb{R}$ is a
            universal covering space of $S^{1}$. Up to homotopy equivalence,
            $B\mathbb{Z}^{n}=T^{n}$, the $n$ torus. This is because
            $\pi_{1}(B\mathbb{Z}^{n})=\mathbb{Z}^{n}$, and
            $\pi_{n}(B\mathbb{Z}^{n})=0$ for $n\geq{2}$.
            For $S^{n}$, if $n\geq{2}$ then $S^{n}$ is simply connected,
            $\pi_{1}(S^{n})=0$. But then the identity map makes
            $S^{n}$ a covering space for itself. That is,
            $id_{S^{n}}$ is a covering map. But since $S^{n}$ is
            simply connected ($n\geq{2}$), we have that
            $S^{n}$ is a universal covering of itself.
            Moreover, it can be shown that
            $S^{n}$ is a universal covering space of
            $\mathbb{R}P^{n}$ for $n\geq{2}$.
            All universal covering spaces are homotopy
            equivalent to each other.
        \subsubsection{Eilenburg-MacLane Spaces}
            \begin{definition}
                An Eilenberg-MacLane space is a topological space
                $X$ such that there exists a non-trivial group
                $G$ and an $n\in\mathbb{N}$ such that
                $\pi_{n}(X)=G$ and, for all $m\ne{n}$,
                $\pi_{m}(X)=0$.
            \end{definition}
            This $B$ functor takes a group $G$ and spits out
            an Eilenberg-MacLane space. That is,
            $\pi_{1}(BG)=G$, and $\pi_{n}(BG)=0$ for all
            $n\geq{2}$. Eilenberg-MacLane spaces are analogous to
            prime numbers in Number Theory, but for the
            study of topological spaces. These have a
            special notation:
            \begin{notation}
                An Eilenberg-MacLane space $X$ is of the type
                $K(G,n)$ if $\pi_{n}(X)=G$ and, for all
                $m\ne{n}$, $\pi_{m}(X)=0$. We write
                $X\in{K(G,n)}$.
            \end{notation}
            Every principal $G$ bundle over $\mathcal{M}$
            can be imagined as $[\mathcal{M},BG]$.
            A principal $\mathcal{O}(n)$ bundle can be
            identitifed with a map
            $f:\mathcal{M}\rightarrow{B}\mathcal{O}(n)$.
            For example, $f$ be the constant map.
            Constant maps are homotopic to each other. This
            is the easiest bundle.
\end{document}