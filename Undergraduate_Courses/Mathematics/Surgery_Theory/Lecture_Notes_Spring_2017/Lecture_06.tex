\documentclass[crop=false,class=article,oneside]{standalone}
%----------------------------Preamble-------------------------------%
%---------------------------Packages----------------------------%
\usepackage{geometry}
\geometry{b5paper, margin=1.0in}
\usepackage[T1]{fontenc}
\usepackage{graphicx, float}            % Graphics/Images.
\usepackage{natbib}                     % For bibliographies.
\bibliographystyle{agsm}                % Bibliography style.
\usepackage[french, english]{babel}     % Language typesetting.
\usepackage[dvipsnames]{xcolor}         % Color names.
\usepackage{listings}                   % Verbatim-Like Tools.
\usepackage{mathtools, esint, mathrsfs} % amsmath and integrals.
\usepackage{amsthm, amsfonts, amssymb}  % Fonts and theorems.
\usepackage{tcolorbox}                  % Frames around theorems.
\usepackage{upgreek}                    % Non-Italic Greek.
\usepackage{fmtcount, etoolbox}         % For the \book{} command.
\usepackage[newparttoc]{titlesec}       % Formatting chapter, etc.
\usepackage{titletoc}                   % Allows \book in toc.
\usepackage[nottoc]{tocbibind}          % Bibliography in toc.
\usepackage[titles]{tocloft}            % ToC formatting.
\usepackage{pgfplots, tikz}             % Drawing/graphing tools.
\usepackage{imakeidx}                   % Used for index.
\usetikzlibrary{
    calc,                   % Calculating right angles and more.
    angles,                 % Drawing angles within triangles.
    arrows.meta,            % Latex and Stealth arrows.
    quotes,                 % Adding labels to angles.
    positioning,            % Relative positioning of nodes.
    decorations.markings,   % Adding arrows in the middle of a line.
    patterns,
    arrows
}                                       % Libraries for tikz.
\pgfplotsset{compat=1.9}                % Version of pgfplots.
\usepackage[font=scriptsize,
            labelformat=simple,
            labelsep=colon]{subcaption} % Subfigure captions.
\usepackage[font={scriptsize},
            hypcap=true,
            labelsep=colon]{caption}    % Figure captions.
\usepackage[pdftex,
            pdfauthor={Ryan Maguire},
            pdftitle={Mathematics and Physics},
            pdfsubject={Mathematics, Physics, Science},
            pdfkeywords={Mathematics, Physics, Computer Science, Biology},
            pdfproducer={LaTeX},
            pdfcreator={pdflatex}]{hyperref}
\hypersetup{
    colorlinks=true,
    linkcolor=blue,
    filecolor=magenta,
    urlcolor=Cerulean,
    citecolor=SkyBlue
}                           % Colors for hyperref.
\usepackage[toc,acronym,nogroupskip,nopostdot]{glossaries}
\usepackage{glossary-mcols}
%------------------------Theorem Styles-------------------------%
\theoremstyle{plain}
\newtheorem{theorem}{Theorem}[section]

% Define theorem style for default spacing and normal font.
\newtheoremstyle{normal}
    {\topsep}               % Amount of space above the theorem.
    {\topsep}               % Amount of space below the theorem.
    {}                      % Font used for body of theorem.
    {}                      % Measure of space to indent.
    {\bfseries}             % Font of the header of the theorem.
    {}                      % Punctuation between head and body.
    {.5em}                  % Space after theorem head.
    {}

% Italic header environment.
\newtheoremstyle{thmit}{\topsep}{\topsep}{}{}{\itshape}{}{0.5em}{}

% Define environments with italic headers.
\theoremstyle{thmit}
\newtheorem*{solution}{Solution}

% Define default environments.
\theoremstyle{normal}
\newtheorem{example}{Example}[section]
\newtheorem{definition}{Definition}[section]
\newtheorem{problem}{Problem}[section]

% Define framed environment.
\tcbuselibrary{most}
\newtcbtheorem[use counter*=theorem]{ftheorem}{Theorem}{%
    before=\par\vspace{2ex},
    boxsep=0.5\topsep,
    after=\par\vspace{2ex},
    colback=green!5,
    colframe=green!35!black,
    fonttitle=\bfseries\upshape%
}{thm}

\newtcbtheorem[auto counter, number within=section]{faxiom}{Axiom}{%
    before=\par\vspace{2ex},
    boxsep=0.5\topsep,
    after=\par\vspace{2ex},
    colback=Apricot!5,
    colframe=Apricot!35!black,
    fonttitle=\bfseries\upshape%
}{ax}

\newtcbtheorem[use counter*=definition]{fdefinition}{Definition}{%
    before=\par\vspace{2ex},
    boxsep=0.5\topsep,
    after=\par\vspace{2ex},
    colback=blue!5!white,
    colframe=blue!75!black,
    fonttitle=\bfseries\upshape%
}{def}

\newtcbtheorem[use counter*=example]{fexample}{Example}{%
    before=\par\vspace{2ex},
    boxsep=0.5\topsep,
    after=\par\vspace{2ex},
    colback=red!5!white,
    colframe=red!75!black,
    fonttitle=\bfseries\upshape%
}{ex}

\newtcbtheorem[auto counter, number within=section]{fnotation}{Notation}{%
    before=\par\vspace{2ex},
    boxsep=0.5\topsep,
    after=\par\vspace{2ex},
    colback=SeaGreen!5!white,
    colframe=SeaGreen!75!black,
    fonttitle=\bfseries\upshape%
}{not}

\newtcbtheorem[use counter*=remark]{fremark}{Remark}{%
    fonttitle=\bfseries\upshape,
    colback=Goldenrod!5!white,
    colframe=Goldenrod!75!black}{ex}

\newenvironment{bproof}{\textit{Proof.}}{\hfill$\square$}
\tcolorboxenvironment{bproof}{%
    blanker,
    breakable,
    left=3mm,
    before skip=5pt,
    after skip=10pt,
    borderline west={0.6mm}{0pt}{green!80!black}
}

\AtEndEnvironment{lexample}{$\hfill\textcolor{red}{\blacksquare}$}
\newtcbtheorem[use counter*=example]{lexample}{Example}{%
    empty,
    title={Example~\theexample},
    boxed title style={%
        empty,
        size=minimal,
        toprule=2pt,
        top=0.5\topsep,
    },
    coltitle=red,
    fonttitle=\bfseries,
    parbox=false,
    boxsep=0pt,
    before=\par\vspace{2ex},
    left=0pt,
    right=0pt,
    top=3ex,
    bottom=1ex,
    before=\par\vspace{2ex},
    after=\par\vspace{2ex},
    breakable,
    pad at break*=0mm,
    vfill before first,
    overlay unbroken={%
        \draw[red, line width=2pt]
            ([yshift=-1.2ex]title.south-|frame.west) to
            ([yshift=-1.2ex]title.south-|frame.east);
        },
    overlay first={%
        \draw[red, line width=2pt]
            ([yshift=-1.2ex]title.south-|frame.west) to
            ([yshift=-1.2ex]title.south-|frame.east);
    },
}{ex}

\AtEndEnvironment{ldefinition}{$\hfill\textcolor{Blue}{\blacksquare}$}
\newtcbtheorem[use counter*=definition]{ldefinition}{Definition}{%
    empty,
    title={Definition~\thedefinition:~{#1}},
    boxed title style={%
        empty,
        size=minimal,
        toprule=2pt,
        top=0.5\topsep,
    },
    coltitle=Blue,
    fonttitle=\bfseries,
    parbox=false,
    boxsep=0pt,
    before=\par\vspace{2ex},
    left=0pt,
    right=0pt,
    top=3ex,
    bottom=0pt,
    before=\par\vspace{2ex},
    after=\par\vspace{1ex},
    breakable,
    pad at break*=0mm,
    vfill before first,
    overlay unbroken={%
        \draw[Blue, line width=2pt]
            ([yshift=-1.2ex]title.south-|frame.west) to
            ([yshift=-1.2ex]title.south-|frame.east);
        },
    overlay first={%
        \draw[Blue, line width=2pt]
            ([yshift=-1.2ex]title.south-|frame.west) to
            ([yshift=-1.2ex]title.south-|frame.east);
    },
}{def}

\AtEndEnvironment{ltheorem}{$\hfill\textcolor{Green}{\blacksquare}$}
\newtcbtheorem[use counter*=theorem]{ltheorem}{Theorem}{%
    empty,
    title={Theorem~\thetheorem:~{#1}},
    boxed title style={%
        empty,
        size=minimal,
        toprule=2pt,
        top=0.5\topsep,
    },
    coltitle=Green,
    fonttitle=\bfseries,
    parbox=false,
    boxsep=0pt,
    before=\par\vspace{2ex},
    left=0pt,
    right=0pt,
    top=3ex,
    bottom=-1.5ex,
    breakable,
    pad at break*=0mm,
    vfill before first,
    overlay unbroken={%
        \draw[Green, line width=2pt]
            ([yshift=-1.2ex]title.south-|frame.west) to
            ([yshift=-1.2ex]title.south-|frame.east);},
    overlay first={%
        \draw[Green, line width=2pt]
            ([yshift=-1.2ex]title.south-|frame.west) to
            ([yshift=-1.2ex]title.south-|frame.east);
    }
}{thm}

%--------------------Declared Math Operators--------------------%
\DeclareMathOperator{\adjoint}{adj}         % Adjoint.
\DeclareMathOperator{\Card}{Card}           % Cardinality.
\DeclareMathOperator{\curl}{curl}           % Curl.
\DeclareMathOperator{\diam}{diam}           % Diameter.
\DeclareMathOperator{\dist}{dist}           % Distance.
\DeclareMathOperator{\Div}{div}             % Divergence.
\DeclareMathOperator{\Erf}{Erf}             % Error Function.
\DeclareMathOperator{\Erfc}{Erfc}           % Complementary Error Function.
\DeclareMathOperator{\Ext}{Ext}             % Exterior.
\DeclareMathOperator{\GCD}{GCD}             % Greatest common denominator.
\DeclareMathOperator{\grad}{grad}           % Gradient
\DeclareMathOperator{\Ima}{Im}              % Image.
\DeclareMathOperator{\Int}{Int}             % Interior.
\DeclareMathOperator{\LC}{LC}               % Leading coefficient.
\DeclareMathOperator{\LCM}{LCM}             % Least common multiple.
\DeclareMathOperator{\LM}{LM}               % Leading monomial.
\DeclareMathOperator{\LT}{LT}               % Leading term.
\DeclareMathOperator{\Mod}{mod}             % Modulus.
\DeclareMathOperator{\Mon}{Mon}             % Monomial.
\DeclareMathOperator{\multideg}{mutlideg}   % Multi-Degree (Graphs).
\DeclareMathOperator{\nul}{nul}             % Null space of operator.
\DeclareMathOperator{\Ord}{Ord}             % Ordinal of ordered set.
\DeclareMathOperator{\Prin}{Prin}           % Principal value.
\DeclareMathOperator{\proj}{proj}           % Projection.
\DeclareMathOperator{\Refl}{Refl}           % Reflection operator.
\DeclareMathOperator{\rk}{rk}               % Rank of operator.
\DeclareMathOperator{\sgn}{sgn}             % Sign of a number.
\DeclareMathOperator{\sinc}{sinc}           % Sinc function.
\DeclareMathOperator{\Span}{Span}           % Span of a set.
\DeclareMathOperator{\Spec}{Spec}           % Spectrum.
\DeclareMathOperator{\supp}{supp}           % Support
\DeclareMathOperator{\Tr}{Tr}               % Trace of matrix.
%--------------------Declared Math Symbols--------------------%
\DeclareMathSymbol{\minus}{\mathbin}{AMSa}{"39} % Unary minus sign.
%------------------------New Commands---------------------------%
\DeclarePairedDelimiter\norm{\lVert}{\rVert}
\DeclarePairedDelimiter\ceil{\lceil}{\rceil}
\DeclarePairedDelimiter\floor{\lfloor}{\rfloor}
\newcommand*\diff{\mathop{}\!\mathrm{d}}
\newcommand*\Diff[1]{\mathop{}\!\mathrm{d^#1}}
\renewcommand*{\glstextformat}[1]{\textcolor{RoyalBlue}{#1}}
\renewcommand{\glsnamefont}[1]{\textbf{#1}}
\renewcommand\labelitemii{$\circ$}
\renewcommand\thesubfigure{%
    \arabic{chapter}.\arabic{figure}.\arabic{subfigure}}
\addto\captionsenglish{\renewcommand{\figurename}{Fig.}}
\numberwithin{equation}{section}

\renewcommand{\vector}[1]{\boldsymbol{\mathrm{#1}}}

\newcommand{\uvector}[1]{\boldsymbol{\hat{\mathrm{#1}}}}
\newcommand{\topspace}[2][]{(#2,\tau_{#1})}
\newcommand{\measurespace}[2][]{(#2,\varSigma_{#1},\mu_{#1})}
\newcommand{\measurablespace}[2][]{(#2,\varSigma_{#1})}
\newcommand{\manifold}[2][]{(#2,\tau_{#1},\mathcal{A}_{#1})}
\newcommand{\tanspace}[2]{T_{#1}{#2}}
\newcommand{\cotanspace}[2]{T_{#1}^{*}{#2}}
\newcommand{\Ckspace}[3][\mathbb{R}]{C^{#2}(#3,#1)}
\newcommand{\funcspace}[2][\mathbb{R}]{\mathcal{F}(#2,#1)}
\newcommand{\smoothvecf}[1]{\mathfrak{X}(#1)}
\newcommand{\smoothonef}[1]{\mathfrak{X}^{*}(#1)}
\newcommand{\bracket}[2]{[#1,#2]}

%------------------------Book Command---------------------------%
\makeatletter
\renewcommand\@pnumwidth{1cm}
\newcounter{book}
\renewcommand\thebook{\@Roman\c@book}
\newcommand\book{%
    \if@openright
        \cleardoublepage
    \else
        \clearpage
    \fi
    \thispagestyle{plain}%
    \if@twocolumn
        \onecolumn
        \@tempswatrue
    \else
        \@tempswafalse
    \fi
    \null\vfil
    \secdef\@book\@sbook
}
\def\@book[#1]#2{%
    \refstepcounter{book}
    \addcontentsline{toc}{book}{\bookname\ \thebook:\hspace{1em}#1}
    \markboth{}{}
    {\centering
     \interlinepenalty\@M
     \normalfont
     \huge\bfseries\bookname\nobreakspace\thebook
     \par
     \vskip 20\p@
     \Huge\bfseries#2\par}%
    \@endbook}
\def\@sbook#1{%
    {\centering
     \interlinepenalty \@M
     \normalfont
     \Huge\bfseries#1\par}%
    \@endbook}
\def\@endbook{
    \vfil\newpage
        \if@twoside
            \if@openright
                \null
                \thispagestyle{empty}%
                \newpage
            \fi
        \fi
        \if@tempswa
            \twocolumn
        \fi
}
\newcommand*\l@book[2]{%
    \ifnum\c@tocdepth >-3\relax
        \addpenalty{-\@highpenalty}%
        \addvspace{2.25em\@plus\p@}%
        \setlength\@tempdima{3em}%
        \begingroup
            \parindent\z@\rightskip\@pnumwidth
            \parfillskip -\@pnumwidth
            {
                \leavevmode
                \Large\bfseries#1\hfill\hb@xt@\@pnumwidth{\hss#2}
            }
            \par
            \nobreak
            \global\@nobreaktrue
            \everypar{\global\@nobreakfalse\everypar{}}%
        \endgroup
    \fi}
\newcommand\bookname{Book}
\renewcommand{\thebook}{\texorpdfstring{\Numberstring{book}}{book}}
\providecommand*{\toclevel@book}{-2}
\makeatother
\titleformat{\part}[display]
    {\Large\bfseries}
    {\partname\nobreakspace\thepart}
    {0mm}
    {\Huge\bfseries}
\titlecontents{part}[0pt]
    {\large\bfseries}
    {\partname\ \thecontentslabel: \quad}
    {}
    {\hfill\contentspage}
\titlecontents{chapter}[0pt]
    {\bfseries}
    {\chaptername\ \thecontentslabel:\quad}
    {}
    {\hfill\contentspage}
\newglossarystyle{longpara}{%
    \setglossarystyle{long}%
    \renewenvironment{theglossary}{%
        \begin{longtable}[l]{{p{0.25\hsize}p{0.65\hsize}}}
    }{\end{longtable}}%
    \renewcommand{\glossentry}[2]{%
        \glstarget{##1}{\glossentryname{##1}}%
        &\glossentrydesc{##1}{~##2.}
        \tabularnewline%
        \tabularnewline
    }%
}
\newglossary[not-glg]{notation}{not-gls}{not-glo}{Notation}
\newcommand*{\newnotation}[4][]{%
    \newglossaryentry{#2}{type=notation, name={\textbf{#3}, },
                          text={#4}, description={#4},#1}%
}
%--------------------------LENGTHS------------------------------%
% Spacings for the Table of Contents.
\addtolength{\cftsecnumwidth}{1ex}
\addtolength{\cftsubsecindent}{1ex}
\addtolength{\cftsubsecnumwidth}{1ex}
\addtolength{\cftfignumwidth}{1ex}
\addtolength{\cfttabnumwidth}{1ex}

% Indent and paragraph spacing.
\setlength{\parindent}{0em}
\setlength{\parskip}{0em}
%--------------------------Main Document----------------------------%
\begin{document}
    \ifx\ifmathcoursessurgery\undefined
        \section*{Surgery Theory}
        \setcounter{section}{1}
        \renewcommand\thefigure{%
            \arabic{section}.\arabic{figure}%
        }
        \renewcommand\thesubfigure{%
            \arabic{section}.\arabic{figure}.\arabic{subfigure}%
        }
    \fi
    \subsection{Lecture 6: The Brown Representation Theorem}
        A functor $f:\textrm{Spaces}\rightarrow\textrm{Groups}$
        takes a topological space and returns a group. There are
        many examples, such as homology, cohomoloy, and K-Theory.
        Under certain circumstances there is a space
        $B_{\circ{f}}$ such that there is a one-to-one functor
        $f(\mathcal{M})\leftrightarrow[\mathcal{M},B_{\circ{f}}]$,
        where $\mathcal{M}$ is a manifold.
        \begin{example}
            Let $G$ be a group, and $X\in{K}(G,n)$ an
            Eilenberg-MacLane space. This is not usually a
            manifold, but may be a complex, for example.
            As $X\in{K}(G,n)$, we have that
            $\pi_{n}(X)=G$ and, for all $m\ne{n}$,
            $\pi_{m}(X)=0$. THen $K(G,n)$ is the
            $B_{\circ{f}}$, where the $\circ{f}$ is cohomology
            with coefficients in $G$. That is, we have
            $H^{n}(\mathcal{M};G)%
             \leftrightarrow[\mathcal{M},K(G,n)]$
            is a one-to-one mapping.
        \end{example}
        Conside a manifold $\mathcal{M}$ and the semi-group of
        vector bundles $V(\mathcal{M})$ on $\mathcal{M}$
        with $\oplus$ give by the \textit{Whitney Sum}
        (More on that later). The Grothendique construction gives
        us a group $E(\mathcal{M})$ where the elements are vector
        bundles and ``negative,'' vector bundles (Virtual bundels).
        $E$ can then be thought of as a functor from spaces to
        groups: $E:\textrm{Spaces}\rightarrow\textrm{Groups}$.
        There is some sloppiness ahead that will be clarified later.
        There os a space $BO$ such that
        $E(\mathcal{M})\leftrightarrow[\mathcal{M},BO]$
        is a one-to-one mapping. Note that the
        $B_{\circ{f}}$ are classifying spaces. $BO$ is also
        a classifying space. Let $\mathcal{O}(n)$ be the
        set of orthogonal matrices, as defined in a previous
        lecture. $n\times{n}$ matrices such that $A^{T}=A^{-1}$.
        We saw before that there is a natual mapping $\psi_{n}$
        of $\mathcal{O}(n)$ into $\mathcal{O}(n+1)$. We can
        then form the sequence:
        \begin{equation*}
            \mathcal{O}(1)
            \overset{\psi_{1}}{\longrightarrow}
            \mathcal{O}(2)
            \overset{\psi_{2}}{\longrightarrow}
            \mathcal{O}(3)
            \overset{\psi_{3}}{\longrightarrow}
            \mathcal{O}(4)
            \overset{\psi_{4}}{\longrightarrow}
            \cdots
            \mathcal{O}(n)
            \overset{\psi_{n}}{\longrightarrow}
            \cdots
        \end{equation*}
        And define $\mathcal{O}$ to be the
        \textit{direct limit} of this sequence.
        This is a subset of ``infinite'' dimensional
        matrices. Orthogonal matrices act on vector bundles,
        there are ``rotations,'' of the fibers in
        various dimensions. Let $H$ be a vector bundle over
        $\mathcal{M}$. ``Compactify,'' the fibers,
        which are homeomorphic to $\mathbb{R}^{n}$, making
        them now homeomorphic to $S^{n}$. This is, in a way,
        adding a point ``at infinity,'' or performing the
        one point compactification of $\mathbb{R}^{n}$.
        An example is the stereographic projection of the
        sphere onto the plane. The compactification of
        this is adding the ``North Pole,'' which was
        previously ignored as it is projected
        ``to infinity.'' The stereographic projection
        gives a bijection
        $f:S^{n}\setminus\{\textrm{North Pole}\}%
         \rightarrow\mathbb{R}^{n}$.
        We now have the mapping
        $f:\mathbb{R}^{n}\cup\{\infty\}%
         \rightarrow(S^{n}\setminus\{\textrm{North Pole}\})%
         \cup\{\textrm{North Pole}\}=S^{n}$.
        How to we make $\mathbb{R}^{n}\cup\{\infty\}$ into
        a topological space? If
        $\mathcal{U}$ is open in $\mathbb{R}^{n}$, we say
        that it is still open. Moreover, we say that
        $[-\infty,a)=(-\infty,a)\cup\{\infty\}$ and
        $(a,\infty]=(a,\infty)\cup\{\infty\}$ are also
        open sets. The topology on $\mathbb{R}^{n}$ is
        then the topology generated by the three types
        of sets. Compactifying the fibers of a vector bundle
        creates something called a sphere bundle. An
        example is shown in
        Fig.~\ref{fig:Surgery_Theory_%
                  Compactification_of_Vector_Bundle}.
        \begin{figure}
            \centering
            \captionsetup{type=figure}
            \subimport{../../../../tikz/}
                      {Compactification_of_Vector_Bundle}
            \caption{Turning a Vector Bundle into a Sphere Bundle.}
            \label{fig:Surgery_Theory_%
                   Compactification_of_Vector_Bundle}
        \end{figure}
        A spherical fibration is a space $\mathcal{M}$
        where at each point $x$, the fiber of $x$ is
        equivalent to a sphere $S^{n}$ and all
        \textit{transition maps} are homotopy equivalent,
        rather than homeomorphic. The collection of all
        spherical fibrations on a manifold $\mathcal{M}$
        is a semigroup. The Grothendique group associated
        with this is denoted $S(\mathcal{M})$. This group
        is a classifying space. So we have that a vector
        bundle on $\mathcal{M}$ gives rise to a spherical
        fibration on $\mathcal{M}$.
        \begin{equation*}
            \underset{\textrm{Vector Bundle}}{[\mathcal{M},BO]}
            \longrightarrow
            \underset{\textrm{Spherical Fibration}}{[\mathcal{M},BG]}
        \end{equation*}
        We have the following diagrams to consider:
        \begin{figure}
            \centering
            \captionsetup{type=figure}
            \subimport{../../../../tikz/}
                      {Lifting_Property_Commutative_Diagram}
            \caption{Diagrams for the Lifting Propery}
            \label{fig:Surgery_Theory_Lifting_Property_Diagram}
        \end{figure}
        Given $\varphi$ and $f$, we can form
        $\hat{\varphi}:\mathcal{M}\rightarrow{BO}$ by taking
        the composition, $\hat{\varphi}(x)=(f\circ\varphi)(x)$.
        The lifting problem is, given the central diagram,
        can we find a continuous map
        $\hat{\varphi}:\mathcal{M}\rightarrow{BG}$ such that
        the diagram becomes commutative? The answer is not always.
        The final diagram gives rise to all ``kernels,'' like
        in exactness: $BG/BO\dashrightarrow{BO}\rightarrow{BG}$.
        We thus define $G/O\equiv{BG/BO}$. The homotopy collection
        $[\mathcal{M},G/O]$ ``computes,'' the extent of which a map
        $\psi:\mathcal{M}\rightarrow{B}$ can be ``lifted,'' to
        $\hat{\varphi}:\mathcal{M}\rightarrow{BO}$.
        Loops on $\mathbb{R}^{n}$, that is, continuous functions
        $f:S^{1}\rightarrow\mathbb{R}^{n}$, can be
        contracted to a point. To put another way, no loops
        can wrap around ``holes'' in $\mathbb{R}^{n}$.
        Therefore, $[\mathbb{R}^{n},G/O]$ is a point. For
        more examples, we have
        $[S^{1},G/O]=\pi_{1}(G/O)$, and
        $[S^{n},G/P]=\pi_{n}(G/O)$. In our discussions
        $O$ gives rise to a differentiable manifold $BO$.
        We can replace this with piece-wise linear or
        topological and obtain spaces like
        $BPL$ or $BTop$, respectively.
        From the Poincare Conjecture we have that
        $S^{Top}(S^{n})=S^{PL}(S^{n})=\{e\}$. That is,
        the trivial group. Our Surgery Exact Sequence now
        becomes:
        \begin{equation*}
            L_{n+1}(\mathbb{Z}\pi)\rightarrow
            0\rightarrow
            [\Sigma\mathcal{M},G/Top]\rightarrow
            L_{n}(\mathbb{Z}\pi)\rightarrow
            0\rightarrow
            [\mathcal{M},G/Top]\rightarrow
            L_{n-1}(\mathbb{Z}\pi)\rightarrow\cdots
        \end{equation*}
        So $L_{n+1}(\mathbb{Ze}\simeq[S^{n+1},G/Top]%
            =\pi_{n}(G/Top)$,
        therefore $L_{n+1}(G/Top)\simeq{L_{n}(\mathbb{Z}e)}$.
        
        
        
        \par\hfill\par
        \textbf{Learn about Suspending Space, Wall Groups,
                and Grothendique complete of semigroup}
\end{document}