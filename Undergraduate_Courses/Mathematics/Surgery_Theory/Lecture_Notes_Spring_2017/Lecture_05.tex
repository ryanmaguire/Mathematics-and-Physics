\documentclass[crop=false,class=article,oneside]{standalone}
%----------------------------Preamble-------------------------------%
%---------------------------Packages----------------------------%
\usepackage{geometry}
\geometry{b5paper, margin=1.0in}
\usepackage[T1]{fontenc}
\usepackage{graphicx, float}            % Graphics/Images.
\usepackage{natbib}                     % For bibliographies.
\bibliographystyle{agsm}                % Bibliography style.
\usepackage[french, english]{babel}     % Language typesetting.
\usepackage[dvipsnames]{xcolor}         % Color names.
\usepackage{listings}                   % Verbatim-Like Tools.
\usepackage{mathtools, esint, mathrsfs} % amsmath and integrals.
\usepackage{amsthm, amsfonts, amssymb}  % Fonts and theorems.
\usepackage{tcolorbox}                  % Frames around theorems.
\usepackage{upgreek}                    % Non-Italic Greek.
\usepackage{fmtcount, etoolbox}         % For the \book{} command.
\usepackage[newparttoc]{titlesec}       % Formatting chapter, etc.
\usepackage{titletoc}                   % Allows \book in toc.
\usepackage[nottoc]{tocbibind}          % Bibliography in toc.
\usepackage[titles]{tocloft}            % ToC formatting.
\usepackage{pgfplots, tikz}             % Drawing/graphing tools.
\usepackage{imakeidx}                   % Used for index.
\usetikzlibrary{
    calc,                   % Calculating right angles and more.
    angles,                 % Drawing angles within triangles.
    arrows.meta,            % Latex and Stealth arrows.
    quotes,                 % Adding labels to angles.
    positioning,            % Relative positioning of nodes.
    decorations.markings,   % Adding arrows in the middle of a line.
    patterns,
    arrows
}                                       % Libraries for tikz.
\pgfplotsset{compat=1.9}                % Version of pgfplots.
\usepackage[font=scriptsize,
            labelformat=simple,
            labelsep=colon]{subcaption} % Subfigure captions.
\usepackage[font={scriptsize},
            hypcap=true,
            labelsep=colon]{caption}    % Figure captions.
\usepackage[pdftex,
            pdfauthor={Ryan Maguire},
            pdftitle={Mathematics and Physics},
            pdfsubject={Mathematics, Physics, Science},
            pdfkeywords={Mathematics, Physics, Computer Science, Biology},
            pdfproducer={LaTeX},
            pdfcreator={pdflatex}]{hyperref}
\hypersetup{
    colorlinks=true,
    linkcolor=blue,
    filecolor=magenta,
    urlcolor=Cerulean,
    citecolor=SkyBlue
}                           % Colors for hyperref.
\usepackage[toc,acronym,nogroupskip,nopostdot]{glossaries}
\usepackage{glossary-mcols}
%------------------------Theorem Styles-------------------------%
\theoremstyle{plain}
\newtheorem{theorem}{Theorem}[section]

% Define theorem style for default spacing and normal font.
\newtheoremstyle{normal}
    {\topsep}               % Amount of space above the theorem.
    {\topsep}               % Amount of space below the theorem.
    {}                      % Font used for body of theorem.
    {}                      % Measure of space to indent.
    {\bfseries}             % Font of the header of the theorem.
    {}                      % Punctuation between head and body.
    {.5em}                  % Space after theorem head.
    {}

% Italic header environment.
\newtheoremstyle{thmit}{\topsep}{\topsep}{}{}{\itshape}{}{0.5em}{}

% Define environments with italic headers.
\theoremstyle{thmit}
\newtheorem*{solution}{Solution}

% Define default environments.
\theoremstyle{normal}
\newtheorem{example}{Example}[section]
\newtheorem{definition}{Definition}[section]
\newtheorem{problem}{Problem}[section]

% Define framed environment.
\tcbuselibrary{most}
\newtcbtheorem[use counter*=theorem]{ftheorem}{Theorem}{%
    before=\par\vspace{2ex},
    boxsep=0.5\topsep,
    after=\par\vspace{2ex},
    colback=green!5,
    colframe=green!35!black,
    fonttitle=\bfseries\upshape%
}{thm}

\newtcbtheorem[auto counter, number within=section]{faxiom}{Axiom}{%
    before=\par\vspace{2ex},
    boxsep=0.5\topsep,
    after=\par\vspace{2ex},
    colback=Apricot!5,
    colframe=Apricot!35!black,
    fonttitle=\bfseries\upshape%
}{ax}

\newtcbtheorem[use counter*=definition]{fdefinition}{Definition}{%
    before=\par\vspace{2ex},
    boxsep=0.5\topsep,
    after=\par\vspace{2ex},
    colback=blue!5!white,
    colframe=blue!75!black,
    fonttitle=\bfseries\upshape%
}{def}

\newtcbtheorem[use counter*=example]{fexample}{Example}{%
    before=\par\vspace{2ex},
    boxsep=0.5\topsep,
    after=\par\vspace{2ex},
    colback=red!5!white,
    colframe=red!75!black,
    fonttitle=\bfseries\upshape%
}{ex}

\newtcbtheorem[auto counter, number within=section]{fnotation}{Notation}{%
    before=\par\vspace{2ex},
    boxsep=0.5\topsep,
    after=\par\vspace{2ex},
    colback=SeaGreen!5!white,
    colframe=SeaGreen!75!black,
    fonttitle=\bfseries\upshape%
}{not}

\newtcbtheorem[use counter*=remark]{fremark}{Remark}{%
    fonttitle=\bfseries\upshape,
    colback=Goldenrod!5!white,
    colframe=Goldenrod!75!black}{ex}

\newenvironment{bproof}{\textit{Proof.}}{\hfill$\square$}
\tcolorboxenvironment{bproof}{%
    blanker,
    breakable,
    left=3mm,
    before skip=5pt,
    after skip=10pt,
    borderline west={0.6mm}{0pt}{green!80!black}
}

\AtEndEnvironment{lexample}{$\hfill\textcolor{red}{\blacksquare}$}
\newtcbtheorem[use counter*=example]{lexample}{Example}{%
    empty,
    title={Example~\theexample},
    boxed title style={%
        empty,
        size=minimal,
        toprule=2pt,
        top=0.5\topsep,
    },
    coltitle=red,
    fonttitle=\bfseries,
    parbox=false,
    boxsep=0pt,
    before=\par\vspace{2ex},
    left=0pt,
    right=0pt,
    top=3ex,
    bottom=1ex,
    before=\par\vspace{2ex},
    after=\par\vspace{2ex},
    breakable,
    pad at break*=0mm,
    vfill before first,
    overlay unbroken={%
        \draw[red, line width=2pt]
            ([yshift=-1.2ex]title.south-|frame.west) to
            ([yshift=-1.2ex]title.south-|frame.east);
        },
    overlay first={%
        \draw[red, line width=2pt]
            ([yshift=-1.2ex]title.south-|frame.west) to
            ([yshift=-1.2ex]title.south-|frame.east);
    },
}{ex}

\AtEndEnvironment{ldefinition}{$\hfill\textcolor{Blue}{\blacksquare}$}
\newtcbtheorem[use counter*=definition]{ldefinition}{Definition}{%
    empty,
    title={Definition~\thedefinition:~{#1}},
    boxed title style={%
        empty,
        size=minimal,
        toprule=2pt,
        top=0.5\topsep,
    },
    coltitle=Blue,
    fonttitle=\bfseries,
    parbox=false,
    boxsep=0pt,
    before=\par\vspace{2ex},
    left=0pt,
    right=0pt,
    top=3ex,
    bottom=0pt,
    before=\par\vspace{2ex},
    after=\par\vspace{1ex},
    breakable,
    pad at break*=0mm,
    vfill before first,
    overlay unbroken={%
        \draw[Blue, line width=2pt]
            ([yshift=-1.2ex]title.south-|frame.west) to
            ([yshift=-1.2ex]title.south-|frame.east);
        },
    overlay first={%
        \draw[Blue, line width=2pt]
            ([yshift=-1.2ex]title.south-|frame.west) to
            ([yshift=-1.2ex]title.south-|frame.east);
    },
}{def}

\AtEndEnvironment{ltheorem}{$\hfill\textcolor{Green}{\blacksquare}$}
\newtcbtheorem[use counter*=theorem]{ltheorem}{Theorem}{%
    empty,
    title={Theorem~\thetheorem:~{#1}},
    boxed title style={%
        empty,
        size=minimal,
        toprule=2pt,
        top=0.5\topsep,
    },
    coltitle=Green,
    fonttitle=\bfseries,
    parbox=false,
    boxsep=0pt,
    before=\par\vspace{2ex},
    left=0pt,
    right=0pt,
    top=3ex,
    bottom=-1.5ex,
    breakable,
    pad at break*=0mm,
    vfill before first,
    overlay unbroken={%
        \draw[Green, line width=2pt]
            ([yshift=-1.2ex]title.south-|frame.west) to
            ([yshift=-1.2ex]title.south-|frame.east);},
    overlay first={%
        \draw[Green, line width=2pt]
            ([yshift=-1.2ex]title.south-|frame.west) to
            ([yshift=-1.2ex]title.south-|frame.east);
    }
}{thm}

%--------------------Declared Math Operators--------------------%
\DeclareMathOperator{\adjoint}{adj}         % Adjoint.
\DeclareMathOperator{\Card}{Card}           % Cardinality.
\DeclareMathOperator{\curl}{curl}           % Curl.
\DeclareMathOperator{\diam}{diam}           % Diameter.
\DeclareMathOperator{\dist}{dist}           % Distance.
\DeclareMathOperator{\Div}{div}             % Divergence.
\DeclareMathOperator{\Erf}{Erf}             % Error Function.
\DeclareMathOperator{\Erfc}{Erfc}           % Complementary Error Function.
\DeclareMathOperator{\Ext}{Ext}             % Exterior.
\DeclareMathOperator{\GCD}{GCD}             % Greatest common denominator.
\DeclareMathOperator{\grad}{grad}           % Gradient
\DeclareMathOperator{\Ima}{Im}              % Image.
\DeclareMathOperator{\Int}{Int}             % Interior.
\DeclareMathOperator{\LC}{LC}               % Leading coefficient.
\DeclareMathOperator{\LCM}{LCM}             % Least common multiple.
\DeclareMathOperator{\LM}{LM}               % Leading monomial.
\DeclareMathOperator{\LT}{LT}               % Leading term.
\DeclareMathOperator{\Mod}{mod}             % Modulus.
\DeclareMathOperator{\Mon}{Mon}             % Monomial.
\DeclareMathOperator{\multideg}{mutlideg}   % Multi-Degree (Graphs).
\DeclareMathOperator{\nul}{nul}             % Null space of operator.
\DeclareMathOperator{\Ord}{Ord}             % Ordinal of ordered set.
\DeclareMathOperator{\Prin}{Prin}           % Principal value.
\DeclareMathOperator{\proj}{proj}           % Projection.
\DeclareMathOperator{\Refl}{Refl}           % Reflection operator.
\DeclareMathOperator{\rk}{rk}               % Rank of operator.
\DeclareMathOperator{\sgn}{sgn}             % Sign of a number.
\DeclareMathOperator{\sinc}{sinc}           % Sinc function.
\DeclareMathOperator{\Span}{Span}           % Span of a set.
\DeclareMathOperator{\Spec}{Spec}           % Spectrum.
\DeclareMathOperator{\supp}{supp}           % Support
\DeclareMathOperator{\Tr}{Tr}               % Trace of matrix.
%--------------------Declared Math Symbols--------------------%
\DeclareMathSymbol{\minus}{\mathbin}{AMSa}{"39} % Unary minus sign.
%------------------------New Commands---------------------------%
\DeclarePairedDelimiter\norm{\lVert}{\rVert}
\DeclarePairedDelimiter\ceil{\lceil}{\rceil}
\DeclarePairedDelimiter\floor{\lfloor}{\rfloor}
\newcommand*\diff{\mathop{}\!\mathrm{d}}
\newcommand*\Diff[1]{\mathop{}\!\mathrm{d^#1}}
\renewcommand*{\glstextformat}[1]{\textcolor{RoyalBlue}{#1}}
\renewcommand{\glsnamefont}[1]{\textbf{#1}}
\renewcommand\labelitemii{$\circ$}
\renewcommand\thesubfigure{%
    \arabic{chapter}.\arabic{figure}.\arabic{subfigure}}
\addto\captionsenglish{\renewcommand{\figurename}{Fig.}}
\numberwithin{equation}{section}

\renewcommand{\vector}[1]{\boldsymbol{\mathrm{#1}}}

\newcommand{\uvector}[1]{\boldsymbol{\hat{\mathrm{#1}}}}
\newcommand{\topspace}[2][]{(#2,\tau_{#1})}
\newcommand{\measurespace}[2][]{(#2,\varSigma_{#1},\mu_{#1})}
\newcommand{\measurablespace}[2][]{(#2,\varSigma_{#1})}
\newcommand{\manifold}[2][]{(#2,\tau_{#1},\mathcal{A}_{#1})}
\newcommand{\tanspace}[2]{T_{#1}{#2}}
\newcommand{\cotanspace}[2]{T_{#1}^{*}{#2}}
\newcommand{\Ckspace}[3][\mathbb{R}]{C^{#2}(#3,#1)}
\newcommand{\funcspace}[2][\mathbb{R}]{\mathcal{F}(#2,#1)}
\newcommand{\smoothvecf}[1]{\mathfrak{X}(#1)}
\newcommand{\smoothonef}[1]{\mathfrak{X}^{*}(#1)}
\newcommand{\bracket}[2]{[#1,#2]}

%------------------------Book Command---------------------------%
\makeatletter
\renewcommand\@pnumwidth{1cm}
\newcounter{book}
\renewcommand\thebook{\@Roman\c@book}
\newcommand\book{%
    \if@openright
        \cleardoublepage
    \else
        \clearpage
    \fi
    \thispagestyle{plain}%
    \if@twocolumn
        \onecolumn
        \@tempswatrue
    \else
        \@tempswafalse
    \fi
    \null\vfil
    \secdef\@book\@sbook
}
\def\@book[#1]#2{%
    \refstepcounter{book}
    \addcontentsline{toc}{book}{\bookname\ \thebook:\hspace{1em}#1}
    \markboth{}{}
    {\centering
     \interlinepenalty\@M
     \normalfont
     \huge\bfseries\bookname\nobreakspace\thebook
     \par
     \vskip 20\p@
     \Huge\bfseries#2\par}%
    \@endbook}
\def\@sbook#1{%
    {\centering
     \interlinepenalty \@M
     \normalfont
     \Huge\bfseries#1\par}%
    \@endbook}
\def\@endbook{
    \vfil\newpage
        \if@twoside
            \if@openright
                \null
                \thispagestyle{empty}%
                \newpage
            \fi
        \fi
        \if@tempswa
            \twocolumn
        \fi
}
\newcommand*\l@book[2]{%
    \ifnum\c@tocdepth >-3\relax
        \addpenalty{-\@highpenalty}%
        \addvspace{2.25em\@plus\p@}%
        \setlength\@tempdima{3em}%
        \begingroup
            \parindent\z@\rightskip\@pnumwidth
            \parfillskip -\@pnumwidth
            {
                \leavevmode
                \Large\bfseries#1\hfill\hb@xt@\@pnumwidth{\hss#2}
            }
            \par
            \nobreak
            \global\@nobreaktrue
            \everypar{\global\@nobreakfalse\everypar{}}%
        \endgroup
    \fi}
\newcommand\bookname{Book}
\renewcommand{\thebook}{\texorpdfstring{\Numberstring{book}}{book}}
\providecommand*{\toclevel@book}{-2}
\makeatother
\titleformat{\part}[display]
    {\Large\bfseries}
    {\partname\nobreakspace\thepart}
    {0mm}
    {\Huge\bfseries}
\titlecontents{part}[0pt]
    {\large\bfseries}
    {\partname\ \thecontentslabel: \quad}
    {}
    {\hfill\contentspage}
\titlecontents{chapter}[0pt]
    {\bfseries}
    {\chaptername\ \thecontentslabel:\quad}
    {}
    {\hfill\contentspage}
\newglossarystyle{longpara}{%
    \setglossarystyle{long}%
    \renewenvironment{theglossary}{%
        \begin{longtable}[l]{{p{0.25\hsize}p{0.65\hsize}}}
    }{\end{longtable}}%
    \renewcommand{\glossentry}[2]{%
        \glstarget{##1}{\glossentryname{##1}}%
        &\glossentrydesc{##1}{~##2.}
        \tabularnewline%
        \tabularnewline
    }%
}
\newglossary[not-glg]{notation}{not-gls}{not-glo}{Notation}
\newcommand*{\newnotation}[4][]{%
    \newglossaryentry{#2}{type=notation, name={\textbf{#3}, },
                          text={#4}, description={#4},#1}%
}
%--------------------------LENGTHS------------------------------%
% Spacings for the Table of Contents.
\addtolength{\cftsecnumwidth}{1ex}
\addtolength{\cftsubsecindent}{1ex}
\addtolength{\cftsubsecnumwidth}{1ex}
\addtolength{\cftfignumwidth}{1ex}
\addtolength{\cfttabnumwidth}{1ex}

% Indent and paragraph spacing.
\setlength{\parindent}{0em}
\setlength{\parskip}{0em}
%--------------------------Main Document----------------------------%
\begin{document}
    \ifx\ifmathcoursessurgery\undefined
        \section*{Surgery Theory}
        \setcounter{section}{1}
        \renewcommand\thefigure{%
            \arabic{section}.\arabic{figure}%
        }
        \renewcommand\thesubfigure{%
            \arabic{section}.\arabic{figure}.\arabic{subfigure}%
        }
    \fi
    \subsection{Lecture 5: The Wall L-Groups}
        The Wall L-Groups are defined on all commutative rings.
        In fact, there is a functor $L_{n}$ which takes
        commutative rings to groups. Some facts about this:
        \begin{enumerate}
            \item L-Groups are $4$ periodic.
                  For all commutative rings $R$, we have
                  $L_{n}(R)\simeq L_{n+4}(R)$.
                  This is hard to prove.
            \item Surgery theory requires for
                  $R=\mathbb{Z}G$, where $G$ is the
                  fundamental group of the manifold in
                  question, and $\mathbb{Z}G$ is all
                  finite linear combinations of the elements
                  in $G$.
            \item There is a whole theory of computing
                  $L_{n}(R)$ when $R$ is a field,
                  usually denoted $\mathbb{F}$.
            \item Potentially true statement: L-groups
                  only have 2 or 4 torsion, if they have
                  torsion at all. This is hard to prove,
                  as well.
            \item For $G$ equal to the trivial group,
                  $L_{n}(\mathbb{Z}[e])$ we have
                  $L_{n}(\mathbb{Z}[e])%
                   =\begin{cases}%
                        \mathbb{Z},&n\cong{0}(4)\\%
                        0,&n\cong{1}(4)\\%
                        \mathbb{Z}_{2},&n\cong{2}(4)\\%
                        0,&n\cong{3}(4)%
                    \end{cases}$
        \end{enumerate}
        Suppose $\mathcal{M}^{n}$ is a closed
        manifold and $n\geq 5$, and $\pi_{1}(M)=e$.
        Suppose $n=5$. Then:
        \begin{align*}
            L_{6}(\mathbb{Z}[e])
            &\longrightarrow[\mathcal{M},G/Cat]
            \longrightarrow{S^{Cat}}(\mathcal{M})
            \longrightarrow{L_{5}}(\mathbb{Z}[e]\\
            \mathbb{Z}_{2}&\longrightarrow
            [\mathcal{M},G/Cat]
            \overset{f}{\longrightarrow}S^{Cat}(\mathcal{M})
            \longrightarrow{0}
        \end{align*}
        If $n=6$, we have:
        \begin{align*}
            L_{7}(\mathbb{Z}[e])
            &\rightarrow[M,G/Cat]\rightarrow
            S^{cat}(\mathcal{M})\rightarrow L_{6}(\mathbb{Z}[e])\\
            0&\rightarrow [M,G/Cat]
            \overset{g}{\rightarrow}S^{Cat}(\mathcal{M})
            \overset{?}{\rightarrow}\mathbb{Z}_{2}
        \end{align*}
        In the case of $n=4$, there are these
        things called 'Good' groups in which some
        of these results still hold. The
        dimensions can be broken up like this:
        \begin{itemize}
            \item $2$ Completely solved.
            \item $3$ This is Knot Theory.
            \item $4$ Very hard.
            \item $\geq 5$ Surgery Theory.
        \end{itemize}
        How do you classify manifolds?
        \begin{itemize}
            \item In two dimensions the genus
                  (number of wholes) and the
                  orientation (Euler characteristic)
                  gives you everything.
            \item In three dimensions, Thurnston and
                  Perelman did the classification of this.
            \item Four is a big vacuum of unsolved
                  stuff. 'Good' groups come up here.
            \item For every group $G$ there is a
                  manifold $\mathcal{M}$ of dimensions
                  $5$ or greater such that
                  $\pi_{1}(\mathcal{M})=G$
        \end{itemize}
        Let's study $L_{n}(\mathbb{Z}[e])$.
        This is surprisingly hard enough to study.
        The computation of this was known by Brouder,
        but the use of the surgery exact sequences
        wasn't done until Wall (Hence, Wall groups).
        \subsubsection{The Witt Group}
            $L_{0}(\mathbb{Z}[e])$ is equal to something
            called the Witt group $Witt(\mathbb{Z})$.
            First let's talk about the Witt group of fields.
            The Witt group of a field $\mathbb{F}$ is
            the set of symmetric bilinear forms
            $B:V\times{V}\rightarrow\mathbb{F}$
            of finite dimensional vector spaces
            $V$ over $\mathbb{F}$, modulo some
            equivalence relation. So $B$ can be
            represented by some symmetric matrix
            in $M_{n}(\mathbb{F})$ with respect to
            some basis $\{\beta\}$. So if
            $B:V\times{V}\rightarrow\mathbb{F}$
            has matrix $A_{\beta}$ and if
            $D:V\times{V}\rightarrow\mathbb{F}$
            has matrix $A'_{\delta}$,
            then construct the matrix:
            \begin{equation*}
                \tilde{A}=
                \begin{pmatrix}
                    A&0\\
                    0&A'
                \end{pmatrix}_{\beta,\delta}
            \end{equation*}
            Then one can get a Bilinear form
            $B\perp{D}:V\oplus{V}\rightarrow{V}\oplus{V}$
            using this matrix. This is called the
            orthogonal sum. The equivalence relation
            on these Bilinear forms is a bit complicated.
            Consider the matrix:
            \begin{equation*}
                \begin{bmatrix}
                    x&y
                \end{bmatrix}
                \begin{bmatrix}
                    1&0\\
                    0&-1
                \end{bmatrix}
                \begin{bmatrix}
                    x&y
                \end{bmatrix}
                =x^{2}-y^{2}
            \end{equation*}
            This is called a Hyperbolic form $H_{2}(\mathbb{F})$
            \begin{enumerate}
                \item Two forms 
                      $B_{1}:V\times V\rightarrow \mathbb{F}$
                      and $B_{2}:W\times W\rightarrow \mathbb{F}$,
                      with $\dim(V)=\dim(W)$.
                      If $A^{T}[B_{1}]A=[B_{2}]$,
                      then $B_{1}\sim B_{2}$.
                \item We can also write
                      $B_{1}\sim B_{2}$ is
                      $B_{1}%
                       =B_{2}\underset{k}{\perp}H_{2}(\mathbb{F})$.
                      So $H_{2}(\mathbb{F})$ is the $0$
                      element in $Witt(\mathbb{F})$.
                      Note that, since $H_{2}(\mathbb{F})$
                      has dimension $2$,
                      $\dim(B_{1})=\dim(B_{2})\mod{2}$.
                \item What is the inverse of $B_{1}$?
                      It is a form $B_{2}$ for which
                      $B_{1}\perp B_{2}\simeq H_{2}(\mathbb{F})$
            \end{enumerate}
            There is a map, called the signature map,
            of a matrix
            $W(\mathbb{F})\rightarrow%
             L_{0}(\mathbb{Z}[e])\simeq \mathbb{Z}$.
            It is defined for matrices with real eigenvalues.
            It is the number of positive eigenvalues minus
            the number of negative eignvalues.
        \subsubsection{Manifold Structures}
            Let $X$ be a closed manifold. Then a homotopy
            equivalence $f:\mathcal{N}\rightarrow{X}$ is called a
            manifold structure on $X$. Two manifold structures,
            $f_{1}:\mathcal{N}_{1}\rightarrow{\mathcal{M}}$ and
            $f_{2}:\mathcal{N}_{2}\rightarrow{\mathcal{M}}$, are called
            equivalent on $S(X)$ if there is a homeomorphism
            $g:\mathcal{M}\rightarrow\mathcal{N}$ that
            Fig.~\subref{fig:Surgery_Theory_Equivalent_Manifold_Structure_Diagram}
            is a commutative diagram. Since the composition of homeomorphisms
            is a homeomorphism, if $f_{1}:\mathcal{L}\rightarrow{X}$ and
            $f_{2}:\mathcal{M}\rightarrow{X}$ are equivalent manifold structures
            on $X$, and if $f_{2}:\mathcal{M}\rightarrow{X}$
            and $f_{3}:\mathcal{N}\rightarrow{X}$ are equivalent
            manifold structures, then $f_{1}:\mathcal{L}\rightarrow{X}$
            and $f_{3}:\mathcal{N}\rightarrow{X}$ are equivalent manifold
            structures. That is, the diagram shown in
            Fig.~\subref{fig:Surgery_Theory_Equivalent_Manifold_%
                         Structure_Diagram_Equivalence_Relation}
            is a commutative diagram.
            \begin{figure}[H]
                \captionsetup{type=figure}
                \begin{subfigure}[b]{0.49\textwidth}
                    \centering
                    \captionsetup{type=figure}
                    \subimport{../../../../tikz/}
                              {Equivalent_Manifold_Structure_Diagram}
                    \subcaption{Commutative Diagram for Manifold Structures}
                    \label{fig:Surgery_Theory_Equivalent_Manifold_Structure_Diagram}
                \end{subfigure}
                \begin{subfigure}[b]{0.49\textwidth}
                    \centering
                    \captionsetup{type=figure}
                    \subimport{../../../../tikz/}
                              {Equivalent_Manifold_Structure_%
                               Diagram_Equivalence_Relation}
                    \subcaption{Equivalent Manifolds form an Equivalence Relation.}
                    \label{fig:Surgery_Theory_Equivalent_%
                           Manifold_Structure_Diagram_Equivalence_Relation}
                \end{subfigure}
                \label{Commutative Diagrams for Manifold Structures.}
                \label{fig:Commutative_Diagrams_for_Manifold_Structures}
            \end{figure}
\end{document}