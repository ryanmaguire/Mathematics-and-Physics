\documentclass[crop=false,class=article,oneside]{standalone}
%----------------------------Preamble-------------------------------%
%---------------------------Packages----------------------------%
\usepackage{geometry}
\geometry{b5paper, margin=1.0in}
\usepackage[T1]{fontenc}
\usepackage{graphicx, float}            % Graphics/Images.
\usepackage{natbib}                     % For bibliographies.
\bibliographystyle{agsm}                % Bibliography style.
\usepackage[french, english]{babel}     % Language typesetting.
\usepackage[dvipsnames]{xcolor}         % Color names.
\usepackage{listings}                   % Verbatim-Like Tools.
\usepackage{mathtools, esint, mathrsfs} % amsmath and integrals.
\usepackage{amsthm, amsfonts, amssymb}  % Fonts and theorems.
\usepackage{tcolorbox}                  % Frames around theorems.
\usepackage{upgreek}                    % Non-Italic Greek.
\usepackage{fmtcount, etoolbox}         % For the \book{} command.
\usepackage[newparttoc]{titlesec}       % Formatting chapter, etc.
\usepackage{titletoc}                   % Allows \book in toc.
\usepackage[nottoc]{tocbibind}          % Bibliography in toc.
\usepackage[titles]{tocloft}            % ToC formatting.
\usepackage{pgfplots, tikz}             % Drawing/graphing tools.
\usepackage{imakeidx}                   % Used for index.
\usetikzlibrary{
    calc,                   % Calculating right angles and more.
    angles,                 % Drawing angles within triangles.
    arrows.meta,            % Latex and Stealth arrows.
    quotes,                 % Adding labels to angles.
    positioning,            % Relative positioning of nodes.
    decorations.markings,   % Adding arrows in the middle of a line.
    patterns,
    arrows
}                                       % Libraries for tikz.
\pgfplotsset{compat=1.9}                % Version of pgfplots.
\usepackage[font=scriptsize,
            labelformat=simple,
            labelsep=colon]{subcaption} % Subfigure captions.
\usepackage[font={scriptsize},
            hypcap=true,
            labelsep=colon]{caption}    % Figure captions.
\usepackage[pdftex,
            pdfauthor={Ryan Maguire},
            pdftitle={Mathematics and Physics},
            pdfsubject={Mathematics, Physics, Science},
            pdfkeywords={Mathematics, Physics, Computer Science, Biology},
            pdfproducer={LaTeX},
            pdfcreator={pdflatex}]{hyperref}
\hypersetup{
    colorlinks=true,
    linkcolor=blue,
    filecolor=magenta,
    urlcolor=Cerulean,
    citecolor=SkyBlue
}                           % Colors for hyperref.
\usepackage[toc,acronym,nogroupskip,nopostdot]{glossaries}
\usepackage{glossary-mcols}
%------------------------Theorem Styles-------------------------%
\theoremstyle{plain}
\newtheorem{theorem}{Theorem}[section]

% Define theorem style for default spacing and normal font.
\newtheoremstyle{normal}
    {\topsep}               % Amount of space above the theorem.
    {\topsep}               % Amount of space below the theorem.
    {}                      % Font used for body of theorem.
    {}                      % Measure of space to indent.
    {\bfseries}             % Font of the header of the theorem.
    {}                      % Punctuation between head and body.
    {.5em}                  % Space after theorem head.
    {}

% Italic header environment.
\newtheoremstyle{thmit}{\topsep}{\topsep}{}{}{\itshape}{}{0.5em}{}

% Define environments with italic headers.
\theoremstyle{thmit}
\newtheorem*{solution}{Solution}

% Define default environments.
\theoremstyle{normal}
\newtheorem{example}{Example}[section]
\newtheorem{definition}{Definition}[section]
\newtheorem{problem}{Problem}[section]

% Define framed environment.
\tcbuselibrary{most}
\newtcbtheorem[use counter*=theorem]{ftheorem}{Theorem}{%
    before=\par\vspace{2ex},
    boxsep=0.5\topsep,
    after=\par\vspace{2ex},
    colback=green!5,
    colframe=green!35!black,
    fonttitle=\bfseries\upshape%
}{thm}

\newtcbtheorem[auto counter, number within=section]{faxiom}{Axiom}{%
    before=\par\vspace{2ex},
    boxsep=0.5\topsep,
    after=\par\vspace{2ex},
    colback=Apricot!5,
    colframe=Apricot!35!black,
    fonttitle=\bfseries\upshape%
}{ax}

\newtcbtheorem[use counter*=definition]{fdefinition}{Definition}{%
    before=\par\vspace{2ex},
    boxsep=0.5\topsep,
    after=\par\vspace{2ex},
    colback=blue!5!white,
    colframe=blue!75!black,
    fonttitle=\bfseries\upshape%
}{def}

\newtcbtheorem[use counter*=example]{fexample}{Example}{%
    before=\par\vspace{2ex},
    boxsep=0.5\topsep,
    after=\par\vspace{2ex},
    colback=red!5!white,
    colframe=red!75!black,
    fonttitle=\bfseries\upshape%
}{ex}

\newtcbtheorem[auto counter, number within=section]{fnotation}{Notation}{%
    before=\par\vspace{2ex},
    boxsep=0.5\topsep,
    after=\par\vspace{2ex},
    colback=SeaGreen!5!white,
    colframe=SeaGreen!75!black,
    fonttitle=\bfseries\upshape%
}{not}

\newtcbtheorem[use counter*=remark]{fremark}{Remark}{%
    fonttitle=\bfseries\upshape,
    colback=Goldenrod!5!white,
    colframe=Goldenrod!75!black}{ex}

\newenvironment{bproof}{\textit{Proof.}}{\hfill$\square$}
\tcolorboxenvironment{bproof}{%
    blanker,
    breakable,
    left=3mm,
    before skip=5pt,
    after skip=10pt,
    borderline west={0.6mm}{0pt}{green!80!black}
}

\AtEndEnvironment{lexample}{$\hfill\textcolor{red}{\blacksquare}$}
\newtcbtheorem[use counter*=example]{lexample}{Example}{%
    empty,
    title={Example~\theexample},
    boxed title style={%
        empty,
        size=minimal,
        toprule=2pt,
        top=0.5\topsep,
    },
    coltitle=red,
    fonttitle=\bfseries,
    parbox=false,
    boxsep=0pt,
    before=\par\vspace{2ex},
    left=0pt,
    right=0pt,
    top=3ex,
    bottom=1ex,
    before=\par\vspace{2ex},
    after=\par\vspace{2ex},
    breakable,
    pad at break*=0mm,
    vfill before first,
    overlay unbroken={%
        \draw[red, line width=2pt]
            ([yshift=-1.2ex]title.south-|frame.west) to
            ([yshift=-1.2ex]title.south-|frame.east);
        },
    overlay first={%
        \draw[red, line width=2pt]
            ([yshift=-1.2ex]title.south-|frame.west) to
            ([yshift=-1.2ex]title.south-|frame.east);
    },
}{ex}

\AtEndEnvironment{ldefinition}{$\hfill\textcolor{Blue}{\blacksquare}$}
\newtcbtheorem[use counter*=definition]{ldefinition}{Definition}{%
    empty,
    title={Definition~\thedefinition:~{#1}},
    boxed title style={%
        empty,
        size=minimal,
        toprule=2pt,
        top=0.5\topsep,
    },
    coltitle=Blue,
    fonttitle=\bfseries,
    parbox=false,
    boxsep=0pt,
    before=\par\vspace{2ex},
    left=0pt,
    right=0pt,
    top=3ex,
    bottom=0pt,
    before=\par\vspace{2ex},
    after=\par\vspace{1ex},
    breakable,
    pad at break*=0mm,
    vfill before first,
    overlay unbroken={%
        \draw[Blue, line width=2pt]
            ([yshift=-1.2ex]title.south-|frame.west) to
            ([yshift=-1.2ex]title.south-|frame.east);
        },
    overlay first={%
        \draw[Blue, line width=2pt]
            ([yshift=-1.2ex]title.south-|frame.west) to
            ([yshift=-1.2ex]title.south-|frame.east);
    },
}{def}

\AtEndEnvironment{ltheorem}{$\hfill\textcolor{Green}{\blacksquare}$}
\newtcbtheorem[use counter*=theorem]{ltheorem}{Theorem}{%
    empty,
    title={Theorem~\thetheorem:~{#1}},
    boxed title style={%
        empty,
        size=minimal,
        toprule=2pt,
        top=0.5\topsep,
    },
    coltitle=Green,
    fonttitle=\bfseries,
    parbox=false,
    boxsep=0pt,
    before=\par\vspace{2ex},
    left=0pt,
    right=0pt,
    top=3ex,
    bottom=-1.5ex,
    breakable,
    pad at break*=0mm,
    vfill before first,
    overlay unbroken={%
        \draw[Green, line width=2pt]
            ([yshift=-1.2ex]title.south-|frame.west) to
            ([yshift=-1.2ex]title.south-|frame.east);},
    overlay first={%
        \draw[Green, line width=2pt]
            ([yshift=-1.2ex]title.south-|frame.west) to
            ([yshift=-1.2ex]title.south-|frame.east);
    }
}{thm}

%--------------------Declared Math Operators--------------------%
\DeclareMathOperator{\adjoint}{adj}         % Adjoint.
\DeclareMathOperator{\Card}{Card}           % Cardinality.
\DeclareMathOperator{\curl}{curl}           % Curl.
\DeclareMathOperator{\diam}{diam}           % Diameter.
\DeclareMathOperator{\dist}{dist}           % Distance.
\DeclareMathOperator{\Div}{div}             % Divergence.
\DeclareMathOperator{\Erf}{Erf}             % Error Function.
\DeclareMathOperator{\Erfc}{Erfc}           % Complementary Error Function.
\DeclareMathOperator{\Ext}{Ext}             % Exterior.
\DeclareMathOperator{\GCD}{GCD}             % Greatest common denominator.
\DeclareMathOperator{\grad}{grad}           % Gradient
\DeclareMathOperator{\Ima}{Im}              % Image.
\DeclareMathOperator{\Int}{Int}             % Interior.
\DeclareMathOperator{\LC}{LC}               % Leading coefficient.
\DeclareMathOperator{\LCM}{LCM}             % Least common multiple.
\DeclareMathOperator{\LM}{LM}               % Leading monomial.
\DeclareMathOperator{\LT}{LT}               % Leading term.
\DeclareMathOperator{\Mod}{mod}             % Modulus.
\DeclareMathOperator{\Mon}{Mon}             % Monomial.
\DeclareMathOperator{\multideg}{mutlideg}   % Multi-Degree (Graphs).
\DeclareMathOperator{\nul}{nul}             % Null space of operator.
\DeclareMathOperator{\Ord}{Ord}             % Ordinal of ordered set.
\DeclareMathOperator{\Prin}{Prin}           % Principal value.
\DeclareMathOperator{\proj}{proj}           % Projection.
\DeclareMathOperator{\Refl}{Refl}           % Reflection operator.
\DeclareMathOperator{\rk}{rk}               % Rank of operator.
\DeclareMathOperator{\sgn}{sgn}             % Sign of a number.
\DeclareMathOperator{\sinc}{sinc}           % Sinc function.
\DeclareMathOperator{\Span}{Span}           % Span of a set.
\DeclareMathOperator{\Spec}{Spec}           % Spectrum.
\DeclareMathOperator{\supp}{supp}           % Support
\DeclareMathOperator{\Tr}{Tr}               % Trace of matrix.
%--------------------Declared Math Symbols--------------------%
\DeclareMathSymbol{\minus}{\mathbin}{AMSa}{"39} % Unary minus sign.
%------------------------New Commands---------------------------%
\DeclarePairedDelimiter\norm{\lVert}{\rVert}
\DeclarePairedDelimiter\ceil{\lceil}{\rceil}
\DeclarePairedDelimiter\floor{\lfloor}{\rfloor}
\newcommand*\diff{\mathop{}\!\mathrm{d}}
\newcommand*\Diff[1]{\mathop{}\!\mathrm{d^#1}}
\renewcommand*{\glstextformat}[1]{\textcolor{RoyalBlue}{#1}}
\renewcommand{\glsnamefont}[1]{\textbf{#1}}
\renewcommand\labelitemii{$\circ$}
\renewcommand\thesubfigure{%
    \arabic{chapter}.\arabic{figure}.\arabic{subfigure}}
\addto\captionsenglish{\renewcommand{\figurename}{Fig.}}
\numberwithin{equation}{section}

\renewcommand{\vector}[1]{\boldsymbol{\mathrm{#1}}}

\newcommand{\uvector}[1]{\boldsymbol{\hat{\mathrm{#1}}}}
\newcommand{\topspace}[2][]{(#2,\tau_{#1})}
\newcommand{\measurespace}[2][]{(#2,\varSigma_{#1},\mu_{#1})}
\newcommand{\measurablespace}[2][]{(#2,\varSigma_{#1})}
\newcommand{\manifold}[2][]{(#2,\tau_{#1},\mathcal{A}_{#1})}
\newcommand{\tanspace}[2]{T_{#1}{#2}}
\newcommand{\cotanspace}[2]{T_{#1}^{*}{#2}}
\newcommand{\Ckspace}[3][\mathbb{R}]{C^{#2}(#3,#1)}
\newcommand{\funcspace}[2][\mathbb{R}]{\mathcal{F}(#2,#1)}
\newcommand{\smoothvecf}[1]{\mathfrak{X}(#1)}
\newcommand{\smoothonef}[1]{\mathfrak{X}^{*}(#1)}
\newcommand{\bracket}[2]{[#1,#2]}

%------------------------Book Command---------------------------%
\makeatletter
\renewcommand\@pnumwidth{1cm}
\newcounter{book}
\renewcommand\thebook{\@Roman\c@book}
\newcommand\book{%
    \if@openright
        \cleardoublepage
    \else
        \clearpage
    \fi
    \thispagestyle{plain}%
    \if@twocolumn
        \onecolumn
        \@tempswatrue
    \else
        \@tempswafalse
    \fi
    \null\vfil
    \secdef\@book\@sbook
}
\def\@book[#1]#2{%
    \refstepcounter{book}
    \addcontentsline{toc}{book}{\bookname\ \thebook:\hspace{1em}#1}
    \markboth{}{}
    {\centering
     \interlinepenalty\@M
     \normalfont
     \huge\bfseries\bookname\nobreakspace\thebook
     \par
     \vskip 20\p@
     \Huge\bfseries#2\par}%
    \@endbook}
\def\@sbook#1{%
    {\centering
     \interlinepenalty \@M
     \normalfont
     \Huge\bfseries#1\par}%
    \@endbook}
\def\@endbook{
    \vfil\newpage
        \if@twoside
            \if@openright
                \null
                \thispagestyle{empty}%
                \newpage
            \fi
        \fi
        \if@tempswa
            \twocolumn
        \fi
}
\newcommand*\l@book[2]{%
    \ifnum\c@tocdepth >-3\relax
        \addpenalty{-\@highpenalty}%
        \addvspace{2.25em\@plus\p@}%
        \setlength\@tempdima{3em}%
        \begingroup
            \parindent\z@\rightskip\@pnumwidth
            \parfillskip -\@pnumwidth
            {
                \leavevmode
                \Large\bfseries#1\hfill\hb@xt@\@pnumwidth{\hss#2}
            }
            \par
            \nobreak
            \global\@nobreaktrue
            \everypar{\global\@nobreakfalse\everypar{}}%
        \endgroup
    \fi}
\newcommand\bookname{Book}
\renewcommand{\thebook}{\texorpdfstring{\Numberstring{book}}{book}}
\providecommand*{\toclevel@book}{-2}
\makeatother
\titleformat{\part}[display]
    {\Large\bfseries}
    {\partname\nobreakspace\thepart}
    {0mm}
    {\Huge\bfseries}
\titlecontents{part}[0pt]
    {\large\bfseries}
    {\partname\ \thecontentslabel: \quad}
    {}
    {\hfill\contentspage}
\titlecontents{chapter}[0pt]
    {\bfseries}
    {\chaptername\ \thecontentslabel:\quad}
    {}
    {\hfill\contentspage}
\newglossarystyle{longpara}{%
    \setglossarystyle{long}%
    \renewenvironment{theglossary}{%
        \begin{longtable}[l]{{p{0.25\hsize}p{0.65\hsize}}}
    }{\end{longtable}}%
    \renewcommand{\glossentry}[2]{%
        \glstarget{##1}{\glossentryname{##1}}%
        &\glossentrydesc{##1}{~##2.}
        \tabularnewline%
        \tabularnewline
    }%
}
\newglossary[not-glg]{notation}{not-gls}{not-glo}{Notation}
\newcommand*{\newnotation}[4][]{%
    \newglossaryentry{#2}{type=notation, name={\textbf{#3}, },
                          text={#4}, description={#4},#1}%
}
%--------------------------LENGTHS------------------------------%
% Spacings for the Table of Contents.
\addtolength{\cftsecnumwidth}{1ex}
\addtolength{\cftsubsecindent}{1ex}
\addtolength{\cftsubsecnumwidth}{1ex}
\addtolength{\cftfignumwidth}{1ex}
\addtolength{\cfttabnumwidth}{1ex}

% Indent and paragraph spacing.
\setlength{\parindent}{0em}
\setlength{\parskip}{0em}
%--------------------------Main Document----------------------------%
\begin{document}
    \ifx\ifmathcoursessurgery\undefined
        \section*{Surgery Theory}
        \setcounter{section}{1}
        \renewcommand\thefigure{%
            \arabic{section}.\arabic{figure}%
        }
        \renewcommand\thesubfigure{%
            \arabic{section}.\arabic{figure}.\arabic{subfigure}%
        }
    \fi
    \subsection{Lecture 1: Homotopy}
        \begin{wrapfigure}[8]{l}{0.35\textwidth}
            \centering
            \captionsetup{type=figure}
            \subimport{../../../../tikz/}{Homotopy_Example}
            \caption[Homotopy Diagram]{%
                Diagram for a homotopy between two functions
                $f,g:X\rightarrow Y$.%
            }
            \label{%
                fig:surgery_theory_course_homotopy_diagram_%
                for_depicting_what_a_homotopy_is%
            }
        \end{wrapfigure}
        Let $X$ and $Y$ be topological spaces, and let
        $f:{X}\rightarrow{Y}$ and $g:{X}\rightarrow{Y}$
        be continuous functions. We now define what it
        means for $f$ and $g$ to be \textit{homotopic}.
        \begin{definition}
            A homotopy between continuous functions
            $f,g:{X}\rightarrow{Y}$ is a continuous
            function $H:{X}\times{I}\rightarrow{Y}$
            such that $H(x,0)=f(x)$ and $H(x,1)=g(x)$.
        \end{definition}
        \begin{definition}
            Homotopic functions are continuous functions
            $f,g:{X}\rightarrow{Y}$, denoted ${f}\simeq{g}$,
            with a homotopy between them.
        \end{definition}
        \begin{notation}
            The set of continuous functions
            $f:{X}\rightarrow{Y}$ is denoted $C(X,Y)$.
        \end{notation}
        Fig.~\ref{%
            fig:surgery_theory_course_homotopy_diagram_%
            for_depicting_what_a_homotopy_is%
        }
        shows two topological spaces and two homotopic
        continuous functions $f,g:{X}\rightarrow{Y}$.
        \begin{example}
            Let $X=\mathbb{R}^{n}$ and $Y=\mathbb{R}^{m}$. Let
            $f,g:\mathbb{R}^{n}\rightarrow\mathbb{R}^{m}$
            be arbitrary continuous functions.
            The `straight line' homotopy is a homotopy
            between any such functions. Let
            $H:\mathbb{R}^{n}\times{I}\rightarrow\mathbb{R}^{m}$
            be defined by $H(x,t)=(1-t)f(x)+tg(x)$. Then
            $H(x,0)=f(x)$, $H(x,1)=g(x)$, and
            $H$ is continuous. Thus, ${f}\simeq{g}$.
            Note that $g(x)=constant$ is possible.
            Any continuous function
            $f:\mathbb{R}^{n}\rightarrow\mathbb{R}^{m}$
            is homotopic to a point.
        \end{example}
        \begin{wrapfigure}[8]{r}{0.32\textwidth}
            \centering
            \captionsetup{type=figure}
            \vspace{-1ex}
            \subimport{../../../../tikz/}{Homotopy_on_Unit_Interval}
            \caption[Another Homotopy Diagram]{%
                Straight-line homotopy%
            }
            \label{%
                fig:surgery_theory_course_homotopy_diagram_%
                for_straight_line_homotopy%
            }
        \end{wrapfigure}
        One can visualize a homotopy by letting $X=[0,1]$,
        and $Y\subset\mathbb{R}^{2}$ be a nice blob,
        like the one shown in
        Fig.~\ref{%
            fig:surgery_theory_course_homotopy_diagram_for_%
            straight_line_homotopy%
        }.
        Let $f:[0,1]\rightarrow Y$ and $g:[0,1]\rightarrow Y$
        be smooth curves within the blob. Then the homotopy
        $H(x,t)=(1-t)f(x)+tg(x)$ is the map that drags $f(x)$
        to $g(x)$ via the straight line connecting the two
        points. This is done for every point $x\in [0,1]$.
        The next thing to show is that the notion of
        homotopy $\simeq$ is an equivalence relation on the
        set $C(X,Y)$.
        \begin{theorem}
            Homotopic is an equivalence relation.
        \end{theorem}
        \begin{proof}
            We must show that $\simeq$ is reflexive,
            symmetric, and transitive.
            \begin{enumerate}
                \item If ${f}\in{C(X,Y)}$,
                    let $H(x,t)=f(x)$. Then
                    $H\in{C({X}\times{I},Y)}$, $H(x,0)=f(x)$,
                    and $H(x,1)=f(x)$.
                    Therefore $H$ is a homotopy between $f$ and
                    itself. That is, $f\simeq f$.
                \item If ${f,g}\in{C(X,Y)}$ and ${f}\simeq{g}$,
                    then there exists a homotopy $H$
                    between $f$ and $g$.
                    Let $G=H(x,1-t)$. Then
                    $G\in{C({X}\times{I},Y)}$,
                    $G(x,0)=g(x)$, and $G(x,1)=f(x)$.
                    Therefore, ${g}\simeq{f}$.
                \item If ${f,g,h}\in{C(X,Y)}$, ${f}\simeq{g}$,
                    and ${g}\simeq{h}$, then there exists
                    homotopy $H_{1}$ between $f$ and $g$ and
                    a homotopy $H_{2}$ between $g$ and $h$.
                    Let $H:{X}\times{I}\rightarrow{Y}$
                    be defined by:
                    \begin{equation*}
                        H_{3}(x,t)=
                        \begin{cases}
                            H_{1}(x,2t),
                            &{0}\leq{t}\leq\frac{1}{2}\\
                            H_{2}(x,2t-1),
                            &\frac{1}{2}<{t}\leq{1}
                        \end{cases}
                    \end{equation*}
                    By the pasting lemma, 
                    $H_{3}\in{C({X}\times{I},Y)}$.
                    But $H_{3}(x,0)=f(x)$ and
                    $H_{3}(x,1)=h(x)$.
                    Thus, $f\simeq h$.
            \end{enumerate}
        \end{proof}
        \clearpage
        \begin{definition}
            Homotopy equivalent spaces are topological
            spaces $X$ and $Y$ such that there exists functions
            ${f}\in{C(X,Y)}$ and ${g}\in{C(Y,X)}$
            such that
            ${f}\circ{g}\simeq{id_{Y}}$
            and ${g}\circ{f}\simeq{id_{X}}$.
        \end{definition}
        \begin{definition}
            A homeomorphism from a topological space $X$ to a
            topological space $Y$ is a continuous bijection
            $f:{X}\rightarrow{Y}$ such that
            $f^{-1}:{Y}\rightarrow{X}$ is continuous.
        \end{definition}
        \begin{definition}
            Homeomorphic topological spaces are topological
            spaces $X$ and $Y$ such that there
            exists a homeomorphism
            $f:{X}\rightarrow{Y}$ between them.
        \end{definition}
        \begin{theorem}
            \label{%
                theorem:surgery_theory_homeomorphic_%
                implies_homotopy_equivalent%
            }
            If $X$ and $Y$ are homeomorphic, then they
            are homotopy equivalent.
        \end{theorem}
        \begin{proof}
            If $X$ and $Y$ are homeomorphic, then there is a
            homeomorphism $f:X\rightarrow Y$. But then $f$ is a
            continuous map from $X$ to $Y$, and $f^{-1}$ is a
            continuous map from $Y$ to $X$. Moreover,
            ${f}\circ{f^{-1}}=id_{Y}$, and
            ${f^{-1}}\circ{f}=id_{X}$,
            for $f$ is a bijection. But
            ${id_{X}}\simeq{id_{X}}$,
            and ${id_{Y}}\simeq{id_{Y}}$. Therefore, etc.
        \end{proof}
        The study of surgery theory ask about the
        converse of theorem
        \ref{%
            theorem:surgery_theory_homeomorphic_%
            implies_homotopy_equivalent%
        }.
        The converse of this theorem is not always true,
        as we will now demonstrate.
        \begin{theorem}
            \label{%
                theorem:surgery_theory_homotopic_does_%
                not_imply_homeomorphic%
            }
            There exist homotopy equivalent spaces
            that are not homeomorphic.
        \end{theorem}
        \begin{proof}
            Let $X=\mathbb{R}^{2}$ and $Y=\{(0,0)\}$.
            Let $f:{X}\rightarrow{Y}$ be defined by
            $f(x,y)=(0,0)$. Let $g=id_{Y}$. Then
            $g\circ{f}=(0,0)$. Let $H(x,y,t)=(1-t)(x,y)$.
            Then $H$ is continuous, $H(x,y,0)=(x,y)$,
            and $H(x,y,1)=(0,0)$. Thus, $H$ is a
            homotopy between ${g}\circ{f}$ and $id_{X}$, and
            therefore ${g}\circ{f}\simeq{id_{X}}$. But also
            ${f}\circ{g}=id_{Y}$, and
            ${id_{Y}}\simeq{id_{Y}}$.
            Therefore $X$ and $Y$ are homotopy equivalent.
            If $h:{X}\rightarrow{Y}$ is a
            homeomorphism, then it is a bijection. But if $h$
            is a bijection, then $|X|=|Y|$. But
            $\mathbb{R}^{2}$ is uncountable, and $|Y|=1$. A
            contradiction. Therefore $X$ and $Y$ are
            not homeomorphic.
        \end{proof}
        Fig.~\subref{%
            fig:surgery_theory_course_homotopy_equivalence_%
            diagram_of_plane_with_point%
        }
        shows the mapping $f$ between $\mathbb{R}^{2}$ and
        $\{(0,0)\}$.
        Theorem~\ref{%
            theorem:surgery_theory_homotopic_does_not_%
            imply_homeomorphic%
        }
        relies on the fact that $\mathbb{R}^{2}$ and $\{(0,0)\}$
        are of different \textit{cardinality}. However, even
        if the topological spaces $X$ and $Y$ are homotopy
        equivalent, and are of the same cardinality, it is
        still possible that they are not homeomorphic.
        We will need to show that homeomorphisms preserve
        the notion of \textit{compactness}.
        \begin{definition}
            An open cover of a subset $A$ of a topological
            space $X$ with topology $\tau$
            is a set of open sets
            $\mathcal{O}\subset\tau$ such that
            $A\subset\cup_{\mathcal{U}\in\mathcal{O}}\mathcal{U}$.
        \end{definition}
        There's is a fundamental theorem from topology and
        the study of Euclidean spaces that we will use frequently.
        There are some building blocks to get to it.
        \begin{definition}
            A compact subset of a topological space $X$ is a
            set $A\subset{X}$ such that for every open cover
            $\mathcal{O}$ of $A$, there is a finite subcover
            $\Delta\subset\mathcal{O}$. That is,
            $\Delta$ is finite and an open cover of $A$.
        \end{definition}
        \begin{theorem}
            If $X$ is compact and
            $S\subset{X}$ is closed, then
            $S$ is compact.
        \end{theorem}
        \begin{proof}
            For let $\mathcal{O}$ be an open cover
            of $S$. Then
            $\mathcal{O}\cup\{S^{C}\}$ is an open
            cover of $X$, as $S$ is closed and therefore
            $S^{C}$ is open. But $X$ is compact and therefore
            there is an open subcover $\Delta$. But then
            $\Delta\setminus\{S^{C}\}$ is an finite subcover
            of $S$. Therefore, etc.
        \end{proof}
        \begin{theorem}
            If $a,b\in\mathbb{R}$ and $a<b$, then
            $[a,b]$ is compact.
        \end{theorem}
        \begin{proof}
            For suppose not. Then there is an open
            cover $\mathcal{O}$ of $[a,b]$ with no finite
            subcover. Let $A$ be the set
            $A=\{r\in\mathbb{R}:[a,r]%
                 \textrm{ has a finite subcover}\}$.
            As $\mathcal{O}$ is an open cover, there is
            an open subset $\mathcal{U}_{1}\in\mathcal{O}$ such
            that $a\in\mathcal{U}_{1}$. Therefore $A$ is
            not empty. Moreoever, as $[a,b]$ is not compact,
            for all $r\in{A}$, $r<b$. Therefore $A$ is bounded
            above. By the least upper bound property there
            is a $\gamma\in\mathbb{R}$ such that for
            all $r\in{A}$, $r\leq\gamma$. But, as
            $\mathcal{U}_{1}$ is open and $a\in\mathcal{U}_{1}$,
            $a<\gamma\leq{b}$. But then $\gamma\in[a,b]$, and
            thus there is a $\mathcal{U}_{2}$ such that
            $\gamma\in\mathcal{U}_{2}$. But as
            $\mathcal{U}_{2}$ is open, there is an $r>0$ such
            that $(\gamma-r,\gamma+r)\subset\mathcal{U}_{2}$.
            But then $[a,\gamma+r/2]$ has a finite subcover,
            a contradiction as $\gamma$ is the least upper bound
            of $A$. Therefore $[a,b]$ is compact.
        \end{proof}
        \begin{theorem}
            \label{%
                theorem:surgery_theory_%
                product_of_compact_is_compact
            }
            If $A$ and $B$ are compact, then
            $A\times{B}$ is compact (With respect to the
            product topology).
        \end{theorem}
        \begin{proof}
            For let $\mathcal{O}$ be an open cover
            of $A\times{B}$.  Then
            $\{\pi_{A}(\mathcal{U}):\mathcal{U}\in\mathcal{O}\}$,
            that is, the set of projections of open sets in
            $\mathcal{O}$ onto $A$, is an open cover of $A$.
            Similarly for $B$. But $A$ and $B$ are compact, and
            therefore there exists finite subcovers. Taking the
            union of these two gives
            a finite subcover of $A\times{B}$.
        \end{proof}
        \begin{theorem}
            \label{%
                theorem:surgery_theory_finite_%
                product_of_compact_is_compact%
            }
            If $A_{1},\hdots,A_{n}$ are compact,
            then $A_{1}\times\hdots\times{A_{n}}$ is compact.
        \end{theorem}
        The finiteness of the product in Thm.~\ref{%
            theorem:surgery_theory_finite_%
            product_of_compact_is_compact%
        } is
        unnecessary (But it makes the proof easier). There
        is a result called Tychonoff's Theorem, which is
        actually equivalent to the axiom of choice, which
        says that given an arbitrary collection of compact sets,
        the space formed by the product of these sets is also
        compact, with respect to the product topology.
        We can now prove our main result.
        \begin{theorem}[Heine-Borel Theorem]
            \label{theorem:surgery_theory_Heine_Borel}
            A subset $S\subset\mathbb{R}^{n}$ is compact if
            and only if it closed and bounded.
        \end{theorem}
        \begin{proof}
            Suppose $S$ is compact and
            suppose it is unbounded. Then the set of open
            balls about the origin
            $B_{n}(0)=\{\mathbf{x}\in\mathbb{R}^{n}%
                        :\norm{\mathbf{x}}<n\}$
            is an open cover of $S$, since it is an open
            cover of $\mathbb{R}^{n}$, and yet no
            finite subcover exists. For if one did, then there
            is a least $N\in\mathbb{N}$ such that
            $S\subset{B_{N}(0)}$, a contradiction as
            $S$ is unbounded. Therefore $S$ is bounded.
            Furthermore, suppose $S$ is
            not closed. Then there exists a point
            $\mathbf{x}\in{S^{C}}$ such that, for all
            $r>0$, $B_{r}(\mathbf{x})\cap{S}\ne\emptyset$,
            where
            $B_{r}(\mathbf{x})=\{\mathbf{y}\in\mathbb{R}^{n}:%
             \norm{\mathbf{x}-\mathbf{y}}<r\}$.
            Let $\overline{B}_{r}(\mathbf{x})$ be the
            closure of these sets (That is, the closed ball
            about $\mathbf{x}$). Then the set of complements
            $\overline{B}_{n}(\mathbf{x})^{C}$ is an open
            cover of of $S$, for it is an open cover of
            $\mathbb{R}^{n}\setminus\{\mathbf{x}\}$, but no
            finite subcover exists. Thus $S$ is closed. Therefore,
            if $S$ is compact then it is closed and bounded.
            If $S$ is bounded, then there is an $r\in\mathbb{R}$
            such that $S\subset[-r,r]^{n}$. But
            $[-r,r]^{n}$ is the product of compact sets, and
            is therefore compact. But $S$ is closed, and closed
            subsets of compact spaces are compact. Therefore
            $S$ is compact.
        \end{proof}
        This will help find examples and counterexamples for
        the converse of Thm.~\ref{%
            theorem:surgery_theory_homeomorphic_%
            implies_homotopy_equivalent%
        }. Homeomorphisms preserve the notion of compactness.
        \begin{theorem}
            If $X$ and $Y$ are homeomorphic, and if
            $X$ is compact, then $Y$ is compact.
        \end{theorem}
        \begin{proof}
            For if $X$ and $Y$ are homeomorphic, then there
            is a continuous function $f:X\rightarrow{Y}$.
            Let $\mathcal{O}$ be an open cover of $Y$.
            Then $\{f^{-1}(\mathcal{U}):\mathcal{U}\in\mathcal{O}\}$
            is an open cover of $X$. But $X$ is compact, and
            therefore there is a finite subcover
            $\Delta$. But then
            $\{\mathcal{U}\in\mathcal{O}:%
             f^{-1}(\mathcal{U})\in\Delta\}$ is a finite
            subcover of $Y$.
        \end{proof}
        \begin{theorem}
            \label{%
                theorem:surgery_theory_Homotopy_%
                equivalance_of_plane_without_point_and_unit_%
                disc_but_not_homeomorphic%
            }
            $\mathbb{R}^{2}\setminus\{(0,0)\}$ is homotopy
            equivalent to $S^{1}$, but not homeomorphic.
        \end{theorem}
        \begin{proof}
            For let $X=\mathbb{R}^{2}\setminus\{(0,0)\}$,
            and let
            $Y=S^{1}=\{(x,y)\in\mathbb{R}^{2}:x^{2}+y^{2}=1\}$.
            Let $f:{X}\rightarrow{Y}$ be defined by
            $f(x,y)=(x,y)/\norm{(x,y)}$. Let
            $g:{Y}\rightarrow{X}$ be defined by $g(x,y)=(x,y)$.
            Define the function $H$ by
            $H(x,y,t)=(1-t)f(x,y)+tg(x,y)$.
            But then $H(x,y,0)=f(x,y)$,
            and $H(x,y,1)=g(x,y)$. Thus $H$ is a
            homotopy between ${g}\circ{f}$ and $id_{X}$. But also
            $({f}\circ{g})(x,y)=(x,y)$, for all $(x,y)\in S^{1}$.
            Therefore ${f}\circ{g}=id_{Y}$, and
            ${id_{Y}}\simeq{id_{Y}}$. Therefore, $X$ and $Y$ are
            homotopy equivalent. But $X$ is unbounded,
            and is therefore not compact,
            and $Y$ is closed and bounded,
            and is thus compact. But homeomorphisms
            preserve compactness. Therefore $X$ and $Y$ are
            not homeomorphic.
        \end{proof}
        \begin{figure}[H]
            \centering
            \captionsetup{type=figure}
            \begin{subfigure}[b]{0.33\textwidth}
                \captionsetup{type=figure}
                \centering
                \subimport{../../../../tikz/}{Retraction_of_Plane_to_Point}
                \subcaption{%
                    Retraction of $\mathbb{R}^{2}$ to $(0,0)$
                }
                \label{%
                    fig:surgery_theory_course_homotopy_%
                    equivalence_diagram_of_plane_with_point%
                }
            \end{subfigure}
            \begin{subfigure}[b]{0.66\textwidth}
                \captionsetup{type=figure}
                \centering
                \subimport{../../../../tikz/}
                          {HE_Plane_Without_Point_to_Circle}
                \subcaption{%
                        Homotopy Equivalence of
                        $\mathbb{R}^{2}\setminus\{(0,0)\}$
                        and $S^{1}$%
                    }
                \label{%
                    fig:surgery_theory_homotopy_equivalence_%
                    between_the_plane_with_a_point_removed_%
                    and_the_unit_circle%
                }
            \end{subfigure}
            \caption[%
                Examples of Homotopy Equivalences
                That are not Homeomorphic
            ]{%
                Examples of Homotopy Equivalences
                That are not Homeomorphic.
            }
            \label{%
                fig:surgery_theory_course_various_HE_%
                but_not_homeo_examples%
            }
        \end{figure}
        Theorem \ref{%
            theorem:surgery_theory_Homotopy_%
            equivalance_of_plane_without_point_%
            and_unit_disc_but_not_homeomorphic%
        }
        relies on the fact that $S^{1}$ is compact and
        $\mathbb{R}^{2}\setminus\{(0,0)\}$ isn't.
        However, even if $X$ and $Y$ are both
        compact, and of the same cardinality, it is possible
        that they are homotopy equivalent but not homeomorphic.
        We'll need some results about connectedness to show this.
        \begin{definition}
            A disconnected subset of a topological space $X$
            is a set $S\subset{X}$ such that there exists
            disjoint non-empty open sets $X_{1},X_{2}$ such
            that $A=X_{1}\cup{X_{2}}$.
        \end{definition}
        \begin{definition}
            A connected subset is a set that is not
            disconnected.
        \end{definition}
        \begin{theorem}
            If $X$ and $Y$ are homeomorphic and
            $X$ is connected, then $Y$ is connected.
        \end{theorem}
        \begin{proof}
            Suppose not. If $Y$ is disconnected, then
            there are disjoint non-empty open sets $Y_{1},Y_{2}$
            such that $Y=Y_{1}\cup{Y_{2}}$. But as $X$ and $Y$
            are homeomorphic, there is a continuous function
            $f:X\rightarrow{Y}$. But then
            $f^{-1}(Y_{1})$ and $f^{-1}(Y_{2})$ are
            non-empty, as $f$ is a bijection, and moreoever
            they are disjoint open subsets of $X$, as
            $f$ is continuous. But then $X$ is disconnected,
            a contradiction. Therefore $Y$ is connected.
        \end{proof}
        \begin{theorem}
            $[-1,1]$ and $[-1,1]^{2}$ are homotopy equivalent,
            but not homeomorphic.
        \end{theorem}
        \begin{proof}
            Let $X=[-1,1]$ and $Y=[-1,1]^{2}$.
            Let $f:X\rightarrow{Y}$ be defined by
            $f(x)=(x,0)$ and $g:Y\rightarrow{Y}$ be defined
            by $g(x,y)=x$.
            Then $H(x,t)=f(x)$ is a homotopy between
            $g\circ{f}$ and $id_{X}$, and thus
            $g\circ{f}\simeq{id_{X}}$. But also
            $H(x,y,t)=(1-t)g(x,0)+(x,y)$ is a homotopy
            between $f\circ{g}$ and $id_{Y}$, and thus
            $f\circ{g}\simeq{id_{Y}}$. Therefore $X$ and $Y$
            are homotopy equivalent. Suppose $h$ is a
            homeomorphism $h:X\rightarrow{Y}$ and let
            $h(0)=\mathbf{x}\in{Y}$. If $h$ is a homeomorphism
            between $X$ and $y$, then the restriction of
            $h$ to $X\setminus\{0\}$ is a homeomophism
            between $[-1,0)\cup(0,1]$ and
            $[-1,1]^{2}\setminus\{\mathbf{x}\}$. But
            $[-1,1]^{2}\setminus\{\mathbf{x}\}$ is connected,
            and $[-1,0)\cup(0,1]$ is not. But homeomorphisms
            preserve connectedness. Therefore, $X$ and
            $Y$ are not homeomorphic.
        \end{proof}
        If $X$ and $Y$ are of the same dimension,
        it is still possible that they are homotopy equivalent,
        but not homeomorphic. First, we show that
        $S^{2}\setminus\{(0,0,1)\}$ is homeomorphic to $D^{2}$.
        \begin{theorem}
            \label{%
                theorem:surgery_theory_the_sphere_%
                with_a_point_removed_is_homeomorphic_%
                to_the_plane%
            }
            $S^{2}\setminus\{(0,0,1)\}$ is
            homeomorphic to $\mathbb{R}^{2}$
        \end{theorem}
        \begin{proof}
            For let
            $f:S^{2}\setminus\{(0,0,1)\}%
             \rightarrow \mathbb{R}^{2}$
            be the stereographic projection mapping,
            $f(x,y,z)=(\frac{x}{1-z},\frac{y}{1-z})$,
            for $(x,y,z)\in S^{2}\setminus\{(0,0,1)\}$.
            If $(X,Y)\in\mathbb{R}^{2}$, let:
            \begin{align*}
                x&=\frac{2X}{\norm{(X,Y)}^{2}+1}&
                y&=\frac{2Y}{\norm{(X,Y)}^{2}+1}&
                z&=\frac{\norm{(X,Y)}^{2}-1}{\norm{(X,Y)}^{2}+1} 
            \end{align*}
            Then:
            \begin{equation*}
                \bigg(\frac{x}{1-z},\frac{y}{1-z}\bigg)
                =
                \bigg(
                    \frac{\frac{2X}{\norm{(X,Y)}^{2}+ 1}}
                    {\frac{2}{\norm{(X,Y)}^{2}+1}},
                    \frac{\frac{2Y}{\norm{(X,Y)}^{2}+1}}
                    {\frac{2}{\norm{(X,Y)}^{2}+1}}
                \bigg)
                =(X,Y)    
            \end{equation*}
            and
            \begin{align*}
                \norm{(x,y,z)}
                &=\sqrt{
                    \frac{4X^{2}}
                    {\big(\norm{(X,Y)}^{2}+1\big)^{2}}
                    +
                    \frac{4Y^{2}}
                    {\big(\norm{(X,Y)}^{2}+1\big)^{2}}
                    + \frac{(\norm{(X,Y)}^{2}-1)^{2}}
                    {(\norm{(X,Y)}^{2}+1)^{2}}
                }\\
                &=\sqrt{
                    \frac{
                        4\norm{(X,Y)}^{2}
                        +\norm{(X,Y)}^{4}
                        -2\norm{(X,Y)}^{2}+1}
                    {(\norm{(X,Y)}+1)^{2}}
                }\\
                &=\sqrt{
                    \frac{\norm{(X,Y)}^{4}+2\norm{(X,Y)}^{2}+1}
                    {(\norm{(X,Y)}^{2}+1)^{2}}
                }
                =\sqrt{
                    \frac{(\norm{(X,Y)}^{2}+1)^{2}}
                    {(\norm{(X,Y)}^{2}+1)^{2}}
                }
                =1
            \end{align*}
            Thus, $(x,y,z)\in S^{2}\setminus\{(0,0,1)\}$,
            and $f$ is surjective.
            If $f(x_{1},y_{1},z_{1})=f(x_{2},y_{2},z_{2})$,
            then $z_{1}=z_{2}$.
            For as
            $(x_{1},y_{1},z_{1})\in S^{2}\setminus\{(0,0,1)\}$,
            and therefore
            $x_{1}^{2}+y_{1}^{2}=1-z_{1}^{2}$, we have:
            \begin{equation*}
                \norm{(X,Y)}^{2}
                =\frac{x_{1}^2+y_{1}^2}{(1-z_{1})^{2}}
                =\frac{1-z_{1}^{2}}{(1-z_{1})^{2}}
                =\frac{x_{2}^{2}+y_{2}^{2}}{(1-z_{2})^{2}}
                =\frac{1-z_{2}^{2}}{(1-z_{2})^{2}}
            \end{equation*}
            So we have
            $\frac{1-z_{1}^{2}}{(1-z_{1})^{2}}%
             =\frac{1-z_{2}^{2}}{(1-z_{2})^{2}}$.
            Simplifying, we get
            $\frac{1+z_{1}}{1-z_{1}}=\frac{1+z_{2}}{1-z_{2}}$.
            But $f(x)=\frac{1+x}{1-x}$ is an injective function,
            and therefore $z_{1}=z_{2}$.
            From this $x_{1}=x_{2}$ and $y_{1}=y_{2}$.
            Thus, $f$ is a bijection. Moreoever,
            $f$ is continuous and $f^{-1}(X,Y)%
             =(\frac{2X}{\norm{(X,Y)}^{2}+1},%
               \frac{2Y}{\norm{(X,Y)}^{2}+1},%
               \frac{\norm{(X,Y)}^{2}-1}{\norm{(X,Y)}^{2}+1})$,
            which is continuous. $f$ is a homeomorphism.
        \end{proof}
        \begin{figure}[H]
            \captionsetup{type=figure}
            \centering
            \subimport{../../../../tikz/}{Stereographic_Projection}
            \caption[%
                Stereographic Projection%
            ]{%
                Stereographic Projection of the
                Sphere onto the Plane%
            }
            \label{%
                fig:surgery_theory_stereographic_%
                projection_of_sphere_to_plane_homeomorphism%
            }
        \end{figure}
        Fig.~\ref{%
            fig:surgery_theory_stereographic_%
            projection_of_sphere_to_plane_homeomorphism%
        }
        depicts the stereographic projection used
        to prove theorem
        \ref{%
            theorem:surgery_theory_the_sphere_with%
            _a_point_removed_is_homeomorphic_to_the_plane%
        }.
        It can be seen that $(0,0,1)$ projects `to infinity'.
        Because of this, it is not uncommon to call this point
        infinity. Next, we prove that $\mathbb{R}^{2}$ is
        homeomorphic to $D^{2}$, almost completing our claim
        that $S^{2}\setminus\{(0,0,1)\}$
        is homeomorphic to $D^{2}$.
        \begin{theorem}
            $\mathbb{R}^{2}$ is homeomorphic to $D^{2}$.
        \end{theorem}
        \begin{proof}
            Let $f:D^{2}\rightarrow\mathbb{R}^{2}$
            be defined by
            $f(\mathbf{x})%
             =\frac{\mathbf{x}}{1-\norm{\mathbf{x}}}$.
            $f$ is surjective.
            For $\mathbf{0}\mapsto\mathbf{0}$.
            If $\mathbf{y}\in\mathbb{R}^2\setminus\mathbf{0}$,
            then let
            $\mathbf{x}=\frac{\mathbf{y}}{1+\norm{\mathbf{y}}}$.
            Then
            $\norm{\mathbf{x}}%
             =\frac{\norm{\mathbf{y}}}{1+\norm{\mathbf{y}}}<1$,
            and thus $\mathbf{x}\in D^{2}$.
            But
            $f(\mathbf{x})%
             =\frac{\mathbf{y}}{1+\norm{\mathbf{y}}}%
              (1-\frac{\norm{\mathbf{y}}}%
                      {1+\norm{\mathbf{y}}})^{-1}%
             =\mathbf{y}$.
            Moreover, $f$ is injective.
            For if
            $f(\mathbf{x}_{1})=f(\mathbf{x}_{2})$,
            then
            $\frac{\norm{\mathbf{x}_{1}}}%
             {1+\norm{\mathbf{x}_{1}}}%
             =\norm{f(\mathbf{x}_{1})}%
             =\norm{f(\mathbf{x}_{2})}%
             =\frac{\norm{\mathbf{x}_{2}}}%
              {1+\norm{\mathbf{x}}_{2}}$,
            and therefore
            $\norm{\mathbf{x}}_{1}=\norm{\mathbf{x}_{2}}$.
            But
            $\frac{\mathbf{x}_{1}}{1+\norm{\mathbf{x}_{1}}}%
             =\frac{\mathbf{x}_{2}}{1+\norm{\mathbf{x}_{2}}}$,
            and therefore $\mathbf{x}_{1}=\mathbf{x}_{2}$.
            $f$ is bijective.
            Moreover, $f$ is continuous. Finally,
            $f^{-1}(\mathbf{y})%
             =\frac{\mathbf{y}}{1+\norm{\mathbf{y}}}$
            is continuous. $f$ is a homeomorphism.
        \end{proof}
        \begin{theorem}
            $S^{2}\setminus\{(0,0,1)\}$
            is homeomorphic to $D^{2}$.
        \end{theorem}
        \begin{proof}
            For $S^{2}\setminus\{(0,0,1)\}$ is
            homeomorphic to $\mathbb{R}^{2}$, and
            $\mathbb{R}^{2}$ is homeomorphic to $D^{2}$.
            But homeomorphism is an equivalence relation, so
            $S^{2}\setminus\{(0,0,1)\}$
            is homeomorphic to $D^{2}$.
        \end{proof}
        \begin{figure}[H]
            \centering
            \captionsetup{type=figure}
            \subimport{../../../../tikz/}{Turning_Sphere_Into_Plane}
            \caption[%
                Homeomorphism From
                $S^{2}\setminus\{(0,0,1)\}$ and $D^{2}$%
            ]{%
                The units phere with a point removed
                can be continuously deformed
                into the open unit disc.%
            }
            \label{fig:my_label}
        \end{figure}
        We can use the fact that $S^{2}\setminus \{(0,0,1)\}$
        is homeomorphic to $D^{2}$ to construct examples of
        topological manifolds of the same dimensions that are
        homotopy equivalent, but not homeomorphic. We may
        generalize to $S^{2}$ with $n$ points removed is
        homeomorphic to $D^{2}$ with $n-1$ points removed.
        We now define the notion of
        \textit{manifold} and \textit{dimension}.
        \begin{definition}
            An $n$ dimensional manifold is a Hausdorff
            topological space $X$ such that for all
            $p\in{X}$ there is an open neighborhood
            $\mathcal{U}$ of $p$, such that $\mathcal{U}$
            is homeomorphic to $\mathbb{R}^{n}$.
        \end{definition}
        It can be shown that if $X$ and $Y$ are homeomorphic
        manifolds, then they are of the same dimension.
        This is simply because $\mathbb{R}^{n}$ is homeomorphic
        to $\mathbb{R}^{m}$ if and only if $n=m$. Therefore,
        homeomorphisms preserve dimension. We use the fact that
        a sphere is not homeomorphic to a torus. We also use
        the following visual representation of a torus:
        \begin{figure}[H]
            \centering
            \captionsetup{type=figure}
            \subimport{../../../../tikz/}
                      {Plane_Representation_of_Torus}
            \caption[Plane Representation of a Torus]{%
                The unit square with a particular
                equivalence relation on it can be
                used to represent a torus.%
            }
            \label{%
                fig:surgery_theory_plane_representation_%
                of_a_torus%
            }
        \end{figure}
        \begin{theorem}
            There exist manifolds $X$ and $Y$ such that
            $\dim(X)=\dim(Y)$, ${X}\simeq{Y}$,
            yet $X$ and $Y$ are not homeomorphic.
        \end{theorem}
        \begin{proof}
            For let
            $X=S^{2}\setminus\{(0,0,1),(0,1,0),(1,0,0)\}$,
            and let
            $Y=T^{2}\setminus\{(1,0,0)\}$.
            That is, $X$ is a sphere with three points removed,
            and $Y$ is a torus with one point removed. Then
            $\dim(X)=\dim(Y)=2$.
            Moreover, $X\simeq Y$. For $X$ is homeomorphic
            to the plane with $2$ points removed. This is
            homotopy equivalent to a figure $8$.
            Using the square representation of a torus in
            Fig.~\ref{%
                fig:surgery_theory_plane_%
                representation_of_a_torus%
            }
            we see that the torus with a point removed
            is also homotopy equivalent to a figure $8$.
            But homotopy equivalence is an equivalence
            relation, and thus $X\simeq Y$. But the a sphere
            is not homeomorphic to a torus, and similarly a
            sphere with $3$ points removed is not homeomorphic
            to a torus with $1$ point removed.
        \end{proof}
        \begin{figure}[H]
                \centering
                \captionsetup{type=figure}
                \subimport{../../../../tikz/}{Figure_8_HE}
                \caption{%
                    Equivalency of $S^{2}\setminus\{a,b,c\}$,
                    $T^{2}\setminus\{\alpha\}$,
                    and a figure-8.%
                }
                \label{%
                    fig:surgery_theory_homotopy_%
                    equivalence_sphere_with_3_%
                    holes_torus_with_1_hole%
                }
        \end{figure} 
        Fig.~\ref{%
            fig:surgery_theory_homotopy_equivalence_%
            sphere_with_3_holes_torus_with_1_hole%
        }
        shows how both $S^{2}$ with three points
        removed and $T^{2}$ with one point removed are
        homotopy equivalent. Recall that
        $\mathbb{R}^{2}\setminus \{(0,0)\}$ is homotopy
        equivalent to $S^{1}$. In a similar manner,
        the plane with two points removed is homotopy
        equivalent to two circles whose intersection
        contains a single points (That is, a figure-$8$).
        While the ``Proof,'' given was hand wavy,
        the fact that the sphere is not homeomorphic to the
        torus comes from the fact that these two objects
        have different boundary components, something
        preserved by homeomorphism. Intiutively,
        one can think of removing a great circle
        (Or a ``line'') from the sphere.
        Removing such an object creates two disconnected
        components. However, removing a circle from the
        torus still leaves one connected surface.
        The next question is
        ``What about compact manifolds without boundary?''
        \begin{theorem}[The Generalized Poincare-Conjecture]
            If $X$ is an $n$ dimensional manifold that
            is homotopy equivalent to $S^{n+1}$, then $X$
            is homeomorphic to $S^{n+1}$.
        \end{theorem}
        \begin{definition}
            A rigid manifold is a manifold $X$ such that
            for all homotopy equivalent closed manifolds $Y$,
            $X$ is homeomorphic to $Y$.
        \end{definition}
        The question then becomes
        ``Which manifolds are rigid, and which are not?''
        From the Poincare theorem, $S^{n}$ is rigid for all
        $n\in\mathbb{N}$. The first example of a non-rigid
        manifold came in the 1930's from Franz, Reidemeister,
        and de Rham, and is called a Lens Space.
        Let $p$ and $q$ be coprime positive integers.
        Divide $S^{3}$ into $p$ equal parts, and then divide
        this into its northern and southern hemispheres.
        Take a piece of the northern hemisphere and move
        it over $q$ pieces, and then glue this to the
        southern hemisphere. Take the piece that is already
        there and move it over $q$ pieces, and then glue
        that to the northern hemisphere. Repeat this
        process until all slices are done. The is called
        the Lens Space $L(p,q)$. $L(1,1)$ is simply the
        sphere. $L(2,1)$ is the real projective plane
        $\mathbb{RP}^{2}$.
        See Fig.~\ref{%
            fig:surgery_theory_lens_space_drawing%
        }
        to see how this construction occurs.
        It can be shown that for distinct pairs
        $(p,q)$, $(p',q')$, that $L(p,q)$ is homotopy
        equivalent to $L(p',q')$, but not homeomorphic.
        \begin{figure}[H]
            \centering
            \captionsetup{type=figure}
            \subimport{../../../../tikz/}{Lens_Space}
            \caption{How to construct $L(p,q)$.}
            \label{fig:surgery_theory_lens_space_drawing}
        \end{figure}
        We move on to the structure set of topological spaces,
        in particular closed topological
        manifolds $\mathcal{M}$.
        \begin{definition}
                Equivalent homotopies are homotopy equivalences
                $f_{1}:X_{1}\rightarrow Y$,
                $f_{2}:X_{2}\rightarrow Y$,
                denoted $f_{1}\sim{f_{2}}$, such that there
                exists a continuous function
                $g:X_{1}\rightarrow{X_{2}}$ and
                $f_{2}\circ{g}\simeq{f_{1}}$.
            \end{definition}
        The equivalent classes of $Y$ is called the
        structure set of $Y$,.
        denoted $S(Y)$This set contains maps
        like $f_{1}$, $f_{2}$.
        If $g$ is a homeomorphism, then $f_{1}=f_{2}$.
        \begin{example}
                $S(S^{n})=\{S^{n}\}$
            \end{example}
        If $|S(Y)|>1$, then $Y$ is non-rigid.
        \begin{example}
                $|S(L(p,q))|\ne{1}$
            \end{example}
        A few questions naturally arise from the
        definition of the structure set:
        \begin{enumerate}
                \begin{multicols}{2}
                    \item Is $S(Y)$ a group?
                    \begin{itemize}
                        \item Sometimes.
                    \end{itemize}
                    \item Is $S(Y)$ finite?
                    \begin{itemize}
                        \item Sometimes.
                        \begin{itemize}
                            \item $|S(S^{n})| = 1$
                            \item $|S(T^{n})| = 2^{n}$
                        \end{itemize}
                    \end{itemize}
                    \item Can $S(Y)$ be infinite?
                    \begin{itemize}
                        \item Yes.
                        \begin{itemize}
                            \item $|S(\mathbb{RP}^{5})|$ - Finite.
                            \item $|S(\mathbb{RP}^{6})|$ - Finite.
                            \item $|S(\mathbb{RP}^{7})|$ -
                                  \underline{Infinite}.
                            \item $|S(\mathbb{RP}^{8})|$ - Finite.
                        \end{itemize}
                    \end{itemize}
                \end{multicols}
            \end{enumerate}
        A review of some concepts from algebraic topology.
        \begin{definition}
                A path in a topological space $X$ is a
                continuous function $f:I\rightarrow X$
            \end{definition}
        \begin{definition}
                A loop in a topological space $X$ is a
                path $f$ such that $f(0)=f(1)$.
            \end{definition}
        \begin{definition}
            The fundamental group of a topological space
            $X$ is the set
            $\pi_{1}(X)=\{f\in{C(I,X)}:f(0)=f(1)\}/h$,
            where $h$ is the modulo of homotopy,
            equipped with the concatenation operation:
            \begin{equation*}
                (f*g)(t)=
                \begin{cases}
                    f(2t),&0\leq{t}<\frac{1}{2}\\
                    g(2t-1),&\frac{1}{2}\leq{t}<1
                \end{cases}
            \end{equation*}
        \end{definition}
        \begin{theorem}
            If $X$ and $Y$ are homeomorphic topological
            spaces, then $\pi_{1}(X)$ is isomorphic
            to $\pi_{1}(Y)$.
        \end{theorem}
        \begin{proof}
            If $X$ and $Y$ are homemorphic, then there is
            a continuous bijection
            $f:X\rightarrow{Y}$ such that $f^{-1}$ is
            continuous.
            Let $\phi:\pi_{1}(X)\rightarrow\pi_{1}(Y)$
            be the map defined by the image
            $\phi(x(t))=(f\circ{x})(t)$. As $f$ is
            continuous, $\phi(x(t))\in\pi_{1}(Y)$.
            But if $x_{1},x_{2}\in\pi_{1}(X)$, then
            $\phi(x_{1}(t)*x_{2}(t))%
             =\phi(x_{1}(t))*\phi(x_{2}(t))$. Thus
            $\phi$ is a homomorphism. But as
            $f$ is a bijection, so is $\phi$, and
            therefore $\phi$ is an isomorphism.
            Thus, $\pi_{1}(X)$ and
            $\pi_{1}(Y)$ are isomorphic.
        \end{proof}
        \begin{theorem}
            If $X$ and $Y$ are topological spaces,
            and if $\pi_{1}(X)$ and $\pi_{1}(Y)$
            are not isomorphic, then
            $X$ and $Y$ are not homeomorphic.
        \end{theorem}
        Using this theorem we can tell whether or not
        certain spaces are homeomorphic. That is,
        the fundamental group is a
        \textit{topological invariant}.
        \begin{example}
                \
                \begin{enumerate}
                    \begin{multicols}{2}
                        \item $\pi_{1}(S^{n})=\{e\}$ - No Torsion.
                        \item $\pi_{1}(T^{n})=\mathbb{Z}^{n}$ -
                              No Torsion.
                        \item $\pi_{1}(\mathbb{RP}^{n})%
                               =\mathbb{Z}_{2}$ - Torsion.
                        \item $\pi_{1}(L(p,q))=\mathbb{Z}_{p}$
                              - Torsion.
                    \end{multicols}
                \end{enumerate}
            \end{example}
        \begin{definition}
                The order of an element $g$ of a group $G$
                is $O(g)=\inf\{n\in\mathbb{N}:a=a^{n}\}$.
            \end{definition}
        \begin{definition}
                A torsion group is a group $G$ such that
                there exists $g\in G$ such that $1<O(g)<\infty$.
            \end{definition}
        We now arrive at the first ``Surgery Theory''
        based theorem.
        \begin{theorem}
            If $n\geq 5$, $n\equiv{3}\mod{4}$,
            and $\pi_{1}(X)$ is a torsion group,
            then $|S(X^{n})|=\infty$.
        \end{theorem}
        Some other gems:
        $S(\mathbb{C}\mathbb{P}^{n})=\mathbb{Z}_{2}$.
        Chern Manifolds are a thing.
        \subsubsection{The Unsolvable Word Problem}
            \begin{definition}
                A presentation of a group $G$ is a set
                $H\subset{G}$ of generators and a set $R$
                of relations on $H$.
                This is denoted $G=\langle{H}|S\rangle$.
            \end{definition}
            \begin{example}
                \
                \begin{enumerate}
                    \item $\langle{a}|a^{n}=e\rangle$
                          is a the cyclic group of order $n$
                          generated by $a$.
                    \item $\langle{g},h|hg=gh\rangle%
                           =\mathbb{Z}^{2}$
                    \item $\langle{g},h|g^{2}=e,h^{2}=e\rangle%
                           =\mathbb{Z}_{2}*\mathbb{Z}_{n}$
                    \item $\langle{g},h|f^{2}=e,h^{2}=e,%
                           gh=h^{-1}g\rangle=D_{2n}$
                \end{enumerate}
            \end{example}
            \begin{remark}
                The word problem on unsolvability:
                Given two group presentations,
                there is no algorithm to show that
                they are isomorphic.
            \end{remark}
            \begin{definition}
                A finitely presented group is a group with
                a presentation $\langle{H}|R\rangle$ such that $H$
                and $R$ are finite.
            \end{definition}
            \begin{theorem}
                If $n\geq 5$ and $G$ is finitely presented,
                then there is a closed $n$ dimensional manifold
                $\mathcal{M}$ such that $\pi_{1}(\mathcal{M})=G$.
            \end{theorem}
        \subsubsection{%
            Exact Sequences and Surgery Exact Sequences
        }
            \begin{definition}
                An exact sequence
                $\cdots G_{3}%
                 \overset{f_{3}}{\rightarrow}G_{2}%
                 \overset{f_{2}}{\rightarrow}G_{1}%
                 \overset{f_{1}}{\rightarrow}G_{0}$
                is a sequence $f_{n}$ of homomorphisms
                and a sequence $G_{n}$ of groups such that
                $\Ima(f_{n+1})=\ker(f_{n})$
            \end{definition}
            \begin{remark}
                Note, the definition requires that the $f_{n}$
                are \textit{homomorphisms}, not homeomorphisms.
                Homeomorphism is a topological notion,
                not an algebraic one.
            \end{remark}
            \begin{example}
                $O\overset{f}{\rightarrow}G%
                 \overset{g}{\rightarrow}H$.
                $\Ima(f)=0\Rightarrow\ker(g)=0$.
                So $g$ is injective.
            \end{example}
            \begin{example}
                $G\overset{f}{\rightarrow}%
                 H\overset{g}{\rightarrow}O$,
                $\ker(g)=H\Rightarrow\Ima(f)=H$.
                So $f$ is surjective.
            \end{example}
            \begin{definition}
                A short exact sequence is an exact sequence
                $0\overset{f}{\rightarrow}%
                 G\overset{g}{\rightarrow}%
                 H\overset{h}{\rightarrow}%
                 L\overset{\ell}{\rightarrow}0$
            \end{definition}
            We have, from the previous examples,
            that in a short exact sequence $f$ must be
            injective and $g$ must be surjective. We now
            move onto surgery exact sequences (See Wall et. al).
            Let $n\geq 5$, and $\mathcal{M}$ be a closed
            manifold of dimension $n$. Let
            $\pi=\pi_{1}(\mathcal{M})$.
            Let Cat have the following meaning:
            \begin{itemize}
                \item Top: Category of continuous maps.
                      That is, the topological catagory.
                \item PL: Piece-Wise linear category.
                      Maps are piece-wise linear.
                \item Diff: Differentiable category.
                      Maps are diffeomorphisms.
            \end{itemize}
            \begin{example}
                \
                \begin{enumerate}
                    \begin{multicols}{2}
                        \item $S^{Top}(S^{n})=\{S^{n}\}$
                        \item $S^{PL}(S^{n})=\{S^{n}\}$
                        \item $|S^{Diff}(S^{2})|=28$ (Milnor)
                        \item $S^{PL}(T^{n})=\{S^{n}\}$ - Rigid
                        \item $|S^{PL}(T^{n})|=2^{n}$ - Non-Rigid.
                        \item $S^{Diff}(T^{n})$ - Difficult.
                    \end{multicols}
                \end{enumerate}
            \end{example}
            A surgery exact sequence is a sequence of the form:
            \begin{align*}
                S^{Cat}(M\times S')\rightarrow[M\times S',G/Cat]
                &\rightarrow L_{n+1}(\pi_{1}(\mathcal{M}))
                \rightarrow{S^{Cat}}(\mathcal{M})
                \rightarrow\cdots\\
                \cdots
                &\rightarrow{[M,G/Cat]}
                \rightarrow{L_{n}}(\pi_{1}(\mathcal{M}))
            \end{align*}
            Here, $L_{n}(X)$ is a \textit{Wall Group},
            and $[A,B]$ is a type of classifiying space.
\end{document}