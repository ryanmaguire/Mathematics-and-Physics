\documentclass[crop=false,class=article,oneside]{standalone}
%----------------------------Preamble-------------------------------%
%---------------------------Packages----------------------------%
\usepackage{geometry}
\geometry{b5paper, margin=1.0in}
\usepackage[T1]{fontenc}
\usepackage{graphicx, float}            % Graphics/Images.
\usepackage{natbib}                     % For bibliographies.
\bibliographystyle{agsm}                % Bibliography style.
\usepackage[french, english]{babel}     % Language typesetting.
\usepackage[dvipsnames]{xcolor}         % Color names.
\usepackage{listings, lstlinebgrd}      % Verbatim-Like Tools.
\usepackage{mathtools, esint, mathrsfs} % amsmath and integrals.
\usepackage{amsthm, amsfonts}           % Fonts and theorems.
\usepackage{tabularx}
\usepackage{tcolorbox}                  % Frames around theorems.
\usepackage{upgreek}                    % Non-Italic Greek.
\usepackage{paracol}                    % Two-column styling.
\usepackage{wrapfig}                    % Wrap text around figure.
\usepackage{fmtcount, etoolbox}         % For the \book{} command.
\usepackage[newparttoc]{titlesec}       % Formatting chapter, etc.
\usepackage{titletoc}                   % Allows \book in toc.
\usepackage[nottoc]{tocbibind}          % Bibliography in toc.
\usepackage[titles]{tocloft}            % ToC formatting.
\usepackage{multicol, enumitem}         % Multi-column/enumerate.
\usepackage{import}                     % Import external files.
\usepackage{pgfplots, tikz}             % Drawing/graphing tools.
\usetikzlibrary{
    calc,                   % Calculating right angles and more.
    angles,                 % Drawing angles within triangles.
    arrows.meta,            % Latex and Stealth arrows.
    quotes,                 % Adding labels to angles.
    positioning,            % Relative positioning of nodes.
    decorations.markings,   % Adding arrows in the middle of a line.
    patterns,
    arrows,
    shapes,
    shapes.geometric,
    cd,
    hobby,
    babel
}                                       % Libraries for tikz.
\pgfplotsset{compat=1.9}                % Version of pgfplots.
\usepackage[font=scriptsize,
            labelformat=simple,
            labelsep=colon]{subcaption} % Subfigure captions.
\usepackage[font={scriptsize},
            hypcap=true,
            labelsep=colon]{caption}    % Figure captions.
\usepackage{hyperref}                   % Allows for hyperlinks.
\hypersetup{
    colorlinks=true,
    linkcolor=blue,
    filecolor=magenta,
    urlcolor=Cerulean,
    citecolor=SkyBlue
}                           % Colors for hyperref.
\usepackage[toc,acronym,nogroupskip]{glossaries} % Glossaries and acronyms.
\usepackage[subpreambles=false]{standalone}      % Complileable sub files.

% Various font stuff from kiwi.
% Use this for Times text and Computer Modern math
%\usepackage{times}

% Quite nice
%\usepackage[charter, greekfamily=, greekuppercase=italicized]{mathdesign}
%\usepackage[utopia, greekuppercase=italicized]{mathdesign}    % Math is narrower

% Use this for Times text and math
%\usepackage{newtxtext}
%\usepackage[libertine,cmintegrals]{newtxmath}
%\usepackage{fix-cm}

%\usepackage{txfontsb}
% or
%\usepackage{mathptmx}

%\usepackage[scaled=0.92]{helvet}
%\renewcommand{\rmdefault}{ptm}

%\usepackage{mathpazo}    % add possibly `sc` and `osf` options
%\usepackage{eulervm}

%\usepackage{fourier}
%\renewcommand{\rmdefault}{ptm}
%\usepackage{mathptm}

%\usepackage{fontspec}
%\setmainfont{lmodern}

%\usepackage[varg]{txfonts}
%\usepackage{fouriernc}
%\usepackage{mathpazo}

%\usepackage{bookman}
%\usepackage[scaled]{uarial}
%\usepackage[scaled]{helvet}
%\renewcommand*\familydefault{\sfdefault}
%\usepackage[math]{anttor}

%\newcommand\fgeorgia{\fontfamily{jvn}\selectfont}
%\newcommand\ftimes{\fontfamily{ptm}\selectfont}
%\newcommand\fhelvetica{\fontfamily{phv}\selectfont}
%\newcommand\fcourier{\fontfamily{pcr}\selectfont}
%\newcommand\fbookman{\fontfamily{pbk}\selectfont}
%\newcommand\fnewcentury{\fontfamily{pnc}\selectfont}
%\newcommand\fpalatino{\fontfamily{ppl}\selectfont}
%\newcommand\favantgarde{\fontfamily{pag}\selectfont}
%\newcommand\fnormal{\normalfont}
%\newcommand\fsize[1]{\ifnum#1>0\fontsize{#1}{#1}\selectfont\else\normalsize\fi}
%------------------------Theorem Styles-------------------------%
% Define theorem style for default spacing and normal font.
\newtheoremstyle{normal}
    {\topsep}               % Amount of space above the theorem.
    {\topsep}               % Amount of space below the theorem.
    {}                      % Font used for body of theorem.
    {}                      % Measure of space to indent.
    {\bfseries}             % Font of the header of the theorem.
    {}                      % Punctuation between head and body.
    {.5em}                  % Space after theorem head.
    {}

% Define theorem style for default spacing with italicized font.
\newtheoremstyle{normalit}{\topsep}{\topsep}
                {\itshape}{}{\bfseries}{}{.5em}{}

% Italic header environment.
\newtheoremstyle{thmit}{\topsep}{\topsep}{}{}{\itshape}{}{0.5em}{}

% Define italicized environments.
\theoremstyle{normalit}
\newtheorem{theorem}{Theorem}[section]
\newtheorem{lemma}{Lemma}[section]
\newtheorem{corollary}{Corollary}[section]
\newtheorem{proposition}{Proposition}[section]
\newtheorem*{theorem*}{Theorem}

% Define environments with italic headers.
\theoremstyle{thmit}
\newtheorem*{solution}{Solution}
\newtheorem*{fsolution}{Solution}

% Define default environments.
\theoremstyle{normal}
\newtheorem{example}{Example}[section]
\newtheorem{definition}{Definition}[section]
\newtheorem{problem}{Problem}[section]
\newtheorem{question}{Question}[section]
\newtheorem{remark}{Remark}[section]
\newtheorem{properties}{Properties}[section]
\newtheorem{notation}{Notation}[section]
\newtheorem{axiom}{Axiom}[section]
\newtheorem*{properties*}{Properties}
\newtheorem*{remark*}{Remark}
\newtheorem*{definition*}{Definition}
\theoremstyle{plain}

% Define framed environment.
\tcbuselibrary{most}
\newtcbtheorem[use counter*=theorem]{ftheorem}{Theorem}%
    {colback=green!5,colframe=green!35!black,
     fonttitle=\bfseries\upshape}{th}

\newtcbtheorem[use counter*=example]{fdefinition}{Definition}%
    {fonttitle=\bfseries\upshape,
     colback=blue!5!white,colframe=blue!75!black}{def}

\newtcbtheorem[use counter*=example]{fexample}{Example}%
    {fonttitle=\bfseries\upshape,
     colback=red!5!white,colframe=red!75!black}{ex}

\newtcbtheorem[use counter*=notation]{fnotation}{Notation}%
    {fonttitle=\bfseries\upshape,
     colback=SeaGreen!5!white,colframe=SeaGreen!75!black}{ex}

\newtcbtheorem[use counter*=corollary]{fcorollary}{Corollary}%
    {fonttitle=\bfseries\upshape,
     colback=Orchid!5!white,colframe=Orchid!75!black}{ex}

\newenvironment{bproof}{\textit{Proof.}}{\hfill$\square$}
\tcolorboxenvironment{bproof}{blanker,breakable,left=5mm,
                             before skip=10pt,after skip=10pt,
                             borderline west={1mm}{0pt}{red}}
\tcolorboxenvironment{fsolution}
    {enhanced jigsaw,colframe=cyan,interior hidden,breakable}

%--------------------Declared Math Operators--------------------%
\DeclareMathOperator{\Refl}{Refl}           % Reflection operator.
\DeclareMathOperator{\Span}{Span}           % Span of a set of vectors.
\DeclareMathOperator{\Card}{Card}           % Cardinality of set.
\DeclareMathOperator{\Ord}{Ord}             % Ordinal of ordered set.
\DeclareMathOperator{\Tr}{Tr}               % Trace of matrix.
\DeclareMathOperator{\adjoint}{adj}         % Adjoint of matrix.
\DeclareMathOperator{\rk}{rk}               % Rank of operator.
\DeclareMathOperator{\nul}{nul}             % Null space of operator.
\DeclareMathOperator{\sgn}{sgn}             % Sign of a number.
\DeclareMathOperator{\multideg}{mutlideg}   % Multi-Degree (Graphs).
\DeclareMathOperator{\GCD}{GCD}             % Greatest common denominator.
\DeclareMathOperator{\LM}{LM}               % Leading monomial
\DeclareMathOperator{\LC}{LC}               % Leading coefficient.
\DeclareMathOperator{\LT}{LT}               % Leading term.
\DeclareMathOperator{\LCM}{LCM}             % Least common multiple.
\DeclareMathOperator{\Mon}{Mon}             % Monomial.
\DeclareMathOperator{\Spec}{Spec}           % Spectrum.
\DeclareMathOperator{\proj}{proj}           % Projection.
\DeclareMathOperator{\comp}{comp}           % Component.
\DeclareMathOperator{\sinc}{sinc}           % Sinc function.
\DeclareMathOperator{\Ima}{Im}              % Image of operator.
\DeclareMathOperator{\Prin}{Prin}           % Principal value.
\DeclareMathOperator{\Mod}{mod}             % Modulus.
%------------------------New Commands---------------------------%
\DeclarePairedDelimiter\norm{\lVert}{\rVert}
\DeclarePairedDelimiter\ceil{\lceil}{\rceil}
\DeclarePairedDelimiter\floor{\lfloor}{\rfloor}
\newcommand*\diff{\mathop{}\!\mathrm{d}}
\newcommand*\Diff[1]{\mathop{}\!\mathrm{d^#1}}
\renewcommand{\mod}{\ \Mod}
\renewcommand*{\glstextformat}[1]{\textcolor{RoyalBlue}{#1}}
\renewcommand{\glsnamefont}[1]{\textbf{#1}}
\renewcommand\labelitemii{$\circ$}
\renewcommand\thesubfigure{\arabic{chapter}.\arabic{figure}}
\renewcommand\thesubfigure{%
    \arabic{chapter}.\arabic{figure}.\arabic{subfigure}}
\addto\captionsenglish{\renewcommand{\figurename}{Fig.}}
%------------------------Book Command---------------------------%
\makeatletter
\renewcommand\@pnumwidth{1cm}
\newcounter{book}
\renewcommand\thebook{\@Roman\c@book}
\newcommand\book{%
    \if@openright
        \cleardoublepage
    \else
        \clearpage
    \fi
    \thispagestyle{plain}%
    \if@twocolumn
        \onecolumn
        \@tempswatrue
    \else
        \@tempswafalse
    \fi
    \null\vfil
    \secdef\@book\@sbook
}
\def\@book[#1]#2{%
    \ifnum \c@secnumdepth >-3\relax
        \refstepcounter{book}%
        \addcontentsline{toc}{book}{
            \bookname\ \thebook:\hspace{1em}#1
        }
    \else
        \addcontentsline{toc}{book}{#1}%
    \fi
    \markboth{}{}%
    {\centering
     \interlinepenalty \@M
     \normalfont
     \ifnum \c@secnumdepth >-2\relax
       \huge\bfseries \bookname\nobreakspace\thebook
       \par
       \vskip 20\p@
     \fi
     \Huge \bfseries #2\par}%
    \@endbook}
\def\@sbook#1{%
    {\centering
     \interlinepenalty \@M
     \normalfont
     \Huge \bfseries #1\par}%
    \@endbook}
\def\@endbook{
    \vfil\newpage
        \if@twoside
            \if@openright
                \null
                \thispagestyle{empty}%
                \newpage
            \fi
        \fi
        \if@tempswa
            \twocolumn
        \fi
}
\newcommand*\l@book[2]{%
    \ifnum \c@tocdepth >-2\relax
        \addpenalty{-\@highpenalty}%
        \addvspace{2.25em \@plus\p@}%
        \setlength\@tempdima{3em}%
        \begingroup
            \parindent \z@ \rightskip \@pnumwidth
            \parfillskip -\@pnumwidth
            {
                \leavevmode
                \Large \bfseries #1\hfil \hb@xt@\@pnumwidth{
                    \hss #2
                }
            }
            \par
            \nobreak
            \global\@nobreaktrue
            \everypar{\global\@nobreakfalse\everypar{}}%
        \endgroup
    \fi}
\newcommand\bookname{Book}
\renewcommand{\thebook}{\texorpdfstring{\Numberstring{book}}{book}}
\providecommand*{\toclevel@book}{-2}
\makeatother
\titlecontents{chapter}[0pt]
    {\bfseries}
    {\chaptername\ \thecontentslabel:\quad}
    {}
    {\hfill\contentspage}
\titleformat{\part}[display]
    {\Large\bfseries}
    {\partname\nobreakspace\thepart}
    {0mm}
    {\Huge\bfseries}
    \titlecontents{part}[0pt]
    {\large\bfseries}
    {\partname\ \thecontentslabel: \quad}
    {}
    {\hfill\contentspage}
\newcommand{\MarkRightAngle}[4][.3cm]
    {\coordinate (tempa) at ($(#3)!#1!(#2)$);
     \coordinate (tempb) at ($(#3)!#1!(#4)$);
     \coordinate (tempc) at ($(tempa)!0.5!(tempb)$);%midpoint
     \draw (tempa) -- ($(#3)!2!(tempc)$) -- (tempb);}
%--------------------------LENGTHS------------------------------%
% Spacings for the Table of Contents.
\addtolength{\cftsecnumwidth}{1ex}
\addtolength{\cftsubsecindent}{1ex}
\addtolength{\cftsubsecnumwidth}{1ex}
\addtolength{\cftfignumwidth}{1ex}
\addtolength{\cfttabnumwidth}{1ex}

% Spacing for multi-column and enumerate environments.
\setlength{\multicolsep}{6pt}
\setlist[enumerate]{itemsep=0pt,topsep=3pt}

% Indent and paragraph spacing.
\setlength{\parindent}{0em}
\setlength{\parskip}{0em}
%--------------------------Main Document----------------------------%
\begin{document}
    \ifx\ifmathcoursessurgery\undefined
        \section*{Surgery Theory}
        \setcounter{section}{2}
        \renewcommand\thesubfigure{%
            \arabic{section}.\arabic{figure}.\arabic{subfigure}%
        }
    \else
        \section{Lecture Notes from Fall 2018 (Wellesley College)}
    \fi
    \subsection{Lecture 1: Singular and Simplicial Homology}
        The fundamental group $\pi_{1}(X)$ is useful for
        studying low dimensional spaces. However, it is poor for
        studying higher dimensional spaces since, for example,
        it is unable to distinguish spheres of dimensions
        $n\geq 2$. The first solution to this is to study
        the homotopy groups $\pi_{n}(X)$. For this, we have
        that $\pi_{i}(X)=0$ for $i<n$, and $\mathbb{Z}$ for
        $i=n$. A drawback is that homotopy groups are
        difficult to compute. The problem of $\pi_{i}(S^{n})$
        is very difficult for when $i>n$. Homology groups,
        $H_{n}(X)$ are one such solution to this difficulty.
        Homology groups share some characteristics with
        homotopy groups. If $X$ is a CW complex, then $H_{n}(X)$
        depends only on the $(n+1)$-skeleton of $X$. Also,
        $H_{i}(S^{n})$ and $\pi_{i}(S^{n})$ are isomorphic for
        $1\leq i\leq n$. One benefit is that $H_{i}(S^{n})=0$
        for $i>n$.
        \subsubsection{Simplicial Homology}
            The torus, projective plane, and Klein bottle can
            be created from a square by identifying opposite
            edges in certain ways. We can divide the square
            into two triangles, meaning these surfaces can by
            built from two triangles and then identifying
            edges. This can be done for all closed polygons
            as well. We can generalize this to $n$ dimensions
            by considering the $n-\textrm{simplex}$, which is
            the convex hull of a set of $n+1$ points in
            $\mathbb{R}^{n}$ that do not lie in an
            $n-\textrm{dimensional}$ hyperplane.
            $n-\textrm{Simplexes}$ are denoted $\Delta^{n}$.
            The interior of $\Delta^{n}$ is denoted
            $\mathring{\Delta}^{n}$. A face of an
            $n-\textrm{simplex}$ is a simplex formed by
            removing one of the vertices from $\Delta^{n}$.
            \begin{definition}
                A $\Delta-\textrm{Complex}$ on a space $X$ is
                a set of maps
                $\sigma_{\alpha}:\Delta^{n}\rightarrow X$
                such that:
                \begin{enumerate}
                    \item The restriction
                        $\sigma_{\alpha}|\mathring{\Delta}^{n}$
                        is injected. Each point $x\in X$ is
                        in the image of only one such
                        $\mathring{\Delta}^{n}$.
                    \item Each restriction of $\sigma_{\alpha}$
                        to a face is equal to one of the maps
                        $\sigma_{\beta}:%
                         \Delta^{n-1}\rightarrow X$.
                    \item A set $A\subset X$ is open if and only
                        if $\sigma_{\alpha}^{-1}$ is open in
                        $\Delta^{n}$ for all $\sigma_{\alpha}$.
                \end{enumerate}
            \end{definition}
            Let $X$ be a $\Delta-\textrm{Complex}$, and let
            $\Delta_{n}(X)$ be the free abelian group whose
            basis is the open $n-\textrm{simplices}$ of $X$.
            Elements of $\Delta_{n}(X)$ are called
            $n-\textrm{chains}$ and can be written as
            $\sum_{\alpha}n_{\alpha}e_{\alpha}^{n}$, where
            $n_{\alpha}$ is an integer.
            \begin{definition}
                The Boundary Homomorphism
                $\partial_{n}:%
                 \Delta_{n}(X)\rightarrow\Delta_{n-1}(X)$
                is the map:
                \begin{equation*}
                    \partial_{n}(\sigma_{\alpha})=
                    \sum_{i}(-1)^{i}\sigma_{\alpha}|
                    (v_{0},\hdots,v_{i-1},v_{i+1},\hdots,v_{n})
                \end{equation*}
            \end{definition}
            \begin{theorem}
                The composition of
                $\Delta_{n}(X)%
                 \overset{\partial_{n}}{\longrightarrow}%
                 \Delta_{n-1}(X)%
                 \overset{\partial_{n-1}}{\longrightarrow}%
                 \Delta_{n-2}(X)$
                is zero.
            \end{theorem}
            \begin{definition}
                The $n^{th}$ simplicial homology group of
                $X$, denoted $H_{n}^{\Delta}(X)$,
                is the quotient group
                $\ker(\partial_{n})/\Ima(\partial_{n+1})$
                formed from the chain complex
                $\cdots\longrightarrow\Delta_{n}(X)%
                 \overset{\partial_{n}}{\longrightarrow}%
                 \Delta_{n-1}(X)%
                 \overset{\partial_{n-1}}{\longrightarrow}%
                 \Delta_{n-2}(X)\longrightarrow\cdots$
            \end{definition}
            The triangle on vertices $a,b,c$ can be defined
            by $[a,b,c]$. We have:
            $\partial_{2}([a,b,c])=[b,c]-[a,c]+[a,b]$.
            The negative sign on $[a,c]$ is used to preserve
            orientation. That is, if you start at $b$,
            travel to $c$, then to $a$, and finally loop
            back to $b$, this should be the ``same'' as going
            from $b$ to $c$, then ``negative'' $a$ to $c$, and
            finally $a$ to $b$. Note that then:
            \begin{equation*}
                \partial_{2}\partial_{1}([a,b,c])
                =\partial_{1}\big([b,c]-[a,c]+[a,b]\big)
                =\partial_{1}\big([b,c]\big)
                -\partial_{1}\big([a,c]\big)
                +\partial_{1}\big([a,b]\big)
                =(b-c)+(c-a)+(a-b)=0
            \end{equation*}
            We can perform the same calculation for the tetrahedron:
            \begin{align*}
                \partial_{3}\partial_{2}\big([a,b,c,d]\big)
                =&\hspace{0.5em}\partial_{2}\big([b,c,d]\big)
                 -\partial_{2}\big([a,c,d]\big)
                 +\partial_{2}\big([a,b,d]\big)
                 -\partial_{2}\big([a,b,c]\big)\\
                =&\hspace{0.5em}\big([c,d]-[b,d]+[b,c]\big)
                 -\big([c,d]-[a,d]+[a,c]\big)+\\
                &\hspace{0.5em}\big([b,d]-[a,d]+[a,b]\big)
                 -\big([b,c]-[a,c]+[a,b]\big)\\
                =&\hspace{0.5em}0
            \end{align*}
            Given a complex $W$, we look at the following chain:
            \begin{equation*}
                C_{n+1}\overset{\partial_{n}}{\longrightarrow}
                C_{n}\overset{\partial_{n-1}}{\longrightarrow}
                C_{n-1}\overset{\partial_{n-2}}{\longrightarrow}
                \cdots\longrightarrow
                C_{2}\overset{\partial_{1}}{\longrightarrow}C_{1}
            \end{equation*}
            Where $\Ima(\partial_{n+1)})\subset\ker(\partial_{n})$
            and $H_{r}^{\Delta}(W)=\ker(\partial_{n})/\Ima(\partial_{n+1})$.
            We call the elements of $\Ima(\partial_{n+1})$
            \textit{boundaries}, and the elements of
            $\ker(\partial_{n})$ \textit{cycles}.
            \begin{example}
                Let $W=[a,b]$. That is, the line connecting
                $a$ and $b$. Then $C_{2}=0$ and
                $C_{1}=\mathbb{Z}\{[a,b]\}\simeq\mathbb{Z}$.
                But also
                $C_{0}=\mathbb{Z}\{[a],[b]\}\simeq\mathbb{Z}^{2}$.
                So we have
                $0\rightarrow{C_{1}}\rightarrow{C_{0}}\rightarrow0$.
                Now $\Ima(\partial_{1})\subset\ker(\partial_{0})$,
                and $\partial_{0}$ is a mapping from
                $\mathbb{Z}^{2}$ into $0$, and thus the kernel
                of $\partial_{0}$ is all of $\mathbb{Z}^{2}$.
                Moreover, since $\partial_{1}$ is a
                homomorphism, either the image of $\partial_{1}$
                is $0$ or $\partial_{1}$ is injective. But
                $\partial_{1}([a,b])=b-a\ne0$, and therefore
                $\partial_{1}$ is injective. So we have
                $H_{0}^{\Delta}(W)=\mathbb{Z}^{2}/\mathbb{Z}=\mathbb{Z}$.
                For $n>0$, $H_{n}^{\Delta}(W)=0$.
            \end{example}
            \begin{example}
                Let $W=[a,b,c]$, a triangle including it's interior.
                Then $C_{3}=0$ and
                $C_{2}=\mathbb{Z}\{[a,b,c]\}\simeq\mathbb{Z}$.
                Also $C_{1}=\mathbb{Z}\{b-a,c-a,c-b\}=\mathbb{Z}^{3}$.
                $\partial_{2}$ is not the zero mapping, and thus
                $\Ima(\partial_{2})=\mathbb{Z}$. Also $\ker(\partial_{2})=0$,
                and thus $H_{2}^{\Delta}(W)=0/0=0$. For $\partial_{1}$ we have
                $\partial_{1}([b,c])=\partial_{1}([a,c])-\partial_{1}([a,b])$,
                and thus $\Ima(\partial_{1})=\mathbb{Z}^{2}$. But then
                $\ker(\partial_{1})=\mathbb{Z}$. Therefore
                $H_{1}^{\Delta}(W)=\mathbb{Z}/\mathbb{Z}=0$.
                Finally, $\partial_{0}$ is the zero mapping, and
                thus the kernel is $\ker(\partial_{0})=\mathbb{Z}^{3}$.
                Therefore
                $H_{0}^{\Delta}(W)=\mathbb{Z}^{3}/\mathbb{Z}^{2}=\mathbb{Z}$
            \end{example}
            \begin{example}
                Let $W=\{[a,b],[b,c],[a,c]\}$, a triangle without
                its interior. Then we have that $C_{2}=0$ and
                $C_{1}=\mathbb{Z}\{[a,b],[b,c],[a,c]\}=\mathbb{Z}^{3}$.
                $\partial_{2}$ is the zero mapping and thus
                $\Ima(\partial_{2})=0$. From the previous example
                we saw that $\Ima(\partial_{1})=\mathbb{Z}^{2}$,
                and thus $\ker(\partial_{1})=\mathbb{Z}$.
                Thus $H_{1}^{\Delta}(W)=\mathbb{Z}/0=\mathbb{Z}$.
                Finally,
                $H_{0}^{\Delta}(W)=\mathbb{Z}^{3}/\mathbb{Z}^{2}=\mathbb{Z}$.
            \end{example}
            We see that removing the interior of the triangle changed what
            the homology groups are. As will be seen later, the homology
            groups are way of determining what the ``holes,''
            of the space are.
            \begin{example}
                Let $W$ be the union of a triangle $[a,b,c]$
                and a line segment $[c,d]$. Then $C_{3}=0$, and
                $C_{n}=0$ for all $n>3$. The chain is then
                $0\rightarrow{C_{2}}\rightarrow{C_{1}}\rightarrow{C_{0}}$.
            \end{example}
            \begin{theorem}
                If $W$ is contractible and connected,
                then $H_{0}^{\Delta}(W)=\mathbb{Z}$ and, for all
                $n>0$, $H_{n}^{\Delta}(W)=0$.
            \end{theorem}
            
        \subsubsection{Singular Homology}
            Singular homology is formed in the same way that
            simplicial homology is, with the requirement
            that the chains be formed by $\Delta_{n}(X)$ being
            relaxed. Now, only the continuity of the $\sigma$
            are required. Therefore, when $X$ is a space that can
            be ``triangulated,'' we have that the simplicial
            and singular homologies are the same.
            Suppose X and Y are orientable mans f:X-Y continuous. Whats degree?
            Take the nth top homology, $h_n(Z)-h_n(Y,Z)$. This is the map
            $f^*$ which is induced by $f$. Each one of these is a copy of
            $\mathbb{Z}$. So each one of these is a homomorphism. So it has
            to send the number $1\in\mathbb{N}$ somewhere. So the image of
            $1$ is the degree of the map $f$. For example
            $f$ constant, the degree is 0. If $f$ is the identity,
            then the degree is 1. If you take a loop that wraps around
            $S^{1}$ twice then the degree is $2$.
            This looks like the winding number, but is not quite that.
\end{document}