\documentclass[crop=false,class=book,oneside]{standalone}
%----------------------------Preamble-------------------------------%
%---------------------------Packages----------------------------%
\usepackage{geometry}
\geometry{b5paper, margin=1.0in}
\usepackage[T1]{fontenc}
\usepackage{graphicx, float}            % Graphics/Images.
\usepackage{natbib}                     % For bibliographies.
\bibliographystyle{agsm}                % Bibliography style.
\usepackage[french, english]{babel}     % Language typesetting.
\usepackage[dvipsnames]{xcolor}         % Color names.
\usepackage{listings}                   % Verbatim-Like Tools.
\usepackage{mathtools, esint, mathrsfs} % amsmath and integrals.
\usepackage{amsthm, amsfonts, amssymb}  % Fonts and theorems.
\usepackage{tcolorbox}                  % Frames around theorems.
\usepackage{upgreek}                    % Non-Italic Greek.
\usepackage{fmtcount, etoolbox}         % For the \book{} command.
\usepackage[newparttoc]{titlesec}       % Formatting chapter, etc.
\usepackage{titletoc}                   % Allows \book in toc.
\usepackage[nottoc]{tocbibind}          % Bibliography in toc.
\usepackage[titles]{tocloft}            % ToC formatting.
\usepackage{pgfplots, tikz}             % Drawing/graphing tools.
\usepackage{imakeidx}                   % Used for index.
\usetikzlibrary{
    calc,                   % Calculating right angles and more.
    angles,                 % Drawing angles within triangles.
    arrows.meta,            % Latex and Stealth arrows.
    quotes,                 % Adding labels to angles.
    positioning,            % Relative positioning of nodes.
    decorations.markings,   % Adding arrows in the middle of a line.
    patterns,
    arrows
}                                       % Libraries for tikz.
\pgfplotsset{compat=1.9}                % Version of pgfplots.
\usepackage[font=scriptsize,
            labelformat=simple,
            labelsep=colon]{subcaption} % Subfigure captions.
\usepackage[font={scriptsize},
            hypcap=true,
            labelsep=colon]{caption}    % Figure captions.
\usepackage[pdftex,
            pdfauthor={Ryan Maguire},
            pdftitle={Mathematics and Physics},
            pdfsubject={Mathematics, Physics, Science},
            pdfkeywords={Mathematics, Physics, Computer Science, Biology},
            pdfproducer={LaTeX},
            pdfcreator={pdflatex}]{hyperref}
\hypersetup{
    colorlinks=true,
    linkcolor=blue,
    filecolor=magenta,
    urlcolor=Cerulean,
    citecolor=SkyBlue
}                           % Colors for hyperref.
\usepackage[toc,acronym,nogroupskip,nopostdot]{glossaries}
\usepackage{glossary-mcols}
%------------------------Theorem Styles-------------------------%
\theoremstyle{plain}
\newtheorem{theorem}{Theorem}[section]

% Define theorem style for default spacing and normal font.
\newtheoremstyle{normal}
    {\topsep}               % Amount of space above the theorem.
    {\topsep}               % Amount of space below the theorem.
    {}                      % Font used for body of theorem.
    {}                      % Measure of space to indent.
    {\bfseries}             % Font of the header of the theorem.
    {}                      % Punctuation between head and body.
    {.5em}                  % Space after theorem head.
    {}

% Italic header environment.
\newtheoremstyle{thmit}{\topsep}{\topsep}{}{}{\itshape}{}{0.5em}{}

% Define environments with italic headers.
\theoremstyle{thmit}
\newtheorem*{solution}{Solution}

% Define default environments.
\theoremstyle{normal}
\newtheorem{example}{Example}[section]
\newtheorem{definition}{Definition}[section]
\newtheorem{problem}{Problem}[section]

% Define framed environment.
\tcbuselibrary{most}
\newtcbtheorem[use counter*=theorem]{ftheorem}{Theorem}{%
    before=\par\vspace{2ex},
    boxsep=0.5\topsep,
    after=\par\vspace{2ex},
    colback=green!5,
    colframe=green!35!black,
    fonttitle=\bfseries\upshape%
}{thm}

\newtcbtheorem[auto counter, number within=section]{faxiom}{Axiom}{%
    before=\par\vspace{2ex},
    boxsep=0.5\topsep,
    after=\par\vspace{2ex},
    colback=Apricot!5,
    colframe=Apricot!35!black,
    fonttitle=\bfseries\upshape%
}{ax}

\newtcbtheorem[use counter*=definition]{fdefinition}{Definition}{%
    before=\par\vspace{2ex},
    boxsep=0.5\topsep,
    after=\par\vspace{2ex},
    colback=blue!5!white,
    colframe=blue!75!black,
    fonttitle=\bfseries\upshape%
}{def}

\newtcbtheorem[use counter*=example]{fexample}{Example}{%
    before=\par\vspace{2ex},
    boxsep=0.5\topsep,
    after=\par\vspace{2ex},
    colback=red!5!white,
    colframe=red!75!black,
    fonttitle=\bfseries\upshape%
}{ex}

\newtcbtheorem[auto counter, number within=section]{fnotation}{Notation}{%
    before=\par\vspace{2ex},
    boxsep=0.5\topsep,
    after=\par\vspace{2ex},
    colback=SeaGreen!5!white,
    colframe=SeaGreen!75!black,
    fonttitle=\bfseries\upshape%
}{not}

\newtcbtheorem[use counter*=remark]{fremark}{Remark}{%
    fonttitle=\bfseries\upshape,
    colback=Goldenrod!5!white,
    colframe=Goldenrod!75!black}{ex}

\newenvironment{bproof}{\textit{Proof.}}{\hfill$\square$}
\tcolorboxenvironment{bproof}{%
    blanker,
    breakable,
    left=3mm,
    before skip=5pt,
    after skip=10pt,
    borderline west={0.6mm}{0pt}{green!80!black}
}

\AtEndEnvironment{lexample}{$\hfill\textcolor{red}{\blacksquare}$}
\newtcbtheorem[use counter*=example]{lexample}{Example}{%
    empty,
    title={Example~\theexample},
    boxed title style={%
        empty,
        size=minimal,
        toprule=2pt,
        top=0.5\topsep,
    },
    coltitle=red,
    fonttitle=\bfseries,
    parbox=false,
    boxsep=0pt,
    before=\par\vspace{2ex},
    left=0pt,
    right=0pt,
    top=3ex,
    bottom=1ex,
    before=\par\vspace{2ex},
    after=\par\vspace{2ex},
    breakable,
    pad at break*=0mm,
    vfill before first,
    overlay unbroken={%
        \draw[red, line width=2pt]
            ([yshift=-1.2ex]title.south-|frame.west) to
            ([yshift=-1.2ex]title.south-|frame.east);
        },
    overlay first={%
        \draw[red, line width=2pt]
            ([yshift=-1.2ex]title.south-|frame.west) to
            ([yshift=-1.2ex]title.south-|frame.east);
    },
}{ex}

\AtEndEnvironment{ldefinition}{$\hfill\textcolor{Blue}{\blacksquare}$}
\newtcbtheorem[use counter*=definition]{ldefinition}{Definition}{%
    empty,
    title={Definition~\thedefinition:~{#1}},
    boxed title style={%
        empty,
        size=minimal,
        toprule=2pt,
        top=0.5\topsep,
    },
    coltitle=Blue,
    fonttitle=\bfseries,
    parbox=false,
    boxsep=0pt,
    before=\par\vspace{2ex},
    left=0pt,
    right=0pt,
    top=3ex,
    bottom=0pt,
    before=\par\vspace{2ex},
    after=\par\vspace{1ex},
    breakable,
    pad at break*=0mm,
    vfill before first,
    overlay unbroken={%
        \draw[Blue, line width=2pt]
            ([yshift=-1.2ex]title.south-|frame.west) to
            ([yshift=-1.2ex]title.south-|frame.east);
        },
    overlay first={%
        \draw[Blue, line width=2pt]
            ([yshift=-1.2ex]title.south-|frame.west) to
            ([yshift=-1.2ex]title.south-|frame.east);
    },
}{def}

\AtEndEnvironment{ltheorem}{$\hfill\textcolor{Green}{\blacksquare}$}
\newtcbtheorem[use counter*=theorem]{ltheorem}{Theorem}{%
    empty,
    title={Theorem~\thetheorem:~{#1}},
    boxed title style={%
        empty,
        size=minimal,
        toprule=2pt,
        top=0.5\topsep,
    },
    coltitle=Green,
    fonttitle=\bfseries,
    parbox=false,
    boxsep=0pt,
    before=\par\vspace{2ex},
    left=0pt,
    right=0pt,
    top=3ex,
    bottom=-1.5ex,
    breakable,
    pad at break*=0mm,
    vfill before first,
    overlay unbroken={%
        \draw[Green, line width=2pt]
            ([yshift=-1.2ex]title.south-|frame.west) to
            ([yshift=-1.2ex]title.south-|frame.east);},
    overlay first={%
        \draw[Green, line width=2pt]
            ([yshift=-1.2ex]title.south-|frame.west) to
            ([yshift=-1.2ex]title.south-|frame.east);
    }
}{thm}

%--------------------Declared Math Operators--------------------%
\DeclareMathOperator{\adjoint}{adj}         % Adjoint.
\DeclareMathOperator{\Card}{Card}           % Cardinality.
\DeclareMathOperator{\curl}{curl}           % Curl.
\DeclareMathOperator{\diam}{diam}           % Diameter.
\DeclareMathOperator{\dist}{dist}           % Distance.
\DeclareMathOperator{\Div}{div}             % Divergence.
\DeclareMathOperator{\Erf}{Erf}             % Error Function.
\DeclareMathOperator{\Erfc}{Erfc}           % Complementary Error Function.
\DeclareMathOperator{\Ext}{Ext}             % Exterior.
\DeclareMathOperator{\GCD}{GCD}             % Greatest common denominator.
\DeclareMathOperator{\grad}{grad}           % Gradient
\DeclareMathOperator{\Ima}{Im}              % Image.
\DeclareMathOperator{\Int}{Int}             % Interior.
\DeclareMathOperator{\LC}{LC}               % Leading coefficient.
\DeclareMathOperator{\LCM}{LCM}             % Least common multiple.
\DeclareMathOperator{\LM}{LM}               % Leading monomial.
\DeclareMathOperator{\LT}{LT}               % Leading term.
\DeclareMathOperator{\Mod}{mod}             % Modulus.
\DeclareMathOperator{\Mon}{Mon}             % Monomial.
\DeclareMathOperator{\multideg}{mutlideg}   % Multi-Degree (Graphs).
\DeclareMathOperator{\nul}{nul}             % Null space of operator.
\DeclareMathOperator{\Ord}{Ord}             % Ordinal of ordered set.
\DeclareMathOperator{\Prin}{Prin}           % Principal value.
\DeclareMathOperator{\proj}{proj}           % Projection.
\DeclareMathOperator{\Refl}{Refl}           % Reflection operator.
\DeclareMathOperator{\rk}{rk}               % Rank of operator.
\DeclareMathOperator{\sgn}{sgn}             % Sign of a number.
\DeclareMathOperator{\sinc}{sinc}           % Sinc function.
\DeclareMathOperator{\Span}{Span}           % Span of a set.
\DeclareMathOperator{\Spec}{Spec}           % Spectrum.
\DeclareMathOperator{\supp}{supp}           % Support
\DeclareMathOperator{\Tr}{Tr}               % Trace of matrix.
%--------------------Declared Math Symbols--------------------%
\DeclareMathSymbol{\minus}{\mathbin}{AMSa}{"39} % Unary minus sign.
%------------------------New Commands---------------------------%
\DeclarePairedDelimiter\norm{\lVert}{\rVert}
\DeclarePairedDelimiter\ceil{\lceil}{\rceil}
\DeclarePairedDelimiter\floor{\lfloor}{\rfloor}
\newcommand*\diff{\mathop{}\!\mathrm{d}}
\newcommand*\Diff[1]{\mathop{}\!\mathrm{d^#1}}
\renewcommand*{\glstextformat}[1]{\textcolor{RoyalBlue}{#1}}
\renewcommand{\glsnamefont}[1]{\textbf{#1}}
\renewcommand\labelitemii{$\circ$}
\renewcommand\thesubfigure{%
    \arabic{chapter}.\arabic{figure}.\arabic{subfigure}}
\addto\captionsenglish{\renewcommand{\figurename}{Fig.}}
\numberwithin{equation}{section}

\renewcommand{\vector}[1]{\boldsymbol{\mathrm{#1}}}

\newcommand{\uvector}[1]{\boldsymbol{\hat{\mathrm{#1}}}}
\newcommand{\topspace}[2][]{(#2,\tau_{#1})}
\newcommand{\measurespace}[2][]{(#2,\varSigma_{#1},\mu_{#1})}
\newcommand{\measurablespace}[2][]{(#2,\varSigma_{#1})}
\newcommand{\manifold}[2][]{(#2,\tau_{#1},\mathcal{A}_{#1})}
\newcommand{\tanspace}[2]{T_{#1}{#2}}
\newcommand{\cotanspace}[2]{T_{#1}^{*}{#2}}
\newcommand{\Ckspace}[3][\mathbb{R}]{C^{#2}(#3,#1)}
\newcommand{\funcspace}[2][\mathbb{R}]{\mathcal{F}(#2,#1)}
\newcommand{\smoothvecf}[1]{\mathfrak{X}(#1)}
\newcommand{\smoothonef}[1]{\mathfrak{X}^{*}(#1)}
\newcommand{\bracket}[2]{[#1,#2]}

%------------------------Book Command---------------------------%
\makeatletter
\renewcommand\@pnumwidth{1cm}
\newcounter{book}
\renewcommand\thebook{\@Roman\c@book}
\newcommand\book{%
    \if@openright
        \cleardoublepage
    \else
        \clearpage
    \fi
    \thispagestyle{plain}%
    \if@twocolumn
        \onecolumn
        \@tempswatrue
    \else
        \@tempswafalse
    \fi
    \null\vfil
    \secdef\@book\@sbook
}
\def\@book[#1]#2{%
    \refstepcounter{book}
    \addcontentsline{toc}{book}{\bookname\ \thebook:\hspace{1em}#1}
    \markboth{}{}
    {\centering
     \interlinepenalty\@M
     \normalfont
     \huge\bfseries\bookname\nobreakspace\thebook
     \par
     \vskip 20\p@
     \Huge\bfseries#2\par}%
    \@endbook}
\def\@sbook#1{%
    {\centering
     \interlinepenalty \@M
     \normalfont
     \Huge\bfseries#1\par}%
    \@endbook}
\def\@endbook{
    \vfil\newpage
        \if@twoside
            \if@openright
                \null
                \thispagestyle{empty}%
                \newpage
            \fi
        \fi
        \if@tempswa
            \twocolumn
        \fi
}
\newcommand*\l@book[2]{%
    \ifnum\c@tocdepth >-3\relax
        \addpenalty{-\@highpenalty}%
        \addvspace{2.25em\@plus\p@}%
        \setlength\@tempdima{3em}%
        \begingroup
            \parindent\z@\rightskip\@pnumwidth
            \parfillskip -\@pnumwidth
            {
                \leavevmode
                \Large\bfseries#1\hfill\hb@xt@\@pnumwidth{\hss#2}
            }
            \par
            \nobreak
            \global\@nobreaktrue
            \everypar{\global\@nobreakfalse\everypar{}}%
        \endgroup
    \fi}
\newcommand\bookname{Book}
\renewcommand{\thebook}{\texorpdfstring{\Numberstring{book}}{book}}
\providecommand*{\toclevel@book}{-2}
\makeatother
\titleformat{\part}[display]
    {\Large\bfseries}
    {\partname\nobreakspace\thepart}
    {0mm}
    {\Huge\bfseries}
\titlecontents{part}[0pt]
    {\large\bfseries}
    {\partname\ \thecontentslabel: \quad}
    {}
    {\hfill\contentspage}
\titlecontents{chapter}[0pt]
    {\bfseries}
    {\chaptername\ \thecontentslabel:\quad}
    {}
    {\hfill\contentspage}
\newglossarystyle{longpara}{%
    \setglossarystyle{long}%
    \renewenvironment{theglossary}{%
        \begin{longtable}[l]{{p{0.25\hsize}p{0.65\hsize}}}
    }{\end{longtable}}%
    \renewcommand{\glossentry}[2]{%
        \glstarget{##1}{\glossentryname{##1}}%
        &\glossentrydesc{##1}{~##2.}
        \tabularnewline%
        \tabularnewline
    }%
}
\newglossary[not-glg]{notation}{not-gls}{not-glo}{Notation}
\newcommand*{\newnotation}[4][]{%
    \newglossaryentry{#2}{type=notation, name={\textbf{#3}, },
                          text={#4}, description={#4},#1}%
}
%--------------------------LENGTHS------------------------------%
% Spacings for the Table of Contents.
\addtolength{\cftsecnumwidth}{1ex}
\addtolength{\cftsubsecindent}{1ex}
\addtolength{\cftsubsecnumwidth}{1ex}
\addtolength{\cftfignumwidth}{1ex}
\addtolength{\cfttabnumwidth}{1ex}

% Indent and paragraph spacing.
\setlength{\parindent}{0em}
\setlength{\parskip}{0em}
%----------------------------GLOSSARY-------------------------------%
\makeglossaries
\loadglsentries{../../../glossary}
\loadglsentries{../../../acronym}
%--------------------------Main Document----------------------------%
\begin{document}
    \newif\ifphysicscourseselectromagnetismI
    \ifx\ifphysicscourses\undefined
        \title{Electromagnetism I}
        \author{Ryan Maguire}
        \date{\vspace{-5ex}}
        \maketitle
        \tableofcontents
        \chapter*{Electromagnetism I}
        \markboth{}{ELECTROMAGNETISM I}
        \setcounter{chapter}{1}
    \else
        \chapter{Electromagnetism I}
    \fi
    \section{Homework From UML PHYS.3530/PHYS.5530}
        \subimport{./Homework/}{Homework_01}
        \subimport{./Homework/}{Homework_02}
        \subimport{./Homework/}{Homework_03}
        \subimport{./Homework/}{Homework_04}
        \subimport{./Homework/}{Homework_05}
        \subimport{./Homework/}{Homework_06}
        \subimport{./Homework/}{Homework_07}
        \subimport{./Homework/}{Homework_08}
        \subimport{./Homework/}{Homework_09}
        \subimport{./Homework/}{Homework_10}
        \subimport{./Homework/}{Homework_11}
        \subimport{./Homework/}{Homework_12}
        \subimport{./Homework/}{Homework_13}
    \section{Exams}
        \subsection{Exam I}
        \subsubsection{Question I}
        Give the vector field field $\mathbf{A} = c\hat{\boldsymbol{\uptheta}}$, where $c$ is a constant, find $\nabla \times \mathbf{A}$. Is this a conservative vector field? Explain.
        \subsubsection{Question II}
        Verify the Divergence Theorem for $\mathbf{A}$ given in problem 1 in spherical coordinates for a hemisphere of radius $a_0$ resting on the $xy-plane$ with the center of the flat base of the hemisphere at the origin and the symmetry axis of the hemisphere along the positive $z-axis$.
        \subsubsection{Question III}
        A semicircular charged line of radius $a$ carries uniform linear charge density $\lambda$. It has the equation $x^2+y^2 = a^2$, $x\geq 0$, $z=0$. That is, the half circle resting on the $xy-plane$ of radius $a$. Find the electring field at a point $P$ on the $z$ axis a distance $z$ from the origin.
        \subsection{Exam II}
        \subsubsection{Question I}
        A conducting sphere of radius $a$, centered at the origin carries charge $Q_1$. This sphere is surrounded by a hollow concentric conducting spherical shell of inner radius $b$ and outer radius $c$ with $a<b<c$. The outer hollow conducting shell caries a total charge $Q_2$. 
        \begin{enumerate}
            \item What is the electric field everywhere?
            \item What is the potential everywhere, assuming $\underset{r\rightarrow \infty} \lim \phi(r) = 0$?
            \item How much charge is on the inner and outer surfaces of the conducting shell at $r=b$ and $r=c$?
        \end{enumerate}
        \subsubsection{Question II}
        The outer conductor of problem I is now grounded.
        \begin{enumerate}
            \item What is the electric field everywhere?
            \item What is the potential everywhere?
            \item How much charge is on the surfaces at $r=b$ and $r=c$?
            \item What is the capacitance of the system of conductors?
            \item Calculate the electrostatic potential energy of the configuration assuming the energy resides in the charges.
            \item Calculate the electrostatic potential energy of the configuration assuming the energy resides in the electric field.
        \end{enumerate}
        \subsubsection{Question III}
        A semicircular arc of radius $a$ in the $y-z$ plane with center on the $y-axis$ at the origin and the top of the arc on the positive $y-axis$ carries linear charge density $\lambda = \lambda_0 \cos(\theta')$, where $\lambda_0$ is constant and $\theta'$ is measured with respected to the positive $z-axis$.
        \begin{enumerate}
            \item What is the electric monopole moment of this distribution?
            \item What is the electric dipole moment of this distribution?
            \item What is the electric potential at a distance $r$ from the origin for this distribution where $r>a$, accurate to order $\frac{1}{r^2}$?
        \end{enumerate}
        \subsection{Exam III}
        \subsubsection{Question I}
        The electric field in a spherical region of space of radius $a$ is given by $\mathbf{E} = E_0 \frac{r^2}{a^2}\hat{\mathbf{r}}$ for $r< a$, where $E_0$ is a constant. This region is surrounded concentrically by a grounded conducting spherical shell of inner radius $b$ and outer radius $c$ with $a<b<c$. There is no charge in the region $a<r<b$. 
        \begin{enumerate}
            \item What is the electric charge density in the region $r<a$?
            \item Wher is the electric field in the region $a<r<b$?
            \item How much charge is on the surfaces at $r=b$ and $r=c$?
            \item What is the electric field for $r>c$?
            \item What is the electric potential $\phi$ at $r=0$ assuming that ground potential is $\phi = 0$.
        \end{enumerate}
        \subsubsection{Question II}
        A capacitor $C_{1}$ is charged to a potential difference $\Delta \phi$ between its plates. A second capacitor $C_{2}$ is uncharged. One plate of $C_2$ is now connected to a plate of $C_1$ by a conductor of negligible capacitance, the remaining plates are similarly connected. 
        \begin{enumerate}
            \item For the resultant equilibrium state, find the charge on each capacitor and the potential difference $\Delta \phi$ between their respective plates.
            \item Compare the energy stored in capacitor $C_1$ before connecting it to $C_2$, to the energy of the combination after connected them. Are these energies the same? If not, which is larger and where did any additional energy come from, or where did any lost energy go?
        \end{enumerate}
        \subsubsection{Question III}
        Find the electric dipole moment of an hourglass configuration of charge consisting of two identical right circular cones of radius $a$ and height $a$ with symmetry axes aligned apex to apex along the $z-$axis with the apexes touching at the origin. The top cone has charge density $\sigma_0$ on its surface, and the bottom cone has charge density $-\sigma_0$ on its surface.
        \subsection{Practice Final Exam}
        \subsubsection{Problem I}
        The electric field in a region of space is given in spherical coordinates as $\mathbf{E} = cr\hat{\mathbf{r}}$, where $c$ is constant. 
        \begin{enumerate}
            \item Find the charge density at a point $(r,\theta,\phi)$
            \item Find the total charge inside a sphere of radius $a$ centered at the origin.
        \end{enumerate}
        \subsubsection{Problem II}
        A battery is used to charge an ideal parallel plate capacitor to a potential difference $\Delta \phi = V_0$. The battery is then disconnected. The separation between the plates is now increasing from $d$ to $\alpha d$, where $\alpha >1$. The area of the plates is $A$.
        \begin{enumerate}
            \item What is the ratio of the new energy to the original energy>
            \item Is the energy increases or decreased?
            \item Where does this energy come from or go to?
            \item Compute the change in energy $\Delta U_e$ expressing your answer in terms of the given quantities $V_0,d,A,\alpha$ and fundamental constants.
        \end{enumerate}
        \subsubsection{Problem III}
        A dielectric sphere of radius $a$ and permittivity $\varepsilon$ contains a free charge density. $\rho_f = cr$, where $c$ is a constant. The sphere is centered at the origin. Find the electric potential at the center of the sphere assuming that the potential is zero at an infinite distance from the center.
        \subsubsection{Problem IV}
        A thick slab extending from $z=-a$ to $z=a$ carries a uniform vlume current density $\mathbf{J} = J_0 \hat{\mathbf{x}}$. The slab is infinite in the $xy-$plane. Find the magnetic field $B$ as a function of $z$ inside and outside the slab. Plot $B$ as a function of $z$ for $-b<z<b$ where $b>a$.
        \subsubsection{Problem V}
        An ideal long solenoid of radius $a$, carrying $n$ turns per unit length, is looped by a wire with resistance $R$. 
        \begin{enumerate}
            \item If the current in the solenoid is increasing at a constant rate $\frac{dI}{dt} = k$, what current flows in the lopp, and which way (Left or right) does it pass through the resistor?
            \item If the current $I$ in the solenoid is constant but the solenoid is pulled out of the loop and reinserted in the opposite direction, what total charge passes through the resistor?
        \end{enumerate}
        \subsection{Final Exam}
        \subsubsection{Problem I}
        \begin{enumerate}
            \item Write down Maxwell's Equations in differential form.
            \item Convert them to integral form and show derivations.
            \item Name each equation.
        \end{enumerate}
        \begin{proof}[Solution]
        \
        \begin{enumerate}
        \item Gauss' Law: $\nabla \cdot \mathbf{E} = \frac{\rho}{\epsilon_0}\Rightarrow\frac{Q_{encl}}{\epsilon_0}=\iiint_{V} \frac{\rho}{\epsilon_0}d\tau=\iiint_{V} \big(\nabla \cdot \mathbf{E}\big) d\tau = \oiint_{\partial V} \mathbf{E}\cdot \mathbf{da}$
        \item Faraday's Law: $\nabla \times \mathbf{E} = -\frac{\partial \mathbf{B}}{\partial t}\Rightarrow-\frac{d \Phi_{B}}{dt} = \iint_{S} -\frac{\partial \mathbf{B}}{\partial t}da = \iint_{S} \big(\nabla \times \mathbf{E}\big)da = \oint_{\partial S}\mathbf{E}\cdot \mathbf{d\ell}$
        \item Gauss' Law of Magnetism: $\nabla \cdot \mathbf{B} = 0\Rightarrow 0 = \iiint_{V} \big(\nabla \cdot\mathbf{B}\big)d\tau = \oiint_{\partial V} \mathbf{B}\cdot \mathbf{da}$
        \item Ampere's Law: $\nabla \times \mathbf{B} = \mu_0 \mathbf{J} + \mu_0 \epsilon_0 \frac{\partial \mathbf{E}}{\partial t}\Rightarrow\mu_0 I_{encl}+ \mu_0 \epsilon_0 \frac{d\Phi_{E}}{dt} = \iint_{S}\big(\mu_0 \mathbf{J} + \mu_0 \epsilon_0 \frac{\partial \mathbf{E}}{\partial t}\big)da = \iint_{S}\big(\nabla \times \mathbf{B}\big)da = \oint_{\partial S}\mathbf{B}\cdot \mathbf{d\ell}$
        \end{enumerate}
        \end{proof}
        \subsubsection{Problem II}
        A conduction sphere of radius $a$ carries a charge $Q_{1}$. It is surrounded by a conducting spherical shell of inner radius $b$ and outer radius $c$ with $a<b<c$. The charge on the conducting shell is $Q_{2}$. The region between the conductors $a<r<b$ is filled with linear isotropic dielectric of permittivity $\varepsilon$. Find the following in the regions $r<a.a<r<b.b<r.b<r<c.c<r$:
        \begin{enumerate}
            \item The electric displacement $\mathbf{D}$
            \item The electric field $\mathbf{E}$
            \item The polarization vector $\mathbf{P}$
            \item Find the free charge on the conductors at $r=a,b,c$.
            \item The bound volume charge in the dielectric.
            \item The bound surface charge density at the inner and outer surfaces of the dielectric.
            \item The electric potential at the origin assuming the potential is zero as $r$ goes to infinity.
        \end{enumerate}
        \begin{proof}[Solution]
        \
        \begin{enumerate}
        \begin{multicols}{2}
            \item $\mathbf{D}=\begin{cases}\mathbf{0}&r<a\\ \frac{Q_{1}}{4\pi r^{2}} &a<r<b\\\mathbf{0} & b<r<c\\ \frac{Q_{1}+Q_{2}}{4\pi r^{2}} & c<r \end{cases}$
            \item $\mathbf{E}=\begin{cases}\mathbf{0}&r<a\\ \frac{Q_{1}}{4\pi\epsilon_{0} r^{2}} & a<r<b\\ \mathbf{0} & b<r<c\\ \frac{Q_{1}+Q_{2}}{4\pi\epsilon_{0}r^{2}} & c<r\end{cases}$
            \item $\mathbf{P}=\begin{cases}\mathbf{0}&r<a\\ \frac{Q_{1}}{4\pi r^{2}}(1-\frac{\epsilon_{0}}{\varepsilon}) & a<r<b\\ \mathbf{0} & b<r<c\\ \mathbf{0} & c<r \end{cases}$
            \item $Q = \begin{cases} Q_1 & r=a\\ -Q_1 & r=b\\ Q_1+Q_2 & r=c\end{cases}$
        \end{multicols}
        \begin{multicols}{2}
            \item $\rho_b = \nabla \cdot \mathbf{P}$, $\rho_b = 0$.
            \item $\sigma_{b,a}=-\frac{Q_1}{4\pi a^2}\big(1-\frac{\epsilon_0}{\varepsilon}\big)$, $\sigma_{b,b}=\frac{Q_1}{4\pi b^2}\big(1-\frac{\epsilon_0}{\varepsilon}\big)$.
            \end{multicols}
            \item $\phi = -\int_{0}^{\infty}\mathbf{E}\cdot \mathbf{d\ell} = \int_{c}^{\infty} \mathbf{E}\cdot \mathbf{d\ell}+\int_{b}^{c}\mathbf{E}\cdot \mathbf{d\ell}+\int_{a}^{b}\mathbf{E}\cdot \mathbf{d\ell} + \int_{0}^{a} \mathbf{E}\cdot \mathbf{d\ell} = \frac{Q_1+Q_2}{4\pi \epsilon_0 c}+\frac{Q_1}{4\pi \epsilon_0 a}-\frac{Q_1}{4\pi \epsilon_0 b}$.
        \end{enumerate}
        \end{proof}
        \subsubsection{Problem III}
        A sphere of radius $a$ carries charge density $\rho = \rho_0(r/a)$, where $\rho_0$ is a constant. Find the work done to assemble the charge distribution.
        \begin{proof}[Solution]
        We find $\mathbf{E}$ inside and outside using Gauss' law.
        \begin{equation*}
        \oiint_{S} \mathbf{E}\cdot \mathbf{da} = \frac{Q_{encl}}{\epsilon_0} = \int_{0}^{r}\int_{0}^{\pi} \int_{0}^{2\pi} \rho_0 \frac{r}{a}r^2\sin(\theta)d\varphi d\theta dr = \frac{4\pi \rho_0}{a \epsilon_0}\frac{r^4}{4} = E(4\pi r^2).
        \end{equation*}
        So $\mathbf{E} = \frac{\rho_0 r^2}{4a\epsilon_0}\hat{\mathbf{r}}$. Outside we have $\oiint_{S} \mathbf{E}\cdot \mathbf{da} = \int_{0}^{2\pi}\int_{0}^{\pi} \int_{0}^{a} \rho \frac{r}{a}r^2 \sin(\theta) dr d\theta d\varphi$, so $\mathbf{E} = \frac{\rho_0 a^3}{4r^2 \epsilon_0}\hat{\mathbf{r}}$. The work is
        \begin{align*}
        \frac{\epsilon_0}{2}\int_{All\ Space}E^2 d\tau &= \frac{\epsilon_0}{2}\int_{0}^{2\pi}\int_{0}^{\pi}\int_{0}^{\infty} E^2 r^2\sin(\theta) drd\theta d\varphi\\
        &= \int_{0}^{2\pi}\int_{0}^{\pi}\int_{0}^{a} E^2r^2\sin(\theta)dr d\theta d\varphi + \int_{0}^{2\pi}\int_{0}^{\pi}\int_{a}^{\infty} E^2r^2\sin(\theta)drd\theta d\varphi\\
        &= \frac{\pi \rho_0^2 a^5}{7\epsilon_0}
        \end{align*}
        So $\mathbf{E} = \frac{\rho_0 r^2}{4a\epsilon_0}\hat{\mathbf{r}}$
        \end{proof}
        \subsubsection{Problem IV}
        \begin{enumerate}
            \item Could the vector field $\mathbf{F} = ax\hat{\mathbf{x}}+by\hat{\mathbf{y}}+cz\hat{\mathbf{z}}$ be a possible magnetic field, where $a+b+c\ne 0$? Explain why or why not.
            \item An electric field is given by $\mathbf{E} = ax\hat{\mathbf{y}}$, where $a$ is a constant. Is this a conservative field? Explain why or why not.
            \item Find the possible magnetic field $\mathbf{B}$ associated to $\mathbf{E}$. 
        \end{enumerate}
        \begin{proof}[Solution]
        \
        \begin{enumerate}
            \item No, for $\nabla \cdot \mathbf{F} = a+b+c \ne 0$, and therefore $\mathbf{F}$ cannot be a magnetic field.
            \item No, for $\nabla \times \mathbf{E} = a\hat{\mathbf{z}} \ne 0$, and thus $\mathbf{E}$ is not a conservative field.
            \item $\nabla \times \mathbf{E} = -\frac{\partial \mathbf{B}}{\partial t} = a\hat{\mathbf{z}}$, so $\mathbf{B} = -at\hat{\mathbf{z}}+\mathbf{B}_0$, where $\mathbf{B}_0$ is some constant vector. Here $\mathbf{B}$ is increasing with time in the $-z$ direction.
        \end{enumerate}
        \end{proof}
        \subsubsection{Problem V}
        Two infinitely long coaxial cylindrical infinitesimally
        thin conducting shells concentric with the $z-$axis carry
        oppositely directed currents of equal magnitude in the $+$
        and $-$ $z-$directions. The radius of the inner shell is $a$
        and that of the outer shell is $b$. What is the self-inductance
        of a length $\ell$ of this system?
        \begin{proof}[Solution]
        The flux carried by the inner shell cuts through the area of a rectangle of lenght $\ell$ and width $b-a$. So $\Phi = \int \mathbf{B}\cdot \mathbf{da} = \int_{a}^{b} \frac{\mu_0 I}{2\pi \rho}\ell d\rho = \frac{\mu_0 I\ell}{2\pi}\ln\big(\frac{b}{a}\big)$. So, $L = \frac{\Phi}{I} = \frac{\mu_0 \ell}{2\pi}\ln\big(\frac{b}{a}\big)$.
        \end{proof}
\end{document}