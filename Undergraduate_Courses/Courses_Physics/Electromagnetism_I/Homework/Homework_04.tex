\documentclass[crop=false,class=article,oneside]{standalone}
%----------------------------Preamble-------------------------------%
%---------------------------Packages----------------------------%
\usepackage{geometry}
\geometry{b5paper, margin=1.0in}
\usepackage[T1]{fontenc}
\usepackage{graphicx, float}            % Graphics/Images.
\usepackage{natbib}                     % For bibliographies.
\bibliographystyle{agsm}                % Bibliography style.
\usepackage[french, english]{babel}     % Language typesetting.
\usepackage[dvipsnames]{xcolor}         % Color names.
\usepackage{listings}                   % Verbatim-Like Tools.
\usepackage{mathtools, esint, mathrsfs} % amsmath and integrals.
\usepackage{amsthm, amsfonts, amssymb}  % Fonts and theorems.
\usepackage{tcolorbox}                  % Frames around theorems.
\usepackage{upgreek}                    % Non-Italic Greek.
\usepackage{fmtcount, etoolbox}         % For the \book{} command.
\usepackage[newparttoc]{titlesec}       % Formatting chapter, etc.
\usepackage{titletoc}                   % Allows \book in toc.
\usepackage[nottoc]{tocbibind}          % Bibliography in toc.
\usepackage[titles]{tocloft}            % ToC formatting.
\usepackage{pgfplots, tikz}             % Drawing/graphing tools.
\usepackage{imakeidx}                   % Used for index.
\usetikzlibrary{
    calc,                   % Calculating right angles and more.
    angles,                 % Drawing angles within triangles.
    arrows.meta,            % Latex and Stealth arrows.
    quotes,                 % Adding labels to angles.
    positioning,            % Relative positioning of nodes.
    decorations.markings,   % Adding arrows in the middle of a line.
    patterns,
    arrows
}                                       % Libraries for tikz.
\pgfplotsset{compat=1.9}                % Version of pgfplots.
\usepackage[font=scriptsize,
            labelformat=simple,
            labelsep=colon]{subcaption} % Subfigure captions.
\usepackage[font={scriptsize},
            hypcap=true,
            labelsep=colon]{caption}    % Figure captions.
\usepackage[pdftex,
            pdfauthor={Ryan Maguire},
            pdftitle={Mathematics and Physics},
            pdfsubject={Mathematics, Physics, Science},
            pdfkeywords={Mathematics, Physics, Computer Science, Biology},
            pdfproducer={LaTeX},
            pdfcreator={pdflatex}]{hyperref}
\hypersetup{
    colorlinks=true,
    linkcolor=blue,
    filecolor=magenta,
    urlcolor=Cerulean,
    citecolor=SkyBlue
}                           % Colors for hyperref.
\usepackage[toc,acronym,nogroupskip,nopostdot]{glossaries}
\usepackage{glossary-mcols}
%------------------------Theorem Styles-------------------------%
\theoremstyle{plain}
\newtheorem{theorem}{Theorem}[section]

% Define theorem style for default spacing and normal font.
\newtheoremstyle{normal}
    {\topsep}               % Amount of space above the theorem.
    {\topsep}               % Amount of space below the theorem.
    {}                      % Font used for body of theorem.
    {}                      % Measure of space to indent.
    {\bfseries}             % Font of the header of the theorem.
    {}                      % Punctuation between head and body.
    {.5em}                  % Space after theorem head.
    {}

% Italic header environment.
\newtheoremstyle{thmit}{\topsep}{\topsep}{}{}{\itshape}{}{0.5em}{}

% Define environments with italic headers.
\theoremstyle{thmit}
\newtheorem*{solution}{Solution}

% Define default environments.
\theoremstyle{normal}
\newtheorem{example}{Example}[section]
\newtheorem{definition}{Definition}[section]
\newtheorem{problem}{Problem}[section]

% Define framed environment.
\tcbuselibrary{most}
\newtcbtheorem[use counter*=theorem]{ftheorem}{Theorem}{%
    before=\par\vspace{2ex},
    boxsep=0.5\topsep,
    after=\par\vspace{2ex},
    colback=green!5,
    colframe=green!35!black,
    fonttitle=\bfseries\upshape%
}{thm}

\newtcbtheorem[auto counter, number within=section]{faxiom}{Axiom}{%
    before=\par\vspace{2ex},
    boxsep=0.5\topsep,
    after=\par\vspace{2ex},
    colback=Apricot!5,
    colframe=Apricot!35!black,
    fonttitle=\bfseries\upshape%
}{ax}

\newtcbtheorem[use counter*=definition]{fdefinition}{Definition}{%
    before=\par\vspace{2ex},
    boxsep=0.5\topsep,
    after=\par\vspace{2ex},
    colback=blue!5!white,
    colframe=blue!75!black,
    fonttitle=\bfseries\upshape%
}{def}

\newtcbtheorem[use counter*=example]{fexample}{Example}{%
    before=\par\vspace{2ex},
    boxsep=0.5\topsep,
    after=\par\vspace{2ex},
    colback=red!5!white,
    colframe=red!75!black,
    fonttitle=\bfseries\upshape%
}{ex}

\newtcbtheorem[auto counter, number within=section]{fnotation}{Notation}{%
    before=\par\vspace{2ex},
    boxsep=0.5\topsep,
    after=\par\vspace{2ex},
    colback=SeaGreen!5!white,
    colframe=SeaGreen!75!black,
    fonttitle=\bfseries\upshape%
}{not}

\newtcbtheorem[use counter*=remark]{fremark}{Remark}{%
    fonttitle=\bfseries\upshape,
    colback=Goldenrod!5!white,
    colframe=Goldenrod!75!black}{ex}

\newenvironment{bproof}{\textit{Proof.}}{\hfill$\square$}
\tcolorboxenvironment{bproof}{%
    blanker,
    breakable,
    left=3mm,
    before skip=5pt,
    after skip=10pt,
    borderline west={0.6mm}{0pt}{green!80!black}
}

\AtEndEnvironment{lexample}{$\hfill\textcolor{red}{\blacksquare}$}
\newtcbtheorem[use counter*=example]{lexample}{Example}{%
    empty,
    title={Example~\theexample},
    boxed title style={%
        empty,
        size=minimal,
        toprule=2pt,
        top=0.5\topsep,
    },
    coltitle=red,
    fonttitle=\bfseries,
    parbox=false,
    boxsep=0pt,
    before=\par\vspace{2ex},
    left=0pt,
    right=0pt,
    top=3ex,
    bottom=1ex,
    before=\par\vspace{2ex},
    after=\par\vspace{2ex},
    breakable,
    pad at break*=0mm,
    vfill before first,
    overlay unbroken={%
        \draw[red, line width=2pt]
            ([yshift=-1.2ex]title.south-|frame.west) to
            ([yshift=-1.2ex]title.south-|frame.east);
        },
    overlay first={%
        \draw[red, line width=2pt]
            ([yshift=-1.2ex]title.south-|frame.west) to
            ([yshift=-1.2ex]title.south-|frame.east);
    },
}{ex}

\AtEndEnvironment{ldefinition}{$\hfill\textcolor{Blue}{\blacksquare}$}
\newtcbtheorem[use counter*=definition]{ldefinition}{Definition}{%
    empty,
    title={Definition~\thedefinition:~{#1}},
    boxed title style={%
        empty,
        size=minimal,
        toprule=2pt,
        top=0.5\topsep,
    },
    coltitle=Blue,
    fonttitle=\bfseries,
    parbox=false,
    boxsep=0pt,
    before=\par\vspace{2ex},
    left=0pt,
    right=0pt,
    top=3ex,
    bottom=0pt,
    before=\par\vspace{2ex},
    after=\par\vspace{1ex},
    breakable,
    pad at break*=0mm,
    vfill before first,
    overlay unbroken={%
        \draw[Blue, line width=2pt]
            ([yshift=-1.2ex]title.south-|frame.west) to
            ([yshift=-1.2ex]title.south-|frame.east);
        },
    overlay first={%
        \draw[Blue, line width=2pt]
            ([yshift=-1.2ex]title.south-|frame.west) to
            ([yshift=-1.2ex]title.south-|frame.east);
    },
}{def}

\AtEndEnvironment{ltheorem}{$\hfill\textcolor{Green}{\blacksquare}$}
\newtcbtheorem[use counter*=theorem]{ltheorem}{Theorem}{%
    empty,
    title={Theorem~\thetheorem:~{#1}},
    boxed title style={%
        empty,
        size=minimal,
        toprule=2pt,
        top=0.5\topsep,
    },
    coltitle=Green,
    fonttitle=\bfseries,
    parbox=false,
    boxsep=0pt,
    before=\par\vspace{2ex},
    left=0pt,
    right=0pt,
    top=3ex,
    bottom=-1.5ex,
    breakable,
    pad at break*=0mm,
    vfill before first,
    overlay unbroken={%
        \draw[Green, line width=2pt]
            ([yshift=-1.2ex]title.south-|frame.west) to
            ([yshift=-1.2ex]title.south-|frame.east);},
    overlay first={%
        \draw[Green, line width=2pt]
            ([yshift=-1.2ex]title.south-|frame.west) to
            ([yshift=-1.2ex]title.south-|frame.east);
    }
}{thm}

%--------------------Declared Math Operators--------------------%
\DeclareMathOperator{\adjoint}{adj}         % Adjoint.
\DeclareMathOperator{\Card}{Card}           % Cardinality.
\DeclareMathOperator{\curl}{curl}           % Curl.
\DeclareMathOperator{\diam}{diam}           % Diameter.
\DeclareMathOperator{\dist}{dist}           % Distance.
\DeclareMathOperator{\Div}{div}             % Divergence.
\DeclareMathOperator{\Erf}{Erf}             % Error Function.
\DeclareMathOperator{\Erfc}{Erfc}           % Complementary Error Function.
\DeclareMathOperator{\Ext}{Ext}             % Exterior.
\DeclareMathOperator{\GCD}{GCD}             % Greatest common denominator.
\DeclareMathOperator{\grad}{grad}           % Gradient
\DeclareMathOperator{\Ima}{Im}              % Image.
\DeclareMathOperator{\Int}{Int}             % Interior.
\DeclareMathOperator{\LC}{LC}               % Leading coefficient.
\DeclareMathOperator{\LCM}{LCM}             % Least common multiple.
\DeclareMathOperator{\LM}{LM}               % Leading monomial.
\DeclareMathOperator{\LT}{LT}               % Leading term.
\DeclareMathOperator{\Mod}{mod}             % Modulus.
\DeclareMathOperator{\Mon}{Mon}             % Monomial.
\DeclareMathOperator{\multideg}{mutlideg}   % Multi-Degree (Graphs).
\DeclareMathOperator{\nul}{nul}             % Null space of operator.
\DeclareMathOperator{\Ord}{Ord}             % Ordinal of ordered set.
\DeclareMathOperator{\Prin}{Prin}           % Principal value.
\DeclareMathOperator{\proj}{proj}           % Projection.
\DeclareMathOperator{\Refl}{Refl}           % Reflection operator.
\DeclareMathOperator{\rk}{rk}               % Rank of operator.
\DeclareMathOperator{\sgn}{sgn}             % Sign of a number.
\DeclareMathOperator{\sinc}{sinc}           % Sinc function.
\DeclareMathOperator{\Span}{Span}           % Span of a set.
\DeclareMathOperator{\Spec}{Spec}           % Spectrum.
\DeclareMathOperator{\supp}{supp}           % Support
\DeclareMathOperator{\Tr}{Tr}               % Trace of matrix.
%--------------------Declared Math Symbols--------------------%
\DeclareMathSymbol{\minus}{\mathbin}{AMSa}{"39} % Unary minus sign.
%------------------------New Commands---------------------------%
\DeclarePairedDelimiter\norm{\lVert}{\rVert}
\DeclarePairedDelimiter\ceil{\lceil}{\rceil}
\DeclarePairedDelimiter\floor{\lfloor}{\rfloor}
\newcommand*\diff{\mathop{}\!\mathrm{d}}
\newcommand*\Diff[1]{\mathop{}\!\mathrm{d^#1}}
\renewcommand*{\glstextformat}[1]{\textcolor{RoyalBlue}{#1}}
\renewcommand{\glsnamefont}[1]{\textbf{#1}}
\renewcommand\labelitemii{$\circ$}
\renewcommand\thesubfigure{%
    \arabic{chapter}.\arabic{figure}.\arabic{subfigure}}
\addto\captionsenglish{\renewcommand{\figurename}{Fig.}}
\numberwithin{equation}{section}

\renewcommand{\vector}[1]{\boldsymbol{\mathrm{#1}}}

\newcommand{\uvector}[1]{\boldsymbol{\hat{\mathrm{#1}}}}
\newcommand{\topspace}[2][]{(#2,\tau_{#1})}
\newcommand{\measurespace}[2][]{(#2,\varSigma_{#1},\mu_{#1})}
\newcommand{\measurablespace}[2][]{(#2,\varSigma_{#1})}
\newcommand{\manifold}[2][]{(#2,\tau_{#1},\mathcal{A}_{#1})}
\newcommand{\tanspace}[2]{T_{#1}{#2}}
\newcommand{\cotanspace}[2]{T_{#1}^{*}{#2}}
\newcommand{\Ckspace}[3][\mathbb{R}]{C^{#2}(#3,#1)}
\newcommand{\funcspace}[2][\mathbb{R}]{\mathcal{F}(#2,#1)}
\newcommand{\smoothvecf}[1]{\mathfrak{X}(#1)}
\newcommand{\smoothonef}[1]{\mathfrak{X}^{*}(#1)}
\newcommand{\bracket}[2]{[#1,#2]}

%------------------------Book Command---------------------------%
\makeatletter
\renewcommand\@pnumwidth{1cm}
\newcounter{book}
\renewcommand\thebook{\@Roman\c@book}
\newcommand\book{%
    \if@openright
        \cleardoublepage
    \else
        \clearpage
    \fi
    \thispagestyle{plain}%
    \if@twocolumn
        \onecolumn
        \@tempswatrue
    \else
        \@tempswafalse
    \fi
    \null\vfil
    \secdef\@book\@sbook
}
\def\@book[#1]#2{%
    \refstepcounter{book}
    \addcontentsline{toc}{book}{\bookname\ \thebook:\hspace{1em}#1}
    \markboth{}{}
    {\centering
     \interlinepenalty\@M
     \normalfont
     \huge\bfseries\bookname\nobreakspace\thebook
     \par
     \vskip 20\p@
     \Huge\bfseries#2\par}%
    \@endbook}
\def\@sbook#1{%
    {\centering
     \interlinepenalty \@M
     \normalfont
     \Huge\bfseries#1\par}%
    \@endbook}
\def\@endbook{
    \vfil\newpage
        \if@twoside
            \if@openright
                \null
                \thispagestyle{empty}%
                \newpage
            \fi
        \fi
        \if@tempswa
            \twocolumn
        \fi
}
\newcommand*\l@book[2]{%
    \ifnum\c@tocdepth >-3\relax
        \addpenalty{-\@highpenalty}%
        \addvspace{2.25em\@plus\p@}%
        \setlength\@tempdima{3em}%
        \begingroup
            \parindent\z@\rightskip\@pnumwidth
            \parfillskip -\@pnumwidth
            {
                \leavevmode
                \Large\bfseries#1\hfill\hb@xt@\@pnumwidth{\hss#2}
            }
            \par
            \nobreak
            \global\@nobreaktrue
            \everypar{\global\@nobreakfalse\everypar{}}%
        \endgroup
    \fi}
\newcommand\bookname{Book}
\renewcommand{\thebook}{\texorpdfstring{\Numberstring{book}}{book}}
\providecommand*{\toclevel@book}{-2}
\makeatother
\titleformat{\part}[display]
    {\Large\bfseries}
    {\partname\nobreakspace\thepart}
    {0mm}
    {\Huge\bfseries}
\titlecontents{part}[0pt]
    {\large\bfseries}
    {\partname\ \thecontentslabel: \quad}
    {}
    {\hfill\contentspage}
\titlecontents{chapter}[0pt]
    {\bfseries}
    {\chaptername\ \thecontentslabel:\quad}
    {}
    {\hfill\contentspage}
\newglossarystyle{longpara}{%
    \setglossarystyle{long}%
    \renewenvironment{theglossary}{%
        \begin{longtable}[l]{{p{0.25\hsize}p{0.65\hsize}}}
    }{\end{longtable}}%
    \renewcommand{\glossentry}[2]{%
        \glstarget{##1}{\glossentryname{##1}}%
        &\glossentrydesc{##1}{~##2.}
        \tabularnewline%
        \tabularnewline
    }%
}
\newglossary[not-glg]{notation}{not-gls}{not-glo}{Notation}
\newcommand*{\newnotation}[4][]{%
    \newglossaryentry{#2}{type=notation, name={\textbf{#3}, },
                          text={#4}, description={#4},#1}%
}
%--------------------------LENGTHS------------------------------%
% Spacings for the Table of Contents.
\addtolength{\cftsecnumwidth}{1ex}
\addtolength{\cftsubsecindent}{1ex}
\addtolength{\cftsubsecnumwidth}{1ex}
\addtolength{\cftfignumwidth}{1ex}
\addtolength{\cfttabnumwidth}{1ex}

% Indent and paragraph spacing.
\setlength{\parindent}{0em}
\setlength{\parskip}{0em}
%----------------------------GLOSSARY-------------------------------%
\makeglossaries
\loadglsentries{../../../glossary}
\loadglsentries{../../../acronym}
%--------------------------Main Document----------------------------%
\begin{document}
    \ifx\ifsub\undefined
        \section*{Electromagnetism I}
        \setcounter{section}{4}
        \renewcommand\thesubfigure{%
            \arabic{section}.\arabic{figure}.\arabic{subfigure}%
        }
    \fi    
    \subsection{Homework IV}
    Wangsness Chapter 2: Problems 3, 7, 8\\
    Wangsness Chapter 3: Problems 9, 10
    \begin{problem}[Wangsness 2-3]
        \label{problem:EMAG_wangsness_2_3}
        Consider $8$ equal point charges $q$ located on the corners of a cube of length
        $a$, as in Fig.~\subref{fig:EMAG_1_Wangsness_2_3}. Find the total force on the
        charge at the origin.
    \end{problem}
    \begin{proof}[Solution]
        \begin{equation*}
            \mathbf{F}_q=\sum_{i=1}^{N}\frac{qq'_{i}}{4\pi \epsilon_0}
            \frac{\hat{\mathbf{r}}_{i}}{R_{i}^{2}}=-\frac{q^{2}}{4\pi \epsilon_0 a^{2}}
            (1+\frac{2}{2^{3/2}}+\frac{1}{3^{3/2}})(\hat{\mathbf{x}}+\hat{\mathbf{y}}
            +\hat{\mathbf{z}})\approx-1.9\frac{q^{2}}{4\pi\epsilon_{0}a^{2}}
            (\hat{\mathbf{x}}+\hat{\mathbf{y}}+\hat{\mathbf{z}})
        \end{equation*}
    \end{proof}
    \begin{figure}[H]
      \begin{subfigure}[b]{0.49\textwidth}
         \centering
        \begin{tikzpicture}[line width=0.4pt,line cap = round,>={triangle 45}]
            \draw[->] (0.0,0.0,0.0) -- (3.5,0.0,0.0) node[right] {$y$};
            \draw[->] (0.0,0.0,0.0) -- (0.0,3.5,0.0) node[above] {$z$};
            \draw[->] (0.0,0.0,0.0) -- (0.0,0.0,3.5) node[below left] {$x$};
            \draw[-]  (0.0,2.5,0.0) -- (2.7,2.5,0.0) {};
            \draw[-]  (0.0,2.5,0.0) -- (0.0,2.5,2.0) {};
            \draw[-]  (2.7,0.0,0.0) -- (2.7,2.5,0.0) {};
            \draw[-]  (2.7,0.0,0.0) -- (2.7,0.0,2.0) {};
            \draw[-]  (0.0,0.0,2.0) -- (0.0,2.5,2.0) {};
            \draw[-]  (0.0,0.0,2.0) -- (2.7,0.0,2.0) {};
            \draw[-]  (2.7,2.5,0.0) -- (2.7,2.5,2.0) {};
            \draw[-]  (2.7,0.0,2.0) -- (2.7,2.5,2.0) {};
            \draw[-]  (0.0,2.5,2.0) -- (2.7,2.5,2.0) {};
            \filldraw[black] (0.0,0.0,0.0) circle (0.4mm);
            \filldraw[black] (0.0,2.5,0.0) circle (0.4mm);
            \filldraw[black] (0.0,0.0,2.0) circle (0.4mm);
            \filldraw[black] (2.7,0.0,0.0) circle (0.4mm);
            \filldraw[black] (2.7,2.5,2.0) circle (0.4mm);
            \filldraw[black] (0.0,2.5,2.0) circle (0.4mm);
            \filldraw[black] (2.7,0.0,2.0) circle (0.4mm);
            \filldraw[black] (2.7,2.5,0.0) circle (0.4mm);
            \node at (0.0,2.5,0.0) [above right] {$a$};
            \node at (0.0,0.0,2.0) [left] {$a$};
            \node at (2.7,0.0,0.0) [above right] {$a$};
            \node at (0.0,0.0,0.0) [below right] {$q$};
        \end{tikzpicture}
        \caption{Drawing for Wangsness 2-3}
        \label{fig:EMAG_1_Wangsness_2_3}
      \end{subfigure}
      \begin{subfigure}[b]{0.49\textwidth}
        \centering
        \begin{tikzpicture}[line width=0.6pt,line cap = round,>={triangle 45}]
            \draw[->] (0.0,0.0,0.0) -- (3.0,0.0,0.0) node[right] {$y$};
            \draw[->] (0.0,0.0,0.0) -- (0.0,3.0,0.0) node[above] {$z$};
            \draw[->] (0.0,0.0,0.0) -- (0.0,0.0,5.0) node[below left] {$x$};
            \filldraw[ball color=gray!90!white,opacity=0.3] (0,0,0) circle (2.0);
            \coordinate (z)     at (0.0,1.0,0.0)        {};
            \coordinate (x)     at (0.0,0.0,1.0)        {};
            \coordinate (o)     at (0.0,0.0,0.0)        {};
            \coordinate (p)     at (0.9,0.7,0.0)        {};
            \coordinate (p1)    at (0.9,-0.5,0.0)       {};
            \coordinate (p2)    at (1.4,0.0,0.0)        {};
            \coordinate (p3)    at (-0.5,-0.5,0.0)      {};
            \coordinate (Q)     at (0.0,2.5,0.0)        {};
            \node               at (Q) [right]         {Q};
            \node               at (p) [right]       {$p$};
            \node               at (-1,0.8,0)   {$\sigma$};
            \node               at (0.5,0.18,0) {$\mathbf{r}'$};
            \draw[-,densely dashed,draw=black] (o) -- (p1);
            \draw[-,densely dashed,draw=black] (p) -- (p1);
            \draw[-,densely dashed,draw=black] (p1) -- (p2);
            \draw[-,densely dashed,draw=black] (p1) -- (p3);
            \draw[->,>=stealth,draw=blue,semithick] (o) -- (p);
            \draw[->,>=stealth,draw=blue,semithick] (p) -- node[right]
            {\scriptsize{$\mathbf{R}$}}(Q);
            \filldraw[black] (Q) circle (0.5mm);
            \filldraw[black] (p) circle (0.5mm);
            \pic[draw=black, -, "$\theta$",angle eccentricity=1.5,angle radius = 0.4cm]
            {angle = p--o--z};
            \pic[draw=black, -,"$\varphi$", angle eccentricity=1.5,angle radius=0.25cm]
            {angle = x--o--p1};
        \end{tikzpicture}
        \caption{Drawing for Wangsness 2-8}
        \label{fig:EMAG_1_wangsness_2_8}
      \end{subfigure}
      \caption[Figures for Wangsness 2-3 and 2-8]{Figures for problems \ref{problem:EMAG_wangsness_2_3} and \ref{problem:EMAG_wangsness_2_8}, respectively.}
    \end{figure}
    \begin{problem}[Wangsness 2-8]
        \label{problem:EMAG_wangsness_2_8}
        Consider the sphere in Fig.~\subref{fig:EMAG_1_wangsness_2_8} of radius $a$ with
        a constant surface change density $\sigma$. What is the total charge $Q'$ on the
        sphere? Find the force produced by this charge distribution on a point $q$ on the
        $z$ axis for both $z>a$ and $z<a$.
    \end{problem}
    \begin{proof}[Solution]
        We have that:
        \begin{equation*}
            Q'=\iint_{S}\sigma da=\sigma\iint_{S}da=\sigma 4\pi a^{2}
        \end{equation*}
        The relative position vector $\mathbf{R}$ of the point $Q$ with respect to a
        point $\mathbf{r}'$ on the sphere is $z\hat{\mathbf{z}} - a\hat{\mathbf{r}}'$.
        So we have:
        \begin{equation*}
            \mathbf{F}_{q}
            =\frac{q}{4\pi\epsilon_0}\iint_{S}\frac{\sigma da'\mathbf{R}}{R^{3}}
            =\frac{q\sigma}{4\pi\epsilon_0}\int_{0}^{2\pi}\int_{0}^{\pi}
            \frac{(z\hat{\mathbf{z}}-a\hat{\mathbf{r}}')a^{2}
            \sin(\theta')d\theta'd\varphi'}{\big(z^2+a^2-2az\cos(\theta')\big)^{3/2}}
        \end{equation*}
        Writing $\hat{\mathbf{r}}'=\sin(\theta')\cos(\varphi')\hat{\mathbf{x}}+
        \sin(\theta')\sin(\varphi')\hat{\mathbf{y}}+\cos(\theta')\hat{\mathbf{z}}$
        leads us to conclude the $x$ and $y$ component vanish as
        $\int_{0}^{2\pi}\cos(\varphi')d\varphi'
        =\int_{0}^{2\pi}\sin(\varphi')d\varphi'=0$.
        Thus, we have:
        \begin{equation*}
            \mathbf{F}_{q}=\frac{q\sigma\hat{\mathbf{z}}}{4\pi\epsilon_{0}}
            \int_{0}^{2\pi}\int_{0}^{\pi}\frac{\big(z-a\cos(\theta)\big)a^{2}
            \sin(\theta')}{\big(z^{2}+a^{2}-2az\cos(\theta')\big)^{3/2}}d\theta'd\varphi'
        \end{equation*}
        Let $u=cos(\theta')$. Then $du=-\sin(\theta')d\theta'$. We obtain:
        \begin{align*}
            \mathbf{F}_{q}&=a^{2}\frac{q\sigma \hat{\mathbf{z}}}
            {2\epsilon_{0}}\int_{-1}^{1}\frac{u}{\big(z^{2}+a^{2}-2azu\big)^{3/2}}du&
            &=a^{2}\frac{q\sigma \hat{\mathbf{z}}}{2\epsilon_{0}}
            \frac{\partial}{\partial z}
            \bigg[\frac{1}{za}\sqrt{a^{2}+z^{2}-2azu}\bigg]_{-1}^{1}\\
            &=a^{2}\frac{q\sigma \hat{\mathbf{z}}}{2\epsilon_{0}}
            \int_{-1}^{1}\frac{\partial}{\partial z}
            \bigg(\frac{1}{\sqrt{a^{2}+z^{2}-2azu}}\bigg)du
            &&=a^{2}\frac{q\sigma \hat{\mathbf{z}}}{2\epsilon_{0}}
            \frac{\partial}{\partial z}\bigg(\frac{|z-a|-|z+a|}{az}\bigg)\\
            &=a^{2}\frac{q\sigma \hat{\mathbf{z}}}{2\epsilon_{0}}
            \frac{\partial}{\partial z} \int_{-1}^{1}
            \frac{1}{\sqrt{a^{2}+z^{2}-2azu}}du\\
        \end{align*}
        Now, if $z>a$, then $|z-a|-|z+a|=(z-a)-(z+a)=-2a$, and thus:
        \begin{equation*}
            \mathbf{F}_{q}=a^{2}\frac{q\sigma \hat{\mathbf{z}}}{2\epsilon_{0}}
            \frac{\partial}{\partial z}\bigg(\frac{-2a}{az}\bigg)
            =a^{2}\frac{q\sigma}{\epsilon_{0}z^{2}}\hat{\mathbf{z}}
            =\frac{qQ}{4\pi\epsilon_{0}z^{2}}\tag{$z>a$}
        \end{equation*}
        If $z<a$, then $|z-a|-|z+a|=2z$, and thus:
        \begin{equation*}
            \mathbf{F}=\mathbf{F}_{q}=a^{2}\frac{q\sigma
            \hat{\mathbf{z}}}{2\epsilon_{0}}
            \frac{\partial}{\partial z}\bigg(\frac{2z}{az}\bigg)
            =\mathbf{0}\tag{$z<a$}
        \end{equation*}
    \end{proof}
    \begin{problem}[Wangsness 2-7]
        Given a line change of length $L$ with constant charge density lying along the
        positive $z$ axis with its ends located at $z_{0}$ and $z_{0}+L$, find the total
        force exerted on this by a uniform spherical charge distribution with center at
        the origin and radius $a<z_{0}$.
    \end{problem}
    \begin{proof}[Solution]
        If the charge is distributed over a length $L$ along the $z$-axis with charge per
        unit length $\lambda$, the force over the length $L$ is given by:
        \begin{equation*}
            \mathbf{F}_{Lz}=\frac{\rho a^{3}}{3\epsilon_{0}}\hat{\mathbf{z}}
            \int_{z_{0}}^{z_{0}+L}\frac{\lambda dz}{z^{2}}=
            \frac{\rho\lambda a^{3}}{3\epsilon_{0}}\bigg(\frac{L}{z_{0}(z_{0}+L)}\bigg)
            \hat{\mathbf{z}}
        \end{equation*}
    \end{proof}
    \begin{problem}[Wangsness 3-9]
        Given two infinite plane sheets with equal and opposite constant surface charge
        densities $\sigma$ that are parallel and a distance $\pm a$ to the $xy$ plane,
        find $\mathbf{E}$ everywhere.
    \end{problem}
    \begin{proof}[Solution]
        For an infinite sheet, $\mathbf{E}=\frac{\sigma}{2\epsilon_{0}}$.
        Using the principle of superposition, we get:
        \begin{equation*}
            \mathbf{E}=
            \begin{cases}
                0,&|z|>a\\
                -\frac{\sigma}{\epsilon_{0}}\hat{\mathbf{z}},&|z|\leq a
            \end{cases}
        \end{equation*}
    \end{proof}
    \begin{problem}[Wangsness 3-10]
        A circular arc of radius $a$ with arc angle $2\alpha$ that lies in the $xy$
        plane and has a constant linear charge density $\lambda$ and center of curvature
        at the origin. Find $\mathbf{E}$ at an arbitrary point on the $z$ axis. Show that
        when the arc becomes a complete circle you obtain
        $\mathbf{E}=\frac{\lambda az\hat{\mathbf{z}}}{2\epsilon_{0}(a^{2}+z^{2})^{3/2}}$
    \end{problem}
    \begin{proof}[Solution]
        We have that:
        \begin{equation*}
            \hat{\mathbf{E}}=\frac{\lambda}{4\pi\epsilon_{0}}
            \int\frac{dl\hat{\mathbf{r}}}{R^{2}}=
            \frac{\lambda}{4\pi\epsilon_0}\int\frac{ad\phi'\hat{\mathbf{r}}}{a^{2}+z^{2}}
        \end{equation*}
        And
        \begin{equation*}
            \hat{\mathbf{r}}=\frac{\mathbf{R}}{R}=
            \frac{-\rho'\hat{\boldsymbol{\uprho}}
            +z\hat{\mathbf{z}}}{\sqrt{a^{2}+z^{2}}}=\frac{-a\cos(\phi')\hat{\mathbf{x}}
            -a\sin(\phi')\hat{\mathbf{y}}+z\hat{\mathbf{z}}}{\sqrt{a^{2}+z^{2}}}    
        \end{equation*}
        So, we obtain the following:
        \begin{equation*}
            \mathbf{E}=\frac{\lambda}{4\pi\epsilon_0}\int_{-\alpha}^{\alpha}
            \frac{-a\cos(\phi')\hat{\mathbf{x}}-a\sin(\phi')\hat{\mathbf{y}}
            +z\hat{\mathbf{z}}}{\sqrt{a^{2}+z^{2}}}ad\phi'
            =\frac{\lambda a\big[-a\sin(\alpha)\hat{\mathbf{x}}+z\alpha
            \hat{\mathbf{z}}\big]}{2\pi\epsilon_{0}(a^{2}+z^{2})^{3/2}}
        \end{equation*}
        If $\alpha=\pi$, we get:
        \begin{equation*}
            \mathbf{E}
            =\frac{\lambda az}{2\epsilon_{0}(a^{2}+z^{2})^{3/2}}\hat{\mathbf{z}}
        \end{equation*}
    \end{proof}
    \newpage
\end{document}