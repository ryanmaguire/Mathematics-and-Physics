\documentclass[crop=false,class=article,oneside]{standalone}
%----------------------------Preamble-------------------------------%
%---------------------------Packages----------------------------%
\usepackage{geometry}
\geometry{b5paper, margin=1.0in}
\usepackage[T1]{fontenc}
\usepackage{graphicx, float}            % Graphics/Images.
\usepackage{natbib}                     % For bibliographies.
\bibliographystyle{agsm}                % Bibliography style.
\usepackage[french, english]{babel}     % Language typesetting.
\usepackage[dvipsnames]{xcolor}         % Color names.
\usepackage{listings, lstlinebgrd}      % Verbatim-Like Tools.
\usepackage{mathtools, esint, mathrsfs} % amsmath and integrals.
\usepackage{amsthm, amsfonts}           % Fonts and theorems.
\usepackage{tabularx}
\usepackage{tcolorbox}                  % Frames around theorems.
\usepackage{upgreek}                    % Non-Italic Greek.
\usepackage{paracol}                    % Two-column styling.
\usepackage{wrapfig}                    % Wrap text around figure.
\usepackage{fmtcount, etoolbox}         % For the \book{} command.
\usepackage[newparttoc]{titlesec}       % Formatting chapter, etc.
\usepackage{titletoc}                   % Allows \book in toc.
\usepackage[nottoc]{tocbibind}          % Bibliography in toc.
\usepackage[titles]{tocloft}            % ToC formatting.
\usepackage{multicol, enumitem}         % Multi-column/enumerate.
\usepackage{import}                     % Import external files.
\usepackage{pgfplots, tikz}             % Drawing/graphing tools.
\usetikzlibrary{
    calc,                   % Calculating right angles and more.
    angles,                 % Drawing angles within triangles.
    arrows.meta,            % Latex and Stealth arrows.
    quotes,                 % Adding labels to angles.
    positioning,            % Relative positioning of nodes.
    decorations.markings,   % Adding arrows in the middle of a line.
    patterns,
    arrows,
    shapes,
    shapes.geometric,
    cd,
    hobby,
    babel
}                                       % Libraries for tikz.
\pgfplotsset{compat=1.9}                % Version of pgfplots.
\usepackage[font=scriptsize,
            labelformat=simple,
            labelsep=colon]{subcaption} % Subfigure captions.
\usepackage[font={scriptsize},
            hypcap=true,
            labelsep=colon]{caption}    % Figure captions.
\usepackage{hyperref}                   % Allows for hyperlinks.
\hypersetup{
    colorlinks=true,
    linkcolor=blue,
    filecolor=magenta,
    urlcolor=Cerulean,
    citecolor=SkyBlue
}                           % Colors for hyperref.
\usepackage[toc,acronym,nogroupskip]{glossaries} % Glossaries and acronyms.
\usepackage[subpreambles=false]{standalone}      % Complileable sub files.

% Various font stuff from kiwi.
% Use this for Times text and Computer Modern math
%\usepackage{times}

% Quite nice
%\usepackage[charter, greekfamily=, greekuppercase=italicized]{mathdesign}
%\usepackage[utopia, greekuppercase=italicized]{mathdesign}    % Math is narrower

% Use this for Times text and math
%\usepackage{newtxtext}
%\usepackage[libertine,cmintegrals]{newtxmath}
%\usepackage{fix-cm}

%\usepackage{txfontsb}
% or
%\usepackage{mathptmx}

%\usepackage[scaled=0.92]{helvet}
%\renewcommand{\rmdefault}{ptm}

%\usepackage{mathpazo}    % add possibly `sc` and `osf` options
%\usepackage{eulervm}

%\usepackage{fourier}
%\renewcommand{\rmdefault}{ptm}
%\usepackage{mathptm}

%\usepackage{fontspec}
%\setmainfont{lmodern}

%\usepackage[varg]{txfonts}
%\usepackage{fouriernc}
%\usepackage{mathpazo}

%\usepackage{bookman}
%\usepackage[scaled]{uarial}
%\usepackage[scaled]{helvet}
%\renewcommand*\familydefault{\sfdefault}
%\usepackage[math]{anttor}

%\newcommand\fgeorgia{\fontfamily{jvn}\selectfont}
%\newcommand\ftimes{\fontfamily{ptm}\selectfont}
%\newcommand\fhelvetica{\fontfamily{phv}\selectfont}
%\newcommand\fcourier{\fontfamily{pcr}\selectfont}
%\newcommand\fbookman{\fontfamily{pbk}\selectfont}
%\newcommand\fnewcentury{\fontfamily{pnc}\selectfont}
%\newcommand\fpalatino{\fontfamily{ppl}\selectfont}
%\newcommand\favantgarde{\fontfamily{pag}\selectfont}
%\newcommand\fnormal{\normalfont}
%\newcommand\fsize[1]{\ifnum#1>0\fontsize{#1}{#1}\selectfont\else\normalsize\fi}
%------------------------Theorem Styles-------------------------%
% Define theorem style for default spacing and normal font.
\newtheoremstyle{normal}
    {\topsep}               % Amount of space above the theorem.
    {\topsep}               % Amount of space below the theorem.
    {}                      % Font used for body of theorem.
    {}                      % Measure of space to indent.
    {\bfseries}             % Font of the header of the theorem.
    {}                      % Punctuation between head and body.
    {.5em}                  % Space after theorem head.
    {}

% Define theorem style for default spacing with italicized font.
\newtheoremstyle{normalit}{\topsep}{\topsep}
                {\itshape}{}{\bfseries}{}{.5em}{}

% Italic header environment.
\newtheoremstyle{thmit}{\topsep}{\topsep}{}{}{\itshape}{}{0.5em}{}

% Define italicized environments.
\theoremstyle{normalit}
\newtheorem{theorem}{Theorem}[section]
\newtheorem{lemma}{Lemma}[section]
\newtheorem{corollary}{Corollary}[section]
\newtheorem{proposition}{Proposition}[section]
\newtheorem*{theorem*}{Theorem}

% Define environments with italic headers.
\theoremstyle{thmit}
\newtheorem*{solution}{Solution}
\newtheorem*{fsolution}{Solution}

% Define default environments.
\theoremstyle{normal}
\newtheorem{example}{Example}[section]
\newtheorem{definition}{Definition}[section]
\newtheorem{problem}{Problem}[section]
\newtheorem{question}{Question}[section]
\newtheorem{remark}{Remark}[section]
\newtheorem{properties}{Properties}[section]
\newtheorem{notation}{Notation}[section]
\newtheorem{axiom}{Axiom}[section]
\newtheorem*{properties*}{Properties}
\newtheorem*{remark*}{Remark}
\newtheorem*{definition*}{Definition}
\theoremstyle{plain}

% Define framed environment.
\tcbuselibrary{most}
\newtcbtheorem[use counter*=theorem]{ftheorem}{Theorem}%
    {colback=green!5,colframe=green!35!black,
     fonttitle=\bfseries\upshape}{th}

\newtcbtheorem[use counter*=example]{fdefinition}{Definition}%
    {fonttitle=\bfseries\upshape,
     colback=blue!5!white,colframe=blue!75!black}{def}

\newtcbtheorem[use counter*=example]{fexample}{Example}%
    {fonttitle=\bfseries\upshape,
     colback=red!5!white,colframe=red!75!black}{ex}

\newtcbtheorem[use counter*=notation]{fnotation}{Notation}%
    {fonttitle=\bfseries\upshape,
     colback=SeaGreen!5!white,colframe=SeaGreen!75!black}{ex}

\newtcbtheorem[use counter*=corollary]{fcorollary}{Corollary}%
    {fonttitle=\bfseries\upshape,
     colback=Orchid!5!white,colframe=Orchid!75!black}{ex}

\newenvironment{bproof}{\textit{Proof.}}{\hfill$\square$}
\tcolorboxenvironment{bproof}{blanker,breakable,left=5mm,
                             before skip=10pt,after skip=10pt,
                             borderline west={1mm}{0pt}{red}}
\tcolorboxenvironment{fsolution}
    {enhanced jigsaw,colframe=cyan,interior hidden,breakable}

%--------------------Declared Math Operators--------------------%
\DeclareMathOperator{\Refl}{Refl}           % Reflection operator.
\DeclareMathOperator{\Span}{Span}           % Span of a set of vectors.
\DeclareMathOperator{\Card}{Card}           % Cardinality of set.
\DeclareMathOperator{\Ord}{Ord}             % Ordinal of ordered set.
\DeclareMathOperator{\Tr}{Tr}               % Trace of matrix.
\DeclareMathOperator{\adjoint}{adj}         % Adjoint of matrix.
\DeclareMathOperator{\rk}{rk}               % Rank of operator.
\DeclareMathOperator{\nul}{nul}             % Null space of operator.
\DeclareMathOperator{\sgn}{sgn}             % Sign of a number.
\DeclareMathOperator{\multideg}{mutlideg}   % Multi-Degree (Graphs).
\DeclareMathOperator{\GCD}{GCD}             % Greatest common denominator.
\DeclareMathOperator{\LM}{LM}               % Leading monomial
\DeclareMathOperator{\LC}{LC}               % Leading coefficient.
\DeclareMathOperator{\LT}{LT}               % Leading term.
\DeclareMathOperator{\LCM}{LCM}             % Least common multiple.
\DeclareMathOperator{\Mon}{Mon}             % Monomial.
\DeclareMathOperator{\Spec}{Spec}           % Spectrum.
\DeclareMathOperator{\proj}{proj}           % Projection.
\DeclareMathOperator{\comp}{comp}           % Component.
\DeclareMathOperator{\sinc}{sinc}           % Sinc function.
\DeclareMathOperator{\Ima}{Im}              % Image of operator.
\DeclareMathOperator{\Prin}{Prin}           % Principal value.
\DeclareMathOperator{\Mod}{mod}             % Modulus.
%------------------------New Commands---------------------------%
\DeclarePairedDelimiter\norm{\lVert}{\rVert}
\DeclarePairedDelimiter\ceil{\lceil}{\rceil}
\DeclarePairedDelimiter\floor{\lfloor}{\rfloor}
\newcommand*\diff{\mathop{}\!\mathrm{d}}
\newcommand*\Diff[1]{\mathop{}\!\mathrm{d^#1}}
\renewcommand{\mod}{\ \Mod}
\renewcommand*{\glstextformat}[1]{\textcolor{RoyalBlue}{#1}}
\renewcommand{\glsnamefont}[1]{\textbf{#1}}
\renewcommand\labelitemii{$\circ$}
\renewcommand\thesubfigure{\arabic{chapter}.\arabic{figure}}
\renewcommand\thesubfigure{%
    \arabic{chapter}.\arabic{figure}.\arabic{subfigure}}
\addto\captionsenglish{\renewcommand{\figurename}{Fig.}}
%------------------------Book Command---------------------------%
\makeatletter
\renewcommand\@pnumwidth{1cm}
\newcounter{book}
\renewcommand\thebook{\@Roman\c@book}
\newcommand\book{%
    \if@openright
        \cleardoublepage
    \else
        \clearpage
    \fi
    \thispagestyle{plain}%
    \if@twocolumn
        \onecolumn
        \@tempswatrue
    \else
        \@tempswafalse
    \fi
    \null\vfil
    \secdef\@book\@sbook
}
\def\@book[#1]#2{%
    \ifnum \c@secnumdepth >-3\relax
        \refstepcounter{book}%
        \addcontentsline{toc}{book}{
            \bookname\ \thebook:\hspace{1em}#1
        }
    \else
        \addcontentsline{toc}{book}{#1}%
    \fi
    \markboth{}{}%
    {\centering
     \interlinepenalty \@M
     \normalfont
     \ifnum \c@secnumdepth >-2\relax
       \huge\bfseries \bookname\nobreakspace\thebook
       \par
       \vskip 20\p@
     \fi
     \Huge \bfseries #2\par}%
    \@endbook}
\def\@sbook#1{%
    {\centering
     \interlinepenalty \@M
     \normalfont
     \Huge \bfseries #1\par}%
    \@endbook}
\def\@endbook{
    \vfil\newpage
        \if@twoside
            \if@openright
                \null
                \thispagestyle{empty}%
                \newpage
            \fi
        \fi
        \if@tempswa
            \twocolumn
        \fi
}
\newcommand*\l@book[2]{%
    \ifnum \c@tocdepth >-2\relax
        \addpenalty{-\@highpenalty}%
        \addvspace{2.25em \@plus\p@}%
        \setlength\@tempdima{3em}%
        \begingroup
            \parindent \z@ \rightskip \@pnumwidth
            \parfillskip -\@pnumwidth
            {
                \leavevmode
                \Large \bfseries #1\hfil \hb@xt@\@pnumwidth{
                    \hss #2
                }
            }
            \par
            \nobreak
            \global\@nobreaktrue
            \everypar{\global\@nobreakfalse\everypar{}}%
        \endgroup
    \fi}
\newcommand\bookname{Book}
\renewcommand{\thebook}{\texorpdfstring{\Numberstring{book}}{book}}
\providecommand*{\toclevel@book}{-2}
\makeatother
\titlecontents{chapter}[0pt]
    {\bfseries}
    {\chaptername\ \thecontentslabel:\quad}
    {}
    {\hfill\contentspage}
\titleformat{\part}[display]
    {\Large\bfseries}
    {\partname\nobreakspace\thepart}
    {0mm}
    {\Huge\bfseries}
    \titlecontents{part}[0pt]
    {\large\bfseries}
    {\partname\ \thecontentslabel: \quad}
    {}
    {\hfill\contentspage}
\newcommand{\MarkRightAngle}[4][.3cm]
    {\coordinate (tempa) at ($(#3)!#1!(#2)$);
     \coordinate (tempb) at ($(#3)!#1!(#4)$);
     \coordinate (tempc) at ($(tempa)!0.5!(tempb)$);%midpoint
     \draw (tempa) -- ($(#3)!2!(tempc)$) -- (tempb);}
%--------------------------LENGTHS------------------------------%
% Spacings for the Table of Contents.
\addtolength{\cftsecnumwidth}{1ex}
\addtolength{\cftsubsecindent}{1ex}
\addtolength{\cftsubsecnumwidth}{1ex}
\addtolength{\cftfignumwidth}{1ex}
\addtolength{\cfttabnumwidth}{1ex}

% Spacing for multi-column and enumerate environments.
\setlength{\multicolsep}{6pt}
\setlist[enumerate]{itemsep=0pt,topsep=3pt}

% Indent and paragraph spacing.
\setlength{\parindent}{0em}
\setlength{\parskip}{0em}
%--------------------------Main Document----------------------------%
\begin{document}
    \ifx\ifsub\undefined
        \section*{Electromagnetism I}
        \setcounter{section}{2}
        \renewcommand\thesubfigure{%
            \arabic{section}.\arabic{figure}.\arabic{subfigure}%
        }
    \fi    
    \subsection{Homework II}
        Wangsness Chapter 1 - Problems: 11, 12, 13, 14, 15
        \begin{problem}[Wangsness 1-11]
            \label{problem:EMAG_1_Wangsness_1_11}
            Calculate the path integral of
            $\mathbf{A}=x^{2}\hat{\mathbf{x}}
            +y^{2}\hat{\mathbf{y}}+z^{2}\hat{\mathbf{z}}$
            along the path shown in figure
            \subref{fig:EMAG_1_path_of_integration_for_wangsness_1_11}
            by integrating over $y$.
        \end{problem}
        \begin{proof}[Solution]
            The \textit{path integral} of $\mathbf{A}$ along a path $C$ is:
            \begin{equation*}
                \int_{C}\mathbf{A}\cdot\boldsymbol{d\ell}
                =\int_{C}\mathbf{A}\cdot\big(dx\hat{\mathbf{x}}
                +dy\hat{\mathbf{y}}+dz\hat{\mathbf{z}}\big)
            \end{equation*}
            We have $\mathbf{A}=
            x^{2}\hat{\mathbf{x}}+y^{2}\hat{\mathbf{y}}+z^{2}\hat{\mathbf{z}}$.
            Using this, we obtain:
            \begin{equation*}
                \int_{C}\mathbf{A}\cdot\boldsymbol{d\ell}
                =\int_{C}\big(x^{2}\hat{\mathbf{x}}
                +y^{2}\hat{\mathbf{y}}+z^{2}\hat{\mathbf{z}}\big)
                \cdot\big(dx\hat{\mathbf{x}}
                +dy\hat{\mathbf{y}}+dz\hat{\mathbf{z}}\big)
                =\int_{c}\big(x^{2}dx+y^{2}dy\big)
            \end{equation*}
            Along the path of integration, we have $x=y^{2}$, and therefore $dx=2ydy$.
            Substituting this back in:
            \begin{align*}
                \int_{C}\mathbf{A}\cdot\boldsymbol{d\ell}
                &=\int_{C}\big(x^{2}dx+y^{2}dy\big)
                &
                &=\bigg[\frac{1}{3}y^{6}+\frac{1}{3}y^{3}\bigg]_{0}^{\sqrt{2}}\\
                &=\int_{0}^{\sqrt{2}}\big((y^{2})^{2}(2ydy)+y^{2}dy\big)
                &
                &=\frac{1}{3}\big((\sqrt{2})^{6}+(\sqrt{2})^{3}\big)\\
                &=\int_{0}^{\sqrt{2}}\big(2y^{5}+y^{2}\big)dy
                &
                &=\frac{2}{3}\big(4+\sqrt{2}\big)\\
            \end{align*}
        \end{proof}
        \begin{figure}[H]
            \centering
            \begin{subfigure}[b]{0.49\textwidth}
                \centering
                \begin{tikzpicture}[>=triangle 45]
                    \begin{axis}[width=\linewidth,axis lines=center,
                    axis line style={->},
                    xtick distance=1,xlabel = $x$,xmin=-0.1,xmax=2.2,
                    ytick distance=1,ylabel = $y$,ymin=-0.1,ymax=2.1,
                    ->-/.style={decoration={markings,mark=at position .55 with
                    {\arrow{>}}},postaction={decorate}}]
                        \addplot[->-,line width=0.2mm,
                        samples=25,domain=0:1.4141,draw=blue]({\x^2},{\x});
                        \draw[dashed] (axis cs:2,0) -- (axis cs:2,1.4141);
                        \draw[dashed] (axis cs:0,1.4141) -- (axis cs:2,1.4141);
                        \node at (axis cs:1,0.7) {$y^2=x$};
                    \end{axis}
                \end{tikzpicture}
                \caption{Path of Integration for Wangsness 1-11}
                \label{fig:EMAG_1_path_of_integration_for_wangsness_1_11}
            \end{subfigure}
            \begin{subfigure}[b]{0.49\textwidth}
                \centering
                \begin{tikzpicture}[line width=1pt,line cap = round,>={Stealth},
                every edge/.style={draw=black,very thick}]
                    \draw[->] (0,0,0) -- (3,0,0) node[right] {$y$};
                    \draw[->] (0,0,0) -- (0,3,0) node[above] {$z$};
                    \draw[->] (0,0,0) -- (0,0,4) node[below left] {$x$};
                    \shade[fill=gray!60!white,opacity=0.5,draw=black,thick]
                    (2,0) arc (0:90:2) {[x={(0,0,1.33)}]
                    arc (90:0:2)} {[y={(0,0,1.33)}] arc (90:0:2)};
                    \draw[->,thick,draw=blue] (0.9,0.65) -- node [left]
                    {$\hat{\mathbf{z}}$} (0.9,1.3);
                    \draw[->,thick,draw=red] (0.9,0.65) -- node [below right]
                    {$\hat{\mathbf{n}}$} (1.3,0.9);
                    \draw[fill=orange] 
                    (0.8,0.6) -- (0.9,0.6) -- (1,0.7) -- (0.9,0.7) --cycle;
                    \node at (0.73,0.67) [below] {$da$};
                \end{tikzpicture}
                \caption{Geometry for Wangsness 1-12}
                \label{fig:EMAG_1_geometry_for_wangsness_1_12}
            \end{subfigure}
            \caption[Figures for Wangsness 1-11 and 1-12]{Figures for Problems
            \ref{problem:EMAG_1_Wangsness_1_11} and
            \ref{problem:EMAG_1_wangsness_1_12}, Respectively.}
            \label{fig:EMAG_1_figures_for_wangsness_1_11_and_1_12}
        \end{figure}
        \begin{problem}[Wangsness 1-12]
            \label{problem:EMAG_1_wangsness_1_12}
            Find the surface integral of $\mathbf{r}$ and the volume integral of
            $\nabla\cdot\mathbf{r}$ for a sphere of radius $a_{0}$ centered
            at the origin.
        \end{problem}
        \begin{proof}[Solution]
            The \textit{surface integral} of $\mathbf{A}$ over a closed surface
            $\partial\Sigma$ is defined as:
            \begin{equation*}
                \oiint_{\partial\Sigma}\mathbf{A}\cdot\boldsymbol{da}
                =\oiint_{\partial\Sigma}\mathbf{A}\cdot\hat{\boldsymbol{n}}da
            \end{equation*}
            Where $\hat{\mathbf{n}}$ is the unit normal to the surface
            $\partial\Sigma$.
            For a sphere, we have:
            \begin{equation*}
                \hat{\mathbf{n}}
                =\frac{\nabla(u)}{\norm{\nabla(u)}}
                =\frac{2x\hat{\mathbf{x}}+2y\hat{\mathbf{y}}+2z\hat{\mathbf{z}}}
                {\sqrt{4x^{2}+4y^{2}+4z^{2}}}
                = \frac{x\hat{\mathbf{x}}+\hat{\mathbf{y}}+z\hat{\mathbf{z}}}
                {\sqrt{x^{2}+y^{2}+z^{2}}}
            \end{equation*}
            Thus, we have:
            \begin{equation*}
                \oiint_{\partial\Sigma}\mathbf{r}\cdot\hat{\mathbf{n}}da
                =\oiint_{\partial\Sigma}\bigg(x\hat{\mathbf{x}}
                +y\hat{\mathbf{y}}+z\hat{\mathbf{z}}\bigg)\cdot
                \bigg(\frac{x\hat{\mathbf{x}}
                +y\hat{\mathbf{y}}+z\hat{\mathbf{z}}}{\sqrt{x^{2}+y^{2}+z^{2}}}\bigg)da
                =\oiint_{\partial\Sigma}\sqrt{x^{2}+y^{2}+z^{2}}da
            \end{equation*}
            But recall that $x^{2}+y^{2}+z^{2}=a_{0}^{2}$, so we have:
            \begin{equation*}
                \oiint_{\partial\Sigma}\mathbf{r}\cdot\boldsymbol{da}
                =a_{0}\oiint_{\partial\Sigma}da\\
            \end{equation*}
            But $\oiint_{\partial\Sigma}da$ is just the surface area of
            $\partial\Sigma$.
            And the surface area of the sphere is $4\pi a_{0}^{2}$. So:
            \begin{equation*}
                \oiint_{\partial\Sigma}\mathbf{r}\cdot \boldsymbol{da}=4\pi a_{0}^{3}
            \end{equation*}
            Using spherical coordinates is much easier.
            \begin{equation*}
                \oiint_{\partial\Sigma}\mathbf{r}\cdot\boldsymbol{da}
                =\int_{0}^{2\pi}\int_{0}^{\pi}a_{0}\hat{\mathbf{r}}\cdot
                \hat{\mathbf{r}}a_{0}^{2}\sin(\theta)d\theta d\varphi
                =\int_{0}^{2\pi}\int_{0}^{\pi}a_{0}^{3}\sin(\theta)d\theta d\varphi
                =2\pi a_{0}^{3}\int_{0}^{\pi}\sin(\theta)d\theta=4\pi a_{0}^{3}
            \end{equation*}
            To compute the \textit{volume integral} of $\nabla \cdot \mathbf{r}$ within
            $\Sigma$, we compute $\nabla\cdot \mathbf{r}$ and then integrate:
            \begin{align*}
                \nabla\cdot\mathbf{r}&=\frac{\partial x}{\partial x}
                +\frac{\partial y}{\partial y}+\frac{\partial z}{\partial z}=3\\
                \iiint_{\Sigma}\nabla\cdot\mathbf{r}d\tau&=\iiint_{\Sigma}3d\tau
                =3\iiint_{\Sigma}d\tau=3\frac{4}{3}\pi a_{0}^{3}=4\pi a_{0}^{3}
            \end{align*}
        \end{proof} 
        \begin{problem}[Wangsness 1-13]
            \label{problem:EMAG_1_wangsness_1_13}
            Given the vector field
            $\mathbf{A}=xy\hat{\mathbf{x}}+yz\hat{\mathbf{y}}+xz\hat{\mathbf{z}}$,
            evaluate the flux of $\mathbf{A}$ through a parallelepiped of sides $a,b,c$
            shown in figure \subref{fig:EMAG_1_wangsness_1_13_region_of_integration}.
            Compute $\int\nabla\cdot\mathbf{A}d\tau$ over the volume.
        \end{problem}
        \begin{proof}[Solution]
            There are six sides we must integrate over. Given $\mathbf{A}=xy\hat{\mathbf{x}}+yz\hat{\mathbf{y}}+xz\hat{\mathbf{z}}$,
            we have:
            \begin{align*}
                \oiint_{\partial\Sigma}\mathbf{A}\cdot\boldsymbol{da}
                &=\oiint_{\partial\Sigma}(xydydz+yzdxdz+xzdxdz)\\
                &=\underset{\textrm{Front}}{\iint}xydydz
                -\underset{\textrm{Back}}{\iint}xydydz
                +\underset{\textrm{Right}}{\iint}yzdxdz
                -\underset{\textrm{Left}}{\iint}yzdxdz
                +\underset{\textrm{Top}}{\iint}xzdxdy
                -\underset{\textrm{Bottom}}{\iint}xzdxdy\\
                &=\int_{0}^{c}\int_{0}^{b}(a)ydydz
                +\int_{0}^{c}\int_{0}^{a}(b)zdxdz
                +\int_{0}^{b}\int_{0}^{a}x(c)dxdy=\frac{abc}{2}(a+b+c)
            \end{align*}
            To compute $\iiint_{V}\nabla\cdot\mathbf{A}d\tau$, we have:
            $\nabla\cdot\mathbf{A}=
            \frac{\partial(xy)}{\partial x}
            +\frac{\partial(yz)}{\partial y}+\frac{\partial(xz)}{\partial z}=x+y+z$.
            Thus:
            \begin{align*}
                \iiint_{\Sigma}\nabla\cdot\mathbf{A}d\tau
                &=\iiint_{\Sigma}(x+y+z)d\tau=
                \int_{0}^{c}\int_{0}^{b}\int_{0}^{a}(x+y+z)dxdydz\\
                &=\int_{0}^{c}\int_{0}^{b}\int_{0}^{a}xdxdydz
                +\int_{0}^{c}\int_{0}^{b}\int_{0}^{a}ydxdydz
                +\int_{0}^{c}\int_{0}^{b}\int_{0}^{a}zdxdydz\\
                &=\frac{a^{2}bc}{2}+\frac{ab^{2}c}{2}
                +\frac{abc^{2}}{2}=\frac{abc}{2}(a+b+c)
            \end{align*}
        \end{proof}
        \begin{figure}[H]
            \centering
            \begin{subfigure}[b]{0.49\textwidth}
                \begin{tikzpicture}[line width=0.4pt,line cap = round,>={Stealth}]
                    \draw[->,semithick] (0,0) -- (4,0) node[right] {$y$};
                    \draw[->,semithick] (0,0) -- (0,3) node[above] {$z$};
                    \draw[->,semithick] (0,0) -- (-2,-2) node[below left] {$x$};
                    \draw[ball color=gray!10!white,opacity=0.6]
                    (-1.2,-1.2) -- (1.5,-1.2) -- (1.5,0.8) -- (-1.2,0.8) -- cycle;
                    \draw[ball color = gray!90!white,opacity=0.6]
                    (-1.2,0.8) -- (0,1.6) -- (2.7,1.6) -- (1.5,0.8) -- cycle;
                    \draw[fill=gray,opacity=0.6]
                    (1.5,0.8) -- (2.7,1.6) -- (2.7,0) -- (1.5,-1.2) -- cycle;
                    \filldraw[fill=black] (-1.2,-1.2) circle (0.04) node [below] {$a$};
                    \filldraw[fill=black]
                    (0,1.6) circle (0.04) node [above right] {$c$};
                    \filldraw[fill=black] (2.7,0) circle (0.04) node [below] {$b$};
                \end{tikzpicture}
            \caption{Wangsness 1-13}
            \label{fig:EMAG_1_wangsness_1_13_region_of_integration}
            \end{subfigure}
            \begin{subfigure}[b]{0.49\textwidth}
                \centering
                \begin{tikzpicture}[>=triangle 45,
                ->-/.style={decoration={markings,
                mark=at position .55 with {\arrow{>}}},postaction={decorate}}]
                    \begin{axis}[width=\textwidth,axis lines=center,
                    xlabel = $x$,xmin=-0.15,xmax=4,xtick distance=3,
                    ylabel = $y$,ymin=-0.15,ymax=5,ytick distance=4]
                        \draw[->-,line width=0.2mm,draw=blue]
                        (axis cs:0,0) -- (axis cs:3,0);
                        \draw[->-,line width=0.2mm,draw=blue]
                        (axis cs:3,0) -- (axis cs:3,4);
                        \draw[->-,line width=0.2mm,draw=blue]
                        (axis cs:3,4) -- (axis cs:0,4);
                        \draw[->-,line width=0.2mm,draw=blue]
                        (axis cs:0,4) -- (axis cs:0,0);
                    \end{axis}
                \end{tikzpicture}
                \caption{Wangsness 1-14}
                \label{fig:EMAG_1_wangsness_1_14}
            \end{subfigure}
            \caption[Figures for Wangsness 1-13 and 1-14]
            {Figures for problems
            \ref{problem:EMAG_1_wangsness_1_13} and
            \ref{problem:EMAG_1_wangsness_1_14}, Respectively.}
        \end{figure}
        \begin{problem}[Wangsness 1-14]
            \label{problem:EMAG_1_wangsness_1_14}
            Given $\mathbf{A}=-y\hat{\mathbf{x}}+x\hat{\mathbf{y}}$, calculate the line
            integral $\oint\mathbf{A}\cdot\mathbf{ds}$ over the closed path in the $xy$
            plane shown in figure \subref{fig:EMAG_1_wangsness_1_14}.
        \end{problem}
        \begin{proof}[Solution]
            Given $\mathbf{A}=-y\hat{\mathbf{x}}+x\hat{\mathbf{y}}$, we have:
            \begin{align*}
                \oint_{\partial S}\mathbf{A}\cdot\boldsymbol{d\ell}
                &=\oint_{\partial S}\big(-y\hat{\mathbf{x}}
                +x\hat{\mathbf{y}}\big)\cdot\big(dx\hat{\mathbf{x}}
                +dy\hat{\mathbf{y}}\big)\\
                &=\underbrace{\int_{0}^{3}(-ydx+xdy)}_{y=0,\ dy=0}
                +\underbrace{\int_{0}^{4}(-ydx+xdy)}_{x=3,\ dx=0}
                +\underbrace{\int_{3}^{0}(-ydx+xdy)}_{y =4,\ dy=0}
                +\underbrace{\int_{4}^{0}(-ydx+xdy)}_{x=0,\ dx=0}\\
                &=0+12+12+0=24
            \end{align*}
            Next, we compute $\iint(\nabla\times\mathbf{A})\cdot\boldsymbol{da}$. We
            have $\nabla\times\mathbf{A}=2\hat{\mathbf{z}}$. Thus:
            \begin{equation*}
                \iint_{S}\big(\nabla\times\mathbf{A}\big)\cdot\boldsymbol{da}
                =\iint_{S}\big(2\hat{\mathbf{z}}\big)\cdot \big(dydz\hat{\mathbf{x}}
                +dxdz\hat{\mathbf{y}}+dxdy\hat{\mathbf{z}}\big)
                =\int_{0}^{4}\int_{0}^{3}2dydx=24
            \end{equation*}
        \end{proof}
        \begin{problem}[Wangsness 1-15]
            \label{problem:EMAG_1_wangsness_1_15}
            Given $\mathbf{A}=x^{2}y\hat{\mathbf{x}}
            +xy^{2}\hat{\mathbf{y}}+a^{3}e^{-\beta y}\cos(\alpha x)\hat{\mathbf{z}}$,
            compute $\oint\mathbf{A}\cdot \boldsymbol{d\ell}$ along the path in figure
            \ref{fig:EMAG_1_wangsness_1_15}. Compute
            $\iint(\nabla\times\mathbf{A})\cdot\boldsymbol{da}$ over the same region.
        \end{problem}
        \begin{proof}[Solution]
            Along the entire contour, we have $z=0$ and $dz=0$. Thus, we have:
            \begin{equation*}
                \oint_{\partial S}\mathbf{A}\cdot\boldsymbol{d\ell}
                =\oint_{\partial S}\big(x^{2}y\hat{\mathbf{x}}+xy^{2}\hat{\mathbf{y}}
                +a^{3}e^{\beta y}\cos(\alpha x)\hat{\mathbf{z}}\big)
                \cdot\big(dx\hat{\mathbf{x}}+dy\hat{\mathbf{y}}\big)
                =\oint_{\partial S}\big(x^{2}ydx+xy^{2}dy\big)
            \end{equation*}
            Along the first path, we have $x=0$ and $dx=0$. Along the second path, we
            have $y=\sqrt{2k}, dy=0$. Along the third path we have $y^2=kx$, and thus
            $dx=\frac{2ydy}{k}$. So:
            \begin{align*}
                \oint_{\partial S}\mathbf{A}\cdot\boldsymbol{d\ell}
                &=\int_{C_{1}}\mathbf{A}\cdot\boldsymbol{d\ell}
                +\int_{C_{2}}\mathbf{A}\cdot\boldsymbol{d\ell}
                +\int_{C_{3}}\mathbf{A}\cdot\boldsymbol{d\ell}\\
                &=\int_{0}^{\sqrt{2k}}(0)y^{2}dy+\int_{0}^{2}x^{2}\sqrt{2k}dx
                +\int_{\sqrt{2k}}^{0}\big(\frac{y^{4}}{k^2}y\frac{2y}{k}dy
                +\frac{y^{2}}{k}y^{2}dy\big)\\
                &=\frac{8}{3}\sqrt{2k}+\int_{\sqrt{2k}}^{0}\big(2\frac{y^{6}}{k^{3}}
                +\frac{y^{4}}{k}\big)dy
                =\frac{8}{3}\sqrt{2k}-\frac{16}{7}\sqrt{2k}-\frac{4k\sqrt{2k}}{5}
                =\sqrt{2k}\big(\frac{8}{21}-\frac{4}{5}k\big)
            \end{align*}
            Next we compute $\iint(\nabla\times\mathbf{A})\cdot\boldsymbol{da}$. Note
            that $\boldsymbol{da}=\hat{\mathbf{z}}dxdy$. The $\hat{\mathbf{z}}$
            component for $\nabla\times\mathbf{A}$ is $(y^{2}-x^{2})\hat{\mathbf{z}}$.
            We have:
            \begin{align*}
                \iint_{\Sigma}\big(\nabla\times\mathbf{A}\big)\cdot\boldsymbol{da}
                &=\int_{0}^{2}\int_{\sqrt{kx}}^{\sqrt{2k}} \big(x^{2}-y^{2}\big)dydx
                =\int_{0}^{2}\bigg[x^{2}y
                -\frac{y^{3}}{3}\bigg]_{\sqrt{kx}}^{\sqrt{2k}}dx\\
                &=\int_{0}^{2}\bigg[\bigg(x^{2}\sqrt{2k}-\frac{2k\sqrt{2k}}{3}\bigg)
                -\bigg(x^{2}\sqrt{kx}-\frac{kx\sqrt{kx}}{3}\bigg)\bigg]dx\\
                &=\sqrt{2k}\int_{0}^{2}x^{2}dx-\frac{2k}{3}\sqrt{2k}
                \int_{0}^{2}dx-\sqrt{k}\int_{0}^{2}x^{\frac{5}{2}}dx
                +\frac{k\sqrt{k}}{3}\int_{0}^{2}x^{\frac{3}{2}}dx\\
                &=\frac{8}{3}\sqrt{2k}-\frac{4k}{3}\sqrt{2k}-\frac{16}{7}\sqrt{2k}
                +\frac{8k}{15}\sqrt{2k}=\frac{8}{21}\sqrt{2k}-\frac{4}{5}k\sqrt{2k}
                =\sqrt{2k}\big(\frac{8}{21}-\frac{4k}{5}\big)
            \end{align*}
        \end{proof}
        \begin{figure}[H]
            \centering
            \begin{tikzpicture}[>=triangle 45,
            ->-/.style={decoration={markings,
            mark=at position .55 with {\arrow{>}}},postaction={decorate}}]
                \begin{axis}[width=0.4\textwidth,axis lines=center,
                axis lines=middle,
                xlabel = $x$,xmin=-0.1,xmax=2.3,xtick distance=1,
                ylabel = $y$,ymin=-0.1,ymax=2.5,ytick distance=2,
                yticklabel={$\sqrt{\pgfmathprintnumber{\tick}k}$}]
                    \draw[->-,line width=0.2mm,draw=blue]
                    (axis cs:0,0) -- (axis cs:0,2);
                    \draw[->-,line width=0.2mm,draw=blue]
                    (axis cs:0,2) -- (axis cs:2,2);
                    \addplot[->-,line width=0.2mm,
                    samples=25,domain=1.414214:0,draw=blue] (x*x,1.4142*x);
                    \node at (axis cs:1,0.75) {$y^2=kx$};
                \end{axis}
            \end{tikzpicture}
            \caption[Figure for Wangsness 1-15]
            {Figure for problem \ref{problem:EMAG_1_wangsness_1_15}}
            \label{fig:EMAG_1_wangsness_1_15}
        \end{figure}
\end{document}