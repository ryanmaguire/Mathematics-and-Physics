\documentclass[crop=false,class=article,oneside]{standalone}
%----------------------------Preamble-------------------------------%
%---------------------------Packages----------------------------%
\usepackage{geometry}
\geometry{b5paper, margin=1.0in}
\usepackage[T1]{fontenc}
\usepackage{graphicx, float}            % Graphics/Images.
\usepackage{natbib}                     % For bibliographies.
\bibliographystyle{agsm}                % Bibliography style.
\usepackage[french, english]{babel}     % Language typesetting.
\usepackage[dvipsnames]{xcolor}         % Color names.
\usepackage{listings}                   % Verbatim-Like Tools.
\usepackage{mathtools, esint, mathrsfs} % amsmath and integrals.
\usepackage{amsthm, amsfonts, amssymb}  % Fonts and theorems.
\usepackage{tcolorbox}                  % Frames around theorems.
\usepackage{upgreek}                    % Non-Italic Greek.
\usepackage{fmtcount, etoolbox}         % For the \book{} command.
\usepackage[newparttoc]{titlesec}       % Formatting chapter, etc.
\usepackage{titletoc}                   % Allows \book in toc.
\usepackage[nottoc]{tocbibind}          % Bibliography in toc.
\usepackage[titles]{tocloft}            % ToC formatting.
\usepackage{pgfplots, tikz}             % Drawing/graphing tools.
\usepackage{imakeidx}                   % Used for index.
\usetikzlibrary{
    calc,                   % Calculating right angles and more.
    angles,                 % Drawing angles within triangles.
    arrows.meta,            % Latex and Stealth arrows.
    quotes,                 % Adding labels to angles.
    positioning,            % Relative positioning of nodes.
    decorations.markings,   % Adding arrows in the middle of a line.
    patterns,
    arrows
}                                       % Libraries for tikz.
\pgfplotsset{compat=1.9}                % Version of pgfplots.
\usepackage[font=scriptsize,
            labelformat=simple,
            labelsep=colon]{subcaption} % Subfigure captions.
\usepackage[font={scriptsize},
            hypcap=true,
            labelsep=colon]{caption}    % Figure captions.
\usepackage[pdftex,
            pdfauthor={Ryan Maguire},
            pdftitle={Mathematics and Physics},
            pdfsubject={Mathematics, Physics, Science},
            pdfkeywords={Mathematics, Physics, Computer Science, Biology},
            pdfproducer={LaTeX},
            pdfcreator={pdflatex}]{hyperref}
\hypersetup{
    colorlinks=true,
    linkcolor=blue,
    filecolor=magenta,
    urlcolor=Cerulean,
    citecolor=SkyBlue
}                           % Colors for hyperref.
\usepackage[toc,acronym,nogroupskip,nopostdot]{glossaries}
\usepackage{glossary-mcols}
%------------------------Theorem Styles-------------------------%
\theoremstyle{plain}
\newtheorem{theorem}{Theorem}[section]

% Define theorem style for default spacing and normal font.
\newtheoremstyle{normal}
    {\topsep}               % Amount of space above the theorem.
    {\topsep}               % Amount of space below the theorem.
    {}                      % Font used for body of theorem.
    {}                      % Measure of space to indent.
    {\bfseries}             % Font of the header of the theorem.
    {}                      % Punctuation between head and body.
    {.5em}                  % Space after theorem head.
    {}

% Italic header environment.
\newtheoremstyle{thmit}{\topsep}{\topsep}{}{}{\itshape}{}{0.5em}{}

% Define environments with italic headers.
\theoremstyle{thmit}
\newtheorem*{solution}{Solution}

% Define default environments.
\theoremstyle{normal}
\newtheorem{example}{Example}[section]
\newtheorem{definition}{Definition}[section]
\newtheorem{problem}{Problem}[section]

% Define framed environment.
\tcbuselibrary{most}
\newtcbtheorem[use counter*=theorem]{ftheorem}{Theorem}{%
    before=\par\vspace{2ex},
    boxsep=0.5\topsep,
    after=\par\vspace{2ex},
    colback=green!5,
    colframe=green!35!black,
    fonttitle=\bfseries\upshape%
}{thm}

\newtcbtheorem[auto counter, number within=section]{faxiom}{Axiom}{%
    before=\par\vspace{2ex},
    boxsep=0.5\topsep,
    after=\par\vspace{2ex},
    colback=Apricot!5,
    colframe=Apricot!35!black,
    fonttitle=\bfseries\upshape%
}{ax}

\newtcbtheorem[use counter*=definition]{fdefinition}{Definition}{%
    before=\par\vspace{2ex},
    boxsep=0.5\topsep,
    after=\par\vspace{2ex},
    colback=blue!5!white,
    colframe=blue!75!black,
    fonttitle=\bfseries\upshape%
}{def}

\newtcbtheorem[use counter*=example]{fexample}{Example}{%
    before=\par\vspace{2ex},
    boxsep=0.5\topsep,
    after=\par\vspace{2ex},
    colback=red!5!white,
    colframe=red!75!black,
    fonttitle=\bfseries\upshape%
}{ex}

\newtcbtheorem[auto counter, number within=section]{fnotation}{Notation}{%
    before=\par\vspace{2ex},
    boxsep=0.5\topsep,
    after=\par\vspace{2ex},
    colback=SeaGreen!5!white,
    colframe=SeaGreen!75!black,
    fonttitle=\bfseries\upshape%
}{not}

\newtcbtheorem[use counter*=remark]{fremark}{Remark}{%
    fonttitle=\bfseries\upshape,
    colback=Goldenrod!5!white,
    colframe=Goldenrod!75!black}{ex}

\newenvironment{bproof}{\textit{Proof.}}{\hfill$\square$}
\tcolorboxenvironment{bproof}{%
    blanker,
    breakable,
    left=3mm,
    before skip=5pt,
    after skip=10pt,
    borderline west={0.6mm}{0pt}{green!80!black}
}

\AtEndEnvironment{lexample}{$\hfill\textcolor{red}{\blacksquare}$}
\newtcbtheorem[use counter*=example]{lexample}{Example}{%
    empty,
    title={Example~\theexample},
    boxed title style={%
        empty,
        size=minimal,
        toprule=2pt,
        top=0.5\topsep,
    },
    coltitle=red,
    fonttitle=\bfseries,
    parbox=false,
    boxsep=0pt,
    before=\par\vspace{2ex},
    left=0pt,
    right=0pt,
    top=3ex,
    bottom=1ex,
    before=\par\vspace{2ex},
    after=\par\vspace{2ex},
    breakable,
    pad at break*=0mm,
    vfill before first,
    overlay unbroken={%
        \draw[red, line width=2pt]
            ([yshift=-1.2ex]title.south-|frame.west) to
            ([yshift=-1.2ex]title.south-|frame.east);
        },
    overlay first={%
        \draw[red, line width=2pt]
            ([yshift=-1.2ex]title.south-|frame.west) to
            ([yshift=-1.2ex]title.south-|frame.east);
    },
}{ex}

\AtEndEnvironment{ldefinition}{$\hfill\textcolor{Blue}{\blacksquare}$}
\newtcbtheorem[use counter*=definition]{ldefinition}{Definition}{%
    empty,
    title={Definition~\thedefinition:~{#1}},
    boxed title style={%
        empty,
        size=minimal,
        toprule=2pt,
        top=0.5\topsep,
    },
    coltitle=Blue,
    fonttitle=\bfseries,
    parbox=false,
    boxsep=0pt,
    before=\par\vspace{2ex},
    left=0pt,
    right=0pt,
    top=3ex,
    bottom=0pt,
    before=\par\vspace{2ex},
    after=\par\vspace{1ex},
    breakable,
    pad at break*=0mm,
    vfill before first,
    overlay unbroken={%
        \draw[Blue, line width=2pt]
            ([yshift=-1.2ex]title.south-|frame.west) to
            ([yshift=-1.2ex]title.south-|frame.east);
        },
    overlay first={%
        \draw[Blue, line width=2pt]
            ([yshift=-1.2ex]title.south-|frame.west) to
            ([yshift=-1.2ex]title.south-|frame.east);
    },
}{def}

\AtEndEnvironment{ltheorem}{$\hfill\textcolor{Green}{\blacksquare}$}
\newtcbtheorem[use counter*=theorem]{ltheorem}{Theorem}{%
    empty,
    title={Theorem~\thetheorem:~{#1}},
    boxed title style={%
        empty,
        size=minimal,
        toprule=2pt,
        top=0.5\topsep,
    },
    coltitle=Green,
    fonttitle=\bfseries,
    parbox=false,
    boxsep=0pt,
    before=\par\vspace{2ex},
    left=0pt,
    right=0pt,
    top=3ex,
    bottom=-1.5ex,
    breakable,
    pad at break*=0mm,
    vfill before first,
    overlay unbroken={%
        \draw[Green, line width=2pt]
            ([yshift=-1.2ex]title.south-|frame.west) to
            ([yshift=-1.2ex]title.south-|frame.east);},
    overlay first={%
        \draw[Green, line width=2pt]
            ([yshift=-1.2ex]title.south-|frame.west) to
            ([yshift=-1.2ex]title.south-|frame.east);
    }
}{thm}

%--------------------Declared Math Operators--------------------%
\DeclareMathOperator{\adjoint}{adj}         % Adjoint.
\DeclareMathOperator{\Card}{Card}           % Cardinality.
\DeclareMathOperator{\curl}{curl}           % Curl.
\DeclareMathOperator{\diam}{diam}           % Diameter.
\DeclareMathOperator{\dist}{dist}           % Distance.
\DeclareMathOperator{\Div}{div}             % Divergence.
\DeclareMathOperator{\Erf}{Erf}             % Error Function.
\DeclareMathOperator{\Erfc}{Erfc}           % Complementary Error Function.
\DeclareMathOperator{\Ext}{Ext}             % Exterior.
\DeclareMathOperator{\GCD}{GCD}             % Greatest common denominator.
\DeclareMathOperator{\grad}{grad}           % Gradient
\DeclareMathOperator{\Ima}{Im}              % Image.
\DeclareMathOperator{\Int}{Int}             % Interior.
\DeclareMathOperator{\LC}{LC}               % Leading coefficient.
\DeclareMathOperator{\LCM}{LCM}             % Least common multiple.
\DeclareMathOperator{\LM}{LM}               % Leading monomial.
\DeclareMathOperator{\LT}{LT}               % Leading term.
\DeclareMathOperator{\Mod}{mod}             % Modulus.
\DeclareMathOperator{\Mon}{Mon}             % Monomial.
\DeclareMathOperator{\multideg}{mutlideg}   % Multi-Degree (Graphs).
\DeclareMathOperator{\nul}{nul}             % Null space of operator.
\DeclareMathOperator{\Ord}{Ord}             % Ordinal of ordered set.
\DeclareMathOperator{\Prin}{Prin}           % Principal value.
\DeclareMathOperator{\proj}{proj}           % Projection.
\DeclareMathOperator{\Refl}{Refl}           % Reflection operator.
\DeclareMathOperator{\rk}{rk}               % Rank of operator.
\DeclareMathOperator{\sgn}{sgn}             % Sign of a number.
\DeclareMathOperator{\sinc}{sinc}           % Sinc function.
\DeclareMathOperator{\Span}{Span}           % Span of a set.
\DeclareMathOperator{\Spec}{Spec}           % Spectrum.
\DeclareMathOperator{\supp}{supp}           % Support
\DeclareMathOperator{\Tr}{Tr}               % Trace of matrix.
%--------------------Declared Math Symbols--------------------%
\DeclareMathSymbol{\minus}{\mathbin}{AMSa}{"39} % Unary minus sign.
%------------------------New Commands---------------------------%
\DeclarePairedDelimiter\norm{\lVert}{\rVert}
\DeclarePairedDelimiter\ceil{\lceil}{\rceil}
\DeclarePairedDelimiter\floor{\lfloor}{\rfloor}
\newcommand*\diff{\mathop{}\!\mathrm{d}}
\newcommand*\Diff[1]{\mathop{}\!\mathrm{d^#1}}
\renewcommand*{\glstextformat}[1]{\textcolor{RoyalBlue}{#1}}
\renewcommand{\glsnamefont}[1]{\textbf{#1}}
\renewcommand\labelitemii{$\circ$}
\renewcommand\thesubfigure{%
    \arabic{chapter}.\arabic{figure}.\arabic{subfigure}}
\addto\captionsenglish{\renewcommand{\figurename}{Fig.}}
\numberwithin{equation}{section}

\renewcommand{\vector}[1]{\boldsymbol{\mathrm{#1}}}

\newcommand{\uvector}[1]{\boldsymbol{\hat{\mathrm{#1}}}}
\newcommand{\topspace}[2][]{(#2,\tau_{#1})}
\newcommand{\measurespace}[2][]{(#2,\varSigma_{#1},\mu_{#1})}
\newcommand{\measurablespace}[2][]{(#2,\varSigma_{#1})}
\newcommand{\manifold}[2][]{(#2,\tau_{#1},\mathcal{A}_{#1})}
\newcommand{\tanspace}[2]{T_{#1}{#2}}
\newcommand{\cotanspace}[2]{T_{#1}^{*}{#2}}
\newcommand{\Ckspace}[3][\mathbb{R}]{C^{#2}(#3,#1)}
\newcommand{\funcspace}[2][\mathbb{R}]{\mathcal{F}(#2,#1)}
\newcommand{\smoothvecf}[1]{\mathfrak{X}(#1)}
\newcommand{\smoothonef}[1]{\mathfrak{X}^{*}(#1)}
\newcommand{\bracket}[2]{[#1,#2]}

%------------------------Book Command---------------------------%
\makeatletter
\renewcommand\@pnumwidth{1cm}
\newcounter{book}
\renewcommand\thebook{\@Roman\c@book}
\newcommand\book{%
    \if@openright
        \cleardoublepage
    \else
        \clearpage
    \fi
    \thispagestyle{plain}%
    \if@twocolumn
        \onecolumn
        \@tempswatrue
    \else
        \@tempswafalse
    \fi
    \null\vfil
    \secdef\@book\@sbook
}
\def\@book[#1]#2{%
    \refstepcounter{book}
    \addcontentsline{toc}{book}{\bookname\ \thebook:\hspace{1em}#1}
    \markboth{}{}
    {\centering
     \interlinepenalty\@M
     \normalfont
     \huge\bfseries\bookname\nobreakspace\thebook
     \par
     \vskip 20\p@
     \Huge\bfseries#2\par}%
    \@endbook}
\def\@sbook#1{%
    {\centering
     \interlinepenalty \@M
     \normalfont
     \Huge\bfseries#1\par}%
    \@endbook}
\def\@endbook{
    \vfil\newpage
        \if@twoside
            \if@openright
                \null
                \thispagestyle{empty}%
                \newpage
            \fi
        \fi
        \if@tempswa
            \twocolumn
        \fi
}
\newcommand*\l@book[2]{%
    \ifnum\c@tocdepth >-3\relax
        \addpenalty{-\@highpenalty}%
        \addvspace{2.25em\@plus\p@}%
        \setlength\@tempdima{3em}%
        \begingroup
            \parindent\z@\rightskip\@pnumwidth
            \parfillskip -\@pnumwidth
            {
                \leavevmode
                \Large\bfseries#1\hfill\hb@xt@\@pnumwidth{\hss#2}
            }
            \par
            \nobreak
            \global\@nobreaktrue
            \everypar{\global\@nobreakfalse\everypar{}}%
        \endgroup
    \fi}
\newcommand\bookname{Book}
\renewcommand{\thebook}{\texorpdfstring{\Numberstring{book}}{book}}
\providecommand*{\toclevel@book}{-2}
\makeatother
\titleformat{\part}[display]
    {\Large\bfseries}
    {\partname\nobreakspace\thepart}
    {0mm}
    {\Huge\bfseries}
\titlecontents{part}[0pt]
    {\large\bfseries}
    {\partname\ \thecontentslabel: \quad}
    {}
    {\hfill\contentspage}
\titlecontents{chapter}[0pt]
    {\bfseries}
    {\chaptername\ \thecontentslabel:\quad}
    {}
    {\hfill\contentspage}
\newglossarystyle{longpara}{%
    \setglossarystyle{long}%
    \renewenvironment{theglossary}{%
        \begin{longtable}[l]{{p{0.25\hsize}p{0.65\hsize}}}
    }{\end{longtable}}%
    \renewcommand{\glossentry}[2]{%
        \glstarget{##1}{\glossentryname{##1}}%
        &\glossentrydesc{##1}{~##2.}
        \tabularnewline%
        \tabularnewline
    }%
}
\newglossary[not-glg]{notation}{not-gls}{not-glo}{Notation}
\newcommand*{\newnotation}[4][]{%
    \newglossaryentry{#2}{type=notation, name={\textbf{#3}, },
                          text={#4}, description={#4},#1}%
}
%--------------------------LENGTHS------------------------------%
% Spacings for the Table of Contents.
\addtolength{\cftsecnumwidth}{1ex}
\addtolength{\cftsubsecindent}{1ex}
\addtolength{\cftsubsecnumwidth}{1ex}
\addtolength{\cftfignumwidth}{1ex}
\addtolength{\cfttabnumwidth}{1ex}

% Indent and paragraph spacing.
\setlength{\parindent}{0em}
\setlength{\parskip}{0em}
%--------------------------Main Document----------------------------%
\begin{document}
    \ifx\ifemagi\undefined
        \section*{Electromagnetism I}
        \setcounter{section}{2}
        \renewcommand\thesubfigure{%
            \arabic{section}.\arabic{figure}.\arabic{subfigure}%
        }
    \fi    
    \subsection{Homework II}
        Wangsness Chapter 1 - Problems: 11, 12, 13, 14, 15
        \begin{problem}[Wangsness 1-11]
            \label{problem:EMAG_1_Wangsness_1_11}
            Calculate the path integral of
            $\mathbf{A}=x^{2}\hat{\mathbf{x}}
            +y^{2}\hat{\mathbf{y}}+z^{2}\hat{\mathbf{z}}$
            along the path shown in figure
            \subref{fig:EMAG_1_path_of_integration_for_wangsness_1_11}
            by integrating over $y$.
        \end{problem}
        \begin{proof}[Solution]
            The \textit{path integral} of $\mathbf{A}$ along a path $C$ is:
            \begin{equation*}
                \int_{C}\mathbf{A}\cdot\boldsymbol{d\ell}
                =\int_{C}\mathbf{A}\cdot\big(dx\hat{\mathbf{x}}
                +dy\hat{\mathbf{y}}+dz\hat{\mathbf{z}}\big)
            \end{equation*}
            We have $\mathbf{A}=
            x^{2}\hat{\mathbf{x}}+y^{2}\hat{\mathbf{y}}+z^{2}\hat{\mathbf{z}}$.
            Using this, we obtain:
            \begin{equation*}
                \int_{C}\mathbf{A}\cdot\boldsymbol{d\ell}
                =\int_{C}\big(x^{2}\hat{\mathbf{x}}
                +y^{2}\hat{\mathbf{y}}+z^{2}\hat{\mathbf{z}}\big)
                \cdot\big(dx\hat{\mathbf{x}}
                +dy\hat{\mathbf{y}}+dz\hat{\mathbf{z}}\big)
                =\int_{c}\big(x^{2}dx+y^{2}dy\big)
            \end{equation*}
            Along the path of integration, we have $x=y^{2}$, and therefore $dx=2ydy$.
            Substituting this back in:
            \begin{align*}
                \int_{C}\mathbf{A}\cdot\boldsymbol{d\ell}
                &=\int_{C}\big(x^{2}dx+y^{2}dy\big)
                &
                &=\bigg[\frac{1}{3}y^{6}+\frac{1}{3}y^{3}\bigg]_{0}^{\sqrt{2}}\\
                &=\int_{0}^{\sqrt{2}}\big((y^{2})^{2}(2ydy)+y^{2}dy\big)
                &
                &=\frac{1}{3}\big((\sqrt{2})^{6}+(\sqrt{2})^{3}\big)\\
                &=\int_{0}^{\sqrt{2}}\big(2y^{5}+y^{2}\big)dy
                &
                &=\frac{2}{3}\big(4+\sqrt{2}\big)\\
            \end{align*}
        \end{proof}
        \begin{figure}[H]
            \centering
            \begin{subfigure}[b]{0.49\textwidth}
                \centering
                \begin{tikzpicture}[>=triangle 45]
                    \begin{axis}[width=\linewidth,axis lines=center,
                    axis line style={->},
                    xtick distance=1,xlabel = $x$,xmin=-0.1,xmax=2.2,
                    ytick distance=1,ylabel = $y$,ymin=-0.1,ymax=2.1,
                    ->-/.style={decoration={markings,mark=at position .55 with
                    {\arrow{>}}},postaction={decorate}}]
                        \addplot[->-,line width=0.2mm,
                        samples=25,domain=0:1.4141,draw=blue]({\x^2},{\x});
                        \draw[dashed] (axis cs:2,0) -- (axis cs:2,1.4141);
                        \draw[dashed] (axis cs:0,1.4141) -- (axis cs:2,1.4141);
                        \node at (axis cs:1,0.7) {$y^2=x$};
                    \end{axis}
                \end{tikzpicture}
                \caption{Path of Integration for Wangsness 1-11}
                \label{fig:EMAG_1_path_of_integration_for_wangsness_1_11}
            \end{subfigure}
            \begin{subfigure}[b]{0.49\textwidth}
                \centering
                \begin{tikzpicture}[line width=1pt,line cap = round,>={Stealth},
                every edge/.style={draw=black,very thick}]
                    \draw[->] (0,0,0) -- (3,0,0) node[right] {$y$};
                    \draw[->] (0,0,0) -- (0,3,0) node[above] {$z$};
                    \draw[->] (0,0,0) -- (0,0,4) node[below left] {$x$};
                    \shade[fill=gray!60!white,opacity=0.5,draw=black,thick]
                    (2,0) arc (0:90:2) {[x={(0,0,1.33)}]
                    arc (90:0:2)} {[y={(0,0,1.33)}] arc (90:0:2)};
                    \draw[->,thick,draw=blue] (0.9,0.65) -- node [left]
                    {$\hat{\mathbf{z}}$} (0.9,1.3);
                    \draw[->,thick,draw=red] (0.9,0.65) -- node [below right]
                    {$\hat{\mathbf{n}}$} (1.3,0.9);
                    \draw[fill=orange] 
                    (0.8,0.6) -- (0.9,0.6) -- (1,0.7) -- (0.9,0.7) --cycle;
                    \node at (0.73,0.67) [below] {$da$};
                \end{tikzpicture}
                \caption{Geometry for Wangsness 1-12}
                \label{fig:EMAG_1_geometry_for_wangsness_1_12}
            \end{subfigure}
            \caption[Figures for Wangsness 1-11 and 1-12]{Figures for Problems
            \ref{problem:EMAG_1_Wangsness_1_11} and
            \ref{problem:EMAG_1_wangsness_1_12}, Respectively.}
            \label{fig:EMAG_1_figures_for_wangsness_1_11_and_1_12}
        \end{figure}
        \begin{problem}[Wangsness 1-12]
            \label{problem:EMAG_1_wangsness_1_12}
            Find the surface integral of $\mathbf{r}$ and the volume integral of
            $\nabla\cdot\mathbf{r}$ for a sphere of radius $a_{0}$ centered
            at the origin.
        \end{problem}
        \begin{proof}[Solution]
            The \textit{surface integral} of $\mathbf{A}$ over a closed surface
            $\partial\Sigma$ is defined as:
            \begin{equation*}
                \oiint_{\partial\Sigma}\mathbf{A}\cdot\boldsymbol{da}
                =\oiint_{\partial\Sigma}\mathbf{A}\cdot\hat{\boldsymbol{n}}da
            \end{equation*}
            Where $\hat{\mathbf{n}}$ is the unit normal to the surface
            $\partial\Sigma$.
            For a sphere, we have:
            \begin{equation*}
                \hat{\mathbf{n}}
                =\frac{\nabla(u)}{\norm{\nabla(u)}}
                =\frac{2x\hat{\mathbf{x}}+2y\hat{\mathbf{y}}+2z\hat{\mathbf{z}}}
                {\sqrt{4x^{2}+4y^{2}+4z^{2}}}
                = \frac{x\hat{\mathbf{x}}+\hat{\mathbf{y}}+z\hat{\mathbf{z}}}
                {\sqrt{x^{2}+y^{2}+z^{2}}}
            \end{equation*}
            Thus, we have:
            \begin{equation*}
                \oiint_{\partial\Sigma}\mathbf{r}\cdot\hat{\mathbf{n}}da
                =\oiint_{\partial\Sigma}\bigg(x\hat{\mathbf{x}}
                +y\hat{\mathbf{y}}+z\hat{\mathbf{z}}\bigg)\cdot
                \bigg(\frac{x\hat{\mathbf{x}}
                +y\hat{\mathbf{y}}+z\hat{\mathbf{z}}}{\sqrt{x^{2}+y^{2}+z^{2}}}\bigg)da
                =\oiint_{\partial\Sigma}\sqrt{x^{2}+y^{2}+z^{2}}da
            \end{equation*}
            But recall that $x^{2}+y^{2}+z^{2}=a_{0}^{2}$, so we have:
            \begin{equation*}
                \oiint_{\partial\Sigma}\mathbf{r}\cdot\boldsymbol{da}
                =a_{0}\oiint_{\partial\Sigma}da\\
            \end{equation*}
            But $\oiint_{\partial\Sigma}da$ is just the surface area of
            $\partial\Sigma$.
            And the surface area of the sphere is $4\pi a_{0}^{2}$. So:
            \begin{equation*}
                \oiint_{\partial\Sigma}\mathbf{r}\cdot \boldsymbol{da}=4\pi a_{0}^{3}
            \end{equation*}
            Using spherical coordinates is much easier.
            \begin{equation*}
                \oiint_{\partial\Sigma}\mathbf{r}\cdot\boldsymbol{da}
                =\int_{0}^{2\pi}\int_{0}^{\pi}a_{0}\hat{\mathbf{r}}\cdot
                \hat{\mathbf{r}}a_{0}^{2}\sin(\theta)d\theta d\varphi
                =\int_{0}^{2\pi}\int_{0}^{\pi}a_{0}^{3}\sin(\theta)d\theta d\varphi
                =2\pi a_{0}^{3}\int_{0}^{\pi}\sin(\theta)d\theta=4\pi a_{0}^{3}
            \end{equation*}
            To compute the \textit{volume integral} of $\nabla \cdot \mathbf{r}$ within
            $\Sigma$, we compute $\nabla\cdot \mathbf{r}$ and then integrate:
            \begin{align*}
                \nabla\cdot\mathbf{r}&=\frac{\partial x}{\partial x}
                +\frac{\partial y}{\partial y}+\frac{\partial z}{\partial z}=3\\
                \iiint_{\Sigma}\nabla\cdot\mathbf{r}d\tau&=\iiint_{\Sigma}3d\tau
                =3\iiint_{\Sigma}d\tau=3\frac{4}{3}\pi a_{0}^{3}=4\pi a_{0}^{3}
            \end{align*}
        \end{proof} 
        \begin{problem}[Wangsness 1-13]
            \label{problem:EMAG_1_wangsness_1_13}
            Given the vector field
            $\mathbf{A}=xy\hat{\mathbf{x}}+yz\hat{\mathbf{y}}+xz\hat{\mathbf{z}}$,
            evaluate the flux of $\mathbf{A}$ through a parallelepiped of sides $a,b,c$
            shown in figure \subref{fig:EMAG_1_wangsness_1_13_region_of_integration}.
            Compute $\int\nabla\cdot\mathbf{A}d\tau$ over the volume.
        \end{problem}
        \begin{proof}[Solution]
            There are six sides we must integrate over. Given $\mathbf{A}=xy\hat{\mathbf{x}}+yz\hat{\mathbf{y}}+xz\hat{\mathbf{z}}$,
            we have:
            \begin{align*}
                \oiint_{\partial\Sigma}\mathbf{A}\cdot\boldsymbol{da}
                &=\oiint_{\partial\Sigma}(xydydz+yzdxdz+xzdxdz)\\
                &=\underset{\textrm{Front}}{\iint}xydydz
                -\underset{\textrm{Back}}{\iint}xydydz
                +\underset{\textrm{Right}}{\iint}yzdxdz
                -\underset{\textrm{Left}}{\iint}yzdxdz
                +\underset{\textrm{Top}}{\iint}xzdxdy
                -\underset{\textrm{Bottom}}{\iint}xzdxdy\\
                &=\int_{0}^{c}\int_{0}^{b}(a)ydydz
                +\int_{0}^{c}\int_{0}^{a}(b)zdxdz
                +\int_{0}^{b}\int_{0}^{a}x(c)dxdy=\frac{abc}{2}(a+b+c)
            \end{align*}
            To compute $\iiint_{V}\nabla\cdot\mathbf{A}d\tau$, we have:
            $\nabla\cdot\mathbf{A}=
            \frac{\partial(xy)}{\partial x}
            +\frac{\partial(yz)}{\partial y}+\frac{\partial(xz)}{\partial z}=x+y+z$.
            Thus:
            \begin{align*}
                \iiint_{\Sigma}\nabla\cdot\mathbf{A}d\tau
                &=\iiint_{\Sigma}(x+y+z)d\tau=
                \int_{0}^{c}\int_{0}^{b}\int_{0}^{a}(x+y+z)dxdydz\\
                &=\int_{0}^{c}\int_{0}^{b}\int_{0}^{a}xdxdydz
                +\int_{0}^{c}\int_{0}^{b}\int_{0}^{a}ydxdydz
                +\int_{0}^{c}\int_{0}^{b}\int_{0}^{a}zdxdydz\\
                &=\frac{a^{2}bc}{2}+\frac{ab^{2}c}{2}
                +\frac{abc^{2}}{2}=\frac{abc}{2}(a+b+c)
            \end{align*}
        \end{proof}
        \begin{figure}[H]
            \centering
            \begin{subfigure}[b]{0.49\textwidth}
                \begin{tikzpicture}[line width=0.4pt,line cap = round,>={Stealth}]
                    \draw[->,semithick] (0,0) -- (4,0) node[right] {$y$};
                    \draw[->,semithick] (0,0) -- (0,3) node[above] {$z$};
                    \draw[->,semithick] (0,0) -- (-2,-2) node[below left] {$x$};
                    \draw[ball color=gray!10!white,opacity=0.6]
                    (-1.2,-1.2) -- (1.5,-1.2) -- (1.5,0.8) -- (-1.2,0.8) -- cycle;
                    \draw[ball color = gray!90!white,opacity=0.6]
                    (-1.2,0.8) -- (0,1.6) -- (2.7,1.6) -- (1.5,0.8) -- cycle;
                    \draw[fill=gray,opacity=0.6]
                    (1.5,0.8) -- (2.7,1.6) -- (2.7,0) -- (1.5,-1.2) -- cycle;
                    \filldraw[fill=black] (-1.2,-1.2) circle (0.04) node [below] {$a$};
                    \filldraw[fill=black]
                    (0,1.6) circle (0.04) node [above right] {$c$};
                    \filldraw[fill=black] (2.7,0) circle (0.04) node [below] {$b$};
                \end{tikzpicture}
            \caption{Wangsness 1-13}
            \label{fig:EMAG_1_wangsness_1_13_region_of_integration}
            \end{subfigure}
            \begin{subfigure}[b]{0.49\textwidth}
                \centering
                \begin{tikzpicture}[>=triangle 45,
                ->-/.style={decoration={markings,
                mark=at position .55 with {\arrow{>}}},postaction={decorate}}]
                    \begin{axis}[width=\textwidth,axis lines=center,
                    xlabel = $x$,xmin=-0.15,xmax=4,xtick distance=3,
                    ylabel = $y$,ymin=-0.15,ymax=5,ytick distance=4]
                        \draw[->-,line width=0.2mm,draw=blue]
                        (axis cs:0,0) -- (axis cs:3,0);
                        \draw[->-,line width=0.2mm,draw=blue]
                        (axis cs:3,0) -- (axis cs:3,4);
                        \draw[->-,line width=0.2mm,draw=blue]
                        (axis cs:3,4) -- (axis cs:0,4);
                        \draw[->-,line width=0.2mm,draw=blue]
                        (axis cs:0,4) -- (axis cs:0,0);
                    \end{axis}
                \end{tikzpicture}
                \caption{Wangsness 1-14}
                \label{fig:EMAG_1_wangsness_1_14}
            \end{subfigure}
            \caption[Figures for Wangsness 1-13 and 1-14]
            {Figures for problems
            \ref{problem:EMAG_1_wangsness_1_13} and
            \ref{problem:EMAG_1_wangsness_1_14}, Respectively.}
        \end{figure}
        \begin{problem}[Wangsness 1-14]
            \label{problem:EMAG_1_wangsness_1_14}
            Given $\mathbf{A}=-y\hat{\mathbf{x}}+x\hat{\mathbf{y}}$, calculate the line
            integral $\oint\mathbf{A}\cdot\mathbf{ds}$ over the closed path in the $xy$
            plane shown in figure \subref{fig:EMAG_1_wangsness_1_14}.
        \end{problem}
        \begin{proof}[Solution]
            Given $\mathbf{A}=-y\hat{\mathbf{x}}+x\hat{\mathbf{y}}$, we have:
            \begin{align*}
                \oint_{\partial S}\mathbf{A}\cdot\boldsymbol{d\ell}
                &=\oint_{\partial S}\big(-y\hat{\mathbf{x}}
                +x\hat{\mathbf{y}}\big)\cdot\big(dx\hat{\mathbf{x}}
                +dy\hat{\mathbf{y}}\big)\\
                &=\underbrace{\int_{0}^{3}(-ydx+xdy)}_{y=0,\ dy=0}
                +\underbrace{\int_{0}^{4}(-ydx+xdy)}_{x=3,\ dx=0}
                +\underbrace{\int_{3}^{0}(-ydx+xdy)}_{y =4,\ dy=0}
                +\underbrace{\int_{4}^{0}(-ydx+xdy)}_{x=0,\ dx=0}\\
                &=0+12+12+0=24
            \end{align*}
            Next, we compute $\iint(\nabla\times\mathbf{A})\cdot\boldsymbol{da}$. We
            have $\nabla\times\mathbf{A}=2\hat{\mathbf{z}}$. Thus:
            \begin{equation*}
                \iint_{S}\big(\nabla\times\mathbf{A}\big)\cdot\boldsymbol{da}
                =\iint_{S}\big(2\hat{\mathbf{z}}\big)\cdot \big(dydz\hat{\mathbf{x}}
                +dxdz\hat{\mathbf{y}}+dxdy\hat{\mathbf{z}}\big)
                =\int_{0}^{4}\int_{0}^{3}2dydx=24
            \end{equation*}
        \end{proof}
        \begin{problem}[Wangsness 1-15]
            \label{problem:EMAG_1_wangsness_1_15}
            Given $\mathbf{A}=x^{2}y\hat{\mathbf{x}}
            +xy^{2}\hat{\mathbf{y}}+a^{3}e^{-\beta y}\cos(\alpha x)\hat{\mathbf{z}}$,
            compute $\oint\mathbf{A}\cdot \boldsymbol{d\ell}$ along the path in figure
            \ref{fig:EMAG_1_wangsness_1_15}. Compute
            $\iint(\nabla\times\mathbf{A})\cdot\boldsymbol{da}$ over the same region.
        \end{problem}
        \begin{proof}[Solution]
            Along the entire contour, we have $z=0$ and $dz=0$. Thus, we have:
            \begin{equation*}
                \oint_{\partial S}\mathbf{A}\cdot\boldsymbol{d\ell}
                =\oint_{\partial S}\big(x^{2}y\hat{\mathbf{x}}+xy^{2}\hat{\mathbf{y}}
                +a^{3}e^{\beta y}\cos(\alpha x)\hat{\mathbf{z}}\big)
                \cdot\big(dx\hat{\mathbf{x}}+dy\hat{\mathbf{y}}\big)
                =\oint_{\partial S}\big(x^{2}ydx+xy^{2}dy\big)
            \end{equation*}
            Along the first path, we have $x=0$ and $dx=0$. Along the second path, we
            have $y=\sqrt{2k}, dy=0$. Along the third path we have $y^2=kx$, and thus
            $dx=\frac{2ydy}{k}$. So:
            \begin{align*}
                \oint_{\partial S}\mathbf{A}\cdot\boldsymbol{d\ell}
                &=\int_{C_{1}}\mathbf{A}\cdot\boldsymbol{d\ell}
                +\int_{C_{2}}\mathbf{A}\cdot\boldsymbol{d\ell}
                +\int_{C_{3}}\mathbf{A}\cdot\boldsymbol{d\ell}\\
                &=\int_{0}^{\sqrt{2k}}(0)y^{2}dy+\int_{0}^{2}x^{2}\sqrt{2k}dx
                +\int_{\sqrt{2k}}^{0}\big(\frac{y^{4}}{k^2}y\frac{2y}{k}dy
                +\frac{y^{2}}{k}y^{2}dy\big)\\
                &=\frac{8}{3}\sqrt{2k}+\int_{\sqrt{2k}}^{0}\big(2\frac{y^{6}}{k^{3}}
                +\frac{y^{4}}{k}\big)dy
                =\frac{8}{3}\sqrt{2k}-\frac{16}{7}\sqrt{2k}-\frac{4k\sqrt{2k}}{5}
                =\sqrt{2k}\big(\frac{8}{21}-\frac{4}{5}k\big)
            \end{align*}
            Next we compute $\iint(\nabla\times\mathbf{A})\cdot\boldsymbol{da}$. Note
            that $\boldsymbol{da}=\hat{\mathbf{z}}dxdy$. The $\hat{\mathbf{z}}$
            component for $\nabla\times\mathbf{A}$ is $(y^{2}-x^{2})\hat{\mathbf{z}}$.
            We have:
            \begin{align*}
                \iint_{\Sigma}\big(\nabla\times\mathbf{A}\big)\cdot\boldsymbol{da}
                &=\int_{0}^{2}\int_{\sqrt{kx}}^{\sqrt{2k}} \big(x^{2}-y^{2}\big)dydx
                =\int_{0}^{2}\bigg[x^{2}y
                -\frac{y^{3}}{3}\bigg]_{\sqrt{kx}}^{\sqrt{2k}}dx\\
                &=\int_{0}^{2}\bigg[\bigg(x^{2}\sqrt{2k}-\frac{2k\sqrt{2k}}{3}\bigg)
                -\bigg(x^{2}\sqrt{kx}-\frac{kx\sqrt{kx}}{3}\bigg)\bigg]dx\\
                &=\sqrt{2k}\int_{0}^{2}x^{2}dx-\frac{2k}{3}\sqrt{2k}
                \int_{0}^{2}dx-\sqrt{k}\int_{0}^{2}x^{\frac{5}{2}}dx
                +\frac{k\sqrt{k}}{3}\int_{0}^{2}x^{\frac{3}{2}}dx\\
                &=\frac{8}{3}\sqrt{2k}-\frac{4k}{3}\sqrt{2k}-\frac{16}{7}\sqrt{2k}
                +\frac{8k}{15}\sqrt{2k}=\frac{8}{21}\sqrt{2k}-\frac{4}{5}k\sqrt{2k}
                =\sqrt{2k}\big(\frac{8}{21}-\frac{4k}{5}\big)
            \end{align*}
        \end{proof}
        \begin{figure}[H]
            \centering
            \begin{tikzpicture}[>=triangle 45,
            ->-/.style={decoration={markings,
            mark=at position .55 with {\arrow{>}}},postaction={decorate}}]
                \begin{axis}[width=0.4\textwidth,axis lines=center,
                axis lines=middle,
                xlabel = $x$,xmin=-0.1,xmax=2.3,xtick distance=1,
                ylabel = $y$,ymin=-0.1,ymax=2.5,ytick distance=2,
                yticklabel={$\sqrt{\pgfmathprintnumber{\tick}k}$}]
                    \draw[->-,line width=0.2mm,draw=blue]
                    (axis cs:0,0) -- (axis cs:0,2);
                    \draw[->-,line width=0.2mm,draw=blue]
                    (axis cs:0,2) -- (axis cs:2,2);
                    \addplot[->-,line width=0.2mm,
                    samples=25,domain=1.414214:0,draw=blue] (x*x,1.4142*x);
                    \node at (axis cs:1,0.75) {$y^2=kx$};
                \end{axis}
            \end{tikzpicture}
            \caption[Figure for Wangsness 1-15]
            {Figure for problem \ref{problem:EMAG_1_wangsness_1_15}}
            \label{fig:EMAG_1_wangsness_1_15}
        \end{figure}
\end{document}