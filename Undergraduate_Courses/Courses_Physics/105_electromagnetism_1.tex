\documentclass[crop=false,class=book,oneside]{standalone}
%----------------------------Preamble-------------------------------%
%---------------------------Packages----------------------------%
\usepackage{geometry}
\geometry{b5paper, margin=1.0in}
\usepackage[T1]{fontenc}
\usepackage{graphicx, float}            % Graphics/Images.
\usepackage{natbib}                     % For bibliographies.
\bibliographystyle{agsm}                % Bibliography style.
\usepackage[french, english]{babel}     % Language typesetting.
\usepackage[dvipsnames]{xcolor}         % Color names.
\usepackage{listings, lstlinebgrd}      % Verbatim-Like Tools.
\usepackage{mathtools, esint, mathrsfs} % amsmath and integrals.
\usepackage{amsthm, amsfonts}           % Fonts and theorems.
\usepackage{tabularx}
\usepackage{tcolorbox}                  % Frames around theorems.
\usepackage{upgreek}                    % Non-Italic Greek.
\usepackage{paracol}                    % Two-column styling.
\usepackage{wrapfig}                    % Wrap text around figure.
\usepackage{fmtcount, etoolbox}         % For the \book{} command.
\usepackage[newparttoc]{titlesec}       % Formatting chapter, etc.
\usepackage{titletoc}                   % Allows \book in toc.
\usepackage[nottoc]{tocbibind}          % Bibliography in toc.
\usepackage[titles]{tocloft}            % ToC formatting.
\usepackage{multicol, enumitem}         % Multi-column/enumerate.
\usepackage{import}                     % Import external files.
\usepackage{pgfplots, tikz}             % Drawing/graphing tools.
\usetikzlibrary{
    calc,                   % Calculating right angles and more.
    angles,                 % Drawing angles within triangles.
    arrows.meta,            % Latex and Stealth arrows.
    quotes,                 % Adding labels to angles.
    positioning,            % Relative positioning of nodes.
    decorations.markings,   % Adding arrows in the middle of a line.
    patterns,
    arrows,
    shapes,
    shapes.geometric,
    cd,
    hobby,
    babel
}                                       % Libraries for tikz.
\pgfplotsset{compat=1.9}                % Version of pgfplots.
\usepackage[font=scriptsize,
            labelformat=simple,
            labelsep=colon]{subcaption} % Subfigure captions.
\usepackage[font={scriptsize},
            hypcap=true,
            labelsep=colon]{caption}    % Figure captions.
\usepackage{hyperref}                   % Allows for hyperlinks.
\hypersetup{
    colorlinks=true,
    linkcolor=blue,
    filecolor=magenta,
    urlcolor=Cerulean,
    citecolor=SkyBlue
}                           % Colors for hyperref.
\usepackage[toc,acronym,nogroupskip]{glossaries} % Glossaries and acronyms.
\usepackage[subpreambles=false]{standalone}      % Complileable sub files.

% Various font stuff from kiwi.
% Use this for Times text and Computer Modern math
%\usepackage{times}

% Quite nice
%\usepackage[charter, greekfamily=, greekuppercase=italicized]{mathdesign}
%\usepackage[utopia, greekuppercase=italicized]{mathdesign}    % Math is narrower

% Use this for Times text and math
%\usepackage{newtxtext}
%\usepackage[libertine,cmintegrals]{newtxmath}
%\usepackage{fix-cm}

%\usepackage{txfontsb}
% or
%\usepackage{mathptmx}

%\usepackage[scaled=0.92]{helvet}
%\renewcommand{\rmdefault}{ptm}

%\usepackage{mathpazo}    % add possibly `sc` and `osf` options
%\usepackage{eulervm}

%\usepackage{fourier}
%\renewcommand{\rmdefault}{ptm}
%\usepackage{mathptm}

%\usepackage{fontspec}
%\setmainfont{lmodern}

%\usepackage[varg]{txfonts}
%\usepackage{fouriernc}
%\usepackage{mathpazo}

%\usepackage{bookman}
%\usepackage[scaled]{uarial}
%\usepackage[scaled]{helvet}
%\renewcommand*\familydefault{\sfdefault}
%\usepackage[math]{anttor}

%\newcommand\fgeorgia{\fontfamily{jvn}\selectfont}
%\newcommand\ftimes{\fontfamily{ptm}\selectfont}
%\newcommand\fhelvetica{\fontfamily{phv}\selectfont}
%\newcommand\fcourier{\fontfamily{pcr}\selectfont}
%\newcommand\fbookman{\fontfamily{pbk}\selectfont}
%\newcommand\fnewcentury{\fontfamily{pnc}\selectfont}
%\newcommand\fpalatino{\fontfamily{ppl}\selectfont}
%\newcommand\favantgarde{\fontfamily{pag}\selectfont}
%\newcommand\fnormal{\normalfont}
%\newcommand\fsize[1]{\ifnum#1>0\fontsize{#1}{#1}\selectfont\else\normalsize\fi}
%------------------------Theorem Styles-------------------------%
% Define theorem style for default spacing and normal font.
\newtheoremstyle{normal}
    {\topsep}               % Amount of space above the theorem.
    {\topsep}               % Amount of space below the theorem.
    {}                      % Font used for body of theorem.
    {}                      % Measure of space to indent.
    {\bfseries}             % Font of the header of the theorem.
    {}                      % Punctuation between head and body.
    {.5em}                  % Space after theorem head.
    {}

% Define theorem style for default spacing with italicized font.
\newtheoremstyle{normalit}{\topsep}{\topsep}
                {\itshape}{}{\bfseries}{}{.5em}{}

% Italic header environment.
\newtheoremstyle{thmit}{\topsep}{\topsep}{}{}{\itshape}{}{0.5em}{}

% Define italicized environments.
\theoremstyle{normalit}
\newtheorem{theorem}{Theorem}[section]
\newtheorem{lemma}{Lemma}[section]
\newtheorem{corollary}{Corollary}[section]
\newtheorem{proposition}{Proposition}[section]
\newtheorem*{theorem*}{Theorem}

% Define environments with italic headers.
\theoremstyle{thmit}
\newtheorem*{solution}{Solution}
\newtheorem*{fsolution}{Solution}

% Define default environments.
\theoremstyle{normal}
\newtheorem{example}{Example}[section]
\newtheorem{definition}{Definition}[section]
\newtheorem{problem}{Problem}[section]
\newtheorem{question}{Question}[section]
\newtheorem{remark}{Remark}[section]
\newtheorem{properties}{Properties}[section]
\newtheorem{notation}{Notation}[section]
\newtheorem{axiom}{Axiom}[section]
\newtheorem*{properties*}{Properties}
\newtheorem*{remark*}{Remark}
\newtheorem*{definition*}{Definition}
\theoremstyle{plain}

% Define framed environment.
\tcbuselibrary{most}
\newtcbtheorem[use counter*=theorem]{ftheorem}{Theorem}%
    {colback=green!5,colframe=green!35!black,
     fonttitle=\bfseries\upshape}{th}

\newtcbtheorem[use counter*=example]{fdefinition}{Definition}%
    {fonttitle=\bfseries\upshape,
     colback=blue!5!white,colframe=blue!75!black}{def}

\newtcbtheorem[use counter*=example]{fexample}{Example}%
    {fonttitle=\bfseries\upshape,
     colback=red!5!white,colframe=red!75!black}{ex}

\newtcbtheorem[use counter*=notation]{fnotation}{Notation}%
    {fonttitle=\bfseries\upshape,
     colback=SeaGreen!5!white,colframe=SeaGreen!75!black}{ex}

\newtcbtheorem[use counter*=corollary]{fcorollary}{Corollary}%
    {fonttitle=\bfseries\upshape,
     colback=Orchid!5!white,colframe=Orchid!75!black}{ex}

\newenvironment{bproof}{\textit{Proof.}}{\hfill$\square$}
\tcolorboxenvironment{bproof}{blanker,breakable,left=5mm,
                             before skip=10pt,after skip=10pt,
                             borderline west={1mm}{0pt}{red}}
\tcolorboxenvironment{fsolution}
    {enhanced jigsaw,colframe=cyan,interior hidden,breakable}

%--------------------Declared Math Operators--------------------%
\DeclareMathOperator{\Refl}{Refl}           % Reflection operator.
\DeclareMathOperator{\Span}{Span}           % Span of a set of vectors.
\DeclareMathOperator{\Card}{Card}           % Cardinality of set.
\DeclareMathOperator{\Ord}{Ord}             % Ordinal of ordered set.
\DeclareMathOperator{\Tr}{Tr}               % Trace of matrix.
\DeclareMathOperator{\adjoint}{adj}         % Adjoint of matrix.
\DeclareMathOperator{\rk}{rk}               % Rank of operator.
\DeclareMathOperator{\nul}{nul}             % Null space of operator.
\DeclareMathOperator{\sgn}{sgn}             % Sign of a number.
\DeclareMathOperator{\multideg}{mutlideg}   % Multi-Degree (Graphs).
\DeclareMathOperator{\GCD}{GCD}             % Greatest common denominator.
\DeclareMathOperator{\LM}{LM}               % Leading monomial
\DeclareMathOperator{\LC}{LC}               % Leading coefficient.
\DeclareMathOperator{\LT}{LT}               % Leading term.
\DeclareMathOperator{\LCM}{LCM}             % Least common multiple.
\DeclareMathOperator{\Mon}{Mon}             % Monomial.
\DeclareMathOperator{\Spec}{Spec}           % Spectrum.
\DeclareMathOperator{\proj}{proj}           % Projection.
\DeclareMathOperator{\comp}{comp}           % Component.
\DeclareMathOperator{\sinc}{sinc}           % Sinc function.
\DeclareMathOperator{\Ima}{Im}              % Image of operator.
\DeclareMathOperator{\Prin}{Prin}           % Principal value.
\DeclareMathOperator{\Mod}{mod}             % Modulus.
%------------------------New Commands---------------------------%
\DeclarePairedDelimiter\norm{\lVert}{\rVert}
\DeclarePairedDelimiter\ceil{\lceil}{\rceil}
\DeclarePairedDelimiter\floor{\lfloor}{\rfloor}
\newcommand*\diff{\mathop{}\!\mathrm{d}}
\newcommand*\Diff[1]{\mathop{}\!\mathrm{d^#1}}
\renewcommand{\mod}{\ \Mod}
\renewcommand*{\glstextformat}[1]{\textcolor{RoyalBlue}{#1}}
\renewcommand{\glsnamefont}[1]{\textbf{#1}}
\renewcommand\labelitemii{$\circ$}
\renewcommand\thesubfigure{\arabic{chapter}.\arabic{figure}}
\renewcommand\thesubfigure{%
    \arabic{chapter}.\arabic{figure}.\arabic{subfigure}}
\addto\captionsenglish{\renewcommand{\figurename}{Fig.}}
%------------------------Book Command---------------------------%
\makeatletter
\renewcommand\@pnumwidth{1cm}
\newcounter{book}
\renewcommand\thebook{\@Roman\c@book}
\newcommand\book{%
    \if@openright
        \cleardoublepage
    \else
        \clearpage
    \fi
    \thispagestyle{plain}%
    \if@twocolumn
        \onecolumn
        \@tempswatrue
    \else
        \@tempswafalse
    \fi
    \null\vfil
    \secdef\@book\@sbook
}
\def\@book[#1]#2{%
    \ifnum \c@secnumdepth >-3\relax
        \refstepcounter{book}%
        \addcontentsline{toc}{book}{
            \bookname\ \thebook:\hspace{1em}#1
        }
    \else
        \addcontentsline{toc}{book}{#1}%
    \fi
    \markboth{}{}%
    {\centering
     \interlinepenalty \@M
     \normalfont
     \ifnum \c@secnumdepth >-2\relax
       \huge\bfseries \bookname\nobreakspace\thebook
       \par
       \vskip 20\p@
     \fi
     \Huge \bfseries #2\par}%
    \@endbook}
\def\@sbook#1{%
    {\centering
     \interlinepenalty \@M
     \normalfont
     \Huge \bfseries #1\par}%
    \@endbook}
\def\@endbook{
    \vfil\newpage
        \if@twoside
            \if@openright
                \null
                \thispagestyle{empty}%
                \newpage
            \fi
        \fi
        \if@tempswa
            \twocolumn
        \fi
}
\newcommand*\l@book[2]{%
    \ifnum \c@tocdepth >-2\relax
        \addpenalty{-\@highpenalty}%
        \addvspace{2.25em \@plus\p@}%
        \setlength\@tempdima{3em}%
        \begingroup
            \parindent \z@ \rightskip \@pnumwidth
            \parfillskip -\@pnumwidth
            {
                \leavevmode
                \Large \bfseries #1\hfil \hb@xt@\@pnumwidth{
                    \hss #2
                }
            }
            \par
            \nobreak
            \global\@nobreaktrue
            \everypar{\global\@nobreakfalse\everypar{}}%
        \endgroup
    \fi}
\newcommand\bookname{Book}
\renewcommand{\thebook}{\texorpdfstring{\Numberstring{book}}{book}}
\providecommand*{\toclevel@book}{-2}
\makeatother
\titlecontents{chapter}[0pt]
    {\bfseries}
    {\chaptername\ \thecontentslabel:\quad}
    {}
    {\hfill\contentspage}
\titleformat{\part}[display]
    {\Large\bfseries}
    {\partname\nobreakspace\thepart}
    {0mm}
    {\Huge\bfseries}
    \titlecontents{part}[0pt]
    {\large\bfseries}
    {\partname\ \thecontentslabel: \quad}
    {}
    {\hfill\contentspage}
\newcommand{\MarkRightAngle}[4][.3cm]
    {\coordinate (tempa) at ($(#3)!#1!(#2)$);
     \coordinate (tempb) at ($(#3)!#1!(#4)$);
     \coordinate (tempc) at ($(tempa)!0.5!(tempb)$);%midpoint
     \draw (tempa) -- ($(#3)!2!(tempc)$) -- (tempb);}
%--------------------------LENGTHS------------------------------%
% Spacings for the Table of Contents.
\addtolength{\cftsecnumwidth}{1ex}
\addtolength{\cftsubsecindent}{1ex}
\addtolength{\cftsubsecnumwidth}{1ex}
\addtolength{\cftfignumwidth}{1ex}
\addtolength{\cfttabnumwidth}{1ex}

% Spacing for multi-column and enumerate environments.
\setlength{\multicolsep}{6pt}
\setlist[enumerate]{itemsep=0pt,topsep=3pt}

% Indent and paragraph spacing.
\setlength{\parindent}{0em}
\setlength{\parskip}{0em}
%----------------------------GLOSSARY-------------------------------%
\makeglossaries
\loadglsentries{../../glossary}
\loadglsentries{../../acronym}
%--------------------------Main Document----------------------------%
\begin{document}
\ifx\ifmain\undefined
    \title{Electromagnetism I}
    \author{Ryan Maguire}
    \date{\vspace{-5ex}}
    \maketitle
    \tableofcontents
    \chapter*{Electromagnetism I}
    \markboth{}{ELECTROMAGNETISM I}
    \setcounter{chapter}{1}
\else
    \chapter{Electromagnetism I}
\fi
\section{Homework From UML PHYS.5530 (95.553)}
\subsection{Homework I}
Wangsness Chapter 1 - Problems: 2, 3, 4, 5, 8, 9
\begin{problem}[Wangsness 1-2]
    Given $\mathbf{A}=2\hat{\mathbf{x}}-3\hat{\mathbf{y}}-4\hat{\mathbf{z}}$
    and $\mathbf{B}=6\hat{\mathbf{x}}+5\hat{\mathbf{y}}+\hat{\mathbf{z}}$,
    find the magnitudes and angles made with the $x$, $y$, and $z$ axes for
    $\mathbf{A}+\mathbf{B}$ and $\mathbf{A}-\mathbf{B}$.
\end{problem}
\begin{proof}[Solution]
    First, we need to find $\mathbf{A}+\mathbf{B}$ and $\mathbf{A}-\mathbf{B}$:
    \begin{align*}
        \mathbf{A}+\mathbf{B}&=
            (2\hat{\mathbf{x}}-3\hat{\mathbf{y}}-4\hat{\mathbf{z}})
            +(6\hat{\mathbf{x}}+5\hat{\mathbf{y}}+\hat{\mathbf{z}})
        &
        \mathbf{A}-\mathbf{B}&=
            (2\hat{\mathbf{x}}-3\hat{\mathbf{y}}-4\hat{\mathbf{z}})
            -(6\hat{\mathbf{x}}+5\hat{\mathbf{y}}+\hat{\mathbf{z}})\\
        &=(2+6)\hat{\mathbf{x}}+(5-3)\hat{\mathbf{y}}+(1-4)\hat{\mathbf{z}}
        &
        &=(2-6)\hat{\mathbf{x}}-(3+5)\hat{\mathbf{y}}-(4+1)\hat{\mathbf{z}}\\
        &=8\hat{\mathbf{x}}+2\hat{\mathbf{y}}-3\hat{\mathbf{z}}
        &
        &=-4\hat{\mathbf{x}}-8\hat{\mathbf{y}}-5\hat{\mathbf{z}}
    \end{align*}
    The magnitude of a vector
    $\mathbf{A}=a_{1}\hat{\mathbf{x}}_{1}+\hdots+a_{N}\hat{\mathbf{x}}_{N}$,
    also called its \textit{norm}, is:
    \begin{equation*}
        \norm{\mathbf{A}}=\sqrt{\sum_{i=1}^{N}a_{i}^{2}}
    \end{equation*}
    Using this, we have:
    \begin{align*}
        \norm{\mathbf{A+B}}&=(8^{2}+2^{2}+3^{3})^{1/2}
        &
        \norm{\mathbf{A-B}}&=(4^{2}+8^{2}+5^{2})^{1/2}\\
        &=\sqrt{77}
        &
        &=\sqrt{105}
    \end{align*}
    The \textit{direction angle} between $\mathbf{A}$ and the $\xi$-axis is:
    \begin{equation*}
        \alpha_{\xi}
        =\cos^{-1}\bigg(\frac{\mathbf{A}\cdot\hat{\boldsymbol{\upxi}}}
            {\norm{\mathbf{A}}\norm{\hat{\boldsymbol{\upxi}}}}\bigg)
        =\cos^{-1}\bigg(\frac{\mathbf{A}\cdot\hat{\boldsymbol{\upxi}}}
            {\norm{\mathbf{A}}}\bigg)
    \end{equation*}
    The direction angles of $\mathbf{A}+\mathbf{B}$ and
    $\mathbf{A}-\mathbf{B}$ for $\hat{\mathbf{x}},\hat{\mathbf{y}}$,
    and $\hat{\mathbf{z}}$ are:
    \begin{align*}
        \alpha&=
        \cos^{-1}\bigg(\frac{(\mathbf{A}+\mathbf{B})\cdot\hat{\mathbf{x}}}
            {\norm{\mathbf{A+B}}}\bigg)
        &
        \beta&=
        \cos^{-1}\bigg(\frac{(\mathbf{A}+\mathbf{B})\cdot \hat{\mathbf{y}}}
            {\norm{\mathbf{A}+\mathbf{B}}}\bigg)
        &
        \gamma&=
        \cos^{-1}\bigg(\frac{(\mathbf{A}+\mathbf{B})\cdot \hat{\mathbf{z}}}
            {\norm{\mathbf{A}+\mathbf{B}}}\bigg)\\
        &=\cos^{-1}\bigg(\frac{8}{\sqrt{77}}\bigg)
        &
        &=\cos^{-1}\bigg(\frac{2}{\sqrt{77}}\bigg)&
        &=\cos^{-1}\bigg(\frac{-3}{\sqrt{77}}\bigg)\\
        &= 24.3^{\circ} & &= 76.8^{\circ} & &= 110^{\circ}
    \end{align*}
    For $\mathbf{A}-\mathbf{B}$:
    \begin{align*}
        \alpha
        &=\cos^{-1}\bigg(\frac{(\mathbf{A}-\mathbf{B})\cdot\hat{\mathbf{x}}}
            {\norm{\mathbf{A-B}}}\bigg)
        &
        \beta
        &=\cos^{-1}\bigg(\frac{(\mathbf{A}-\mathbf{B})\cdot\hat{\mathbf{y}}}
            {\norm{\mathbf{A}-\mathbf{B}}}
            \bigg)
        &
        \gamma
        &=\cos^{-1}\bigg(\frac{(\mathbf{A}-\mathbf{B})\cdot\hat{\mathbf{z}}}
            {\norm{\mathbf{A}-\mathbf{B}}}\bigg)\\
        &=
        \cos^{-1}\bigg(\frac{-4}{\sqrt{77}}\bigg)
        &
        &=\cos^{-1}\bigg(\frac{-8}{\sqrt{77}}\bigg)
        &
        &=\cos^{-1}\bigg(\frac{-5}{\sqrt{77}}\bigg)\\
        &=113^{\circ}
        &
        &=141.3^{\circ}
        &
        &=119.2^{\circ}
    \end{align*}
\end{proof}
\begin{problem}[Wangsness 1-3]
    Find the relative position vector $\mathbf{R}$ of $\mathbf{P}=(2,-2,3)$
    with respect to $\mathbf{P}'=(-3,1,4)$. What are the direction angles of
    $\mathbf{R}$?
\end{problem}
\begin{proof}[Solution]
    The relative position vector of $\mathbf{B}$ with respect
    to $\mathbf{A}$ is:
    \begin{equation*}
        \mathbf{R}_{\mathbf{A}\rightarrow\mathbf{B}}=\mathbf{B}-\mathbf{A}
    \end{equation*}
    Thus, we have:
    \begin{align*}
        \mathbf{R}&=\mathbf{P}-\mathbf{P}'\\
        &=(2\hat{\mathbf{x}}-2\hat{\mathbf{y}}+3\hat{\mathbf{z}})
            -(-3\hat{\mathbf{x}}+\hat{\mathbf{y}}+4\hat{\mathbf{z}})\\
        &=(2+3)\hat{\mathbf{x}}+(-2-1)\hat{\mathbf{y}}+(3-4)\hat{\mathbf{z}}\\
        &=5\hat{\mathbf{x}}-3\hat{\mathbf{y}}-\hat{\mathbf{z}}
    \end{align*}
    The direction angles for $\mathbf{R}$ are:
    \begin{align*}
        \alpha&=\cos^{-1}\bigg(\frac{\mathbf{R}\cdot\hat{\mathbf{x}}}
            {\norm{\mathbf{R}}}\bigg)
        &
        \beta&=\cos^{-1}\bigg(\frac{\mathbf{R}\cdot\hat{\mathbf{y}}}
            {\norm{\mathbf{R}}}\bigg)
        &
        \gamma&=\cos^{-1}\bigg(\frac{\mathbf{R}\cdot\hat{\mathbf{z}}}
            {\norm{\mathbf{R}}}\bigg)\\
        &=\cos^{-1}\bigg(\frac{5}{\sqrt{35}}\bigg)
        &
        &=\cos^{-1}\bigg(\frac{-3}{\sqrt{35}}\bigg)
        &
        &=\cos^{-1}\bigg(\frac{-1}{\sqrt{35}}\bigg)\\
        &=32.5^{\circ}
        &
        &=120^{\circ}
        &
        &=99.7^{\circ}
    \end{align*}
\end{proof}
\begin{problem}[Wangsness 1-4]
    Given $\mathbf{A}=\hat{\mathbf{x}}+2\hat{\mathbf{y}}+3\hat{\mathbf{z}}$
    and $\mathbf{B}=4\hat{\mathbf{x}}-5\hat{\mathbf{y}}+6\hat{\mathbf{z}}$,
    find the angle between them. Find the component of $\mathbf{A}$ in the
    direction of $\mathbf{B}$.
\end{problem}
\begin{proof}[Solution]
    The definition of the \textit{angle} between two vectors
    $\mathbf{A}$ and $\mathbf{B}$ is:
    \begin{equation*}
        \theta=\cos^{-1}\bigg(\frac{\mathbf{A}\cdot\mathbf{B}}
            {\norm{\mathbf{A}}\norm{\mathbf{B}}}\bigg)
    \end{equation*}
    We have that:
    \begin{align*}
        \mathbf{A}\cdot\mathbf{B}&=1\cdot4-2\cdot5+3\cdot6
        &
        \norm{\mathbf{A}}&=\sqrt{1^{2}+2^{2}+3^{2}}
        &
        \norm{\mathbf{B}}&=\sqrt{4^{2}+5^{2}+6^{2}}\\
        &=12
        &
        &=\sqrt{14}
        &
        &=\sqrt{77}\\
    \end{align*}
    Using this, we have:
    \begin{equation*}
        \theta=\cos^{-1}\bigg(\frac{12}{\sqrt{14}{\sqrt{77}}}\bigg)
        =68.6^{\circ}
    \end{equation*}
    The \textit{component} of $\mathbf{A}$ in the direction of
    $\mathbf{B}$ is defined as:
    \begin{equation*}
        \comp_{\mathbf{B}}(\mathbf{A})=\mathbf{A}\cdot\frac{\mathbf{B}}
            {\norm{\mathbf{B}}}
    \end{equation*}
    Using this, we have:
    \begin{equation*}
        \comp_{\mathbf{B}}(\mathbf{A})=\mathbf{A}\cdot\frac{\mathbf{B}}
            {\norm{\mathbf{B}}}
        =\frac{\mathbf{A}\cdot\mathbf{B}}{\norm{\mathbf{B}}}
        =\frac{12}{\sqrt{77}}\approx 1.37
    \end{equation*}
\end{proof}
\begin{problem}[Wangsness 1-5]
    Given $\mathbf{A}=2\hat{\mathbf{x}}+3\hat{\mathbf{y}}-4\hat{\mathbf{z}}$
    and $\mathbf{B}=-6\hat{\mathbf{x}}-4\hat{\mathbf{y}}+\hat{\mathbf{z}}$,
    find the component of $\mathbf{A}\times\mathbf{B}$ along the direction
    of $\mathbf{C}=\hat{\mathbf{x}}-\hat{\mathbf{y}}+\hat{\mathbf{z}}$.
\end{problem}
\begin{proof}[Solution]
    The \textit{cross product} of $\mathbf{A}$ with $\mathbf{B}$ is:
    \begin{equation*}
        \mathbf{A}\times\mathbf{B}
            =(A_{y}B_{z}-A_{z}B_{y})\hat{\mathbf{x}}
            +(A_{z}B_{x}-A_{x}B_{z})\hat{\mathbf{y}}
            +(A_{x}B_{y}-A_{y}B_{x})\hat{\mathbf{z}}
    \end{equation*}
    Note that $\mathbf{A}\times\mathbf{B}=-\mathbf{B}\times\mathbf{A}$.
    A way to remember this formula is using matrices:
    \begin{equation*}
        \mathbf{A}\times\mathbf{B}=\det\Bigg(
        \begin{bmatrix}
            \hat{\mathbf{x}}&\hat{\mathbf{y}}&\hat{\mathbf{z}}\\
            A_{x}&A_{y}&A_{z}\\
            B_{x}&B_{y}&B_{z}
        \end{bmatrix}\Bigg)
        =
        \begin{vmatrix}
            \hat{\mathbf{x}}&\hat{\mathbf{y}}&\hat{\mathbf{z}}\\
            A_{x}&A_{y}&A_{z}\\
            B_{x}&B_{y}&B_{z}
        \end{vmatrix}
    \end{equation*}
    We have:
    \begin{align*}
        \mathbf{A}\times\mathbf{B}
        &=(2\hat{\mathbf{x}}+3\hat{\mathbf{y}}-4\hat{\mathbf{z}})
        \times(-6\hat{\mathbf{x}}-4\hat{\mathbf{y}}+\hat{\mathbf{z}})\\
        &=(3-16)\hat{\mathbf{x}}
        +(24-2)\hat{\mathbf{y}}+(-8+18)\hat{\mathbf{z}}\\
        &=-13\hat{\mathbf{x}}+22\hat{\mathbf{y}}+10\hat{\mathbf{z}}
    \end{align*}
    The component along the direction of $\mathbf{C}$ is:
    \begin{align*}
        \comp_{\mathbf{C}}(\mathbf{A}\times\mathbf{B})
        &=(\mathbf{A}\times\mathbf{B})
        \cdot\frac{\mathbf{C}}{\norm{\mathbf{C}}}
        &
        &=\frac{-13-22+10}{\sqrt{3}}\\
        &=\frac{(-13\hat{\mathbf{x}}+22\hat{\mathbf{y}}+10\hat{\mathbf{z}})
        \cdot(\hat{\mathbf{x}}-\hat{\mathbf{y}}+\hat{\mathbf{z}})}
        {\sqrt{1^2+(-1)^2+1^2}}
        &
        &=-\frac{25}{\sqrt{3}}
    \end{align*}
\end{proof}
\begin{problem}[Wangsness 1-8]
    Given a family of hyperbolas in the $xy$ plane $u=xy$, find $\nabla(u)$.
    If $\mathbf{A}=3\hat{\mathbf{x}}+2\hat{\mathbf{y}}+4\hat{\mathbf{z}}$,
    find the component of $\mathbf{A}$ in the direction of $\nabla(u)$ at
    the point on the curve for which $u=3$ and $x=2$.
\end{problem}
\begin{proof}[Solution]
    $\nabla(u)$ is called the \textit{gradient} of $u$.
    In Cartesian coordinates this is defined as:
    \begin{equation*}
        \nabla(u)=\sum_{i=1}^{N}\frac{\partial u}
            {\partial x_{i}}\hat{\mathbf{x}}_{i}
    \end{equation*}
    Where $\frac{\partial u}{\partial x_{i}}$ is the partial derivative
    of $u$ with respect to the $i^{th}$ coordinate. We have $u=xy$, so:
    \begin{equation*}
        \nabla(u)=\frac{\partial(xy)}{\partial x}\hat{\mathbf{x}}
        +\frac{\partial(xy)}{\partial y}\hat{\mathbf{y}}
        =y\hat{\mathbf{x}}+x\hat{\mathbf{y}}
    \end{equation*}
    When $u=3$ and $x=2$, we have $y=\frac{3}{2}$.
    So the component of $\mathbf{A}$ along $\nabla(u)$ when $u=3$ and $x=2$
    is:
    \begin{align*}
        \comp_{\nabla(u)}(\mathbf{A})
        &=\mathbf{A}\cdot\frac{\nabla(u)}{\norm{\nabla(u)}}
        &
        &=\frac{9+8}{2\sqrt{\frac{9+16}{4}}}\\
        &=(3\hat{\mathbf{x}}+2\hat{\mathbf{y}}+4\hat{\mathbf{z}})\cdot
        \frac{\frac{3}{2}\hat{\mathbf{x}}+2\hat{\mathbf{y}}}
            {\sqrt{(\frac{3}{2})^{2}+2^{2}}}
        &
        &=\frac{17}{\sqrt{25}}\\
        &=\frac{\frac{9}{2}+4}{\sqrt{\frac{9}{4}+4}}& &=\frac{17}{5}\\
    \end{align*}
\end{proof}
\begin{problem}[Wangsness 1-9]
    An ellipsoid is define by
    $u=\frac{x^{2}}{a^{2}}+\frac{y^{2}}{b^{2}}+\frac{z^{2}}{c^{2}}$.
    Find the unit vector normal to the surface of each point of an
    ellipsoid.
\end{problem}
\begin{proof}[Solution]
    The vector normal to a surface $u$ is the gradient: $\nabla(u)$.
    The unit vector normal to a surface would then be
    $\frac{\nabla(u)}{\norm{\nabla(u)}}$. We have:
    \begin{equation*}
        \nabla(u)=\frac{\partial u}{\partial x}\hat{\mathbf{x}}
        +\frac{\partial u}{\partial y}\hat{\mathbf{y}}
        +\frac{\partial y}{\partial z}\hat{\mathbf{z}}
        =2\frac{x}{a^2}\hat{\mathbf{x}}
        +2\frac{y}{b^2}\hat{\mathbf{y}}+2\frac{z}{c^2}\hat{\mathbf{z}}
    \end{equation*}
    The norm of $\nabla(u)$ and the unit vector normal to the surface $u$
    are:
    \begin{align*}
        \norm{\nabla(u)}&=\sqrt{\big(\nabla_{x}(u)\big)^{2}
        +\big(\nabla_{y}(u)\big)^{2}
        +\big(\nabla_{z}(u)\big)^{2}}
        &\hat{\mathbf{n}}&=\frac{\nabla(u)}{\norm{\nabla(u)}}\\
        &=\sqrt{\frac{4x^{2}}{a^{4}}
        +\frac{4y^{2}}{b^{4}}+\frac{4z^{2}}{c^{4}}}
        &
        &=\frac{2\frac{x}{a^{2}}\hat{\mathbf{x}}
        +2\frac{y}{b^{2}}+2\frac{z}{c^{2}}}{2\sqrt{\frac{x^{2}}{a^{4}}
        +\frac{y^{2}}{b^{4}}+\frac{z^{2}}{c^{4}}}}\\
        &=2\sqrt{\frac{x^{2}}{a^{4}}
        +\frac{y^{2}}{b^{4}}
        +\frac{z^{2}}{c^{4}}}
        &
        &=\frac{\frac{x}{a^{2}}\hat{\mathbf{x}}+\frac{y}{b^{2}}
        +\frac{z}{c^{2}}}{\sqrt{\frac{x^{2}}{a^{4}}
        +\frac{y^{2}}{b^{4}}+\frac{z^{2}}{c^{4}}}}
    \end{align*}
\end{proof}
\subsection{Homework II}
Wangsness Chapter 1 - Problems: 11, 12, 13, 14, 15
\begin{problem}[Wangsness 1-11]
    \label{problem:EMAG_1_Wangsness_1_11}
    Calculate the path integral of
    $\mathbf{A}=x^{2}\hat{\mathbf{x}}
    +y^{2}\hat{\mathbf{y}}+z^{2}\hat{\mathbf{z}}$
    along the path shown in figure
    \subref{fig:EMAG_1_path_of_integration_for_wangsness_1_11}
    by integrating over $y$.
\end{problem}
\begin{proof}[Solution]
    The \textit{path integral} of $\mathbf{A}$ along a path $C$ is:
    \begin{equation*}
        \int_{C}\mathbf{A}\cdot\boldsymbol{d\ell}
        =\int_{C}\mathbf{A}\cdot\big(dx\hat{\mathbf{x}}
        +dy\hat{\mathbf{y}}+dz\hat{\mathbf{z}}\big)
    \end{equation*}
    We have $\mathbf{A}=
    x^{2}\hat{\mathbf{x}}+y^{2}\hat{\mathbf{y}}+z^{2}\hat{\mathbf{z}}$.
    Using this, we obtain:
    \begin{equation*}
        \int_{C}\mathbf{A}\cdot\boldsymbol{d\ell}
        =\int_{C}\big(x^{2}\hat{\mathbf{x}}
        +y^{2}\hat{\mathbf{y}}+z^{2}\hat{\mathbf{z}}\big)
        \cdot\big(dx\hat{\mathbf{x}}
        +dy\hat{\mathbf{y}}+dz\hat{\mathbf{z}}\big)
        =\int_{c}\big(x^{2}dx+y^{2}dy\big)
    \end{equation*}
    Along the path of integration, we have $x=y^{2}$, and therefore $dx=2ydy$.
    Substituting this back in:
    \begin{align*}
        \int_{C}\mathbf{A}\cdot\boldsymbol{d\ell}
        &=\int_{C}\big(x^{2}dx+y^{2}dy\big)
        &
        &=\bigg[\frac{1}{3}y^{6}+\frac{1}{3}y^{3}\bigg]_{0}^{\sqrt{2}}\\
        &=\int_{0}^{\sqrt{2}}\big((y^{2})^{2}(2ydy)+y^{2}dy\big)
        &
        &=\frac{1}{3}\big((\sqrt{2})^{6}+(\sqrt{2})^{3}\big)\\
        &=\int_{0}^{\sqrt{2}}\big(2y^{5}+y^{2}\big)dy
        &
        &=\frac{2}{3}\big(4+\sqrt{2}\big)\\
    \end{align*}
\end{proof}
\begin{figure}[H]
    \centering
    \begin{subfigure}[b]{0.49\textwidth}
        \centering
        \begin{tikzpicture}[>=triangle 45]
            \begin{axis}[width=\linewidth,axis lines=center,
            axis line style={->},
            xtick distance=1,xlabel = $x$,xmin=-0.1,xmax=2.2,
            ytick distance=1,ylabel = $y$,ymin=-0.1,ymax=2.1,
            ->-/.style={decoration={markings,mark=at position .55 with
            {\arrow{>}}},postaction={decorate}}]
                \addplot[->-,line width=0.2mm,
                samples=25,domain=0:1.4141,draw=blue]({\x^2},{\x});
                \draw[dashed] (axis cs:2,0) -- (axis cs:2,1.4141);
                \draw[dashed] (axis cs:0,1.4141) -- (axis cs:2,1.4141);
                \node at (axis cs:1,0.7) {$y^2=x$};
            \end{axis}
        \end{tikzpicture}
        \caption{Path of Integration for Wangsness 1-11}
        \label{fig:EMAG_1_path_of_integration_for_wangsness_1_11}
    \end{subfigure}
    \begin{subfigure}[b]{0.49\textwidth}
        \centering
        \begin{tikzpicture}[line width=1pt,line cap = round,>={Stealth},
        every edge/.style={draw=black,very thick}]
            \draw[->] (0,0,0) -- (3,0,0) node[right] {$y$};
            \draw[->] (0,0,0) -- (0,3,0) node[above] {$z$};
            \draw[->] (0,0,0) -- (0,0,4) node[below left] {$x$};
            \shade[fill=gray!60!white,opacity=0.5,draw=black,thick]
            (2,0) arc (0:90:2) {[x={(0,0,1.33)}]
            arc (90:0:2)} {[y={(0,0,1.33)}] arc (90:0:2)};
            \draw[->,thick,draw=blue] (0.9,0.65) -- node [left]
            {$\hat{\mathbf{z}}$} (0.9,1.3);
            \draw[->,thick,draw=red] (0.9,0.65) -- node [below right]
            {$\hat{\mathbf{n}}$} (1.3,0.9);
            \draw[fill=orange] 
            (0.8,0.6) -- (0.9,0.6) -- (1,0.7) -- (0.9,0.7) --cycle;
            \node at (0.73,0.67) [below] {$da$};
        \end{tikzpicture}
        \caption{Geometry for Wangsness 1-12}
        \label{fig:EMAG_1_geometry_for_wangsness_1_12}
    \end{subfigure}
    \caption[Figures for Wangsness 1-11 and 1-12]{Figures for Problems
    \ref{problem:EMAG_1_Wangsness_1_11} and
    \ref{problem:EMAG_1_wangsness_1_12}, Respectively.}
    \label{fig:EMAG_1_figures_for_wangsness_1_11_and_1_12}
\end{figure}
\begin{problem}[Wangsness 1-12]
    \label{problem:EMAG_1_wangsness_1_12}
    Find the surface integral of $\mathbf{r}$ and the volume integral of
    $\nabla\cdot\mathbf{r}$ for a sphere of radius $a_{0}$ centered
    at the origin.
\end{problem}
\begin{proof}[Solution]
    The \textit{surface integral} of $\mathbf{A}$ over a closed surface
    $\partial\Sigma$ is defined as:
    \begin{equation*}
        \oiint_{\partial\Sigma}\mathbf{A}\cdot\boldsymbol{da}
        =\oiint_{\partial\Sigma}\mathbf{A}\cdot\hat{\boldsymbol{n}}da
    \end{equation*}
    Where $\hat{\mathbf{n}}$ is the unit normal to the surface
    $\partial\Sigma$.
    For a sphere, we have:
    \begin{equation*}
        \hat{\mathbf{n}}
        =\frac{\nabla(u)}{\norm{\nabla(u)}}
        =\frac{2x\hat{\mathbf{x}}+2y\hat{\mathbf{y}}+2z\hat{\mathbf{z}}}
        {\sqrt{4x^{2}+4y^{2}+4z^{2}}}
        = \frac{x\hat{\mathbf{x}}+\hat{\mathbf{y}}+z\hat{\mathbf{z}}}
        {\sqrt{x^{2}+y^{2}+z^{2}}}
    \end{equation*}
    Thus, we have:
    \begin{equation*}
        \oiint_{\partial\Sigma}\mathbf{r}\cdot\hat{\mathbf{n}}da
        =\oiint_{\partial\Sigma}\bigg(x\hat{\mathbf{x}}
        +y\hat{\mathbf{y}}+z\hat{\mathbf{z}}\bigg)\cdot
        \bigg(\frac{x\hat{\mathbf{x}}
        +y\hat{\mathbf{y}}+z\hat{\mathbf{z}}}{\sqrt{x^{2}+y^{2}+z^{2}}}\bigg)da
        =\oiint_{\partial\Sigma}\sqrt{x^{2}+y^{2}+z^{2}}da
    \end{equation*}
    But recall that $x^{2}+y^{2}+z^{2}=a_{0}^{2}$, so we have:
    \begin{equation*}
        \oiint_{\partial\Sigma}\mathbf{r}\cdot\boldsymbol{da}
        =a_{0}\oiint_{\partial\Sigma}da\\
    \end{equation*}
    But $\oiint_{\partial\Sigma}da$ is just the surface area of
    $\partial\Sigma$.
    And the surface area of the sphere is $4\pi a_{0}^{2}$. So:
    \begin{equation*}
        \oiint_{\partial\Sigma}\mathbf{r}\cdot \boldsymbol{da}=4\pi a_{0}^{3}
    \end{equation*}
    Using spherical coordinates is much easier.
    \begin{equation*}
        \oiint_{\partial\Sigma}\mathbf{r}\cdot\boldsymbol{da}
        =\int_{0}^{2\pi}\int_{0}^{\pi}a_{0}\hat{\mathbf{r}}\cdot
        \hat{\mathbf{r}}a_{0}^{2}\sin(\theta)d\theta d\varphi
        =\int_{0}^{2\pi}\int_{0}^{\pi}a_{0}^{3}\sin(\theta)d\theta d\varphi
        =2\pi a_{0}^{3}\int_{0}^{\pi}\sin(\theta)d\theta=4\pi a_{0}^{3}
    \end{equation*}
    To compute the \textit{volume integral} of $\nabla \cdot \mathbf{r}$ within
    $\Sigma$, we compute $\nabla\cdot \mathbf{r}$ and then integrate:
    \begin{align*}
        \nabla\cdot\mathbf{r}&=\frac{\partial x}{\partial x}
        +\frac{\partial y}{\partial y}+\frac{\partial z}{\partial z}=3\\
        \iiint_{\Sigma}\nabla\cdot\mathbf{r}d\tau&=\iiint_{\Sigma}3d\tau
        =3\iiint_{\Sigma}d\tau=3\frac{4}{3}\pi a_{0}^{3}=4\pi a_{0}^{3}
    \end{align*}
\end{proof} 
\begin{problem}[Wangsness 1-13]
    \label{problem:EMAG_1_wangsness_1_13}
    Given the vector field
    $\mathbf{A}=xy\hat{\mathbf{x}}+yz\hat{\mathbf{y}}+xz\hat{\mathbf{z}}$,
    evaluate the flux of $\mathbf{A}$ through a parallelepiped of sides $a,b,c$
    shown in figure \subref{fig:EMAG_1_wangsness_1_13_region_of_integration}.
    Compute $\int\nabla\cdot\mathbf{A}d\tau$ over the volume.
\end{problem}
\begin{proof}[Solution]
    There are six sides we must integrate over. Given $\mathbf{A}=xy\hat{\mathbf{x}}+yz\hat{\mathbf{y}}+xz\hat{\mathbf{z}}$,
    we have:
    \begin{align*}
        \oiint_{\partial\Sigma}\mathbf{A}\cdot\boldsymbol{da}
        &=\oiint_{\partial\Sigma}(xydydz+yzdxdz+xzdxdz)\\
        &=\underset{\textrm{Front}}{\iint}xydydz
        -\underset{\textrm{Back}}{\iint}xydydz
        +\underset{\textrm{Right}}{\iint}yzdxdz
        -\underset{\textrm{Left}}{\iint}yzdxdz
        +\underset{\textrm{Top}}{\iint}xzdxdy
        -\underset{\textrm{Bottom}}{\iint}xzdxdy\\
        &=\int_{0}^{c}\int_{0}^{b}(a)ydydz
        +\int_{0}^{c}\int_{0}^{a}(b)zdxdz
        +\int_{0}^{b}\int_{0}^{a}x(c)dxdy=\frac{abc}{2}(a+b+c)
    \end{align*}
    To compute $\iiint_{V}\nabla\cdot\mathbf{A}d\tau$, we have:
    $\nabla\cdot\mathbf{A}=
    \frac{\partial(xy)}{\partial x}
    +\frac{\partial(yz)}{\partial y}+\frac{\partial(xz)}{\partial z}=x+y+z$.
    Thus:
    \begin{align*}
        \iiint_{\Sigma}\nabla\cdot\mathbf{A}d\tau
        &=\iiint_{\Sigma}(x+y+z)d\tau=
        \int_{0}^{c}\int_{0}^{b}\int_{0}^{a}(x+y+z)dxdydz\\
        &=\int_{0}^{c}\int_{0}^{b}\int_{0}^{a}xdxdydz
        +\int_{0}^{c}\int_{0}^{b}\int_{0}^{a}ydxdydz
        +\int_{0}^{c}\int_{0}^{b}\int_{0}^{a}zdxdydz\\
        &=\frac{a^{2}bc}{2}+\frac{ab^{2}c}{2}
        +\frac{abc^{2}}{2}=\frac{abc}{2}(a+b+c)
    \end{align*}
\end{proof}
\begin{figure}[H]
    \centering
    \begin{subfigure}[b]{0.49\textwidth}
        \begin{tikzpicture}[line width=0.4pt,line cap = round,>={Stealth}]
            \draw[->,semithick] (0,0) -- (4,0) node[right] {$y$};
            \draw[->,semithick] (0,0) -- (0,3) node[above] {$z$};
            \draw[->,semithick] (0,0) -- (-2,-2) node[below left] {$x$};
            \draw[ball color=gray!10!white,opacity=0.6]
            (-1.2,-1.2) -- (1.5,-1.2) -- (1.5,0.8) -- (-1.2,0.8) -- cycle;
            \draw[ball color = gray!90!white,opacity=0.6]
            (-1.2,0.8) -- (0,1.6) -- (2.7,1.6) -- (1.5,0.8) -- cycle;
            \draw[fill=gray,opacity=0.6]
            (1.5,0.8) -- (2.7,1.6) -- (2.7,0) -- (1.5,-1.2) -- cycle;
            \filldraw[fill=black] (-1.2,-1.2) circle (0.04) node [below] {$a$};
            \filldraw[fill=black]
            (0,1.6) circle (0.04) node [above right] {$c$};
            \filldraw[fill=black] (2.7,0) circle (0.04) node [below] {$b$};
        \end{tikzpicture}
    \caption{Wangsness 1-13}
    \label{fig:EMAG_1_wangsness_1_13_region_of_integration}
    \end{subfigure}
    \begin{subfigure}[b]{0.49\textwidth}
        \centering
        \begin{tikzpicture}[>=triangle 45,
        ->-/.style={decoration={markings,
        mark=at position .55 with {\arrow{>}}},postaction={decorate}}]
            \begin{axis}[width=\textwidth,axis lines=center,
            xlabel = $x$,xmin=-0.15,xmax=4,xtick distance=3,
            ylabel = $y$,ymin=-0.15,ymax=5,ytick distance=4]
                \draw[->-,line width=0.2mm,draw=blue]
                (axis cs:0,0) -- (axis cs:3,0);
                \draw[->-,line width=0.2mm,draw=blue]
                (axis cs:3,0) -- (axis cs:3,4);
                \draw[->-,line width=0.2mm,draw=blue]
                (axis cs:3,4) -- (axis cs:0,4);
                \draw[->-,line width=0.2mm,draw=blue]
                (axis cs:0,4) -- (axis cs:0,0);
            \end{axis}
        \end{tikzpicture}
        \caption{Wangsness 1-14}
        \label{fig:EMAG_1_wangsness_1_14}
    \end{subfigure}
    \caption[Figures for Wangsness 1-13 and 1-14]
    {Figures for problems
    \ref{problem:EMAG_1_wangsness_1_13} and
    \ref{problem:EMAG_1_wangsness_1_14}, Respectively.}
\end{figure}
\begin{problem}[Wangsness 1-14]
    \label{problem:EMAG_1_wangsness_1_14}
    Given $\mathbf{A}=-y\hat{\mathbf{x}}+x\hat{\mathbf{y}}$, calculate the line
    integral $\oint\mathbf{A}\cdot\mathbf{ds}$ over the closed path in the $xy$
    plane shown in figure \subref{fig:EMAG_1_wangsness_1_14}.
\end{problem}
\begin{proof}[Solution]
    Given $\mathbf{A}=-y\hat{\mathbf{x}}+x\hat{\mathbf{y}}$, we have:
    \begin{align*}
        \oint_{\partial S}\mathbf{A}\cdot\boldsymbol{d\ell}
        &=\oint_{\partial S}\big(-y\hat{\mathbf{x}}
        +x\hat{\mathbf{y}}\big)\cdot\big(dx\hat{\mathbf{x}}
        +dy\hat{\mathbf{y}}\big)\\
        &=\underbrace{\int_{0}^{3}(-ydx+xdy)}_{y=0,\ dy=0}
        +\underbrace{\int_{0}^{4}(-ydx+xdy)}_{x=3,\ dx=0}
        +\underbrace{\int_{3}^{0}(-ydx+xdy)}_{y =4,\ dy=0}
        +\underbrace{\int_{4}^{0}(-ydx+xdy)}_{x=0,\ dx=0}\\
        &=0+12+12+0=24
    \end{align*}
    Next, we compute $\iint(\nabla\times\mathbf{A})\cdot\boldsymbol{da}$. We
    have $\nabla\times\mathbf{A}=2\hat{\mathbf{z}}$. Thus:
    \begin{equation*}
        \iint_{S}\big(\nabla\times\mathbf{A}\big)\cdot\boldsymbol{da}
        =\iint_{S}\big(2\hat{\mathbf{z}}\big)\cdot \big(dydz\hat{\mathbf{x}}
        +dxdz\hat{\mathbf{y}}+dxdy\hat{\mathbf{z}}\big)
        =\int_{0}^{4}\int_{0}^{3}2dydx=24
    \end{equation*}
\end{proof}
\begin{problem}[Wangsness 1-15]
    \label{problem:EMAG_1_wangsness_1_15}
    Given $\mathbf{A}=x^{2}y\hat{\mathbf{x}}
    +xy^{2}\hat{\mathbf{y}}+a^{3}e^{-\beta y}\cos(\alpha x)\hat{\mathbf{z}}$,
    compute $\oint\mathbf{A}\cdot \boldsymbol{d\ell}$ along the path in figure
    \ref{fig:EMAG_1_wangsness_1_15}. Compute
    $\iint(\nabla\times\mathbf{A})\cdot\boldsymbol{da}$ over the same region.
\end{problem}
\begin{proof}[Solution]
    Along the entire contour, we have $z=0$ and $dz=0$. Thus, we have:
    \begin{equation*}
        \oint_{\partial S}\mathbf{A}\cdot\boldsymbol{d\ell}
        =\oint_{\partial S}\big(x^{2}y\hat{\mathbf{x}}+xy^{2}\hat{\mathbf{y}}
        +a^{3}e^{\beta y}\cos(\alpha x)\hat{\mathbf{z}}\big)
        \cdot\big(dx\hat{\mathbf{x}}+dy\hat{\mathbf{y}}\big)
        =\oint_{\partial S}\big(x^{2}ydx+xy^{2}dy\big)
    \end{equation*}
    Along the first path, we have $x=0$ and $dx=0$. Along the second path, we
    have $y=\sqrt{2k}, dy=0$. Along the third path we have $y^2=kx$, and thus
    $dx=\frac{2ydy}{k}$. So:
    \begin{align*}
        \oint_{\partial S}\mathbf{A}\cdot\boldsymbol{d\ell}
        &=\int_{C_{1}}\mathbf{A}\cdot\boldsymbol{d\ell}
        +\int_{C_{2}}\mathbf{A}\cdot\boldsymbol{d\ell}
        +\int_{C_{3}}\mathbf{A}\cdot\boldsymbol{d\ell}\\
        &=\int_{0}^{\sqrt{2k}}(0)y^{2}dy+\int_{0}^{2}x^{2}\sqrt{2k}dx
        +\int_{\sqrt{2k}}^{0}\big(\frac{y^{4}}{k^2}y\frac{2y}{k}dy
        +\frac{y^{2}}{k}y^{2}dy\big)\\
        &=\frac{8}{3}\sqrt{2k}+\int_{\sqrt{2k}}^{0}\big(2\frac{y^{6}}{k^{3}}
        +\frac{y^{4}}{k}\big)dy
        =\frac{8}{3}\sqrt{2k}-\frac{16}{7}\sqrt{2k}-\frac{4k\sqrt{2k}}{5}
        =\sqrt{2k}\big(\frac{8}{21}-\frac{4}{5}k\big)
    \end{align*}
    Next we compute $\iint(\nabla\times\mathbf{A})\cdot\boldsymbol{da}$. Note
    that $\boldsymbol{da}=\hat{\mathbf{z}}dxdy$. The $\hat{\mathbf{z}}$
    component for $\nabla\times\mathbf{A}$ is $(y^{2}-x^{2})\hat{\mathbf{z}}$.
    We have:
    \begin{align*}
        \iint_{\Sigma}\big(\nabla\times\mathbf{A}\big)\cdot\boldsymbol{da}
        &=\int_{0}^{2}\int_{\sqrt{kx}}^{\sqrt{2k}} \big(x^{2}-y^{2}\big)dydx
        =\int_{0}^{2}\bigg[x^{2}y
        -\frac{y^{3}}{3}\bigg]_{\sqrt{kx}}^{\sqrt{2k}}dx\\
        &=\int_{0}^{2}\bigg[\bigg(x^{2}\sqrt{2k}-\frac{2k\sqrt{2k}}{3}\bigg)
        -\bigg(x^{2}\sqrt{kx}-\frac{kx\sqrt{kx}}{3}\bigg)\bigg]dx\\
        &=\sqrt{2k}\int_{0}^{2}x^{2}dx-\frac{2k}{3}\sqrt{2k}
        \int_{0}^{2}dx-\sqrt{k}\int_{0}^{2}x^{\frac{5}{2}}dx
        +\frac{k\sqrt{k}}{3}\int_{0}^{2}x^{\frac{3}{2}}dx\\
        &=\frac{8}{3}\sqrt{2k}-\frac{4k}{3}\sqrt{2k}-\frac{16}{7}\sqrt{2k}
        +\frac{8k}{15}\sqrt{2k}=\frac{8}{21}\sqrt{2k}-\frac{4}{5}k\sqrt{2k}
        =\sqrt{2k}\big(\frac{8}{21}-\frac{4k}{5}\big)
    \end{align*}
\end{proof}
\begin{figure}[H]
    \centering
    \begin{tikzpicture}[>=triangle 45,
    ->-/.style={decoration={markings,
    mark=at position .55 with {\arrow{>}}},postaction={decorate}}]
        \begin{axis}[width=0.4\textwidth,axis lines=center,
        axis lines=middle,
        xlabel = $x$,xmin=-0.1,xmax=2.3,xtick distance=1,
        ylabel = $y$,ymin=-0.1,ymax=2.5,ytick distance=2,
        yticklabel={$\sqrt{\pgfmathprintnumber{\tick}k}$}]
            \draw[->-,line width=0.2mm,draw=blue]
            (axis cs:0,0) -- (axis cs:0,2);
            \draw[->-,line width=0.2mm,draw=blue]
            (axis cs:0,2) -- (axis cs:2,2);
            \addplot[->-,line width=0.2mm,
            samples=25,domain=1.414214:0,draw=blue] (x*x,1.4142*x);
            \node at (axis cs:1,0.75) {$y^2=kx$};
        \end{axis}
    \end{tikzpicture}
    \caption[Figure for Wangsness 1-15]
    {Figure for problem \ref{problem:EMAG_1_wangsness_1_15}}
    \label{fig:EMAG_1_wangsness_1_15}
\end{figure}
\subsection{Homework III}
Wangsness Chapter 1 - Problems: 19, 20, 21, 22, 23, 24, 26
\begin{problem}[Wangsness 1-19]
    Let $\mathbf{A}=a\hat{\boldsymbol{\uprho}}+b\hat{\boldsymbol{\upvarphi}}
    +c\hat{\mathbf{z}}$, where $a,b,c$ are constants. 
    Is $\mathbf{A}$ a constant vector? Find $\nabla\cdot\mathbf{A}$ and
    $\nabla\times\mathbf{A}$. Find the rectangular and spherical components of
    $\mathbf{A}$, expressing in terms of $x,y,z$ and $r,\theta,\varphi$,
    respectively.
\end{problem}
\begin{proof}[Solution]
    If $a$ or $b$ are non-zero, then $\mathbf{A}$ is not a constant, for
    $\hat{\boldsymbol{\uprho}}$ and $\hat{\boldsymbol{\upvarphi}}$ are
    non-constant functions of $\varphi$. To compute $\nabla\cdot\mathbf{A}$,
    we use $\nabla$ in cylindrical coordinates and do:
    \begin{equation*}
        \nabla\cdot\mathbf{A}=
        \frac{\partial A_{\rho}}{\partial \rho}+\frac{A_{\rho}}{\rho}
        +\frac{1}{\rho}\frac{\partial A_{\phi}}{\partial\varphi}
        +\frac{\partial A_{z}}{\partial z}
        =\frac{\partial a}{\partial\rho}+\frac{a}{\rho}+\frac{1}{\rho}
        \frac{\partial b}{\partial\varphi}
        +\frac{\partial A_{z}}{\partial z}=\frac{a}{\rho}
    \end{equation*}
    For $\nabla\times\mathbf{A}$, we again use cylindrical coordinates and do:
    \begin{align*}
        \nabla\times\mathbf{A}=
        \begin{vmatrix}
            \frac{1}{\rho}\hat{\boldsymbol{\uprho}}
            &\hat{\boldsymbol{\upvarphi}}&\frac{1}{\rho}\hat{\mathbf{z}}\\
            \frac{\partial}{\partial\rho}&\frac{\partial}{\partial\varphi}
            &\frac{\partial}{\partial z}\\
            A_{\rho}&\rho A_{\varphi}&A_{z}
        \end{vmatrix}
        &=\frac{\hat{\boldsymbol{\uprho}}}{\rho}
        \bigg(\frac{\partial A_{z}}{\partial \varphi}
        -\frac{\partial(\rho A_{\varphi})}{\partial z}\bigg)
        +\hat{\boldsymbol{\upvarphi}}
        \bigg(\frac{\partial A_{\rho}}{\partial z}
        -\frac{\partial A_{z}}{\partial\rho}\bigg)
        +\frac{\hat{\mathbf{z}}}{\rho}
        \bigg(\frac{\partial(\rho A_{\varphi})}{\partial\rho}
        -\frac{\partial A_{\rho}}{\partial\varphi}\bigg)\\
        &=\hat{\boldsymbol{\uprho}}
        \bigg(\frac{1}{\rho}\frac{\partial A_{z}}{\partial\varphi}
        -\frac{\partial A_{\varphi}}{\partial z}\bigg)
        +\hat{\boldsymbol{\upvarphi}}
        \bigg(\frac{\partial A_{\rho}}{\partial z}
        -\frac{\partial A_{z}}{\partial\rho}\bigg)
        +\hat{\mathbf{z}}
        \bigg(\frac{1}{\rho}\frac{\partial(\rho A_{\varphi})}{\partial\rho}
        -\frac{1}{\rho}\frac{\partial A_{\rho}}{\partial\varphi}\bigg)\\
        &=\hat{\boldsymbol{\uprho}}(0-0)
        +\hat{\boldsymbol{\upvarphi}}(0-0)
        +\hat{\mathbf{z}}(\frac{A_{\phi}}{\rho}+0-0)
        =\frac{b}{\rho}\hat{\mathbf{z}}
    \end{align*}
    Rectangular coordinates of $\mathbf{A}$:
    \begin{align*}
        \mathbf{A}
        &=a(\cos(\phi)\hat{\mathbf{x}}+\sin(\phi)\hat{\mathbf{y}})
        +b(-\sin(\phi)\hat{\mathbf{x}}
        +\cos(\phi)\hat{\mathbf{y}})+c\hat{\mathbf{z}}\\
        &=a\bigg(\frac{x\hat{\mathbf{x}}
        +y\hat{\mathbf{y}}}{\sqrt{x^{2}+y^{2}}}\bigg)
        +b\bigg(\frac{-y\hat{\mathbf{x}}
        +x\hat{\mathbf{y}}}{\sqrt{x^{2}+y^{2}}}\bigg)
        +c\hat{\mathbf{z}}
        =\frac{ax-by}{\sqrt{x^{2}+y^{2}}}\hat{\mathbf{x}}
        +\frac{ay+bx}{\sqrt{x^{2}+y^{2}}}\hat{\mathbf{y}}+c\hat{\mathbf{z}}
    \end{align*}
    For spherical coordinates:
    \begin{align*}
        \hat{\boldsymbol{\uprho}}&=\sin(\theta)\hat{\mathbf{r}}
        +\cos(\theta)\hat{\boldsymbol{\uptheta}}
        &\mathbf{A}&=a(\sin(\theta)\hat{\mathbf{r}}
        +\cos(\theta)\hat{\boldsymbol{\uptheta}})+b\hat{\boldsymbol{\upvarphi}}
        +c(\cos(\theta)\hat{\mathbf{r}}
        -\sin(\theta)\hat{\boldsymbol{\uptheta}})\\
        \hat{\mathbf{z}}&=\cos(\theta)\hat{\mathbf{r}}
        -\cos(\theta)\hat{\boldsymbol{\uptheta}}
        &&=(a\sin(\theta)+c\cos(\theta))\hat{\mathbf{r}}
        +b\hat{\boldsymbol{\upvarphi}}
        +(a\cos(\theta)-c\sin(\theta))\hat{\boldsymbol{\uptheta}}
    \end{align*}
\end{proof}
\begin{problem}[Wangsness 1-20]
    Let $\mathbf{A}=a\hat{\mathbf{r}}+b\hat{\boldsymbol{\uptheta}}+
    c\hat{\boldsymbol{\upvarphi}}$. Is $\mathbf{A}$ a constant vector? Find
    $\nabla\cdot \mathbf{A}$ and $\nabla\times \mathbf{A}$. Find the
    rectangular and cylindrical components of $\mathbf{A}$, expressing in
    terms of $x,y,z$ and $\rho,\varphi,z$, respectively.
\end{problem}
\begin{proof}[Solution]
    If $a$ or $b$ or $c$ are non-zero, then $\mathbf{A}$ is not a constant
    vector, for $\hat{\mathbf{r}}$, $\hat{\boldsymbol{\uptheta}}$, and
    $\hat{\boldsymbol{\upvarphi}}$ are non-constant functions of $r,\theta,\varphi$.
    To compute $\nabla\cdot\mathbf{A}$, we use spherical coordinates and do:
    \begin{equation*}
        \nabla\cdot\mathbf{A}
        =\frac{1}{r^{2}}\frac{\partial (r^{2}A_{r})}{\partial r}
        +\frac{1}{r\sin(\theta)}
        \frac{\partial(\sin(\theta)A_{\theta})}{\partial \theta}
        +\frac{1}{r\sin(\theta)}\frac{\partial A_{\varphi}}{\partial\varphi}
        =\frac{2a}{r}+\frac{b\cos(\theta)}{r\sin(\theta)}
    \end{equation*}
    For $\nabla\times\mathbf{A}$:
    \begin{align*}
        \nabla\times\mathbf{A}&=
        \begin{vmatrix}
            \frac{1}{r^{2}\sin(\theta)}\hat{\mathbf{r}}
            &\frac{1}{r\sin(\theta)}\hat{\boldsymbol{\uptheta}}
            &\frac{1}{r}\hat{\boldsymbol{\upvarphi}}\\
            \frac{\partial}{\partial r}
            &\frac{\partial}{\partial\theta}
            &\frac{\partial}{\partial\varphi}\\
            A_{r}&rA_{\theta}&r\sin(\theta)A_{\varphi}
        \end{vmatrix}\\
        &=\frac{\hat{\mathbf{r}}}{r\sin(\theta)}
        \bigg(\frac{\partial (\sin(\theta)A_{\varphi})}{\partial\theta}
        -\frac{\partial A_{\theta}}{\partial \varphi}\bigg)
        +\frac{\hat{\boldsymbol{\uptheta}}}{r}\bigg(\frac{1}{\sin(\theta)}
        \frac{\partial A_{r}}{\partial \varphi}
        -\frac{\partial (rA_{\varphi})}{\partial r}\bigg)
        +\frac{\hat{\boldsymbol{\upvarphi}}}{r}
        \bigg(\frac{\partial (rA_{\theta})}{\partial r}
        -\frac{\partial A_{r}}{\partial \theta}\bigg)\\
        &=\frac{\cos(\theta)}{r\sin(\theta)}\hat{\mathbf{r}}
        -\frac{c}{r}\hat{\boldsymbol{\uptheta}}
        +\frac{b}{r}\hat{\boldsymbol{\upvarphi}}
    \end{align*}
    In rectangular coordinates, we have:
    \begin{align*}
        \mathbf{A}=\hspace{0.5em}&a\big(\sin(\theta)\cos(\varphi)
        \hat{\mathbf{x}}+\sin(\theta)\sin(\varphi)
        \hat{\mathbf{y}}+\cos(\theta)\hat{\mathbf{z}}\big)+\\
        &b\big(\cos(\theta)\cos(\varphi)\hat{\mathbf{x}}+\cos(\theta)\sin(\varphi)
        \hat{\mathbf{y}}-\sin(\theta)\hat{\mathbf{z}}\big)+\\
        &c\big(-\sin(\varphi)\hat{\mathbf{x}}+\cos(\varphi)\hat{\mathbf{y}}\big)\\
        =\hspace{0.5em}&\frac{a}{\sqrt{x^{2}+y^{2}+z^{2}}}\bigg(x\hat{\mathbf{x}}+
        y\hat{\mathbf{y}}+z\hat{\mathbf{z}}\bigg)+\\
        &\frac{b}{\sqrt{x^{2}+y^{2}+z^{2}}}\bigg(\frac{xz}{\sqrt{x^{2}+y^{2}}}
        \hat{\mathbf{x}}+\frac{yz}{\sqrt{x^{2}+y^{2}}}\hat{\mathbf{y}}
        -\sqrt{x^{2}+y^{2}}\hat{\mathbf{z}}\bigg)+\\
        &-\frac{y}{\sqrt{x^{2}+y^{2}}}\hat{\mathbf{x}}+
        \frac{x}{\sqrt{x^{2}+y^{2}}}\hat{\mathbf{y}}\\
        =\hspace{0.5em}&\bigg(\frac{ax}{\sqrt{x^{2}+y^{2}+z^{2}}}
        +\frac{bxz}{\sqrt{x^{2}+y^{2}}\sqrt{x^{2}+y^{2}+z^{2}}}
        -\frac{y}{\sqrt{x^{2}+y^{2}}}\bigg)\hat{\mathbf{x}}\\
        &+\bigg(\frac{ay}{\sqrt{x^{2}+y^{2}+z^{2}}}
        +\frac{byz}{\sqrt{x^{2}+y^{2}}\sqrt{x^{2}+y^{2}+z^{2}}}+
        \frac{x}{\sqrt{x^{2}+y^{2}}}\bigg)\hat{\mathbf{y}}\\
        &+\frac{az}{\sqrt{x^{2}+y^{2}+z^{2}}}-
        \frac{b\sqrt{x^{2}+y^{2}}}{\sqrt{x^{2}+y^{2}+z^{2}}}\bigg)\hat{\mathbf{z}}
    \end{align*}
    We can then use this to convert to cylindrical, recalling that
    $r^{2}=\rho^{2}+z^{2}$:
    \begin{equation*}
        \mathbf{A}=\bigg(\frac{a\rho+bz}{\sqrt{\rho^{2}+z^{2}}}\bigg)
        \hat{\boldsymbol{\uprho}}+c\hat{\boldsymbol{\upvarphi}}+
        \bigg(\frac{az-b\rho}{\sqrt{\rho^{2}+z^{2}}}\bigg)\hat{\mathbf{z}}
    \end{equation*}
\end{proof}
\begin{problem}[Wangsness 1-21]
    Find $\nabla\cdot\mathbf{r}$ for the position vector $\mathbf{r}$ expressed in
    rectangular, cylindrical, and spherical coordinates.
\end{problem}
\begin{proof}[Solution]
    In rectangular coordinates we have
    $\mathbf{r}=x\hat{\mathbf{x}}+y\hat{\mathbf{y}}+z\hat{\mathbf{z}}$. So:
    \begin{equation*}
        \nabla\cdot\mathbf{r}=\frac{\partial x}{\partial x}
        +\frac{\partial y}{\partial y}+\frac{\partial z}{\partial z}=1+1+1=3
    \end{equation*}
    In cylindrical coordinates,
    $\mathbf{r}=\rho\hat{\boldsymbol{\uprho}}+ z\hat{\mathbf{z}}$, So:
    \begin{equation*}
        \nabla\cdot\mathbf{r}=\frac{1}{\rho}
        \frac{\partial}{\partial \rho}\big(\rho^2\big)
        +\frac{1}{\rho}\frac{\partial}{\partial\phi}\big(0\big)
        +\frac{\partial z}{\partial z}=2+0+1=3
    \end{equation*}
    In spherical coordinates we have:
    \begin{equation*}
        \nabla\cdot\mathbf{r}=\frac{1}{r^{2}}\frac{\partial}{\partial r}
        \big(r^{2}r\big)=3
    \end{equation*}
\end{proof}
\begin{problem}[Wangsness 1-22]
\label{problem:EMAG_wangsness_1_22}
    Let $\mathbf{A}=a\rho\hat{\boldsymbol{\uprho}}
    +b\hat{\boldsymbol{\upvarphi}}+cz\hat{\mathbf{z}}$, where $a,b,c$ are constants.
    Find $\oiint\mathbf{A}\cdot\boldsymbol{da}$ over the surface of a right
    circular cylinder of length $L$ and radius $\rho_{0}$ whose axis is along the
    positive $z$ axis and the origin is the center of the lower circular face
    (See Fig.~\subref{fig:EMAG_1_wangsness_1_22}). Find
    $\iiint\nabla\cdot \mathbf{A}d\tau$ over the volume of the cylinder.
\end{problem}
\begin{proof}[Solution]
    We have that:
    \begin{equation*}
        \oint\mathbf{A}\cdot\boldsymbol{da}=\int_{Top}\mathbf{A}\cdot\boldsymbol{da}
        +\int_{Cylinder}\mathbf{A}\cdot\boldsymbol{da}
        +\int_{Bottom}\mathbf{A}\cdot \boldsymbol{da}
    \end{equation*}
    On the cylindrical surface, $da = \rho_0 d\phi dz$, so the integral is:
    \begin{align*}
        \int_{0}^{L}\int_{0}^{2\pi}(a\rho_0\hat{\boldsymbol{\uprho}}
        +b\hat{\boldsymbol{\upvarphi}}+cz\hat{\mathbf{z}})\cdot\rho_{0}
        \hat{\boldsymbol{\uprho}}d\varphi dz
        &=\int_{0}^{L}\int_{0}^{2\pi}a\rho_0^{2}d\phi dz=a\rho_0^{2}(2\pi)L    
    \end{align*}
    For the top and bottom, $da=\pm\rho d\rho d\phi\hat{\mathbf{z}}$, respectively.
    On the bottom surface $z=0$ and thus the integral is zero. On the top we get:
    \begin{equation*}
        \int_{0}^{2\pi}\int_{0}^{\rho_0}cL\rho d\rho d\phi=\pi cL\rho_0^{2}
    \end{equation*}
    Thus:
    \begin{equation*}
        \oiint\mathbf{A}\cdot\boldsymbol{da}=\pi L\rho_0^{2}(2a+c)
    \end{equation*}
    Computing the divergence, we get $\nabla\cdot\mathbf{A}=2a+c$. Therefore:
    \begin{equation*}
        \iiint_C\nabla\cdot\mathbf{A}d\tau=(2a+c)V=\pi L\rho_0^{2}(2a+c)
    \end{equation*}
\end{proof}
\begin{figure}[H]
    \centering
    \begin{subfigure}[b]{0.49\textwidth}
        \centering
        \begin{tikzpicture}[>=triangle 45]
            \coordinate (x) at (0,0,1.6);
            \coordinate (y) at (2,0,0);
            \coordinate (z) at (0,5,0);
            \coordinate (xt) at (0,4,0.76);
            \coordinate (yt) at (1.5,4,0);
            \coordinate (zt) at (0,4,0);
            \coordinate (p) at (1.05,4,0.66);
            \draw[->] (0,0,0) -- (2,0,0) node[right] {$y$};
            \draw[->] (0,0,0) -- (0,5,0) node[above] {$z$};
            \draw[->] (0,0,0) -- (0,0,1.6) node[below left] {$x$};
            \draw[-] (0,4,0) -- (0,4,0.76) {};
            \draw[-] (0,4,0) -- (1.5,4,0) {};
            \draw[-,blue] (0,4,0) node [below=0.2cm,right,black]
            {\scriptsize{$\rho$}} -- (1.05,4,0.66) {};
            \pic [draw=black, -, "\scriptsize{${\varphi}$}", angle eccentricity=1.7,
            angle radius =0.5cm] {angle = p--zt--yt};
            \fill[top color=gray!50!black,bottom color=gray!10,middle color=gray,
            shading=axis,opacity=0.25] (0,0) circle (1.5cm and 0.3cm);
            \fill[left color=gray!50!black,right color=gray!50!black,
            middle color=gray!50,shading=axis,opacity=0.25] (1.5,0) -- (1.5,4) arc
            (360:180:1.5cm and 0.3cm) -- (-1.5,0) arc (180:360:1.5cm and 0.3cm);
            \fill[top color=gray!90!,bottom color=gray!2,middle color=gray!30,
            shading=axis,opacity=0.25] (0,4) circle (1.5cm and 0.3cm);
            \draw (-1.5,4) -- (-1.5,0) arc (180:360:1.5cm and 0.3cm) -- (1.5,4) ++
            (-1.5,0) circle (1.5cm and 0.3cm);
            \draw[densely dashed] (-1.5,0) arc (180:0:1.5cm and 0.3cm);
        \end{tikzpicture}
        \caption{Wangsness 1-22}
        \label{fig:EMAG_1_wangsness_1_22}
    \end{subfigure}
    \begin{subfigure}[b]{0.49\textwidth}
        \centering
        \begin{tikzpicture}[>=triangle 45,->-/.style={decoration={markings,
        mark=at position .55 with {\arrow{>}}},postaction={decorate}}]
        \begin{axis}[width=\textwidth,axis equal,axis lines=middle,
            xlabel=$x$,ylabel=$y$,
            xmin=-0.1,xmax=1.3,ymin=-0.1,ymax=1.3,
            axis lines=middle,
            xticklabel={$r_{0}$},yticklabel={$r_{0}$},
            xtick distance=1,ytick distance=1,
            ->-/.style={decoration={markings,
            mark=at position .55 with {\arrow{>}}},postaction={decorate}}]
            \draw[->-,line width=0.2mm,draw=blue] (axis cs:0,0) -- (axis cs:1,0);
            \draw[->-,line width=0.2mm,draw=blue] (axis cs:0,1) -- (axis cs:0,0);
            \addplot[->-,line width=0.2mm,samples=30,domain=0:1,draw=blue]
            ({(1-x^2)/(1+x^2)},{2*x/(1+x^2)});
            \node at (axis cs:1,0.9) {$x^2+y^2=r_{0}^{2}$};
            \coordinate (a) at (axis cs:1,0);
            \coordinate (b) at (axis cs:0.8,0.6);
            \coordinate (o) at (axis cs:0,0);
            \draw[draw=red,thick] (axis cs:0,0) -- node [above] {$r_{0}$}
            (axis cs:4/5,3/5);
            \pic[draw=black,-,"$\theta$", angle eccentricity=1.5,
            angle radius = 0.6cm] {angle = a--o--b};
        \end{axis}
    \end{tikzpicture}
        \caption{Wangsness 1-23}
        \label{fig:EMAG_1_wangsness_1_23}
    \end{subfigure}
    \caption[Figures for Wangsness 1-22 and 1-23]{Figures for problems \ref{problem:EMAG_wangsness_1_22} and \ref{problem:EMAG_wangsness_1_23}, respectively.}
\end{figure}
\begin{problem}[Wangsness 1-23]
    \label{problem:EMAG_wangsness_1_23}
    Let $\mathbf{A}=4\hat{\mathbf{r}}+3\hat{\boldsymbol{\uptheta}}
    -2\hat{\boldsymbol{\upvarphi}}$. Find the line integral around the closed path
    shown in Fig.~\subref{fig:EMAG_1_wangsness_1_23}. Find the surface integral of
    $\nabla \times \mathbf{A}$ over the enclosed area.
\end{problem}
\begin{proof}[Solution]
    We have:
    \begin{equation*}
        \oint_{\partial S}\mathbf{A}\cdot\boldsymbol{d\ell}=\sum_{i}
        \int_{\partial S_{i}}\mathbf{A}\cdot\boldsymbol{d\ell}
    \end{equation*}
    Along the first path $\varphi=0$, $\theta=\frac{\pi}{2}$, and
    $\boldsymbol{d\ell}=dr\hat{\mathbf{r}}$. The integral is then $4r_{0}$.
    \begin{equation*}
        \int_{\partial S_{1}}\mathbf{A}\cdot\boldsymbol{d\ell}
        =\int_{0}^{r_{0}}(4\hat{\mathbf{r}}+3\hat{\boldsymbol{\uptheta}}
        -2\hat{\boldsymbol{\upvarphi}})\cdot(\hat{\mathbf{r}}dr)
        =4\int_{0}^{r_{0}}dr=4r_{0}
    \end{equation*}
    Along the second path $r=r_{0}$, $\theta=\frac{\pi}{2}$, and 
    $\boldsymbol{d\ell}=r_{0} d\varphi\hat{\boldsymbol{\upvarphi}}$.
    \begin{equation*}
        \int_{\partial S_{2}}\mathbf{A}\cdot\hat{\boldsymbol{d\ell}}
        =\int_{0}^{\frac{\pi}{2}}(4\hat{\mathbf{r}}
        +3\hat{\boldsymbol{\uptheta}}-2\hat{\boldsymbol{\upvarphi}})
        \cdot(r_{0}d\varphi\hat{\boldsymbol{\upvarphi}})
        =-2\int_{0}^{\frac{\pi}{2}}r_{0}d\varphi=-\pi r_{0}
    \end{equation*}
    Along the final path, $\varphi=\frac{pi}{2}$, $\theta=\frac{\pi}{2}$, and
    $\boldsymbol{d\ell}=dr\hat{\mathbf{r}}$.
    \begin{equation*}
        \int_{\partial S_{3}}\mathbf{A}\cdot \boldsymbol{d\ell}
        =\int_{r_{0}}^{0}(4\hat{\mathbf{r}}+3\hat{\boldsymbol{\uptheta}}
        -2\hat{\boldsymbol{\upvarphi}})\cdot (\hat{\mathbf{r}}dr)=4\int_{r_{0}}^{0}dr
        =-4r_{0}
    \end{equation*}
    Therefore:
    \begin{equation*}
        \oint_{\partial S}\mathbf{A}\cdot\boldsymbol{d\ell}
        =\sum_{i}\int_{\partial S_{i}}\mathbf{A}\cdot\boldsymbol{d\ell}
        =4r_{0}-\pi r_{0}-4r_{0}=-\pi r_{0}
    \end{equation*}
    The curl is: $\nabla\times\mathbf{A}
    =\frac{-2\cot(\theta)}{r}\hat{\mathbf{r}}+\frac{2}{r}\hat{\boldsymbol{\uptheta}}
    +\frac{3}{r}\hat{\boldsymbol{\upvarphi}}$. For the plane,
    $\boldsymbol{da}=\hat{\mathbf{z}}rdrd\varphi$ $\theta=\frac{\pi}{2}$. So:
    \begin{align*}
        \iint\nabla\times\mathbf{A}\cdot\boldsymbol{da}&=
        \int_{0}^{\frac{\pi}{2}}\int_{0}^{r_{0}}
        \bigg(\frac{-2\cot(\theta)}{r}\hat{\mathbf{r}}
        +\frac{2}{r}\hat{\boldsymbol{\uptheta}}
        +\frac{3}{r}\hat{\boldsymbol{\upvarphi}}\bigg)
        \cdot\hat{\mathbf{z}} rdrd\varphi\\
        &=\int_{0}^{\frac{\pi}{2}}\int_{0}^{r_{0}}
        \bigg(-2\cot(\theta)\cos(\theta)-2\sin(\theta)\bigg)drd\varphi
        =-2\int_{0}^{\frac{\pi}{2}}\int_{0}^{r_{0}}drd\varphi=-\pi r_{0}
    \end{align*}
\end{proof}
\begin{problem}[Wangsness 1-24]
    Verify that
    $\nabla\times (u\mathbf{A})=\nabla(u)\times\mathbf{A}+u(\nabla\times \mathbf{A})$
\end{problem}
\begin{proof}[Solution]
    Let $\mathbf{A}=\langle A_{x},A_{y},A_{z}\rangle$ and $u=u(x,y,z)$.
    Using the product rule, we get:
    \begin{align*}
        \nabla\times (u\mathbf{A})&=
        \begin{vmatrix}
            \hat{\mathbf{x}}&\hat{\mathbf{y}}&\hat{\mathbf{z}}\\
            \frac{\partial}{\partial x}&\frac{\partial}{\partial y}&
            \frac{\partial}{\partial z}\\uA_{x}&uA_{y}&uA_{z}
        \end{vmatrix}\\
        &=\hat{\mathbf{x}}\big(\frac{\partial u}{\partial y}A_{z}+u
        \frac{\partial A_z}{\partial y}-\frac{\partial u}{\partial z}A_{y}-u
        \frac{\partial A_y}{\partial z}\big)+\hat{\mathbf{y}}
        \big(\frac{\partial u}{\partial z}A_{x}+u\frac{\partial A_x}{\partial z}-
        \frac{\partial u}{\partial x}A_{z}
        -u\frac{\partial A_{z}}{\partial x}\big)+\\
        &\hspace{1.4em}\hat{\mathbf{z}}\big(\frac{\partial u}{\partial x}A_{y}+u
        \frac{\partial A_y}{\partial x}-\frac{\partial u}{\partial y}A_{x}-
        u\frac{\partial A_{x}}{\partial y}\big)
    \end{align*}
    But:
    \begin{equation*}
        \nabla(u)\times\mathbf{A}=\hat{\mathbf{x}}\big(\frac{\partial u}{\partial y}
        A_{z}-\frac{\partial u}{\partial z}A_{y}\big)+\hat{\mathbf{y}}
        \big(\frac{\partial u}{\partial z}A_{x}-\frac{\partial u}{\partial x}
        A_{z}\big)+\hat{\mathbf{z}}\big(\frac{\partial u}{\partial x}A_{y}-
        \frac{\partial u}{\partial y}A_{x}\big)
    \end{equation*}
    and
    \begin{equation*}
        u(\nabla\times\mathbf{A})=\hat{\mathbf{x}}
        \big(u\frac{\partial A_{z}}{\partial y}
        -u\frac{\partial A_{y}}{\partial z}\big)
        +\hat{\mathbf{y}}\big(u\frac{\partial A_{y}}{\partial z}
        -u\frac{\partial A_{z}}{\partial y}A_{x}\big)+\hat{\mathbf{z}}
        \big(\frac{u\partial A_{y}}{\partial x}
        -u\frac{\partial A_{x}}{\partial y}\big)
    \end{equation*}
    Summing these, we have
    $\nabla\times (u\mathbf{A})=\nabla(u)\times \mathbf{A}+u\nabla \times \mathbf{A}$
\end{proof}
\begin{problem}[Wangsness 1-26]
    Verify that $\oint_{S}u\boldsymbol{da}=\int_{V}\nabla(u)d\tau$ and
    $\oint_{S}\mathbf{A}\times\boldsymbol{da}=-\int_{V}\nabla\times\mathbf{A}d\tau$
\end{problem}
\begin{proof}[Solution]
    Let $\mathbf{C}$ be an arbitrary constant vector. Then:
    \begin{equation*}
        \mathbf{C}\cdot\bigg(\oiint_{S}u\boldsymbol{da}
        -\iiint_{V}\nabla(u)d\tau\bigg)
        =\oiint_{S}u\bigg(\mathbf{C}\cdot\boldsymbol{da}\bigg)
        -\iiint_{V}\bigg(\mathbf{C}\cdot\nabla(u)\bigg)d\tau    
    \end{equation*}
    But, as $\mathbf{C}$ is constant, $\nabla\cdot\mathbf{C}=0$, and thus we have:
    \begin{align*}
        \iiint_{V}\mathbf{C}\cdot\nabla(u)d\tau &=\iiint_{V}\nabla\cdot(u \mathbf{C})
        d\tau-\iiint_{V}u\nabla\cdot\mathbf{C}d\tau\\
        &=\iiint_{V}\nabla(u\mathbf{C})d\tau
    \end{align*}
    But from the divergence theorem:
    \begin{equation*}
        \iiint_{V}\nabla(u\mathbf{C})d\tau=\oiint_{S}u\mathbf{C}\cdot\boldsymbol{da}
    \end{equation*}
    Thus,
    $\mathbf{C}\cdot\big(\oint_{S}u\boldsymbol{da}-\int_{V}\nabla(u)d\tau\big) =0$.
    As $\mathbf{C}$ is any arbitrary vector,
    $\oint_{S}u\boldsymbol{da}-\int_{V}\nabla(u)d\tau=0$ and thus
    $\oint_{S}u\boldsymbol{da}=\int_{V}\nabla(u)d\tau$. It is a simple exercise in
    vector geometry to show that if $\mathbf{A}\cdot \mathbf{C} = 0$ for all vectors
    $\mathbf{C}$, then $\mathbf{A} = \mathbf{0}$. In an analogous manner:
    \begin{align*}
        \mathbf{C}\cdot\bigg(\oiint_{S}\mathbf{A}\times\boldsymbol{da}
        +\iiint_{V}\nabla\times\mathbf{A}d\tau\bigg)
        =0\Rightarrow\oiint_{S}\mathbf{A}\times\boldsymbol{da}
        =-\iiint_{V}\nabla\times\mathbf{A}d\tau
    \end{align*}
\end{proof}
\subsection{Homework IV}
Wangsness Chapter 2: Problems 3, 7, 8\\
Wangsness Chapter 3: Problems 9, 10
\begin{problem}[Wangsness 2-3]
    \label{problem:EMAG_wangsness_2_3}
    Consider $8$ equal point charges $q$ located on the corners of a cube of length
    $a$, as in Fig.~\subref{fig:EMAG_1_Wangsness_2_3}. Find the total force on the
    charge at the origin.
\end{problem}
\begin{proof}[Solution]
    \begin{equation*}
        \mathbf{F}_q=\sum_{i=1}^{N}\frac{qq'_{i}}{4\pi \epsilon_0}
        \frac{\hat{\mathbf{r}}_{i}}{R_{i}^{2}}=-\frac{q^{2}}{4\pi \epsilon_0 a^{2}}
        (1+\frac{2}{2^{3/2}}+\frac{1}{3^{3/2}})(\hat{\mathbf{x}}+\hat{\mathbf{y}}
        +\hat{\mathbf{z}})\approx-1.9\frac{q^{2}}{4\pi\epsilon_{0}a^{2}}
        (\hat{\mathbf{x}}+\hat{\mathbf{y}}+\hat{\mathbf{z}})
    \end{equation*}
\end{proof}
\begin{figure}[H]
  \begin{subfigure}[b]{0.49\textwidth}
     \centering
    \begin{tikzpicture}[line width=0.4pt,line cap = round,>={triangle 45}]
        \draw[->] (0.0,0.0,0.0) -- (3.5,0.0,0.0) node[right] {$y$};
        \draw[->] (0.0,0.0,0.0) -- (0.0,3.5,0.0) node[above] {$z$};
        \draw[->] (0.0,0.0,0.0) -- (0.0,0.0,3.5) node[below left] {$x$};
        \draw[-]  (0.0,2.5,0.0) -- (2.7,2.5,0.0) {};
        \draw[-]  (0.0,2.5,0.0) -- (0.0,2.5,2.0) {};
        \draw[-]  (2.7,0.0,0.0) -- (2.7,2.5,0.0) {};
        \draw[-]  (2.7,0.0,0.0) -- (2.7,0.0,2.0) {};
        \draw[-]  (0.0,0.0,2.0) -- (0.0,2.5,2.0) {};
        \draw[-]  (0.0,0.0,2.0) -- (2.7,0.0,2.0) {};
        \draw[-]  (2.7,2.5,0.0) -- (2.7,2.5,2.0) {};
        \draw[-]  (2.7,0.0,2.0) -- (2.7,2.5,2.0) {};
        \draw[-]  (0.0,2.5,2.0) -- (2.7,2.5,2.0) {};
        \filldraw[black] (0.0,0.0,0.0) circle (0.4mm);
        \filldraw[black] (0.0,2.5,0.0) circle (0.4mm);
        \filldraw[black] (0.0,0.0,2.0) circle (0.4mm);
        \filldraw[black] (2.7,0.0,0.0) circle (0.4mm);
        \filldraw[black] (2.7,2.5,2.0) circle (0.4mm);
        \filldraw[black] (0.0,2.5,2.0) circle (0.4mm);
        \filldraw[black] (2.7,0.0,2.0) circle (0.4mm);
        \filldraw[black] (2.7,2.5,0.0) circle (0.4mm);
        \node at (0.0,2.5,0.0) [above right] {$a$};
        \node at (0.0,0.0,2.0) [left] {$a$};
        \node at (2.7,0.0,0.0) [above right] {$a$};
        \node at (0.0,0.0,0.0) [below right] {$q$};
    \end{tikzpicture}
    \caption{Drawing for Wangsness 2-3}
    \label{fig:EMAG_1_Wangsness_2_3}
  \end{subfigure}
  \begin{subfigure}[b]{0.49\textwidth}
    \centering
    \begin{tikzpicture}[line width=0.6pt,line cap = round,>={triangle 45}]
        \draw[->] (0.0,0.0,0.0) -- (3.0,0.0,0.0) node[right] {$y$};
        \draw[->] (0.0,0.0,0.0) -- (0.0,3.0,0.0) node[above] {$z$};
        \draw[->] (0.0,0.0,0.0) -- (0.0,0.0,5.0) node[below left] {$x$};
        \filldraw[ball color=gray!90!white,opacity=0.3] (0,0,0) circle (2.0);
        \coordinate (z)     at (0.0,1.0,0.0)        {};
        \coordinate (x)     at (0.0,0.0,1.0)        {};
        \coordinate (o)     at (0.0,0.0,0.0)        {};
        \coordinate (p)     at (0.9,0.7,0.0)        {};
        \coordinate (p1)    at (0.9,-0.5,0.0)       {};
        \coordinate (p2)    at (1.4,0.0,0.0)        {};
        \coordinate (p3)    at (-0.5,-0.5,0.0)      {};
        \coordinate (Q)     at (0.0,2.5,0.0)        {};
        \node               at (Q) [right]         {Q};
        \node               at (p) [right]       {$p$};
        \node               at (-1,0.8,0)   {$\sigma$};
        \node               at (0.5,0.18,0) {$\mathbf{r}'$};
        \draw[-,densely dashed,draw=black] (o) -- (p1);
        \draw[-,densely dashed,draw=black] (p) -- (p1);
        \draw[-,densely dashed,draw=black] (p1) -- (p2);
        \draw[-,densely dashed,draw=black] (p1) -- (p3);
        \draw[->,>=stealth,draw=blue,semithick] (o) -- (p);
        \draw[->,>=stealth,draw=blue,semithick] (p) -- node[right]
        {\scriptsize{$\mathbf{R}$}}(Q);
        \filldraw[black] (Q) circle (0.5mm);
        \filldraw[black] (p) circle (0.5mm);
        \pic[draw=black, -, "$\theta$",angle eccentricity=1.5,angle radius = 0.4cm]
        {angle = p--o--z};
        \pic[draw=black, -,"$\varphi$", angle eccentricity=1.5,angle radius=0.25cm]
        {angle = x--o--p1};
    \end{tikzpicture}
    \caption{Drawing for Wangsness 2-8}
    \label{fig:EMAG_1_wangsness_2_8}
  \end{subfigure}
  \caption[Figures for Wangsness 2-3 and 2-8]{Figures for problems \ref{problem:EMAG_wangsness_2_3} and \ref{problem:EMAG_wangsness_2_8}, respectively.}
\end{figure}
\begin{problem}[Wangsness 2-8]
    \label{problem:EMAG_wangsness_2_8}
    Consider the sphere in Fig.~\subref{fig:EMAG_1_wangsness_2_8} of radius $a$ with
    a constant surface change density $\sigma$. What is the total charge $Q'$ on the
    sphere? Find the force produced by this charge distribution on a point $q$ on the
    $z$ axis for both $z>a$ and $z<a$.
\end{problem}
\begin{proof}[Solution]
    We have that:
    \begin{equation*}
        Q'=\iint_{S}\sigma da=\sigma\iint_{S}da=\sigma 4\pi a^{2}
    \end{equation*}
    The relative position vector $\mathbf{R}$ of the point $Q$ with respect to a
    point $\mathbf{r}'$ on the sphere is $z\hat{\mathbf{z}} - a\hat{\mathbf{r}}'$.
    So we have:
    \begin{equation*}
        \mathbf{F}_{q}
        =\frac{q}{4\pi\epsilon_0}\iint_{S}\frac{\sigma da'\mathbf{R}}{R^{3}}
        =\frac{q\sigma}{4\pi\epsilon_0}\int_{0}^{2\pi}\int_{0}^{\pi}
        \frac{(z\hat{\mathbf{z}}-a\hat{\mathbf{r}}')a^{2}
        \sin(\theta')d\theta'd\varphi'}{\big(z^2+a^2-2az\cos(\theta')\big)^{3/2}}
    \end{equation*}
    Writing $\hat{\mathbf{r}}'=\sin(\theta')\cos(\varphi')\hat{\mathbf{x}}+
    \sin(\theta')\sin(\varphi')\hat{\mathbf{y}}+\cos(\theta')\hat{\mathbf{z}}$
    leads us to conclude the $x$ and $y$ component vanish as
    $\int_{0}^{2\pi}\cos(\varphi')d\varphi'
    =\int_{0}^{2\pi}\sin(\varphi')d\varphi'=0$.
    Thus, we have:
    \begin{equation*}
        \mathbf{F}_{q}=\frac{q\sigma\hat{\mathbf{z}}}{4\pi\epsilon_{0}}
        \int_{0}^{2\pi}\int_{0}^{\pi}\frac{\big(z-a\cos(\theta)\big)a^{2}
        \sin(\theta')}{\big(z^{2}+a^{2}-2az\cos(\theta')\big)^{3/2}}d\theta'd\varphi'
    \end{equation*}
    Let $u=cos(\theta')$. Then $du=-\sin(\theta')d\theta'$. We obtain:
    \begin{align*}
        \mathbf{F}_{q}&=a^{2}\frac{q\sigma \hat{\mathbf{z}}}
        {2\epsilon_{0}}\int_{-1}^{1}\frac{u}{\big(z^{2}+a^{2}-2azu\big)^{3/2}}du&
        &=a^{2}\frac{q\sigma \hat{\mathbf{z}}}{2\epsilon_{0}}
        \frac{\partial}{\partial z}
        \bigg[\frac{1}{za}\sqrt{a^{2}+z^{2}-2azu}\bigg]_{-1}^{1}\\
        &=a^{2}\frac{q\sigma \hat{\mathbf{z}}}{2\epsilon_{0}}
        \int_{-1}^{1}\frac{\partial}{\partial z}
        \bigg(\frac{1}{\sqrt{a^{2}+z^{2}-2azu}}\bigg)du
        &&=a^{2}\frac{q\sigma \hat{\mathbf{z}}}{2\epsilon_{0}}
        \frac{\partial}{\partial z}\bigg(\frac{|z-a|-|z+a|}{az}\bigg)\\
        &=a^{2}\frac{q\sigma \hat{\mathbf{z}}}{2\epsilon_{0}}
        \frac{\partial}{\partial z} \int_{-1}^{1}
        \frac{1}{\sqrt{a^{2}+z^{2}-2azu}}du\\
    \end{align*}
    Now, if $z>a$, then $|z-a|-|z+a|=(z-a)-(z+a)=-2a$, and thus:
    \begin{equation*}
        \mathbf{F}_{q}=a^{2}\frac{q\sigma \hat{\mathbf{z}}}{2\epsilon_{0}}
        \frac{\partial}{\partial z}\bigg(\frac{-2a}{az}\bigg)
        =a^{2}\frac{q\sigma}{\epsilon_{0}z^{2}}\hat{\mathbf{z}}
        =\frac{qQ}{4\pi\epsilon_{0}z^{2}}\tag{$z>a$}
    \end{equation*}
    If $z<a$, then $|z-a|-|z+a|=2z$, and thus:
    \begin{equation*}
        \mathbf{F}=\mathbf{F}_{q}=a^{2}\frac{q\sigma
        \hat{\mathbf{z}}}{2\epsilon_{0}}
        \frac{\partial}{\partial z}\bigg(\frac{2z}{az}\bigg)
        =\mathbf{0}\tag{$z<a$}
    \end{equation*}
\end{proof}
\begin{problem}[Wangsness 2-7]
    Given a line change of length $L$ with constant charge density lying along the
    positive $z$ axis with its ends located at $z_{0}$ and $z_{0}+L$, find the total
    force exerted on this by a uniform spherical charge distribution with center at
    the origin and radius $a<z_{0}$.
\end{problem}
\begin{proof}[Solution]
    If the charge is distributed over a length $L$ along the $z$-axis with charge per
    unit length $\lambda$, the force over the length $L$ is given by:
    \begin{equation*}
        \mathbf{F}_{Lz}=\frac{\rho a^{3}}{3\epsilon_{0}}\hat{\mathbf{z}}
        \int_{z_{0}}^{z_{0}+L}\frac{\lambda dz}{z^{2}}=
        \frac{\rho\lambda a^{3}}{3\epsilon_{0}}\bigg(\frac{L}{z_{0}(z_{0}+L)}\bigg)
        \hat{\mathbf{z}}
    \end{equation*}
\end{proof}
\begin{problem}[Wangsness 3-9]
    Given two infinite plane sheets with equal and opposite constant surface charge
    densities $\sigma$ that are parallel and a distance $\pm a$ to the $xy$ plane,
    find $\mathbf{E}$ everywhere.
\end{problem}
\begin{proof}[Solution]
    For an infinite sheet, $\mathbf{E}=\frac{\sigma}{2\epsilon_{0}}$.
    Using the principle of superposition, we get:
    \begin{equation*}
        \mathbf{E}=
        \begin{cases}
            0,&|z|>a\\
            -\frac{\sigma}{\epsilon_{0}}\hat{\mathbf{z}},&|z|\leq a
        \end{cases}
    \end{equation*}
\end{proof}
\begin{problem}[Wangsness 3-10]
    A circular arc of radius $a$ with arc angle $2\alpha$ that lies in the $xy$
    plane and has a constant linear charge density $\lambda$ and center of curvature
    at the origin. Find $\mathbf{E}$ at an arbitrary point on the $z$ axis. Show that
    when the arc becomes a complete circle you obtain
    $\mathbf{E}=\frac{\lambda az\hat{\mathbf{z}}}{2\epsilon_{0}(a^{2}+z^{2})^{3/2}}$
\end{problem}
\begin{proof}[Solution]
    We have that:
    \begin{equation*}
        \hat{\mathbf{E}}=\frac{\lambda}{4\pi\epsilon_{0}}
        \int\frac{dl\hat{\mathbf{r}}}{R^{2}}=
        \frac{\lambda}{4\pi\epsilon_0}\int\frac{ad\phi'\hat{\mathbf{r}}}{a^{2}+z^{2}}
    \end{equation*}
    And
    \begin{equation*}
        \hat{\mathbf{r}}=\frac{\mathbf{R}}{R}=
        \frac{-\rho'\hat{\boldsymbol{\uprho}}
        +z\hat{\mathbf{z}}}{\sqrt{a^{2}+z^{2}}}=\frac{-a\cos(\phi')\hat{\mathbf{x}}
        -a\sin(\phi')\hat{\mathbf{y}}+z\hat{\mathbf{z}}}{\sqrt{a^{2}+z^{2}}}    
    \end{equation*}
    So, we obtain the following:
    \begin{equation*}
        \mathbf{E}=\frac{\lambda}{4\pi\epsilon_0}\int_{-\alpha}^{\alpha}
        \frac{-a\cos(\phi')\hat{\mathbf{x}}-a\sin(\phi')\hat{\mathbf{y}}
        +z\hat{\mathbf{z}}}{\sqrt{a^{2}+z^{2}}}ad\phi'
        =\frac{\lambda a\big[-a\sin(\alpha)\hat{\mathbf{x}}+z\alpha
        \hat{\mathbf{z}}\big]}{2\pi\epsilon_{0}(a^{2}+z^{2})^{3/2}}
    \end{equation*}
    If $\alpha=\pi$, we get:
    \begin{equation*}
        \mathbf{E}
        =\frac{\lambda az}{2\epsilon_{0}(a^{2}+z^{2})^{3/2}}\hat{\mathbf{z}}
    \end{equation*}
\end{proof}
\newpage
\subsection{Homework V}
Wangsness Chapter 4 - Problems: 3, 5, 6, 7, 11, 12
\begin{problem}[Wangsness 4-3]
    \label{problem:EMAG_wangsness_4_3}
    An infinitely long line is surrounded by an infinitely long cylinder of radius
    $\rho_{0}$ whose axis coincides with the line charge
    (See Fig.~\ref{fig:EMAG_1_Wangsness_4_3}). The surface of the 
    cylinder carries a charge of constant surface density $\sigma$. Find $\mathbf{E}$
    everywhere. What particular value of $\sigma$ will make $\mathbf{E}=\mathbf{0}$
    for all points outside of the charged cylinder?
\end{problem}
\begin{proof}[Solution]
    For $\rho<\rho_0$ choose a Gaussian cylinder concentric with the line. From
    Gauss' Law we have:
    \begin{equation*}
        \oiint \mathbf{E}\cdot \mathbf{da} = \frac{Q_{in}}{\epsilon_0}
        =E(2\pi\rho\ell)\Rightarrow
        \mathbf{E}=\frac{\lambda}{2\pi\epsilon_{0}\rho}\hat{\boldsymbol{\uprho}}
    \end{equation*}
    For $\rho>\rho_0$ choose a similar Gaussian cylinder. We get:
    \begin{equation*}
        \oiint\mathbf{E}\cdot\mathbf{da}=\frac{Q_{in}}{\epsilon_{0}}\Rightarrow
        \mathbf{E}=\frac{\lambda\ell+\sigma 2\pi\rho_{0}\ell}
        {2\pi\epsilon_{0}\rho\ell}\hat{\boldsymbol{\uprho}}
        =\frac{\lambda+2\pi\rho_{0}\sigma}{2\pi\epsilon_{0}\rho}
        \hat{\boldsymbol{\uprho}}
    \end{equation*}
    To make $\hat{\mathbf{E}}=\mathbf{0}$ for $\rho>\rho_0$ we need
    $\sigma=\frac{-\lambda}{2\pi\rho_0}$.
\end{proof}
\begin{figure}[H]
    \centering
    \begin{tikzpicture}[every path/.style={thick}]
        \draw[draw=black,thick=semithick]
            (-6,0) -- (6,0) node[above left] {$\lambda$};
        \draw[draw=blue]                (-4,0) circle (1);
        \draw[draw=blue]                (-4,1) -- (3,1);
        \draw[draw=blue]                (3,1) arc (90:-90:1);
        \draw[draw=blue]                (-4,-1) -- node[below] {$\sigma$} (3,-1);
        \draw[draw=red,densely dashed]  (-2,0) circle (0.75);
        \draw[draw=black]
            (-4,0) -- node[above right=0.02cm and 0.01cm] {$\rho_{0}$}(-4.71,0.71);
        \draw[draw=red,densely dashed]  (-2,0.75) -- node[below] {$\ell$} (2,0.75);
        \draw[draw=red,densely dashed]  (-2,-0.75) -- (2,-0.75);
        \draw[draw=red,densely dashed]  (2,0.75) arc (90:-90:0.75);
    \end{tikzpicture}
    \caption[Drawing for Wangsness 4-3]
    {Drawing for problem \ref{problem:EMAG_wangsness_4_3}}
    \label{fig:EMAG_1_Wangsness_4_3}
\end{figure}
\begin{problem}[Wangsness 4-5]
    \label{problem:EMAG_wangsness_4_5}
    A sphere of radius $a$ has a charge density that varies with distance $r$ from
    the center according to $\rho=Ar^{1/2}$, where $A$ is a constant. Find
    $\mathbf{E}$ everywhere.
\end{problem}
\begin{proof}[Solution]
    For $r>a$
    \begin{equation*}
        \oiint\mathbf{E}\cdot\mathbf{da}=\frac{Q_{in}}{\epsilon_{0}}\Rightarrow
        E(4\pi r^{2})=\iiint Ar'^{1/2}r'^{2}\sin(\theta')dr'd\theta'd\phi'
        =4\pi\frac{2}{7}\frac{a^{7/2}}{\epsilon_{0}}A\Rightarrow
        \mathbf{E}=\frac{2A a^{7/2}}{7\epsilon_{0}r^{2}}\hat{\mathbf{r}}
    \end{equation*}
    For $r<a$
    \begin{equation*}
        E(4\pi r^{2})=4\pi A\frac{2}{7}r^{7/2}\Rightarrow
        \mathbf{E}=\frac{2Ar^{3/2}}{7\epsilon_{0}}\hat{\mathbf{r}}
    \end{equation*}
\end{proof}
\begin{figure}[H]
    \centering
    \begin{tikzpicture}[every path/.style={thick}]
        \draw[draw=blue]                (-3.5,0) circle (2);
        \draw[draw=red,densely dashed]  (-3.5,0) circle (3);
        \draw[draw=red,densely dashed]  (3.5,0) circle (2);
        \draw[draw=blue]                (3.5,0) circle (3);
        \draw[draw=black] (-3.5,0) -- node[below left] {$a$} (-4.914,1.414);
        \draw[draw=black] (-3.5,0) -- node[below right] {$r$} (-1.378,2.121);
        \filldraw[fill=black] (-1.378,2.121) circle (0.5mm) node[above right] {$P$};
        \draw[draw=black] (3.5,0) -- node[below right] {$r$} (4.914,1.414);
        \draw[draw=black] (3.5,0) -- node[below left] {$a$} (1.378,2.121);
        \filldraw[fill=black] (4.914,1.414) circle (0.5mm) node[above right] {$P$};
    \end{tikzpicture}
    \caption[Drawing for Wangsness 4-3]
    {Drawing for problem \ref{problem:EMAG_wangsness_4_5}.
    Dashed red represents the Gaussian surface, and blue represents
    the charged sphere.}
    \label{fig:EMAG_1_Wangsness_4_5}
\end{figure}
\begin{problem}[Wangsness 4-6]
    Two concentric spheres have radii $a$ and $b$ with $b>a$. The between them
    $(a\leq r\leq b)$ is filled with a charge of constant density. The charge density
    is zero everywhere else. Find $\mathbf{E}$ everywhere and express it in terms of
    the total charge $Q$. What happens as $a\rightarrow 0$?
\end{problem}
\begin{proof}[Solution]
    From the symmetry of the problem we have, for $r<a$, that:
    \begin{equation*}
        \oiint\mathbf{E}\cdot\mathbf{da}=0\Rightarrow\mathbf{E}=\mathbf{0}
    \end{equation*}
    For $a\leq r\leq b$:
    \begin{equation*}
        \oiint\mathbf{E}\cdot\mathbf{da}=\frac{Q_{in}}{\epsilon_0}=
        \frac{\rho_{c}(\frac{4}{3}\pi)(r^{3}-a^{3})}{\epsilon_0}    
    \end{equation*}
    Where $\rho_{c}$ is the charge density
    $\rho_{c}=\frac{Q}{\frac{4}{3}\pi(b^{3}-a^{3})}$. Therefore:
    \begin{equation*}
        \mathbf{E}=\frac{Q}{4\pi\epsilon_{0}r^{2}}
        \bigg(\frac{r^{3}-a^{3}}{b^{3}-a^{3}}\bigg)\hat{\mathbf{r}}    
    \end{equation*}
    For $r\geq b$, this is similar to the uniform sphere problem:
    \begin{equation*}
        \oiint\mathbf{E}\cdot\mathbf{da}=\frac{Q_{in}}{\epsilon_{0}}\Rightarrow
        \mathbf{E}=\frac{Q}{4\pi\epsilon_{0}r^{2}}
    \end{equation*}
    In the limit as $a\rightarrow 0$ we have:
    \begin{equation*}
        \mathbf{E}=
        \begin{cases}
            \frac{Qr}{4\pi\epsilon_{0}b^{3}},&r\leq b\\
            \frac{Q}{4\pi\epsilon_{0}r^{2}},&r>b
        \end{cases}
    \end{equation*}
    This is expected, for in the limit as $a\rightarrow 0$ we obtain a uniformly
    charged sphere of radius $b$.
\end{proof}
\subsubsection{Wangsness 4-7}
Choose as a Gaussian surface a cylinder of length $L$ and radius $\rho$ that is concentric with the infinite cylinder. For $\rho<a$, $\oiint \mathbf{E}\cdot \mathbf{da} = \frac{Q_{in}}{\epsilon_0}$. Now $\oiint \mathbf{E} \cdot \mathbf{da} = \iint_{Right\ Side} \mathbf{E}\cdot \mathbf{da} + \iint_{Left\ Side}\mathbf{E}\cdot \mathbf{da} + \iint_{Cylinder} \mathbf{E}\cdot \mathbf{da}$. From symmetry, we have that $\mathbf{E}$ and $\mathbf{da}$ are orthogonal along the left and right faces of the cylinder, leaving only the cylindrical body left to integrate over. We get $E(2\pi \rho L) = \frac{\rho_c \pi \rho^2 L}{\epsilon_0}$, where $\rho_c$ is the charge density. Combining this together, we obtain $\mathbf{E} = \frac{\rho_c \rho}{2\epsilon_0} \hat{\boldsymbol{\uprho}}$. For $\rho>a$, $\mathbf{E} = \frac{\rho_{c} a^2}{2\epsilon_0 \rho}\hat{\boldsymbol{\uprho}}$. We see that the electric field goes like $\frac{1}{\rho}$, which is consistent with the result obtained from $4-11$.
\subsubsection{Wangsness 4-11}
$\mathbf{E} = E_0 \big(\frac{\rho}{a}\big)^3 \hat{\boldsymbol{\uprho}}$ for $0 < \rho < a$, and $\mathbf{E} = 0$ otherwise. Thus, $\nabla \cdot \mathbf{E} = \frac{1}{\rho} \frac{\partial}{\partial \rho}\big(\rho E_{\rho}\big) + \frac{1}{\rho} \frac{\partial E_{\phi}}{\partial \phi} + \frac{\partial E_z}{\partial z}$ = $\frac{4E_0 \rho^2}{a^3}$. From Gauss' Law, $\nabla \cdot \mathbf{E} = \frac{\rho_c}{\epsilon_0}$. Thus, $\rho_c = \epsilon_0 \nabla \cdot \mathbf{E} = \frac{4\epsilon_0 E_0 \rho^2}{a^3}$ for $\rho<a$. For $\rho>a$, $\nabla \cdot \mathbf{E} = \nabla \cdot \mathbf{0} = 0$, and thus $\rho_c = 0$.
\subsubsection{Wangsness 4-12}
$\mathbf{E} = E_r \hat{\mathbf{r}}+E_{\theta} \hat{\boldsymbol{\uptheta}}$, where $E_r = \frac{2A\cos(\theta)}{r^3}$ and $E_{\theta} = \frac{A\sin(\theta)}{r^3}$. From Gauss' Law, $\nabla \cdot \mathbf{E} = \frac{\rho_c}{\epsilon_0}$, and thus $\rho_c = \epsilon_0 \nabla \cdot \mathbf{E}$. But $\nabla \cdot \mathbf{E} = 0$, and thus $\rho_c = 0$.
\subsection{Homework VI}
\subsubsection{Wangsness 5-1}
$\mathbf{E} = (yz-2x)\hat{\mathbf{x}}+xz\hat{\mathbf{y}}+xy\hat{\mathbf{z}}$. So, $\nabla \times \mathbf{E} = (x-x)\hat{\mathbf{x}}-(y-y)\hat{\mathbf{y}}+(z-z)\hat{\mathbf{z}} = \mathbf{0}$. Therefore $\mathbf{E}$ is a possible electrostatic field. Indeed, writen $\mathbf{E} = \nabla(\phi)$, we get $-\frac{\partial \phi}{\partial x} = yz-2x$, so $-\phi = xyz - x^2 + g(y,z)$, where $g$ is a function of $y$ and $z$ (Note: $\frac{\partial g}{\partial x} = 0$ as $g$ is not a function of $x$). Now $-\frac{\partial \phi}{\partial y} = xz$ and $-\frac{\partial \phi}{\partial z} = xy$, and thus $g(y,z) = constant$. So, $\phi(x,y,z) = -xyz+x^2 + C$, where $C$ is some constant. We may choose $C$ is we desire, so let $C=0$ to make things easy. The integral $\int \mathbf{E}\cdot \mathbf{d\ell}$ from the origin $(0,0,0)$ to a point $(x,y,z)$ is thus independent of path and may be computed by using the fundamental theorem of gradients. That is, $\int \mathbf{E}\cdot \mathbf{d\ell} = -\int _{(0,0,0)}^{(x,y,z)} \nabla(\phi)\cdot \mathbf{d\ell} = -\big(\phi(x,y,z) -\phi(0,0,0)\big) = \phi(0,0,0)-\phi(x,y,z)$. Now, $\phi(0,0,0)= 0$, and thus $\int_{(0,0,0)}^{(x,y,z)}\mathbf{E}\cdot \mathbf{d\ell} = -\phi(x,y,z) = xyz-x^2$.
\subsubsection{Wangsness 5-3}
$\phi = \frac{1}{4\pi \epsilon_0}\big[ \frac{q}{R_{+}} - \frac{q}{R_{-}}\big] = \frac{q}{4\pi \epsilon_0}\big[\frac{1}{\sqrt{x^2+y^2+(z-a)^2}}-\frac{1}{\sqrt{x^2+y^2+(z+a)^2}}\big]$. At $z=0$, we get $\phi = \frac{q}{4\pi \epsilon_0}\big[ \frac{1}{\sqrt{x^2+y^2+a^2}}-\frac{1}{\sqrt{x^2+y^2+a^2}}\big] = 0$. Thus, the entire $xy-$plane is an equipotential surface with $\phi = 0$.
\begin{figure}[htbp]
  \centering
    {\includegraphics[scale=0.4]{5-3.png}}
    \caption{Drawing for Wangsness 5-3}
\end{figure}
\subsubsection{Wangsness 5-4}
$\phi = \frac{1}{4\pi \epsilon_0} \sum \frac{q_i}{R_i} = \frac{1}{4\pi \epsilon_0}\bigg[ \frac{q}{\sqrt{a^2/4}}+\frac{2q}{\sqrt{a^2/2}}-\frac{4q}{\sqrt{a^2/2}}+\frac{3q}{\sqrt{a^2/2}}\bigg] = \frac{q}{\sqrt{2}\pi \epsilon_0 a}$. To know $\mathbf{E}$ from $\phi$, $\phi$ must be known in some region about the point, not just at the point. To be precise in mathematical terms, we must know $\phi$ in some open set about the point in order to compute $\nabla(\phi)$. This is analogous to functions in calculus. Suppose $f$ is a function and $a$ is a real number, and suppose we know the value of $f(a)$. Can we determine what $f'(a)$ is? The answer is no, there is not enough information. If we know what $f(x)$ is in some interval $(a-\epsilon,a+\epsilon)$, then we can compute $f'(a)$.
\begin{figure}[htbp]
    \centering
    {\includegraphics[scale=0.4]{5-4.png}}
    \caption{Drawing for Wangsness 5-4}
\end{figure}
\subsubsection{Wangsness 5-10}
We know that $\mathbf{E} = \begin{cases} \frac{2A a^{7/2}}{7\epsilon_0 r^2}\hat{\mathbf{r}}, & r>a \\ \frac{2A r^{3/2}}{7\epsilon_0}, & r<a\end{cases}$. So, $\Delta \phi = \int \mathbf{E} \cdot \mathbf{d\ell}$. Now, $\mathbf{d\ell} = -dr(-\hat{\mathbf{r}}) = \mathbf{dr}$. We split the integral into two parts and compute: $\int \mathbf{E}\cdot \mathbf{d\ell} = \int_{0}^{a} \mathbf{E}\cdot \mathbf{dr} + \int_{a}^{\infty} \mathbf{E}\cdot \mathbf{dr} = \frac{4A a^{5/4}}{35 \epsilon_0}\bigg[ \frac{7}{2}-\big(\frac{r}{a}\big)^{7/2}\bigg]$.
\subsubsection{Wangsness 5-11}
We have that $\mathbf{E} = \begin{cases} \mathbf{0}, & r<a\\ \frac{Q}{4\pi \epsilon_0}\bigg(\frac{r^3-a^3}{b^3-a^3}\bigg)\hat{\mathbf{r}}, & a\leq r \leq b\\ \frac{Q}{4\pi \epsilon_0 r^2}\hat{\mathbf{r}}, & r>b\end{cases}$ We thus split the integral into three regions and compute: $\Delta\phi=\int \mathbf{E}\cdot \mathbf{d\ell} = \int_{0}^{a} \mathbf{E}\cdot \mathbf{dr}+\int_{a}^{b} \mathbf{E}\cdot \mathbf{dr}+\int_{b}^{\infty} \mathbf{E}\cdot \mathbf{dr}$. We then obtain $\phi(\mathbf{r}) = \begin{cases} \frac{\rho_c(b^3-a^3)}{3\epsilon_0 r}, & r>b \\ \frac{\rho}{3\epsilon_0}\bigg[ \frac{3}{2}b^2-\frac{r^2}{2}-\frac{a^3}{r}\bigg], & a\leq r \leq b\\ \frac{\rho_c}{2\epsilon_0}(b^2-a^2), & r<a\end{cases}$
\subsubsection{Wangsness 5-14}
$\phi = \frac{1}{4\pi \epsilon_0}\iint \frac{\sigma_c da'}{R}$. Here, $R = |\mathbf{r}-\mathbf{r}'| = \sqrt{a^2+r^2-2ar\cos(\theta')}$, and $da' = a^2\sin(\theta')d\theta'd\phi'$. So, we have $\phi = \frac{\sigma_c}{4\pi \epsilon_0}\int_{0}^{2\pi}\int_{0}^{\pi} \frac{a^2\sin(\theta')d\theta' d\phi'}{\sqrt{a^2+r^2-2ar\cos(\theta')}} = \frac{\sigma_c a^2}{2\epsilon_0}\int_{0}^{\pi} \frac{\sin(\theta')d\theta'}{\sqrt{a^2-r^2-2ar\cos(\theta')}}$. Let $u = \cos(\theta')$. Then $du = -\sin(\theta')d\theta'$, and we have $-\int\frac{du}{\sqrt{a^2+r^2-2aru}}$. Note that $r$ and $a$ are constants in the integral, and thus this can be computed by trigonometric substitution (Or wolframalpha/integral tables if you're lazy). We thus have $\phi(\mathbf{r}) = \begin{cases}\frac{a^2\sigma}{\epsilon_0 r}, & r>a \\ \frac{a\sigma}{\epsilon_0}, & r<a\end{cases}$.
\subsection{Homework VII}
\subsubsection{Wangsness 6-6}
Before the connection, $Q=C_1 \Delta \phi$. After the connection, $Q_1 = C_1 \Delta\phi'$, $Q_2 = C_2 \Delta \phi'$, $Q_1+Q_2=Q$, and $Q_1+Q_2=(C_1+C_2)\Delta \phi' = Q = C_1\Delta \phi$. Thus, $\Delta \phi' = \frac{C_1}{C_1+C_2}\Delta \phi$, $Q_1 = \frac{C_1^2}{C_1+C_2}\Delta \phi$, and $Q_2 = \frac{C_1 C_2}{C_1+C_2}\Delta \phi$.
\subsubsection{Wangsness 6-7}
For Parallel:\\
$Q_1 = C_1\Delta \phi$, $Q_2 = C_2\Delta \phi$, where $Q_1$ and $Q_2$ are the charges on the plates $C_1$ and $C_2$, respectively. The total charge is $Q_1+Q_2$. Thus, $Q=Q_1+Q_2 = C_1\Delta\phi + C_2 \Delta \phi =(C_1+C_2)\Delta\phi = C_p \Delta \phi$, where $C_p$ is $C_1+C_2$. \\
For Series:\\
$\Delta \phi = \Delta\phi_1 + \Delta \phi_2$, where $\Delta\phi_1$ and $\Delta \phi_2$ are the potential differences across $C_1$ and $C_2$, respectively. If a charge $Q$ is on the left plate of $C_1$, then there is a charge $-Q$ on the right plate, and therefore there is a charge $Q$ on the left plate of $C_2$ as well. Thus, $Q=Q_1=Q_2$. So, $\Delta \phi = \frac{Q_1}{C_1} + \frac{Q_2}{C_2} = \frac{Q}{C_1}+\frac{Q}{C_2} = Q\big(\frac{1}{C_1}+\frac{1}{C_2}\big) = \frac{Q}{C_s}$, where $\frac{1}{C_s} = \frac{1}{C_1}+\frac{1}{C_2}$.
\subsubsection{Wangsness 6-9}
At $r=a$, $E=\frac{Q}{4\pi \epsilon_0 a^2}$, $Q=C\Delta \phi$, and $C=\frac{4\pi \epsilon_0 ab}{b-a}$. Thus, we may write $E$ as $E=\frac{b\Delta \phi}{a(b-a)}$. To minimize this, we solve $\frac{\partial E}{\partial a} = 0$. This gives us $a=\frac{b}{2}$. To check that is is a minimum, we check $\frac{\partial^2 E}{\partial a^2}\bigg|_{a=\frac{b}{2}}$. This gives us $\frac{32\Delta \phi}{b}$, which is positive. Therefore $a=\frac{b}{2}$ is a minimum.
\subsubsection{Wangsness 6-10}
$E=\frac{\lambda}{2\pi \epsilon_0 \rho}$, where $\lambda$ is the linear charge density, $\lambda = \frac{Q}{L}$. Thus, $\Delta\phi = - \int_{b}^{a} \frac{\lambda}{2\pi \epsilon_0\rho}d\rho = \frac{\lambda}{2\pi \epsilon_0}\ln\big(\frac{b}{a}\big)$. We have that $C=\frac{Q}{\Delta\phi}$, and thus $C = \frac{2\pi \epsilon_0 L}{\ln\big(\frac{b}{a}\big)}$
\subsection{Homework VIII}
\subsubsection{Wangsness 7-2}
$U_e = \underset{All\ Pairs}\sum\frac{q_i q_j}{r\pi \epsilon_0 R_{ij}}$, where $R_{ij} = |\mathbf{r}_i-\mathbf{r}_j| = \sqrt{r_i^2+r_j^2 -2\mathbf{r}_i\cdot \mathbf{r}_j}$. Computing the sum, we get $U_e = \frac{q^2}{4\pi \epsilon_0 a}\big(12 + \frac{12}{\sqrt{2}}+\frac{4}{\sqrt{3}}\big)$.
\begin{figure}[htbp]
    \centering
    {\includegraphics[scale=0.4]{7-2.png}}
    \caption{Drawing for Wangsness 7-2}
\end{figure}
\subsubsection{Wangsness 7-4}
$\rho_c = Ar^n$, where $A$ is a constant and $n\geq 0$. Thus, $U_e = \frac{1}{2} \int \rho_c(\mathbf{r})\phi(\mathbf{r})d\tau$. We have that $E = \frac{Aa^{n+3}}{\epsilon_0 (n+3)r^2}$ for $r\geq a$, and $E=\frac{Ar^{n+1}}{\epsilon_0(n+3)}$ for $r\leq a$ from Gauss' Law. Now, all of the charges are located within $r\leq a$, and so we must compute $\phi$ in this region. Letting $\phi(r)\rightarrow 0$ as $r\rightarrow \infty$, we may compute $\phi$ as $\phi(r) = -\int_{\infty}^{r}\mathbf{E}\cdot \mathbf{d\ell} = -\int_{\infty}^{a} \mathbf{E}\cdot \mathbf{d\ell} - \int_{a}^{r}\mathbf{E}\cdot \mathbf{d\ell} = \frac{Aa^{n+3}}{\epsilon_0(n+3)a}+\frac{Aa^{n+2}}{\epsilon_0(n+2(n+3)}-\frac{Ar^{n+2}}{\epsilon_0(n+3)(n+2)}$. We can now compute  the potential energy. $U_e =\frac{1}{2}\int \rho_c(\mathbf{r})\phi(\mathbf{r})d\tau = \int_{0}^{2\pi} \int_{0}^{\pi}\int_{0}^{a} A r^n \bigg(\frac{Aa^{n+3}}{\epsilon_0(n+3)a}+\frac{Aa^{n+2}}{\epsilon_0(n+2)(n+3)}-\frac{Ar^{n+2}}{\epsilon_0(n+3)(n+2)}\bigg)r^2 \sin(\theta)dr d\theta d\phi = \frac{2\pi A^2}{\epsilon_0 (n+3)}\bigg[\frac{a^{2n+5}}{n+3}+\frac{a^{2n+5}}{(n+2)(n+3)}-\frac{a^{2n+5}}{(n+2)(2n+5)}\bigg]$. Taking the limit as $n\rightarrow 0$, we get $\frac{3}{5}\bigg[\frac{Q^2}{4\pi \epsilon_0 a}\bigg]$, as expected for a constant spherical charge density.
\subsubsection{Wangsness 7-6}
\begin{figure}[htbp]
    \centering
    {\includegraphics[scale=0.4]{7-6.png}}
    \caption{Drawing for Wangsness 7-6}
\end{figure}
$U_e = \frac{1}{2}\int_{S}\sigma_c(\mathbf{r})\phi(\mathbf{r})da$. In the region between the cylinders we have that $E = \frac{\lambda}{2\pi \epsilon_0 \rho}$, and thus $\phi= \frac{\lambda 2\pi \epsilon_0}\ln\big(\frac{\rho_0}{\rho}\big)$, where $\rho_0$ is the zero of $\phi$. We can now compute $U_e$ and we get $U_e = \frac{\lambda L}{4\pi \epsilon_0}\ln\big(\frac{b}{a}\big)$. As $U_e = \frac{1}{2}\frac{Q^2}{C}$, we get $C= \frac{2\pi \epsilon_0 L}{\ln\big(\frac{b}{a}\big)}$
\subsubsection{Wangsness 7-9}
The electric field is $\mathbf{E}_i = \frac{Qr}{4\pi \epsilon_0 a^3}\hat{\mathbf{r}}$ inside the distribution, and $\mathbf{E}_o = \frac{Q}{4\pi\epsilon_0r^2}\hat{\mathbf{r}}$. The energy density inside is thus $\mu_{e_i} = \frac{1}{2}\epsilon_0 E_i^2=\frac{Q^2r^2}{32\pi^2 \epsilon_0 a^6}$ and outside is $\mu_{e_o} = \frac{1}{2}\epsilon_0 E_0^2 = \frac{Q^2}{32\pi^2 \epsilon_0 r^4}$. The total energy is $\int_{Inside} \mu_{e_i}d\tau + \int_{Outside} \mu_{e_o}d\tau$. Computing this integral, we get $U_e = \frac{3}{5}\bigg( \frac{Q^2}{4\pi \epsilon_0}\bigg)$, in agreement with before.
\subsubsection{Wangsness 7-10}
$U_e = \frac{\epsilon_0}{2} \underset{All\ Space}\int E^2 d\tau$. The $\mathbf{E}-$Field in between the spheres is $\frac{Q}{4\pi \epsilon_0 r^2}\hat{\mathbf{r}}$. The energy associated to this region is thus $\int_{0}^{2\pi}\int_{0}^{\pi}\int_{a}^{b} \frac{\epsilon}{2} \frac{Q^2}{16\pi^2 \epsilon_0^2 r^4}r^2\sin(\theta) dr d\theta d\phi = \frac{Q^2}{8\pi \epsilon_0}\bigg(\frac{b-a}{ab}\bigg)$. We have that $C = \frac{1}{2} \frac{Q^2}{U_e}$, and thus $C = \frac{4\pi \epsilon_0 ab}{b-a}$.
\subsubsection{Wangsness 7-17}
$\mathbf{E} = \frac{\lambda}{2\pi \epsilon_0 \rho}\hat{\boldsymbol{\uprho}}$, $\Delta\phi = \frac{\lambda}{2\pi \epsilon_0 \ln\big(\frac{b}{a}\big)}$, and therefore $\lambda = \frac{2\pi \epsilon_0 \Delta\phi}{\ln\big(\frac{b}{a}\big)}$ and $E = \frac{\Delta \phi}{\rho \ln\big(\frac{b}{a}\big)}$. So, $f_e = \mu_e = \frac{1}{2} \epsilon_0 E^2\bigg|_{\rho = a} = \frac{1}{2} \epsilon_0 \bigg[ \frac{\Delta\phi}{a \ln\big(\frac{b}{a}\big)}$. Thus, $\mathbf{F}_{Tot} = \int f_e \mathbf{da} = f_e \int \mathbf{da} = 0$.
\subsection{Homework IX}
\subsubsection{Wangsness 8-5}
The monopole moment is $Q = \sum q_i = -3q-2q-q+q+2q+3q+4q+5q=9q$.
The dipole moment is $\mathbf{p} = \sum q_i \hat{\mathbf{r}}_i = (-3q)\mathbf{0} + (-2q)a\hat{\mathbf{x}} + (-q)(a\hat{\mathbf{x}}+a\hat{\mathbf{y}})+qa\hat{\mathbf{y}} + 2q(a\hat{\mathbf{y}}+a\hat{\mathbf{z}})+3q(a\hat{\mathbf{x}}+a\hat{\mathbf{y}}+a\hat{\mathbf{z}})+4q(a\hat{\mathbf{x}}+a\hat{\mathbf{z}})+5qa\hat{\mathbf{z}}=4qa\hat{\mathbf{x}}+5qa\hat{\mathbf{y}}+14aq\hat{\mathbf{z}}$
\begin{figure}[htbp]
    \centering
    {\includegraphics[scale=0.4]{8-5.png}}
    \caption{Drawing for Wangsness 8-5}
\end{figure}
It is possible to find an origin about which the dipole moment will vanish. Consider an arbitrary charge distribution with the center of charge designated as c.c. The position vector of this is $\mathbf{r}_{c.c.} = \frac{\int \mathbf{r}' \rho d\tau '}{\int \rho d\tau '} = \frac{\sum \mathbf{r}_i q_i}{\sum q_i}$. Shifting the origin by $\mathbf{r}_{c.c.}$ the dipole moment becomes zero. The original dipole moment is $\mathbf{p}_{o} = \int \mathbf{r}' d\tau ' = \sum \mathbf{r}_i q_i$. The new dipole moment will be $\mathbf{p}_N = \int (\mathbf{r}' - \mathbf{r}_{c.c.})d\tau' = \int \mathbf{r}' d\tau' - \mathbf{r}_{c.c.} \int \rho d\tau' = \sum \mathbf{r}_i q_i - \mathbf{r}_{c.c} \sum q_i = \sum \mathbf{r}_i q_i - \frac{\sum \mathbf{r}_i q_i }{\sum q_i}\sum q_i = \mathbf{0}$. For the problem at hand, this equates to $\mathbf{r}_{c.c} = \langle \frac{4}{9}a, \frac{5}{9}a, \frac{14}{9}a\rangle$ (In Cartesian Coordinates).
\begin{problem}[Wangsness 8-8]
\end{problem}
\begin{proof}[Solution]
\begin{equation*}
    Q = \int_{S'} \sigma da'= \int_{0}^{2\pi} \sin(\theta)\cos(\theta)d\theta = \int_{0}^{2\pi} \int_{0}^{\pi} \sigma_{0} \cos(\theta) a^2 \sin(\theta) d\theta d\phi = 2\pi \sigma_{0} = 0
\end{equation*}
\begin{align*}
    \mathbf{p} &= \int_{S'}\sigma \mathbf{r}' da'\\
    &= \sigma a^{3}\int_{0}^{2\pi} \int_{0}^{\pi} \cos(\theta) \big(\sin(\theta) \cos(\phi) \hat{\mathbf{x}} + \sin(\theta)\sin(\phi) \hat{\mathbf{y}} + \cos(\theta) \hat{\mathbf{z}}\big)\sin(\theta)d\theta d\phi\\
    &= \frac{4\pi \sigma_0 a^3}{3} \hat{\mathbf{z}}
\end{align*}
$\phi \approx \frac{1}{4\pi \epsilon_0} \frac{\mathbf{p}\cdot \hat{\mathbf{r}}}{r^2} = \frac{\sigma_0 a^3}{3\epsilon_0^2 r^2}\cos(\theta)$
\end{proof}
\subsubsection{Wangsness 9-1}
Surface of Separation between regions $1$ and $2$ is a plane $f=2x+y+z=1$. $\mathbf{E}_1 = 4\hat{\mathbf{x}}+\hat{\mathbf{y}}-3\hat{\mathbf{z}}$ is given. Find the normal and tangential component of $\mathbf{E}_1$: The unit vector is the normal to the plane which is $\hat{\mathbf{n}}=\frac{\nabla(f)}{|\nabla(f)|} = \frac{2\hat{\mathbf{x}}+\hat{\mathbf{y}}+\hat{\mathbf{z}}}{\sqrt{6}}$. The normal component of $\mathbf{E}_1$ is $\mathbf{E}_1 \cdot \hat{\mathbf{n}} = \sqrt{6}$. Thus, $\mathbf{E}_{1n} = (\mathbf{E}_1 \cdot \hat{\mathbf{n}})\hat{\mathbf{n}} = 2\hat{\mathbf{x}}+\hat{\mathbf{y}}+\hat{\mathbf{z}}$. The tangential component is $\mathbf{E}_1 - \mathbf{E}_{1n} = 2\hat{\mathbf{x}}-4\hat{\mathbf{z}}$.
\subsubsection{Wangsness 9-3}
We are given the density $\sigma = \sigma_0 \cos(\theta) = \frac{\sigma_0 z}{a}$ and $\mathbf{E}_1 = \alpha \hat{\mathbf{x}}+\beta \hat{\mathbf{y}}+ \gamma \hat{\mathbf{z}}$. The boundary conditions are $E_{2t} = E_{1t}$ and $E_{2n}-E_{1n} = \frac{\sigma}{\epsilon_0}$. We can write $\mathbf{E}_1 = E_{1t}\hat{\boldsymbol{\upmu}}+E_{1n} \hat{\mathbf{r}}$ where $\hat{\mathbf{r}}$ is the normal to the spherical surface and $\hat{\boldsymbol{\upmu}}$ is the tangent to the sphere. On the outside, $\mathbf{E}_2 = E_{2t}\hat{\boldsymbol{\upmu}}+E_{2n}\hat{\mathbf{r}}$. Now, using the boundary conditions the $E-$field on the outside is $\mathbf{E}_2 = E_{1t}\hat{\boldsymbol{\upmu}}+(\frac{\sigma}{\epsilon_0}+E_{1n})\hat{\mathbf{r}} = E_{1t}\hat{\boldsymbol{\upmu}}+E_{1n}\hat{\mathbf{r}}+\frac{\sigma}{\epsilon_0}\hat{\mathbf{r}} = \alpha \hat{\mathbf{x}}+\beta \hat{\mathbf{y}}+\gamma \hat{\mathbf{z}} + \frac{\sigma z}{\epsilon_0 a}\bigg(\frac{x\hat{\mathbf{x}}+y\hat{\mathbf{y}}+z\hat{\mathbf{z}}}{a}\bigg)$. So, $\mathbf{E}_2 = \big(\alpha+\frac{\sigma_0 zx}{\epsilon_0 a^2}\big)\hat{\mathbf{x}}+\big(\beta + \frac{\sigma_0 zy}{\epsilon_0 a^2}\big)\hat{\mathbf{y}}+\big(\gamma+ \frac{\sigma_0 z^2}{\epsilon_0 a^2}\big)\hat{\mathbf{z}}$
\subsection{Homework X}
\subsubsection{Wangsness 10-3}
$\mathbf{P}= P(1+\alpha z)\hat{\mathbf{z}}$, where $P$ and $\alpha$ are constants. Volume charge density is $\rho_{b} = -\nabla \cdot \mathbf{P} = -\alpha P$. The surface charge densities are $\mathbf{P}\cdot \hat{\mathbf{n}}$, so $\sigma_{top} = P(1+\alpha t)$ and $\sigma_{bottom} = -P$. On the left and right sides the charge density is zero as the normals to the sides are at right angles with $\mathbf{P}$. So, $Q = \int_{V} \rho d\tau' + \int_{top} \sigma_{top} da' + \int_{bottom} \sigma_{bottom} da' = \int_{V}-\alpha P d\tau' + \int_{top}P(1+\alpha t) da' + \int_{bottom} - Pda' = \alpha PAt + PA - \alpha PA t - PA = 0$
\begin{figure}[htbp]
    \centering
    {\includegraphics[scale=0.4]{10-3.png}}
    \caption{Drawing for Wangsness 10-3}
\end{figure}
\subsubsection{Wangsness 10-6}
We are given that $\mathbf{P} = P_0 \hat{\mathbf{k}}$. Now $\rho_{b} = -\nabla \cdot \mathbf{P}$, and as $\mathbf{P}$ is uniform, $-\nabla \cdot \mathbf{P} = 0$. Thus $\rho_b = 0$. $\sigma_b = \mathbf{P}\cdot \hat{\mathbf{n}} = P_0 \hat{\mathbf{k}} \cdot \hat{\mathbf{n}} = P_0 \cos(\theta)$. The positive charge is thus located in the region $\theta < \frac{\pi}{2}$. So $Q_b^+ = \int_{0}^{\pi/2}\int_{0}^{2\pi} P_0 \cos(\theta) a^2 \sin(\theta) d\theta d\phi = \pi a^2 P_0$.
\begin{figure}[htbp]
    \centering
    {\includegraphics[scale=0.3]{10-6.png}}
    \caption{Drawing for Wangsness 10-6}
\end{figure}
\subsubsection{Wangsness 10-17}
Choose a spherical Gaussian surface outside the sphere concentric with the given sphere. $\int \mathbf{D}\cdot \mathbf{da} = Q_f$, so $D_o (4\pi r^2) = q$, and thus $\mathbf{D}_o = \frac{q}{4\pi r^2} \hat{\mathbf{r}}$. From $\mathbf{D} = \epsilon_0 \mathbf{E}+\mathbf{P}$, we have that $\mathbf{D}_o - \epsilon_0 \mathbf{E}_o = \mathbf{P}_o$. $\mathbf{E}_0 = \frac{q}{4\pi \epsilon_0 r^2}\hat{\mathbf{r}}$, and thus $\mathbf{P}_o = 0$. There is no dielectric outside of the sphere. Choosing a Gaussian surface inside of the sphere, we get $\int \mathbf{D}\cdot \mathbf{da} = Q_f$, for $D(4\pi r^2) = q$, and thus $\mathbf{D}_i = \frac{q}{4\pi r^2} \hat{\mathbf{r}}$. $\mathbf{E}_i = \frac{\mathbf{D}_i}{\epsilon} = \frac{\mathbf{D}_i}{\kappa_e \epsilon_0} = \frac{q}{4\pi \kappa_{e} \epsilon_0 r^2}\hat{\mathbf{r}}$. So $\mathbf{P}_i = \mathbf{D}_i - \epsilon_0 \mathbf{E}_i = (1-\frac{1}{\kappa_{e}}) \frac{q}{4\pi r^2} \hat{\mathbf{r}}$. Finally, $Q_b^{surface} = \int_{S} \sigma_{b} da' = \iint \mathbf{P}\cdot \hat{\mathbf{n}} da' = \int_{0}^{\pi} \int_{0}^{2\pi} \frac{\kappa_e-1}{\kappa_e} \frac{q}{4\pi} \sin(\theta) d\theta d\phi = \frac{\kappa_e-1}{\kappa_e} q$.
\subsubsection{Wangsness 10-18}
$\oint \mathbf{D} \cdot \mathbf{da} = q$. $\mathbf{D} = \frac{q}{4\pi r^2} \hat{\mathbf{r}}$ for all $r$ inside the cavity or in the dielectric. $\rho_b = 0$ since $\rho_f = 0$ in the dielectric. In the dielectric $\mathbf{E} = \frac{\mathbf{D}}{\epsilon} = \frac{\mathbf{D}}{\kappa_e \epsilon_0}$, so $\mathbf{E} = \frac{q}{4\pi \kappa_e \epsilon_0 r^2}\hat{\mathbf{r}}$. $\mathbf{P} = \mathbf{D}- \epsilon_0 \mathbf{E} = \frac{\kappa_e-1}{\kappa_e} \frac{q}{4\pi r^2} \hat{\mathbf{r}}$ at the surface of the cavity $r=a$. $\sigma_b = \mathbf{P}\cdot \hat{\mathbf{n}} = \frac{\kappa_e-1}{\kappa_e} \frac{q}{4\pi a^2} \hat{\mathbf{r}}\cdot (-\hat{\mathbf{r}}) = - \frac{\kappa_e-1}{\kappa_e} \frac{q}{4\pi a^2}$. $Q_b^{cavity} = \int \sigma da = - \frac{\kappa_e-1}{\kappa_e}q$.
\subsubsection{Wangsness 10-25}
$\kappa_e(x) = \alpha+\beta x$ (The dielectric constant varies linearly with $x$. $\alpha$ and $\beta$ are constants). Find $\mathbf{D}$ between the plates. $\int_{Gaussian Surface} \mathbf{D}\cdot \mathbf{da} = Q_f^{enc}$ (D is uniform between plates). $D\Delta a = Q_f^{enc}$, and thus $D = \frac{Q_f^{enc}}{\Delta a} = \sigma = \frac{Q}{A}$, where $Q$ is the total charge of the plate and $A$ is the area of the plate. $E = \frac{D}{\epsilon} = \frac{Q}{\kappa \epsilon_0 A} = \frac{Q}{\epsilon_0 A(\alpha + \beta x)}$. At $x=0$, $\kappa_e = \kappa_{e_1}$, so $\alpha+\beta(0) = \kappa_{e_1}$, and thus $\alpha = \kappa_{e_1}$. At $x=d$, $\kappa_{e} = \kappa_{e_2}$, and so $\beta = \frac{\kappa_{e_2}-\kappa_{e_1}}{d}$. The potential difference between the plates is $\Delta \phi = -\int_{-}^{+} \mathbf{E}\cdot \mathbf{d\ell} = \int_{+}^{-} Edx = \frac{Q}{\epsilon_0 A} \int_{0}^{d} \frac{dx}{\alpha+\beta x} = \frac{Q}{\epsilon A} \frac{1}{\beta} \ln(\alpha+\beta x)\big|_{0}^{d} = \frac{Q}{\epsilon_0 A\beta} \ln(\frac{\alpha+\beta d}{\alpha}) = \frac{Q}{\epsilon_0 A\beta} \ln(\frac{\kappa_{e_2}}{\kappa_{e_1}}) = \frac{Q}{C}$. Hence $C = \frac{\epsilon_0 A\beta}{\ln(\frac{\kappa_{e_2}}{\kappa_{e_1}})} = \frac{(\kappa_{e_2}-\kappa_{e_1})\epsilon_0 A}{d\ln(\frac{\kappa_{e_2}}{\kappa_{e_1}})}$
\begin{figure}[H]
  \begin{subfigure}[b]{0.49\textwidth}
     \centering
    \includegraphics[width=\textwidth]{10-25.png}
    \caption{Drawing for Wangsness 10-25}
  \end{subfigure}
  \begin{subfigure}[b]{0.49\textwidth}
    \centering
    \includegraphics[width=\textwidth]{12-3.png}
    \caption{Drawing for Wangsness 12-3}
  \end{subfigure}
\end{figure}
\subsubsection{Wangsness 10-27}
$\kappa = \kappa_{e_1}$ for $a\leq \rho < \rho_0$, $\kappa = \kappa_{e_2}$ for $\rho_0 \leq \rho \leq b$. First get $D$ by assuming a charge per unit length $\lambda $on the inner cylinder and $-\lambda$ on the outer. $\int \mathbf{D}\cdot \mathbf{da} = Q_{f}^{enc} = D(2\pi \rho L) = \lambda L$. So $\mathbf{D} = \frac{\lambda}{2\pi \rho} \hat{\boldsymbol{\uprho}}$. $\Delta \phi = -\int_{-}^{+} \mathbf{E} \cdot \mathbf{d\ell} = \int_{a}^{b} \frac{\lambda}{2\pi \rho \epsilon}d\rho = \int_{a}^{\rho_0} \frac{\lambda}{2\pi \epsilon_0 \kappa_{e_1}\rho}d\rho + \int_{\rho_0}^{b} \frac{\lambda}{2\pi \epsilon_0 \kappa_{e_2}\rho}d\rho = \frac{\lambda}{2\pi \epsilon_0}\big[\frac{1}{\kappa_{e_1}}\ln(\frac{\rho_0}{a}) + \frac{1}{\kappa_{e_2}}\ln(\frac{b}{\rho_0})\big]$. From $\Delta \phi = \frac{Q}{C}$, and $Q=\lambda L$, we get $C = \frac{2\pi \epsilon_0 L}{\frac{1}{\kappa_{e_1}}\ln(\frac{\rho_0}{a}) + \frac{1}{\kappa_{e_2}}\ln(\frac{b}{\rho_0})}$
\subsection{Homework XI}
\subsubsection{Wangsness 12-3}
$\mathbf{J} = \rho \mathbf{v}$. $\rho = \frac{Q}{\frac{4}{3}\pi a^3} = \frac{3q}{4\pi a^3}$. $\mathbf{u} = \mathbf{\omega}\times\mathbf{r} = \omega \hat{\mathbf{z}} \times r \hat{\mathbf{r}} = \omega r \sin(\theta) \hat{\boldsymbol{\upvarphi}}$. So, we have that $\mathbf{J} = \frac{3Q}{4\pi a^3} \omega r \sin(\theta) \hat{\boldsymbol{\upvarphi}}$. $\mathbf{da} = rdrd\theta \hat{\boldsymbol{\upvarphi}}$, so $I = \int \mathbf{J} \cdot \mathbf{da} = \frac{3Q \omega}{4\pi a^3} \int_{0}^{\pi} \int_{0}^{a} r^2\sin(\theta)drd\theta = \frac{Q\omega}{2\pi}$
\subsubsection{Wangsness 13-4}
We will calculate the force exerted by $C'$ on $C$. $\mathbf{F}_{C'\rightarrow C} = \frac{\mu_0}{4\pi} \oint_{C} \oint_{C'} \frac{I \mathbf{d\ell}\times (I' \mathbf{d\ell}'\times \hat{\mathbf{r}})}{R^2}$. We use the $BAC-CAB$ rule: $\mathbf{A}\times(\mathbf{B}\times \mathbf{C}) = \mathbf{B}(\mathbf{A}\cdot \mathbf{C}) - \mathbf{C}(\mathbf{A}\cdot \mathbf{B})$. We can rewrite the previous integral as  $\mathbf{F}_{C'\rightarrow C} = -\frac{\mu_0 II'}{4\pi} \oint_{C} \oint_{C'} \big[ \mathbf{d\ell}'\cdot(\mathbf{d\ell}\times \frac{\hat{\mathbf{r}}}{R^2}) - \frac{\hat{\mathbf{r}}}{R^2} \mathbf{d\ell}\cdot \mathbf{d\ell}\big]$. Recall that $\nabla(\frac{1}{R}) = \frac{\hat{\mathbf{r}}}{R^2}$. Using this, we have $\mathbf{F}_{C'\rightarrow C} = -\frac{\mu_0 II'}{4\pi} \oint_{C}\oint_{C'} \mathbf{d\ell'}\big[ \mathbf{d\ell}\cdot \nabla(\frac{1}{R})- \frac{\hat{\mathbf{r}}}{R^2} \mathbf{d\ell'} \cdot \mathbf{d\ell}\big]$. From the fundamental theorem of gradients, $\oint \nabla(f) \cdot \mathbf{d\ell} = 0$ for any function $f$. Thus $\oint \nabla(\frac{1}{R}) \cdot \mathbf{d\ell} = 0$. From this we have $\mathbf{F}_{C\rightarrow C'} = -\frac{\mu_0 II'}{4\pi} \oint_{C}\oint_{C'} \frac{\hat{\mathbf{r}}}{R^2} \mathbf{d\ell}'\cdot \mathbf{d\ell}$. We now compute this integral along all four paths of the problem. $\mathbf{d\ell}' = dz' \hat{\mathbf{z}}$ for all paths. Along path $I$, $\mathbf{r} = \hat{\mathbf{x}}d+\hat{\mathbf{z}}z$, $\mathbf{d\ell} = \hat{\mathbf{z}}dz$. Along path $III$, $\mathbf{r} = \hat{\mathbf{x}}(a+d)+\hat{\mathbf{z}}z$, $\mathbf{d\ell} = \hat{\mathbf{z}}dz$. Along paths $II$ and $IV$, $\mathbf{d\ell}\cdot \mathbf{d\ell}' = 0$. Piecing this together, $\mathbf{F}_{C\rightarrow C'} = -\frac{\mu_0 II'}{4\pi}\int_{0}^{b} \int_{-\infty}^{\infty} \frac{\hat{\mathbf{x}}d+\hat{\mathbf{z}}(z-z')}{\big(d^2+(z-z')^2\big)^{3/2}}dz'dz - \frac{\mu_0 II'}{4\pi} \int_{b}^{0} \int_{-\infty}^{\infty} \frac{\hat{\mathbf{x}}(d+a)+\hat{\mathbf{z}}(z-z')}{\big((d+a)^2+(z-z')^2\big)^{3/2}}dz'dz$. Making the substitution $t=z'-z$, we get an integral of the form $\int_{-\infty}^{\infty} \frac{t+z'}{(A+t^2)^{3/2}}dt$. This is an odd function that is integrated over symmetric bounds, and thus the integral is zero. The only part left is the $\hat{\mathbf{x}}$ contribution. Evaluating this integral, we get $\mathbf{F}_{C\rightarrow C'} = -\frac{\mu_0 II' ab}{2\pi d(a+d)}\hat{\mathbf{x}}$.
\begin{figure}[htbp]
    \centering
    {\includegraphics[scale=0.4]{13-4.png}}
    \caption{Drawing for Wangsness 13-4}
\end{figure}
\subsubsection{Wangsness 14-7}
$\mathbf{R} = \mathbf{r}-\mathbf{r}'$, where $\mathbf{r} = z\hat{\mathbf{z}}$ and $\mathbf{r}' = a\cos(\phi')\hat{\mathbf{x}}+a\sin(\phi')\hat{\mathbf{y}}$. We have $\boldsymbol{d\ell}' = ad\varphi' \hat{\boldsymbol{\upvarphi}}$. Putting this together, we have $\mathbf{R} = z\hat{\mathbf{z}} - a(\cos(\phi')\hat{\mathbf{x}}+\sin(\phi')\hat{\mathbf{y}})$. So:
\begin{align*}
    \mathbf{B} &= \frac{\mu_0 I'}{4\pi}\int \frac{\boldsymbol{d\ell}\times \mathbf{R}}{R^3} = \frac{\mu_0I'}{4\pi} \int_{-\alpha}^{\alpha} \frac{ad\phi' \hat{\boldsymbol{\upvarphi}}\times (z\hat{\mathbf{z}}-a\cos(\phi')\hat{\mathbf{x}}-a\sin(\phi')\hat{\mathbf{y}})}{(z^2+a^2)^{3/2}}\\
    &= \frac{\mu_0 I'a}{4\pi(z^2+a^2)^{3/2}}\int_{-\alpha}^{\alpha} (-\sin(\phi')\hat{\mathbf{x}}+\cos(\phi')\hat{\mathbf{y}})\times (-a\cos(\phi')\hat{\mathbf{x}}-a\sin(\phi')\hat{\mathbf{y}}+z\hat{\mathbf{z}})d\phi'\\
    &= \frac{\mu_0 I'a}{4\pi (z^2+a^2)^{3/2}}\int_{-\alpha}^{\alpha} (z\cos(\phi')\hat{\mathbf{x}}+z\sin(\phi')\hat{\mathbf{y}}+a\hat{\mathbf{z}})d\phi'
\end{align*}
Sine is an odd function, and the limit is over a symmetric interval, and thus the $\hat{\mathbf{y}}$ component is zero. So we have:
\begin{equation*}
    \mathbf{B} = \frac{\mu_0 I' a}{2\pi (z^2+a^2)^{3/2}}\big(z\sin(\alpha)\hat{\mathbf{x}}+a\alpha \hat{\mathbf{z}}\big)    
\end{equation*}
\subsubsection{Wangsness 14-15}
The force on $q$ is given by $\mathbf{F} = q\mathbf{v}\times \mathbf{B}$. We first get $\mathbf{B}$ at $q$. $\mathbf{B} = \frac{\mu_0}{4\pi} \int \frac{I' d\ell' \times \hat{\mathbf{r}}}{R^2}$. For this problem, $\mathbf{R} = -\rho' \hat{\boldsymbol{\uprho}}$. We need only compute the integral along paths $I$ and $III$, for along $II$ and $IV$ we have that $\mathbf{d\ell}$ and $\mathbf{R}$ are parallel. So, we have $\mathbf{B} = \frac{\mu_0}{4\pi} \int_{0}^{\pi} \frac{I'(-ad\phi' \hat{\boldsymbol{\upvarphi}})\times (-a\hat{\boldsymbol{\uprho}})}{a^3}+ \frac{\mu_0}{4\pi} \int_{0}^{\pi} \frac{I'(bd\phi' \hat{\boldsymbol{\upvarphi}})\times (-b\hat{\boldsymbol{\uprho}})}{b^3} = \frac{\mu_0 I'}{4} \frac{b-a}{ab} \hat{\mathbf{z}}$. The force is $\mathbf{F} = qv\hat{\mathbf{y}} \times \frac{\mu_0 I}{4} \frac{b-a}{ab} \hat{\mathbf{z}} = \frac{qv\mu_0 I'}{4} \frac{b-a}{ab} \hat{\mathbf{x}}$
\begin{figure}[htbp]
    \centering
    {\includegraphics[scale=0.4]{14-15.png}}
    \caption{Drawing for Wangsness 14-15}
\end{figure}
\subsection{Homework XII}
\subsubsection{Wangsness 15-7}
For $\rho\leq a$, path $(1)$ has $\oint \mathbf{B}\cdot \mathbf{d\ell}= \mu_0 I_{enc}$, where $I_{enc} = I\frac{\rho^2}{a^2}$. So $\mathbf{B} = \frac{\mu_0 I\rho}{2\pi a^2} \hat{\boldsymbol{\upvarphi}}$. For $a\leq \rho \leq b$, $I_{enc} = I$. So $B = \frac{\mu_0 I}{2\pi \rho} \hat{\boldsymbol{\upvarphi}}$. For $b\leq \rho \leq c$, $I_{enc} = I =I\frac{\rho^2-b^2}{c^2-b^2} = I\frac{c^2-\rho^2}{c^2-b^2}$. So $\mathbf{B} = \frac{\mu_0 I}{2\pi \rho} \frac{c^2-\rho^2}{c^2-b^2}\hat{\boldsymbol{\upvarphi}}$. FInally, or $\rho \geq c$, $I_{enc} = 0$, so $\mathbf{B} = 0$.
\begin{figure}[htbp]
    \centering
    {\includegraphics[scale=0.4]{15-7.png}}
    \caption{Drawing for Wangsness 15-7}
\end{figure}
\subsubsection{Wangsness 15-8}
\begin{equation*}
    \mathbf{B} = \begin{cases} 0, & \rho < a \\ \frac{\mu_0 I}{2\pi \rho}\frac{\rho^2-a^2}{b^2-a^2}\hat{\boldsymbol{\upvarphi}}, & a<\rho < b \\ \frac{\mu_0 I}{2\pi \rho} \hat{\boldsymbol{\upvarphi}}, & \rho>b\end{cases}    
\end{equation*}
By definition, $\mu_0 \mathbf{J} = \nabla \times \mathbf{B}$. So:
\begin{equation*}
    \mathbf{J} = \begin{cases} 0, & \rho<a\\ \frac{I}{\pi(b^2-a^2)}, & a<\rho < b\\ 0, & \rho>b \end{cases}    
\end{equation*}
The current $I$ i distributed uniformly over the volume between two coaxial cylinders of inner radius $a$ and outer radius $b$ in the direction of the cylinder axis.
\subsubsection{Wangsness 16-10}
The field point is on the $z-$axis. $\mathbf{r} = z\hat{\mathbf{z}}$, $\mathbf{r}' =  a\cos(\phi')\hat{\mathbf{x}}+a\sin(\phi')\hat{\mathbf{y}}$. $\mathbf{d\ell}' = \mathbf{dr}' = ad\phi' \hat{\boldsymbol{\upvarphi}} = ad\phi' (-\sin(\phi')\hat{\mathbf{x}}+\cos(\phi')\hat{\mathbf{y}})$. $\mathbf{A} = \frac{\mu_0 I'}{4\pi} \int \frac{\mathbf{d\ell}'}{R} = \frac{\mu_0 I'}{4\pi} \frac{a}{\sqrt{a^2+z^2}}\int_{-\alpha}^{\alpha} (-\sin(\phi')\hat{\mathbf{x}}+\cos(\phi')\hat{\mathbf{y}})d\phi' = \frac{\mu_0 I'a}{2\pi \sqrt{a^2+z^2}}\sin(\alpha)\hat{\mathbf{y}}$. To find $\mathbf{B}$ from $\mathbf{A}$, we need to evaluate $\nabla \times \mathbf{A}$. We don't know about $\mathbf{A}$ for a general point, and thus we can't evaluate the $x$ and $y$ derivatives.
\subsection{Homework XIII}
\subsubsection{Wangsness 17-3}
The $\mathbf{B}$ field associated with $I$ is $\mathbf{B} = \frac{\mu_0 I}{2\pi \rho} \hat{\boldsymbol{\upvarphi}}$. In the plane of the paper, $\phi$ is into the paper. $\Phi = \int \mathbf{B}\cdot \mathbf{da} = \int \frac{\mu_0 I}{2\pi \rho} \cdot b d\rho \hat{\boldsymbol{\upvarphi}} = \frac{\mu_0Ib}{2\pi} \int_{d}^{d+a} \frac{d\rho}{\rho}= \frac{\mu_0 Ib}{2\pi} \ln(\frac{d+a}{d}) = \frac{\mu_0 bI_0 e^{-\lambda t}}{2\pi} \ln(\frac{d+a}{d}) = \frac{\mu_0 I_0 \lambda b}{2\pi} \ln(\frac{d+a}{d})e^{-\lambda t}$. The induced current is clockwise around the loop to produce a field which goes into the paper to counteract the decreasing $\mathbf{B}$ due to $I_0$.
\subsubsection{Wangsness 17-4}
The $\mathbf{B}-$field at distance $\rho$ from the wire at points in the plane of the paper is $\mathbf{B} = \frac{\mu_0 I}{2\pi \rho} \hat{\mathbf{y}}$. The flux of $\mathbf{B}$ through the loop is $\Phi = \int \mathbf{B}\cdot \mathbf{da} = \iint \frac{\mu_0 I}{2\pi \rho}\rho d\theta d\rho$. We have $\rho = b+r\cos(\theta)$. So $\Phi = \frac{\mu_0 I}{2\pi} \int_{0}^{a} \int_{0}^{2\pi} \frac{r d\theta dr}{b+r\cos(\theta)} = \frac{\mu_0 I}{2\pi} \int_{0}^{a} \frac{2r}{\sqrt{b^2-r^2}}\tan^{-1}\big[\frac{\sqrt{b^2-r^2}\tan(\theta/2)}{b+r}\big]_{0}^{2\pi} \Rightarrow \tan^{-1}\big[\frac{\sqrt{b^2-r^2}}{b+r}\tan(\pi)\big] - \tan^{-1}\big[ \frac{\sqrt{b^2-r^2}}{b+r}\tan(0)\big]$. So $\Phi = \mu_0 I\big[b-\sqrt{b^2-a^2}\big]$. The loop moves with constant speed $v$ along the $x-$axis away from the current $I$, $v = \frac{db}{dt}$. So $\xi = -\frac{d\Phi}{dt} = -\mu_0 I \frac{d}{dt}\big[b-\sqrt{b^2-a^2}\big] = -\mu_0 I\big[ v-\frac{bv}{\sqrt{b^2-a^2}}\big] = \mu_0 NIv\big[ \frac{b}{\sqrt{b^2-a^2}}-1\big]$. The current will be clockwise trying to increase the flux which is decreasing due to motion away from the wire.
\begin{figure}[htbp]
    \centering
    {\includegraphics[scale=0.4]{17-4.png}}
    \caption[Drawing for Wangsness 17-4]{Drawing for Wangness 17-4}
\end{figure}
\subsubsection{Wangsness 17-19}
$\Phi_{12} = IM_{12}$. The flux due to $1$ through $2$ is $\Phi_{12} = \int \mathbf{B}_1 \cdot \mathbf{da}_2$. $\mathbf{B}_1 = \frac{\mu_0 I}{2\pi} \big( \frac{1}{\rho+d}- \frac{1}{\rho+d+D}\big)$. So we have that $\Phi_{12} = \int_{0}^{a} \frac{\mu_0 I}{2\pi} \big(\frac{1}{\rho+d}- \frac{1}{\rho+d+D}\big) bd\rho = \frac{\mu_0 Ib}{2\pi}\big[ \ln(\frac{a+d}{a+d+D}) - \ln(\frac{d}{d+D})\big]$. Thus, we have $M = \frac{\mu_0 b}{2\pi} \ln\big(\frac{a+d}{d}\big)$
\subsubsection{Wangsness 17-20}
The field inside the toroid is $\mathbf{B} = \frac{\mu_0 NI}{2\pi \rho} \hat{\boldsymbol{\upvarphi}}$. The flux through a single turn is $\Phi^1 = \frac{\mu_0 NI}{2\pi} \int_{0}^{a} \int_{0}^{2\pi} \frac{r}{b+r\cos(\theta)}d\theta dr$. We've done this integral before, and we get $\Phi^1= \mu_0 NI\big[b-\sqrt{b^2-r^2}\big]$. $\Phi = N\Phi^1$. $L = \frac{\Phi}{I} = \frac{\mu_0 N^2 I}{I} \big[b-\sqrt{b^2-r^2}\big] = \mu_0 N^2 \big[b-\sqrt{b^2-r^2}\big]$
\section{Exams}
\subsection{Exam I}
\subsubsection{Question I}
Give the vector field field $\mathbf{A} = c\hat{\boldsymbol{\uptheta}}$, where $c$ is a constant, find $\nabla \times \mathbf{A}$. Is this a conservative vector field? Explain.
\subsubsection{Question II}
Verify the Divergence Theorem for $\mathbf{A}$ given in problem 1 in spherical coordinates for a hemisphere of radius $a_0$ resting on the $xy-plane$ with the center of the flat base of the hemisphere at the origin and the symmetry axis of the hemisphere along the positive $z-axis$.
\subsubsection{Question III}
A semicircular charged line of radius $a$ carries uniform linear charge density $\lambda$. It has the equation $x^2+y^2 = a^2$, $x\geq 0$, $z=0$. That is, the half circle resting on the $xy-plane$ of radius $a$. Find the electring field at a point $P$ on the $z$ axis a distance $z$ from the origin.
\subsection{Exam II}
\subsubsection{Question I}
A conducting sphere of radius $a$, centered at the origin carries charge $Q_1$. This sphere is surrounded by a hollow concentric conducting spherical shell of inner radius $b$ and outer radius $c$ with $a<b<c$. The outer hollow conducting shell caries a total charge $Q_2$. 
\begin{enumerate}
    \item What is the electric field everywhere?
    \item What is the potential everywhere, assuming $\underset{r\rightarrow \infty} \lim \phi(r) = 0$?
    \item How much charge is on the inner and outer surfaces of the conducting shell at $r=b$ and $r=c$?
\end{enumerate}
\subsubsection{Question II}
The outer conductor of problem I is now grounded.
\begin{enumerate}
    \item What is the electric field everywhere?
    \item What is the potential everywhere?
    \item How much charge is on the surfaces at $r=b$ and $r=c$?
    \item What is the capacitance of the system of conductors?
    \item Calculate the electrostatic potential energy of the configuration assuming the energy resides in the charges.
    \item Calculate the electrostatic potential energy of the configuration assuming the energy resides in the electric field.
\end{enumerate}
\subsubsection{Question III}
A semicircular arc of radius $a$ in the $y-z$ plane with center on the $y-axis$ at the origin and the top of the arc on the positive $y-axis$ carries linear charge density $\lambda = \lambda_0 \cos(\theta')$, where $\lambda_0$ is constant and $\theta'$ is measured with respected to the positive $z-axis$.
\begin{enumerate}
    \item What is the electric monopole moment of this distribution?
    \item What is the electric dipole moment of this distribution?
    \item What is the electric potential at a distance $r$ from the origin for this distribution where $r>a$, accurate to order $\frac{1}{r^2}$?
\end{enumerate}
\subsection{Exam III}
\subsubsection{Question I}
The electric field in a spherical region of space of radius $a$ is given by $\mathbf{E} = E_0 \frac{r^2}{a^2}\hat{\mathbf{r}}$ for $r< a$, where $E_0$ is a constant. This region is surrounded concentrically by a grounded conducting spherical shell of inner radius $b$ and outer radius $c$ with $a<b<c$. There is no charge in the region $a<r<b$. 
\begin{enumerate}
    \item What is the electric charge density in the region $r<a$?
    \item Wher is the electric field in the region $a<r<b$?
    \item How much charge is on the surfaces at $r=b$ and $r=c$?
    \item What is the electric field for $r>c$?
    \item What is the electric potential $\phi$ at $r=0$ assuming that ground potential is $\phi = 0$.
\end{enumerate}
\subsubsection{Question II}
A capacitor $C_{1}$ is charged to a potential difference $\Delta \phi$ between its plates. A second capacitor $C_{2}$ is uncharged. One plate of $C_2$ is now connected to a plate of $C_1$ by a conductor of negligible capacitance, the remaining plates are similarly connected. 
\begin{enumerate}
    \item For the resultant equilibrium state, find the charge on each capacitor and the potential difference $\Delta \phi$ between their respective plates.
    \item Compare the energy stored in capacitor $C_1$ before connecting it to $C_2$, to the energy of the combination after connected them. Are these energies the same? If not, which is larger and where did any additional energy come from, or where did any lost energy go?
\end{enumerate}
\subsubsection{Question III}
Find the electric dipole moment of an hourglass configuration of charge consisting of two identical right circular cones of radius $a$ and height $a$ with symmetry axes aligned apex to apex along the $z-$axis with the apexes touching at the origin. The top cone has charge density $\sigma_0$ on its surface, and the bottom cone has charge density $-\sigma_0$ on its surface.
\subsection{Practice Final Exam}
\subsubsection{Problem I}
The electric field in a region of space is given in spherical coordinates as $\mathbf{E} = cr\hat{\mathbf{r}}$, where $c$ is constant. 
\begin{enumerate}
    \item Find the charge density at a point $(r,\theta,\phi)$
    \item Find the total charge inside a sphere of radius $a$ centered at the origin.
\end{enumerate}
\subsubsection{Problem II}
A battery is used to charge an ideal parallel plate capacitor to a potential difference $\Delta \phi = V_0$. The battery is then disconnected. The separation between the plates is now increasing from $d$ to $\alpha d$, where $\alpha >1$. The area of the plates is $A$.
\begin{enumerate}
    \item What is the ratio of the new energy to the original energy>
    \item Is the energy increases or decreased?
    \item Where does this energy come from or go to?
    \item Compute the change in energy $\Delta U_e$ expressing your answer in terms of the given quantities $V_0,d,A,\alpha$ and fundamental constants.
\end{enumerate}
\subsubsection{Problem III}
A dielectric sphere of radius $a$ and permittivity $\varepsilon$ contains a free charge density. $\rho_f = cr$, where $c$ is a constant. The sphere is centered at the origin. Find the electric potential at the center of the sphere assuming that the potential is zero at an infinite distance from the center.
\subsubsection{Problem IV}
A thick slab extending from $z=-a$ to $z=a$ carries a uniform vlume current density $\mathbf{J} = J_0 \hat{\mathbf{x}}$. The slab is infinite in the $xy-$plane. Find the magnetic field $B$ as a function of $z$ inside and outside the slab. Plot $B$ as a function of $z$ for $-b<z<b$ where $b>a$.
\subsubsection{Problem V}
An ideal long solenoid of radius $a$, carrying $n$ turns per unit length, is looped by a wire with resistance $R$. 
\begin{enumerate}
    \item If the current in the solenoid is increasing at a constant rate $\frac{dI}{dt} = k$, what current flows in the lopp, and which way (Left or right) does it pass through the resistor?
    \item If the current $I$ in the solenoid is constant but the solenoid is pulled out of the loop and reinserted in the opposite direction, what total charge passes through the resistor?
\end{enumerate}
\subsection{Final Exam}
\subsubsection{Problem I}
\begin{enumerate}
    \item Write down Maxwell's Equations in differential form.
    \item Convert them to integral form and show derivations.
    \item Name each equation.
\end{enumerate}
\begin{proof}[Solution]
\
\begin{enumerate}
\item Gauss' Law: $\nabla \cdot \mathbf{E} = \frac{\rho}{\epsilon_0}\Rightarrow\frac{Q_{encl}}{\epsilon_0}=\iiint_{V} \frac{\rho}{\epsilon_0}d\tau=\iiint_{V} \big(\nabla \cdot \mathbf{E}\big) d\tau = \oiint_{\partial V} \mathbf{E}\cdot \mathbf{da}$
\item Faraday's Law: $\nabla \times \mathbf{E} = -\frac{\partial \mathbf{B}}{\partial t}\Rightarrow-\frac{d \Phi_{B}}{dt} = \iint_{S} -\frac{\partial \mathbf{B}}{\partial t}da = \iint_{S} \big(\nabla \times \mathbf{E}\big)da = \oint_{\partial S}\mathbf{E}\cdot \mathbf{d\ell}$
\item Gauss' Law of Magnetism: $\nabla \cdot \mathbf{B} = 0\Rightarrow 0 = \iiint_{V} \big(\nabla \cdot\mathbf{B}\big)d\tau = \oiint_{\partial V} \mathbf{B}\cdot \mathbf{da}$
\item Ampere's Law: $\nabla \times \mathbf{B} = \mu_0 \mathbf{J} + \mu_0 \epsilon_0 \frac{\partial \mathbf{E}}{\partial t}\Rightarrow\mu_0 I_{encl}+ \mu_0 \epsilon_0 \frac{d\Phi_{E}}{dt} = \iint_{S}\big(\mu_0 \mathbf{J} + \mu_0 \epsilon_0 \frac{\partial \mathbf{E}}{\partial t}\big)da = \iint_{S}\big(\nabla \times \mathbf{B}\big)da = \oint_{\partial S}\mathbf{B}\cdot \mathbf{d\ell}$
\end{enumerate}
\end{proof}
\subsubsection{Problem II}
A conduction sphere of radius $a$ carries a charge $Q_{1}$. It is surrounded by a conducting spherical shell of inner radius $b$ and outer radius $c$ with $a<b<c$. The charge on the conducting shell is $Q_{2}$. The region between the conductors $a<r<b$ is filled with linear isotropic dielectric of permittivity $\varepsilon$. Find the following in the regions $r<a.a<r<b.b<r.b<r<c.c<r$:
\begin{enumerate}
    \item The electric displacement $\mathbf{D}$
    \item The electric field $\mathbf{E}$
    \item The polarization vector $\mathbf{P}$
    \item Find the free charge on the conductors at $r=a,b,c$.
    \item The bound volume charge in the dielectric.
    \item The bound surface charge density at the inner and outer surfaces of the dielectric.
    \item The electric potential at the origin assuming the potential is zero as $r$ goes to infinity.
\end{enumerate}
\begin{proof}[Solution]
\
\begin{enumerate}
\begin{multicols}{2}
    \item $\mathbf{D}=\begin{cases}\mathbf{0}&r<a\\ \frac{Q_{1}}{4\pi r^{2}} &a<r<b\\\mathbf{0} & b<r<c\\ \frac{Q_{1}+Q_{2}}{4\pi r^{2}} & c<r \end{cases}$
    \item $\mathbf{E}=\begin{cases}\mathbf{0}&r<a\\ \frac{Q_{1}}{4\pi\epsilon_{0} r^{2}} & a<r<b\\ \mathbf{0} & b<r<c\\ \frac{Q_{1}+Q_{2}}{4\pi\epsilon_{0}r^{2}} & c<r\end{cases}$
    \item $\mathbf{P}=\begin{cases}\mathbf{0}&r<a\\ \frac{Q_{1}}{4\pi r^{2}}(1-\frac{\epsilon_{0}}{\varepsilon}) & a<r<b\\ \mathbf{0} & b<r<c\\ \mathbf{0} & c<r \end{cases}$
    \item $Q = \begin{cases} Q_1 & r=a\\ -Q_1 & r=b\\ Q_1+Q_2 & r=c\end{cases}$
\end{multicols}
\begin{multicols}{2}
    \item $\rho_b = \nabla \cdot \mathbf{P}$, $\rho_b = 0$.
    \item $\sigma_{b,a}=-\frac{Q_1}{4\pi a^2}\big(1-\frac{\epsilon_0}{\varepsilon}\big)$, $\sigma_{b,b}=\frac{Q_1}{4\pi b^2}\big(1-\frac{\epsilon_0}{\varepsilon}\big)$.
    \end{multicols}
    \item $\phi = -\int_{0}^{\infty}\mathbf{E}\cdot \mathbf{d\ell} = \int_{c}^{\infty} \mathbf{E}\cdot \mathbf{d\ell}+\int_{b}^{c}\mathbf{E}\cdot \mathbf{d\ell}+\int_{a}^{b}\mathbf{E}\cdot \mathbf{d\ell} + \int_{0}^{a} \mathbf{E}\cdot \mathbf{d\ell} = \frac{Q_1+Q_2}{4\pi \epsilon_0 c}+\frac{Q_1}{4\pi \epsilon_0 a}-\frac{Q_1}{4\pi \epsilon_0 b}$.
\end{enumerate}
\end{proof}
\subsubsection{Problem III}
A sphere of radius $a$ carries charge density $\rho = \rho_0(r/a)$, where $\rho_0$ is a constant. Find the work done to assemble the charge distribution.
\begin{proof}[Solution]
We find $\mathbf{E}$ inside and outside using Gauss' law.
\begin{equation*}
\oiint_{S} \mathbf{E}\cdot \mathbf{da} = \frac{Q_{encl}}{\epsilon_0} = \int_{0}^{r}\int_{0}^{\pi} \int_{0}^{2\pi} \rho_0 \frac{r}{a}r^2\sin(\theta)d\varphi d\theta dr = \frac{4\pi \rho_0}{a \epsilon_0}\frac{r^4}{4} = E(4\pi r^2).
\end{equation*}
So $\mathbf{E} = \frac{\rho_0 r^2}{4a\epsilon_0}\hat{\mathbf{r}}$. Outside we have $\oiint_{S} \mathbf{E}\cdot \mathbf{da} = \int_{0}^{2\pi}\int_{0}^{\pi} \int_{0}^{a} \rho \frac{r}{a}r^2 \sin(\theta) dr d\theta d\varphi$, so $\mathbf{E} = \frac{\rho_0 a^3}{4r^2 \epsilon_0}\hat{\mathbf{r}}$. The work is
\begin{align*}
\frac{\epsilon_0}{2}\int_{All\ Space}E^2 d\tau &= \frac{\epsilon_0}{2}\int_{0}^{2\pi}\int_{0}^{\pi}\int_{0}^{\infty} E^2 r^2\sin(\theta) drd\theta d\varphi\\
&= \int_{0}^{2\pi}\int_{0}^{\pi}\int_{0}^{a} E^2r^2\sin(\theta)dr d\theta d\varphi + \int_{0}^{2\pi}\int_{0}^{\pi}\int_{a}^{\infty} E^2r^2\sin(\theta)drd\theta d\varphi\\
&= \frac{\pi \rho_0^2 a^5}{7\epsilon_0}
\end{align*}
So $\mathbf{E} = \frac{\rho_0 r^2}{4a\epsilon_0}\hat{\mathbf{r}}$
\end{proof}
\subsubsection{Problem IV}
\begin{enumerate}
    \item Could the vector field $\mathbf{F} = ax\hat{\mathbf{x}}+by\hat{\mathbf{y}}+cz\hat{\mathbf{z}}$ be a possible magnetic field, where $a+b+c\ne 0$? Explain why or why not.
    \item An electric field is given by $\mathbf{E} = ax\hat{\mathbf{y}}$, where $a$ is a constant. Is this a conservative field? Explain why or why not.
    \item Find the possible magnetic field $\mathbf{B}$ associated to $\mathbf{E}$. 
\end{enumerate}
\begin{proof}[Solution]
\
\begin{enumerate}
    \item No, for $\nabla \cdot \mathbf{F} = a+b+c \ne 0$, and therefore $\mathbf{F}$ cannot be a magnetic field.
    \item No, for $\nabla \times \mathbf{E} = a\hat{\mathbf{z}} \ne 0$, and thus $\mathbf{E}$ is not a conservative field.
    \item $\nabla \times \mathbf{E} = -\frac{\partial \mathbf{B}}{\partial t} = a\hat{\mathbf{z}}$, so $\mathbf{B} = -at\hat{\mathbf{z}}+\mathbf{B}_0$, where $\mathbf{B}_0$ is some constant vector. Here $\mathbf{B}$ is increasing with time in the $-z$ direction.
\end{enumerate}
\end{proof}
\subsubsection{Problem V}
Two infinitely long coaxial cylindrical infinitesimally thin conducting shells concentric with the $z-$axis carry oppositely directed currents of equal magnitude in the $+$ and $-$ $z-$directions. The radius of the inner shell is $a$ and that of the outer shell is $b$. What is the self-inductance of a length $\ell$ of this system?
\begin{proof}[Solution]
The flux carried by the inner shell cuts through the area of a rectangle of lenght $\ell$ and width $b-a$. So $\Phi = \int \mathbf{B}\cdot \mathbf{da} = \int_{a}^{b} \frac{\mu_0 I}{2\pi \rho}\ell d\rho = \frac{\mu_0 I\ell}{2\pi}\ln\big(\frac{b}{a}\big)$. So, $L = \frac{\Phi}{I} = \frac{\mu_0 \ell}{2\pi}\ln\big(\frac{b}{a}\big)$.
\end{proof}
\end{document}