\documentclass[crop=false,class=article,oneside]{standalone}
%----------------------------Preamble-------------------------------%
%---------------------------Packages----------------------------%
\usepackage{geometry}
\geometry{b5paper, margin=1.0in}
\usepackage[T1]{fontenc}
\usepackage{graphicx, float}            % Graphics/Images.
\usepackage{natbib}                     % For bibliographies.
\bibliographystyle{agsm}                % Bibliography style.
\usepackage[french, english]{babel}     % Language typesetting.
\usepackage[dvipsnames]{xcolor}         % Color names.
\usepackage{listings}                   % Verbatim-Like Tools.
\usepackage{mathtools, esint, mathrsfs} % amsmath and integrals.
\usepackage{amsthm, amsfonts, amssymb}  % Fonts and theorems.
\usepackage{tcolorbox}                  % Frames around theorems.
\usepackage{upgreek}                    % Non-Italic Greek.
\usepackage{fmtcount, etoolbox}         % For the \book{} command.
\usepackage[newparttoc]{titlesec}       % Formatting chapter, etc.
\usepackage{titletoc}                   % Allows \book in toc.
\usepackage[nottoc]{tocbibind}          % Bibliography in toc.
\usepackage[titles]{tocloft}            % ToC formatting.
\usepackage{pgfplots, tikz}             % Drawing/graphing tools.
\usepackage{imakeidx}                   % Used for index.
\usetikzlibrary{
    calc,                   % Calculating right angles and more.
    angles,                 % Drawing angles within triangles.
    arrows.meta,            % Latex and Stealth arrows.
    quotes,                 % Adding labels to angles.
    positioning,            % Relative positioning of nodes.
    decorations.markings,   % Adding arrows in the middle of a line.
    patterns,
    arrows
}                                       % Libraries for tikz.
\pgfplotsset{compat=1.9}                % Version of pgfplots.
\usepackage[font=scriptsize,
            labelformat=simple,
            labelsep=colon]{subcaption} % Subfigure captions.
\usepackage[font={scriptsize},
            hypcap=true,
            labelsep=colon]{caption}    % Figure captions.
\usepackage[pdftex,
            pdfauthor={Ryan Maguire},
            pdftitle={Mathematics and Physics},
            pdfsubject={Mathematics, Physics, Science},
            pdfkeywords={Mathematics, Physics, Computer Science, Biology},
            pdfproducer={LaTeX},
            pdfcreator={pdflatex}]{hyperref}
\hypersetup{
    colorlinks=true,
    linkcolor=blue,
    filecolor=magenta,
    urlcolor=Cerulean,
    citecolor=SkyBlue
}                           % Colors for hyperref.
\usepackage[toc,acronym,nogroupskip,nopostdot]{glossaries}
\usepackage{glossary-mcols}
%------------------------Theorem Styles-------------------------%
\theoremstyle{plain}
\newtheorem{theorem}{Theorem}[section]

% Define theorem style for default spacing and normal font.
\newtheoremstyle{normal}
    {\topsep}               % Amount of space above the theorem.
    {\topsep}               % Amount of space below the theorem.
    {}                      % Font used for body of theorem.
    {}                      % Measure of space to indent.
    {\bfseries}             % Font of the header of the theorem.
    {}                      % Punctuation between head and body.
    {.5em}                  % Space after theorem head.
    {}

% Italic header environment.
\newtheoremstyle{thmit}{\topsep}{\topsep}{}{}{\itshape}{}{0.5em}{}

% Define environments with italic headers.
\theoremstyle{thmit}
\newtheorem*{solution}{Solution}

% Define default environments.
\theoremstyle{normal}
\newtheorem{example}{Example}[section]
\newtheorem{definition}{Definition}[section]
\newtheorem{problem}{Problem}[section]

% Define framed environment.
\tcbuselibrary{most}
\newtcbtheorem[use counter*=theorem]{ftheorem}{Theorem}{%
    before=\par\vspace{2ex},
    boxsep=0.5\topsep,
    after=\par\vspace{2ex},
    colback=green!5,
    colframe=green!35!black,
    fonttitle=\bfseries\upshape%
}{thm}

\newtcbtheorem[auto counter, number within=section]{faxiom}{Axiom}{%
    before=\par\vspace{2ex},
    boxsep=0.5\topsep,
    after=\par\vspace{2ex},
    colback=Apricot!5,
    colframe=Apricot!35!black,
    fonttitle=\bfseries\upshape%
}{ax}

\newtcbtheorem[use counter*=definition]{fdefinition}{Definition}{%
    before=\par\vspace{2ex},
    boxsep=0.5\topsep,
    after=\par\vspace{2ex},
    colback=blue!5!white,
    colframe=blue!75!black,
    fonttitle=\bfseries\upshape%
}{def}

\newtcbtheorem[use counter*=example]{fexample}{Example}{%
    before=\par\vspace{2ex},
    boxsep=0.5\topsep,
    after=\par\vspace{2ex},
    colback=red!5!white,
    colframe=red!75!black,
    fonttitle=\bfseries\upshape%
}{ex}

\newtcbtheorem[auto counter, number within=section]{fnotation}{Notation}{%
    before=\par\vspace{2ex},
    boxsep=0.5\topsep,
    after=\par\vspace{2ex},
    colback=SeaGreen!5!white,
    colframe=SeaGreen!75!black,
    fonttitle=\bfseries\upshape%
}{not}

\newtcbtheorem[use counter*=remark]{fremark}{Remark}{%
    fonttitle=\bfseries\upshape,
    colback=Goldenrod!5!white,
    colframe=Goldenrod!75!black}{ex}

\newenvironment{bproof}{\textit{Proof.}}{\hfill$\square$}
\tcolorboxenvironment{bproof}{%
    blanker,
    breakable,
    left=3mm,
    before skip=5pt,
    after skip=10pt,
    borderline west={0.6mm}{0pt}{green!80!black}
}

\AtEndEnvironment{lexample}{$\hfill\textcolor{red}{\blacksquare}$}
\newtcbtheorem[use counter*=example]{lexample}{Example}{%
    empty,
    title={Example~\theexample},
    boxed title style={%
        empty,
        size=minimal,
        toprule=2pt,
        top=0.5\topsep,
    },
    coltitle=red,
    fonttitle=\bfseries,
    parbox=false,
    boxsep=0pt,
    before=\par\vspace{2ex},
    left=0pt,
    right=0pt,
    top=3ex,
    bottom=1ex,
    before=\par\vspace{2ex},
    after=\par\vspace{2ex},
    breakable,
    pad at break*=0mm,
    vfill before first,
    overlay unbroken={%
        \draw[red, line width=2pt]
            ([yshift=-1.2ex]title.south-|frame.west) to
            ([yshift=-1.2ex]title.south-|frame.east);
        },
    overlay first={%
        \draw[red, line width=2pt]
            ([yshift=-1.2ex]title.south-|frame.west) to
            ([yshift=-1.2ex]title.south-|frame.east);
    },
}{ex}

\AtEndEnvironment{ldefinition}{$\hfill\textcolor{Blue}{\blacksquare}$}
\newtcbtheorem[use counter*=definition]{ldefinition}{Definition}{%
    empty,
    title={Definition~\thedefinition:~{#1}},
    boxed title style={%
        empty,
        size=minimal,
        toprule=2pt,
        top=0.5\topsep,
    },
    coltitle=Blue,
    fonttitle=\bfseries,
    parbox=false,
    boxsep=0pt,
    before=\par\vspace{2ex},
    left=0pt,
    right=0pt,
    top=3ex,
    bottom=0pt,
    before=\par\vspace{2ex},
    after=\par\vspace{1ex},
    breakable,
    pad at break*=0mm,
    vfill before first,
    overlay unbroken={%
        \draw[Blue, line width=2pt]
            ([yshift=-1.2ex]title.south-|frame.west) to
            ([yshift=-1.2ex]title.south-|frame.east);
        },
    overlay first={%
        \draw[Blue, line width=2pt]
            ([yshift=-1.2ex]title.south-|frame.west) to
            ([yshift=-1.2ex]title.south-|frame.east);
    },
}{def}

\AtEndEnvironment{ltheorem}{$\hfill\textcolor{Green}{\blacksquare}$}
\newtcbtheorem[use counter*=theorem]{ltheorem}{Theorem}{%
    empty,
    title={Theorem~\thetheorem:~{#1}},
    boxed title style={%
        empty,
        size=minimal,
        toprule=2pt,
        top=0.5\topsep,
    },
    coltitle=Green,
    fonttitle=\bfseries,
    parbox=false,
    boxsep=0pt,
    before=\par\vspace{2ex},
    left=0pt,
    right=0pt,
    top=3ex,
    bottom=-1.5ex,
    breakable,
    pad at break*=0mm,
    vfill before first,
    overlay unbroken={%
        \draw[Green, line width=2pt]
            ([yshift=-1.2ex]title.south-|frame.west) to
            ([yshift=-1.2ex]title.south-|frame.east);},
    overlay first={%
        \draw[Green, line width=2pt]
            ([yshift=-1.2ex]title.south-|frame.west) to
            ([yshift=-1.2ex]title.south-|frame.east);
    }
}{thm}

%--------------------Declared Math Operators--------------------%
\DeclareMathOperator{\adjoint}{adj}         % Adjoint.
\DeclareMathOperator{\Card}{Card}           % Cardinality.
\DeclareMathOperator{\curl}{curl}           % Curl.
\DeclareMathOperator{\diam}{diam}           % Diameter.
\DeclareMathOperator{\dist}{dist}           % Distance.
\DeclareMathOperator{\Div}{div}             % Divergence.
\DeclareMathOperator{\Erf}{Erf}             % Error Function.
\DeclareMathOperator{\Erfc}{Erfc}           % Complementary Error Function.
\DeclareMathOperator{\Ext}{Ext}             % Exterior.
\DeclareMathOperator{\GCD}{GCD}             % Greatest common denominator.
\DeclareMathOperator{\grad}{grad}           % Gradient
\DeclareMathOperator{\Ima}{Im}              % Image.
\DeclareMathOperator{\Int}{Int}             % Interior.
\DeclareMathOperator{\LC}{LC}               % Leading coefficient.
\DeclareMathOperator{\LCM}{LCM}             % Least common multiple.
\DeclareMathOperator{\LM}{LM}               % Leading monomial.
\DeclareMathOperator{\LT}{LT}               % Leading term.
\DeclareMathOperator{\Mod}{mod}             % Modulus.
\DeclareMathOperator{\Mon}{Mon}             % Monomial.
\DeclareMathOperator{\multideg}{mutlideg}   % Multi-Degree (Graphs).
\DeclareMathOperator{\nul}{nul}             % Null space of operator.
\DeclareMathOperator{\Ord}{Ord}             % Ordinal of ordered set.
\DeclareMathOperator{\Prin}{Prin}           % Principal value.
\DeclareMathOperator{\proj}{proj}           % Projection.
\DeclareMathOperator{\Refl}{Refl}           % Reflection operator.
\DeclareMathOperator{\rk}{rk}               % Rank of operator.
\DeclareMathOperator{\sgn}{sgn}             % Sign of a number.
\DeclareMathOperator{\sinc}{sinc}           % Sinc function.
\DeclareMathOperator{\Span}{Span}           % Span of a set.
\DeclareMathOperator{\Spec}{Spec}           % Spectrum.
\DeclareMathOperator{\supp}{supp}           % Support
\DeclareMathOperator{\Tr}{Tr}               % Trace of matrix.
%--------------------Declared Math Symbols--------------------%
\DeclareMathSymbol{\minus}{\mathbin}{AMSa}{"39} % Unary minus sign.
%------------------------New Commands---------------------------%
\DeclarePairedDelimiter\norm{\lVert}{\rVert}
\DeclarePairedDelimiter\ceil{\lceil}{\rceil}
\DeclarePairedDelimiter\floor{\lfloor}{\rfloor}
\newcommand*\diff{\mathop{}\!\mathrm{d}}
\newcommand*\Diff[1]{\mathop{}\!\mathrm{d^#1}}
\renewcommand*{\glstextformat}[1]{\textcolor{RoyalBlue}{#1}}
\renewcommand{\glsnamefont}[1]{\textbf{#1}}
\renewcommand\labelitemii{$\circ$}
\renewcommand\thesubfigure{%
    \arabic{chapter}.\arabic{figure}.\arabic{subfigure}}
\addto\captionsenglish{\renewcommand{\figurename}{Fig.}}
\numberwithin{equation}{section}

\renewcommand{\vector}[1]{\boldsymbol{\mathrm{#1}}}

\newcommand{\uvector}[1]{\boldsymbol{\hat{\mathrm{#1}}}}
\newcommand{\topspace}[2][]{(#2,\tau_{#1})}
\newcommand{\measurespace}[2][]{(#2,\varSigma_{#1},\mu_{#1})}
\newcommand{\measurablespace}[2][]{(#2,\varSigma_{#1})}
\newcommand{\manifold}[2][]{(#2,\tau_{#1},\mathcal{A}_{#1})}
\newcommand{\tanspace}[2]{T_{#1}{#2}}
\newcommand{\cotanspace}[2]{T_{#1}^{*}{#2}}
\newcommand{\Ckspace}[3][\mathbb{R}]{C^{#2}(#3,#1)}
\newcommand{\funcspace}[2][\mathbb{R}]{\mathcal{F}(#2,#1)}
\newcommand{\smoothvecf}[1]{\mathfrak{X}(#1)}
\newcommand{\smoothonef}[1]{\mathfrak{X}^{*}(#1)}
\newcommand{\bracket}[2]{[#1,#2]}

%------------------------Book Command---------------------------%
\makeatletter
\renewcommand\@pnumwidth{1cm}
\newcounter{book}
\renewcommand\thebook{\@Roman\c@book}
\newcommand\book{%
    \if@openright
        \cleardoublepage
    \else
        \clearpage
    \fi
    \thispagestyle{plain}%
    \if@twocolumn
        \onecolumn
        \@tempswatrue
    \else
        \@tempswafalse
    \fi
    \null\vfil
    \secdef\@book\@sbook
}
\def\@book[#1]#2{%
    \refstepcounter{book}
    \addcontentsline{toc}{book}{\bookname\ \thebook:\hspace{1em}#1}
    \markboth{}{}
    {\centering
     \interlinepenalty\@M
     \normalfont
     \huge\bfseries\bookname\nobreakspace\thebook
     \par
     \vskip 20\p@
     \Huge\bfseries#2\par}%
    \@endbook}
\def\@sbook#1{%
    {\centering
     \interlinepenalty \@M
     \normalfont
     \Huge\bfseries#1\par}%
    \@endbook}
\def\@endbook{
    \vfil\newpage
        \if@twoside
            \if@openright
                \null
                \thispagestyle{empty}%
                \newpage
            \fi
        \fi
        \if@tempswa
            \twocolumn
        \fi
}
\newcommand*\l@book[2]{%
    \ifnum\c@tocdepth >-3\relax
        \addpenalty{-\@highpenalty}%
        \addvspace{2.25em\@plus\p@}%
        \setlength\@tempdima{3em}%
        \begingroup
            \parindent\z@\rightskip\@pnumwidth
            \parfillskip -\@pnumwidth
            {
                \leavevmode
                \Large\bfseries#1\hfill\hb@xt@\@pnumwidth{\hss#2}
            }
            \par
            \nobreak
            \global\@nobreaktrue
            \everypar{\global\@nobreakfalse\everypar{}}%
        \endgroup
    \fi}
\newcommand\bookname{Book}
\renewcommand{\thebook}{\texorpdfstring{\Numberstring{book}}{book}}
\providecommand*{\toclevel@book}{-2}
\makeatother
\titleformat{\part}[display]
    {\Large\bfseries}
    {\partname\nobreakspace\thepart}
    {0mm}
    {\Huge\bfseries}
\titlecontents{part}[0pt]
    {\large\bfseries}
    {\partname\ \thecontentslabel: \quad}
    {}
    {\hfill\contentspage}
\titlecontents{chapter}[0pt]
    {\bfseries}
    {\chaptername\ \thecontentslabel:\quad}
    {}
    {\hfill\contentspage}
\newglossarystyle{longpara}{%
    \setglossarystyle{long}%
    \renewenvironment{theglossary}{%
        \begin{longtable}[l]{{p{0.25\hsize}p{0.65\hsize}}}
    }{\end{longtable}}%
    \renewcommand{\glossentry}[2]{%
        \glstarget{##1}{\glossentryname{##1}}%
        &\glossentrydesc{##1}{~##2.}
        \tabularnewline%
        \tabularnewline
    }%
}
\newglossary[not-glg]{notation}{not-gls}{not-glo}{Notation}
\newcommand*{\newnotation}[4][]{%
    \newglossaryentry{#2}{type=notation, name={\textbf{#3}, },
                          text={#4}, description={#4},#1}%
}
%--------------------------LENGTHS------------------------------%
% Spacings for the Table of Contents.
\addtolength{\cftsecnumwidth}{1ex}
\addtolength{\cftsubsecindent}{1ex}
\addtolength{\cftsubsecnumwidth}{1ex}
\addtolength{\cftfignumwidth}{1ex}
\addtolength{\cfttabnumwidth}{1ex}

% Indent and paragraph spacing.
\setlength{\parindent}{0em}
\setlength{\parskip}{0em}
%--------------------------Main Document----------------------------%
\begin{document}
    \ifx\ifsurgery\undefined
        \section*{Surgery Theory}
        \setcounter{section}{1}
        \renewcommand\thesubfigure{%
            \arabic{section}.\arabic{figure}.\arabic{subfigure}%
        }
    \fi
    \subsection{Lecture 2: Surgery Structure Sets}
        \begin{wrapfigure}[6]{r}{0.2\textwidth}
            \vspace{-7ex}
            \centering
            \captionsetup{type=figure}
            \begin{tikzcd}[row sep=small,column sep=large]
                N_{1}\arrow[dd,"g"]\arrow[dr, "f_{1}"]\\
                &M\\
                N_{2}\arrow[ur, "f_{2}" below]
            \end{tikzcd}
            \caption[Surgery Theory Commutative Diagram]{%
                An example of a Commutative Diagram.
            }
            \label{%
                fig:wellesley_surgery_theory_%
                commutative_diagram_for_g_for_%
                two_homotopy_equivalences%
            }
        \end{wrapfigure}
        Let $X,M_{1}$, and $M_{2}$ be closed,
        compact $n$-dimensional manifolds without boundary.
        Two homotopy equivalences $f_{i}:M_{i}\rightarrow X$
        are called equivalent if there exists a cobordism
        $(W;M_{1},M_{2})$ and a map
        $(F;f_{1},f_{2}):(W;M_{1},M_{2})%
         \rightarrow(X\times[0,1];X\times\{0\},X\times\{1\})$
        such that $F,f_{1},f_{2}$ are homotopy equivalences.
        The structure set $S(X)$ is the set of equivalence
        classes of homotopy equivalences $f:M\rightarrow X$
        from closed manifolds of dimension $n$ to $X$.
        \hfill
        \begin{definition}
            The surgery structure set of a closed
            (without boundary) compact manifold $M$ is
            $S(M)=\{f:N^{n}\rightarrow{M^{n}}|f%
             \textrm{ is a Homotopy Equivalence}\}$
        \end{definition}
        \begin{definition}
            The base point of a surgery structure
            set is the map $id_{X}:X\rightarrow X$.
        \end{definition}
        Let $N_{1}$ and $N_{2}$ be two manifold structures.
        And let $f_{1}:N_{1}^{n}\rightarrow M^{n}$ and
        $f_{2}:N_{2}^{n}\rightarrow M^{n}$ be two homotopy
        equivalences. We call $g:N_{1}\rightarrow N_{2}$ a
        cat-homeomorphism if $g$, together with $f_{1}$ and
        $f_{2}$, form the commutative diagram in figure
        \ref{%
            fig:wellesley_surgery_theory_commutative_%
            diagram_for_g_for_two_homotopy_equivalences%
        }.
        That is, $g$ is a cat-homeomorphism if it
        homotopy commutes.
        \begin{remark}
            Cat means category. There are three types:
            Top, PL, and Diff. 
            \begin{itemize}
                \begin{multicols}{3}
                    \item Top: Topological
                    \item PL: Piece-wise Linear
                    \item Diff: Diffeomorphism
                \end{multicols}
            \end{itemize}
        \end{remark}
        \begin{example}
            Some examples of surgery structure sets:
            \begin{enumerate}
                \begin{multicols}{3}
                    \item $S^{Top}(S^{n})=\{S^{n}\}$
                    \item $S^{PL}(S^{n})=\{S^{n}\}$
                    \item $S^{Diff}(S^{7})=\mathbb{Z}_{28}$
                \end{multicols}
            \end{enumerate}
        \end{example}
        \subsubsection{Orientable and Non-Orientable}
            Stiefel-Whitney classes $w_{1},\hdots, w_{n}$ are
            cohomological classes.
            Orientiable means that $w_{1}=0$.
            \begin{example}
                \
                \begin{enumerate}
                    \begin{multicols}{2}
                    \item $\mathbb{RP}^{2}$ - Non-Orientable
                    \item $\mathbb{RP}^{4}$ - Non-Orientable
                    \item $\mathbb{RP}^{6}$ - Non-Orientable
                    \item $\mathbb{RP}^{8}$ - Non-Orientable
                    \item $\mathbb{RP}^{3}$ - Orientable
                    \item $\mathbb{RP}^{5}$ - Orientable
                    \item $\mathbb{RP}^{7}$ - Orientable
                    \item $\mathbb{CP}^{n}$ - Orientable for all
                          $n\in\mathbb{N}$
                    \end{multicols}
                \end{enumerate}
            \end{example}
            Returning to surgery exact sequences, the goal
            is to compute $S^{Cat}(\mathcal{M}^{n})$, where $n$
            is the dimension of $\mathcal{M}^{n}$. The notion
            of a surgery helps solve this question. Let
            $X=\mathbb{S}^{2}\setminus%
             \{(a_{1},b_{1},c_{1}),(a_{2},b_{2},c_{2})\}$.
            That is, the sphere with two points removed.
            Stretch these two points out to create a sphere
            with two holes removed. One could imagine taking
            a hollow cylinder and stretching it to connect
            the two holes in the sphere. The result is a
            spherical coffee cup, as shown in
            Fig.~\ref{%
                fig:surgery_theory_example_of_a_surgery%
            }.
            This figure can be continuously deformed into a torus.
            \begin{figure}[H]
                \centering
                \captionsetup{type=figure}
                \begin{tikzpicture}
                    \fill[ball color = gray!50!white,draw=black]
                        (0,0) circle (2);
                    \fill[fill=white,draw=black,thick]
                        (1,1.1) circle (0.1);
                    \fill[fill=white,draw=black,thick]
                        (1.2,0.7) circle (0.1);
                    \node at (3,0) {$+$};
                    \fill[%
                        bottom color=black,
                        top color=white,
                        draw=black
                    ]   (5,1) ellipse (1 and 0.5);
                    \fill[%
                        left color=gray!50!black,
                        right color=gray!50!black,
                        middle color=gray!50,
                        shading=axis
                    ]   (4,1)--(4,-1.5)
                        arc(180:360:1 and 0.5)--(6,1)
                        arc(360:180:1 and 0.5);
                    \draw[draw=black,thick,>=triangle 45,->]
                        (6.5,0)--(7.5,0);
                    \fill[ball color=gray!50!white,draw=black]
                        (10,0) circle (2);
                    \fill[%
                        left color=black!50!gray,
                        right color=white,
                        draw=black,
                        thin
                    ]   (11.7,1) arc(90:-90:1)
                        arc(-90:-270:0.2) arc(-90:90:0.6)
                        arc(-90:-270:0.2);
                \end{tikzpicture}
                \caption{Simple Surgery Example}
                \label{fig:surgery_theory_example_of_a_surgery}
            \end{figure}
            Recall that $S^{0}$ is two points, and that
            $D^{2}$ is the open unit disc. Then $S^{0}\times D^{2}$
            is simply two disjoint open unit discs. This is a good
            representation of the idea of the disjoint union,
            denoted $X\coprod Y$. We have that:
            \begin{equation*}
                S^{0}\times{D^{2}}=D^{2}\coprod{D^{2}}
            \end{equation*}
            We can also represent a cylinder as the closed
            $S^{1}\times \overline{D}^{1}$. The codimension
            of a surgery is the dimension of the object minus
            the dimension of a surgery. So, for the surgery
            in Fig.~\ref{fig:surgery_theory_example_of_a_surgery},
            the dimension of the entire thing is $2$, the dimension
            of the surgery is $2$, so the codimension is $0$.
            This is called a Zero-Surgery. A zero-surgery takes
            out $2$ holes and connects them with a tube.
            \begin{figure}[H]
                \centering
                \captionsetup{type=figure}
                \begin{tikzpicture}[%
                    line width=1pt,
                    line cap=round,
                    >={Stealth[black]},
                    every edge/.style={draw=black,very thick}
                ]
                    \node[%
                        fill=black,
                        circle,
                        inner sep=0pt,
                        outer sep=0pt
                    ]   at (0,0) (a) {};
                    \node[%
                        fill=black,
                        circle,
                        inner sep=0pt,
                        outer sep=0pt
                    ]   at (-0.6,0.1) (b) {};
                    \node[%
                        fill=black,
                        circle,
                        inner sep=0pt,
                        outer sep=0pt
                    ]   at (-1,1) (c) {};
                    \node[%
                        fill=black,
                        circle,
                        inner sep=0pt,
                        outer sep=0pt
                    ]   at (-0.4,1.5) (d) {};
                    \node[%
                        fill=black,
                        circle,
                        inner sep=0pt,
                        outer sep=0pt
                    ]   at (0, 1.8) (e) {};
                    \node[%
                        fill=black,
                        circle,
                        inner sep=0pt,
                        outer sep=0pt
                    ]   at (0.5, 1.7) (f) {};
                    \node[%
                        fill=black,
                        circle,
                        inner sep=0pt,
                        outer sep=0pt
                    ]   at (1,1.2) (g) {};
                    \node[%
                        fill=black,
                        circle,
                        inner sep=0pt,
                        outer sep=0pt
                    ]   at (0.7,0.4) (h) {};
                    \node at (-0.3,1) (i) {$M$};
                    \path[draw,use Hobby shortcut,closed=true]
                        (a)..(b)..(c)..(d)..(e)..(f)..(g)..(h);
                    \node[%
                        fill=black,
                        circle,
                        inner sep=0pt,
                        outer sep=0pt
                    ]   at (3,0) (a1) {};
                    \node[%
                        fill=black,
                        circle,
                        inner sep=0pt,
                        outer sep=0pt
                    ]   at (2.5,0.4) (b1) {};
                    \node[%
                        fill=black,
                        circle,
                        inner sep=0pt,
                        outer sep=0pt
                    ]   at (2,1.2) (c1) {};
                    \node[%
                        fill=black,
                        circle,
                        inner sep=0pt,
                        outer sep=0pt
                    ]   at (2.6,2.1) (d1) {};
                    \node[%
                        fill=black,
                        circle,
                        inner sep=0pt,
                        outer sep=0pt
                    ]   at (3, 2.2) (e1) {};
                    \node[%
                        fill=black,
                        circle,
                        inner sep=0pt,
                        outer sep=0pt
                    ]   at (3.7, 1.7) (f1) {};
                    \node[%
                        fill=black,
                        circle,
                        inner sep=0pt,
                        outer sep=0pt
                    ]   at (4,1) (g1) {};
                    \node[%
                        fill=black,
                        circle,
                        inner sep=0pt,
                        outer sep=0pt
                    ]   at (3.7,0.4) (h1) {};
                    \node at (3.4,1) (i1) {$N$};
                    \draw[draw=black,densely dashed]
                        (0.2,1) circle (0.2);
                    \draw[draw=black,densely dashed]
                        (2.8,1.3) circle (0.2);
                    \draw (0.2,1.2) -- (2.8,1.5);
                    \draw (0.2,0.8) -- (2.8,1.1);
                    \path[draw,use Hobby shortcut,closed=true]
                        (a1)..(b1)..(c1)..(d1)..
                        (e1)..(f1)..(g1)..(h1);
                \end{tikzpicture}
                \caption[Surgery Theory - A Zero Surgery]{A Zero Surgery between $N$ and $M$.}
                \label{fig:surgery_theory_a_zero_surgery}
            \end{figure}
            Let $\mathcal{M}^{n}$ be an $n$ dimensional manifold.
            Embed $S^{k}\times D^{n-k}$ into $\mathcal{M}^{n}$.
            Then
            $\partial(S^{k}\times {D^{n-k}})%
             =S^{k}\times{S^{n-k-1}}$,
            where $\partial(X)$ is the boundary of $X$.
            Remove $\partial(S^{k}\times{D^{n-k}})$ and
            glue $S^{k+1}D^{n-k-1}$.
            Note that
            $\dim(S^{k+1}\times{D^{n-k-1})}%
             =\dim(S^{k}\times{D^{n-k}})=n$.
            We alse have that
            $\partial(D^{k+1}\times{S^{n-k-1}})%
             =S^{k}\times{S^{n-k-1}}$.
            Glue $\mathcal{M}^{n}\cup(D^{k+1}\times S^{n-k-1})$
            along $\partial(S^{k}\times{S^{n-k-1}})$.
            \begin{figure}[H]
                \centering
                \captionsetup{type=figure}
                \resizebox{!}{0.2\textheight}{
                \begin{tikzpicture}
                    \draw[draw=black,densely dashed]
                        (0,0) circle (1.2);
                    \draw[draw=black,thick] (0,0) circle (3);
                    \node at (0,2.7) {\scriptsize{$\mathcal{M}$}};
                    \node at (0,0) {$S^{k}\times D^{n-k}$};
                    \node at (0,1.4)
                        {%
                            \scriptsize{%
                                $\partial(S^{k}\times{D^{n-k}})%
                                 =S^{k}\times{S^{n-k-1}}$
                            }
                        };
                    \node[%
                        fill=black,
                        circle,
                        inner sep=0pt,
                        outer sep=0pt
                    ]   at (5,0) (a) {};
                    \node[%
                        fill=black,
                        circle,
                        inner sep=0pt,
                        outer sep=0pt
                    ]   at (4.4,0.1) (b) {};
                    \node[%
                        fill=black,
                        circle,
                        inner sep=0pt,
                        outer sep=0pt
                    ]   at (4,1) (c) {};
                    \node[%
                        fill=black,
                        circle,
                        inner sep=0pt,
                        outer sep=0pt
                    ]   at (4.6,1.5) (d) {};
                    \node[%
                        fill=black,
                        circle,
                        inner sep=0pt,
                        outer sep=0pt
                    ]   at (5, 1.8) (e) {};
                    \node[%
                        fill=black,
                        circle,
                        inner sep=0pt,
                        outer sep=0pt
                    ]   at (5.5, 1.7) (f) {};
                    \node[%
                        fill=black,
                        circle,
                        inner sep=0pt,
                        outer sep=0pt
                    ]   at (6,1.2) (g) {};
                    \node[%
                        fill=black,
                        circle,
                        inner sep=0pt,
                        outer sep=0pt
                    ]   at (5.7,0.4) (h) {};
                    \node at (5,0.8)
                        {\scriptsize{$D^{k+1}\times{S^{n-k}}$}};
                    \draw[>=latex,draw=black,->]
                        (3.6,0.7)--(1.4,0.2);
                    \node at (5,-0.5)
                        {\scriptsize{Glue Along Boundary}};
                    \path[draw,use Hobby shortcut,closed=true,thick]
                        (a)..(b)..(c)..(d)..(e)..(f)..(g)..(h);
                \end{tikzpicture}}
                \caption[Surgery Theory - Surgery Example]{%
                    Gluing $D^{k+1}\times S^{n-k}$ along
                    $\partial(S^{k}\times D^{n-k})$.
                    The new manifold is
                    $\mathcal{M}^{n}\setminus
                     (S^{k}\times D^{n-k}\coprod
                     (D^{k+1}\times S^{n-k-1})$
                }
                \label{fig:surgery_theory_glueing_S_k_D_n_k_to_M}
            \end{figure}
            We now consider $k$ surgeries
            $\mathcal{M}%
             \overset{\textrm{k-surgery}}{\longrightarrow}%
             \mathcal{N}$.
            We have seen
            $S^{2}%
             \overset{\textrm{0-surgery}}{\longrightarrow}
             T^{2}$.
            Note: $\pi_{1}(S^{2})$ is trivial,
            and $\pi_{1}(T^{2})=\mathbb{Z}^{2}$.
            This happens because $n<5$. When $n\geq{5}$,
            we have the following result.
            \begin{theorem}
                If $\mathcal{M}$ is an $n$ dimensional manifold,
                $n\geq{5}$, and if $\mathcal{N}$ is the result of
                a $k$ surgery on $\mathcal{M}$, then
                $\pi_{1}(\mathcal{M})=\pi_{1}(\mathcal{N})$.
            \end{theorem}
        \subsubsection{More On Surgery Exact Sequences}
            Recall that a surgery exact sequence
            looks like the following:
            \begin{equation*}
                \underset{\textrm{Group}}
                {\underbrace{L_{n+1}(\mathbb{Z}\pi_{1}\mathcal{M})}}
                \rightarrow\cdots\rightarrow
                \underset{\textrm{Not a Group}}
                {\underbrace{S^{Cat}(\mathcal{M}^{n})}}
                \rightarrow
                \underset{\textrm{Group}}
                {\underbrace{[M,G/0]}}
                \rightarrow \underset{\textrm{Group}}
                {\underbrace{L_{n}(\mathbb{Z}\pi_{1}(\mathcal{M}))}} 
            \end{equation*}
            An exact sequence of groups is of then form
            $G_{n+1}\overset{g_{n}}{\rightarrow}%
             G_{n}\rightarrow \hdots$,
             where $\Im(g_n)=\ker(g_{n-1})$.
             We refine our notion of a surgery exact sequence:
            \begin{equation*}
                \cdots\rightarrow
                L_{n+1}(\mathbb{Z}\pi_{1}(\mathcal{M}))
                \dashrightarrow{S^{Cat}}(\mathcal{M})
                \overset{g}{\rightarrow}[M,G/o]
                \overset{\sigma}{\rightarrow}
                L_{n}(\mathbb{Z}\pi_{1}(\mathcal{M}))
            \end{equation*}
            The dotted line means
            $L_{n+1}(\mathbb{Z}\pi_{1}(\mathcal{M}))$
            acts on $S^{Cat}(\mathcal{M})$.
            Exact means that $\Im(g)=\ker(\sigma)$.
            Each element $f\in{[M,G/o]}$
            either pulls back to $\emptyset$ or
            something non-empty. If non-empty,
            you get a blob in
            $S^{Cat}(\mathcal{M})$: $f^{-1}(\{x\})$.
            But:
            \begin{equation*}
                \underset{f\in[M,G/o]}{\cup}g^{-1}(\{f\})
                =S^{Cat}(\mathcal{M})
            \end{equation*}
            This process creates a partition of
            $S^{Cat}(\mathcal{M})$. Now,
            $L_{n+1}(\mathbb{Z}(\pi_{1}\mathcal{M}))$
            acts on $S^{Cat}(\mathcal{M})$ in some fashion.
            Partition the space into orbits. Exactness
            here means that partitioning by point inverses
            is the same as partitioning by orbits. That is,
            the two partitions are identical. See
            Fig.~\ref{fig:surgery_theory_partition_of_S_Cat}
            for a partioning into orbits.
            \begin{figure}[H]
                \centering
                \captionsetup{type=figure}
                \resizebox{!}{0.15\textheight}{
                \begin{tikzpicture}
                    \path   coordinate (aux0) at (0,1.5)
                            coordinate (aux1) at (0,3.5)
                            coordinate (aux2) at (10,3.5)
                            coordinate (aux3) at (9,6)
                            coordinate (aux4) at (4,0)
                            coordinate (aux5) at (7,0)
                            coordinate (aux6) at (2,6)
                            coordinate (aux7) at (5,6)
                            coordinate (esp1) at (0.2,2.5)
                            coordinate (esp2) at (1.5,1.5)
                            coordinate (esp3) at (3,0.1)
                            coordinate (esp4) at (5.5,1.1)
                            coordinate (esp5) at (8,0.5)
                            coordinate (esp6) at (8.75,2)
                            coordinate (esp7) at (9.7,3)
                            coordinate (esp8) at (6.5,4.5)
                            coordinate (esp9) at (3.8,5.8)
                            coordinate (esp10) at (1.5,4);
                    \draw[line width=0.8pt]
                        (esp1) to[out=-90,in=170]
                        (esp2) to[out=-10,in=170]
                        (esp3) to[out=-10,in=180]
                        (esp4) to[out=0,in=180]
                        (esp5) to[out=10,in=-150]
                        (esp6) to[out=20,in=-90]
                        (esp7) to[out=90,in=-60]
                        (esp8) to[out=120,in=0]
                        (esp9) to[out=180,in=0]
                        (esp10) to[out=180,in=90] cycle;
                    \clip   (esp1) to[out=-90,in=170]
                            (esp2) to[out=-10,in=170]
                            (esp3) to[out=-10,in=180]
                            (esp4) to[out=0,in=180]
                            (esp5) to[out=10,in=-150]
                            (esp6) to[out=20,in=-90]
                            (esp7) to[out=90,in=-60]
                            (esp8) to[out=120,in=0]
                            (esp9) to[out=180,in=0]
                            (esp10) to[out=180,in=90] cycle;
                    \draw   (aux4) to[bend right=10]
                            (aux6) -- (aux7) to[bend left=10]
                            (aux5) -- cycle;
                    \draw   (aux5) to[bend right=10]
                            (aux7) -- (10,6) -- (10,0) -- cycle;
                    \draw   (aux0) -- (aux1) to[bend right=10]
                            (aux3) -- (10,6) -- (aux2)
                            to[bend left=10] cycle;
                    \draw   (0,0) -- (aux4) to[bend right=10]
                            (aux6) -- (0,6) -- (0,0) -- cycle;
                    \draw   (0,6) -- (aux1) to[bend right=10]
                            (aux3) -- (0,6) -- cycle;
                    \node at (2,2.5) {Orbit};  
                    \node at (5,3) {$S^{Cat}(\mathcal{M})$};
                \end{tikzpicture}}
                \caption[%
                    Surgery Theory - Partion of $S^{Cat}$%
                ]{%
                    Partition of $S^{Cat}(\mathcal{M})$%
                }
                \label{fig:surgery_theory_partition_of_S_Cat}
            \end{figure}
            The next object to talk about is
            $L_{n}(\mathbb{Z}\pi_{1}(\mathcal{M}))$.
            These are called Wall groups.
            They are difficult to compute,
            but there are some facts that are known about them:
            \begin{itemize}
                \item Wall groups only have 2-torsion.
                \begin{itemize}
                    \item 2-torsion means that elements
                          of finite order have order $2$.
                    \item This implies the groups are Abelian.
                \end{itemize}
                \item They can be orientable or not.
                \begin{itemize}
                    \item $L_{n}(%
                           \mathbb{Z}\pi_{1}(\mathcal{M})^{\pm})$
                           indicates orientable or not.
                \end{itemize}
            \end{itemize}
\end{document}