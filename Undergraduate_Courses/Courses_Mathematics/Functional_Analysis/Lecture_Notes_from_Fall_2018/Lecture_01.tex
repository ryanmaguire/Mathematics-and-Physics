\documentclass[crop=false,class=article,oneside]{standalone}
%----------------------------Preamble-------------------------------%
%---------------------------Packages----------------------------%
\usepackage{geometry}
\geometry{b5paper, margin=1.0in}
\usepackage[T1]{fontenc}
\usepackage{graphicx, float}            % Graphics/Images.
\usepackage{natbib}                     % For bibliographies.
\bibliographystyle{agsm}                % Bibliography style.
\usepackage[french, english]{babel}     % Language typesetting.
\usepackage[dvipsnames]{xcolor}         % Color names.
\usepackage{listings, lstlinebgrd}      % Verbatim-Like Tools.
\usepackage{mathtools, esint, mathrsfs} % amsmath and integrals.
\usepackage{amsthm, amsfonts}           % Fonts and theorems.
\usepackage{tabularx}
\usepackage{tcolorbox}                  % Frames around theorems.
\usepackage{upgreek}                    % Non-Italic Greek.
\usepackage{paracol}                    % Two-column styling.
\usepackage{wrapfig}                    % Wrap text around figure.
\usepackage{fmtcount, etoolbox}         % For the \book{} command.
\usepackage[newparttoc]{titlesec}       % Formatting chapter, etc.
\usepackage{titletoc}                   % Allows \book in toc.
\usepackage[nottoc]{tocbibind}          % Bibliography in toc.
\usepackage[titles]{tocloft}            % ToC formatting.
\usepackage{multicol, enumitem}         % Multi-column/enumerate.
\usepackage{import}                     % Import external files.
\usepackage{pgfplots, tikz}             % Drawing/graphing tools.
\usetikzlibrary{
    calc,                   % Calculating right angles and more.
    angles,                 % Drawing angles within triangles.
    arrows.meta,            % Latex and Stealth arrows.
    quotes,                 % Adding labels to angles.
    positioning,            % Relative positioning of nodes.
    decorations.markings,   % Adding arrows in the middle of a line.
    patterns,
    arrows,
    shapes,
    shapes.geometric,
    cd,
    hobby,
    babel
}                                       % Libraries for tikz.
\pgfplotsset{compat=1.9}                % Version of pgfplots.
\usepackage[font=scriptsize,
            labelformat=simple,
            labelsep=colon]{subcaption} % Subfigure captions.
\usepackage[font={scriptsize},
            hypcap=true,
            labelsep=colon]{caption}    % Figure captions.
\usepackage{hyperref}                   % Allows for hyperlinks.
\hypersetup{
    colorlinks=true,
    linkcolor=blue,
    filecolor=magenta,
    urlcolor=Cerulean,
    citecolor=SkyBlue
}                           % Colors for hyperref.
\usepackage[toc,acronym,nogroupskip]{glossaries} % Glossaries and acronyms.
\usepackage[subpreambles=false]{standalone}      % Complileable sub files.

% Various font stuff from kiwi.
% Use this for Times text and Computer Modern math
%\usepackage{times}

% Quite nice
%\usepackage[charter, greekfamily=, greekuppercase=italicized]{mathdesign}
%\usepackage[utopia, greekuppercase=italicized]{mathdesign}    % Math is narrower

% Use this for Times text and math
%\usepackage{newtxtext}
%\usepackage[libertine,cmintegrals]{newtxmath}
%\usepackage{fix-cm}

%\usepackage{txfontsb}
% or
%\usepackage{mathptmx}

%\usepackage[scaled=0.92]{helvet}
%\renewcommand{\rmdefault}{ptm}

%\usepackage{mathpazo}    % add possibly `sc` and `osf` options
%\usepackage{eulervm}

%\usepackage{fourier}
%\renewcommand{\rmdefault}{ptm}
%\usepackage{mathptm}

%\usepackage{fontspec}
%\setmainfont{lmodern}

%\usepackage[varg]{txfonts}
%\usepackage{fouriernc}
%\usepackage{mathpazo}

%\usepackage{bookman}
%\usepackage[scaled]{uarial}
%\usepackage[scaled]{helvet}
%\renewcommand*\familydefault{\sfdefault}
%\usepackage[math]{anttor}

%\newcommand\fgeorgia{\fontfamily{jvn}\selectfont}
%\newcommand\ftimes{\fontfamily{ptm}\selectfont}
%\newcommand\fhelvetica{\fontfamily{phv}\selectfont}
%\newcommand\fcourier{\fontfamily{pcr}\selectfont}
%\newcommand\fbookman{\fontfamily{pbk}\selectfont}
%\newcommand\fnewcentury{\fontfamily{pnc}\selectfont}
%\newcommand\fpalatino{\fontfamily{ppl}\selectfont}
%\newcommand\favantgarde{\fontfamily{pag}\selectfont}
%\newcommand\fnormal{\normalfont}
%\newcommand\fsize[1]{\ifnum#1>0\fontsize{#1}{#1}\selectfont\else\normalsize\fi}
%------------------------Theorem Styles-------------------------%
% Define theorem style for default spacing and normal font.
\newtheoremstyle{normal}
    {\topsep}               % Amount of space above the theorem.
    {\topsep}               % Amount of space below the theorem.
    {}                      % Font used for body of theorem.
    {}                      % Measure of space to indent.
    {\bfseries}             % Font of the header of the theorem.
    {}                      % Punctuation between head and body.
    {.5em}                  % Space after theorem head.
    {}

% Define theorem style for default spacing with italicized font.
\newtheoremstyle{normalit}{\topsep}{\topsep}
                {\itshape}{}{\bfseries}{}{.5em}{}

% Italic header environment.
\newtheoremstyle{thmit}{\topsep}{\topsep}{}{}{\itshape}{}{0.5em}{}

% Define italicized environments.
\theoremstyle{normalit}
\newtheorem{theorem}{Theorem}[section]
\newtheorem{lemma}{Lemma}[section]
\newtheorem{corollary}{Corollary}[section]
\newtheorem{proposition}{Proposition}[section]
\newtheorem*{theorem*}{Theorem}

% Define environments with italic headers.
\theoremstyle{thmit}
\newtheorem*{solution}{Solution}
\newtheorem*{fsolution}{Solution}

% Define default environments.
\theoremstyle{normal}
\newtheorem{example}{Example}[section]
\newtheorem{definition}{Definition}[section]
\newtheorem{problem}{Problem}[section]
\newtheorem{question}{Question}[section]
\newtheorem{remark}{Remark}[section]
\newtheorem{properties}{Properties}[section]
\newtheorem{notation}{Notation}[section]
\newtheorem{axiom}{Axiom}[section]
\newtheorem*{properties*}{Properties}
\newtheorem*{remark*}{Remark}
\newtheorem*{definition*}{Definition}
\theoremstyle{plain}

% Define framed environment.
\tcbuselibrary{most}
\newtcbtheorem[use counter*=theorem]{ftheorem}{Theorem}%
    {colback=green!5,colframe=green!35!black,
     fonttitle=\bfseries\upshape}{th}

\newtcbtheorem[use counter*=example]{fdefinition}{Definition}%
    {fonttitle=\bfseries\upshape,
     colback=blue!5!white,colframe=blue!75!black}{def}

\newtcbtheorem[use counter*=example]{fexample}{Example}%
    {fonttitle=\bfseries\upshape,
     colback=red!5!white,colframe=red!75!black}{ex}

\newtcbtheorem[use counter*=notation]{fnotation}{Notation}%
    {fonttitle=\bfseries\upshape,
     colback=SeaGreen!5!white,colframe=SeaGreen!75!black}{ex}

\newtcbtheorem[use counter*=corollary]{fcorollary}{Corollary}%
    {fonttitle=\bfseries\upshape,
     colback=Orchid!5!white,colframe=Orchid!75!black}{ex}

\newenvironment{bproof}{\textit{Proof.}}{\hfill$\square$}
\tcolorboxenvironment{bproof}{blanker,breakable,left=5mm,
                             before skip=10pt,after skip=10pt,
                             borderline west={1mm}{0pt}{red}}
\tcolorboxenvironment{fsolution}
    {enhanced jigsaw,colframe=cyan,interior hidden,breakable}

%--------------------Declared Math Operators--------------------%
\DeclareMathOperator{\Refl}{Refl}           % Reflection operator.
\DeclareMathOperator{\Span}{Span}           % Span of a set of vectors.
\DeclareMathOperator{\Card}{Card}           % Cardinality of set.
\DeclareMathOperator{\Ord}{Ord}             % Ordinal of ordered set.
\DeclareMathOperator{\Tr}{Tr}               % Trace of matrix.
\DeclareMathOperator{\adjoint}{adj}         % Adjoint of matrix.
\DeclareMathOperator{\rk}{rk}               % Rank of operator.
\DeclareMathOperator{\nul}{nul}             % Null space of operator.
\DeclareMathOperator{\sgn}{sgn}             % Sign of a number.
\DeclareMathOperator{\multideg}{mutlideg}   % Multi-Degree (Graphs).
\DeclareMathOperator{\GCD}{GCD}             % Greatest common denominator.
\DeclareMathOperator{\LM}{LM}               % Leading monomial
\DeclareMathOperator{\LC}{LC}               % Leading coefficient.
\DeclareMathOperator{\LT}{LT}               % Leading term.
\DeclareMathOperator{\LCM}{LCM}             % Least common multiple.
\DeclareMathOperator{\Mon}{Mon}             % Monomial.
\DeclareMathOperator{\Spec}{Spec}           % Spectrum.
\DeclareMathOperator{\proj}{proj}           % Projection.
\DeclareMathOperator{\comp}{comp}           % Component.
\DeclareMathOperator{\sinc}{sinc}           % Sinc function.
\DeclareMathOperator{\Ima}{Im}              % Image of operator.
\DeclareMathOperator{\Prin}{Prin}           % Principal value.
\DeclareMathOperator{\Mod}{mod}             % Modulus.
%------------------------New Commands---------------------------%
\DeclarePairedDelimiter\norm{\lVert}{\rVert}
\DeclarePairedDelimiter\ceil{\lceil}{\rceil}
\DeclarePairedDelimiter\floor{\lfloor}{\rfloor}
\newcommand*\diff{\mathop{}\!\mathrm{d}}
\newcommand*\Diff[1]{\mathop{}\!\mathrm{d^#1}}
\renewcommand{\mod}{\ \Mod}
\renewcommand*{\glstextformat}[1]{\textcolor{RoyalBlue}{#1}}
\renewcommand{\glsnamefont}[1]{\textbf{#1}}
\renewcommand\labelitemii{$\circ$}
\renewcommand\thesubfigure{\arabic{chapter}.\arabic{figure}}
\renewcommand\thesubfigure{%
    \arabic{chapter}.\arabic{figure}.\arabic{subfigure}}
\addto\captionsenglish{\renewcommand{\figurename}{Fig.}}
%------------------------Book Command---------------------------%
\makeatletter
\renewcommand\@pnumwidth{1cm}
\newcounter{book}
\renewcommand\thebook{\@Roman\c@book}
\newcommand\book{%
    \if@openright
        \cleardoublepage
    \else
        \clearpage
    \fi
    \thispagestyle{plain}%
    \if@twocolumn
        \onecolumn
        \@tempswatrue
    \else
        \@tempswafalse
    \fi
    \null\vfil
    \secdef\@book\@sbook
}
\def\@book[#1]#2{%
    \ifnum \c@secnumdepth >-3\relax
        \refstepcounter{book}%
        \addcontentsline{toc}{book}{
            \bookname\ \thebook:\hspace{1em}#1
        }
    \else
        \addcontentsline{toc}{book}{#1}%
    \fi
    \markboth{}{}%
    {\centering
     \interlinepenalty \@M
     \normalfont
     \ifnum \c@secnumdepth >-2\relax
       \huge\bfseries \bookname\nobreakspace\thebook
       \par
       \vskip 20\p@
     \fi
     \Huge \bfseries #2\par}%
    \@endbook}
\def\@sbook#1{%
    {\centering
     \interlinepenalty \@M
     \normalfont
     \Huge \bfseries #1\par}%
    \@endbook}
\def\@endbook{
    \vfil\newpage
        \if@twoside
            \if@openright
                \null
                \thispagestyle{empty}%
                \newpage
            \fi
        \fi
        \if@tempswa
            \twocolumn
        \fi
}
\newcommand*\l@book[2]{%
    \ifnum \c@tocdepth >-2\relax
        \addpenalty{-\@highpenalty}%
        \addvspace{2.25em \@plus\p@}%
        \setlength\@tempdima{3em}%
        \begingroup
            \parindent \z@ \rightskip \@pnumwidth
            \parfillskip -\@pnumwidth
            {
                \leavevmode
                \Large \bfseries #1\hfil \hb@xt@\@pnumwidth{
                    \hss #2
                }
            }
            \par
            \nobreak
            \global\@nobreaktrue
            \everypar{\global\@nobreakfalse\everypar{}}%
        \endgroup
    \fi}
\newcommand\bookname{Book}
\renewcommand{\thebook}{\texorpdfstring{\Numberstring{book}}{book}}
\providecommand*{\toclevel@book}{-2}
\makeatother
\titlecontents{chapter}[0pt]
    {\bfseries}
    {\chaptername\ \thecontentslabel:\quad}
    {}
    {\hfill\contentspage}
\titleformat{\part}[display]
    {\Large\bfseries}
    {\partname\nobreakspace\thepart}
    {0mm}
    {\Huge\bfseries}
    \titlecontents{part}[0pt]
    {\large\bfseries}
    {\partname\ \thecontentslabel: \quad}
    {}
    {\hfill\contentspage}
\newcommand{\MarkRightAngle}[4][.3cm]
    {\coordinate (tempa) at ($(#3)!#1!(#2)$);
     \coordinate (tempb) at ($(#3)!#1!(#4)$);
     \coordinate (tempc) at ($(tempa)!0.5!(tempb)$);%midpoint
     \draw (tempa) -- ($(#3)!2!(tempc)$) -- (tempb);}
%--------------------------LENGTHS------------------------------%
% Spacings for the Table of Contents.
\addtolength{\cftsecnumwidth}{1ex}
\addtolength{\cftsubsecindent}{1ex}
\addtolength{\cftsubsecnumwidth}{1ex}
\addtolength{\cftfignumwidth}{1ex}
\addtolength{\cfttabnumwidth}{1ex}

% Spacing for multi-column and enumerate environments.
\setlength{\multicolsep}{6pt}
\setlist[enumerate]{itemsep=0pt,topsep=3pt}

% Indent and paragraph spacing.
\setlength{\parindent}{0em}
\setlength{\parskip}{0em}
%--------------------------Main Document----------------------------%
\begin{document}
    \ifx\ifsub\undefined
        \section*{Functional Analysis}
        \setcounter{section}{1}
    \fi
    \subsection{Lecture 1: September 10, 2018}
        \subsubsection{Completeness}
            We will be talking about a bunch of different sets
            throughout the course, so let's begin with some of
            their notations:
            \begin{itemize}
                \item $\mathbb{N}$\quad The Natural Numbers.
                      We'll use this a lot.
                \item $\mathbb{Z}$\quad The Integers.
                      We'll never talk about these.
                \item $\mathbb{Q}$\quad The Rational Numbers.
                      Good for examples and counterexamples.
                \item $\mathbb{R}$\quad The Real Numbers.
                      These are the numbers we'll primarily
                      be concerned with.
                \item $\mathbb{C}$\quad The Complex Numbers.
                      See comment about $\mathbb{Z}$.
            \end{itemize}
            One of the fundamental properties of $\mathbb{R}$ is
            that is is ``Complete.'' This property is fundamental
            to many theorems involved in a standard calculus or
            real analysis course. The concepts of differentiation
            and convergence rely on completeness, and the
            intermediate value theorem fails without it. On the
            other hand, $\mathbb{Q}$ is not complete. $\mathbb{R}$
            is also something called a ``Field,'' and an ordered
            field at that. $\mathbb{Q}$ is also an ordered field.
            This means addition, subtraction, multiplication, and
            division are well defined operations (No dividing by
            zero, though), and that there is a sense of order on
            the set. For example, zero is less than one. The
            special fact about $\mathbb{R}$ is that it is a
            complete ordered field. In fact, $\mathbb{R}$ is the
            \textit{only} complete ordered field (Up to isomorphism,
            whatever that means). Completeness in $\mathbb{R}$ can
            be stated as the fact that the real numbers have the
            least upper bound property.
            \begin{definition}
                A subset of $\mathbb{R}$ that is bounded above is
                a nonempty set $S\subset{\mathbb{R}}$ such that
                there exists an $M\in\mathbb{R}$ such that for all
                $x\in{S}$, $x\leq{M}$.
            \end{definition}
            \begin{definition}
                An upper bound of a bounded above subset
                $S\subset\mathbb{R}$ is a real number $M\in\mathbb{R}$
                such that for all $x\in{S}$, $x\leq{M}$.
            \end{definition}
            For a subset $S$ of $\mathbb{R}$ is bounded above, then
            there exists infinitely many bounds. The real numbers
            have a special property that every bounded above set
            has a smallest upper bound.
            \begin{theorem}[Least Upper Bound Theorem]
                If $S\subset{\mathbb{R}}$ is bounded above, then
                there exists an $s\in\mathbb{R}$, called the least
                upper bound, such that $s$ and an upper bound
                and for all upper bounds $M$ of $S$, $s\leq{M}$.
            \end{theorem}
            \begin{theorem}
                There exist subsets $S$ of $\mathbb{Q}$ such that
                $S$ is bounded above, yet for all upper bounds
                $M$ there exists an $s$ such that $s$ is an upper
                bound of $S$ and $s<M$.
            \end{theorem}
            \begin{example}
                As an example, consider the set
                $\{x\in\mathbb{Q}:x^{2}\leq{2}\}$.
                This set has no least upper bound. The reason
                being is somewhat related to the fact that the
                least upper bound ``Wants,'' to be $\sqrt{2}$,
                but $\sqrt{2}$ is not a rational number. Thus
                there is no rational number to fill the gap.
                The rationals are incomplete.
            \end{example}
            The least upper bound property gives rise
            to many theorems, many of which are equivalent
            to this axiom.
            \begin{theorem}
                Bounded monotonic sequences converge.
            \end{theorem}
            \begin{proof}
                Let $x_{n}$ be a bounded monotonic sequence that is
                increasing in $\mathbb{R}$. If $x_{n}$ is decreasing, we
                replace the least upper bound with the greatest lower
                bound, and the proof is symmetric.
                Then $S=\{x_{n}:n\in\mathbb{N}\}$ is a
                non-empty subset of $\mathbb{R}$. But $x_{n}$ is
                a bounded sequence, and therefore $S$ is a bounded
                subset of $\mathbb{R}$.
                By the least upper bound property
                there exists a least upper bound $s\in\mathbb{R}$ of
                $S$. We now show that $x_{n}\rightarrow{s}$. Let
                $\varepsilon>0$ be given.
                Since $s$ is the least upper bound,
                $s-\varepsilon$ is not an upper bound of $S$, since
                $s-\varepsilon<s$. Therefore there exists a point
                $x_{N}\in{S}$ such that $s-\varepsilon<x_{n}$.
                Let $N\in\mathbb{N}$ be such that $s-\varepsilon<x_{N}$.
                But $x_{n}$ is monotonically increasing, and therefore
                for all $n>N$, $x_{N}<x_{n}$.
                But, as $s$ is a least upper
                bound of $S$, $x_{n}\leq{s}$. But then, for all $n>N$,
                $0<s-x_{n}<s-x_{N}<\varepsilon$.
                Therefore, $x_{n}\rightarrow{s}$
            \end{proof}
            The least upper bound is, in a sense, the
            reason why decimal expansions of
            real numbers work. For example, let $x_{n}$ be the
            sequence 3, 3.1, 3.14, 3.141, 3.1415, 3.14159, ...
            This sequence, which is the decimal expansion of $\pi$,
            is bounded by $4$. Therefore it has a least upper
            bound. We define the number $\pi$ to be the least upper
            bound of this sequence. Completeness is a very important
            property, but so far it relies on ordering.
            We want to find an equivalent definition of completeness
            that does not rely on ordering, so that we may speak of
            complete spaces, or sets, that have no notion of
            ordering on them.
            We start with a different definiton for the completeness
            of $\mathbb{R}$.
            \begin{definition}
                A Cauchy sequence in $\mathbb{R}$ is a sequence
                $x_{n}:\mathbb{N}\rightarrow\mathbb{R}$ such that
                for all $\varepsilon>0$ there is an $N\in\mathbb{N}$
                such that for all $n,m>N$, $|x_{n}-x_{m}|<\varepsilon$
            \end{definition}
            \begin{theorem}
                \label{%
                    FUNCTIONAL_ANALYSIS:CONVERGENT_SEEQUENCES_%
                    ARE_CAUCHY_SEQUENCES%
                }
                If $x_{n}:\mathbb{N}\rightarrow\mathbb{R}$ is a 
                convergent sequence, then it is a Cauchy-Sequence.
            \end{theorem}
            The converse of Thm.~\ref{%
                FUNCTIONAL_ANALYSIS:CONVERGENT_SEEQUENCES_%
                ARE_CAUCHY_SEQUENCES%
            }
            turns out to be a more general notion of completeness.
            That is, we can apply this to spaces that don't have
            a notion of order, but do have a notion of completeness.
            \begin{theorem}
                If $x_{n}:\mathbb{N}\rightarrow\mathbb{R}$ is
                a Cauchy sequence, then it converges.
            \end{theorem}
            \begin{proof}
                First we prove that Cauchy sequences are bounded.
                Let $\varepsilon=1$. Then $\varepsilon>0$. But
                as $x_{n}$ is a Cauchy sequence, there is an
                $N\in\mathbb{N}$ such that for all $n,m>N$,
                $|x_{n}-x_{m}|<\varepsilon$. That is, for all
                $n,m>N$, $-1<x_{n}-x_{m}<1$. Let $m=N+1$. Then,
                for all $n>N$, $x_{N+1}-1<x_{n}<x_{N+1}+1$.
                Then, for all $n\in\mathbb{N}$,
                $x_{n}\leq\max(\{x_{0},\hdots,1+x_{N+1}\})$. Therefore,
                $x_{n}$ is bounded. Next we need to talk about
                some fundamental properties of subsequences. Let
                $k:\mathbb{N}\rightarrow\mathbb{N}$ be a strictly
                increasing function. Then $x_{n_{k}}$ is a
                subsequence of $x_{n}$. The notation is somewhat
                strange here. We must first note that
                $x_{n}:\mathbb{N}\rightarrow\mathbb{R}$ is really just
                a function from the natural numbers to the real
                numbers. When we write $x_{n}$, what we really
                mean is $x(n)$. But nobody writes $x(0)$, $x(1),$
                $\hdots$ even if that's what they really mean.
                So a subsequence is merely function composition
                of the two functions
                $x:\mathbb{N}\rightarrow\mathbb{R}$
                and $k:\mathbb{N}\rightarrow\mathbb{N}$. So
                $x_{n_{k}}=x(k(n))$. Since $k$ is strictly increasing,
                the ``ordering'' of the sequence remains the same,
                we've simply skipped some
                (Perhaps a lot) of the elements.
                As an example, conside $k=n$. The most boring
                subsequence in the world: It's the exact same
                sequence. Or perhaps $k=2n$, which skips every other
                point. There is an important theorem associated
                to subsequences of bounded sequences, which we will
                use now but prove later. This is the
                Bolzano-Weierstrass theorem: Every bounded sequence
                has a convergent subsequence. But if $x_{n}$ is
                Cauchy, then it is bounded. By the
                Bolzano-Weiestrass theorem there is a convergent
                subsequence $x_{k_{n}}$. Let $x$ be the limit of
                $x_{k_{n}}$. We now must show that
                $x_{n}\rightarrow{x}$. Let $\varepsilon>0$ be given.
                As $x_{k_{n}}\rightarrow{x}$, there is an
                $N_{0}\in\mathbb{N}$ such that for all
                $n>N_{0}$, $|x_{k_{x}}-x|<\frac{\varepsilon}{2}$.
                But as $x_{n}$ is a Cauchy sequence, there is an
                $N_{1}$ such that for all $n,m>N_{1}$, 
                $|x_{n}-x_{m}|<\frac{\varepsilon}{2}$. Let
                $N=\max\{N_{0},N_{1}\}$. Then for all $n>N$,
                $|x-x_{n}|\leq|x-x_{k_{n}}|+|x_{k_{n}}-x_{n}|$.
                But as $k_{n}$ is a subsequence,
                $k_{n}\geq{n}$ and therefore $k_{n}>N$.
                But then $|x_{k_{n}}-x_{n}|<\frac{\varepsilon}{2}$.
                Therefore
                $|x-x_{n}|\leq|x-x_{k_{}}|+|x_{k_{n}}-x_{n}|\leq%
                 \frac{\varepsilon}{2}+\frac{\varepsilon}{2}%
                 =\varepsilon$. Therefore, etc.
            \end{proof}
            Now to prove the Bolzano-Weierstrass Theorem.
            \begin{theorem}
                Every sequence in $\mathbb{R}$ has a monotonic
                subsequence.
            \end{theorem}
            \begin{proof}
                Let $x_{n}$ be a sequence in $\mathbb{R}$. Call
                $n$ a ``peak point'' if $x_{n}\geq{x_{m}}$ for all
                ${m}\geq{n}$. If there are infinitely many of
                these ``peak points,'' then we have obtained
                a decreasing sequence, since the $n^{th}$ peak point
                will be greater than or equal to
                the $(n+1)^{th}$ peak point. We have thus obtained
                a monotonically increasing  subsequence.
                If there are finitely many,
                there are either $0$ or there is a last one,
                $x_{n_{0}}$. But then $x_{n_{0}+1}$ is not a
                peak point. But then there is a $k\in\mathbb{N}$
                such that $k>n_{0}+1$ and $x_{k}\geq{x_{n_{0}+1}}$, for
                otherwise $x_{n_{0}+1}$ would be a peak point.
                But $x_{k}$ is also not a peak point, and so there is
                a $k_{1}$ such that $k_{1}>k$ and
                $x_{k_{1}}\geq{x_{k}}$. This pattern continues, and
                we thus have a monotonically increasing subsequence.
                If there are zero peak points,
                repeat the argument above with $x_{n_{0}}=x_{0}$.
            \end{proof}
            There's probabbly some axiom of choice stuff going on
            here, but oh well.
            \begin{theorem}[Bolzano-Weierstrass Theorem]
                Bounded sequences have a convergent subsequence.
            \end{theorem}
            \begin{proof}
                By the previous theorem, all sequences have a
                monotonic subsequence. But bounded monotonic
                sequences converge. Therefore,
                there is a convergent subsequence.
            \end{proof}
            This notion is so important it has a name.
            \begin{definition}
                A sequentially compact space is a space such that
                every bounded sequence has a convergent subsequence.
            \end{definition}
            \begin{theorem}
                $\mathbb{R}$ is sequentially compact.
            \end{theorem}
            For shit's and giggles, let's prove the intermediate
            value theorem. A result used a lot in calculus, but
            never quite ``proved'' in the rigorous sense of the word.
            \begin{theorem}
                If $f:[a,b]\rightarrow\mathbb{R}$ is continuous and
                if $f(a)<f(b)$, then for all $z\in(f(a),f(b))$,
                there is a $c\in(a,b)$ such that $f(c)=z$.
            \end{theorem}
            \begin{proof}
                Let $x_{1}=\frac{a+b}{2}$. By trichotomy,
                which is one of the ordering properties, either
                $f(x_{1})=z$, $f(x_{1})<z$, or $f(x_{1})>z$. If
                $f(x_{1})=z$, we are done. If not, suppose
                $f(x_{1})<z$. The proof is symmetric for
                $f(x_{1})>z$. Let $x_{2}=\frac{x_{1}+b}{2}$. We
                continue checking whether $f(x_{2})=z$, and continue
                dividing the region in two. If $f(x_{2})<z$, 
                we set $x_{3}=\frac{x_{1}+x_{2}}{2}$, and if
                $f(x_{2})>z$ we set $x_{3}=\frac{x_{2}+b}{2}$.
                Note that $|x_{n+1}-x_{n}|=\frac{b-a}{2^{n+1}}$.
                Moreover, $x_{n}$ converges. Suppose it converges to
                $c$. Then $c\in[a,b]$. That is, the limit of a
                function $x_{n}:\mathbb{N}\rightarrow[a,b]$ is
                contained in $[a,b]$. This is related to the
                ``compactness'' of $[a,b]$. Moreover, it is related
                to the ``closedness'' of $[a,b]$. If there is an
                $N\in\mathbb{N}$ such that $f(x_{N})=z$, then we are
                done. Suppose not. Let $k_{n}$ be the
                elements such that $f(x_{k_{n}})<z$ and $\ell_{n}$
                be the elements such that $f(x_{\ell_{n}})>z$.
                Both of these must be infinite. For suppose not.
                Suppose there is a final $N$ such that
                $f(x_{N})<z$. Then for all $n>N$, $f(x_{n})>z$.
                But from how we've defined the sequence $x_{n}$,
                we have that $x_{n}\rightarrow{x_{N}}$. From the
                continuity of $f$, there is an open interval
                about $X_{N}$ such that for all elements $x$
                inside that interval, we have that$f(x)<z$.
                A contradiction, since eventually some of the
                $x_{n}$ will be in this interval, and thus
                $f(x_{n})<z$. So, both $k_{n}$ and $\ell_{n}$ are
                infinite. From continuity, we have
                $\lim_{n\rightarrow\infty}f(x_{k_{n}})\leq{z}$
                and
                $\lim_{n\rightarrow\infty}f(x_{\ell_{n}})\geq{z}$.
                Thus, $f(c)\leq{z}$ and $f(c)\geq{z}$, and therefore
                $f(c)=z$.
            \end{proof}
            This theorem fails in $\mathbb{Q}$, for it relies on
            the completeness of $\mathbb{R}$. For example,
            $f(x)=x^{2}$ defined on $[0,4]$. Then $2\in[0,4]$, but
            there is no rational such that $x^{2}=2$
            (We use the irrationality of $\sqrt{2}$ a lot, huh?)
            A cuter way to phrase this, in a more topological
            sense, is that the image of $[a,b]$, which is an
            interval, or a connected subset of $\mathbb{R}$,
            is again an interval, or a connected subset of
            $\mathbb{R}$. The proof that continuous functions take
            connected sets (Intervals) to connected sets
            (Again, intervals) is a lot easier than the one presented
            here, but relies on notions from topology.
            So we'll skip that.
            Another commonly used theorem in calculus
            (Again, usually not proved) is the extreme value theorem.
            The extreme value is used to proved Rolle's theorem,
            which says that if $f$ is differentiable on $(a,b)$
            and if $f(a)=f(b)$, then there is a point
            $c\in(a,b)$ such that $f'(c)=0$. This is used to
            prove the mean value theorem, which says that
            if $f$ is differentiable on $(a,b)$, then there is a point
            $c\in(a,b)$ such that
            $f'(c)=\frac{f(b)-f(a)}{b-a}$. This is in turned used to
            prove the Fundamental Theorem of Calculus. So some very
            important stuff going on here. First we prove that
            continuous functions on closed and bounded sets (That is,
            compact sets) are bounded. We stick to closed intervals
            for now.
            \begin{theorem}
                If $f:[a,b]\rightarrow\mathbb{R}$ is continuous,
                then it is bounded.
            \end{theorem}
            \begin{proof}
                Suppose not. Then for all $n\in\mathbb{N}$, there is
                an $x_{n}\in[a,b]$ such that $f(x_{n})>n$. But then
                $x_{n}$ is a bounded sequence, and thus by
                Bolzano-Weierstrass there is a convergent subsequence.
                Let $x$ be the limit of this convergent subsequence.
                But then $f(x_{k_{n}})\rightarrow{f(x)}$, this is
                the fundamental property of continuous functions.
                This is indeed equivalent to the standard
                $\varepsilon-\delta$ definition of continuity.
                But $f(x_{k_{n}})\rightarrow\infty$, and $f(x)<\infty$,
                a contradition. Therefore, etc.
            \end{proof}
            \begin{theorem}[Exreme Value Theorem]
                If $f:[a,b]\rightarrow\mathbb{R}$ is continuous,
                then there exists $c\in[a,b]$ such that
                for all $x\in[a,b]$, $f(x)\leq{f(c)}$
            \end{theorem}
            \begin{proof}
                By the previous theorem,
                $\{f(x):x\in[a,b]\}$ is bounded. By completeness,
                there is a least upper bound. Let $s$ be such
                a bound. If $s$ is the least upper bound, then
                $s-1$ is not a least upper bound. Therefore this is
                an $x_{0}\in[a,b]$ such that $s-1<f(x_{0})$. Similarly,
                for all $n\in\mathbb{N}$, there is an
                $x_{n}\in\mathbb{N}$ such that
                $s-\frac{1}{n}<f(x_{n})$. But $x_{n}$ is a bounded
                sequence, and bounded sequences have a convergent
                subsequence. Let $x$ be the limit of this
                subsequence. From continuity,
                $f(x)=\lim_{n\rightarrow\infty}f(x_{k_{n}})$.
                But $s-\frac{1}{n}\leq{f(x_{k_{n}})}\leq{s}$,
                and therefore $f(x_{k_{n}})\rightarrow{s}$.
                Thus, $f(x)=s$.
            \end{proof}
            Much the way the intermediate value theorem can be
            generalized to say that the continuous image of
            connected sets is connected, the extreme value
            theorem can be generalized to say that the
            continuous image of compact sets is compact.
            The proof is rather easy, but requires
            topology. So, we'll skip that too.
            The requirement of these previous theorems on
            continuity is crucial. Without continuity, functions
            on $[a,b]$ need not be bounded. Without continuity,
            functions on $(a,b)$ can just ``jump,'' right over
            other points. Continuity is very important. It's so
            important, let's talk about it for a moment.
        \subsubsection{Continuity}
            \begin{definition}
                A function $f:S\rightarrow\mathbb{R}$
                on a subset $S\subset\mathbb{R}$ continuous
                at a point $x\in{S}$ is a function such that
                for all $\varepsilon>0$ there is a $\delta>0$
                such that for all $x_{0}\in{S}$ where
                $|x-x_{0}|<\delta$, we have
                $|f(x)-f(x_{0}|<\varepsilon$.
            \end{definition}
            We have used this several times already. It is
            equivalent to the following.
            \begin{theorem}
                A function $f:S\rightarrow\mathbb{R}$
                is continuous at a point $x\in{S}$ if
                and only if for all sequences
                $x_{n}:\mathbb{N}\rightarrow{S}$ such that
                $x_{n}\rightarrow{x}$, we have
                $f(x_{n})\rightarrow{f(x)}$.
            \end{theorem}
            \begin{definition}
                A continuous function $f:S\rightarrow\mathbb{R}$
                is a function that is continuous at all $x\in{S}$.
            \end{definition}
            This definition comes from the fact that
            continuity is a point-wise property, and not a
            ``curve'' property. Continuous functions are
            functions that have point-wise continuity at
            every point. The statement ``A continuous function
            is a curve that you can draw,'' which many have
            heard in calculus is slightly misleading. There
            are functions that are continuous at one point and
            no where else. There are functions that are
            continuous on the irrationals and discontinuous
            on the rationals. For this beast, if $x$ is
            rational, write it as $x=\frac{p}{q}$ where $p$ and
            $q$ are integers and relatively prime. Define
            $f(x)=\frac{1}{q}$. If $x$ is irrational, define
            $f(x)=0$. This function is continuous at every
            irrational number and discontinuous at every
            rational number. There is no ``reverse,'' of this
            function. That is, there is no function continuous
            on $\mathbb{Q}$ and discontinuous on every
            irrational. Uniform continuity is a property of
            all points in the domain of a function. In
            ``fancy,'' notation, we can write the definition
            of a continuous function as follows:
            \begin{definition}
                A continuous function on a set $S$ is a
                function $f:S\rightarrow\mathbb{R}$ such that
                $\forall_{x\in{S}}\forall_{\varepsilon>0}%
                 \exists_{\delta>0}:\forall_{x_{0}\in{S}}:%
                 |x-x_{0}|<\delta%
                 \Rightarrow|f(x)-f(x_{0})|<\varepsilon$
            \end{definition}
            This says, give me a point $x$
            and a positive number
            $\varepsilon$ and I can find a $\delta$ satisfying
            this property. The key part is that you must
            specify the point first. That is, the $\delta$ I
            choose may be very dependent on the $x$ you choose.
            Uniform continuity occurs when a $\delta>0$ can be
            chosen regardless of the $x$. The $\delta$ is
            only dependent on the $\varepsilon$ chosen.
            \begin{definition}
                A uniformly continuous function on a set $S$
                is a function $f:S\rightarrow\mathbb{R}$ such that
                $\forall_{\varepsilon>0}\exists_{\delta>0}%
                 \forall_{x\in{S}}:\forall_{x_{0}\in{S}}:%
                 |x-x_{0}|<\delta,|f(x)-f(x_{0})|<\varepsilon$
            \end{definition}
            \begin{theorem}
                If $f:S\rightarrow\mathbb{R}$ is uniformly
                continuous, and if $x_{n}$ and $y_{n}$
                are sequences in $S$ such that
                $x_{n}-y_{n}\rightarrow{0}$, then
                $f(x_{n})-f(y_{n})\rightarrow{0}$.
            \end{theorem}
            The requirement of uniform continuity is crucial.
            Let $f:(0,1)\rightarrow\mathbb{R}$ be defined by
            $f(x)=\frac{1}{x}$. Then $f$ is continuous, but
            not uniformly continuous. Let $x_{n}=\frac{1}{n}$
            and $y_{n}=\frac{1}{2n}$. Then
            $y_{n}-x_{n}=\frac{1}{2n}\rightarrow{0}$, but
            $f(y_{n})-f(x_{n})=2n-n=n$, and that diverges.
            \begin{theorem}
                If $f:[a,b]\rightarrow\mathbb{R}$ is
                continuous, then it is uniformly continuous.
            \end{theorem}
            The above theorem relies on the fact that
            $[a,b]$ is closed and bounded. Indeed, this is
            the only thing it relies on, the fact that it's
            an interval (Or connected) it unnecessary. We can
            write a more general result.
            \begin{theorem}
                If $f:S\rightarrow\mathbb{R}$ is continuous
                and if $S$ is compact, then $f$ is
                uniformly continuous.
            \end{theorem}
            We end with a brief discussion on sequences of
            functions.
        \subsubsection{Sequences of Functions}
            \begin{definition}
                A sequence of functions
                $f_{n}:S\rightarrow\mathbb{R}$ converges
                point-wise to a function $f$ if
                for all $x\in{S}$, $f_{n}(x)\rightarrow{f(x)}$
            \end{definition}
            Rewriting this in ``fancy,'' notation:
            \begin{definition}
                $f_{n}\rightarrow{f}$ point-wise if
                $\forall_{x\in{S}}\forall_{\varepsilon>0}%
                 \exists_{N\in\mathbb{N}}:\forall_{n>N},%
                 |f(x)-f_{n}(x)|<\varepsilon$
            \end{definition}
            Uniform continuity requires that all of the
            points of the domain converge to $f(x)$ at
            the same speed. That is, given any $\varepsilon>0$
            there is an $N\in\mathbb{N}$ that works for
            all points. Point-Wise convergence does not
            have this property. It is possible for a sequence
            of functions to converge, point-wise, to zero,
            and yet there is a sequence $x_{n}$ such that
            $f_{n}(x_{n})\rightarrow\infty$. Uniform
            convergence does not allow this.
            \begin{definition}
                $f_{n}\rightarrow{f}$ uniformly if
                $\forall_{\varepsilon>0}%
                 \exists_{N\in\mathbb{N}}%
                 \forall_{x\in{S}}:\forall_{n>N},%
                 |f(x)-f_{n}(x)|<\varepsilon$
            \end{definition}
            \begin{theorem}
                If $f_{n}$ is a sequence of
                continuous functions and if
                $f_{n}\rightarrow{f}$ uniformly, then
                $f$ is continuous.
            \end{theorem}
            The word ``uniformly,'' is crucial in
            this definition. This theorem is not
            necessarily true of point-wise converging
            functions. Let
            $f_{n}:[0,1]\rightarrow[0,1]$ be defined by
            $f(x)=x^{n}$. Then $f_{n}$ converges to
            $0$ if $x\ne{1}$, and $1$ if $x=1$. That is,
            the limit function is discontinuous. This is
            possible because $f_{n}$ does not
            converge uniformly, only point-wise.
\end{document}