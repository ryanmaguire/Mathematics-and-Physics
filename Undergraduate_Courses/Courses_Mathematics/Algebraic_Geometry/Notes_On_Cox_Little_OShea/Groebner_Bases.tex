\documentclass[crop=false,class=article,oneside]{standalone}
%----------------------------Preamble-------------------------------%
%---------------------------Packages----------------------------%
\usepackage{geometry}
\geometry{b5paper, margin=1.0in}
\usepackage[T1]{fontenc}
\usepackage{graphicx, float}            % Graphics/Images.
\usepackage{natbib}                     % For bibliographies.
\bibliographystyle{agsm}                % Bibliography style.
\usepackage[french, english]{babel}     % Language typesetting.
\usepackage[dvipsnames]{xcolor}         % Color names.
\usepackage{listings, lstlinebgrd}      % Verbatim-Like Tools.
\usepackage{mathtools, esint, mathrsfs} % amsmath and integrals.
\usepackage{amsthm, amsfonts}           % Fonts and theorems.
\usepackage{tabularx}
\usepackage{tcolorbox}                  % Frames around theorems.
\usepackage{upgreek}                    % Non-Italic Greek.
\usepackage{paracol}                    % Two-column styling.
\usepackage{wrapfig}                    % Wrap text around figure.
\usepackage{fmtcount, etoolbox}         % For the \book{} command.
\usepackage[newparttoc]{titlesec}       % Formatting chapter, etc.
\usepackage{titletoc}                   % Allows \book in toc.
\usepackage[nottoc]{tocbibind}          % Bibliography in toc.
\usepackage[titles]{tocloft}            % ToC formatting.
\usepackage{multicol, enumitem}         % Multi-column/enumerate.
\usepackage{import}                     % Import external files.
\usepackage{pgfplots, tikz}             % Drawing/graphing tools.
\usetikzlibrary{
    calc,                   % Calculating right angles and more.
    angles,                 % Drawing angles within triangles.
    arrows.meta,            % Latex and Stealth arrows.
    quotes,                 % Adding labels to angles.
    positioning,            % Relative positioning of nodes.
    decorations.markings,   % Adding arrows in the middle of a line.
    patterns,
    arrows,
    shapes,
    shapes.geometric,
    cd,
    hobby,
    babel
}                                       % Libraries for tikz.
\pgfplotsset{compat=1.9}                % Version of pgfplots.
\usepackage[font=scriptsize,
            labelformat=simple,
            labelsep=colon]{subcaption} % Subfigure captions.
\usepackage[font={scriptsize},
            hypcap=true,
            labelsep=colon]{caption}    % Figure captions.
\usepackage{hyperref}                   % Allows for hyperlinks.
\hypersetup{
    colorlinks=true,
    linkcolor=blue,
    filecolor=magenta,
    urlcolor=Cerulean,
    citecolor=SkyBlue
}                           % Colors for hyperref.
\usepackage[toc,acronym,nogroupskip]{glossaries} % Glossaries and acronyms.
\usepackage[subpreambles=false]{standalone}      % Complileable sub files.

% Various font stuff from kiwi.
% Use this for Times text and Computer Modern math
%\usepackage{times}

% Quite nice
%\usepackage[charter, greekfamily=, greekuppercase=italicized]{mathdesign}
%\usepackage[utopia, greekuppercase=italicized]{mathdesign}    % Math is narrower

% Use this for Times text and math
%\usepackage{newtxtext}
%\usepackage[libertine,cmintegrals]{newtxmath}
%\usepackage{fix-cm}

%\usepackage{txfontsb}
% or
%\usepackage{mathptmx}

%\usepackage[scaled=0.92]{helvet}
%\renewcommand{\rmdefault}{ptm}

%\usepackage{mathpazo}    % add possibly `sc` and `osf` options
%\usepackage{eulervm}

%\usepackage{fourier}
%\renewcommand{\rmdefault}{ptm}
%\usepackage{mathptm}

%\usepackage{fontspec}
%\setmainfont{lmodern}

%\usepackage[varg]{txfonts}
%\usepackage{fouriernc}
%\usepackage{mathpazo}

%\usepackage{bookman}
%\usepackage[scaled]{uarial}
%\usepackage[scaled]{helvet}
%\renewcommand*\familydefault{\sfdefault}
%\usepackage[math]{anttor}

%\newcommand\fgeorgia{\fontfamily{jvn}\selectfont}
%\newcommand\ftimes{\fontfamily{ptm}\selectfont}
%\newcommand\fhelvetica{\fontfamily{phv}\selectfont}
%\newcommand\fcourier{\fontfamily{pcr}\selectfont}
%\newcommand\fbookman{\fontfamily{pbk}\selectfont}
%\newcommand\fnewcentury{\fontfamily{pnc}\selectfont}
%\newcommand\fpalatino{\fontfamily{ppl}\selectfont}
%\newcommand\favantgarde{\fontfamily{pag}\selectfont}
%\newcommand\fnormal{\normalfont}
%\newcommand\fsize[1]{\ifnum#1>0\fontsize{#1}{#1}\selectfont\else\normalsize\fi}
%------------------------Theorem Styles-------------------------%
% Define theorem style for default spacing and normal font.
\newtheoremstyle{normal}
    {\topsep}               % Amount of space above the theorem.
    {\topsep}               % Amount of space below the theorem.
    {}                      % Font used for body of theorem.
    {}                      % Measure of space to indent.
    {\bfseries}             % Font of the header of the theorem.
    {}                      % Punctuation between head and body.
    {.5em}                  % Space after theorem head.
    {}

% Define theorem style for default spacing with italicized font.
\newtheoremstyle{normalit}{\topsep}{\topsep}
                {\itshape}{}{\bfseries}{}{.5em}{}

% Italic header environment.
\newtheoremstyle{thmit}{\topsep}{\topsep}{}{}{\itshape}{}{0.5em}{}

% Define italicized environments.
\theoremstyle{normalit}
\newtheorem{theorem}{Theorem}[section]
\newtheorem{lemma}{Lemma}[section]
\newtheorem{corollary}{Corollary}[section]
\newtheorem{proposition}{Proposition}[section]
\newtheorem*{theorem*}{Theorem}

% Define environments with italic headers.
\theoremstyle{thmit}
\newtheorem*{solution}{Solution}
\newtheorem*{fsolution}{Solution}

% Define default environments.
\theoremstyle{normal}
\newtheorem{example}{Example}[section]
\newtheorem{definition}{Definition}[section]
\newtheorem{problem}{Problem}[section]
\newtheorem{question}{Question}[section]
\newtheorem{remark}{Remark}[section]
\newtheorem{properties}{Properties}[section]
\newtheorem{notation}{Notation}[section]
\newtheorem{axiom}{Axiom}[section]
\newtheorem*{properties*}{Properties}
\newtheorem*{remark*}{Remark}
\newtheorem*{definition*}{Definition}
\theoremstyle{plain}

% Define framed environment.
\tcbuselibrary{most}
\newtcbtheorem[use counter*=theorem]{ftheorem}{Theorem}%
    {colback=green!5,colframe=green!35!black,
     fonttitle=\bfseries\upshape}{th}

\newtcbtheorem[use counter*=example]{fdefinition}{Definition}%
    {fonttitle=\bfseries\upshape,
     colback=blue!5!white,colframe=blue!75!black}{def}

\newtcbtheorem[use counter*=example]{fexample}{Example}%
    {fonttitle=\bfseries\upshape,
     colback=red!5!white,colframe=red!75!black}{ex}

\newtcbtheorem[use counter*=notation]{fnotation}{Notation}%
    {fonttitle=\bfseries\upshape,
     colback=SeaGreen!5!white,colframe=SeaGreen!75!black}{ex}

\newtcbtheorem[use counter*=corollary]{fcorollary}{Corollary}%
    {fonttitle=\bfseries\upshape,
     colback=Orchid!5!white,colframe=Orchid!75!black}{ex}

\newenvironment{bproof}{\textit{Proof.}}{\hfill$\square$}
\tcolorboxenvironment{bproof}{blanker,breakable,left=5mm,
                             before skip=10pt,after skip=10pt,
                             borderline west={1mm}{0pt}{red}}
\tcolorboxenvironment{fsolution}
    {enhanced jigsaw,colframe=cyan,interior hidden,breakable}

%--------------------Declared Math Operators--------------------%
\DeclareMathOperator{\Refl}{Refl}           % Reflection operator.
\DeclareMathOperator{\Span}{Span}           % Span of a set of vectors.
\DeclareMathOperator{\Card}{Card}           % Cardinality of set.
\DeclareMathOperator{\Ord}{Ord}             % Ordinal of ordered set.
\DeclareMathOperator{\Tr}{Tr}               % Trace of matrix.
\DeclareMathOperator{\adjoint}{adj}         % Adjoint of matrix.
\DeclareMathOperator{\rk}{rk}               % Rank of operator.
\DeclareMathOperator{\nul}{nul}             % Null space of operator.
\DeclareMathOperator{\sgn}{sgn}             % Sign of a number.
\DeclareMathOperator{\multideg}{mutlideg}   % Multi-Degree (Graphs).
\DeclareMathOperator{\GCD}{GCD}             % Greatest common denominator.
\DeclareMathOperator{\LM}{LM}               % Leading monomial
\DeclareMathOperator{\LC}{LC}               % Leading coefficient.
\DeclareMathOperator{\LT}{LT}               % Leading term.
\DeclareMathOperator{\LCM}{LCM}             % Least common multiple.
\DeclareMathOperator{\Mon}{Mon}             % Monomial.
\DeclareMathOperator{\Spec}{Spec}           % Spectrum.
\DeclareMathOperator{\proj}{proj}           % Projection.
\DeclareMathOperator{\comp}{comp}           % Component.
\DeclareMathOperator{\sinc}{sinc}           % Sinc function.
\DeclareMathOperator{\Ima}{Im}              % Image of operator.
\DeclareMathOperator{\Prin}{Prin}           % Principal value.
\DeclareMathOperator{\Mod}{mod}             % Modulus.
%------------------------New Commands---------------------------%
\DeclarePairedDelimiter\norm{\lVert}{\rVert}
\DeclarePairedDelimiter\ceil{\lceil}{\rceil}
\DeclarePairedDelimiter\floor{\lfloor}{\rfloor}
\newcommand*\diff{\mathop{}\!\mathrm{d}}
\newcommand*\Diff[1]{\mathop{}\!\mathrm{d^#1}}
\renewcommand{\mod}{\ \Mod}
\renewcommand*{\glstextformat}[1]{\textcolor{RoyalBlue}{#1}}
\renewcommand{\glsnamefont}[1]{\textbf{#1}}
\renewcommand\labelitemii{$\circ$}
\renewcommand\thesubfigure{\arabic{chapter}.\arabic{figure}}
\renewcommand\thesubfigure{%
    \arabic{chapter}.\arabic{figure}.\arabic{subfigure}}
\addto\captionsenglish{\renewcommand{\figurename}{Fig.}}
%------------------------Book Command---------------------------%
\makeatletter
\renewcommand\@pnumwidth{1cm}
\newcounter{book}
\renewcommand\thebook{\@Roman\c@book}
\newcommand\book{%
    \if@openright
        \cleardoublepage
    \else
        \clearpage
    \fi
    \thispagestyle{plain}%
    \if@twocolumn
        \onecolumn
        \@tempswatrue
    \else
        \@tempswafalse
    \fi
    \null\vfil
    \secdef\@book\@sbook
}
\def\@book[#1]#2{%
    \ifnum \c@secnumdepth >-3\relax
        \refstepcounter{book}%
        \addcontentsline{toc}{book}{
            \bookname\ \thebook:\hspace{1em}#1
        }
    \else
        \addcontentsline{toc}{book}{#1}%
    \fi
    \markboth{}{}%
    {\centering
     \interlinepenalty \@M
     \normalfont
     \ifnum \c@secnumdepth >-2\relax
       \huge\bfseries \bookname\nobreakspace\thebook
       \par
       \vskip 20\p@
     \fi
     \Huge \bfseries #2\par}%
    \@endbook}
\def\@sbook#1{%
    {\centering
     \interlinepenalty \@M
     \normalfont
     \Huge \bfseries #1\par}%
    \@endbook}
\def\@endbook{
    \vfil\newpage
        \if@twoside
            \if@openright
                \null
                \thispagestyle{empty}%
                \newpage
            \fi
        \fi
        \if@tempswa
            \twocolumn
        \fi
}
\newcommand*\l@book[2]{%
    \ifnum \c@tocdepth >-2\relax
        \addpenalty{-\@highpenalty}%
        \addvspace{2.25em \@plus\p@}%
        \setlength\@tempdima{3em}%
        \begingroup
            \parindent \z@ \rightskip \@pnumwidth
            \parfillskip -\@pnumwidth
            {
                \leavevmode
                \Large \bfseries #1\hfil \hb@xt@\@pnumwidth{
                    \hss #2
                }
            }
            \par
            \nobreak
            \global\@nobreaktrue
            \everypar{\global\@nobreakfalse\everypar{}}%
        \endgroup
    \fi}
\newcommand\bookname{Book}
\renewcommand{\thebook}{\texorpdfstring{\Numberstring{book}}{book}}
\providecommand*{\toclevel@book}{-2}
\makeatother
\titlecontents{chapter}[0pt]
    {\bfseries}
    {\chaptername\ \thecontentslabel:\quad}
    {}
    {\hfill\contentspage}
\titleformat{\part}[display]
    {\Large\bfseries}
    {\partname\nobreakspace\thepart}
    {0mm}
    {\Huge\bfseries}
    \titlecontents{part}[0pt]
    {\large\bfseries}
    {\partname\ \thecontentslabel: \quad}
    {}
    {\hfill\contentspage}
\newcommand{\MarkRightAngle}[4][.3cm]
    {\coordinate (tempa) at ($(#3)!#1!(#2)$);
     \coordinate (tempb) at ($(#3)!#1!(#4)$);
     \coordinate (tempc) at ($(tempa)!0.5!(tempb)$);%midpoint
     \draw (tempa) -- ($(#3)!2!(tempc)$) -- (tempb);}
%--------------------------LENGTHS------------------------------%
% Spacings for the Table of Contents.
\addtolength{\cftsecnumwidth}{1ex}
\addtolength{\cftsubsecindent}{1ex}
\addtolength{\cftsubsecnumwidth}{1ex}
\addtolength{\cftfignumwidth}{1ex}
\addtolength{\cfttabnumwidth}{1ex}

% Spacing for multi-column and enumerate environments.
\setlength{\multicolsep}{6pt}
\setlist[enumerate]{itemsep=0pt,topsep=3pt}

% Indent and paragraph spacing.
\setlength{\parindent}{0em}
\setlength{\parskip}{0em}
%--------------------------Main Document----------------------------%
\begin{document}
    \ifx\ifgeoalg\undefined
        \section*{Electromagnetism I}
        \setcounter{section}{1}
    \fi
    \subsection{Groebner Bases}
        \subsubsection{Introduction}
        There are three problems we wish to address:
        \begin{enumerate}
            \item Does every Ideal $I\subset k[x_1,\hdots ,x_n]$ have a finite generating set?
            \item Given $f\in k[x_1,\hdots ,x_n]$, and $I = \langle f_1,\hdots, f_s\rangle$, can we determine if $f\in I?$
            \item For $f_1,\hdots, f_s \in k[x_1,\hdots ,x_n]$, can we determine what $\mathbf{V}(f_1,\hdots, f_s)$ is?
        \end{enumerate}
        We've already solved this in the case of one variable, $n=1$. The case of $n\in \mathbb{N}$ where $f_1,\hdots, f_s$ are linear functions is the subject of linear algebra. Both the Eucldiean algorithm and the methods of linear algebra require a notion of ordering of terms. In the case of one variable, if $n>m$ we write $x^n>x^m$. In the case of linear algebra we usually write $x_n>x_{n-1}>\hdots > x_2 > x_1$. 
        \subsubsection{Orderings on the Monomials in \texorpdfstring{$k[x_1,\hdots ,x_n]$}{kx}}
        \begin{definition}
        A monomial ordering on $k[x_1,\hdots, x_n]$ is any relation $\succ$ on $\mathbb{N}^n$ such that:
        \begin{enumerate}
            \item $\succ$ is a total ordering.
            \item If $\alpha \succ \beta$ and $\gamma \in \mathbb{N}^n$, then $\alpha+\gamma \succ \beta + \gamma$.
            \item $\succ$ is a well-ordering on $\mathbb{N}^n$. 
        \end{enumerate}
        \end{definition}
        \begin{theorem}
        An ordering $\prec$ on $\mathbb{N}^n$ is a well-ordering if and only if for any monotonically decreasing sequence $\{a_n\}_{n=1}^{\infty}$, there is an $N\in \mathbb{N}$ such that for all $n>N$, $a_n = a_N$.
        \end{theorem}
        \begin{proof}
        For if $\prec$ is a well ordering, then $\{a_n\}_{n=1}^{\infty}$ contains a least element $x$. Suppose $a_n$ contains a strictly decreasing subsequence. But $\prec$ is a well ordering, and therefore $\{a_n\}_{n=1}^{\infty}$ contains a least element $x$. But again $\prec$ is a well ordering, and thus $\{a_n\}_{n=1}^{\infty} \setminus \{x\}$ contains a least element $y$. But then $x\prec y$, and $x$ is the least element of $\{a_n\}_{n=1}^{\infty}$. Therefore there is an $a_n$ such that $x\preceq a_n \preceq y$. But $a_n$ is strictly decreasing, and therefore $a_{n+1} \preceq x$, and thus $a_{n+2} \prec x$. But $x$ is the least element of $\{a_n\}_{n=1}^{\infty}$, a contradiction. Therefore $a_n$ contains no strictly increasing subsequence. Suppose every decreasing sequence eventually terminates. Let $E\subset \mathbb{N}^n$. Suppose there is no least element. Then we can construct a strictly decreasing sequence. But every decreasing sequence eventually terminates, a contradiction. Therefore, etc.
        \end{proof}
        \begin{definition}
        If $\alpha,\beta \in \mathbb{N}^n$, then $\alpha$ is said to be lexicographically greater than $\beta$, denoted $\underset{Lex}{>}$, if the left-most entry of $\alpha-\beta$ is positive.
        \end{definition}
        \begin{theorem}
        The Lexicographic Ordering is a monomial ordering.
        \end{theorem}
        \begin{definition}
        The graded lexicographic ordering $\underset{GrLex}{>}$ on $\mathbb{N}^n$ is an ordering on $\mathbb{N}^n$ such that $\alpha \underset{GrLex}{>}\beta$ if and only if either $|\alpha|>|\beta|$, or $|\alpha| = |\beta|$ and $\alpha \underset{Lex}{>}\beta$.
        \end{definition}
        \begin{theorem}
        The graded lexicographic ordering is a monomial ordering.
        \end{theorem}
        \begin{definition}
        For $f=\sum_{\alpha} a_{\alpha} x^\alpha \in k[x_1,\hdots ,x_n]$, and $\prec$ a monomial ordering, the multidegree of $f$ is $\multideg(f) = \max\{\alpha\in\mathbb{N}^n: a_{\alpha} \ne 0\}$.
        \end{definition}
        \begin{definition}
        For $f=\sum_{\alpha}a_\alpha x^\alpha \in k[x_1,\hdots ,x_n]$ and monomial order $>$, the leading coefficient of $f$ is $LC(f) =a_{\multideg(f)}\in k$
        \end{definition}
        \begin{definition}
        For $f=\sum_{\alpha} a_{\alpha} x^\alpha \in k[x_1,\hdots ,x_n]$, and $\prec$ a monomial ordering, the leading monomial of $f$ is $\LM(f) = x^{\multideg(f)}$
        \end{definition}
        \begin{definition}
        For $f=\sum_{\alpha} a_{\alpha} x^{\alpha} \in k[x_1,\hdots ,x_n]$, and $\prec$ a monomial ordering, the leading term of $f$ is $\LT(f) = \LC(f)\cdot \LM(f)$.
        \end{definition}
        \begin{theorem}
        If $f,g\in k[x_1,\hdots ,x_n]$ are non-zero, then $\multideg(fg)=\multideg(f)+\multideg(g)$
        \end{theorem}
        \begin{theorem}
        If $f,g\in k[x_1,\hdots ,x_n]$ are non-zero, and if $f+g \ne 0$, then $\multideg(f+g) \leq \max\{\multideg(f),\multideg(g)\}$.
        \end{theorem}
        \begin{theorem}
        If $f,g\in k[x_1,\hdots ,x_n]$ are non-zero, $f+g\ne 0$, and if $\multideg(f)\ne \multideg(g)$, then $\multideg(f+g) = \max\{\multideg(f),\multideg(g)\}$.
        \end{theorem}
        \begin{theorem}
        If $>$ is a monomial ordering on $\mathbb{N}^n$, and $F = (f_1,\hdots, f_s)$ is an ordered $s-$tuple of polynomials in $k[x_1,\hdots ,x_n]$, then every $f\in k[x_1,\hdots ,x_n]$ can be written as $f = r+\sum_{k=1}^{s} a_k f_k$, where $a_k,r\in k[x_1,\hdots ,x_n]$, and either $r=0$ or $r$ is a linear combination, with coefficients in $k$, of monomials, none of which is divisible by any of $\LT(f_1),\hdots, \LT(f_s)$. We call $r$ the remainder of $f$ with respect to $F$.
        \end{theorem}
        \begin{definition}
        An ideal $I\subset k[x_1,\hdots ,x_n]$ is a monomial ideal if there is a subset $A\subset \mathbb{N}^n$ such that $I$ consists of all polynomials which are finite sums of the form $\sum_{\alpha} h_{\alpha} x^\alpha$, where $h_{\alpha} \in k[x_1,\hdots ,x_n]$. 
        \end{definition}
        \begin{theorem}
        If $I=\langle x^\alpha: \alpha \in A\}$ is a monomial ideal, then a monomial $x^\beta$ lies in $I$ if and only if $x^\beta$ is divisible by $x^\alpha$ for some $\alpha \in A$.
        \end{theorem}
        \begin{theorem}
        If $I$ is a monomial ideal, and $f\in k[x_1,\hdots ,x_n]$, then the following are equivalent:
        \begin{enumerate}
                \item $f\in I$
                \item Every term of $f$ lies in $I$.
                \item $f$ is a $k-$linear combination of the monomials in $I$.
        \end{enumerate}
        \end{theorem}
        \begin{theorem}[Dickson's Lemma]
        If $I=\langle x^\alpha: \alpha \in A\rangle$ is a monomial ideal, then $I$ can be written as $\langle x^{\alpha(1)}, \hdots, x^{\alpha(s)}\rangle$, where $\alpha(1),\hdots, \alpha(s) \in A$. 
        \end{theorem}
        \begin{theorem}
        If $>$ is a relation on $\mathbb{N}^n$ such that $>$ is a total ordering and for $\alpha>\beta$ and $\gamma\in \mathbb{N}^n$, $\alpha+\gamma>\beta+\gamma$, then $>$ is a well-ordering if and only if for all $\alpha \in \mathbb{N}^n$, $\alpha \geq 0$.
        \end{theorem}
        \subsubsection{The Hilbert Basis Theorem and Groebner Bases}
        \begin{definition}
        For a non-zero ideal $I\subset k[x_1,\hdots ,x_n]$, $\LT(I)$ is the set of leading terms of elements of $I$. $\langle \LT(I)\rangle$ is the ideal generated by this set.
        \end{definition}
        \begin{theorem}
        If $I\subset k[x_1,\hdots ,x_n]$ is an ideal, then $\langle \LT(I)\rangle$ is a monomial ideal.
        \end{theorem}
        \begin{theorem}
        If $I\subset k[x_1,\hdots ,x_n]$ is an ideal, then there are $g_1,\hdots, g_t\in I$ such that $\langle \LT(I)\rangle = \langle \LT(g_1),\hdots, \LT(g_t)\rangle$
        \end{theorem}
        \begin{theorem}[Hilbert Basis Theorem]
        Every ideal $I\subset k[x_1,\hdots ,x_n]$ has a finite generating set.
        \end{theorem}
        \begin{definition}
        For a monomial order $>$, a finite subset $G=\{g_1,\hdots, g_t\}$ of an ideal $I$ is said to be a Groebner Basis if $\langle \LT(g_1),\hdots, \LT(g_t)\rangle = \langle \LT(I)\rangle$
        \end{definition}
        \begin{theorem}
        If $>$ is a monomial order, then every non-zero ideal $I\subset k[x_1,\hdots ,x_n]$ has a Groebner basis.
        \end{theorem}
        \begin{theorem}
        If $I\subset k[x_1,\hdots ,x_n]$ is a non-zero ideal and $G$ is a Groebner Basis, then $G$ is also a generated set of $I$.
        \end{theorem}
        \begin{theorem}[The Ascending Chain Condition]
        If $I_n$ is a sequence of ideals such that $I_{n}\subset I_{n+1}$, then there is an $N\in \mathbb{N}$ such that for all $n>N$, $I_n = I_N$.
        \end{theorem}
        \begin{definition}
        If $I\subset k[x_1,\hdots ,x_n]$ is an ideal, then $\textbf{V}(I)$ is the set $\{\alpha \in k^n: \forall_{f\in I},f(\alpha) = 0\}$
        \end{definition}
        \begin{theorem}
        If $I\subset k[x_1,\hdots ,x_n]$ is an ideal, then $\textbf{V}(I)$ is an affine variety.
        \end{theorem}
        \begin{theorem}
        If $I = \langle f_1,\hdots, f_s\rangle$, then $\textbf{V}(I) = \mathbf{V}(f_1,\hdots, f_s)$.
        \end{theorem}
        \subsubsection{Properties of Groebner Bases}
        \begin{theorem}
        If $G=\{g_1,\hdots, g_t\}$ is a Groebner basis of $I\subset k[x_1,\hdots ,x_n]$ and $f\in k[x_1,\hdots ,x_n]$, then there is a unique $r\in k[x_1,\hdots ,x_n]$ such that $r$ is not divisible by any of $\LT(g_1),\hdots, \LT(g_t)$, and there is a $g\in I$ such that $f = g+r$. 
        \end{theorem}
        \begin{notation}
        We write $\overline{f}^{F}$ for the remainder on division of $f$ by $F = (f_{1},\hdots,f_{s})$
        \end{notation}
        \begin{definition}
        If $f,g\in k[x_1,\hdots ,x_n]$ are non-zero polynomials, $\multideg(f) = \alpha$, $\multideg(g) = \beta$, and if $\gamma = (\gamma_1,\hdots, \gamma_n)$, where $\gamma_k = \max\{\alpha_k,\beta_k\}$, then $x^y$ is the least common multiple of $\LM(f)$ and $\LM(g)$, denoted $x^y = \LCM(\LM(f),\LM(g))$.
        \end{definition}
        \begin{definition}
        If $f,g\in k[x_1,\hdots ,x_n]$ are non-zero, then the $S-$polynomial of $f$ and $g$ is $S(f,g) = \frac{x^y}{\LT(f)}f - \frac{x^y}{\LT(g)}g$
        \end{definition}
        \begin{theorem}[Buchberger's Criterion]
        If $I$ is a polynomial ideal, then a basis $G=\{g_1,\hdots, g_t\}$ for $I$ is a Groebner basis for $I$ if and only if for all pairs $i\ne j$, the remainder on division of $S(g_i,g_j)$ by $G$ is zero.
        \end{theorem}
\end{document}