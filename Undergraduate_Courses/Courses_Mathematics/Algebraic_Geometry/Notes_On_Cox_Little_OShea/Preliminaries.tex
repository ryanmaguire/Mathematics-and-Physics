\documentclass[crop=false,class=article,oneside]{standalone}
%----------------------------Preamble-------------------------------%
%---------------------------Packages----------------------------%
\usepackage{geometry}
\geometry{b5paper, margin=1.0in}
\usepackage[T1]{fontenc}
\usepackage{graphicx, float}            % Graphics/Images.
\usepackage{natbib}                     % For bibliographies.
\bibliographystyle{agsm}                % Bibliography style.
\usepackage[french, english]{babel}     % Language typesetting.
\usepackage[dvipsnames]{xcolor}         % Color names.
\usepackage{listings, lstlinebgrd}      % Verbatim-Like Tools.
\usepackage{mathtools, esint, mathrsfs} % amsmath and integrals.
\usepackage{amsthm, amsfonts}           % Fonts and theorems.
\usepackage{tabularx}
\usepackage{tcolorbox}                  % Frames around theorems.
\usepackage{upgreek}                    % Non-Italic Greek.
\usepackage{paracol}                    % Two-column styling.
\usepackage{wrapfig}                    % Wrap text around figure.
\usepackage{fmtcount, etoolbox}         % For the \book{} command.
\usepackage[newparttoc]{titlesec}       % Formatting chapter, etc.
\usepackage{titletoc}                   % Allows \book in toc.
\usepackage[nottoc]{tocbibind}          % Bibliography in toc.
\usepackage[titles]{tocloft}            % ToC formatting.
\usepackage{multicol, enumitem}         % Multi-column/enumerate.
\usepackage{import}                     % Import external files.
\usepackage{pgfplots, tikz}             % Drawing/graphing tools.
\usetikzlibrary{
    calc,                   % Calculating right angles and more.
    angles,                 % Drawing angles within triangles.
    arrows.meta,            % Latex and Stealth arrows.
    quotes,                 % Adding labels to angles.
    positioning,            % Relative positioning of nodes.
    decorations.markings,   % Adding arrows in the middle of a line.
    patterns,
    arrows,
    shapes,
    shapes.geometric,
    cd,
    hobby,
    babel
}                                       % Libraries for tikz.
\pgfplotsset{compat=1.9}                % Version of pgfplots.
\usepackage[font=scriptsize,
            labelformat=simple,
            labelsep=colon]{subcaption} % Subfigure captions.
\usepackage[font={scriptsize},
            hypcap=true,
            labelsep=colon]{caption}    % Figure captions.
\usepackage{hyperref}                   % Allows for hyperlinks.
\hypersetup{
    colorlinks=true,
    linkcolor=blue,
    filecolor=magenta,
    urlcolor=Cerulean,
    citecolor=SkyBlue
}                           % Colors for hyperref.
\usepackage[toc,acronym,nogroupskip]{glossaries} % Glossaries and acronyms.
\usepackage[subpreambles=false]{standalone}      % Complileable sub files.

% Various font stuff from kiwi.
% Use this for Times text and Computer Modern math
%\usepackage{times}

% Quite nice
%\usepackage[charter, greekfamily=, greekuppercase=italicized]{mathdesign}
%\usepackage[utopia, greekuppercase=italicized]{mathdesign}    % Math is narrower

% Use this for Times text and math
%\usepackage{newtxtext}
%\usepackage[libertine,cmintegrals]{newtxmath}
%\usepackage{fix-cm}

%\usepackage{txfontsb}
% or
%\usepackage{mathptmx}

%\usepackage[scaled=0.92]{helvet}
%\renewcommand{\rmdefault}{ptm}

%\usepackage{mathpazo}    % add possibly `sc` and `osf` options
%\usepackage{eulervm}

%\usepackage{fourier}
%\renewcommand{\rmdefault}{ptm}
%\usepackage{mathptm}

%\usepackage{fontspec}
%\setmainfont{lmodern}

%\usepackage[varg]{txfonts}
%\usepackage{fouriernc}
%\usepackage{mathpazo}

%\usepackage{bookman}
%\usepackage[scaled]{uarial}
%\usepackage[scaled]{helvet}
%\renewcommand*\familydefault{\sfdefault}
%\usepackage[math]{anttor}

%\newcommand\fgeorgia{\fontfamily{jvn}\selectfont}
%\newcommand\ftimes{\fontfamily{ptm}\selectfont}
%\newcommand\fhelvetica{\fontfamily{phv}\selectfont}
%\newcommand\fcourier{\fontfamily{pcr}\selectfont}
%\newcommand\fbookman{\fontfamily{pbk}\selectfont}
%\newcommand\fnewcentury{\fontfamily{pnc}\selectfont}
%\newcommand\fpalatino{\fontfamily{ppl}\selectfont}
%\newcommand\favantgarde{\fontfamily{pag}\selectfont}
%\newcommand\fnormal{\normalfont}
%\newcommand\fsize[1]{\ifnum#1>0\fontsize{#1}{#1}\selectfont\else\normalsize\fi}
%------------------------Theorem Styles-------------------------%
% Define theorem style for default spacing and normal font.
\newtheoremstyle{normal}
    {\topsep}               % Amount of space above the theorem.
    {\topsep}               % Amount of space below the theorem.
    {}                      % Font used for body of theorem.
    {}                      % Measure of space to indent.
    {\bfseries}             % Font of the header of the theorem.
    {}                      % Punctuation between head and body.
    {.5em}                  % Space after theorem head.
    {}

% Define theorem style for default spacing with italicized font.
\newtheoremstyle{normalit}{\topsep}{\topsep}
                {\itshape}{}{\bfseries}{}{.5em}{}

% Italic header environment.
\newtheoremstyle{thmit}{\topsep}{\topsep}{}{}{\itshape}{}{0.5em}{}

% Define italicized environments.
\theoremstyle{normalit}
\newtheorem{theorem}{Theorem}[section]
\newtheorem{lemma}{Lemma}[section]
\newtheorem{corollary}{Corollary}[section]
\newtheorem{proposition}{Proposition}[section]
\newtheorem*{theorem*}{Theorem}

% Define environments with italic headers.
\theoremstyle{thmit}
\newtheorem*{solution}{Solution}
\newtheorem*{fsolution}{Solution}

% Define default environments.
\theoremstyle{normal}
\newtheorem{example}{Example}[section]
\newtheorem{definition}{Definition}[section]
\newtheorem{problem}{Problem}[section]
\newtheorem{question}{Question}[section]
\newtheorem{remark}{Remark}[section]
\newtheorem{properties}{Properties}[section]
\newtheorem{notation}{Notation}[section]
\newtheorem{axiom}{Axiom}[section]
\newtheorem*{properties*}{Properties}
\newtheorem*{remark*}{Remark}
\newtheorem*{definition*}{Definition}
\theoremstyle{plain}

% Define framed environment.
\tcbuselibrary{most}
\newtcbtheorem[use counter*=theorem]{ftheorem}{Theorem}%
    {colback=green!5,colframe=green!35!black,
     fonttitle=\bfseries\upshape}{th}

\newtcbtheorem[use counter*=example]{fdefinition}{Definition}%
    {fonttitle=\bfseries\upshape,
     colback=blue!5!white,colframe=blue!75!black}{def}

\newtcbtheorem[use counter*=example]{fexample}{Example}%
    {fonttitle=\bfseries\upshape,
     colback=red!5!white,colframe=red!75!black}{ex}

\newtcbtheorem[use counter*=notation]{fnotation}{Notation}%
    {fonttitle=\bfseries\upshape,
     colback=SeaGreen!5!white,colframe=SeaGreen!75!black}{ex}

\newtcbtheorem[use counter*=corollary]{fcorollary}{Corollary}%
    {fonttitle=\bfseries\upshape,
     colback=Orchid!5!white,colframe=Orchid!75!black}{ex}

\newenvironment{bproof}{\textit{Proof.}}{\hfill$\square$}
\tcolorboxenvironment{bproof}{blanker,breakable,left=5mm,
                             before skip=10pt,after skip=10pt,
                             borderline west={1mm}{0pt}{red}}
\tcolorboxenvironment{fsolution}
    {enhanced jigsaw,colframe=cyan,interior hidden,breakable}

%--------------------Declared Math Operators--------------------%
\DeclareMathOperator{\Refl}{Refl}           % Reflection operator.
\DeclareMathOperator{\Span}{Span}           % Span of a set of vectors.
\DeclareMathOperator{\Card}{Card}           % Cardinality of set.
\DeclareMathOperator{\Ord}{Ord}             % Ordinal of ordered set.
\DeclareMathOperator{\Tr}{Tr}               % Trace of matrix.
\DeclareMathOperator{\adjoint}{adj}         % Adjoint of matrix.
\DeclareMathOperator{\rk}{rk}               % Rank of operator.
\DeclareMathOperator{\nul}{nul}             % Null space of operator.
\DeclareMathOperator{\sgn}{sgn}             % Sign of a number.
\DeclareMathOperator{\multideg}{mutlideg}   % Multi-Degree (Graphs).
\DeclareMathOperator{\GCD}{GCD}             % Greatest common denominator.
\DeclareMathOperator{\LM}{LM}               % Leading monomial
\DeclareMathOperator{\LC}{LC}               % Leading coefficient.
\DeclareMathOperator{\LT}{LT}               % Leading term.
\DeclareMathOperator{\LCM}{LCM}             % Least common multiple.
\DeclareMathOperator{\Mon}{Mon}             % Monomial.
\DeclareMathOperator{\Spec}{Spec}           % Spectrum.
\DeclareMathOperator{\proj}{proj}           % Projection.
\DeclareMathOperator{\comp}{comp}           % Component.
\DeclareMathOperator{\sinc}{sinc}           % Sinc function.
\DeclareMathOperator{\Ima}{Im}              % Image of operator.
\DeclareMathOperator{\Prin}{Prin}           % Principal value.
\DeclareMathOperator{\Mod}{mod}             % Modulus.
%------------------------New Commands---------------------------%
\DeclarePairedDelimiter\norm{\lVert}{\rVert}
\DeclarePairedDelimiter\ceil{\lceil}{\rceil}
\DeclarePairedDelimiter\floor{\lfloor}{\rfloor}
\newcommand*\diff{\mathop{}\!\mathrm{d}}
\newcommand*\Diff[1]{\mathop{}\!\mathrm{d^#1}}
\renewcommand{\mod}{\ \Mod}
\renewcommand*{\glstextformat}[1]{\textcolor{RoyalBlue}{#1}}
\renewcommand{\glsnamefont}[1]{\textbf{#1}}
\renewcommand\labelitemii{$\circ$}
\renewcommand\thesubfigure{\arabic{chapter}.\arabic{figure}}
\renewcommand\thesubfigure{%
    \arabic{chapter}.\arabic{figure}.\arabic{subfigure}}
\addto\captionsenglish{\renewcommand{\figurename}{Fig.}}
%------------------------Book Command---------------------------%
\makeatletter
\renewcommand\@pnumwidth{1cm}
\newcounter{book}
\renewcommand\thebook{\@Roman\c@book}
\newcommand\book{%
    \if@openright
        \cleardoublepage
    \else
        \clearpage
    \fi
    \thispagestyle{plain}%
    \if@twocolumn
        \onecolumn
        \@tempswatrue
    \else
        \@tempswafalse
    \fi
    \null\vfil
    \secdef\@book\@sbook
}
\def\@book[#1]#2{%
    \ifnum \c@secnumdepth >-3\relax
        \refstepcounter{book}%
        \addcontentsline{toc}{book}{
            \bookname\ \thebook:\hspace{1em}#1
        }
    \else
        \addcontentsline{toc}{book}{#1}%
    \fi
    \markboth{}{}%
    {\centering
     \interlinepenalty \@M
     \normalfont
     \ifnum \c@secnumdepth >-2\relax
       \huge\bfseries \bookname\nobreakspace\thebook
       \par
       \vskip 20\p@
     \fi
     \Huge \bfseries #2\par}%
    \@endbook}
\def\@sbook#1{%
    {\centering
     \interlinepenalty \@M
     \normalfont
     \Huge \bfseries #1\par}%
    \@endbook}
\def\@endbook{
    \vfil\newpage
        \if@twoside
            \if@openright
                \null
                \thispagestyle{empty}%
                \newpage
            \fi
        \fi
        \if@tempswa
            \twocolumn
        \fi
}
\newcommand*\l@book[2]{%
    \ifnum \c@tocdepth >-2\relax
        \addpenalty{-\@highpenalty}%
        \addvspace{2.25em \@plus\p@}%
        \setlength\@tempdima{3em}%
        \begingroup
            \parindent \z@ \rightskip \@pnumwidth
            \parfillskip -\@pnumwidth
            {
                \leavevmode
                \Large \bfseries #1\hfil \hb@xt@\@pnumwidth{
                    \hss #2
                }
            }
            \par
            \nobreak
            \global\@nobreaktrue
            \everypar{\global\@nobreakfalse\everypar{}}%
        \endgroup
    \fi}
\newcommand\bookname{Book}
\renewcommand{\thebook}{\texorpdfstring{\Numberstring{book}}{book}}
\providecommand*{\toclevel@book}{-2}
\makeatother
\titlecontents{chapter}[0pt]
    {\bfseries}
    {\chaptername\ \thecontentslabel:\quad}
    {}
    {\hfill\contentspage}
\titleformat{\part}[display]
    {\Large\bfseries}
    {\partname\nobreakspace\thepart}
    {0mm}
    {\Huge\bfseries}
    \titlecontents{part}[0pt]
    {\large\bfseries}
    {\partname\ \thecontentslabel: \quad}
    {}
    {\hfill\contentspage}
\newcommand{\MarkRightAngle}[4][.3cm]
    {\coordinate (tempa) at ($(#3)!#1!(#2)$);
     \coordinate (tempb) at ($(#3)!#1!(#4)$);
     \coordinate (tempc) at ($(tempa)!0.5!(tempb)$);%midpoint
     \draw (tempa) -- ($(#3)!2!(tempc)$) -- (tempb);}
%--------------------------LENGTHS------------------------------%
% Spacings for the Table of Contents.
\addtolength{\cftsecnumwidth}{1ex}
\addtolength{\cftsubsecindent}{1ex}
\addtolength{\cftsubsecnumwidth}{1ex}
\addtolength{\cftfignumwidth}{1ex}
\addtolength{\cfttabnumwidth}{1ex}

% Spacing for multi-column and enumerate environments.
\setlength{\multicolsep}{6pt}
\setlist[enumerate]{itemsep=0pt,topsep=3pt}

% Indent and paragraph spacing.
\setlength{\parindent}{0em}
\setlength{\parskip}{0em}
%--------------------------Main Document----------------------------%
\begin{document}
    \ifx\ifmathcoursesalgebraicgeometry\undefined
        \section*{Algebraic Geometry}
        \setcounter{section}{1}
    \fi
    \subsection{Preliminaries}
        \subsubsection{Groups}
            \begin{definition}
                A binary operation on a set $S$ is a
                function $*:S\times S \rightarrow S$.
            \end{definition}
            \begin{definition}
                A group is a set $G$ and a binary operation $*$,
                denoted $\langle G,*\rangle$, such that:
                \begin{enumerate}
                    \item $\forall_{a,b,c\in G}$, $a*(b*c)=(a*b)*c$
                          \hfill[Associativity]
                    \item $\exists_{e\in G}$ such that
                          $\forall_{a\in G}$, $a*e=e*a=a$
                          \hfill[Existence of Neutral Element]
                    \item $\forall_{a\in G}$, $\exists_{b\in G}$
                          such that $a*b=b*a=e$.
                          We write $b=a^{-1}$.
                          \hfill[Existence of Inverse Elements]
                \end{enumerate}
            \end{definition}
            \begin{definition}
                An Abelian group is a group $\langle G,*\rangle$
                such that $\forall_{a\in G},a*b=b*a$
            \end{definition}
            \begin{theorem}
                If $\langle G, *\rangle$ is a group with neutral
                element $e$, then $e$ is unique.
            \end{theorem}
            \begin{theorem}
                If $\langle G,*\rangle$ is a group and $a\in G$,
                then $a^{-1}$ is unique.
            \end{theorem}
            \begin{theorem}
                If $p$ is prime, then
                $\mathbb{Z}_p\setminus \{0\}$ is a group
                under multiplication modulo $p$.
            \end{theorem}
            \begin{remark}
                $\mathbb{Z}$ is NOT a group under
                multiplication. Multiplicative inverses may
                not be integers.
            \end{remark}
            \begin{definition}
                An injective function is a function
                $f:A\rightarrow B$ such that
                $\forall_{a,b\in A}$,
                $f(a)=f(b)\Rightarrow a=b$.
            \end{definition}
            \begin{definition}
                A surjective functions is a function
                $f:A\rightarrow B$ such that
                $\forall_{b\in B}$,
                $\exists_{a\in A}:f(a)=b$.
            \end{definition}
            \begin{definition}
                A bijective function is a function that
                is both injective and surjective.
            \end{definition}
            \begin{definition}
                A permutation on a set $S$ is a
                bijective function $\sigma:S\rightarrow S$.
            \end{definition}
            \begin{definition}
                The composition $f:A\rightarrow B$ and
                $g:B\rightarrow C$ is $g\circ f:A\rightarrow C$
                defined by $x\mapsto g(f(x))$.
            \end{definition}
            \begin{definition}
                A subgroup of a group $\langle G,*\rangle$ is
                a set $H\subset G$ such that $\langle H,*\rangle$
                is a group.
            \end{definition}
        \subsubsection{Fields and Rings}
            \begin{definition}
                A field is a set $k$ with two binary
                operations $+$ and $\cdot$ such that:
                \begin{enumerate}
                    \item $\forall_{a,b,c\in k}$,
                          $a+(b+c)=(a+b)+c$
                          \hfill[Associativity of Addition]
                    \item $\forall_{a,b,c\in k}$,
                          $a\cdot(b\cdot c)=(a\cdot b)\cdot c$
                          \hfill[Associativity of Multiplication]
                    \item $\forall_{a,b\in k}$,
                          $a+b=b+a$
                          \hfill[Commutativity of Addition]
                    \item $\forall_{a,b\in k}$,
                          $a\cdot b=b\cdot a$
                          \hfill[Commutativity of Multiplication]
                    \item $\exists_{0 \in k}$ such that
                          $\forall_{a\in k}$,
                          $a+0=0+a=a$
                          \hfill[Existence of Additive Identity]
                    \item $\exists_{1\in k}$ such that
                          $\forall_{a\in k}$,
                          $1\cdot a=a\cdot 1=a$
                          \hfill[Existence of Multiplicative Identity]
                    \item $\forall_{a\in k}$ there is a
                          $b\in k$ such that $a+b=0$
                          \hfill[Existence of Additive Inverse]
                    \item $\forall_{a\in k}$, $a\ne 0$,
                          there is a $b\in k$ such that
                          $a\cdot b=1$
                          \hfill [Existence of Multiplicative Inverses]
                    \item $\forall_{a,b,c\in k}$,
                          $a\cdot(b+c)=a\cdot b+a\cdot c$
                          \hfill[Multiplication Distributes Over Addition]
                \end{enumerate}
            \end{definition}
            \begin{remark}
                We usually omit the multiplication
                symbol $\cdot$ and just write $ab$ instead
                of $a\cdot b$
            \end{remark}
            \begin{theorem}
                If $k$ is a field, then
                $\langle k,+ \rangle$ is an Abelian group.
            \end{theorem}
            \begin{theorem}
                If $k$ is a field and $a\in k$,
                then $a\cdot 0=0$
            \end{theorem}
            \begin{remark}
                If $k$ is a field and $0=1$, then $k=\{0\}$.
                This makes the "Zero Field," rather boring.
            \end{remark}
            \begin{theorem}
                If $-1$ is the additive inverse of $1$,
                then $(-1)^2=1$
            \end{theorem}
            \begin{definition}
                A ring is a set $R$ with two binary
                operations $+$ and $\cdot$ such that:
                \begin{enumerate}
                    \item $\forall_{a,b,c\in k}$,
                          $a+(b+c)=(a+b)+c$
                          \hfill[Associativity of Addition]
                    \item $\forall_{a,b,c\in k}$,
                          $a\cdot(b\cdot c)=(a\cdot b)\cdot c$
                          \hfill[Associativity of Multiplication]
                    \item $\forall_{a,b\in k}$,
                          $a+b=b+a$
                          \hfill[Commutativity of Addition]
                    \item $\exists_{0 \in k}$ such that
                          $\forall_{a\in k}$,
                          $a+0=0+a=a$
                          \hfill[Existence of Additive Identity]
                    \item $\forall_{a\in k}$,
                          $\exists_{b\in k}$ such that $a+b=0$
                          \hfill[Existence of Additive Inverse]
                    \item $\forall_{a,b,c\in k}$,
                          $a\cdot(b+c)=a\cdot b+a\cdot c$
                          and $(b+c)\cdot a=b\cdot a+c\cdot a$
                          \hfill [Distributive Property]
                \end{enumerate}
            \end{definition}
            \begin{definition}
                A commutative ring is a ring $R$
                such that $\forall_{a,b\in R},ab=ba$
            \end{definition}
            \begin{definition}
                A commutative ring with unity is a
                commutative ring such that
                $\exists_{1\in R}\forall_{a\in R}:1a=a$
            \end{definition}
            \begin{remark}
                In rings and fields, $+$ is usually called
                addition and $\cdot$ is usually called multiplication.
            \end{remark}
            \begin{corollary}
                If $R$ is a ring and $a\in R$,
                then $a\cdot 0 = 0\cdot a=0$
            \end{corollary}
            \begin{definition}
                An integral domain is a commutative
                ring such that $ab=0\Rightarrow a=0$ or $b=0$
            \end{definition}
            \begin{definition}
                A divisor of zero in a ring $R$ is an
                element $a\in R$ such that
                $\exists_{b\in R\setminus\{0\}}:ab=0$
            \end{definition}
            \begin{theorem}
                $a$ divisor of zero in a ring $R$
                if and only if $f:R\rightarrow R$,
                $f(x)=ax$ is not injective.
            \end{theorem}
            \begin{theorem}
                Any field $k$ is an integral domain.
            \end{theorem}
            \begin{definition}
                An ideal of a commutative ring is
                a set $I\subset R$ such that:
                \begin{enumerate}
                    \item $0\in I$
                          \hfill[Existence of Additive Inverse]
                    \item $\forall_{a,b\in I}$,
                          $a+b\in I$
                          \hfill[Closure Under Addition]
                    \item $\forall_{a\in I,b\in R}$,
                          $a b \in I$
                          \hfill[Absorption Property]
                \end{enumerate}
            \end{definition}
        \subsubsection{Determinants}
            The elementary definitions from linear algebra
            are presumed. The set of all permutations of
            $\mathbb{Z}_{n}$ is denoted $S_n$. $S_{n}$ is a
            group under composition,
            $\langle S_{n},\circ\rangle$
            \begin{definition}
                The permutation matrix of $\sigma \in S_{n}$,
                denoted $P_{\sigma}$, is the matrix formed
                by the image of the identity matrix $I_{n}$
                under the mapping
                $(a_{ij})\mapsto (a_{i\sigma(j)})$
            \end{definition}
            \begin{example}
                Consider the permutation on $\mathbb{Z}_3$
                defined by the cycle
                $1\rightarrow 3\rightarrow 2\rightarrow 1$.
                We can make this a matrix equation as follows:
                \begin{equation*}
                    \begin{bmatrix}
                        0&0&1\\
                        1&0&0\\
                        0&1&0
                    \end{bmatrix}
                    \begin{bmatrix}
                        1\\
                        2\\
                        3
                    \end{bmatrix}
                    =
                    \begin{bmatrix}
                        3\\
                        1\\
                        2
                    \end{bmatrix}    
                \end{equation*}
                The leftmost matrix is obtained by
                permuting the columns of the identity
                matrix $I_{3}$ by $\sigma$.
            \end{example}
            \begin{definition}
                The sign of a permutation $\sigma\in S_{n}$
                is $\sgn(\sigma) = \det(P_{\sigma})$.
            \end{definition}
            \begin{remark}
                From the way $P_{\sigma}$ is defined,
                $\sgn(\sigma)=\det(P_{\sigma})=\pm 1$,
                depending on $\sigma$.
            \end{remark}
            \begin{theorem}
                If $A=(a_{ij})$ is an $n\times n$ matrix, then
                $\det(A)=\underset{\sigma\in S_n}%
                 \sum\sgn(\sigma)\prod_{k=1}^{n}a_{k\sigma(k)}$.
            \end{theorem}
\end{document}