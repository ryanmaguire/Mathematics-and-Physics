\documentclass[crop=false,class=article,oneside]{standalone}
%----------------------------Preamble-------------------------------%
%---------------------------Packages----------------------------%
\usepackage{geometry}
\geometry{b5paper, margin=1.0in}
\usepackage[T1]{fontenc}
\usepackage{graphicx, float}            % Graphics/Images.
\usepackage{natbib}                     % For bibliographies.
\bibliographystyle{agsm}                % Bibliography style.
\usepackage[french, english]{babel}     % Language typesetting.
\usepackage[dvipsnames]{xcolor}         % Color names.
\usepackage{listings}                   % Verbatim-Like Tools.
\usepackage{mathtools, esint, mathrsfs} % amsmath and integrals.
\usepackage{amsthm, amsfonts, amssymb}  % Fonts and theorems.
\usepackage{tcolorbox}                  % Frames around theorems.
\usepackage{upgreek}                    % Non-Italic Greek.
\usepackage{fmtcount, etoolbox}         % For the \book{} command.
\usepackage[newparttoc]{titlesec}       % Formatting chapter, etc.
\usepackage{titletoc}                   % Allows \book in toc.
\usepackage[nottoc]{tocbibind}          % Bibliography in toc.
\usepackage[titles]{tocloft}            % ToC formatting.
\usepackage{pgfplots, tikz}             % Drawing/graphing tools.
\usepackage{imakeidx}                   % Used for index.
\usetikzlibrary{
    calc,                   % Calculating right angles and more.
    angles,                 % Drawing angles within triangles.
    arrows.meta,            % Latex and Stealth arrows.
    quotes,                 % Adding labels to angles.
    positioning,            % Relative positioning of nodes.
    decorations.markings,   % Adding arrows in the middle of a line.
    patterns,
    arrows
}                                       % Libraries for tikz.
\pgfplotsset{compat=1.9}                % Version of pgfplots.
\usepackage[font=scriptsize,
            labelformat=simple,
            labelsep=colon]{subcaption} % Subfigure captions.
\usepackage[font={scriptsize},
            hypcap=true,
            labelsep=colon]{caption}    % Figure captions.
\usepackage[pdftex,
            pdfauthor={Ryan Maguire},
            pdftitle={Mathematics and Physics},
            pdfsubject={Mathematics, Physics, Science},
            pdfkeywords={Mathematics, Physics, Computer Science, Biology},
            pdfproducer={LaTeX},
            pdfcreator={pdflatex}]{hyperref}
\hypersetup{
    colorlinks=true,
    linkcolor=blue,
    filecolor=magenta,
    urlcolor=Cerulean,
    citecolor=SkyBlue
}                           % Colors for hyperref.
\usepackage[toc,acronym,nogroupskip,nopostdot]{glossaries}
\usepackage{glossary-mcols}
%------------------------Theorem Styles-------------------------%
\theoremstyle{plain}
\newtheorem{theorem}{Theorem}[section]

% Define theorem style for default spacing and normal font.
\newtheoremstyle{normal}
    {\topsep}               % Amount of space above the theorem.
    {\topsep}               % Amount of space below the theorem.
    {}                      % Font used for body of theorem.
    {}                      % Measure of space to indent.
    {\bfseries}             % Font of the header of the theorem.
    {}                      % Punctuation between head and body.
    {.5em}                  % Space after theorem head.
    {}

% Italic header environment.
\newtheoremstyle{thmit}{\topsep}{\topsep}{}{}{\itshape}{}{0.5em}{}

% Define environments with italic headers.
\theoremstyle{thmit}
\newtheorem*{solution}{Solution}

% Define default environments.
\theoremstyle{normal}
\newtheorem{example}{Example}[section]
\newtheorem{definition}{Definition}[section]
\newtheorem{problem}{Problem}[section]

% Define framed environment.
\tcbuselibrary{most}
\newtcbtheorem[use counter*=theorem]{ftheorem}{Theorem}{%
    before=\par\vspace{2ex},
    boxsep=0.5\topsep,
    after=\par\vspace{2ex},
    colback=green!5,
    colframe=green!35!black,
    fonttitle=\bfseries\upshape%
}{thm}

\newtcbtheorem[auto counter, number within=section]{faxiom}{Axiom}{%
    before=\par\vspace{2ex},
    boxsep=0.5\topsep,
    after=\par\vspace{2ex},
    colback=Apricot!5,
    colframe=Apricot!35!black,
    fonttitle=\bfseries\upshape%
}{ax}

\newtcbtheorem[use counter*=definition]{fdefinition}{Definition}{%
    before=\par\vspace{2ex},
    boxsep=0.5\topsep,
    after=\par\vspace{2ex},
    colback=blue!5!white,
    colframe=blue!75!black,
    fonttitle=\bfseries\upshape%
}{def}

\newtcbtheorem[use counter*=example]{fexample}{Example}{%
    before=\par\vspace{2ex},
    boxsep=0.5\topsep,
    after=\par\vspace{2ex},
    colback=red!5!white,
    colframe=red!75!black,
    fonttitle=\bfseries\upshape%
}{ex}

\newtcbtheorem[auto counter, number within=section]{fnotation}{Notation}{%
    before=\par\vspace{2ex},
    boxsep=0.5\topsep,
    after=\par\vspace{2ex},
    colback=SeaGreen!5!white,
    colframe=SeaGreen!75!black,
    fonttitle=\bfseries\upshape%
}{not}

\newtcbtheorem[use counter*=remark]{fremark}{Remark}{%
    fonttitle=\bfseries\upshape,
    colback=Goldenrod!5!white,
    colframe=Goldenrod!75!black}{ex}

\newenvironment{bproof}{\textit{Proof.}}{\hfill$\square$}
\tcolorboxenvironment{bproof}{%
    blanker,
    breakable,
    left=3mm,
    before skip=5pt,
    after skip=10pt,
    borderline west={0.6mm}{0pt}{green!80!black}
}

\AtEndEnvironment{lexample}{$\hfill\textcolor{red}{\blacksquare}$}
\newtcbtheorem[use counter*=example]{lexample}{Example}{%
    empty,
    title={Example~\theexample},
    boxed title style={%
        empty,
        size=minimal,
        toprule=2pt,
        top=0.5\topsep,
    },
    coltitle=red,
    fonttitle=\bfseries,
    parbox=false,
    boxsep=0pt,
    before=\par\vspace{2ex},
    left=0pt,
    right=0pt,
    top=3ex,
    bottom=1ex,
    before=\par\vspace{2ex},
    after=\par\vspace{2ex},
    breakable,
    pad at break*=0mm,
    vfill before first,
    overlay unbroken={%
        \draw[red, line width=2pt]
            ([yshift=-1.2ex]title.south-|frame.west) to
            ([yshift=-1.2ex]title.south-|frame.east);
        },
    overlay first={%
        \draw[red, line width=2pt]
            ([yshift=-1.2ex]title.south-|frame.west) to
            ([yshift=-1.2ex]title.south-|frame.east);
    },
}{ex}

\AtEndEnvironment{ldefinition}{$\hfill\textcolor{Blue}{\blacksquare}$}
\newtcbtheorem[use counter*=definition]{ldefinition}{Definition}{%
    empty,
    title={Definition~\thedefinition:~{#1}},
    boxed title style={%
        empty,
        size=minimal,
        toprule=2pt,
        top=0.5\topsep,
    },
    coltitle=Blue,
    fonttitle=\bfseries,
    parbox=false,
    boxsep=0pt,
    before=\par\vspace{2ex},
    left=0pt,
    right=0pt,
    top=3ex,
    bottom=0pt,
    before=\par\vspace{2ex},
    after=\par\vspace{1ex},
    breakable,
    pad at break*=0mm,
    vfill before first,
    overlay unbroken={%
        \draw[Blue, line width=2pt]
            ([yshift=-1.2ex]title.south-|frame.west) to
            ([yshift=-1.2ex]title.south-|frame.east);
        },
    overlay first={%
        \draw[Blue, line width=2pt]
            ([yshift=-1.2ex]title.south-|frame.west) to
            ([yshift=-1.2ex]title.south-|frame.east);
    },
}{def}

\AtEndEnvironment{ltheorem}{$\hfill\textcolor{Green}{\blacksquare}$}
\newtcbtheorem[use counter*=theorem]{ltheorem}{Theorem}{%
    empty,
    title={Theorem~\thetheorem:~{#1}},
    boxed title style={%
        empty,
        size=minimal,
        toprule=2pt,
        top=0.5\topsep,
    },
    coltitle=Green,
    fonttitle=\bfseries,
    parbox=false,
    boxsep=0pt,
    before=\par\vspace{2ex},
    left=0pt,
    right=0pt,
    top=3ex,
    bottom=-1.5ex,
    breakable,
    pad at break*=0mm,
    vfill before first,
    overlay unbroken={%
        \draw[Green, line width=2pt]
            ([yshift=-1.2ex]title.south-|frame.west) to
            ([yshift=-1.2ex]title.south-|frame.east);},
    overlay first={%
        \draw[Green, line width=2pt]
            ([yshift=-1.2ex]title.south-|frame.west) to
            ([yshift=-1.2ex]title.south-|frame.east);
    }
}{thm}

%--------------------Declared Math Operators--------------------%
\DeclareMathOperator{\adjoint}{adj}         % Adjoint.
\DeclareMathOperator{\Card}{Card}           % Cardinality.
\DeclareMathOperator{\curl}{curl}           % Curl.
\DeclareMathOperator{\diam}{diam}           % Diameter.
\DeclareMathOperator{\dist}{dist}           % Distance.
\DeclareMathOperator{\Div}{div}             % Divergence.
\DeclareMathOperator{\Erf}{Erf}             % Error Function.
\DeclareMathOperator{\Erfc}{Erfc}           % Complementary Error Function.
\DeclareMathOperator{\Ext}{Ext}             % Exterior.
\DeclareMathOperator{\GCD}{GCD}             % Greatest common denominator.
\DeclareMathOperator{\grad}{grad}           % Gradient
\DeclareMathOperator{\Ima}{Im}              % Image.
\DeclareMathOperator{\Int}{Int}             % Interior.
\DeclareMathOperator{\LC}{LC}               % Leading coefficient.
\DeclareMathOperator{\LCM}{LCM}             % Least common multiple.
\DeclareMathOperator{\LM}{LM}               % Leading monomial.
\DeclareMathOperator{\LT}{LT}               % Leading term.
\DeclareMathOperator{\Mod}{mod}             % Modulus.
\DeclareMathOperator{\Mon}{Mon}             % Monomial.
\DeclareMathOperator{\multideg}{mutlideg}   % Multi-Degree (Graphs).
\DeclareMathOperator{\nul}{nul}             % Null space of operator.
\DeclareMathOperator{\Ord}{Ord}             % Ordinal of ordered set.
\DeclareMathOperator{\Prin}{Prin}           % Principal value.
\DeclareMathOperator{\proj}{proj}           % Projection.
\DeclareMathOperator{\Refl}{Refl}           % Reflection operator.
\DeclareMathOperator{\rk}{rk}               % Rank of operator.
\DeclareMathOperator{\sgn}{sgn}             % Sign of a number.
\DeclareMathOperator{\sinc}{sinc}           % Sinc function.
\DeclareMathOperator{\Span}{Span}           % Span of a set.
\DeclareMathOperator{\Spec}{Spec}           % Spectrum.
\DeclareMathOperator{\supp}{supp}           % Support
\DeclareMathOperator{\Tr}{Tr}               % Trace of matrix.
%--------------------Declared Math Symbols--------------------%
\DeclareMathSymbol{\minus}{\mathbin}{AMSa}{"39} % Unary minus sign.
%------------------------New Commands---------------------------%
\DeclarePairedDelimiter\norm{\lVert}{\rVert}
\DeclarePairedDelimiter\ceil{\lceil}{\rceil}
\DeclarePairedDelimiter\floor{\lfloor}{\rfloor}
\newcommand*\diff{\mathop{}\!\mathrm{d}}
\newcommand*\Diff[1]{\mathop{}\!\mathrm{d^#1}}
\renewcommand*{\glstextformat}[1]{\textcolor{RoyalBlue}{#1}}
\renewcommand{\glsnamefont}[1]{\textbf{#1}}
\renewcommand\labelitemii{$\circ$}
\renewcommand\thesubfigure{%
    \arabic{chapter}.\arabic{figure}.\arabic{subfigure}}
\addto\captionsenglish{\renewcommand{\figurename}{Fig.}}
\numberwithin{equation}{section}

\renewcommand{\vector}[1]{\boldsymbol{\mathrm{#1}}}

\newcommand{\uvector}[1]{\boldsymbol{\hat{\mathrm{#1}}}}
\newcommand{\topspace}[2][]{(#2,\tau_{#1})}
\newcommand{\measurespace}[2][]{(#2,\varSigma_{#1},\mu_{#1})}
\newcommand{\measurablespace}[2][]{(#2,\varSigma_{#1})}
\newcommand{\manifold}[2][]{(#2,\tau_{#1},\mathcal{A}_{#1})}
\newcommand{\tanspace}[2]{T_{#1}{#2}}
\newcommand{\cotanspace}[2]{T_{#1}^{*}{#2}}
\newcommand{\Ckspace}[3][\mathbb{R}]{C^{#2}(#3,#1)}
\newcommand{\funcspace}[2][\mathbb{R}]{\mathcal{F}(#2,#1)}
\newcommand{\smoothvecf}[1]{\mathfrak{X}(#1)}
\newcommand{\smoothonef}[1]{\mathfrak{X}^{*}(#1)}
\newcommand{\bracket}[2]{[#1,#2]}

%------------------------Book Command---------------------------%
\makeatletter
\renewcommand\@pnumwidth{1cm}
\newcounter{book}
\renewcommand\thebook{\@Roman\c@book}
\newcommand\book{%
    \if@openright
        \cleardoublepage
    \else
        \clearpage
    \fi
    \thispagestyle{plain}%
    \if@twocolumn
        \onecolumn
        \@tempswatrue
    \else
        \@tempswafalse
    \fi
    \null\vfil
    \secdef\@book\@sbook
}
\def\@book[#1]#2{%
    \refstepcounter{book}
    \addcontentsline{toc}{book}{\bookname\ \thebook:\hspace{1em}#1}
    \markboth{}{}
    {\centering
     \interlinepenalty\@M
     \normalfont
     \huge\bfseries\bookname\nobreakspace\thebook
     \par
     \vskip 20\p@
     \Huge\bfseries#2\par}%
    \@endbook}
\def\@sbook#1{%
    {\centering
     \interlinepenalty \@M
     \normalfont
     \Huge\bfseries#1\par}%
    \@endbook}
\def\@endbook{
    \vfil\newpage
        \if@twoside
            \if@openright
                \null
                \thispagestyle{empty}%
                \newpage
            \fi
        \fi
        \if@tempswa
            \twocolumn
        \fi
}
\newcommand*\l@book[2]{%
    \ifnum\c@tocdepth >-3\relax
        \addpenalty{-\@highpenalty}%
        \addvspace{2.25em\@plus\p@}%
        \setlength\@tempdima{3em}%
        \begingroup
            \parindent\z@\rightskip\@pnumwidth
            \parfillskip -\@pnumwidth
            {
                \leavevmode
                \Large\bfseries#1\hfill\hb@xt@\@pnumwidth{\hss#2}
            }
            \par
            \nobreak
            \global\@nobreaktrue
            \everypar{\global\@nobreakfalse\everypar{}}%
        \endgroup
    \fi}
\newcommand\bookname{Book}
\renewcommand{\thebook}{\texorpdfstring{\Numberstring{book}}{book}}
\providecommand*{\toclevel@book}{-2}
\makeatother
\titleformat{\part}[display]
    {\Large\bfseries}
    {\partname\nobreakspace\thepart}
    {0mm}
    {\Huge\bfseries}
\titlecontents{part}[0pt]
    {\large\bfseries}
    {\partname\ \thecontentslabel: \quad}
    {}
    {\hfill\contentspage}
\titlecontents{chapter}[0pt]
    {\bfseries}
    {\chaptername\ \thecontentslabel:\quad}
    {}
    {\hfill\contentspage}
\newglossarystyle{longpara}{%
    \setglossarystyle{long}%
    \renewenvironment{theglossary}{%
        \begin{longtable}[l]{{p{0.25\hsize}p{0.65\hsize}}}
    }{\end{longtable}}%
    \renewcommand{\glossentry}[2]{%
        \glstarget{##1}{\glossentryname{##1}}%
        &\glossentrydesc{##1}{~##2.}
        \tabularnewline%
        \tabularnewline
    }%
}
\newglossary[not-glg]{notation}{not-gls}{not-glo}{Notation}
\newcommand*{\newnotation}[4][]{%
    \newglossaryentry{#2}{type=notation, name={\textbf{#3}, },
                          text={#4}, description={#4},#1}%
}
%--------------------------LENGTHS------------------------------%
% Spacings for the Table of Contents.
\addtolength{\cftsecnumwidth}{1ex}
\addtolength{\cftsubsecindent}{1ex}
\addtolength{\cftsubsecnumwidth}{1ex}
\addtolength{\cftfignumwidth}{1ex}
\addtolength{\cfttabnumwidth}{1ex}

% Indent and paragraph spacing.
\setlength{\parindent}{0em}
\setlength{\parskip}{0em}
%--------------------------Main Document----------------------------%
\begin{document}
    \ifx\ifmathcoursesalgebraicgeometry\undefined
        \section*{Algebraic Geometry}
        \setcounter{section}{1}
    \fi
    \subsection{Elimination Theory}
    \subsubsection{The Elimination and Extension Theorems}
    \begin{definition}
    If $I = \langle f_1,\hdots, f_s\rangle \subset k[x_1,\hdots ,x_n]$, the $\ell-$th elimination ideal, denoted $I_{\ell}$, is the ideal defined as $I_{\ell} = I \cap k[x_{\ell+1},\hdots, x_n]$.
    \end{definition}
    \begin{theorem}
    For $\ell \in \mathbb{Z}_{n-1}$, if $I = \langle f_1,\hdots, f_s\rangle\subset k[x_1,\hdots ,x_n]$ is an ideal, then $I_{\ell}$ is an ideal of $k[x_1,\hdots ,x_n]$.
    \end{theorem}
    \begin{theorem}[The Elimination Theorem]
    If $I\subset k[x_1,\hdots ,x_n]$ is an ideal and $G$ is a Groebner Basis of $I$ with respect to the lexicographic ordering $x_1>x_2>\hdots > x_n$, then for all $\ell \in \mathbb{Z}_{n}$, $G_{\ell} = G\cap k[x_{\ell+1},\hdots, x_n]$ is a Groebner Basis of $I_{\ell}$.
    \end{theorem}
    \begin{theorem}[The Extension Theorem]
    If $I = \langle f_1,\hdots, g_s\rangle \subset \mathbb{C}[x_1,\hdots ,x_n]$, and if $I_1$ is the first elimination ideal of $I$, and if for all $i\in \mathbb{Z}_s$ $f_i = g(x_2,\hdots, x_n)x_1^{N_i}+h$, where the degree of the $x_1$ component of $h$ is less than $N_i$, and if $(a_2,\hdots, a_n)\notin \textbf{V}(g_1,\hdots, g_s)$, then there is an $a_1 \in \mathbb{C}$ such that $(a_1,\hdots, a_n)\in \textbf{V}(I)$
    \end{theorem}
    \begin{remark}
    The requirement that we work in $\mathbb{C}$ is crucial. This theorem does not hold in $\mathbb{R}$. 
    \end{remark}
    \begin{theorem}
    If $I = \langle f_1,\hdots, f_s\rangle \subset \mathbb{C}[x_1,\hdots, x_n]$ if for some $i$, $f_i$ is of the form $f_i = cx_1^N + g(x_1,\hdots, x_n)$, where the degree of the $x_1$ term in $g$ is less than $N$, and $c \ne 0$, and if $(a_2,\hdots, a_n) \in \textbf{V}(I_{1})$, then there is an $a_1 \in \mathbb{C}$ such that $(a_1,\hdots, a_n) \in \textbf{V}(I)$.
    \end{theorem}
    \subsubsection{The Geometry of Elimination}
    \begin{definition}
    The projectiom map $\pi_{\ell}: \mathbb{C}^n \rightarrow \mathbb{C}^{n-\ell}$ is defined as $\pi_{\ell}(a_1,\hdots, a_n) = (a_{\ell+1},\hdots, a_n)$.
    \end{definition}
    \begin{theorem}
    If $V=\mathbf{V}(f_1,\hdots, f_s) \subset \mathbb{C}^n$, and $I_{\ell}$ is the $\ell-$th elimination ideal of $\langle f_1,\hdots, f_s\rangle$, then $\pi_{\ell}(V) \subset \textbf{V}(I_{\ell})$
    \end{theorem}
    \begin{theorem}
    If $V = \mathbf{V}(f_1,\hdots, f_s) \subset \mathbb{C}^n$, and $G_{\ell}$ is as defined in the extension theorem, then $\textbf{V}(I_{\ell}) = \pi_{\ell}(V)\cup G_{\ell}$
    \end{theorem}
    \begin{theorem}[The First Closure Theorem]
    If $V = \mathbf{V}(f_1,\hdots, f_s) \subset \mathbb{C}^n$ and $I_{\ell}$ is the $\ell-$th elimination ideal of $\langle f_1,\hdots, f_s\rangle$, then $\textbf{V}(I_{\ell})$ is the smallest affine variety containing $\pi_{\ell}(V)\subset \mathbb{C}^{n-\ell}$.
    \end{theorem}
    \begin{theorem}[The Second Closure Theorem]
    If $V = \mathbf{V}(f_1,\hdots, f_s) \subset \mathbb{C}^n$, $V\ne \emptyset$, and if $I_{\ell}$ is the $\ell-$th elimination ideal of $\langle f_1,\hdots, f_s\rangle$, then there is an affine variety $W\underset{Proper}{\subset} \textbf{V}(I_{\ell})$ such that $\textbf{V}(I_{\ell})\setminus W \subset \pi_{\ell}(V)$.
    \end{theorem}
    \begin{theorem}
    If $V = \mathbf{V}(f_1,\hdots, f_s)\subset \mathbb{C}^n$ and if for some $i$, $f_i$ is of the form $f_i = cx_1^N + g$, where the $x_1$ terms in $g$ are of degree less than $N$, and $c\ne 0$, then $\pi_{1}(V) = \textbf{V}(I_{1})$.
    \end{theorem}
    \subsubsection{Implicitization}
    \begin{definition}
    A polynomial parametrization is a finite set of equations $x_k = f_k(t_1,\hdots, t_m)\in k[t_1,\hdots, t_m]$. The function $F:k^m\rightarrow k^n$ is the image defined by $(t_1,\hdots, t_m)\mapsto (x_1,\hdots, x_n)$
    \end{definition}
    \begin{theorem}[The Polynomial Implicitization Theorem]
    If $k$ is an infinite field and $F:k^m\rightarrow k^n$ is a function determined by some polynomial parametrization, and if $I$ is an ideal $I = \langle x_1-f_1,\hdots, x_n - f_n\rangle \subset k[t_1,\hdots, t_m, x_1,\hdots, x_n]$, then $\textbf{V}(I_m)$ is the smallest variety in $k^n$ containing $F(k^n)$, where $I_m$ is the $m^{th}$ elimination ideal.
    \end{theorem}
    \begin{definition}
    A rational parametrization is a finite set of equations $x_k = f_k(t_1,\hdots, t_m)\in k(t_1,\hdots, t_m)$
    \end{definition}
    \begin{theorem}[Rational Implicitization]
    If $k$ is an infinite field, $f_k, g_k, k=1,2,\hdots, n$ are a rational parametrization, $W = \mathbf{V}(g_1,\hdots, g_s)$, and if $F:k^m\setminus W \rightarrow k^n$ is the function determined by the rational parametrization, if $J = \langle g_1 x_1 - g_1,\hdots, g_n x_n - g_n, 1-gy\rangle \subset k[y,t_1,\hdots, g_m, x_1,\hdots, x_n]$, where $g = g_1\cdots g_n$, and if $J_{m+1}$ is the $(m+1)^{th}$ elimination ideal, then $\textbf{V}(J_{m+1})$ is the smallest variety in $k^n$ containing $F(K^m\setminus W)$.
    \end{theorem}
    \subsubsection{Singular Points and Envelopes}
    \begin{definition}
    A singular point on an affine variety $\mathbf{V}(f)$ is a point $x\in k$ such that there exists no tangent line at $x$.
    \end{definition}
    \begin{remark}
    For curves in the plane, this usually happens when either the curve intersects itself or has a kink in it.
    \end{remark}
    \begin{definition}
    If $k\in \mathbb{N}$, if $(a,b)\in \mathbf{V}(f)$, and if $L$ is a line through $(a,b)$, then $L$ meets $\mathbf{V}(f)$ with multiplicity $k$ at $(a,b)$ if $L$ can be linearly parametrized in $x$ and $y$ so that $t=0$ is a root of multiplicity $k$ of the polynomial $g(t) = f(a+ct,b+dt)$.
    \end{definition}
    \begin{theorem}
    If $f\in k[x,y]$, $(a,b) \in \mathbf{V}(f)$, and if $\nabla f(a,b) \ne (0,0)$, then there is a unique line through $(a,b)$ which meets $\mathbf{V}(f)$ with multiplicity $k\geq 2$.
    \end{theorem}
    \begin{theorem}
    If $f\in k[x,y]$, $(a,b) \in \mathbf{V}(f)$, and if $\nabla f(a,b) = 0$, then every line through $(a,b)$ meets $\mathbf{V}(f)$ with multiplicity $k \geq 2$.
    \end{theorem}
    \begin{definition}
    If $f\in k[x,y]$, $(a,b) \in \mathbf{V}(f)$, and if $\nabla f(a,b) \ne (0,0)$, then the tangent line of $\mathbf{V}(f)$ at $(a,b)$ is the unique line through $(a,b)$ with multiplicity $k\geq 2$. We say that $(a,b)$ is a non-singular point of $\mathbf{V}(f)$.
    \end{definition}
    \begin{definition}
    If $f\in k[x,y]$, $(a,b) \in \mathbf{V}(f)$, and if $\nabla f(a,b) = (0,0)$, then we say that $(a,b)$ is a singular point of $\mathbf{V}(f)$.
    \end{definition}
    \begin{definition}
    If $\mathbf{V}(F_t)$ is a family of curves in $\mathbb{R}^2$, its envelope consists of all points $(x,y) \in \mathbb{R}^2$ such that $F(x,y,t) = 0$ and $\frac{\partial}{\partial t}F(x,y,t) = 0$ for some $t\in \mathbb{R}$.
    \end{definition}
    \subsubsection{Unique Factorization and Resultants}
    \begin{definition}
    If $k$ is a field, then a polynomial $f\in k[x_1,\hdots ,x_n]$ is said to be irreducible if $f$ is non-constant and is not the product of two non-constant polynomials in $k[x_1,\hdots ,x_n]$.
    \end{definition}
    \begin{theorem}
    Every non-constant polynomial $f\in k[x_1,\hdots ,x_n]$ can be written as a product of polynomials which are irreducible over $k$
    \end{theorem}
    \begin{theorem}
    If $f,g\in k[x_1,\hdots ,x_n]$ have positive degree in $x_1$, then $f$ and $g$ have a common factor in $k[x_1,\hdots ,x_n]$ of positive degree in $x_1$ if and only if they have a common factor in $k(x_2,\hdots, x_n)[x_1]$
    \end{theorem}
    \begin{theorem}
    Every non-constant $f\in k[x_1,\hdots ,x_n]$ can be written as a product $f = f_1\cdots f_r$ of irreducibles of $k$. Furthermore, if $f = g_1\cdots g_s$, where the $g_k$ are irreducible, then $r=s$ and there are constants $\alpha_1,\hdots, \alpha_n$ such that $\{f_1,\hdots, f_r\} = \{\alpha_1 g_1, \hdots, \alpha_r g_r\}$.
    \end{theorem}
    \begin{theorem}
    If $f,g \in k[x]$ are polynomials of degree $\ell>0$ and $m>0$, respectively, then $f$ and $g$ have a common factor if and only if there are polynomials $A,B\in k[x]$ such that $A$ and $B$ are not both zero, $A$ has degree at most $m-1$ and $B$ has degree at most $\ell-1$, and $Af+Bg = 0$.
    \end{theorem}
    \begin{definition}
    If $f = a_0 x^{\ell} +\hdots + a_{\ell}$ and $g = b_0 x^m + \hdots b_m$, then the Sylvester Matrix is:
    \begin{equation*}
        \begin{pmatrix} a_0 & 0 & 0 & 0 & b_0 & 0 & 0 & 0 \\ a_1 & a_0 & 0 & 0 & b_1 & b_0 & 0 & 0 \\ \vdots & \vdots & \ddots & 0 & \vdots & \vdots & \ddots & 0 \\ \vdots & \vdots & \ddots & a_{0} & \vdots & \vdots & \ddots & b_0 \\ a_{\ell} & \hdots & \hdots & a_{1} & b_{m} & \hdots & \hdots & 0 \\ 0 & a_{\ell} & \hdots & \vdots & 0 & b_{m} & \hdots & \vdots\\ 0 & 0 & \ddots & 0 & 0 & \hdots & \ddots & 0 \\ 0 & \hdots & \hdots & a_{\ell} & 0 & \hdots & \hdots & b_{m} \end{pmatrix}
    \end{equation*}
    \end{definition}
    \begin{theorem}
    If $f,g \in k[x]$, then the resultant of $f$ and $g$ is the determinant of the Sylvester matrix of $f$ and $g$.
    \end{theorem}
    \begin{theorem}
    If $f,g\in k[x]$ are polynomials of positive degree, then the resultant of $f$ and $g$ is an integer polynomial in the coefficients of $f$ and $g$.
    \end{theorem}
    \begin{theorem}
    If $f,g\in k[x]$ are polynomials of positive degree, then $f$ and $g$ have a common factor if and only if their resultant is zero.
    \end{theorem}
    \begin{theorem}
    If $f,g\in k[x]$ are of positive degree, then there are polynomials $A,B \in k[x]$ such that $Af + Bg = Resultant(f,g)$
    \end{theorem}
\end{document}