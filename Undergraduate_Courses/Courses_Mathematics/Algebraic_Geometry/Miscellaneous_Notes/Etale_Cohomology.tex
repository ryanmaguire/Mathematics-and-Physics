\documentclass[crop=false,class=article,oneside]{standalone}
%----------------------------Preamble-------------------------------%
%---------------------------Packages----------------------------%
\usepackage{geometry}
\geometry{b5paper, margin=1.0in}
\usepackage[T1]{fontenc}
\usepackage{graphicx, float}            % Graphics/Images.
\usepackage{natbib}                     % For bibliographies.
\bibliographystyle{agsm}                % Bibliography style.
\usepackage[french, english]{babel}     % Language typesetting.
\usepackage[dvipsnames]{xcolor}         % Color names.
\usepackage{listings, lstlinebgrd}      % Verbatim-Like Tools.
\usepackage{mathtools, esint, mathrsfs} % amsmath and integrals.
\usepackage{amsthm, amsfonts}           % Fonts and theorems.
\usepackage{tabularx}
\usepackage{tcolorbox}                  % Frames around theorems.
\usepackage{upgreek}                    % Non-Italic Greek.
\usepackage{paracol}                    % Two-column styling.
\usepackage{wrapfig}                    % Wrap text around figure.
\usepackage{fmtcount, etoolbox}         % For the \book{} command.
\usepackage[newparttoc]{titlesec}       % Formatting chapter, etc.
\usepackage{titletoc}                   % Allows \book in toc.
\usepackage[nottoc]{tocbibind}          % Bibliography in toc.
\usepackage[titles]{tocloft}            % ToC formatting.
\usepackage{multicol, enumitem}         % Multi-column/enumerate.
\usepackage{import}                     % Import external files.
\usepackage{pgfplots, tikz}             % Drawing/graphing tools.
\usetikzlibrary{
    calc,                   % Calculating right angles and more.
    angles,                 % Drawing angles within triangles.
    arrows.meta,            % Latex and Stealth arrows.
    quotes,                 % Adding labels to angles.
    positioning,            % Relative positioning of nodes.
    decorations.markings,   % Adding arrows in the middle of a line.
    patterns,
    arrows,
    shapes,
    shapes.geometric,
    cd,
    hobby,
    babel
}                                       % Libraries for tikz.
\pgfplotsset{compat=1.9}                % Version of pgfplots.
\usepackage[font=scriptsize,
            labelformat=simple,
            labelsep=colon]{subcaption} % Subfigure captions.
\usepackage[font={scriptsize},
            hypcap=true,
            labelsep=colon]{caption}    % Figure captions.
\usepackage{hyperref}                   % Allows for hyperlinks.
\hypersetup{
    colorlinks=true,
    linkcolor=blue,
    filecolor=magenta,
    urlcolor=Cerulean,
    citecolor=SkyBlue
}                           % Colors for hyperref.
\usepackage[toc,acronym,nogroupskip]{glossaries} % Glossaries and acronyms.
\usepackage[subpreambles=false]{standalone}      % Complileable sub files.

% Various font stuff from kiwi.
% Use this for Times text and Computer Modern math
%\usepackage{times}

% Quite nice
%\usepackage[charter, greekfamily=, greekuppercase=italicized]{mathdesign}
%\usepackage[utopia, greekuppercase=italicized]{mathdesign}    % Math is narrower

% Use this for Times text and math
%\usepackage{newtxtext}
%\usepackage[libertine,cmintegrals]{newtxmath}
%\usepackage{fix-cm}

%\usepackage{txfontsb}
% or
%\usepackage{mathptmx}

%\usepackage[scaled=0.92]{helvet}
%\renewcommand{\rmdefault}{ptm}

%\usepackage{mathpazo}    % add possibly `sc` and `osf` options
%\usepackage{eulervm}

%\usepackage{fourier}
%\renewcommand{\rmdefault}{ptm}
%\usepackage{mathptm}

%\usepackage{fontspec}
%\setmainfont{lmodern}

%\usepackage[varg]{txfonts}
%\usepackage{fouriernc}
%\usepackage{mathpazo}

%\usepackage{bookman}
%\usepackage[scaled]{uarial}
%\usepackage[scaled]{helvet}
%\renewcommand*\familydefault{\sfdefault}
%\usepackage[math]{anttor}

%\newcommand\fgeorgia{\fontfamily{jvn}\selectfont}
%\newcommand\ftimes{\fontfamily{ptm}\selectfont}
%\newcommand\fhelvetica{\fontfamily{phv}\selectfont}
%\newcommand\fcourier{\fontfamily{pcr}\selectfont}
%\newcommand\fbookman{\fontfamily{pbk}\selectfont}
%\newcommand\fnewcentury{\fontfamily{pnc}\selectfont}
%\newcommand\fpalatino{\fontfamily{ppl}\selectfont}
%\newcommand\favantgarde{\fontfamily{pag}\selectfont}
%\newcommand\fnormal{\normalfont}
%\newcommand\fsize[1]{\ifnum#1>0\fontsize{#1}{#1}\selectfont\else\normalsize\fi}
%------------------------Theorem Styles-------------------------%
% Define theorem style for default spacing and normal font.
\newtheoremstyle{normal}
    {\topsep}               % Amount of space above the theorem.
    {\topsep}               % Amount of space below the theorem.
    {}                      % Font used for body of theorem.
    {}                      % Measure of space to indent.
    {\bfseries}             % Font of the header of the theorem.
    {}                      % Punctuation between head and body.
    {.5em}                  % Space after theorem head.
    {}

% Define theorem style for default spacing with italicized font.
\newtheoremstyle{normalit}{\topsep}{\topsep}
                {\itshape}{}{\bfseries}{}{.5em}{}

% Italic header environment.
\newtheoremstyle{thmit}{\topsep}{\topsep}{}{}{\itshape}{}{0.5em}{}

% Define italicized environments.
\theoremstyle{normalit}
\newtheorem{theorem}{Theorem}[section]
\newtheorem{lemma}{Lemma}[section]
\newtheorem{corollary}{Corollary}[section]
\newtheorem{proposition}{Proposition}[section]
\newtheorem*{theorem*}{Theorem}

% Define environments with italic headers.
\theoremstyle{thmit}
\newtheorem*{solution}{Solution}
\newtheorem*{fsolution}{Solution}

% Define default environments.
\theoremstyle{normal}
\newtheorem{example}{Example}[section]
\newtheorem{definition}{Definition}[section]
\newtheorem{problem}{Problem}[section]
\newtheorem{question}{Question}[section]
\newtheorem{remark}{Remark}[section]
\newtheorem{properties}{Properties}[section]
\newtheorem{notation}{Notation}[section]
\newtheorem{axiom}{Axiom}[section]
\newtheorem*{properties*}{Properties}
\newtheorem*{remark*}{Remark}
\newtheorem*{definition*}{Definition}
\theoremstyle{plain}

% Define framed environment.
\tcbuselibrary{most}
\newtcbtheorem[use counter*=theorem]{ftheorem}{Theorem}%
    {colback=green!5,colframe=green!35!black,
     fonttitle=\bfseries\upshape}{th}

\newtcbtheorem[use counter*=example]{fdefinition}{Definition}%
    {fonttitle=\bfseries\upshape,
     colback=blue!5!white,colframe=blue!75!black}{def}

\newtcbtheorem[use counter*=example]{fexample}{Example}%
    {fonttitle=\bfseries\upshape,
     colback=red!5!white,colframe=red!75!black}{ex}

\newtcbtheorem[use counter*=notation]{fnotation}{Notation}%
    {fonttitle=\bfseries\upshape,
     colback=SeaGreen!5!white,colframe=SeaGreen!75!black}{ex}

\newtcbtheorem[use counter*=corollary]{fcorollary}{Corollary}%
    {fonttitle=\bfseries\upshape,
     colback=Orchid!5!white,colframe=Orchid!75!black}{ex}

\newenvironment{bproof}{\textit{Proof.}}{\hfill$\square$}
\tcolorboxenvironment{bproof}{blanker,breakable,left=5mm,
                             before skip=10pt,after skip=10pt,
                             borderline west={1mm}{0pt}{red}}
\tcolorboxenvironment{fsolution}
    {enhanced jigsaw,colframe=cyan,interior hidden,breakable}

%--------------------Declared Math Operators--------------------%
\DeclareMathOperator{\Refl}{Refl}           % Reflection operator.
\DeclareMathOperator{\Span}{Span}           % Span of a set of vectors.
\DeclareMathOperator{\Card}{Card}           % Cardinality of set.
\DeclareMathOperator{\Ord}{Ord}             % Ordinal of ordered set.
\DeclareMathOperator{\Tr}{Tr}               % Trace of matrix.
\DeclareMathOperator{\adjoint}{adj}         % Adjoint of matrix.
\DeclareMathOperator{\rk}{rk}               % Rank of operator.
\DeclareMathOperator{\nul}{nul}             % Null space of operator.
\DeclareMathOperator{\sgn}{sgn}             % Sign of a number.
\DeclareMathOperator{\multideg}{mutlideg}   % Multi-Degree (Graphs).
\DeclareMathOperator{\GCD}{GCD}             % Greatest common denominator.
\DeclareMathOperator{\LM}{LM}               % Leading monomial
\DeclareMathOperator{\LC}{LC}               % Leading coefficient.
\DeclareMathOperator{\LT}{LT}               % Leading term.
\DeclareMathOperator{\LCM}{LCM}             % Least common multiple.
\DeclareMathOperator{\Mon}{Mon}             % Monomial.
\DeclareMathOperator{\Spec}{Spec}           % Spectrum.
\DeclareMathOperator{\proj}{proj}           % Projection.
\DeclareMathOperator{\comp}{comp}           % Component.
\DeclareMathOperator{\sinc}{sinc}           % Sinc function.
\DeclareMathOperator{\Ima}{Im}              % Image of operator.
\DeclareMathOperator{\Prin}{Prin}           % Principal value.
\DeclareMathOperator{\Mod}{mod}             % Modulus.
%------------------------New Commands---------------------------%
\DeclarePairedDelimiter\norm{\lVert}{\rVert}
\DeclarePairedDelimiter\ceil{\lceil}{\rceil}
\DeclarePairedDelimiter\floor{\lfloor}{\rfloor}
\newcommand*\diff{\mathop{}\!\mathrm{d}}
\newcommand*\Diff[1]{\mathop{}\!\mathrm{d^#1}}
\renewcommand{\mod}{\ \Mod}
\renewcommand*{\glstextformat}[1]{\textcolor{RoyalBlue}{#1}}
\renewcommand{\glsnamefont}[1]{\textbf{#1}}
\renewcommand\labelitemii{$\circ$}
\renewcommand\thesubfigure{\arabic{chapter}.\arabic{figure}}
\renewcommand\thesubfigure{%
    \arabic{chapter}.\arabic{figure}.\arabic{subfigure}}
\addto\captionsenglish{\renewcommand{\figurename}{Fig.}}
%------------------------Book Command---------------------------%
\makeatletter
\renewcommand\@pnumwidth{1cm}
\newcounter{book}
\renewcommand\thebook{\@Roman\c@book}
\newcommand\book{%
    \if@openright
        \cleardoublepage
    \else
        \clearpage
    \fi
    \thispagestyle{plain}%
    \if@twocolumn
        \onecolumn
        \@tempswatrue
    \else
        \@tempswafalse
    \fi
    \null\vfil
    \secdef\@book\@sbook
}
\def\@book[#1]#2{%
    \ifnum \c@secnumdepth >-3\relax
        \refstepcounter{book}%
        \addcontentsline{toc}{book}{
            \bookname\ \thebook:\hspace{1em}#1
        }
    \else
        \addcontentsline{toc}{book}{#1}%
    \fi
    \markboth{}{}%
    {\centering
     \interlinepenalty \@M
     \normalfont
     \ifnum \c@secnumdepth >-2\relax
       \huge\bfseries \bookname\nobreakspace\thebook
       \par
       \vskip 20\p@
     \fi
     \Huge \bfseries #2\par}%
    \@endbook}
\def\@sbook#1{%
    {\centering
     \interlinepenalty \@M
     \normalfont
     \Huge \bfseries #1\par}%
    \@endbook}
\def\@endbook{
    \vfil\newpage
        \if@twoside
            \if@openright
                \null
                \thispagestyle{empty}%
                \newpage
            \fi
        \fi
        \if@tempswa
            \twocolumn
        \fi
}
\newcommand*\l@book[2]{%
    \ifnum \c@tocdepth >-2\relax
        \addpenalty{-\@highpenalty}%
        \addvspace{2.25em \@plus\p@}%
        \setlength\@tempdima{3em}%
        \begingroup
            \parindent \z@ \rightskip \@pnumwidth
            \parfillskip -\@pnumwidth
            {
                \leavevmode
                \Large \bfseries #1\hfil \hb@xt@\@pnumwidth{
                    \hss #2
                }
            }
            \par
            \nobreak
            \global\@nobreaktrue
            \everypar{\global\@nobreakfalse\everypar{}}%
        \endgroup
    \fi}
\newcommand\bookname{Book}
\renewcommand{\thebook}{\texorpdfstring{\Numberstring{book}}{book}}
\providecommand*{\toclevel@book}{-2}
\makeatother
\titlecontents{chapter}[0pt]
    {\bfseries}
    {\chaptername\ \thecontentslabel:\quad}
    {}
    {\hfill\contentspage}
\titleformat{\part}[display]
    {\Large\bfseries}
    {\partname\nobreakspace\thepart}
    {0mm}
    {\Huge\bfseries}
    \titlecontents{part}[0pt]
    {\large\bfseries}
    {\partname\ \thecontentslabel: \quad}
    {}
    {\hfill\contentspage}
\newcommand{\MarkRightAngle}[4][.3cm]
    {\coordinate (tempa) at ($(#3)!#1!(#2)$);
     \coordinate (tempb) at ($(#3)!#1!(#4)$);
     \coordinate (tempc) at ($(tempa)!0.5!(tempb)$);%midpoint
     \draw (tempa) -- ($(#3)!2!(tempc)$) -- (tempb);}
%--------------------------LENGTHS------------------------------%
% Spacings for the Table of Contents.
\addtolength{\cftsecnumwidth}{1ex}
\addtolength{\cftsubsecindent}{1ex}
\addtolength{\cftsubsecnumwidth}{1ex}
\addtolength{\cftfignumwidth}{1ex}
\addtolength{\cfttabnumwidth}{1ex}

% Spacing for multi-column and enumerate environments.
\setlength{\multicolsep}{6pt}
\setlist[enumerate]{itemsep=0pt,topsep=3pt}

% Indent and paragraph spacing.
\setlength{\parindent}{0em}
\setlength{\parskip}{0em}
%--------------------------Main Document----------------------------%
\begin{document}
    \ifx\ifmathcoursesalgebraicgeometry\undefined
        \section*{Algebraic Geometry}
        \setcounter{section}{1}
    \fi
    \subsection{\'{E}tale Cohomology}
        \subsubsection{Review of Schemes}
            \begin{remark}
                Limitations of Affine Varieties:
                \begin{enumerate}
                    \item One would like to construct spaces
                          by gluing together simpler pieces,
                          like in geometry and topology.
                    \item Difficult over non-algebraically
                          closed fields.
                    \item Keeping track of multiplicities.
                \end{enumerate}
            \end{remark}
            Grothendieck's Theory of Schemes gives solutions to
            these problems. Should $x^{2}+y^{2}=-1$ and
            $x^{2}+y^{2}=3$ be regarded as the same over
            $\mathbb{A}_{\mathbb{R}}^{2}$? They both have no
            solution. The answer is no. An isomorphism should
            be given by an invertible transformation. In general,
            the affine variety $X\subset\mathbb{A}_{R}^n$ is
            completely determined by the coordinate ring
            $S=\mathcal{O}(X)%
              =R[x_{1},\hdots,x_{n}]/(f_{1},\hdots,f_{N})$.
            Given a compact Hausdorff space $X$, let $C(X)$
            denote the set of continous complex valued functions.
            This is a commutative ring with identity. With the
            supremum norm, it becomes a unital $C^{*}-$algebra.
            \begin{theorem}
                The map $X\rightarrow\max\{C(X)\}$
                is a homeomorphism.
            \end{theorem}
            Given a continuous map of spaces $f:X\rightarrow Y$,
            we get a homomorphism $C(Y)\rightarrow C(X)$ given
            by $g\mapsto{g}\circ f$. Thus $C(X)$ can be
            regarded as a contravariant functor. 
            \begin{theorem}[Gelfand]
                The functor $X\mapsto C(X)$ induces an
                equivalence between the category of compact
                Hausdorff spaces and the opposite category of
                commutative unital $C^{*}-$algebras.
            \end{theorem}
            \begin{definition}
                The spectrum of $R$, denoted $\Spec(R)$,
                is the set of prime ideals of $R$.
            \end{definition}
            \begin{theorem}
                The Zariski topology on $\Spec(R)$
                contains open sets
                $D(f)=\{p\in\Spec(R):f\notin{p}\}$
            \end{theorem}
            A function $f:\mathbb{R}^{n}\rightarrow\mathbb{R}$
            is $C^{\infty}$ if and only if its restriction to
            the neighborhood of every point is $C^{\infty}$.
            That is, $f\in{C}^{\infty}(X)$ if and only if
            for any open cover
            $\{U_{i}\},f\big|_{U_{i}}\in{C}^{\infty}(U_i)$.
            \begin{definition}
                If $X$ is a topological space, a presheaf
                of sets $\mathcal{F}$ is a collection of
                sets $\mathcal{F}(U)$ for each open set
                $U\subset X$ together with maps
                $\rho_{UV}:\mathcal{F}(U)\rightarrow \mathcal{F}(V)$
                for each pair $U\subset V$ such that $\rho_{UU}=id$
                and $\rho_{WV}\circ\rho_{VU}=\rho_{WU}$
                whenever $U\subset{V}\subset{W}$.
            \end{definition}
            \begin{definition}
                A sheaf is a presheaf such that for any open
                cover $\{U_i\}$ of an open $U\subset{X}$ and
                section $f_{i}\in\mathcal{F}(U_{i})$ such that
                $F_{i}\big|_{U_{i}\cap I_{j}}%
                 =f_{j}\big|_{U_{i}\cap U_{j}}$,
                there is a unique $f\in\mathcal{F}(U)$
                such that $f_{i}=f\big|_{U_{i}}$.
            \end{definition}
            \begin{definition}
                A ringed space is a pair $(X,\mathcal{O}_{X})$,
                where $X$ is a topological space and
                $\mathcal{O}_{X}$ is a sheaf of commutative rings.
            \end{definition}
            The collection of presheaves of a topological space
            form a category, denoted $Sh(X)$. 
            \begin{definition}
                A scheme is a ringed space $(X,\mathcal{O}_X)$
                which is locally an affine space.
            \end{definition}
            \begin{theorem}
                A property of commutative rings extends
                to schemes if it is local.
            \end{theorem}
        \subsubsection{Differential Calculus of Schemes}
            \begin{definition}
                The tangent space of an affine variety
                $X=V(f_{1},\hdots,f_{N})\subset\mathbb{A}_{k}^{n}$,
                denoted $T_{X,p}$, is the set of points
                $v\in{k}^{n}$ such that
                $\sum\frac{\partial{f_{j}}}%
                          {\partial{x_{i}}}p)v_{i})%
                 =0$
            \end{definition}
            \begin{definition}
                A domain
                $R=k[x_{1},\hdots,x_{n}]/(f_{1},\hdots,f_{N})$
                or $\Spec(R)$ is smooth if and only if the rank of
                $\big(\frac{\partial{f_{j}}}{\partial{x_{i}}}(p)\big)$
                is $n=\dim(R)$ for all $p\in\max(R)$. 
            \end{definition}
            \begin{definition}
                An \'{e}tale $R$ algebra is smooth of relative
                dimension 0, where
                $\det(\frac{\partial{f_{i}}}{\partial{x_{j}}})$
                is a unit in $S$.
            \end{definition}
            \begin{theorem}
                If $k$ is a field, then an algebra over $k$
                is \'{e}tale if and only if it is a finite Cartesian
                product of separable field extensions. 
            \end{theorem}
            \begin{theorem}
                The tensor product of two
                \'{e}tale algebras is \'{e}tale.
            \end{theorem}
            \begin{theorem}
                If $S$ is \'{e}tale over $R$ and $T$ is
                \'{e}tale over $S$, then $T$ is \'{e}tale over $R$.
            \end{theorem}
            \begin{definition}
                If $R$ is a commutative ring and $S$ is an
                $R$ algebra and $M$ is an $S$ module,
                then an $R$ linear derivation from $S$ to $M$ is
                a map $\delta:S\rightarrow M$ such that
                $\delta(s_{1}+s_{2})=\delta(s_{1})+\delta(s_{2})$,
                $\delta(s_{1}s_{2})%
                 =s_{1}\delta(s_{2})+s_{2}\delta(s_{1})$,
                 and $\delta(r)=0$ for all $r\in{R}$.
            \end{definition}
            \begin{theorem}
                There exists an $S$ module $\Omega_{S/R}$
                with a universal $R$ linear derivation
                $d:S\rightarrow \Omega_{S/R}$.
            \end{theorem}
            \begin{theorem}
                If $M$ is a finitely generated module over a
                Noetherian ring $R$, then these are equivalent:
                \begin{enumerate}
                    \begin{multicols}{3}
                        \item $M$ is locally free.
                        \item $\forall_{p\in\Spec(R)},R_{p}\otimes M$
                              is free.
                        \item $M$ is projective.
                    \end{multicols}
                \end{enumerate}
            \end{theorem}
            \begin{definition}
                If $M$ is an $S-$module, then $M$ is called
                flat if $M\otimes{i}$ is injective for any $i$.
            \end{definition}
            \begin{theorem}
                If $S$ is an $R$ algebra and $M$ is an $S$ module,
                $f\in S$ is an element such that multiplication
                by $f$ is injective on $M\otimes k(m)$ for all
                $m\in\max(R)$, and if $M$ is flat over $R$,
                then $M/fM$ is flat over $R$.
            \end{theorem}
            \begin{theorem}
                A smooth algebra is flat.
            \end{theorem}
            \begin{theorem}
                If $R$ is a Noetherian ring, then a homomorphism
                $R\rightarrow S$ is \'{e}tale if and only if:
                \begin{enumerate}
                    \item $S$ is finitely generated as an algebra.
                    \item $S$ is flat as an $R$-module.
                    \item $\Omega_{S/R}=0$.
                \end{enumerate}
            \end{theorem}
            \begin{definition}
                A sheaf on a scheme is quasi-coherent if
                it is with respect to some affine open cover.
            \end{definition}
            \begin{theorem}
                If $f:X\rightarrow Y$ is a morphism, there
                exists a quasi-coherent sheaf $\Omega_{X/Y}$
                such that
                $\Omega_{X/Y}\big|_{\Spec(S_{ij})}=\Omega_{S_{ij}/R_i}$
                for open affine covers $\Spec(R_i)=U_{i}$.
            \end{theorem}
        \subsubsection{The Fundamental \'{E}tale Group}
            \begin{definition}
                A topological group is a topological space
                $(X,\tau)$ with a group structure $(X,*)$ such
                that $*:X\times X\rightarrow X$ is a continuous
                function with respect to the product topology.
            \end{definition}
            \begin{theorem}
                A topological space is profinite if and only
                if it is compact Hausdorff and totally disconnected.
            \end{theorem}
            \begin{definition}
                The topological fundamental group of a
                topological space $X$, denoted $\pi_1(X)$,
                is the group of homotopy classes of loops in $X$
                with a given base point.
            \end{definition}
            \begin{theorem}
                Any \'{e}tale morphism $Y\rightarrow X$ is
                a finite to one covering space of $X$ with
                the usual topology.
            \end{theorem}
            \begin{theorem}[Grothendieck's Theorem]
                If $X$ is a scheme of finite type of
                $\mathbb{C}$, then $\pi_{1}^{et}(X)$ is the
                profinite completion of $\pi_{1}{X}$.
            \end{theorem}
        \subsubsection{\'{E}tale Topology}
            \begin{remark}
                Given a topological space $(X,\tau)$,
                the topology $\tau$
                (That is, the collection of open sets) forms a
                partially ordered set with respect to set
                inclusion. There also exists a notion of
                open covering $U=\cup U_{i}$.
            \end{remark}
            \begin{definition}
                A Groethendieck Topology on a category $C$ with
                fibre products is a collection of families of
                morphisms $U_{i}\rightarrow U$ such that:
                \begin{enumerate}
                    \item The family consisting of a single
                          isomorphism $\{U\sim{U}\}$ is a covering.
                    \item If $\{U_{i}\rightarrow{U}\}$ and
                          $\{V_{ij}\rightarrow{U_{i}}\}$ are coverings,
                          then so is the composition
                          $\{V_{ij}\rightarrow{U}\}$.
                \end{enumerate}
            \end{definition}
            \begin{definition}
                A site is a category with a Grothendieck Topology.
            \end{definition}
\end{document}