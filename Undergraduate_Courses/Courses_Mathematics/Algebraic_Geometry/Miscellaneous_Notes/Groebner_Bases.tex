\documentclass[crop=false,class=article,oneside]{standalone}
%----------------------------Preamble-------------------------------%
%---------------------------Packages----------------------------%
\usepackage{geometry}
\geometry{b5paper, margin=1.0in}
\usepackage[T1]{fontenc}
\usepackage{graphicx, float}            % Graphics/Images.
\usepackage{natbib}                     % For bibliographies.
\bibliographystyle{agsm}                % Bibliography style.
\usepackage[french, english]{babel}     % Language typesetting.
\usepackage[dvipsnames]{xcolor}         % Color names.
\usepackage{listings, lstlinebgrd}      % Verbatim-Like Tools.
\usepackage{mathtools, esint, mathrsfs} % amsmath and integrals.
\usepackage{amsthm, amsfonts}           % Fonts and theorems.
\usepackage{tabularx}
\usepackage{tcolorbox}                  % Frames around theorems.
\usepackage{upgreek}                    % Non-Italic Greek.
\usepackage{paracol}                    % Two-column styling.
\usepackage{wrapfig}                    % Wrap text around figure.
\usepackage{fmtcount, etoolbox}         % For the \book{} command.
\usepackage[newparttoc]{titlesec}       % Formatting chapter, etc.
\usepackage{titletoc}                   % Allows \book in toc.
\usepackage[nottoc]{tocbibind}          % Bibliography in toc.
\usepackage[titles]{tocloft}            % ToC formatting.
\usepackage{multicol, enumitem}         % Multi-column/enumerate.
\usepackage{import}                     % Import external files.
\usepackage{pgfplots, tikz}             % Drawing/graphing tools.
\usetikzlibrary{
    calc,                   % Calculating right angles and more.
    angles,                 % Drawing angles within triangles.
    arrows.meta,            % Latex and Stealth arrows.
    quotes,                 % Adding labels to angles.
    positioning,            % Relative positioning of nodes.
    decorations.markings,   % Adding arrows in the middle of a line.
    patterns,
    arrows,
    shapes,
    shapes.geometric,
    cd,
    hobby,
    babel
}                                       % Libraries for tikz.
\pgfplotsset{compat=1.9}                % Version of pgfplots.
\usepackage[font=scriptsize,
            labelformat=simple,
            labelsep=colon]{subcaption} % Subfigure captions.
\usepackage[font={scriptsize},
            hypcap=true,
            labelsep=colon]{caption}    % Figure captions.
\usepackage{hyperref}                   % Allows for hyperlinks.
\hypersetup{
    colorlinks=true,
    linkcolor=blue,
    filecolor=magenta,
    urlcolor=Cerulean,
    citecolor=SkyBlue
}                           % Colors for hyperref.
\usepackage[toc,acronym,nogroupskip]{glossaries} % Glossaries and acronyms.
\usepackage[subpreambles=false]{standalone}      % Complileable sub files.

% Various font stuff from kiwi.
% Use this for Times text and Computer Modern math
%\usepackage{times}

% Quite nice
%\usepackage[charter, greekfamily=, greekuppercase=italicized]{mathdesign}
%\usepackage[utopia, greekuppercase=italicized]{mathdesign}    % Math is narrower

% Use this for Times text and math
%\usepackage{newtxtext}
%\usepackage[libertine,cmintegrals]{newtxmath}
%\usepackage{fix-cm}

%\usepackage{txfontsb}
% or
%\usepackage{mathptmx}

%\usepackage[scaled=0.92]{helvet}
%\renewcommand{\rmdefault}{ptm}

%\usepackage{mathpazo}    % add possibly `sc` and `osf` options
%\usepackage{eulervm}

%\usepackage{fourier}
%\renewcommand{\rmdefault}{ptm}
%\usepackage{mathptm}

%\usepackage{fontspec}
%\setmainfont{lmodern}

%\usepackage[varg]{txfonts}
%\usepackage{fouriernc}
%\usepackage{mathpazo}

%\usepackage{bookman}
%\usepackage[scaled]{uarial}
%\usepackage[scaled]{helvet}
%\renewcommand*\familydefault{\sfdefault}
%\usepackage[math]{anttor}

%\newcommand\fgeorgia{\fontfamily{jvn}\selectfont}
%\newcommand\ftimes{\fontfamily{ptm}\selectfont}
%\newcommand\fhelvetica{\fontfamily{phv}\selectfont}
%\newcommand\fcourier{\fontfamily{pcr}\selectfont}
%\newcommand\fbookman{\fontfamily{pbk}\selectfont}
%\newcommand\fnewcentury{\fontfamily{pnc}\selectfont}
%\newcommand\fpalatino{\fontfamily{ppl}\selectfont}
%\newcommand\favantgarde{\fontfamily{pag}\selectfont}
%\newcommand\fnormal{\normalfont}
%\newcommand\fsize[1]{\ifnum#1>0\fontsize{#1}{#1}\selectfont\else\normalsize\fi}
%------------------------Theorem Styles-------------------------%
% Define theorem style for default spacing and normal font.
\newtheoremstyle{normal}
    {\topsep}               % Amount of space above the theorem.
    {\topsep}               % Amount of space below the theorem.
    {}                      % Font used for body of theorem.
    {}                      % Measure of space to indent.
    {\bfseries}             % Font of the header of the theorem.
    {}                      % Punctuation between head and body.
    {.5em}                  % Space after theorem head.
    {}

% Define theorem style for default spacing with italicized font.
\newtheoremstyle{normalit}{\topsep}{\topsep}
                {\itshape}{}{\bfseries}{}{.5em}{}

% Italic header environment.
\newtheoremstyle{thmit}{\topsep}{\topsep}{}{}{\itshape}{}{0.5em}{}

% Define italicized environments.
\theoremstyle{normalit}
\newtheorem{theorem}{Theorem}[section]
\newtheorem{lemma}{Lemma}[section]
\newtheorem{corollary}{Corollary}[section]
\newtheorem{proposition}{Proposition}[section]
\newtheorem*{theorem*}{Theorem}

% Define environments with italic headers.
\theoremstyle{thmit}
\newtheorem*{solution}{Solution}
\newtheorem*{fsolution}{Solution}

% Define default environments.
\theoremstyle{normal}
\newtheorem{example}{Example}[section]
\newtheorem{definition}{Definition}[section]
\newtheorem{problem}{Problem}[section]
\newtheorem{question}{Question}[section]
\newtheorem{remark}{Remark}[section]
\newtheorem{properties}{Properties}[section]
\newtheorem{notation}{Notation}[section]
\newtheorem{axiom}{Axiom}[section]
\newtheorem*{properties*}{Properties}
\newtheorem*{remark*}{Remark}
\newtheorem*{definition*}{Definition}
\theoremstyle{plain}

% Define framed environment.
\tcbuselibrary{most}
\newtcbtheorem[use counter*=theorem]{ftheorem}{Theorem}%
    {colback=green!5,colframe=green!35!black,
     fonttitle=\bfseries\upshape}{th}

\newtcbtheorem[use counter*=example]{fdefinition}{Definition}%
    {fonttitle=\bfseries\upshape,
     colback=blue!5!white,colframe=blue!75!black}{def}

\newtcbtheorem[use counter*=example]{fexample}{Example}%
    {fonttitle=\bfseries\upshape,
     colback=red!5!white,colframe=red!75!black}{ex}

\newtcbtheorem[use counter*=notation]{fnotation}{Notation}%
    {fonttitle=\bfseries\upshape,
     colback=SeaGreen!5!white,colframe=SeaGreen!75!black}{ex}

\newtcbtheorem[use counter*=corollary]{fcorollary}{Corollary}%
    {fonttitle=\bfseries\upshape,
     colback=Orchid!5!white,colframe=Orchid!75!black}{ex}

\newenvironment{bproof}{\textit{Proof.}}{\hfill$\square$}
\tcolorboxenvironment{bproof}{blanker,breakable,left=5mm,
                             before skip=10pt,after skip=10pt,
                             borderline west={1mm}{0pt}{red}}
\tcolorboxenvironment{fsolution}
    {enhanced jigsaw,colframe=cyan,interior hidden,breakable}

%--------------------Declared Math Operators--------------------%
\DeclareMathOperator{\Refl}{Refl}           % Reflection operator.
\DeclareMathOperator{\Span}{Span}           % Span of a set of vectors.
\DeclareMathOperator{\Card}{Card}           % Cardinality of set.
\DeclareMathOperator{\Ord}{Ord}             % Ordinal of ordered set.
\DeclareMathOperator{\Tr}{Tr}               % Trace of matrix.
\DeclareMathOperator{\adjoint}{adj}         % Adjoint of matrix.
\DeclareMathOperator{\rk}{rk}               % Rank of operator.
\DeclareMathOperator{\nul}{nul}             % Null space of operator.
\DeclareMathOperator{\sgn}{sgn}             % Sign of a number.
\DeclareMathOperator{\multideg}{mutlideg}   % Multi-Degree (Graphs).
\DeclareMathOperator{\GCD}{GCD}             % Greatest common denominator.
\DeclareMathOperator{\LM}{LM}               % Leading monomial
\DeclareMathOperator{\LC}{LC}               % Leading coefficient.
\DeclareMathOperator{\LT}{LT}               % Leading term.
\DeclareMathOperator{\LCM}{LCM}             % Least common multiple.
\DeclareMathOperator{\Mon}{Mon}             % Monomial.
\DeclareMathOperator{\Spec}{Spec}           % Spectrum.
\DeclareMathOperator{\proj}{proj}           % Projection.
\DeclareMathOperator{\comp}{comp}           % Component.
\DeclareMathOperator{\sinc}{sinc}           % Sinc function.
\DeclareMathOperator{\Ima}{Im}              % Image of operator.
\DeclareMathOperator{\Prin}{Prin}           % Principal value.
\DeclareMathOperator{\Mod}{mod}             % Modulus.
%------------------------New Commands---------------------------%
\DeclarePairedDelimiter\norm{\lVert}{\rVert}
\DeclarePairedDelimiter\ceil{\lceil}{\rceil}
\DeclarePairedDelimiter\floor{\lfloor}{\rfloor}
\newcommand*\diff{\mathop{}\!\mathrm{d}}
\newcommand*\Diff[1]{\mathop{}\!\mathrm{d^#1}}
\renewcommand{\mod}{\ \Mod}
\renewcommand*{\glstextformat}[1]{\textcolor{RoyalBlue}{#1}}
\renewcommand{\glsnamefont}[1]{\textbf{#1}}
\renewcommand\labelitemii{$\circ$}
\renewcommand\thesubfigure{\arabic{chapter}.\arabic{figure}}
\renewcommand\thesubfigure{%
    \arabic{chapter}.\arabic{figure}.\arabic{subfigure}}
\addto\captionsenglish{\renewcommand{\figurename}{Fig.}}
%------------------------Book Command---------------------------%
\makeatletter
\renewcommand\@pnumwidth{1cm}
\newcounter{book}
\renewcommand\thebook{\@Roman\c@book}
\newcommand\book{%
    \if@openright
        \cleardoublepage
    \else
        \clearpage
    \fi
    \thispagestyle{plain}%
    \if@twocolumn
        \onecolumn
        \@tempswatrue
    \else
        \@tempswafalse
    \fi
    \null\vfil
    \secdef\@book\@sbook
}
\def\@book[#1]#2{%
    \ifnum \c@secnumdepth >-3\relax
        \refstepcounter{book}%
        \addcontentsline{toc}{book}{
            \bookname\ \thebook:\hspace{1em}#1
        }
    \else
        \addcontentsline{toc}{book}{#1}%
    \fi
    \markboth{}{}%
    {\centering
     \interlinepenalty \@M
     \normalfont
     \ifnum \c@secnumdepth >-2\relax
       \huge\bfseries \bookname\nobreakspace\thebook
       \par
       \vskip 20\p@
     \fi
     \Huge \bfseries #2\par}%
    \@endbook}
\def\@sbook#1{%
    {\centering
     \interlinepenalty \@M
     \normalfont
     \Huge \bfseries #1\par}%
    \@endbook}
\def\@endbook{
    \vfil\newpage
        \if@twoside
            \if@openright
                \null
                \thispagestyle{empty}%
                \newpage
            \fi
        \fi
        \if@tempswa
            \twocolumn
        \fi
}
\newcommand*\l@book[2]{%
    \ifnum \c@tocdepth >-2\relax
        \addpenalty{-\@highpenalty}%
        \addvspace{2.25em \@plus\p@}%
        \setlength\@tempdima{3em}%
        \begingroup
            \parindent \z@ \rightskip \@pnumwidth
            \parfillskip -\@pnumwidth
            {
                \leavevmode
                \Large \bfseries #1\hfil \hb@xt@\@pnumwidth{
                    \hss #2
                }
            }
            \par
            \nobreak
            \global\@nobreaktrue
            \everypar{\global\@nobreakfalse\everypar{}}%
        \endgroup
    \fi}
\newcommand\bookname{Book}
\renewcommand{\thebook}{\texorpdfstring{\Numberstring{book}}{book}}
\providecommand*{\toclevel@book}{-2}
\makeatother
\titlecontents{chapter}[0pt]
    {\bfseries}
    {\chaptername\ \thecontentslabel:\quad}
    {}
    {\hfill\contentspage}
\titleformat{\part}[display]
    {\Large\bfseries}
    {\partname\nobreakspace\thepart}
    {0mm}
    {\Huge\bfseries}
    \titlecontents{part}[0pt]
    {\large\bfseries}
    {\partname\ \thecontentslabel: \quad}
    {}
    {\hfill\contentspage}
\newcommand{\MarkRightAngle}[4][.3cm]
    {\coordinate (tempa) at ($(#3)!#1!(#2)$);
     \coordinate (tempb) at ($(#3)!#1!(#4)$);
     \coordinate (tempc) at ($(tempa)!0.5!(tempb)$);%midpoint
     \draw (tempa) -- ($(#3)!2!(tempc)$) -- (tempb);}
%--------------------------LENGTHS------------------------------%
% Spacings for the Table of Contents.
\addtolength{\cftsecnumwidth}{1ex}
\addtolength{\cftsubsecindent}{1ex}
\addtolength{\cftsubsecnumwidth}{1ex}
\addtolength{\cftfignumwidth}{1ex}
\addtolength{\cfttabnumwidth}{1ex}

% Spacing for multi-column and enumerate environments.
\setlength{\multicolsep}{6pt}
\setlist[enumerate]{itemsep=0pt,topsep=3pt}

% Indent and paragraph spacing.
\setlength{\parindent}{0em}
\setlength{\parskip}{0em}
%--------------------------Main Document----------------------------%
\begin{document}
    \ifx\ifmathcoursesalgebraicgeometry\undefined
        \section*{Algebraic Geometry}
        \setcounter{section}{1}
    \fi
    \subsection{Groebner Bases}
        \begin{definition}
            A ring is a set $R$ with two binary operations $+$
            and $\cdot$, called addition and multiplication,
            such that the following are true:
            \begin{enumerate}
                \begin{multicols}{3}
                    \item $(R,+)$ is an Abelian Group
                    \item $(a\cdot{b})\cdot{c}=a\cdot(b\cdot{c})$
                    \item $a\cdot(b+c)=(a\cdot b)+(a\cdot c)$
                \end{multicols}
            \end{enumerate}
        \end{definition}
        \begin{definition}
            A commutative ring is a ring $R$ such that
            $\forall_{a,b\in R},a\cdot{b}=b\cdot{a}$
        \end{definition}
        \begin{definition}
            A ring with identity is a ring $R$ such that
            $\exists_{1_{R}\in R}:\forall_{a\in R}, 1_{R}\cdot a=a\cdot 1_{R}=a$
        \end{definition}
        \begin{definition}
            A subring of a ring with identity $R$ is a set
            $S\subset R$ such that $1_{R}\in S$, and $S$ is
            closed under the ring operations.
        \end{definition}
        \begin{remark}
            Any field is a ring.
        \end{remark}
        \begin{definition}
            A monomial in variables $x_1,\hdots, x_n$ over a
            ring $R$ is a product
            $x^\alpha=\prod_{k=1}^{n} x_1^{\alpha_1}$,
            where $(\alpha_1,\hdots,\alpha_n)\in \mathbb{N}^n$.
        \end{definition}
        \begin{notation}
            The set of monomials in $n$ variables over
            $R$ is denoted $\Mon_{R}(x_1,\hdots, x_n)$
        \end{notation}
        \begin{definition}
            If $\alpha,\beta \in \mathbb{N}^n$ such that
            $\alpha_i \leq \beta_i$, then $x^{\alpha}$ is said
            to divide $x^\beta$, denoted $x^\alpha \vert x^\beta$,
            if $x^\beta = x^\alpha \cdot x^\gamma$ for some
            $\gamma\in\mathbb{N}^n$.
        \end{definition}
        \begin{definition}
            A term is a monomial multiplied by a coefficient in $R$.
        \end{definition}
        \begin{definition}
            A polnyomial over $R$ is a finite $R-$linear
            combination of monomials,
            $f=\sum_{\alpha} a_{\alpha}\cdot x^{\alpha}$.
        \end{definition}
        \begin{notation}
            The set of all polynomials in $n$ variables over
            a ring $R$ is denoted $R[x_1,\hdots, x_n]$.
        \end{notation}
        \begin{theorem}
            If $R$ is a commutative ring with identity,
            then $R[x_1,\hdots, x_n]$ is a commutative
            ring with identity.
        \end{theorem}
        \begin{definition}
            A polynomial $f\in R[x_1,\hdots, x_n]$ is
            called a constant polynomial if $f\in R$.
        \end{definition}
        \begin{definition}
            A field $k$ is a commutative ring with identity
            such that for all $a\in k$, $a\ne 0$, there is a
            $b\in k$ such that $a\cdot b=1$
        \end{definition}
        \begin{remark}
            We usually work with fields and consider
            polynomial rings of the form $k[x_1,\hdots ,x_n]$.
        \end{remark}
        \begin{definition}
            A total ordering on a set $A$ is a relation
            $>$ such that $\forall_{a,b\in A}$, precisely one
            of the following truee:
            \begin{enumerate}
                \begin{multicols}{3}
                    \item $a<b$
                    \item $a=b$
                    \item $b<a$
                \end{multicols}
            \end{enumerate}
        \end{definition}
        \begin{definition}
            A relation $\sim$ on a set $A$ is said to be
            transitive if for all $a,b,c\in A$, if $a\sim b$ and
            $b\sim c$, then $a\sim c$.
        \end{definition}
        \begin{definition}
            A well ordering on a set $A$ is a relation $<$
            such that for every subset $E\subset A$, there is an
            element $x\in E$ such that for all $y\in E$, $y\ne x$,
            we have $x<y$.
        \end{definition}
        \begin{remark}
            Equivalently, a well ordering on a set $A$
            is a relation $<$ such that for every monotonically
            decreasing sequence $\alpha_n$, there is an
            $N\in \mathbb{N}$ such that for all $n>N$,
            $\alpha_n = \alpha_N$. That is,
            decreasing sequences terminate.
        \end{remark}
        \begin{definition}
            A monomial ordering on $\mathbb{N}^n$ is a relation
            $>$ such that $>$ is total, transitive, well
            ordering. A well ordering on $k[x_1,\hdots ,x_n]$
            is a well ordering on
            $\alpha=(\alpha_1,\hdots,\alpha_n)\in\mathbb{N}^n$.
        \end{definition}
        \begin{definition}
            The lexicographic ordering on $\mathbb{N}^n$ is
            defined as
            $(\alpha_1,\hdots,\alpha_n)\underset{Lex}{>}%
             (\beta_1,\hdots,\beta_n)$
            if the left-most non-zero entry of
            $(\alpha_1-\beta_1,\hdots, \alpha_n-\beta_n)$
            is positive.
        \end{definition}
        \begin{theorem}
            The lexicographic ordering is a monomial ordering.
        \end{theorem}
        \begin{definition}
            The graded lexicographic ordering is defined as
            $(\alpha_1,\hdots,\alpha_n)\underset{GrLex}{>}%
             (\beta_1,\hdots, \beta_n)$
            if $|\alpha|>|\beta|$ or $|\alpha|=|\beta|$
            and $\alpha\underset{Lex}{>}\beta$.
        \end{definition}
        \begin{theorem}
            The graded lexicographic ordering is a monomial ordering.
        \end{theorem}
        \begin{theorem}[The Division Algorithm]
            If $f_1,\hdots, f_s\in k[x_1,\hdots ,x_n]$ are
            non-zero polynomials and if $>$ is a monomial ordering,
            then there are $r,q_1,\hdots, q_n\in k[x_1,\hdots ,x_n]$
            such that the following are true:
            \begin{enumerate}
                \item $f=q_{1}f_{1}+\hdots+q_{s}f_{s}+r$
                \item No term of $r$ is divisible by
                      any of $\LT(f_{1}),\hdots,\LT(f_{s})$.
                \item $\LT(f)=\max_{>}\{\LT(q_{i})%
                       \cdot\LT(f_i):q_i\ne{0}\}$
            \end{enumerate}
        \end{theorem}
        \begin{definition}
            An ideal
            $I=\langle{x}^{\alpha}:\alpha\in{A}\rangle%
              =\{\sum_{\alpha}h_{\alpha}x^\alpha,h_{\alpha}%
               \in k[x_1,\hdots ,x_n]\}$
            is called a monomial ideal.
        \end{definition}
        \begin{theorem}
            If $I=\langle{x}^\alpha:\alpha\in{A}\rangle$
            is a monomial ideal,
            $\beta\in\mathbb{N}^n$, then $x^\beta\in{I}$
            if and only if there is an $\alpha\in{A}$
            such that $x^{\alpha}$ divides $x^{\beta}$.
        \end{theorem}
        \begin{theorem}
            If $I$ is a monomial ideal,
            $f\in{k}[x_1,\hdots ,x_n]$,
            then the following are equivalent:
            \begin{enumerate}
                \item $f\in I$
                \item Every term of $f$ lies in $I$.
                \item $f$ is a $k-$linear combination of
                      monomials in $I$.
            \end{enumerate}
        \end{theorem}
        \begin{theorem}[Dickson's Lemma]
            Every monomial ideal of $k[x_{1},\hdots,x_{n}]$
            is finitely generated.
        \end{theorem}
        \begin{theorem}[Hilbert's Basis Theorem]
            Every ideal $I\subset{k}[x_{1},\hdots,x_{n}]$
            is finitely generated.
        \end{theorem}
        \begin{definition}
            If $>$ is a monomial ordering on $k[x_{1},\hdots,x_{n}]$,
            then a Groebner Basis of $I\subset k[x_{1},\hdots,x_{n}]$
            is a set $G=\{g_{1},\hdots,g_{s}\}$ such that
            $\langle\LT(I)\rangle%
             =\langle \LT(g_1),\hdots,\LT(g_s)\rangle$
        \end{definition}
        \begin{theorem}
            Every non-zero ideal
            $I\subset{k}[x_{1},\hdots,x_{n}]$
            has a Groebner Basis.
        \end{theorem}
\end{document}