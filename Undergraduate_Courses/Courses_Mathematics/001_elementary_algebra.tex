\documentclass[crop=false,class=book]{standalone}
%---------------PREAMBLE------------%
%---------------------------Packages----------------------------%
\usepackage{geometry}
\geometry{b5paper, margin=1.0in}
\usepackage[T1]{fontenc}
\usepackage{graphicx, float}            % Graphics/Images.
\usepackage{natbib}                     % For bibliographies.
\bibliographystyle{agsm}                % Bibliography style.
\usepackage[french, english]{babel}     % Language typesetting.
\usepackage[dvipsnames]{xcolor}         % Color names.
\usepackage{listings}                   % Verbatim-Like Tools.
\usepackage{mathtools, esint, mathrsfs} % amsmath and integrals.
\usepackage{amsthm, amsfonts, amssymb}  % Fonts and theorems.
\usepackage{tcolorbox}                  % Frames around theorems.
\usepackage{upgreek}                    % Non-Italic Greek.
\usepackage{fmtcount, etoolbox}         % For the \book{} command.
\usepackage[newparttoc]{titlesec}       % Formatting chapter, etc.
\usepackage{titletoc}                   % Allows \book in toc.
\usepackage[nottoc]{tocbibind}          % Bibliography in toc.
\usepackage[titles]{tocloft}            % ToC formatting.
\usepackage{pgfplots, tikz}             % Drawing/graphing tools.
\usepackage{imakeidx}                   % Used for index.
\usetikzlibrary{
    calc,                   % Calculating right angles and more.
    angles,                 % Drawing angles within triangles.
    arrows.meta,            % Latex and Stealth arrows.
    quotes,                 % Adding labels to angles.
    positioning,            % Relative positioning of nodes.
    decorations.markings,   % Adding arrows in the middle of a line.
    patterns,
    arrows
}                                       % Libraries for tikz.
\pgfplotsset{compat=1.9}                % Version of pgfplots.
\usepackage[font=scriptsize,
            labelformat=simple,
            labelsep=colon]{subcaption} % Subfigure captions.
\usepackage[font={scriptsize},
            hypcap=true,
            labelsep=colon]{caption}    % Figure captions.
\usepackage[pdftex,
            pdfauthor={Ryan Maguire},
            pdftitle={Mathematics and Physics},
            pdfsubject={Mathematics, Physics, Science},
            pdfkeywords={Mathematics, Physics, Computer Science, Biology},
            pdfproducer={LaTeX},
            pdfcreator={pdflatex}]{hyperref}
\hypersetup{
    colorlinks=true,
    linkcolor=blue,
    filecolor=magenta,
    urlcolor=Cerulean,
    citecolor=SkyBlue
}                           % Colors for hyperref.
\usepackage[toc,acronym,nogroupskip,nopostdot]{glossaries}
\usepackage{glossary-mcols}
%------------------------Theorem Styles-------------------------%
\theoremstyle{plain}
\newtheorem{theorem}{Theorem}[section]

% Define theorem style for default spacing and normal font.
\newtheoremstyle{normal}
    {\topsep}               % Amount of space above the theorem.
    {\topsep}               % Amount of space below the theorem.
    {}                      % Font used for body of theorem.
    {}                      % Measure of space to indent.
    {\bfseries}             % Font of the header of the theorem.
    {}                      % Punctuation between head and body.
    {.5em}                  % Space after theorem head.
    {}

% Italic header environment.
\newtheoremstyle{thmit}{\topsep}{\topsep}{}{}{\itshape}{}{0.5em}{}

% Define environments with italic headers.
\theoremstyle{thmit}
\newtheorem*{solution}{Solution}

% Define default environments.
\theoremstyle{normal}
\newtheorem{example}{Example}[section]
\newtheorem{definition}{Definition}[section]
\newtheorem{problem}{Problem}[section]

% Define framed environment.
\tcbuselibrary{most}
\newtcbtheorem[use counter*=theorem]{ftheorem}{Theorem}{%
    before=\par\vspace{2ex},
    boxsep=0.5\topsep,
    after=\par\vspace{2ex},
    colback=green!5,
    colframe=green!35!black,
    fonttitle=\bfseries\upshape%
}{thm}

\newtcbtheorem[auto counter, number within=section]{faxiom}{Axiom}{%
    before=\par\vspace{2ex},
    boxsep=0.5\topsep,
    after=\par\vspace{2ex},
    colback=Apricot!5,
    colframe=Apricot!35!black,
    fonttitle=\bfseries\upshape%
}{ax}

\newtcbtheorem[use counter*=definition]{fdefinition}{Definition}{%
    before=\par\vspace{2ex},
    boxsep=0.5\topsep,
    after=\par\vspace{2ex},
    colback=blue!5!white,
    colframe=blue!75!black,
    fonttitle=\bfseries\upshape%
}{def}

\newtcbtheorem[use counter*=example]{fexample}{Example}{%
    before=\par\vspace{2ex},
    boxsep=0.5\topsep,
    after=\par\vspace{2ex},
    colback=red!5!white,
    colframe=red!75!black,
    fonttitle=\bfseries\upshape%
}{ex}

\newtcbtheorem[auto counter, number within=section]{fnotation}{Notation}{%
    before=\par\vspace{2ex},
    boxsep=0.5\topsep,
    after=\par\vspace{2ex},
    colback=SeaGreen!5!white,
    colframe=SeaGreen!75!black,
    fonttitle=\bfseries\upshape%
}{not}

\newtcbtheorem[use counter*=remark]{fremark}{Remark}{%
    fonttitle=\bfseries\upshape,
    colback=Goldenrod!5!white,
    colframe=Goldenrod!75!black}{ex}

\newenvironment{bproof}{\textit{Proof.}}{\hfill$\square$}
\tcolorboxenvironment{bproof}{%
    blanker,
    breakable,
    left=3mm,
    before skip=5pt,
    after skip=10pt,
    borderline west={0.6mm}{0pt}{green!80!black}
}

\AtEndEnvironment{lexample}{$\hfill\textcolor{red}{\blacksquare}$}
\newtcbtheorem[use counter*=example]{lexample}{Example}{%
    empty,
    title={Example~\theexample},
    boxed title style={%
        empty,
        size=minimal,
        toprule=2pt,
        top=0.5\topsep,
    },
    coltitle=red,
    fonttitle=\bfseries,
    parbox=false,
    boxsep=0pt,
    before=\par\vspace{2ex},
    left=0pt,
    right=0pt,
    top=3ex,
    bottom=1ex,
    before=\par\vspace{2ex},
    after=\par\vspace{2ex},
    breakable,
    pad at break*=0mm,
    vfill before first,
    overlay unbroken={%
        \draw[red, line width=2pt]
            ([yshift=-1.2ex]title.south-|frame.west) to
            ([yshift=-1.2ex]title.south-|frame.east);
        },
    overlay first={%
        \draw[red, line width=2pt]
            ([yshift=-1.2ex]title.south-|frame.west) to
            ([yshift=-1.2ex]title.south-|frame.east);
    },
}{ex}

\AtEndEnvironment{ldefinition}{$\hfill\textcolor{Blue}{\blacksquare}$}
\newtcbtheorem[use counter*=definition]{ldefinition}{Definition}{%
    empty,
    title={Definition~\thedefinition:~{#1}},
    boxed title style={%
        empty,
        size=minimal,
        toprule=2pt,
        top=0.5\topsep,
    },
    coltitle=Blue,
    fonttitle=\bfseries,
    parbox=false,
    boxsep=0pt,
    before=\par\vspace{2ex},
    left=0pt,
    right=0pt,
    top=3ex,
    bottom=0pt,
    before=\par\vspace{2ex},
    after=\par\vspace{1ex},
    breakable,
    pad at break*=0mm,
    vfill before first,
    overlay unbroken={%
        \draw[Blue, line width=2pt]
            ([yshift=-1.2ex]title.south-|frame.west) to
            ([yshift=-1.2ex]title.south-|frame.east);
        },
    overlay first={%
        \draw[Blue, line width=2pt]
            ([yshift=-1.2ex]title.south-|frame.west) to
            ([yshift=-1.2ex]title.south-|frame.east);
    },
}{def}

\AtEndEnvironment{ltheorem}{$\hfill\textcolor{Green}{\blacksquare}$}
\newtcbtheorem[use counter*=theorem]{ltheorem}{Theorem}{%
    empty,
    title={Theorem~\thetheorem:~{#1}},
    boxed title style={%
        empty,
        size=minimal,
        toprule=2pt,
        top=0.5\topsep,
    },
    coltitle=Green,
    fonttitle=\bfseries,
    parbox=false,
    boxsep=0pt,
    before=\par\vspace{2ex},
    left=0pt,
    right=0pt,
    top=3ex,
    bottom=-1.5ex,
    breakable,
    pad at break*=0mm,
    vfill before first,
    overlay unbroken={%
        \draw[Green, line width=2pt]
            ([yshift=-1.2ex]title.south-|frame.west) to
            ([yshift=-1.2ex]title.south-|frame.east);},
    overlay first={%
        \draw[Green, line width=2pt]
            ([yshift=-1.2ex]title.south-|frame.west) to
            ([yshift=-1.2ex]title.south-|frame.east);
    }
}{thm}

%--------------------Declared Math Operators--------------------%
\DeclareMathOperator{\adjoint}{adj}         % Adjoint.
\DeclareMathOperator{\Card}{Card}           % Cardinality.
\DeclareMathOperator{\curl}{curl}           % Curl.
\DeclareMathOperator{\diam}{diam}           % Diameter.
\DeclareMathOperator{\dist}{dist}           % Distance.
\DeclareMathOperator{\Div}{div}             % Divergence.
\DeclareMathOperator{\Erf}{Erf}             % Error Function.
\DeclareMathOperator{\Erfc}{Erfc}           % Complementary Error Function.
\DeclareMathOperator{\Ext}{Ext}             % Exterior.
\DeclareMathOperator{\GCD}{GCD}             % Greatest common denominator.
\DeclareMathOperator{\grad}{grad}           % Gradient
\DeclareMathOperator{\Ima}{Im}              % Image.
\DeclareMathOperator{\Int}{Int}             % Interior.
\DeclareMathOperator{\LC}{LC}               % Leading coefficient.
\DeclareMathOperator{\LCM}{LCM}             % Least common multiple.
\DeclareMathOperator{\LM}{LM}               % Leading monomial.
\DeclareMathOperator{\LT}{LT}               % Leading term.
\DeclareMathOperator{\Mod}{mod}             % Modulus.
\DeclareMathOperator{\Mon}{Mon}             % Monomial.
\DeclareMathOperator{\multideg}{mutlideg}   % Multi-Degree (Graphs).
\DeclareMathOperator{\nul}{nul}             % Null space of operator.
\DeclareMathOperator{\Ord}{Ord}             % Ordinal of ordered set.
\DeclareMathOperator{\Prin}{Prin}           % Principal value.
\DeclareMathOperator{\proj}{proj}           % Projection.
\DeclareMathOperator{\Refl}{Refl}           % Reflection operator.
\DeclareMathOperator{\rk}{rk}               % Rank of operator.
\DeclareMathOperator{\sgn}{sgn}             % Sign of a number.
\DeclareMathOperator{\sinc}{sinc}           % Sinc function.
\DeclareMathOperator{\Span}{Span}           % Span of a set.
\DeclareMathOperator{\Spec}{Spec}           % Spectrum.
\DeclareMathOperator{\supp}{supp}           % Support
\DeclareMathOperator{\Tr}{Tr}               % Trace of matrix.
%--------------------Declared Math Symbols--------------------%
\DeclareMathSymbol{\minus}{\mathbin}{AMSa}{"39} % Unary minus sign.
%------------------------New Commands---------------------------%
\DeclarePairedDelimiter\norm{\lVert}{\rVert}
\DeclarePairedDelimiter\ceil{\lceil}{\rceil}
\DeclarePairedDelimiter\floor{\lfloor}{\rfloor}
\newcommand*\diff{\mathop{}\!\mathrm{d}}
\newcommand*\Diff[1]{\mathop{}\!\mathrm{d^#1}}
\renewcommand*{\glstextformat}[1]{\textcolor{RoyalBlue}{#1}}
\renewcommand{\glsnamefont}[1]{\textbf{#1}}
\renewcommand\labelitemii{$\circ$}
\renewcommand\thesubfigure{%
    \arabic{chapter}.\arabic{figure}.\arabic{subfigure}}
\addto\captionsenglish{\renewcommand{\figurename}{Fig.}}
\numberwithin{equation}{section}

\renewcommand{\vector}[1]{\boldsymbol{\mathrm{#1}}}

\newcommand{\uvector}[1]{\boldsymbol{\hat{\mathrm{#1}}}}
\newcommand{\topspace}[2][]{(#2,\tau_{#1})}
\newcommand{\measurespace}[2][]{(#2,\varSigma_{#1},\mu_{#1})}
\newcommand{\measurablespace}[2][]{(#2,\varSigma_{#1})}
\newcommand{\manifold}[2][]{(#2,\tau_{#1},\mathcal{A}_{#1})}
\newcommand{\tanspace}[2]{T_{#1}{#2}}
\newcommand{\cotanspace}[2]{T_{#1}^{*}{#2}}
\newcommand{\Ckspace}[3][\mathbb{R}]{C^{#2}(#3,#1)}
\newcommand{\funcspace}[2][\mathbb{R}]{\mathcal{F}(#2,#1)}
\newcommand{\smoothvecf}[1]{\mathfrak{X}(#1)}
\newcommand{\smoothonef}[1]{\mathfrak{X}^{*}(#1)}
\newcommand{\bracket}[2]{[#1,#2]}

%------------------------Book Command---------------------------%
\makeatletter
\renewcommand\@pnumwidth{1cm}
\newcounter{book}
\renewcommand\thebook{\@Roman\c@book}
\newcommand\book{%
    \if@openright
        \cleardoublepage
    \else
        \clearpage
    \fi
    \thispagestyle{plain}%
    \if@twocolumn
        \onecolumn
        \@tempswatrue
    \else
        \@tempswafalse
    \fi
    \null\vfil
    \secdef\@book\@sbook
}
\def\@book[#1]#2{%
    \refstepcounter{book}
    \addcontentsline{toc}{book}{\bookname\ \thebook:\hspace{1em}#1}
    \markboth{}{}
    {\centering
     \interlinepenalty\@M
     \normalfont
     \huge\bfseries\bookname\nobreakspace\thebook
     \par
     \vskip 20\p@
     \Huge\bfseries#2\par}%
    \@endbook}
\def\@sbook#1{%
    {\centering
     \interlinepenalty \@M
     \normalfont
     \Huge\bfseries#1\par}%
    \@endbook}
\def\@endbook{
    \vfil\newpage
        \if@twoside
            \if@openright
                \null
                \thispagestyle{empty}%
                \newpage
            \fi
        \fi
        \if@tempswa
            \twocolumn
        \fi
}
\newcommand*\l@book[2]{%
    \ifnum\c@tocdepth >-3\relax
        \addpenalty{-\@highpenalty}%
        \addvspace{2.25em\@plus\p@}%
        \setlength\@tempdima{3em}%
        \begingroup
            \parindent\z@\rightskip\@pnumwidth
            \parfillskip -\@pnumwidth
            {
                \leavevmode
                \Large\bfseries#1\hfill\hb@xt@\@pnumwidth{\hss#2}
            }
            \par
            \nobreak
            \global\@nobreaktrue
            \everypar{\global\@nobreakfalse\everypar{}}%
        \endgroup
    \fi}
\newcommand\bookname{Book}
\renewcommand{\thebook}{\texorpdfstring{\Numberstring{book}}{book}}
\providecommand*{\toclevel@book}{-2}
\makeatother
\titleformat{\part}[display]
    {\Large\bfseries}
    {\partname\nobreakspace\thepart}
    {0mm}
    {\Huge\bfseries}
\titlecontents{part}[0pt]
    {\large\bfseries}
    {\partname\ \thecontentslabel: \quad}
    {}
    {\hfill\contentspage}
\titlecontents{chapter}[0pt]
    {\bfseries}
    {\chaptername\ \thecontentslabel:\quad}
    {}
    {\hfill\contentspage}
\newglossarystyle{longpara}{%
    \setglossarystyle{long}%
    \renewenvironment{theglossary}{%
        \begin{longtable}[l]{{p{0.25\hsize}p{0.65\hsize}}}
    }{\end{longtable}}%
    \renewcommand{\glossentry}[2]{%
        \glstarget{##1}{\glossentryname{##1}}%
        &\glossentrydesc{##1}{~##2.}
        \tabularnewline%
        \tabularnewline
    }%
}
\newglossary[not-glg]{notation}{not-gls}{not-glo}{Notation}
\newcommand*{\newnotation}[4][]{%
    \newglossaryentry{#2}{type=notation, name={\textbf{#3}, },
                          text={#4}, description={#4},#1}%
}
%--------------------------LENGTHS------------------------------%
% Spacings for the Table of Contents.
\addtolength{\cftsecnumwidth}{1ex}
\addtolength{\cftsubsecindent}{1ex}
\addtolength{\cftsubsecnumwidth}{1ex}
\addtolength{\cftfignumwidth}{1ex}
\addtolength{\cfttabnumwidth}{1ex}

% Indent and paragraph spacing.
\setlength{\parindent}{0em}
\setlength{\parskip}{0em}
%---------------GLOSSARY------------%
\makeglossaries
\loadglsentries{../../glossary}
\loadglsentries{../../acronym}
%--------------Title Page-----------%
\begin{document}
\chapter{Elementary Algebra}
\section{North Shore Placement Exam}
\subsection{Order of Operations}
First we start with the fundamental properties of addition,
multiplication, and exponentiation.
\begin{properties}[Arithmetic Properties]
\label{property:North_Shore_Arithmetic_Properties}
\
\begin{enumerate}
    \item\label{property:North_Shore_Arithmetic_Properties_Com_Add}
    $a+b=b+a$\hfill [Commutativity of Addition]
    \item\label{property:north_shore_arithmetic_properties_assoc_add}
    $a+(b+c)=(a+b)+c$\hfill [Associativity of Addition]
    \item\label{property:north_shore_arithmetic_properties_comm_mult}
    $a\cdot b=b\cdot a$\hfill [Commutativity of Multiplication]
    \item \label{property:north_shore_arithmetic_properties_assoc_mult} $a\cdot (b\cdot c) = (a\cdot b)\cdot c$ \hfill [Associativity of Multiplication]
    \item \label{property:north_shore_arithmetic_properties_add_idenity} $a+0 = a$ \hfill [Identity Property of Addition]
    \item \label{property:north_shore_arithmetic_properties_mult_identity} $a\cdot 1 = a$ \hfill [Identity Property of Multiplication]
    \item \label{property:north_shore_arithmetic_properties_add_inverse} $a + (-a) = 0$ \hfill [Inverse Property of Addition]
    \item \label{property:north_shore_arithmetic_properties_mult_inverse} If $a\ne 0$, then $\frac{a}{a} = 1$ \hfill [Inverse Property of Multiplication]
    \item \label{property:north_shore_arithmetic_properties_distributive_property} $a\cdot (b+c) = a\cdot b + a\cdot c$ \hfill [Distributive Property of Multiplication over Addition]
\end{enumerate}
\end{properties}
\begin{properties}[Properties of Exponents]
\
\label{property:North_Shore_Exponent_Rules}
\begin{enumerate}
    \item \label{property:north_shore_distributive_property_of_expo} $(x\cdot y)^n = x^n \cdot y^n$ \hfill [Distributive Property of Exponents]
    \item \label{property:north_shore_inverse_proerty_of_expo} $x^{-n} = \frac{1}{x^n}$ \hfill [Inverse Property of Exponents]
    \item \label{property:north_shore_power_property_of_expo} $(x^n)^m = x^{n\cdot m}$ \hfill [Power Property of Exponents]
    \item \label{property:north_shore_product_property_of_expo} $x^{n} x^{m} = x^{n+m}$ \hfill [Multiplicative Property of Exponents]
\end{enumerate}
\end{properties}
The order of operations \gls{pemdas} tells one how to simplify expressions.
\begin{enumerate}
\label{North_Shore_PEMDAS}
    \item \textbf{P}arenthesis.
    \item \textbf{E}xponents.
    \item \textbf{M}ultiplication or \textbf{D}ivision, in the order they appear from left to right.
    \item \textbf{A}ddition or \textbf{S}ubtraction, in the order they appear from left to right.
\end{enumerate}
\begin{example}
\
\begin{enumerate}
    \item $2\cdot 3+4\cdot 5\overset{\textrm{\tiny{Multiplication}}}{=}6+20\overset{\textrm{\tiny{Addition}}}{=}26$
    \item $2\cdot 3+4 \overset{\textrm{\tiny{Multiplication}}}{=} 6+4\overset{\textrm{\tiny{Addition}}}{=}\boxed{10}$
    \item $3\cdot 3 + 2^4 \overset{\textrm{\tiny{Exponents}}}{=} 3\cdot 3 + 16 \overset{\textrm{\tiny{Multiplication}}}{=} 9+16 \overset{\textrm{\tiny{Addition}}}{=} \boxed{25}$
    \item $(4+1)^2-17\cdot 3\overset{\textrm{\tiny{Parenthesis}}}{=}5^{2}-17\cdot 3\overset{\textrm{\tiny{Exponents}}}{=}25-17\cdot 3 \overset{\textrm{\tiny{Multiplication}}}{=} 25-51 \overset{\textrm{\tiny{Subtraction}}}{=}\boxed{-26}$
    \item $(1+1)^{(2+3)}\cdot 5-2\cdot 3\overset{\textrm{\tiny{Parenthesis}}}{=}2^{5}\cdot 5-2\cdot 3\overset{\textrm{\tiny{Exponents}}}{=}32\cdot 5-2\cdot 3\overset{\textrm{\tiny{Multiplication}}}{=} 160-6\overset{\textrm{\tiny{Subtraction}}}{=}\boxed{154}$
    \item $(2+2)-(16-11)^{(1+1)}\overset{\textrm{\tiny{Parenthesis}}}{=}4-5^{2}\overset{\textrm{\tiny{Exponents}}}{=}4-25\overset{\textrm{\tiny{Subtraction}}}{=}\boxed{-21}$
\end{enumerate}
\end{example}
\begin{remark}
\label{remark:North_Shore_Radicals_Def} Many problems involve radicals. These are defined by $\sqrt{x}=x^{\frac{1}{2}}$ and $\sqrt[n]{x}=x^{\frac{1}{n}}$
\end{remark}
\newpage
\begin{example}
\
\begin{enumerate}
    \begin{multicols}{4}
        \item $\sqrt{x} = x^{\frac{1}{2}}$
        \item $\sqrt[3]{x} = x^{\frac{1}{3}}$
        \item $\sqrt[5]{x} = x^{\frac{1}{5}}$
        \item $\sqrt[27]{x} = x^{\frac{1}{27}}$
        \end{multicols}
    \end{enumerate}
\end{example}
\begin{remark}
\label{remark:North_Shore_Exponent_and_Radical_Def}
Radicals can be combined with exponents. These are defined by $\sqrt[n]{x^m} = x^{\frac{m}{n}}$
\end{remark}
\begin{example}
\
\begin{enumerate}
    \begin{multicols}{4}
        \item $\sqrt{x^3}=x^{\frac{3}{2}}$
        \item $\sqrt[3]{x^2}=x^{\frac{2}{3}}$
        \item $\sqrt[15]{x^3}=x^{\frac{1}{5}}$
        \item $\sqrt[3]{x^3}=x$
    \end{multicols}
\end{enumerate}
\end{example}
\begin{theorem}
\label{thm:North_Shore_square_root_of_product_of_positive_reals}
If $a$ and $b$ are \textbf{positive} real numbers, then $\sqrt{a\cdot b}=\sqrt{a}\cdot\sqrt{b}$
\end{theorem}
\begin{example}
\
\begin{enumerate}
    \begin{multicols}{4}
    \item $\sqrt{6}=\sqrt{2}\cdot\sqrt{3}$
    \item $\sqrt{8}=2\sqrt{2}$
    \item $\sqrt{1,\!000}=10\sqrt{10}$
    \item $\sqrt{50}=5\sqrt{2}$
    \end{multicols}
\end{enumerate}
\end{example}
\begin{theorem}
\label{thm:North_Shore_square_root_of_negative_real}
If $x$ is \textbf{positive}, and $i$ is the imaginary unit $(i^{2}=-1)$, then $\sqrt{-x}=i\sqrt{x}$
\end{theorem}
\begin{example}
\
\begin{enumerate}
    \begin{multicols}{4}
    \item $\sqrt{-7}=i\sqrt{7}$
    \item $\sqrt{-9}=3i$
    \item $\sqrt{-4}=2i$
    \item $\sqrt{-2}=i\sqrt{2}$
    \end{multicols}
\end{enumerate}
\end{example}
\begin{theorem}
\label{thm:North_Shore_odd_nth_root_of_real}
If $n$ is an \textbf{odd} integer, and $x$ is a \textbf{real} number, then $\sqrt[n]{-x}=-\sqrt[n]{x}$
\end{theorem}
\begin{example}
\
\begin{enumerate}
    \begin{multicols}{4}
    \item $\sqrt[3]{-7}=-\sqrt[3]{7}$
    \item $\sqrt[17]{-1}=-1$
    \item $\sqrt[5]{-9}=-\sqrt[5]{9}$
    \item $\sqrt[3]{-27}=-3$
    \end{multicols}
\end{enumerate}
\end{example}
\begin{remark}
\label{remark:North_Shore_non_zero_raised_to_zero}
For all non-zero  real numbers, $x^{0}=1$. $0^0$ is left undefined.
\end{remark}
\begin{example}
\
\begin{enumerate}
    \begin{multicols}{4}
    \item $\pi^{0}=1$
    \item $(-5)^{0}=1$
    \item $(\frac{1}{2})^{0}=1$
    \item $0^{0}$ is undefined.
    \end{multicols}
\end{enumerate}
\end{example}
\begin{definition}
\label{definiiton:North_Shore_Absolute_Value_Def}
The absolute value of a real number $x$ is defined as $|x|=\begin{cases} x, & x\geq 0\\-x, & x<0\end{cases}$
\end{definition}
\begin{example}
\
\begin{enumerate}
    \begin{multicols}{5}
        \item[1.] $|3|=3$
        \item[6.] $|24|=24$
        \item[2.] $|-3|=3$
        \item[7.] $|-24|=24$
        \item[3.] $|\pi|=\pi$
        \item[8.] $|\frac{1}{2}|=\frac{1}{2}$
        \item[4.] $|-\pi|=\pi$
        \item[9.] $|-\frac{1}{2}|=\frac{1}{2}$
        \item[5.] $|0|=0$
        \item[10.] $|-1|=1$
    \end{multicols}
\end{enumerate}
\end{example}
\begin{remark}
\label{remark:North_Shore_Rational_Expressions}
Expressions like $\frac{a+b}{c+d}$ should be treated as equivalent to $(a+b)\div (c+d)$
\end{remark}
\subsubsection{Problems}
\begin{problem}
Simplify $3^{2}+5-\sqrt{4}+4^{0}$
\end{problem}
\begin{proof}[Solution]
\begin{flalign*}
3^{2}+5-\sqrt{4}+4^{0}&=3^{2}+5-4^{\frac{1}{2}}+4^{0} & \tag{Remark~\ref{remark:North_Shore_Radicals_Def}}\\[-0.5ex]
&=3^{2}+5-4^{\frac{1}{2}}+1&\tag{Remark~\ref{remark:North_Shore_non_zero_raised_to_zero}}\\
&=9+5-2+1&\tag{Exponentiation}\\
&=14-2+1&\tag{Addition}\\
&=12+1&\tag{Subtraction}\\
&=\boxed{13}&\tag{Addition}
\end{flalign*}
\end{proof}
\begin{problem}
Simplify $(5+1)(4-2)-3$
\end{problem}
\begin{proof}[Solution]
\begin{flalign*}
    (5+1)(4-2)-3&=6\cdot 2-3&\tag{Parenthesis}\\
    &=12-3&\tag{Multiplication}\\
    &=\boxed{9}&\tag{Subtraction}
\end{flalign*}
\end{proof}
\begin{problem}
Simplify $3\cdot 7^{2}$
\end{problem}
\begin{proof}[Solution]
\begin{flalign*}
    3\cdot 7^{2}&=3\cdot 49&\tag{Exponentiation}\\
    &=\boxed{147}&\tag{Multiplication}
\end{flalign*}
\end{proof}
\begin{problem}
Simplify $2\cdot (7+3)^{2}$
\end{problem}
\begin{proof}[Solution]
\begin{flalign*}
    2\cdot(7+3)^{2}&=2\cdot 10^{2}&\tag{Parenthesis}\\
    &=2\cdot 100&\tag{Exponentiation}\\
    &=\boxed{200}&\tag{Multiplication}
\end{flalign*}
\end{proof}
\begin{problem}
Simplify $49\div 7-2\cdot 2$
\end{problem}
\begin{proof}[Solution]
\begin{flalign*}
    49\div 7-2\cdot 2&=7-2\cdot 2&\tag{Division}\\
    &=7-4&\tag{Multiplication}\\
    &=\boxed{3}&\tag{Subtraction}
\end{flalign*}
\end{proof}
\begin{problem}
Simplify $9\div 3\cdot 5-8\div 2+27$
\end{problem}
\begin{proof}[Solution]
\begin{flalign*}
    9\div 3\cdot 5-8\div 2+27&=3\cdot 5-4+27&\tag{Multiplication/Division from Left to Right}\\
    &=15-4+27&\tag{Multiplication}\\
    &=11+27&\tag{Subtraction}\\
    &=\boxed{38}&\tag{Addition}
\end{flalign*}
\end{proof}
\begin{problem}
Simplify $3+2(5)-|-7|$
\end{problem}
\begin{proof}[Solution]
\begin{flalign*}
    3+2(5)-|-7|&=3+2(5)-7&\tag{Def.~\ref{definiiton:North_Shore_Absolute_Value_Def}}\\
    &=3+10-7&\tag{Multiplication}\\
    &=13-7&\tag{Addition}\\
    &=\boxed{6}&\tag{Addition}
\end{flalign*}
\end{proof}
\begin{problem}
Simplify $\frac{5\cdot 5-4(4)}{2^{2}-1}$
\end{problem}
\begin{proof}[Solution]
\begin{flalign*}
    \tfrac{5\cdot 5-4(4)}{2^{2}-1}&=(5\cdot 5-4(4))\div (2^{2}-1)&\tag{Remark~\ref{remark:North_Shore_Rational_Expressions}}\\
    &=(5\cdot 5-4(4))\div (4-1)&\tag{Exponentiation Inside Parenthesis}\\
    &=(25-16)\div (4-1)&\tag{Multiplication Inside Parenthesis}\\
    &=9\div 3&\tag{Parenthesis}\\
    &=\boxed{3}&\tag{Division}
\end{flalign*}
\end{proof}
\begin{problem}
Simplify $\frac{4^{2}-5^{2}}{(4-5)^{2}}$
\end{problem}
\begin{proof}[Solution]
\begin{flalign*}
    \tfrac{4^{2}-5^2}{(4-5)^{2}}&=(4^{2}-5^{2})\div ((4-5)^2)&\tag{Remark~\ref{remark:North_Shore_Rational_Expressions}}\\
    &=(4^{2}-5^{2})\div ((-1)^2)&\tag{Parenthesis}\\
    &=(16-25)\div (1)&\tag{Exponentiation Inside Parenthesis}\\
    &=-9\div 1&\tag{Parenthesis}\\
    &=\boxed{-9}&\tag{Division}
\end{flalign*}
\end{proof}
\begin{problem}
Simplify $-5^{2}$
\end{problem}
\begin{proof}[Solution]
\begin{flalign*}
    -5^{2}&=\boxed{-25}&\tag{Exponentiation}
\end{flalign*}
\end{proof}
\begin{problem}
Simplify $48\div 2(3+9)$
\end{problem}
\begin{proof}[Solution]
\begin{flalign*}
    48\div 2(3+9)&=48\div 2(12)&\tag{Parenthesis}\\
    &=24(12)&\tag{Division}\\
    &=\boxed{288}&\tag{Multiplication}
\end{flalign*}
\end{proof}
\begin{problem}
Simplify $6\cdot (3+2)^{2}+14$
\end{problem}
\begin{proof}[Solution]
\begin{flalign*}
    6\cdot(3+2)^{2}+14&=6\cdot 5^{2}+14&\tag{Parenthesis}\\
    &=6\cdot 25+14&\tag{Exponentiation}\\
    &=150+14&\tag{Multiplication}\\
    &=\boxed{164}
\end{flalign*}
\end{proof}
\begin{problem}
Simplify $(5+1)^{(3-2)}$
\end{problem}
\begin{proof}[Solution]
\begin{flalign*}
    (5+1)^{(3-2)}&=6^{1}&\tag{Parenthesis}\\
    &=\boxed{6}&\tag{Exponentiation}
\end{flalign*}
\end{proof}
\subsection{Scientific Notation}
Non-zero real numbers can be written as $r\times 10^{n}$, where $1\leq |r|<10$ and $n$ is an integer.
\begin{example}
\
\begin{enumerate}
\begin{multicols}{3}
    \item[1.] $12,\! 345=1.2345\times 10^{4}$
    \item[4.] $10=1\times 10^{1}$
    \item[2.] $0.01=1\times 10^{-2}$
    \item[5.] $124=1.24 \times 10^{2}$
    \item[3.] $36.24=3.624\times 10^{1}$
    \item[6.] $0.000314=3.14\times 10^{-4}$
\end{multicols}
\end{enumerate}
\end{example}
\begin{remark}
Several constants in chemistry/physics are written in scientific notation.
\begin{enumerate}
    \item Planck's Constant: $h=6.626\times 10^{-34}\textrm{J}\cdot\textrm{s}$
    \item Universal Gravitation Constant: $G=6.67\times 10^{-11}\textrm{Nm}^{2}\textrm{kg}^{-2}$
    \item Avogadro's Number: $N_{A}=6.0221\times 10^{23}\textrm{mol}^{-1}$
    \item Speed of Light: $c=2.998\times 10^{8}\textrm{ms}^{-1}$
\end{enumerate}
\end{remark}
\subsubsection{Problems}
\begin{problem}
Write the following in scientific notation:
\begin{enumerate}
    \begin{multicols}{4}
    \item $350,\!000,\!000$
    \item $120,\!500,\!000,\!000$
    \item $0.0000000523$
    \item $10.01$
    \end{multicols}
\end{enumerate}
\end{problem}
\begin{proof}[Solution]
\
\begin{enumerate}
    \begin{multicols}{4}
    \item $3.5\times 10^8$
    \item $5.23\times 10^{-8}$
    \item $1.205\times 10^{11}$
    \item $1.001\times 10^{1}$
    \end{multicols}
\end{enumerate}
\end{proof}
\begin{problem}
Write in expanded form:
\begin{enumerate}
    \begin{multicols}{3}
    \item $6.02\times 10^{15}$
    \item $3.0\times 10^{8}$
    \item $1.819\times 10^{-9}$
    \end{multicols}
\end{enumerate}
\end{problem}
\begin{proof}[Solution]
\
\begin{enumerate}
\begin{multicols}{3}
    \item $6,\!020,\!000,\!000,\!000,\!000$
    \item $300,\!000,\!000$
    \item $0.000000001819$
\end{multicols}
\end{enumerate}
\end{proof}
\begin{problem}
Simplify:
\begin{enumerate}
    \begin{multicols}{4}
    \item $(3\times 10^{3})(5\times 10^{6})$
    \item $\frac{6\times 10^{9}}{3\times 10^{4}}$
    \item $(3\times 10^{-4})^{2}$
    \item $\frac{(3.2\times 10^{5})(2\times 10^{-3})}{2\times 10^{-5}}$
    \end{multicols}
\end{enumerate}
\end{problem}
\begin{proof}[Solution]
\
\begin{enumerate}
    \item $(3\times 10^{3})(5\times 10^{6})=3\times 5\times 10^{3}\times 10^{6}=15\times 10^{6+3}=15\times 10^{9}=\boxed{1.5\times 10^{10}}$
    \item $\frac{6\times 10^{9}}{3\times 10^{4}}=\frac{6}{3}\times\frac{10^{9}}{10^{4}}=2\times 10^{9-4}=\boxed{2\times 10^{5}}$
    \item $(3\times 10^{-4})^{2}=3^{2}\times (10^{-4})^{2}=\boxed{9\times 10^{-8}}$
    \item $\frac{(3.2\times 10^{5})(2\times 10^{-2})}{2\times 10^{-5}}=3.2\times 10^{5}\times\frac{2\times 10^{-3}}{2\times 10^{-5}}=3.2\times 10^{5}\times 10^{2}=\boxed{3.2\times 10^{7}}$
\end{enumerate}
\end{proof}
\subsection{Substitution}
\subsubsection{Problems}
\begin{problem}
Solve $xyz-4z$ for $x=3,y=-4,z=2$.
\end{problem}
\begin{proof}[Solution]
\begin{flalign*}
    xyz-4z\big|_{x=3,y=-4,z=2}&=(3)(-4)(2)-4(2)&\tag{Substitution}\\
    &=-24-8&\tag{Multiplication}\\
    &=\boxed{-32}&\tag{Subtraction}
\end{flalign*}
\end{proof}
\begin{problem}
Solve $2x-y$ for $x=3,y=-4,z=2$.
\end{problem}
\begin{proof}[Solution]
\begin{flalign*}
    2x-y\big|_{x=3,y=-4,z=2}&=2(3)-(-4)&\tag{Substitution}\\
    &=6+4&\tag{Multiplication}\\
    &=\boxed{10}&\tag{Addition}
\end{flalign*}
\end{proof}
\newpage
\begin{problem}
Solve $x(y-3z)$ for $x=3,y=-4,z=2$.
\end{problem}
\begin{proof}[Solution]
\begin{flalign*}
    x(y-3z)\big|_{x=3,y=-4,z=2}&=(3)((-4)-3(2))&\tag{Substitution}\\
    &=3(-10)&\tag{Parenthesis}\\
    &=\boxed{-30}&\tag{Multiplication}
\end{flalign*}
\end{proof}
\begin{problem}
Solve $\frac{5x-z}{xy}$ for $x=3,y=-4,z=2$.
\end{problem}
\begin{proof}[Solution]
\begin{flalign*}
    \tfrac{5x-z}{xy}\big|_{x=3,y=-4,z=2}&=\tfrac{5(3)-(2)}{(3)(-4)}&\tag{Substitution}\\
    &=\tfrac{15-2}{-12}&\tag{Multiplication}\\
    &=\boxed{-\tfrac{13}{12}}&\tag{Subtraction}
\end{flalign*}
\end{proof}
\begin{problem}
Solve $3y^{2}-2x+4z$ for $x=3,y=-4,z=2$.
\end{problem}
\begin{proof}[Solution]
\begin{flalign*}
    3y^{2}-2x+4z\big|_{x=3,y=-4,z=2}&=3(-4)^{2}-2(3)+4(2)&\tag{Substitution}\\
    &=3(16)-2(3)+4(2)&\tag{Exponentiation}\\
    &=48-6+8&\tag{Multiplication}\\
    &=\boxed{50}&\tag{Addition/Subtraction}
\end{flalign*}
\end{proof}
\begin{problem}
Solve $x+y+z$ for $x=1,y=2,z=3$.
\end{problem}
\begin{proof}[Solution]
\begin{flalign*}
    x+y+z\big|_{x=1,y=2,z=3}&=(1)+(2)+(3)&\tag{Substitution}\\
    &=\boxed{6}&\tag{Addition}
\end{flalign*}
\end{proof}
\begin{problem}
Solve $(x+1)(y-2)$ for $x=1,y=2,z=3$
\end{problem}
\begin{proof}[Solution]
\begin{flalign*}
    (x+1)(y-2)\big|_{x=1,y=2,z=3}&=((1)+1)((2)-2)&\tag{Substitution}\\
    &=2\cdot 0&\tag{Parenthesis}\\
    &=\boxed{0}&\tag{Multipliation}
\end{flalign*}
\end{proof}
\begin{problem}
Solve $x^{2}+y^{2}-z^{2}$ for $x=1,y=2,z=3$.
\end{problem}
\begin{proof}[Solution]
\begin{flalign*}
    x^{2}+y^{2}-z^{2}\big|_{x=1,y=2,z=3}&=(1)^{2}+(2)^{2}-(3)^{2}&\tag{Substitution}\\
    &=1+4-9&\tag{Exponentiation}\\
    &=\boxed{-4}&\tag{Addition/Subtraction}
\end{flalign*}
\end{proof}
\newpage
\begin{problem}
Solve $\frac{z+1}{y(x+1)}$ for $x=1,y=2,z=3$.
\end{problem}
\begin{proof}[Solution]
\begin{flalign*}
    \tfrac{z+1}{y(x+1)}\big|_{x=1,y=2,z=3}&=\tfrac{(3)+1}{(2)((1)+1)}&\tag{Substitution}\\
    &=\tfrac{4}{2(2)}&\tag{Parenthesis}\\
    &=\tfrac{4}{4}&\tag{Multiplication}\\
    &=\boxed{1}&\tag{Division}
\end{flalign*}
\end{proof}
\begin{problem}
Solve $xy+xz+yz$ for $x=0,y=3,z=-1$.
\end{problem}
\begin{proof}[Solution]
\begin{flalign*}
    xy+xz+yz\big|_{x=0,y=3,z=-1}&=(0)(3)+(0)(-1)+(3)(-1)&\tag{Substitution}\\
    &=0+0+(-3)&\tag{Multiplication}\\
    &=\boxed{-3}&\tag{Addition}
\end{flalign*}
\end{proof}
\begin{problem}
Solve $y^{x}+z^{y}$ for $x=0,y=3,z=-1$.
\end{problem}
\begin{proof}[Solution]
\begin{flalign*}
    y^{x}+z^{y}\big|_{x=0,y=3,z=-1}&=(3)^{(0)}+(-1)^{(3)}&\tag{Substitution}\\
    &=1+(-1)&\tag{Exponentiation}\\
    &=\boxed{0}&\tag{Addition}
\end{flalign*}
\end{proof}
\begin{problem}
Solve $\frac{y+z}{x}$ for $x=0,y=3,z=-1$.
\end{problem}
\begin{proof}[Solution]
\begin{flalign*}
    \tfrac{y+z}{x}\big|_{x=0,y=3,z=-1}&=\tfrac{(3)+(-1)}{(0)}&\tag{Substitution}\\
    &=\boxed{\textrm{Undefined}}&\tag{Division by Zero}
\end{flalign*}
\end{proof}
\begin{problem}
Solve $xy^{z^y}+y$ for $x=0,y=3,z=-1$.
\end{problem}
\begin{proof}[Solution]
\begin{flalign*}
    xy^{z^{y}}+y\big|_{x=0,y=3,z=-1}&=(0)(3)^{(-1)^{(3)}}+3&\tag{Substitution}\\
    &=0\cdot 3^{-1}+3&\tag{Exponentiation}\\
    &=0+3&\tag{Multiplication}\\
    &=\boxed{3}&\tag{Addition}
\end{flalign*}
\end{proof}
\newpage
\subsection{Linear Equations in One Variable}
\subsubsection{Problems}
\begin{problem}
Solve for $x$: $6x-48=6$
\end{problem}
\begin{proof}[Solution]
\begin{flalign*}
    6x-48&=6\\
    \Rightarrow 6x&=54&\tag{Add $48$ to Both Sides}\\
    \Rightarrow x&=\tfrac{54}{6}&\tag{Divide Both Sides by $6$}\\
    \Rightarrow x&=\boxed{9}&\tag{Division}
\end{flalign*}
\end{proof}
\begin{problem}
Solve for $x$: $\frac{2}{3}x-5=x-3$
\end{problem}
\begin{proof}[Solution]
\begin{flalign*}
    \tfrac{2}{3}x-5&=x-3\\
    \Rightarrow 2x-15&=3x-9&\tag{Multiply Both Sides by $3$}\\
    \Rightarrow -x-15&=-9&\tag{Subtract $3x$ from Both Sides}\\
    \Rightarrow -x&=6&\tag{Add $15$ to Both Sides}\\
    \Rightarrow x&=\boxed{-6}&\tag{Multiply Both Sides by $-1$}
\end{flalign*}
\end{proof}
\begin{problem}
Solve for $x$: $50-x-(3x+2)=0$
\end{problem}
\begin{proof}[Solution]
\begin{flalign*}
    50-x-(3x+2)&=0\\
    \Rightarrow 50-x-3x-2&=0&\tag{Distribute the Minus Sign}\\
    \Rightarrow 48-4x&=0&\tag{Simplify the Left-Hand Side}\\
    \Rightarrow 4x&=48&\tag{Add $4x$ to Both Sides}\\
    \Rightarrow x&=\tfrac{48}{4}&\tag{Divide Both Sides by $4$}\\
    \Rightarrow x&=\boxed{12}&\tag{Division}
\end{flalign*}
\end{proof}
\begin{problem}
Solve for $x$: $8-4(x-1)=2+3(4-x)$
\end{problem}
\begin{proof}[Solution]
\begin{flalign*}
    8-4(x-1)&=2+3(4-x)\\
    \Rightarrow 8-4x+4&=2+12-3x&\tag{Simplify Both Sides}\\
    \Rightarrow 12-4x&=14-3x&\tag{Simplify Both Sides}\\
    \Rightarrow 12&=14+x&\tag{Add $4x$ to Both Sides}\\
    \Rightarrow x&=\boxed{-2}&\tag{Subtract $14$ from Both Sides}
\end{flalign*}
\end{proof}
\begin{problem}
Solve $x+1=1$ for $x$.
\end{problem}
\begin{proof}[Solution]
\begin{flalign*}
    x+1&=1\\
    \Rightarrow x&=\boxed{0}&\tag{Subtract $1$ from Both Sides}
\end{flalign*}
\end{proof}
\begin{problem}
Solve $4(x-1)+x=0$ for $x$.
\end{problem}
\begin{proof}[Solution]
\begin{flalign*}
    4(x-1)+x&=0\\
    \Rightarrow 5x-4&=0&\tag{Simplify the Left-Hand Side}\\
    \Rightarrow 5x&=4&\tag{Add $4$ to Both Sides}\\
    \Rightarrow x&=\boxed{\tfrac{4}{5}}&\tag{Divide Both Sides by $5$}
\end{flalign*}
\end{proof}
\begin{problem}
Solve $1-x+10=7$ for $x$.
\end{problem}
\begin{proof}[Solution]
\begin{flalign*}
    1-x+10&=7\\
    \Rightarrow 11-x&=7&\tag{Simplify the Left-Hand Side}\\
    \Rightarrow 11&=7+x&\tag{Add $x$ to both sides}\\
    \Rightarrow x&=\boxed{4}&\tag{Subtract $7$ from Both Sides}
\end{flalign*}
\end{proof}
\subsection{Formulas}
\subsubsection{Problems}
\begin{problem}
Solve $PV=nRT$ for $T$.
\end{problem}
\begin{proof}[Solution]
This is the Ideal Gas Law from chemistry: $\boxed{T=\tfrac{PV}{nR}}$
\end{proof}
\begin{problem}
Solve $y=3x+2$ for $x$.
\end{problem}
\begin{proof}[Solution]
$y=3x+2\Rightarrow y-2=3x\Rightarrow\boxed{x=\tfrac{y-2}{3}}$
\end{proof}
\begin{problem}
Solve $C=2\pi r$ for $r$.
\end{problem}
\begin{proof}[Solution]
This is the formula for the circumference $C$ of a circle of radius $r$: $\boxed{r=\tfrac{C}{2\pi}}$
\end{proof}
\begin{problem}
Solve $\tfrac{x}{2}+\tfrac{y}{5}=1$ for $y$.
\end{problem}
\begin{proof}[Solution]
$\tfrac{x}{2}+\tfrac{y}{5}=1\Rightarrow=\tfrac{y}{5}=1-\tfrac{x}{2}\Rightarrow\boxed{y=5-\tfrac{5}{2}x}$.
\end{proof}
\begin{problem}
Solve $y=hx+4x$ for $x$.
\end{problem}
\begin{proof}[Solution]
$y=hx+4x\Rightarrow x(h+4)=y\Rightarrow\boxed{x=\tfrac{y}{h+4}}$
\end{proof}
\subsection{Word Problems}
\subsubsection{Problems}
\begin{problem}
$y$ is $5$ more than twice that of $x$, their sum is $35$. Find $x$ and $y$.
\end{problem}
\begin{proof}[Solution]
We have $y=5+2x$ and $x+y=35$. Substituting $y$, we get $x+2x+5=35\Rightarrow 3x+5=35\Rightarrow 3x=30\Rightarrow\boxed{x=10}$. But $y=2+2x=5+2(10)\Rightarrow\boxed{y=25}$
\end{proof}
\begin{problem}
Ms. Jones invested $\$18,\!000$ in two accounts, one pays $6\%$ and the other $8\%$. Her total interest was $\$1,\!290$. How much did she have in each account?
\end{problem}
\begin{proof}[Solution]
Let $x$ and $y$ be the amounts in the $6\%$ and $8\%$ accounts, respectively. Then $x+y=\$18,\!000$ and $\frac{6}{100}x+\frac{8}{100}y=\$1,\!290$. Solving the first equation we get $y=\$18,\!000-x$. Substituting we have: $\frac{6}{100}x+\frac{8}{100}(\$18,\!000-x)=\$1,\!290\Rightarrow \$1440-\frac{2}{100}x=\$1,\!290\Rightarrow\frac{2}{100}x=\$150\Rightarrow\boxed{x=\$7,\!500}$ But $y=\$18,\!000-x\Rightarrow y=\$18,\!000-\$7,\!500\Rightarrow\boxed{y=\$10,\!500}$
\end{proof}
\begin{problem}
How many liters of $40\%$ and $16\%$ solution are needed to obtain $20$ liters of $22\%$ solution?
\end{problem}
\begin{proof}[Solution]
Let $x$ and $y$ be the number of liters of $40\%$ and $16\%$ solution, respectively. Then $x+y=20$, and $\frac{40}{100}x+\frac{16}{100}y=\frac{22}{100}20\Rightarrow 40x+16y=440$. Solving the first equation, we get $y=20-x$. Substituting, we have: $40x+16(20-x)=440\Rightarrow 24x+320=440\Rightarrow 24x=120\Rightarrow x=\frac{120}{24}\Rightarrow\boxed{x=5}$. But $y=20-x=20-5\Rightarrow \boxed{y=15}$ so there are $\boxed{5\textrm{ liters of }40\%}$ and $\boxed{15\textrm{ liters of }16\%}$ solution.
\end{proof}
\begin{problem}
Sheila bought burgers and fries for her children and some friends. The burgers cost $\$2.05$ each and the fries are $\$0.85$ each. She bought a total of $14$ items for a total cost of $\$19.10$. How many of each did she buy?
\end{problem}
\begin{proof}[Solution]
Let $x$ and $y$ be the number of burgers and fries, respectively. Then $2.05x+0.85y=19.10$, and $x+y=14$. Solving the second equation, we get $y=14-x$. Substitute this into the second equation to $2.05x+0.85(14-x)=19.10\Rightarrow 1.20x+11.90=19.10\Rightarrow 1.20x=7.20\Rightarrow\boxed{x=6}$ But $y=14-x\Rightarrow\boxed{y=8}$ Sheila bought $\boxed{6\textrm{ burgers}}$ and $\boxed{8\textrm{ fries}}$
\end{proof}
\subsection{Inequalities}
There are three main rules for dealing with inequalities:
\begin{properties}\label{property:north_shore_properties_of_inequalities}
\
\begin{enumerate}
    \item \label{property:north_shore_additive_property_inequals}If $c$ is real and $a<b$, then $a+c<b+c$\hfill [Additive Property of Inequalities]
    \item \label{property:north_shore_multiplicative_property_inequals}If $c$ is $\mathbf{positive}$ and $a<b$, then $ac < bc$\hfill [Multiplicative Property of Inequalities]
    \item \label{property:north_shore_inverse_property_inequals}If $c$ is $\mathbf{negative}$ and $a<b$, then $bc<ac$\hfill [Inverse Property of Inequalities]
\end{enumerate}
\end{properties}
\subsubsection{Problems}
\begin{problem}
Solve for $x$: $2x-7\geq 3$
\end{problem}
\begin{proof}[Solution]
    \begin{flalign*}
        2x-7&\geq 3\\
        \Rightarrow 2x&\geq 10&\tag{Add $7$ to Both Sides, property~\ref{property:north_shore_properties_of_inequalities} part~\ref{property:north_shore_additive_property_inequals}}\\
        \Rightarrow x&\geq\tfrac{10}{2}&\tag{Divide Both Sides by $2$, property~\ref{property:north_shore_properties_of_inequalities} part~\ref{property:north_shore_multiplicative_property_inequals}}\\
        \Rightarrow x&\geq 5&\tag{Division}
    \end{flalign*}
\end{proof}
\begin{problem}
Solve for $x$: $-5(2x+3)<2x-3$
\end{problem}
\begin{proof}[Solution]
\begin{flalign*}
-5(2x+3)&<2x-3\\
\Rightarrow -10x-15&<2x-3&\tag{Simplify the Left-Hand Side}\\
\Rightarrow -15+3&<2x+10x&\tag{Add $10x+3$ to Both Sides, property~\ref{property:north_shore_properties_of_inequalities} part~\ref{property:north_shore_additive_property_inequals}}\\
\Rightarrow -12&<12x&\tag{Simplify Both Sides}\\
\Rightarrow x&>-1&\tag{Divide Both Sides by $12$, property~\ref{property:north_shore_properties_of_inequalities} part~\ref{property:north_shore_multiplicative_property_inequals}}
\end{flalign*}
\end{proof}
\begin{problem}
Solve for $x$: $3(x-4)-(x+1)\leq -12$
\end{problem}
\begin{proof}[Solution]
    \begin{flalign*}
        3(x-4)-(x+1)&\leq -12\\
        \Rightarrow 3x-12-x-1&\leq -12&\tag{Simplify the Left-Hand Side}\\
        \Rightarrow 2x-13&\leq -12&\tag{Simplify the Left-Hand Side}\\
        \Rightarrow 2x&\leq 1&\tag{Add $13$ to Both Sides, property~\ref{property:north_shore_properties_of_inequalities} part~\ref{property:north_shore_additive_property_inequals}}\\
        \Rightarrow x&\leq\tfrac{1}{2}&\tag{Divide Both Sides by $2$, property~\ref{property:north_shore_properties_of_inequalities} part~\ref{property:north_shore_multiplicative_property_inequals}}
    \end{flalign*}
\end{proof}
\subsection{Exponents and Polynomials}
The two main rules for problem with polynomials and exponents are:
\begin{align}
    (a_{1}x^{2}+b_{1}x+c_{1})+(a_{2}x^{2}+b_{2}x+c_{2})&=x^{2}(a_{1}+a_{2})+x(b_{1}+b_{2})+(c_{1}+c_{2})\label{equation:north_shore_additive_polynomial_property}\\
    (ax+b)(cx+d)&=acx^{2}+(ad+bc)x+bd\label{equation:north_shore_multiplicative_polynomial_property}
\end{align}
\begin{remark}
Eqn.~\ref{equation:north_shore_additive_polynomial_property} says that the coefficients of like terms can be added together to simplify the expression, and eqn.~\ref{equation:north_shore_multiplicative_polynomial_property} is often called the \gls{foil} rule.
\end{remark}
\begin{remark}
\label{remark:north_shore_square_of_a_sum_foil}
A special case of \gls{foil}, we can write $(a+b)^{2}=a^{2}+ab+ba+b^{2}=a^{2}+2ab+b^{2}$
\begin{equation}
\label{equation:North_Shore_square_of_a_sum}
    (a+b)^{2}=a^{2}+2ab+b^{2}
\end{equation}
\end{remark}
\subsubsection{Problems}
\begin{problem}
Simplify using only positive exponents: $(3x^{0}y^{5}z^{6})(-2xy^{3}z^{-2})$
\end{problem}
\begin{proof}[Solution]
    \begin{flalign*}
        (3x^{0}y^{5}z^{6})(-2xy^{3}z^{-2})&=(3\cdot (-2))(x^{0}\cdot x^{1})(y^{5}\cdot y^{3})(z^{6}\cdot z^{-2})&\tag{Property~\ref{property:North_Shore_Arithmetic_Properties} part~\ref{property:north_shore_arithmetic_properties_assoc_mult}}\\
        &=-6x^{0+1}y^{5+3}z^{6-2}&\tag{Property~\ref{property:North_Shore_Exponent_Rules} part~\ref{property:north_shore_product_property_of_expo}}\\
        &=-6xy^{8}z^{4}&\tag{Simplify Exponents}
    \end{flalign*}
\end{proof}
\begin{problem}
Simplify using only positive exponents: $(3x^{2}-5x-6)+(5x^{2}+4x+4)$
\end{problem}
\begin{proof}[Solution]
\begin{flalign*}
    (3x^{2}-5x-6)+(5x^{2}+4x+4)&=(3+5)x^{2}+(-5+4)x+(-6+4)&\tag{Eqn.~\ref{equation:north_shore_additive_polynomial_property}}\\
    &=8x^{2}-x-2&\tag{Simplify Parenthesis}
\end{flalign*}
\end{proof}
\begin{problem}
Simplify using only positive exponents: $\frac{(2a^{-5}b^{4}c^{3})^{-2}}{(3a^{3}b^{-7}c^3)^{2}}$
\end{problem}
\begin{proof}[Solution]
\begin{flalign*}
    \tfrac{(2a^{-5}b^{4}c^{3})^{-2}}{(3a^{3}b^{-7}c^{3})^{2}}&=(2a^{-5}b^{4}c^{3})^{-2}\cdot\tfrac{1}{(3a^{3}b^{-7}c^{3})^{2}}&\tag{Remark~\ref{remark:North_Shore_Rational_Expressions}}\\
    &=\tfrac{1}{(2a^{-5}b^{4}c^{3})^{2}}\cdot\tfrac{1}{(3a^{3}b^{-7}c^{3})^{2}}&\tag{Property~\ref{property:North_Shore_Exponent_Rules} part~\ref{property:north_shore_inverse_proerty_of_expo}}\\
    &=\tfrac{1}{2^{2}(a^{-5})^{2}(b^{4})^{2}(c^{3})^{2}}\cdot\tfrac{1}{3^{2}(a^{3})^{2}(b^{-7})^{2}(c^{3})^{2}}&\tag{Property~\ref{property:North_Shore_Exponent_Rules} part~\ref{property:north_shore_distributive_property_of_expo}}\\
    &=\tfrac{1}{4a^{-10}b^{8}c^{6}}\cdot\tfrac{1}{9a^{6}b^{-14}c^{6}}&\tag{Property~\ref{property:North_Shore_Exponent_Rules} part~\ref{property:north_shore_power_property_of_expo}}\\
    &=\tfrac{a^{10}}{4b^{8}c^{6}}\cdot\tfrac{b^{14}}{9a^{6}c^{6}}&\tag{Property~\ref{property:North_Shore_Exponent_Rules} part~\ref{property:north_shore_inverse_proerty_of_expo}}\\
    &=\tfrac{a^{10}b^{14}}{36a^{6}b^{8}c^{6}c^{6}}&\tag{Multiplication}\\
    &=\tfrac{a^{10}b^{14}}{36a^{6}b^{8}c^{12}}&\tag{Property~\ref{property:North_Shore_Exponent_Rules} part~\ref{property:north_shore_product_property_of_expo}}\\
    &=\tfrac{a^{4}b^{6}}{36c^{12}}&\tag{Division}
\end{flalign*}
\end{proof}
\begin{problem}
Simplify using only positive exponents: $(-a^{5}b^{7}c^{9})^{4}$
\end{problem}
\begin{proof}[Solution]
    \begin{flalign*}
        (-a^{5}b^{7}c^{9})^{4}&=((-1)\cdot a^{5}b^{7}c^{9})^{4}&\tag{Rewrite Expression}\\
        &=(-1)^{4}(a^{5})^{4}(b^{7})^{4}(c^{9})^{4}&\tag{Property~\ref{property:North_Shore_Exponent_Rules} part~\ref{property:north_shore_distributive_property_of_expo}}\\
        &=a^{20}b^{28}c^{36}&\tag{Property~\ref{property:North_Shore_Exponent_Rules} part~\ref{property:north_shore_power_property_of_expo}}
    \end{flalign*}
\end{proof}
\begin{problem}
Simplify using only positive exponents: $\frac{24x^{4}-32x^{3}+16x^{2}}{8x^{2}}$
\end{problem}
\begin{proof}[Solution]
    \begin{flalign*}
        \tfrac{24x^{4}-32x^{3}+16x^{2}}{8x^{2}}&=\tfrac{8x^{2}(3x^{2}-4x+2)}{8x^{2}}&\tag{Property~\ref{property:North_Shore_Arithmetic_Properties} part~\ref{property:north_shore_arithmetic_properties_distributive_property}}\\
        &=3x^{2}-4x+2&\tag{Property~\ref{property:North_Shore_Arithmetic_Properties} part~\ref{property:north_shore_arithmetic_properties_mult_inverse}}
    \end{flalign*}
\end{proof}
\begin{problem}
Simplify using only positive exponents: $(x^{2}-5x)(2x^{3}-7)$
\end{problem}
\begin{proof}[Solution]
    \begin{flalign*}
        (x^{2}-5x)(2x^{3}-7)&=2x^{5}-7x^{2}-10x^{4}+35x&\tag{\gls{foil}}\\
        &=x(2x^{4}-10x^{3}-7x+35)&\tag{Property~\ref{property:North_Shore_Arithmetic_Properties} part~\ref{property:north_shore_arithmetic_properties_distributive_property}}
    \end{flalign*}
\end{proof}
\begin{problem}
Simplify using only positive exponents: $(4x^{2}y^{6}z)^{2}(-2x^{-2}y^{3}z^{4})^{6}$
\end{problem}
\begin{proof}[Solution]
    \begin{flalign*}
        (4x^{2}y^{6}z)^{2}(-x^{-2}y^{3}z^{4})^{6}&=(4^{2}(x^{2})^{2}(y^{6})^{2}(z)^{2})((x^{-2})^{6}(y^{3})^{6}(z^{4})^{6})&\tag{Property~\ref{property:North_Shore_Exponent_Rules} part~\ref{property:north_shore_distributive_property_of_expo}}\\
        &=(16x^{4}y^{12}z^{2})(x^{-12}y^{18}z^{24})&\tag{Property~\ref{property:North_Shore_Exponent_Rules} part~\ref{property:north_shore_power_property_of_expo}}\\
        &=16(x^{4}x^{-12})(y^{12}y^{18})(z^{2}z^{24})&\tag{Property~\ref{property:North_Shore_Arithmetic_Properties} part~\ref{property:north_shore_arithmetic_properties_assoc_mult}}\\
        &=16x^{4-12}y^{18+12}z^{2+24}&\tag{Property~\ref{property:North_Shore_Exponent_Rules} part~\ref{property:north_shore_product_property_of_expo}}\\
        &=16x^{-8}y^{30}z^{26}&\tag{Simplify Exponents}\\ 
        &=\tfrac{16y^{30}z^{26}}{x^{8}}&\tag{Property~\ref{property:North_Shore_Exponent_Rules} part~\ref{property:north_shore_inverse_proerty_of_expo}}
    \end{flalign*}
\end{proof}
\begin{problem}
Simplify using only positive exponents: $(x^{2}-5x)(2x^{3}-7)$
\end{problem}
\begin{proof}[Solution]
    \begin{flalign*}
        (x^{2}-5x)(2x^{3}-7)&=2x^{5}-7x^{2}-10x^{4}+35x&\tag{\gls{foil}}\\
        &=x(2x^{4}-10x^{3}-7x+35)&\tag{Property~\ref{property:North_Shore_Arithmetic_Properties} part~\ref{property:north_shore_arithmetic_properties_distributive_property}}
    \end{flalign*}
\end{proof}
\begin{problem}
Simplify using only positive exponents: $\frac{26a^{2}b^{-5}c^{9}}{-4a^{-6}bc^{9}}$
\end{problem}
\begin{proof}[Solution]
    \begin{flalign*}
        \tfrac{26a^{2}b^{-5}c^{9}}{-4a^{-6}bc^{9}}&=26a^2b^{-5}c^{9}\cdot\tfrac{1}{-4a^{-6}bc^{9}}&\tag{Remark~\ref{remark:North_Shore_Rational_Expressions}}\\
        &=\tfrac{26a^{2}c^{9}}{b^{5}}\cdot\tfrac{a^{6}}{-4bc^{9}}&\tag{Property~\ref{property:North_Shore_Exponent_Rules} part~\ref{property:north_shore_inverse_proerty_of_expo}}\\
        &= -\tfrac{26}{4}\cdot\tfrac{a^{2+6}c^{9}}{b^{5+1}c^{9}}&\tag{Property~\ref{property:North_Shore_Exponent_Rules} part~\ref{property:north_shore_product_property_of_expo}}\\
        &= -\tfrac{13a^{8}}{2b^{6}}&\tag{Property~\ref{property:North_Shore_Arithmetic_Properties} part~\ref{property:north_shore_arithmetic_properties_mult_inverse}}
    \end{flalign*}
\end{proof}
\begin{problem}
Simplify using only positive exponents: $(5a+6)^2$
\end{problem}
\begin{proof}[Solution]
    \begin{flalign*}
        (5a+6)^{2}&=25a^{2}+60a+36&\tag{Remark~\ref{remark:north_shore_square_of_a_sum_foil} Eqn.~\ref{equation:North_Shore_square_of_a_sum}}
    \end{flalign*}
\end{proof}
\begin{problem}
Simplify using only positive exponents: $(5x+1)(x+3)$
\end{problem}
\begin{proof}[Solution]
    \begin{flalign*}
        (5x+1)(x+3)&=5x^{2}+16x+3&\tag{Eqn.~\ref{equation:north_shore_multiplicative_polynomial_property}}
    \end{flalign*}
\end{proof}
\subsection{Factoring}
\begin{definition}
A quadratic is an equation of the form $y=ax^{2}+bx+c$.
\end{definition}
\begin{definition}
The Factorization of a quadratic $y=ax^{2}+bx+c$ is an equivalent expression of the form $y=\gamma(x+\alpha)(x+\beta)$.
\end{definition}
\begin{theorem}
\label{theorem:north_shore_factorization_of_quadratic_when_a_is_equal_to_zero}
If $y=x^{2}+bx+c$ has the factorization $y=(x+\alpha)(x+\beta)$, then $b=\alpha+\beta$ and $c=\alpha\cdot\beta$
\end{theorem}
\begin{proof}
Suppose $y=x^{2}+bx+c$ and $y=(x+\alpha)(x+\beta)$. By \gls{foil}, $y=(x+\alpha)(x+\beta)=x^{2}+x(\alpha+\beta)+\alpha\cdot\beta$. But also $y=x^{2}+bx+c$. But the coefficients must be equal, and therefore $\alpha+\beta=b$ and $\alpha\cdot\beta=c$.
\end{proof}
\begin{remark}\label{remark:north_shore_example_of_using_factorization_when_a_equals_zero}
This result helps us factor quadratic equations when the leading term has a coefficient of $1$ (That is, $a=1$). If we see something like $x^{2}-2x+1$ and we want to factor it, all we need to ask is ``What two numbers add to $-2$ and multiply to $1$?" Often times guessing and checking will get the answer after a few tries. For $x^{2}-2x+1$ We see that $(-1)+(-1)=-2$, and $(-1)\cdot (-1)=1$. So, $x^{2}-2x+1=(x-1)(x-1)=(x-1)^{2}$.
\end{remark}
\begin{theorem}[The Difference of Squares]
\label{theorem:north_shore_difference_of_squares}
If $a$ and $b$ are real numbers, then $a^{2}-b^{2}=(a-b)(a+b)$
\end{theorem}
\begin{proof}
By \gls{foil}, $(a-b)(a+b)=a^{2}-ab+ba-b^{2}=a^{2}-b^{2}$. Therefore, $(a-b)(a+b)=a^{2}-b^{2}$.
\end{proof}
\begin{remark}
This helps factor quadratics quickly if two squares are being subtracted.
\end{remark}
\begin{example}
Factor the expression: $9x^{2}-64$. Note that $9=3^{2}$, so $9x^{2}=(3x)^{2}$. Also note that $64=8^{2}$. So we can write $9x^{2}-64=(3x)^{2}-(8)^{2}$. Using the difference of squares formula, we have $9x^{2}-64=(3x)^{2}-(8)^{2}=(3x-8)(3x+8)$
\end{example}
\begin{example}
Factor the expression: $16x^{4}-81y^{4}$. While this looks like a ``Quartic Equation" (Equations involving $x^4$), it can be rewriten as a quadratic one. Note that $16=4^{2}$ and $81=9^{2}$. But also $x^{4}=(x^{2})^{2}$ and $y^{4}=(y^{2})^{2}$. So we have $16x^{4}-81y^{4}=(4x^{2})^{2}-(9y^{2})^{2}$. This is a difference of squares, and so we can apply the difference of squares formula. $16x^{4}-81y^{4}=(4x^{2})^{2}-(9y^{2})^{2}=(4x^{2}+9y^{2})(4x^{2}-9y^{2})$. We're not quite done yet, for we can simplify $4x^{2}-9y^{2}$ as well. Note that $4x^{2}=(2x)^{2}$, and $9y^{2}=(3y)^{2}$. So we have $4x^{2}-9y^{2}=(2x)^{2}-(3y)^{2}=(2x-3y)(2x+3y)$. Together, we have: $16x^{4}-81y^{4}=(4x^{2}+9y^{2})(2x-3y)(2x+3y)$.
\end{example}
\begin{theorem}[The Difference of Cubes]
\label{theorem:north_shore_difference_of_cubes}
If $a$ and $b$ are real numbers, then $a^{3}-b^{3}=(a-b)(a^{2}+ab+b^{2})$
\end{theorem}
\begin{proof}
By the distributive property, $(a-b)(a^{2}+ab+b^{2})=a(a^{2}+ab+b^{2})-b(a^{2}+ab+b^{2})$. Distributing again, we get $a^{3}+a^{2}b+ab^{2}-ba^{2}-ab^{2}-b^{3}$. Rearranging, we have $(a^{3}-b^{3})+(a^{2}b-ba^{2})+(ab^{2}-b^{2}a)$. By the commutative property, $a^{2}b=ba^{2}$, and $ab^{2}=b^{2}a$. Therefore, $(a^{2}b-ba^{2})+(ab^{2}-b^{2}a)=0$. Thus, $(a-b)(a^{2}+ab+b^{2})=a^{3}-b^{3}$.
\end{proof}
\begin{remark}
We can use the difference of cubes formula to factor cubic expressions.
\end{remark}
\begin{theorem}[The Sum of Squares]
\label{theorem:north_shore_sum_of_squares}
If $x$ and $a$ are real numbers, and if $a$ is non-zero, then there is no \textbf{real} factorization of $x^{2}+a^{2}$.
\end{theorem}
\begin{proof}
If $(x+\alpha)(x+\gamma)$ is a real factorization, then by Thm.~\ref{theorem:north_shore_factorization_of_quadratic_when_a_is_equal_to_zero}, $\alpha+\beta=0$ and $\alpha\cdot\beta=a^{2}$. But then $\alpha=-\beta$ and therefore $-\alpha^{2}=a^{2}$. But $a$ is real, and therefore $a^{2}\geq 0$. But as $\alpha$ is real and non-zero, $\alpha^{2}>0$, and thus $-\alpha^{2}<0$. But $a^{2}=-\alpha^{2}$, a contradiction as $a^{2}\geq 0$. There is no real factorization.
\end{proof}
\begin{remark}
If we see a sum of squares we know that there is no real factorization.
\end{remark}
\subsubsection{Problems}
\begin{problem}
Factor $x^{2}+5x-6$
\end{problem}
\begin{proof}[Solution]
By Thm.~\ref{theorem:north_shore_factorization_of_quadratic_when_a_is_equal_to_zero}, if $x^{2}+5x-6=(x+\alpha)(x+\beta)$, then $\alpha+\beta=5$ and $\alpha\cdot\beta=-6$. By guessing and checking a few common factors of $5$ and $6$, we get $\alpha=6$ and $\beta=-1$. So, $x^{2}+5x-6=(x+6)(x-1)$
\end{proof}
\begin{problem}
$x^{2}-5x-6$
\end{problem}
\begin{proof}[Solution]
By Thm.~\ref{theorem:north_shore_factorization_of_quadratic_when_a_is_equal_to_zero}, if $x^{2}-5x-6=(x+\alpha)(x+\beta)$, then $\alpha+\beta=-5$ and $\alpha\cdot\beta=-6$. After guessing and checking, we have $\alpha=-6$ and $\beta=1$. So $x^{2}-5x-6=(x-6)(x+1)$
\end{proof}
\begin{problem}
$4x^{2}-36$
\end{problem}
\begin{proof}[Solution]
$4x^{2}-36=4(x^{2}-3^{2})$. By Thm.~\ref{theorem:north_shore_difference_of_squares}, $x^{2}-3^{2}=(x+3)(x-3)$. Thus, $4x^{2}-36=4(x-3)(x+3)$
\end{proof}
\begin{problem}
$x^{2}+4$
\end{problem}
\begin{proof}[Solution]
This is a sum of squares. By Thm.~\ref{theorem:north_shore_sum_of_squares}, there is no real factorization.
\end{proof}
\begin{problem}
$64x^{4}-4y^{4}$
\end{problem}
\begin{proof}[Solution]
Note that $64x^{2}-4y^{2}=(8x^{2})^{2}-(2y^{2})^{2}$. By Thm.~\ref{theorem:north_shore_difference_of_squares}, $(8x^{2})^{2}-(2y^{2})^{2}=(8x^{2}+2y^{2})(8x^{2}-2y^{2})$. Note that $8x^{2}$ and $2y^{2}$ share a common factor of $2$ so we can simplify this as $4(4x^{2}+y^{2})(4x^{2}-y^{2})$. Again by the difference of squares, we have $4x^{2}-y^{2}=(2x)^{2}-y^{2}=(2x-y)(2x+y)$. Piecing this all back together, we have $64x^{2}-4y^{2}=4(4x^{2}+y^{2})(2x-y)(2x+y)$.
\end{proof}
\begin{problem}
$8x^{3}-27$
\end{problem}
\begin{proof}[Solution]
By Thm.~\ref{theorem:north_shore_difference_of_cubes}, $8x^{3}-27=(2x)^{3}-(3)^{3}=(2x-3)(4x^{2}+6x+9)$
\end{proof}
\begin{problem}
$49y^{2}+84y+36$
\end{problem}
\begin{proof}[Solution]
First note that $49=7^{2}$, $36=6^{2}$, and $84=2\cdot 6\cdot 7$. So, $49x^{2}+84x+36=(7x)^{2}+2(7)(6)x+(6)^{2}$. But $(ax+b)^{2}=a^{2}x^{2}+2abx+b^{2}$. We have $a=7,b=6$, so $49x^{2}+84x+36=(7x+6)^{2}$.
\end{proof}
\begin{problem}
$12x^{2}+12x+3$
\end{problem}
\begin{proof}[Solution]
$4x^{2}+4x+1=3((2x)^{2}+2(2)(1)(x)+(1)^{2})=3(2x+1)^{2}$
\end{proof}
\subsection{Quadratic Expressions}
\begin{theorem}[Completing the Square]
\label{theorem:north_shore_completing_the_square}
If $y=ax^{2}+bx+c$, then $y=a(x+\frac{b}{2a})^{2}-\frac{b^2}{4a}+c$
\end{theorem}
\begin{proof}
For $(x+\frac{b}{2a})^{2}=x^{2}+\frac{b}{a}x+\frac{b^{2}}{4a^{2}}$. But then $a(x+\frac{b}{2a})^{2}-\frac{b^{2}}{4a}+c=ax^{2}+bx+c$
\end{proof}
\begin{theorem}[The Quadratic Formula]
\label{theorem:north_shore_quadratic_formula_theorem}
If $ax^{2}+bx+c=0$, $a\ne 0$, then $x=\frac{-b\pm\sqrt{b^{2}-4ac}}{2a}$
\end{theorem}
\begin{proof}
By Thm.~\ref{theorem:north_shore_completing_the_square}, $ax^{2}+bx+c=a(x+\frac{b}{2a})^{2}-\frac{b^{2}}{4a}+c$. But $ax^{2}+bx+c=0$, so $a(x+\frac{b}{2a}x)^{2}-\frac{b^{2}}{4a}+c=0$. Therefore $a(x+\frac{b}{2a})^{2}=\frac{b^2}{4a}-c$. But $c=\frac{4ac}{4a}$, so $a(x^2+\frac{b}{2a})^{2}=\frac{b^{2}-4ac}{4a}$. Dividing both sides by $a$, we get $(x+\frac{b}{2a})^{2}=\frac{b^{2}-4ac}{4a^{2}}$. Now we take square roots, but note that there are two possible square roots. So, $x+\frac{b}{2a}=\pm\frac{\sqrt{b^{2}-4ac}}{2a}$. Subtracting $\frac{b}{2a}$ from both sides we get $x=-\frac{b}{2a}\pm\frac{\sqrt{b^{2}-4ac}}{2a}=\frac{-b\pm\sqrt{b^{2}-4ac}}{2a}$
\end{proof}
\begin{remark}
Remember that $ax^{2}+bx+c=0$ has two solutions: $x=\frac{-b+\sqrt{b^{2}-4ac}}{2a}$ and $x=\frac{-b-\sqrt{b^{2}-4ac}}{2a}$
\end{remark}
\begin{theorem}
\label{theorem:north_shore_zeros_of_a_factored_polynomial}
If $\gamma(x-\alpha)(x-\beta)=0$, then either $x=\alpha$ or $x=\beta$.
\end{theorem}
\begin{remark}
If we see $ax^{2}+bx+c=0$ and can \textit{factor} into $\gamma(x-\alpha)(x-\beta)$, then $x=\alpha$ or $x=\beta$.
\end{remark}
\subsubsection{Problems}
\begin{problem}
Find all solutions for a: $4a^{2}+9a+2=0$
\end{problem}
\begin{proof}[Solution]
By Thm.~\ref{theorem:north_shore_quadratic_formula_theorem}, $a=\frac{-9\pm\sqrt{9^{2}-4\cdot 4\cdot 2}}{2\cdot 4}=\frac{-9\pm 7}{8}$. $a=-2$ or $a=-\frac{1}{4}$
\end{proof}
\begin{problem}
Find all solutions for $x$: $9x^{2}-81=0$
\end{problem}
\begin{proof}[Solution]
By Thm.~\ref{theorem:north_shore_difference_of_squares}, $9x^{2}-81=9(x-3)(x+3)$. By Thm.~\ref{theorem:north_shore_zeros_of_a_factored_polynomial}, $x=3$ or $x=-3$.
\end{proof}
\begin{problem}
Find all solutions for $x$: $25x^{2}-6=30$.
\end{problem}
\begin{proof}[Solution]
Subtract $30$ to get $25x^{2}-36=0$. By Thm.~\ref{theorem:north_shore_difference_of_squares}, $25x^{2}-36 =(5x-6)(5x+6)$. By Thm.~\ref{theorem:north_shore_zeros_of_a_factored_polynomial}, $5x=6$ or $5x=-6$. $x=\frac{6}{5}$ or $x=-\frac{6}{5}$.
\end{proof}
\begin{problem}
Find all solutions for $x$: $3x^{2}-5x-2=0$
\end{problem}
\begin{proof}[Solution]
By Thm.~\ref{theorem:north_shore_quadratic_formula_theorem}, $x=\frac{5\pm\sqrt{(-5)^{2}-4(3)(-2)}}{2(3)}=\frac{5\pm 7}{6}$. So $x=2$ or $x=-\frac{1}{3}$
\end{proof}
\begin{problem}
Find all solutions for $x$: $(3x+2)^{2}=16$
\end{problem}
\begin{proof}[Solution]
Subtracting $16$ we get $(3x+2)^{2}-16$. By Thm.~\ref{theorem:north_shore_difference_of_squares}, $(3x+2)^{2}-16=(3x+6)(3x-2)$. By Thm.~\ref{theorem:north_shore_zeros_of_a_factored_polynomial}, $3x=-6$ or $3x=2$. Thus, $x=-2$ or $x=\frac{2}{3}$.
\end{proof}
\begin{problem}
Find all solutions for $r$: $r^{2}-2r-4=0$
\end{problem}
\begin{proof}[Solution]
By Thm.~\ref{theorem:north_shore_quadratic_formula_theorem}, $r=\frac{2\pm\sqrt{(-2)^{2}-4(1)(-4)}}{2(1)}=1\pm\sqrt{5}$. $x=1+\sqrt{5}$ or $x=1-\sqrt{5}$
\end{proof}
\subsection{Rational Expressions}
\begin{theorem}[Cross Multiplying]
\label{theorem:north_shore_cross_multiplying}
If $a,b,c,d$ are real numbers, $b\ne 0$, $d\ne 0$, then $\frac{a}{b}+\frac{c}{d}=\frac{ad+bc}{bd}$
\end{theorem}
\begin{definition}
\label{definition:north_shore_numerator_of_rational_expression}
The numerator of a rational expression $\frac{P(x)}{Q(x)}$ is the polynomial $P(x)$.
\end{definition}
\begin{definition}
\label{definition:north_shore_denominator_of_rational_expression}
The denominator of a rational expression $\frac{P(x)}{Q(x)}$ is the polynomial $Q(x)$.
\end{definition}
\subsubsection{Problems}
\begin{problem}
Simplify: $\frac{4}{2a-2}+\frac{3a}{a^{2}-a}$
\end{problem}
\begin{proof}[Solution]
    \begin{flalign*}
        \tfrac{4}{2a-2}+\tfrac{3a}{a^{2}-a}&=\tfrac{2}{a-1}+\tfrac{3}{a-1}&\tag{Factor and Simplify Terms}\\
        &=\tfrac{5}{a-1}&\tag{Addition}
    \end{flalign*}
\end{proof}
\begin{problem}
Simplify: $\frac{3}{x^{2}-1}-\frac{4}{x^{2}+3x+2}$
\end{problem}
\begin{proof}[Solution]
    \begin{flalign*}
        \tfrac{3}{x^{2}-1}-\tfrac{4}{x^{2}+3x+2}&=\tfrac{3}{(x-1)(x+1)}-\tfrac{4}{(x+2)(x+1)}&\tag{Factor Denominators}\\
        &=\tfrac{3(x+2)-4(x-1)}{(x-1)(x+2)(x+1)}&\tag{Thm.~\ref{theorem:north_shore_cross_multiplying}}\\
        &=\tfrac{10-x}{(x+1)(x+2)(x-1)}&\tag{Simplify Numerator}
    \end{flalign*}
\end{proof}
\begin{problem}
Simplify: $\frac{\frac{2}{x}-\frac{1}{y}}{\frac{1}{xy}}$
\end{problem}
\begin{proof}[Solution]
    \begin{flalign*}
        \tfrac{\frac{2}{x}-\frac{1}{y}}{\frac{1}{xy}}&=xy(\tfrac{2}{x}-\tfrac{1}{y})&\tag{Property~\ref{property:North_Shore_Exponent_Rules} part~\ref{property:north_shore_inverse_proerty_of_expo}}\\
        &=2y-x&\tag{Multiplication}
    \end{flalign*}
\end{proof}
\begin{problem}
Simplify: $\frac{1}{1-\sqrt{x}}+\frac{1}{1+\sqrt{x}}$
\end{problem}
\begin{proof}[Solution]
    \begin{flalign*}
        \tfrac{1}{1-\sqrt{x}}+\tfrac{1}{1+\sqrt{x}}&=\tfrac{1+\sqrt{x} + 1-\sqrt{x}}{(1-\sqrt{x})(1+\sqrt{x})}&\tag{Thm.~\ref{theorem:north_shore_cross_multiplying}}\\
        &=\tfrac{2}{1-x}&\tag{Simplify and \gls{foil}}
    \end{flalign*}
\end{proof}
\newpage
\begin{problem}
$\frac{2}{x-1}+\frac{1}{x+1}=\frac{5}{4}$
\end{problem}
\begin{proof}[Solution]
    \begin{flalign*}
        \tfrac{2}{x-1}+\tfrac{1}{x+1}&=\tfrac{2(x+1)+(x-1)}{(x+1)(x-1)}&\tag{Thm.~\ref{theorem:north_shore_cross_multiplying}}\\
        &=\tfrac{3x+1}{(x+1)(x-1)}&\tag{Simplify the Numerator}\\
        \Rightarrow \tfrac{3x+1}{(x+1)(x-1)}&=\tfrac{5}{4}&\tag{Substitution}\\
        \Rightarrow 3x+1&=\tfrac{5}{4}(x+1)(x-1)&\tag{Multiply Both Sides by $(x+1)(x-1)$}\\
        \Rightarrow 12x+4&=5x^{2}-5&\tag{Multiply Both Sides by $4$ and Simplify}\\
        \Rightarrow 5x^{2}-12x-9&=0&\tag{Subtract $12x+4$ from Both Sides}\\
        \Rightarrow x&=\tfrac{12\pm\sqrt{(12)^{2}-4\cdot (5)\cdot(-9)}}{2\cdot(5)}&\tag{Thm.~\ref{theorem:north_shore_quadratic_formula_theorem}}\\
        \Rightarrow x&=\tfrac{12\pm\sqrt{324}}{10}&\tag{Simplify}\\
        \Rightarrow x&=\tfrac{12\pm 18}{10}&\tag{Simplify}\\
        \Rightarrow x&=3\textrm{ or }-\tfrac{3}{5}&\tag{Simplify}
    \end{flalign*}
\end{proof}
\subsection{Graphing}
\subsubsection{Problems}
\begin{problem}
\label{problem:north_shore_exam_graph_everything}Graph the following:
\begin{enumerate}
\begin{multicols}{4}
    \item $3x-2y=6$
    \item $x=-3$
    \item $y=2$
    \item $y=-\frac{2}{3}x+5$
    \item $y=|x-3|$
    \item $y=-x^2+2$
    \item $y=\sqrt{2-x}$
    \item $y=x-10$
\end{multicols}
\end{enumerate}
\end{problem}
\begin{figure}[H]
    \centering
    \begin{tikzpicture}[>=triangle 45]
        \begin{axis}[width=\linewidth,axis lines=center,axis line style={<->},
            xtick distance=1,xlabel = $x$,xmin=-10.5,xmax=10.5,ytick distance=1,ylabel = $y$,ymin=-10.5,ymax=10.5,
            grid=both,grid style={line width=.1pt, draw=gray!10},major grid style={line width=.2pt,draw=gray!50}]
            \draw[draw=blue,line width=1pt] (axis cs:-7,10) -- (axis cs:3,0) -- node[pos=.45,right=3pt] {$\color{blue}y=|x-3|$} (axis cs:10,7);
            \draw[draw=green,line width=1pt] (axis cs:-3,-10) -- node[pos=.325,left=1pt] {$\color{green}x=-3$} (axis cs:-3,10);
            \draw[draw=purple,line width=1pt] (axis cs:-10,2) -- node[pos=.1,above=1pt] {$\color{purple}y=2$} (axis cs:10,2);
            \draw[draw=cyan,line width=1pt] (axis cs:-8,10.333) -- node[pos=.55,above=0.7cm] {$\color{cyan}y=-\frac{2}{3}x+5$} (axis cs:10,-1.667);
            \draw[draw=pink,line width=1pt] (axis cs:-5,-10.5) -- node[pos=.1,left=0.6cm] {$\color{pink}3x-2y=6$} (axis cs:9,10.5);
            \draw[draw=red,line width=1pt] (axis cs:0,-10) -- node[pos=.55,below right=0.3cm and 0.3cm] {$\color{red}y=x-10$} (axis cs:10,0);
            \addplot[domain=-3.5:3.5,draw=RoyalBlue,samples=50,line width=1pt,variable=\x] {-\x^2+2} node[pos=.9,right=0.5cm] {$\color{RoyalBlue}y=-x^{2}+2$};
            \addplot[domain=0:3.46,draw=SkyBlue,samples=25,line width=1pt,variable=\x] (2-\x^2,\x) node[pos=.85,above=0.3cm] {$\color{SkyBlue}y=\sqrt{2-x}$};
        \end{axis}
    \end{tikzpicture}
    \caption{The chaos that is the solution to problem \ref{problem:north_shore_exam_graph_everything}}
    \label{fig:north_shore_graphing_problem}
\end{figure}
\subsection{Systems of Equations}
A system of equations is a set of 2 or more equations involving the same variables. Solving systems of linear equations is one of the main focuses in the study of linear algebra.
\begin{example}
\label{example:north_shore_example_of_a_system_of_linear_equations}Consider the following system of linear equations:
\begin{align*}
    2x+3y&=1\\
    x+6y&=2
\end{align*}
Solving this system of equations asks ``Which ordered pairs $(x,y)$ solve both of these equations?" The second equations says that $x=2-6y$. We can substitute this back into the first equation to get $2(2-6y)+3y=1$ which simplifies to $4-9y=1$. The solution to this is $y=\frac{1}{3}$. Solving for $x$, we have $x=2-6y=2-6(\frac{1}{3})=2-2=0$. $(0,\frac{1}{3})$ is a solution.
\end{example}
We can picture these types of problems graphically as well. For the example above there are two linear equations, which can be represented graphically as lines. A solution to this system can then be interpreted as a point where the two lines intersect. The graph of the two equations in example \ref{example:north_shore_example_of_a_system_of_linear_equations} are shown in Fig.~\ref{fig:north_shore_systems_of_linear_equations}.
\begin{figure}[H]
    \centering
    \begin{tikzpicture}[>=triangle 45]
        \begin{axis}[width=0.8\linewidth,axis lines=center,axis line style={<->},
            xtick distance=1,xlabel = $x$,xmin=-4.2,xmax=4.2,ytick distance=1,ylabel = $y$,ymin=-4.2,ymax=4.2,
            grid=both,grid style={line width=.1pt, draw=gray!10},major grid style={line width=.2pt,draw=gray!50}]
            \draw[draw=blue,line width=1pt] (axis cs:-4,1) -- node[pos=.1,above=5mm] {$\color{blue}x+6y=2$} (axis cs:4,-0.666);
            \addplot[domain=-4:4,draw=red,samples=2,line width=1pt,variable=\x] {0.33333-0.66666*\x} node[pos=.15,above=0.7cm] {$\color{red}2x+3y=1$};
        \end{axis}
    \end{tikzpicture}
    \caption{The graphs of the two equations shown in example \ref{example:north_shore_example_of_a_system_of_linear_equations}.}
    \label{fig:north_shore_systems_of_linear_equations}
\end{figure}
Any system of linear equations has 3 possible outcomes: No solutions, one solution, or infinitely many solutions. This makes sense if one considers the few possibilities allowed. Given two parallel lines there will be no solutions for parallel lines never intersect. Given two equations that represent the same line there will be infinitely many solutions, for any point on the line will work. Given two equations for two distinct non-parallel lines there will be only one solution, for such lines can only intersect once.
\subsubsection{Problems}
\begin{problem}
Solve the following system of equations:
\begin{align*}
    2x-3y&=-12\\
    x-2y&=-9
\end{align*}
\end{problem}
\begin{proof}[Solution]
Subtracting 2 of the second equation from first equation, we get:
\begin{equation*}
    (2x-3y)-2(x-2y)=-12-2(-9)
\end{equation*}
Which simplifies to $y=6$. The second equation yield $x-2(6)=-9\Rightarrow x=3$. The solution is $(3,6)$
\end{proof}
\begin{problem}
Solve the following system of equations:
\begin{align*}
    4x+6y&=10\\
    2x+3y&=5
\end{align*}
\end{problem}
\begin{proof}[Solution]
Dividing both sides of the first equation by 2 gives us the second equation. These equations represent the same line, and there are infinitely many solutions: $y=\frac{5}{3}-\frac{2}{3}x$
\end{proof}
\begin{problem}
Solve the following system of equations:
\begin{align*}
    x+2y&=5\\
    x+2y&=7
\end{align*}
\end{problem}
\begin{proof}[Solution]
Let $z=x+2y$. Then $z=5$ and $z=7$. But this is impossible, for $5\ne 7$. No solutions.
\end{proof}
\begin{problem}
\begin{align*}
    2x-3y&=-4\\
    2x+y\phantom{3}&=\phantom{-}4
\end{align*}
\end{problem}
\begin{proof}[Solution]
Subtracting the first equation from the second gives us $4y=8\Rightarrow y=2$. But then from the first equation we obtain $2x=3(2)-4=2\Rightarrow x=1$. The solution is $(1,2)$.
\end{proof}
\subsection{Radicals}
To simplify expressions with radicals we use something called the conjugate of a radical expression. 
\begin{definition}
The conjugate of a rational expression $\sqrt{a}-\sqrt{b}$ is $\sqrt{a}+\sqrt{b}$.
\end{definition}
Using the \gls{foil} rule from before, we can obtain the following theorem.
\begin{theorem}
If $a,b\geq 0$, and if $a\ne b$, then $\frac{1}{\sqrt{a}-\sqrt{b}}=\frac{\sqrt{a}+\sqrt{b}}{a-b}$
\end{theorem}
\begin{proof}
As $a,b\geq 0$ and $a\ne b$, $\frac{1}{\sqrt{a}-\sqrt{b}}$ is well-defined and $\sqrt{a}+\sqrt{b}\ne 0$. But then:
\begin{align*}
    \tfrac{1}{\sqrt{a}-\sqrt{b}}&=\tfrac{1}{\sqrt{a}-\sqrt{b}}\tfrac{\sqrt{a}+\sqrt{b}}{\sqrt{a}+\sqrt{b}}\\
    &=\tfrac{\sqrt{a}+\sqrt{b}}{(\sqrt{a}-\sqrt{b})(\sqrt{a}+\sqrt{b})}\\
    &=\tfrac{\sqrt{a}+\sqrt{b}}{a-b}&\tag{Thm.~\ref{theorem:north_shore_difference_of_squares}}
\end{align*}
\end{proof}
\subsubsection{Problems}
\begin{problem}
Simplify the following so that there are no radicals in the denominator:
\begin{enumerate}
\begin{multicols}{3}
    \item $\sqrt{8}\sqrt{10}$
    \item $\sqrt[4]{\frac{81}{x^4}}$
    \item $\sqrt{\frac{4}{3}}$
    \item $\sqrt{\frac{12}{18}}$
    \item $\sqrt[3]{24x^{3}y^{6}}$
    \item $\frac{\sqrt{3}}{5-\sqrt{3}}$
\end{multicols}
\end{enumerate}
\end{problem}
\begin{proof}[Solution]
\
\begin{enumerate}
\begin{multicols}{3}
    \item $\sqrt{8}\sqrt{10}=\sqrt{16\cdot 5}=4\sqrt{5}$
    \item $\sqrt[4]{\frac{81}{x^{4}}}=\frac{3}{|x|}$
    \item $\sqrt{\frac{4}{3}}=\frac{2}{\sqrt{3}}=\frac{2\sqrt{3}}{3}$
    \item $\sqrt{\frac{12}{18}}=\frac{2\sqrt{3}}{3\sqrt{2}}=\frac{2\sqrt{2}\sqrt{3}}{6}$
    \item $\sqrt[3]{24x^{3}y^{6}}=2\sqrt[3]{3}xy^{2}$
    \item $\frac{\sqrt{3}}{5-\sqrt{3}}=\frac{\sqrt{3}(5+\sqrt{3})}{5-3}=\frac{5\sqrt{3}+3}{2}$
\end{multicols}
\end{enumerate}
\end{proof}
\section{Miscellaneous Notes}
\subsection{The Language, Notation, and Numbers of Mathematics}
We begin by discussing sets of numbers, with a primary focus on the set of real numbers.
\begin{definition}
A set is a collection of objects, none of which is the set itself.
\end{definition}
\begin{remark}
The requirement that a set cannot contain itself is to avoid logical paradoxes, such as the one discovered by Bertrand Russell. We won't delve into this matter.
\end{remark}
\begin{remark}
There are two main ways to describe the elements of a set. The first way is to list out the elements, separated by commas, and enclosed in braces $\{\ \}$. For example $\{1,2,3\}$ or $\{a,b,c\}$. When a set has infinitely many elements, but the elements can be listed in a certain pattern, we use an ellipses (Three dots $\hdots$) to indicate the pattern goes on. For example, $\{1,2,3,\hdots\}$ is the set of all positive integers. $\{2,4,6,8,\hdots\}$ is the set of all even integers. $\{1,3,5,7,9,\hdots\}$ is the set of all odd integers. The second way to describe a set is using the so called Set-Builder notation. To write out the set of all positive integers, we could write $\{n: n\ \textrm{is a positive integer}\}$. This reads as the set of all $n$ such that $n$ is a positive integer. Or $\{r: r\ \textrm{is a real number}\}$ is the set of all real numbers. $\{x: x\ \textrm{is a day of the week}\}$ is another way to write the set $\{Sunday, Monday, Tuesday, Wednesday, Thursday, Friday, Saturday\}$.
\end{remark}

\begin{remark}
In mathematics, sets can be very abstract entities that have nothing to do with numbers. For example, $\{\textrm{Apple, Eiffel Tower, Japan, Mathematics}\}$ is a set containing those four objects. The objects themselves need not be related, nor need they be "Mathematical," in nature. For the purpose of precalculus, we study sets of numbers.
\end{remark}

\begin{example}[Examples of sets]
\
\begin{enumerate}
\begin{multicols}{3}
\item $\{1,2,3\}$
\item $\{a,b,c\}$
\item $\{\textrm{Boston, New York, Chicago}\}$
\item $\{\textrm{Atlantic, Pacific, Arctic, Indian}\}$
\item $\{n:n\ \textrm{is an integer}\}$
\item $\{n^2: n\ \textrm{is an integer}\}$
\end{multicols}
\end{enumerate}
\end{example}

\begin{remark}
$\{n^2: n\ \textrm{is an integer}\}$ is a more compact way of writing $\{0,1,4,9,16,25,36,49,\hdots\}$.
\end{remark}

\begin{notation}
If $A$ is a set and $x$ is an element of $A$, we write $x\in A$. If $x$ is not an element of $A$, we write $x\notin A$.
\end{notation}

\begin{example}
Let $A = \{1,2,3,4,5\}$. Then $3\in A$, for the number $3$ appears in the set $A$, however $6\notin A$.
\end{example}

\begin{definition}
A subset of a set $B$ is a set $A$ such that for all $x\in A$, it is true that $x\in B$.
\end{definition}

\begin{remark}
That is to say a set $A$ is a subset of a set $B$ if $A$ is entirely contained within $B$.
\end{remark}

\begin{notation}
If $A$ is a subset of $B$, we write $A\subset B$. If $A$ is not a subset of $B$ we write $A\not\subset B$.
\end{notation}

\begin{example}
Let $A = \{1,2,3,4,5\}$ and $B=\{1,2,3,4,5,6,7,8\}$. Then $A\subset B$ for every number contained in $A$ is contained in $B$. However, $B\not\subset A$, for there are numbers in $B$ that are not in $A$. For example, $7\in B$ but $7\notin A$.
\end{example}

\begin{example}[Examples of Subsets]
\
\begin{enumerate}
\begin{multicols}{2}
\item $\{1,2,3\}\subset \{1,2,3,4,5\}$
\item $\{a,b,c\} \subset \{a,b,c,d,e\}$
\item $\{\textrm{Boston}\} \subset \{\textrm{Boston, New York, Chicago}\}$
\item $\{\textrm{Atlantic, Pacific}\} \subset \{\textrm{Atlantic, Pacific, Arctic, Indian}\}$
\end{multicols}
\end{enumerate}
\end{example}

\begin{notation}
There are special symbols used for the more common sets of numbers that are encountered in mathematics:
\begin{enumerate}
\item $\mathbb{N} = \{1,2,3,4,\hdots\}$ \hfill [The Natural Numbers]
\item $\mathbb{N}_0 = \{0,1,2,3,\hdots\}$ \hfill [The Whole Numbers]
\item $\mathbb{Z} = \{\hdots,-2,-1,0,1,2,\hdots\}$ \hfill [The Integers]
\item $\mathbb{Q} = \{\frac{p}{q}: p,q\in \mathbb{Z}, q\ne 0\}$ \hfill [The Rational Numbers]
\item $\mathbb{R} = \{x:x\ \textrm{is a real number}\}$ \hfill [The Real Numbers]
\item $\\mathbb{R}\setminus\mathbb{Q}=\{x\in\mathbb{R}:x\notin\mathbb{Q}\}$ \hfill [The Irrational Numbers]
\end{enumerate}
\end{notation}
\begin{example}[Examples of Numbers]
\
\begin{enumerate}
\begin{multicols}{2}
\item $\{1,2,5,111,460,123456789\} \subset \mathbb{N}$. 
\item $\{0,1,2,3,10000\}\subset \mathbb{N}_0$
\item $\{-11,47,123456789,-123456789,0\} \subset \mathbb{Z}$
\item $\{0,1,-1,\frac{1}{2}, \frac{117}{211}, 0.123456789, 0.1111111\hdots\} \subset \mathbb{Q}$
\item $\{\pi,\sqrt{2},0,1,2,3,-5,0.101001000100001\hdots,\frac{2}{5}\}\subset \mathbb{R}$
\item $\{\pi,\sqrt{2},0.101001000100001\hdots,\frac{1+\sqrt{5}}{2}\}\subset\\mathbb{R}\setminus\mathbb{Q}$
\end{multicols}
\end{enumerate}
\end{example}
\begin{remark}
$0.111\hdots$ is just a fancy way to write $\frac{1}{9}$. Strange beings like $\pi$ are included in the real numbers, but cannot be expressed as fractions. It can be shown that any rational number has a repeating decimal expansion. Therefore a number like $0.1234567891011121314151617181920\hdots$ cannot possibly be rational, for the decimal never repeats. Similarly $0.123412341234\hdots$ must be a rational number, for its decimal expansion repeats. Indeed this number is equal to $\frac{1234}{9999}$.
\end{remark}
\begin{definition}
The intersection of two set $A$ and $B$, denoted $A\cap B$, is the set $A\cap B = \{x:x\in A\ \textrm{AND}\ x\in B\}$.
\end{definition}
\begin{remark}
That is, the intersection of two sets is the set that contains all elements simultaneously in $A$ and $B$.
\end{remark}
\begin{definition}
The empty set, denoted $\emptyset$ is the set that contains no elements. That is, $\emptyset = \{\}$.
\end{definition}
\begin{remark}
The empty set is a strange concept. Do not confuse $\emptyset$ and $\{\emptyset\}$. $\emptyset$ is the set that contains NO elements, $\{\emptyset\}$ is the set that contains the empty set (And is therefore NOT empty). We won't worry about $\emptyset$ much.
\end{remark}
\begin{example}
If $A = \{1,2,3,4,5\}$ and $B = \{3,4,5,6,7\}$, then $A\cap B = \{3,4,5\}$. If $A = \{1,2,3,4,5\}$ and $C = \{6,7,8,9,10\}$, then $A$ and $C$ contain nothing in common, and therefore $A\cap C = \emptyset$.
\end{example}
\begin{example}[Examples of Intersection]
\
\begin{enumerate}
\begin{multicols}{2}
\item $\{1,2,3\}\cap \{3,4,5\} = \{3\}$
\item $\{a,b,c\} \cap \{c,d,e\} = \{c\}$
\end{multicols}
\item $\{\textrm{Boston, New York}\} \cap \{\textrm{New York, Chicago}\} = \{\textrm{New York}\}$
\item $\{\textrm{Polar Bear, Mars, Spongebob}\} \cap \{1,a,\mathcal{L}\} = \emptyset$
\end{enumerate}
\end{example}
\begin{definition}
The union of sets $A$ and $B$, denoted $A\cup B$, is the set $A\cup B = \{x:x\in A\ \textrm{OR}\ x\in B\}$.
\end{definition}
\begin{remark}
That is, the union of two sets is the set of elements in either set.
\end{remark}
\begin{example}[Examples of Unions]
\
\begin{enumerate}
\begin{multicols}{2}
\item $\{1,2,3\} \cup \{3,4,5\} = \{1,2,3,4,5\}$
\item $\{a,b,c\}\cup \{c,d,e\} = \{a,b,c,d,e\}$
\item $\{1,2,3\} \cup \{\textrm{Panda, Elephant}\} = \{1,2,3,\textrm{Panda, Elephant}\}$
\item $\{\textrm{Japan, Mathematics, Apple}\} \cup \{\textrm{Mars, Zelda}\} = \{\textrm{Japan, Mathematics, Apple, Mars, Zelda}\}$
\end{multicols}
\end{enumerate}
\end{example}
\begin{remark}
It is true that $\mathbb{N}\subset\mathbb{N}_{0}\subset\mathbb{Q}\subset \mathbb{R}$.
Also $\\mathbb{R}\setminus\mathbb{Q}\subset\mathbb{R}$,
and $\mathbb{Q}\cap\big(\\mathbb{R}\setminus\mathbb{Q}\big)=\emptyset$.
\end{remark}
We compare the size of numbers using the inequality symbols $<$ (Less than) and $>$ (Greater than).
\begin{properties}[Order Property of Real Numbers]
For two real numbers $a,b\in \mathbb{R}$:
\begin{enumerate}
\begin{multicols}{2}
\item $a<b$ if $a$ is to the left of $b$ on the number line.
\item $a>b$ if $a$ is to the right of $b$ on the number line.
\end{multicols}
\end{enumerate}
\end{properties}
\begin{definition}
A variable is a symbol used to represent an unknown quantity.
\end{definition}
\begin{example}
$x$ and $y$ are commonly used to represent real numbers. $n$ and $m$ are commonly used to represent integers. $z$ is often used to represent complex numbers, but we won't get into that until later.
\end{example}
\begin{example}
Let's use a variable to represent the sentence "To hit a baseball out of the park, the ball must travel more than 315 feet." Let $d$ be the distance the ball must travel to be a home run. Then $d>315$
\end{example}
\begin{notation}
If we wish to say $a$ is less than or equal to $b$, we write $a\leq b$. This means that either $a<b$ or $a=b$. Similarly, if we wish to write that $a$ is greater than or equal to $b$, we write $a\geq b$. This means that either $a>b$ or $a=b$.
\end{notation}
The absolute value of a real number is the distance from that number to the origin (The number $0$).
\begin{definition}
The absolute value of a real number $x\in \mathbb{R}$ is $|x| = \begin{cases} x, & x \geq 0 \\ -x, & x<0 \end{cases}$
\end{definition}
\begin{theorem}
If $a$ and $b$ are real numbers, then the following are true:
\begin{enumerate}
\begin{multicols}{4}
\item $|a+b| = |-a-b|$
\item $|a-b| = |b-a|$
\item $|-a| = |a|$
\item $|a\cdot b| = |a|\cdot |b|$
\end{multicols}
\end{enumerate}
\end{theorem}

\begin{example}[Examples of Absolute Value]
\
\begin{enumerate}
\begin{multicols}{4}
\item $|1| = 1$
\item $|-1| = 1$
\item $|0| = 0$
\item $|\pi| = \pi$
\item $|-\pi| = \pi$
\item $|\sqrt{2}| = \sqrt{2}$
\item $|-\sqrt{2}| = \sqrt{2}$
\item $|\frac{3}{4}| = \frac{3}{4}$
\item $|-\frac{3}{4}| = \frac{3}{4}$
\item $|\frac{3}{-4}| = \frac{3}{4}$.
\item $|(-3)\cdot4| = 12$
\item $|(-3)\cdot(-4)| = 12$
\end{multicols}
\end{enumerate}
\end{example}
\begin{definition}
The exponentiation of a real number $a\in \mathbb{R}$ by a natural number $n\in \mathbb{N}$ is the number $a^n = \underset{n\ times}{\underbrace{a\cdots a}}$.
\end{definition}
\begin{example}[Examples of Exponentiation]
\
\begin{enumerate}
\begin{multicols}{3}
\item $10^2 = 10\cdot 10 = 100$.
\item $10^4 = 10\cdot 10 \cdot 10 \cdot 10 = 10000$
\item $10^1 = 10$
\item $2^3 = 2\cdot 2 \cdot 2 = 8$
\item $\pi^2 = \pi\cdot \pi = 9.869\hdots$
\item $(\sqrt{2})^2 = \sqrt{2}\cdot \sqrt{2} = 2$.
\end{multicols}
\end{enumerate}
\end{example}
\begin{remark}
The fact that $(\sqrt{2})^2 = 2$ is really the definition of the number $\sqrt{2}$. We'll see this in a bit.
\end{remark}
\begin{theorem}
The following are true:
\begin{enumerate}
\begin{multicols}{3}
\item If $n\in \mathbb{N}$, then $0^n = 0$.
\item $(-1)^2 = 1$
\item $(-x)^2 = x^2$
\item $(-1)^3 = -1$
\item $(-x)^3=-x^3$
\item If $n$ is even, then $(-1)^n = 1$
\item if $n$ is even, then $(-x)^n = x^n$
\item If $n$ is odd, then $(-1)^n = -1$
\item If $n$ is odd, then $(-x)^n = -x^n$. 
\end{multicols}
\end{enumerate}
\end{theorem}
\begin{theorem}
If $y$ is a positive real number, then there is a unique positive real number $x$ such that $y=x^2$.
\end{theorem}
\begin{remark}
The theorem says there is a unique $positive$ real number. There are actually two real numbers satisfying this property. For if $y=x^2$, then $y=(-x)^2$, and thus $x$ and $-x$ are solutions. One of these will be negative, though.
\end{remark}
\begin{definition}
The principal square root of a positive real number $x$, denoted $\sqrt{x}$, is the unique positive real number such that $(\sqrt{x})^2 = x$. The symbol $\sqrt{\ \ }$ is called a radical, and $x$ is called the radicand.
\end{definition}
\begin{example}[Examples of Square Roots]
\
\begin{enumerate}
\begin{multicols}{4}
\item $\sqrt{1} = 1$
\item $\sqrt{4} = 2$
\item $\sqrt{9} = 3$
\item $\sqrt{16} = 4$
\item $\sqrt{25} = 5$
\item $\sqrt{36} = 6$
\item $\sqrt{49} = 7$
\item $\sqrt{64} = 8$
\end{multicols}
\end{enumerate}
\end{example}
\begin{theorem}
If $y\in \mathbb{R}$ is a real number, then there is a unique real number $x$ such that $y=x^3$.
\end{theorem}
\begin{definition}
The cube root of a real number $x$, denoted $\sqrt[3]{x}$, is the unique real number such that $(\sqrt[3]{x})^3 = x$.
\end{definition}
\begin{example}[Examples of Cube Roots]
\
\begin{enumerate}
\begin{multicols}{4}
\item $\sqrt[3]{-1} = -1$
\item $\sqrt[3]{8} = 2$
\item $\sqrt[3]{-8} = -2$
\item $\sqrt[3]{27} = 3$
\item $\sqrt[3]{125} = 5$
\item $\sqrt[3]{-125} = -5$
\item $\sqrt[3]{-64} = -4$
\item $\sqrt[3]{1000} = 10$
\end{multicols}
\end{enumerate}
\end{example}
\begin{remark}
The cube root theorem is more relaxed than the square root theorem. The square root theorem requires that $y$ is positive. Negative real numbers do not have square roots. However, all real number have cube roots.
\end{remark}
\begin{remark}
Note that for any positive real number $a$, $\big(\sqrt{a}\big)^2 = a$, and for any real number $b$, $(\sqrt[3]{b})^3 = b$.
\end{remark}
\begin{theorem}
If $r\in \mathbb{R}$ is positive and $n\in \mathbb{N}$, then there is a unique positive real number $\sqrt[n]{r}$, such that $(\sqrt[n]{r})^n = r$.
\end{theorem}
\begin{definition}
The principal $n^{th}$ root of a positive $r\in \mathbb{R}$ is the unique positive real number such that $(\sqrt[n]{r})^n = r$.
\end{definition}
\begin{example}[Examples of $n^{th}$ Roots]
\
\begin{enumerate}
\begin{multicols}{4}
\item $\sqrt{4} = 2$
\item $\sqrt[3]{27} = 3$
\item $\sqrt[4]{16} = 2$
\item $\sqrt[3]{125} = 5$
\item $\sqrt[3]{8} = 2$
\item $\sqrt[4]{81} = 3$
\item $\sqrt[5]{32} = 2$
\item $\sqrt[6]{729} = 3$
\end{multicols}
\end{enumerate}
\end{example}
\begin{properties}[The Order of Operations]
When performing arithmetic to simplify expressions, use the following order:
\begin{enumerate}
\item Perform operations inside parenthesis, brackets, braces, etc., first.
\item Next, perform exponentiation.
\item Then perform multiplication and division from left to right in the order they appear in the expression.
\item Finally perform addition and subtraction from left to right in the order they appear in the expression.
\end{enumerate}
\end{properties}
\begin{remark}
For some reason, the internet was once obsessed with the expression $48\div 2(9+3)$. Depending on what you do, you either got $288$ or $2$. According to the order of operations, the $correct$ answer is $288$. However the $real$ correct answer is: \textbf{If you write ambiguous expressions like this instead of using parantheses, then you are a bad person}. Parenthesis rid of ambiguity. $48\div\big(2(9+3)\big) = 2$, unambiguously. $\big(48\div 2\big)(9+3) = 288$, again unambiguously. Furthermore, stop using the $\div$ symbol. It's archaic and ambiguous. Writing $\frac{48}{2(9+3)}$ or $\frac{48}{2}(9+3)$ leaves no ambiguity.
\end{remark}
\subsubsection{Solved Problems}
\begin{enumerate}
\begin{multicols}{4}
\item $|-2.75| = 2.75$
\item $|-7.24| = 7.24$
\item $-|-4| = -4$
\item $-|-6| = -6$
\item $|3-(-6)| = 9$
\item $|-4-7| = 11$
\item $|-7.5-2.5| = 10$
\item $|13.4 - (-2.6)| = 16$
\item $|5-2| = 3$
\item $|-1-2| = 3$
\item $7^2 = 49$
\item $(-7)^2 = 49$
\item $-7^2 = -49$
\item $-(-7)^2 = -49$.
\item $3^3 = 27$
\item $(-3)^3 = -27$
\item $-(-3)^3 = 27$
\item $(-1)^2 = 1$
\item $(-1)^3 = -1$
\item $(-1)^4 = 1$
\item $(-1)^5 = -1$
\item $-24 - (-31) = 7$
\item $\frac{-\frac{3}{4}}{\frac{7}{8}} = -\frac{6}{7}$
\item $\frac{-20}{\frac{1}{2}} = -40$.
\end{multicols}
\end{enumerate}
\subsubsection{Algebraic Expressions and the Properties of Real Numbers}
\begin{definition}
An algebraic term is a collection of factors such as numbers, variables, or expressions within parentheses.
\end{definition}
\begin{example}[Examples of Algebraic Terms]
\
\begin{enumerate}
\begin{multicols}{6}
\item $3$
\item $-x$
\item $5xy$
\item $-3n^3$
\item $4y$
\item $2(x+3)$.
\end{multicols}
\end{enumerate}
\end{example}
\begin{definition}
The numerical value in a term is called its coefficient.
\end{definition}
\begin{example}[Examples of Coefficients]
\
\begin{enumerate}
\begin{multicols}{3}
\item $4$ is the coefficient of $4xy$
\item $3$ is the coefficient of $3z^2$
\item $1$ is the coefficient of $xyz$
\item $\frac{1}{\pi}$ is the coefficient of $\frac{1}{p}x^3$.
\item $1$ is the coefficient of $x$
\item $10$ is the coefficient of $10t^2$
\end{multicols}
\end{enumerate}
\end{example}
\begin{definition}
A constant is an algebraic term with only a numerical factor in it.
\end{definition}
\begin{definition}
A variable term is an algebraic term that contains a variable.
\end{definition}
\begin{example}[Examples of Variable Terms]
\
\begin{enumerate}
\begin{multicols}{4}
\item $x^2$
\item $2xyz^2$
\item $5t^4$
\item $3x(y+1)$
\item $t(x+y)(x-y)$
\item $3t^2y$
\item $x^4y^3z^2w$
\item $2\pi r$
\end{multicols}
\end{enumerate}
\end{example}
\begin{definition}
An algebraic expression is a single term or the sum of finitely many terms.
\end{definition}
\begin{example}
Let's translate the following sentences into mathematical expressions:
\begin{enumerate}
\begin{multicols}{2}
\item Twice a number, increased by $5$: $2n+5$
\item Six less than three times a number: $3n-6$
\end{multicols}
\end{enumerate}
\end{example}
\begin{properties}[Evaluating a Mathematical Expression]
To evaluate an expression, do the following:
\begin{enumerate}
\begin{multicols}{2}
\item Substitute the given values for each variable.
\item Simplify.
\end{multicols}
\end{enumerate}
\end{properties}
\begin{example}
Solve $x^3-2x^2+5$ for $x=-3$: $(-3)^3+2(-3)^2+5 = -27-18+5 = -40$.
\end{example}
\begin{properties}[The Commutative Properties]
If $a$ and $b$ are real numbers, then the following is true:
\begin{enumerate}
\item $a+b = b+a$ \hfill [The Commutative Property of Addition]
\item $a\cdot b = b\cdot a$ \hfill [The Commutative Property of Multiplication]
\end{enumerate}
\end{properties}
\begin{example}[Examples of the Commutative Properties]
\
\begin{enumerate}
\begin{multicols}{3}
\item $2+3 = 5,\ 3+2 = 5$
\item $2+(-8) = -6,\ (-8)+2 = - 6$
\item $5\cdot 6 = 30,\ 6\cdot 5 = 30$
\end{multicols}
\end{enumerate}
\end{example}
\begin{properties}[The Associative Properties]
If $a,b,c\in \mathbb{R}$, then the following is true:
\begin{enumerate}
\item $a+(b+c) = (a+b)+c$ \hfill [The Associative Property of Addition]
\item $a\cdot(b\cdot c) = (a\cdot b)\cdot c$ \hfill [The Associative Property of Multiplication]
\end{enumerate}
\end{properties}
\begin{example}[Examples of the Associative Properties]
\
\begin{enumerate}
\begin{multicols}{2}
\item $1+(2+3) = 1+5 = 6,\ (1+2)+3 = 3+3 = 6$
\item $5+(2+8) = 5+10 = 15,\ (5+2)+8 = 7+8 = 15$
\item $2\cdot(3\cdot 4) = 2\cdot 12 = 24,\ (2\cdot 3)\cdot 4 = 6\cdot 4 = 24$
\item $\frac{1}{2}\cdot(2\cdot 3) = \frac{1}{2}\cdot 6 = 3,\ (\frac{1}{2}\cdot 2)\cdot 3 = 1\cdot 3 = 3$.
\end{multicols}
\end{enumerate}
\end{example}
\begin{properties}[The Distributive Property of Multiplication over Addition]
If $a,b,c\in \mathbb{R}$, then the following is true:
\begin{enumerate}
\item $a\cdot(b+c) = (a\cdot b) + (a\cdot c)$\hfill [The Distributive Property of Multiplication over Addition]
\end{enumerate}
\end{properties}
\begin{example}[Examples of the Distributive Property]
\
\begin{enumerate}
\item $2\cdot(1+1) = 2\cdot 2 = 4,\ 2\cdot(1+1) = (2\cdot 1)+(2\cdot 1) = 2+2 = 4$
\item $2\cdot(2+3) = 2\cdot 5 = 10,\ 2\cdot(2+3) = (2\cdot 2)+(2\cdot 3) = 4+6 = 10$
\end{enumerate}
\end{example}
\begin{definition}
Like terms are two algebraic terms that have the same variable factors.
\end{definition}
\begin{example}
In $xy+x+2xy$, $xy$ and $2x$ are like terms so we can combine them to get $3xy+x$.
\end{example}
When trying to simplify an expression, we use the distributive property, the associative properties, and the commutative properties to combine like terms to get and form simplified expressions.
\begin{enumerate}
\begin{multicols}{2}
\item Seven fewer then a number: $n-7$
\item $x$ decreased by $6$: $x-6$
\item The number of a number and four: $n+4$
\item A number increased by $9$: $n+9$
\item The difference between a number and five is squared: $(n-5)^2$
\item The sum of a number and two is cubed: $(n+2)^3$
\item Thirteen less than twice a number: $2n-13$
\item Five less than double a number: $2n-13$
\end{multicols}
\item[] Let $x=2$ and $y=-3$. Evaluate:
\begin{multicols}{4}
\item $4x-2y: 14$
\item $5x-3y: 19$
\item $-2x^2+3y^2: 19$
\item $-5x^2+4y^2: 16$
\item $2y^2+5y-3: 0$
\item $3x^2+2x-5: 11$
\item $(2x-2y)^2:144$
\item $(2x-3y)^2: 169$
\end{multicols}
\end{enumerate}
\subsubsection{Exponents, Scientific Notation, and a Review of Polynomials}
\begin{notation}
If $a\in \mathbb{R}$, $a\ne 0$, and $n\in \mathbb{Z}$, $n<0$, then $a^n = \frac{1}{a^{|n|}}$.
\end{notation}
\begin{example}[Examples of Negative Exponents]
\
\begin{enumerate}
\begin{multicols}{4}
\item $2^{-1} = \frac{1}{2}$
\item $2^{-3} = \frac{1}{2^3} = \frac{1}{8}$
\item $10^{-1} =\frac{1}{10}$
\item $10^{-3} = \frac{1}{10^3} = \frac{1}{1000}$
\item $4^{-2} = \frac{1}{4^2} = \frac{1}{16}$
\item $\pi^{-1} = \frac{1}{\pi}$
\item $\pi^{-2} = \frac{1}{\pi^2}$
\item $\big(\frac{1}{2}\big)^{-1} = 2$
\item $\big(\frac{3}{2}\big)^{-1} = \frac{2}{3}$
\item $\big(\frac{1}{10}\big)^{-2} = 10^2 = 100$
\item $\big(\frac{1}{2}\big)^{-3} = 2^3 = 8$
\item $\big(\frac{3}{2}\big)^{-4} = \big(\frac{2}{3}\big)^4 = \frac{16}{81}$
\end{multicols}
\end{enumerate}
\end{example}
\begin{notation}
If $a \in \mathbb{R}$, $a>0$, and $p,q \in \mathbb{Z}$, then $a^{\frac{p}{q}}$ is the unique positive number such that $\big(a^{\frac{p}{q}}\big)^q = a^p$.
\end{notation}
\begin{example}[Examples of Fractional Exponents]
\
\begin{enumerate}
\begin{multicols}{4}
\item $2^{\frac{1}{2}} = \sqrt{2}$
\item $2^{\frac{1}{3}} = \sqrt[3]{2}$
\item $10^{\frac{1}{n}} = \sqrt[n]{10}$
\item $2^{\frac{3}{2}} = \sqrt{2^3} = \sqrt{8}$
\item $10^{\frac{4}{3}} = \sqrt[3]{10^4} = \sqrt[3]{10000}$
\item $3^{\frac{4}{5}} = \sqrt[5]{3^4} = \sqrt[5]{81}$
\item $2^{\frac{5}{6}} = \sqrt[6]{2^5} = \sqrt[6]{32}$
\item $6^{\frac{5}{5}} = \sqrt[5]{6^5} = 6$. 
\end{multicols}
\end{enumerate}
\end{example}
Thus, we have defined exponentiation for negative integers and fractions as well.
\begin{properties}[The Properties of Exponents]
If $a,b\in \mathbb{R}$ and $n,m,p\in \mathbb{N}$, then the following are true:
\begin{enumerate}
\item $\big(a^n\big)^m = a^{n\cdot m}$ \hfill [Power Property]
\item $a^n \cdot a^m = a^{n+m}$ \hfill [Product Property]
\item $\big(a^m\cdot b^n\big)^p = a^{m \cdot p} \cdot b^{n\cdot p}$ \hfill [Product to a Power Property]
\item If $b\ne 0$, $\big(\frac{a^n}{b^n}\big)^p = \frac{a^{n\cdot p}}{b^{n\cdot p}}$ \hfill [Quotient to a Power Property]
\item If $a\ne 0$, $\frac{a^n}{a^m} = a^{n-m}$ \hfill [Quotient Property]
\item If $a\ne 0$, $a^0 = 1$\hfill [The Zero Property]
\end{enumerate}
\end{properties}
\begin{remark}
WARNING: Some notation from calculus ahead. We leave $0^0$ undefined. In various scenarios in calculus, we use the convention that $0^0 = 1$, but this is no more than a convention. An example is in infinite series. Say we want to add $1+x+x^2+x^3+x^4+\hdots$. We use the notation $\sum_{n=0}^{\infty} x^n=x^0+x^1+x^2+x^3+\hdots$ to represent this. We plug in a value for $x$ and get a number. For every value of $x$ other than $x=0$, we have $x^0 = 1$ and so we get back the original sum. It would be really annoying to continuously talk about the special case when $x=0$, and so we adopt the convention that $0^0 =1$ as well. This is just a convention for this area of mathematics, and $0^0$ is, in general, left undefined.
\end{remark}
\begin{definition}
The scientific notation of a real number $x$ is $x = r\times 10^n$, where $0\leq |r| < 10$, and $n\in \mathbb{Z}$.
\end{definition}
Every real number has a scientific representation.
\begin{example}[Examples of Scientific Notation]
\
\begin{enumerate}
\begin{multicols}{3}
\item $101 = 1.01\times 10^2$
\item $10,000 = 1\times 10^4$
\item $314.15926\hdots = \pi \times 10^2$
\item $-123.456 = -1.23456\times 10^2$
\item $-0.031415926\hdots = -\pi \times 10^{-2}$
\item $0.00001 = 1\times 10^{-5}$
\end{multicols}
\item $0.\underset{33\ times}{\underbrace{0\hdots 0}662607004} = 6.62607004\times 10^{-34} = h$ (Physicist's like this number)
\end{enumerate}
\end{example}
\begin{definition}
A monomial is a term with only whole number variable exponents and no variables in the denominator.
\end{definition}
\begin{example}[Examples of Monomials]
\
\begin{enumerate}
\begin{multicols}{4}
\item $4x^2$
\item $3xyz$
\item $3y^2z$
\item $3z^2$
\end{multicols}
\end{enumerate}
\end{example}
\begin{definition}
A polynomial is a sum of monomials.
\end{definition}
\begin{example}[Examples of Polynomials]
\
\begin{enumerate}
\begin{multicols}{4}
\item $x^2+x+1$
\item $3xy+6z^2+w$
\item $x^2+2xy+y^2$
\item $1+xyz+x^2y^2z^2$
\end{multicols}
\end{enumerate}
\end{example}
\begin{definition}
The degree of a polynomial in one variable is the largest exponent of any of the terms.
\end{definition}
\begin{definition}
The degree of a polynomial in many variables is the largest sum of exponents of any of the terms.
\end{definition}
\begin{definition}
A binomial is a polynomial with two monomial terms.
\end{definition}

\begin{definition}
A trinomial is a polynomial with three monomial terms.
\end{definition}

\begin{example}
\
\begin{enumerate}
\begin{multicols}{2}
\item $5x^2y-2xy$ is a binomial of degree $3$.
\item $3x^2 - 1$ is a binomial of degree $2$.
\item $z^3-3z^2+9z-27$ is a polynomial of degree $3$.
\item $x+5$ is a binomial of degree $1$.
\item $2x^2+x+3$ is a trinomial of degree $2$.
\item $xyz+1$ is a binomial of degree $3$
\item $x^2yz +x+1$ is a trinomial of degree $4$
\item $x^2y^2z + 1$ is a binomial of degree $5$
\item $x^2y^2z^2 + 2$ is a binomial of degree $6$
\item $x^{10} y^{26} z^3 + 2x^2+1$ is a trinomial of degree $39$
\end{multicols}
\end{enumerate}
\end{example}
To add polynomials, we combine like terms and simplify using the commutative and associative properties.
\begin{example}[Adding Polynomials]
\
\begin{enumerate}
\item $(3x^2y+x+y)+(2x^2y-y) = 5x^2y+x$
\item $(1+x+x^2)+(x+x^2+x^3) = 1+2x+2x^2+x^3$
\item $(1+x)+(x+x^2) = 1+2x+x^2$
\item $(xyz+xy+x+z) + (y+xz+yz-xyz) = xy+xz+yz+x+y+z$
\end{enumerate}
\end{example}
To multiply two polynomials, we use the distributive property and then combine like terms.
\begin{example}[Multiplying Polynomials]
\
\begin{enumerate}
\begin{multicols}{2}
\item $xy(x+z) = x^2y+xyz$
\item $x(y+z) = xy+xz$
\item $(x+1)(y^2+z) = xy^2+xz+y^2+z$
\item $(x+y)(x-y) = x^2-y^2$.
\end{multicols}
\end{enumerate}
\end{example}
\begin{theorem}
If $A$ and $B$ are real numbers, the following is true:
\begin{enumerate}
\item $(A+B)(A-B) = A^2-B^2$ \hfill [Binomial Conjugates]
\item $(A+B)^2 = A^2 + 2AB + B^2$ \hfill [Square of a Sum]
\item $(A-B)^2 = A^2-2AB + B^2$ \hfill [Square of a Difference]
\end{enumerate}
\end{theorem}
\begin{enumerate}
\begin{multicols}{3}
\item $n^2\cdot 21n^5= 21n^7$
\item $5x^2 \cdot 7x^2= 35 x^4$
\item $(-6p^2q)(2p^3q^3)= -12p^5q^4$
\item $(a^2)^4\cdot (a^2)^3\cdot b^2\cdot b^5= a^{14}b^7$
\item $(6pq^2)^3= 216p^3q^6$
\item $\frac{-6 w^5}{-2 w^2}= 3w^3$
\item $\frac{8 z^7}{16 z^5}= \frac{1}{2}z^2$
\item $\frac{-12 a^3b^5}{4a^2b^4}= -3ab$
\item $\big(\frac{2}{3}\big)^{-3}= \frac{27}{8}$
\item $\frac{5m^3n^5}{10mn^2}= \frac{1}{2}m^2n^3$
\item $\frac{3}{m^{-2}}= 3m^2$
\item $\big(\frac{2p^4}{q^3}\big)^2= 4\frac{p^8}{q^6}$
\item $\big(\frac{-5 v^4}{7w^3}\big)^2= \frac{25 v^8}{49 w^6}$
\item $\frac{9p^6 q^4}{-12p^4q^6}= -\frac{3}{4} \frac{p^2}{q^2}$
\item $\frac{5m^2 n^2}{10 m^2 n}= \frac{1}{2} n$
\item $\frac{5k^3}{20 k^{-2}}= \frac{1}{4} k^5$
\item $\frac{7x^3}{x} = 7x^2$
\item $\frac{x^2y^3 z^4}{xyz} = xy^2z^3$
\end{multicols}
\item $(x^3+2x^2+x+1)+(3x^3-4x) = 4x^3+2x^2-3x+1$
\item $(xy+1)+(3x^2-2xy+4) = -xy+3x^2+5$
\begin{multicols}{2}
\item $2x(x^2+y^2) = 2x^3+2xy^2$
\item $(1+x)(1-x) = 1+x^2$
\item $(1+x+x^2)(1-x) = 1 - x^3$
\item $(1+x+x^2+x^3)(1-x) = 1 - x^4$
\item $(2x^2+3y)(x+y) = 2x^3+2x^2y+3xy+3y^2$
\item $xy(x+y)^2 = x^3y+2x^2y^2+xy^3$
\end{multicols}
\end{enumerate}
\subsubsection{Factoring Polynomials}
\begin{definition}
To factor an expression is to rewrite the expression as an equivalent product.
\end{definition}
The distributive property of multiplication over addition is an example of factoring.
\begin{example}
Let's factor $x^2+2xy+y^2$. Using the distributive property we obtain $x^2+2xy+y^2=x^2+xy+xy+y^2 = x(x+y)+y(x+y) = (x+y)(x+y) = (x+y)^2$.
\end{example}
\begin{example}
Let's factor $12x^2+18xy-30y$. Using the distributive property, $12x^2+12xy-30y = 6(2x^2+3xy-5y)$.
\end{example}
\begin{example}
$x^5+x^2 = x^2(x^3+1)$.
\end{example}
\begin{example}
Let's factor $3t^3+15t^2-6t-30$. Using the distributive property, we have $3\big(t^3+5t^2-2t-10\big) = 3\big(t^2(2t+5)-2(t+5)\big) = 3\big((t^2-2)(t+5)\big) = 3(t^2-2)(t+5)$.
\end{example}
\begin{example}
$(x+3)x^2+5(x+3) = (x+3)(x^2+5)$.
\end{example}
\begin{definition}
A quadratic polynomial is one of the form $ax^2+bx+c$, where $a,b,c\in \mathbb{R}$.
\end{definition}
There is a special case of quadratic polynomials where $a=1$. That is, quadratics of the form $x^2+bx+c$ for real numbers $b,c\in \mathbb{R}$. 
\begin{theorem}
If $x^2+bx+c$ is a quadratic, where $b,c\in \mathbb{R}$, and if $\alpha,\beta \in \mathbb{R}$ are real numbers such that $\alpha \cdot \beta = c$ and $\alpha+\beta = b$, then $x^2+bx+c = (x+\alpha)(x+\beta)$.
\end{theorem}
\begin{proof}
For $(x+\alpha)(x+\beta) = x^2+\alpha x + \beta x + \alpha\cdot \beta = x^2+x(\alpha + \beta) + \alpha \cdot \beta=x^2+bx+c$.
\end{proof}
\begin{example}
$x^2-11x+24 = (x-3)(x-8)$.
\end{example}
\begin{example}
$x^2-3x-10 = (x-5)(x+2)$
\end{example}
\begin{definition}
A prime polynomial is a polynomial that cannot be factored further.
\end{definition}
\begin{example}
$x^2+9x+15$ is a prime polynomial. We need $\alpha\cdot \beta = 15$ and $\alpha+\beta = 9$. Factoring $15$ into prime numbers, we get $15 = 5\cdot 3$ or $15 = 15\cdot 1$. In either case, $5+3=8\ne 9$ and $15+1 = 16 \ne 9$. So $x^2+9x+15$ cannot be factored further using integer coefficients.
\end{example}
For the general case of $a\ne 0$, we let $d = \frac{b}{a}$ and $e = \frac{c}{a}$. Then we find $\alpha$ and $\beta$ such that $(x+\alpha)(x+\beta) = x^2+dx+e$. Multiplying both sides by $a$ gives us $a(x+\alpha)(x+\beta) = ax^2+adx+ae = ax^2+bx+c$. So, $(ax+a\cdot \alpha)(x+\beta) = ax^2+bx+c$. So the general case reduces to the specific case of $a=1$.
We've seen the following identities before, and they can be used to simplify expressions:
\begin{enumerate}
\item $A^2-B^2 = (A+B)(A-B)$ \hfill [Difference of Squares]
\item $A^2+2AB+B^2 = (A+B)^2$ \hfill [Square of a Sum]
\item $A^2-2AB+B^2 = (A-B)^2$ \hfill [Square of a Difference]
\end{enumerate}
\begin{example}[Examples of Factoring]
\
\begin{enumerate}
\begin{multicols}{2}
\item $4x^2-81 = (2x+9)(2x-9)$
\item $x^2+49$ is prime.
\item $x^2 - 16 = (x+4)(x-4)$.
\item $x^2-9 = (x+3)(x-3)$
\item $x^4 - 81 = (x^2+9)(x^2-9) = (x^2+9)(x+3)(x-3)$.
\item $4x^2+8xy+4y^2 = (2x)^2+2(2x)(2y)+(2y)^2 = (2x+2y)^2$
\end{multicols}
\end{enumerate}
\end{example}
There are two identities that help us factor cubes very easily.
\begin{enumerate}
\item $x^3+y^3 = (x+y)(x^2-xy+y^2)$ \hfill [Sum of Cubes]
\item $x^3-y^3 = (x-y)(x^2+xy+y^2)$ \hfill [Difference of Cubes]
\end{enumerate}
\begin{example}[Examples of Factoring with Cubes]
\
\begin{enumerate}
\begin{multicols}{2}
\item $x^3+125 = (x+5)(x^2-5x+25)$
\item $5x^3y - 40y^4 = 5y\big(x^3-8y^3) = 5y(x-2y)(x^2+2xy+4y^2)$
\end{multicols}
\end{enumerate}
\end{example}
\subsubsection{Solved Problems}
\begin{enumerate}
\begin{multicols}{2}
\item $17x^2-51 = 17(x^2-3)$
\item $21x^3-13x^2+56x = 7x(3x^2-2x+8)$
\item $-3x^4+9x^2-6x^3 =-3x^2(x^2+2x-3)$
\item $-13x^2-52 = -13(x^2+4)$
\item $2x(x+2)+3(x+2) = (2x+3)(x+2)$
\item $(x^2+3)3x+(x^2+3)2 = (3x+2)(x^2+3)$
\item $5x(x-3)-2(x-3) = (5x-3)(x-3)$
\item $3x(x^2+5) - 3(x^2+5) = 3(x-1)(x^2+5)$
\item $-x^2 + 5x +15 = -(x-7)(x+2)$
\item $x^2-4x-45 = (x-9)(x+5)$
\item $x^2-9x+20 = (x-4)(x-5)$
\item $3x^2-13x-10 = (3x+2)(x-5)$
\item $6x^2+x-35 = (2x+5)(3x-7)$
\item $15x^2-22x-48 = (3x-8)(5x+6)$
\item $4x^2-25 = (2x+5)(2x-5)$
\item $50x^2-72 = 2(25x^2-36) = 2(5x+6)(5x-6)$
\item $8x^3-27 = (2x-3)(4x^2+6x+9)$
\item $x^3+8 = (x+2)(x^2-2x+4)$
\item $27x^3-64 = (3x-4)(9x^2+12x +16)$
\item $x^2-1 = (x+1)(x-1)$
\end{multicols}
\end{enumerate}
\subsubsection{Rational Expressions}
\begin{definition}
A rational expression is an expression that can be written as the quotient of two polynomials.
\end{definition}
\begin{example}[Examples of Rational Expressions]
\
\begin{enumerate}
\begin{multicols}{4}
\item $\frac{x^2+1}{x-1}$
\item $\frac{xy+4}{x^2+x+1}$
\item $\frac{x}{y}$
\item $\frac{x+y}{x-y}$
\end{multicols}
\end{enumerate}
\end{example}
\begin{definition}
The simplest form of a rational expression is an equivalent expression such that the numerator and denominator have no common factors.
\end{definition}
\begin{properties}[Fundamental Property of Rational Expressions]
If $P,Q,$ and $R$ are polynomials, $Q,R \ne 0$, then:
\begin{enumerate}
\item $\frac{P\cdot R}{Q\cdot R} = \frac{P}{Q}$\hfill [Simplification of Rational Expressions]
\end{enumerate}
\end{properties}
\begin{example}[Examples of Simplest Forms]
\
\begin{enumerate}
\begin{multicols}{2}
\item $\frac{x^2-1}{x+1} = \frac{(x+1)(x-1)}{x-1} = x+1$
\item $\frac{x^3-y^3}{x^2+xy+y^2} = \frac{(x-y)(x^2+xy+y^2)}{x^2+xy+y^2} = x-y$
\end{multicols}
\end{enumerate}
\end{example}
\begin{example}
$\frac{x^2-1}{x^2-3x+2} = \frac{(x+1)(x-1)}{(x-2)(x-1)} = \frac{x+1}{x-2}$
\end{example}
\begin{properties}[Multiplication of Rational Expressions]
If $P,Q,R,$ and $S$ are polynomials, $Q,S\ne 0$, then:
\begin{enumerate}
\item $\frac{P}{Q}\cdot \frac{R}{S} = \frac{PR}{QS}$
\end{enumerate}
\end{properties}
\begin{example}[Example of Multiplying Rational Expressions]
\
\begin{enumerate}
\begin{multicols}{2}
\item $\frac{2x+2}{3x-3x^2} \cdot \frac{3x^2-x-2}{9x^2-4} =\frac{-2(x+1)}{3x(3x-2}$
\item $\frac{x+1}{x^2-y^2}\frac{x+y}{x+1} = \frac{1}{x-y}$
\end{multicols}
\end{enumerate}
\end{example}
\begin{properties}[Dividing Rational Expressions]
If $P,Q,R,$ and $S$ are polynomials and $Q,R,S\ne 0$, then:
\begin{enumerate}
\item $\frac{P}{Q}\div \frac{R}{S} = \frac{\frac{P}{Q}}{\frac{R}{S}} = \frac{PS}{QR}$
\end{enumerate}
\end{properties}
\begin{example}[Examples of Dividing Rational Expressions]
\
\begin{enumerate}
\begin{multicols}{2}
\item $\frac{\frac{x+y}{y^2+1}}{\frac{x-y}{y^2+1}} = \frac{x+y}{y^2+1}\cdot \frac{y^2+1}{x-y}= \frac{x+y}{x-y}$
\item $\frac{\frac{xyz+y^2}{xy}}{\frac{y}{xy}} = \frac{xyz+y^2}{xy}\cdot \frac{xy}{y}= xz+y$
\end{multicols}
\end{enumerate}
\end{example}
\begin{properties}
If $P,Q,R,$ and $S$ are polynomials and $Q,S\ne 0$, then:
\begin{enumerate}
\item $\frac{P}{Q} + \frac{R}{S} = \frac{PS+QR}{QS}$ \hfill [Sum of Rational Expressions]
\item $\frac{P}{Q}-\frac{R}{S} = \frac{PS-QR}{QS}$\hfill [Difference of Rational Expressions]
\end{enumerate}
\end{properties}
\begin{example}[Addition and Subtraction of Rational Expressions]
\
\begin{enumerate}
\begin{multicols}{4}
\item $\frac{x}{y} + \frac{z}{w} = \frac{xw+yz}{yw}$
\item $\frac{x+1}{x^2} + \frac{x^2}{x-1} = \frac{x^2-1+x^2}{x^2(x-1)}$
\item $\frac{2x+1}{x^2} + 1 = \frac{x^2+2x+1}{x^2}$
\item $\frac{x}{y} - 1 = \frac{x-y}{y}$
\end{multicols}
\end{enumerate}
\end{example}
\begin{definition}
A compound fraction is a fraction whose numerator and denominator are also fractions.
\end{definition}
\begin{properties}[Simplifying Compound Fractions]
If $A,B,C,D,E,F,G,H$ are fractions, $B,D,F,H\ne 0$ and $\frac{E}{F}+\frac{G}{H} \ne 0$, then:
\begin{enumerate}
\item $\frac{\frac{A}{B}+\frac{C}{D}}{\frac{E}{F}+\frac{G}{H}} = \frac{FH(AD+BC)}{BD(EH+FG)}$ \hfill [Simplified Compound Fraction]
\end{enumerate}
\end{properties}
\begin{remark}
From this we see that all compound fractions are just normal fractions in disguise.
\end{remark}
\begin{example}
Simplify $\frac{\frac{2}{3x}-\frac{3}{2}}{\frac{3}{4x}-\frac{3}{x^2}}$. We have $\frac{\frac{2}{3x}-\frac{3}{2}}{\frac{3}{4x}-\frac{3}{x^2}} = \frac{8x-18x^2}{9x-4} = -2x\frac{9x-4}{9x-4} = -2x$, so long as $9x-4 \ne 0$.
\end{example}
\begin{example}
An electrical circuit with two resistors in parallel, with resistances $R_1$ and $R_2$, respectively, will have a total resistance $R$ which has the equation $\frac{1}{R} = \frac{1}{R_1}+\frac{1}{R_2}$. Thus $R = \frac{1}{\frac{1}{R_1}+ \frac{1}{R_2}} = \frac{R_1R_2}{R_1+R_2}$
\end{example}
\begin{enumerate}
\begin{multicols}{4}
\item $\frac{x-7}{-3x+21} = -\frac{1}{3}$
\item $\frac{2x+6}{4x^2-8x} = \frac{x+3}{2x(x-2)}$
\item $\frac{x-4}{7x-28} = \frac{1}{7}$
\item $\frac{x^2-5x-14}{x^2+6x-7}$ is simplified.
\item $\frac{x^2+3x-10}{x^2+x-6} = \frac{x+5}{x+3}$
\item $\frac{x-7}{7-x} = -1$
\item $\frac{x^2-3x-28}{49-x^2} = -\frac{x+4}{x+7}$
\item $\frac{12x^3y^5}{4x^2y^{-4}} = 3xy^9$
\item $\frac{7x+21}{63} = \frac{x+3}{9}$
\item $\frac{x^2-4}{2-x} = -(x+2)$
\item $\frac{x^3+8}{x^2-2x+4} = x+2$
\item $\frac{12x^2-13x+3}{27x^3-1} = \frac{4x-3}{9x^2+3x+1}$
\end{multicols}
\begin{multicols}{2}
\item $\frac{x^2-4x+4}{x^2-9}\cdot \frac{x^2-2x-3}{x^2-4} = \frac{(x-2)(x+1)}{(x+3)(x+2)}$
\item $\frac{x^2+5x-24}{x^2-6x+9}\cdot \frac{x}{x^2-64} = \frac{x}{(x-3)(x-8)}$
\end{multicols}
\end{enumerate}
\subsubsection{Radicals and Radical Expressions}
We have already seen the definition of $a^{\frac{p}{q}}$ for positive real numbers $a$, and $p,q\in \mathbb{Z}, q\ne 0$. Now for some results on radicals.
\begin{theorem}
For all $x\in \mathbb{R}$, $\sqrt{x^2} = |x|$
\end{theorem}
\begin{example}
$\sqrt{169x^2} = 13|x|$
\end{example}
\begin{theorem}
For all real numbers $x\in \mathbb{R}$, $\sqrt[3]{x^3} = x$
\end{theorem}
\begin{example}
$\sqrt[3]{-8} = -2$
\end{example}
\begin{remark}
The following is true:
\begin{equation}
\nonumber (x+y)^2 = x^2+2xy+y^2
\end{equation}
The following is \textbf{NOT TRUE}:
\begin{equation}
\nonumber (x+y)^2 = x^2 + y^2
\end{equation}
\textbf{DO NOT WRITE THIS, YOU WILL BE WRONG}.
Similarly, the following is true:
\begin{equation}
\nonumber \sqrt{(x+y)^2} = |x+y|
\end{equation}
The following is \textbf{NOT TRUE}:
\begin{equation}
\nonumber \sqrt{x^2+y^2} = |x|+|y|
\end{equation}
Again, \textbf{DO NO WRITE THIS, YOU WILL BE WRONG}.
Finally, the following is \textbf{NOT TRUE}:
\begin{equation}
\nonumber \sqrt{x+y} = \sqrt{x}+\sqrt{y}
\end{equation}
\textbf{DO NOT WRITE THIS, YOU WILL BE WRONG}.
\end{remark}
Recall that, for positive real numbers $x$, and $p,q\in \mathbb{Z},q\ne 0$, $x^{\frac{p}{q}}$ is the unique positive real number such that $(x^{\frac{p}{q}})^q = x^p$. In other words, $x^{\frac{p}{q}}=\sqrt[q]{x^p}$. Or equivalently, $x^{\frac{p}{q}} = \big(\sqrt[q]{x}\big)^p$
\begin{properties}
If $a,b$ are positive real numbers, $b\ne0$, and $n,m\in \mathbb{Z}, m\ne 0$, then:
\begin{enumerate}
\item $\sqrt[n]{a\cdot b} = \sqrt[n]{a}\cdot \sqrt[n]{b}$
\item $\sqrt[n]{\frac{a}{b}} = \frac{\sqrt[n]{a}}{\sqrt[n]{b}}$
\end{enumerate}
\end{properties}
\begin{remark}
Again, $\sqrt[n]{a+b} \ne \sqrt[n]{a}+\sqrt[n]{b}$. Do not make the mistake that equality holds here.
\end{remark}
To add, subtract, multiply, and divide with radicals, it is often useful to simplify the radical and then treat the remaining part as a variable. For example, consider $\sqrt{8} - \sqrt{2}$. We know that $8 = 4\cdot 2$, so $\sqrt{8} = \sqrt{4\cdot 2}$. But $\sqrt{4\cdot 2} = \sqrt{4}\cdot \sqrt{2}$. And we know $\sqrt{4} = 2$. So we have that $\sqrt{8} = 2\sqrt{2}$. Returning to the original expression, $\sqrt{8} - \sqrt{2} = 2\sqrt{2} - \sqrt{2}$. Factoring out the $\sqrt{2}$ (Like a variable) gives use $\sqrt{2}(2-1) = \sqrt{2}$. So $\sqrt{8}- \sqrt{2} = \sqrt{2}$/
\begin{example}[Examples of Arithmetic with Radicals]
\
\begin{enumerate}
\begin{multicols}{2}
\item $\sqrt{50} - \sqrt{8} = 5\sqrt{2}- 2\sqrt{2} = 3\sqrt{2}$
\item $\sqrt{54} - \sqrt{18} = 3\sqrt{2\cdot 3} - 3\sqrt{2}= 3\sqrt{2}(\sqrt{3}-1)$
\end{multicols}
\end{enumerate}
\end{example}
One application is that of Pythagoras' Theorem, which may be the most important theorem in all of mathematics. A right triangle is one where the largest angle in the triangle is $90^{\circ}$, or $\frac{\pi}{2}$ radians. The longest side of such a triangle is called the hypotenuse, and the other two sides are called the legs. If a right triangle has legs of lengths $a$ and $b$, and a hypotenuse of length $c$, then Pythagoras' Theorem says that $a^2+b^2 = c^2$. This can be used to find the hypotenuse if we only know the lengths of the legs. Since the length of the hypotenuse is a positive real number, if the lengths of the legs are $a$ and $b$, then $c = \sqrt{a^2+b^2}$. 
\begin{example}
A right angle triangle has one leg with length $3$ meters, and another with length $4$ meters. What is the length of the hypotenuse? Well, $c = \sqrt{(3)^2+(4)^2} = \sqrt{9+16} = \sqrt{25} = 5$.
\end{example}
\begin{remark}
If the denominator of an expression contains radicals, it is possible to create a rationalize equivalent expression (One without a radical in the denominator). Given an expression $A+\sqrt{B}$, where $A$ and $B$ are algebraic expressions, we note that $(\sqrt{A}+\sqrt{B})(\sqrt{A}-\sqrt{B}) = A-B$. 
\end{remark}
\begin{example}
Simplify:
\begin{enumerate}
\begin{multicols}{2}
\item $\sqrt{2}{2-\sqrt{6}} = \frac{2(2-\sqrt{6}}{(2-\sqrt{6})(2+\sqrt{6})} = \frac{1}{5}(2-\sqrt{6})$
\item $\frac{1}{\sqrt{2}-\sqrt{6}} = \frac{\sqrt{2}+\sqrt{6}}{(\sqrt{2}+\sqrt{6})(\sqrt{2}-\sqrt{6})} = \frac{\sqrt{2}+\sqrt{6}}{10}$
\end{multicols}
\end{enumerate}
\end{example}
\subsubsection{Solved Problems}
\begin{enumerate}
\begin{multicols}{4}
\item $12\sqrt{72} - 9\sqrt{98} = 9\sqrt{2}$
\item $8\sqrt{48} - 3\sqrt{108} = 14\sqrt{3}$
\item $7\sqrt{18x} - \sqrt{50x} = 16\sqrt{2x}$
\item $2\sqrt{28x} - 3\sqrt{63x} = -5\sqrt{7x}$
\end{multicols}
\end{enumerate}
\subsection{Equations and Inequalities}
\subsubsection{Linear Equations}
\begin{definition}
A family of equations is a set of equations that share a common characteristic.
\end{definition}
\begin{example}
The set of all quadratic polynomials in one variable is a family of equations. So is the set of all polynomials in three variables.
\end{example}
We wish to study the family of linear equations of one variable. These have the form $y=mx+b$.
\begin{definition}
An equations is a statement that two expressions are equal.
\end{definition}
We've already seen plenty of equations in the preliminaries section. There are many arithmetic properties that equations have.
\begin{properties}[Properties of Equality]
If $A,B,$ and $C$ are algebraic expressions, then:
\begin{enumerate}
\item If $A=B$, then $A+C=B+C$ for all $C$. \hfill [Additive Property of Equality]
\item If $A=B$, then $A\cdot C = B\cdot C$ \hfill [Multiplicative Property of Equality]
\item If $C \ne 0$, and $A=B$, then $\frac{A}{C} = \frac{B}{C}$ \hfill [Division Property of Equality]
\item If $C\ne 0$, and $A\cdot C = B\cdot C$, then $A=B$. \hfill [Cancellation Law of Equality]
\end{enumerate}
\end{properties}
\begin{remark}
Note that $A\cdot C = B\cdot C$ is not enough to say that $A=B$. If $C = 0$, then $2\cdot 0 = 0$ and $5\cdot 0 = 0$, so $2\cdot 0 = 5\cdot 0$, but $2\ne 5$. If $C = 0$, then we can say that $A=B$.
\end{remark}
\begin{example}
Solve for $x$: $3(x-1) +x = -x+7$. We have $3x-3+x = 4x-3 = -x+7$. Adding $x$ to both sides, $\big(4x-3)+x = \big(-x+7)+x$. So $5x-3 = 7$. Adding $3$ to both sides, $\big(5x-3\big)+3 = 7+3$, so $5x = 10$. Dividing by $5$, we get $x = \frac{10}{5} = 2$.
\end{example}
\begin{example}
Solve for for $n$: $\frac{1}{4}\big(n+8\big) = \frac{1}{2}\big(n-6\big)$: Multiplying both sides by $4$, we have $(n+8)-8 = 2(n-6)$, so $n = 2n-12$. Subtracting $2n$ from both sides, we get $-n = -12$. Multiplying both sides by $-1$, we get $n=12$.
\end{example}
The two previous examples are examples of conditional equations.
\begin{definition}
A conditional equation is an equation that is only true for certain values of the variables involved in the equation.
\end{definition}
\begin{definition}
An identity is an equation that is true of all values of the variables in the equation.
\end{definition}
\begin{example}
$2x + 4 = 2(x+2)$ is an identity. It is true of all values of $x$.
\end{example}
\begin{example}
$\sqrt{x^2}=|x|$ is an identity. It is true of all values of $x$.
\end{example}
\begin{definition}
A contradiction is an equation that is never true, regardless of the value the variables may take.
\end{definition}
\begin{example}
$x=x+1$ is a contradiction. For $x = x+1$ implies $0=1$, which is false.
\end{example}
\begin{example}
$2x - 5 = 2x$ is a contradiction. This would imply $0=5$, which is false.
\end{example}
\begin{definition}
A literal equation is an equation involving more than one variable.
\end{definition}
\begin{example}
The ideal gas law states that $PV=nRT$, where $P$ is the pressure, $V$ is the volume, $n$ is the number of moles in the system, and $T$ is the temperature. $R$ is a constant known as the universal gas constant. This is a literal equation in $4$ variables ($R$ is not a variable, it is a set constant). Solving for $P$, we get $P = \frac{nRT}{V}$.
\end{example}
\begin{example}
The amount of money $A$ earned from a simple interest model is $A = A_0+A_0Rt$, where $A_0$ is the initial deposit, $R$ is the
interest rate, and $t$ is the time elapsed since the deposit. Solve for $A_0$ in terms of $A,R,$ and $t$.
$A_{0}(1+Rt)=A$, so $A_{0}=\frac{A}{1+Rt}$
\end{example}
\begin{theorem}[General Solution to a Linear Equation]
If $a,b,c,d\in \mathbb{R}$, $a,c, \ne 0$, then:
\begin{enumerate}
\item If $ax+b=0$, then $x=-\frac{b}{a}$
\item If $ax+b=d$, then $x=\frac{c-b}{a}$
\item If $ax+b=cx+d$, and $a\ne c$, then $x =\frac{d-b}{a-c}$
\end{enumerate}
\end{theorem}
\begin{remark}
$ax+b=c$ is a linear equation in $x$. That is, it does not involve higher powers of $x$ ($x^{2},x^{3}$, etc.). It is a literal
equation, as $a$, $b$ and $c$ can be any real number (So long as $a\ne 0$), but we only care about solutions for $x$.
\end{remark}
\begin{example}
    Give a formula for the sum of three consecutive integers. Let $n$ be the smallest integer. The next integer is $n+1$, and the one
    thereafter if $n+2$. We are summing them, so we have $n+(n+1)+(n+2)$. From the associative and commutative properties of addition,
    we have $(n+n+n)+(1+2)=3n+3=3(n+1)$. So the sum of three consecutive integers, starting at $n$, is $3(n+1)$.
\end{example}
\subsubsection{Solved Problems}
\begin{enumerate}
    \begin{multicols}{2}
        \item $4x+3(x-2)=18-x:x=3$
        \item $15-2x=9-4(x+1):x=-5$
        \item $21-(2x+17)=-(7-x):x=-11$
        \item $12+5x=9+(6x+7):x=-4$
        \item $-3(4x+5)=-15x-20+3x:$ Contradiction.
        \item $5x-9-2=-5(2-x)-1:$ Identity
        \item $8-8(3x+5)=-5+6(x+1):x=-\frac{11}{10}$
        \item $-4(4x+5)=-6-2(8x+7):$ Identity
    \end{multicols}
\end{enumerate}
\subsubsection{Linear Inequalities in One Variable}
\begin{definition}
A linear inequality in one variable is an inequality involving two linear expressions in one variable.
\end{definition}
\begin{definition}
The solution set to an inequality is the set of real numbers that satisfy the inequality.
\end{definition}
\begin{example}
$x +1 > 5$ has the solution set $\{x:x>4\}$. That is, all real numbers greater than $4$.
\end{example}
\begin{notation}
The following notations are used to represents intervals of the real line. Let $a,b\in\mathbb{R}$, $a<b$:
    \begin{enumerate}
        \item $(a,\infty)=\{x:a<x<\infty\}$\hfill[Open Right-Half Line
        \item $[a,\infty)=\{x:a\leq x<\infty\}$\hfill[Closed Right-Half Line]
        \item $(a,b)=\{x:a<x<b\}$\hfill[Open Interval]
        \item $[a,b)=\{x:a\leq x<b\}$\hfill[Right Semi-Open Interval]
        \item $(a,b]=\{x:a<x\leq b\}$\hfill [Left Semi-Open Interval]
        \item $[a,b]=\{x:a\leq x\leq b\}$\hfill[Closed Interval]
        \item $(-\infty,b]=\{x:x\leq b\}$\hfill[Closed Left-Half Line]
        \item $(-\infty,b)=\{x:x<b\}$\hfill[Open Left-Half Line]
    \end{enumerate}
\end{notation}
\begin{properties}[Properties of Inequalities]
If $a<b$, $c\in\mathbb{R}$, then the following are true:
\begin{enumerate}
    \item $a+b<b+c$ \hfill [Additive Property of Inequalities]
    \item If $c>0$, then $a\cdot c<b\cdot c$ \hfill [Multiplicative Property of Inequalities]
    \item $-b<-a$ \hfill [Negation Property of Inequalities]
\end{enumerate}
\end{properties}
\begin{example}
$1<2$, and $1-5=-4<2-5=-3$. Also $\pi>0$, and thus $\pi\cdot 1<\pi\cdot 2$. We know that $3<4$, but multiplying
by $-1$ flips the inequality, and we get $-4<-3$ (Or $-3>-4)$.
\end{example}
\begin{definition}
A compound inequality is an inequality where the solution set is multiple intervals.
\end{definition}
\begin{example}
$-3x-1<-4$ or $4x+3<-6$. This has the requirement that $x>1$ or $x< -\frac{9}{4}$. To write this in interval notation,
we do $(-\infty,-\frac{9}{4})\cup (1,\infty)$
\end{example}
\begin{example}
$3x+5>-13$ and $3x+5<-1$. This has the requirement that $x>-6$ and $x<-2$. In interval notation,
we have $(-6,\infty)\cap (-\infty,-2)=(-6,-2)$.
\end{example}
\begin{example}
Solve $-6\leq\frac{2x+5}{-3}<1$. Multiplying by $-3$, we must flip the inequality to get $-3<2x+5\leq 18$.
Subtracting $5$, we have $-8<2x\leq 13$. Dividing by $2$, we have $-4 <x \leq\frac{13}{2}$. In interval notation,
that $(-4,\frac{13}{2}]$.
\end{example}
\begin{definition}
The domain of an expression is the set of values for which the expression is well defined.
\end{definition}
\begin{example}
$\frac{1}{x}$ is undefined at $0$, so its domain is $(-\infty, 0)\cup (0,\infty)$.
\end{example}
\begin{remark}
Remember from interval notation that $(-\infty,0)\cup (0,\infty)$ does not include the number $0$. It includes every
real number except for $0$.
\end{remark}
\subsubsection{Solved Problems}
Determine the domain of the following expressions in interval notation:
\begin{enumerate}
    \begin{multicols}{2}
        \item $\frac{12}{x}:(-\infty,0)\cup(0,\infty)$
        \item $\frac{5}{x+7}:(-\infty,-7)\cup(-7,\infty)$
        \item $\frac{1}{x-7}:(-\infty,7)\cup(7,\infty)$
        \item $\frac{4}{x-3}:(-\infty,3)\cup(3,\infty)$
    \end{multicols}
\end{enumerate}
\subsubsection{Absolute Value Equations and Inequalities}
Inequalities can involve the absolute value of a variable. When this happens, we must be careful when solving for the solution set.
\begin{properties}[Properties of Absolute Value Inequalities]
If $A$ and $B$ are algebraic expressions and $a>0$, then:
    \begin{enumerate}
        \item $|A|=a$ if and only if either $A=a$, or $A=-a$.
        \item $|A|>a$ if and only if either $A>a$ or $-A>a$
        \item $|A|<a$ if and only if $-a<A<a$
        \item $|A\cdot B|=|A|\cdot|B|$
    \end{enumerate}
\end{properties}
\begin{example}[Examples of Absolute Values in Inequalities]
\
    \begin{enumerate}
        \item $-5|x-7|+2=-13$ is equivalent to $|x-7|=-15$, and so $x-7=3$ or $x-7=-3$. The solution set is $\{4,10\}$.
        \item $|5-2x|=7$ implies either $5-2x=7$, or $5-2x=-7$. The solution set is $\{-1,6\}$.
        \item $|x|<7$ implies $-7<x<7$. The solution set is the interval $(-7,7)$.
        \item $|x-2|<7$ implies $-5<x<9$. The solution set is $(-5,9)$.
        \item $|x+1|>4$ implies $x+1>4$ or $x+1<-4$. The solution set is $(-\infty, -5)\cup (3,\infty)$
    \end{enumerate}
\end{example}
\subsubsection{Solved Problems}
Find the solution sets. Write the answer in set notation or using interval notation.
\begin{enumerate}
    \begin{multicols}{3}
        \item $2|x-1|-7=3:\{-4,6\}$
        \item $3|x-5|-14=-2:\{1,9\}$
        \item $-3|x+5|+6=-15:\{2,-12\}$
        \item $|x|=1:\{-1,1\}$
        \item $|x-2|\leq 7:[-5,9]$
        \item $|x-2|<7:(-5,9)$
        \item $5|x-2|-7\leq 8:[-1,5]$
        \item $-|x| > 2:\emptyset$
        \item $-|x|<1:\mathbb{R}$
    \end{multicols}
\end{enumerate}
\subsubsection{Complex Numbers}
There is no real number $x\in \mathbb{R}$ such that $x^2 = -1$. That's because the square of a real number is either zero, or
positive. In fact, $0$ is the only solution to $x^2 = 0$. Thus, every other real number has that property that its square is
positive. Solving the equation $x^2+1 = 0$ doesn't make sense in the realm of real numbers, and up until now we'd simply say
there is no solution. That is, there is no such thing is $\sqrt{-1}$. To solve this problem, we introduce imaginary and
complex numbers.
\begin{definition}
The imaginary unit $i$ is a number such that $i^{2}=-1$.
\end{definition}
\begin{remark}
Note that the imaginary unit is not a real number (Hence the name). It is a part of the larger complex numbers.
\end{remark}
\begin{notation}
If $r$ is a positive real number, we write $\sqrt{-r}=i\sqrt{r}$.
\end{notation}
\begin{remark}
Note that if $r$ is a negative real number, then $\sqrt{-r}=\sqrt{|r|}$, which is a real number.
\end{remark}
\begin{definition}
The principle square root of a negative real number $r$, is the complex number $i\sqrt{|r|}$.
\end{definition}
\begin{definition}
A complex number is a sum $a+ib$, where $a$ and $b$ are real numbers, and $i$ is the imaginary unit.
\end{definition}
\begin{definition}
The set of all complex number, denoted $\mathbb{C}$, is the set $\{a+ib:a,b\in \mathbb{R}\}$.
\end{definition}
\begin{example}[Examples of Complex Numbers]
\
\begin{enumerate}
    \begin{multicols}{4}
        \item $4+\sqrt{-49}=4+7i$
        \item $1-\sqrt{-1}=1-i$
        \item $25+\sqrt{-16}=25+4i$
        \item $1-\sqrt{-100}=1-10i$
    \end{multicols}
\end{enumerate}
\end{example}
\begin{remark}
Note that from the definition of complex numbers, $\mathbb{R}\subset \mathbb{C}$.
\end{remark}
The sum and difference of complex numbers is computed using the distributive property to group purely imaginary and purely
real components together. We multiply using the distributive property and the fact that $i^{2}=-1$.
\begin{example}[Examples of Complex Arithmetic]
\
\begin{enumerate}
    \begin{multicols}{4}
        \item $(2+3i)+(1+i)=3+4i$
        \item $(1+i)+(1-i)=2$
        \item $(1+i)(1-i)=1-i^{2}=1-(-1)=2$
        \item $i(2+3i)=-3+2i$
    \end{multicols}
\end{enumerate}
\end{example}
\begin{remark}
The powers of $i$ go in a cycle:
\begin{enumerate}
    \begin{multicols}{4}
        \item $i^{1}=i$
        \item $i^{2}=-1$ 
        \item $i^{3}=i\cdot i^{2}=-i$
        \item $i^{4}=i^{2}\cdot i^{2}=1$
        \item $i^{5}=i^{4}\cdot i=i$
        \item $i^{6}=i^{4}\cdot i^{2}=-1$
        \item $i^{7}=i^{4}\cdot i^{3}=-i$
        \item $i^{8}=i^{4}\cdot i^{4}=1$
    \end{multicols}
\end{enumerate}
\end{remark}
\begin{example}[Simplifying Powers of $i$]
\
\begin{enumerate}
    \begin{multicols}{3}
        \item $i^{22}=\big(i^{4}\big)^{5}\cdot i^{2}=1\cdot i^{2}=-1$
        \item $i^{57}=\big(i^{4}\big)^{14}\cdot i=1\cdot i=i$
        \item $i^{75}=\big(i^{4}\big)^{18}\cdot i^{3}=1\cdot i^{3}=i^{3}$
    \end{multicols}
\end{enumerate}
\end{example}
\begin{definition}
The complex conjugate of a complex number $z=a+ib$, denoted $\overline{z}$, is the complex number $\overline{z} = a-ib$.
\end{definition}
\begin{remark}
For a real number $z$, $\overline{z}=z$. This is because if $z=a+ib$ is real, then $b=0$.
\end{remark}
\begin{theorem}[The Complex Conjugate Theorem]
If $z=a+ib$ is a complex number, then $z\cdot\overline{z}=a^{2}+b^{2}$
\end{theorem}
\begin{proof}
For $z\cdot\overline{z}=(a+ib)(a-ib)=a^{2}+iab-iab-(ib)^{2}=a^{2}-(i)^{2}b^{2}=a^{2}+b^{2}$
\end{proof}
While $x^{2}+y^{2}$ cannot be simplified, in general, over the real numbers, it can by factored over the complex numbers.
If $x$ and $y$ are real numbers, then $x^{2}+y^{2}=(x+iy)(x-iy)$. This is just an application of the complex conjugate theorem.
We define division the following way: If $z=a+ib$, $w=c+id$, where $c,d\ne 0$, then
$\frac{z}{w}=\frac{z\cdot\overline{w}}{w\cdot\overline{w}}=\frac{(ac-bd)+i(ad+cb)}{c^{2}+d^{2}}$
\begin{example}[Examples of Division by Complex Numbers]
\
\begin{enumerate}
    \begin{multicols}{3}
        \item $\frac{2}{5-i}=\frac{5+i}{13}$
        \item $\frac{3-i}{2+i}=1-i$
        \item $\frac{6+6i}{3+3i}=2$
    \end{multicols}
\end{enumerate}
\end{example}
\subsubsection{Solved Problems}
\begin{enumerate}
    \begin{multicols}{3}
        \item $\sqrt{-16}=4i$
        \item $\sqrt{27}=3\sqrt{3}$
        \item $\sqrt{-81}=9i$
        \item $-\sqrt{-64}=-8i$
        \item $\sqrt{-49}=7i$
        \item $\sqrt{|-25|}=5$
        \item $\sqrt{-17}=i\sqrt{17}$
        \item $\sqrt{-\frac{9}{16}}=\frac{3}{4}i$
        \item $(12-2i)+(7+3i)=19+i$
        \item $(14+i)-(7+3i)=7-2i$
        \item $5+(1-i)=6-i$
        \item $(2+2i)+(-5-i)=-3+i$
        \item $(1+i)(2+i)=1+3i$
        \item $(4+i)(4-i)=17$
        \item $i(6-17i)=17+6i$
        \item $\frac{1+i}{1-i}=i$
        \item $i^{2}7=i^{3}$
        \item $i^{81}=i$
    \end{multicols}
\end{enumerate}
\subsubsection{Solving Quadratic Equations}
\begin{definition}
A quadratic equation is an equation of the form $ax^{2}+bx+c=0$, where $a,b,c\in\mathbb{R}$, and $a\ne 0$.
\end{definition}
\begin{remark}
All quadratic equations have degree $2$. The highest power that occurs is $x^{2}$, hence they are degree $2$.
\end{remark}
Quadratic equations often have two solutions, whereas linear equations have only one. There is a property of real numbers
that allows us to solve quadratic equations, called the Euclidean Property of Real Numbers.
\begin{properties}[The Euclidean Property of Real Numbers]
\
\begin{enumerate}
\item If $a,b\in \mathbb{R}$ and $a\cdot b = 0$, then either $a=0$ or $b=0$, or both.
\end{enumerate}
\end{properties}
\begin{remark}
This says that if the product of two numbers is zero, one of these numbers must be zero. It is named after Euclid of Alexandria,
one of the greatest mathematicians to ever live, who proved this in the context of geometry around $300$ B.C.
\end{remark}
\begin{example}
Solve $x^{2}=x$. We have that $x^{2}-x=0$, and thus $x(x-1)=0$. From the Euclidean property, either $x=0$ or $x-1=0$.
Thus, the solutions are $x=0$ and $x=1$.
\end{example}
\subsubsection{Completing the Square}
Recall that $(x+y)^2 = x^2+2xy+y^2$. Given $ax^2+bx+c$, we can reverse this process to obtain a simplified version. This process is called completing the square. Note that $ax^2+bx = a\big(x^2+\frac{b}{a}x\big) = a\big[\big(x+\frac{b}{2a}\big)^2-\frac{b^2}{4a^2}\big]$. So, $ax^2+bx+c = a\big[\big(x+\frac{b}{2a}\big)^2 - \frac{b^2}{4a^2}\big] + c$. Solving for $ax^2+bx+c=0$ is thus equivalent to solving $a\big[\big(x+\frac{b}{2a}\big)^2-\frac{b^2}{4a^2}\big]+c=0$. We get $\big(x+\frac{b}{2a}\big)^2=\frac{b^2}{4a^2} -\frac{c}{a} = \frac{b^2-4ac}{4a^2}$. Taking square roots, we have $x+\frac{b}{2a} = \pm\frac{\sqrt{b^2 - 4ac}}{2a}$, where $\pm$ means 'Plus or Minus.' Meaning both the $+$ symbol in front or the $-$ symbol in front yield correct answers. Finally, we get $x = \frac{-b\pm \sqrt{b^2-4ac}}{2a}$. With this, we have solved every quadratic equation possible (Remember for it to be quadratic $a$ cannot equal $0$).
\begin{theorem}[The Quadratic Formula]
    If $a,b,c\in\mathbb{R}$, $a\ne 0$, and $ax^{2}+bx+c=0$, then the solution set is
    $\{\frac{-b-\sqrt{b^{2}-4ac}}{2a},\frac{-b+\sqrt{b^{2}-4ac}}{2a}\}$
\end{theorem}
\begin{remark}
It is possible that the solution to a quadratic is complex or purely imaginary. This occurs when $b^{2}-4ac<0$ 
\end{remark}
\begin{definition}
The discriminant of a quadratic $ax^{2}+bx+c$ is the number $b^{2}-4ac$
\end{definition}
The discriminant has a useful property.
\begin{theorem}
    If $a,b,c\in\mathbb{R}$, $a\ne 0$, and $ax^{2}+bx+c=0$, then:
    \begin{enumerate}
        \item There are no real solutions is $b^{2}-4ac<0$
        \item There is one real solution if $b^{2}-4ac=0$
        \item There are two real solutions if $b^{2}-4ac>0$
    \end{enumerate}
\end{theorem}
By considering the discriminant, we can determine how many real solutions there are without having to solve the quadratic.
\end{document}