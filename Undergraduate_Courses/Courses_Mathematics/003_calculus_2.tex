\documentclass[crop=false,class=book]{standalone}
%---------------PREAMBLE------------%
%---------------------------Packages----------------------------%
\usepackage{geometry}
\geometry{b5paper, margin=1.0in}
\usepackage[T1]{fontenc}
\usepackage{graphicx, float}            % Graphics/Images.
\usepackage{natbib}                     % For bibliographies.
\bibliographystyle{agsm}                % Bibliography style.
\usepackage[french, english]{babel}     % Language typesetting.
\usepackage[dvipsnames]{xcolor}         % Color names.
\usepackage{listings}                   % Verbatim-Like Tools.
\usepackage{mathtools, esint, mathrsfs} % amsmath and integrals.
\usepackage{amsthm, amsfonts, amssymb}  % Fonts and theorems.
\usepackage{tcolorbox}                  % Frames around theorems.
\usepackage{upgreek}                    % Non-Italic Greek.
\usepackage{fmtcount, etoolbox}         % For the \book{} command.
\usepackage[newparttoc]{titlesec}       % Formatting chapter, etc.
\usepackage{titletoc}                   % Allows \book in toc.
\usepackage[nottoc]{tocbibind}          % Bibliography in toc.
\usepackage[titles]{tocloft}            % ToC formatting.
\usepackage{pgfplots, tikz}             % Drawing/graphing tools.
\usepackage{imakeidx}                   % Used for index.
\usetikzlibrary{
    calc,                   % Calculating right angles and more.
    angles,                 % Drawing angles within triangles.
    arrows.meta,            % Latex and Stealth arrows.
    quotes,                 % Adding labels to angles.
    positioning,            % Relative positioning of nodes.
    decorations.markings,   % Adding arrows in the middle of a line.
    patterns,
    arrows
}                                       % Libraries for tikz.
\pgfplotsset{compat=1.9}                % Version of pgfplots.
\usepackage[font=scriptsize,
            labelformat=simple,
            labelsep=colon]{subcaption} % Subfigure captions.
\usepackage[font={scriptsize},
            hypcap=true,
            labelsep=colon]{caption}    % Figure captions.
\usepackage[pdftex,
            pdfauthor={Ryan Maguire},
            pdftitle={Mathematics and Physics},
            pdfsubject={Mathematics, Physics, Science},
            pdfkeywords={Mathematics, Physics, Computer Science, Biology},
            pdfproducer={LaTeX},
            pdfcreator={pdflatex}]{hyperref}
\hypersetup{
    colorlinks=true,
    linkcolor=blue,
    filecolor=magenta,
    urlcolor=Cerulean,
    citecolor=SkyBlue
}                           % Colors for hyperref.
\usepackage[toc,acronym,nogroupskip,nopostdot]{glossaries}
\usepackage{glossary-mcols}
%------------------------Theorem Styles-------------------------%
\theoremstyle{plain}
\newtheorem{theorem}{Theorem}[section]

% Define theorem style for default spacing and normal font.
\newtheoremstyle{normal}
    {\topsep}               % Amount of space above the theorem.
    {\topsep}               % Amount of space below the theorem.
    {}                      % Font used for body of theorem.
    {}                      % Measure of space to indent.
    {\bfseries}             % Font of the header of the theorem.
    {}                      % Punctuation between head and body.
    {.5em}                  % Space after theorem head.
    {}

% Italic header environment.
\newtheoremstyle{thmit}{\topsep}{\topsep}{}{}{\itshape}{}{0.5em}{}

% Define environments with italic headers.
\theoremstyle{thmit}
\newtheorem*{solution}{Solution}

% Define default environments.
\theoremstyle{normal}
\newtheorem{example}{Example}[section]
\newtheorem{definition}{Definition}[section]
\newtheorem{problem}{Problem}[section]

% Define framed environment.
\tcbuselibrary{most}
\newtcbtheorem[use counter*=theorem]{ftheorem}{Theorem}{%
    before=\par\vspace{2ex},
    boxsep=0.5\topsep,
    after=\par\vspace{2ex},
    colback=green!5,
    colframe=green!35!black,
    fonttitle=\bfseries\upshape%
}{thm}

\newtcbtheorem[auto counter, number within=section]{faxiom}{Axiom}{%
    before=\par\vspace{2ex},
    boxsep=0.5\topsep,
    after=\par\vspace{2ex},
    colback=Apricot!5,
    colframe=Apricot!35!black,
    fonttitle=\bfseries\upshape%
}{ax}

\newtcbtheorem[use counter*=definition]{fdefinition}{Definition}{%
    before=\par\vspace{2ex},
    boxsep=0.5\topsep,
    after=\par\vspace{2ex},
    colback=blue!5!white,
    colframe=blue!75!black,
    fonttitle=\bfseries\upshape%
}{def}

\newtcbtheorem[use counter*=example]{fexample}{Example}{%
    before=\par\vspace{2ex},
    boxsep=0.5\topsep,
    after=\par\vspace{2ex},
    colback=red!5!white,
    colframe=red!75!black,
    fonttitle=\bfseries\upshape%
}{ex}

\newtcbtheorem[auto counter, number within=section]{fnotation}{Notation}{%
    before=\par\vspace{2ex},
    boxsep=0.5\topsep,
    after=\par\vspace{2ex},
    colback=SeaGreen!5!white,
    colframe=SeaGreen!75!black,
    fonttitle=\bfseries\upshape%
}{not}

\newtcbtheorem[use counter*=remark]{fremark}{Remark}{%
    fonttitle=\bfseries\upshape,
    colback=Goldenrod!5!white,
    colframe=Goldenrod!75!black}{ex}

\newenvironment{bproof}{\textit{Proof.}}{\hfill$\square$}
\tcolorboxenvironment{bproof}{%
    blanker,
    breakable,
    left=3mm,
    before skip=5pt,
    after skip=10pt,
    borderline west={0.6mm}{0pt}{green!80!black}
}

\AtEndEnvironment{lexample}{$\hfill\textcolor{red}{\blacksquare}$}
\newtcbtheorem[use counter*=example]{lexample}{Example}{%
    empty,
    title={Example~\theexample},
    boxed title style={%
        empty,
        size=minimal,
        toprule=2pt,
        top=0.5\topsep,
    },
    coltitle=red,
    fonttitle=\bfseries,
    parbox=false,
    boxsep=0pt,
    before=\par\vspace{2ex},
    left=0pt,
    right=0pt,
    top=3ex,
    bottom=1ex,
    before=\par\vspace{2ex},
    after=\par\vspace{2ex},
    breakable,
    pad at break*=0mm,
    vfill before first,
    overlay unbroken={%
        \draw[red, line width=2pt]
            ([yshift=-1.2ex]title.south-|frame.west) to
            ([yshift=-1.2ex]title.south-|frame.east);
        },
    overlay first={%
        \draw[red, line width=2pt]
            ([yshift=-1.2ex]title.south-|frame.west) to
            ([yshift=-1.2ex]title.south-|frame.east);
    },
}{ex}

\AtEndEnvironment{ldefinition}{$\hfill\textcolor{Blue}{\blacksquare}$}
\newtcbtheorem[use counter*=definition]{ldefinition}{Definition}{%
    empty,
    title={Definition~\thedefinition:~{#1}},
    boxed title style={%
        empty,
        size=minimal,
        toprule=2pt,
        top=0.5\topsep,
    },
    coltitle=Blue,
    fonttitle=\bfseries,
    parbox=false,
    boxsep=0pt,
    before=\par\vspace{2ex},
    left=0pt,
    right=0pt,
    top=3ex,
    bottom=0pt,
    before=\par\vspace{2ex},
    after=\par\vspace{1ex},
    breakable,
    pad at break*=0mm,
    vfill before first,
    overlay unbroken={%
        \draw[Blue, line width=2pt]
            ([yshift=-1.2ex]title.south-|frame.west) to
            ([yshift=-1.2ex]title.south-|frame.east);
        },
    overlay first={%
        \draw[Blue, line width=2pt]
            ([yshift=-1.2ex]title.south-|frame.west) to
            ([yshift=-1.2ex]title.south-|frame.east);
    },
}{def}

\AtEndEnvironment{ltheorem}{$\hfill\textcolor{Green}{\blacksquare}$}
\newtcbtheorem[use counter*=theorem]{ltheorem}{Theorem}{%
    empty,
    title={Theorem~\thetheorem:~{#1}},
    boxed title style={%
        empty,
        size=minimal,
        toprule=2pt,
        top=0.5\topsep,
    },
    coltitle=Green,
    fonttitle=\bfseries,
    parbox=false,
    boxsep=0pt,
    before=\par\vspace{2ex},
    left=0pt,
    right=0pt,
    top=3ex,
    bottom=-1.5ex,
    breakable,
    pad at break*=0mm,
    vfill before first,
    overlay unbroken={%
        \draw[Green, line width=2pt]
            ([yshift=-1.2ex]title.south-|frame.west) to
            ([yshift=-1.2ex]title.south-|frame.east);},
    overlay first={%
        \draw[Green, line width=2pt]
            ([yshift=-1.2ex]title.south-|frame.west) to
            ([yshift=-1.2ex]title.south-|frame.east);
    }
}{thm}

%--------------------Declared Math Operators--------------------%
\DeclareMathOperator{\adjoint}{adj}         % Adjoint.
\DeclareMathOperator{\Card}{Card}           % Cardinality.
\DeclareMathOperator{\curl}{curl}           % Curl.
\DeclareMathOperator{\diam}{diam}           % Diameter.
\DeclareMathOperator{\dist}{dist}           % Distance.
\DeclareMathOperator{\Div}{div}             % Divergence.
\DeclareMathOperator{\Erf}{Erf}             % Error Function.
\DeclareMathOperator{\Erfc}{Erfc}           % Complementary Error Function.
\DeclareMathOperator{\Ext}{Ext}             % Exterior.
\DeclareMathOperator{\GCD}{GCD}             % Greatest common denominator.
\DeclareMathOperator{\grad}{grad}           % Gradient
\DeclareMathOperator{\Ima}{Im}              % Image.
\DeclareMathOperator{\Int}{Int}             % Interior.
\DeclareMathOperator{\LC}{LC}               % Leading coefficient.
\DeclareMathOperator{\LCM}{LCM}             % Least common multiple.
\DeclareMathOperator{\LM}{LM}               % Leading monomial.
\DeclareMathOperator{\LT}{LT}               % Leading term.
\DeclareMathOperator{\Mod}{mod}             % Modulus.
\DeclareMathOperator{\Mon}{Mon}             % Monomial.
\DeclareMathOperator{\multideg}{mutlideg}   % Multi-Degree (Graphs).
\DeclareMathOperator{\nul}{nul}             % Null space of operator.
\DeclareMathOperator{\Ord}{Ord}             % Ordinal of ordered set.
\DeclareMathOperator{\Prin}{Prin}           % Principal value.
\DeclareMathOperator{\proj}{proj}           % Projection.
\DeclareMathOperator{\Refl}{Refl}           % Reflection operator.
\DeclareMathOperator{\rk}{rk}               % Rank of operator.
\DeclareMathOperator{\sgn}{sgn}             % Sign of a number.
\DeclareMathOperator{\sinc}{sinc}           % Sinc function.
\DeclareMathOperator{\Span}{Span}           % Span of a set.
\DeclareMathOperator{\Spec}{Spec}           % Spectrum.
\DeclareMathOperator{\supp}{supp}           % Support
\DeclareMathOperator{\Tr}{Tr}               % Trace of matrix.
%--------------------Declared Math Symbols--------------------%
\DeclareMathSymbol{\minus}{\mathbin}{AMSa}{"39} % Unary minus sign.
%------------------------New Commands---------------------------%
\DeclarePairedDelimiter\norm{\lVert}{\rVert}
\DeclarePairedDelimiter\ceil{\lceil}{\rceil}
\DeclarePairedDelimiter\floor{\lfloor}{\rfloor}
\newcommand*\diff{\mathop{}\!\mathrm{d}}
\newcommand*\Diff[1]{\mathop{}\!\mathrm{d^#1}}
\renewcommand*{\glstextformat}[1]{\textcolor{RoyalBlue}{#1}}
\renewcommand{\glsnamefont}[1]{\textbf{#1}}
\renewcommand\labelitemii{$\circ$}
\renewcommand\thesubfigure{%
    \arabic{chapter}.\arabic{figure}.\arabic{subfigure}}
\addto\captionsenglish{\renewcommand{\figurename}{Fig.}}
\numberwithin{equation}{section}

\renewcommand{\vector}[1]{\boldsymbol{\mathrm{#1}}}

\newcommand{\uvector}[1]{\boldsymbol{\hat{\mathrm{#1}}}}
\newcommand{\topspace}[2][]{(#2,\tau_{#1})}
\newcommand{\measurespace}[2][]{(#2,\varSigma_{#1},\mu_{#1})}
\newcommand{\measurablespace}[2][]{(#2,\varSigma_{#1})}
\newcommand{\manifold}[2][]{(#2,\tau_{#1},\mathcal{A}_{#1})}
\newcommand{\tanspace}[2]{T_{#1}{#2}}
\newcommand{\cotanspace}[2]{T_{#1}^{*}{#2}}
\newcommand{\Ckspace}[3][\mathbb{R}]{C^{#2}(#3,#1)}
\newcommand{\funcspace}[2][\mathbb{R}]{\mathcal{F}(#2,#1)}
\newcommand{\smoothvecf}[1]{\mathfrak{X}(#1)}
\newcommand{\smoothonef}[1]{\mathfrak{X}^{*}(#1)}
\newcommand{\bracket}[2]{[#1,#2]}

%------------------------Book Command---------------------------%
\makeatletter
\renewcommand\@pnumwidth{1cm}
\newcounter{book}
\renewcommand\thebook{\@Roman\c@book}
\newcommand\book{%
    \if@openright
        \cleardoublepage
    \else
        \clearpage
    \fi
    \thispagestyle{plain}%
    \if@twocolumn
        \onecolumn
        \@tempswatrue
    \else
        \@tempswafalse
    \fi
    \null\vfil
    \secdef\@book\@sbook
}
\def\@book[#1]#2{%
    \refstepcounter{book}
    \addcontentsline{toc}{book}{\bookname\ \thebook:\hspace{1em}#1}
    \markboth{}{}
    {\centering
     \interlinepenalty\@M
     \normalfont
     \huge\bfseries\bookname\nobreakspace\thebook
     \par
     \vskip 20\p@
     \Huge\bfseries#2\par}%
    \@endbook}
\def\@sbook#1{%
    {\centering
     \interlinepenalty \@M
     \normalfont
     \Huge\bfseries#1\par}%
    \@endbook}
\def\@endbook{
    \vfil\newpage
        \if@twoside
            \if@openright
                \null
                \thispagestyle{empty}%
                \newpage
            \fi
        \fi
        \if@tempswa
            \twocolumn
        \fi
}
\newcommand*\l@book[2]{%
    \ifnum\c@tocdepth >-3\relax
        \addpenalty{-\@highpenalty}%
        \addvspace{2.25em\@plus\p@}%
        \setlength\@tempdima{3em}%
        \begingroup
            \parindent\z@\rightskip\@pnumwidth
            \parfillskip -\@pnumwidth
            {
                \leavevmode
                \Large\bfseries#1\hfill\hb@xt@\@pnumwidth{\hss#2}
            }
            \par
            \nobreak
            \global\@nobreaktrue
            \everypar{\global\@nobreakfalse\everypar{}}%
        \endgroup
    \fi}
\newcommand\bookname{Book}
\renewcommand{\thebook}{\texorpdfstring{\Numberstring{book}}{book}}
\providecommand*{\toclevel@book}{-2}
\makeatother
\titleformat{\part}[display]
    {\Large\bfseries}
    {\partname\nobreakspace\thepart}
    {0mm}
    {\Huge\bfseries}
\titlecontents{part}[0pt]
    {\large\bfseries}
    {\partname\ \thecontentslabel: \quad}
    {}
    {\hfill\contentspage}
\titlecontents{chapter}[0pt]
    {\bfseries}
    {\chaptername\ \thecontentslabel:\quad}
    {}
    {\hfill\contentspage}
\newglossarystyle{longpara}{%
    \setglossarystyle{long}%
    \renewenvironment{theglossary}{%
        \begin{longtable}[l]{{p{0.25\hsize}p{0.65\hsize}}}
    }{\end{longtable}}%
    \renewcommand{\glossentry}[2]{%
        \glstarget{##1}{\glossentryname{##1}}%
        &\glossentrydesc{##1}{~##2.}
        \tabularnewline%
        \tabularnewline
    }%
}
\newglossary[not-glg]{notation}{not-gls}{not-glo}{Notation}
\newcommand*{\newnotation}[4][]{%
    \newglossaryentry{#2}{type=notation, name={\textbf{#3}, },
                          text={#4}, description={#4},#1}%
}
%--------------------------LENGTHS------------------------------%
% Spacings for the Table of Contents.
\addtolength{\cftsecnumwidth}{1ex}
\addtolength{\cftsubsecindent}{1ex}
\addtolength{\cftsubsecnumwidth}{1ex}
\addtolength{\cftfignumwidth}{1ex}
\addtolength{\cfttabnumwidth}{1ex}

% Indent and paragraph spacing.
\setlength{\parindent}{0em}
\setlength{\parskip}{0em}
%---------------GLOSSARY------------%
\makeglossaries
\loadglsentries{../../glossary}
\loadglsentries{../../acronym}
%--------------Title Page-----------%
\begin{document}
\chapter{Calculus II}
\section{Exams}
\subsection{Practice Exam I}
\begin{problem}
Use the Midpoint Rule to estimate the area under the graph of $f(x)=16-x^{2}$ between $x=-4$ and $x=4$ using four rectangles. Estimate the average of $f$ on the same interval.
\end{problem}
\begin{proof}[Solution]
\
\begin{table}[H]
    \centering
    \begin{tabular}{|c|c|c|c|}
        \hline
        Start&End&\# Pts&$\Delta x$\\
        \hline
        $a=-4$&$b=4$&$n=4$&$\frac{b-a}{n}=2$\\
        \hline
    \end{tabular}
\end{table}
The midpoints are $m_{n}=\frac{x_{n-1}+x_{n}}{2}$, where $x_{0}=a$, and $x_{n}=x_{n-1}+\Delta x$.
\begin{table}[H]
    \centering
    \begin{tabular}{|c|c|c|c|}
        \hline
        $m_{1}$&$m_{2}$&$m_{3}$&$m_{4}$\\
        \hline
        $-3$&$-1$&$1$&$3$\\
        \hline
    \end{tabular}
\end{table}
The midpoint rule says $A_{M}=\sum_{k=1}^{n}f(m_{k})\Delta x$
\begin{table}[H]
    \centering
    \begin{tabular}{|c|c|c|c|}
        \hline
        $f(m_{1})$&$f(m_{2})$&$f(m_{3})$&$f(m_{4})$\\
        \hline
        $7$&$15$&$15$&$7$\\
        \hline
    \end{tabular}
\end{table}
So, $A_{M}=2(7+15+15+7)=2(44)=88$. The estimated average of $f$ is $\frac{A_{m}}{b-a}=\frac{88}{8}=11$
\end{proof}
\begin{problem}
Using the Fundamental Theorem of Calculus, find $\frac{dy}{dx}$:
\begin{enumerate}
    \begin{multicols}{2}
        \item $y=\int_{\sin(x)}^{0}\frac{4dt}{\sqrt{1-t^{2}}}$
        \item $y=\int_{3+x^{2}}^{6}\frac{t}{1+e^{t}}dt$
    \end{multicols}
\end{enumerate}
\end{problem}
\begin{proof}[Solution]
\
\begin{enumerate}
    \item   \begin{align*}
    y&=\int_{\sin(x)}^{0}\frac{4dt}{\sqrt{1-t^{2}}}=-\int_{0}^{\sin(x)}\frac{4dt}{\sqrt{1-t^{2}}}\Rightarrow\frac{dy}{dx}=-\frac{d}{dx}\int_{0}^{\sin(x)}\frac{4dt}{\sqrt{1-t^{2}}}\\
    \Rightarrow\frac{dy}{dx}&=-\frac{4}{\sqrt{1-\sin^{2}(x)}}\frac{d}{dx}(\sin(x))=-\frac{4\cos(x)}{|\cos(x)|}=-4\sgn(x)
            \end{align*}
    \item   \begin{align*}
    y&=\int_{3+x^{2}}^{6}\frac{t}{1+e^{t}}dt=-\int_{6}^{3+x^{2}}\frac{t}{1+e^{t}}dt\Rightarrow\frac{dy}{dx}=-\frac{d}{dx}\int_{6}^{3+x^{2}}\frac{t}{1+e^{t}}dt\\
    &=-\frac{3+x^{2}}{1+e^{3+x^{2}}}\frac{d}{dx}(3+x^{2})=-\frac{2x(3+x^{2})}{1+e^{3+x^{2}}}
            \end{align*}
\end{enumerate}
\end{proof}
\newpage
\begin{problem}
Evaluate $\int_{-2}^{2}(x^{2}+e^{2})dx$
\end{problem}
\begin{proof}[Solution]
\begin{equation*}
    \int_{-2}^{2}(x^{2}+e^{2})dx=\frac{1}{3}x^{3}+e^{2}x\big|_{-2}^{2}=\frac{16}{3}+4e^{2}
\end{equation*}
\end{proof}
\begin{problem}
\
\begin{enumerate}
    \begin{multicols}{3}
        \item $\int_{0}^{1}\frac{4x^{3}}{(x^{4}-3)^{2}}dx$
        \item $\int e^{5x}(e^{5x}+4)^{2}dx$
        \item $\int\cos^{3}(3x)\sin(3x)dx$
    \end{multicols}
\end{enumerate}
\end{problem}
\begin{proof}[Solution]
\
\begin{enumerate}
    \item   \begin{align*}
    u=x^{4}-3,du&=4x^{3}dx,x=0\Rightarrow u=-3,x=1\Rightarrow u=-2\\
    \int_{0}^{1}\frac{4x^{3}}{(x^{4}-3)^{2}}dx&=\int_{-3}^{-2}\frac{du}{u^{2}}=-\frac{1}{u}\big|_{-3}^{-2}=\frac{1}{6}
            \end{align*}
    \item   \begin{align*}
    u=e^{5x}+4,du&=5e^{5x}dx\Rightarrow e^{5x}dx=\frac{1}{5}du\\
    \int e^{5x}(e^{5x}+4)^{2}dx&=\frac{1}{5}\int u^{2}du=\frac{1}{15}u^{3}+C=\frac{1}{15}(e^{5x}+4)^{3}+C
            \end{align*}
    \item   \begin{align*}
    u=\cos(3x),du&=-3\sin(3x)dx\Rightarrow -\frac{1}{3}du=\sin(3x)dx\\
    \int \cos^{3}(3x)\sin(3x)dx&=-\frac{1}{3}\int u^{3}du=-\frac{1}{12}u^{4}+C=-\frac{1}{12}\cos^{4}(3x)+C
            \end{align*}
\end{enumerate}
\end{proof}
\begin{problem}
Find the area of the region bounded by the curves $f(x)=-x^{2}+4x-2$ and $g(x)=x^{2}-2$
\end{problem}
\begin{proof}[Solution]
The points of intersection are: $x^{2}-2=-x^{2}+4x-2\Rightarrow 2x(x-2)=0\Rightarrow x=0,x=2$.
The area between the curves is $A=\int_{0}^{2}(f(x)-g(x))dx$:
\begin{equation*}
    A=\int_{0}^{2}(f(x)-g(x))dx=\int_{0}^{2}(-2x^{2}+4x)dx=-\frac{2}{3}x^{3}+2x^{2}\big|_{0}^{2}=\frac{8}{3}
\end{equation*}
\end{proof}
\begin{problem}
Evaluate $\int xe^{2x}dx$
\end{problem}
\begin{proof}[Solution]
\
\begin{table}[H]
    \centering
    \begin{tabular}{|c|c|}
        \hline
        $u=x$&$dv=e^{2x}dx$\\
        \hline
        $du=dx$&$v=\frac{1}{2}e^{2x}$\\
        \hline
    \end{tabular}
\end{table}
\begin{equation*}
    \int udv=uv-\int vdu\Rightarrow \int xe^{2x}dx=\frac{1}{2}xe^{2x}-\frac{1}{2}\int e^{2x}dx=\frac{1}{2}xe^{2x}-\frac{1}{4}e^{2x}+C
\end{equation*}
\end{proof}
\begin{problem}
Evaluate $\int_{0}^{\pi^{2}}\cos(\sqrt{x})dx$
\end{problem}
\begin{proof}[Solution]
Let $y=\sqrt{x}$. $dy=\frac{1}{2\sqrt{x}}dx\Rightarrow dx=2\sqrt{x}dy=2ydy\Rightarrow\int\cos(\sqrt{x})dx=2\int y\cos(y)dy$
\begin{table}[H]
    \centering
    \begin{tabular}{|c|c|}
        \hline
        $u=y$&$dv=\cos(y)dy$\\
        \hline
        $du=dy$&$v=\sin(y)$\\
        \hline
    \end{tabular}
\end{table}
\begin{align*}
    \int\cos(\sqrt{x})dx&=2\int y\cos(y)dy=2\big(y\sin(y)-\int\sin(y)dy\big)=2y\sin(y)+2\cos(y)\\
    \Rightarrow \int_{0}^{\pi^{2}}\cos(\sqrt{x})dx&=2\sqrt{x}\cos(\sqrt{x})+2\cos(\sqrt{x})\big|_{0}^{\pi^{2}}=-4
\end{align*}
\end{proof}
\subsection{Exam I}
\begin{problem}
Using the Fundamental Theorem of Calculus, find $\frac{dy}{dx}$:
\begin{enumerate}
    \begin{multicols}{2}
        \item $y=\int_{\sqrt{x}}^{0}\cos(t^{4})dt$
        \item $y=\int_{5+2x^{2}}^{6}(e^{t}+2t)dt$
    \end{multicols}
\end{enumerate}
\end{problem}
\begin{proof}[Solution]
\
\begin{enumerate}
    \item   \begin{align*}
        y&=\int_{\sqrt{x}}^{0}\cos(t^{4})dt=-\int_{0}^{\sqrt{x}}\cos(t^{4})dt\Rightarrow     \frac{dy}{dx}=-\frac{d}{dx}\int_{0}^{\sqrt{x}}\cos(t^{4})dt\\
        &=\cos(x^{2})\frac{d}{dx}(\sqrt{x})=-\frac{\cos(x^{2})}{2\sqrt{x}}
            \end{align*}
    \item   \begin{align*}
        y&=\int_{5+2x^{2}}^{6}(e^{t}+2t)dt=-\int_{6}^{5+2x^{2}}(e^{t}+2t)dt\Rightarrow \frac{dy}{dx}=-\frac{d}{dx}\int_{6}^{5+2x^{2}}(e^{t}+2t)dt\\
        &=-[e^{5+2x^{2}}+2(5+2x^{2})]\frac{d}{dx}(5+2x^{2})=-4x[e^{5+2x^{2}}+10+4x^{2}]
            \end{align*}
\end{enumerate}
\end{proof}
\begin{problem}
Evaluate the integral: $\int(\frac{3}{x}-\frac{2}{x^{3}}-\sqrt[3]{x}+2e^{2x}+\cos(\pi x)-\pi)dx$
\end{problem}
\begin{proof}[Solution]
$\int(\frac{3}{x}-\frac{2}{x^{3}}-\sqrt[3]{x}+2e^{2x}+\cos(\pi x)-\pi)dx=3\ln(|x|)+\frac{1}{x^{2}}-\frac{3}{4}x^{\frac{4}{3}}+e^{2x}+\frac{\sin(\pi x)}{\pi}-\pi x+C$
\end{proof}
\begin{problem}
Evaluate the following integrals:
\begin{enumerate}
    \begin{multicols}{4}
        \item $\int_{0}^{2}3x^{2}\sqrt{x^{3}+8}dx$
        \item $\int\frac{e^{3y}}{e^{3y}-5)^{2}}dy$
        \item $\int\frac{1}{\arctan^{2}(x)(1+x^{2})}dx$
        \item $\int_{0}^{\pi/4}\sin^{3}(2\theta)\cos(2\theta)d\theta$
    \end{multicols}
\end{enumerate}
\end{problem}
\begin{proof}[Solution]
\
\begin{enumerate}
    \item   \begin{align*}
    u&=x^{3}+8, du=3x^{2}dx\\
    \int 3x^{2}\sqrt{x^{3}+8}dx&=\int\sqrt{u}du=\frac{3}{2}u^{\frac{3}{2}}+C=\frac{3}{2}(x^{3}+8)^{\frac{3}{2}}+C\\
    \Rightarrow\int_{0}^{2}3x^{2}\sqrt{x^{3}+8}dx&=\frac{3}{2}(x^{3}+8)^{\frac{3}{2}}\big|_{0}^{2}=\frac{3}{2}[(2^{2}+8)^{\frac{3}{2}}-(8)^{\frac{3}{2}}]=\frac{32}{3}(4-\sqrt{2})
            \end{align*}
    \item   \begin{align*}
    u&=e^{3y}-5,du=3e^{3y}dy\Rightarrow e^{3y}dy=\frac{1}{3}du\\
    \int\frac{e^{3y}}{(e^{3y}-5)^{3}}dy&=\frac{1}{3}\int\frac{du}{u^{3}}=\frac{1}{3}\int u^{-3}du=-\frac{1}{6}u^{-2}+C=-\frac{1}{6}(e^{3y}-5)^{-2}+C
            \end{align*}
    \item   \begin{align*}
    u&=\arctan(x),du=\frac{1}{1+x^{2}}dx\\
    \int\frac{1}{\arctan^{2}(x)(1+x^{2})}dx&=\int\frac{du}{u^{2}}=-\frac{1}{u}+C=-\frac{1}{\arctan(x)}+C
            \end{align*}
    \item   \begin{align*}
    u=\sin(2\theta),du=2\cos(2\theta)d\theta\Rightarrow\cos(2\theta)d\theta&=\frac{1}{2}du,\theta=0\Rightarrow u=0,\theta=\frac{\pi}{4}\Rightarrow u=1\\
    \int_{0}^{\frac{\pi}{4}}\sin^{3}(2\theta)\cos(2\theta)d\theta&=\frac{1}{2}\int_{0}^{1}u^{3}du=\frac{1}{8}u^{4}\big|_{0}^{1}=\frac{1}{8}
            \end{align*}
\end{enumerate}
\end{proof}
\begin{problem}
FInd the area of the region bounded by the curves of $y=x^{2}+4$ and $y=-x^{2}+6x+4$
\end{problem}
\begin{proof}[Solution]
We first find the points of intersection of the two curves:
\begin{equation*}
    x^{2}+4=-x^{2}+6x+4\Rightarrow2x^{2}-6x=0\Rightarrow 2x(x-3)=0\Rightarrow x=0,x=3
\end{equation*}
The area bounded by $f(x)=-x^{2}+6x+4$ and $g(x)=x^{2}+4$ is $\int_{0}^{3}(f(x)-g(x))dx$
\begin{equation*}
    A=\int_{0}^{3}(f(x)-g(x))dx=\int_{0}^{3}(-2x^{2}+6x)dx=[-\frac{2}{3}+3x^{2}]_{0}^{3}=-\frac{2}{3}(3)^{3}+3(3)^{2}-0=9
\end{equation*}
\end{proof}
\begin{problem}
Evaluate $\int x^{2}e^{3x}dx$
\end{problem}
\begin{proof}[Solution]
\
\begin{table}[H]
    \centering
    \begin{tabular}{|c|c|}
        \hline
        $u=x^{2}$&$dv=e^{3x}dx$\\
        \hline
        $du=2xdx$&$v=\frac{1}{3}e^{3x}$\\
        \hline
    \end{tabular}
\end{table}
\begin{equation*}
    \int udv=uv-\int vdu\Rightarrow \int x^{2}e^{3x}dx=\frac{1}{3}x^{2}e^{3x}-\frac{2}{3}\int xe^{3x}dx
\end{equation*}
\begin{table}[H]
    \centering
    \begin{tabular}{|c|c|}
        \hline
        $u=x$&$dv=e^{3x}dx$\\
        \hline
        $du=dx$&$v=\frac{1}{3}e^{3x}$\\
        \hline
    \end{tabular}
\end{table}
\begin{align*}
    \frac{1}{3}x^{2}e^{3x}-\frac{2}{3}\int xe^{3x}dx&=\frac{1}{3}x^{2}e^{3x}-\frac{2}{3}\big(\frac{1}{3}xe^{3x}-\frac{1}{3}\int e^{3x}dx\big)=\frac{1}{3}x^{2}e^{3x}-\frac{2}{3}\big(\frac{1}{3}xe^{3x}-\frac{1}{9}e^{3x}\big)+C\\
    &=\frac{1}{3}e^{3x}(x^{2}-\frac{2}{3}x+\frac{2}{9})+C
\end{align*}
\end{proof}
\begin{problem}
Evaluate $\int_{0}^{4\pi^{2}}\sin(\sqrt{y})dy$
\end{problem}
\begin{proof}[Solution]
Let $x=\sqrt{y}$. Then $dx=\frac{1}{2\sqrt{y}}dy\Rightarrow dy=2\sqrt{y}dx=2xdx$.
\begin{table}[H]
    \centering
    \begin{tabular}{|c|c|}
        \hline
        $u=x$&$dv=2\sin(x)dx$\\
        \hline
        $du=dx$&$v=-2\cos(x)$\\
        \hline
    \end{tabular}
\end{table}
\begin{align*}
    \int\sin(\sqrt{y})dy&=\int x\sin(y)dx=-2x\cos(x)-\int(-2\cos(x))dx=-2x\cos(x)+2\sin(x)\\
    \Rightarrow \int_{0}^{4\pi^{2}}\sin(\sqrt{y})dy&=[-2\sqrt{y}\cos(\sqrt{y})+2\sin(\sqrt{y})]_{0}^{4\pi^{2}}=-4\pi
\end{align*}
\end{proof}
\begin{problem}
Evaluate $\int_{0}^{\ln(\sqrt{3})}\frac{1}{\pi}\frac{e^{t}}{1+e^{2t}}dt$
\end{problem}
\begin{proof}[Solution]
Let $u=e^{t}$. Then $du=e^{t}dt$. So:
\begin{align*}
    \int\frac{1}{\pi}\frac{e^{t}}{1+e^{2t}}dt&=\int\frac{1}{\pi}\frac{1}{1+u^{2}}du=\frac{1}{\pi}\arctan(u)=\frac{1}{\pi}\arctan(e^{t})\\
    \Rightarrow\int_{0}^{\ln(\sqrt{3})}\frac{1}{\pi}\frac{e^{t}}{1+e^{2t}}dt&=\frac{1}{\pi}\arctan(e^{t})\big|_{0}^{\ln(\sqrt{3})}=\frac{1}{\pi}(\arctan(e^{\ln(\sqrt{3})})-\arctan(e^{0}))\\
    &=\frac{1}{\pi}(\arctan(\sqrt{3})-\arctan(1))=\frac{1}{\pi}(\frac{\pi}{\pi}{3}-\frac{\pi}{4})=\frac{1}{12}
\end{align*}
\end{proof}
\newpage
\subsection{Exam II}
\begin{problem}
Evaluate the integral $\int\sin^{5}(x)dx$
\end{problem}
\begin{proof}[Solution]
\begin{equation*}
    \int\sin^{5}(x)dx=\int\sin(x)\sin^{4}(x)dx=\int\sin(x)(1-\cos^{2}(x))^{2}dx
\end{equation*}
Let $u=\cos(x)$. Then $du=-\sin(x)dx$.
\begin{align*}
    \int\sin^{5}(x)dx&=-\int(1-u^{2})^{2}du=-u+\frac{2}{3}u^{3}-\frac{1}{5}u^{5}+C\\
    &=-\cos(x)+\frac{2}{3}\cos^{3}(x)-\frac{1}{5}\cos^{5}(x)+C
\end{align*}
\end{proof}
\begin{problem}
Estimate the integral $\int_{0}^{4}(x^{3}+1)dx$ with $n=4$ using the trapezoidal rule and Simpson's rule.
\end{problem}
\begin{proof}[Solution]
The integral is $f(x)=x^{3}+1$. We have that $\Delta x=\frac{b-a}{n}=\frac{4-0}{4}=1$
\begin{enumerate}
    \item Using the trapazoidal rule:
    \begin{equation*}
        T=\frac{\Delta x}{2}[f(x_{0})+2f(x_{1})+2f(x_{2})+2f(x_{3})+f(x_{4})]=\frac{1}{2}[1+2(2)+2(9)+2(28)+65]=\frac{144}{2}=72
    \end{equation*}
    \item Using Simpson's Rule:
    \begin{equation*}
        T=\frac{\Delta x}{3}[f(x_{0})+4f(x_{1})+2f(x_{2})+4f(x_{3})+f(x_{4})]=\frac{1}{3}[1+4(2)+2(9)+4(28)+65]=\frac{204}{3}=68
    \end{equation*}
\end{enumerate}
\end{proof}
\begin{problem}
Evaluate the integral of $\int\frac{x^{2}}{x^{2}+25}dx$ using trigonometric substitution.
\end{problem}
\begin{proof}[Solution]

\end{proof}
\end{document}