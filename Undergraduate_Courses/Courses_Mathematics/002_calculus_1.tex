\documentclass[crop=false,class=book]{standalone}
%---------------PREAMBLE------------%
%---------------------------Packages----------------------------%
\usepackage{geometry}
\geometry{b5paper, margin=1.0in}
\usepackage[T1]{fontenc}
\usepackage{graphicx, float}            % Graphics/Images.
\usepackage{natbib}                     % For bibliographies.
\bibliographystyle{agsm}                % Bibliography style.
\usepackage[french, english]{babel}     % Language typesetting.
\usepackage[dvipsnames]{xcolor}         % Color names.
\usepackage{listings}                   % Verbatim-Like Tools.
\usepackage{mathtools, esint, mathrsfs} % amsmath and integrals.
\usepackage{amsthm, amsfonts, amssymb}  % Fonts and theorems.
\usepackage{tcolorbox}                  % Frames around theorems.
\usepackage{upgreek}                    % Non-Italic Greek.
\usepackage{fmtcount, etoolbox}         % For the \book{} command.
\usepackage[newparttoc]{titlesec}       % Formatting chapter, etc.
\usepackage{titletoc}                   % Allows \book in toc.
\usepackage[nottoc]{tocbibind}          % Bibliography in toc.
\usepackage[titles]{tocloft}            % ToC formatting.
\usepackage{pgfplots, tikz}             % Drawing/graphing tools.
\usepackage{imakeidx}                   % Used for index.
\usetikzlibrary{
    calc,                   % Calculating right angles and more.
    angles,                 % Drawing angles within triangles.
    arrows.meta,            % Latex and Stealth arrows.
    quotes,                 % Adding labels to angles.
    positioning,            % Relative positioning of nodes.
    decorations.markings,   % Adding arrows in the middle of a line.
    patterns,
    arrows
}                                       % Libraries for tikz.
\pgfplotsset{compat=1.9}                % Version of pgfplots.
\usepackage[font=scriptsize,
            labelformat=simple,
            labelsep=colon]{subcaption} % Subfigure captions.
\usepackage[font={scriptsize},
            hypcap=true,
            labelsep=colon]{caption}    % Figure captions.
\usepackage[pdftex,
            pdfauthor={Ryan Maguire},
            pdftitle={Mathematics and Physics},
            pdfsubject={Mathematics, Physics, Science},
            pdfkeywords={Mathematics, Physics, Computer Science, Biology},
            pdfproducer={LaTeX},
            pdfcreator={pdflatex}]{hyperref}
\hypersetup{
    colorlinks=true,
    linkcolor=blue,
    filecolor=magenta,
    urlcolor=Cerulean,
    citecolor=SkyBlue
}                           % Colors for hyperref.
\usepackage[toc,acronym,nogroupskip,nopostdot]{glossaries}
\usepackage{glossary-mcols}
%------------------------Theorem Styles-------------------------%
\theoremstyle{plain}
\newtheorem{theorem}{Theorem}[section]

% Define theorem style for default spacing and normal font.
\newtheoremstyle{normal}
    {\topsep}               % Amount of space above the theorem.
    {\topsep}               % Amount of space below the theorem.
    {}                      % Font used for body of theorem.
    {}                      % Measure of space to indent.
    {\bfseries}             % Font of the header of the theorem.
    {}                      % Punctuation between head and body.
    {.5em}                  % Space after theorem head.
    {}

% Italic header environment.
\newtheoremstyle{thmit}{\topsep}{\topsep}{}{}{\itshape}{}{0.5em}{}

% Define environments with italic headers.
\theoremstyle{thmit}
\newtheorem*{solution}{Solution}

% Define default environments.
\theoremstyle{normal}
\newtheorem{example}{Example}[section]
\newtheorem{definition}{Definition}[section]
\newtheorem{problem}{Problem}[section]

% Define framed environment.
\tcbuselibrary{most}
\newtcbtheorem[use counter*=theorem]{ftheorem}{Theorem}{%
    before=\par\vspace{2ex},
    boxsep=0.5\topsep,
    after=\par\vspace{2ex},
    colback=green!5,
    colframe=green!35!black,
    fonttitle=\bfseries\upshape%
}{thm}

\newtcbtheorem[auto counter, number within=section]{faxiom}{Axiom}{%
    before=\par\vspace{2ex},
    boxsep=0.5\topsep,
    after=\par\vspace{2ex},
    colback=Apricot!5,
    colframe=Apricot!35!black,
    fonttitle=\bfseries\upshape%
}{ax}

\newtcbtheorem[use counter*=definition]{fdefinition}{Definition}{%
    before=\par\vspace{2ex},
    boxsep=0.5\topsep,
    after=\par\vspace{2ex},
    colback=blue!5!white,
    colframe=blue!75!black,
    fonttitle=\bfseries\upshape%
}{def}

\newtcbtheorem[use counter*=example]{fexample}{Example}{%
    before=\par\vspace{2ex},
    boxsep=0.5\topsep,
    after=\par\vspace{2ex},
    colback=red!5!white,
    colframe=red!75!black,
    fonttitle=\bfseries\upshape%
}{ex}

\newtcbtheorem[auto counter, number within=section]{fnotation}{Notation}{%
    before=\par\vspace{2ex},
    boxsep=0.5\topsep,
    after=\par\vspace{2ex},
    colback=SeaGreen!5!white,
    colframe=SeaGreen!75!black,
    fonttitle=\bfseries\upshape%
}{not}

\newtcbtheorem[use counter*=remark]{fremark}{Remark}{%
    fonttitle=\bfseries\upshape,
    colback=Goldenrod!5!white,
    colframe=Goldenrod!75!black}{ex}

\newenvironment{bproof}{\textit{Proof.}}{\hfill$\square$}
\tcolorboxenvironment{bproof}{%
    blanker,
    breakable,
    left=3mm,
    before skip=5pt,
    after skip=10pt,
    borderline west={0.6mm}{0pt}{green!80!black}
}

\AtEndEnvironment{lexample}{$\hfill\textcolor{red}{\blacksquare}$}
\newtcbtheorem[use counter*=example]{lexample}{Example}{%
    empty,
    title={Example~\theexample},
    boxed title style={%
        empty,
        size=minimal,
        toprule=2pt,
        top=0.5\topsep,
    },
    coltitle=red,
    fonttitle=\bfseries,
    parbox=false,
    boxsep=0pt,
    before=\par\vspace{2ex},
    left=0pt,
    right=0pt,
    top=3ex,
    bottom=1ex,
    before=\par\vspace{2ex},
    after=\par\vspace{2ex},
    breakable,
    pad at break*=0mm,
    vfill before first,
    overlay unbroken={%
        \draw[red, line width=2pt]
            ([yshift=-1.2ex]title.south-|frame.west) to
            ([yshift=-1.2ex]title.south-|frame.east);
        },
    overlay first={%
        \draw[red, line width=2pt]
            ([yshift=-1.2ex]title.south-|frame.west) to
            ([yshift=-1.2ex]title.south-|frame.east);
    },
}{ex}

\AtEndEnvironment{ldefinition}{$\hfill\textcolor{Blue}{\blacksquare}$}
\newtcbtheorem[use counter*=definition]{ldefinition}{Definition}{%
    empty,
    title={Definition~\thedefinition:~{#1}},
    boxed title style={%
        empty,
        size=minimal,
        toprule=2pt,
        top=0.5\topsep,
    },
    coltitle=Blue,
    fonttitle=\bfseries,
    parbox=false,
    boxsep=0pt,
    before=\par\vspace{2ex},
    left=0pt,
    right=0pt,
    top=3ex,
    bottom=0pt,
    before=\par\vspace{2ex},
    after=\par\vspace{1ex},
    breakable,
    pad at break*=0mm,
    vfill before first,
    overlay unbroken={%
        \draw[Blue, line width=2pt]
            ([yshift=-1.2ex]title.south-|frame.west) to
            ([yshift=-1.2ex]title.south-|frame.east);
        },
    overlay first={%
        \draw[Blue, line width=2pt]
            ([yshift=-1.2ex]title.south-|frame.west) to
            ([yshift=-1.2ex]title.south-|frame.east);
    },
}{def}

\AtEndEnvironment{ltheorem}{$\hfill\textcolor{Green}{\blacksquare}$}
\newtcbtheorem[use counter*=theorem]{ltheorem}{Theorem}{%
    empty,
    title={Theorem~\thetheorem:~{#1}},
    boxed title style={%
        empty,
        size=minimal,
        toprule=2pt,
        top=0.5\topsep,
    },
    coltitle=Green,
    fonttitle=\bfseries,
    parbox=false,
    boxsep=0pt,
    before=\par\vspace{2ex},
    left=0pt,
    right=0pt,
    top=3ex,
    bottom=-1.5ex,
    breakable,
    pad at break*=0mm,
    vfill before first,
    overlay unbroken={%
        \draw[Green, line width=2pt]
            ([yshift=-1.2ex]title.south-|frame.west) to
            ([yshift=-1.2ex]title.south-|frame.east);},
    overlay first={%
        \draw[Green, line width=2pt]
            ([yshift=-1.2ex]title.south-|frame.west) to
            ([yshift=-1.2ex]title.south-|frame.east);
    }
}{thm}

%--------------------Declared Math Operators--------------------%
\DeclareMathOperator{\adjoint}{adj}         % Adjoint.
\DeclareMathOperator{\Card}{Card}           % Cardinality.
\DeclareMathOperator{\curl}{curl}           % Curl.
\DeclareMathOperator{\diam}{diam}           % Diameter.
\DeclareMathOperator{\dist}{dist}           % Distance.
\DeclareMathOperator{\Div}{div}             % Divergence.
\DeclareMathOperator{\Erf}{Erf}             % Error Function.
\DeclareMathOperator{\Erfc}{Erfc}           % Complementary Error Function.
\DeclareMathOperator{\Ext}{Ext}             % Exterior.
\DeclareMathOperator{\GCD}{GCD}             % Greatest common denominator.
\DeclareMathOperator{\grad}{grad}           % Gradient
\DeclareMathOperator{\Ima}{Im}              % Image.
\DeclareMathOperator{\Int}{Int}             % Interior.
\DeclareMathOperator{\LC}{LC}               % Leading coefficient.
\DeclareMathOperator{\LCM}{LCM}             % Least common multiple.
\DeclareMathOperator{\LM}{LM}               % Leading monomial.
\DeclareMathOperator{\LT}{LT}               % Leading term.
\DeclareMathOperator{\Mod}{mod}             % Modulus.
\DeclareMathOperator{\Mon}{Mon}             % Monomial.
\DeclareMathOperator{\multideg}{mutlideg}   % Multi-Degree (Graphs).
\DeclareMathOperator{\nul}{nul}             % Null space of operator.
\DeclareMathOperator{\Ord}{Ord}             % Ordinal of ordered set.
\DeclareMathOperator{\Prin}{Prin}           % Principal value.
\DeclareMathOperator{\proj}{proj}           % Projection.
\DeclareMathOperator{\Refl}{Refl}           % Reflection operator.
\DeclareMathOperator{\rk}{rk}               % Rank of operator.
\DeclareMathOperator{\sgn}{sgn}             % Sign of a number.
\DeclareMathOperator{\sinc}{sinc}           % Sinc function.
\DeclareMathOperator{\Span}{Span}           % Span of a set.
\DeclareMathOperator{\Spec}{Spec}           % Spectrum.
\DeclareMathOperator{\supp}{supp}           % Support
\DeclareMathOperator{\Tr}{Tr}               % Trace of matrix.
%--------------------Declared Math Symbols--------------------%
\DeclareMathSymbol{\minus}{\mathbin}{AMSa}{"39} % Unary minus sign.
%------------------------New Commands---------------------------%
\DeclarePairedDelimiter\norm{\lVert}{\rVert}
\DeclarePairedDelimiter\ceil{\lceil}{\rceil}
\DeclarePairedDelimiter\floor{\lfloor}{\rfloor}
\newcommand*\diff{\mathop{}\!\mathrm{d}}
\newcommand*\Diff[1]{\mathop{}\!\mathrm{d^#1}}
\renewcommand*{\glstextformat}[1]{\textcolor{RoyalBlue}{#1}}
\renewcommand{\glsnamefont}[1]{\textbf{#1}}
\renewcommand\labelitemii{$\circ$}
\renewcommand\thesubfigure{%
    \arabic{chapter}.\arabic{figure}.\arabic{subfigure}}
\addto\captionsenglish{\renewcommand{\figurename}{Fig.}}
\numberwithin{equation}{section}

\renewcommand{\vector}[1]{\boldsymbol{\mathrm{#1}}}

\newcommand{\uvector}[1]{\boldsymbol{\hat{\mathrm{#1}}}}
\newcommand{\topspace}[2][]{(#2,\tau_{#1})}
\newcommand{\measurespace}[2][]{(#2,\varSigma_{#1},\mu_{#1})}
\newcommand{\measurablespace}[2][]{(#2,\varSigma_{#1})}
\newcommand{\manifold}[2][]{(#2,\tau_{#1},\mathcal{A}_{#1})}
\newcommand{\tanspace}[2]{T_{#1}{#2}}
\newcommand{\cotanspace}[2]{T_{#1}^{*}{#2}}
\newcommand{\Ckspace}[3][\mathbb{R}]{C^{#2}(#3,#1)}
\newcommand{\funcspace}[2][\mathbb{R}]{\mathcal{F}(#2,#1)}
\newcommand{\smoothvecf}[1]{\mathfrak{X}(#1)}
\newcommand{\smoothonef}[1]{\mathfrak{X}^{*}(#1)}
\newcommand{\bracket}[2]{[#1,#2]}

%------------------------Book Command---------------------------%
\makeatletter
\renewcommand\@pnumwidth{1cm}
\newcounter{book}
\renewcommand\thebook{\@Roman\c@book}
\newcommand\book{%
    \if@openright
        \cleardoublepage
    \else
        \clearpage
    \fi
    \thispagestyle{plain}%
    \if@twocolumn
        \onecolumn
        \@tempswatrue
    \else
        \@tempswafalse
    \fi
    \null\vfil
    \secdef\@book\@sbook
}
\def\@book[#1]#2{%
    \refstepcounter{book}
    \addcontentsline{toc}{book}{\bookname\ \thebook:\hspace{1em}#1}
    \markboth{}{}
    {\centering
     \interlinepenalty\@M
     \normalfont
     \huge\bfseries\bookname\nobreakspace\thebook
     \par
     \vskip 20\p@
     \Huge\bfseries#2\par}%
    \@endbook}
\def\@sbook#1{%
    {\centering
     \interlinepenalty \@M
     \normalfont
     \Huge\bfseries#1\par}%
    \@endbook}
\def\@endbook{
    \vfil\newpage
        \if@twoside
            \if@openright
                \null
                \thispagestyle{empty}%
                \newpage
            \fi
        \fi
        \if@tempswa
            \twocolumn
        \fi
}
\newcommand*\l@book[2]{%
    \ifnum\c@tocdepth >-3\relax
        \addpenalty{-\@highpenalty}%
        \addvspace{2.25em\@plus\p@}%
        \setlength\@tempdima{3em}%
        \begingroup
            \parindent\z@\rightskip\@pnumwidth
            \parfillskip -\@pnumwidth
            {
                \leavevmode
                \Large\bfseries#1\hfill\hb@xt@\@pnumwidth{\hss#2}
            }
            \par
            \nobreak
            \global\@nobreaktrue
            \everypar{\global\@nobreakfalse\everypar{}}%
        \endgroup
    \fi}
\newcommand\bookname{Book}
\renewcommand{\thebook}{\texorpdfstring{\Numberstring{book}}{book}}
\providecommand*{\toclevel@book}{-2}
\makeatother
\titleformat{\part}[display]
    {\Large\bfseries}
    {\partname\nobreakspace\thepart}
    {0mm}
    {\Huge\bfseries}
\titlecontents{part}[0pt]
    {\large\bfseries}
    {\partname\ \thecontentslabel: \quad}
    {}
    {\hfill\contentspage}
\titlecontents{chapter}[0pt]
    {\bfseries}
    {\chaptername\ \thecontentslabel:\quad}
    {}
    {\hfill\contentspage}
\newglossarystyle{longpara}{%
    \setglossarystyle{long}%
    \renewenvironment{theglossary}{%
        \begin{longtable}[l]{{p{0.25\hsize}p{0.65\hsize}}}
    }{\end{longtable}}%
    \renewcommand{\glossentry}[2]{%
        \glstarget{##1}{\glossentryname{##1}}%
        &\glossentrydesc{##1}{~##2.}
        \tabularnewline%
        \tabularnewline
    }%
}
\newglossary[not-glg]{notation}{not-gls}{not-glo}{Notation}
\newcommand*{\newnotation}[4][]{%
    \newglossaryentry{#2}{type=notation, name={\textbf{#3}, },
                          text={#4}, description={#4},#1}%
}
%--------------------------LENGTHS------------------------------%
% Spacings for the Table of Contents.
\addtolength{\cftsecnumwidth}{1ex}
\addtolength{\cftsubsecindent}{1ex}
\addtolength{\cftsubsecnumwidth}{1ex}
\addtolength{\cftfignumwidth}{1ex}
\addtolength{\cfttabnumwidth}{1ex}

% Indent and paragraph spacing.
\setlength{\parindent}{0em}
\setlength{\parskip}{0em}
%---------------GLOSSARY------------%
\makeglossaries
\loadglsentries{../../glossary}
\loadglsentries{../../acronym}
%--------------Title Page-----------%
\begin{document}
\chapter{Calculus I}
\section{Exams}
\subsection{CLEP Exam}
\begin{problem}
    If $f(x)=-2x^{-3}$, then $f'(x)=$
    \begin{enumerate}[label=(\Alph*)]
        \begin{multicols}{4}
            \item Bob
            \item Bill
            \item Alice
            \item George
        \end{multicols}
    \end{enumerate}
\end{problem}
\clearpage
\subsection{Exam I}
\begin{problem}
Compute the derivative of $y(x)=\frac{1}{2}(x^{4}+7)$
\end{problem}
\begin{proof}[Solution]
$\frac{dy}{dx}=\frac{d}{dx}(\frac{1}{2}(x^{4}+7))=\frac{1}{2}\frac{d}{dx}(x^{4}+7)=\frac{1}{2}(4x^{3})=2x^{3}$
\end{proof}
\begin{problem}
Compute the derivative of $y(x)=\frac{x^{2}+1}{5}$
\end{problem}
\begin{proof}[Solution]
$\frac{dy}{dx}=\frac{d}{dx}(\frac{x^{2}+1}{5})=\frac{1}{5}\frac{d}{dx}(x^{2}+1)=\frac{1}{5}(2x)=\frac{2}{5}x$
\end{proof}
\begin{problem}
Compute the derivative of $y(x)=-3x^{-8}+2\sqrt{x}$
\end{problem}
\begin{proof}[Solution]
$\frac{dy}{dx}=\frac{d}{dx}(3x^{-8}+2\sqrt{x})=3\frac{d}{dx}(x^{-8})+2\frac{d}{dx}(x^{\frac{1}{2}})=-24x^{9}+x^{-\frac{1}{2}}$
\end{proof}
\begin{problem}
Compute the derivative of $y(x)=\frac{x\sqrt{x}+1}{x}$
\end{problem}
\begin{proof}[Solution]
$\frac{dy}{dx}=\frac{d}{dx}(\frac{x\sqrt{x}+1}{x})=\frac{d}{dx}(x^{\frac{1}{2}}+x^{-1})=\frac{d}{dx}(x^{\frac{1}{2}})+\frac{d}{dx}(x^{-1})=\frac{1}{2}x^{-\frac{1}{2}}-x^{-2}$
\end{proof}
\begin{problem}
Compute the derivative of $y(x)=\frac{\sqrt{x}+\sqrt[3]{x}}{\sqrt[4]{x^{3}}}$
\end{problem}
\begin{proof}[Solution]
$\frac{dy}{dx}=\frac{d}{dx}(\frac{\sqrt{x}+\sqrt[3]{x}}{\sqrt[4]{x^{3}}})=\frac{d}{dx}(x^{\frac{1}{2}-\frac{3}{4}}+x^{\frac{1}{3}-\frac{3}{4}})=\frac{d}{dx}(x^{-\frac{1}{4}})+\frac{d}{dx}(x^{-\frac{5}{12}})=-\frac{1}{4}x^{-\frac{5}{4}}-\frac{5}{12}x^{-\frac{17}{12}}$
\end{proof}
\begin{problem}
Compute the derivative of $y(t)=\frac{t^{2}+1}{3t}$
\end{problem}
\begin{proof}[Solution]
$\frac{dy}{dt}=\frac{d}{dt}(\frac{t^{2}+1}{3t})=\frac{1}{3}\frac{d}{dt}(t+t^{-1})=\frac{1}{3}(\frac{d}{dt}(t)+\frac{d}{dt}(t^{-1}))=\frac{1-t^{-2}}{3}$
\end{proof}
\begin{problem}
What are the horizontal tangents of $g(t)=e^{t}-4t+5$?
\end{problem}
\begin{proof}[Solution]
Horizontal tangents occur when $\frac{dg}{dt}=0$. Computing, we have: $\frac{dg}{dt}=\frac{d}{dt}(e^{t}-4t+5) = e^{t}-4$. So $\frac{dg}{dt}=0\Rightarrow e^{t}-4=0\Rightarrow e^{t}=4\Rightarrow t=\ln(4)=2\ln(2)$. There is a horizontal tangent at $t=2\ln(2)$.
\end{proof}
\begin{problem}
Compute $\frac{d^{2}y}{dx^{2}}$ of $x\sin(x)$. Simplify.
\end{problem}
\begin{proof}[Solution]
$\frac{d^{2}y}{dx^{2}}=\frac{d}{dx}(\frac{dy}{dx})$. Computing the first derivative, we have:
\begin{equation*}
    \frac{dy}{dx}=\frac{d}{dx}(x\sin(x))=x\frac{d}{dx}(\sin(x))+\sin(x)\frac{d(x)}{dx}=x\cos(x)+\sin(x)    
\end{equation*}
Computing the second derivative, we have:
\begin{equation*}
\frac{d^{2}y}{dx^{2}}=\frac{d}{dx}(\frac{dy}{dx})=\frac{d}{dx}(x\cos(x)+\sin(x))=\frac{d}{dx}(x\cos(x))+\frac{d}{dx}(\sin(x)) =-x\sin(x)+\cos(x)+\cos(x)   
\end{equation*}
Simplifying, we have $\frac{d^{2}y}{dx^{2}} = 2\cos(x)-x\sin(x)$
\end{proof}
\begin{problem}
Compute the derivative of $f(t)=\frac{2t-1}{t+3}$
\end{problem}
\begin{proof}[Solution]
$\frac{df}{dt}=\frac{d}{dt}(\frac{2t-1}{t+3})=\frac{(t+3)\frac{d}{dt}(2t-1)-(2t-1)\frac{d}{dt}(t+3)}{(t+3)^{2}}=\frac{2(t+3)-(2t-1)}{(t+3)^{2})}=\frac{7}{(t+3)^{2}}$
\end{proof}
\begin{problem}
Compute the derivative of $g(x)=\frac{x^{2}-1}{x^{2}+1}$
\end{problem}
\begin{proof}[Solution]
$\frac{dg}{dx}=\frac{d}{dx}(\frac{x^{2}-1}{x^{2}+1})=\frac{(x^{2}+1)\frac{d}{dx}(x^{2}-1)-(x^{2}-1)\frac{d}{dx}(x^{2}+1)}{(x^{2}+1)^{2}}=\frac{2x((x^{2}+1)-(x^{2}-1))}{(x^{2}+1)^{2}}=\frac{4x}{(x^{2}+1)^{2}}$
\end{proof}
\begin{problem}
Compute the derivative of $h(x)=\frac{x}{x^{2}+1}$
\end{problem}
\begin{proof}[Solution]
$\frac{dh}{dx}=\frac{d}{dx}(\frac{x}{x^{2}+1})=\frac{(x^{2}+1)\frac{d(x)}{dx}-x\frac{d}{dx}(x^{2}+1)}{(x^{2}+1)^{2}}=\frac{(x^{2}+1)-2x^{2}}{(x^{2}+1)^{2}}=\frac{1-x^{2}}{(x^{2}+1)^{2}}$
\end{proof}
\begin{problem}
Compute the derative of $y(t)=\frac{3t^{2}-5}{2t+1}$
\end{problem}
\begin{proof}[Solution]
$\frac{dy}{dt}=\frac{d}{dt}(\frac{3t^{2}-5}{2t+1})=\frac{(2t+1)\frac{d}{dt}(3t^{2}-5)-(3t^{2}-5)\frac{d}{dt}(2t+1)}{(2t+1)^{2}}=\frac{(2t+1)(6t)-(3t^{2}-5)(2)}{(2t+1)^{2}}=\frac{6t^{2}+6t+10}{2}$
\end{proof}
\begin{problem}
Find the equation of the tangent line to the graph of $f(x)=x-2\tan(x)$ at $x=\frac{\pi}{4}$.
\end{problem}
\begin{proof}[Solution]
The equation of the tangent is $y(x)=f'(x_{0})(x-x_{0})+f(x_{0})$. Computing the derivative, we have: $\frac{df}{dx}=\frac{d}{dx}(x-2\tan(x))=1-2\sec^{2}(x)$. So $f'(\frac{\pi}{4})=-3$. And $f(\frac{\pi}{4})=\frac{\pi}{4}-2\tan(\frac{\pi}{4})=\frac{\pi}{4}-2$. The tangent is $y(x)=-3(x-\frac{\pi}{4})+\frac{\pi}{4}-2$. Simplifying, we get $y(x)=-3x+\pi-2$
\end{proof}
\newpage
\begin{problem}
Use the limit definition to compute the derivative of $f(x)=\frac{1}{x}$.
\end{problem}
\begin{proof}[Solution]
\begin{equation*}
    \frac{df}{dx}(x)=\underset{h\rightarrow 0}{\lim}\frac{f(x+h)-f(x)}{h}=\underset{h\rightarrow 0}{\lim}\frac{\frac{1}{x+h}-\frac{1}{x}}{h}=\underset{h\rightarrow 0}{\lim}\frac{\frac{x-(x+h)}{x(x+h)}}{h}=\underset{h\rightarrow 0}{\lim}-\frac{1}{x(x+h)}=-\frac{1}{x^{2}}
\end{equation*}
\end{proof}
\begin{problem}
Use the limit definition to compute the derivative of $f(x)=\frac{1}{x+1}$
\end{problem}
\begin{proof}[Solution]
\begin{align*}
    \frac{df}{dx}(x)&=\underset{h\rightarrow 0}{\lim}\frac{f(x+h)-f(x)}{h}=\underset{h\rightarrow 0}{\lim}\frac{\frac{1}{x+h+1}-\frac{1}{x+1}}{h}=\underset{h\rightarrow 0}{\lim}\frac{\frac{x+1-(x+h+1)}{(x+1)(x+h+1)}}{h}\\
    &=\underset{h\rightarrow 0}{\lim}-\frac{1}{(x+h+1)(x+1)}=-\frac{1}{(x+1)^{2}}
\end{align*}
\end{proof}
\begin{problem}
Use the limit definition to compute the derivative of $f(x)=\frac{1}{\sqrt{x}}$
\end{problem}
\begin{proof}[Solution]
\begin{align*}
    \frac{df}{dx}(x)&=\underset{h\rightarrow 0}{\lim}\frac{f(x+h)-f(x)}{h}=\underset{h\rightarrow 0}{\lim}\frac{\frac{1}{\sqrt{x+h}}-\frac{1}{\sqrt{x}}}{h}=\underset{h\rightarrow 0}{\lim}\frac{\frac{\sqrt{x}-\sqrt{x+h}}{\sqrt{x+h}\sqrt{x}}}{h}=\underset{h\rightarrow 0}{\lim}\frac{x-(x+h)}{h\sqrt{x+h}\sqrt{x}(\sqrt{x+h}+\sqrt{x})}\\
    &=\underset{h\rightarrow 0}{\lim}-\frac{1}{\sqrt{x+h}\sqrt{x}(\sqrt{x+h}+\sqrt{x})}=-\frac{1}{x\sqrt{x}}
\end{align*}
\end{proof}
\begin{problem}
use the limit definition to compute the derivative of $f(x)=\frac{1}{x^{2}}$
\end{problem}
\begin{proof}[Solution]
\begin{equation*}
    \frac{df}{dx}(x)=\underset{h\rightarrow 0}{\lim}\frac{f(x+h)-f(x)}{h}=\underset{h\rightarrow 0}{\lim}\frac{\frac{1}{(x+h)^{2}}-\frac{1}{x^{2}}}{h}=\underset{h\rightarrow 0}{\lim}\frac{\frac{x^{2}-(x+h)^{2}}{x^{2}(x+h)^{2}}}{h}=\underset{h\rightarrow 0}{\lim}-\frac{2x+h^{2}}{x^{2}(x+h)^{2}}=-\frac{2}{x^{3}}
\end{equation*}
\end{proof}
\begin{problem}
Use the limit definition to compute the derivative of $f(x)=\sqrt{x+1}$
\end{problem}
\begin{proof}[Solution]
\begin{align*}
    \frac{df}{dx}(x)&=\underset{h\rightarrow 0}{\lim}\frac{f(x+h)-f(x)}{h}=\underset{h\rightarrow 0}{\lim}\frac{\sqrt{x+h+1}-\sqrt{x+1}}{h}=\underset{h\rightarrow 0}{\lim}\frac{(x+h+1)-(x+1)}{h(\sqrt{x+h+1}+\sqrt{x+1})}\\
    &=\underset{h\rightarrow 0}{\lim}\frac{1}{\sqrt{x+h+1}+\sqrt{x+1}}=\frac{1}{2\sqrt{x+1}}
\end{align*}
\end{proof}
\begin{problem}
Find the horizontal and vertical asymptotes of $f(x)=\frac{2x^{3}-3}{x^{4}-16}$
\end{problem}
\begin{proof}[Solution]
Horizontal asymptotes occur when the limit $\underset{x\rightarrow\pm}{\lim}f(x)$ exists. For both limits $f(x)\rightarrow 0$, from L'H\^{o}pital's rule. Vertical asymptotes occur when the denominator approaches 0. This occurs when $x^{4}=16$. The real solutions to this are $x=2$ and $x=-2$. From this we get:
\begin{align*}
    \underset{x\rightarrow 2^{+}}{\lim}f(x)&=+\infty&\underset{x\rightarrow 2^{-}}{\lim}f(x)&=-\infty&\underset{x\rightarrow -2^{+}}{\lim}f(x)&=-\infty&\underset{x\rightarrow -2^{-}}{\lim}f(x)&=+\infty
\end{align*}
\end{proof}
\begin{problem}
Evaluate $\underset{x\rightarrow\infty}{\lim}\frac{3x^{4}+5}{2x^{4}}$
\end{problem}
\begin{proof}[Solution]
From L'H\^{o}pital's Rule: $\underset{x\rightarrow\infty}{\lim}\frac{3x^{4}+5}{2x^{4}}=\underset{x\rightarrow\infty}{\lim}\frac{12x^{3}}{8x^{3}}=\underset{x\rightarrow \infty}{\lim}\frac{3}{2}=\frac{3}{2}$
\end{proof}
\subsection{Exam II}
\begin{problem}
Compute the derivative of $f(x)=e^{\sec(x)}$
\end{problem}
\begin{proof}[Solution]
From the chain rule: $\frac{df}{dx}(x)=\frac{d}{dx}(e^{\sec(x)})=e^{\sec(x)}\frac{d}{dx}(\sec(x))=e^{\sec(x)}\sec(x)\tan(x)$
\end{proof}
\begin{problem}
Compute the derivative of $g(x)=\ln(\sin^{2}(x))$
\end{problem}
\begin{proof}[Solution]
Chain rule: $\frac{dg}{dx}(x)=\frac{d}{dx}(\ln(\sin^{2}(x)))=\frac{1}{\sin^{2}(x)}\frac{d}{dx}(\sin^{2}(x))=\frac{2\sin(x)\cos(x)}{\sin^{2}(x)}=2\cot(x)$
\end{proof}
\begin{problem}
Compute the derivative of $s(t)=\sin(\sqrt[5]{t})$
\end{problem}
\begin{proof}[Solution]
Chain rule: $\frac{ds}{dt}(t)=\frac{d}{dt}(\sin(\sqrt[5]{t}))=\cos(\sqrt[5]{t})\frac{d}{dt}(\sqrt[5]{t})=\frac{\cos(\sqrt[5]{t})}{5\sqrt[5]{t^{4}}}$
\end{proof}
\begin{problem}
Compute the derivative of $f(x)=\ln(\cos^{-1}(x))$
\end{problem}
\begin{proof}[Solution]
Chain rule: $\frac{df}{dx}(x)=\frac{d}{dx}(\ln(\cos^{-1}(x)))=\frac{1}{\cos^{-1}(x)}\frac{d}{dx}(\cos^{-1}(x))=-\frac{1}{\cos^{-1}(x)\sqrt{1-x^{2}}}$
\end{proof}
\begin{problem}
Compute the derivative of $y(x)=(x+2)^{x+2}$
\end{problem}
\begin{proof}[Solution]
Let $f(x)=\ln(y(x))$. Then $f(x)=\ln((x+2)^{x+2})=(x+2)\ln(x+2)$. Then:
\begin{equation*}
    \frac{df}{dx}(x)=(x+2)\frac{d}{dx}(\ln(x+2))+\ln(x+2)\frac{d}{dx}(x+2)=1+\ln(x+2)
\end{equation*}
But $\frac{df}{dx}(x)=\frac{d}{dx}(\ln(y(x)))=\frac{1}{y(x)}\frac{dy}{dx}(x)$, so $\frac{dy}{dx}(x)=y(x)\frac{df}{dx}(x)$. Thus, $\frac{dy}{dx}(x)=(x+2)^{x+2}(1+\ln(x+2))$
\end{proof}
\begin{problem}
Compute the derivative of $\sqrt{xy}=1$
\end{problem}
\begin{proof}[Solution]
$\frac{d}{dx}(\sqrt{xy})=\frac{d}{dx}(1)=0$. And $\sqrt{xy}=1\Rightarrow xy=1\Rightarrow y=\frac{1}{x}$. So:
\begin{equation*}
    \frac{1}{\sqrt{xy}}\frac{d}{dx}(xy)=0\Rightarrow \frac{y}{\sqrt{xy}}+\frac{x}{\sqrt{xy}}\frac{dy}{dx}=0\Rightarrow y+x\frac{dy}{dx}=0\Rightarrow\frac{dy}{dx}=-\frac{y}{x}=\frac{dy}{dx}=-\frac{1}{x^{2}}
\end{equation*}
\end{proof}
\begin{problem}
Compute the derivative of $y(x)=x^{2}\cos^{2}(2x^{2})$
\end{problem}
\begin{proof}[Solution]
$\frac{d}{dx}(x^{2}\cos^{2}(2x^{2}))=\cos^{2}(2x^{2})\frac{d}{dx}(x^{2})+x^{2}\frac{d}{dx}(\cos^{2}(2x^{2}))=2x\cos^{2}(2x^{2})+8x^{3}\sin^{2}(2x^{2})$
\end{proof}
\begin{problem}
Compute the derivative of $y(t)=t\tan^{-1}(t)-\frac{1}{2}\ln(t)$
\end{problem}
\begin{proof}[Solution]
$\frac{dy}{dt}(t)=\frac{d}{dt}(t\tan^{-1}(t))-\frac{1}{2}\frac{d}{dt}(\ln(t))=\tan^{-1}(t)+t\frac{d}{dt}(\tan^{-1}(t))-\frac{1}{2t}=\tan^{-1}(t)+\frac{t}{1+t^{2}}-\frac{1}{2t}$
\end{proof}
\begin{problem}
Compute the derivative of $f(x)=xe^{\sqrt[3]{x}}$
\end{problem}
\begin{proof}[Solution]
$\frac{df}{dx}(x)=\frac{d}{dx}(xe^{\sqrt[3]{x}})=e^{\sqrt[3]{x}}+xe^{\sqrt[3]{x}}\frac{d}{dx}(\sqrt[3]{x})=e^{\sqrt[3]{x}}+\frac{1}{3}\sqrt[3]{x}e^{\sqrt[3]{x}}$
\end{proof}
\begin{problem}
Compute the derivative of $g(x)=\ln(\csc(x^{2}))$
\end{problem}
\begin{proof}[Solution]
$\frac{d}{dx}(\ln(\csc(x^{2})))=\frac{1}{\csc(x^{2})}\frac{d}{dx}(\csc(x^{2}))=-\sin(x^{2})\cot(x^{2})\csc(x^{2})\frac{d}{dx}(x^{2})=-2x\cot(x^{2})$
\end{proof}
\begin{problem}
Compute the derivative of $s(t)=\sin(\tan(t))$
\end{problem}
\begin{proof}[Solution]
$\frac{ds}{dt}(t)=\frac{d}{dt}(\sin(\tan(t)))=\cos(\tan(t))\frac{d}{dt}(\tan(t))=\cos(\tan(t))\sec^{2}(t)$
\end{proof}
\begin{problem}
Compute the derivative of $f(x)=e^{\tan(x)}$
\end{problem}
\begin{proof}[Solution]
$\frac{df}{dx}(x)=\frac{d}{dx}(e^{\tan(x)})=e^{\tan(x)}\frac{d}{dx}(\tan(x))=e^{\tan(x)}\sec^{2}(x)$
\end{proof}
\begin{problem}
Compute the derivative of $g(x)=\ln(\sec(x))$
\end{problem}
\begin{proof}[Solution]
$\frac{dg}{dx}(x)=\frac{d}{dx}(\ln(\sec(x)))=\frac{1}{\sec(x)}\frac{d}{dx}(\sec(x))=\tan(x)$
\end{proof}
\begin{problem}
Compute the derivative of $s(t)=\cos(\sqrt{1+t^{2}})$
\end{problem}
\begin{proof}[Solution]
$\frac{ds}{dt}(t)=\frac{d}{dt}(\cos(\sqrt{1+t^{2}}))=-\sin(\sqrt{1+t^{2}})\frac{d}{dt}(\sqrt{1+t^{2}})=-\frac{t}{\sqrt{1+t^{2}}}\sin(\sqrt{1+t^{2}})$
\end{proof}
\begin{problem}
Compute the second derivative of $y(x)=\sin^{-1}(2x)$
\end{problem}
\begin{proof}[Solution]
The second derivative is $\frac{d^{2}y}{dx^{2}}(x)=\frac{d}{dx}(\frac{dy}{dx}(x))$. So, we have:
\begin{align*}
    \frac{d^{2}y}{dx^{2}}(x)&=\frac{d}{dx}\big(\frac{dy}{dx}\big)=\frac{d}{dx}\big(\frac{d}{dx}(\sin^{-1}(2x))\big)=\frac{d}{dx}\big(\frac{1}{\sqrt{1-4x^{2}}}\frac{d}{dx}(2x)\big)=\frac{d}{dx}\big(\frac{2}{\sqrt{1-4x^{2}}}\big)\\
    &=2\frac{d}{dx}((1-4x^{2})^{-\frac{1}{2}})=-(1-4x^{2})^{-\frac{3}{2}}\frac{d}{dx}(1-4x^{2})=8x(1-4x^{2})^{-\frac{3}{2}}=\frac{8x}{(1-4x^{2})^{\frac{3}{2}}}
\end{align*}
\end{proof}
\begin{problem}
Let $f(x)=-3x^{4}+4x^{3}$
\begin{enumerate}
    \item Find the critical points of $f$.
    \item Find the intervals for which $f$ is increasing and decreasing.
    \item Find possible inflection points of $f$.
    \item Find the intervals on which the function is concave up and down.
\end{enumerate}
\end{problem}
\begin{proof}[Solution]
\
\begin{enumerate}
    \item Critical points occur when $f'(x)=0$. We have $f'(x)=-12x^{3}+12x^{2}$. Solving for $f'(x)=0$, we have $-12x^{3}+12x^{2}=0 \Rightarrow x^{2}(1-x)=0$. This occurs only when $x=0$ or $x=1$. 
    \item $f$ is increasing when $f'(x)>0$. Solving for this, we have $x^{2}(1-x)>0\Rightarrow x<1$. $f$ is decreasing when $f'(x)<0$, which occurs when $x>1$.
    \item Possible inflection points occur when $f''(x)=0$. Solving for this, we have $f''(x) = -36x^{2}+24x = -12x^{2}(3x-2)$. So $f''(x)=0$ when $x=0$ and when $x=\frac{2}{3}$.
    \item $f$ is concave up when $f''(x)>0$ and concave down when $f''(x)<0$. If $x>0$, then $-12x<0$, so $-12x(3x-2)>0\Rightarrow 3x-2<0$. So $0<x<\frac{2}{3}$. If $x<0$, then $-12x>0$ and $3x-2<0$, and therefore $-12x(3x-2)<0$. If $x>\frac{2}{3}$, then $(3x-2)>0$ and $-12x<0$, so $-12x(3x-2)<0$. Thus, $f$ is concave up on $(0,\frac{2}{3})$ and concanve down on $(-\infty,0)\cup(\frac{2}{3},\infty)$
\end{enumerate}
\end{proof}
\begin{problem}
Find the equation of the line tangent to the graph of $x+\sqrt{xy}=6$ at the point $(4,1)$.
\end{problem}
\begin{proof}[Solution]
The tangent line is $y=m(x-x_{0})+y_{0}$, where $m$ is the derivative of the graph at $(x_{0},y_{0})$. We have $x_{0}=4$ and $y_{0}=1$. From $\sqrt{xy}=6-x$, and $y=\frac{(6-x)^{2}}{x}$, we have:
\begin{equation*}
    \frac{d}{dx}(x+\sqrt{xy})=\frac{d}{dx}(6)=0\Rightarrow 1+\frac{1}{2\sqrt{xy}}(y+x\frac{dy}{dx})=0\Rightarrow\frac{dy}{dx}=-\frac{2\sqrt{xy}}{x}-\frac{y}{x}=-\frac{2(6-x)}{x}-\frac{(6-x)^{2}}{x^{2}}
\end{equation*}
Evaluating at $x=4$, we get $m=-\frac{5}{4}$. The line is $y=-\frac{5}{4}(x-4)+1$
\end{proof}
\begin{problem}
Compute the second derivative of $y(x)x^{2}\cos(-x)$.
\end{problem}
\begin{proof}[Solution]
First, $\cos(-x)=\cos(x)$, so we can simplify. We have:
\begin{equation*}
    \frac{dy}{dx}(x)=\frac{d}{dx}(x^{2}\cos(x))=2x\cos(x)-x^{2}\sin(x)\Rightarrow\frac{d^{2}y}{dx^{2}}=2\cos(x)-2x\sin(x)-2x\sin(x)-x^{2}\cos(x)
\end{equation*}
Simplifying, we have $\frac{d^{2}y}{dx^{2}}=(2-x^{2})\cos(x)-4x\sin(x)$
\end{proof}
\begin{problem}
Find the extreme values (Global and Relative) of $f(x)=4x^{2}-4x+1$ for $0\leq x\leq 1$.
\end{problem}
\begin{proof}[Solution]
Extreme values occur either at endpoints, or when $f'(x)=0$. $f'(x)=8x-4$. So $f'(x)=0$ when $x=\frac{1}{2}$. At this point, $f(\frac{1}{2})=0$. At the endpoints, $f(0)=1$ and $f(1)=1$, so $x=\frac{1}{2}$ is a global minimum. $x=0$ and $x=1$ are relative maxmima.
\end{proof}
\begin{problem}
Compute the derivative of $g(t)=(t^{3}-2)^{\tan(t)}$
\end{problem}
\begin{proof}[Solution]
Let $y(t)=\ln(g(t))$. Then $\frac{dy}{dt}(t)=\frac{d}{dt}(\tan(t)\ln(t^{3}-2))=\sec^{2}(t)\ln(t^{3}-2)+\tan(t)\frac{3t^{2}}{t^{3}-2}$. But $\frac{dy}{dt}(t)=\frac{d}{dt}(\ln(g(t)))=\frac{1}{g(t)}\frac{dg}{dt}(t)$. So $\frac{dg}{dt}(t)=(t^{3}-2)^{\tan(t)}(\sec^{2}(t)\ln(t^{3}-2)+\tan(t)\frac{3t^{2}}{t^{3}-2})$
\end{proof}
\end{document}