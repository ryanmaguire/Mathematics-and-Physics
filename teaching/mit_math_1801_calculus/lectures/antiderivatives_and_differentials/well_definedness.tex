%-----------------------------------LICENSE------------------------------------%
%   This file is part of Mathematics-and-Physics.                              %
%                                                                              %
%   Mathematics-and-Physics is free software: you can redistribute it and/or   %
%   modify it under the terms of the GNU General Public License as             %
%   published by the Free Software Foundation, either version 3 of the         %
%   License, or (at your option) any later version.                            %
%                                                                              %
%   Mathematics-and-Physics is distributed in the hope that it will be useful, %
%   but WITHOUT ANY WARRANTY; without even the implied warranty of             %
%   MERCHANTABILITY or FITNESS FOR A PARTICULAR PURPOSE.  See the              %
%   GNU General Public License for more details.                               %
%                                                                              %
%   You should have received a copy of the GNU General Public License along    %
%   with Mathematics-and-Physics.  If not, see <https://www.gnu.org/licenses/>.%
%------------------------------------------------------------------------------%
\ifx\noPreamble\undefined
\documentclass{article}
\usepackage{graphicx}
\usepackage{float}
\usepackage{color}
\usepackage{amsmath, amssymb}
\usepackage{hyperref}
\graphicspath{{../../../../../images/}}
\title{18.01: Antiderivatives and Differentials - Uniqueness and Existence}
\author{Ryan Maguire}
\date{\today}
\setlength{\parindent}{0em}
\setlength{\parskip}{0em}
\newcommand{\setGlossaryItem}[2]{%
    \phantomsection
    #1\def\@currentlabel{\unexpanded{#1}}\label{#2}%
}
\makeatother
\begin{document}
    \maketitle
\fi
    When we talk about antiderivatives, one often say
    \textit{the} antiderivative of a function
    $F:\mathbb{R}\rightarrow\mathbb{R}$. This is a poor
    use of the word \textit{the} because it seems to imply there is only one.
    Unfortunately for any continuous
    real-valued function $F$,
    there are \textit{infinitely many} functions
    $f:\mathbb{R}\rightarrow\mathbb{R}$ such that $f(x)=F'(x)$
    for each real number $x$.
    \par\hfill\par
    This rarely causes problems since there is essentially a unique
    antiderivative. By this it is meant that if
    $f_{0}:\mathbb{R}\rightarrow\mathbb{R}$ and
    $f_{1}:\mathbb{R}\rightarrow\mathbb{R}$ are
    antiderivatives of $F:\mathbb{R}\rightarrow\mathbb{R}$,
    then there is some constant real number $C$
    such that $f_{0}(x)=f_{1}(x)+C$ for all $x$. To see this, note that by
    the linearity of the derivative we have:
    \begin{equation}
        \begin{aligned}
            \left(f_{0}(x)-f_{1}(x)\right)^{\prime}
                &=f_{0}^{\prime}(x)-f_{1}^{\prime}(x)\\
                &=F(x)-F(x)\\
                &=0
        \end{aligned}
    \end{equation}
    If we let $C:\mathbb{R}\rightarrow\mathbb{R}$ be defined by
    $C(x)=f_{0}(x)-f_{1}(x)$, then this shows that $C$ satisfies
    $C'(x)=0$ for all real numbers $x$. Now suppose $C$ is not a constant
    function. Then there are real numbers $x_{0}$ and $x_{1}$ such that
    $C(x_{0})\ne{C}(x_{1})$. But then $C(x_{0})-C(x_{1})\ne{0}$. Then by
    the mean value theorem for derivatives, there is some point
    $x_{c}$ between $x_{0}$ and $x_{1}$ such that:
    \begin{equation}
        C'(x_{c})=\frac{C(x_{0})-C(x_{1})}{x_{0}-x_{1}}
    \end{equation}
    The left-hand side is zero, but the right-hand side is non-zero, which is
    a contradiction. So $C$ is constant. But remember, $C$ we defined as
    $C(x)=f_{0}(x)-f_{1}(x)$, and hence $f_{0}(x)=f_{1}(x)+C$.
    \par\hfill\par
    This argument shows two things. First, \textit{if} there exists one
    solution to $f_{1}'(x)=F(x)$, then there exists infinitely many since
    $f_{0}(x)=f_{1}(x)+C$ is another solution for any real number $C$.
    Secondly, it shows that \textit{all} solutions are just vertical shifts of
    each other. All that is left to do is show that there is at least one
    antiderivative, and then we've classified them all. This is achievable once
    we have studied the integral. Given some fixed real number $x_{0}$,
    one solution is given by defining $f:\mathbb{R}\rightarrow\mathbb{R}$
    via:
    \begin{equation}
        f(x)=\int_{x_{0}}^{x}F^{\prime}(t)\,\textrm{d}t
    \end{equation}
    This equation may not make sense for now, and that is perfectly okay.
    All it means is that there is always at least one antiderivative for any
    continuous function $F:\mathbb{R}\rightarrow\mathbb{R}$.
\ifx\noPreamble\undefined
    \newpage
    \Large{\textbf{Glossary}}
    \par\hfill\par
    \textbf{Continuous Function}
    \par
    Intuitively, this is a function that can be drawn without \textit{jumps},
    or without lifting up your pencil. More rigorously, it is a real-valued
    function $f:\mathbb{R}\rightarrow\mathbb{R}$ such that for each real number
    $x$ and every positive number $\varepsilon$ there
    \par\hfill\par
    \textbf{Mean Value Theorem}
    \newpage
    I, the copyright holder of this work, release it into the public domain.
    This applies worldwide. In some countries this may not be legally possible;
    if so: I grant anyone the right to use this work for any purpose, without
    any conditions, unless such conditions are required by law.
    \par\hfill\par
    The source code used to generate this document is free software and released
    under version 3 of the GNU General Public License.
\end{document}
\fi
