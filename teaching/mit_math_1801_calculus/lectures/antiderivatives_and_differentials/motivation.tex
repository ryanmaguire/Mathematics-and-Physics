%-----------------------------------LICENSE------------------------------------%
%   This file is part of Mathematics-and-Physics.                              %
%                                                                              %
%   Mathematics-and-Physics is free software: you can redistribute it and/or   %
%   modify it under the terms of the GNU General Public License as             %
%   published by the Free Software Foundation, either version 3 of the         %
%   License, or (at your option) any later version.                            %
%                                                                              %
%   Mathematics-and-Physics is distributed in the hope that it will be useful, %
%   but WITHOUT ANY WARRANTY; without even the implied warranty of             %
%   MERCHANTABILITY or FITNESS FOR A PARTICULAR PURPOSE.  See the              %
%   GNU General Public License for more details.                               %
%                                                                              %
%   You should have received a copy of the GNU General Public License along    %
%   with Mathematics-and-Physics.  If not, see <https://www.gnu.org/licenses/>.%
%------------------------------------------------------------------------------%
\ifx\noPreamble\undefined
\documentclass{article}
\usepackage{graphicx}
\usepackage{float}
\usepackage{color}
\usepackage{amsmath}
\graphicspath{{../../../../../images/}}
\title{18.01: Antiderivatives and Differentials - Motivation}
\author{Ryan Maguire}
\date{\today}
\setlength{\parindent}{0em}
\setlength{\parskip}{0em}
\newif\ifsolution
\solutiontrue
\begin{document}
    \maketitle
\fi
    Given a real-valued function $F$, if $F$ is smooth enough we can take its
    derivative at each real number $x$ and define a new function:
    \begin{equation}
        y(x)=F^{\prime}(x)
    \end{equation}
    Or, using the Liebniz notation:
    \begin{equation}
        y(x)=\frac{\textrm{d}F}{\textrm{d}x}(x)
    \end{equation}
    \textit{Undoing} a derivative is a common problem in
    mathematics and physics. Take momentum as an example.
    The defining equation is:
    \begin{equation}
        \frac{\textrm{d}\mathbf{p}}{\textrm{d}t}=\mathbf{F}(t)
    \end{equation}
    where $\mathbf{p}$ is the momentum and $\mathbf{F}$ is the force.
    If we know the force on a particle, the momentum could be
    calculated if we could somehow \textit{undo} the derivative.
    Further, one could study torque. This is related to angular momentum
    via:
    \begin{equation}
        \frac{\textrm{d}\mathbf{L}}{\textrm{d}t}=\boldsymbol{\tau}
    \end{equation}
    where $\mathbf{L}$ is the angular momentum, and $\boldsymbol{\tau}$ is the
    torque.\footnote{%
        For the curious observer, these quantities are in \textbf{bold} because
        they are vectors. But you don't need to worry about that now.
    }
    Again, undoing the derivative of angular momentum could tell us the
    torque on a particle.
    \par\hfill\par
    Significantly trickier examples are found in biology. In a given
    ecosystem, if $x(t)$ denotes the population of rabbits at time $t$, and if
    $y(t)$ denotes the population of foxes at time $t$, then we can model
    population growth of these two species via:
    \begin{equation}
        \begin{aligned}
            \frac{\textrm{d}x}{\textrm{d}t}(t)
                &=ax(t)-bx(t)y(t)\\
            \frac{\textrm{d}y}{\textrm{d}t}(t)
                &=-cy(t)+dx(t)y(t)
        \end{aligned}
    \end{equation}
    where the constants $a$, $b$, $c$, and $d$ describe the birth and death
    rates of the respective species. Undoing this \textit{system} of
    derivatives could allow biologists to predict future populations.
    \par\hfill\par
    The usefulness of this concept is not tied solely to physics,
    and pervades most branches of mathematics.
    Even some of the more modern and abstract ones.
    In differential topology
    one studies \textit{geodesic flow} where some quantity is related to
    another by a derivative, and one wishes to \textit{undo} the derivative.
    \par\hfill\par
    This process of undoing a derivative is called the
    \textbf{antiderivative}. One of the most important results in calculus are
    the two \textit{fundamental theorems of calculus} which tell us that the
    notion of antiderivative and the concept of \textit{integration} are very
    closely related. Indeed, one may say that they are one in the same.
\ifx\noPreamble\undefined
    \newpage
    I, the copyright holder of this work, release it into the public domain.
    This applies worldwide. In some countries this may not be legally possible;
    if so: I grant anyone the right to use this work for any purpose, without
    any conditions, unless such conditions are required by law.
    \par\hfill\par
    The source code used to generate this document is free software and released
    under version 3 of the GNU General Public License.
\end{document}
\fi