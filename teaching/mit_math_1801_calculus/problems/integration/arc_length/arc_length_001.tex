%-----------------------------------LICENSE------------------------------------%
%   This file is part of Mathematics-and-Physics.                              %
%                                                                              %
%   Mathematics-and-Physics is free software: you can redistribute it and/or   %
%   modify it under the terms of the GNU General Public License as             %
%   published by the Free Software Foundation, either version 3 of the         %
%   License, or (at your option) any later version.                            %
%                                                                              %
%   Mathematics-and-Physics is distributed in the hope that it will be useful, %
%   but WITHOUT ANY WARRANTY; without even the implied warranty of             %
%   MERCHANTABILITY or FITNESS FOR A PARTICULAR PURPOSE.  See the              %
%   GNU General Public License for more details.                               %
%                                                                              %
%   You should have received a copy of the GNU General Public License along    %
%   with Mathematics-and-Physics.  If not, see <https://www.gnu.org/licenses/>.%
%------------------------------------------------------------------------------%
\documentclass{article}
\usepackage{graphicx}
\usepackage{float}
\usepackage{color}
\usepackage{amsmath}
\graphicspath{{../../../../../images/}}
\title{18.01: Integration - Arc Length}
\author{Ryan Maguire}
\date{\today}
\setlength{\parindent}{0em}
\setlength{\parskip}{0em}
\newif\ifsolution
\solutiontrue
\begin{document}
    \maketitle
    \textbf{Problem:}
    \par\hfill\par
    A curve $\gamma(t)=\left(x(t),\,y(t)\right)$
    in the $xy$ plane is given parametrically by the following equations:
    \begin{subequations}
        \begin{align}
            x(t)&=1+3t^{2}\\
            y(t)&=2+2t^{3}
        \end{align}
    \end{subequations}
    Find its arc length between the points where $t=0$ and $t=1$.
    \begin{figure}[H]
        \centering
        \includegraphics{parametric_equations_001}
        \caption{The parametric curve plotted between $t=0$ and $t=1$.}
    \end{figure}
    \ifsolution
        \newpage
        \color{blue}
        \textbf{Solution:}
        \par\hfill\par
        Letting $\dot{x}(t)$ and $\dot{y}(t)$ denote the derivative of
        $x$ and $y$ with respective to $t$,
        we have:
        \begin{subequations}
            \begin{align}
                L&=\int_{0}^{1}\sqrt{\dot{x}(t)^{2}+\dot{y}(t)^{2}}
                    \,\textrm{d}t\\
                &=\int_{0}^{1}\sqrt{(6t)^2+(6t^2)^2}\,\textrm{d}t\\
                &=\int_{0}^{1}6t\sqrt{1+t^{2}}\,\textrm{d}t
            \end{align}
        \end{subequations}
        This final integral can be evaluated using $u$ substitution
        by choosing $u=1+t^{2}$. Using this we obtain
        $\textrm{d}u=2t\,\textrm{d}t$. Remembering to change the limits,
        we get:
        \begin{subequations}
            \begin{align}
                \int_{0}^{1}6t\sqrt{1+t^{2}}\,\textrm{d}t
                &=\int_{1}^{2}3\sqrt{u}\,\textrm{d}u\\
                &=3\int_{1}^{2}u^{1/2}\,\textrm{d}u
            \end{align}
        \end{subequations}
        This can be computed with the power rule.
        \begin{subequations}
            \begin{align}
                3\int_{1}^{2}u^{1/2}\,\textrm{d}u
                &=3\frac{u^{3/2}}{3/2}\Big|_{1}^{2}\\
                &=2u^{3/2}\Big|_{1}^{2}\\
                &=4\sqrt{2}-2
            \end{align}
        \end{subequations}
        The arc length is $L=4\sqrt{2}-2$.
        \color{black}
    \fi
    \newpage
    I, the copyright holder of this work, release it into the public domain.
    This applies worldwide. In some countries this may not be legally possible;
    if so: I grant anyone the right to use this work for any purpose, without
    any conditions, unless such conditions are required by law.
    \par\hfill\par
    The source code used to generate this document is free software and released
    under version 3 of the GNU General Public License.
\end{document}
