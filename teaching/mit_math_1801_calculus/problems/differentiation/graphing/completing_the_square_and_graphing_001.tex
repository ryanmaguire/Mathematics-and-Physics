%-----------------------------------LICENSE------------------------------------%
%   This file is part of Mathematics-and-Physics.                              %
%                                                                              %
%   Mathematics-and-Physics is free software: you can redistribute it and/or   %
%   modify it under the terms of the GNU General Public License as             %
%   published by the Free Software Foundation, either version 3 of the         %
%   License, or (at your option) any later version.                            %
%                                                                              %
%   Mathematics-and-Physics is distributed in the hope that it will be useful, %
%   but WITHOUT ANY WARRANTY; without even the implied warranty of             %
%   MERCHANTABILITY or FITNESS FOR A PARTICULAR PURPOSE.  See the              %
%   GNU General Public License for more details.                               %
%                                                                              %
%   You should have received a copy of the GNU General Public License along    %
%   with Mathematics-and-Physics.  If not, see <https://www.gnu.org/licenses/>.%
%------------------------------------------------------------------------------%
\documentclass{article}
\usepackage{graphicx}
\usepackage{color}
\graphicspath{{../../../../../images/}}
\title{18.01: Differentiation - Graphing}
\author{Ryan Maguire}
\date{\today}
\setlength{\parindent}{0em}
\setlength{\parskip}{0em}
\newif\ifsolution
\solutiontrue
\begin{document}
    \maketitle
    \textbf{Problem:}
    \par\hfill\par
    Let $f$ be defined by:
    \[
        f(x)=x^{2}-4x+3
    \]
    By completing the square, shifting, and scaling, sketch a graph of
    $f$.
    \par\hfill\par
    \ifsolution
        \color{blue}
        \textbf{Solution:}
        \par\hfill\par
        We can factor $f$ directly, obtaining:
        \[
            f(x)=(x-3)(x-1)
        \]
        which tells us the roots of $f$ are $x=1$ and $x=3$. Let's complete
        the square. We have:
        \[
            x^{2}-4x=(x-2)^{2}-4
        \]
        And hence:
        \[
            f(x)=x^{2}-4x+3=(x-2)^{2}-1
        \]
        We know what the graph of $y=x^{2}$ looks like, it is an ordinary
        parabola. This final formula tells us we are shifting this familiar
        image vertically \textit{downwards} by 1 and horizontally to the
        \textit{right} by 2. This creates the image below.
        \begin{figure}
            \centering
            \includegraphics{completing_the_square_and_graphing_001}
            \caption{Graph of $f$}
        \end{figure}
    \fi
\end{document}
