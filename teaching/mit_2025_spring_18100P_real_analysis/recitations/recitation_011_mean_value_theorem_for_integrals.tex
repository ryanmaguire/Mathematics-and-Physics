%-----------------------------------LICENSE------------------------------------%
%   This file is part of Mathematics-and-Physics.                              %
%                                                                              %
%   Mathematics-and-Physics is free software: you can redistribute it and/or   %
%   modify it under the terms of the GNU General Public License as             %
%   published by the Free Software Foundation, either version 3 of the         %
%   License, or (at your option) any later version.                            %
%                                                                              %
%   Mathematics-and-Physics is distributed in the hope that it will be useful, %
%   but WITHOUT ANY WARRANTY; without even the implied warranty of             %
%   MERCHANTABILITY or FITNESS FOR A PARTICULAR PURPOSE.  See the              %
%   GNU General Public License for more details.                               %
%                                                                              %
%   You should have received a copy of the GNU General Public License along    %
%   with Mathematics-and-Physics.  If not, see <https://www.gnu.org/licenses/>.%
%------------------------------------------------------------------------------%
\documentclass{article}
\usepackage{amssymb}
\usepackage{amsthm}
\usepackage{xcolor}
\title{Lemmas for the Fundamental Theorem of Calculus}
\author{Ryan Maguire}
\date{\today}
\setlength{\parindent}{0em}
\setlength{\parskip}{0em}
\newtheorem{theorem}{Theorem}
\newif\ifsolution
\solutiontrue

\begin{document}
    \maketitle
    \section{Mean Value Theorem for Integrals}
        Here's an intuitive argument for the fundamental theorem of
        calculus. Let $f:[a,\,b]\rightarrow\mathbb{R}$ be a continuous
        function and define:
        \begin{equation}
            F(x)=
            \int_{a}^{x}f(t)\,\textrm{d}t
        \end{equation}
        The derivative produces the following difference quotient:
        \begin{equation}
            \begin{array}{rcl}
                \displaystyle
                F^{\prime}(x)
                &=&
                \displaystyle
                \lim_{h\rightarrow{0}}
                \frac{
                    \displaystyle
                    \int_{a}^{x+h}f(t)\,\textrm{d}t-
                    \int_{a}^{x}f(t)\,\textrm{d}t
                }{h}\\[2em]
                &=&
                \displaystyle
                \lim_{h\rightarrow{0}}
                \frac{1}{h}
                \int_{x}^{x+h}f(t)\,\textrm{d}t
            \end{array}
        \end{equation}
        where we have made use of the additive property of the
        limits of the Riemann / Darboux integral:
        \begin{equation}
            \int_{a}^{c}f(t)\,\textrm{d}t
            =\int_{a}^{b}f(t)\,\textrm{d}t+\int_{b}^{c}f(t)\,\textrm{d}t
        \end{equation}
        Since $[x,\,x+h]$ is small, and $f$ is continuous, $f(t)$ varies
        very little on this interval. We may as well replace it
        with a constant $f(x_{h})$, giving us:
        \begin{equation}
            F^{\prime}(x)
            =\lim_{h\rightarrow{0}}\frac{f(x_{h})}{h}
                \int_{x}^{x+h}\textrm{d}t
            =\lim_{h\rightarrow{0}}\frac{f(x_{h})}{h}h
            =\lim_{h\rightarrow{0}}f(x_{h})
        \end{equation}
        Since $x_{h}\in[x,\,x+h]$, as $h$ tends to zero $x_{h}$ must
        converge to $x$. From the continuity of $f$ we may conclude
        that $F^{\prime}(x)=f(x)$.
        \par\hfill\par
        This argument can be made precise if we can rigorously prove that
        there is indeed some $x_{h}\in[x,\,x+h]$ with the property that
        the integral of $f$ over $[x,\,x+h]$ is just $f(x_{h})h$. There is,
        this is the \textit{mean value theorem for integrals}.
        \begin{theorem}[\textbf{Mean Value Theorem for Integrals}]
            If $f:[a,\,b]\rightarrow\mathbb{R}$ is continuous, then there
            is a $c\in[a,\,b]$ such that:
            \begin{equation}
                \int_{a}^{b}f(x)\,\textrm{d}x
                =f(c)(b-a)
            \end{equation}
        \end{theorem}
        \begin{proof}
            Since $f$ is continuous and $[a,\,b]$ is compact, by the
            extreme value theorem there exists $x_{m},x_{M}\in[a,\,b]$ such
            that $f(x_{m})\leq{f}(x)\leq{f}(x_{M})$ for all
            $x\in[a,\,b]$. But then:
            \begin{equation}
                \frac{1}{b-a}
                \int_{a}^{b}f(x_{m})\,\textrm{d}x
                \leq
                \frac{1}{b-a}
                \int_{a}^{b}f(x)\,\textrm{d}x
                \leq
                \frac{1}{b-a}
                \int_{a}^{b}f(x_{M})\,\textrm{d}x
            \end{equation}
            We may factor constants out of the integral to give us:
            \begin{equation}
                f(x_{m})
                \leq
                \frac{1}{b-a}
                \int_{a}^{b}f(x)\,\textrm{d}x
                \leq
                f(x_{M})
            \end{equation}
            Labeling the ratio in middle as $L$, this tells us that
            $L$ is wedged between $f(x_{m})$ and $f(x_{M})$. Since $f$ is
            continuous, by the intermediate value theorem there is some
            $c\in[a,\,b]$ such that $f(c)=L$. But then, multiplying both sides
            by $b-a$, we get:
            \begin{equation}
                f(c)(b-a)
                =
                \int_{a}^{b}f(x)\,\textrm{d}x
            \end{equation}
        \end{proof}
        There are some things in this proof that you should be very explicit
        about in your writing, some we have proven in class and some we have
        not.
        \begin{itemize}
            \item
                The integral is linear in the limits (proved in class):
                \begin{equation}
                    \int_{a}^{c}f(t)\,\textrm{d}t
                    =\int_{a}^{b}f(t)\,\textrm{d}t+\int_{b}^{c}f(t)\,\textrm{d}t
                \end{equation}
            \item
                Continuous functions on closed bounded intervals have
                Riemann integrals (sketch of proof given, you should provide
                a detailed one).
        \end{itemize}
        Try to prove the mean value theorem for integrals rigorously from the
        definition of the Riemann / Darboux integral. You may freely use
        the extreme value theorem, which comes in hand a few times.
\end{document}
