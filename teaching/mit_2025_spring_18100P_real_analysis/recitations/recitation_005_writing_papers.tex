%-----------------------------------LICENSE------------------------------------%
%   This file is part of Mathematics-and-Physics.                              %
%                                                                              %
%   Mathematics-and-Physics is free software: you can redistribute it and/or   %
%   modify it under the terms of the GNU General Public License as             %
%   published by the Free Software Foundation, either version 3 of the         %
%   License, or (at your option) any later version.                            %
%                                                                              %
%   Mathematics-and-Physics is distributed in the hope that it will be useful, %
%   but WITHOUT ANY WARRANTY; without even the implied warranty of             %
%   MERCHANTABILITY or FITNESS FOR A PARTICULAR PURPOSE.  See the              %
%   GNU General Public License for more details.                               %
%                                                                              %
%   You should have received a copy of the GNU General Public License along    %
%   with Mathematics-and-Physics.  If not, see <https://www.gnu.org/licenses/>.%
%------------------------------------------------------------------------------%
\documentclass{article}
\usepackage{amsmath, amsthm, amssymb}
\usepackage{xcolor}
\theoremstyle{normal}
\newtheorem{theorem}{Theorem}
\newtheorem{definition}{Definition}
\title{Organizing a Paper}
\author{Ryan Maguire}
\date{\today}
\setlength{\parindent}{0em}
\setlength{\parskip}{0em}
\begin{document}
    \maketitle
    \tableofcontents
    \section{Mathematical Writing}
        Sections, subsections, and subsubsections are incredibly useful tools
        for organizing books and papers, but one should be careful about abusing
        them.
        \subsection{Getting to the point}
            I probably didn't need to make a subsection just for this sentence.
            \subsubsection{Emphasizing the point}
                And now I'm simply nesting sentence after sentence. This may
                actually \textit{harm} my writing.
        \subsection{Using \LaTeX{}'s Organizational Features}
            The current section we are in is called
            \textit{Mathematical writing}, and this subsection is about using
            \LaTeX{} to help organize our paper.
            This seems like an appropriate subsection, but to warrant declaring
            some section of the text a subsection there should be some length
            to the writing.
        \subsection{New subsection}
            Otherwise
        \subsection{Newer subsection}
            you may
        \subsection{Even newer subsection}
            be needlessly
        \subsection{The newest subsection}
            breaking up your train of thought.
            \par\hfill\par
            Let's step back and see how we me benefit our writing using these
            \LaTeX{} tools.
    \section{Fresnel Inversion Through Planetary Rings}
        The rings of Saturn (insert Ryan's research here, blah blah blah,
        talk about some of the \textit{surface-level} features. This is the
        outermost layer of the organization).
        \subsection{Modeling Saturn's A Ring Using a Straight-Edge}
            The outer edge of the A ring can be accurately modelled using
            a straight-edge. That is, we define the transmittance by:
            \begin{equation}
                \hat{T}(\rho)=
                \begin{cases}
                    1,&\rho\geq{R}\\
                    0,&\rho<R
                \end{cases}
            \end{equation}
            (Discuss more details, we are being more specific here, but this
            makes sense since our subsection is now related to a
            specific topic that falls under the umbrella of the current
            section).
            \subsubsection{Numerical Evaluation of the Fresnel Integrals}
                Now that we understand how to model the diffraction pattern
                presented by Saturn's A ring (using Fresnel sine and cosine
                integrals), we now take on the challenge of numerically
                evaluating these functions at an arbitrary real number. This
                is done using a combination of
                \textit{Remez minimax approximations} for small inputs, and
                \textit{asymptotic expansions} for large values.
                (We're getting really specific here, but again, this makes
                sense. Our subsubsection is about a very particular part of
                the current subsection, which itself is part of the current
                section).
            \subsubsection{The Stationary Phase Approximation}
                Highly oscillatory integrands, such as
                $f(\xi)\exp(i\pi\xi^{2})$, can be treated numerically using
                the stationary value of the function being exponentiated
                (which in our case is $i\pi\xi^{2}$). This trick, which dates
                back to Lord Kelvin's work in the late 1800's, is often
                referred to as the \textit{stationary phase approximation}.
                (Go into details, yadda yadda. This subsubsection is still
                related to the current subsection, but is not directly
                tied to the previous subsubsection, so we break away and
                create a new one).
        \subsection{Using Square Wells to Model Ringlets}
            Several features across the rings of Saturn, such as the
            Maxwell ringlet, Encke gap, Huygen's ringlet, and many more,
            are better modelled using \textit{square wells} and
            \textit{inverted square wells}. (Talk about the mathematics behind
            this, so on and so forth. We are still talking about diffraction
            through the rings, so this subsection is directly related to the
            current section).
    \section{Refraction via Titan's Atmosphere}
        Titan, one of the largest moons in the solar system, has an
        atmosphere that is denser than Earth's. As radio waves pass through
        they are \textit{refracted}, bending the angle of their trajectory.
        (Talk about more stuff related to this. Maybe have subsections and
        subsubsections related to the mathematics and physics of refractions.
        Note, we are \textbf{not} talking about diffraction and planetary
        rings any more, we are instead focusing on refraction and atmospheres.
        This needs to be a new section since it is unrelated to the title of
        the previous one).
    \section{Enough About Saturn}
        The previous example somewhat captures how section, subsection, and
        subsubsection are used in scientific and mathematical writing, but
        to break things down to the subsubsection level often means you have a
        lot to say about these specific things. If you only have a sentence,
        or even a short paragraph, using \textbf{boldface} text for keywords,
        or using newlines
        \par\hfill\par
        like this, may better serve your paper and reduce cluttering the page
        (and the table of contents) with potentially unnecessary labels.
    \section{Working with Theorems}
        Pure math is often nothing more than
        \textit{definition-theorem-proof-example-figure}. Real analysis and
        set theory certainly fall into the realm of pure mathematics.
        The applied mathematician, and the mathematical physicist, often does
        not need to worry about structuring theorems and proofs since these
        are quite rare in such papers. Rare, but not unheard of. Let's use an
        example to motivate how a theorem-proof environment should work.
        \begin{theorem}[\textbf{Equioscillation Theorem}]
            If $f:[a,\,b]\rightarrow\mathbb{R}$ is a continuous function on the
            closed bounded interval $[a,\,b]$ (with $a<b$), and if
            $N\in\mathbb{N}$, then there is a polynomial
            $P_{N}:[a,\,b]\rightarrow\mathbb{R}$ of degree at most $N$ that
            minimizes $|f(x)-P_{N}(x)|$ with respect to the supremum norm,
            such that either $P_{N}(x)=f(x)$ (meaning $f$ is a polynomial),
            or this polynomial must satisfy:
            \begin{enumerate}
                \item
                    There are $N+2$ local maxima for $|f(x)-P_{N}(x)|$
                    on the interval $[a,\,b]$.
                \item
                    The local extrema oscillate in sign. That is, if
                    $x_{n}$ and $x_{n+1}$ are successive maxima, then
                    $f(x_{n})-f_{N}(x_{n})>0$ implies
                    $f(x_{n+1})-f_{N}(x_{n+1})<0$, and vice-versa.
                \item
                    The local extrema all have the same magnitude. That is,
                    $|f(x_{n})-P_{N}(x_{n})|=|f(x_{m})-P_{N}(x_{m})|$ for all
                    extremal points $x_{n}$ and $x_{m}$
            \end{enumerate}
        \end{theorem}
        \begin{proof}
            Dive into the Remez exchange algorithm.
        \end{proof}
        A few things to note.
        \begin{itemize}
            \item
                The theorem got its own theorem environment. It was not
                placed in a section title, or subsection title, but instead
                given its own clear labeling as a theorem on the page. Using
                \textit{Equioscillation Theorem} as a section header is
                perfectly valid, and may even improve the organization, but it
                is not a substitute. The theorem needs its own theorem
                environment.
            \item
                The proof needs its own proof environment. Important definitions
                should not be placed in the proof, but instead should come in
                the discussion leading up to the theorem.
            \item
                Non-important definitions (such as \textbf{towers} in our
                proof of Zorn's lemma) can be placed in the proof environment,
                but \textbf{do not give them their own definition environment}.
                Splitting the proof environment in two by placing a definition
                in between throws off the flow and the reader may think the
                proof is done, and then struggle to realize it is not.
        \end{itemize}
\end{document}
