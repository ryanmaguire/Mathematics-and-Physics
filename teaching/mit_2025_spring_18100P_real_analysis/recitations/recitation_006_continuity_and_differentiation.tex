%-----------------------------------LICENSE------------------------------------%
%   This file is part of Mathematics-and-Physics.                              %
%                                                                              %
%   Mathematics-and-Physics is free software: you can redistribute it and/or   %
%   modify it under the terms of the GNU General Public License as             %
%   published by the Free Software Foundation, either version 3 of the         %
%   License, or (at your option) any later version.                            %
%                                                                              %
%   Mathematics-and-Physics is distributed in the hope that it will be useful, %
%   but WITHOUT ANY WARRANTY; without even the implied warranty of             %
%   MERCHANTABILITY or FITNESS FOR A PARTICULAR PURPOSE.  See the              %
%   GNU General Public License for more details.                               %
%                                                                              %
%   You should have received a copy of the GNU General Public License along    %
%   with Mathematics-and-Physics.  If not, see <https://www.gnu.org/licenses/>.%
%------------------------------------------------------------------------------%
\documentclass{article}
\usepackage{graphicx}
\usepackage{amsmath, amsthm, amssymb}
\usepackage{xcolor}
\theoremstyle{normal}
\newtheorem{theorem}{Theorem}
\newtheorem{definition}{Definition}
\title{Continuity and Differentiation}
\author{Ryan Maguire}
\date{\today}
\setlength{\parindent}{0em}
\setlength{\parskip}{0em}
\newif\ifsolution
\solutiontrue
\graphicspath{{../../../images/}}

\begin{document}
    \maketitle
    \section{Continuity}
        We proved in class that continuity has three equivalent
        definitions. Given $A\subseteq\mathbb{R}^{n}$ and
        $f:A\rightarrow\mathbb{R}^{m}$, continuity means:
        \begin{enumerate}
            \item
                For every convergent sequence
                $a:\mathbb{N}\rightarrow{A}$, the sequence
                given by $f(a_{n})$ is convergent.
            \item
                For every $x_{0}\in{A}$ and for all $\varepsilon>0$ there is
                a $\delta>0$ such that $x\in{A}$ and
                $|x-x_{0}|<\delta$ implies $|f(x)-f(x_{0})|<\varepsilon$
                (Note: in $\mathbb{R}^{n}$ we replace the absolute value
                function with the Euclidean distance function
                $\|\mathbf{x}-\mathbf{y}\|$ which is given by the
                Pythagorean formula).
            \item
                If $\mathcal{V}\subseteq\mathbb{R}^{n}$ is open, then
                $f^{-1}[\mathcal{V}]\subseteq{A}$ is open (with respect to $A$).
        \end{enumerate}
        Where $\mathcal{U}\subseteq{A}$ is open \textit{open with respect to}
        $A$ if for each $x\in\mathcal{U}$ there is a radius
        $r>0$ such that $y\in\mathcal{U}$ and $|x-y|<r$
        (or $\|\mathbf{x}-\mathbf{y}\|<r$) implies $y\in\mathcal{U}$.
        \par\hfill\par
        The third definition is where \textit{topology} starts, since it
        only uses open sets. But we are interested in analysis (particularly
        the analysis of $\mathbb{R}$, $\mathbb{R}^{n}$, and
        $\mathbb{C}$). In these settings the second definition is perhaps the
        most intuitive. This says that as long as our accuracy in measuring
        $x_{0}$ is within $\delta$, the error in computing $f(x_{0})$ is no
        more than $\varepsilon$ (see Fig.~\ref{fig:eps_delta_def}).
        \begin{figure}
            \centering
            \resizebox{0.75\textwidth}{!}{%
                \includegraphics{continuity_epsilon_delta_definition}
            }
            \caption{Visual for the $\varepsilon-\delta$ definition.}
            \label{fig:eps_delta_def}
        \end{figure}
        \par\hfill\par
        In the following problems you may use any of the three definitions.
        \par\hfill\par
        \textbf{Problem:}
        \par\hfill\par
        Show that the function $f:\mathbb{R}\rightarrow\mathbb{R}$ defined by:
        \begin{equation}
            f(x)
            =\begin{cases}
                x,&x\in\mathbb{Q}\\
                -x,&x\not\in\mathbb{Q}
            \end{cases}
        \end{equation}
        is continuous at the origin and \textbf{nowhere else}.
        \par\hfill\par
        \ifsolution
            \color{blue}
            \textit{Solution.}
            \par\hfill\par
            Let $\varepsilon>0$. The value of $f$ at the origin is
            $f(0)=0$. We want to show that there is some $\delta>0$ such that
            if $|x-0|<\delta$, then $|f(x)-f(0)|<\varepsilon$. that is,
            $|x|<\delta$ implies $|f(x)|<\varepsilon$.
            Pick $\delta=\varepsilon$. By the definition of $f$ we have
            $|f(x)|=|x|$. Note that if $|x|<\delta$, then
            $|x|<\varepsilon$, by the definition of $\delta$, which means that
            $|f(x)|<\varepsilon$ since $|f(x)|=|x|$. Hence $f$ is continuous
            at the origin. If $x\in\mathbb{R}$ is \textit{not} the origin,
            we argue as follows. If $x$ is rational, find a sequence
            $a:\mathbb{N}\rightarrow\mathbb{R}$ of irrationals that
            converges to $x$. This can be done explicitly:
            \begin{equation}
                a_{n}=x+\frac{\sqrt{2}}{n+1}
            \end{equation}
            Then $f(a_{n})=-a_{n}$ and this converges to $-x$. But
            $f(x)=x$, and $x\ne{-x}$ since $x$ is non-zero.
            \par\hfill\par
            If $x\in\mathbb{R}$ is irrational, then by definition
            there is a Cauchy sequence $a:\mathbb{N}\rightarrow\mathbb{Q}$
            such that $a_{n}\rightarrow{x}$. But then
            $f(a_{n})=a_{n}$ and this converges to $x$. However
            $f(x)=-x$ and $x\ne{-x}$ since $x$ is non-zero. Hence if
            $x$ is non-zero (either rational or irrational), then $f$ is
            not continuous at $x$.
            \par\hfill\par
            \color{black}
        \fi
        \textbf{Problem:}
        \par\hfill\par
        Define $f:\mathbb{R}\rightarrow\mathbb{R}$ as follows:
        \begin{equation}
            f(x)
            =\begin{cases}
                0,&x\not\in\mathbb{Q}\\
                \frac{\textrm{GCD}(p,\,q)}{q},&x=\frac{p}{q}
            \end{cases}
        \end{equation}
        Since rational numbers $p/q$ can always be chosen to consist of
        co-prime pairs (by dividing the numerator and denominator by their
        greatest common divisor), we may write:
        \begin{equation}
            f\left(\frac{p}{q}\right)
            =\frac{1}{q}
        \end{equation}
        given a co-prime pair $(p,\,q)$ (and non-zero $q\in\mathbb{Z}$).
        The function is plotted in Fig.~\ref{fig:popcorn}.
        Show that $f$ is continuous at every irrational number, and
        discontinuous at every rational number.
        \begin{figure}
            \centering
            \includegraphics{dirichlets_popcorn_function}
            \caption{The Dirichlet Popcorn Function}
            \label{fig:popcorn}
        \end{figure}
        \par\hfill\par
        \ifsolution
            \color{blue}
            \textit{Solution.}
            \par\hfill\par
            Suppose $x\in\mathbb{R}$ is a rational number.
            Define $a:\mathbb{N}\rightarrow\mathbb{R}$ by:
            \begin{equation}
                a_{n}=x+\frac{\sqrt{2}}{n+1}
            \end{equation}
            Then $a_{n}\rightarrow{x}$, but $f(a_{n})=0$ for all
            $n\in\mathbb{N}$, and hence $f(a_{n})\rightarrow{0}$.
            But $f(x)=1/q$, where $x=p/q$ and $p$ and $q$ are co-prime.
            That is, $f(a_{n})\not\rightarrow{f}(x)$, so $f$ is not continuous
            at $x$.
            \par\hfill\par
            If $x_{0}$ is irrational, then we can prove $f$ is continuous at
            $x_{0}$ by using the $\varepsilon-\delta$ definition. Let
            $\varepsilon>0$ be given. There is some positive integer
            $N\in\mathbb{N}$ such that $1/N<\varepsilon$. Let
            $\delta=1/N$. If $x$ satisfies $|x-x_{0}|<\delta$ and $x$
            is irrational, then $|f(x)-f(x_{0})|=|0-0|=0$, and this is indeed
            bounded by $\varepsilon$. For rational $x$, $x=p/q$, if
            $|x-x_{0}|<\delta$, then $q$ must be very large, larger than
            $N$ (otherwise $|x-p/q|$ would be greater than $1/N$). But then
            $|f(x)|=|1/q|$, and since $|q|>N$, we have $|1/q|<1/N$, and hence
            $|1/q|<\varepsilon$. But then $|f(x)-f(x_{0})|<\varepsilon$.
            Since $\varepsilon>0$ is arbitrary, $f$ is continuous at
            $x_{0}$.
            \par\hfill\par
            \color{black}
        \fi
        There is no reverse of this. That is, there is no function
        $f:\mathbb{R}\rightarrow\mathbb{R}$ that is continuous on the
        rationals, but discontinuous on the irrationals. The proof is
        a little too difficult for our current discussion. Roughly speaking
        there are types of sets called $F_{\sigma}$ sets and sets called
        $G_{\delta}$ sets. The rationals form an $F_{\sigma}$ set, the
        irrationals yield a $G_{\delta}$ set (whatever this means). Because of
        this it is possible to have a function that is continuous on
        $\mathbb{R}\setminus\mathbb{Q}$ and discontinuous on $\mathbb{Q}$,
        but not the other way around.
        \par\hfill\par
    \section{Differentiation}
        Differentiation is defined in a similar manner to continuity, but
        we now consider a certain \textit{ratio}. For the time being we
        restrict ourselves to functions $f:A\rightarrow\mathbb{R}$ with
        $A\subseteq\mathbb{R}$. Due to technical reasons, life is a lot
        simpler if we require $A$ to be an open subset. $f$ is
        differentiable at $x_{0}\in{A}$ provided that for any sequence
        $a:\mathbb{N}\rightarrow{A}$ that converges to $x_{0}$
        (i.e. $a_{n}\rightarrow{x}_{0}$) with the property that
        $a_{n}\ne{x}_{0}$ for all $n\in\mathbb{N}$, we have that:
        \begin{equation}
            \frac{f(x)-f(a_{n})}{x-a_{n}}
        \end{equation}
        is a convergent sequence.
        \begin{theorem}
            If $f:A\rightarrow\mathbb{R}$ is differentiable at
            $x_{0}\in{A}$, and if $a,b:\mathbb{N}\rightarrow{A}$ are
            convergent sequences with $a_{n}\rightarrow{x}_{0}$ and
            $b_{n}\rightarrow{x}_{0}$, and $a_{n}\ne{x}_{0}$ and
            $b_{n}\ne{x}_{0}$ for all $n\in\mathbb{N}$, then:
            \begin{equation}
                \lim_{n\rightarrow\infty}
                \frac{f(x)-f(a_{n})}{x-a_{n}}
                =
                \lim_{n\rightarrow\infty}
                \frac{f(x)-f(b_{n})}{x-b_{n}}
            \end{equation}
            That is, the limit is unique.
        \end{theorem}
        \begin{proof}
            For suppose the limits are $A$ and $B$, respectively.
            Form a new sequence $c:\mathbb{N}\rightarrow{A}$ by:
            \begin{equation}
                c_{n}=
                \begin{cases}
                    a_{n/2},&n\textrm{ is even.}\\
                    b_{(n-1)/2},&n\textrm{ is odd.}
                \end{cases}
            \end{equation}
            Then $c_{n}\rightarrow{x}_{0}$ since the $a$ and $b$ sequences
            do this, and moreover $c_{n}\ne{x}_{0}$ for all $n\in\mathbb{N}$.
            Since $f$ is differentiable at $x_{0}$, the sequence:
            \begin{equation}
                \frac{f(x)-f(c_{n})}{x-c_{n}}
            \end{equation}
            must converge, call the limit $C$. But this sequence contains a
            convergent subsequence, namely the even entries, that converges to
            $A$. Since limits are unique, we have $A=C$.
            But this sequence of ratios also contains a convergent subsequence,
            namely the odd entries, with limit $B$. Since limits are unique,
            we see that $B=C$ must also be true. By the transitivity of
            equality we may conclude $A=B$.
        \end{proof}
        We call the unique limit of ratios the \textit{derivative} of
        $f$ at $x_{0}$, often denoted $f^{\prime}(x_{0})$ or
        $\dot{f}(x_{0})$. Like continuity, there is an alternative way to
        think about differentiation.
        \begin{theorem}
            If $A\subseteq\mathbb{R}$ is open, if $f:A\rightarrow\mathbb{R}$ is
            a function, and if $x_{0}\in{A}$, then $f$ is differentiable at
            $x_{0}$ if and only if there is an $L\in\mathbb{R}$ such that
            for all $\varepsilon>0$ there is a $\delta>0$ such that $x\in{A}$
            and $0<|x-x_{0}|<\delta$ implies:
            \begin{equation}
                \left|
                    L-\frac{f(x)-f(x_{0})}{x-x_{0}}
                \right|<\varepsilon
            \end{equation}
        \end{theorem}
        \begin{proof}
            Going one way, suppose $f$ is differentiable. Pick any
            sequence $a:\mathbb{N}\rightarrow{A}$ such that
            $a_{n}\rightarrow{x}_{0}$ and $a_{n}\ne{x}_{0}$ for all
            $n\in\mathbb{N}$. Such a sequence exists since $A$ is open
            (this is precisely why we require this in the definition).
            Define $L$ via:
            \begin{equation}
                L=
                \lim_{n\rightarrow\infty}
                \frac{f(a_{n})-f(x_{0})}{a_{n}-x_{0}}
            \end{equation}
            If the $\varepsilon-\delta$ property is not satisfied,
            then there is an $\varepsilon>0$ such that for all $\delta>0$
            there exists $x\in{A}$ with $0<|x-x_{0}|<\delta$, and yet
            $\left|L-\frac{f(x)-f(x_{0})}{x-x_{0}}\right|\geq\varepsilon$.
            Substituting $\delta=\frac{1}{n+1}$ we may find a sequence of
            points (the axiom of countable choice is being subtly used here)
            $b_{n}$ such that $0<|x-b_{n}|<\frac{1}{n+1}$ and:
            \begin{equation}
                \left|
                    L-\frac{f(b_{n})-f(x{0})}{b_{n}-x_{0}}
                \right|
                \geq\varepsilon
            \end{equation}
            But such a sequence must converge to $x_{0}$,
            $b_{n}\rightarrow{x}_{0}$, and since $f$ is differentiable at
            $x_{0}$ the sequence of ratios must converge to $L$, which is a
            contradiction. So the $\varepsilon-\delta$ criterion is satisfied.
            \par\hfill\par
            In the other direction, suppose $f$ satisfies the
            $\varepsilon-\delta$ criterion. We wish to prove that $f$ is
            differentiable at $x_{0}$. Let $a:\mathbb{N}\rightarrow{A}$ be
            a sequence such that $a_{n}\rightarrow{x}_{0}$ and
            $a_{n}\ne{x}_{0}$ for all $n\in\mathbb{N}$. Given
            $\varepsilon>0$, since the $\varepsilon-\delta$ property is
            satisfied, there exists $\delta>0$ such that
            $x\in{A}$ and $0<|x-x_{0}|<\delta$ implies
            $\left|L-\frac{f(x)-f(x_{0})}{x-x_{0}}\right|<\varepsilon$.
            But if $\delta>0$ and $a_{n}\rightarrow{x}_{0}$, then by the
            definition of convergence there is some $N\in\mathbb{N}$ such
            that $n\in\mathbb{N}$ and $n>N$ implies
            $|a_{n}-x_{0}|<\delta$. But then for all $n>N$ we have:
            \begin{equation}
                \left|
                    L-\frac{f(a_{n})-f(x{0})}{a_{n}-x_{0}}
                \right|<\varepsilon
            \end{equation}
            Hence the sequence of ratios converges and has $L$ as its limit.
            That is, $f$ is differentiable at $x_{0}$.
        \end{proof}
        \par\hfill\par
        \textbf{Problem:}
        \par\hfill\par
        Show that $f:\mathbb{R}\rightarrow\mathbb{R}$ defined by:
        \begin{equation}
            f(x)
            =\begin{cases}
                x^{2},&x\in\mathbb{Q}\\
                -x^{2},&x\not\in\mathbb{Q}
            \end{cases}
        \end{equation}
        is differentiable at the origin and nowhere else.
        \par\hfill\par
        \ifsolution
            \color{blue}
            \textit{Solution.}
            \par\hfill\par
            The function is wedged between $x^{2}$ and $-x^{2}$, both of
            which are pretty flat near the origin. It is likely that the
            derivative is $0$, if it exists at all. Given $\varepsilon>0$,
            we want to find some $\delta>0$ such that $0<|x-0|<\delta$ implies:
            \begin{equation}
                \begin{array}{lrcl}
                    &
                    \displaystyle
                    \left|
                        0-\frac{f(x)-f(0)}{x-0}
                    \right|
                    &<&
                    \varepsilon\\[1em]
                    \Longleftrightarrow
                    &
                    \displaystyle
                    \left|
                        \frac{\pm{x}^{2}}{x}
                    \right|
                    &<&
                    \varepsilon\\[1em]
                    \Longleftrightarrow
                    &
                    \displaystyle
                    \left|\pm{x}\right|
                    &<&
                    \varepsilon\\[1em]
                    \Longleftrightarrow
                    &
                    |x|
                    &<&
                    \varepsilon
                \end{array}
            \end{equation}
            Choosing $\delta=\varepsilon$ solves our problems. For any
            non-zero $x$ the function is not even continuous, meaning it cannot
            possibly be differentiable.
        \fi
\end{document}
