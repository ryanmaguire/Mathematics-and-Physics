%-----------------------------------LICENSE------------------------------------%
%   This file is part of Mathematics-and-Physics.                              %
%                                                                              %
%   Mathematics-and-Physics is free software: you can redistribute it and/or   %
%   modify it under the terms of the GNU General Public License as             %
%   published by the Free Software Foundation, either version 3 of the         %
%   License, or (at your option) any later version.                            %
%                                                                              %
%   Mathematics-and-Physics is distributed in the hope that it will be useful, %
%   but WITHOUT ANY WARRANTY; without even the implied warranty of             %
%   MERCHANTABILITY or FITNESS FOR A PARTICULAR PURPOSE.  See the              %
%   GNU General Public License for more details.                               %
%                                                                              %
%   You should have received a copy of the GNU General Public License along    %
%   with Mathematics-and-Physics.  If not, see <https://www.gnu.org/licenses/>.%
%------------------------------------------------------------------------------%
\documentclass{article}
\usepackage{graphicx}
\usepackage{amsmath, amsthm, amssymb}
\usepackage{xcolor}
\theoremstyle{normal}
\newtheorem{theorem}{Theorem}
\newtheorem{definition}{Definition}
\title{Integrability and Measure}
\author{Ryan Maguire}
\date{\today}
\setlength{\parindent}{0em}
\setlength{\parskip}{0em}
\newif\ifsolution
\solutiontrue
\graphicspath{{../../../images/}}

\begin{document}
    \maketitle
    \section{Integrability}
        From calculus we know that the definition of the integral follows
        from the intuitive notion of \textit{area under a curve},
        see Fig.~\ref{fig:integral_of_continuous_function}.
        \begin{figure}
            \centering
            \includegraphics{integral_of_continuous_function}
            \caption{Integration as Area Under a Curve}
            \label{fig:integral_of_continuous_function}
        \end{figure}
        A more rigorous definition is required if we are to prove things.
        In particular, our motivating goal is to prove the Riemann-Lebesgue
        theorem that says if a Fourier series converges, then the coefficients
        tend to zero. That is, given $f:[a,\,b]\rightarrow\mathbb{R}$, define:
        \begin{equation}
            a_{n}=
            \frac{1}{L}\int_{0}^{L}
                f(x)\sin\left(\frac{n\pi}{L}\right)\,\textrm{d}x
        \end{equation}
        If these coefficients produce a convergent Fourier sine series, meaning:
        \begin{equation}
            f(x)=\sum_{n=0}^{\infty}a_{n}\sin\left(\frac{n\pi{x}}{L}\right)
        \end{equation}
        then $a_{n}\rightarrow{0}$.
        \par\hfill\par
        Darboux integration, which is equivalent to Riemannian integration,
        only requires partitions (whereas Riemannian integration needs
        tagged partitions).
        \begin{definition}[\textbf{Partition}]
            A partition of an interval $[a,\,b]$ is a function
            $P:\mathbb{Z}_{n+1}\rightarrow[a,\,b]$ such that
            $a=P_{0}<P_{1}<\cdots<P_{n-1}<P_{n}=b$. That is, a finite sequence
            of points in $[a,\,b]$ that is strictly increasing and contains
            the two endpoints.
        \end{definition}
        The Darboux integral is defined using the upper and lower sums of a
        partition:
        \begin{equation}
            \begin{array}{rcl}
                \displaystyle
                L(f,\,P)
                &=&
                \displaystyle
                \sum_{k=0}^{n-1}
                    \inf\left(
                        \left\{\,
                            f(x)\;\big|\;P_{k}\leq{x}\leq{P}_{k+1}\,
                        \right\}
                    \right)
                    \Delta{x}_{k}\\[1.5em]
                    \displaystyle
                    U(f,\,P)
                    &=&
                    \displaystyle
                    \sum_{k=0}^{n-1}
                        \sup\left(
                            \left\{\,
                                f(x)\;\big|\;P_{k}\leq{x}\leq{P}_{k+1}\,
                            \right\}
                        \right)
                        \Delta{x}_{k}
            \end{array}
        \end{equation}
        Where $\Delta_{k}$ is the width of the subinterval, $P_{k+1}-P_{k}$.
        If there is a nested sequence of partitions $\mathfrak{P}$, meaning
        $\mathfrak{P}_{n}$ is a partition of $[a,\,b]$ for each
        $n\in\mathbb{N}$, and $\mathfrak{P}_{n+1}$ contains all of the points
        in $\mathfrak{P}_{n}$, such that:
        \begin{equation}
            \lim_{n\rightarrow\infty}\left(
                U(f,\,\mathfrak{P}_{n})-L(f,\,\mathfrak{P}_{n})
            \right)
            =0
        \end{equation}
        and such that the \textbf{mesh}, $m(\mathfrak{P}_{k})$, which is the
        maximum width $\Delta{x}_{k}$ for all subintervals in
        $\mathfrak{P}_{n}$, tends to zero as well,
        then the function is said to be Darboux (and hence Riemann) integrable.
        We write:
        \begin{equation}
            \lim_{n\rightarrow\infty}L(f,\,\mathfrak{P}_{n})
            =\lim_{n\rightarrow\infty}U(f,\,\mathfrak{P}_{n})
            =\int_{a}^{b}f(x)\,\textrm{d}x
        \end{equation}
        Prove that the popcorn function $f:[0,\,2]\rightarrow\mathbb{R}$
        given by:
        \begin{equation}
            f(x)=
            \begin{cases}
                \frac{1}{q},&x=p/q\textrm{ with }\operatorname{GCD}(p,\,q)=1\\
                0,&\textrm{otherwise}
            \end{cases}
        \end{equation}
        is Darboux integrable (see Fig.~\ref{fig:dirichlets_popcorn_function}).
        \begin{figure}
            \centering
            \includegraphics{dirichlets_popcorn_function}
            \caption{The Popcorn Function}
            \label{fig:dirichlets_popcorn_function}
        \end{figure}
        \ifsolution
            \color{blue}
            \par\hfill\par
            The integral is zero. There is a symmetry for the function between
            $[0,\,1]$ and $[1,\,2]$, so we need only prove the integral
            of $f$ on $[0,\,1]$ is zero. Let $\varepsilon>0$, and let
            $N\in\mathbb{N}^{+}$ be such that $\frac{2}{N+1}<\varepsilon$.
            Define $R$ to be the \textit{range} of $f$ with values greater
            than or equal to $1/N$. That is:
            \begin{equation}
                R=
                \left\{
                    \,
                    1,\,\frac{1}{2},\,\frac{1}{3},\,\frac{2}{3},\,
                    \frac{1}{4},\,\frac{3}{4},\,\cdots,\,
                    \frac{1}{N},\,\cdots,\,\frac{N-1}{N}
                \right\}
            \end{equation}
            Since we are now considering $x\in[0,\,1]$, if $x\not\in{R}$,
            then $f(x)<\frac{1}{N+1}$, and hence $f(x)<\frac{\varepsilon}{2}$.
            Let $M=\operatorname{Card}(R)$, the number of points in $R$,
            and choose a partition
            $P$ for $[0,\,1]$ consisting of equally spaced subintervals, each
            of width $\frac{\varepsilon}{4M}$. Since there are $M$ points in
            $R$, there are at most $2M$ subintervals of $P$ containing points
            in $R$. That is, any given point in $R$ may either lie in a single
            interval, or it may fall on the boundary of two different
            subintervals. If each point in $R$ lies on the boundary of two
            subintervals, then there would be $2M$ subintervals containing
            points in $R$. Hence, in general, there are at most $2M$
            subintervals containing points in $R$, but possibly fewer.
            On the subintervals which contain points in $R$, we may replace
            $\sup(\{\,f(x)\;|\;P_{k}\leq{x}\leq{P}_{k+1}\,\})$ with $1$, and on
            the remaining subintervals we may replace this supremum with
            $\frac{1}{N+1}$. This will produce and upper bound for the sum
            $U(f,\,P)$. Letting $n$ denote the total number of subintervals
            produced by the partition $P$, we have:
            \begin{equation}
                \begin{array}{rcl}
                    \displaystyle
                    U(f,\,P_{N})
                    &=&
                    \displaystyle
                    \sum_{k=0}^{n-1}
                        \sup\left(
                            \left\{\,
                                f(x)\;\big|\;P_{k}\leq{x}\leq{P}_{k+1}\,
                            \right\}
                        \right)
                        \Delta{x}_{k}\\[1.5em]
                    &=&
                    \displaystyle
                    \sum_{[x_{k},\,x_{k+1}]\cap{R}\ne\emptyset}
                        \sup\left(
                            \left\{\,
                                f(x)\;\big|\;P_{k}\leq{x}\leq{P}_{k+1}\,
                            \right\}
                        \right)
                        \Delta{x}_{k}\\[1.5em]
                    &&
                    \displaystyle
                    \quad
                    +\sum_{[x_{k},\,x_{k+1}]\cap{R}=\emptyset}
                        \sup\left(
                            \left\{\,
                                f(x)\;\big|\;P_{k}\leq{x}\leq{P}_{k+1}\,
                            \right\}
                        \right)
                        \Delta{x}_{k}\\[1.5em]
                    &\leq&
                    \displaystyle
                    \sum_{[x_{k},\,x_{k+1}]\cap{R}\ne\emptyset}
                        \Delta{x}_{k}
                    +\sum_{[x_{k},\,x_{k+1}]\cap{R}=\emptyset}
                        \sup\left(
                            \left\{\,
                                f(x)\;\big|\;P_{k}\leq{x}\leq{P}_{k+1}\,
                            \right\}
                        \right)
                        \Delta{x}_{k}\\[1.5em]
                    &\leq&
                    \displaystyle
                    \sum_{[x_{k},\,x_{k+1}]\cap{R}\ne\emptyset}
                        \Delta{x}_{k}
                    +\sum_{[x_{k},\,x_{k+1}]\cap{R}=\emptyset}
                        \frac{1}{N+1}
                        \Delta{x}_{k}\\[1.5em]
                    &\leq&
                    \displaystyle
                    2M\frac{\varepsilon}{4M}
                    +\sum_{[x_{k},\,x_{k+1}]\cap{R}=\emptyset}
                        \frac{1}{N+1}
                        \Delta{x}_{k}\\[1.5em]
                    &\leq&
                    \displaystyle
                    2M\frac{\varepsilon}{4M}+\frac{1}{N+1}\\[1.5em]
                    &=&
                    \displaystyle
                    \frac{\varepsilon}{2}+\frac{1}{N+1}\\[1.5em]
                    &<&
                    \frac{\varepsilon}{2}+\frac{\varepsilon}{2}\\[1.5em]
                    &=&
                    \displaystyle
                    \varepsilon
                \end{array}
            \end{equation}
            The upper sum is hence bounded by $\varepsilon$, and from the
            definition of $f$ the lower sum is always zero. The integral is
            therefore zero.
            \color{black}
        \fi
    \section{Measure}
        Riemann and Darboux integration are not the only forms of integration.
        A more powerful theory comes from Lebesgue which uses measures.
        A \textbf{measure space} is an ordered triple $(X,\,\mathcal{A},\,\mu)$
        where $X$ is a set, $\mathcal{A}\subseteq\mathcal{P}(X)$
        is something called a \textit{sigma algebra}, and
        $\mu:\mathcal{A}\rightarrow\mathbb{R}$ is a \textit{measure}. This is a
        generalization of the likely more familiar notion of probability spaces.
        $\mathcal{A}$ satisfies the following properties:
        \begin{itemize}
            \item
                $\emptyset\in\mathcal{A}$.
            \item
                $X\in\mathcal{A}$.
            \item
                If $A\in\mathcal{A}$, then so is $X\setminus{A}$.
            \item
                If $\mathcal{O}\subseteq\mathcal{A}$ is \textit{countable},
                then $\bigcup\mathcal{O}\in\mathcal{A}$ as well.
        \end{itemize}
        The connection to probability theory goes like this. The empty set is
        an \textit{event}, it is the \textit{nothing happens} event.
        Similarly, the entire set $X$ is an event, it is the
        \textit{anything happens} event. If $A$ is an event, meaning
        $A\in\mathcal{A}$, then the complement of $A$ is also an event, it
        is the \textit{not} $A$ event. Lastly, the union of events is also an
        event, it is the \textit{one of these happens} event.
        \par\hfill\par
        The \textit{measure} $\mu$, is a function
        $\mu:\mathcal{A}\rightarrow\mathbb{R}$. The connection to probability
        theory is that $\mu$ assigns a \textit{probability} to an event
        $A\in\mathcal{A}$. It must satisfy some of the familiar rules from
        probability theory:
        \begin{itemize}
            \item
                For all $A\in\mathcal{A}$, the measure of $A$ is non-negative.
                That is, $\mu(A)\geq{0}$. This means there are no negative
                lengths.
            \item
                Given a sequence of sets in $\mathcal{A}$, $A_{0}$, $A_{1}$,
                $\cdots$, the measure of the union is not greater than the
                sum of the individual measures. That is:
                \begin{equation}
                    \mu\left(\bigcup_{n=0}^{\infty}A_{n}\right)
                    \leq
                    \sum_{n=0}^{\infty}
                        \mu\left(A_{n}\right)
                \end{equation}
        \end{itemize}
        For $\mathbb{R}$, the standard measure is given by defining
        $\mu\big((a,\,b)\big)=b-a$. That is, the length of the interval
        $(a,\,b)$ is just $b-a$. Given a set $A$, an upper bound for the
        length of $A$ is obtained by covering $A$ with
        (possibly infinitely many) open intervals, $(a_{n},\,b_{n})$.
        The length of $A$ is then bounded by:
        \begin{equation}
            \mu(A)\leq\sum\left(b_{n}-a_{n}\right)
        \end{equation}
        The measure of $A$ is the infimum of the measures of all possible
        open covers of $A$.
        \par\hfill\par
        To show that $A$ has measure zero you must show that for any
        $\varepsilon>0$ there is a collection of open intervals that covers
        $A$ and such that the measure of this collection (the sum of the
        individual widths) is less than $\varepsilon$.
        \par\hfill\par
        We showed in class that the following function defined on $[0,\,1]$
        is not Riemann or Darboux integrable:
        \begin{equation}
            f(x)=
            \begin{cases}
                1,&x\in\mathbb{Q}\\
                0,&x\not\in\mathbb{Q}
            \end{cases}
        \end{equation}
        For any partition $P$ we have $U(f,\,P)=1$ and $L(f,\,P)=0$, meaning
        the lower sums and upper sums do not converge to the same value.
        The Lebesgue integral (which we have not formally defined) allows us
        to throw away sets of measure zero without changing the integral.
        If we can show that $\mu(\mathbb{Q})=0$, then the integral of
        $f$ must be zero as well since on $[0,\,1]\setminus\mathbb{Q}$ the
        function is just zero.
        \par\hfill\par
        Prove that $\mu(\mathbb{Q})=0$.
        \ifsolution
            \color{blue}
            \par\hfill\par
            This will be true for any countable subset of $\mathbb{R}$.
            Let $\varepsilon>0$ and pick a bijection
            $a:\mathbb{N}\rightarrow\mathbb{Q}$. Define $\mathcal{U}_{n}$ by:
            \begin{equation}
                \mathcal{U}_{n}=
                \left(
                    a_{n}-\frac{\varepsilon}{2^{n+2}},\,
                    a_{n}+\frac{\varepsilon}{2^{n+2}}
                \right)
            \end{equation}
            Then $\mu(\mathcal{U}_{n})=\varepsilon/2^{n+1}$. The sum is then:
            \begin{equation}
                \mu(\mathbb{Q})
                <\sum_{n=0}^{\infty}
                    \mu\left(\mathcal{U}_{n}\right)
                =\sum_{n=0}^{\infty}
                    \frac{\varepsilon}{2^{n+1}}
                =\varepsilon
            \end{equation}
        \fi
\end{document}
