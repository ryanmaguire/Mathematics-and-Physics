%-----------------------------------LICENSE------------------------------------%
%   This file is part of Mathematics-and-Physics.                              %
%                                                                              %
%   Mathematics-and-Physics is free software: you can redistribute it and/or   %
%   modify it under the terms of the GNU General Public License as             %
%   published by the Free Software Foundation, either version 3 of the         %
%   License, or (at your option) any later version.                            %
%                                                                              %
%   Mathematics-and-Physics is distributed in the hope that it will be useful, %
%   but WITHOUT ANY WARRANTY; without even the implied warranty of             %
%   MERCHANTABILITY or FITNESS FOR A PARTICULAR PURPOSE.  See the              %
%   GNU General Public License for more details.                               %
%                                                                              %
%   You should have received a copy of the GNU General Public License along    %
%   with Mathematics-and-Physics.  If not, see <https://www.gnu.org/licenses/>.%
%------------------------------------------------------------------------------%
\documentclass{article}
\usepackage{amssymb}
\usepackage{xcolor}
\title{Logic and Quantifiers}
\author{Ryan Maguire}
\date{\today}
\setlength{\parindent}{0em}
\setlength{\parskip}{0em}
\newif\ifsolution
\solutiontrue

\begin{document}
    \maketitle
    \section{Various Types of Logic}
        \begin{enumerate}
            \item
                Zeroth Order Logic
                \begin{itemize}
                    \item
                        We allow \textit{propositions} (sentences that can be
                        affirmed  true or false), \textit{inference rules}
                        (rules of deduction) and \textit{axioms} (statements we
                        accept as true without proof). No quantifying at all.
                    \item
                        Hilbert systems (without quantifiers) are a type of
                        zeroth order logic.
                    \item
                        Also called sentential logic and propositional logic.
                \end{itemize}
            \item
                First Order Logic
                \begin{itemize}
                    \item
                        Everything from zeroth order logic is permissible.
                    \item
                        We allow ourselves to use the terms \textit{for all}
                        or \textit{there exists} on \textbf{sets} or
                        \textbf{elements} of sets, but not
                        \textit{properties}.
                    \item
                        Most of the Peano axioms are first order. For example,
                        \textit{for all} $n\in\mathbb{N}$, \textit{for all}
                        $m\in\mathbb{N}$, \textit{if} $n+1=m+1$
                        \textit{is true, then} $n=m$ \textit{is true}.
                        This quantifies over the elements of the
                        natural numbers.
                    \item
                        Peano arithmetic is \textit{almost} a first order
                        system,  but the induction hypothesis quantifies over
                        \textit{predicates} on the natural numbers.
                \end{itemize}
            \item
                Second Order Logic
                \begin{itemize}
                    \item
                        Most of mathematics works in this framework.
                    \item
                        All of first order and zeroth order
                        logic is allowed.
                    \item
                        In addition, we can quantify over \textit{predicates},
                        or sentences
                        $P(x)$ that take in \textit{sets} or
                        \textit{elements} $x$ and such
                        that $P(x)$ can be asserted to be true or false.
                    \item
                        The final Peano axiom does this. It says:
                        \[
                            \forall(P)\Big(
                                P(0)
                                \land
                                \big(P(n)\Rightarrow{P}(n+1)\big)
                            \Big)
                            \Rightarrow
                            \forall(n\in\mathbb{N})\big(P(n)\big)
                        \]
                    \item
                        \textbf{DO NOT WRITE LIKE THIS}.
                        Write this out in your preferred
                        \textit{human readable} language. Since I prefer
                        English, this would read:
                        \begin{center}
                            For all predicates $P$
                            on the natural numbers, if $P(0)$ is true,
                            and if $P(n)$ being
                            true implies that $P(n+1)$ is true, then for all
                            $n\in\mathbb{N}$ the proposition $P(n)$ is true.
                        \end{center}
                        This is far easier for the readier to comprehend.
                \end{itemize}
        \end{enumerate}
    \section{Peano Arithmetic and Practice with Quantifiers}
        \begin{enumerate}
            \item
                Try to come up with some everyday examples of quantifiers. For
                example, for all \textit{towns in Massachusetts}, the town is
                \textit{north of the equator.} If we let $P(x)$ be the
                sentence $x$ \textit{is north of the equator},
                then this sentence can be cryptically re-written as:
                \[
                    \forall(x\in\textrm{MA})\big(P(x)\big)
                \]
                \begin{itemize}
                    \item
                        \textbf{Question:}
                        What type of logic does this statement fall under?
                        \ifsolution
                            \color{blue}
                            \item
                                \textbf{Answer}:
                                First order logic. We are quantifying
                                over \textit{things} (towns in Massachusetts),
                                but not \textit{properties about things}.
                                There is a property under consideration
                                (being north of the equator), but it is
                                \textit{fixed} and not being quantified over.
                            \color{black}
                        \fi
                \end{itemize}
            \item
                Try to provide an intuitive \textit{proof}
                of the induction axiom.
                \begin{itemize}
                    \item
                        \textbf{Question:}
                        What does the negation of Peano's induction axiom say?
                        \ifsolution
                            \color{blue}
                            \item
                                \textbf{Answer:}
                                There exists a sentence $P$ for the natural
                                numbers such that $P(0)$ is true, and
                                $P(n)$ implies $P(n+1)$,
                                but $P$ is \textbf{not} true for all $n$.
                            \color{black}
                        \fi
                    \item
                        \textbf{Question:}
                        Can you describe an \textit{algorithm}
                        to show that the negation is false?
                        \ifsolution
                            \color{blue}
                            \item
                                \textbf{Answer:}
                                We are told $P(0)$ is true. Suppose there
                                is some $n$ such that $P(n)$ is false.
                                Could $n=1$? No, because $P(0)$ is true,
                                and $P(0)$ implies $P(1)$ is true, and
                                therefore by \textit{modus ponens}, $P(1)$ is
                                also true. Could $P(2)$ be false?
                                No, because $P(1)$ is true, and we know that
                                $P(1)$ implies $P(2)$. If we iterate
                                through the natural numbers we will find that
                                $P(n)$ can never be \textit{false}, and so
                                $P(n)$ is true for all $n$.
                                \color{black}
                            \item
                                \textbf{NOTE:}
                                This is not a \textit{proof} of the last Peano
                                axiom since there is no axiom saying such an
                                algorithm is permissible in mathematics.
                                Indeed, the axiom that this is legal is indeed
                                Peano's axiom, and this argument is
                                \textit{circular}, but it provides intuition
                                as to why we accept it as true.
                        \fi
                \end{itemize}
            \item
                Consider the statement:
                \[
                    \forall(P)(P\lor\neg{P})
                \]
                That is, for all propositions $P$, either $P$ is true,
                or $P$ is false. This is the
                \textit{law of the excluded middle}.
                \begin{itemize}
                    \item
                        \textbf{Question:}
                        What type of logic is this?
                        \ifsolution
                            \color{blue}
                            \item
                                \textbf{Answer:}
                                Second order.  We are quantifying over all
                                \textit{properties}.
                            \color{black}
                        \fi
                \end{itemize}
        \end{enumerate}
\end{document}
