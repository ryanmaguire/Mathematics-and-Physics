%-----------------------------------LICENSE------------------------------------%
%   This file is part of Mathematics-and-Physics.                              %
%                                                                              %
%   Mathematics-and-Physics is free software: you can redistribute it and/or   %
%   modify it under the terms of the GNU General Public License as             %
%   published by the Free Software Foundation, either version 3 of the         %
%   License, or (at your option) any later version.                            %
%                                                                              %
%   Mathematics-and-Physics is distributed in the hope that it will be useful, %
%   but WITHOUT ANY WARRANTY; without even the implied warranty of             %
%   MERCHANTABILITY or FITNESS FOR A PARTICULAR PURPOSE.  See the              %
%   GNU General Public License for more details.                               %
%                                                                              %
%   You should have received a copy of the GNU General Public License along    %
%   with Mathematics-and-Physics.  If not, see <https://www.gnu.org/licenses/>.%
%------------------------------------------------------------------------------%
\documentclass{article}
\usepackage{xcolor}
\usepackage{hyperref}
\title{Communicating Mathematics}
\author{Ryan Maguire}
\date{\today}
\setlength{\parindent}{0em}
\setlength{\parskip}{0em}
\newif\ifsolution
\solutiontrue

\begin{document}
    \maketitle
    \ifsolution
        \color{blue}
        Ask the students what they should expect in their writings.
        Consider things such as:
        \begin{itemize}
            \item
                Length of the paper.
            \item
                How much can be assumed in their writing?
                They do not need to reinvent the wheel, but the reading
                should be mostly self-contained.
            \item
                Which symbols are allowed? Their papers should not
                overly abuse notation (symbols like $\forall$ or
                $\exists$) and should revert to English when possible.
                If symbols \textit{help} the reader, they can be used.
        \end{itemize}
        \color{black}
    \fi
    Read the following short paper,
    \href{%
        https://arxiv.org/pdf/math/9608204%
    }{A Nonstandard Proof of the Jordan Curve Theorem}.
    Ask yourself the following.
    \begin{itemize}
        \item
            What is the purpose of the paper?
        \item
            Which areas of mathematics may benefit from the result?
        \item
            How is the main result proved?
        \item
            What is the notation used by the authors? Is this notation
            common in other areas of mathematics? Is it dated, or still used
            in modern mathematics?
    \end{itemize}
    No writing is perfect. Critique the authors style.
    \begin{itemize}
        \item
            Do you wish the authors phrased something differently?
        \item
            Are the some paragraphs that are more confusing than authors? Why?
    \end{itemize}
    \ifsolution
    \color{blue}
        At the end of the recitation, stress some of the writing requirements.
        \begin{itemize}
            \item
                Set up definitions. They may not assume the reader is another
                student in 18.100P. If something needs defining, it should be
                defined.
            \item
                The writing is in \LaTeX{}. They should use the
                \texttt{amsmath} and \texttt{amsthm} environments to set up
                theorem / proof environments. This helps the reader easily
                spot where the main result in the paper is, and helps them
                organize their writing.
            \item
                Stress that good writing often does not overly use abstract
                mathematical notations. Good writing has a nice blend of
                notation and \textit{words}. Pictures are also nice, but not
                required at this level.
        \end{itemize}
        \color{black}
    \fi

\end{document}
