%-----------------------------------LICENSE------------------------------------%
%   This file is part of Mathematics-and-Physics.                              %
%                                                                              %
%   Mathematics-and-Physics is free software: you can redistribute it and/or   %
%   modify it under the terms of the GNU General Public License as             %
%   published by the Free Software Foundation, either version 3 of the         %
%   License, or (at your option) any later version.                            %
%                                                                              %
%   Mathematics-and-Physics is distributed in the hope that it will be useful, %
%   but WITHOUT ANY WARRANTY; without even the implied warranty of             %
%   MERCHANTABILITY or FITNESS FOR A PARTICULAR PURPOSE.  See the              %
%   GNU General Public License for more details.                               %
%                                                                              %
%   You should have received a copy of the GNU General Public License along    %
%   with Mathematics-and-Physics.  If not, see <https://www.gnu.org/licenses/>.%
%------------------------------------------------------------------------------%
\documentclass{article}
\usepackage{amsmath, amsthm, amssymb}
\usepackage{xcolor}
\theoremstyle{normal}
\newtheorem{theorem}{Theorem}
\newtheorem{definition}{Definition}
\title{Limits}
\author{Ryan Maguire}
\date{\today}
\setlength{\parindent}{0em}
\setlength{\parskip}{0em}
\newif\ifsolution
\solutionfalse

\begin{document}
    \maketitle
    \section{Cauchy Sequences}
        We defined real numbers using equivalence classes of Cauchy
        sequences. Here's a quick reminder on some of the definitions.
        \begin{definition}
            A \textbf{sequence} in a set $A$ is a function
            $a:\mathbb{N}\rightarrow{A}$.
        \end{definition}
        \begin{definition}
            A \textbf{convergent sequence} in
            $\mathbb{Q}$ is sequence $a:\mathbb{N}\rightarrow\mathbb{Q}$ such
            that there exists a rational number $L\in\mathbb{Q}$ with the
            property that for all $\epsilon>0$ there is an index
            $N\in\mathbb{N}$ such that $n\in\mathbb{N}$ and $n>N$ implies
            $|a_{n}-L|<\varepsilon$.
        \end{definition}
        \begin{definition}
            A \textbf{Cauchy sequence} in $\mathbb{Q}$ is a sequence
            $a:\mathbb{N}\rightarrow\mathbb{Q}$ such that for all
            $\varepsilon>0$ there exists an index $N\in\mathbb{N}$ such that
            for all $n,m\in\mathbb{N}$ with $n,m>N$ we have
            $|a_{n}-a_{m}|<\varepsilon$.
        \end{definition}
        We can be more general and replace $\mathbb{Q}$ with any set that has
        a \textit{distance function}. For the real and rational numbers this
        is the absolute value function, and for the Euclidean / complex plane
        this is given by the Pythagorean formula:
        \begin{equation}
            \textrm{dist}\Big((x_{0},\,y_{0}),\,(x_{1},\,y_{1})\Big)
            =\sqrt{(x_{1}-x_{0})^{2}+(y_{1}-y_{0})^{2}}
        \end{equation}
        This is more commonly written using the \textit{Euclidean norm}:
        \begin{equation}
            \|(x,\,y)\|=\sqrt{x^{2}+y^{2}}
        \end{equation}
        The distance is then:
        \begin{equation}
            \textrm{dist}(\mathbf{x},\,\mathbf{y})
            =\|\mathbf{x}-\mathbf{y}\|
        \end{equation}
        Because of this we may speak about Cauchy sequences in
        $\mathbb{Q}$, $\mathbb{R}$, and $\mathbb{C}$ freely. $\mathbb{Q}$ has
        a problem: Not all Cauchy sequences converge. Applying Heron's formula
        to $x_{0}=2$ gives us a sequence of rationals that better approximates
        $\sqrt{2}$ (via a Cauchy sequence), but there is no rational that
        squares to two. Regardless, we always have the following.
        \par\hfill\par
        \textbf{Problem 1:}
        Prove the following theorem.
        \begin{theorem}
            If $a:\mathbb{N}\rightarrow{X}$ is a convergent sequence
            ($X=\mathbb{Q}$, $X=\mathbb{R}$, or $X=\mathbb{C}$), then it is
            a Cauchy sequence.
        \end{theorem}
        \ifsolution
            \color{blue}
            \begin{proof}
                Since $a$ converges, there is a limit $L$. But then for all
                $\varepsilon>0$ there is an $N\in\mathbb{N}$ such that
                $n\in\mathbb{N}$ and $n>N$ implies $|a_{n}-L|<\varepsilon/2$.
                Let $n,m\in\mathbb{N}$ be two natural numbers that are
                larger than $N$. Then:
                \begin{align}
                    |a_{n}-a_{m}|
                    &=|a_{n}-L+L-a_{m}|\\
                    &\leq|a_{n}-L|+|L-a_{m}|\\
                    &<\frac{\varepsilon}{2}+\frac{\varepsilon}{2}\\
                    &=\varepsilon
                \end{align}
                That is, the sequence $a$ is a Cauchy sequence.
            \end{proof}
            \color{black}
        \fi
        \par\hfill\par
        \textbf{Problem 2:}
        \par
        Consider $a:\mathbb{N}\rightarrow\mathbb{R}$ given by
        $a_{n}=\ln(n+2)-\ln(n+1)$. Is this a convergent sequence?
        Prove or disprove this.
        \par\hfill\par
        \ifsolution
            \color{blue}
            The natural logarithm gets very flat as we go to infinity,
            meaning the difference $\ln(n+2)-\ln(n+1)$ gets very small.
            Zero is a likely candidate for the limit. Suppose we know
            $|\ln(n+2)-\ln(n+1)|<\varepsilon$ for large $n$. What does this
            say? We have:
                \begin{align}
                    &&
                    |\ln(n+2)-\ln(n+1)|
                    &<
                    \varepsilon\\
                    &
                    \Longrightarrow
                    &
                    \ln(n+2)-\ln(n+1)
                    &<
                    \varepsilon\\
                    &
                    \Longrightarrow
                    &
                    \ln\left(\frac{n+2}{n+1}\right)
                    &<
                    \varepsilon\\
                    &
                    \Longrightarrow
                    &
                    \frac{n+2}{n+1}<\exp(\varepsilon)
                    &&
                \end{align}
                This is a little hard to work with, so let's suppose that
                we've already shown $\frac{n+2}{n}<\exp(\varepsilon)$. If this
                were true, than since $\frac{n+2}{n+1}<\frac{n+2}{n}$ we would
                be able to conclude $\frac{n+2}{n+1}<\varepsilon$. We now
                divert our attention to this new inequality. We continue:
                \begin{align}
                    &&
                    \frac{n+2}{n}
                    &<\exp(\varepsilon)\\
                    &
                    \Longrightarrow
                    &
                    1+\frac{2}{n}
                    &<\exp(\varepsilon)\\
                    &
                    \Longrightarrow
                    &
                    \frac{2}{n}
                    &<
                    \exp(\varepsilon)-1\\
                    &
                    \Longrightarrow
                    &
                    n
                    &>
                    \frac{2}{\exp(\varepsilon)-1}
                \end{align}
                This was very backward reasoning, and not quite valid, but we
                now have a candidate for $N$. Given $\varepsilon>0$, choose:
                \begin{equation}
                    N=\textrm{ceil}\left(
                        \frac{2}{\exp(\varepsilon)-1}
                    \right)
                \end{equation}
                Now we begin the rigorous (and laborious) process of proving
                $a_{n}\rightarrow{0}$. Given $n>N$ we have:
                \begin{align}
                    |\ln(n+2)-\ln(n+1)-0|
                    &=\ln(n+2)-\ln(n+1)\\
                    &=\ln\left(
                        \frac{n+2}{n+1}
                    \right)\\
                    &<\ln\left(
                        \frac{n+2}{n}
                    \right)\\
                    &=
                    \ln\left(1+\frac{2}{n}\right)\\
                    &<\ln\left(1+\frac{2}{N}\right)\\
                    &=\ln\left(
                        1+\
                        \frac{2}{
                            \textrm{ceil}\left(
                                \frac{2}{\exp(\varepsilon)-1}
                            \right)
                        }
                    \right)\\
                    &<\ln\left(
                        1+\
                        \frac{2}{\frac{2}{\exp(\varepsilon)-1}}
                    \right)\\
                    &=\ln\left(\exp(\varepsilon)\right)\\
                    &=\varepsilon
                \end{align}
            \color{black}
        \fi
        \par\hfill\par
        \textbf{Problem 3:}
        \par
        Define $a_{n}=\exp(n+1)-\exp(n)$. Does this converge? Prove or
        disprove this.
        \par\hfill\par
        \ifsolution
            It does not. The negation of convergence is that for all
            $L\in\mathbb{R}$ there exists an $\varepsilon>0$ such that for all
            $N\in\mathbb{N}$ there is an $n\in\mathbb{N}$ with $n>N$ such that
            $|L-a_{n}|\geq\varepsilon$. But we do not need to rely on the
            definition since we have proven that a convergent sequence is a
            Cauchy sequence, and hence only need to show that this is not
            Cauchy. Pick $\varepsilon=1$. Given $N\in\mathbb{N}$ choose
            $n=N+1$ and $m=N+2$. The difference $|a_{n}-a_{m}|$ becomes
            arbitrary large as $N$ increased, so this is not a Cauchy sequence.
        \fi
\end{document}
