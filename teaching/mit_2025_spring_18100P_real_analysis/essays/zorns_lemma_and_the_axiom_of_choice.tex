%-----------------------------------LICENSE------------------------------------%
%   This file is part of Mathematics-and-Physics.                              %
%                                                                              %
%   Mathematics-and-Physics is free software: you can redistribute it and/or   %
%   modify it under the terms of the GNU General Public License as             %
%   published by the Free Software Foundation, either version 3 of the         %
%   License, or (at your option) any later version.                            %
%                                                                              %
%   Mathematics-and-Physics is distributed in the hope that it will be useful, %
%   but WITHOUT ANY WARRANTY; without even the implied warranty of             %
%   MERCHANTABILITY or FITNESS FOR A PARTICULAR PURPOSE.  See the              %
%   GNU General Public License for more details.                               %
%                                                                              %
%   You should have received a copy of the GNU General Public License along    %
%   with Mathematics-and-Physics.  If not, see <https://www.gnu.org/licenses/>.%
%------------------------------------------------------------------------------%
\documentclass{article}
\usepackage{geometry}
\geometry{margin = 1.5in}
\usepackage{amsthm}
\usepackage{amssymb}
\usepackage{xcolor}
\usepackage{hyperref}
\title{Zorn's Lemma and the Axiom of Choice}
\author{MIT 18.100P}
\date{\today}
\setlength{\parindent}{0em}
\setlength{\parskip}{0em}
\newtheorem*{theorem*}{Theorem}
\newtheorem*{axiom*}{Axiom}

\begin{document}
    \maketitle
    Zorn's lemma is equivalent to the axiom of choice. You assignment is
    to write an article proving this, written in a general style that does
    not assume the reader is also taking this course.
    \par\hfill\par
    The axiom of choice states:
    \begin{axiom*}
        If $\mathcal{O}$ is a non-empty set such that each
        $\mathcal{U}\in\mathcal{O}$ is also non-empty, then there is a
        function $f:\mathcal{O}\rightarrow\bigcup\mathcal{O}$, called a
        \textit{choice function}, such that $f(\mathcal{U})\in\mathcal{U}$
        for all $\mathcal{U}\in\mathcal{O}$.
    \end{axiom*}
    Note that the \textit{product} of a collection of sets is defined by:
    \[
        \prod\mathcal{O}
        =\left\{\,
            f:\mathcal{O}\rightarrow\bigcup\mathcal{O}\;\Big|\;
                f\textrm{ is a choice function}
        \right\}
    \]
    The axiom of choice then reads:
    \begin{center}
        \textit{The product of non-empty sets is non-empty}.
    \end{center}
    Zorn's lemma then states the following.
    \begin{theorem*}
        If $(X,\,\leq)$ is a partially ordered set such that $X$ is non-empty,
        and such that for each chain $C\subseteq{X}$ there exists an upper
        bound for $C$, then there exists a maximal element
        $x\in{X}$. That is, there is an element $x$ is such that for all
        $y\in{X}$, $x\leq{y}$ implies $x=y$.
    \end{theorem*}
    Both Zorn's lemma and the axiom of choice have repeated use in real
    analysis, so it is worthwhile to absorb the material and think critically
    about your arguments. You may use any external resource while trying to
    learn \textit{how} to prove Zorn's lemma, but
    \textbf{you must use your own words when writing your article}.
    This is such a well-known and well-studied theorem that there are
    thousands of references, and the Wikipedia article gives an excellent
    starting point (\url{https://en.wikipedia.org/wiki/Zorn's_lemma}).
    \par\hfill\par
    \textbf{\Large{Guidelines}}
    \par\hfill\par
    Your article must be written in \LaTeX{} or \TeX, the primary typesetting
    languages for mathematics and other sciences. \LaTeX{} is preferred since it
    is much easier to hit the ground running and start writing documents.
    \begin{itemize}
        \item
            Your article is expected to be at least 3 pages.
            It would be hard to use less than 3 pages, but if you have
            a good argument and a solid article, do not feel like you
            must stretch it out just to meet a quota. This can actually
            \textit{harm} your writing.
        \item
            50\% of the grade is for \textit{mathematical correctness}.
            This addresses the following:
            \begin{itemize}
                \item
                    Are you using definitions correctly?
                \item
                    Are you applying axioms and other theorems
                    appropriately in your arguments?
                \item
                    Is your proof valid?
                \item
                    If you are using figures, are they relevant and explained
                    properly?
            \end{itemize}
        \item
            50\% comes from \textit{writing style}:
            \begin{itemize}
                \item
                     Does the paper have an introduction that identifies
                     and signals the purpose and the main result for a
                     general audience (i.e. not someone currently enrolled
                     in 18.100P or necessarily currently taking real analysis)?
                \item
                    Does the paper set up key definitions and give the
                    intuition of the approach used to prove the result
                    so that the reader can point to the places in the
                    text where these are given?
                \item
                    Does the paper use of statement environments
                    and proof environments?
                \item
                    Does the paper write in paragraphs,
                    integrating words and symbols?
            \end{itemize}
    \end{itemize}
    \par\hfill\par
    \textbf{\Large{Deadlines}}
    \par\hfill\par
    \textbf{Rough Draft: 2025/02/21}
    \par
    A full draft of your article. You will do peer review with your fellow
    students during the recitation, and then the course team will provide
    feedback on your work.
    \par\hfill\par
    \textbf{Feedback Returned: 2025/02/28}
    \par
    Your rough draft will be returned with comments from the course team.
    \par\hfill\par
    \textbf{Final Due: 2025/03/07}
    \par
    After making revisions, you will turn in your final draft for grading.
\end{document}
