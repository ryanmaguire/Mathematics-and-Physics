%-----------------------------------LICENSE------------------------------------%
%   This file is part of Mathematics-and-Physics.                              %
%                                                                              %
%   Mathematics-and-Physics is free software: you can redistribute it and/or   %
%   modify it under the terms of the GNU General Public License as             %
%   published by the Free Software Foundation, either version 3 of the         %
%   License, or (at your option) any later version.                            %
%                                                                              %
%   Mathematics-and-Physics is distributed in the hope that it will be useful, %
%   but WITHOUT ANY WARRANTY; without even the implied warranty of             %
%   MERCHANTABILITY or FITNESS FOR A PARTICULAR PURPOSE.  See the              %
%   GNU General Public License for more details.                               %
%                                                                              %
%   You should have received a copy of the GNU General Public License along    %
%   with Mathematics-and-Physics.  If not, see <https://www.gnu.org/licenses/>.%
%------------------------------------------------------------------------------%
\documentclass{article}
\usepackage{geometry}
\geometry{a4paper, margin = 1in}
\usepackage{amsthm}
\usepackage{amssymb}
\usepackage{amsmath}
\usepackage[nottoc]{tocbibind}
\usepackage{graphicx}
\usepackage{hyperref}
\hypersetup{colorlinks = true, linkcolor = blue}
\title{The Cauchy-Riemann Equations}
\author{Ryan Maguire}
\date{\today}
\setlength{\parindent}{0em}
\setlength{\parskip}{0em}
\newtheorem{theorem}{Theorem}[section]
\newtheorem{axiom}{Axiom}[section]
\theoremstyle{definition}
\newtheorem{definition}{Definition}[section]
\graphicspath{{../../../images/}}
\newcommand*{\twiddle}[1]{%
    \tilde{#1}
}

\begin{document}
    \maketitle
    \begin{abstract}
        Over the past few centuries complex analysis has found its way into all
        of the physical and applied sciences. From electromagnetic waves and
        circuits to computer graphics, the ability to add and multiply points
        in the plane in a meaningful and geometry way has far applications,
        but perhaps most import is the ability to differentiate functions of
        a complex variable. It is well known that a function
        $f:\mathbb{C}\rightarrow\mathbb{C}$ is differentiable if and only if
        it satisfies the Cauchy-Riemann equations. We prove this rigorously
        in a mostly self-contained manner and provide physical geometric
        interpretations of this theorem.
    \end{abstract}
    \tableofcontents
    \section{Introduction}
        There is a natural way to add points in the Euclidean plane
        $\mathbb{R}^{2}$ by using \textit{vector addition}. Given
        $P=(x_{0},\,y_{0})$ and $Q=(x_{1},\,y_{1})$, we may sum $P$ and $Q$
        together by simply adding their components:
        \begin{equation}
            P+Q
            =(x_{0},\,y_{0})+(x_{1},\,y_{1})
            =(x_{0}+x_{1},\,y_{0}+y_{1})
        \end{equation}
        See Fig.~\ref{fig:complex_number_addition_001}.
        \begin{figure}
            \centering
            \includegraphics{complex_number_addition_001}
            \caption{Visual for Vector Addition in $\mathbb{R}^{2}$}
            \label{fig:complex_number_addition_001}
        \end{figure}
        The definiiton of multiplying points is not so clear at first glance.
        Cardano, in his \textit{Ars Magna} (1545 C.E.), writes that
        one may solve $x(10-x)=40$ if they allow for negative square roots,
        such as $5-\sqrt{-15}$ and $5+\sqrt{-15}$
        \cite[p.~252]{KlineMathematicalHistory}.
        History, solutions to cubic equations of the form
        $x^{3}+mx=n$ lead to auxiliary terms $t$ and $u$ given by
        \cite[p.~264]{KlineMathematicalHistory}:
        \begin{equation}
            \begin{array}{rcl}
                \displaystyle
                t
                &=&
                \displaystyle
                \sqrt{\left(\frac{n}{2}\right)^{2}+\left(\frac{m}{3}\right)^{3}}
                +\frac{n}{2}\\[1.5em]
                \displaystyle
                u
                &=&
                \displaystyle
                \sqrt{\left(\frac{n}{2}\right)^{2}+\left(\frac{m}{3}\right)^{3}}
                -\frac{n}{2}\\
            \end{array}
        \end{equation}
        For a cubic such as $x^{3}-6x-2=0$, which has three \textit{real} roots,
        we require square roots of \textit{negative} numbers. The multiplicative
        operation on the complex plane can be defined by modeling this behavior.
        We set $i=(0,\,1)$ and require $i^{2}=-1$. Algebraically, we are
        constructing a \textit{field extension} over the real numbers,
        $\mathbb{C}=\mathbb{R}[i]$, where we are \textit{adjoining} $i$ to
        $\mathbb{R}$ and writing $z\in\mathbb{C}$ as $z=a+ib$ with
        $a,b\in\mathbb{R}$ \cite[p.~512]{DummitAndFooteAbstractAlgebra}.
        While this is very formal and precise, it is far from concrete or
        geometric. If we set out to define a multiplication on $\mathbb{R}^{2}$,
        then by the Frobenius theorem \cite[p.~452]{JacobsonBasicAlgebraI}
        there is only one way (up to isomorphism) to this in a \textit{nice}
        manner. That is, to produce a field structure
        $(\mathbb{R}^{2},\,+,\,\times)$ that is \textit{compatible} with the
        real number system $(\mathbb{R},\,+,\,\times)$, there must be some
        fixed negative real number $r$ such that:
        \begin{equation}
            (x_{0},\,y_{0})\times(x_{1},\,y_{1})
            =(x_{0}y_{1}+ry_{0}y_{1},\,x_{0}y_{1}+x_{1}y_{0})
        \end{equation}
        By compatible, formally we mean that the complex numbers form an
        \textit{algebra} over the reals.
        Algebraically any choice of $r<0$ will do.
        Choosing $r=-1$ is the geometrically reasonable selection since now
        our multiplication is compatible with the Pythagorean formula:
        \begin{equation}
            \|(x_{0},\,y_{0})\times(x_{1},\,y_{1})\|
            =\|(x_{0},\,y_{0})\|\;\|(x_{1},\,y_{1})\|
        \end{equation}
        Where $\|\cdot\|$ denotes the Euclidean norm given by the
        Pythagorean formula. This produces the same structure as the more
        abstract field extension method, we may write
        $z=(x,\,y)=x+iy$ to jump back and forth between the two notions.
        By treating complex numbers purely as points in the plane, there is
        a third interpretations that is very geometric. Complex numbers are
        \textit{matrices} (see \cite[p.~25]{PetersenLinearAlgebra} for a
        simple example). Since multiplication satisfies the
        distributive law, $(z_{0}+z_{1})w=z_{0}w+z_{1}w$, we may think of
        $w=a+ib$ as a linear function
        $f:\mathbb{R}^{2}\rightarrow\mathbb{R}^{2}$ given by $f(z)=zw$.
        There is thus a representative matrix. To calculate the entries
        we simply compute $f\big((1,\,0)\big)$ and $f\big((0,\,1)\big)$,
        giving us:
        \begin{equation}
            w=
            \begin{bmatrix}
                a&-b\\
                b&\phantom{+}a
            \end{bmatrix}
        \end{equation}
        If $\|w\|=1$, we may write $w=\cos(\theta)+i\sin(\theta)$ for some
        real angle $\theta$. The representative matrix is then:
        \begin{equation}
            w=
            \begin{bmatrix}
                \cos(\theta)&-\sin(\theta)\\[0.5em]
                \sin(\theta)&\phantom{+}\cos(\theta)
            \end{bmatrix}
        \end{equation}
        This is a \textbf{rotation matrix}. That is, complex arithmetic
        describes geometric transformations of the plane. The combination
        of these interpretations result in complex analysis invading every
        branch of mathematics: algebra, analysis, combinatorics, geometry,
        mathematical physics, number theory, topology, and applied mathematics.
        In physics we describe electromagnetic waves using complex numbers
        \cite[p.~380]{WangsnessElectromagneticFields}, the central postulate
        behind quantum mechanics is the Schr\"{o}dinger equation which models
        quantum systems using complex numbers
        (\cite[p.~1]{GriffithsQuantumMechanics},
        \cite[p.~68]{McIntyreQuantumMechanics}), in acoustics and
        audio engineering the Fourier transform is used decompose sounds into
        their component frequencies
        \cite[p.~28]{MorseIngbardTheoreticalAcoustics}, and in computer
        graphics we use complex numbers (and \textit{quaternions}) to describe
        how we scale and rotate vectors. The list of applications goes on
        and on. Most of these applications use more than the fact that complex
        arithmetic has a geometric meaning, they make regular use of the fact
        that \textit{ratios} are well-defined when the denominator is
        non-zero. To keep the notation fixed, we provide the following
        definition.
        \begin{definition}[\textbf{Complex Numbers}]
            A complex number is a point in the plane $z\in\mathbb{R}^{2}$
            represented by writing $z=x+iy$ with $x,y\in\mathbb{R}$ where $x$
            is the \textit{real part} and $y$ is the \textit{imaginary part}.
            Arithmetic is given by the associative, commutative, and
            distributive laws (i.e., real arithmetic plus factoring the $i$
            symbol), and by using $i^{2}=-1$. We write
            $\mathbb{R}^{2}=\mathbb{C}$ when we explicitly wish to think of
            points as complex numbers, rather than geometric points in the
            Euclidean plane.
        \end{definition}
        This arithmetic allows us to define division as follows.
        Given $z=x+iy$, with $(x,\,y)\ne(0,\,0)$, we have:
        \begin{equation}
            z^{-1}=\frac{x-iy}{x^{2}+y^{2}}
        \end{equation}
        The geometric interpretation is shown in
        Fig.~\ref{fig:complex_number_division_001}. We subtract angles and
        scale by the ratio of the magnitudes for the two vectors.
        \begin{figure}
            \centering
            \includegraphics{complex_number_division_001}
            \caption{Visual for Division of Complex Numbers}
            \label{fig:complex_number_division_001}
        \end{figure}
        There is a straight-forward metric we can place on the complex plane
        given by the Pythagorean formula. Namely, given
        $z=x_{0}+iy_{0}$ and $w=x_{1}+iy_{1}$, we write:
        \begin{equation}
            \operatorname{dist}(z,\,w)
            =\|(x_{0},\,y_{0})-(x_{1},\,y_{1})\|
            =\sqrt{(x_{0}-x_{1})^{2}+(y_{0}-y_{1})^{2}}
        \end{equation}
        The \textbf{modulus} of a complex number $z=x+iy$ is given by
        $|z|=\|(x,\,y)\|$, meaning we may also write
        $\operatorname{dist}(z,\,w)=|z-w|$. With this, convergence of sequences
        is well-defined, and the complex numbers have a notion of
        \textit{Cauchy sequences} as well. Indeed, the complex numbers are
        \textbf{complete}: all Cauchy sequences converge
        \cite[p.~33]{AlhforsComplexAnalysis}. Continuity may then be defined
        as follows.
        \begin{definition}[\textbf{Continuous Function}]
            A continuous complex-valued function of a complex variable is a
            function $f:\mathbb{C}\rightarrow\mathbb{C}$ such that for every
            convergent sequence $a:\mathbb{N}\rightarrow\mathbb{C}$ we have:
            \begin{equation}
                \lim_{n\rightarrow\infty}f(a_{n})
                =f\left(\lim_{n\rightarrow\infty}a_{n}\right)
            \end{equation}
            That is, the image of a convergent sequence under $f$ is again a
            convergent sequence.
        \end{definition}
        \subsection{Differentiation}
            Since the complex numbers form a field, we may take two sequences
            of non-zero complex numbers and take their quotient. Differentiation
            may be defined in the exact same manner as real numbers in this
            respect.
            \begin{definition}[\textbf{Differentiable Function}]
                A differentiable complex-valued function of a complex variable
                is a function $f:\mathbb{C}\rightarrow\mathbb{C}$ such that
                for all $z_{0}\in\mathbb{C}$ and for any sequence
                $a:\mathbb{N}\rightarrow\mathbb{C}$ that convergences to
                $z_{0}$, the sequence of ratios converges. That is:
                \begin{equation}
                    f^{\prime}(z_{0})
                    =\lim_{n\rightarrow\infty}
                    \frac{f(a_{n})-f(z_{0})}{a_{n}-z_{0}}
                \end{equation}
                is well-defined.
            \end{definition}
            Checking differentiability using only the definition is far more
            straight forward for real-valued functions than in the complex
            setting. There are only two directions one may approach a real
            number, from the left and from the right. There are infinitely
            many directions to approach a fixed complex number from, which
            dramatically increases the difficulty. The
            \textbf{Cauchy-Riemann equations} are the primary tool used to
            overcome this hurdle.
        \subsection{The Cauchy-Riemann Equations}
        \subsection{Harmonic Functions}
    \section{Sketch of Proof}
    \section{Proof of the Cauchy-Riemann Theorem}
        We now prove the crucial main theorem, providing a necessary and
        sufficient condition of differentiability.
        \begin{theorem}[\textbf{The Cauchy-Riemann Equations: Part 1}]
            If $f:\mathbb{C}\rightarrow\mathbb{C}$ is given by
            $f(x+iy)=u(x,\,y)+iv(x,\,y)$, and if $f$ is differentiable, then
            the Cauchy-Riemann equations are satisfied:
            \begin{equation}
                \begin{array}{rcl}
                    \displaystyle
                    \frac{\partial{u}}{\partial{x}}
                    &=&
                    \displaystyle
                    \phantom{+}
                    \frac{\partial{v}}{\partial{y}}\\[1.5em]
                    \displaystyle
                    \frac{\partial{u}}{\partial{y}}
                    &=&
                    \displaystyle
                    -\frac{\partial{v}}{\partial{x}}
                \end{array}
            \end{equation}
        \end{theorem}
        \begin{proof}
            If $f:\mathbb{C}\rightarrow\mathbb{C}$ is differentiable,
            then letting $h\rightarrow{0}$ using solely real numbers produces:
            \begin{equation}
                \begin{array}{rcl}
                    \displaystyle
                    f^{\prime}(z)
                    &=&
                    \displaystyle
                    \lim_{h\rightarrow{0}}
                    \frac{f(z+h)-f(z)}{h}\\[1.5em]
                    \displaystyle
                    &=&
                    \displaystyle
                    \lim_{h\rightarrow{0}}
                    \frac{u(x+h,\,y)+iv(x+h,\,y)-u(x,\,y)-iv(x,\,y)}{h}\\[1.5em]
                    \displaystyle
                    &=&
                    \displaystyle
                    \lim_{h\rightarrow{0}}
                    \frac{u(x+h,\,y)-u(x,\,y)}{h}
                    +i\lim_{h\rightarrow{0}}
                        \frac{v(x+h,\,y)-v(x,\,y)}{h}\\[1.5em]
                    &=&
                    \displaystyle
                    \frac{\partial{u}}{\partial{x}}
                        +i\frac{\partial{v}}{\partial{x}}
                \end{array}
            \end{equation}
            Replacing $h$ with $ih$, we allow this quantity to approach zero
            using purely imaginary numbers. We obtain:
            \begin{equation}
                \begin{array}{rcl}
                    \displaystyle
                    f^{\prime}(z)
                    &=&
                    \displaystyle
                    \lim_{h\rightarrow{0}}
                    \frac{f(z+ih)-f(z)}{ih}\\[1.5em]
                    \displaystyle
                    &=&
                    \displaystyle
                    \lim_{h\rightarrow{0}}
                    \frac{u(x,\,y+h)+iv(x,\,y+h)-u(x,\,y)-iv(x,\,y)}{ih}\\[1.5em]
                    \displaystyle
                    &=&
                    \displaystyle
                    \lim_{h\rightarrow{0}}
                    \frac{u(x,\,y+h)-u(x,\,y)}{ih}
                    +i\lim_{h\rightarrow{0}}
                        \frac{v(x,\,y+h)-v(x,\,y)}{ih}\\[1.5em]
                    &=&
                    \displaystyle
                    -i\frac{\partial{u}}{\partial{y}}
                        +\frac{\partial{v}}{\partial{y}}
                \end{array}
            \end{equation}
            Since the function is differentiable, the limit is independent of
            the path to zero. Equating we get:
            \begin{equation}
                \frac{\partial{u}}{\partial{x}}
                +i\frac{\partial{v}}{\partial{x}}
                =
                \frac{\partial{v}}{\partial{y}}
                -i\frac{\partial{u}}{\partial{y}}
            \end{equation}
            Comparing the real and imaginary parts of both sides
            produces the Cauchy-Riemann equations.
        \end{proof}
        \begin{theorem}
            If $f:\mathbb{C}\rightarrow\mathbb{C}$ is given by
            $f(x+iy)=u(x,\,y)+iv(x,\,y)$ and if $u$ and $v$ both have
            continuous partial derivatives with respect to $x$ and $y$,
            and if the Cauchy-Riemann equations are satisfied, then $f$ is
            complex differentiable.
        \end{theorem}
        \begin{proof}
            For let $h:\mathbb{N}\rightarrow\mathbb{C}$ be a sequence
            that converges to $0$. For each $n\in\mathbb{N}$
            we may write $h_{n}=a_{n}+ib_{n}$ with $a_{n},b_{n}\in\mathbb{R}$.
            We have:
            \begin{equation}
                \begin{array}{rcl}
                    \displaystyle
                    \frac{f(z+h_{n})-f(z)}{h_{n}}
                    &=&
                    \displaystyle
                    \frac{f(z+h_{n})-f(z+a_{n})+f(z+a_{n})-f(z)}{h_{n}}\\[1.5em]
                    &=&
                    \displaystyle
                    \frac{f(z+h_{n})-f(z+a_{n})}{h_{n}}
                        +\frac{f(z+a_{n})-f(z)}{h_{n}}
                \end{array}
            \end{equation}
            Using $f(x+iy)=u(x,\,y)+iv(x,\,y)$, this final expression splits
            into two equations, one for $u$ and one for $v$. Concentrating
            on the real part of the equation ($u$), we get:
            \begin{equation}
                \label{eqn:u_difference_quotient}
                \frac{u(x+a_{n},\,y+b_{n})-u(x,\,y)}{h_{n}}
                =
                \frac{u(x+a_{n},\,y+b_{n})-u(x+a_{n},\,y)}{h_{n}}
                    +\frac{u(x+a_{n},\,y)-u(x,\,y)}{h_{n}}
            \end{equation}
            By the mean value theorem there is some real number
            ${b}_{n}^{\prime}$ between $0$ and $b_{n}$ such that:
            \begin{equation}
                u(x+a_{n},\,y+b_{n})-u(x+a_{n},\,y)
                =\frac{\partial{u}}{\partial{y}}
                \left(x+a_{n},\,y+b_{n}^{\prime}\right)b_{n}
            \end{equation}
            Similarly there is a real number $a_{n}^{\prime}$ between $0$ and
            $a_{n}$ such that:
            \begin{equation}
                u(x+a_{n},\,y)-u(x,\,y)
                =\frac{\partial{u}}{\partial{x}}
                \left(x+a_{n}^{\prime},\,y\right)a_{n}
            \end{equation}
            Plugging these two expressions into
            Eqn.~\ref{eqn:u_difference_quotient}, we obtain:
            \begin{equation}
                \frac{u(x+a_{n},\,y+b_{n})-u(x,\,y)}{h_{n}}
                =\frac{\partial{u}}{\partial{y}}
                \left(x+a_{n},\,y+b_{n}^{\prime}\right)\frac{b_{n}}{h_{n}}
                +\frac{\partial{u}}{\partial{x}}
                \left(x+a_{n}^{\prime},\,y\right)\frac{a_{n}}{h_{n}}
            \end{equation}
            Similarly for $v$ there are numbers $a_{n}^{\prime\prime}$ and
            $b_{n}^{\prime\prime}$ between $0$ and $a_{n}$ and $b_{n}$,
            respectively, such that:
            \begin{equation}
                \frac{v(x+a_{n},\,y+b_{n})-v(x,\,y)}{h_{n}}
                =\frac{\partial{v}}{\partial{y}}
                \left(x+a_{n},\,y+b_{n}^{\prime\prime}\right)
                \frac{b_{n}}{h_{n}}
                +\frac{\partial{v}}{\partial{x}}
                \left(x+a_{n}^{\prime\prime},\,y\right)
                \frac{a_{n}}{h_{n}}
            \end{equation}
            By hypothesis $u$ and $v$ satisfy the Cauchy-Riemann equations,
            and hence:
            \begin{equation}
                \begin{array}{rcl}
                    \displaystyle
                    \frac{\partial{u}}{\partial{x}}
                    \left(x+a_{n},\,y+b_{n}^{\prime\prime}\right)
                    &=&
                    \displaystyle
                    \phantom{+}
                    \frac{\partial{v}}{\partial{y}}
                    \left(x+a_{n},\,y+b_{n}^{\prime\prime}\right)\\[1.5em]
                    \displaystyle
                    \frac{\partial{v}}{\partial{x}}
                    \left(x+a_{n},\,y+b_{n}^{\prime}\right)
                    &=&
                    \displaystyle
                    -\frac{\partial{u}}{\partial{y}}
                    \left(x+a_{n},\,y+b_{n}^{\prime}\right)
                \end{array}
            \end{equation}
            Combining the previous equations we obtain:
            \begin{equation}
                \begin{array}{rcl}
                    \displaystyle
                    \frac{f(x+a_{n},\,y+b_{n})-f(x,\,y)}{h_{n}}
                    &=&
                    \displaystyle
                    \left(
                        \frac{\partial{u}}{\partial{y}}
                        \left(x+a_{n},\,y+b_{n}^{\prime}\right)
                        \frac{b_{n}}{h_{n}}
                        +\frac{\partial{u}}{\partial{x}}
                        \left(x+a_{n}^{\prime},\,y\right)
                        \frac{a_{n}}{h_{n}}
                    \right)\\[1.5em]
                    &+&
                    \displaystyle
                    i\left(
                        \frac{\partial{v}}{\partial{y}}
                        \left(x+a_{n},\,y+b_{n}^{\prime\prime}\right)
                        \frac{b_{n}}{h_{n}}
                        +\frac{\partial{v}}{\partial{x}}
                        \left(x+a_{n}^{\prime\prime},\,y\right)
                        \frac{a_{n}}{h_{n}}
                    \right)\\[2em]
                    &=&
                    \displaystyle
                    \left(
                        \frac{\partial{u}}{\partial{x}}
                        (x+a_{n}^{\prime},\,y)
                        \frac{a_{n}}{h_{n}}
                        +\frac{\partial{u}}{\partial{x}}
                        (x+a_{n},\,y+b_{n}^{\prime\prime})
                        \frac{ib_{n}}{h_{n}}
                    \right)\\[1.5em]
                    &+&
                    \displaystyle
                    \left(
                        \frac{\partial{v}}{\partial{x}}
                        \left(x+a_{n}^{\prime\prime},\,y\right)
                        \frac{a_{n}}{h_{n}}
                        +\frac{\partial{v}}{\partial{x}}
                        \left(x+a_{n},\,y+b_{n}^{\prime}\right)
                        \frac{ib_{n}}{h_{n}}
                    \right)
                \end{array}
            \end{equation}
            Define $U_{n}$ and $V_{n}$ as follows:
            \begin{equation}
                \begin{array}{rcl}
                    U_{n}
                    &=&
                    \displaystyle
                    \left(
                        \frac{\partial{u}}{\partial{x}}
                        (x+a_{n},\,y+b_{n}^{\prime\prime})
                        -
                        \frac{\partial{u}}{\partial{x}}
                        (x+a_{n}^{\prime},\,y)
                    \right)
                    \frac{ib_{n}}{h_{n}}\\[2em]
                    V_{n}
                    &=&
                    \displaystyle
                    \left(
                        \frac{\partial{v}}{\partial{x}}
                        (x+a_{n},\,y+b_{n}^{\prime\prime})
                        -
                        \frac{\partial{v}}{\partial{x}}
                        (x+a_{n}^{\prime\prime},\,y)
                    \right)
                    \frac{ib_{n}}{h_{n}}
                \end{array}
            \end{equation}
            The difference quotient is then:
            \begin{equation}
                \frac{f(x+a_{n},\,y+b_{n})-f(x,\,y)}{h_{n}}
                =\frac{\partial{u}}{\partial{x}}
                    (x+a_{n}^{\prime},\,y)
                    \frac{a+ib_{n}}{h_{n}}
                +\frac{\partial{v}}{\partial{x}}
                    (x+a_{n}^{\prime\prime},\,y)
                    \frac{a+ib_{n}}{h_{n}}
                +U_{n}+V_{n}
            \end{equation}
            But $h_{n}=a_{n}+ib_{n}$ by definition, and hence:
            \begin{equation}
                \frac{f(x+a_{n},\,y+b_{n})-f(x,\,y)}{h_{n}}
                =\frac{\partial{u}}{\partial{x}}
                    (x+a_{n}^{\prime},\,y)
                +\frac{\partial{v}}{\partial{x}}
                    (x+a_{n}^{\prime\prime},\,y)
                +U_{n}+V_{n}
            \end{equation}
            Since $h_{n}\rightarrow{0}$ we may conclude that
            $a_{n}\rightarrow{0}$ and $b_{n}\rightarrow{0}$ as well.
            But $a_{n}^{\prime}$ and $a_{n}^{\prime\prime}$ are wedged between
            $0$ and $a_{n}$, and hence these sequences must converge to $0$
            as well. A similar statement applies to $b_{n}^{\prime}$ and
            $b_{n}^{\prime\prime}$.
            If we take the limit as $n\rightarrow\infty$, then since
            $u$ and $v$ are continuously differentiable, the partial
            derivatives are continuous, and hence:
            \begin{equation}
                \lim_{n\rightarrow\infty}
                \left(
                    \frac{\partial{u}}{\partial{x}}
                    (x+a_{n}^{\prime},\,y)
                    +i\frac{\partial{v}}{\partial{x}}
                    (x+a_{n}^{\prime\prime},\,y)
                    +U_{n}+V_{n}
                \right)
                =\frac{\partial{u}}{\partial{x}}(x,\,y)
                +\frac{\partial{v}}{\partial{x}}(x,\,y)
                +\lim_{n\rightarrow\infty}\left(U_{n}+V_{n}\right)
            \end{equation}
            Again from continuity, from the definition of $U_{n}$
            (and since $a_{n}^{\prime}$ and $b_{n}^{\prime\prime}$ both
            converge to zero) we have $U_{n}\rightarrow{0}$. Similarly
            $V_{n}\rightarrow{0}$, and hence:
            \begin{equation}
                \lim_{n\rightarrow\infty}
                \frac{f(z+h_{n})-f(z)}{h_{n}}
                =\frac{\partial{u}}{\partial{x}}(x,\,y)
                +i\frac{\partial{v}}{\partial{x}}(x,\,y)
            \end{equation}
            and therefore $f$ is differentiable.
        \end{proof}
    \section{Geometric Interpretations}
    \bibliographystyle{plain}
    \bibliography{bib.bib}
\end{document}
