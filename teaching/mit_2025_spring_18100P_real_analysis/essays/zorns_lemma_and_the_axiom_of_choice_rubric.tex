%-----------------------------------LICENSE------------------------------------%
%   This file is part of Mathematics-and-Physics.                              %
%                                                                              %
%   Mathematics-and-Physics is free software: you can redistribute it and/or   %
%   modify it under the terms of the GNU General Public License as             %
%   published by the Free Software Foundation, either version 3 of the         %
%   License, or (at your option) any later version.                            %
%                                                                              %
%   Mathematics-and-Physics is distributed in the hope that it will be useful, %
%   but WITHOUT ANY WARRANTY; without even the implied warranty of             %
%   MERCHANTABILITY or FITNESS FOR A PARTICULAR PURPOSE.  See the              %
%   GNU General Public License for more details.                               %
%                                                                              %
%   You should have received a copy of the GNU General Public License along    %
%   with Mathematics-and-Physics.  If not, see <https://www.gnu.org/licenses/>.%
%------------------------------------------------------------------------------%
\documentclass{article}
\usepackage{geometry}
\geometry{margin = 1.5in}
\usepackage{amsthm}
\usepackage{amssymb}
\usepackage{xcolor}
\usepackage{hyperref}
\title{Zorn's Lemma and the Axiom of Choice: Rubric}
\author{MIT 18.100P}
\date{\today}
\setlength{\parindent}{0em}
\setlength{\parskip}{0em}

\begin{document}
    \textbf{Zorn's Lemma (25 pts)}
    \par\hfill\par
    25
    \par
    Works from the definitions, axioms, and basic theorems that
    someone with (very) basic set theory background could
    understand. Fills in all of the steps. If using towers, clearly
    defines what they are and works with them properly. If using
    ordinals, clearly defines what they are. Should \textbf{not}
    use \textit{classes} unless they describe what NGB set theory
    is (highly unlikely). Proof is logically valid.
    \par\hfill\par
    18-24
    \par
    Same as above, but with a few poorly explained step or slightly
    flawed reasoning / logic.
    \par\hfill\par
    12-17
    \par
    Uses non-obvious terms that have not been defined, may contain a few
    logical misteps.
    \par\hfill\par
    6-11
    \par
    Uses several undefined things. Towers not explained, or ordinals used
    without definition, or classes used without discussing NGB set theory.
    \par\hfill\par
    0-5
    \par
    Incomplete, logically invalid, or missing many steps.
    \par\hfill\par
    \textbf{Axiom of Choice (25 pts)}
    \par\hfill\par
    25
    \par
    Works from the definitions, axioms, and basic theorems that
    someone with (very) basic set theory background could
    understand. Fills in all of the steps. If using extensions, explains what
    they are. Verifies that the end result is indeed a choice function.
    If declaring something is a poset, proves that this is indeed true.
    \par\hfill\par
    18-24
    \par
    Same as above, but with a few poorly explained step or slightly
    flawed reasoning / logic.
    \par\hfill\par
    12-17
    \par
    Uses non-obvious terms that have not been defined, may contain a few
    logical misteps.
    \par\hfill\par
    6-11
    \par
    Uses several undefined things. Does not prove that extensions produce
    a poset. Does not verify that the final function is a choice function.
    \par\hfill\par
    0-5
    \par
    Incomplete, logically invalid, or missing many steps.
    \par\hfill\par
    \textbf{Indicates purpose of paper in the introduction (10 pts)}
    \par\hfill\par
    10
    \par
    Gives reason for caring about equivalence of AC and ZL
    with proper citations of sources.
    \par\hfill\par
    8-9
    \par
    Gives reason for... without adequate citation.
    \par\hfill\par
    6-7
    \par
    Gives minimal background for ZL and AC without motivating relationship
    \par\hfill\par
    4-5
    \par
    Conflates purpose and main result, so no real background motivation given
    \par\hfill\par
    0
    \par
    Lacks any set up of main result
    \par\hfill\par
    \textbf{Indicates main result in introduction (10 pts)}
    \par\hfill\par
    10
    \par
    Statement in thm environment of equivalence of AC and ZL and/or
    ``The main result is...'' phrase in easily visible location in
    introduction with statement in thm environment elsewhere in paper.
    \par\hfill\par
    8-9
    \par
    The main result is...
    type statement with no formal statement of the equivalence anywhere
    in the paper, but includes formal statements that add up to a
    statement of the equivalence.
    \par\hfill\par
    6-7
    \par
    Introduction only has a ``we prove....'' but there is a formal theorem
    statement of the equivalence or formal theorems that add up to the
    equivalence in the body of the paper.
    \par\hfill\par
    5
    \par
    Introduction only has ``we prove...'' statement in introduction,
    no formal statements of the main claim anywhere in body of paper
    \par\hfill\par
    0
    \par
    no statement of goal of equivalence in introduction
    \par\hfill\par
    \textbf{Gives key definitions (10 pts)}
    10
    \par
    Definitions defined in context and uses standard publication
    formatting to identify terms (italics, bold, etc.)
    \par\hfill\par
    8-9
    \par
    All definitions defined in context but not using standard
    publication formatting, or some definitions defined in context
    using standard publication formatting but some just listed without context.
    \par\hfill\par
    6-7
    \par
    All definitions presented using standard publication formatting
    but none given in context.
    \par\hfill\par
    4-5
    \par
    All definitions given but presented in non-standard publication
    formatting and in lists rather than contextualized.
    \par\hfill\par
    0
    \par
    Definitions not provided.
    \par\hfill\par
    \textbf{Gives proof approach (10 pts)}
    \par\hfill\par
    10
    \par
    Overview of approach appears in introduction so reader can
    understand structure of paper, similar analogues appear locally
    at appropriate moments in the paper.
    \par\hfill\par
    8-9
    \par
    Overview of approach appears in the introduction but with too
    much detail for an introduction.
    \par\hfill\par
    6-7
    Overview of approach appears somewhere in the paper that
    transitions into the body but not necessarily structurally
    in the introduction.
    \par\hfill\par
    4-5
    \par
    Overview of the approach given but not in location
    that makes utility relevant.
    \par\hfill\par
    0
    \par
    missing overview of approach
    \par\hfill\par
    \textbf{Uses statement and proof environments (5 pts)}
    \par\hfill\par
    5
    \par
    Technical definitions appear in def environments, all proofs are
    preceded by claim statements in environments, proofs appear in proof
    environments, no nested environments, clear that students have used
    LaTeX appropriately and not hard coded the formatting.
    \par\hfill\par
    4
    \par
    Technical definitions appear in def environments, all proofs are
    preceded by claim statements in environments, proofs appear in proof
    environments, no nested environments, LaTeX looks funny so indicative
    of not using code correctly.
    \par\hfill\par
    3
    \par
    Some items that need to be in environments aren't, e.g. proofs
    without formal claim. This does not mean if you feel something
    would have been better as a lemma but the student hasn't chosen
    to make it a lemma then an environment is missing OR nesting
    environments makes it hard to follow.
    \par\hfill\par
    2
    \par
    Using environments but in a way that shows the student doesn't
    understand their role in organizing a math paper.
    \par\hfill\par
    0
    \par
    No environments.
    \par\hfill\par
    \textbf{Integrates English and symbols (5 pts)}
    \par\hfill\par
    5
    \par
    Dominant mode is writing in English paragraphs with symbols
    integrated into sentence structure and equations displayed
    for readability. No banned blackboard shorthands used.
    \par\hfill\par
    4
    \par
    Dominant mode is writing in English paragraphs with symbols
    integrated into sentence structure and equations displayed for
    readability. Banned blackboard shorthands used occasionally.
    \par\hfill\par
    3
    \par
    Sentences are integrated but lacks paragraph structure or
    banned blackboard shorthands used throughout, so end product
    looks like pset more than paper.
    \par\hfill\par
    0
    \par
    All equations with minimal text
\end{document}
