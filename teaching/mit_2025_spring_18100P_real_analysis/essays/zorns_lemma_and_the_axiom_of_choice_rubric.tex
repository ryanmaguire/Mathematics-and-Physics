%-----------------------------------LICENSE------------------------------------%
%   This file is part of Mathematics-and-Physics.                              %
%                                                                              %
%   Mathematics-and-Physics is free software: you can redistribute it and/or   %
%   modify it under the terms of the GNU General Public License as             %
%   published by the Free Software Foundation, either version 3 of the         %
%   License, or (at your option) any later version.                            %
%                                                                              %
%   Mathematics-and-Physics is distributed in the hope that it will be useful, %
%   but WITHOUT ANY WARRANTY; without even the implied warranty of             %
%   MERCHANTABILITY or FITNESS FOR A PARTICULAR PURPOSE.  See the              %
%   GNU General Public License for more details.                               %
%                                                                              %
%   You should have received a copy of the GNU General Public License along    %
%   with Mathematics-and-Physics.  If not, see <https://www.gnu.org/licenses/>.%
%------------------------------------------------------------------------------%
\documentclass{article}
\usepackage{geometry}
\geometry{margin = 1.5in}
\usepackage{amsthm}
\usepackage{amssymb}
\usepackage{xcolor}
\usepackage{hyperref}
\title{Zorn's Lemma and the Axiom of Choice: Rubric}
\author{MIT 18.100P}
\date{\today}
\setlength{\parindent}{0em}
\setlength{\parskip}{0em}

\begin{document}
    \textbf{Introduction}
    \par\hfill\par
    9-10:
    \par
    Motivates the theorems, provides background information,
    clearly labels relevant definitions using either bold text or
    a definition environment, provides examples when necessary,
    contains references.
    \par\hfill\par
    7-8
    \par
    Provides background information, has appropriate definitions
    that are clearly labeled. Has some examples.
    \par\hfill\par
    5-6
    \par
    Provides basic background information, has appropriate
    definitions that are clearly labeled.
    \par\hfill\par
    3-4
    \par
    Provides basic background information with definitions
    that are not marked well.
    \par\hfill\par
    0-2
    \par
    Lacks most definitions or theorem statements. Incomplete work.
    \par\hfill\par
    \textbf{Zorn's Lemma}
    \par\hfill\par
    9-10
    \par
    Works from the definitions, axioms, and basic theorems that
    someone with (very) basic set theory background could
    understand. Fills in all of the steps. If using towers, clearly
    defines what they are and works with them properly. If using
    ordinals, clearly defines what they are. Should \textbf{not}
    use \textit{classes} unless they describe what NGB set theory
    is (highly unlikely). Proof is logically valid.
    \par\hfill\par
    7-8
    \par
    Same as above, but with a few poorly explained step or slightly
    flawed reasoning / logic.
    \par\hfill\par
    5-6
    \par
    Uses non-obvious terms that have not been defined, may contain a few
    logical misteps.
    \par\hfill\par
    3-4
    \par
    Uses several undefined things. Towers not explained, or ordinals used
    without definition, or classes used without discussing NGB set theory.
    \par\hfill\par
    0-2
    \par
    Incomplete, logically invalid, or missing many steps.
    \par\hfill\par
    \textbf{Axiom of Choice}
    \par\hfill\par
    9-10
    \par
    Works from the definitions, axioms, and basic theorems that
    someone with (very) basic set theory background could
    understand. Fills in all of the steps. If using extensions, explains what
    they are. Verifies that the end result is indeed a choice function.
    If declaring something is a poset, proves that this is indeed true.
    \par\hfill\par
    7-8
    \par
    Same as above, but with a few poorly explained step or slightly
    flawed reasoning / logic.
    \par\hfill\par
    5-6
    \par
    Uses non-obvious terms that have not been defined, may contain a few
    logical misteps.
    \par\hfill\par
    3-4
    \par
    Uses several undefined things. Does not prove that extensions produce
    a poset. Does not verify that the final function is a choice function.
    \par\hfill\par
    0-2
    \par
    Incomplete, logically invalid, or missing many steps.
    \par\hfill\par
    \textbf{Total}
    \par
    Tally up your grade and multiply by 5/3. End result is out of 50.
    The remaining 50 points are for communication. I'll let Malcah fill this
    part in.
\end{document}
