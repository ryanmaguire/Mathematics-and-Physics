%-----------------------------------LICENSE------------------------------------%
%   This file is part of Mathematics-and-Physics.                              %
%                                                                              %
%   Mathematics-and-Physics is free software: you can redistribute it and/or   %
%   modify it under the terms of the GNU General Public License as             %
%   published by the Free Software Foundation, either version 3 of the         %
%   License, or (at your option) any later version.                            %
%                                                                              %
%   Mathematics-and-Physics is distributed in the hope that it will be useful, %
%   but WITHOUT ANY WARRANTY; without even the implied warranty of             %
%   MERCHANTABILITY or FITNESS FOR A PARTICULAR PURPOSE.  See the              %
%   GNU General Public License for more details.                               %
%                                                                              %
%   You should have received a copy of the GNU General Public License along    %
%   with Mathematics-and-Physics.  If not, see <https://www.gnu.org/licenses/>.%
%------------------------------------------------------------------------------%
\documentclass{article}
\usepackage{amssymb}
\usepackage{hyperref}
\hypersetup{colorlinks = true}
\title{The Fundamental Theorem of Calculus}
\author{MIT 18.100P}
\date{\today}
\setlength{\parindent}{0em}
\setlength{\parskip}{0em}
\begin{document}
    \maketitle
    Your final paper covers the true highlight of the class, the fundamental
    theorem of calculus. Starting with the basics of Riemann or Darboux
    integration, you are to prove that if $f:[a,\,b]\rightarrow\mathbb{R}$
    is continuous on the closed bounded interval $[a,\,b]$,
    then not only is it integrable, but the function
    $F:[a,\,b]\rightarrow\mathbb{R}$ given by:
    \begin{equation}
        F(x)=\int_{a}^{x}f(t)\,\textrm{d}t
    \end{equation}
    is differentiable on the open interval $(a,\,b)$ with
    $F^{\prime}(x)=f(x)$. Moreover, the integral of $f$ is related to $F$
    by:
    \begin{equation}
        \int_{a}^{b}f(x)\,\textrm{d}x=F(b)-F(a)
    \end{equation}
    You may freely use any of the results \textbf{from class} on continuity,
    differentiation, sequences, and series such as:
    \begin{enumerate}
        \item
            Bolzano-Weierstrass.
        \item
            Heine-Borel.
        \item
            Intermediate Value Theorem.
        \item
            Extreme Value Theorem.
        \item
            Mean Value Theorem.
        \item
            Taylor's Theorem.
    \end{enumerate}
    Some of these are more useful than others in proving this theorem,
    but regardless, if you use a major result you must be explicit.
    A brief mention of the theorem should occur somewhere in the
    preliminaries section (proof sketch, introduction, etc.).
    \par\hfill\par
    Your proof should come in three parts:
    \begin{enumerate}
        \item
            $f$ is indeed integrable. That is, $\int_{a}^{x}f(t)\,\textrm{d}t$
            exists and is well-defined.
        \item
            $F(x)=\int_{a}^{x}f(t)\,\textrm{d}t$ is differentiable on $(a,\,b)$
            and $F^{\prime}(x)=f(x)$.
        \item
            If $F:[a,\,b]\rightarrow\mathbb{R}$ is a continuous function that
            is differentiable on $(a,\,b)$ with $F^{\prime}(x)=f(x)$, then
            $\int_{a}^{b}f(x)\,\textrm{d}x=F(b)-F(a)$.
    \end{enumerate}
    You should also clearly explain how to use your choice of integration
    (tagged partitions if going the Riemann route, lower and upper sums if
    choosing Darboux's method).
    \par\hfill\par
    \textbf{\Large{Guidelines}}
    \par\hfill\par
    Your article must be written in \LaTeX{} or \TeX, the primary typesetting
    languages for mathematics and other sciences. \LaTeX{} is preferred since it
    is much easier to hit the ground running and start writing documents.
    \par\hfill\par
    \textbf{Note}: Writing assignment 3 does \textbf{not} have the figure
    requirement that assignment 2 had. Figures are encouraged, but you will
    not be penalized if you are missing one.
    \begin{itemize}
        \item
            Your article is expected to be at least 3 pages.
            It would be hard to use less than 3 pages, but if you have
            a good argument and a solid article, do not feel like you
            must stretch it out just to meet a quota. This can actually
            \textit{harm} your writing.
        \item
            50\% of the grade is for \textit{mathematical correctness}.
            This addresses the following:
            \begin{itemize}
                \item
                    Are you using definitions correctly?
                \item
                    Are you applying axioms and other theorems
                    appropriately in your arguments?
                \item
                    Is your proof valid?
                \item
                    If you are using figures, are they relevant and explained
                    properly?
            \end{itemize}
        \item
            50\% comes from \textit{writing style}:
            \begin{itemize}
                \item
                    Does the paper meet all the expectations
                    from writing assignment 1, i.e.:
                    \begin{itemize}
                        \item
                            Does the paper have an introduction that identifies
                            and signals the purpose and the main result for a
                            general audience (i.e. not someone currently
                            enrolled in 18.100P or necessarily currently
                            taking real analysis)?
                        \item
                            Does the paper set up key definitions and give the
                            intuition of the approach used to prove the result
                            so that the reader can point to the places in the
                            text where these are given?
                        \item
                            Does the paper use of statement environments
                            and proof environments?
                        \item
                            Does the paper write in paragraphs,
                            integrating words and symbols?
                    \end{itemize}
                \item
                    Does the paper guide the reader through the ideas of the
                    paper, signposting the structure, motivating sections, and
                    motivating definitions and lemmas?
                \item
                    Does the paper guide the reader through the reasoning of
                    the proofs by giving overviews of the proof strategies,
                    justification for intermediate claims,
                    and identifying key steps in calculations?
                \item
                    Does the introduction of the paper sets up the
                    mathematical object of interest, the problem/question,
                    main result(s), and roadmap to the rest of the paper?
            \end{itemize}
    \end{itemize}
    \par\hfill\par
    \textbf{\Large{Deadlines}}
    \par\hfill\par
    \textbf{Rough Draft: 2025/05/02}
    \par
    A full draft of your article. You will do peer review with your fellow
    students during the recitation, and then the course team will provide
    feedback on your work.
    \par\hfill\par
    \textbf{Feedback Returned: 2025/05/07}
    \par
    Your rough draft will be returned with comments from the course team.
    \par\hfill\par
    \textbf{Final Due: 2025/05/13}
    \par
    After making revisions, you will turn in your final draft for grading.
\end{document}
