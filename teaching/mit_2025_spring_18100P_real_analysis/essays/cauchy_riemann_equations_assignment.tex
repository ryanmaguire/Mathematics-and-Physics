%-----------------------------------LICENSE------------------------------------%
%   This file is part of Mathematics-and-Physics.                              %
%                                                                              %
%   Mathematics-and-Physics is free software: you can redistribute it and/or   %
%   modify it under the terms of the GNU General Public License as             %
%   published by the Free Software Foundation, either version 3 of the         %
%   License, or (at your option) any later version.                            %
%                                                                              %
%   Mathematics-and-Physics is distributed in the hope that it will be useful, %
%   but WITHOUT ANY WARRANTY; without even the implied warranty of             %
%   MERCHANTABILITY or FITNESS FOR A PARTICULAR PURPOSE.  See the              %
%   GNU General Public License for more details.                               %
%                                                                              %
%   You should have received a copy of the GNU General Public License along    %
%   with Mathematics-and-Physics.  If not, see <https://www.gnu.org/licenses/>.%
%------------------------------------------------------------------------------%
\documentclass{article}
\usepackage{amssymb}
\usepackage{hyperref}
\hypersetup{colorlinks = true}
\title{The Cauchy-Riemann Equations}
\author{MIT 18.100P}
\date{\today}
\setlength{\parindent}{0em}
\setlength{\parskip}{0em}
\begin{document}
    \maketitle
    Continuous functions are those that \textbf{map convergent sequences to}
    \textbf{convergent sequences}. That is, if $A\subseteq\mathbb{R}$, given
    a function $f:A\rightarrow\mathbb{R}$ and a point $x_{0}\in{A}$, we say
    that $f$ is continuous at $x_{0}$ provided that for any sequence
    $a:\mathbb{N}\rightarrow{A}$ that converges to $x_{0}$,
    $a_{n}\rightarrow{x}_{0}$, we have $f(a_{n})\rightarrow{f}(x_{0})$.
    Written more succinctly, this says:
    \begin{equation}
        \lim_{n\rightarrow\infty}f(a_{n})
        =f\left(\lim_{n\rightarrow\infty}a_{n}\right)
        =f(x_{0})
    \end{equation}
    Continuity is \textit{pointwise}. A function can be continuous at a single
    point and nowhere else. Conversely a function can be continuous everywhere
    except for a single point. \textit{Differentiability} is defined similarly,
    but now we take a ratio. Given any convergent sequence
    $a:\mathbb{N}\rightarrow{A}$ with $a_{n}\rightarrow{x}_{0}$, and
    $a_{n}\ne{x}_{0}$ for all $n$ (to avoid division-by-zero), we have:
    \begin{equation}
        f^{\prime}(x_{0})
        =\lim_{n\rightarrow\infty}
        \frac{f(a_{n})-f(x_{0})}{a_{n}-x_{0}}
    \end{equation}
    That is, if $a$ is a convergent sequence with limit $x_{0}$ (and is never
    exactly equal to $x_{0}$), then this sequence of ratios is also a
    convergent sequence.
    \par\hfill\par
    A small note, while continuity allows us to consider arbitrary subsets
    $A$, differentiability is infinitely easier to work with when $A$ is open.
    If $A$ is not open, and if $x_{0}\in{A}$, then there may not be any
    convergent sequences $a_{n}\rightarrow{x}_{0}$ such that $a_{n}\ne{x}_{0}$
    for all $n$, which is a problem. Please restrict your writing to
    the case where $A$ is open.
    \par\hfill\par
    The definition requires two things: a notion of convergence (which comes
    from the \textit{distance} function, $d(x,\,y)=|x-y|$), and a well-defined
    meaning for division by non-zero elements. That is, we need a
    \textit{field} and a \textit{metric}. The rationals have such a structure,
    but Cauchy sequences rarely converge in $\mathbb{Q}$, which may cause
    issues. The complex numbers $\mathbb{C}$ also form a field, and moreover
    they have a metric given by the Pythagorean formula. That is, if we have
    $z=(x_{0},\,y_{0})$ and $w=(x_{1},\,y_{1})$, the distance is then:
    \begin{equation}
        |z-w|
        =\|(x_{0},\,y_{0})-(x_{1},\,y_{1})\|
        =\sqrt{(x_{1}-x_{0})^{2}+(y_{1}-y_{0})^{2}}
    \end{equation}
    Complex differentiation is then defined in precisely the same way.
    Given $A\subseteq\mathbb{C}$ (again, for simplicity stick to open subsets),
    $f:A\rightarrow\mathbb{C}$, and
    $z_{0}\in{A}$, we say $f$ is differentiable at $z_{0}$ provided that for
    any sequence $a:\mathbb{N}\rightarrow\mathbb{C}$ with
    $a_{n}\rightarrow{z}_{0}$ and $a_{n}\ne{z}_{0}$ for all $n$, the sequence
    of ratios converges:
    \begin{equation}
        f^{\prime}(z_{0})
        =\lim_{n\rightarrow\infty}
        \frac{f(a_{n})-f(z_{0})}{a_{n}-z_{0}}
    \end{equation}
    In practice, this is hard to work with. Note that a function
    $f:A\rightarrow\mathbb{C}$ is really two functions. That is:
    \begin{equation}
        f\big((x,\,y)\big)
        =\big(u(x,\,y),\,v(x,\,y))
    \end{equation}
    or, using complex number notation:
    \begin{equation}
        f(z)=f(x+iy)=u(x,\,y)+iv(x,\,y)
    \end{equation}
    Where $u,v:A\rightarrow\mathbb{R}$ are real-valued functions defined on
    $A$. A function is complex differentiable if and only if it satisfies the
    \textbf{Cauchy-Riemann Equations}:
    \begin{equation}
        \begin{array}{rcl}
            \displaystyle
            \frac{\partial{u}}{\partial{x}}
            &=&
            \displaystyle
            \frac{\partial{v}}{\partial{y}}\\
            [1.5em]
            \displaystyle
            \frac{\partial{u}}{\partial{y}}
            &=&
            \displaystyle
            -\frac{\partial{v}}{\partial{x}}
        \end{array}
    \end{equation}
    Your goal is to prove this. You should mention:
    \begin{enumerate}
        \item
            Partial differentiation.
        \item
            Directional derivatives.
        \item
            The Laplace equation and harmonic functions.
        \item
            Bonus: Geometric / physical interpretations.
    \end{enumerate}
    See the references on the final page for further reading on these topics.
    \par\hfill\par
    \textbf{\Large{Guidelines}}
    \par\hfill\par
    Your article must be written in \LaTeX{} or \TeX, the primary typesetting
    languages for mathematics and other sciences. \LaTeX{} is preferred since it
    is much easier to hit the ground running and start writing documents.
    \begin{itemize}
        \item
            Your article is expected to be at least 3 pages.
            It would be hard to use less than 3 pages, but if you have
            a good argument and a solid article, do not feel like you
            must stretch it out just to meet a quota. This can actually
            \textit{harm} your writing.
        \item
            50\% of the grade is for \textit{mathematical correctness}.
            This addresses the following:
            \begin{itemize}
                \item
                    Are you using definitions correctly?
                \item
                    Are you applying axioms and other theorems
                    appropriately in your arguments?
                \item
                    Is your proof valid?
                \item
                    If you are using figures, are they relevant and explained
                    properly?
            \end{itemize}
        \item
            50\% comes from \textit{writing style}:
            \begin{itemize}
                \item
                     Does the paper have an introduction that identifies
                     and signals the purpose and the main result for a
                     general audience (i.e. not someone currently enrolled
                     in 18.100P or necessarily currently taking real analysis)?
                \item
                    Does the paper set up key definitions and give the
                    intuition of the approach used to prove the result
                    so that the reader can point to the places in the
                    text where these are given?
                \item
                    Does the paper use of statement environments
                    and proof environments?
                \item
                    Does the paper write in paragraphs,
                    integrating words and symbols?
            \end{itemize}
    \end{itemize}
    \par\hfill\par
    \textbf{\Large{Deadlines}}
    \par\hfill\par
    \textbf{Rough Draft: 2025/03/21}
    \par
    A full draft of your article. You will do peer review with your fellow
    students during the recitation, and then the course team will provide
    feedback on your work.
    \par\hfill\par
    \textbf{Feedback Returned: 2025/04/04}
    \par
    Your rough draft will be returned with comments from the course team.
    \par\hfill\par
    \textbf{Final Due: 2025/04/11}
    \par
    After making revisions, you will turn in your final draft for grading.
    \par\hfill\par
    \section*{External Links}
    \def\libretextURL{%
        https://math.libretexts.org/%
            Bookshelves/%
                Analysis/%
                    Complex_Variables_with_Applications_(Orloff)/%
                        02\%3A_Analytic_Functions/%
                            2.06\%3A_Cauchy-Riemann_Equations%
    }
    \def\planetMathURL{%
        https://planetmath.org/ProofOfTheCauchyRiemannEquations%
    }
    \def\encyclopediaURL{%
        https://encyclopediaofmath.org/index.php?title=Laplace_equation%
    }
    \def\byuURL{%
        https://physics.byu.edu/%
            faculty/%
                berrondo/%
                    docs/%
                        physics-441/%
                            laplace.pdf%
    }
    \begin{enumerate}
        \item
            \href{\libretextURL}{math.libretexts.org}
            \par
            Proof that differentiable implies the Cauchy-Riemann equations.
        \item
            \href{\planetMathURL}{planetmath.org}
            \par
            Proof that Cauchy-Riemann equations imply differentiable.
        \item
            \href{\encyclopediaURL}{encyclopediaofmath.org}
            \par
            Introduction to Laplace's Equation.
        \item
            \href{\byuURL}{physics.byu.edu}
            \par
            Proving that the Cauchy-Riemann equations
            produce Laplace's equation.
    \end{enumerate}
\end{document}
