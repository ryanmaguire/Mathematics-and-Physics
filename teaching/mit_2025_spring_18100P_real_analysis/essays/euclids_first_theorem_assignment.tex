%-----------------------------------LICENSE------------------------------------%
%   This file is part of Mathematics-and-Physics.                              %
%                                                                              %
%   Mathematics-and-Physics is free software: you can redistribute it and/or   %
%   modify it under the terms of the GNU General Public License as             %
%   published by the Free Software Foundation, either version 3 of the         %
%   License, or (at your option) any later version.                            %
%                                                                              %
%   Mathematics-and-Physics is distributed in the hope that it will be useful, %
%   but WITHOUT ANY WARRANTY; without even the implied warranty of             %
%   MERCHANTABILITY or FITNESS FOR A PARTICULAR PURPOSE.  See the              %
%   GNU General Public License for more details.                               %
%                                                                              %
%   You should have received a copy of the GNU General Public License along    %
%   with Mathematics-and-Physics.  If not, see <https://www.gnu.org/licenses/>.%
%------------------------------------------------------------------------------%
\documentclass{article}
\usepackage{geometry}
\geometry{margin = 1.5in}
\usepackage{amsthm}
\usepackage{amssymb}
\usepackage{xcolor}
\usepackage{hyperref}
\title{Euclid's First Theorem}
\author{MIT 18.100P}
\date{\today}
\setlength{\parindent}{0em}
\setlength{\parskip}{0em}
\newtheorem{theorem}{Theorem}
\newtheorem{axiom}{Axiom}

\begin{document}
    \maketitle
    Euclid begins \textit{The Elements} by stating five \textit{axioms},
    which are statements he claims are true and do not need proving.
    They are:
    \begin{axiom}[Euclid's Axioms]
        \par\hfill\par
        \begin{enumerate}
            \item
                Given a point $P$ and a point $Q$, there is a line segment
                $\overline{PQ}$ between them.
            \item
                Given a line segment $\overline{PQ}$, it may be extended
                indefinitely in either direction.
            \item
                Given a point $P$ and a point $Q$ there is a circle
                $C$ that has $P$ as the center and contains the point $Q$.
            \item
                Any two right-angles are equal in measurement.
            \item
                Given a point $P$ and a line $L$ that does not contain $P$,
                there is a unique line passing through $P$ that is parallel
                to $L$.
        \end{enumerate}
    \end{axiom}
    Euclid then proves his first theorem using this.
    \begin{theorem}
        If $A$ and $B$ are distinct points in the plane, then there exists an
        equilateral triangle with line segment $\overline{AB}$ as one of its
        edges.
    \end{theorem}
    \begin{proof}
        Using axiom 1, construct the line segment $\overline{AB}$.
        Using axiom 3, construct the circle with $A$ as the center and which
        contains $B$. Again using axiom 3, construct the circle with $B$ as
        the center which contains $A$. Let $C$ be the intersection of these
        two circles, and construct the triangle $\Delta{ABC}$ using axiom 1.
    \end{proof}
    As we discussed in class, the existence of $C$ does not follow from the
    axioms. Fill in the blanks by adding a new axiom and justifying the proof.
    \par\hfill\par
    \textbf{\Large{Guidelines}}
    \par\hfill\par
    Your article must be written in \LaTeX{} or \TeX, the primary typesetting
    languages for mathematics and other sciences. \LaTeX{} is preferred since it
    is much easier to hit the ground running and start writing documents.
    \begin{itemize}
        \item
            This is not a geometry course, you may use any material to
            figure out what the missing sixth axiom should be. This includes
            AI / LLM tools. You should cite your work when appropriate.
        \item
            You may \textbf{not} use AI / LLM tools to write your essay.
            That would constitute academic dishonesty.
        \item
            Your article is expected to be 2-3 pages in length.
        \item
            60\% of the grade is for \textit{mathematical correctness}.
            This addresses the following:
            \begin{itemize}
                \item
                    Have you defined relevant concepts (even \textit{primitive}
                    notions like \textit{point} or \textit{line} should get a
                    brief intuitive definition for a paper like this).
                \item
                    Are you using definitions correctly?
                \item
                    Are you stating your axioms clearly?
                \item
                    Are you applying axioms and other theorems
                    appropriately in your arguments?
                \item
                    Is your proof valid?
                \item
                    If you are using figures, are they relevant and explained
                    properly?
            \end{itemize}
        \item
            40\% comes from \textit{writing style}:
            \begin{itemize}
                \item
                    Does the paper use of statement environments
                    and proof environments?
                \item
                    Does the paper write in paragraphs,
                    integrating words and symbols?
            \end{itemize}
    \end{itemize}
    \par\hfill\par
    \textbf{\Large{Deadlines}}
    \par\hfill\par
    \textbf{Final Draft: 2025/02/27}
    \par
    A full paper is to be submitted.
    This first writing assingment is simply to get you started with
    mathematical writing will not include a peer-review or a
    feedback / revision stage.
\end{document}
