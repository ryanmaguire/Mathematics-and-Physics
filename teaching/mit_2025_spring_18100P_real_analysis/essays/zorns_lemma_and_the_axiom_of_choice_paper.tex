%-----------------------------------LICENSE------------------------------------%
%   This file is part of Mathematics-and-Physics.                              %
%                                                                              %
%   Mathematics-and-Physics is free software: you can redistribute it and/or   %
%   modify it under the terms of the GNU General Public License as             %
%   published by the Free Software Foundation, either version 3 of the         %
%   License, or (at your option) any later version.                            %
%                                                                              %
%   Mathematics-and-Physics is distributed in the hope that it will be useful, %
%   but WITHOUT ANY WARRANTY; without even the implied warranty of             %
%   MERCHANTABILITY or FITNESS FOR A PARTICULAR PURPOSE.  See the              %
%   GNU General Public License for more details.                               %
%                                                                              %
%   You should have received a copy of the GNU General Public License along    %
%   with Mathematics-and-Physics.  If not, see <https://www.gnu.org/licenses/>.%
%------------------------------------------------------------------------------%
\documentclass{article}
\usepackage{geometry}
\geometry{a4paper, margin = 1in}
\usepackage{amsthm}
\usepackage{amssymb}
\usepackage{amsmath}
\usepackage[nottoc]{tocbibind}
\usepackage{graphicx}
\usepackage{hyperref}
\hypersetup{colorlinks = true, linkcolor = blue}
\title{Zorn's Lemma and the Axiom of Choice}
\author{Ryan Maguire}
\date{\today}
\setlength{\parindent}{0em}
\setlength{\parskip}{0em}
\newtheorem{theorem}{Theorem}[section]
\newtheorem{axiom}{Axiom}[section]
\theoremstyle{definition}
\newtheorem{definition}{Definition}[section]
\graphicspath{{../../../images/}}
\newcommand*{\twiddle}[1]{%
    \tilde{#1}
}

\begin{document}
    \maketitle
    \begin{abstract}
        The axiom of choice is one of the central postulates of modern
        set theory and is commonly used in analysis, algebra, topology,
        and geometry. The application is often implicit, as mathematicians
        will instead invoke some form of Zorn's lemma in their work to avoid
        explicit mention of a choice function. It is well-known that under the
        assumptions of Zermelo-Fraenkel set theory the axiom of choice and
        Zorn's lemma are equivalent. We provide a (nearly) self-contained proof
        of this equivalence and do not assume the reader is deeply familiar
        with either of the two notions.
    \end{abstract}
    \tableofcontents
    \section{Introduction}
        Consider a collection of buckets, perhaps infinitely many. The buckets
        contain various objects, but each one contains at least one
        \textit{thing}. The primary claim under investigation is that it is
        possible to go to each bucket and choose one of the items from
        each, keeping a record of which ones we pick. One may have reservations
        about doing this with infinitely many buckets, but we use similar ideas
        regularly without objection. For example, the number $\pi$ is the
        ratio of the circumference of a circle to its diameter and has a
        value of $\pi=3.1415926\dots$ but what do these ellipsis represent?
        An \textit{infinite decimal}, and few would have any real quarrel
        with such a notion. To further justify our bucket proposition,
        the \textit{Cartesian product} over all of the buckets can be thought
        of as an infinite tuple where the first entry is an item in the
        first bucket, the second element corresponds to something in the
        second bucket, so on and so forth. Claiming that we may go about the
        buckets and collect an item from each is equivalent to saying that
        \textbf{the Cartesian product of non-empty sets is non-empty}
        \cite[p.~59]{Halmos1974}. Buckets
        are not very formal, mathematically, but we may model this idea using
        the language of set theory. We have the following.
        \begin{axiom}[\textbf{Axiom of Choice}]
            If $\mathcal{O}$ is a non-empty set such that for all
            $\mathcal{U}\in\mathcal{O}$ it is true that $\mathcal{U}$ is
            non-empty, then there is a function
            $f:\mathcal{O}\rightarrow\bigcup\mathcal{O}$, called a
            \textit{choice function}, with the property that for all
            $\mathcal{U}\in\mathcal{O}$ we have $f(\mathcal{U})\in\mathcal{U}$.
        \end{axiom}
        There are many equivalent formulations of this axiom
        (see, for example, \cite[p.~109]{MoschovakisNotesOnSetTheory} or
        \cite[p.~167]{Cunningham2010}).
        We have used the set-theoretic notation for union, but we may just as
        easily use the (perhaps) more familiar notation found in analysis:
        \begin{equation}
            \bigcup\mathcal{O}
            =\bigcup_{\mathcal{U}\in\mathcal{O}}\mathcal{U}
        \end{equation}
        The axiom of choice has far reaching applications in nearly all
        branches of mathematics. The topologist relies on it when they invoke
        \textit{Tychonoff's theorem} \cite[p.~234]{Munkres2000},
        the graph theorist regularly uses it to assert the existence of
        spanning trees, and most algebraists enjoy the fact that every
        vector space has a basis \cite[p.~518]{ArtinAlgebra2014}.
        There are about a dozen more statements in set theory that are
        equivalent to the axiom of choice, some of which may seem even
        more obvious. Consider the following.
        \begin{theorem}
            If $A$ and $B$ are sets, and if $f:A\rightarrow{B}$ is a
            surjective function, then there is a \textit{right inverse}
            $g:B\rightarrow{A}$, meaning that $(f\circ{g})(b)=b$ for all
            $b\in{B}$.
        \end{theorem}
        \begin{proof}
            Since $f$ is surjective, for all $b\in{B}$ there is an $a\in{A}$
            such that $f(a)=b$. Define $g(b)=a$. Then, by definition, we have
            $(f\circ{g})(b)=b$ for all $b\in{B}$.
        \end{proof}
        This theorem is equivalent to the axiom of choice. We are
        \textit{choosing} $g(b)=a$. If this theorem seems more believable,
        then you may want to except the axiom as true. But now consider this.
        \begin{theorem}[\textbf{Banach-Tarski Paradox}]
            It is possible to break the unit sphere $\mathbb{S}^{2}$,
            defined by:
            \begin{equation}
                \mathbb{S}^{2}
                =\left\{\,(x,\,y,\,z)\;\big|\;x^{2}+y^{2}+z^{2}=1\,\right\}
            \end{equation}
            into five pieces, and then, using only rotations and translations,
            create two new copies of $\mathbb{S}^{2}$, each being the exact
            same size as the original one.
        \end{theorem}
        For a proof, see \cite[p.~29]{WagonBanachTarski}.
        In other words, you may start with a pea and eventually end up with
        the sun. This is \textit{clearly}
        false, but the proof only relies on the axiom of choice
        (together with the other familiar notions from set theory).
        \par\hfill\par
        Regardless of these philosophical issues, the analysts and the
        algebraists are not likely to invoke the axiom of choice explicitly
        in their work. What is more common is for \textit{Zorn's lemma} to
        be required, which is a foundational result on the properties of
        partially ordered sets (or posets). Zorn's lemma states that if every
        chain in a non-empty poset is bounded,
        then there exists a maximal element.
        Intuitively, if you have a tree with a trunk and branches, and if
        every branch ends at a leaf, then there is a \textit{tallest} leaf.
        There may be several tallest leaves (each of equal height), but there
        is at least one. This analogy will be made concrete later when we
        provide rigorous definitions. The goal of this paper is to prove the
        following equivalence:
        \begin{theorem}[\textbf{Equivalence of AC and Zorn's Lemma}]
            Assuming the axioms of Zermelo-Fraenkel set theory,
            the axiom of choice (AC) is logically equivalent to Zorn's lemma.
        \end{theorem}
        \subsection{Zermelo-Fraenkel Set Theory}
            We will not start from complete scratch in our proof, but instead
            rely on the solid foundations provided by modern set theory.
            But as a matter of fact, several of the Zermelo-Fraenkel set theory
            axioms (ZF) are not required. To be precise we must
            adopt the axioms of union, pairing, power set, specification,
            and extensionality, and we also assume the existence of the empty
            set. The axioms of regularity, replacement, and infinity are not
            used at any point in our discussion. Of the axioms we are accepting,
            only the axioms of union, specification, and the empty set are
            explicitly used.
            The axiom of union guarantees that unions over collections always
            exist and it is central to proving Zorn's lemma implies the axiom
            of choice, and indeed the language of the axiom of choice requires
            that unions have already been defined. The axiom of the empty set
            simply asserts we are not working in a \textit{vacuum} where
            nothing exists at all. There is, at the very least, the empty set.
            Finally, specification allows us to construct new sets from old
            ones by applying sentences and collecting the elements that
            satisfy this statement. Using \textit{set-builder} notation,
            this is often written:
            \begin{equation}
                B=\left\{\,
                    x\in{A}\;|\;P(x)\,
                \right\}
            \end{equation}
            where $A$ is some set and $P$ is a sentence that depends on the
            elements of $A$.
            \par\hfill\par
            The use of the remaining axioms is subtle, and at first glance the
            reader may not think we actually need them. Pairing, power set,
            union, and specification allow us to define Cartesian products
            \cite[p.~24]{Halmos1974},
            which in turn allow us to provide a non-circular definition of
            functions (i.e. without saying that functions are just
            \textit{rules}, which is circular). Indeed, we may go further and
            construct the \textit{set of all functions} from a given set $A$
            to another set $B$. For brevity, we omit this discussion, but the
            reader can rest assurred that precise constructions can be made,
            albeit with increased effort and patience.
            \cite{Halmos1974} is the standard reference for this endeavor.
        \subsection{Partially Ordered Sets}
            The crucial definitions for our result lie in order theory. Indeed,
            Zorn's lemma is an order theory result above all else. We are
            primarily concerned with \textit{partially ordered sets}, more
            commonly referred to as posets. These arise naturally in just
            about every branch of mathematics one can think of, from analysis
            to algebra, geometry to topology, applied mathematics to
            combinatorics. Their widespread use stem from the simplicity
            of their definition.
            \begin{definition}[\textbf{Partial Ordered Set}]
                A partially ordered set is an ordered pair $(A,\,\leq)$
                where $A$ is a set and $\leq$ is a relation on $A$ such that:
                \begin{enumerate}
                    \item
                        For all $a\in{A}$ we have $a\leq{a}$.
                        \hfill
                        (Reflexivity)
                    \item
                        For all $a,b\in{A}$, if $a\leq{b}$ and $b\leq{a}$,
                        then $a=b$
                        \hfill
                        (Anti-Symmetry)
                    \item
                        For all $a,b,c\in{A}$, if $a\leq{b}$ and
                        $b\leq{c}$, then $a\leq{c}$
                        \hfill
                        (Transitivity)
                \end{enumerate}
                Partially ordered sets are often referred to as
                \textbf{posets}
            \end{definition}
            \begin{figure}
                \centering
                \includegraphics{power_set_001}
                \caption{Example of a Poset}
                \label{fig:power_set}
            \end{figure}
            That's it, no more, no less. The quintessential example to have
            in mind when thinking about posets is the
            \textit{inclusion relation} ($\subseteq$) defined on the power set
            of some set $A$. That is, $\big(\mathcal{P}(A),\,\subseteq\big)$
            is a poset, where $\mathcal{P}(A)$ is the power set of $A$.
            This structure is starkly different from the usual objects studied
            in mathematics that have an order. For the real numbers,
            $\mathbb{R}$, given $x,y\in\mathbb{R}$ either $x\leq{y}$ is true
            or $y\leq{x}$ is. This is called \textbf{totality}. Partial orders
            do not need to have such a concept, and indeed if we set
            $A=\{\,1,\,2,\,3\,\}$, then $\{\,1\,\}$ and $\{\,2\,\}$ are elements
            of $\mathcal{P}(A)$, but neither is a subset of the other. That is,
            these subsets are \textit{incomparable}.
            We visually represent posets by connecting edges between
            the elements that are comparable. The
            result is a \textit{Hasse diagram} (see Fig.~\ref{fig:power_set}).
            \par\hfill\par
            The more familiar orders we experience in everyday life are
            called \textit{total orders}.
            \begin{definition}[\textbf{Totally Ordered Set}]
                A totally ordered set is a partially ordered set
                $(A,\,\leq)$ such that for all $a,b\in{A}$ either
                $a\leq{b}$ or $b\leq{a}$. That is, the ordering satisfies
                \textbf{totality}.
            \end{definition}
            A \textbf{maximal element} in a poset $(A,\,\leq)$ is some element
            $x\in{A}$ such that for all $a\in{A}$, $x\leq{a}$ implies $x=a$.
            In a totally ordered set this simply means $x$ is the
            \textit{greatest} element. Because of this
            Zorn's lemma is almost trivial in a totally ordered set. It merely
            says if a totally ordered says has a greatest element, then it has
            a greatest element. Not very controversial at all. The real power
            of Zorn's lemma comes from its application to arbitrary posets
            with the property that \textit{chains are bounded}, where a chain
            is a totally ordered subset. Formally, we define:
            \begin{definition}[\textbf{Chain}]
                A chain in a partially ordered set $(A,\,\leq)$ is a subset
                $C\subseteq{A}$ such that $(C,\,\leq_{C})$ is a totally ordered
                set, where $\leq_{C}$ is simply the restriction of the relation
                $\leq$ to the subset $C$.
            \end{definition}
            Maximality can be extended to chains. A
            \textbf{maximal chain} is a chain $C\subseteq{A}$ such that if
            $D\subseteq{A}$ is a chain and $C\subseteq{D}$, then $C=D$.
    \section{Plan of Attack}
        We now have the prerequisites for our main result.
        The difficult direction in our proof is using the axiom of choice to
        imply Zorn's lemma. The converse is relatively straight forward, we
        form a partial ordering on the set of all
        \textit{partial choice functions}, ordered by \textit{extensions}.
        That is, given sets $A$ and $B$, and functions
        $f:A\rightarrow\bigcup{A}$ and $g:B\rightarrow\bigcup{B}$, we will
        write $f\leq{g}$ provided $A\subseteq{B}$ and
        $g|_{A}=f$. That is, the restriction of $g$ to $A$ yields the function
        $f$. Zorn's lemma produces a maximal element which we then prove is
        a choice function. This is really all there is to it, we simply need to
        rigorously fill in the details.
        \par\hfill\par
        We tackle Zorn's lemma by first showing that the axiom of choice
        implies \textit{Hausdorff's maximality principle}. Indeed, Hausdorff's
        theorem is equivalent to the axiom of choice
        \cite[p.~69]{Munkres2000}, much like Zorn's lemma, and is important in
        its own right due to many applications in topology
        \cite[p.~32]{KelleyTopology}.
        Using this stepping stone we then prove Zorn's lemma by
        reducing our search for a \textit{maximal element} to a search for a
        \textit{maximal chain}, which is granted by Hausdorff's theorem.
        Under the hypothesis of Zorn's lemma, such a chain has an upper bound,
        and we then prove that this element is maximal. All that is left is to
        prove that the axiom of choice implies Hausdorff's maximality principle.
        This is the most difficult result presented, and the proof is the
        certainly the more challenging one to grasp. We'll wade through
        with care.
    \section{The Proofs}
        Diving into the main part of the paper, we must now confront
        Zorn's lemma. The standard formulation of set theory is to accept the
        axiom of choice, together with the remaining ZF axioms, to produce
        ZFC. Pedagically it seems best to prove Zorn's lemma first, and indeed
        this is how most textbooks handle the ordering.
        \subsection{Stepping Stone: Hausdorff's Maximality Principle}
            As alluded to, the pieces fall together once Hausdorff's theorem
            is given. We prove this first.
            \begin{theorem}[\textbf{Hausdorff's Maximality Principle}]
                If $(A,\,\leq)$ is a partially ordered set, and if
                $C\subseteq{A}$ is a chain, then there is a maximal chain
                $D\subseteq{A}$ such that $C\subseteq{D}$.
            \end{theorem}
            \begin{proof}
                Let $\mathcal{C}$ denote the set of all chains in $A$.
                Note that $\mathcal{C}$ satisfies a few properties:
                \begin{enumerate}
                    \item
                        $\mathcal{C}$ is non-empty. Since $A$ is non-empty
                        we may choose $x\in{A}$. The set $B=\{\,a\,\}$ is
                        a chain, and hence an element of $\mathcal{C}$.
                    \item
                        If $\mathcal{O}\subseteq\mathcal{C}$ is such that for
                        all $\mathcal{U},\mathcal{V}\in\mathcal{O}$ we have
                        either $\mathcal{U}\subseteq\mathcal{V}$ or
                        $\mathcal{V}\subseteq\mathcal{U}$, then
                        $\bigcup\mathcal{O}\in\mathcal{C}$. That is, the union
                        of \textbf{nested} chains is again a chain.
                    \item
                        If $B\in\mathcal{C}$ and if $B^{\prime}\subseteq{B}$,
                        then $B^{\prime}\in\mathcal{C}$. That is, a subset of
                        a chain is still a chain.
                \end{enumerate}
                We prove Hausdorff's maximality principle from these three
                properties. Since $A$ is non-empty, the power set
                $\mathcal{P}(A)$ contains more than just the empty set, and
                hence $\mathcal{P}(A)\setminus\{\,\emptyset\,\}$ is non-empty.
                By the axiom of choice there is a function
                $f:\mathcal{P}(A)\setminus\{\,\emptyset\,\}%
                    \rightarrow\bigcup\big(%
                        \mathcal{P}(A)\setminus\{\,\emptyset\,\}%
                    \big)$
                such that $f(\mathcal{U})\in\mathcal{U}$
                for all
                $\mathcal{U}\in\mathcal{P}(A)\setminus\{\,\emptyset\,\}$.
                Recognizing that
                $\bigcup\big(\mathcal{P}(A)\setminus\{\,\emptyset\,\}\big)=A$,
                the choice function may be rewritten as
                $f:\mathcal{P}(A)\setminus\{\,\emptyset\,\}\rightarrow{A}$.
                Given $B\in\mathcal{C}$, define $B^{\prime}$ by:
                \begin{equation}
                    B^{\prime}
                    =\left\{\,
                        x\in{A}\setminus{B}
                        \;\big|\;
                        B\cup\{\,x\,\}\in\mathcal{C}
                    \right\}
                \end{equation}
                That is, starting with a chain $B$, $B^{\prime}$ consists of
                all points $x\in{A}$ that may be appended to $B$ while still
                defining a chain. We define $\twiddle{B}$ as follows:
                \begin{equation}
                    \twiddle{B}
                    =\begin{cases}
                        B\cup\left\{\,f(B^{\prime})\,\right\},
                            &B^{\prime}\ne\emptyset\\
                        B\,
                            &\textrm{otherwise}
                    \end{cases}
                \end{equation}
                Note that if $B=\twiddle{B}$, then $B$ is maximal. To see this,
                note that if $B^{\prime}\ne\emptyset$, and if
                $B=\twiddle{B}$, then $f(B^{\prime})\in{B}$ must hold by the
                definition of $\twiddle{B}$. But $f$ is a choice function, so
                $f(B^{\prime})\in{B}^{\prime}$, and $B$ and $B^{\prime}$ are
                disjoint, a contradiction. Hence, $B^{\prime}=\emptyset$.
                If $B^{\prime}$ is empty, then there are no elements we can
                add to $B$ while still maintaining a chain. This means $B$ must
                be maximal. Conversely, if $B$ is maximal, then $B=\twiddle{B}$,
                and hence this equality holds if and only if $B$ is maximal.
                We prove Hausdorff's maximality principle by finding an element
                $B\in\mathcal{C}$ such that $B=\twiddle{B}$ and $C\subseteq{B}$,
                where $C$ is the initial chain from our hypothesis.
                \par\hfill\par
                Given our starting chain $C\in\mathcal{C}$, we will call
                $T\subseteq\mathcal{C}$ a \textbf{tower} over $C$ provided
                that the following properties hold:
                \begin{enumerate}
                    \item
                        $C\in{T}$.
                    \item
                        If $\mathcal{O}\subseteq{T}$, then
                        $\bigcup\mathcal{O}\in{T}$. That is, $T$ is closed
                        with respect to unions.
                    \item
                        If $B\in{T}$, then $\twiddle{B}\in{T}$.
                \end{enumerate}
                The set $\mathcal{C}$ is a tower over $C$, so towers certainly
                exist. Define $T_{0}$ to be the intersection of all towers
                containing $C$. The defining properties of a tower are closed
                under intersections, so $T_{0}$ is also a tower over $C$.
                Moreover, $T_{0}$ is totally ordered with respect to inclusion.
                To prove this, let $T_{1}$ consist of all sets in $T_{0}$ that
                are comparable with all other sets in $T_{0}$. That is, for
                each $\mathcal{U}\in{T}_{1}$ and for all
                $\mathcal{V}\in{T}_{0}$, either
                $\mathcal{U}\subseteq\mathcal{V}$,
                or $\mathcal{V}\subseteq\mathcal{U}$. $T_{1}$ is also a tower.
                $C\in{T}_{1}$ follows from the fact that $T_{0}$ is a tower
                over $C$, and $T_{1}$ being closed to unions comes
                from $T_{1}$ being totally ordered with respect to inclusion
                (this is its defining property). To conclude that $T_{1}$ is a
                tower, we must now show that $B\in{T}_{1}$ implies
                $\twiddle{B}\in{T}_{1}$.
                \par\hfill\par
                Given $B\in{T}_{1}$, define:
                \begin{equation}
                    T_{B}=\left\{\,
                        \mathcal{U}\in{T}_{1}
                        \;\big|\;
                        \twiddle{B}\subseteq\mathcal{U}
                        \textrm{ or }
                        \mathcal{U}\subseteq{B}
                    \right\}
                \end{equation}
                Then $T_{B}$ is also a tower over $C$. The first two properties
                follow straight from the definition, the third one requires
                some effort. Suppose $\mathcal{U}\in{T}_{B}$. Note, by the
                definition of $T_{B}$ and $T_{1}$, $\mathcal{U}$ must be an
                element of $T_{0}$ as well. Since $T_{0}$ is a tower,
                $\twiddle{\mathcal{U}}\in{T}_{0}$ is also true.
                By definition of $T_{B}$, $\mathcal{U}\in{T}_{B}$ implies that
                either $\mathcal{U}\subseteq{B}$ or
                $\twiddle{B}\subseteq\mathcal{U}$. If $\mathcal{U}\subseteq{B}$,
                then since $B\in{T}_{1}$, $B$ must be comparable to
                $\twiddle{\mathcal{U}}$
                (since $\twiddle{\mathcal{U}}\in{T}_{0}$), and hence either
                $\twiddle{\mathcal{U}}\subseteq{B}$ or
                $B\subseteq\twiddle{\mathcal{U}}$. If
                $\twiddle{\mathcal{U}}\subseteq{B}$, then
                $\twiddle{\mathcal{U}}\in{T}_{B}$, by definition. If
                $B\subseteq\twiddle{\mathcal{U}}$, then we have the string of
                inclusions
                $\mathcal{U}\subseteq{B}\subseteq\twiddle{\mathcal{U}}$,
                and by definition of $\twiddle{\mathcal{U}}$ (given by our
                choice function), this means either $\mathcal{U}=B$ or
                $B=\tilde{\mathcal{U}}$. Either way,
                $\tilde{\mathcal{U}}\in{T}_{B}$ is true. Similarly, if
                $B\subseteq\mathcal{U}$, then $\tilde{\mathcal{U}}\in{T}_{B}$,
                and hence $T_{B}$ is a tower. But $T_{B}\subseteq{T}_{0}$,
                and $T_{0}$ is the intersection of \textbf{all} towers
                above $C$, meaning $T_{0}\subseteq{T}_{B}$, and therefore
                $T_{0}=T_{B}$. Since $T_{B}\subseteq{T}_{1}\subseteq{T}_{0}$,
                and since $T_{0}=T_{B}$, if $T_{1}$ is squeezed in the
                middle, then $T_{0}=T_{1}$ as well. But $T_{1}$ is, by
                definition, totally ordered with respect to inclusion, and
                hence $T_{0}$ is totally ordered as well.
                \par\hfill\par
                The finale is within reach, and far simpler than the proceeding
                paragraphs. Define $B=\bigcup{T}_{0}$. Then, since $T_{0}$ is
                totally ordered, by property 2 of towers this union is
                in $T_{0}$. That is, $B$ is a chain and $C\subseteq{B}$.
                But moreover, since $T_{0}$ is a tower and $B\in{T}_{0}$,
                we have $\twiddle{B}\in{T}_{0}$. But $B$ is the union of
                all things in $T_{0}$, so $\twiddle{B}\subseteq{B}$.
                But $B\subseteq\twiddle{B}$ by definition, and therefore
                $B=\twiddle{B}$. That is, $B$ is a maximal chain and
                $C\subseteq{B}$, completing the proof.
            \end{proof}
        \subsection{Proving Zorn's Lemma}
            With Hausdorff's maximality principle in our toolbelt, we proceed
            with relative ease (or at least with increased confidence).
            \begin{theorem}[\textbf{Zorn's Lemma}]
                If $(A,\,\leq)$ is a non-empty poset such that every chain
                chain $C\subseteq{A}$ has an upper bound, then there exists
                a maximal element $s\in{A}$.
            \end{theorem}
            \begin{proof}
                Since $A$ is non-empty, there is some $x\in{A}$. But the set
                $C=\{\,x\,\}$ is a chain (there is only one element, so every
                element of $C$ is comparable to every other element). By
                Hausdorff's maximality principle there exists a maximal
                chain $D$ such that $C\subseteq{D}$. By
                hypethesis every chain has an upper bound. Let
                $s\in{A}$ be an upper bound of $D$. Then $s$ must be
                maximal. For if not, then there is some $y\in{A}$ such that
                $s<y$. But then $D\cup\{\,y\,\}$ is a chain since
                every element of $D$ is comparable with $s$ and $s$
                is comparable with $y$. But this chain is strictly larger than
                $D$ since it contains $y$, contradicting the fact
                that $D$ is a maximal chain. Hence, $s$ is a
                maximal element.
            \end{proof}
        \subsection{The Axiom of Choice from Zorn's Lemma}
            The latter direction, proving the axiom of choice from Zorn's lemma,
            is far more straight-forward. Let's begin.
            \begin{theorem}
                Assuming the axioms from Zermelo-Fraenkel set theory, if
                every non-empty poset $(A,\,\leq)$, with the property that
                every chain has an upper bounded, must necessarily have a
                maximal element, then the axiom of choice is true.
            \end{theorem}
            \begin{proof}
                Zermelo-Fraenkel set theory allows us to consider the set of
                all functions from a set $A$ to a set $B$.  This may be
                constructed using the axioms of the power set, union, pairing,
                and specification, and we will use this freely throughout.
                Suppose $\mathcal{O}$ is a non-empty
                set such that for all $\mathcal{U}\in\mathcal{O}$ the set
                $\mathcal{U}$ is also non-empty. That is, suppose the hypotheses
                from the axiom of choice are given. Define $\mathcal{X}$ as
                follows:
                \begin{equation}
                    \mathcal{X}
                    =\left\{\,
                        f:\mathcal{A}
                            \rightarrow\bigcup\mathcal{A}
                        \;\big|\;
                        \mathcal{A}
                        \subseteq\mathcal{O}
                        \textrm{ and }
                        f(\mathcal{U})\in\mathcal{U}
                        \textrm{ for each }
                        \mathcal{U}\in\mathcal{A}
                    \right\}
                \end{equation}
                That is, $\mathcal{X}$ is the set of all
                \textit{partial choice functions}. Since $\mathcal{O}$ is
                non-empty, there is some $\mathcal{U}\in\mathcal{O}$.
                But by hypothesis $\mathcal{U}$ is non-empty, so there is some
                $x\in\mathcal{U}$. Let
                $\mathcal{A}=\{\,\mathcal{U}\,\}$ and define
                $f:\mathcal{A}\rightarrow\bigcup\mathcal{A}$
                by $f(\mathcal{U})=x$. Then $f$ is an example of a partial
                choice function, meaning $\mathcal{X}$ is \textbf{non-empty}.
                Order $\mathcal{X}$ as follows. We write
                $f\leq{g}$ provided the domain of $f$ is a subset of the domain
                of $g$, and if the restriction of $g$ to the domain of $f$ is
                just $f$. That is, given
                $f:\mathcal{A}\rightarrow\bigcup\mathcal{A}$
                and
                $g:\mathcal{B}%
                    \rightarrow\bigcup\mathcal{B}$,
                we write $f\leq{g}$ if
                $\mathcal{A}\subseteq\mathcal{B}$ and
                $g|_{\mathcal{A}}=f$. This means $g$ is an
                \textit{extension} of $f$. We now prove that
                $(\mathcal{X},\,\leq)$ is a poset.
                Reflexivity is realized, restricting a function to its entire
                domain simply produces the same function. Anti-symmetry is
                a consequence of the anti-symmetry of the inclusion relation.
                That is,
                $\mathcal{A}\subseteq\mathcal{B}$
                and
                $\mathcal{B}\subseteq\mathcal{A}$
                implies
                $\mathcal{A}=\mathcal{B}$. If
                $f\leq{g}$ and $g\leq{f}$, the functions must then just be
                equal to each other since their domains are the same and their
                restrictions to these domains are identical. Lastly,
                transitivity follows from the transitivity of inclusion.
                If $\mathcal{A}\subseteq\mathcal{B}$ and
                $\mathcal{B}\subseteq\mathcal{C}$,
                then
                $\mathcal{A}\subseteq\mathcal{C}$.
                If $f\leq{g}$ and $g\leq{h}$, then $g$ is an extension of
                $f$ and $h$ is an extension of $g$. Restricting
                $h$ to the domain of $f$ is equivalent to first restricting
                it to the domain of $g$ (which yields $g$ since $h$ is an
                extension of $g$), and then restricting this to the domain of
                $f$ (which produces $f$ since $g$ is an extension of $f$).
                Hence, $h$ is an extension of $f$.
                \par\hfill\par
                Our argument thus far shows that $\mathcal{X}$ is non-empty
                and $(\mathcal{X},\,\leq)$ is a poset. We now prove that
                all chains have an upper bound. Given a chain
                $\Delta\subseteq\mathcal{X}$, define
                $\mathcal{A}$ to be the union of the domains. That is:
                \begin{equation}
                    \mathcal{A}=
                    \bigcup_{f\in\Delta}\textrm{dom}(f)
                \end{equation}
                By the definition of union, and of $\mathcal{X}$, this produces
                a subset of $\mathcal{O}$. We
                define $f:\mathcal{A}\rightarrow\bigcup\mathcal{A}$ as follows.
                Given $\mathcal{U}\in\mathcal{A}$, by the definition of
                union there is some function
                $f_{\mathcal{B}}:\mathcal{B}\rightarrow\bigcup\mathcal{B}$
                contained in $\Delta$ such that
                $\mathcal{U}\in\textrm{dom}(f_{\mathcal{B}})=\mathcal{B}$.
                Define
                $f(\mathcal{U})=f_{\mathcal{B}}(\mathcal{U})$.
                There is no ambiguity in the definition here. If
                $f_{\mathcal{C}}:\mathcal{C}\rightarrow\bigcup\mathcal{C}$ is
                another such function with
                $\mathcal{U}\in\textrm{dom}(f_{\mathcal{C}})$, then since
                $\Delta$ is a chain either $\mathcal{B}\subseteq\mathcal{C}$ or
                $\mathcal{C}\subseteq\mathcal{B}$. Then either
                $f_{\mathcal{C}}$ is an extensive of $f_{\mathcal{B}}$, or
                vice-versa. In either case they produce the same output
                for the input $\mathcal{U}$, meaning $f(\mathcal{U})$ is
                well-defined. Since $f_{\mathcal{B}}$ is a partial choice
                function, and since we have defined
                $f(\mathcal{U})=f_{\mathcal{B}}(\mathcal{U})$, we may conclude
                that $f(\mathcal{U})\in\mathcal{U}$ is true and hence
                $f:\mathcal{A}\rightarrow\bigcup\mathcal{A}$ is a partial
                choice function. That is, $f\in\mathcal{X}$. But by
                construction $f$ is an extension of all of the elements of
                $\Delta$, and hence the chain $\Delta$ has (at least one)
                upper bound, namely the function $f$.
                \par\hfill\par
                We now have that $(\mathcal{X},\,\leq)$ is a non-empty poset
                where each chain has an upper bound. By Zorn's lemma there is
                a maximal element $f$. We conclude the proof by showing that
                the domain of $f$ is all of $\mathcal{O}$, meaning $f$ is a
                choice function on $\mathcal{O}$. If not then there is some
                $\mathcal{U}$ not contained in the domain of $f$. But if
                $\mathcal{U}\in\mathcal{O}$, then by hypothesis $\mathcal{U}$
                is non-empty and contains some element $x$. We may then extend
                $f$ by defining $f(\mathcal{U})=x$. But $f$ is maximal, meaning
                it cannot be extended, a contradiction. Therefore the domain of
                $f$ is all of $\mathcal{O}$ and we have found a choice function.
            \end{proof}
            From this we have that Zorn's lemma and the axiom of choice are
            equivalent. We have actually established the equivalence of the
            Axiom of Choice (AC), Zorn's Lemma (ZL), and
            Hausdorff's Maximality Principle (HMP) since
            $\textrm{AC}\Rightarrow\textrm{HMP}$, and
            $\textrm{HMP}\Rightarrow\textrm{ZL}$ (the first part of the paper).
            Since we have now proven $\textrm{ZL}\Rightarrow\textrm{AC}$, by
            the transitivity of implication all three of these statements
            are logically equivalent (assuming the remaining axioms of
            Zermelo-Fraenkel set theory).
    \bibliographystyle{plain}
    \bibliography{bib.bib}
\end{document}
