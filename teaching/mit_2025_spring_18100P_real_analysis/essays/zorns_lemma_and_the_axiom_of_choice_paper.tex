%-----------------------------------LICENSE------------------------------------%
%   This file is part of Mathematics-and-Physics.                              %
%                                                                              %
%   Mathematics-and-Physics is free software: you can redistribute it and/or   %
%   modify it under the terms of the GNU General Public License as             %
%   published by the Free Software Foundation, either version 3 of the         %
%   License, or (at your option) any later version.                            %
%                                                                              %
%   Mathematics-and-Physics is distributed in the hope that it will be useful, %
%   but WITHOUT ANY WARRANTY; without even the implied warranty of             %
%   MERCHANTABILITY or FITNESS FOR A PARTICULAR PURPOSE.  See the              %
%   GNU General Public License for more details.                               %
%                                                                              %
%   You should have received a copy of the GNU General Public License along    %
%   with Mathematics-and-Physics.  If not, see <https://www.gnu.org/licenses/>.%
%------------------------------------------------------------------------------%
\documentclass{article}
\usepackage{amsthm}
\usepackage{amssymb}
\usepackage{hyperref}
\title{Zorn's Lemma and the Axiom of Choice}
\author{Ryan Maguire}
\date{\today}
\setlength{\parindent}{0em}
\setlength{\parskip}{0em}
\newtheorem{theorem}{Theorem}
\newtheorem{axiom}{Axiom}

\begin{document}
    \maketitle
    \begin{abstract}
        The axiom of choice is one of the central postulates of modern
        set theory is commonly used in analysis, algebra, topology,
        and geometry. The application is often implicit, as mathematicians
        will instead invoke some form of Zorn's lemma in their work to avoid
        explicit mention of a choice function. It is well-known that under the
        assumptions of Zermelo-Fraenkel set theory the axiom of choice and
        Zorn's lemma are equivalent. We provide a (nearly) self-contained proof
        of this equivalence and do not assume the reader is deeply familiar
        with either of the two notions.
    \end{abstract}
    \section{Introduction}

\end{document}