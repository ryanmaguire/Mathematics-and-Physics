%-----------------------------------LICENSE------------------------------------%
%   This file is part of Mathematics-and-Physics.                              %
%                                                                              %
%   Mathematics-and-Physics is free software: you can redistribute it and/or   %
%   modify it under the terms of the GNU General Public License as             %
%   published by the Free Software Foundation, either version 3 of the         %
%   License, or (at your option) any later version.                            %
%                                                                              %
%   Mathematics-and-Physics is distributed in the hope that it will be useful, %
%   but WITHOUT ANY WARRANTY; without even the implied warranty of             %
%   MERCHANTABILITY or FITNESS FOR A PARTICULAR PURPOSE.  See the              %
%   GNU General Public License for more details.                               %
%                                                                              %
%   You should have received a copy of the GNU General Public License along    %
%   with Mathematics-and-Physics.  If not, see <https://www.gnu.org/licenses/>.%
%------------------------------------------------------------------------------%
\documentclass{article}
\usepackage{amsthm}
\usepackage{amssymb}
\usepackage{hyperref}
\hypersetup{colorlinks = true, linkcolor = blue}
\title{Zorn's Lemma and the Axiom of Choice}
\author{Ryan Maguire}
\date{\today}
\setlength{\parindent}{0em}
\setlength{\parskip}{0em}
\newtheorem{theorem}{Theorem}
\newtheorem{axiom}{Axiom}

\begin{document}
    \maketitle
    \begin{abstract}
        The axiom of choice is one of the central postulates of modern
        set theory and is commonly used in analysis, algebra, topology,
        and geometry. The application is often implicit, as mathematicians
        will instead invoke some form of Zorn's lemma in their work to avoid
        explicit mention of a choice function. It is well-known that under the
        assumptions of Zermelo-Fraenkel set theory the axiom of choice and
        Zorn's lemma are equivalent. We provide a (nearly) self-contained proof
        of this equivalence and do not assume the reader is deeply familiar
        with either of the two notions.
    \end{abstract}
    \tableofcontents
    \section{Introduction}
        Consider a collection of buckets, perhaps infinitely many. The buckets
        contains various objects, but each one contains at least one
        \textit{thing}. The primary claim under investigation is that it is
        possible to go to each bucket and choose one of the items from
        each, keeping a record of which ones we pick. One may have reservations
        about doing this with infinitely many buckets, but we use similar ideas
        regularly without objection. For example, the number $\pi$ is the
        ratio of the circumference of a circle to its diameter and has a
        value of $\pi=3.1415926\dots$ but what do these ellipsis represent?
        An \textit{infinite decimal}, and few would have any real quarrel
        with such a notion. To further justify our bucket proposition,
        the \textit{Cartesian product} over all of the buckets can be thought
        of as an infinite tuple where the first entry is an item in the
        first bucket, the second element corresponds to something in the
        second bucket, so on and so forth. Claiming that we may go about the
        buckets and collect an item from each is equivalent to saying that
        \textbf{the Cartesian product of non-empty sets is non-empty}. Buckets
        are not very formal, mathematically, but we may model this idea using
        the language of set theory. We have the following.
        \begin{axiom}[\textbf{Axiom of Choice}]
            If $\mathcal{O}$ is a non-empty set such that for all
            $\mathcal{U}\in\mathcal{O}$ it is true that $\mathcal{U}$ is
            non-empty, then there is a function
            $f:\mathcal{O}\rightarrow\bigcup\mathcal{O}$ with the property
            that for all
            $\mathcal{U}\in\mathcal{O}$ we have $f(\mathcal{U})\in\mathcal{U}$.
        \end{axiom}
        We have used the set-theoretic notation for union, but we may just as
        easily use the (perhaps) more familiar notation found in analysis:
        \begin{equation}
            \bigcup\mathcal{O}
            =\bigcup_{\mathcal{U}\in\mathcal{O}}\mathcal{U}
        \end{equation}
        The axiom of choice has far reaching applications in nearly all
        branches of mathematics. The topologist relies on it when they invoke
        \textit{Tychonoff's theorem}, the graph theorist regularly
        uses it to assert the existence of spanning trees, and most algebraists
        enjoy the fact that every vector space has a basis. There are about
        a dozen more statements in set theory that are equivalent to the
        axiom of choice, some of which may seem even more obvious. Consider
        the following.
        \begin{theorem}
            If $A$ and $B$ are sets, and if $f:A\rightarrow{B}$ is a
            surjective function, then there is a \textit{right inverse}
            $g:B\rightarrow{A}$ such that $(f\circ{g})(b)=b$ for all
            $b\in{B}$.
        \end{theorem}
        \begin{proof}
            Since $f$ is surjective, for all $b\in{B}$ there is an $a\in{A}$
            such that $f(a)=b$. Define $g(b)=a$. Then, by definition, we have
            $(f\circ{g})(b)=b$ for all $b\in{B}$.
        \end{proof}
        This theorem is equivalent to the axiom of choice. We are
        \textit{choosing} $g(b)=a$. If this theorem seems more believable,
        then you may want to except the axiom as true. But now consider this.
        \begin{theorem}[\textbf{Banach-Tarski Paradox}]
            It is possible to break the unit sphere $\mathbb{S}^{2}$, which
            is defined by:
            \begin{equation}
                \mathbb{S}^{2}
                =\left\{\,(x,\,y,\,z)\;\big|\;x^{2}+y^{2}+z^{2}=1\,\right\}
            \end{equation}
            into five pieces, and then, using only rotations and translations,
            create two new copies of $\mathbb{S}^{2}$, each being the exact
            same size as the original one.
        \end{theorem}
    \section{Plan of Attack}
        The difficult direction in our proof is using the axiom of choice to
        imply Zorn's lemma. The converse is relatively straight forward, we
        form a partial ordering on the set of all
        \textit{partial choice functions}, ordered by \textit{extensions}.
        That is, given sets $A$ and $B$, and functions
        $f:A\rightarrow\bigcup{A}$ and $g:B\rightarrow\bigcup{B}$, we will
        write $f\leq{g}$ provided $A\subseteq{B}$ and
        $g|_{A}=f$. That is, the restriction of $g$ to $A$ yields the function
        $f$.
\end{document}
