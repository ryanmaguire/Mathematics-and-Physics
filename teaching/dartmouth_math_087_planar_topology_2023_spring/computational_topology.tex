\documentclass{article}
\usepackage{mathtools, esint, mathrsfs} % amsmath and integrals.
\usepackage{amsthm, amsfonts, amssymb}  % Fonts and theorems.
\usepackage{hyperref}                   % Hyperlinks.

% Colors for hyperref.
\hypersetup{
    colorlinks=true,
    linkcolor=blue,
    filecolor=magenta,
    urlcolor=Cerulean,
    citecolor=SkyBlue
}

\title{Computational Topology}
\author{Ryan Maguire}
\date{Fall 2021}
\setlength{\parindent}{0em}
\setlength{\parskip}{0em}

\newtheoremstyle{normal}
    {\topsep}               % Amount of space above the theorem.
    {\topsep}               % Amount of space below the theorem.
    {}                      % Font used for body of theorem.
    {}                      % Measure of space to indent.
    {\bfseries}             % Font of the header of the theorem.
    {}                      % Punctuation between head and body.
    {.5em}                  % Space after theorem head.
    {}

\theoremstyle{plain}
\newtheorem{theorem}{Theorem}[section]
\theoremstyle{normal}
\newtheorem{example}{Example}[section]

\begin{document}
    \maketitle
    These notes come from the homework, lectures, and project work for
    the computational topology course offered by the CS department at
    Dartmouth College during the Fall 2021 semester. Any errors contained
    in these notes are my own.
    \tableofcontents
    \section{Week 1}
        \subsection{Fun with Groups}
            A group is an ordered pair $(G,*)$ where $G$ is a set and $*$ is a
            \textit{binary operation}, a function $*:G\times{G}\rightarrow{G}$,
            such that:
            \begin{enumerate}
                \item For all $a,b,c\in{G}$ it is true that $a*(b*c)=(a*b)*c$
                      (Associativity).
                \item There exists an element $e\in{G}$ such that for all
                      $a\in{G}$ it is true that $a*e=e*a=a$
                      (Existence of Identity).
                \item For all $a\in{G}$ there exists $b\in{G}$ such that
                      $a*b=b*a=e$. (Existence of Inverses).
            \end{enumerate}
            Given $a,b\in{G}$ such that $a*b=e$, the element $b$ is often
            labeled $b=a^{-1}$.
            \begin{example}
                Letting $\mathbb{R}$ denote the real numbers,
                $\mathbb{R}^{+}$ denote the set of all positive real numbers,
                and $+$ and $\cdot$ denoting addition and multiplication of
                real numbers, respectively, $(\mathbb{R},+)$ and
                $(\mathbb{R}^{+},\cdot)$ are groups. This is because addition
                and multiplication are both associative:
                \begin{align}
                    a+(b+c)&=(a+b)+c\\
                    a\cdot(b\cdot{c})&=(a\cdot{b})\cdot{c}
                \end{align}
                $(\mathbb{R},+)$ has 0 as the identity since $a+0=0+a=a$ for
                any real number $a$. $(\mathbb{R}^{+},\cdot)$ has 1 as the
                identity since $a\cdot{1}=1\cdot{a}=a$. Lastly, inverses always
                exist. Given $a\in\mathbb{R}$, the negative of $a$, denoted
                $-a$, acts as the additive inverse since $a+(-a)=0$. Given
                $a\in\mathbb{R}^{+}$, the reciprocal $1/a$ is the
                multiplicative inverse since $a\cdot(1/a)=1$. Since all three
                properties are satisfied, both $(\mathbb{R},+)$ and
                $(\mathbb{R}^{+},\cdot)$ are groups.
            \end{example}
            \begin{example}
                \label{ex:sym_diff_is_group}%
                If $A$ is a set, if $G=\mathcal{P}(A)$ is the power set of $A$,
                and if $\ominus$ is the \textit{symmetric difference} of sets,
                then $(G,\ominus)$ is a group. The symmetric difference of two
                sets $\mathcal{U}$ and $\mathcal{V}$ is the set all of elements
                that are in either $\mathcal{U}$ or $\mathcal{V}$, but not
                both. In set theory notation we can write:
                \begin{equation}
                    \mathcal{U}\ominus\mathcal{V}=
                        (\mathcal{U}\cup\mathcal{V})\setminus
                        (\mathcal{U}\cap\mathcal{V})
                \end{equation}
                The identity element is the empty set $\emptyset=\{\}$ since:
                \begin{equation}
                    \mathcal{U}\ominus\emptyset
                        =(\mathcal{U}\cup\emptyset)\setminus
                            (\mathcal{U}\cap\emptyset)
                        =\mathcal{U}\setminus\emptyset
                        =\mathcal{U}
                \end{equation}
                And similarly for $\emptyset\ominus\mathcal{U}$.
                It is associative, but this requires a little bit of work.
                The inverse of $\mathcal{U}\in{G}$ is $\mathcal{U}$ since:
                \begin{align}
                    \mathcal{U}\ominus\mathcal{U}
                    &=(\mathcal{U}\cup\mathcal{U})\setminus
                      (\mathcal{U}\cap\mathcal{U})\\
                    &=\mathcal{U}\setminus\mathcal{U}\\
                    &=\emptyset
                \end{align}
                Since this structure satisfies the three properties,
                $(\mathcal{P}(A),\ominus)$ is a group.
            \end{example}
            A group homomorphism between two groups $(G,*)$ and $(H,\circ)$ is
            a function $f:G\rightarrow{H}$ such that, for all $a,b\in{G}$, it
            is true that:
            \begin{equation}
                f(a*b)=f(a)\circ{f}(b)
            \end{equation}
            That is, $f$ \textit{respects} the group operations.
            \begin{example}
                If $G=\mathbb{R}$, the set of all real numbers,
                $H=\mathbb{R}^{+}$, the set of all positive real numbers, and
                if $f:G\rightarrow{H}$ is the function $f(x)=\exp(x)$, then
                $f$ is a homomorphism with respect to the group structures
                $(\mathbb{R},+)$ and $(\mathbb{R}^{+},\cdot)$, where
                $+$ and $\cdot$ are the familiar notions of addition and
                multiplication, respectively. This is because of one of the
                fundamental identities for exponentiation:
                \begin{equation}
                    \exp(x+y)=\exp(x)\cdot\exp(y)
                \end{equation}
                This is also written as $e^{x+y}=e^{x}e^{y}$ and the
                multiplication symbol is dropped.
            \end{example}
            The kernel of a homomorphism $f:G\rightarrow{H}$ between two
            groups $(G,*)$ and $(H,\circ)$ is the set all elements in
            $G$ which map to the identity element of $H$. In set theory
            notation this is the fiber of the identity $e_{H}$ denoted
            $f^{-1}[\{e_{H}\}]$, where $e_{H}$ is the identity of $H$. We can
            also use set builder notation as follows:
            \begin{equation}
                \ker(f)=\{\,x\in{G}\,|\,f(x)=e_{H}\,\}
            \end{equation}
            A graph is an ordered pair $(V,E)$, where $V$ is a set, and
            $E\subseteq\mathcal{P}(V)$, the power set of $V$, with the property
            that for all $e\in{E}$, there are exactly two distinct elements
            $v_{0},v_{1}\in{e}$. The set $V$ is called the \textit{vertex set},
            and the set $E$ is called the \textit{edge set}. An element
            $e\in{E}$ tells us two vertices are connected by an edge, hence
            the requirement that $e=\{v_{0},v_{1}\}$ is a subset of $V$ with
            exactly two distinct elements.
            \par\hfill\par
            If $(V,E)$ is a graph, there are two groups we can associate to the
            graph. First is the \textit{vertex space}, which is the set of
            all subsets of vertices of the graph. In set theory notation, this
            is the set $\mathcal{P}(V)$, the power set of $V$. This is made a
            group with the symmetric difference operation $\ominus$, as we saw
            in Ex,~\ref{ex:sym_diff_is_group}. Similarly, there is the set of
            all subgraphs and this is called the \textit{edge space}, and it
            too becomes a group with $\ominus$
            \begin{theorem}
                If $(V,E)$ is a graph, if $(\mathcal{P}(V),\ominus)$ is the
                vertex space, if $(\mathcal{P}(E),\ominus)$ is the edge space,
                and if $f:\mathcal{P}(E)\rightarrow\mathcal{P}(V)$ is the
                function defined by $f(\mathcal{W})=\mathcal{U}$ where
                $\mathcal{W}\in\mathcal{P}(E)$ is a subset of edges, and
                $\mathcal{U}$ is the set of all elements of $V$ that have
                odd degree in the graph $(V,\mathcal{W})$, then $f$ is a
                group homomorphism.
            \end{theorem}
            \begin{proof}
                We must prove that, for two collections of edges
                $\mathcal{W}_{0}$ and $\mathcal{W}_{1}$, it is true that:
                \begin{equation}
                    f(\mathcal{W}_{0}\ominus\mathcal{W}_{1})
                    =f(\mathcal{W}_{0})\ominus{f}(\mathcal{W}_{1})
                \end{equation}
                Let $\mathcal{U}_{0}$ be the set of all vertices with odd
                degree in the graph $(V,\mathcal{W}_{0}\ominus\mathcal{W}_{1})$
                and $\mathcal{U}_{1}$ be the set
                $f(\mathcal{W}_{0})\ominus{f}(\mathcal{W}_{1})$. If
                $v\in\mathcal{U}_{0}$, then $v$ has odd degree in the subgraph
                $\mathcal{W}_{0}\ominus\mathcal{W}_{1}$ by definition.
                If $v$ has odd degree in $\mathcal{W}_{0}$, then it cannot have
                odd degree in $\mathcal{W}_{1}$. For if $v$ had odd degree in
                both $\mathcal{W}_{0}$ and $\mathcal{W}_{1}$, then the degree
                of $v$ in $\mathcal{W}_{0}\ominus\mathcal{W}_{1}$ would be
                the degree in $\mathcal{W}_{0}$ plus the degree in
                $\mathcal{W}_{0}$ minus twice the degree in
                $\mathcal{W}_{0}\cap\mathcal{W}_{1}$, per definition of
                symmetric difference and the degree of a vertex. But this
                would be an even number, though we've stated $v$ has odd
                degree in $\mathcal{W}_{0}\ominus\mathcal{W}_{1}$. Thus, if
                $v$ has odd degree in $\mathcal{W}_{0}$, then it can't have
                odd degree in $\mathcal{W}_{1}$. Similarly, if
                $v$ has odd degree in $\mathcal{W}_{1}$, it can't have
                odd degree in $\mathcal{W}_{0}$. By the same argument, $v$
                cannot have even degree in both $\mathcal{W}_{0}$ and
                $\mathcal{W}_{1}$, otherwise it would have even degree in
                $\mathcal{W}_{0}\ominus\mathcal{W}_{1}$. Thus, if
                $v\in\mathcal{U}_{0}$, then $v\in\mathcal{U}_{1}$ and therefore
                $\mathcal{U}_{0}\subseteq\mathcal{U}_{1}$. If
                $v\in\mathcal{U}_{1}$, then either
                $v\in{f}(\mathcal{W}_{0})$ or $v\in{f}(\mathcal{W}_{1})$, but
                not both, per definition of $\ominus$. Per definition of $f$,
                either $v$ has odd degree in $\mathcal{W}_{0}$, or $v$ has
                odd degree in $\mathcal{W}_{1}$, but not both. That is,
                $v\in{f}(\mathcal{W}_{0}\ominus\mathcal{W}_{1})$. Hence,
                $\mathcal{U}_{1}\subseteq\mathcal{U}_{0}$. But it was proved
                that $\mathcal{U}_{0}\subseteq\mathcal{U}_{1}$ and therefore
                $\mathcal{U}_{0}=\mathcal{U}_{1}$.
            \end{proof}
            The identity of the vertex set is the empty set, so the kernel of
            this homomorphism is the set of all subgraphs where every vertex
            has even degree in the subgraph.
            \par\hfill\par
            The edge and vertex spaces can be thought of as vector spaces over
            the two element field $\mathbb{Z}_{2}$ where the basis elements are
            the elements of the edge and vertex spaces, respectively. The
            dimension of the kernel of this homomorphism is the number of
            subgraphs that are even, meaning every vertex has even degree.
            A theorem from graph theory and combinatorics tells us that if
            $G$ is a connected graph, then there are
            $2^{|E|-|V|+1}$ even subgraphs, where $|E|$ is the number of edges
            and $|V|$ is the number of vertex. Breaking the graph into
            its connected components, we can compute the dimension of the
            kernel via multiplying over $2^{|E_{n}|-|V_{n}|+1}$ where
            $|E_{n}|$ and $|V_{n}|$ are the number of edges and vertices,
            respectively, in the $n^{th}$ connected component.
        \subsection{Shortest Paths}
            To find a shortest path through a graph with a strictly
            monotonically decreasing sequence of edge lengths we can take
            Dijkstra's algorithm and modify. Label all of the vertices as
            unvisited and set the distance from the starting node to all of
            the others to infinity. From the start node, look at all of the
            adjacent nodes and change the distance for these points from
            infinity to the length of the edge. Pick one of the adjacent
            vertices and repeat this process, looking at all of the vertices
            adjacent to this one and, if the path from the start node to one of
            these new adjacent vertices is less than the distance currently
            assigned to that vertex, update the vertex with this lower value.
            This is Dijkstra's algorithm, but we need to modify it. If, at a
            certain vertex, we are unable to select an edge that has a length
            strictly less than the edge we just came from, we must terminate
            the path there. If all paths terminate before we reach the
            desired node, return a value indicating such
            (perhaps $-1$ as per most programming standards).
        \subsection{Winter}
            Let $f:E\rightarrow\mathbb{N}$ be the function such that for each
            $e\in{E}$, $f(e)$ is the minimum of the number of vertices in each
            of the components of $E\setminus\{e\}$. So, for an edge to a leaf,
            $f(e)=1$. Let $e^{*}$ be the edge with the maximum value of $f$
            (Such a value exists since $E$ is finite). Since $f(e)=1$ if and
            only if $e$ is an edge to a leaf, if $f(e^{*})=1$ then there are
            at most 4 vertices (since the graph is a bush) and then removing
            this edge results in the components having at least $n/4$ vertices.
            If $f(e^{*})>1$, then $e^{*}$ is an internal edge. Let $u$ and
            $v$ be the vertices for $e^{*}$. Since $e^{*}$ is an internal edge,
            and since this is a bush, both $u$ and $v$ have two other edges
            connected to them. If the connected component in $E\setminus\{e\}$
            containing $u$ has 2 or more vertices fewer than the
            component containing $v$, by selecting one of the other edges
            connected to $v$ we obtain an edge $\tilde{e}$ such that
            $f(\tilde{e})=f(e^{*})+1$, violating the maximality of $e^{*}$.
            Hence, $e^{*}$ will be such that the components have at least
            $n/4$ vertices.
            \par\hfill\par
            Analysis of the firewood algorithm. It terminates because at
            each cut the number of internal edges is reduced by 1, or if the
            edge being cut contains a leaf, the total number of edges is
            being reduced. So each edge can be cut at most twice (once as an
            internal edge, and again if it's an edge to a leaf). There are at
            most $2|E|$ operations. This shows the maximum numbers of bushes
            returned. If we expanded out this recursion we would get a chain
            of unions and the number of bushes is the number of elements in
            this chain of unions. The maximum number is the floor of
            $2|E|/r$ which is $O(n/r)$. This is tight, since if we had the
            4 elements tree consisting of one 3-degree vertex and three
            1-degree vertex, and requested $r=1$, we would get exactly 4
            bushes.
\end{document}
