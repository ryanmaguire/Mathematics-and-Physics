%-----------------------------------LICENSE------------------------------------%
%   This file is part of Mathematics-and-Physics.                              %
%                                                                              %
%   Mathematics-and-Physics is free software: you can redistribute it and/or   %
%   modify it under the terms of the GNU General Public License as             %
%   published by the Free Software Foundation, either version 3 of the         %
%   License, or (at your option) any later version.                            %
%                                                                              %
%   Mathematics-and-Physics is distributed in the hope that it will be useful, %
%   but WITHOUT ANY WARRANTY; without even the implied warranty of             %
%   MERCHANTABILITY or FITNESS FOR A PARTICULAR PURPOSE.  See the              %
%   GNU General Public License for more details.                               %
%                                                                              %
%   You should have received a copy of the GNU General Public License along    %
%   with Mathematics-and-Physics.  If not, see <https://www.gnu.org/licenses/>.%
%------------------------------------------------------------------------------%
\documentclass{article}
\usepackage{amsmath}
\usepackage{geometry}
\usepackage{hyperref}
\hypersetup{colorlinks = true, citecolor = black}
\geometry{margin=1.0in}
\title{MIT PRIMES: Computational Knot Theory}
\author{Ryan Maguire}
\date{\today}
\begin{document}
    \maketitle
    \section{The Problem}
        Knots have become a central part of modern mathematics.
        Topologists and geometers use them because of their central connections
        to manifolds and contact structures, and computer scientists are
        interested in the difficult problems that arise from their study.
        One of the oldest problems in knot theory is the simplest to describe:
        distinguishing two knots. That is, given two knots as input, determine
        if it is possible to deform one into the other without cutting or
        tearing. While this sounds simple enough, this is an amazingly
        difficult problems. Given a knot diagram with $n$ crossings, which is
        known \textit{a priori} to be a simple circle, it may take up to
        $(231n)^{11}$ steps just to unknot it \cite{Lackenby2015Unknotting}.
        For non-trivial knots, the number of steps required may involve
        astronomical numbers that humans cannot comprehend
        \cite{CowardLackenbyReidemeisterUpperBound}.
        \par\hfill\par
        Even with these obstacles, mathematicians still \textit{try} to
        overcome them and find effecient means of distinguishing two knots.
        This is done using \textbf{knot invariants}, which are mathematical
        objects (usually algebraic) that may be assigned to a knot, and which
        do not change as one continuously perturbs the knot. Some of the simpler
        knot invariants are the knot polynomials, such as the single variable
        \textbf{Jones} and \textbf{Alexander} polynomials, or the
        two-variable \textbf{HOMFLY-PT} and \textbf{Khovanov} polynomials.
        These invariants are not perfect, meaning
        it is possible for two \textit{distinct} knots to have the same
        polynomial, but they are quite good at distinguishing knots.
        Out of the 352 millions prime knots with up to 19 crossings, about
        25\% are uniquely determined by their Jones polynomial
        \cite{JonesData}, and 60\% are distinguished by the HOMFLY-PT
        polynomial \cite{HOMFLYData}. Efficiently computing these polynomials
        gives us a quick first line-of-attack when trying to determine if
        two knots are truly the same.
        \par\hfill\par
        Unfortunately, for each of these polynomials (besides the Alexander
        polynomial) the best algorithms are exponential or sub-exponential
        (\cite{Burton2018HOMFLFixedParameter},
        \cite{ThistlethwaiteSpanningTree},
        \cite{Przytycka1991SubexponentiallyCT},
        \cite{MaguireJones}),
        and their computations are known to be \textbf{NP-Hard}
        \cite{HOMFLYPTNPHard}. In short, this means efficiently calculating
        these invariants for a general knot or link would earn the
        mathematician one million dollars \cite{Blank2000TheMP}.
        While most think this is not possible, it shouldn't stop us from
        trying to improve algorithms and implementations. That is precisely
        the aim of this research project.
        \par\hfill\par
        During PRIMES 2025 we worked on improving methods for computing the
        Jones polynomial, and this culminated in being able to efficiently
        compute this invariant for all primes knots of less than 21 crossings
        in just a few days. The tabulation of the Khovanov polynomial is still
        stuck at 17 crossings, our goal to is to improve this all the way to
        20 crossings. We will do this by analyzing the current best method
        (\texttt{FastKh} by Bar-Natan \cite{BarNatan2006FASTKH}) and then
        implement it in C. We will stick to fixed-width integers to try to
        speed things up, even though this will limit us to knots with less than
        a few dozen crossings. An emphasis on speed will be made, but we will
        also certify that our algorithms are correct.
    \section{What to Expect}
        By participating in the research project, you should expect to study
        knot theory, topology, complexity theory, computer programming, and a
        bit of geometry. You will need to develop some comfort with the
        C programming language, and a bit of Python and Sage. You are (k)not
        expected to be an expert programmer or topologist: these things will be
        taught to you.
    \par\hfill\par
    \bibliographystyle{plain}
    \bibliography{../../bib.bib}
\end{document}
