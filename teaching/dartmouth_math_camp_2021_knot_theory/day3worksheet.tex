\documentclass[12pt,letterpaper]{article}
\usepackage{fullpage}
\usepackage[top=1.5cm, bottom=2cm, left=2.5cm, right=2.5cm]{geometry}
\usepackage{amsmath,amsthm,amsfonts,amssymb,amscd,dsfont}
\usepackage{lastpage}
\usepackage{enumerate}
\usepackage{fancyhdr}
\usepackage{mathrsfs}
\usepackage{xcolor}
\usepackage{graphicx}
\usepackage{listings}
\usepackage{hyperref}
\usepackage{multicol}
\usepackage{tcolorbox}

\newcommand{\R}{\mathds{R}}
\hypersetup{%
  colorlinks=true,
  linkcolor=blue,
  linkbordercolor={0 0 1}
}


% Edit these as appropriate
\newcommand\course{}
\newcommand\hwnumber{1}                  % <-- homework number
\usepackage{dsfont}

\usepackage{pgfplots}
\pgfplotsset{compat=1.13}
\usepackage{mathrsfs}
\usetikzlibrary{arrows}
\usepackage{bm}
\setlength\parindent{0pt}

\pagestyle{fancyplain}

\chead{{\Large \textbf{Worksheet: Tricoloring and Reidermeister Moves}}}
\rhead{}
\lfoot{}
\cfoot{}
\rfoot{\small\thepage}
\headsep 1.5em

\begin{document}
\textcolor{white}{hi}

Have you ever asked yourself how can you be sure that the things that you learned in Algebra are actually correct? This is one of the things that mathematicians do: we find new statements that we think are true and then we \textbf{\textit{prove}} they are correct. 

For example, you have probably learned that the roots of a quadratic polynomial $ax^2+bx+c$ are:
$$\frac{-b+\sqrt{b^2-4ac}}{2a}\;\;\;\text{and}\;\;\;\frac{-b-\sqrt{b^2-4ac}}{2a}$$

But can you show that this is true for every polynomial you can come up with and not only with the ones you have checked until now?

Here is a proof of that fact taken from Khan Academy

\begin{tcolorbox}[arc=3mm,boxsep=5mm]
\textbf{Theorem.} Let $f(x)=ax^2+bx+c$ be a polynomial with coefficients in the real numbers. Then, the roots of $f(x)$ are given by $$\frac{-b+\sqrt{b^2-4ac}}{2a}\;\;\;\text{and}\;\;\;\frac{-b-\sqrt{b^2-4ac}}{2a}$$
\end{tcolorbox}

\begin{proof}
We set $ax^2+bx+c=0$ and solve for $x$ as follows,
\begin{multicols}{2}
\includegraphics[scale=0.6,trim={0 0 0cm 1.3cm},clip]{proof1.png}

\includegraphics[scale=0.5]{proof2.png}
\end{multicols}
\end{proof}

In this worksheet, we are going to construct the proof of a fact about knots and tricoloring.


\begin{tcolorbox}[arc=3mm,boxsep=5mm]
\textbf{Theorem:} If a knot diagram is tricolorable, then every other knot diagram equivalent to it is also tricolorable.
\end{tcolorbox}

To start with the proof, we check that applying any Reidermeister move to a tricolorable knot diagram still gives you a tricolorable knot diagram. Please, tricolor all of the new knot diagrams that you get from applying the Reidermeister moves. As you do this, think why your coloring does not affect the tricoloring of the rest of the knot diagram.

\begin{center}
    \includegraphics[scale=0.3]{problemsRMT.jpeg}
\end{center}

Now we know that applying any Reidermeister move to a tricolorable knot diagram gives you a new tricolorable knot diagram. Is this enough to conclude that the Theorem is true? Please discuss in your groups why/why not.

\newpage
\section*{Instructions for the breakout room facilitators}

\begin{itemize}
    \item Students might not be used to seeing mathematical proofs, so this is an activity in which I expect you to interact a lot with them. You are welcome to give hints and to help them write what they think in a formal enough way.
    
    \item When they are deciding about the coloring, make sure that they don't change the coloring of the strings that are part of the rest of the knot.
    
    \item You can cite the theorem that we learned about yesterday:
    
    \textbf{Theorem:} If two knot diagrams are equivalent, then you can get from one to the other by a finite sequence of Reidermeister moves.
    
    \item If you are done early, please ask students to try to tricolor the figure 8 knot:
    
    \includegraphics[]{Juanita's_lesson_plans/Day_3/Figure_Eight_Knot.jpg}
    
\end{itemize}

\newpage

\section*{Solution}

We first note that given a tricolorable knot diagram, performing any Reidermeister move gives a new tricolorable knot diagram.

\includegraphics[scale=0.3]{solutionsRMT.jpeg}

Because of the theorem that states that if two knot diagrams are equivalent, then you can get from one to the other by a finite sequence of Reidermeister moves, we can conclude that every knot diagram equivalent to a tricolorable knot diagram is also tricolorable.
\end{document}
