\documentclass[12pt,letterpaper]{article}
\usepackage{fullpage}
\usepackage[top=1.5cm, bottom=2cm, left=2.5cm, right=2.5cm]{geometry}
\usepackage{amsmath,amsthm,amsfonts,amssymb,amscd,dsfont}
\usepackage{lastpage}
\usepackage{enumerate}
\usepackage{fancyhdr}
\usepackage{mathrsfs}
\usepackage{xcolor}
\usepackage{graphicx}
\usepackage{listings}
\usepackage{hyperref}
\usepackage{graphics}

\newcommand{\R}{\mathds{R}}
\hypersetup{%
  colorlinks=true,
  linkcolor=blue,
  linkbordercolor={0 0 1}
}


% Edit these as appropriate
\newcommand\course{}
\newcommand\hwnumber{1}                  % <-- homework number
\usepackage{dsfont}

\usepackage{pgfplots}
\pgfplotsset{compat=1.13}
\usepackage{mathrsfs}
\usetikzlibrary{arrows}
\usepackage{bm}

\pagestyle{fancyplain}

\chead{{\Large \textbf{Math camp, day 3}}}
\rhead{}
\lfoot{}
\cfoot{}
\rfoot{\small\thepage}
\headsep 1.5em

\begin{document}

    \section*{Objectives}
    \begin{enumerate}
        \item Students will be able to \textbf{recognize} a mathematical proof.
        \item Students will be able to \textbf{give examples} of tricoloring of knots.
    \end{enumerate}
	\section{Check in and recap}
	    \textit{10 minutes}
	    
	    
	    \section{Virtual Knots}
	
	\textbf{Challenge:} Try to draw a knot picture with no crossings that is not a picture of the unknot.
	
	\textit{13 minutes}
	
	We cannot do this! Is there a better place to draw knot pictures, so we can draw harder knots that have no crossings? Example: a torus knot. \textbf{We need the plastic torus to see this}.
	
	\hspace{1cm}\includegraphics[scale=0.4]{extorus.jpeg}
	
	Using a torus, we can draw the trefoil knot without any crossings!
	
	\includegraphics[scale=0.3]{trefiol.png}
	
	\textit{20 minutes}
	
	But do we really want to draw tori or more surfaces? You just saw me struggling with it, we want to draw things in 2D for as long as we can.
	
    This is why we add a \textbf{virtual crossing}:
    
    \includegraphics[scale=0.4]{vk1.jpeg}
    
	\textbf{Definition:} A \textbf{virtual knot diagram} is described by a knot picture, as before, but with a second type of crossing, a virtual crossing, indicated by encircling the vertex (just as in the example before).
	
	\textbf{Example:}
	How would we draw a virtual knot diagram for the trefoil knot that we saw drawn in the torus?
	
	\textit{27 minutes}
	
	For you to know, we can use generalized Reidermeister moves to find equivalences of virtual knot diagrams. We can use the old moves, but we can also do:
	
	\includegraphics[scale=0.5]{rm1.png}\hspace{1cm}\includegraphics[scale=0.5]{rm2.png}
	
	\textit{30 minutes}
	
	\section{Tricoloring}  
	
    We need to know that an arc is a continuous line segment. Give one example of an arc and one non-example
    
    So, a knot diagram is composed of arcs.
	
	\textbf{Definition:} A knot diagram is tricolorable if every arc of the knot can be coloured one of three different colours such that:
	\begin{enumerate}
	    \item At each crossing, either arcs of all three colors meet, or only arcs of a single color meet.
	    \item At least three colors are used.
	\end{enumerate}
	
	Things that are valid: Show example with 1 color or 3 colors.
	
	Things that are not valid: Show examples with two colors.
	
	\textbf{Problem: }Can we decide what knots are tricolorable and what knots are not?
	
		\textit{40 minutes}
	
	Examples
	\begin{enumerate}
	    \item The unknot is not tricolorable. We cannot use more than three colors. Can we choose a different picture of the unknot to do this?
	    Examples:
	    
	    \includegraphics[scale=0.3]{triunknot.jpeg}
	    
	    
	    What if we choose an equivalent knot?
	    
	    Activity: tricolor the knots that are all equivalent to the trefoil knot.
	    
	    \includegraphics[scale=0.3]{trefoileq.jpeg}
	    
	    So, it looks like it doesn't matter what equivalent knot do we choose, we get the same answer. How do we make sure that this is the case?
	    
	    The statement that we want to prove is: "If a knot is (not) tricolorable, then any equivalent knot is (not) tricolorable. Think about this during the break.
	    
	    \textit{55 minutes}
	    
	    \textit{Break: 70 minutes}
	    
	    Any ideas on how to do this?
	    
	    How do we know if two knots are equivalent? What did we learn yesterday?
	    
	    Yes! Reidemeister moves! Two knots are equivalent if we can get from one to the other by using Reidemeister moves. So, if we take a tricolorable knot, can we tricolor any new knot that we get from a Reidemeister move.
	    
	    Activity: Tricoloring and Reidemeister moves.
	    
	    \textit{90 minutes}
	    
	    
	    Whenever we convince ourselves that something is true, as mathematicians, we want to figure out if some generalizations are true. For example, let us try to tricolor the figure 8 knot.
	    
	    \includegraphics{Juanita's_lesson_plans/Day_3/Figure_Eight_Knot.jpg}
	    
	    Definitely not possible! So, can we add one more color and see what we get?
	    
	    \includegraphics[]{Juanita's_lesson_plans/Day_3/Figure_Eight_Knot_Quadcoloring.jpg}
        	    
	    So we gave a negative answer to our problem, but then took some restrictions away until we were able to say that the problem is solvable.
	    
	    \textit{95 minutes}
	    
	    
	\end{enumerate}
\end{document}
