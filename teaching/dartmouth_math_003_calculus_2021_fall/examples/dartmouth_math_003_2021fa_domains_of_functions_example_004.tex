%-----------------------------------LICENSE------------------------------------%
%   This file is part of Mathematics-and-Physics.                              %
%                                                                              %
%   Mathematics-and-Physics is free software: you can redistribute it and/or   %
%   modify it under the terms of the GNU General Public License as             %
%   published by the Free Software Foundation, either version 3 of the         %
%   License, or (at your option) any later version.                            %
%                                                                              %
%   Mathematics-and-Physics is distributed in the hope that it will be useful, %
%   but WITHOUT ANY WARRANTY; without even the implied warranty of             %
%   MERCHANTABILITY or FITNESS FOR A PARTICULAR PURPOSE.  See the              %
%   GNU General Public License for more details.                               %
%                                                                              %
%   You should have received a copy of the GNU General Public License along    %
%   with Mathematics-and-Physics.  If not, see <https://www.gnu.org/licenses/>.%
%------------------------------------------------------------------------------%
\documentclass{article}
\usepackage{amssymb}  % Needed for mathbb.
\usepackage{graphicx} % Needed for figures.
\usepackage{hyperref} % Hyperlinks for figures.
\hypersetup{colorlinks = true, linkcolor = blue}
\graphicspath{{../../../images/}}

\title{Domains of Functions - Example 4}
\author{Ryan Maguire}
\date{\today}

% No indent and no paragraph skip.
\setlength{\parindent}{0em}
\setlength{\parskip}{0em}

\begin{document}
    \maketitle
    The trigonometric function sine measures the height of a point on the
    unit circle as a function of the angle this point makes with the $x$ axis.
    From this we can see that $\sin(x)$ will lie between $-1$ and $1$ for any
    input $x$, but we can also see that there are angles that result in
    $0$. In particular, for every integer $n$ (positive, negative, or zero),
    $\sin(n\pi)=0$. The graph of $\sin(x)$ is shown in
    Fig.~\ref{fig:sin_x}.
    \par\hfill\par
    Consider the expression:
    \begin{equation}
        f(x)=\frac{x}{\sin(x)}
    \end{equation}
    For which values of $x$ is $f(x)$ well-defined? $\sin(x)$ is well-defined
    for any real number, so the only thing to look out for is a
    division-by-zero.
    This occurs precisely when $\sin(x)=0$, which we've stated happens at the
    values $x=n\pi$ where $n$ is any integer. So the domain is the set:
    \begin{equation}
        D=\{\,x\in\mathbb{R}\,|\,x\ne{n\pi}\textit{ for any }n\in\mathbb{Z}\,\}
    \end{equation}
    This reads that $D$ is the set of all real numbers $x$ such that $x$ is
    not equal to $n\pi$ for any $n\in\mathbb{Z}$. The set $\mathbb{Z}$ is the
    set of all integers, positive, negative, or zero. Use of the letter Z stems
    from the German word \textit{Zahl}, meaning number.
    \par\hfill\par
    If we examine Fig.~\ref{fig:x_by_sin_x} we see something peculiar happen
    at $x=0$. Even though we have a division-by-zero, the function seems well
    behaved in this region. This is because the \textit{limit} as $x$
    approaches zero for $x/\sin(x)$ is well behaved and tends towards 1. We
    don't yet have the machinery to prove this, but this is an example of a
    function such that $f(0)$ is undefined, yet the limit of $f(x)$ as $x$
    approaches 0 is well-defined.
    \begin{figure}
        \centering
        \includegraphics{sin_x.pdf}
        \caption{The sine function}
        \label{fig:sin_x}
    \end{figure}
    \begin{figure}
        \centering
        \includegraphics{x_by_sin_x.pdf}
        \caption{The function $f(x)$}
        \label{fig:x_by_sin_x}
    \end{figure}
    \newpage
    I, the copyright holder of this work, release it into the public domain.
    This applies worldwide. In some countries this may not be legally possible;
    if so: I grant anyone the right to use this work for any purpose, without
    any conditions, unless such conditions are required by law.
    \par\hfill\par
    The source code used to generate this document is free software and released
    under version 3 of the GNU General Public License.
\end{document}
