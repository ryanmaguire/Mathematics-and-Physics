%-----------------------------------LICENSE------------------------------------%
%   This file is part of Mathematics-and-Physics.                              %
%                                                                              %
%   Mathematics-and-Physics is free software: you can redistribute it and/or   %
%   modify it under the terms of the GNU General Public License as             %
%   published by the Free Software Foundation, either version 3 of the         %
%   License, or (at your option) any later version.                            %
%                                                                              %
%   Mathematics-and-Physics is distributed in the hope that it will be useful, %
%   but WITHOUT ANY WARRANTY; without even the implied warranty of             %
%   MERCHANTABILITY or FITNESS FOR A PARTICULAR PURPOSE.  See the              %
%   GNU General Public License for more details.                               %
%                                                                              %
%   You should have received a copy of the GNU General Public License along    %
%   with Mathematics-and-Physics.  If not, see <https://www.gnu.org/licenses/>.%
%------------------------------------------------------------------------------%
\documentclass{article}
\usepackage{amsmath} % Needed for align.
\usepackage{amssymb} % Needed for mathbb.
\usepackage{amsthm}  % For the theorem environment.

\newtheoremstyle{normal}
    {\topsep}               % Amount of space above the theorem.
    {\topsep}               % Amount of space below the theorem.
    {}                      % Font used for body of theorem.
    {}                      % Measure of space to indent.
    {\bfseries}             % Font of the header of the theorem.
    {}                      % Punctuation between head and body.
    {.5em}                  % Space after theorem head.
    {}

\theoremstyle{normal}
\newtheorem{definition}{Definition}

\title{$\varepsilon-\delta$ Continuity - Linear Functions}
\author{Ryan Maguire}
\date{\today}

% No indent and no paragraph skip.
\setlength{\parindent}{0em}
\setlength{\parskip}{0em}

\begin{document}
    \maketitle
    The definition of continuity is as follows:
    \begin{definition}
        A real-valued function that is continuous at a point
        $x_{0}\in\mathbb{R}$ is a function $f:\mathbb{R}\rightarrow\mathbb{R}$
        such that for all $\varepsilon>0$ there exists a $\delta>0$ such that
        for all $x\in\mathbb{R}$ with $|x-x_{0}|<\delta$ it is true that
        $|f(x)-f(x_{0})|<\varepsilon$.
    \end{definition}
    Let's prove linear functions $f(x)=ax+b$ are continuous ($a\ne{0}$).
    \begin{equation}
        \textbf{Want:}\quad
        |x-x_{0}|<\delta\Rightarrow
        |f(x)-f(x_{0})|<\varepsilon
    \end{equation}
    Plugging in $f$:
    \begin{equation}
        \textbf{Want:}\quad
        |x-x_{0}|<\delta
        \Rightarrow|(ax+b)-(ax_{0}+b)|<\varepsilon
    \end{equation}
    We can simplify this:
    \begin{equation}
        \textbf{Want:}\quad
        |x-x_{0}|<\delta
        \Rightarrow
        |ax+b-ax_{0}-b|<\varepsilon
    \end{equation}
    Cancel the $b$ to get:
    \begin{equation}
        \textbf{Want:}\quad
        |x-x_{0}|<\delta
        \Rightarrow
        |ax-ax_{0}|<\varepsilon
    \end{equation}
    Factor the $a$:
    \begin{equation}
        \textbf{Want:}\quad
        |x-x_{0}|<\delta
        \Rightarrow
        |a|\cdot|x-x_{0}|<\varepsilon
    \end{equation}
    Since we only care about $|x-x_{0}|<\delta$, we have
    $|a||x-x_{0}|<|a|\delta$. We update our wish-list one last time:
    \begin{equation}
        \textbf{Want:}\quad
        |x-x_{0}|<\delta
        \Rightarrow|a|\delta\leq\varepsilon
    \end{equation}
    If we made $|a|\delta\leq\varepsilon$, we would then have the following:
    \begin{align}
        |x-x_{0}|<\delta&\Rightarrow|a|\delta\leq\varepsilon\\
        &\Rightarrow|a||x-x_{0}|<\varepsilon\\
        &\Rightarrow|ax-ax_{0}|<\varepsilon\\
        &\Rightarrow|ax+b-ax_{0}-b|<\varepsilon\\
        &\Rightarrow|(ax+b)-(ax_{0}+b)|<\varepsilon
    \end{align}
    And we'd be done. All we need to do is choose a $\delta$ such that
    $|a|\cdot\delta\leq\varepsilon$. Since $a\ne{0}$, we can simply choose
    $\delta=\varepsilon/|a|$. Note, any \textit{smaller} positive value would
    work just fine. If you wanted to choose
    $\delta=\varepsilon/(2|a|)$, you could still prove
    $|x-x_{0}|<\delta$ implies $|f(x)-f(x_{0})|<\varepsilon$. There is no
    \textit{best} choice of $\delta$. The goal is just find some positive
    number $\delta>0$ that works.
    \par\hfill\par
    All of this was work to find a candidate $\delta$. Now that we've found
    such a candidate, let's show that it works. Let $\varepsilon>0$. Choose
    $\delta=\varepsilon/|a|$. If $|x-x_{0}|<\delta$, then:
    \begin{align}
        &|x-x_{0}|<\delta\tag{Hypothesis}\\
        \Rightarrow&|x-x_{0}|<\frac{\varepsilon}{|a|}
            \tag{Definition of $\delta$}\\
        \Rightarrow&|a|\cdot|x-x_{0}|<\varepsilon
            \tag{Multiply both sides by $|a|$}\\
        \Rightarrow&|ax-ax_{0}|<\varepsilon
            \tag{Distribute the $|a|$ using absolute value laws}\\
        \Rightarrow&|ax-ax_{0}+0|<\varepsilon
            \tag{Adding zero doesn't change anything}\\
        \Rightarrow&|ax-ax_{0}+(b-b)|<\varepsilon
            \tag{Since $b-b=0$}\\
        \Rightarrow&|ax+b-ax_{0}-b|<\varepsilon
            \tag{Arithmetic}\\
        \Rightarrow&|(ax+b)-(ax_{0}+b)|<\varepsilon
            \tag{Factor the minus sign}\\
        \Rightarrow&|f(x)-f(x_{0})|<\varepsilon
            \tag{Definition of $f$}
    \end{align}
    That is, given any $\varepsilon>0$, choose
    $\delta=\varepsilon/|a|$. With this, if $|x-x_{0}|<\delta$, then
    $|f(x)-f(x_{0})|<\varepsilon$, which is what we wanted to prove.
    \newpage
    I, the copyright holder of this work, release it into the public domain.
    This applies worldwide. In some countries this may not be legally possible;
    if so: I grant anyone the right to use this work for any purpose, without
    any conditions, unless such conditions are required by law.
    \par\hfill\par
    The source code used to generate this document is free software and released
    under version 3 of the GNU General Public License.
\end{document}
