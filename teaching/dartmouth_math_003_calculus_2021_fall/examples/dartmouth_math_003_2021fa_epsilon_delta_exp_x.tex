%-----------------------------------LICENSE------------------------------------%
%   This file is part of Mathematics-and-Physics.                              %
%                                                                              %
%   Mathematics-and-Physics is free software: you can redistribute it and/or   %
%   modify it under the terms of the GNU General Public License as             %
%   published by the Free Software Foundation, either version 3 of the         %
%   License, or (at your option) any later version.                            %
%                                                                              %
%   Mathematics-and-Physics is distributed in the hope that it will be useful, %
%   but WITHOUT ANY WARRANTY; without even the implied warranty of             %
%   MERCHANTABILITY or FITNESS FOR A PARTICULAR PURPOSE.  See the              %
%   GNU General Public License for more details.                               %
%                                                                              %
%   You should have received a copy of the GNU General Public License along    %
%   with Mathematics-and-Physics.  If not, see <https://www.gnu.org/licenses/>.%
%------------------------------------------------------------------------------%
\documentclass{article}
\usepackage{amsmath} % Needed for align.
\usepackage{amssymb} % Needed for mathbb.
\usepackage{amsthm}  % For the theorem environment.

\newtheoremstyle{normal}{\topsep}{\topsep}{}{}{\bfseries}{}{.5em}{}
\theoremstyle{normal}
\newtheorem{definition}{Definition}

\title{$\varepsilon-\delta$ Continuity - Exponential Function}
\author{Ryan Maguire}
\date{\today}

% No indent and no paragraph skip.
\setlength{\parindent}{0em}
\setlength{\parskip}{0em}

\begin{document}
    \maketitle
    The definition of continuity is as follows:
    \begin{definition}
        A real-valued function that is continuous at a point
        $x_{0}\in\mathbb{R}$ is a function $f:\mathbb{R}\rightarrow\mathbb{R}$
        such that for all $\varepsilon>0$ there exists a $\delta>0$ such that
        for all $x\in\mathbb{R}$ with $|x-x_{0}|<\delta$ it is true that
        $|f(x)-f(x_{0})|<\varepsilon$.
    \end{definition}
    Let's prove $f(x)=\exp(x)$ is continuous. Alternatively, $f(x)=e^{x}$.
    \begin{equation}
        \textbf{Want:}\quad
        |x-x_{0}|<\delta
        \Rightarrow|f(x)-f(x_{0})|<\varepsilon
    \end{equation}
    Substituting $f$:
    \begin{equation}
        \textbf{Want:}\quad
        |x-x_{0}|<\delta
        \Rightarrow
        |\exp(x)-\exp(x_{0})|<\varepsilon
    \end{equation}
    One of the fundamental identities of the exponential function is:
    \begin{equation}
        \exp(a)\exp(b)=\exp(a+b)
    \end{equation}
    Alternatively, $e^{a}e^{b}=e^{a+b}$. Let's use this:
    \begin{align}
        \exp(x)-\exp(x_{0})
            &=\big(\exp(x)-\exp(x_{0})\big)\frac{\exp(-x_{0})}{\exp(-x_{0})}\\
            &=\frac{\exp(x)\exp(-x_{0})-\exp(x_{0})\exp(-x_{0})}{\exp(-x_{0})}\\
            &=\frac{\exp(x-x_{0})-\exp(x_{0}-x_{0})}{\exp(-x_{0})}\\
            &=\frac{\exp(x-x_{0})-1}{\exp(-x_{0})}
    \end{align}
    Using the fact that $\exp(-x_{0})=1/\exp(x_{0})$, we get:
    \begin{equation}
        \exp(x)-\exp(x_{0})
        =\exp(x_{0})\big(\exp(x-x_{0})-1\big)
    \end{equation}
    We update our wish-list:
    \begin{equation}
        \textbf{Want:}\quad
        |x-x_{0}|<\delta
        \Rightarrow
        \exp(x_{0})\big|\exp(x-x_{0})-1\big|<\varepsilon
    \end{equation}
    Now we note that $x_{0}$ is a fixed-point, not a variable like $x$. We can
    divide to obtain:
    \begin{equation}
        \textbf{Want:}\quad
        |x-x_{0}|<\delta
        \Rightarrow
        \big|\exp(x-x_{0})-1\big|<\frac{\varepsilon}{\exp(x_{0})}
    \end{equation}
    The inequality $|a|<b$ is equivalent to $-b<a<b$. Using this, we rewrite
    the previous inequality as follows:
    \begin{equation}
        \textbf{Want:}\quad
        |x-x_{0}|<\delta
        \Rightarrow
        -\frac{\varepsilon}{\exp(x_{0})}
            <\exp(x-x_{0})-1
            <\frac{\varepsilon}{\exp(x_{0})}
    \end{equation}
    Simplifying by adding $1$ to each side, we have:
    \begin{equation}
        \textbf{Want:}\quad
        |x-x_{0}|<\delta
        \Rightarrow
        1-\frac{\varepsilon}{\exp(x_{0})}
            <\exp(x-x_{0})
            <1+\frac{\varepsilon}{\exp(x_{0})}
    \end{equation}
    If $\varepsilon\geq\exp(x_{0})$, we could restrict our attention to making
    $|x-x_{0}|<\delta\Rightarrow{0}<\exp(x-x_{0})<2$, since this would then
    imply the previous inequality. This could by done by choosing
    $\delta=\frac{1}{2}$ since $0<\exp(\frac{1}{2})<2$. So we now consider the
    case where $\varepsilon<\exp(x_{0})$. In this case both
    $1-\varepsilon/\exp(x_{0})$ and $1+\varepsilon/\exp(x_{0})$ are positive
    numbers, so we may take their natural logs. The natural logarithm is
    also a strictly increasing function, so if $a<b<c$ is true, then
    $\ln(a)<\ln(b)<\ln(c)$ is also true. Using this, we update our wishlist:
    \begin{equation}
        \textbf{Want:}\quad
        |x-x_{0}|<\delta
        \Rightarrow
        \ln\big(1-\frac{\varepsilon}{\exp(x_{0})}\big)
            <x-x_{0}
            <\ln\big(1+\frac{\varepsilon}{\exp(x_{0})}\big)
    \end{equation}
    This hints at making the following definition:
    \begin{equation}
        \delta=\textrm{min}\Big(
            \big|\ln\big(1-\frac{\varepsilon}{\exp(x_{0})}\big)\big|,\,
            \ln\big(1+\frac{\varepsilon}{\exp(x_{0})}\big)
        \Big)
    \end{equation}
    Now let's show that this works. Let $\varepsilon>0$ and $x_{0}\in\mathbb{R}$
    be given. If $\varepsilon\geq\exp(x_{0})$, choose $\delta=\frac{1}{2}$.
    We then have:
    \begin{align}
        |x-x_{0}|&<\delta
        \tag{Hypothesis}\\
        \Rightarrow
        |x-x_{0}|&<\frac{1}{2}
        \tag{Definition of $\delta$}\\
        \Rightarrow
        -\frac{1}{2}<x-x_{0}&<\frac{1}{2}
        \tag{Definition of absolute value}\\
        \Rightarrow
        \exp\big(-\frac{1}{2}\big)
        <\exp(x-x_{0})
        &<\exp\big(\frac{1}{2}\big)
        \tag{Since $\exp$ is increasing}\\
        \Rightarrow
        0<\exp(x-x_{0})&<\exp\big(\frac{1}{2}\big)
        \tag{Since $\exp$ is positive}\\
        \Rightarrow
        0<\exp(x-x_{0})&<2
        \tag{Since $\exp(1/2)<2$}\\
        \Rightarrow
        -1<\exp(x-x_{0})-1&<1
        \tag{Subtract 1 from all sides}\\
        \Rightarrow
        |\exp(x-x_{0})-1|&<1
        \tag{Definition of absolute value}\\
        \Rightarrow
        |\exp(x)\exp(-x_{0})-1|&<1
        \tag{Exponential rules}\\
        \Rightarrow
        \Big|(\exp(x)\exp(-x_{0})-1)\frac{\exp(x_{0})}{\exp(x_{0})}\Big|&<1
        \tag{Multiply by 1}\\
        \Rightarrow
        \Big|\frac{\exp(x)-\exp(x_{0})}{\exp(x_{0})}\Big|&<1
        \tag{Simplify}\\
        \Rightarrow
        |\exp(x)-\exp(x_{0})|&<\exp(x_{0})
        \tag{Multiply by $\exp(x_{0})$}\\
        \Rightarrow
        |\exp(x)-\exp(x_{0})|&<\varepsilon
        \tag{Since $\exp(x_{0})<\varepsilon$}\\
        \Rightarrow
        |f(x)-f(x_{0})|<\varepsilon
        \tag{Definition of $f$}
    \end{align}
    Before handling the case of $\varepsilon<\exp(x_{0})$ let's get some
    intuition. If $x_{0}$ is very big, then $\exp(x_{0})$ is huge.
    If $\exp(x_{0})\leq\varepsilon$ is true, then $\varepsilon$ must be a
    very large number as well. That is, the \textit{error} we're allowed is
    very big. Because of this it is very easy to choose a value for $\delta$,
    we simply set $\delta=\frac{1}{2}$.
    Similarly, if $x_{0}$ is a large negative
    number, then $\exp(x_{0})$ is roughly zero. Varying $x_{0}$ doesn't change
    this much since the exponential function gets very flat as we go to the
    left on the real axis. This flatness allows us to pick a simply candidate
    for $\delta$, namely $\delta=\frac{1}{2}$.
    \par\hfill\par
    Now we handle the case where $\varepsilon<\exp(x_{0})$. Choose:
    \begin{equation}
        \delta=\textrm{min}\Big(
            \big|\ln\big(1-\frac{\varepsilon}{\exp(x_{0})}\big)\big|,\,
            \ln\big(1+\frac{\varepsilon}{\exp(x_{0})}\big)
        \Big)
    \end{equation}
    This is well defined since $\varepsilon<\exp(x_{0})$, and hence
    $\frac{\varepsilon}{\exp(x_{0})}<1$. Then we have:
    \begin{align}
        |x-x_{0}|&<\delta
        \tag{Hypothesis}\\
        \Rightarrow
        -\delta<x-x_{0}&<\delta
        \tag{Definition of absolute value}\\
        \Rightarrow
        \ln\big(1-\frac{\varepsilon}{\exp(x_{0})}\big)
        <x-x_{0}
        &<\ln\big(1+\frac{\varepsilon}{\exp(x_{0})}\big)
        \tag{Definition of $\delta$}\\
        \Rightarrow
        1-\frac{\varepsilon}{\exp(x_{0})}
        <\exp(x-x_{0})
        &<1+\frac{\varepsilon}{\exp(x_{0})}
        \tag{Since $\exp$ is increasing}\\
        \Rightarrow
        -\frac{\varepsilon}{\exp(x_{0})}
        <\exp(x-x_{0})-1
        &<\frac{\varepsilon}{\exp(x_{0})}
        \tag{Subtract 1 from both sides}\\
        \Rightarrow
        |\exp(x-x_{0})-1|&<\frac{\varepsilon}{\exp(x_{0})}
        \tag{Definition of absolute value}\\
        \Rightarrow
        |\exp(x)\exp(-x_{0})-1|&<\frac{\varepsilon}{\exp(x_{0})}
        \tag{Exponential rules}\\
        \Rightarrow
        |\exp(x)-\exp(x_{0})|&<\varepsilon
        \tag{Multiply by $\exp(x_{0})$}\\
        \Rightarrow
        |f(x)-f(x_{0})|<\varepsilon
        \tag{Definition of $f$}
    \end{align}
    Which is what we wanted to prove.
    \newpage
    I, the copyright holder of this work, release it into the public domain.
    This applies worldwide. In some countries this may not be legally possible;
    if so: I grant anyone the right to use this work for any purpose, without
    any conditions, unless such conditions are required by law.
    \par\hfill\par
    The source code used to generate this document is free software and released
    under version 3 of the GNU General Public License.
\end{document}
