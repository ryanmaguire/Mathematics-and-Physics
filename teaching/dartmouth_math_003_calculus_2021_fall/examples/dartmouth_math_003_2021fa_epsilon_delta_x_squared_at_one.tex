%-----------------------------------LICENSE------------------------------------%
%   This file is part of Mathematics-and-Physics.                              %
%                                                                              %
%   Mathematics-and-Physics is free software: you can redistribute it and/or   %
%   modify it under the terms of the GNU General Public License as             %
%   published by the Free Software Foundation, either version 3 of the         %
%   License, or (at your option) any later version.                            %
%                                                                              %
%   Mathematics-and-Physics is distributed in the hope that it will be useful, %
%   but WITHOUT ANY WARRANTY; without even the implied warranty of             %
%   MERCHANTABILITY or FITNESS FOR A PARTICULAR PURPOSE.  See the              %
%   GNU General Public License for more details.                               %
%                                                                              %
%   You should have received a copy of the GNU General Public License along    %
%   with Mathematics-and-Physics.  If not, see <https://www.gnu.org/licenses/>.%
%------------------------------------------------------------------------------%
\documentclass{article}
\usepackage{amsmath}                            % Needed for align.
\usepackage{amssymb}                            % Needed for mathbb.
\usepackage{amsthm}                             % For the theorem environment.

\newtheoremstyle{normal}
    {\topsep}               % Amount of space above the theorem.
    {\topsep}               % Amount of space below the theorem.
    {}                      % Font used for body of theorem.
    {}                      % Measure of space to indent.
    {\bfseries}             % Font of the header of the theorem.
    {}                      % Punctuation between head and body.
    {.5em}                  % Space after theorem head.
    {}

\theoremstyle{normal}
\newtheorem{definition}{Definition}

\title{$\varepsilon-\delta$ Continuity - Continuity at a Point}
\author{Ryan Maguire}
\date{\today}

% No indent and no paragraph skip.
\setlength{\parindent}{0em}
\setlength{\parskip}{0em}

\begin{document}
    \maketitle
    The definition of continuity is as follows:
    \begin{definition}
        A real-valued function that is continuous at a point
        $x_{0}\in\mathbb{R}$ is a function $f:\mathbb{R}\rightarrow\mathbb{R}$
        such that for all $\varepsilon>0$ there exists a $\delta>0$ such that
        for all $x\in\mathbb{R}$ with $|x-x_{0}|<\delta$ it is true that
        $|f(x)-f(x_{0})|<\varepsilon$.
    \end{definition}
    Let's prove that $f(x)=x^{2}$ is continuous at $x_{0}=1$. As always, we
    have the following wish-list:
    \begin{equation}
        \textbf{Want:}\quad|x-1|<\delta
        \Rightarrow|f(x)-f(1)|<\varepsilon
    \end{equation}
    Since we have an expression for $f$, we may as well substitute that in:
    \begin{equation}
        \textbf{Want:}\quad|x-1|<\delta
        \Rightarrow|x^{2}-1|<\varepsilon
    \end{equation}
    Let's now search for a candidate for $\delta$. Remember, $\varepsilon$ is
    given to us. If we factor the expression $|x^{2}-1|$ we get
    $|(x-1)(x+1)|$, and using the product rule for the absolute value function
    we can simplify this to $|x-1||x+1|$. After this step we see that $\delta$
    can now appear in the expression for $\varepsilon$. Again, we have the
    following wish-list:
    \begin{equation}
        \textbf{Want:}\quad|x-1|<\delta
        \Rightarrow|x-1||x+1|<\varepsilon
    \end{equation}
    Since we're only going to look at values of $x$ satisfying
    $|x-1|<\delta$, we get the following inequality:
    \begin{equation}
        |x-1||x+1|<\delta|x+1|
    \end{equation}
    If we can somehow make this new expression, $\delta|x+1|$, bounded by
    $\varepsilon$, then we'd be done! That is, we'd
    have the following string of inequalities:
    \begin{equation}
        |x^{2}-1|=|x-1||x+1|<\delta|x+1|\leq\varepsilon
    \end{equation}
    And from that we can conclude $|x^{2}-1|<\varepsilon$. So how do we make
    $\delta|x+1|\leq\varepsilon$ a valid inequality? The expression
    $|x+1|$ gets arbitrarily large as $x$ gets bigger and bigger, and so
    $\delta|x+1|\leq\varepsilon$ is \textbf{not} true for all $x$. So what to
    do? Well, we don't care about all $x$, we only care about $x$ values that
    are close to the point of interest, $x_{0}=1$ in this example. So, let's
    look no further than 1 away from this point to begin with. That is, we are
    restricting ourselves to $\delta\leq{1}$. With this newly imposed
    restriction, we have:
    \begin{equation}
        |x-1|<\delta\Rightarrow-\delta<x-1<\delta
        \Rightarrow-1<x-1<1\Rightarrow0<x<2
    \end{equation}
    Intuitively, if $x$ has a distance of less than 1 from the point
    $x_{0}=1$ on the number line, than $x$ must be between 0 and 2. Since
    $x$ is between 0 and 2, the expression $|x+1|$ is never larger than 3
    (the largest it will be is when $x=2$). In other words, since we
    introduced this new restriction that $\delta\leq{1}$, we can now write the
    following inequality:
    \begin{equation}
        \delta|x+1|<3\delta
    \end{equation}
    If we can make $3\delta\leq{\varepsilon}$, we'd be done! It is tempting to
    write \textit{choose} $\delta=\varepsilon/3$, but hold on! We already
    imposed $\delta\leq{1}$. What if $\varepsilon/3$ is greater than this
    value? So we must choose the smaller of these two.
    Choose $\delta=\min(1,\varepsilon/3)$.
    \par\hfill\par
    All of this was a search for a candidate $\delta$. Now that we have such
    a candidate, let's show that it works. Let $\varepsilon>0$. Choose
    $\delta=\min(1,\varepsilon/3)$. If $|x-1|<\delta$, then:
    \begin{align}
        |x-1|&<\min\Big(1,\,\frac{\varepsilon}{3}\Big)
            \tag{Definition of $\delta$}\\
        \Rightarrow
            |x-1|&<\frac{\varepsilon}{3}
                \tag{Definition of $\min$}\\
        \Rightarrow
            3|x-1|&<\varepsilon
                \tag{Multiplication by a Positive Number}
    \end{align}
    But since $|x-1|<\min(1,\varepsilon/3)$, we have $|x-1|<1$, and hence
    $|x+1|<3$. But then:
    \begin{align}
        |x+1||x-1|&<3|x-1|
            \tag{Since $|x+1|<3$}\\
        \Rightarrow
            |x+1||x-1|&<\varepsilon
                \tag{Since $3|x-1|<\varepsilon$}\\
        \Rightarrow
            |x^{2}-1|&<\varepsilon
                \tag{Simplify the Expression}\\
        \Rightarrow
            |f(x)-f(1)|<\varepsilon
                \tag{Definition of $f$}
    \end{align}
    And hence $f(x)=x^{2}$ is continuous at $x_{0}=1$.
    \newpage
    I, the copyright holder of this work, release it into the public domain.
    This applies worldwide. In some countries this may not be legally possible;
    if so: I grant anyone the right to use this work for any purpose, without
    any conditions, unless such conditions are required by law.
    \par\hfill\par
    The source code used to generate this document is free software and released
    under version 3 of the GNU General Public License.
\end{document}
