%-----------------------------------LICENSE------------------------------------%
%   This file is part of Mathematics-and-Physics.                              %
%                                                                              %
%   Mathematics-and-Physics is free software: you can redistribute it and/or   %
%   modify it under the terms of the GNU General Public License as             %
%   published by the Free Software Foundation, either version 3 of the         %
%   License, or (at your option) any later version.                            %
%                                                                              %
%   Mathematics-and-Physics is distributed in the hope that it will be useful, %
%   but WITHOUT ANY WARRANTY; without even the implied warranty of             %
%   MERCHANTABILITY or FITNESS FOR A PARTICULAR PURPOSE.  See the              %
%   GNU General Public License for more details.                               %
%                                                                              %
%   You should have received a copy of the GNU General Public License along    %
%   with Mathematics-and-Physics.  If not, see <https://www.gnu.org/licenses/>.%
%------------------------------------------------------------------------------%
\documentclass{article}
\usepackage{amsmath}                            % Needed for align.
\usepackage{graphicx}                           % Needed for figures.
\usepackage{hyperref}                           % Hyperlinks for figures.
\hypersetup{colorlinks=true, linkcolor=blue}    % Colors for hyperref.
\graphicspath{{../../../images/}}

\title{Inverse Functions - Example 2}
\author{Ryan Maguire}
\date{\today}

% No indent and no paragraph skip.
\setlength{\parindent}{0em}
\setlength{\parskip}{0em}

\begin{document}
    \maketitle
    Consider the following expression:
    \begin{equation}
        f(x)=\frac{1}{\ln(x)}
    \end{equation}
    The largest possible domain is $(0,1)\cup(1,\infty)$ and the range of the
    function is $(-\infty,0)\cup(0,\infty)$. To see this, note that
    $\ln(x)$ has the range of $(-\infty,\infty)$, so $1/\ln(x)$ will only be
    missing zero in it's range. That is, since for every non-zero number $y$
    there is a non-zero $x$ such that $y=1/x$, the range of $1/\ln(x)$ will be
    every non-zero real number. With respect to the domain
    $(0,1)\cup(1,\infty)$ and the range $(-\infty,0)\cup(0,\infty)$, $f$ is
    \textit{bijective} meaning $f(x_0)=f(x_{1})$ implies $x_{0}=x_{1}$,
    and for any value $y\in(-\infty,0)\cup(0,\infty)$ there is a value $x$
    such that $y=f(x)$. We won't prove this, but it can be verified from the
    graph of $1/\ln(x)$ in Fig.~\ref{fig:one_by_ln_x}. This means we can define
    the inverse of $f$. Let's try to compute what it is. To do this, we set up
    the equation $y=f(x)$ and apply various operations to both sides until
    we've isolated $x$ in the form $x=g(y)$. This function $g$ is the inverse
    of $f$. Let's try this.
    \begin{align}
        y&=f(x)\\
        y&=\frac{1}{\ln(x)}\\
        \frac{1}{y}&=\ln(x)\\
        \exp\big(\frac{1}{y}\big)&=x
    \end{align}
    So, the inverse function is:
    \begin{equation}
        f^{-1}(x)=\exp\big(\frac{1}{x}\big)
    \end{equation}
    The domain is $(-\infty,0)\cup(0,\infty)$, which is precisely the range of
    $f$. This is not a coincidence. Like all inverse functions, the plot can
    be made by reflecting $f$ across the line $y=x$. This is done in
    Fig.~\ref{fig:one_by_ln_x_and_inverse}.
    \begin{figure}
        \centering
        \includegraphics{one_by_ln_x.pdf}
        \caption{The function $f(x)=1/\ln(x)$}
        \label{fig:one_by_ln_x}
    \end{figure}
    \begin{figure}
        \centering
        \includegraphics{one_by_ln_x_and_exp_one_by_x.pdf}
        \caption{The function $f(x)=1/\ln(x)$ and it's inverse}
        \label{fig:one_by_ln_x_and_inverse}
    \end{figure}
    \newpage
    I, the copyright holder of this work, release it into the public domain.
    This applies worldwide. In some countries this may not be legally possible;
    if so: I grant anyone the right to use this work for any purpose, without
    any conditions, unless such conditions are required by law.
    \par\hfill\par
    The source code used to generate this document is free software and released
    under version 3 of the GNU General Public License.
\end{document}
