%-----------------------------------LICENSE------------------------------------%
%   This file is part of Mathematics-and-Physics.                              %
%                                                                              %
%   Mathematics-and-Physics is free software: you can redistribute it and/or   %
%   modify it under the terms of the GNU General Public License as             %
%   published by the Free Software Foundation, either version 3 of the         %
%   License, or (at your option) any later version.                            %
%                                                                              %
%   Mathematics-and-Physics is distributed in the hope that it will be useful, %
%   but WITHOUT ANY WARRANTY; without even the implied warranty of             %
%   MERCHANTABILITY or FITNESS FOR A PARTICULAR PURPOSE.  See the              %
%   GNU General Public License for more details.                               %
%                                                                              %
%   You should have received a copy of the GNU General Public License along    %
%   with Mathematics-and-Physics.  If not, see <https://www.gnu.org/licenses/>.%
%------------------------------------------------------------------------------%
\documentclass{article}
\usepackage{amsmath} % Needed for align.
\usepackage{amssymb} % Needed for mathbb.
\usepackage{amsthm}  % For the theorem environment.

\newtheoremstyle{normal}{\topsep}{\topsep}{}{}{\bfseries}{}{.5em}{}
\theoremstyle{normal}
\newtheorem{definition}{Definition}

\title{$\varepsilon-\delta$ Continuity - Quadratics}
\author{Ryan Maguire}
\date{\today}

% No indent and no paragraph skip.
\setlength{\parindent}{0em}
\setlength{\parskip}{0em}

\begin{document}
    \maketitle
    The definition of continuity is as follows:
    \begin{definition}
        A real-valued function that is continuous at a point
        $x_{0}\in\mathbb{R}$ is a function $f:\mathbb{R}\rightarrow\mathbb{R}$
        such that for all $\varepsilon>0$ there exists a $\delta>0$ such that
        for all $x\in\mathbb{R}$ with $|x-x_{0}|<\delta$ it is true that
        $|f(x)-f(x_{0})|<\varepsilon$.
    \end{definition}
    Let's prove $f(x)=x^{2}$ is continuous for all real numbers. Let's first
    work out $x_{0}=0$ so we can avoid any annoying divisions by zero that may
    occur in our work. For $x_{0}=0$, we have:
    \begin{equation}
        \textbf{Want:}\quad
        |x-0|<\delta
        \Rightarrow|x^{2}-0|<\varepsilon
    \end{equation}
    Simplifying, we want $|x|<\delta$ implies $|x|^{2}<\varepsilon$. If we
    choose $\delta=\sqrt{\varepsilon}$, we'd be done since
    $|x|<\delta$ implies $|x|<\sqrt{\varepsilon}$, which then implies
    $|x|^{2}<\varepsilon$.
    \par\hfill\par
    Now let's prove $f(x)=x^{2}$ is continuous at any non-zero value. We have
    the following:
    \begin{equation}
        \textbf{Want:}\quad
        |x-x_{0}|<\delta
        \Rightarrow|x^{2}-x_{0}^{2}|<\varepsilon
    \end{equation}
    Factoring this expression on the right, we can rephrase this:
    \begin{equation}
        \textbf{Want:}\quad
        |x-x_{0}|<\delta
        \Rightarrow
        |x-x_{0}||x+x_{0}|<\varepsilon
    \end{equation}
    Since we will only look at $x$ values satisfying $|x-x_{0}|<\delta$,
    we have:
    \begin{equation}
        |x-x_{0}||x+x_{0}|<\delta|x+x_{0}|
    \end{equation}
    If we can make $\delta|x+x_{0}|\leq\varepsilon$ we'd be done, since:
    \begin{equation}
        |x^{2}-x_{0}^{2}|=|x-x_{0}||x+x_{0}|
        <\delta|x+x_{0}|\leq\varepsilon
    \end{equation}
    So, how do we make $\delta|x+x_{0}|\leq\varepsilon$ a reality? The
    expression $\delta|x+x_{0}|\leq\varepsilon$ is \textbf{not} always true
    since $|x+x_{0}|$ gets arbitrarily large as $x$ gets bigger. But we don't
    care about larger and larger values of $x$, we only care about values of
    $x$ that are close to $x_{0}$. So let's suppose we only look at values that
    are no more than $|x_{0}|/2$ away from $x_{0}$. That is, we are introducing
    the restriction that $\delta\leq|x_{0}|/2$. If this were true, the largest
    $|x+x_{0}|$ could be is $5|x_{0}|/2$. To see this, we invoke the
    \textit{triangle inequality} that says for any real numbers $a$ and $b$,
    the following is true:
    \begin{equation}
        |a+b|\leq|a|+|b|
    \end{equation}
    Applying this to $a=x_{0}$ and $b=x_{0}\pm{x}_{0}/2$, we'd get:
    \begin{equation}
        |x+x_{0}|\leq|x|+|x_{0}|\leq|3x_{0}/2|+|x_{0}|=5|x_{0}|/2
    \end{equation}
    If we choose $\delta$ such that $5|x_{0}|\delta/2\leq\varepsilon$ we'd be
    done. It is tempting to write \textit{choose}
    $\delta=2\varepsilon/5|x_{0}|$, but remember we've already imposed the
    restriction that $\delta\leq|x_{0}|/2$. We must choose the smaller of these
    two.
    \par\hfill\par
    We now have a candidate for $\delta$. Let's show that it works. Let
    $\varepsilon>0$. Choose $\delta=\min(|x_{0}|/2,2\varepsilon/5|x_{0}|)$. If
    $|x-x_{0}|<\delta$, then:
    \begin{align}
        |x-x_{0}|&<\min\Big(
            \frac{|x_{0}|}{2},\,
            \frac{2\varepsilon}{5|x_{0}|}
        \Big)
            \tag{Definition of $\delta$}\\
        \Rightarrow
            |x-x_{0}|&<\frac{2\varepsilon}{5|x_{0}|}
                \tag{Definition of $\min$}\\
        \Rightarrow
            \frac{5|x_{0}|}{2}|x-x_{0}|&<\varepsilon
                \tag{Multiplication by a Positive Number}
    \end{align}
    But since $|x-x_{0}|<\min(|x_{0}|/2,2\varepsilon/5|x_{0}|)$, we have
    $|x-x_{0}|<|x_{0}|/2$ by the definition of $\min$. But then:
    \begin{align}
        |x+x_{0}|&<\frac{5|x_{0}|}{2}
            \tag{Since $|x-x_{0}|<\frac{|x_{0}|}{2}$}\\
        \Rightarrow
            |x+x_{0}||x-x_{0}|&<\frac{5|x_{0}|}{2}|x-x_{0}|
                \tag{Multiplication by a Positive Number}\\
        \Rightarrow
            |x+x_{0}||x-x_{0}|&<\varepsilon
                \tag{Since $\frac{5|x_{0}|}{2}|x-x_{0}|<\varepsilon$}\\
        \Rightarrow
            |x^{2}-x_{0}^{2}|&<\varepsilon
                \tag{Simplify the Expression}\\
        \Rightarrow
            |f(x)-f(x_{0})|&<\varepsilon
                \tag{Definition of $f$}
    \end{align}
    And hence $f$ is continuous at $x_{0}$.
    \newpage
    I, the copyright holder of this work, release it into the public domain.
    This applies worldwide. In some countries this may not be legally possible;
    if so: I grant anyone the right to use this work for any purpose, without
    any conditions, unless such conditions are required by law.
    \par\hfill\par
    The source code used to generate this document is free software and released
    under version 3 of the GNU General Public License.
\end{document}
