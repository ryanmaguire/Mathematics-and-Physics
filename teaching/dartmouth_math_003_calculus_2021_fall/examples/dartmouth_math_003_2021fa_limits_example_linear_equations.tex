%-----------------------------------LICENSE------------------------------------%
%   This file is part of Mathematics-and-Physics.                              %
%                                                                              %
%   Mathematics-and-Physics is free software: you can redistribute it and/or   %
%   modify it under the terms of the GNU General Public License as             %
%   published by the Free Software Foundation, either version 3 of the         %
%   License, or (at your option) any later version.                            %
%                                                                              %
%   Mathematics-and-Physics is distributed in the hope that it will be useful, %
%   but WITHOUT ANY WARRANTY; without even the implied warranty of             %
%   MERCHANTABILITY or FITNESS FOR A PARTICULAR PURPOSE.  See the              %
%   GNU General Public License for more details.                               %
%                                                                              %
%   You should have received a copy of the GNU General Public License along    %
%   with Mathematics-and-Physics.  If not, see <https://www.gnu.org/licenses/>.%
%------------------------------------------------------------------------------%
\documentclass{article}
\usepackage{amsmath}                            % Needed for align.
\usepackage{graphicx}                           % Needed for figures.
\usepackage{hyperref}                           % Hyperlinks for figures.
\hypersetup{colorlinks=true, linkcolor=blue}    % Colors for hyperref.
\graphicspath{{../../../images/}}

\title{Limits - Linear Equations}
\author{Ryan Maguire}
\date{\today}

% No indent and no paragraph skip.
\setlength{\parindent}{0em}
\setlength{\parskip}{0em}

\begin{document}
    \maketitle
    One of the basic laws of limits is the following identity:
    \begin{equation}
        \lim_{x\rightarrow{x_{0}}}x=x_{0}
    \end{equation}
    Let's think about what this says. As $x$ approaches $x_{0}$, we have that
    $x$ approaches $x_{0}$. This is a tautology. It should be intuitively clear
    from the graph of the function $f(x)=x$. As we pick points that get
    closer and closer to $x_{0}$, the value of $f$ also gets closer and closer
    to $x_{0}$. This is shown in Fig.~\ref{fig:lim_f_of_x_equals_x}.
    \par\hfill\par
    The next limit law that's rather easy to believe is that the limit of a
    constant is the constant. That is, if $c$ is a real number, then:
    \begin{equation}
        \lim_{x\rightarrow{x_{0}}}c=c
    \end{equation}
    The function $f(x)=c$ never changes, so as $x$ varies around $x_{0}$,
    the value of $f(x)$ is still $c$. Graphing a constant function should
    convince us that this statement is true. This is done in
    Fig.~\ref{fig:limit_of_constant}.
    \par\hfill\par
    Next, if $f$ and $g$ are two functions, if $f(x)$ tends to $a$ as
    $x$ approaches $x_{0}$, and if $g(x)$ tends to $b$ as $x$ approaches
    $x_{0}$, then the following is true:
    \begin{equation}
        \label{eqn:add_limit_law}%
        \lim_{x\rightarrow{x}_{0}}\big(f(x)+g(x)\big)
        =\lim_{x\rightarrow{x}_{0}}f(x)+
            \lim_{x\rightarrow{x}_{0}}g(x)
        =a+b
    \end{equation}
    There are a few ways to convince ourselves of this. Firstly, when we
    say $f(x)$ tends to $a$ we mean that $f(x)-a$ tends to zero. That is, we
    can make $f(x)-a$ arbitrary small as long as we choose values that are
    close enough to $x_{0}$. So if $f(x)-a$ and $g(x)-b$ are getting really
    close to zero as $x$ gets closer to $x_{0}$, let's look at what happens
    to $f(x)+g(x)$:
    \begin{align}
        f(x)+g(x)-(a+b)&=f(x)+g(x)-a-b\\
        &=\big(f(x)-a\big)+\big(g(x)-b\big)\\
        &\approx{0}+0\\
        &=0
    \end{align}
    So, for values close to $x_{0}$, $f(x)+g(x)$ is close to $a+b$. This is
    precisely the statement made by Eqn.~\ref{eqn:add_limit_law}. This is
    depicted in Fig.~\ref{fig:limit_of_sum}.
    \par\hfill\par
    Lastly, if $c$ is a constant, and if the limit of $f$ exists as $x$
    approaches $x_{0}$, then:
    \begin{equation}
        \lim_{x\rightarrow{x}_{0}}af(x)=a\lim_{x\rightarrow{x}_{0}}f(x)
    \end{equation}
    That is, we can factor out constants. A visual is shown in
    Fig.~\ref{fig:limit_of_constant_times_function}. Using these four
    properties we can prove the limit always exists for a linear equation.
    Letting $f(x)=ax+b$, we have:
    \begin{align}
        \lim_{x\rightarrow{x}_{0}}f(x)
        &=\lim_{x\rightarrow{x}_{0}}\big(ax+b\big)\\
        &=\Big(\lim_{x\rightarrow{x}_{0}}ax\Big)
            +\Big(\lim_{x\rightarrow{x}_{0}}b\Big)
            &\text{(Additive Law)}\\
        &=\Big(\lim_{x\rightarrow{x}_{0}}ax\Big)+b
            &\text{(Limit of a Constant)}\\
        &=a\Big(\lim_{x\rightarrow{x}_{0}}x\Big)+b
            &\text{(Factoring a Constant)}\\
        &=ax_{0}+b
            &\text{(Limit of $x$)}
    \end{align}
    So the limit of a linear function always exists and it is given by what
    we'd expect.
    \begin{figure}
        \centering
        \includegraphics{limit_of_x.pdf}
        \caption{The limit of $f(x)=x$ as $x$ approaches $x_{0}$}
        \label{fig:lim_f_of_x_equals_x}
    \end{figure}
    \begin{figure}
        \centering
        \includegraphics{limit_of_constant.pdf}
        \caption{The limit of $f(x)=c$ as $x$ approaches $x_{0}$}
        \label{fig:limit_of_constant}
    \end{figure}
    \begin{figure}
        \centering
        \includegraphics{limit_of_sum.pdf}
        \caption{The limit of $f(x)+g(x)$ as $x$ approaches $x_{0}$}
        \label{fig:limit_of_sum}
    \end{figure}
    \begin{figure}
        \centering
        \includegraphics{limit_of_constant_times_function.pdf}
        \caption{The limit of $c\cdot{f}(x)$ as $x$ approaches $x_{0}$}
        \label{fig:limit_of_constant_times_function}
    \end{figure}
    \newpage
    I, the copyright holder of this work, release it into the public domain.
    This applies worldwide. In some countries this may not be legally possible;
    if so: I grant anyone the right to use this work for any purpose, without
    any conditions, unless such conditions are required by law.
    \par\hfill\par
    The source code used to generate this document is free software and released
    under version 3 of the GNU General Public License.
\end{document}
