%-----------------------------------LICENSE------------------------------------%
%   This file is part of Mathematics-and-Physics.                              %
%                                                                              %
%   Mathematics-and-Physics is free software: you can redistribute it and/or   %
%   modify it under the terms of the GNU General Public License as             %
%   published by the Free Software Foundation, either version 3 of the         %
%   License, or (at your option) any later version.                            %
%                                                                              %
%   Mathematics-and-Physics is distributed in the hope that it will be useful, %
%   but WITHOUT ANY WARRANTY; without even the implied warranty of             %
%   MERCHANTABILITY or FITNESS FOR A PARTICULAR PURPOSE.  See the              %
%   GNU General Public License for more details.                               %
%                                                                              %
%   You should have received a copy of the GNU General Public License along    %
%   with Mathematics-and-Physics.  If not, see <https://www.gnu.org/licenses/>.%
%------------------------------------------------------------------------------%
\documentclass{article}
\usepackage{amsmath}                            % Needed for align.
\usepackage{amssymb}                            % Needed for mathbb.
\usepackage{amsthm}                             % For the theorem environment.

\newtheoremstyle{normal}
    {\topsep}               % Amount of space above the theorem.
    {\topsep}               % Amount of space below the theorem.
    {}                      % Font used for body of theorem.
    {}                      % Measure of space to indent.
    {\bfseries}             % Font of the header of the theorem.
    {}                      % Punctuation between head and body.
    {.5em}                  % Space after theorem head.
    {}

\theoremstyle{normal}
\newtheorem{definition}{Definition}

\title{$\varepsilon-\delta$ Continuity - $f(x)=\frac{1}{x}$, $x\in[0,\,\infty)$}
\author{Ryan Maguire}
\date{\today}

% No indent and no paragraph skip.
\setlength{\parindent}{0em}
\setlength{\parskip}{0em}

\begin{document}
    \maketitle
    The definition of continuity is as follows:
    \begin{definition}
        A real-valued function that is continuous at a point
        $x_{0}\in\mathbb{R}$ is a function $f:\mathbb{R}\rightarrow\mathbb{R}$
        such that for all $\varepsilon>0$ there exists a $\delta>0$ such that
        for all $x\in\mathbb{R}$ with $|x-x_{0}|<\delta$ it is true that
        $|f(x)-f(x_{0})|<\varepsilon$.
    \end{definition}
    Let's prove $f(x)=1/x$ is continuous on $x\in[1,\infty)$. We want:
    \begin{equation}
        \textbf{Want:}\quad
        |x-x_{0}|<\delta
        \Rightarrow|f(x)-f(x_{0})|<\varepsilon
    \end{equation}
    Substituting in the formula for $f$, we want:
    \begin{equation}
        \textbf{Want:}\quad
        |x-x_{0}|<\delta
        \Rightarrow
        \big|\frac{1}{x}-\frac{1}{x_{0}}\big|<\varepsilon
    \end{equation}
    If we simplify the latter expression, we want:
    \begin{equation}
        \textbf{Want:}\quad
        |x-x_{0}|<\delta
        \Rightarrow
        \big|\frac{x-x_{0}}{xx_{0}}\big|<\varepsilon
    \end{equation}
    And with this we've found a way to get $\delta$ in the expression. Let's
    simplify again:
    \begin{equation}
        \textbf{Want:}\quad
        |x-x_{0}|<\delta
        \Rightarrow
        \frac{1}{|xx_{0}|}|x-x_{0}|<\varepsilon
    \end{equation}
    Since we are only looking at $|x-x_{0}|<\delta$, we have:
    \begin{equation}
        \frac{1}{|xx_{0}|}|x-x_{0}|<\frac{1}{|xx_{0}|}\delta
    \end{equation}
    If we can make this last expression bounded by $\varepsilon$, we'd be done.
    That is, we update our wish-list to:
    \begin{equation}
        \textbf{Want:}\quad
        \frac{\delta}{|xx_{0}|}\leq\varepsilon
    \end{equation}
    Now we recall that we are only considering $x\in[1,\infty)$, and similarly
    for $x_{0}$. If $x$ and $x_{0}$ are confined to the domain $[1,\infty)$,
    then $1/|xx_{0}|$ is bounded by 1. That is, the largest $1/|xx_{0}|$ can
    be is when $x$ and $x_{0}$ are at their smallest. And since
    $x,x_{0}\in[1,\infty)$, $1/|xx_{0}|$ is largest when $x=1$ and $x_{0}=1$,
    yielding $1/|xx_{0}|=1$. Because of this we can update our wish-list one
    last time:
    \begin{equation}
        \textbf{Want:}\quad
        \delta\leq\varepsilon
    \end{equation}
    We can now just choose $\delta=\varepsilon$. Let's show that this works.
    Let $\varepsilon>0$. Choose $\delta=\varepsilon$. If $|x-x_{0}|<\delta$,
    then:
    \begin{align}
        |x-x_{0}|&<\varepsilon
            \tag{Definition of $\delta$}\\
        \Rightarrow
            \frac{1}{|xx_{0}|}|x-x_{0}|&<\varepsilon
                \tag{Since $\frac{1}{|xx_{0}|}\leq{1}$}\\
        \Rightarrow
            \Big|\frac{x-x_{0}}{xx_{0}}\Big|&<\varepsilon
                \tag{Absolute Value Rules}\\
        \Rightarrow
            \Big|\frac{1}{x}-\frac{1}{x_{0}}\Big|&<\varepsilon
                \tag{Simplify}\\
        \Rightarrow
            |f(x)-f(x_{0})|&<\varepsilon
                \tag{Definition of $f$}
    \end{align}
    Hence for all $\varepsilon>0$ there is a $\delta>0$ such that for all
    $x\in[1,\,\infty)$ with $|x-x_{0}|<\delta$ it is true that
    $|f(x)-f(x_{0})|<\varepsilon$. That is, $f$ is continuous at $x_{0}$.
    \newpage
    I, the copyright holder of this work, release it into the public domain.
    This applies worldwide. In some countries this may not be legally possible;
    if so: I grant anyone the right to use this work for any purpose, without
    any conditions, unless such conditions are required by law.
    \par\hfill\par
    The source code used to generate this document is free software and released
    under version 3 of the GNU General Public License.
\end{document}
