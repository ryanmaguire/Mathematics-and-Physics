%-----------------------------------LICENSE------------------------------------%
%   This file is part of Mathematics-and-Physics.                              %
%                                                                              %
%   Mathematics-and-Physics is free software: you can redistribute it and/or   %
%   modify it under the terms of the GNU General Public License as             %
%   published by the Free Software Foundation, either version 3 of the         %
%   License, or (at your option) any later version.                            %
%                                                                              %
%   Mathematics-and-Physics is distributed in the hope that it will be useful, %
%   but WITHOUT ANY WARRANTY; without even the implied warranty of             %
%   MERCHANTABILITY or FITNESS FOR A PARTICULAR PURPOSE.  See the              %
%   GNU General Public License for more details.                               %
%                                                                              %
%   You should have received a copy of the GNU General Public License along    %
%   with Mathematics-and-Physics.  If not, see <https://www.gnu.org/licenses/>.%
%------------------------------------------------------------------------------%
\documentclass{article}
\usepackage{amsmath}  % align environment here.
\usepackage{graphicx} % Needed for figures.
\usepackage{hyperref} % Hyperlinks for figures.
\hypersetup{colorlinks = true, linkcolor = blue}
\graphicspath{{../../../images/}}

\title{Domains of Functions - Example 6}
\author{Ryan Maguire}
\date{\today}

% No indent and no paragraph skip.
\setlength{\parindent}{0em}
\setlength{\parskip}{0em}

\begin{document}
    \maketitle
    Consider the expression:
    \begin{equation}
        f(x)=\frac{3x}{\frac{2}{x}-1}
    \end{equation}
    For which real numbers is this well-defined? The only thing to look out for
    is division-by-zero. The denominator is $2/x-1$, so if we are to avoid
    division-by-zero we must exclude $x=2$. The expression $2/x$ also
    contains a division, and so we must also exclude $x=0$. In set theory
    notation, we can write the domain as:
    \begin{equation}
        D=(-\infty,0)\cup(0,2)\cup(2,\infty)
    \end{equation}
    If we were to simplify, we get:
    \begin{align}
        f(x)&=\frac{3x}{\frac{2}{x}-1}\\
            &=\frac{x}{x}\cdot\frac{3x}{\frac{2}{x}-1}\\
            &=\frac{3x^{2}}{2-x}
    \end{align}
    And from this we can conclude that the \textit{limit} as $x$ approaches
    0 is 0. However, the simplification step involved multiplying by
    $\frac{x}{x}$ which is undefined for $x=0$. The function, as it was
    originally written, must exclude 0 from it's domain. This is
    plotted in Fig.~\ref{fig:graph_of_func}.
    \begin{figure}
        \centering
        \includegraphics{three_x_squared_by_2_minus_x.pdf}
        \caption{Graph of the function $f$}
        \label{fig:graph_of_func}
    \end{figure}
    \newpage
    I, the copyright holder of this work, release it into the public domain.
    This applies worldwide. In some countries this may not be legally possible;
    if so: I grant anyone the right to use this work for any purpose, without
    any conditions, unless such conditions are required by law.
    \par\hfill\par
    The source code used to generate this document is free software and released
    under version 3 of the GNU General Public License.
\end{document}
