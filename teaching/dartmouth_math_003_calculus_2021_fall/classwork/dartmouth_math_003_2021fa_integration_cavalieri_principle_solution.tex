%-----------------------------------LICENSE------------------------------------%
%   This file is part of Mathematics-and-Physics.                              %
%                                                                              %
%   Mathematics-and-Physics is free software: you can redistribute it and/or   %
%   modify it under the terms of the GNU General Public License as             %
%   published by the Free Software Foundation, either version 3 of the         %
%   License, or (at your option) any later version.                            %
%                                                                              %
%   Mathematics-and-Physics is distributed in the hope that it will be useful, %
%   but WITHOUT ANY WARRANTY; without even the implied warranty of             %
%   MERCHANTABILITY or FITNESS FOR A PARTICULAR PURPOSE.  See the              %
%   GNU General Public License for more details.                               %
%                                                                              %
%   You should have received a copy of the GNU General Public License along    %
%   with Mathematics-and-Physics.  If not, see <https://www.gnu.org/licenses/>.%
%------------------------------------------------------------------------------%
\documentclass{article}
\usepackage{graphicx} % Needed for figures.
\usepackage{amsmath}  % Needed for align.
\graphicspath{{../../../images/}}

\title{Cavalieri's Principle - Solution}
\author{Ryan Maguire}
\date{\today}

% No indent and no paragraph skip.
\setlength{\parindent}{0em}
\setlength{\parskip}{0em}

\begin{document}
    \maketitle
    \begin{enumerate}
        \item
            Moving around a single coin does not change its volume, and
            hence the sum of the volumes for each of the coins will remain
            constant as we perturb their positions.
        \item
            The area between $f$ and $g$ can be expressed via:
            \begin{equation}
                \textrm{Area}_{f,\,g}
                =\int_{a}^{b}\big(f(x)-g(x)\big)\;\textrm{d}x
            \end{equation}
        \item
            Applying Cavalieri's principle, we see that the area shouldn't
            change. That is, the area between $F$ and $G$ should be equal to
            the area between $f$ and $g$. This is because $F$ and $G$ are
            obtained by equally perturbing $f$ and $g$ by the continuous
            function $h$.
        \item
            We use the linearity of the integral. We have:
            \begin{align}
                \textrm{Area}_{F,\,G}
                &=\int_{a}^{b}\big(F(x)-G(x)\big)\;\textrm{d}x\\
                &=\int_{a}^{b}\Big(\big(f(x)+h(x)\big)-\big(g(x)+h(x)\big)\Big)
                    \;\textrm{d}x\\
                &=\int_{a}^{b}\big(f(x)+h(x)-g(x)-h(x)\big)\;\textrm{d}x\\
                &=\int_{a}^{b}\big(f(x)-g(x)\big)\;\textrm{d}x\\
                &=\textrm{Area}_{f,\,g}
            \end{align}
            And so we've proven Cavalieri's principle using calculus.
    \end{enumerate}
    \newpage
    I, the copyright holder of this work, release it into the public domain.
    This applies worldwide. In some countries this may not be legally possible;
    if so: I grant anyone the right to use this work for any purpose, without
    any conditions, unless such conditions are required by law.
    \par\hfill\par
    The source code used to generate this document is free software and released
    under version 3 of the GNU General Public License.
\end{document}
