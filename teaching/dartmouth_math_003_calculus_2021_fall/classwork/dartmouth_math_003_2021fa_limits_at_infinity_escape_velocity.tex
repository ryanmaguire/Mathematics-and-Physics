%-----------------------------------LICENSE------------------------------------%
%   This file is part of Mathematics-and-Physics.                              %
%                                                                              %
%   Mathematics-and-Physics is free software: you can redistribute it and/or   %
%   modify it under the terms of the GNU General Public License as             %
%   published by the Free Software Foundation, either version 3 of the         %
%   License, or (at your option) any later version.                            %
%                                                                              %
%   Mathematics-and-Physics is distributed in the hope that it will be useful, %
%   but WITHOUT ANY WARRANTY; without even the implied warranty of             %
%   MERCHANTABILITY or FITNESS FOR A PARTICULAR PURPOSE.  See the              %
%   GNU General Public License for more details.                               %
%                                                                              %
%   You should have received a copy of the GNU General Public License along    %
%   with Mathematics-and-Physics.  If not, see <https://www.gnu.org/licenses/>.%
%------------------------------------------------------------------------------%
\documentclass{article}
\usepackage{amsmath}

\title{Escape Velocity}
\author{Ryan Maguire}
\date{\today}

% No indent and no paragraph skip.
\setlength{\parindent}{0em}
\setlength{\parskip}{0em}

\begin{document}
    \maketitle
    If you throw a ball in the air, it will come back down.
    \textit{Or will it?} There is a certain height the ball will get
    where at that point it will momentarily stop moving, before
    falling back down towards the earth. What if you threw the ball so hard
    that this height was \textit{infinity}? The idea is called escape velocity.
    The laws of Newtonian mechanics tell us that the potential and kinetic
    energy of a ball being thrown are:
    \begin{equation}
        U+V=-\frac{GMm}{r}+\frac{1}{2}mv^{2}
    \end{equation}
    Here, $G$ is a universal constant, $M$ is the mass of the Earth, $m$ is the
    mass of the ball, $r$ is the distance from the core of the Earth to the
    ball, and $v$ is the speed of the ball. The law of conservation of energy
    says that the sum of kinetic and potential energy is conserved:
    \begin{equation}
        U_{0}+V_{0}=U_{1}+V_{1}
    \end{equation}
    Where $U_{0}$ and $V_{0}$ are the initial potential and kinetic energy,
    respectively, and $U_{1}$ and $V_{1}$ are the kinetic
    and potential energies at infinity. The initial values are given by:
    \begin{equation}
        U_{0}+V_{0}=-\frac{GMm}{R}+\frac{1}{2}mv_{0}^{2}
    \end{equation}
    Here, $R$ is the radius of the Earth, and $v_{0}$ is the initial speed of
    the ball.
    \par\hfill\par
    \begin{enumerate}
        \item Try to find the formula for escape velocity. You want to solve
              the values of $U_{1}$ and $V_{1}$ \textit{at infinity}. Remember,
              we want the ball to reach zero velocity at infinity.
        \item What would happen if you threw the ball with a speed
              \textit{greater} than escape velocity? What would the final speed
              be \textit{at infinity} (Again, you want to solve the limit at
              infinity for $V_{1}$)?
        \item Using the following quantities, what is escape velocity for Earth?
            \begin{itemize}
                \item $M=5.9722\times10^{24}\text{kg}$
                \item $G=6.674\times10^{-11}\text{m}^{3}\text{kg}^{-1}\text{s}^{-2}$
                \item $R=6.3781\times{10}^{6}\text{m}$
            \end{itemize}
        \item What would happen if $M$ was so big that escape velocity was
              greater than the speed of light?
        \item The speed of light is
              $2.99792458\times{10}^{8}\text{m}/\text{s}$. How big would
              the radius of Earth need to be for it to turn into a black hole?
    \end{enumerate}
    \newpage
    I, the copyright holder of this work, release it into the public domain.
    This applies worldwide. In some countries this may not be legally possible;
    if so: I grant anyone the right to use this work for any purpose, without
    any conditions, unless such conditions are required by law.
    \par\hfill\par
    The source code used to generate this document is free software and released
    under version 3 of the GNU General Public License.
\end{document}
