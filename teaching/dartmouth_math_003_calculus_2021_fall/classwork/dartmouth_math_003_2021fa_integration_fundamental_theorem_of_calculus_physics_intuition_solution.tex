%-----------------------------------LICENSE------------------------------------%
%   This file is part of Mathematics-and-Physics.                              %
%                                                                              %
%   Mathematics-and-Physics is free software: you can redistribute it and/or   %
%   modify it under the terms of the GNU General Public License as             %
%   published by the Free Software Foundation, either version 3 of the         %
%   License, or (at your option) any later version.                            %
%                                                                              %
%   Mathematics-and-Physics is distributed in the hope that it will be useful, %
%   but WITHOUT ANY WARRANTY; without even the implied warranty of             %
%   MERCHANTABILITY or FITNESS FOR A PARTICULAR PURPOSE.  See the              %
%   GNU General Public License for more details.                               %
%                                                                              %
%   You should have received a copy of the GNU General Public License along    %
%   with Mathematics-and-Physics.  If not, see <https://www.gnu.org/licenses/>.%
%------------------------------------------------------------------------------%
\documentclass{article}
\usepackage{amsmath}  % Needed for align.
\usepackage{amssymb}  % Needed for mathbb.
\usepackage{listings} % Used for Python code.

\title{Velocity and Displacement}
\author{Ryan Maguire}
\date{\today}

% No indent and no paragraph skip.
\setlength{\parindent}{0em}
\setlength{\parskip}{0em}

\begin{document}
    \maketitle
    \begin{enumerate}
        \item
            $D$ represent the total displacement of the particle.
            The integral is summing up all of the tiny displacements
            $v(t)\;\textrm{d}t$ that occur by moving with velocity
            $v(t)$ over a time increment $\textrm{d}t$. By integrating we
            obtain the total displacement over the time interval.
        \item
            Using a left-sided Riemann sum we can numerically compute
            with a displacement of $10^{-4}$ to obtain an approximation of
            $D\approx-4.995$. This can be done in a few lines in Python.
            \begin{lstlisting}[language = Python, gobble = 16]
                displacement = 1.0E-4
                start = 0
                end = 1
                number_of_samples = int((end - start) / displacement)
                x = [n * displacement for n in range(number_of_samples)]
                v = [-10*r for r in x]
                numerical_integral = sum(v) * displacement
                print(numerical_integral)
            \end{lstlisting}
        \item
            The exact value of $r(1)-r(0)$ is $-5$.
            From here we should be able to guess that displacement, velocity,
            and integration are all connected. Indeed they are, and this is
            fundamental theorem of calculus. It should be noted that intuition
            is not a proof. It is quite fortunate, however, that a rigorous
            proof of this theorem is about $\varepsilon$ away once one
            understands the physical notions at play here.
    \end{enumerate}
    \newpage
    I, the copyright holder of this work, release it into the public domain.
    This applies worldwide. In some countries this may not be legally possible;
    if so: I grant anyone the right to use this work for any purpose, without
    any conditions, unless such conditions are required by law.
    \par\hfill\par
    The source code used to generate this document is free software and released
    under version 3 of the GNU General Public License.
\end{document}
