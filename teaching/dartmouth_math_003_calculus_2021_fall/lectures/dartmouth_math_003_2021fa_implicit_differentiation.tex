%-----------------------------------LICENSE------------------------------------%
%   This file is part of Mathematics-and-Physics.                              %
%                                                                              %
%   Mathematics-and-Physics is free software: you can redistribute it and/or   %
%   modify it under the terms of the GNU General Public License as             %
%   published by the Free Software Foundation, either version 3 of the         %
%   License, or (at your option) any later version.                            %
%                                                                              %
%   Mathematics-and-Physics is distributed in the hope that it will be useful, %
%   but WITHOUT ANY WARRANTY; without even the implied warranty of             %
%   MERCHANTABILITY or FITNESS FOR A PARTICULAR PURPOSE.  See the              %
%   GNU General Public License for more details.                               %
%                                                                              %
%   You should have received a copy of the GNU General Public License along    %
%   with Mathematics-and-Physics.  If not, see <https://www.gnu.org/licenses/>.%
%------------------------------------------------------------------------------%
\documentclass{article}
\usepackage{amssymb} % mathbb font.
\usepackage{amsmath} % Needed for align.
\usepackage{amsthm}  % For the theorem environment.

\theoremstyle{plain}
\newtheorem{theorem}{Theorem}

\title{Implicit Differentiation}
\author{Ryan Maguire}
\date{\today}

% No indent and no paragraph skip.
\setlength{\parindent}{0em}
\setlength{\parskip}{0em}

\begin{document}
    \maketitle
    Algebraic expressions can be used to describe geometric ideas. The
    simplest example of this is the unit circle:
    \begin{equation}
        x^{2}+y^{2}=1
    \end{equation}
    \textit{Locally} it is possible to describe the circle by writing either
    $y$ as a function of $x$, $y=y(x)$, or by writing $x$ as a function of
    $y$, $x=x(y)$. That is, if we consider the parts of the unit circle with
    $y>0$ we can write:
    \begin{equation}
        y(x)=\sqrt{1-x^{2}}
    \end{equation}
    The slope at a point $(x,\,y)$ on the unit circle with $y>0$ can then be
    computed from this. Invoking the chain rule we get:
    \begin{equation}
        \label{eqn:y_prime}
        y'(x)=\frac{-x}{\sqrt{1-x^{2}}}
    \end{equation}
    We can skip the algebra by using \textit{implicit differentiation}. We can
    write:
    \begin{equation}
        x^{2}+y(x)^{2}=1
    \end{equation}
    By differentiating both sides of this equation with respect to $x$ and
    invoking the chain rule we get:
    \begin{equation}
        2x+2y\frac{\textrm{d}y}{\textrm{d}x}=0
    \end{equation}
    By solving for $\textrm{d}y/\textrm{d}x$ we obtain:
    \begin{equation}
        \label{eqn:y_prime_implicit}
        \frac{\textrm{d}y}{\textrm{d}x}=-\frac{x}{y}
    \end{equation}
    This is in agreement with our previous calculation where we had a formula
    for $y$, if we substitute $y(x)=\sqrt{1-x^{2}}$ into
    Eqn.~\ref{eqn:y_prime_implicit} we obtain
    Eqn.~\ref{eqn:y_prime}. The usefulness in this equation is that the slope
    can be directly read from the coordinates. That is, if $(x,\,y)$ is a point
    on the circle, the slope is given by $-x/y$. This does not require the
    assumption that $y>0$, unlike our previous effort.
    \par\hfill\par
    As another application, the \textit{jerk} of a particle is the change in
    its acceleration as a function of time. That is, if $a(t)$ is the
    acceleration of a particle, the jerk is given by:
    \begin{equation}
        j(t)=\dot{a}(t)=\frac{\textrm{d}}{\textrm{d}\,t}\big(a(t)\big)
    \end{equation}
    the time derivative of the acceleration. For something like a simple
    pendulum the acceleration is given by the angle the pendulum makes with
    the verticle axis $\theta$. That is:
    \begin{equation}
        a(t)=-g\sin\big(\theta(t)\big)
    \end{equation}
    where $g$ is the constant of gravity. The jerk is then given by implicit
    differentiation. We don't know the explicit formula for $\theta(t)$ but
    we can still write:
    \begin{align}
        j(t)
        &=\dot{a}(t)\\
        &=\frac{\textrm{d}}{\textrm{d}\,t}\Big(-g\sin\big(\theta(t)\big)\Big)\\
        &=-g\frac{\textrm{d}}{\textrm{d}\,t}\Big(\sin\big(\theta(t)\big)\Big)\\
        &=-g\cos\big(\theta(t)\big)\,\dot{\theta}(t)
    \end{align}
    where $\dot{\theta}(t)$ is the \textit{angular velocity}, the rate of
    change in the angle of the pendulum.
\end{document}
