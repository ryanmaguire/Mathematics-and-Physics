%-----------------------------------LICENSE------------------------------------%
%   This file is part of Mathematics-and-Physics.                              %
%                                                                              %
%   Mathematics-and-Physics is free software: you can redistribute it and/or   %
%   modify it under the terms of the GNU General Public License as             %
%   published by the Free Software Foundation, either version 3 of the         %
%   License, or (at your option) any later version.                            %
%                                                                              %
%   Mathematics-and-Physics is distributed in the hope that it will be useful, %
%   but WITHOUT ANY WARRANTY; without even the implied warranty of             %
%   MERCHANTABILITY or FITNESS FOR A PARTICULAR PURPOSE.  See the              %
%   GNU General Public License for more details.                               %
%                                                                              %
%   You should have received a copy of the GNU General Public License along    %
%   with Mathematics-and-Physics.  If not, see <https://www.gnu.org/licenses/>.%
%------------------------------------------------------------------------------%
\documentclass{article}
\usepackage{graphicx} % Needed for figures.
\usepackage{amsmath}  % Needed for cases environment.
\usepackage{amssymb}  % Needed for mathbb.
\usepackage{amsthm}   % For the theorem environment.
\theoremstyle{plain}
\newtheorem{theorem}{Theorem}
\graphicspath{{../../../images/}}

\title{Integrability}
\author{Ryan Maguire}
\date{\today}

% No indent and no paragraph skip.
\setlength{\parindent}{0em}
\setlength{\parskip}{0em}

\begin{document}
    \maketitle
    Intuitively, a continuous function on a closed bounded interval
    $f:[a,b]\rightarrow\mathbb{R}$ should have a well-defined area below
    the curve. The extreme value theorem tells us there is a minimum value
    $m$ that $f$ reaches, and a maximum value $M$. The area under $f$ should
    then be bounded between $m(b-a)$ and $M(b-a)$. $m(b-a)$ is the (red) area
    of the rectangle with height $m$ and width $b-a$ and $M(b-a)$ is the
    (green) area of the rectangle width height $M$ and width $b-a$. The
    function $f$ is contained within the area $M(b-a)$ and above the area
    $m(b-a)$, so the (blue) area under $f$ should lie between these two values.
    \begin{figure}
        \centering
        \includegraphics{continuous_bounded_function_has_bounded_area.pdf}
        \caption{Continuous functions have a well-defined area.}
    \end{figure}
    A partition of a closed interval $[a,b]$ is a finite sequence of points
    $a=x_{0}<x_{1}<\cdots<x_{N}=b$. The lower sum of a continuous function
    $f:[a,b]\rightarrow\mathbb{R}$ is the sum:
    \begin{equation}
        L(f,x_{n})=\sum_{n=0}^{N-1}m_{n}\Delta{x}_{n}
    \end{equation}
    Where $\Delta{x}_{n}=x_{n+1}-x_{n}$ is the distance between the
    $n^{th}$ point and the $(n+1)^{th}$ point, and
    $m_{n}$ is the minimum value of $f(x)$ on the interval $[x_{n},x_{n+1}]$.
    By the extreme value theorem, $m_{n}$ is well-defined since $f$ is
    continuous on $[x_{n},x_{n+1}]$.
    \begin{figure}
        \centering
        \includegraphics{lower_sum_of_continuous_function.pdf}
        \caption{Lower Sum of a Continuous Function}
    \end{figure}
    This sum represents the area underneath the rectangles of height $m_{n}$
    and width $\Delta_{n}$, as shown above. The upper sum of a continuous
    function is similarly defined:
    \begin{equation}
        U(f,x_{n})=\sum_{n=0}^{N-1}M_{n}\Delta{x}_{n}
    \end{equation}
    Where $M_{n}$ is the maximum of $f(x)$ on the interval $[x_{n},x_{n+1}]$.
    Again, this is well-defined by the extreme value theorem.
    \begin{figure}
        \centering
        \includegraphics{upper_sum_of_continuous_function.pdf}
        \caption{Upper Sum of a Continuous Function}
    \end{figure}
    This has a similar interpretation as the area beneath rectangles of height
    $M_{n}$ and width $\Delta{x}_{n}$.
    \begin{theorem}
        If $f:[a,b]\rightarrow\mathbb{R}$ is a continuous function, then for
        every $\varepsilon>0$ there is a partition
        $a=x_{0}<x_{1}<\cdots<x_{N}=b$ with
        $U(f,x_{n})-L(f,x_{n})<\varepsilon$.
    \end{theorem}
    This theorem shows that, as the partitions get finer and finer, the
    values $U(f,x_{n})$ and $L(f,x_{n})$ converge, and they converge to the
    same number. We define this value as the \textit{integral} of $f$ and write
    this as:
    \begin{equation}
        \int_{a}^{b}f(x)\;\textrm{d}x
            =\lim_{\textrm{partition gets finer}}L(f,x_{n})
            =\lim_{\textrm{partition gets finer}}U(f,x_{n})
    \end{equation}
    \begin{figure}
        \centering
        \includegraphics{integral_of_continuous_function.pdf}
        \caption{Integral of a Continuous Function}
    \end{figure}
    We can be more general. If we have a bounded function
    $f:[a,b]\rightarrow\mathbb{R}$ we can give the same definitions, and can
    say what it means for a function to be \textit{integrable}. From the above
    discussion we have the following:
    \begin{theorem}
        If $f:[a,b]\rightarrow\mathbb{R}$ is a continuous function, then it is
        integrable.
    \end{theorem}
    Is \textit{every} function integrable? Consider the function
    $f:[0,1]\rightarrow\mathbb{R}$ defined by:
    \begin{equation}
        f(x)=
        \begin{cases}
            0&x\in\mathbb{Q}\\
            1&x\notin\mathbb{Q}
        \end{cases}
    \end{equation}
    For any partition $0=x_{0}<x_{1}\cdots<x_{N}=1$, and for any interval
    $[x_{n},x_{n+1}]$, the minimum is $m_{n}=0$ and the maximum is $M_{n}=1$.
    The lower sum is 0 and the upper sum is 1, regardless of partition. Because
    of this the function is not integrable.
    \par\hfill\par
    Courses like real analysis and measure theory handle all sorts of
    pathological functions like this. For calculus, we'll stick with continuous
    functions for the most part.
    \newpage
    I, the copyright holder of this work, release it into the public domain.
    This applies worldwide. In some countries this may not be legally possible;
    if so: I grant anyone the right to use this work for any purpose, without
    any conditions, unless such conditions are required by law.
    \par\hfill\par
    The source code used to generate this document is free software and released
    under version 3 of the GNU General Public License.
\end{document}
