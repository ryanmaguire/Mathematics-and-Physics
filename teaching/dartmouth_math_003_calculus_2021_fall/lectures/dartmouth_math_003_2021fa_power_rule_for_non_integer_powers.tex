%-----------------------------------LICENSE------------------------------------%
%   This file is part of Mathematics-and-Physics.                              %
%                                                                              %
%   Mathematics-and-Physics is free software: you can redistribute it and/or   %
%   modify it under the terms of the GNU General Public License as             %
%   published by the Free Software Foundation, either version 3 of the         %
%   License, or (at your option) any later version.                            %
%                                                                              %
%   Mathematics-and-Physics is distributed in the hope that it will be useful, %
%   but WITHOUT ANY WARRANTY; without even the implied warranty of             %
%   MERCHANTABILITY or FITNESS FOR A PARTICULAR PURPOSE.  See the              %
%   GNU General Public License for more details.                               %
%                                                                              %
%   You should have received a copy of the GNU General Public License along    %
%   with Mathematics-and-Physics.  If not, see <https://www.gnu.org/licenses/>.%
%------------------------------------------------------------------------------%
\documentclass{article}
\usepackage{amssymb} % mathbb font.
\usepackage{amsmath} % Needed for align.
\usepackage{amsthm}  % For the theorem environment.

\theoremstyle{plain}
\newtheorem{theorem}{Theorem}

\title{Power Rule for Non-Integer Powers}
\author{Ryan Maguire}
\date{\today}

% No indent and no paragraph skip.
\setlength{\parindent}{0em}
\setlength{\parskip}{0em}

\begin{document}
    \maketitle
    When we prove the power rule we assume the raised index is an integer in our
    proof. Many textbooks then \textit{assume} the formula holds for arbitrary
    real numbers. That is, if $r$ is a real number, then:
    \begin{equation}
        \frac{\textrm{d}}{\textrm{d}\,x}\big(x^{r}\big)=rx^{r-1}
    \end{equation}
    To verify this formula you must first ask
    \textit{what does} $x^{r}$ \textit{even mean}? For integer powers this is
    clearer, you iteratively multiply by $x$ and you does this $r$ times.
    For fractions this also has an interpretation. To compute $x^{p/q}$ you
    first find a number $y$ such that $y^{q}=x$, and then you raise $y$ to the
    $p$ power. That is, you take the $q^{\textrm{th}}$ root of $x$ raise it to
    the $p^{\textrm{th}}$ power. But what on Earth is $2^{\pi}$? Or
    $\pi^{\pi}$? You could define this in terms of \textit{limits} of fractions.
    That is, find a sequence of rational numbers $p_{n}/q_{n}$ that get closer
    to $\pi$ and define $x^{\pi}$ as the \textit{limit} of $x^{p_{n}/q_{n}}$.
    \par\hfill\par
    This is fairly cumbersome. Instead, let's insist that whatever the
    definition is it should satisfy some of the usual power rules. Namely,
    $\ln(x^{r})=r\ln(x)$. If we exponentiate both sides we get:
    \begin{equation}
        x^{r}=\exp\big(r\ln(x)\big)
    \end{equation}
    We take this as our definition of $x^{r}$. Note that it agrees with our
    previous definition for integers:
    \begin{equation}
        x^{n}=\exp\big(n\ln(x)\big)
    \end{equation}
    We can differentiate $x^{r}$
    using the chain rule. We have:
    \begin{align}
        \frac{\textrm{d}}{\textrm{d}\,x}\big(x^{r}\big)
            &=\frac{\textrm{d}}{\textrm{d}\,x}\Big(\exp\big(r\ln(x)\big)\Big)
                \tag{Definition}\\
            &=\exp'\big(r\ln(x)\big)\cdot
                \frac{\textrm{d}}{\textrm{d}\,x}\big(r\ln(x)\big)
                    \tag{Chain Rule}\\
            &=\exp\big(r\ln(x)\big)\cdot
                \frac{\textrm{d}}{\textrm{d}\,x}\big(r\ln(x)\big)
                    \tag{Derivative of $\exp$}\\
            &=r\exp\big(r\ln(x)\big)\cdot
                \frac{\textrm{d}}{\textrm{d}\,x}\big(\ln(x)\big)
                    \tag{Factor Constants}\\
            &=r\exp\big(r\ln(x)\big)\cdot\frac{1}{x}
                \tag{Derivative of $\ln$}\\
            &=r\exp\big(r\ln(x)\big)\cdot\exp\big(-\ln(x)\big)
                \tag{Definition}\\
            &=r\exp\big(r\ln(x)-\ln(x)\big)
                \tag{Exponent Rule}\\
            &=r\exp\big((r-1)\ln(x)\big)
                \tag{Simplify}\\
            &=rx^{r-1}
                \tag{Definition}
    \end{align}
    And so the formula is valid for all real numbers. But note the definition
    requires that $x>0$ since we are taking the natural log of $x$. That is,
    the \textit{domain} of $f(x)=x^{r}$ is $(0,\,\infty)$. On this set we have,
    for all real numbers $r\in\mathbb{R}$, the formula:
    \begin{equation}
        \frac{\textrm{d}}{\textrm{d}\,x}\big(x^{r}\big)=rx^{r-1}
    \end{equation}
    The domain can be expanded to all real numbers for certain $r$, such as
    integer $r$ or $r=p/q$ where $q$ is an odd number, but in general
    $x^{r}$ may not be a real number for negative $x$ and certain $r$.
    Consider $x=-1$ and $r=1/2$ as an example. Similarly, problems may occur
    at zero, consider $x=0$ and $r=-1$. For the general $r\in\mathbb{R}$ we
    can only consider $x^{r}$ for $x\in(0,\,\infty)$.
\end{document}
