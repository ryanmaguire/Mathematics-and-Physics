%-----------------------------------LICENSE------------------------------------%
%   This file is part of Mathematics-and-Physics.                              %
%                                                                              %
%   Mathematics-and-Physics is free software: you can redistribute it and/or   %
%   modify it under the terms of the GNU General Public License as             %
%   published by the Free Software Foundation, either version 3 of the         %
%   License, or (at your option) any later version.                            %
%                                                                              %
%   Mathematics-and-Physics is distributed in the hope that it will be useful, %
%   but WITHOUT ANY WARRANTY; without even the implied warranty of             %
%   MERCHANTABILITY or FITNESS FOR A PARTICULAR PURPOSE.  See the              %
%   GNU General Public License for more details.                               %
%                                                                              %
%   You should have received a copy of the GNU General Public License along    %
%   with Mathematics-and-Physics.  If not, see <https://www.gnu.org/licenses/>.%
%------------------------------------------------------------------------------%
\documentclass{article}

\usepackage{amssymb} % mathbb font.
\usepackage{amsmath} % Needed for align.
\usepackage{amsthm}  % For the theorem environment.

\theoremstyle{plain}
\newtheorem{theorem}{Theorem}

\title{Power Rule}
\author{Ryan Maguire}
\date{\today}

% No indent and no paragraph skip.
\setlength{\parindent}{0em}
\setlength{\parskip}{0em}

\begin{document}
    \maketitle
    The power rule tells us how to differentiate integer powers of a variable.
    Things like $x^{2}$, $x^{3}$, and even $x^{-1}$. The proof is an
    application of the product rule and the use of
    \textit{mathematical induction}.
    \par\hfill\par
    If you have a sequence of sentences
    $P_{n}$ and you know that $P_{0}$ is true and for every $n\in\mathbb{N}$
    the truth of $P_{n}$ implies the truth of $P_{n+1}$, then every sentence
    $P_{n}$ is true. The argument for this is that $P_{0}$ being true implies
    $P_{1}$ is true, and we know $P_{0}$ is true, so $P_{1}$ must be true.
    Similarly $P_{1}$ being true implies $P_{2}$ is true, and we've just shown
    that $P_{1}$ is true, so $P_{2}$ is also true. Since $P_{n}$ implies
    $P_{n+1}$, we can continue up this ladder indefinitely and claim that every
    sentence is true.
    \par\hfill\par
    For our purposes the sentence $P_{n}$ is an equation. It says:
    \begin{equation}
        \frac{\textrm{d}}{\textrm{d}\,x}\big(x^{n}\big)=nx^{n-1}
    \end{equation}
    The case for $P_{0}$ is true since $x^{0}\equiv{1}$, the derivative of this
    is zero, and the resulting formula $0\cdot{x}^{-1}$ is also zero.
    Now we need to prove that the truth of the $n^{th}$ equation implies the
    validity of the $(n+1)^{th}$ one. That is, we may \textit{assume} that
    this equation is true for a fixed $n\in\mathbb{N}$, and then we must prove
    this implies the equation holds for $n+1$. This is done with the product
    rule. We have:
    \begin{align}
        \frac{\textrm{d}}{\textrm{d}\,x}\big(x^{n+1}\big)
            &=\frac{\textrm{d}}{\textrm{d}\,x}\big(x^{n}\cdot{x}\big)
                \tag{Definition}\\
            &=\frac{\textrm{d}}{\textrm{d}\,x}\big(x^{n}\big)\cdot{x}
                +x^{n}\frac{\textrm{d}}{\textrm{d}\,x}\big(x\big)
                    \tag{Product Rule}\\
            &=\frac{\textrm{d}}{\textrm{d}\,x}\big(x^{n}\big)\cdot{x}+x^{n}
                    \tag{Derivative of $x$}\\
            &=\big(nx^{n-1}\big)\cdot{x}+x^{n}
                \tag{Induction Hypothesis}\\
            &=nx^{n}+x^{n}
                \tag{Simplify}\\
            &=(n+1)x^{n}
                \tag{Simplify}
    \end{align}
    which is exactly the formula for $n+1$. By induction, the formula is always
    true. Our first use of this is the differentiation of polynomials. Given:
    \begin{equation}
        f(x)=\sum_{n=0}^{N}a_{n}x^{n}
    \end{equation}
    we may differentiate this using the sum and power rules. We have:
    \begin{align}
        f'(x)
            &=\frac{\textrm{d}}{\textrm{d}\,x}\sum_{n=0}^{N}a_{n}x^{n}
                \tag{Definition}\\
            &=\sum_{n=0}^{N}\frac{\textrm{d}}{\textrm{d}\,x}\big(a_{n}x^{n}\big)
                \tag{Sum Rule}\\
            &=\sum_{n=0}^{N}a_{n}\frac{\textrm{d}}{\textrm{d}\,x}\big(x^{n}\big)
                \tag{Factoring Constants}\\
            &=\sum_{n=0}^{N}a_{n}nx^{n-1}
                \tag{Power Rule}
    \end{align}
    Let's conclude by proving the power rule holds for negative integers.
    First, let's differentiate $x^{-1}=\frac{1}{x}$. We'll do this in two ways.
    The limit definition says:
    \begin{align}
        \lim_{h\rightarrow{0}}\frac{\frac{1}{x+h}-\frac{1}{x}}{h}
            &=\lim_{h\rightarrow{0}}\frac{x-(x+h)}{hx(x+h)}\\
            &=\lim_{h\rightarrow{0}}\frac{-h}{hx(x+h)}\\
            &=\lim_{h\rightarrow{0}}\frac{-1}{x(x+h)}\\
            &=\frac{-1}{x^{2}}\\
            &=-x^{-2}
    \end{align}
    We can also use the product rule. We have $x\cdot{x}^{-1}=1$ and so can
    write:
    \begin{align}
        0
            &=\frac{\textrm{d}}{\textrm{d}\,x}\big(1\big)\\
            &=\frac{\textrm{d}}{\textrm{d}\,x}\big(x\cdot{x}^{-1}\big)\\
            &=\frac{\textrm{d}}{\textrm{d}\,x}\big(x)\cdot{x}^{-1}
                +x\cdot\frac{\textrm{d}}{\textrm{d}\,x}\big(x^{-1}\big)\\
            &=x^{-1}+x\frac{\textrm{d}}{\textrm{d}\,x}\big(x^{-1}\big)
    \end{align}
    We now solve for $\frac{\textrm{d}}{\textrm{d}\,x}(x^{-1})$ and obtain:
    \begin{align}
        x^{-1}+x\frac{\textrm{d}}{\textrm{d}\,x}\big(x^{-1}\big)&=0\\
        \Rightarrow
            x\frac{\textrm{d}}{\textrm{d}\,x}\big(x^{-1}\big)&=-x^{-1}\\
        \Rightarrow
            \frac{\textrm{d}}{\textrm{d}\,x}\big(x^{-1}\big)&=-x^{-2}
    \end{align}
    We now prove the power rule for negative integers using induction.
    We have proven the result for $-1$ and need to show that formula being true
    for $-n$ implies it is true for $-(n+1)$.
    \begin{align}
        \frac{\textrm{d}}{\textrm{d}\,x}\big(x^{-(n+1)}\big)
            &=\frac{\textrm{d}}{\textrm{d}\,x}\big(x^{-n-1}\big)\\
            &=\frac{\textrm{d}}{\textrm{d}\,x}\big(x^{-n}\cdot{x}^{-1}\big)\\
            &=\frac{\textrm{d}}{\textrm{d}\,x}\big(x^{-n}\big)\cdot{x}^{-1}
                +x^{-n}\cdot\frac{\textrm{d}}{\textrm{d}\,x}\big(x^{-1}\big)\\
            &=\frac{\textrm{d}}{\textrm{d}\,x}\big(x^{-n}\big)\cdot{x}^{-1}
                +x^{-n}\cdot\big(-x^{-2}\big)\\
            &=\frac{\textrm{d}}{\textrm{d}\,x}\big(x^{-n}\big)\cdot{x}^{-1}
                -x^{-n-2}\\
            &=-nx^{-n-1}\cdot{x}^{-1}-x^{-n-2}\\
            &=-nx^{-n-2}-x^{-n-2}\\
            &=-(n+1)x^{-n-2}\\
            &=-(n+1)x^{-(n+1)-1}
    \end{align}
    Hence for all integers, positive, negative, or zero, we have:
    \begin{equation}
        \frac{\textrm{d}}{\textrm{d}\,x}\big(x^{n}\big)
            =nx^{n-1}
    \end{equation}
\end{document}
