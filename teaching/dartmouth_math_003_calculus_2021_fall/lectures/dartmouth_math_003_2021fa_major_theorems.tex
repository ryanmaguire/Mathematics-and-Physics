%-----------------------------------LICENSE------------------------------------%
%   This file is part of Mathematics-and-Physics.                              %
%                                                                              %
%   Mathematics-and-Physics is free software: you can redistribute it and/or   %
%   modify it under the terms of the GNU General Public License as             %
%   published by the Free Software Foundation, either version 3 of the         %
%   License, or (at your option) any later version.                            %
%                                                                              %
%   Mathematics-and-Physics is distributed in the hope that it will be useful, %
%   but WITHOUT ANY WARRANTY; without even the implied warranty of             %
%   MERCHANTABILITY or FITNESS FOR A PARTICULAR PURPOSE.  See the              %
%   GNU General Public License for more details.                               %
%                                                                              %
%   You should have received a copy of the GNU General Public License along    %
%   with Mathematics-and-Physics.  If not, see <https://www.gnu.org/licenses/>.%
%------------------------------------------------------------------------------%
\documentclass{article}
\usepackage{amssymb} % mathbb font.
\usepackage{amsmath} % Needed for align.
\usepackage{amsthm}  % For the theorem environment.

\theoremstyle{plain}
\newtheorem{theorem}{Theorem}

\title{Major Theorems}
\author{Ryan Maguire}
\date{\today}

% No indent and no paragraph skip.
\setlength{\parindent}{0em}
\setlength{\parskip}{0em}

\begin{document}
    \maketitle
    Fermat's theorem relates the local extrema of differentiable functions to
    their derivatives. It is quite useful in applied mathematics and physics
    where one often wishes to find the points where local extrema occur.
    \begin{theorem}[Fermat's Theorem]
        If $f$ is a differentiable function, and if $x_{0}$ is a local extremum
        of $f$ (either a local minimum or local maximum), then $f'(x_{0})=0$.
    \end{theorem}
    \begin{proof}
        Suppose that $x_{0}$ is a local minimum of $f$. The proof is symmetric
        if $x_{0}$ is a local maximum. Since this point is a local min, there
        is some interval about the point such that $f(x)\geq{f}(x_{0})$ for all
        $x$ in this interval. But then $f(x)-f(x_{0})\geq{0}$ is true for all
        such $x$. But then for all $x>x_{0}$ in this interval we have
        $x-x_{0}>0$, and hence $(f(x)-f(x_{0}))/(x-x_{0})>0$. For all
        $x<x_{0}$ in this interval we have $x-x_{0}<0$ and hence
        $(f(x)-f(x_{0}))/(x-x_{0})<0$. That is, to the \textit{right} of
        $x_{0}$ the slope of the secant line is positive, and to the
        \textit{left} of the point the slope of the secant line is negative.
        Since $f$ is differentiable at $x_{0}$ the limit of this ratio exists
        and must be the same as one approaches either from the left or the
        right. For this to be true, the limit must be zero, and hence
        $f'(x_{0})=0$.
    \end{proof}
    The extreme value theorem is needed to prove some of the other big theorems.
    Tradition states that the proof must be skipped in a calculus course since
    it is very hard and requires notions from topology and real analysis. We
    will slightly break from tradition and present some of the ideas, but we'll
    do this loosely.
    \begin{theorem}[Extreme Value Theorem]
        If $f:[a,\,b]\rightarrow\mathbb{R}$ is continuous, then there are points
        $x_{m},\,x_{M}\in[a,\,b]$ such that
        $f(x_{m})\leq{f}(x)\leq{f}(x_{M})$ for all $x\in[a,\,b]$.
    \end{theorem}
    \begin{proof}[Sketch of Proof]
        Firstly, $f$ must be bounded. For if not, we can find a point
        $x_{1}\in[a,\,b]$ such that $|f(x_{1})|\geq{1}$. Then we can find a
        point $x_{2}\in[a,\,b]$ where $|f(x_{2})|\geq{2}$. Similarly for all
        $n\in\mathbb{N}$ we can find a point $x_{n}\in[a,\,b]$ such that
        $|f(x_{n})|\geq{n}$. If we pick infinitely many points $x_{n}$ in the
        interval $[a,\,b]$ there must be some point $x\in[a,\,b]$ that is
        arbitrarily close to these points. This is the Bolzano-Weierstrass
        theorem, and the proof is usually reserved for a real analysis course,
        or introductory topology. Since $f$ is continuous, the values of
        $f(x_{n})$ should approach $f(x)$ as the points $x_{n}$ get closer to
        $x$. But the values of $f(x_{n})$ diverge off to infinity and cannot
        possibly converge to $f(x)$, contradicting the fact that $f$ is
        continuous. So $f$ must be bounded. One of the defining properties of
        the real numbers is that there exists least upper bounds and greatest
        lower bounds of bounded sets. Let $y_{m}$ and $y_{M}$ be the greatest
        lower bound and least upper bound of the range of $f$, respectively.
        Since $[a,\,b]$ is a closed and bounded set, and since $f$ is
        continuous, the range of $f$ is also closed (this is a fact about
        \textit{compactness} from topology, but it should seem to be
        \textit{intuitively} true). Since the range is closed,
        the values $y_{m}$ and $y_{M}$ must be a part of the range. But then
        there must be points $x_{m},x_{M}\in[a,\,b]$ such that
        $f(x_{m})=y_{m}$ and $f(x_{M})=y_{M}$. But since $y_{m}$ and $y_{M}$
        are the greatest lower bound and least upper bound of the range,
        respectively, for all $x\in[a,\,b]$ we have
        $f(x_{m})\leq{f}(x)\leq{f}(x_{M})$.
    \end{proof}
    The are a lot of subleties here. If we let $A$ be the set of all
    rational numbers between $0$ and $2$, $A=[0,\,2]\cap\mathbb{Q}$,
    a continuous function $f:A\rightarrow\mathbb{R}$ need not be bounded.
    For example, define:
    \begin{equation}
        f(x)=\frac{1}{x^{2}-2}
    \end{equation}
    This is well defined for all elements of $A$. The denominator is zero
    only at $x=\sqrt{2}$, and $\sqrt{2}$ is not a rational number and hence not
    a part of $A$. This function is not bounded, even though it is continuous.
    The difference between the reals and rationals is that the rational numbers
    have lots of \textit{holes} in them, whereas the real numbers do not.
    These details must be taken into consideration for a full valid proof.
    \par\hfill\par
    We can use the extreme value theorem to prove Rolle's theorem.
    \begin{theorem}[Rolle's Theorem]
        If $f:[a,\,b]\rightarrow\mathbb{R}$ is continuous, if $f$ is
        differentiable on $(a,\,b)$, and if $f(a)=f(b)$, then there is a
        point $x_{0}\in(a,\,b)$ such that $f'(x_{0})=0$.
    \end{theorem}
    \begin{proof}
        We split the proof into two cases. Firstly, if $f$ is constant, then
        $f'(x_{0})=0$ for all $x_{0}\in(a,\,b)$, and we're done. Next we handle
        the case where $f$ is not a constant function. There must then be
        points $x\in(a,\,b)$ such that either $f(x)<f(a)$ or $f(x)>f(a)$.
        Suppose the former. The proof is symmetric. By the extreme value theorem
        there is a minimum value $x_{m}\in[a,\,b]$ such that
        $f(x_{m})\leq{f}(x)$ for all $x\in[a,\,b]$. Since there is a value
        $x\in(a,\,b)$ with $f(x)<f(a)$, and since $f(a)=f(b)$, we have that
        $x_{m}\in(a,\,b)$. That is, $x_{m}\ne{a}$ and $x_{m}\ne{b}$. By
        Fermat's theorem, since $x_{m}$ is a global (hence local) minimum,
        we have that $f'(x_{m})=0$, completing the proof.
    \end{proof}
    The major application of Rolle's theorem is the mean value theorem for
    derivatives.
    \begin{theorem}[Mean Value Theorem for Derivatives]
        If $f:[a,\,b]\rightarrow\mathbb{R}$ is continuous, and if
        $f$ is differentiable on $(a,\,b)$, then there is a point
        $x_{0}\in(a,\,b)$ such that:
        \begin{equation}
            f'(x_{0})=\frac{f(b)-f(a)}{b-a}
        \end{equation}
    \end{theorem}
    \begin{proof}
        The function $L(x)$ defined by:
        \begin{equation}
            L(x)=\frac{f(b)-f(a)}{b-a}(x-a)+f(a)
        \end{equation}
        has the property that $L(a)=f(a)$ and $L(b)=f(b)$
        and it is differentiable. The function
        $g(x)=f(x)-L(x)$ is then differentiable on $(a,\,b)$ and
        $g(a)=g(b)=0$. By Rolle's theorem there is a point
        $x_{0}\in(a,\,b)$ such that $g'(x_{0})=0$. But
        $g'(x_{0})=f'(x_{0})-L'(x_{0})$. Solving for $f'(x_{0})$ gives us:
        \begin{equation}
            f'(x_{0})=\frac{f(b)-f(a)}{b-a}
        \end{equation}
        completing the proof.
    \end{proof}
\end{document}
