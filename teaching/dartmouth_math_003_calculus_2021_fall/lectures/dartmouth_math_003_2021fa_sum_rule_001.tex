%-----------------------------------LICENSE------------------------------------%
%   This file is part of Mathematics-and-Physics.                              %
%                                                                              %
%   Mathematics-and-Physics is free software: you can redistribute it and/or   %
%   modify it under the terms of the GNU General Public License as             %
%   published by the Free Software Foundation, either version 3 of the         %
%   License, or (at your option) any later version.                            %
%                                                                              %
%   Mathematics-and-Physics is distributed in the hope that it will be useful, %
%   but WITHOUT ANY WARRANTY; without even the implied warranty of             %
%   MERCHANTABILITY or FITNESS FOR A PARTICULAR PURPOSE.  See the              %
%   GNU General Public License for more details.                               %
%                                                                              %
%   You should have received a copy of the GNU General Public License along    %
%   with Mathematics-and-Physics.  If not, see <https://www.gnu.org/licenses/>.%
%------------------------------------------------------------------------------%
\documentclass{article}
\usepackage{graphicx}                           % Needed for figures.
\usepackage{amsmath}                            % Needed for align.
\usepackage{amssymb}                            % Needed for mathbb.
\usepackage{amsthm}                             % For the theorem environment.

\newtheoremstyle{normal}
    {\topsep}               % Amount of space above the theorem.
    {\topsep}               % Amount of space below the theorem.
    {}                      % Font used for body of theorem.
    {}                      % Measure of space to indent.
    {\bfseries}             % Font of the header of the theorem.
    {}                      % Punctuation between head and body.
    {.5em}                  % Space after theorem head.
    {}

\theoremstyle{normal}
\newtheorem{definition}{Definition}

\theoremstyle{plain}
\newtheorem{theorem}{Theorem}

\title{Differentiation}
\author{Ryan Maguire}
\date{Fall 2021}

% No indent and no paragraph skip.
\setlength{\parindent}{0em}
\setlength{\parskip}{0em}

\begin{document}
    \maketitle
    Let's compute the derivative of a linear function $f(x)=ax+b$ in two ways.
    First, by direct evaluation. We have:
    \begin{align}
        f'(x)&=\lim_{h\rightarrow{0}}\frac{f(x+h)-f(x)}{h}\\
            &=\lim_{h\rightarrow{0}}\frac{a(x+h)+b-(ax+b)}{h}\\
            &=\lim_{h\rightarrow{0}}\frac{ax+ah+b-ax-b}{h}\\
            &=\lim_{h\rightarrow{0}}\frac{ah}{h}\\
            &=\lim_{h\rightarrow{0}}a\\
            &=a
    \end{align}
    This makes sense since the derivative of $f$ at $x$ gives us the
    \textit{slope} of $f$ at $x$. A function $f(x)=ax+b$ has constant slope,
    particularly the slope is $a$. Let's now recompute this using the
    \textit{sum rule}.
    \par\hfill\par
    \begin{theorem}[The Sum Rule for Derivatives]
        If $f:\mathbb{R}\rightarrow\mathbb{R}$ and
        $g:\mathbb{R}\rightarrow\mathbb{R}$ are functions, and if both
        $f$ and $g$ are differentiable at $x_{0}\in\mathbb{R}$, then
        $f+g$ is differentiable at $x_{0}$ and
        \begin{equation}
            (f+g)'(x_{0})=f'(x_{0})+g'(x_{0})
        \end{equation}
    \end{theorem}
    \begin{proof}
        We prove this directly from the limit definition. We have:
        \begin{align}
            (f+g)'(x_{0})
                &=\lim_{h\rightarrow{0}}
                    \frac{\big(f(x+h)+g(x+h)\big)-\big(f(x)+g(x)\big)}{h}\\
                &=\lim_{h\rightarrow{0}}
                    \frac{\big(f(x+h)-f(x)\big)+\big(g(x+h)-g(x)\big)}{h}\\
                &=\lim_{h\rightarrow{0}}\frac{f(x+h)-f(x)}{h}+
                    \lim_{h\rightarrow{0}}\frac{g(x+h)-g(x)}{h}\\
                &=f'(x_{0})+g'(x_{0})
        \end{align}
    \end{proof}
    To prove the derivative of $f(x)=ax+b$ is $f'(x)=a$ we can use the sum
    rule. We have:
    \begin{equation}
        \frac{\textrm{d}(ax+b)}{\textrm{d}x}
            =\frac{\textrm{d}(ax)}{\textrm{d}x}+
                \frac{\textrm{d}(b)}{\textrm{d}x}
            =a\frac{\textrm{d}(x)}{\textrm{d}x}+
                \frac{\textrm{d}(b)}{\textrm{d}x}
    \end{equation}
    The derivative of a constant is zero, and the derivative of $x$ is 1.
    This gives us:
    \begin{equation}
        \frac{\textrm{d}(ax+b)}{\textrm{d}x}=a
    \end{equation}
    \newpage
    I, the copyright holder of this work, release it into the public domain.
    This applies worldwide. In some countries this may not be legally possible;
    if so: I grant anyone the right to use this work for any purpose, without
    any conditions, unless such conditions are required by law.
    \par\hfill\par
    The source code used to generate this document is free software and released
    under version 3 of the GNU General Public License.
\end{document}
