%-----------------------------------LICENSE------------------------------------%
%   This file is part of Mathematics-and-Physics.                              %
%                                                                              %
%   Mathematics-and-Physics is free software: you can redistribute it and/or   %
%   modify it under the terms of the GNU General Public License as             %
%   published by the Free Software Foundation, either version 3 of the         %
%   License, or (at your option) any later version.                            %
%                                                                              %
%   Mathematics-and-Physics is distributed in the hope that it will be useful, %
%   but WITHOUT ANY WARRANTY; without even the implied warranty of             %
%   MERCHANTABILITY or FITNESS FOR A PARTICULAR PURPOSE.  See the              %
%   GNU General Public License for more details.                               %
%                                                                              %
%   You should have received a copy of the GNU General Public License along    %
%   with Mathematics-and-Physics.  If not, see <https://www.gnu.org/licenses/>.%
%------------------------------------------------------------------------------%
\documentclass{article}
\usepackage{graphicx} % Needed for figures.
\usepackage{amsmath}  % Needed for align.
\usepackage{amssymb}  % Needed for mathbb.
\usepackage{amsthm}   % For the theorem environment.

\newtheoremstyle{normal}
    {\topsep}   % Amount of space above the theorem.
    {\topsep}   % Amount of space below the theorem.
    {}          % Font used for body of theorem.
    {}          % Measure of space to indent.
    {\bfseries} % Font of the header of the theorem.
    {}          % Punctuation between head and body.
    {.5em}      % Space after theorem head.
    {}

\theoremstyle{normal}
\newtheorem{definition}{Definition}
\newtheorem{notation}{Notation}
\newtheorem{example}{Example}

\theoremstyle{plain}
\newtheorem{theorem}{Theorem}
\newcommand{\ceil}[2][]{#1\lceil#2#1\rceil}

\title{Limits of Sequences}
\author{Ryan Maguire}
\date{\today}

% No indent and no paragraph skip.
\setlength{\parindent}{0em}
\setlength{\parskip}{0em}

\begin{document}
    \maketitle
    We've discussed what it means for a limit to exist at a point. The precise
    definition is that a function $f$ has a limit $L$ at a point $x_{0}$ if,
    for every $\varepsilon>0$ there exists a $\delta>0$ such that, if $x$ is a
    real number and $0<|x-x_{0}|<\delta$, then it is true that
    $|f(x)-L|<\varepsilon$. In other words, points that are no more than
    $\delta$ away from $x_{0}$ in the $x$ axis have values $f(x)$ that are no
    more than $\varepsilon$ away from $L$ in the $y$ axis. This definition
    works well, but we need to expand on it slightly.
    \par\hfill\par
    Consider the function $f$ given by:
    \begin{equation}
        f(x)=\frac{1}{x}
    \end{equation}
    What happens as $x$ gets really big? In other words, what happens as
    $x$ \textit{tends to infinity}? At first glance, we might say $f(x)$
    approaches zero as $x$ approaches infinity. This makes sense, if $x$ is
    really large, then $1/x$ is really small, so $f(x)=1/x$ is getting closer
    and closer to zero as $x$ approaches infinity. How do we make this precise?
    Our limit definition gives the definition of the limit at a real number
    $x_{0}\in\mathbb{R}$ but $\infty$ is not a real number. More importantly,
    the expression $0<|x-\infty|<\delta$ would never be true for any real
    number $\delta>0$, so how do we fix this?
    \par\hfill\par
    Let's first look at sequences. A sequence is an ordered list of numbers.
    For example 0, 1, 4, 9, 16, 25, and so forth. This is $n^{2}$ for the
    natural numbers $n=0$, $1$, $2$, and so on. To be more precise, a sequence
    is a \textit{function} that takes in an integer and returns a real number.
    We can write this $a:\mathbb{N}\rightarrow\mathbb{R}$. The set
    $\mathbb{N}$ is the set of natural numbers:
    \begin{equation}
        \mathbb{N}=\{\,0,\,1,\,2,\,3,\,4,\,\dots\,\}
    \end{equation}
    And as always, $\mathbb{R}$ denotes the real numbers. There is some special
    notation with sequences, however.
    \begin{notation}
        If $a:\mathbb{N}\rightarrow\mathbb{R}$ is a sequence, and if
        $n\in\mathbb{N}$ is a non-negative integer, then we write:
        \begin{equation}
            a(n)=a_{n}
        \end{equation}
    \end{notation}
    This notation is usually reserved for sequences only. The notation is best
    demonstrated by example.
    \begin{example}
        Let $a:\mathbb{N}\rightarrow\mathbb{R}$ be the sequence defined by:
        \begin{equation}
            a_{n}=n^{2}
        \end{equation}
        Then $a_{0}=0^{2}=0$, $a_{1}=1^{2}=1$, and $a_{2}=2^{2}=4$. If we keep
        writing this sequence out, we get 0, 1, 4, 9, 16, 25, 36, 49, and so on.
    \end{example}
    \begin{example}
        Let $a:\mathbb{N}\rightarrow\mathbb{R}$ be the sequence defined by:
        \begin{equation}
            a_{n}=\frac{1}{n+1}
        \end{equation}
        Using this definition, $a_{0}=\frac{1}{0+1}=1$,
        $a_{1}=\frac{1}{1+1}=\frac{1}{2}$, and so on. Writing out the sequence,
        we'd get 1, $\frac{1}{2}$, $\frac{1}{3}$, $\frac{1}{4}$,
        $\frac{1}{5}$, and so forth.
    \end{example}
    Let's examine this second example. It's very similar to the function
    $f(x)=1/x$, defined for all $x\in(0,\infty)$. When we look at the sequence
    $a_{n}=1/(n+1)$ it's easy to convince ourselves that this sequence
    \textit{tends to zero} as $n$ tends to infinity. How do we precisely say
    this? What does it mean for $a_{n}$ to get arbitrarily close to zero?
    How about for any positive number $\varepsilon>0$, there is an integer
    $N\in\mathbb{N}$ such that, for every larger integer $n>N$ we have that
    $|a_{n}|<\varepsilon$. That is, if $a_{n}$ is getting arbitrarily close to
    zero as $n$ gets bigger, we can make $|a_{n}|$ smaller than $\varepsilon$
    as long as we only consider $n$ large enough. We can use this to give the
    definition of a sequence in general.
    \begin{definition}
        The limit of a sequence $a:\mathbb{N}\rightarrow\mathbb{R}$ is a real
        number $L\in\mathbb{R}$ such that for all $\varepsilon>0$ there exists
        a non-negative integer $N\in\mathbb{N}$ such that for all larger
        integers $n>N$ it is true that $|a_{n}-L|<\varepsilon$.
    \end{definition}
    Let's examine $a_{n}=\frac{1}{n+1}$. We believe, intuitively, that the
    limit is zero. Suppose $\varepsilon=0.1$. We want to
    find a non-negative integer $N$ such that for all larger integers $n>N$ it
    is true that $\frac{1}{n+1}<0.1$. Rewriting $0.1$ as $1/10$, we want an
    $N$ such that $n>N$ implies $\frac{1}{n+1}<\frac{1}{10}$. This last
    inequality seems to indicate that we should simply choose $N=9$. If
    $N=9$ and $n>N$ we have:
    \begin{equation}
        \frac{1}{n+1}<\frac{1}{N+1}=\frac{1}{9+1}=\frac{1}{10}
    \end{equation}
    So, as along as we skip the first 10 integers ($n=0$ through $n=9$), the
    value of $a_{n}=\frac{1}{n+1}$ is closer than $0.1$ to zero. What about
    $\varepsilon=0.01$? Can we formulate an argument for this value? We now
    want a non-negative integer $N\in\mathbb{N}$ such that, if $n$ is a
    non-negative integer and $n>N$, it is true that
    $\frac{1}{n+1}<0.01$. Rewriting $0.01$ as $1/100$, we want
    $\frac{1}{n+1}<\frac{1}{100}$. Again it seems natural, from the formula of
    $a_{n}$, to pick $N=99$. If $N=99$ and $n>N$, we have:
    \begin{equation}
        \frac{1}{n+1}<\frac{1}{N+1}=\frac{1}{99+1}=\frac{1}{100}
    \end{equation}
    So, if we skip the first 100 entries in our sequence, the remaining
    elements have $a_{n}$ closer than $0.01$ to zero. What about
    \textit{any} positive real number $\varepsilon>0$? Well, we want an
    integer $N\in\mathbb{N}$ such that for every $n\in\mathbb{N}$ with $n>N$
    it is true that $|a_{n}|<\varepsilon$. That is, we want to make the
    following chain of inequalities work:
    \begin{equation}
        \frac{1}{n+1}<\frac{1}{N+1}<\frac{1}{N}\leq\varepsilon
    \end{equation}
    If we find an $N$ that makes this true, we may conclude that
    $\frac{1}{n+1}<\varepsilon$, that is, that $|a_{n}|<\varepsilon$. The
    last inequality seems to indicate we should pick $N=1/\varepsilon$. The
    problem is this is almost certainly \textit{not} going to be an integer,
    and $N$ is required to be an integer. So, let's pick the first integer that
    is \textit{greater than or equal to} $1/\varepsilon$. This is called the
    ceiling function and is denoted $\ceil{x}$. So, for example,
    $\ceil{\pi}=4$, $\ceil{\sqrt{2}}=2$, $\ceil{4.1}=5$, $\ceil{3}=3$. We can
    now show that $a_{n}=\frac{1}{n+1}$ tends to zero as $n$ tends to infinity.
    \begin{theorem}
        If $a:\mathbb{N}\rightarrow\mathbb{R}$ is the sequence defined by:
        \begin{equation}
            a_{n}=\frac{1}{n+1}
        \end{equation}
        Then the limit of $a_{n}$ as $n$ tends to infinity is zero. That is:
        \begin{equation}
            \lim_{n\rightarrow\infty}\frac{1}{n+1}=0
        \end{equation}
    \end{theorem}
    \begin{proof}
        Let $\varepsilon>0$ and choose $N=\ceil{1/\varepsilon}$. If $n$ is a
        non-negative integer and $n>N$, then:
        \begin{align}
            |a_{n}-0|&=\Big|\frac{1}{n+1}-0\Big|\tag{Definition of $a_{n}$}\\
                &=\Big|\frac{1}{n+1}\Big|\tag{Subtract Zero}\\
                &=\frac{1}{n+1}\tag{$\frac{1}{n+1}$ is positive for $n>0$}\\
                &<\frac{1}{N+1}\tag{Since $N<n$}\\
                &<\frac{1}{N}\tag{Since $N<N+1$}\\
                &=\frac{1}{\ceil{1/\varepsilon}}\tag{Definition of $N$}\\
                &\leq\frac{1}{1/\varepsilon}
                    \tag{Since $x\leq\ceil{x}$ for all $x$}\\
                &=\varepsilon\tag{Simplify}
        \end{align}
        That is, if $n>N$, then $|a_{n}-0|<\varepsilon$.
    \end{proof}
    \newpage
    I, the copyright holder of this work, release it into the public domain.
    This applies worldwide. In some countries this may not be legally possible;
    if so: I grant anyone the right to use this work for any purpose, without
    any conditions, unless such conditions are required by law.
    \par\hfill\par
    The source code used to generate this document is free software and released
    under version 3 of the GNU General Public License.
\end{document}
