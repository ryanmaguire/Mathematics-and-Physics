%-----------------------------------LICENSE------------------------------------%
%   This file is part of Mathematics-and-Physics.                              %
%                                                                              %
%   Mathematics-and-Physics is free software: you can redistribute it and/or   %
%   modify it under the terms of the GNU General Public License as             %
%   published by the Free Software Foundation, either version 3 of the         %
%   License, or (at your option) any later version.                            %
%                                                                              %
%   Mathematics-and-Physics is distributed in the hope that it will be useful, %
%   but WITHOUT ANY WARRANTY; without even the implied warranty of             %
%   MERCHANTABILITY or FITNESS FOR A PARTICULAR PURPOSE.  See the              %
%   GNU General Public License for more details.                               %
%                                                                              %
%   You should have received a copy of the GNU General Public License along    %
%   with Mathematics-and-Physics.  If not, see <https://www.gnu.org/licenses/>.%
%------------------------------------------------------------------------------%
\documentclass{article}

\title{Differentiation}
\author{Ryan Maguire}
\date{\today}

% No indent and no paragraph skip.
\setlength{\parindent}{0em}
\setlength{\parskip}{0em}

\begin{document}
    \maketitle
    Let's use the sum rule to evaluate the derivative of any polynomial. A
    polynomial is a function $P(x)$ of the form:
    \begin{equation}
        P(x)=\sum_{n=0}^{N}a_{n}x^{n}
    \end{equation}
    This \textit{sigma notation} is notation for a sum. It is equivalent to the
    following:
    \begin{equation}
        P(x)=a_{0}+a_{1}x+a_{2}x^{2}+\cdots+a_{N-1}x^{N-1}+a_{N}x^{N}
    \end{equation}
    The numbers above and below the $\Sigma$ tell us where the sum starts and
    ends, and the formula after the $\Sigma$ is the thing we're summing over.
    \par\hfill\par
    The sum rule for differentiation tells us that if we have two
    differentiable functions $f$ and $g$, then $f+g$ is differentiable and the
    derivative is given by:
    \begin{equation}
        (f+g)'(x)=f'(x)+g'(x)
    \end{equation}
    By applying this rule twice, we can show that if $f$, $g$, and $h$ are
    differentiable functions, then:
    \begin{equation}
        (f+g+h)'(x)=f'(x)+g'(x)+h'(x)
    \end{equation}
    And by applying the sum rule $N$ times, if $f_{0}$, $f_{1}$, ...,
    $f_{N}$ are differentiable functions, then:
    \begin{equation}
        \frac{\textrm{d}}{\textrm{d}x}\sum_{n=0}^{N}f_{n}(x)
        =\sum_{n=0}^{N}\frac{\textrm{d}f_{n}}{\textrm{d}x}(x)
    \end{equation}
    Let's use this for a polynomial. The power rule says that if
    $f(x)=x^{n}$, then $f'(x)=nx^{n-1}$. Combining this, for a polynomial
    $P$, the derivative is:
    \begin{equation}
        P'(x)=\sum_{n=0}^{N}na_{n}x^{n-1}
    \end{equation}
    The case $n=0$ is the derivative of the constant term of the polynomial,
    which is zero, so we can just skip that and start the sum at 1. We get:
    \begin{equation}
        P'(x)=\sum_{n=1}^{N}na_{n}x^{n-1}
    \end{equation}
    Let's apply this to real examples. Suppose $P(x)=1+2x+3x^{2}$. The
    derivative is then computed as $2+6x$. This follows directly from the
    above formula. Suppose we have $Q(x)=1+x+x^{2}+x^{3}+x^{4}$. The derivative
    is $1+2x+3x^{2}+4x^{3}$.
    \newpage
    I, the copyright holder of this work, release it into the public domain.
    This applies worldwide. In some countries this may not be legally possible;
    if so: I grant anyone the right to use this work for any purpose, without
    any conditions, unless such conditions are required by law.
    \par\hfill\par
    The source code used to generate this document is free software and released
    under version 3 of the GNU General Public License.
\end{document}
