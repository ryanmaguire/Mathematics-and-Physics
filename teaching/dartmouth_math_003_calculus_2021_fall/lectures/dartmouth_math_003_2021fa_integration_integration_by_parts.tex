%-----------------------------------LICENSE------------------------------------%
%   This file is part of Mathematics-and-Physics.                              %
%                                                                              %
%   Mathematics-and-Physics is free software: you can redistribute it and/or   %
%   modify it under the terms of the GNU General Public License as             %
%   published by the Free Software Foundation, either version 3 of the         %
%   License, or (at your option) any later version.                            %
%                                                                              %
%   Mathematics-and-Physics is distributed in the hope that it will be useful, %
%   but WITHOUT ANY WARRANTY; without even the implied warranty of             %
%   MERCHANTABILITY or FITNESS FOR A PARTICULAR PURPOSE.  See the              %
%   GNU General Public License for more details.                               %
%                                                                              %
%   You should have received a copy of the GNU General Public License along    %
%   with Mathematics-and-Physics.  If not, see <https://www.gnu.org/licenses/>.%
%------------------------------------------------------------------------------%
\documentclass{article}
\usepackage{graphicx} % Needed for figures.
\usepackage{amsmath}  % Needed for align.
\graphicspath{{../../../images/}}

\title{Integration by Parts}
\author{Ryan Maguire}
\date{\today}

% No indent and no paragraph skip.
\setlength{\parindent}{0em}
\setlength{\parskip}{0em}

\begin{document}
    \maketitle
    $u$ substitution is the reverse of the chain rule, validated by the
    fundamental theorem of calculus. Integration by parts is the reverse of the
    product rule. It says, if $f$ and $g$ are differentiable functions, then:
    \begin{align}
        \int_{a}^{b}f(x)g'(x)\;\textrm{d}x
            &=f(x)g(x)\big|_{a}^{b}-\int_{a}^{b}f'(x)g(x)\;\textrm{d}x\\
            &=\Big(f(b)g(b)-f(a)g(a)\Big)-\int_{a}^{b}f'(x)g(x)\;\textrm{d}x
    \end{align}
    This is often written (though, slightly abusing notation), as follows:
    \begin{equation}
        \int_{a}^{b}f\;\textrm{d}g=fg\big|_{a}^{b}-\int_{a}^{b}g\;\textrm{d}f
    \end{equation}
    Before we give a rigorous proof, let's give a geometric and intuitive one.
    Suppose $f$ is a function with inverse $f^{-1}$. We can write
    $y=f(x)$ and $x=f^{-1}(y)$. The area under the curve $f$ from $a$ to $b$
    can be computed via the figure below. We compute the area under $y=f(x)$
    by noting this area, plus the area to the left of $x=f^{-1}(y)$, plus the
    area of the grey rectangle, is equal to the area of the large rectangle
    with $x$ values 0 to $b$ and $y$ values 0 to $f(b)$. If we know the area to
    the left of the curve $x=f^{-1}(y)$, we can compute the area under $f$ as
    follows:
    \begin{equation}
        \int_{a}^{b}y\;\textrm{d}x
            =f(b)b-f(a)a-\int_{f(a)}^{f(b)}x\;\textrm{d}y
    \end{equation}
    See the image below for a visual.
    \begin{figure}
        \centering
        \includegraphics{integration_by_parts.pdf}
        \caption{Visual for Integration by Parts}
    \end{figure}
    This idea is best remembered via the abuse-of-notation equation. The
    indefinite integral is:
    \begin{equation}
        \int{f}\;\textrm{d}g=fg-\int{g}\;\textrm{d}f
    \end{equation}
    Now you ask \textit{what if} $f$ \textit{is not invertible?} Well, the
    picture isn't quite as nice. So let's prove this is true in general.
    The product rule says that if $f$ and $g$ are differentiable functions,
    then:
    \begin{equation}
        (fg)'(x)=f'(x)g(x)+f(x)g'(x)
    \end{equation}
    By the fundamental theorem of calculus, integrating gives us:
    \begin{align}
        \int_{a}^{b}\Big(f'(x)g(x)+f(x)g'(x)\Big)\;\textrm{d}x
            &=\int_{a}^{b}(fg)'(x)\;\textrm{d}x\\
            &=(fg)(b)-(fg)(a)\\
            &=f(b)g(b)-f(a)g(a)
    \end{align}
    Rearranging, we get:
    \begin{equation}
        \int_{a}^{b}f(x)g'(x)\;\textrm{d}x
            =\Big(f(b)g(b)-f(a)g(a)\Big)-\int_{a}^{b}f'(x)g(x)\;\textrm{d}x
    \end{equation}
    Now let's use it. Integration by parts works well when we see products of
    functions. Let's integrate $x\sin(x)$. We have a choice when trying to
    integrate this: Do we set $f(x)=x$ and $g'(x)=\sin(x)$, or
    $f(x)=\sin(x)$ and $g'(x)=x$? We should set $f$ to be the function that's
    gets simpler when we differentiate it. In this case, $f(x)=x$ looks ideal
    because differentiating gives us $f'(x)=1$. Then we set $g'(x)=\sin(x)$ and
    compute the anti-derivative, which is $g(x)=-\cos(x)$. We have:
    \begin{align}
        \int{x}\sin(x)\;\textrm{d}x
            &=\int{f}(x)g'(x)\;\textrm{d}x\\
            &=f(x)g(x)-\int{f}'(x)g(x)\;\textrm{d}x\\
            &=-x\cos(x)-\int{1}\big(-\cos(x)\big)\;\textrm{d}x\\
            &=-x\cos(x)+\int\cos(x)\;\textrm{d}x\\
            &=-x\cos(x)+\sin(x)+C
    \end{align}
    We can differentiate to verify our answer:
    \begin{align}
        \frac{\textrm{d}}{\textrm{d}x}\Big(-x\cos(x)+\sin(x)+C\Big)
            &=\frac{\textrm{d}}{\textrm{d}x}\Big(-x\cos(x)\Big)
                +\frac{\textrm{d}}{\textrm{d}x}\Big(\sin(x)\Big)\\
            &=-\cos(x)+x\sin(x)+\cos(x)\\
            &=x\sin(x)
    \end{align}
    As expected.
    \newpage
    I, the copyright holder of this work, release it into the public domain.
    This applies worldwide. In some countries this may not be legally possible;
    if so: I grant anyone the right to use this work for any purpose, without
    any conditions, unless such conditions are required by law.
    \par\hfill\par
    The source code used to generate this document is free software and released
    under version 3 of the GNU General Public License.
\end{document}
