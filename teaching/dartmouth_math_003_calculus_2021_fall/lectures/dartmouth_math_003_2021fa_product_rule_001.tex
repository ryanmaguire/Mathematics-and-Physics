%-----------------------------------LICENSE------------------------------------%
%   This file is part of Mathematics-and-Physics.                              %
%                                                                              %
%   Mathematics-and-Physics is free software: you can redistribute it and/or   %
%   modify it under the terms of the GNU General Public License as             %
%   published by the Free Software Foundation, either version 3 of the         %
%   License, or (at your option) any later version.                            %
%                                                                              %
%   Mathematics-and-Physics is distributed in the hope that it will be useful, %
%   but WITHOUT ANY WARRANTY; without even the implied warranty of             %
%   MERCHANTABILITY or FITNESS FOR A PARTICULAR PURPOSE.  See the              %
%   GNU General Public License for more details.                               %
%                                                                              %
%   You should have received a copy of the GNU General Public License along    %
%   with Mathematics-and-Physics.  If not, see <https://www.gnu.org/licenses/>.%
%------------------------------------------------------------------------------%
\documentclass{article}
\usepackage{graphicx}                           % Needed for figures.
\usepackage{amsmath}                            % Needed for align.
\usepackage{amssymb}                            % Needed for mathbb.
\usepackage{amsthm}                             % For the theorem environment.

\newtheoremstyle{normal}
    {\topsep}               % Amount of space above the theorem.
    {\topsep}               % Amount of space below the theorem.
    {}                      % Font used for body of theorem.
    {}                      % Measure of space to indent.
    {\bfseries}             % Font of the header of the theorem.
    {}                      % Punctuation between head and body.
    {.5em}                  % Space after theorem head.
    {}

\theoremstyle{normal}
\newtheorem{definition}{Definition}

\theoremstyle{plain}
\newtheorem{theorem}{Theorem}

\title{Differentiation}
\author{Ryan Maguire}
\date{Fall 2021}

% No indent and no paragraph skip.
\setlength{\parindent}{0em}
\setlength{\parskip}{0em}

\begin{document}
    \maketitle
    The product rule allows us to easily compute the derivative of a function
    $h(x)=f(x)g(x)$ if we already know how to differentiate $f$ and $g$.
    The rule says $(fg)'(x)=f'(x)g(x)+f(x)g'(x)$. The proof of this requires
    the following fact: A differentiable function is continuous. The intuition
    behind differentiable and continuous should imply this. A continuous
    function is one that has no \textit{jumps}, whereas a differentiable
    function is one where if you zoom in enough, the function starts to
    look like a straight line (the tangent line). It seems natural that
    differentiable should imply continuous. Let's prove this.
    \par\hfill\par
    Differentiable implies the following limit exists:
    \begin{equation}
        \lim_{x\rightarrow{x}_{0}}\frac{f(x)-f(x_{0})}{x-x_{0}}
    \end{equation}
    Let's label this limit $L$. By the definition of limit, that means for
    all $\varepsilon>0$ there is a $\delta_{0}>0$ such that:
    \begin{equation}
        0<|x-x_{0}|<\delta_{0}\Rightarrow
        \Big|\frac{f(x)-f(x_{0})}{x-x_{0}}-L\Big|<\varepsilon
    \end{equation}
    Remember that the $\Rightarrow$ symbol reads as \textit{implies}. So,
    $0<|x-x_{0}|<\delta_{0}$ implies that
    $|(f(x)-f(x_{0}))/(x-x_{0})-L|<\varepsilon$. This implies
    $|f(x)-f(x_{0})-L(x-x_{0})|<\varepsilon|x-x_{0}|$ by simply
    multiplying the inequality by $|x-x_{0}|$ on both sides. Thus:
    \begin{equation}
        -\varepsilon|x-x_{0}|<f(x)-f(x_{0})-L(x-x_{0})<\varepsilon|x-x_{0}|
    \end{equation}
    Adding $L(x-x_{0})$ to all sides, we get:
    \begin{equation}
        L(x-x_{0})-\varepsilon|x-x_{0}|<f(x)-f(x_{0})
            <L(x-x_{0})+\varepsilon|x-x_{0}|
    \end{equation}
    Using the laws of the absolute value symbol we can reduce this to
    \begin{equation}
        -|x-x_{0}|(|L|+\varepsilon)<f(x)-f(x_{0})<|x-x_{0}|(|L|+\varepsilon)
    \end{equation}
    This is equivalent to:
    \begin{equation}
        |f(x)-f(x_{0})|<|x-x_{0}|(|L|+\varepsilon)
    \end{equation}
    To show that $f$ is continuous, we need $|x-x_{0}|$ being small to
    imply $|f(x)-f(x_{0})|$ is small. From this last inequality,
    $|f(x)-f(x_{0})|<|x-x_{0}|(|L|+\varepsilon)$, we should choose
    $|x-x_{0}|$ smaller than $\varepsilon/(|L|+\varepsilon)$. To make the
    above argument work we also restricted to $|x-x_{0}|<\delta_{0}$. So we
    need to pick the minimum of these two.
    \begin{theorem}
        If $f:\mathbb{R}\rightarrow\mathbb{R}$ is a function that is
        differentiable at $x_{0}\in\mathbb{R}$, then $f$ is continuous
        at $x_{0}$.
    \end{theorem}
    \begin{proof}
        Like every continuity proof, let $\varepsilon>0$. Since $f$ is
        differentiable at $x_{0}$, the limit
        \begin{equation}
            \lim_{x\rightarrow{x}_{0}}\frac{f(x)-f(x_{0})}{x-x_{0}}
        \end{equation}
        exists. Label this limit $L$. By the definition of limit, since
        $\varepsilon>0$, there is a $\delta_{0}>0$ such that
        \begin{equation}
            0<|x-x_{0}|<\delta_{0}\Rightarrow
            \Big|\frac{f(x)-f(x_{0})}{x-x_{0}}-L\Big|<\varepsilon
        \end{equation}
        Choose $\delta=\min(\delta_{0},\varepsilon/(|L|+\varepsilon))$.
        If $|x-x_{0}|<\delta$, then we have the following:
        \begin{align}
            &\Big|\frac{f(x)-f(x_{0})}{x-x_{0}}-L\Big|<\varepsilon\\
            \Rightarrow&|f(x)-f(x_{0})-L(x-x_{0})|<\varepsilon|x-x_{0}|\\
            \Rightarrow&-\varepsilon|x-x_{0}|<
                f(x)-f(x_{0})-L(x-x_{0})<\varepsilon|x-x_{0}|\\
            \Rightarrow&
                L(x-x_{0})-\varepsilon|x-x_{0}|<
                    f(x)-f(x_{0})<L(x-x_{0})+\varepsilon|x-x_{0}|\\
            \Rightarrow&
                -|x-x_{0}|(|L|+\varepsilon)<f(x)-f(x_{0})
                    <|x-x_{0}|(|L|+\varepsilon)\\
            \Rightarrow&
                |f(x)-f(x_{0})|<|x-x_{0}|(|L|+\varepsilon)\\
            \Rightarrow&
                |f(x)-f(x_{0})|<\delta(|L|+\varepsilon)\\
            \Rightarrow&
                |f(x)-f(x_{0})|<\frac{\varepsilon}{|L|+\varepsilon}
                    (|L|+\varepsilon)\\
            \Rightarrow&|f(x)-f(x_{0})|<\varepsilon
        \end{align}
        That is, $f$ is continuous at $x_{0}$.
    \end{proof}
    If the proof is too complex, don't worry. The important thing is the
    result: Differentiable implies continuous. Another way to see this is by
    rephrasing continuity as follows. A function that is continuous at $x_{0}$
    is a function such that $\lim_{x\rightarrow{x_{0}}}(f(x)-f(x_{0}))=0$. Using
    this definition, we have that if $f$ is differentiable at $x_{0}$, then
    \begin{align}
        \lim_{x\rightarrow{x}_{0}}\big(f(x)-f(x_{0})\big)
        &=\lim_{x\rightarrow{x}_{0}}\frac{f(x)-f(x_{0})}{x-x_{0}}(x-x_{0})\\
        &=\lim_{x\rightarrow{x}_{0}}\frac{f(x)-f(x_{0})}{x-x_{0}}
            \lim_{x\rightarrow{x_{0}}}(x-x_{0})\\
        &=f'(x_{0})\lim_{x\rightarrow{x}_{0}}(x-x_{0})\\
        &=0
    \end{align}
    Of course, this \textit{proof} really sweeps the $\epsilon-\delta$
    definition under the rug. The $\varepsilon-\delta$ definition is still
    needed to rigorously define limit in the first place.
    \par\hfill\par
    Anyways, let's use this.
    \begin{theorem}[The Product Rule for Derivatives]
        If $f:\mathbb{R}\rightarrow\mathbb{R}$ and
        $g:\mathbb{R}\rightarrow\mathbb{R}$ are functions, and if $f$ and $g$
        are differentiable at $x_{0}\in\mathbb{R}$, then $fg$ is differentiable
        at $x_{0}$ and:
        \begin{equation}
            (fg)'(x)=f'(x)g(x)+f(x)g'(x)
        \end{equation}
    \end{theorem}
    \begin{proof}
        We have:
        \begin{align}
            &\lim_{h\rightarrow{0}}
                \frac{f(x+h)g(x+h)-f(x)g(x)}{h}\nonumber\\
            &=\lim_{h\rightarrow{0}}\frac{f(x+h)g(x+h)+0-f(x)g(x)}{h}\\
            &=\lim_{h\rightarrow{0}}
                \frac{f(x+h)g(x+h)+\big(-f(x+h)g(x)+f(x+h)g(x)\big)-f(x)g(x)}
                     {h}\\
            &=\lim_{h\rightarrow{0}}
                \frac{\big(f(x+h)g(x+h)-f(x+h)g(x)\big)
                      +\big(f(x+h)g(x)-f(x)g(x)\big)}{h}\\
            &=
                \lim_{h\rightarrow{0}}
                \frac{f(x+h)g(x+h)-f(x+h)g(x)}{h}+
                \lim_{h\rightarrow{0}}
                    \frac{f(x+h)g(x)-f(x)g(x)}{h}\\
            &=\lim_{h\rightarrow{0}}
                f(x+h)\frac{g(x+h)-g(x)}{h}+
                \lim_{h\rightarrow{0}}g(x)
                    \frac{f(x+h)-f(x)}{h}\\
                    &=\lim_{h\rightarrow{0}}
                        f(x+h)\frac{g(x+h)-g(x)}{h}+
                        g(x)\lim_{h\rightarrow{0}}
                        \frac{f(x+h)-f(x)}{h}\\
                    &=\lim_{h\rightarrow{0}}
                        f(x+h)\frac{g(x+h)-g(x)}{h}+
                        g(x)f'(x)
        \end{align}
        Now to simplify the expression on the left. This is where
        continuity comes in. Since $f$ is differentiable, it is continuous.
        But then $f(x+h)$ tends to $f(x)$ as $h$ approaches zero. Using this, we
        have:
        \begin{align}
            &\lim_{h\rightarrow{0}}
                f(x+h)\frac{g(x+h)-g(x)}{h}+g(x)f'(x)\nonumber\\
            &=f(x)\lim_{h\rightarrow{0}}\frac{g(x+h)-g(x)}{h}+g(x)f'(x)\\
            &=f(x)g'(x)+g(x)f'(x)\\
            &=f'(x)g(x)+f(x)g'(x)
        \end{align}
        The last step just uses commutativity of addition and multiplication.
        That is, the fact that $a+b=b+a$ and $a\cdot{b}=b\cdot{a}$.
    \end{proof}
    The product rule allows us to prove the \textit{power rule}. Let's first
    examine $f(x)=x^{2}$. We can differentiate this using the product rule
    by writing $f(x)=x\cdot{x}$. The derivative is then:
    \begin{equation}
        \frac{\textrm{d}f}{\textrm{d}x}(x)=
        \frac{\textrm{d}(x)}{\textrm{d}x}x+x\frac{\textrm{d}(x)}{\textrm{d}x}
        =x+x=2x
    \end{equation}
    We can differentiate $f(x)=x^{3}$ as well by writing
    $f(x)=x^{2}\cdot{x}$. We then have:
    \begin{equation}
        \frac{\textrm{d}f}{\textrm{d}x}(x)=
        \frac{\textrm{d}(x^{2})}{\textrm{d}x}x
            +x^{2}\frac{\textrm{d}(x)}{\textrm{d}x}
        =(2x)x+x^{2}=3x^{2}
    \end{equation}
    By continuing we can prove, for any positive integer $n$, that the
    derivative of $f(x)=x^{n}$ is $f'(x)=nx^{n-1}$. This is the
    \textit{power rule}. It turns out that for any $r\in\mathbb{R}$, if
    $f(x)=x^{r}$, then $f'(x)=rx^{r-1}$. So in particular, if
    $f(x)=\sqrt{x}=x^{1/2}$, we have
    $f'(x)=\frac{1}{2}x^{-1/2}=\frac{1}{2\sqrt{x}}$.
    \newpage
    I, the copyright holder of this work, release it into the public domain.
    This applies worldwide. In some countries this may not be legally possible;
    if so: I grant anyone the right to use this work for any purpose, without
    any conditions, unless such conditions are required by law.
    \par\hfill\par
    The source code used to generate this document is free software and released
    under version 3 of the GNU General Public License.
\end{document}
