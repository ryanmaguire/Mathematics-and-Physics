%-----------------------------------LICENSE------------------------------------%
%   This file is part of Mathematics-and-Physics.                              %
%                                                                              %
%   Mathematics-and-Physics is free software: you can redistribute it and/or   %
%   modify it under the terms of the GNU General Public License as             %
%   published by the Free Software Foundation, either version 3 of the         %
%   License, or (at your option) any later version.                            %
%                                                                              %
%   Mathematics-and-Physics is distributed in the hope that it will be useful, %
%   but WITHOUT ANY WARRANTY; without even the implied warranty of             %
%   MERCHANTABILITY or FITNESS FOR A PARTICULAR PURPOSE.  See the              %
%   GNU General Public License for more details.                               %
%                                                                              %
%   You should have received a copy of the GNU General Public License along    %
%   with Mathematics-and-Physics.  If not, see <https://www.gnu.org/licenses/>.%
%------------------------------------------------------------------------------%
\documentclass{article}
\usepackage{amsmath}
\usepackage{geometry}
\geometry{margin=1.0in}
\title{MIT PRIMES: Fresnel Inversion and the NASA Cassini Mission}
\author{Ryan Maguire}
\date{\today}

% No indent and no paragraph skip.
\setlength{\parindent}{0em}
\setlength{\parskip}{0em}

\begin{document}
    \maketitle
    \section{The Problem}
        When a spacecraft transmits a radio signal to Earth, it usually goes
        uninterrupted, traveling through the vacuum of space until it enters
        the Earth's atmosphere. If the observer records that the signal has
        dipped in power, or vanishes entirely, then they may rightly conclude
        that some object now falls between the spacecraft and Earth. This is
        called an \textbf{occultation}. Certain objects in the solar system
        can be studied in great detail using occultation observations, which is
        precisely how the Cassini space probe studied the rings of Saturn.
        \par\hfill\par
        When radio waves pass through Saturn's rings they are diffracted, and
        the observer on Earth records a diffraction pattern given by
        Huygen's principle \cite{Goodman1969IntroductionTF}:
        \begin{equation}
            \hat{T}(\rho_{0},\,\phi_{0})
            =\int_{0}^{\infty}\int_{0}^{2\pi}
                T(\rho,\,\phi)\,
                K(\rho,\,\phi,\,\rho_{0},\,\phi_{0})\,
                \rho\,\textrm{d}\rho\,\textrm{d}\phi
        \end{equation}
        where $\hat{T}$ is the \textit{diffracted transmission} (the measured
        quantity), $T$ is the actual transmission, and $K$ is a geometric
        quantity known as the Fresnel kernel. The function $T$ is what
        describes the rings, and we must solve this inverse problem if we
        wish to study them. Using some numerical approximations, and the
        cylindrical symmetry of the rings, one can simplify this to:
        \begin{equation}
            \hat{T}(\rho_{0})
            =\int_{0}^{\infty}T(\rho)\,
                K^{\prime}(\rho,\,\phi_{s},\,\rho_{0},\,\phi_{0})\,
                \rho\,\textrm{d}\rho
        \end{equation}
        where $\phi_{s}$ is the \textbf{stationary value} of $\phi$, and
        $K^{\prime}$ is a modified Fresnel kernel. In the classic Fresnel
        approximation we use a complex exponential with a quadratic to write:
        \begin{equation}
            \hat{T}(\rho_{0})
            =\int_{0}^{\infty}
            T(\rho)\exp\left(
                \frac{i\pi}{2}\left(\frac{\rho-\rho_{0}}{F}\right)^{2}
            \right)\,\textrm{d}\rho
        \end{equation}
        where $F$ is the so-called \textbf{Fresnel scale}
        \cite{Marouf1982TheoryOR}.
        Invoking the convolution theorem tells us how to solve for $T$ exactly.
        Let $\mathcal{F}_{\xi}(f)$ denote the Fourier transform of $f$,
        $f*g$ denote the convolution of $f$ and $g$, and define
        $f(x)=\exp(i\pi{x}^{2}/2F^{2})$. We have:
        \begin{align}
            \mathcal{F}_{\xi}(\hat{T})
            &=\mathcal{F}_{\xi}\left(
                \int_{0}^{\infty}
                T(\rho)\exp\left(
                    \frac{i\pi}{2}\left(\frac{\rho-\rho_{0}}{F}\right)^{2}
                \right)\,\textrm{d}\rho
            \right)\\
            &=\mathcal{F}_{\xi}(T*f)\\
            &=\mathcal{F}_{\xi}(T)\,\mathcal{F}_{\xi}(f)\\
            \Longrightarrow
            \mathcal{F}_{\xi}(T)
            &=\frac{\mathcal{F}_{\xi}(\hat{T})}{\mathcal{F}_{\xi}(f)}\\
            \Longrightarrow
            T(\rho)
            &=\mathcal{F}^{-1}_{\rho}\left(
                \frac{\mathcal{F}_{\xi}(\hat{T})}{\mathcal{F}_{\xi}(f)}
            \right)
        \end{align}
        where $\mathcal{F}^{-1}_{\rho}$ denotes the inverse Fourier transform.
        \par\hfill\par
        Unfortunately, reality is far more difficult,
        and the Fresnel approximation does not hold. This is the
        case for much of the data obtained by Cassini. Here we must explore
        other numerical methods to properly analyze the rings of Saturn.
    \section{What to Expect}
        By participating in the research project, you should expect to study
        Fourier analysis, numerical analysis, partial differential equations,
        numerical Euclidean geometry, elementary physics, and optics. You will
        also get to work with \textbf{real data from the NASA Cassini mission},
        and construct your own profiles of the rings of Saturn. To do this you
        will need to develop some comfort with the C programming language, and
        with Python. You are not expected to be an expert programmer or
        applied mathematician: this will be taught to you.
    \par\hfill\par
    \bibliographystyle{plain}
    \bibliography{../../bib.bib}
\end{document}
