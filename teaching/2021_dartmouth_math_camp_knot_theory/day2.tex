\documentclass[12pt,letterpaper]{article}
\usepackage{fullpage}
\usepackage[top=1.5cm, bottom=2cm, left=2.5cm, right=2.5cm]{geometry}
\usepackage{amsmath,amsthm,amsfonts,amssymb,amscd,dsfont}
\usepackage{lastpage}
\usepackage{enumerate}
\usepackage{fancyhdr}
\usepackage{mathrsfs}
\usepackage{xcolor}
\usepackage{graphicx}
\usepackage{listings}
\usepackage{hyperref}

\newcommand{\R}{\mathds{R}}
\hypersetup{%
  colorlinks=true,
  linkcolor=blue,
  linkbordercolor={0 0 1}
}


% Edit these as appropriate
\newcommand\course{}
\newcommand\hwnumber{1}                  % <-- homework number
\usepackage{dsfont}

\usepackage{pgfplots}
\pgfplotsset{compat=1.13}
\usepackage{mathrsfs}
\usetikzlibrary{arrows}
\usepackage{bm}

\pagestyle{fancyplain}

\chead{{\Large \textbf{Math camp, day 2}}}
\rhead{}
\lfoot{}
\cfoot{}
\rfoot{\small\thepage}
\headsep 1.5em

\begin{document}

    \section*{Objectives}
    \begin{enumerate}
        \item Students will be able to \textbf{apply} Reidermeister moves to find knots that are equivalent to a given knot.
        \item Students will be able to \textbf{define} a link and \textbf{recognize} the difference between links and knots.
        \item Students will be able to \textbf{define} virtual knots diagrams and \textbf{give examples} of them.
    \end{enumerate}
	\section{Reidermeister Moves}
	    We just learned about equivalence of knots. The problem is that picturing things in 3D and finding the equivalences might be hard. We have a solution using knot diagrams!
	    
	    As we learned in Kameron's lecture, we can think of knots as their diagram in two dimensions. For example:
	    \begin{center}
	        \includegraphics[scale=0.3]{knot1.jpeg}
	    \end{center}
	    Earlier today, we learned that the two following knots are equivalent. 
	    \textbf{Make an example with shoelaces and draw it on the ipad.}
	    
	    The question is: what moves can we perform that do not change the "knot".
	    
	   
	    (Here, we will see an example with a shoelace that I will show on camera.)
	    
	    \textit{5 minutes}
	    
	    The movements are
	    \begin{enumerate}
	        \item \includegraphics[scale=0.2]{knot2.jpeg}
	        
	        \item \includegraphics[scale=0.2]{knot3.jpeg}
	        
	        \item \includegraphics[scale=0.2]{knot4.jpeg}
	    \end{enumerate}
	    
	    These are called the Reidermeister moves. You may ask why are they special since there are definitely more moves that don't change the knot. This is because of the following theorem
	  
	    \textbf{Theorem:} If two knots are equivalent, their diagrams are related by a sequence of Reidermeister moves.
	    
	    
	    \textbf{We need to explain why are these valid moves in the sense of Kameron's lecture:}
	    
	    \includegraphics[scale=0.3]{knotRM.jpeg}
	    
	    The other two moves are an exercise, unless we have extra time.
	    
	    \textbf{Caution: Some movements are not allowed!} For example:
	    
	    Make an example with the trefoil
	    
	    \textit{12 minutes}
	    
	    \textbf{Example:}  
	    
	    \begin{center}
	        \includegraphics[scale=0.3]{knot1.jpeg}
	    \end{center}
	    
	    Is the unknot because we can perform move number 1.
	    
	    
	    \textbf{Example:} The one from Kameron's notes
	    
	    \textbf{Example:}
	    \begin{center}
	        \includegraphics[scale=0.3]{knot5.jpeg}
	    \end{center}
	    
	    \textit{20 minutes}
	    
	    \textbf{Practice activity (see Worksheet: Reidermeister moves)}
	    
	    \textbf{Note:} If everything went too fast, we can give students more time for this activity
	     
	    \textit{40 minutes}
	    
	    \newpage
	    \section{Links}
	    
	    We have been studying knots, but we can also have different knots that are connected, this is why we need links. 
	    
	    \textbf{Definition:} A link is a collection of knots which may be linked (or knotted) together.
	    
	    \textbf{Examples:}
	    
	    \begin{center}
	    Hopf link
	    
	        \includegraphics[scale=0.3]{link1.jpeg}
	        
	   Whitehead link
	   
	        \includegraphics[scale=0.3]{link4.jpeg}
	        
        Borromean Ring
        
        \includegraphics[scale=0.1]{link5.png}
        
        
	        \includegraphics[scale=0.3]{link2.jpeg}
	        
	        \includegraphics[scale=0.3]{link3.jpeg}
	    \end{center}
	    
	\textit{50 minutes}
	
	\textit{Break: 65 minutes}
	
	\section{Virtual Knots}
	
	\textbf{Challenge:} Try to draw a knot picture with no crossings that is not a picture of the unknot.
	
	\textit{68 minutes}
	
	We cannot do this! Is there a better place to draw knot pictures, so we can draw harder knots that have no crossings? Example: a torus knot. \textbf{We need the plastic torus to see this}.
	
	\hspace{1cm}\includegraphics[scale=0.4]{extorus.jpeg}
	
	Using a torus, we can draw the trefoil knot without any crossings!
	
	\includegraphics[scale=0.3]{trefiol.png}
	
	\textit{65 minutes}
	
	But do we really want to draw tori or more surfaces? You just saw me struggling with it, we want to draw things in 2D for as long as we can.
	
    This is why we add a \textbf{virtual crossing}:
    
    \includegraphics[scale=0.4]{vk1.jpeg}
    
	\textbf{Definition:} A \textbf{virtual knot diagram} is described by a knot picture, as before, but with a second type of crossing, a virtual crossing, indicated by encircling the vertex (just as in the example before).
	
	\textbf{Example:}
	How would we draw a virtual knot diagram for the trefoil knot that we saw drawn in the torus?
	
	\textit{70 minutes}
	
	For you to know, we can use generalized Reidermeister moves to find equivalences of virtual knot diagrams. We can use the old moves, but we can also do:
	
	\includegraphics[scale=0.5]{rm1.png}\hspace{1cm}\includegraphics[scale=0.5]{rm2.png}
	
	\textit{75 minutes}
	
	\textit{End of the day survey: 15 minutes}
\end{document}
