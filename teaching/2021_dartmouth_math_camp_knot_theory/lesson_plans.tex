\documentclass{article}
\newtheorem{definition}{Definition}
%Kameron: I know next to nothing about formatting, but I think its best we have put our lesson plans in the same place. If anyone wants to add anything to this document please feel free to do so.
\begin{document}

\begin{section}{Day 1 Introduction}
Hello everyone! Thank you for signing up for math camp. We would like to start by taking a minute to introduce ourselves. My name is Kameron McCombs. I am a third year PhD student in the math department. Rest of introductions follow. This week we are going to be talking about knots and graphs! Yay! Don't worry if you don't know any of the terms yet. We will take you through the basics. We would like to show you a side of math that you don't really get to see in elementary and high school. Hopefully as we get through the week you will start to see how different areas of math can be, and maybe get a sense of what people are doing currently. At any point in time if you have any questions please feel free to ask!

\end{section}


\begin{section}{Day 2 Lecture 1: Knots}
Learning Objectives:


	Kids will know the definition of a knot.
	
	
	Kids will know see how we come up with definitions.
	
	
	Kids will see different examples of knots.
	
	
	Kids will know the basic idea behind an equivalence relation.
	
	
	Kids will know the basic idea of a deformation and how that relates to equivalence of knots.\medbreak
	
	
	Today we are going to move to a different topic: knots. When you hear the word knot you might think of something like a knot tied in your shoelaces or a knot tied in a rope. How do we go from this intuitive thought of what a knot is to a precise mathematical definition? Why do we even care?\medbreak
	
	 Well, let's look at some examples of knots we would like to have and some knots that we would not like to have.
	 \textbf{Examples} Trefoil knot(Ryan has 3d and 2d images). Torus knots(have these drawn on Ipad). Pretzel knots(Have these drawn on Ipad).
	 
	 It is important to note that our knots exist in three-dimensional space. However, it is often convenient to depict the knot in two-dimensional space. Our two-dimensional depiction of a knot is called a knot diagram. Are we losing any information by doing this? No, because the only three-dimensional data we need is at each crossing, and our two-dimensional sketch tells us that by the over/under crossings. Note that if we twist the knot around in three-dimensional space we can get different knot diagrams for the same knot!
	 \textbf{Example} Give an example using the trefoil knot of different projections.
	 
	 \textbf{Example} Give an example of a wild knot.
	 
	 This is an example of something we don't want to consider. Why? Well, we are trying to model a physical problem with our knots, and there's no way we could actually construct a knot like this with our shoelaces right?\medbreak
	 
	 How do we make a definition of a knot that includes the examples we like and doesn't include the examples we don't like? We would like for our knot to begin and end at the same point, and we want it to be clear, no matter what knot diagram we choose, at every crossing which part of the knot is crossing over and which is crossing under. Is this enough? No, because the wild knot from before is included under this definition. Then we must restrict the definition more. What should we add? This is a little tricky, and we will see we need some technical way to get around this. Before we do that, let's start an activity to help us make our own definitions.\medbreak
	 
	 \textbf{Activity: Making Definitions} Often in math it seems like we are just being told what to do by the people that came before us, but they had to come up with the definitions right!? It is actually very common that we make our own definitions to categorize objects for our purposes. This is very important in mathematics. So now we are going to split up into breakout rooms. Each group will receive the same set of diagrams. Your goal is to separate a group of diagrams that are like each other and not like the rest of the diagrams. Come up with a name for them in your group(can be nonsensical). Then try to come up with a precise mathematical definition for your group's diagrams. You want to help other people decide if a given diagram would fit into your group's chosen diagrams. At the end we will reconvene in the main room and discuss the definitions each group made. \medbreak
	 
	 \textbf{Activity Guide for Proctors} The students are asked to find distinguishing characteristics of the diagrams. We have specifically avoided using the word knot so that we can throw in diagrams of things that are not knots like links. Possible answers for definitions are things like some knots have this number of crossings. Some knots have crossings that are unnecessary(think figure 8 is an unknot). Some knots have different loops(they might discover that links are different than knots). As the proctor please let the students come up with definitions. If they are getting absolutely nowhere start with the number of crossings as a hint and see if you can foster discussion. Otherwise just try to take their idea and format it in a suitable definition. At the end I will pull everyone into the main room and share the definitions. Afterwards I will ask if any of the diagrams should not be a knot.
	 
	 Now let's return to our problem. Remember, we are trying to figure out how to make a definition that distinguishes the knots we want from wild knots. Does anyone have any ideas? Instead of looking at smooth curves, why don't we just restrict to line segments between points? Are we losing any information by doing this? Does this eliminate the wild knot problem?  If we let $[p,q]$ denote the line segment between points $p$ and $q$, then we can construct a polygonal curve by taking an ordered set of points $(p_1,...,p_n)$ and looking at the connection of the line segments $[p_1,p_2], ..., [p_{n-1},p_n], [p_n,p_1]$. 
	 \textbf{Example} Show the example Ryan has of a polygonal knot.
	 
	 This is called a closed polygonal curve because it starts and ends at the same point. If each line segment only touches other line segments at the endpoints, then this is known as a simple polygonal curve. Putting this all together, we have the following definition:\medbreak
	 
	 \begin{definition}
	 A knot is a simple closed polygonal curve in three-dimensional space.
	 \end{definition}
	 
	 What happens in our definition if $p_2$ is on the line segment $[p_1,p_3]$? If we remove $p_2$ does anything change? This means that we can define the same knot in different ways. What if we have two different knots? Are these knots really all that different?
	 \textbf{Example} Show already drawn trefoil knots that have different points. 
	 
	 Well what are the important pieces of a knot? The crossings right? If I take this part of the knot and move it around does that change any of the important aspects of it? Ok, so if we think that some knots really are the same then how do we distinguish between different knots? Given a knot $(p_1,...,p_n)$, we can deform this knot by adding a point $p_0$ to the beginning as long as $p_0$ is not on the line between $p_1$ and $p_n$, and such that the new knot $(p_0,p_1,...,p_n)$ does not have any new crossings.
	 \textbf{Example} Show an example of a deformation that is allowed. Then show an example of a deformation that is not allowed. The deformation that is not allowed is because we are changing from an under-over crossing to an over-under crossing(pulling the string through itself). Show another example of a "deformation" that isn't allowed by shrinking knot to a point.
	 \textbf{Does anyone have any questions?}
	 
	 We say that two knots are equivalent if we can find a combination of deformations that turn one knot into the other. So now we are left with some questions. How do we tell if two knots are the same? How do we tell if two knots are different? We will see a way to start answering these questions in the next lecture.\medbreak
	 
	 \textbf{If we are running out of time move this to the beginning of Seifert surfaces}
	 Ok before we move to a break, we have one last topic to discuss. Notice that when we defined a knot we used a collection of points $(p_1,...,p_n)$. Did it matter what order we listed the points? Yes, because if we switch two points, we switch what lines are drawn between the points. However, we can change the order in some ways. For instance what about this knot: $(p_2,...,p_n,p_1)$. If we trace this we get the same knot. Notice that we are going the same "direction" through the knot. What about $(p_n,p_{n-1},...,p_1)$? It's the same knot, but we are tracing it "backwards" right? At least compared to how we were just tracing it. Can we do this for any knot? We can do this for any knot we choose. We call the direction that we traverse the knot its orientation. Does this remind anyone of anything? \textbf{Does anyone have any questions?} 

\end{section}

\begin{section}{Day 3 Lecture 2: Seifert Surfaces}
Learning Objectives:

The students will be able to construct a Seifert surface from a knot diagram.

The students will be able to compute the genus of the above Seifert surface of a given knot diagram.

The students will be able to compute the genus of a knot in at least 1 instance of a knot diagram(trefoil knot).\medbreak

Hello everyone! In this lecture we will be talking about Seifert surfaces. We just discussed the idea of a knot invariant. This is a really usefult tool to tell if two knots are different. However, it can often be hard to calculate these invariants. That's where Seifert surfaces come in. They are useful for calculating some invariants of knots. So what is a Seifert surface?

\begin{definition}
For a given oriented knot, we say an orientable surface is a Seifert surface if the boundary of the surface is the knot.
\end{definition}

\textbf{Ask for questions}

Given a knot, do you think we can always find a Seifert surface? If we can find a Seifert surface will it be unique? It turns out that we can always find a Seifert surface for an oriented knot.

\textbf{Example} Draw two Seifert surfaces for a knot diagram that have different genuses.

\textbf{Note: Follow along with an example as you describe this}
Ok, now let's actually show that if we have a knot we can create a Seifert surface for it. First we will start with the oriented knot diagram. Pick any point on the circle and trace along it in the direction of the knot. When you hit a crossing, mark that you are at a crossing, switch to the other arc, and continue tracing in the direction of the knot. When you are retracing the same path, repeat this process with a point on the knot diagram that you haven't traced yet. Notice that we have broken up the knot into disjoint circles. These are called Seifert circles.\medbreak

Now our goal is to connect these circles to form a surface. Notice that some circles are inside of other circles. If this is the case, we can just lift the inner circle so that it lies above the outer circle in three-dimensional space. Now let's fill in the empty space inside of the circles to form disjoint surfaces. Now we need to conect these surfaces. At each point where we marked a crossing from the original diagram, we will now attach a rectangular surface. We want to twist the rectangle so that it will mimic the original crossing. In the original knot diagram at each crossing we have four different points. These will correspond to the corners of the rectangle, with corners separated by the long side representing an arc on the knot. Attach two of the corners to the first circle. We might notice that to connect the other side to the other circle we must twist the rectangle. The way we twist it must correspond to which arc was the over crossing and which was the under crossing. Afterwards, we attach rectangles to every circle corresponding to each crossing. The resulting object is a surface whose boundary is the original knot. Ok so we have a way of constructing a Seifert surface for any oriented knot. Is it possible to find a different Seifert surface for a knot? As it turns out, we can. Even worse, the surfaces might not have the same genus. 

\textbf{Example} Unknot (Show that we can just add a genus to the surface by adding a torus).

\textbf{Example} Left Trefoil

\textbf{Example} Right Trefoil

\textbf{Example} Something that has an inner circle(Use the textbook example)



\textbf{Activity} Break students up into groups and give each one of them a different oriented knot diagram. Ask each group to construct a Seifert surface from the given diagram. Now we can come back into the main room to discuss. Afterwards rotate the diagrams so that each group has a different Seifert surface than the one they constructed. Ask the groups to find the genus of the Seifert surface. Ask them if it is possible to deduce the genus of the knot(optional). This is going to be a difficult exercise so it might be necessary for the proctor to draw the surface for the students(or ask the students to present a drawing to the proctor and have the proctor fix it).

\textbf{Proctor Guide} This might be a difficult exercise. You might have them annotate a whiteboard or you might have to draw things for them if they are getting stuck. Be very helpful as it is confusing! A strategy for organization that might work is controlling the figure writing. Ask the students to one at a time try to draw a Seifert circle and then ask the other students to evaluate. Don't be afraid to nudge them to get them started and please ask if they are confused!

\end{section}

\begin{section}{Day 5 Lecture 2: Applications(Gauss Code)}

Learning Objectives: Students will be able to construct a knot using Gauss code.\medbreak

Students will be able to write the Gauss code of a knot diagram.\medbreak

Hi everyone! Today we are going to discuss some applications of knot theory. We will start with Gauss codes for knots. Gauss code is a way to encode an oriented knot diagram in a way that another person can use to reconstruct the diagram. This is also how computers handle knots. How would we go about encoding an oriented knot diagram? Let's take a look at a particular knot diagram:

\textbf{Example} Look at this knot diagram(Written already in Onenote). Let's first pick any point on the diagram. Then we follow using the orientation. Every time we reach a crossing we label it: For the first crossing we label it 1. The next time we reach a new crossing we label it 2. Every time we reach a new crossing we label it 1 more than the previous crossing. Repeat this until we have labeled every crossing with a number. Now restart at the beginning point and traverse the diagram again. Every time we reach a crossing we look at the number of the crossing and whether we are on the over strand or the under strand. If we are on the over strand we record + the number. If we are on the under strand we record - the number.\medbreak

If we start with a Gauss code how do we come up with an oriented knot diagram? Well, we first pick a point and start there. Then we create a crossing for each number, where we are on an over strand if the number is positive, and an under strand if the number is negative. Does this produce an equivalent diagram? Unfortunately it does not. As it stands that's a pretty big disadvantage right? How can we fix this? We should try to encode more information in the notation! Ok how should we do it?\medbreak

Extended Gauss notation will allow us to recreate equivalent oriented knot diagrams by adding data for the handedness of a knot.  \medbreak

So now we will amend our rules for creating a Gauss code. Starting at a point P, traverse the diagram according to the orientation. When we encounter a crossing for the first time we will record a number the way we did previously. Whenever we encounter a crossing again, we will assign a - number if the crossing is right-handed and a + number if the crossing is left-handed.

\textbf{Activity} We are going to split up into our usual groups. Each group will be given a different oriented knot diagram. Your goal will be to find the Extended Gauss code for the given diagram. You will have 5 minutes to do this. Afterwards, we will join the main room again and we will rotate the Gauss codes to different groups. Then we will go back into breakout rooms and each group will draw an oriented knot diagram with their new Gauss codes. You will have 5 minutes to do this. Afterwards we will join the main room again and compare the knot diagrams to their original counterparts. If we need to fill out more time we will rotate these diagrams again to another group and then separate again. This time their goal will be to take 5-10 minutes to see if they can show the knot diagrams are equivalent.

\end{section}

\end{document}