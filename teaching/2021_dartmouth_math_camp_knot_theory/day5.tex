\documentclass{beamer}
\usepackage[utf8]{inputenc}
\usepackage{caption}
\usepackage{subcaption}
\usepackage{amsmath}
\usepackage{tikz}
\newcommand{\floor}[2][]{#1\lfloor#2#1\rfloor}

\title{Math Camp Day 5}
\author{Ben Adenbaum, Juanita Duque-Rosero, Ryan Maguire, Kameron McCombs}
\date{June 2021}
\usenavigationsymbolstemplate{}
\setbeamertemplate{footline}[frame number]
\begin{document}
    \maketitle
    \begin{frame}{Outline}
        \begin{itemize}
            \item Recap of the week.
            \item Tait Graphs
            \item Gauss code.
            \item The Hyperbolic plane.
            \item Making knots out of different surfaces.
        \end{itemize}
    \end{frame}
    \begin{frame}{Genus of a Graph}
        On day 1 we talked about how any graph can be drawn on some higher
        genus surface without crossings. The proof involves adding
        \textit{coffee mug handles} to the picture every time we see a crossing.
        \begin{figure}
            \centering
            \includegraphics{figs/graph_has_genus_002.eps}
            \caption{A graph without crossings}
            \label{fig:graph_has_genus_002}
        \end{figure}
    \end{frame}
    \begin{frame}{Genus of a Graph}
        We also saw that this algorithm does \textit{not} give
        us the genus of the graph, only an upper bound for the number of
        holes needed. The graph before is planar! If we did this with $K_{5}$
        shown below we'd add 5 holes, but only need one (a torus).
        \begin{figure}
            \centering
            \includegraphics{figs/complete_graph_K5.eps}
            \caption{$K_{5}$}
            \label{fig:complete_graph_K5}
        \end{figure}
    \end{frame}
    \begin{frame}{Genus of a Graph}
        The genus of a graph is the \textit{smallest} genus of a surface needed
        to draw the graph without crossings. If we take $K_{5}$, we can draw it on
        the flat torus like this.
        \begin{figure}
            \centering
            \includegraphics{figs/complete_graph_K5_on_flat_torus.eps}
            \caption{$K_{5}$ on a torus}
            \label{fig:complete_graph_K5_on_flat_torus}
        \end{figure}
    \end{frame}
    \begin{frame}{Dual Graph}
        Let's redraw $K_{5}$ in the following way so we can talk about the
        dual graph. This is obtained by placing a vertex at every faces,
        and connecting two vertices if the original faces shared an edge.
        \begin{figure}
            \centering
            \includegraphics{figs/complete_graph_k5_on_flat_torus_002.eps}
            \caption{$K_{5}$ on a torus}
            \label{fig:complete_graph_K5_on_flat_torus2}
        \end{figure}
    \end{frame}
    \begin{frame}{Dual Graph}
        It's hard to tell what a face is on the torus, so we zoom out and make
        copies of the picture to help us.
        \begin{figure}
            \centering
            \includegraphics{figs/complete_graph_k5_on_torus_plane_cover_002.eps}
            \caption{$K_{5}$ on a torus}
            \label{fig:complete_graph_k5_on_torus_plane_cover_002}
        \end{figure}
    \end{frame}
    \begin{frame}{Dual Graph}
        Remember, the blue and red lines are not actually there, they are only to help
        us understand life on the torus. Similar to how the edge of a map isn't the
        edge of the world. If you got to the edge, you'd wrap around to the other side.
        To help visualize, we remove these lines.
        \begin{figure}
            \centering
            \includegraphics{figs/complete_graph_k5_on_torus_plane_cover_003.eps}
            \caption{$K_{5}$ on a torus}
            \label{fig:complete_graph_k5_on_torus_plane_cover_003}
        \end{figure}
    \end{frame}
    \begin{frame}{Dual Graph}
        This central square comprised of an octagon and 4 triangles is called a
        \textit{fundamental domain}. It has all of the points in the torus, and
        it also has all of the faces for $K_{5}$.
        \begin{figure}
            \centering
            \includegraphics{figs/complete_graph_k5_on_torus_plane_cover_003.eps}
            \caption{$K_{5}$ on a torus}
            \label{fig:complete_graph_k5_on_torus_plane_cover_003b}
        \end{figure}
    \end{frame}
    \begin{frame}{Dual Graph}
        From this, we can construct the \textit{dual graph}.
        \begin{figure}
            \centering
            \includegraphics{figs/dual_graph_k5_on_torus_001.eps}
            \caption{Dual graph for $K_{5}$ on a torus.}
            \label{fig:dual_graph_k5_on_torus_001}
        \end{figure}
    \end{frame}
    \begin{frame}{Dual Graph}
        If we zoom out, we get the following.
        \begin{figure}
            \centering
            \includegraphics{figs/dual_graph_k5_on_torus_plane_cover_001.eps}
            \caption{Dual graph for $K_{5}$ on a torus.}
            \label{fig:dual_graph_k5_on_torus_plane_cover_001}
        \end{figure}
    \end{frame}
    \begin{frame}{Dual Graph}
        Removing the red and blue guide lines, we get the following in which we
        can see the faces again.
        \begin{figure}
            \centering
            \includegraphics{figs/dual_graph_k5_on_torus_plane_cover_002.eps}
            \caption{Dual graph for $K_{5}$ on a torus.}
            \label{fig:dual_graph_k5_on_torus_plane_cover_002}
        \end{figure}
    \end{frame}
    \begin{frame}{Dual Graph}
        Note that the dual graph for this graph is once again $K_{5}$ on a torus.
        \begin{figure}
            \centering
            \includegraphics{figs/complete_graph_k5_on_flat_torus_002.eps}
            \caption{$K_{5}$ on a torus}
            \label{fig:complete_graph_K5_on_flat_torus3}
        \end{figure}
    \end{frame}
    \begin{frame}{Dual Graph}
        This phenomenon works for all graphs on the appropriate surface.
        $K_{7}$ can be drawn on a torus, and we can consider it's dual graph.
        \begin{figure}
            \centering
            \includegraphics[scale=0.17]{figs/k7_dual_graph_torus.png}
            \caption{$K_{7}$ and it's dual}
            \label{fig:k7_dual_graph_torus}
        \end{figure}
    \end{frame}
    \begin{frame}{Orienting a Knot}
        Given a knot, there are always two ways to \textit{orient} it.
        Below is the unoriented trefoil, and next to it is the trefoil
        with one of the orientations selected.
        \begin{figure}
            \centering
            \begin{subfigure}[b]{0.49\textwidth}
                \includegraphics{figs_for_kameron/trefoil_knot_diagram.eps}
                \caption{Unoriented Trefoil}
                \label{fig:knot_graph_trefoil_001}
            \end{subfigure}
            \begin{subfigure}[b]{0.49\textwidth}
                \includegraphics{figs/trefoil_knot_diagram_oriented.eps}
                \caption{Oriented Trefoil}
                \label{fig:trefoil_knot_diagram_oriented}
            \end{subfigure}
        \end{figure}
    \end{frame}
    \begin{frame}{Sign of a Crossing}
        An orientation allows us to sign the crossings. If we've
        oriented our knot (or our link), we can always tilt our head in
        such a way that the picture looks like one of the following.
        \begin{figure}
            \centering
            \begin{subfigure}[b]{0.49\textwidth}
                \includegraphics{figs/positive_crossing.eps}
                \caption{Positive Crossing}
                \label{fig:positive_crossing}
            \end{subfigure}
            \begin{subfigure}[b]{0.49\textwidth}
                \includegraphics{figs/negative_crossing.eps}
                \caption{Negative Crossing}
                \label{fig:negative_crossing}
            \end{subfigure}
        \end{figure}
    \end{frame}
    \begin{frame}{Tait Graph}
        And with that, we can now move on to the Tait graph.
        This construction shows us how to get a graph from a knot, and a
        knot from a graph. It helps turn graph theory problems into knot theory
        problems and vice-versa. This is especially useful for knot theorist, since
        graph theory is a computationally rich subject with vast libraries of computer
        code to solve problems, and knot theorists can use these libraries to perform
        computations for their own work.
    \end{frame}
    \begin{frame}{Tait Graph}
        We talked about invarients like tricoloring and knot genus, but these
        invariants are slightly weak since they cannot distinguish many knots apart.
        Stronger invariants like the \textit{Alexander Polynomial} and
        \textit{Jones Polynomial} can distinguish most of the first few thousand knots,
        but their constructions make them hard for computers to compute since they usually
        involve the knot diagram. How do we tell a computer what a diagram is? Luckily, via the Tait graph,
        computing these polynomials amounts to working with trees, like the ones we saw on
        day 1. 
    \end{frame}
    \begin{frame}{Tait Graph}
        We start with a simple link, the Hopf Link, and we give it an
        orientation. We then pick any face we like (there are 4 in this figure) and place
        a vertex in that face. We then look at the crossings that touch this face, and to
        get another vertex we go over this crossing to the next face.
        \begin{figure}
            \centering
            \includegraphics{figs/hopf_link_diagram_oriented.eps}
            \caption{Oriented Hopf Link}
            \label{fig:hopf_link_diagram_oriented}
        \end{figure}
    \end{frame}
    \begin{frame}{Tait Graph}
        We continue doing this until we've gone through every crossing and filled out all of
        the corresponding faces. For the Hopf link, we end up with the following
        \begin{figure}
            \centering
            \includegraphics{figs/hopf_link_tait_graph_001.eps}
            \caption{Vertices of the Tait Graph for the Hopf Link}
            \label{fig:hopf_link_tait_graph_001}
        \end{figure}
    \end{frame}
    \begin{frame}{Tait Graph}
        We then connect the vertices by drawing edges through the crossings, and
        \textit{signing} the edges. We sign the edge based on the sign of the crossing.
        Red for negative, blue for positive.
        \begin{figure}
            \centering
            \includegraphics{figs/hopf_link_tait_graph_002.eps}
            \caption{Tait Graph for the Hopf Link}
            \label{fig:hopf_link_tait_graph_002}
        \end{figure}
    \end{frame}
    \begin{frame}{Tait Graph}
        For the Hopf Link, we are left with a digon. As we mentioned the other
        day, the dual graph of a digon is again a digon. So the Hopf Link has a unique
        Tait graph, but for most knots there are 2 choices. It all depends on which face
        we choose for our first vertex. We end up with a checkerboard coloring of the plane,
        and if we chose a different face we may end up with the dual graph of our original Tait graph.
    \end{frame}
    \begin{frame}{Tait Graph}
        Let's look at some more examples. First, the trefoil. There are five faces.
        If we choose the center one, the face adjacent through the crossings is the
        outer face.
        \begin{figure}
            \centering
            \includegraphics{figs_for_kameron/Trefoil_Knot.eps}
            \caption{Trefoil Knot}
            \label{fig:trefoil_knot_diagramb}
        \end{figure}
    \end{frame}
    \begin{frame}{Tait Graph}
        The result is 2 vertices with 3 edges between them, all of which are colored blue
        since the crossings are positive.
        \begin{figure}
            \centering
            \includegraphics{figs/tait_graph_trefoil_001.eps}
            \caption{Tait Graph of the Trefoil}
            \label{fig:tait_graph_trefoil_001}
        \end{figure}
    \end{frame}
    \begin{frame}{Tait Graph}
        Next, let's try the figure eight knot.%
        \footnote{This figure is not my own work, and was created by Jim Belk in POV-Ray.}
        \begin{figure}
            \centering
            \includegraphics[scale=0.4]{figs/figure_eight_knot.png}
            \caption{Figure-8 Knot}
            \label{fig:figure_eight_knot}
        \end{figure}
    \end{frame}
    \begin{frame}{Tait Graph}
        Depending on how we oriented the graph, and which face we chose,
        we may get something like this:
        \begin{figure}
            \centering
            \includegraphics{figs/tait_graph_figure_8_001.eps}
            \caption{Tait Graph of the Figure 8}
            \label{fig:tait_graph_figure_8_001}
        \end{figure}
    \end{frame}
    \begin{frame}{Tait Graph}
        In breakout rooms, try to solve the following two problems:
        \begin{enumerate}
            \item What is the other Tait graph for the trefoil?
            \item What is the other Tait graph for the figure-eight?
        \end{enumerate}
    \end{frame}
    \begin{frame}{Tait Graph}
        The dual for the figure-8 is the same, but the Trefoil changes.
        \begin{figure}
            \centering
            \includegraphics{figs/tait_graph_trefoil_002.eps}
            \caption{Dual Tait Graph for the Trefoil}
            \label{fig:tait_graph_trefoil_002}
        \end{figure}
    \end{frame}
    \begin{frame}{Tait Graph}
        To go from graph to knot simply requires taking the
        medial graph of the graph and constructing to corresponding
        knot or link. In practice it is often easiest to place crossings
        of the appropriate sign at each edge, and then try to fill in the arcs
        so that the vertices become faces of the knot.
    \end{frame}
    \begin{frame}{Tait Graph}
        Below is an example for the Hopf Link.
        \begin{figure}
            \centering
            \includegraphics{figs/tait_graph_hopf_link_002.eps}
            \caption{Tait Graph of Hopf Link}
            \label{fig:tait_graph_hopf_link_002}
        \end{figure}
    \end{frame}
    \begin{frame}{Tait Graph}
        If we try to enclose the two dots to make faces, we end up with the Hopf link.
        \begin{figure}
            \centering
            \includegraphics{figs/hopf_link_diagram.eps}
            \caption{Hopf Link}
            \label{fig:hopf_link_diagram}
        \end{figure}
    \end{frame}
    \begin{frame}{Tait Frame}
        A final note before moving on to the last exercise. We've seen
        that the digon is the Hopf link and the triangle is the trefoil.
        To add to the pattern, what is a single vertex? It would be unknot.
        What about a line segment? The infinity knot (which is equivalent to the unknot).
        How about a square?
        \begin{figure}
            \centering
            \includegraphics{figs/tait_graph_solomon_link_001.eps}
            \caption{Tait Graph}
            \label{fig:tait_graph_solomon_link_001}
        \end{figure}
    \end{frame}
    \begin{frame}{Tait Graph}
        You end up with the Solomon link.%
        \footnote{On Tuesday I accidentally called this the Whitehead Link}
        \footnote{This graphic was placed in the public domain by an anonymous source.
                  It is written in the PostScript Vector Graphics language}
        What's the pattern?
        \begin{figure}
            \centering
            \includegraphics[scale=0.15]{figs/Solomon_Link.png}
            \caption{Solomon Link}
            \label{fig:Solomon_Link}
        \end{figure}
    \end{frame}
    \begin{frame}{Tait Graph}
        The infinity knot is the boundary of the rectangle with 1 twist.
        The Hopf link, 2 twists. The trefoil, 3 twists. The Solomon link, 4 twists.
        We saw all of these on Tuesday!
    \end{frame}
    \begin{frame}{Tait Graph}
        As a final exercise, what are the Reidemeister moves in terms of the Tait graph?
    \end{frame}
    \begin{frame}{Tait Graph}
        Reidemeister 1 says you can get rid of \textit{leaves} on the graph.
        \begin{figure}
            \centering
            \includegraphics{figs/tait_graph_reidemeister_1.eps}
            \caption{Reidemeister 1 in the Tait Graph}
            \label{fig:tait_graph_reidemeister_1}
        \end{figure}
    \end{frame}
    \begin{frame}{Tait Graph}
        Reidemeister 2 says you can remove \textit{reducible} vertices if
        the edges alternate in sign.
        \begin{figure}
            \centering
            \includegraphics{figs/tait_graph_reidemeister_2_001.eps}
            \caption{Reidemeister 2 in the Tait Graph}
            \label{fig:tait_graph_reidemeister_2}
        \end{figure}
    \end{frame}
    \begin{frame}{Tait Graph}
        Reidemeister 3 says you can perform a \textit{Y-Delta} move. You can transform
        a Y into a $\Delta$.
        \begin{figure}
            \centering
            \includegraphics{figs/tait_graph_reidemeister_3_001.eps}
            \caption{Reidemeister 3 in the Tait Graph}
            \label{fig:tait_graph_reidemeister_3}
        \end{figure}
    \end{frame}
    \begin{frame}{Tait Graph}
        Before moving on, here's Peter Tait, whom Tait graph's are named after.
        He's also the founder of modern knot theory.
        \begin{figure}
            \centering
            \includegraphics[scale=0.3]{figs/Tait_Peter_Guthrie.jpg}
            \caption{Peter Tait}
            \label{fig:Tait_Peter_Guthrie}
        \end{figure}
    \end{frame}
    \begin{frame}{Hyperbolic Tilings}
        We talked about life on a torus quite a bit, and how we
        can represent the torus by squares and hexagons tiling the plane.
        We also talked about higher genus surfaces like the genus 2 and
        genus 3 surfaces for $K_{8}$ and higher. The genus $g$ surfaces
        can be represented by a $4g$ sided polygon, but the hexagon is the
        polygon with the most sides that can tile the plane. How do we
        represent higher genus surfaces?
    \end{frame}
    \begin{frame}{Hyperbolic Tilings}
        The answer is the \textit{hyperbolic plane}. In this plane, a
        lot of weird things can happen. The defining characteristic of this
        plane is the following:
        \begin{itemize}
            \item Given a line, and a point not on the line, there are
                infinitely many parallel lines through that point.
        \end{itemize}
        Contrast this with the usual plane in which there is only
        \textit{one} parallel line.
    \end{frame}
    \begin{frame}{Hyperbolic Tilings}
        For every $n\geq{3}$, you can tile the hyperbolic plane
        with an $n$ sided polygon. Below is the octagonal tiling.
        \begin{figure}
            \centering
            \includegraphics[scale=0.4]{figs/8_3_tiling.png}
            \caption{Octagonal Tiling of the Hyperbolic Plane}
            \label{fig:8_3_tiling}
        \end{figure}
    \end{frame}
    \begin{frame}{Knotting Other Surfaces}
        Can we knot other surfaces? The sphere? The
        two holed donut? The sphere can't be knotted
        in a \textit{tame} way, but it can be knotted in a
        \textit{wild} way. This is the \textit{Alexander Horned-Sphere}.
        \begin{figure}
            \centering
            \includegraphics[scale=0.1]{figs/Alexander_horned_sphere.png}
            \caption{Alexander horned sphere}
            \label{fig:Alexander_horned_sphere}
        \end{figure}
    \end{frame}
    \begin{frame}{Knotting Other Surfaces}
        If we want to draw the sphere in 3 dimensional space in a
        \textit{tame} way, it turns out everything is the same. There is
        only one way to draw a sphere, as far as topology is concerned.
        What about knots in \textit{four} dimensions?
    \end{frame}
    \begin{frame}{Knotting Other Surfaces}
        In 4 dimensions there is enough wiggle room that you always
        undo crossings. Every knot in 4D is the same. So instead of
        trying to knot closed curves in 4D, let's try to knot other surfaces
        in 3D.
        \begin{figure}
            \centering
            \includegraphics[scale=0.15]{figs/top4.jpeg}
            \caption{Are these the same?}
            \label{fig:top4}
        \end{figure}
    \end{frame}
    \begin{frame}{Knotting Other Surfaces}
        Yes, they are the same!
        \begin{figure}
            \centering
            \includegraphics[scale=0.15]{figs/top2.jpeg}
            \caption{Undoing the tangle}
            \label{fig:top2}
        \end{figure}
    \end{frame}
    \begin{frame}{Knotting Other Surfaces}
        What about these?
        \begin{figure}
            \centering
            \includegraphics[scale=0.15]{figs/top3.jpeg}
            \caption{More Tangles}
            \label{fig:top3}
        \end{figure}
    \end{frame}
    \begin{frame}{Knotting Other Surfaces}
        \begin{figure}
            \centering
            \includegraphics[scale=0.15]{figs/top1.jpeg}
            \caption{They're the same}
            \label{fig:top1}
        \end{figure}
    \end{frame}
\end{document}
