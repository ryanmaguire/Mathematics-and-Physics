\documentclass[12pt,letterpaper]{article}
\usepackage{fullpage}
\usepackage[top=1.5cm, bottom=2cm, left=2.5cm, right=2.5cm]{geometry}
\usepackage{amsmath,amsthm,amsfonts,amssymb,amscd,dsfont}
\usepackage{lastpage}
\usepackage{enumerate}
\usepackage{fancyhdr}
\usepackage{mathrsfs}
\usepackage{xcolor}
\usepackage{graphicx}
\usepackage{listings}
\usepackage{hyperref}

\newcommand{\R}{\mathds{R}}
\hypersetup{%
  colorlinks=true,
  linkcolor=blue,
  linkbordercolor={0 0 1}
}


% Edit these as appropriate
\newcommand\course{}
\newcommand\hwnumber{1}                  % <-- homework number
\usepackage{dsfont}

\usepackage{pgfplots}
\pgfplotsset{compat=1.13}
\usepackage{mathrsfs}
\usetikzlibrary{arrows}
\usepackage{bm}
\setlength\parindent{0pt}

\pagestyle{fancyplain}

\chead{{\Large \textbf{Worksheet: Reidermeister moves}}}
\rhead{}
\lfoot{}
\cfoot{}
\rfoot{\small\thepage}
\headsep 1.5em

\begin{document}
\textcolor{white}{hi}

In this worksheet you are going to practice performing Reidermeister moves to find equivalent knots. As an example, the following knots are equivalent because we can use the Reidermeister move number 1 to get from one to the other.

\begin{center}
    \includegraphics[scale=0.3]{knoteq.jpeg}
\end{center}

\textbf{Exercise:} For each of the following knots, perform one Reidermeister move at a time until you get the knot picture of the unknot or the trefoil knot:
\begin{center}
    \includegraphics[scale=0.3,trim={3cm 0cm 0cm 0cm},clip]{basic.jpeg}
\end{center}

\textbf{Hint:} You can try making the knots with a shoelace to check your answers.

\begin{center}
    \includegraphics[scale=0.3,trim={0.2cm 0 0 0},clip]{problems.jpeg}
\end{center}

\newpage

\section*{Instructions for breakout rooms facilitators:}

\begin{itemize}

    \item To increase participation, please ask students to first make the know with shoelaces and decide if they will get the unknot or a trefoil. Whenever you are drawing the solution, you can ask students one by one for a move that they want to perform. Since the order does not matter, they can choose whatever.
    
    \item Make sure that you do one move at a time and you clearly label what move are you making. Reminder:
    
    \includegraphics[scale=0.4]{moves.png}
    
    \item You might need to draw for people in a whiteboard. You can always ask them to annotate and mark what do they want to move.
    
    \item Please remind people that you can perform moves in different orders and that is okay. In the solutions, I made a choice on the order I wanted, but that doesn't mean that their solutions are wrong.
    
    \item If people are stuck, you can make the knot with a shoelace, which tells you if you should end up with the unknot or the trefoil knot. This also helps with checking your answers.
    
    \item The most challenging part for students could be to recognize that they are done and they got the trefoil knot. You can help them with that if they look stuck.
\end{itemize}


\textbf{If there is extra time:}
Ask students to knot an unknot using Reidermeister moves. When I tell you, please send me a screenshot of just the final knot and I will send you another knot to unknot.

\section*{Solutions}
\begin{center}
    \includegraphics[trim={0.2cm 0 0 0},clip, scale=0.3]{solution1.jpeg}
    
    \includegraphics[scale=0.3]{solution2.jpeg}
    
    \includegraphics[scale=0.33]{solution3.jpeg}
    
    \includegraphics[scale=0.38]{solution4.jpeg}
\end{center}
\end{document}
