\begin{problem}
    The axiom of infinity tells us $\mathbb{N}=\{\,0,\,1,\,2,\,\dots\,\}$
    exists. It does not tell us
    $\mathbb{Z}=\{\,\dots,\,-2,\,-1,\,0,\,1,\,2,\,\dots\,\}$ exists, but we
    can construct it. For this problem we may assume addition $(+)$ for
    natural numbers exists and has its familiar properties such as
    commutativity, associativity, and the cancellation law
    ($a+b=a+c$ implies $b=c$).
    \begin{itemize}
        \item
            (2 Points)
            Consider the set $\mathbb{N}\times\mathbb{N}$.
            Define $R$ to be the relation $(a,\,b)R(c,\,d)$ if and only if
            $a+d=b+c$. Prove this is an equivalence relation.
        \item
            (1 Point)
            Describe the equivalence class of $(a,\,b)$ geometrically
            (when viewing $\mathbb{N}\times\mathbb{N}$ as
            a subset of $xy$ plane).
        \item
            (2 Points)
            For equivalence classes $[(a,\,b)]$ and $[(c,\,d)]$, define:

            \begin{equation}
                [(a,\,b)]+[(c,\,d)]
                =
                [(a+c,\,b+d)]
            \end{equation}

            Prove this is well-defined.
        \item
            (1 Point)
            With this construction we now write (for convenience)

            \begin{equation}
                \begin{array}{rcl}
                    \displaystyle
                    0
                    \!\!
                    &=&
                    \!\!
                    \displaystyle
                    [(0,\,0)]\\[1.5em]
                    \displaystyle
                    n
                    \!\!
                    &=&
                    \!\!
                    \displaystyle
                    [(n,\,0)]\\[1.5em]
                    \displaystyle
                    -n
                    \!\!
                    &=&
                    \!\!
                    \displaystyle
                    [(0,\,n)]
                \end{array}
            \end{equation}

            Justifiy this notation.
    \end{itemize}
\end{problem}
