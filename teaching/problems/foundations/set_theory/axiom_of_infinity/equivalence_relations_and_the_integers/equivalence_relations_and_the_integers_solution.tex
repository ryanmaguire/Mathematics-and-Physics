\begin{solution}

    \par\hfill\par

    \begin{itemize}
        \item
            Consider the set $\mathbb{N}\times\mathbb{N}$.
            Define $R$ to be the relation $(a,\,b)R(c,\,d)$ if and only if
            $a+d=b+c$. Prove this is an equivalence relation.
    \end{itemize}
    $R$ is reflexive. Given $(a,\,b)\in\mathbb{N}\times\mathbb{N}$, we have
    $a+b=b+a$ by the commutative law of addition, and therefore
    $(a,\,b)R(a,\,b)$.

    \par\hfill\par

    $R$ is also symmetric. If $(a,\,b)R(c,\,d)$, then
    $a+d=b+c$ by the definition of $R$. But equality is symmetric,
    so $b+c=a+d$. But addition is commutative, and hence $c+b=d+a$.
    That is, $(c,\,d)R(a,\,b)$. Therefore if $(a,\,b)R(c,\,d)$, then
    $(c,\,d)R(a,\,b)$, and thus $R$ is symmetric.

    \par\hfill\par

    Lastly, $R$ is transitive. If $(a,\,b)R(c,\,d)$, then $a+d=b+c$.
    If $(c,\,d)R(e,\,f)$, then $c+f=d+e$. But then:

    \begin{equation}
        \begin{array}{rclll}
            \displaystyle
            (a+f)+(c+d)
            \!\!
            &=&
            \!\!
            \displaystyle
            (a+d)+(c+f)
            &&
            (\textrm{Associativity and Commutativity})\\[1.5em]
            \!\!
            &=&
            \!\!
            \displaystyle
            (b+c)+(c+f)
            &&
            (\textrm{Since $a+d=b+c$})\\[1.5em]
            \!\!
            &=&
            \!\!
            \displaystyle
            (b+c)+(d+e)
            &&
            (\textrm{Since $c+f=d+e$})\\[1.5em]
            \!\!
            &=&
            \!\!
            \displaystyle
            (b+e)+(c+d)
            &&
            (\textrm{Associativity and Commutativity})\\[1.5em]
            \displaystyle
            \Rightarrow(a+f)+(c+d)
            \!\!
            &=&
            \!\!
            \displaystyle
            (b+e)+(c+d)
            &&
            (\textrm{Transitivity of Equality})
        \end{array}
    \end{equation}

    By the cancellative property of addition, if $(a+f)+(c+d)=(b+e)+(c+d)$,
    then $a+f=b+e$. That is, if $(a,b)R(c,\,d)$, and $(c,\,d)R(e,\,f)$,
    then $(a,\,b)R(e,\,f)$, and hence $R$ is transitive.
    Since $R$ is reflexive, symmetric, and transtive,
    $R$ is therefore an equivalence relation by definition.

    \begin{itemize}
        \item
            Describe the equivalence class of $(a,\,b)$ geometrically
            (when viewing $\mathbb{N}\times\mathbb{N}$ as
            a subset of $xy$ plane).
    \end{itemize}

    Consider the equivalence class of $(0,\,0)$, the set
    $[(0,\,0)]$. This is the set of all
    $(m,\,n)\in\mathbb{N}\times\mathbb{N}$ such that
    $0+n=0+m$, which is the same as $n=m$. This is the \textit{diagonal}
    through the Cartesian plane, the set of points $(n,\,n)$ in the lattice
    $\mathbb{N}\times\mathbb{N}$. What about $[(1,\,0)]$? This is the set
    of points $(m,\,n)$ such that $1+n=0+m$, so $m=1+n$. This is the set
    of points that lie on the line of slope 1 passing through the point
    $(1,\,0)$. In general, $[(x_{0},\,y_{0})]$ is the set of all $(x,\,y)$
    with $y+x_{0}=x+y_{0}$. This property is precisely the property of a
    line with slope 1 passing through the point $(x_{0},\,y_{0})$. The
    equivalence class of $[(m,\,n)]$ is a straight line through the
    lattice $\mathbb{N}\times\mathbb{N}$ of slope 1 containing
    $(m,\,n)$. This is depicted in
    Fig.~\ref{fig:grothendique_construction_of_integers_001}.
    \par\hfill\par
    Addition of equivalence classes is well-defined here. Let
    $[(a,\,b)]=[(x,\,y)]$ and $[(c,\,d)]=[(z,\,w)]$. We need to show
    that $[(a+c,\,b+d)]=[(x+z,\,y+w)]$. Since $[(a,\,b)]=[(x,\,y)]$, and
    since $R$ is an equivalence relation, we know that
    $(a,\,b)R(x,\,y)$. That is, $a+y=b+x$. Similarly, $(c,\,d)R(z,\,w)$, so
    $c+w=d+z$. Then:

    \begin{equation}
        \begin{array}{rclll}
            \displaystyle
            a+c+y+w
            \!\!
            &=&
            \!\!
            \displaystyle
            (a+y)+(c+w)
            &&
            (\textrm{Associativity and Commutativity})\\[1.5em]
            \!\!
            &=&
            \!\!
            \displaystyle
            (b+x)+(c+w)
            &&
            (\textrm{Substitution})\\[1.5em]
            \!\!
            &=&
            \!\!
            \displaystyle
            (b+x)+(d+z)
            &&
            (\textrm{Substitution})\\[1.5em]
            \!\!
            &=&
            \!\!
            \displaystyle
            b+d+x+z
            &&
            (\textrm{Associativity and Commutativity})\\[1.5em]
            \displaystyle
            \Rightarrow
            a+c+y+w
            \!\!
            &=&
            \!\!
            \displaystyle
            b+d+x+z
            &&
            (\textrm{Transitivity of Equality})
        \end{array}
    \end{equation}

    So $(a+c,\,b+d)R(x+z,\,y+w)$, and hence
    $[(a+c,\,b+d)]=[(x+z,\,y+w)]$, so addition is well-defined.
    \par\hfill\par
    Lastly, $[(0,\,0)]$ behaves like the additive identity. This is the
    arithmetic property of \textit{zero}. Given $[(m,\,n)]$, we have
    $[(0,\,0)]+[(m,\,n)]=[(0+m,\,0+n)]=[(m,\,n)]$, so addition by
    $[(0,\,0)]$ does not change anything. The \textit{number}
    $[(0,\,n)]$ also behaves like the negative of $[(n,\,0)]$ since
    $[(n,\,0)]+[(0,\,n)]=[(n,\,n)]=[(0,\,0)]$.
    \begin{figure}
        \centering
        \inputImage{grothendique_construction_of_integers_001}
        \caption{Equivalence Class of Points}
        \label{fig:grothendique_construction_of_integers_001}
    \end{figure}
\end{solution}
