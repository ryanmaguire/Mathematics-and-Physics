%-----------------------------------LICENSE------------------------------------%
%   This file is part of Mathematics-and-Physics.                              %
%                                                                              %
%   Mathematics-and-Physics is free software: you can redistribute it and/or   %
%   modify it under the terms of the GNU General Public License as             %
%   published by the Free Software Foundation, either version 3 of the         %
%   License, or (at your option) any later version.                            %
%                                                                              %
%   Mathematics-and-Physics is distributed in the hope that it will be useful, %
%   but WITHOUT ANY WARRANTY; without even the implied warranty of             %
%   MERCHANTABILITY or FITNESS FOR A PARTICULAR PURPOSE.  See the              %
%   GNU General Public License for more details.                               %
%                                                                              %
%   You should have received a copy of the GNU General Public License along    %
%   with Mathematics-and-Physics.  If not, see <https://www.gnu.org/licenses/>.%
%------------------------------------------------------------------------------%
\begin{solution}
    The set $A$ is, in plain English, the \textit{set of all sets}. Since it
    is itself a set, by definition, it must be true that $A\in{A}$.

    \par\hfill\par

    The set $B$ is the set of all sets that do not contain themselves.
    I believe Russell called these \textit{proper} sets.
    To prove $B\in{B}$ we suppose not. Then $B\not\in{B}$. But if
    $B\not\in{B}$, then by the definition of $B$ we have $B\in{B}$, which is a
    contradiction. We may conclude that $B\in{B}$.

    \par\hfill\par

    Now we prove $B\not\in{B}$. Again, we prove by contradiction. Suppose
    $B\in{B}$. But by definition of $B$, $B$ is the set of all sets that
    do not contain themselves, so $B$ can't be an element of $B$, a
    contradiction. Hence, $B\not\in{B}$.

    \par\hfill\par

    This shows the axiom of unrestricted comprehension is invalid. We cannot
    be allowed to form sets via any sentence, we must restrict ourselves
    to applying sentences to sets we already know exist. The principle of
    explosion (which states that if $P$ is a proposition that is both true and
    false, then every proposition $Q$ is also true and false) shows that if we
    allow the axiom of unrestricted comprehension, then \textit{every}
    statement is both true and false.
\end{solution}
