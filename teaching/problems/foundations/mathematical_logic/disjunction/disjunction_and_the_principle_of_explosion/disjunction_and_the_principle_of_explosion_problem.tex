%-----------------------------------LICENSE------------------------------------%
%   This file is part of Mathematics-and-Physics.                              %
%                                                                              %
%   Mathematics-and-Physics is free software: you can redistribute it and/or   %
%   modify it under the terms of the GNU General Public License as             %
%   published by the Free Software Foundation, either version 3 of the         %
%   License, or (at your option) any later version.                            %
%                                                                              %
%   Mathematics-and-Physics is distributed in the hope that it will be useful, %
%   but WITHOUT ANY WARRANTY; without even the implied warranty of             %
%   MERCHANTABILITY or FITNESS FOR A PARTICULAR PURPOSE.  See the              %
%   GNU General Public License for more details.                               %
%                                                                              %
%   You should have received a copy of the GNU General Public License along    %
%   with Mathematics-and-Physics.  If not, see <https://www.gnu.org/licenses/>.%
%------------------------------------------------------------------------------%
\begin{problem}
    This problem explores the use of the word \textit{or} in mathematics.
    You will prove the \textit{principle of explosion}. The principle says
    that if $P$ is a statement that is both true and false, then for any
    statement $Q$, $Q$ is true. A system that contains a sentence that is
    both true and false is called \textit{inconsistent}. The principle of
    explosion shows that inconsistent systems are very boring.
    \begin{itemize}
        \item
            (2 Points)
            Let $P$ be a statement that is true, and let $Q$
            be any other claim. Since $P$ is true, what can you conclude
            about $P$ \textit{or} $Q$?
        \item
            (2 Points)
            Suppose $P$ is also false. Using what you've concluded
            about $P$ \textit{or} $Q$, what can you prove about $Q$?
    \end{itemize}
\end{problem}
