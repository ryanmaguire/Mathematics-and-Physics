\documentclass{article}

%---------------------------Packages----------------------------%
\usepackage{geometry}
\geometry{a5paper, margin=14mm}
\usepackage{graphics, float}            % Graphics/Images.
\usepackage[english]{babel}             % Language typesetting.
\usepackage[dvipsnames]{xcolor}         % Color names.
\usepackage{mathtools, esint, mathrsfs} % amsmath and integrals.
\usepackage{amsthm, amsfonts, amssymb}  % Fonts and theorems.
\usepackage{upgreek}                    % Non-Italic Greek.
\usepackage{titletoc}                   % Allows \book in toc.
\usepackage[nottoc]{tocbibind}          % Bibliography in toc.
\usepackage[titles]{tocloft}            % ToC formatting.
\usepackage[hang,multiple]{footmisc}
\usepackage{imakeidx}                   % Used for index.
\usepackage[font={scriptsize},
            hypcap=true,
            labelsep=colon]{caption}    % Figure captions.
\usepackage[pdftex,
            pdfauthor={Ryan Maguire},
            pdftitle={Mathematics and Physics},
            pdfsubject={Mathematics, Physics, Science},
            pdfkeywords={Mathematics, Physics, Computer Science, Biology},
            pdfproducer={LaTeX},
            pdfcreator={pdflatex}]{hyperref}
\hypersetup{
    colorlinks=true,
    linkcolor=blue,
    filecolor=magenta,
    urlcolor=Cerulean,
    citecolor=SkyBlue
}                           % Colors for hyperref.
\usepackage[toc,acronym,nogroupskip,nopostdot]{glossaries}
\usepackage{glossary-mcols}

%------------------------Theorem Styles-------------------------%
\theoremstyle{plain}
\newtheorem{theorem}{Theorem}[section]

% Define theorem style for default spacing and normal font.
\newtheoremstyle{normal}
    {\topsep}               % Amount of space above the theorem.
    {\topsep}               % Amount of space below the theorem.
    {}                      % Font used for body of theorem.
    {}                      % Measure of space to indent.
    {\bfseries}             % Font of the header of the theorem.
    {}                      % Punctuation between head and body.
    {.5em}                  % Space after theorem head.
    {}

% Italic header environment.
\newtheoremstyle{thmit}{\topsep}{\topsep}{}{}{\itshape}{}{0.5em}{}

% Define environments with italic headers.
\theoremstyle{thmit}
\newtheorem*{solution}{Solution}

% Define default environments.
\theoremstyle{normal}
\newtheorem{example}{Example}[section]
\newtheorem{definition}{Definition}[section]
\newtheorem{axiom}{Axiom}[section]
\newtheorem{notation}{Notation}[section]

\author{Ryan Maguire}
\title{Notes on Lorentz Geometry}
\begin{document}
    \maketitle
    \tableofcontents
    \listoffigures
    \section{Preliminary Material}
        \begin{definition}
            An $(r,s)$ tensor of a Module $M$ over a ring $R$ is a
            multilinear function $A:(M^{*})^{r}\times{M}^{s}\rightarrow{R}$,
            where $M^{*}$ is the dual of $M$, and multilinear means that the
            restriction of $A$ to any of the components is linear.
        \end{definition}
        Of interest in manifold theory are tensor fields. The set
        $\mathfrak{X}(M)$ of vector fields over $M$ can be given the structure
        of a module over $C^{\infty}(M,\mathbb{R})$. Viewing a vector field
        $X:M\rightarrow{TM}$ as a smooth section from a manifold $M$ to it's
        tangent space, we have, for all $X,Y\in\mathfrak{X}(M)$ and all
        $f\in{C}^{\infty}(M,\mathbb{R})$:
        \begin{subequations}
            \begin{align}
                (X+Y)(p)&=X(p)+Y(p)\\
                (fX)(p)&=f(p)X(p)
            \end{align}
        \end{subequations}
        \begin{definition}
            A tensor field $A$ on a smooth manifold $M$ is an $(r,s)$ tensor
            $A:\mathfrak{X}^{*}(M)^{r}\times\mathfrak{X}(M)^{s}\rightarrow{C}^{\infty}(M,\mathbb{R})$,
            where $\mathfrak{X}(M)$ is the module of vector fields over $M$,
            and $\mathfrak{X}^{*}(M)$ is the dual of this.
        \end{definition}
        We can then define a metric tensor on a manifold to be a $(0,2)$ tensor
        field with certain properties. There is another way to do this. We want
        a metric tensor to give us a dot product, or a means of measuring
        angles, for each point. So for each point we want a bilinear function
        $g_{p}$ that takes in two tangent vectors at $p$ and returns a real
        number. It should also be symmetric since the dot product is, and
        non-degenerate. In terms of the tangent space at $p$, this is a function
        $g_{p}:T_{p}M\times{T}_{p}M\Rightarrow\mathbb{R}$. We want our function
        $g$ to provide, for every $p\in{M}$, a metric tensor $g(p)=g_{p}$, and
        we want this to occur smoothly. First we note that a bilinear function
        that takes in two tangent vectors at $p$ and returns a real number is an
        element of the dual $(T_{p}M\otimes{T}_{p}M)^{*}$. Since our manifold
        $M$ is finite dimensional, and since $T_{p}M$ has its vector space
        dimension equal to the manifold dimension of $M$,
        $T_{p}M\otimes{T}_{p}M$ is the tensor product of two finite dimensional
        vector spaces, and hence $(T_{p}M\otimes{T}_{p}M)^{*}$ is isomorphic
        to $T_{p}M^{*}\otimes{T}_{p}M^{*}$. So we can define a metric tensor
        to be a smooth section $g:M\rightarrow{T}M^{*}\otimes{T}M^{*}$, where
        $TM^{*}$ is the cotangent bundle.
        \par\hfill\par
        There is a way to get a quadratic form on $M$ from a metric tensor.
        Define, for each $p\in{M}$, the quadratic form
        $Q_{p}:T_{p}M\rightarrow\mathbb{R}$ by:
        \begin{equation}
            Q_{p}(v)=g_{p}(v,v)
        \end{equation}
        Since $g_{p}$ is a bilinear form, $Q_{p}$ is a quadratic form. By
        Sylvester's Law of Inertia, there is a basis for $T_{p}M$ such that
        $Q_{p}$ has the form:
        \begin{equation}
            Q_{p}(v)=\sum_{k=1}^{n}a_{k}v_{k}^{2}
        \end{equation}
        Where $a_{k}$ is either $0$, $1$, or $-1$. Sylvester's theorem also says
        the number of $0$, $1$, and $-1$ that occur is an invariant. That is,
        choosing another basis with such a representation of $Q_{p}$ will yield
        the same number of $a_{k}$ being 0, the same number being 1, and the
        same number being $-1$. Since a metric tensor is non-degenerate, then
        can be no zeros in this expansion, so a metric tensor has two invariant
        numbers $(p,n)$, the number of positive 1's and the number of negative
        1's. This ordered pair $(p,n)$ is locally constant on the manifold, and
        if the manifold is connected that means it is constant. This gives us
        the following:
        \begin{definition}
            The signature of a metric tensor $g$ on a connected manifold $M$ is
            the unique ordered pair $(p,n)$ where $p$ is the number of postive
            1's given by Sylvester's theorem and $n$ is the number of negative
            1's.
        \end{definition}
        \begin{definition}
            A Lorentz manifold is a smooth manifold $M$ with a metric tensor
            $g:M\rightarrow{T}M^{*}\otimes{T}M^{*}$ with signature $(n-1,1)$,
            where $n$ is the dimension of $M$.
        \end{definition}
        \begin{example}
            From Sylvester's theorem we know that for any Lorentz manifold
            $(M,g)$ and for any point $p\in{M}$ there is a chart
            $(\mathcal{U},\varphi)$ such that for all $v\in{T}_{p}M$ we have:
            \begin{equation}
                g_{p}(v,v)=-\textrm{d}\varphi_{1}^{2}(v)+
                    \sum_{k=2}^{n}\textrm{d}\varphi_{k}^{2}(v)
            \end{equation}
            Where $\textrm{d}\varphi_{k}$ is the 1-form defined by
            $\textrm{d}\varphi_{k}(\partial\varphi_{\ell})=\delta_{k\ell}$,
            where $\delta_{k\ell}$ is the Kronecker delta, and
            $\partial\varphi_{j}$ is the $j^{th}$ standard basis vector for
            $T_{p}M$ given by the chart $(\mathcal{U},\varphi)$. If we let
            $M=\mathbb{R}^{3}$, then we can cover $M$ with a single global chart
            $(\mathbb{R}^{3},\textrm{Id}_{\mathbb{R}^{3}})$ and can define the
            metric tensor $g$ by:
            \begin{equation}
                g=-\textrm{d}t^{2}+\textrm{d}x^{2}+\textrm{d}y^{2}
            \end{equation}
            The $z$ axis is replaced by a $t$ since we usually think of this as
            representing time. This is called the 3-dimensional Minkowski space.
        \end{example}
        A metric tensor does not need to be positive-definite, like in the case
        of a Riemannian metric. It is possible for a non-zero tangent vector to
        have $g_{p}(v,v)=0$. In the case of Minkowski space this occurs when
        the sum of the squares of the spatial components equals the square of
        the time component. Solving
        $\textrm{d}t^{2}=\textrm{d}x^{2}+\textrm{d}y^{2}$ gives us a cone:
        \begin{equation}
            \textrm{d}t=\pm\sqrt{\textrm{d}x^{2}+\textrm{d}y^{2}}
        \end{equation}
        This is called the \textit{light cone} of the point $p$. Since every
        Lorentz manifold locally has some chart that represents the metric
        tensor in this form, the light cone of a point $p$ in an arbitrary
        Lorentz manifold is a well-defined concept. This notion gives us a few
        new definitions. If the time component is greater than the spatial
        component, then $g_{p}(v,v)$ is negative. We give this a name:
        \begin{definition}
            A timelike vector at a point $p$ in a Lorentz manifold $(M,g)$ is a
            tangent vector $v\in{T}_{p}M$ such that $g_{p}(v,v)<0$.
        \end{definition}
        If $g_{p}(v,v)$ is positive, the above equation tells us the spatial
        components sum to a greater value than the square of the time one. This
        is also given a name:
        \begin{definition}
            A space like vector at a point $p$ in a Lorentz manifold $(M,g)$ is
            a tangent vector $v\in{T}_{p}M$ such that $g_{p}(v,v)>0$.
        \end{definition}
        Now for the case when we have equality. The naming convention comes from
        physics. If we try to describe the displacement of a photon of light
        with respect to time, we get the equation $r^{2}=c^{2}t^{2}$, where
        $c$ is the speed of light. Translating this for our Lorentz manifold,
        if a tangent vector $v$ at the point $p$ satisfies
        $g_{p}(v,v)$, then there is some chart where we have:
        \begin{equation}
            \textrm{d}t^{2}(v)=\sum_{k=2}^{n}\textrm{d}x_{k}^{2}(v)
        \end{equation}
        This mimics our $r^{2}=c^{2}t^{2}$ equation if we set $c=1$. And indeed,
        in \textit{natural units}, which is a unit system used in modern
        physics, the speed of light is taken to be 1. This suggest that we name
        such a vector $v$ as follows:
        \begin{definition}
            A lightlike vector at a point $p$ in a Lorentz manifold $(M,g)$ is a
            non-zerotangent vector $v\in{T}_{p}M$ such that $g_{p}(v,v)=0$.
        \end{definition}
        And the last case combines lightlike vectors with the zero vector.
        \begin{definition}
            A null vector at a point $p$ in a Lorentz manifold $(M,g)$ is a
            tangent vector $v\in{T}_{p}M$ such that either $v$ is lightlike, or
            $v$ is zero. That is, any element $v\in{T}_{p}M$ such that
            $g_{p}(v,v)=0$.
        \end{definition}
        \begin{definition}
            The light cone of a point $p$ in a Lorentz manifold $(M,g)$ is the
            set of all points $v\in{T}_{p}M$ such that $g_{p}(v,v)=0$.
        \end{definition}
        At each point in a Lorentz manifold there is a choice of light cone. A
        Lorentz manifold is \textit{time-orientable} if one can make a smooth
        and consistent choice of light cone as one varies throughout the
        manifold. Orientability and time orientability are different
        phenomona. It is possible to have a non-orientable manifold that is
        indeed time orientable. One such example is the M\"{o}bius band cross
        $\mathbb{R}$. If this space were orientable, then so would the
        M\"{o}bius band, which it is not. It is, however, time orientable.
        \begin{theorem}
            A Lorentz Manifold is time orientable if and only if there exists
            a timelike vector field $X$.
        \end{theorem}
        By timelike vector field, it is meant that for all $p\in{M}$ one has
        that $X_{p}$ is a timelike tangent vector at $p$. The vector field $X$
        can be chosen to be complete. A vector field is complete if each
        integral curve can be defined for all time. To show $X$ can be chosen to
        be complete, we first take a complete Riemannian metric $g_{R}$ and
        replace $X$ by $X/|X|_{R}$. The existence of $g_{R}$ is by
        Nomizu and Ozeki.
        \par\hfill\par
        A Riemannian metric allows us to get a 1-form from a vector field.
        Given $X\in\mathfrak{X}(M)$, define for all $p\in{M}$ and all
        $v\in{T}_{p}M$:
        \begin{equation}
            X^{\flat}(v)=g_{p}(X_{p},v)
        \end{equation}
        The fact that $g$ is a metric tensor makes this a 1-form on $M$. If
        $(M,g_{R})$ is a Riemannian manifold and $X$ is a non-vanishing vector
        field on $X$, one obtains a Lorentz manifold $(M,g)$ by defining:
        \begin{equation}
            g=g_{R}-\frac{2}{g_{R}(X,X)}X^{\flat}\otimes{X}^{\flat}
        \end{equation}
        Some results from algebraic topology now tell us when a manifold can be
        given a Lorentz structure. Any smooth manifold always has a
        Riemannian structure that can be attached to it. To see this one can
        either use a partition of unity argument, or one could use Whitney's
        embedding theorem to embed the manifold into $\mathbb{R}^{N}$ for large
        enough $N$, and then \textit{steal} the metric from Euclidean space.
        From this, if we have a connected smooth manifold with Euler
        characteristic zero, then we know we can endow it with a Lorentz
        structure. Euler characteristic zero means there is a non-vanishing
        vector field, and from the above argument we can then use the existence
        of a Riemannian metric to induce a Lorentz metric.
        \begin{definition}
            A spacetime is a time oriented Lorentz manifold $(M,g)$.
        \end{definition}
        There is a difference between time orientable and time oriented.
        Time orientable means $(M,g)$ \textit{can} be given a time orientation,
        whereas time oriented means a time orientation has already been chosen.
        \begin{definition}
            Pointwise conformal Lorentz metrics on a smooth manifold $M$ are
            Lorentz metrics $g$ and $g'$ such that there exists a function
            $u:M\rightarrow\mathbb{R}$ such that $g'=\exp(2u)g$.
        \end{definition}
        \begin{theorem}[Dajczer Criterion]
            If $(V,g)$ is a real vector space with a non-degenerate indefinite
            scalar product, and if $b$ is a bilinear symmetric form on $V$, and
            if $g(v,v)=0$ imples $b(v,v)=0$, then there is a $c\in\mathbb{R}$
            such that $b=cg$.
        \end{theorem}
        As an application, we have the following:
        \begin{theorem}
            If $M$ is a smooth manifold of dimension greater than 2, and if
            $g$ and $g'$ are Lorentz metrics on $M$, then $g$ and $g'$ are
            pointwise conformal if and only if the both have the same lightlike
            tangent vectors.
        \end{theorem}
        \begin{definition}
            Pointwise conformal spacetimes on a manifold $M$ are spacetimes
            $(M,g)$ and $(M,g')$ such that $g$ and $g'$ are pointwise conformal
            and such that the time orientations agree.
        \end{definition}
        \begin{definition}
            A conformal spacetime to a spacetime $(M,g)$ is a spacetime $(M,g')$
            such that there exists a diffeomorphism $f:M\rightarrow{M}'$ such
            that the pullback metric on $M$ induced by $f$ and $g'$ is
            pointwise conformal to $(M,g)$.
        \end{definition}
        The notion of conformal is an equivalence relation. The class of all
        conformal spacetimes $[(M,g)]$ is called the \textit{conformal} or
        \textit{classical causal structure} of the spacetime $(M,g)$. Since a
        diffeomorphism determines the topology and smooth structure, uniquely,
        of the target, we may as well take $M=M'$. We can then talk about the
        \textit{set} of all pointwise equilalent metrics with the same time
        orientation on $M$. This is indeed a set. A Lorentz metric is a smooth
        function $f:M\rightarrow{T}^{*}M\otimes{T}^{*}M$, and the collection of
        all functions $f:A\rightarrow{B}$ for any sets $A$ and $B$ constitutes
        a set itself. This means the question of what is the causal structure
        of a spacetime $(M,g)$ can be reduced to a question about a set of
        particular functions.
        \par\hfill\par
        A semi-Riemannian metric gives rise to the notion of an exponential
        map. Given a point $p\in{M}$ and a tangent vector $v\in{T}_{p}M$, there
        is a unique geodesic $\gamma:\mathbb{R}\rightarrow{M}$ such that
        $\gamma(0)=p$ and $\dot{\gamma}(0)=0$. We can then define the
        exponential map as $\exp_{p}(v)=\gamma(1)$. That is, we flow along the
        geodesic for time 1.
        \par\hfill\par
        In Riemannian geometry, since we have positive-definiteness, we can
        define:
        \begin{equation}
            L(\gamma)=\int_{a}^{b}
                \sqrt{g_{\gamma(t)}\big(
                    \dot{\gamma}(t),\dot{\gamma}(t)
                \big)}\textrm{d}t
        \end{equation}
        A geodesic could then be defined in terms of the calculus of variations
        as a curve which minimizes this functions. In the world of
        semi-Riemannian metrics the expression inside the square could be
        negative. We circumvent this by means of a connection. The fundamental
        theorem of semi-Riemannian geometry says there is a unique torsion-free
        Levi-Civita connection $\nabla$ on a semi-Riemannian manifold. A
        geodesic is then defined as a curve $\gamma$ such that
        $\nabla_{\dot{\gamma}}(\dot{\gamma})=0$.
    \section{Examples of Spacestimes}
        \begin{definition}
            A totally vicious spacetime is a spacetime $(M,g)$ such that for
            all $p,q\in{M}$ it is true that $I_{p}^{+}\cap{I}_{q}^{-}=M$.
        \end{definition}
        Equivalently, the Lorentzian distance function
        $d:M\times{M}\rightarrow\mathbb{R}$ satisfies $d(p,q)=\infty$.
        \begin{example}
            The torus $\mathbb{T}^{2}=\mathbb{S}^{1}\times\mathbb{S}^{1}$ with
            it's usual metric is totally vicious. This can be seen from the
            universal cover of $\mathbb{R}^{2}$, which is the plane, using the
            standard identification of the sides of the square to obtain a
            torus. $I_{p}^{+}$ is a half-cone pointing upwards, which will
            eventually encompass an entire square $[n,n+1]\times[m,m+1]$,
            meaning all of $\mathbb{R}^{2}$ lies in the forward time direction
            eventually. Similarly for the past direction. The intersection for
            any two points will then be the entire manifold.
        \end{example}
        \begin{example}
            The G\"{o}del spacetime is a spacetime on $\mathbb{R}^{4}$ with a
            strange metric. It is defined by:
            \begin{equation}
                g=\frac{1}{2\omega^{2}}\Big[
                    -(\textrm{d}t+\exp(x)\textrm{d}y)^{2}+
                    \frac{1}{2}\exp(2x)\textrm{d}y^{2}
                    +\textrm{d}z^{2}
                \Big]
            \end{equation}
            Where $\omega$ is allowed to be any non-zero real value.
            Kurt G\"{o}del presented this universe to Einstein as a birthday
            present since it is a spacetime that satisfies Einstein's field
            equation but allows for time travel. It also happens to be a
            totally vicious spacetime.
        \end{example}
        All other examples I found were globally hyperbolic, which implies the
        rest of the causal ladder.
    \section{More Stuff}
        Since spacetimes are topological spaces, there is a $\sigma$ algebra
        associated to them, the Borel $\sigma$ algebra generated by the
        topology. We consider \textit{Borel} measures on a spacetime that have
        the property that for some auxiliary semi-Riemannian metric $g$, the
        measure satisfies $\mu(M)$ is equal to the Lorentz volume. The Borel
        $\sigma$ algebra can be completed by taking the Lebesgue completion,
        adding all subsets of measure zero of all measurable sets.
        \par\hfill\par
        \begin{definition}
            The future $t^{-}$ and past $t^{+}$ volume functions are:
            \begin{subequations}
                \begin{align}
                    t^{-}(p)&=\mu(I^{-}(p))\\
                    t^{+}(p)&=\mu(I^{+}(p))
                \end{align}
            \end{subequations}
        \end{definition}
        These functions need not be continuous. We can take some metric on
        $\mathbb{R}^{2}$ with finite volume, and then remove some chunk of the
        form $[-a,a]\times[1,\infty)$. For some points the future cone will be
        small. As we move to the left there will suddenly be a jump in the size
        of the cone, resulting in a discontinuity in this measure function.
        These functions are, however, semi-continuous. This is equivalent to the
        functions $I^{+}$ and $I^{-}$ having the properties of \textit{inner}
        and \textit{outer} continuity.
        \begin{definition}
            The function $I^{-}$ is inner continuous at a point $p$ if for
            any compact set $K\subseteq{I}^{-}(p)$ there is an open subset
            $\mathcal{U}$ containing $p$ such that $K\subseteq{I}^{-}(q)$ for
            all $q\in\mathcal{U}$.
        \end{definition}
        \begin{definition}
            The function $I^{-}$ is outer continuous at a point $p$ if for
            any compact set $K\subseteq{M}\setminus{I}^{-}(p)$ there is an open
            subset $\mathcal{U}$ containing $p$ such that
            $K\subseteq{M}\setminus{I}^{-}(q)$ for all $q\in\mathcal{U}$.
        \end{definition}
        \begin{definition}
            A spacetime that is past reflecting at a point $p\in{M}$ is one
            such that $I^{+}(q)\subset{I}^{+}(p)$ implies
            $I^{-}(p)\subset{I}^{-}(q)$. A similar definition holds for
            future reflecting, and reflecting spacetimes are those that are both
            past and future reflecting.
        \end{definition}
        \begin{theorem}
            The set valued map $I^{-}$ is outer continuous if and only if
            the spacetime is past reflecting. A similar statement holds for
            future reflecting.
        \end{theorem}
    \section{Stably Causal Spacetimes}
        \begin{definition}
            A generalized time function is a function $t:M\rightarrow\mathbb{R}$
            such that $t$ is strictly increasing on any future directed
            causal curve $\gamma:\mathbb{R}\rightarrow{M}$.
        \end{definition}
        There is no requirement that a generalized time function be continuous.
        \begin{definition}
            A time function is a continuous generalized time function.
        \end{definition}
        \begin{definition}
            A temporal function is a smooth generalized time function with a
            past-directed timelike gradient $\nabla(t)$.
        \end{definition}
        A smooth time function need not be a temporal function since this
        gradient condition can fail to hold.
        \begin{theorem}
            A spacetime is distinguishing if and only if the functions
            $t^{-}$ and $t^{+}$ are generalized time functions.
        \end{theorem}
        \begin{example}
            The function $t^{-}$ is constant on any causal loop, however
            causal spacetimes cannot be characterized in this way. See the 
            example in Figure 6 of the paper.
        \end{example}
        \subsection{Stability of Causality and Chronology}
            Let $\textrm{Lor(M)}$ be the set of all lorentz metrics on a smooth
            manifold $M$. There is a way to place a partial ordering on this set
            as follows:
            \begin{equation}
                g<g'\textrm{ if and only if all causal vectors for }g
                \textrm{ are timelike for }g'
            \end{equation}
            This partial ordering naturally induces a partial ordering on the
            equivalence classes of conformal metrics on $M$. The order topology
            induced from this partial ordering is the same as the
            $C^{0}$ topology on $\textrm{Lor}(M)$. This leads into the
            definition of stably causal:
            \begin{definition}
                A stably causal spacetime $(M,g)$ is one such that there exists
                a neighborhood of $g$ in the quotient $C^{0}$ topology such that
                for all $g'$ in this neighbor, $g'$ is causal.
            \end{definition}
            Intuitively, slight perturbations of a causal metric result in a
            causal metric. Using the order topology, this is equivalent to the
            following.
            \begin{theorem}
                A spacetime $(M,g)$ is stably causal if there exists a metric
                $g'$ such that $g'$ is causal and $g<g'$.
            \end{theorem}
            \begin{theorem}
                Any simply connected 2-dimensional spacetime is stably causal.
            \end{theorem}
            Stably causal implies strongly causal, but the converse can fail.
            See Figure 9 for an example.
    \section{Causally Simple Spacetimes}
        Reviewing some of the notation from earlier in the text, we have for a
        spacetime $(M,g)$ and a point $p\in{M}$:
        \begin{equation}
            J^{+}(p)=\{q\in{M}:p\leq{q}\}
        \end{equation}
        WHere $p\leq{q}$ means there is a causal or constant curve
        $\gamma:[a,b]\rightarrow\mathbb{R}$ such that $\gamma(a)=p$ and
        $\gamma(b)=q$. Similarly for $J^{-}(p)$ and $p\geq{q}$.
        \begin{definition}
            A causally simple spacetime is a spacetime $(M,g)$ such that
            it is causal and for all $p\in{M}$ it is true that
            $J^{+}(p)$ and $J^{-}(p)$ are closed.
        \end{definition}
        For an example of a causally simple but non-globally hyperbolic
        spacetime, see Figure 10.
        \begin{definition}
            A globally hyperbolic spacetime is a causal spacetime such that for
            all $p,q\in{M}$ it is true that $J^{+}(p)\cap{J}^{-}(q)$ is compact.
        \end{definition}
        \begin{definition}
            An achronal subset of a spacetime $(M,g)$ is a subset
            $A\subseteq{M}$ such that there exists no timelike curve
            $\gamma:\mathbb{R}\rightarrow{M}$ such that there are two distinct
            real numbers $a,b\in\mathbb{R}$ such that
            $\gamma(a),\gamma(b)\in{A}$.
        \end{definition}
        That is, an achronal subset is one where timelike curves cross the
        subset at most once.
        \begin{definition}
            An acausal subset is similar to an achronal one, but with timelike
            replaced with causal.
        \end{definition}
        \begin{definition}
            The domain of dependence of $A\subseteq{M}$ is the set of all points
            $p\in{M}$ such that every past or future inextendinble curve through
            $p$ intersects $A$.
        \end{definition}
        \begin{definition}
            The Cauchy horizon of $A\subseteq{M}$ is the set:
            \begin{equation}
                H(A)=\{p\in{D}(A)|I^{+}(p)\cap{D}^{+}(A)=\emptyset\}
            \end{equation}
        \end{definition}
        \begin{definition}
            A Cauchy hypersurface is an achronal subset with empty
            Cauchy horizon.
        \end{definition}
        \begin{theorem}[Geroch's Theorem]
            A spacetime is globally hyperbolic if and only if there exists a
            Cauchy hypersurface $S$, there exists a Cauchy time function, and
            all Cauchy hypersurfaces are homeomorphic to $S$, and $M$ is
            homeomorphic to $S\times\mathbb{R}$.
        \end{theorem}
    \section{The Isocausal Ladder}
        If the timecones of one metric $g$ on a manifold $M$ are included in
        those of another metric $g'$, we denote this with
        $g\prec{g}'$. If there exists an auto-diffeomorphism
        $\Phi:M\rightarrow{M}$ such that the pullback preserves this notion,
        $\Phi^{*}g\prec\Phi^{*}g'$, then $g$ and $g'$ are said to be
        \textit{isocausal}. This symbol gives a partial order on the set of
        all spacetimes on $M$. It was originally believed to refine the standard
        heirarchy, but Garci\'{a}-Parrado and S\'{a}nchez observed that two of
        the notions from the standard ladder are not preserved. That is,
        causal continuity and causal simplicity. We can generalize the above
        notion by letting the manifold $M$ vary as well. We get the following:
        \begin{definition}
            Let $(M_{1},g_{1})$ and $(M_{2},g_{2})$ be spacetimes. A causal
            mapping is a diffeomorphism $\Phi:M_{1}\rightarrow{M}_{2}$ such that
            $g_{1}\prec\Phi^{*}g_{2}$. That is, the pullback metric of $g_{2}$
            has timecones that include that timecones of $g_{1}$.
        \end{definition}
        \begin{definition}
            Isocausal spacetimes are spacetimes $(M_{1},g_{1})$ and
            $(M_{2},g_{2})$ such that there are causal mappings
            $\Phi:M_{1}\rightarrow{M}_{2}$ and $\Psi:M_{2}\rightarrow{M}_{1}$.
        \end{definition}
        Since a diffeomorphism completely determines the smooth structure
        and completely determines the topology between two manifolds, one may
        as well take $M_{1}=M_{2}$.
        \begin{theorem}
            If $(M_{1},g_{1})$ and $(M_{2},g_{2})$ are spacetimes, if
            $M_{1}\prec{M}_{2}$, and if $M_{2}$ is globally hyperbolic, then
            $M_{1}$ is globally hyperbolic.
        \end{theorem}
        There are more general results as well.
        \begin{theorem}
            If $(M_{1},g_{1})$ and $(M_{2},g_{2})$ are spacetimes, if
            $(M_{2},g_{2})$ is either causally stable, strongly causal, causal,
            distinguishing, chronological, or non-totally vicious, and if
            $M_{1}\prec{M}_{2}$, then $(M_{1},g_{1})$ is also either
            causally stable, strongly causal, causal,
            distinguishing, chronological, or non-totally vicious.
        \end{theorem}
        \begin{example}
            Consider two Generalized Robertson-Walker spactimes. These are
            obtained from warped products $I\times{f}_{i}S$, where
            $I\subseteq\mathbb{R}$ is an interval, $S$ is a Riemannian manifold,
            and the metric $g$ is:
            \begin{equation}
                g_{i}=-\textrm{d}t^{2}+f_{i}^{2}(t)g_{S}
            \end{equation}
            Where $g_{S}$ is the Riemannian metric on $S$. Let $S$ be compact,
            and furthermore let $I=\mathbb{R}$. If both $f_{i}$ are bounded and
            strictly positive, not tending towards zero, then these two
            spacetimes are isocausal. That is, if we impose
            $0<\textrm{inf}(f_{i})\leq\textrm{sup}(f_{i})<\infty$, then
            the two spacetimes are isocausal. We can show this via explicity
            causal mappings of the form
            $(t,x)\mapsto(\varphi(t),x)$, for a function
            $\varphi:\mathbb{R}\rightarrow\mathbb{R}$.
        \end{example}
        \begin{theorem}
            If $(M,g)$ is a generalized Robertson-Walker spacetime of the
            form $M=I\times_{f}S$, where $S$ is diffeomorphic to
            $\mathbb{S}^{n-1}$ for some $n\in\mathbb{N}$, then
            $(M,g)$ is isocausal to one and only one of the following:
            \begin{enumerate}
                \item $\mathbb{R}\times\mathbb{S}^{n-1}$ with metric
                      $g=-\textrm{d}t^{2}+g_{0}$.
                \item $(0,\infty)\times\mathbb{S}^{n-1}$ with metric
                      $g=-\textrm{d}t^{2}+g_{0}$.
                      \item $(-\infty,0)\times\mathbb{S}^{n-1}$ with metric
                      $g=-\textrm{d}t^{2}+g_{0}$.
                      \item $(0,L)\times\mathbb{S}^{n-1}$ with metric
                      $g=-\textrm{d}t^{2}+g_{0}$ and $L>0$.
            \end{enumerate}
        \end{theorem}
\end{document}