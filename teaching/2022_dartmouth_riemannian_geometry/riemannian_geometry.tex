%-----------------------------------LICENSE------------------------------------%
%   This file is part of Mathematics-and-Physics.                              %
%                                                                              %
%   Mathematics-and-Physics is free software: you can redistribute it and/or   %
%   modify it under the terms of the GNU General Public License as             %
%   published by the Free Software Foundation, either version 3 of the         %
%   License, or (at your option) any later version.                            %
%                                                                              %
%   Mathematics-and-Physics is distributed in the hope that it will be useful, %
%   but WITHOUT ANY WARRANTY; without even the implied warranty of             %
%   MERCHANTABILITY or FITNESS FOR A PARTICULAR PURPOSE.  See the              %
%   GNU General Public License for more details.                               %
%                                                                              %
%   You should have received a copy of the GNU General Public License along    %
%   with Mathematics-and-Physics.  If not, see <https://www.gnu.org/licenses/>.%
%------------------------------------------------------------------------------%

\documentclass{article}
\usepackage{mathtools, esint, mathrsfs} % amsmath and integrals.
\usepackage{amsthm, amsfonts, amssymb}  % Fonts and theorems.
\usepackage{graphics}                   % Embedding PDF images.
\usepackage{subcaption}                 % Subfigures and subcaptions.
\usepackage{hyperref}                   % Hyperlinks.

\renewcommand\thesubfigure{%
    \arabic{section}.\arabic{figure}.\arabic{subfigure}}

% Colors for hyperref.
\hypersetup{
    colorlinks=true,
    linkcolor=blue,
    filecolor=magenta,
    urlcolor=Cerulean,
    citecolor=SkyBlue
}

\title{Riemannian Geometry}
\author{Ryan Maguire}
\date{\today}
\setlength{\parindent}{0em}
\setlength{\parskip}{0em}

\newtheoremstyle{normal}
    {\topsep}               % Amount of space above the theorem.
    {\topsep}               % Amount of space below the theorem.
    {}                      % Font used for body of theorem.
    {}                      % Measure of space to indent.
    {\bfseries}             % Font of the header of the theorem.
    {}                      % Punctuation between head and body.
    {.5em}                  % Space after theorem head.
    {}

\theoremstyle{plain}
\newtheorem{theorem}{Theorem}[section]
\theoremstyle{normal}
\newtheorem{definition}{Definition}[section]
\newtheorem{example}{Example}[section]
\newtheorem{notation}{Notation}[section]

\begin{document}
    \maketitle
    These notes come from my personal study of Riemannian geometry. Many of
    the concepts come from Manfredo do Carmo's Riemannian Geoemtry, John
    M. Lee's Introduction to Riemannian Geometry, and J\"{u}rgen Jost's
    Riemannian Geometry and Geometric Analysis, as well as Vladimir Chernov's
    lectures for his Math 72 course during the Spring 2022 semester at
    Dartmouth College. Any errors in these notes are my own.
    \tableofcontents
    \listoffigures
    \newpage
    There are three notions of manifold that have increasingly more structure.
    A topological manifold is a space that \textit{locally} looks like
    $\mathbb{R}^{n}$ and is topologically \textit{nice}. A smooth manifold is
    a topological manifold that has enough structure on it to perform calculus.
    A Riemannian manifold is a smooth manifold in which one can perform
    geometry (measure angles, lengths of curves, etc.). These notes are on
    Riemannian manifolds. A brief introduction to the other notions is given.
    \section{General Topology}
        An extremely brief review of point-set topology is given.
        \begin{definition}{\textbf{(Topological Space)}}
            A topological space is an ordered pair $(X,\tau)$ where $X$ is a
            set and $\tau\subseteq\mathcal{P}(X)$ is a collection of subsets of
            $X$ such that:
            \begin{enumerate}
                \item $X\in\tau$ and $\emptyset\in\tau$.
                \item For all $\mathcal{U},\mathcal{V}\in\tau$ it is true that
                    $\mathcal{U}\cap\mathcal{V}\in\tau$.
                \item For any $\mathcal{O}\subseteq\tau$ it is true that
                    $\bigcup\mathcal{O}\in\tau$.
            \end{enumerate}
        \end{definition}
        The set $\tau$ is called the \textit{topology} on $X$. Sets
        $\mathcal{U}\in\tau$ are called \textit{open subsets}. Topological
        spaces can be rephrased as requiring that the entire set $X$ and the
        empty set $\emptyset$ are open, the intersection of two open sets is
        open, and the union or arbitrarily many open sets is open. By induction,
        the intersection of finitely many open subsets is open. A
        \textit{closed} set is a set $\mathcal{C}$ such that
        $X\setminus\mathcal{C}$ is open. That is, a closed set is the complement
        of an open set. By DeMorgan's laws the union of finitely many closed
        sets is closed and the intersection of arbitrarily many closed sets is
        closed.
        \begin{example}
            If $X$ is any set, and if $\mathcal{P}(X)$ is the power set of $X$,
            then $(X,\mathcal{P}(X))$ is a topological space. This is called the
            \textit{discrete} topology on $X$.
        \end{example}
        \begin{example}
            If $X$ is any set, and if $\tau=\{\emptyset,X\}$, then
            $(X,\tau)$ is a topological space. This is called the
            \textit{indiscrete} or \textit{chaotic} or \textit{trivial} topology
            on $X$.
        \end{example}
        \begin{example}
            If $(X,d)$ is a metric space, and if $\tau$ is the set of all
            \textit{metrically open} subsets of $X$, which are sets
            $\mathcal{U}\subseteq{X}$ such that for all $x\in\mathcal{U}$ there
            is an $r>0$ such that the open ball $B_{r}(x)$ of radius $r$
            centered at $x$ is such that $B_{r}(x)\subseteq\mathcal{U}$, then
            $(X,\tau)$ is a topological space.
        \end{example}
        \textit{Meterizable} topological spaces are spaces that arise from
        metrics.
        \begin{definition}{\textbf{(Subspace Topology)}}
            The subspace topology of a set $A\subseteq{X}$ with respect to a
            topological space $(X,\tau)$ is the set $\tau|_{A}$ defined by:
            \begin{equation}
                \tau|_{A}=\{\,\mathcal{U}\cap{A}\;|\;\mathcal{U}\in\tau\,\}
            \end{equation}
        \end{definition}
        If $(X,\tau)$ is a topological space, if $A\subseteq{X}$, and if
        $\tau|_{A}$ is the subspace topology, then $(A,\tau|_{A})$ is indeed
        a topological space.
        \par\hfill\par
        Given a collection of topologies $T$ on a set $X$ (that is, for all
        $\tau\in{T}$, $(X,\tau)$ is a topological space), the intersection
        $\bigcap{T}$ is again a topology on $X$. This allows us to define the
        topology \textit{generated} by a collection of subsets. If
        $B\subseteq\mathcal{P}(X)$ is any collection of subsets of $X$, the
        topology generated by $B$ is the \textit{smallest} topology that
        contains $B$. That is, let $T_{B}$ be the set of all topologies
        $\tau$ such that $B\subseteq\tau$. The topology generated by
        $B$ is $\tau(B)=\bigcap{T}_{B}$. Note that $T_{B}$ is non-empty since
        $\mathcal{P}(X)\in{T}_{B}$, so this intersection is always well-defined.
        This allows us to define the \textit{product} topology. If
        $(X,\tau_{X})$ and $(Y,\tau_{Y})$ are topological spaces, the ordered
        pair $(X\times{Y},\tau_{X}\times\tau_{Y})$ may not be since the set
        $\tau_{X}\times\tau_{Y}$ may not be closed under arbitrary unions.
        The fix is as follows.
        \begin{definition}{\textbf{(Product Topological Space)}}
            The product topological space of two topological spaces
            $(X,\tau_{X})$ and $(Y,\tau_{Y})$ is the topological space
            $\big(X\times{Y},\tau(\tau_{X}\times\tau_{Y})\big)$ where
            $\tau(\tau_{X}\times\tau_{Y})$ is the topology generated by
            $\tau_{X}\times\tau_{Y}$.
        \end{definition}
        \begin{figure}
        	\centering
        	\begin{subfigure}[b]{0.49\textwidth}
                \centering
                \includegraphics{%
                    ../images/open_rectangle_r2.pdf%
                }
                \caption{An Open Set in $\tau_{\mathbb{R}}\times\tau_{\mathbb{R}}$}
                \label{fig:open_rectangle_r2}
            \end{subfigure}
            \hfill
            \begin{subfigure}[b]{0.49\textwidth}
                \centering
                \includegraphics{%
                    ../images/open_not_rectangle_r2.pdf%
                }
                \caption{An Open Set Not in $\tau_{\mathbb{R}}\times\tau_{\mathbb{R}}$}
                \label{fig:commutative_diagram_cofree_object_in_topology}
            \end{subfigure}
            \caption{Open Subsets of the Plane}
            \label{fig:open_subsets_of_r2}
        \end{figure}
        To see why this is necessary, consider the plane $\mathbb{R}^{2}$ which
        is the product of the real line with itself
        $\mathbb{R}^{2}=\mathbb{R}\times\mathbb{R}$. Sets in
        $\tau_{\mathbb{R}}\times\tau_{\mathbb{R}}$ are rectangular in nature.
        The open unit disk in the plane is not contained in this collection,
        hence the need not to look at $\tau_{\mathbb{R}}\times\tau_{\mathbb{R}}$
        but rather the topology generated by this.
        See Fig.~\ref{fig:open_subsets_of_r2}.
        \begin{definition}{\textbf{(Continuous Function)}}
            A continuous function from a topological space $(X,\tau_{X})$ to
            a topological space $(Y,\tau_{Y})$ is a function $f:X\rightarrow{Y}$
            such that for all $\mathcal{V}\in\tau_{Y}$ it is true that
            $f^{-1}[\mathcal{V}]\in\tau_{X}$. That is, the pre-image of an open
            set is open.
        \end{definition}
        \begin{example}
            If $(X,\mathcal{P}(X))$ is the discrete topological space on $X$,
            if $(Y,\tau)$ is any topological space, and if $f:X\rightarrow{Y}$
            is any function, then $f$ is continuous. This earns the discrete
            topological space the title of the \textit{free object} in topology.
            This is analogous to the \textit{free group} or \textit{free module}
            of a set that one encounters in abstract algebra.
        \end{example}
        \begin{example}
            If $(Y,\tau)$ is any topological space, if $(X,\{\emptyset,X\})$ is
            the indiscrete topological space on $X$, and if $f:Y\rightarrow{X}$
            is any function, then $f$ is continuous. This earns the indiscrete
            topology the title of the \textit{cofree object} in topology. It is
            \textit{co-free} because the arrows are flipped (See
            Fig.~\ref{fig:free_and_cofree_objects_in_topology}).
            People interested in category theory may care about this.
        \end{example}
        \begin{figure}
        	\centering
        	\begin{subfigure}[b]{0.48\textwidth}
                \centering
                \includegraphics{%
                    ../images/commutative_diagram_free_object_in_topology.pdf%
                }
                \caption{The Free Object in Topology}
                \label{fig:commutative_diagram_free_object_in_topology}
            \end{subfigure}
            \hfill
            \begin{subfigure}[b]{0.49\textwidth}
                \centering
                \includegraphics{%
                    ../images/commutative_diagram_cofree_object_in_topology.pdf%
                }
                \caption{The Cofree Object in Topology}
                \label{fig:commutative_diagram_cofree_object_in_topology}
            \end{subfigure}
            \caption{The Free and Cofree Objects in Topology}
            \label{fig:free_and_cofree_objects_in_topology}
        \end{figure}
        \begin{definition}{\textbf{(Open Mapping)}}
            An open mapping from a topological space $(X,\tau_{X})$ to a
            topological space $(Y,\tau_{Y})$ is a function $f:X\rightarrow{Y}$
            such that for all $\mathcal{U}\in\tau_{X}$ it is true that
            $f[\mathcal{U}]\in\tau_{Y}$. That is, the image of an open set is
            open.
        \end{definition}
        An open mapping need not be continuous and a continuous function need
        not be an open mapping. Continuous bijective open
        mappings are given a name.
        \begin{definition}{\textbf{(Homeomorphism)}}
            A homeomorphism from a topological space $(X,\tau_{X})$ to a
            topological space $(Y,\tau_{Y})$ is a function $f:X\rightarrow{Y}$
            such that $f$ is continuous, bijective, and an open mapping.
        \end{definition}
        There is an equivalent notion of homeomorphism given as follows.
        \begin{theorem}
            If $(X,\tau_{X})$ and $(Y,\tau_{Y})$ are topological spaces, and if
            $f:X\rightarrow{Y}$ is a function, then $f$ is a homeomorphism if
            and only if $f$ is continuous, bijective, and the inverse function
            $f^{-1}:Y\rightarrow{X}$ is continuous.
        \end{theorem}
        \begin{proof}
            If $f$ is a homeomorphism, then for all $\mathcal{U}\in\tau_{X}$ we
            have $(f^{-1})^{-1}[\mathcal{U}]=f[\mathcal{U}]$. That is, the
            pre-image of the inverse function $f^{-1}$ of $\mathcal{U}$ is given
            by the image $f[\mathcal{U}]$. But homeomorphisms are open mappings,
            so $(f^{-1})^{-1}[\mathcal{U}]$ is open, and therefore $f^{-1}$ is
            continuous. Conversely, if $f^{-1}$ is continuous, then $f$ is an
            open mapping.
        \end{proof}
        \begin{definition}{\textbf{(Topological Embedding)}}
            A topological embedding of a topological space $(X,\tau_{X})$ into a
            topological space $(Y,\tau_{Y})$ is a function $f:X\rightarrow{Y}$
            such that $f$ is a homeomorphism between $(X,\tau_{X})$ and
            $(f[X],\tau_{Y}|_{f[X]})$ where $\tau_{Y}|_{f[X]}$ is the subspace
            topology on $f[X]$.
        \end{definition}
        \begin{example}
            Surfaces in $\mathbb{R}^{3}$ of the form $\big(x,y,f(x,y)\big)$
            where $f:\mathbb{R}^{2}\rightarrow\mathbb{R}$ is a continuous
            function are topological embeddings of $\mathbb{R}^{2}$ into
            $\mathbb{R}^{3}$.
        \end{example}
        There are many \textit{separation} properties that ony studies in
        general topology. Two of the weaker ones are usually sufficient for a
        study of smooth manifolds.
        \begin{definition}{\textbf{(Hausdorff Topological Space)}}
            A Hausdorff topological space, also called a $T_{2}$ topological
            space, is a topological space $(X,\tau)$ such that for all
            $x,y\in{X}$ there are open subsets $\mathcal{U},\mathcal{V}\in\tau$
            such that $x\in\mathcal{U}$, $y\in\mathcal{V}$, and
            $\mathcal{U}\cap\mathcal{V}=\emptyset$.
        \end{definition}
        A Hausdorff space is one where points can be separated by open sets.
        Discrete topological spaces are Hausdorff, as are metric spaces, but
        the trivial topology is only Hausdorff if $X$ contains a single point
        (or if $X$ is empty).
        \begin{figure}
            \centering
            \includegraphics{../images/hausdorff_condition_001.pdf}
            \caption{The Hausdorff Condition on a Topological Space}
            \label{fig:hausdorff_condition_001}
        \end{figure}
        The Hausdorff property is usually imposed on manifolds, but need not be.
        Indeed, in Lorentz geometry, Lorentzian manifolds can give rise to
        \textit{non-Hausdorff manifolds}. Because of this it is useful to
        describe the following type of space.
        \begin{definition}{\textbf{(Fr\'{e}chet Topological Spaces)}}
            A Fr\'{e}chet topological space, also called a $T_{1}$ space, is a
            topological space $(X,\tau)$ such that for all $x,y\in{X}$ there
            are open sets $\mathcal{U},\mathcal{V}\in\tau$ such that
            $x\in\mathcal{U}$, $y\in\mathcal{V}$, and also $x\notin\mathcal{V}$
            and $y\notin\mathcal{U}$.
        \end{definition}
        Be careful not to confuse the term Fr\'{e}chet topological space with
        the term Fr\'{e}chet-Urysohn topological space, and do not confuse these
        with Fr\'{e}chet spaces, which occur in functional analysis. To reduce
        confusion many authors refer to Fr\'{e}chet topological spaces
        exclusively as $T_{1}$ spaces.
        \begin{figure}
            \centering
            \includegraphics{../images/frechet_condition_001.pdf}
            \caption{The Fr\'{e}chet Condition on a Topological Space}
            \label{fig:frechet_condition_001}
        \end{figure}
        The difference between a Fr\'{e}chet topological space and a Hausdorff
        topological space is that in a Hausdorff space the set $\mathcal{U}$
        and $\mathcal{V}$ must be disjoint, whereas in a Fr\'{e}chet
        topological space they are allowed to overlap. See
        Figs.~\ref{fig:hausdorff_condition_001} and
        \ref{fig:frechet_condition_001} for a visual comparison.
        \begin{theorem}
            If $(X,\tau)$ is a Hausdorff topological space, then it is a
            Fr\'{e}chet topological space.
        \end{theorem}
        \begin{proof}
            For let $x,y\in{X}$. Since $(X,\tau)$ is Hausdorff, there are
            disjoint open sets $\mathcal{U},\mathcal{V}\in\tau$ such that
            $x\in\mathcal{U}$ and $y\in\mathcal{V}$. But if $\mathcal{U}$ and
            $\mathcal{V}$ are disjoint, then $x\notin\mathcal{V}$ and
            $y\notin\mathcal{U}$. Thus, $(X,\tau)$ is a Fr\'{e}chet topological
            space.
        \end{proof}
        One of the useful properties of Hausdorffness is that it is preserved
        by products. That is, if $(X,\tau_{X})$ and $(Y,\tau_{Y})$ are
        Hausdorff topological spaces, then so is the product space
        $X\times{Y}$ with the topology generated by $\tau_{X}\times\tau_{Y}$.
        Similarly, Hausdorffness is inherited by subspaces. If
        $(X,\tau)$ is a Hausdorff topological space, $A\subseteq{X}$, and
        if $\tau|_{A}$ is the subspace topology, then
        $(A,\tau|_{A})$ is a Hausdorff topological space.
        \par\hfill\par
        Topologies that come from metrics are very nice and have all of the
        intuitive properties that make spaces easier to visualize. Manifolds
        have topologies that come from metrics. The natural way to go about this
        is by defining a metric on a manifold, usually by defining the length
        of a curve and stating that the distance between two points is the
        length of the \textit{shortest} curve between them, but this takes a lot
        of effort. The fact that manifolds have topologies that come from
        metrics stems from a few facts from point-set topology. These facts will
        be listed but not proved.
        \begin{definition}{\textbf{(Regular Topological Space)}}
            A regular topological space is a topological space $(X,\tau)$ such
            that for all $x\in{X}$ and every closed set
            $\mathcal{C}\subseteq{X}$ with $x\notin\mathcal{C}$, there are
            open sets $\mathcal{U},\mathcal{V}\in\tau$ such that
            $x\in\mathcal{U}$, $\mathcal{C}\subseteq\mathcal{V}$, and
            $\mathcal{U}\cap\mathcal{V}=\emptyset$.
        \end{definition}
        Some require regular spaces to be Hausdorff, others don't. Many say that
        a topological space that is both Hausdorff and regular is $T_{3}$.
        \begin{definition}{\textbf{(Open Covering)}}
            An open covering of a topological space $(X,\tau)$ is a set
            $\mathcal{O}\subseteq\tau$ such that $\bigcup\mathcal{O}=X$. That
            is, for all $x\in{X}$ there is an open set
            $\mathcal{U}\in\mathcal{O}$ such that $x\in\mathcal{U}$.
        \end{definition}
        \begin{example}
            If $(X,\tau)$ is a topological space, then $\tau$ is an open
            covering.
        \end{example}
        \begin{example}
            The set of all open intervals in $\mathbb{R}$:
            \begin{equation}
                \mathcal{O}=
                    \{\,(a,\,b)\;|\;a,b\in\mathbb{R}\textrm{ and }a<b\,\}
            \end{equation}
            is an open covering of $\mathbb{R}$.
        \end{example}
        \begin{example}
            The set of all open balls in $\mathbb{R}^{n}$:
            \begin{equation}
                \mathcal{O}=
                    \{\,B_{r}(\mathbf{x})\;|\;
                        r>0\textrm{ and }\mathbf{x}\in\mathbb{R}^{n}\,\}
            \end{equation}
            is an open covering of $\mathbb{R}^{n}$.
        \end{example}
        \begin{example}
            More generally, the set of all open balls in a metric space $(X,d)$:
            \begin{equation}
                \mathcal{O}=
                    \{\,B_{r}(x)\;|\;r>0\textrm{ and }x\in{X}\,\}
            \end{equation}
            is an open covering of $X$.
        \end{example}
        \begin{definition}{\textbf{(Refinement of an Open Covering})}
            A refinement of an open covering $\mathcal{O}$ of a topological
            space $(X,\tau)$ is an open covering $\Delta$ such that for all
            $\mathcal{U}\in\Delta$ there is a set $\mathcal{V}\in\mathcal{O}$
            such that $\mathcal{U}\subseteq\mathcal{V}$.
        \end{definition}
        A refinement of an open covering $\mathcal{O}$ is another open covering
        made by taking the open sets in $\mathcal{O}$ and making them
        \textit{smaller}.
        \begin{example}
            If $\mathcal{O}$ is the set of all open intervals in $\mathbb{R}$,
            and if $\Delta$ is the set of all open intervals with rational
            endpoints, then $\Delta$ is a refinement of $\mathcal{O}$.
        \end{example}
        \begin{example}
            If $\mathcal{O}$ is the set of all open balls in $\mathbb{R}^{n}$,
            and if $\Delta$ is the set of all open balls in $\mathbb{R}^{n}$
            with radius 1, then $\Delta$ is a refinement of $\mathcal{O}$.
        \end{example}
        \begin{example}
            If $\mathcal{O}$ is the set of all open balls in a metric space
            $(X,d)$, and if $\Delta$ is the set of all open balls of radius 1,
            then $\Delta$ is a refinement of $\mathcal{O}$.
        \end{example}
        \begin{definition}{\textbf{(Locally Finite Open Covering)}}
            A locally finite open covering of a topological space $(X,\tau)$
            is an open covering $\mathcal{O}$ of $(X,\tau)$ such that for all
            $x\in{X}$ there is an open set $\mathcal{U}\in\tau$ such
            that $x\in\mathcal{U}$ and only finitely many elements of
            $\mathcal{O}$ have non-empty intersection with $\mathcal{U}$.
        \end{definition}
        \begin{example}
            The set $\mathcal{O}$ defined by:
            \begin{equation}
                \mathcal{O}=\{\,(n-2,\,n+2)\;|\;n\in\mathbb{Z}\,\}
            \end{equation}
            is a locally finite open covering of $\mathbb{R}$. For any point
            $x\in\mathbb{R}$, the set $(x-1,x+1)$ intersects at most 3 of the
            sets in $\mathcal{O}$.
        \end{example}
        \begin{definition}{\textbf{(Compact Topological Space)}}
            A compact topological space is a topological space $(X,\tau)$
            such that for any open covering $\mathcal{O}$ there is a subset
            $\Delta\subseteq\mathcal{O}$ such that $\Delta$ is an open covering
            of $(X,\tau)$ and $\Delta$ is a finite set.
        \end{definition}
        The slogan for compactness is that
        \textit{every open cover has a finite subcover}. In general, manifolds
        need not be compact, but many nice properties exist when they are.
        All manifolds are \textit{locally compact}.
        \begin{definition}{\textbf{(Locally Compact Topological Space)}}
            A locally compact topological space is a topological space
            $(X,\tau)$ such that for all $x\in{X}$ there is an open set
            $\mathcal{U}\in\tau$ and a compact subspace $K\subseteq{X}$ such
            that $x\in\mathcal{U}$ and $\mathcal{U}\subseteq{K}$.
        \end{definition}
        Locally compact Hausdorff spaces are all the rage in functional
        analysis. To every locally compact Hausdorff space there is a
        $C^{*}$ algebra, and vice-versa, something that makes all of the
        analysts very excited. We'll need the following to prove that manifolds
        have topologies that come from metrics.
        \begin{theorem}
            If $(X,\tau)$ is a locally compact Hausdorff topological space,
            then $(X,\tau)$ is regular.
        \end{theorem}
        Another notion, far weaker than compactness, is tied to manifolds.
        \begin{definition}{\textbf{(Paracompact Topological Space)}}
            A paracompact topological space is a topological space $(X,\tau)$
            such that for all open coverings $\mathcal{O}$ there is a refinement
            $\Delta$ of $\mathcal{O}$ such that $\Delta$ is a locally finite
            open covering.
        \end{definition}
        The slogan for paracompactness is that
        \textit{every open cover has a locally finite refinement}.
        \begin{theorem}
            If $(X,\tau)$ is a compact topological space,
            then it is paracompact.
        \end{theorem}
        \begin{proof}
            Since $(X,\tau)$ is compact, given any open cover $\mathcal{O}$,
            there is a finite open cover $\Delta\subseteq\mathcal{O}$. If
            $x\in{X}$, let $\mathcal{U}=X$. Since $\Delta$ is finite, only
            finitely many elements of $\Delta$ have non-empty intersection with
            $\mathcal{U}$. Hence, $\Delta$ is a locally finite open covering,
            and $(X,\tau)$ is therefore paracompact.
        \end{proof}
        Paracompactness is strictly weaker than compactness.
        \begin{theorem}{\textbf{(Stone's Paracompactness Theorem)}}
            If $(X,d)$ is a metric space and if $\tau$ is the set of all
            metrically open subsets of $X$, then $(X,\tau)$ is a paracompact
            topological space.
        \end{theorem}
        That is, every metric space is paracompact, showing paracompactness is
        strictly weaker than compactness. Paracompactness is such a strange
        property, one may ask why it deserves a name. The reason is the
        connection between paracompactness and partitions of unity.
        \begin{definition}{\textbf{(Partition of Unity)}}
            A partition of unity of a topological space $(X,\tau)$ is a set
            $\mathcal{F}$ of continuous functions $f:X\rightarrow[0,1]$ such
            that for all $x\in{X}$ there is an open subset $\mathcal{U}\in\tau$
            such that for all but finitely many $f\in\mathcal{F}$, the
            restriction $f|_{\mathcal{U}}$ is the zero function, and such that
            for all $x\in{X}$ the following holds:
            \begin{equation}
                \sum_{f\in\mathcal{F}}f(x)=1
            \end{equation}
        \end{definition}
        Since only finitely many functions $f$ are such that $f(x)\ne{0}$ there
        is no issue of convergence in this sum, we are adding only finitely
        many numbers. Partitions of unity are usually tied to open coverings.
        \begin{definition}%
            {\textbf{(Partition of Unity Subordinate to an Open Covering)}}
            A partition of unity subordinate to an open covering $\mathcal{O}$
            of a topological space $(X,\tau)$ is a partition of unity
            $\mathcal{F}$ of $(X,\tau)$ such that for all $f\in\mathcal{F}$
            there is an open set $\mathcal{U}\in\mathcal{O}$ such that
            $\textrm{supp}(f)\subseteq\mathcal{O}$.
        \end{definition}
        For those unfamiliar with notation, $\textrm{supp}(f)$ is the
        \textit{support} of $f$. There are two definitions that are common:
        \begin{equation}
            \textrm{supp}(f)=\{\,x\in{X}\;|\;f(x)\ne{0}\,\}
        \end{equation}
        But also:
        \begin{equation}
            \textrm{supp}(f)
                =\textrm{Cl}_{\tau}\Big(\{\,x\in{X}\;|\;f(x)\ne{0}\,\}\Big)
        \end{equation}
        where $\textrm{Cl}_{\tau}(A)$ denotes the topological closure of the
        set $A\subseteq{X}$, which is the \textit{smallest} closed set
        containing $A$. That is, the intersection of all closed sets
        $\mathcal{C}\subseteq{X}$ such that $A\subseteq\mathcal{C}$. Since the
        entire space $X$ is closed, the collection of all closed sets containing
        $A$ is non-empty, meaning this intersection is well-defined. Since the
        arbitrary intersection of closed sets is closed, $\textrm{Cl}_{\tau}(A)$
        is closed. For partitions of unity we'll use the closure definition.
        Note that for a continuous function $f:X\rightarrow[0,1]$, the subset
        $(0,1]$ is open with respect to $[0,1]$, meaning $f^{-1}\big[(0,1]\big]$
        is open.
        \begin{theorem}
            If $(X,\tau)$ is a Fr\'{e}chet topological space, then $(X,\tau)$ is
            Hausdorff and paracompact if and only if for every open covering
            $\mathcal{O}$ of $(X,\tau)$ there is a partition of unity
            $\mathcal{F}$ subordinate to $\mathcal{O}$.
        \end{theorem}
        The proof involves several technical details from point-set topology.
        It is not particularly hard, just long. Partitions of unity have many
        useful applications to smooth manifolds, such as showing the existence
        of a \textit{Riemannian metric} on any smooth manifold.
        \par\hfill\par
        The last point-set property to discuss is that of a \textit{basis}.
        \begin{definition}{\textbf{(Topological Basis)}}
            A topological basis for a topological space $(X,\tau)$ is an open
            covering $\mathcal{B}$ for $(X,\tau)$ such that for all
            $\mathcal{U},\mathcal{V}\in\mathcal{B}$ and for all
            $x\in\mathcal{U}\cap\mathcal{V}$, there is an open set
            $\mathcal{W}\in\mathcal{B}$ such that $x\in\mathcal{W}$ and
            $\mathcal{W}\subseteq\mathcal{U}\cap\mathcal{V}$.
        \end{definition}
        \begin{example}
            If $(X,\tau)$ is a topological space, then $\tau$ is a basis.
        \end{example}
        Most topologies one encounters in the study of manifolds, metric spaces,
        normed spaces, and inner product spaces have uncountably many elements.
        A nice way to distinguish which topological spaces are \textit{small}
        and which are \textit{large} is not by asking if the topology is
        countable, but whether or not there exists a countable basis. Such
        spaces are given a name.
        \begin{definition}{\textbf{(Second-Countable Topological Space)}}
            A second-countable topological space is a topological space
            $(X,\tau)$ such that there exists a countable topological basis
            $\mathcal{B}$ for $(X,\tau)$.
        \end{definition}
        \begin{example}
            The real line $\mathbb{R}$ is second-countable. Take $\mathcal{B}$
            to be set set of all open intervals with rational endpoints. This
            set has size $\textrm{Card}(\mathbb{Q}\times\mathbb{Q})$,
            and is hence countable, and is also a basis for the standard
            topology on $\mathbb{R}$.
        \end{example}
        \begin{example}
            More generally, $\mathbb{R}^{n}$ is second-countable. Take the set
            of all open balls with radii $r\in\mathbb{Q}^{+}$ centered at
            points $\mathbf{x}$ with rational coordinates.
        \end{example}
        The finite product of second-countable spaces is second-countable
        and all subspaces of second-countable spaces are second-countable.
        In metric spaces, second-countability is equivalent to the notion of
        begin \textit{separable}, meaning there exists a countable dense
        subset. Second-countable always implies separable, but separable
        topological spaces (that are not metric spaces) need not be
        second-countable.
        \begin{theorem}{\textbf{(Urysohn's Meterization Theorem)}}
            If $(X,\tau)$ is a regular Hausdorff topological space that is
            second-countable, then there is a metric $d$ on $X$ such that
            $d$ generates the topology $\tau$. That is, $(X,\tau)$ is
            meterizable.
        \end{theorem}
    \section{Topological Manifolds}
        \begin{figure}
            \centering
            \includegraphics{../images/chart_in_a_manifold.pdf}
            \caption{A Chart in a Locally Euclidean Space}
            \label{fig:chart_in_a_manifold}
        \end{figure}
        \begin{definition}{\textbf{(Locally Euclidean Topological Space)}}
            A locally Euclidean topological space is a topological space
            $(X,\tau)$ such that for all $x\in{X}$ there exists an open
            subset $\mathcal{U}\in\tau$ and a continuous injective open
            mapping $f:\mathcal{U}\rightarrow\mathbb{R}^{n}$ for some
            $n\in\mathbb{N}$ such that $x\in\mathcal{U}$.
        \end{definition}
        Since $f$ is an injective continuous open mapping from $\mathcal{U}$ to
        $\mathbb{R}^{n}$, it is a homeomorphism from $\mathcal{U}$ to
        $f[\mathcal{U}]$, which is necessarily an open subset of
        $\mathbb{R}^{n}$ since $f$ is an open mapping. That is, an equivalent
        description of locally Euclidean topological spaces is that for every
        point $x\in{X}$ there is an open subset $\mathcal{U}\subseteq{X}$ with
        $x\in\mathcal{U}$, an open subset $\mathcal{V}\subseteq\mathbb{R}^{n}$,
        and a homeomorphism $f:\mathcal{U}\rightarrow\mathcal{V}$. For manifolds
        we usually reserve the letter $f$ for smooth functions from the manifold
        to the reals. These homeomorphisms are thus labelled either $\varphi$ or
        $\psi$. The ordered pair $(\mathcal{U},\varphi)$ is given a name.
        \begin{definition}{\textbf{(Chart)}}
            A chart in a topological space $(X,\tau)$ is an ordered pair
            $(\mathcal{U},\varphi)$ where $\mathcal{U}\in\tau$ and
            $\varphi:\mathcal{U}\rightarrow\mathbb{R}^{n}$ is a continuous
            injective open mapping.
        \end{definition}
        Yet another way to describe locally Euclidean topological spaces is that
        it is a topological space $(X,\tau)$ such that for all $x\in{X}$ there
        is a chart $(\mathcal{U},\varphi)$ such that $x\in\mathcal{U}$.
        See Fig.~\ref{fig:chart_in_a_manifold}.
        \begin{definition}{\textbf{(Atlas)}}
            An atlas for a topological space $(X,\tau)$ is a set
            $\mathcal{A}$ of charts on $(X,\tau)$ such that for all
            $x\in{X}$ there is a chart $(\mathcal{U},\varphi)\in\mathcal{A}$
            such that $x\in\mathcal{U}$.
        \end{definition}
        The words chart and atlas are not randomly chosen. Imagine the Earth.
        We can describe the Earth locally about a point with a square map.
        We cannot globally describe the Earth as a square map, since a sphere
        and a square are not homeomorphic. Nonetheless, if we collect a bunch
        of square maps into a book we can locally describe any point on Earth.
        Such a book is called an \textit{atlas} and the individual pages are
        called \textit{charts} in the atlas. In a similar way, an atlas for a
        topological space is a book of charts that locally describe any point
        in the space.
        \par\hfill\par
        A few questions immediately arise. Is it possible for a point to be
        locally like $\mathbb{R}^{n}$ and $\mathbb{R}^{m}$ with $m\ne{n}$?
        \begin{theorem}{\textrm{(Invariance of Dimension)}}
            If $\mathcal{U}\subseteq\mathbb{R}^{n}$ and
            $\mathcal{V}\subseteq\mathbb{R}^{m}$ are non-empty open subsets,
            and if $f:\mathcal{U}\rightarrow\mathcal{V}$ is a homeomorphism,
            then $n=m$.
        \end{theorem}
        The proof is surprisingly hard, requiring Brauer's
        \textit{invariance of domain} theorem which is proved using tools from
        algebraic topology. Nevertheless, let us use this.
        \begin{theorem}
            If $(X,\tau)$ is a topological space, if $x\in{X}$, if
            $(\mathcal{U},\varphi)$ is an $n$ dimensional chart,
            $(\mathcal{V},\psi)$ is an $m$ dimensional chart, and if
            $x\in\mathcal{U}$ and $x\in\mathcal{V}$, then $n=m$.
        \end{theorem}
        \begin{proof}
            Since $\mathcal{U}$ and $\mathcal{V}$ are open, so is
            $\mathcal{U}\cap\mathcal{V}$. Let
            $\mathcal{W}=\mathcal{U}\cap\mathcal{V}$. Since $x\in\mathcal{U}$
            and $x\in\mathcal{V}$, $\mathcal{W}$ is non-empty. Since $\varphi$
            is a continuous injective open mapping, $\varphi|_{\mathcal{W}}$ is
            a homeomorphism onto it's image. Similarly,
            $\psi^{-1}|_{\psi[\mathcal{W}]}$ is a homeomorphism from
            $\psi[\mathcal{W}]$ to $\mathcal{W}$. But then
            $\varphi|_{\mathcal{W}}\circ\psi^{-1}|_{\psi[\mathcal{W}]}$ is the
            composition of homeomorphisms, which is a homeomorphism. Moreover,
            $\varphi|_{\mathcal{W}}\circ\psi^{-1}|_{\psi[\mathcal{W}]}$
            is a homeomorphism from $\psi[\mathcal{W}]$ to
            $\varphi[\mathcal{W}]$, a non-empty open subset in $\mathbb{R}^{m}$
            to a non-empty open subset in $\mathbb{R}^{n}$. By the invariance
            of dimension, $m=n$.
        \end{proof}
        The definition of locally Euclidean does not exclude the disjoint
        union of a line with a plane. This theorem prevents a point from being
        locally like $\mathbb{R}^{n}$ and like $\mathbb{R}^{m}$ unless $m=n$.
        If $(X,\tau)$ is a \textit{connected} locally Euclidean topological
        space, then each point $x\in{X}$ is locally like $\mathbb{R}^{n}$ for
        some \textit{fixed} integer $n\in\mathbb{N}$. In this case we can
        unambiguously say that this is the \textit{dimension} of $(X,\tau)$.
        \par\hfill\par
        \begin{figure}
            \centering
            \includegraphics{../images/bug_eyed_line_001.pdf}
            \caption{Bug-Eyed Line Construction}
            \label{fig:bug_eyed_line_construction}
        \end{figure}
        Locally Euclidean spaces need not be Hausdorff, strange as it is.
        Probably the easiest example to describe is the \textit{bug-eyed line},
        also known as the line with two origins.
        Take $\mathbb{R}\times\{-1,1\}$, two disjoint lines in the plane.
        Glue $(x,1)$ together with $(x,-1)$ for all $x\ne{0}$. By \textit{glue}
        it is meant to use the quotient topology. The result is a space that
        is locally Euclidean but not Hausdorff. The points $(0,1)$ and $(0,-1)$
        can not be separated by disjoint open sets. The construction is shown
        in Fig.~\ref{fig:bug_eyed_line_construction}. To see this, note that
        open sets containing $(0,1)$ or $(0,-1)$ must also contain points that
        are $\pm\varepsilon$ away from the center, see
        Fig.~\ref{fig:bug_eyed_line_open_sets}. Thus, if $\mathcal{U}$
        contains $(0,1)$ and $\mathcal{V}$ contains $(0,-1)$, then
        $\mathcal{U}\cap\mathcal{V}$ is non-empty.
        \begin{figure}
            \centering
            \includegraphics{../images/bug_eyed_line_002.pdf}
            \caption{Open Sets in the Bug-Eyed Line}
            \label{fig:bug_eyed_line_open_sets}
        \end{figure}
        This space is, however, locally Euclidean. For points other than
        $(0,1)$ and $(0,-1)$ this is intuitively clear, and for the two origins
        the open sets in Fig.~\ref{fig:bug_eyed_line_open_sets} show that those
        points are also locally Euclidean. This space is a Fr\'{e}chet
        topological space, however, again the sets in
        Fig.~\ref{fig:bug_eyed_line_open_sets} show why this is true. This is
        true of all locally Euclidean spaces.
        \begin{theorem}
            If $(X,\tau)$ is a locally Euclidean topological space, then
            $(X,\tau)$ is a Fr\'{e}chet ($T_{1}$) topological space.
        \end{theorem}
        \begin{proof}
            If $x,y\in{X}$ are distinct points such that there are charts
            $(\mathcal{U},\varphi)$ and $(\mathcal{V},\psi)$ with
            $x\in\mathcal{U}$, $x\notin\mathcal{V}$ and
            $y\in\mathcal{V}$, $y\notin\mathcal{U}$, then we are done.
            Otherwise, there is a chart $(\mathcal{U},\varphi)$ such that
            $x\in\mathcal{U}$ and $y\in\mathcal{U}$. Since
            $\varphi:\mathcal{U}\rightarrow\mathbb{R}^{n}$ is injective and
            $x\ne{y}$ we have $\varphi(x)\ne\varphi(y)$. Since $\mathbb{R}^{n}$
            is Hausdorff there are open sets $\tilde{\mathcal{V}}_{0}$ and
            $\tilde{\mathcal{V}}_{1}$ with
            $\varphi(x)\in\tilde{\mathcal{V}}_{0}$,
            $\varphi(y)\in\tilde{\mathcal{V}}_{1}$, and
            $\tilde{\mathcal{V}}_{0}\cap\tilde{\mathcal{V}}_{1}=\emptyset$.
            Since $\varphi$ is an open mapping, $\varphi[\mathcal{U}]$ is open.
            Thus, $\varphi[\mathcal{U}]\cap\tilde{\mathcal{V}}_{0}$ and
            $\varphi[\mathcal{U}]\cap\tilde{\mathcal{V}}_{1}$ are open disjoint
            sets, with
            $\varphi(x)\in\varphi[\mathcal{U}]\cap\tilde{\mathcal{V}}_{0}$ and
            $\varphi(y)\in\varphi[\mathcal{U}]\cap\tilde{\mathcal{V}}_{0}$.
            Let $\mathcal{V}_{0}=\varphi^{-1}[\varphi[\mathcal{U}]\cap\tilde{\mathcal{V}}_{0}]$
            and $\mathcal{V}_{1}=\varphi^{-1}[\varphi[\mathcal{U}]\cap\tilde{\mathcal{V}}_{1}]$.
            Since $\varphi$ is continuous, $\mathcal{V}_{0}$ and
            $\mathcal{V}_{1}$ are open. Moreover,
            $x\in\mathcal{V}_{0}$, $y\in\mathcal{V}_{1}$, and
            $\mathcal{V}_{0}\cap\mathcal{V}_{1}=\emptyset$. But then
            $x\in\mathcal{V}_{0}$, $x\notin\mathcal{V}_{1}$ and
            $y\in\mathcal{V}_{1}$, $y\notin\mathcal{V}_{0}$. That is,
            $(X,\tau)$ is a Fr\'{e}chet topological space.
        \end{proof}
        \begin{theorem}
            If $(X,\tau)$ is a locally Euclidean topological space, then it
            is locally compact.
        \end{theorem}
        \begin{proof}
            Since $(X,\tau)$ is locally Euclidean, if $x\in{X}$, then there is
            a chart $(\mathcal{U},\varphi)$ such that $x\in\mathcal{U}$. But
            then $\varphi[\mathcal{U}]$ is an open subset of $\mathbb{R}^{n}$.
            Thus there exists an $r>0$ such that the open ball
            $B_{r}\big(\varphi(x)\big)$ is contained in $\varphi[\mathcal{U}]$.
            Let $\varepsilon=r/2$. Then
            $B_{\varepsilon}\big(\varphi(x)\big)\subseteq\varphi[\mathcal{U}]$,
            but also
            $\bar{B}_{\varepsilon}\big(\varphi(x)\big)\subseteq\varphi[\mathcal{U}]$
            where $\bar{B}_{\varepsilon}\big(\varphi(x)\big)$ is the closed ball
            around $\varphi(x)$ of radius $\varepsilon$. Since
            closed balls are closed and bounded, by the Heine-Borel theorem
            $\bar{B}_{\varepsilon}\big(\varphi(x)\big)$ is compact. Let
            $K=\varphi^{-1}\big[\bar{B}_{\varepsilon}\big(\varphi(x)\big)\big]$
            and
            $\mathcal{U}=\varphi^{-1}\big[B_{\varepsilon}\big(\varphi(x)\big)\big]$.
            Since $\varphi$ is a homeomorphism onto it's image,
            $K$ is compact and $\mathcal{U}$ is open. Moreover,
            $x\in\mathcal{U}$ and $\mathcal{U}\subseteq{K}$. Hence,
            $(X,\tau)$ is locally compact.
        \end{proof}
        Lastly, locally Euclidean does not prohibit a space from being massive.
        The space in mind is the \textit{long line}. The real line can be
        thought of as countably many copies of the half-open interval $[0,1)$
        placed next to each other. The long line can be thought of as
        \textit{uncountably} many such copies. The name is very appropriate.
        It is like the real line, but \textit{significantly} longer. This space
        is neither second-countable nor paracompact. Since subspaces of
        second-countable spaces are second-countable, this shows that it is
        impossible to embed the long line into $\mathbb{R}^{n}$ for any
        $n\in\mathbb{N}$. This is at odds with the celebrated Whitney embedding
        theorem which says that every smooth manifold can be embedded into
        $\mathbb{R}^{n}$ for sufficiently large $n$. If one wants this statement
        to be true, the definition of manifold must exclude the long line.
        \begin{definition}{\textbf{(Topological Manifold)}}
            A topological manifold is a topological space $(X,\tau)$ that is
            Hausdorff, locally Euclidean, and second-countable.
        \end{definition}
        I'd like to think the reason for the Hausdorff and second-countable
        requirements is so that the Whitney embedding theorem has the hope of
        being true. Neither the long line nor the bug-eyed line can be
        embedded into $\mathbb{R}^{n}$ for any $n$ since subspaces of
        $\mathbb{R}^{n}$ are both Hausdorff and second-countable. There are
        several \textbf{inequivalent} definitions of topological manifold. Some
        say that topological manifolds are Hausdorff, locally Euclidean, and
        paracompact. This is strictly weaker. The disjoint union of uncountably
        many copies of the unit circle $\mathbb{S}^{1}$ is Hausdorff, locally
        Euclidean, and paracompact, but not second-countable. However, all
        topological manifolds are paracompact.
        \begin{theorem}
            If $(X,\tau)$ is a topological manifold, then it is meterizable.
        \end{theorem}
        \begin{proof}
            Since $(X,\tau)$ is locally Euclidean, it is locally compact.
            But locally compact Hausdorff topological spaces are regular.
            Since manifolds are also second-countable,
            by Urysohn's meterization theorem $(X,\tau)$ is meterizable.
        \end{proof}
        \begin{theorem}
            If $(X,\tau)$ is a topological manifold, then $(X,\tau)$ is
            paracompact.
        \end{theorem}
        \begin{proof}
            Since manifolds are meterizable,
            by Stone's paracompactness theorem $(X,\tau)$ is paracompact.
        \end{proof}
        If we add connectedness, the reverse is true.
        \begin{theorem}
            If $(X,\tau)$ is a Hausdorff, second-countable, connected,
            paracompact topological space, then $(X,\tau)$ is a topological
            manifold.
        \end{theorem}
        The only thing missing is second-countability. The proof involves mostly
        ideas from point-set topology and cannot be made as succinct as the
        previous theorem. Some authors state that a manifold is a
        \textit{connected} locally Euclidean Hausdorff space that is
        paracompact. This is equivalent to a connected locally Euclidean
        Hausdorff space that is second-countable. For us, topological manfifolds
        need not be connected, but they do need to be second-countable.
    \section{Smooth Manifolds}
        \begin{figure}
            \centering
            \includegraphics{../images/smoothly_overlapping_charts.pdf}
            \caption{Smoothly Compatible Charts}
            \label{fig:smoothly_overlapping_charts}
        \end{figure}
        Topological manifolds do not have a notion of calculus. The only way
        we know how to compute derivatives is in $\mathbb{R}^{n}$
        (or in a more general Banach space via the Fr\'{e}chet derivative).
        The notion of the derivative of a function
        $f:X\rightarrow{Y}$ between topological spaces is meaningless, in
        general. Smooth manifolds are topological manifolds with added structure
        so that it makes sense to ask a function
        $f:X\rightarrow{Y}$ from one manifold to another is differentiable.
        \begin{definition}{\textbf{(Smoothly Compatible Charts)}}
            Smoothly compatible charts in a topological space
            $(X,\tau)$ are charts $(\mathcal{U},\varphi)$,
            $(\mathcal{V},\psi)$ is $(X,\tau)$ such that either
            $\mathcal{U}\cap\mathcal{V}=\emptyset$ or such that the functions
            $\varphi\circ\psi^{-1}$ and $\psi\circ\varphi^{-1}$, defined between
            $\psi[\mathcal{U}\cap\mathcal{V}]$ and
            $\varphi[\mathcal{U}\cap\mathcal{V}]$, are smooth functions.
        \end{definition}
        The \textit{transition maps} $\psi\circ\varphi^{-1}$ take points from
        an open subset
        $\varphi[\mathcal{U}\cap\mathcal{V}]\subseteq\mathbb{R}^{n}$ to an
        open subset $\psi[\mathcal{U}\cap\mathcal{V}]\subseteq\mathbb{R}^{n}$
        meaning it is perfectly valid to ask if this function is differentiable.
        See Fig.~\ref{fig:smoothly_overlapping_charts}.
        \begin{definition}{\textbf{(Smooth Atlas)}}
            A smooth atlas on a locally Euclidean topological space
            $(X,\tau)$ is an atlas $\mathcal{A}$ on $(X,\tau)$ such that for
            all charts $(\mathcal{U},\varphi)$ and $(\mathcal{V},\psi)$ in
            $\mathcal{A}$ it is true that they are smoothly compatible.
        \end{definition}
        There can be different smooth atlases on a given locally Euclidean
        space. Two atlases $\mathcal{A}_{1}$ and $\mathcal{A}_{2}$ are
        compatible if every chart in $\mathcal{A}_{1}$ is compatible with
        every chart in $\mathcal{A}_{2}$, and vice-versa. Two atlases are
        compatible if and only if $\mathcal{A}_{1}\cup\mathcal{A}_{2}$ is a
        smooth atlases.
        \par\hfill\par
        Many authors define smooth atlases on topological manifolds, and not
        on the more general locally Euclidean spaces. Several notions from
        differential topology such as tangent spaces and immersions still make
        sense for locally Euclidean spaces equipped with a smooth atlas. Since
        non-Hausdorff manifolds can arise in the study of Lorentz geometry and
        pseudo-Riemannian manifolds, the restriction of such definitions to
        topological manifolds is unnecessary.
        \begin{definition}{\textbf{(Maximal Smooth Atlas)}}
            A maximal smooth atlas on a locally Euclidean topological space
            $(X,\tau)$ is a smooth atlas $\mathcal{A}$ such that for every
            chart $(\mathcal{U},\varphi)$ on $(X,\tau)$ that is smoothly
            compatible with every chart in $\mathcal{A}$, it is true that
            $(\mathcal{U},\varphi)\in\mathcal{A}$.
        \end{definition}
        There are two ways to prove the existence and uniqueness of a maximal
        smooth atlas containing a given atlas. The first way is the standard,
        given a smooth atlas $\mathcal{A}$ you define $\tilde{\mathcal{A}}$ to
        be the set of all charts that are smoothly compatible with charts in
        $\mathcal{A}$. Given two charts $(\tilde{\mathcal{U}},\tilde{\varphi})$
        and $(\tilde{\mathcal{V}},\tilde{\psi})$ in $\tilde{\mathcal{A}}$ such
        that $\tilde{\mathcal{U}}\cap\tilde{\mathcal{V}}\ne\emptyset$, and a
        point $x$ contained in this intersection, choose a
        chart $(\mathcal{W},\phi)\in\mathcal{A}$ such that $x\in\mathcal{W}$.
        Such a chart exists since $\mathcal{A}$ is a smooth atlas and hence
        covers $X$. Then we have, locally around $x$, the following:
        \begin{align}
            \tilde{\psi}\circ\tilde{\varphi}^{-1}
                &=\tilde{\psi}\circ\big(
                    \phi^{-1}\circ\phi
                \big)\circ\varphi^{-1}\\
                &=\big(\tilde{\psi}\circ\phi^{-1}\big)\circ
                    \big(\phi\circ\varphi^{-1}\big)
        \end{align}
        But $(\tilde{\mathcal{U}},\tilde{\varphi})$ and
        $(\tilde{\mathcal{V}},\tilde{\psi})$ are smoothly compatible with all
        charts in $\mathcal{A}$, and hence are smoothly compatible with
        $(\mathcal{W},\phi)$. Therefore
        $\tilde{\psi}\circ\phi^{-1}$ and $\phi\circ\tilde{\varphi}^{-1}$ are
        smooth. Since the composition of smooth functions is smooth,
        $\tilde{\psi}\circ\tilde{\varphi}^{-1}$ is smooth. In a similar manner,
        $\tilde{\varphi}\circ\tilde{\psi}^{-1}$ is smooth. Hence,
        $\tilde{\mathcal{A}}$ is a smooth atlas, and it is certainly maximal.
        \par\hfill\par
        The second proof uses Zorn's lemma. Note that since the union of
        compatible atlases is again a smooth atlas, the set of all smooth
        atlases on $(X,\tau)$ is partially ordered by inclusion $\subseteq$
        and every chain in this partial ordering is bounded (take the union
        over all elements in the chain). By Zorn's lemma, given $\mathcal{A}$,
        there is a unique maximal element $\tilde{\mathcal{A}}$ such that
        $\mathcal{A}\subseteq\tilde{\mathcal{A}}$. A bit quicker, but uses
        Zorn's lemma and some people don't like that.
        \begin{definition}{\textbf{(Smooth Manifold)}}
            A smooth manifold is an ordered triple $(X,\tau,\mathcal{A})$
            such that $(X,\tau)$ is a topological manifold and $\mathcal{A}$
            is a maximal smooth atlas on $(X,\tau)$.
        \end{definition}
        The maximality condition avoids giving two ordered triples
        $(\mathbb{S}^{2},\tau|_{\mathbb{S}^{2}},\mathcal{A}_{1})$ and
        $(\mathbb{S}^{2},\tau|_{\mathbb{S}^{2}},\mathcal{A}_{2})$ different
        names even though they both mean the standard smooth structure on the
        sphere. There are other technical reasons for such a definition.
        \par\hfill\par
        Two comments. The inverse of the function $\varphi\circ\psi^{-1}$
        is $\psi\circ\varphi^{-1}$ meaning the function and its inverse are
        smooth, and bijections onto their image, meaning it is a
        diffeomorphism from an open subset of $\mathbb{R}^{n}$ to another open
        subset of $\mathbb{R}^{n}$. Next, the topology $\tau$ in a smooth
        manifold $(X,\tau,\mathcal{A})$ is recoverable from $X$ and
        $\mathcal{A}$. A set $\mathcal{U}\subseteq{X}$ is open if and only if
        for all charts $(\mathcal{V},\psi)\in\mathcal{A}$ the set
        $\psi\big[\mathcal{U}\cap\mathcal{V}]$ is open in $\mathbb{R}^{n}$. In
        one direction, if $\mathcal{U}$ is open, then
        $\mathcal{U}\cap\mathcal{V}$ is open, and since $\psi$ is an open
        mapping, $\psi[\mathcal{U}\cap\mathcal{V}]$ is open. In the other,
        for all $x\in\mathcal{U}$ there is a chart
        $(\mathcal{V}_{x},\psi_{x})$ with $x\in\mathcal{V}_{x}$. By hypothesis
        $\psi_{x}[\mathcal{U}\cap\mathcal{V}_{x}]$ is open,
        and since $\psi_{x}$ is a continuous function,
        $\mathcal{U}\cap\mathcal{V}_{x}$ is open. But $\mathcal{U}$ can be
        written as the union of all such $\mathcal{U}\cap\mathcal{V}_{x}$ as
        $x$ varies over $\mathcal{U}$. Since the union of open sets is open,
        $\mathcal{U}$ is open.
        \par\hfill\par
        Because of this it is standard to denote a smooth manifold by
        $(X,\mathcal{A})$ where $\mathcal{A}$ is a maximal smooth atlas. It is
        even more standard to denote a smooth manifold as just $X$ (or $M$ for
        \textit{manifold}), but this implicitly assumes there is only one
        possible smooth structure on a topological manifold, and this is false.
        Milnor showed there are \textit{exotic spheres}, non-standard smooth
        structures on the 7 dimensional sphere $\mathbb{S}^{7}$. What's more,
        there are \textit{uncountably} many different smooth structures on
        $\mathbb{R}^{4}$. If this sounds bizarre, it's because it is. For
        dimensions 0, 1, 2, and 3, every topological manifold is also a
        smooth manifold, and there is essentiallya unique smooth structure for
        it. For dimensions beyond our intuition, things become strange. Indeed,
        Kervaire showed there are higher dimensional topological manifolds that
        cannot be smoothed, meaning there is no smooth structure on it at all.
        \begin{figure}
            \centering
            \includegraphics{../images/sphere_orthographic_projection.pdf}
            \caption{Orthographic Projection of $\mathbb{S}^{2}$}
            \label{fig:sphere_orthographic_projection}
        \end{figure}
        \begin{example}
            $\mathbb{R}^{n}$ is a manifold, for all $n\in\mathbb{N}$, by taking
            the single chart $(\mathbb{R}^{n},\textrm{id}_{\mathbb{R}^{n}})$.
            The resulting smooth manifold is called the \textit{standard} smooth
            structure on $\mathbb{R}^{n}$.
        \end{example}
        \begin{example}
            It is easy to endow $\mathbb{R}$ with another smooth structure. Take
            the single chart $(\mathbb{R},\varphi)$ with $\varphi(x)=x^{3}$.
            This chart is not compatible with
            $(\mathbb{R},\textrm{id}_{\mathbb{R}})$ since the composition
            $\textrm{id}_{\mathbb{R}}\circ\varphi^{-1}$ is given by
            $\varphi^{-1}(x)=x^{1/3}$ and this function is not differentiable
            at the origin. The resulting smooth structure generated by
            $(\mathbb{R},\varphi)$ is therefore different then the standard
            smooth structure on $\mathbb{R}$. This smooth structure is, in a
            sense, \textit{the same} as the standard one since it is
            \textit{diffeomorphic} to it, a term to be described soon.
            Indeed, all smooth structures on $\mathbb{R}$ are diffeomorphic to
            the standard one.
        \end{example}
        \begin{example}
            The $N$ dimensional sphere $\mathbb{S}^{N}$ can be given a smooth
            structure in two ways. The first is via
            \textit{orthographic projections} by covering the sphere in
            $2N+2$ charts. The charts are of the form:
            \begin{align}
                \mathcal{U}_{n}^{+}&=
                    \{\,\mathbf{x}\in\mathbb{S}^{N}\;|\;
                        \mathbf{x}_{n}>0\,\}\\
                \mathcal{U}_{n}^{-}&=
                    \{\,\mathbf{x}\in\mathbb{S}^{N}\;|\;
                        \mathbf{x}_{n}<0\,\}
            \end{align}
            The charts $\varphi_{n}$ are defined by:
            \begin{equation}
                \varphi_{n}(\mathbf{x})
                    =(\mathbf{x}_{1},\dots,\mathbf{x}_{n-1},\mathbf{x}_{n+1},
                    \dots,\mathbf{x}_{N+1})
            \end{equation}
            These functions are called the \textit{orthographic projections}
            of $\mathbb{S}^{N}$ along the $n^{th}$ axis. The charts
            $(\mathcal{U}_{n}^{\pm},\varphi_{n})$ cover the sphere
            $\mathbb{S}^{n}$ and the inverse of
            $\varphi_{n}$ is given by:
            \begin{equation}
                \varphi_{n}^{-1}(\mathbf{y})
                    =(\mathbf{y}_{1},\dots,\mathbf{y}_{n-1},
                    \sqrt{1-||\mathbf{y}||^{2}},
                    \mathbf{y}_{n+1},\dots,\mathbf{y}_{N+1})
            \end{equation}
            The composition of $\varphi_{m}$ with $\varphi_{n}^{-1}$ is:
            \begin{equation}
                (\varphi_{m}\circ\varphi_{n}^{-1})(\mathbf{y})
                    =(\mathbf{y}_{1},\dots,\mathbf{y}_{m-1},\mathbf{y}_{m+1},
                    \dots,\mathbf{y}_{n-1},\sqrt{1-||\mathbf{y}||^{2}},
                    \mathbf{y}_{n+1},\dots,\mathbf{y}_{N+1})
            \end{equation}
            For $\mathbf{y}$ to be in the domain of $\varphi_{n}^{-1}$ means
            $||\mathbf{y}||<1$, so $\sqrt{1-||\mathbf{y}||}$ is never zero, and
            hence this function is smooth. These $2N+2$ charts show that
            $\mathbb{S}^{N}$ is a smooth manifold of dimension $N$
            (See Fig.~\ref{fig:sphere_orthographic_projection}).
        \end{example}
        \begin{figure}
            \centering
            \includegraphics{../images/sphere_stereographic_projection.pdf}
            \caption{Stereographic Projection of $\mathbb{S}^{2}$}
            \label{fig:sphere_stereographic_projection}
        \end{figure}
        \begin{example}
            The simpler method of showing $\mathbb{S}^{N}$ is a smooth manifold
            uses only two charts. Define $\mathcal{U}^{+}$ as:
            \begin{equation}
                \mathcal{U}^{+}=
                    \{\,\mathbf{x}\in\mathbb{S}^{N}\;|\;
                        \mathbf{x}_{N+1}\ne{1}\,\}
            \end{equation}
            And similarly define $\mathcal{U}^{-}$ as follows:
            \begin{equation}
                \mathcal{U}^{+}=
                    \{\,\mathbf{x}\in\mathbb{S}^{N}\;|\;
                        \mathbf{x}_{N+1}\ne{-1}\,\}
            \end{equation}
            The functions $\varphi_{+}:\mathcal{U}^{+}\rightarrow\mathbb{R}^{N}$
            and $\varphi_{-}:\mathcal{U}^{-}\rightarrow\mathbb{R}^{N}$ are given
            as follows:
            \begin{align}
                \varphi_{+}(\mathbf{x})=
                    \Big(\frac{\mathbf{x}_{1}}{1-\mathbf{x}_{N+1}},\cdots,
                        \frac{\mathbf{x}_{N}}{1-\mathbf{x}_{N+1}}\Big)\\
                \varphi_{-}(\mathbf{x})=
                    \Big(\frac{\mathbf{x}_{1}}{1+\mathbf{x}_{N+1}},\cdots,
                        \frac{\mathbf{x}_{N}}{1+\mathbf{x}_{N+1}}\Big)
            \end{align}
            There is no fear of division by zero since in both cases we've
            excluded the possible values for $\mathbf{x}_{N+1}$ that may result
            in division by zero from the domain of definition. Both of these
            functions have inverses given as follows:
            \begin{align}
                \varphi_{+}^{-1}(\mathbf{y})&=
                    \Big(\frac{2\mathbf{y}_{1}}{1+||\mathbf{y}||^{2}},\cdots,
                        \frac{2\mathbf{y}_{N}}{1+||\mathbf{y}||^{2}},
                        \frac{-1+||\mathbf{y}||^{2}}{1+||\mathbf{y}||^{2}}
                    \Big)\\
                    \varphi_{-}^{-1}(\mathbf{y})&=
                        \Big(\frac{2\mathbf{y}_{1}}{1+||\mathbf{y}||^{2}},\cdots,
                            \frac{2\mathbf{y}_{N}}{1+||\mathbf{y}||^{2}},
                            \frac{1-||\mathbf{y}||^{2}}{1+||\mathbf{y}||^{2}}
                        \Big)
            \end{align}
            The transition maps $\varphi_{+}\circ\varphi_{-}^{-1}$ and
            $\varphi_{-}\circ\varphi_{+}^{-1}$ are rational functions in each
            coordinate, and are therefore smooth. These are called the
            \textit{stereographic projections} of the sphere about the north
            and south pole, respectively.
        \end{example}
        Stereographic projection may be interpreted as placing an observer on
        the north pole. Given a point $\mathbf{x}$ on the sphere, draw the line
        from the observer passing through this point. Stereographic projection
        takes $\mathbf{x}$ to the intersection of this line with the plane
        $\mathbf{x}_{N+1}=0$ (See
        Fig.~\ref{fig:sphere_stereographic_projection}). We obtain the so-called
        \textit{near-sided} and \textit{far-sided} projections if we place our
        observer somewhere else on the $\mathbf{x}_{N+1}$ axis.
        See Figs.~\ref{fig:near_and_far_sided_projections}. Stereographic
        projection can be seen as the far-sided projection of an observer at
        the north pole, and orthographic projection is the near-sided projection
        of an observer at infinity.
        \begin{figure}
            \centering
            \begin{subfigure}[b]{0.49\textwidth}
                \centering
                \includegraphics{../images/sphere_near_sided_projection.pdf}
                \caption{Near Sided Projection of $\mathbb{S}^{2}$}
                \label{fig:sphere_near_sided_projection}
            \end{subfigure}
            \hfill
            \begin{subfigure}[b]{0.49\textwidth}
                \centering
                \includegraphics{../images/sphere_far_sided_projection.pdf}
                \caption{Far Sided Projection of $\mathbb{S}^{2}$}
                \label{fig:sphere_far_sided_projection}
            \end{subfigure}
            \caption{Near and Far Sided Projections of $\mathbb{S}^{2}$.}
            \label{fig:near_and_far_sided_projections}
        \end{figure}
        Let's use these ideas to visualize a surprising theorem from complex
        analysis. Given a complex polynomial $p(z)$ of degree $N$, one knows by
        the fundamental theorem of algebra that there are at most $N$ roots.
        If one picks a random point in the plane $z_{0}$ and then applies
        Newton's method:
        \begin{equation}
            z_{n+1}=z_{n}+\frac{p(z_{n})}{p'(z_{n})}
        \end{equation}
        one may be surprised to learn that they'll almost always converge to a
        root, regardless of choice of $z_{0}$. The question is then
        \textit{which root}? Let's take $p(z)=z^{3}-1$ for a simplistic example.
        The three roots are 1, $(-1+i\sqrt{3})/2$, and $(-1-i\sqrt{3})/2$. Color
        a point $z_{0}$ \textit{red} if the sequence above it converges to 1,
        \textit{green} if it converges to $(-1+i\sqrt{3})/2$, \textit{blue} if
        it converges to $(-1-i\sqrt{3})/2$, and \textit{black} if it doesn't
        converge. The resulting image is quite surprising
        (See Fig.~\ref{fig:newton_fractal_cubic}. While this may be beautiful,
        one may ask what this has to do with manifolds and why one should care.
        A paper by Hubbard, Schleicher, and Sutherland used the topology of
        these regions to present an algorithm that can find, for any complex
        $N$ degree polynomial $p$, all of the roots of $p$. The key to this
        proof is that these \textit{basis} are connected to infinity. Meaning
        one could start at a root $z_{0}$ of $p$ and walk along a path
        $\alpha(t)$ to infinity in the complex plane in such a way that for all
        $t_{0}$, Newton's method applied to $\alpha(t_{0})$ converges to
        $z_{0}$. To visualize this we need to wrap the entire plane up into a
        ball so that way may see the entire complex plane in one picture.
        Stereographic projection and orthographic projection helps us achieve
        this. Let us invent an algorithm to obtain
        Fig.~\ref{fig:newton_fractal_on_sphere}. Pick a point $\mathbf{x}$ in
        the square $[-1,1]^{2}$. If $||\mathbf{x}||>1$, color this point black
        and move on. Otherwise, compute the inverse orthographic projection of
        this point onto the sphere $\mathbb{S}^{2}$. Label this point
        $\mathbf{p}$. Compute the stereographic projection of $\mathbf{p}$,
        given us a point $\mathbf{z}$ in the plane. Treat $\mathbf{z}$ as a
        complex number $\mathbf{z}=(x,y)=x+iy$. Perform Newton's method for
        this point for the given polynomial and apply our coloring scheme. Do
        this for all points. The result is a drawing of $\mathbb{S}^{2}$ onto
        a 2D canvas depicting the Newton Fractal of $p(z)=z^{3}-1$. This allows
        us to see the behavior \textit{at infinity}.
        \begin{figure}
            \centering
            \resizebox{\textwidth}{!}{%
                \includegraphics{../images/newton_fractal_cubic.png}%
            }
            \caption{Newton Fractal for $z^{3}-1$}
            \label{fig:newton_fractal_cubic}
        \end{figure}
        \begin{figure}
            \centering
            \resizebox{\textwidth}{!}{%
                \includegraphics{../images/newton_fractal_on_sphere.png}%
            }
            \caption{Newton Fractal for $z^{3}-1$ on $\mathbb{S}^{2}$}
            \label{fig:newton_fractal_on_sphere}
        \end{figure}
        \par\hfill\par
        Other examples are obtained by taking examples we know about
        (like $\mathbb{R}^{N}$ and $\mathbb{S}^{N}$) and looking at products,
        subspaces, and quotients of these. The subspace of a topological
        manifold need not be a topological manifold. For example, take
        $\mathbb{R}^{2}$ and consider the set of points $A$ of the form
        \begin{equation}
            A=\{\,(x,y)\in\mathbb{R}^{2}\;|\;y=x\textrm{ or }y=-x\,\}
        \end{equation}
        A is an \textbf{X} in the plane. The center of this is \textit{not}
        locally Euclidean. Quotients of topological manifolds need not be
        manifolds either. As a pathological example, take
        $\mathbb{R}/\mathbb{Q}$. This space is quite horrendous and certainly
        not a manifold. The \textit{finite} product of manifolds is indeed a
        manifold.
        \begin{theorem}
            If $(X,\tau_{X})$ and $(Y,\tau_{Y})$ are locally Euclidean
            topological spaces, then the product topological space
            $\big(X\times{Y},\tau(\tau_{X}\times\tau_{Y})\big)$ is locally
            Euclidean.
        \end{theorem}
        \begin{proof}
            For let $\mathcal{A}_{X}$ be an atlas on $(X,\tau_{X})$ and
            $\mathcal{A}_{Y}$ be an atlas on $(Y,\tau_{Y})$. Define
            $\mathcal{A}_{X\times{Y}}$ as follows:
            \begin{equation}
                \mathcal{A}_{X\times{Y}}=
                \{\,(\mathcal{U}\times\mathcal{V},\varphi\times\psi)\;|\;
                    (\mathcal{U},\varphi)\in\mathcal{A}_{X}\textrm{ and }
                    (\mathcal{V},\psi)\in\mathcal{A}_{Y}\,\}
            \end{equation}
            The sets $\mathcal{U}\times\mathcal{V}$ cover $X\times{Y}$ since
            $\mathcal{A}_{X}$ and $\mathcal{A}_{Y}$ are atlases. Moreover,
            the functions
            $\varphi\times\psi:\mathcal{U}\times\mathcal{V}\rightarrow\mathbb{R}^{N+M}$
            are injective, continuous, and open since the component functions
            $\varphi\times\psi$ are injective, continuous, and open.
            Thus, $\mathcal{A}_{X\times{Y}}$ is an atlas and
            $\big(X\times{Y},\tau(\tau_{X}\times\tau_{Y})\big)$ is locally
            Euclidean.
        \end{proof}
        The atlas $\mathcal{A}_{X\times{Y}}$ in this proof need not be maximal,
        even if the atlases $\mathcal{A}_{X}$ and $\mathcal{A}_{Y}$ are.
        This is a similar issue with $\tau_{X}\times\tau_{Y}$ not necessarily
        being a topology. Nevertheless, $\mathcal{A}_{X\times{Y}}$ is an atlas,
        so there is a unique maximal atlas containing it. This gives us the
        following.
        \begin{theorem}
            If $(X,\tau_{X})$ and $(Y,\tau_{Y})$ are topological manifolds,
            then so is their product topological space.
        \end{theorem}
        \begin{proof}
            By the previous theorem, the product is locally Euclidean. But the
            product of Hausdorff topological spaces is Hausdorff, and the
            product of second-countable spaces is second-countable, and hence
            the product is a topological manifold.
        \end{proof}
        \begin{theorem}
            If $(X,\mathcal{A}_{X})$ and $(Y,\mathcal{A}_{Y})$ are smooth
            manifolds, and if $\mathcal{A}_{X\times{Y}}$ is defined by:
            \begin{equation}
                \mathcal{A}_{X\times{Y}}=
                \{\,(\mathcal{U}\times\mathcal{V},\varphi\times\psi)\;|\;
                    (\mathcal{U},\varphi)\in\mathcal{A}_{X}\textrm{ and }
                    (\mathcal{V},\psi)\in\mathcal{A}_{Y}\,\}
            \end{equation}
            then $\mathcal{A}_{X\times{Y}}$ is a smooth atlas.
        \end{theorem}
        \begin{proof}
            We need only show that charts in $\mathcal{A}_{X\times{Y}}$
            are smoothly compatible. Charts in
            $\mathcal{A}_{X\times{Y}}$ are of the form
            $(\mathcal{U}_{1}\times\mathcal{V}_{1},\varphi_{1}\times\psi_{1})$
            and
            $(\mathcal{U}_{2}\times\mathcal{V}_{2},\varphi_{2}\times\psi_{2})$.
            The intersection is
            $(\mathcal{U}_{1}\cap\mathcal{U}_{2})\times(\mathcal{V}_{1}\cap\mathcal{V}_{2})$,
            which is open. The transition functions are:
            \begin{align}
                (\varphi_{1}\times\psi_{1})\circ(\varphi_{2}\times\psi_{2})^{-1}
                (\mathbf{x},\mathbf{y})
                &=(\varphi_{1}\times\psi_{1})
                    \Big(\varphi_{2}^{-1}(\mathbf{x}),\psi_{2}^{-1}(\mathbf{y})\Big)\\
                &=\Big(
                    \varphi_{1}\big(
                        \varphi_{2}^{-1}(\mathbf{x})
                    \big),
                    \psi_{1}\big(
                        \psi_{2}^{-1}(\mathbf{y})
                    \big)
                \Big)\\
                &=\Big(
                    (\varphi_{1}\circ\varphi_{2}^{-1})\times(
                    (\psi_{1}\circ\psi_{2}^{-1})
                \Big)(\mathbf{x},\mathbf{y})
            \end{align}
            and similarly for
            $(\varphi_{2}\times\psi_{2})\circ(\varphi_{1}\times\psi_{1})^{-1}$.
            But $\varphi_{1}\circ\varphi_{2}^{-1}$ is smooth since these charts
            belong to $\mathcal{A}_{x}$, and $\psi_{1}\circ\psi_{2}^{-1}$ is
            smooth as well. But then this is the product of smooth functions
            between open subsets of Euclidean spaces, and is therefore smooth.
            Hence, the charts in $\mathcal{A}_{X\times{Y}}$ are smoothly
            compatible.
        \end{proof}
        By endowing the product of two smooth manifolds with the unique maximal
        atlas containing $\mathcal{A}_{X\times{Y}}$, we get that the product of
        smooth manifolds is a again a smooth manifold.
        \begin{example}
            The torus $\mathbb{T}^{2}$, defined topologically as the product
            of $\mathbb{S}^{1}$ with itself, is a smooth manifold by the
            previous theorem. See Fig.~\ref{fig:torus}.
        \end{example}
        \begin{figure}
            \centering
            \includegraphics{../images/torus.pdf}
            \caption{The Torus $\mathbb{T}^{2}$ Embedded in $\mathbb{R}^{3}$}
            \label{fig:torus}
        \end{figure}
\end{document}
