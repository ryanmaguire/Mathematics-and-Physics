%-----------------------------------LICENSE------------------------------------%
%   This file is part of Mathematics-and-Physics.                              %
%                                                                              %
%   Mathematics-and-Physics is free software: you can redistribute it and/or   %
%   modify it under the terms of the GNU General Public License as             %
%   published by the Free Software Foundation, either version 3 of the         %
%   License, or (at your option) any later version.                            %
%                                                                              %
%   Mathematics-and-Physics is distributed in the hope that it will be useful, %
%   but WITHOUT ANY WARRANTY; without even the implied warranty of             %
%   MERCHANTABILITY or FITNESS FOR A PARTICULAR PURPOSE.  See the              %
%   GNU General Public License for more details.                               %
%                                                                              %
%   You should have received a copy of the GNU General Public License along    %
%   with Mathematics-and-Physics.  If not, see <https://www.gnu.org/licenses/>.%
%------------------------------------------------------------------------------%
\documentclass{article}
\usepackage{graphicx}                           % Needed for figures.
\usepackage{amsmath}                            % Needed for align.
\usepackage{amssymb}                            % Needed for mathbb.
\usepackage{amsthm}                             % For the theorem environment.

\newtheoremstyle{normal}
    {\topsep}               % Amount of space above the theorem.
    {\topsep}               % Amount of space below the theorem.
    {}                      % Font used for body of theorem.
    {}                      % Measure of space to indent.
    {\bfseries}             % Font of the header of the theorem.
    {}                      % Punctuation between head and body.
    {.5em}                  % Space after theorem head.
    {}

\theoremstyle{normal}
\newtheorem{definition}{Definition}
\newtheorem{notation}{Notation}
\newtheorem{example}{Example}

\theoremstyle{plain}
\newtheorem{theorem}{Theorem}
\newcommand{\ceil}[2][]{#1\lceil#2#1\rceil}

\title{All Horses are the Same Color}
\author{Math 3}
\date{Fall 2021}

% No indent and no paragraph skip.
\setlength{\parindent}{0em}
\setlength{\parskip}{0em}

\begin{document}
    \maketitle
    To prove infinitely many statements simultaneously we often use
    \textit{mathematical induction}. If we have statements
    $P_{0}$, $P_{1}$, $P_{2}$, and so on, and we know that $P_{0}$ is true,
    and also that $P_{n}$ implies $P_{n+1}$, then every statement is true.
    Let's \textit{prove} that all horses are the same color.
    We will prove this by induction. The $n^{th}$ statement is:
    \begin{center}
        Given a collection of $n$ horses, every horse is the same color.
    \end{center}
    \par\hfill\par
    The base case says that if we have 1 horse, then the horse is one color.
    This is true (if the horse is multicolored, we'll say the color is
    multicolored).
    \par\hfill\par
    Having proved the base case, let's prove the inductive step. We suppose
    the statement is true for $n$ horses, and have to prove this implies the
    statement is true for $n+1$ horses. So suppose that if we have $n$ horses,
    then all horses are the same color. Let us prove that $n+1$ horses must
    all have the same color. Number the horses 1, 2, 3, $\cdots$, $n$, $n+1$.
    First, group the horses together 1 to $n$. Since this group of horses has
    size $n$, by the induction hypothesis, each horse must be the same color.
    Next, group the horses $2$ to $n+1$. Since this group of horses has size
    $n$, each horse must be the same color. But since these two groups overlap,
    the two colors must be the same. Therefore, all $n+1$ horses are the same.
    By induction, we have proved that all horses are the same color.
    \par\hfill\par
    This is obviously absurd and false, but \textit{why}? What is wrong with
    the proof? Work together with your group to figure it out.
    \newpage
    I, the copyright holder of this work, release it into the public domain.
    This applies worldwide. In some countries this may not be legally possible;
    if so: I grant anyone the right to use this work for any purpose, without
    any conditions, unless such conditions are required by law.
    \par\hfill\par
    The source code used to generate this document is free software and released
    under version 3 of the GNU General Public License.
\end{document}
