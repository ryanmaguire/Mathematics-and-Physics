%-----------------------------------LICENSE------------------------------------%
%   This file is part of Mathematics-and-Physics.                              %
%                                                                              %
%   Mathematics-and-Physics is free software: you can redistribute it and/or   %
%   modify it under the terms of the GNU General Public License as             %
%   published by the Free Software Foundation, either version 3 of the         %
%   License, or (at your option) any later version.                            %
%                                                                              %
%   Mathematics-and-Physics is distributed in the hope that it will be useful, %
%   but WITHOUT ANY WARRANTY; without even the implied warranty of             %
%   MERCHANTABILITY or FITNESS FOR A PARTICULAR PURPOSE.  See the              %
%   GNU General Public License for more details.                               %
%                                                                              %
%   You should have received a copy of the GNU General Public License along    %
%   with Mathematics-and-Physics.  If not, see <https://www.gnu.org/licenses/>.%
%------------------------------------------------------------------------------%
\documentclass{article}
\usepackage{graphicx}                           % Needed for figures.
\usepackage{amsmath}                            % Needed for align.
\usepackage{amssymb}                            % Needed for mathbb.
\usepackage{amsthm}                             % For the theorem environment.
\usepackage{float}

\newtheoremstyle{normal}
    {\topsep}               % Amount of space above the theorem.
    {\topsep}               % Amount of space below the theorem.
    {}                      % Font used for body of theorem.
    {}                      % Measure of space to indent.
    {\bfseries}             % Font of the header of the theorem.
    {}                      % Punctuation between head and body.
    {.5em}                  % Space after theorem head.
    {}

\theoremstyle{normal}
\newtheorem{definition}{Definition}
\newtheorem{notation}{Notation}
\newtheorem{example}{Example}

\theoremstyle{plain}
\newtheorem{theorem}{Theorem}
\newcommand{\ceil}[2][]{#1\lceil#2#1\rceil}

\title{Induction and Summation}
\author{Ryan Maguire}
\date{Fall 2021}

% No indent and no paragraph skip.
\setlength{\parindent}{0em}
\setlength{\parskip}{0em}

\begin{document}
    \maketitle
    Suppose we have a sequence of sentences, $P_{0}$, $P_{1}$, $P_{2}$, so on
    and so forth. Suppose also that I know $P_{0}$ is true. Lastly, suppose I
    know that if $P_{n}$ is true, then $P_{n+1}$ is true. The question is then:
    are there any sentences that are \textit{false}? Well, is $P_{1}$ false?
    $P_{0}$ is true by hypothesis, and $P_{0}$ being true implies $P_{1}$ is
    true, again by hypothesis, and therefore $P_{1}$ is true. What about
    $P_{2}$? Well, we just showed that $P_{1}$ is true, and by hypothesis
    $P_{1}$ being true implies $P_{2}$ is true, so we may conclude $P_{2}$ is
    true. Continuing up the ladder, we have that $P_{n}$ is always
    true for all $n$.
    \par\hfill\par
    This is very efficient. If we have some statement about the natural numbers
    1, 2, 3, and so on, instead of proving infinitely many statements, we need
    to prove only 2. We must prove the statement for 0 is true, and we must
    prove that the $n$ statement implies the $n+1$ statement.
    \begin{example}
        Let's prove that $1+1+\cdots+1=n$ when there are $n$ 1's. What's the
        base case? The base case is $1=1$, which is true. The next case is
        $1+1=2$, which is also true. Suppose:
        \begin{equation}
            \underbrace{1+\cdots+1}_{n}=n
        \end{equation}
        is true. Let's now prove that this implies
        \begin{equation}
            \underbrace{1+\cdots+1}_{n+1}=n+1
        \end{equation}
        We have:
        \begin{align}
            n+1&=(n)+1\\
                &=(\underbrace{1+\cdots+1}_{n})+1\\
                &=\underbrace{1+\cdots+1}_{n}+1\\
                &=\underbrace{1+\cdots+1}_{n+1}
        \end{align}
        Since the base case is true, and since the $n$ case implies the
        $n+1$ case, by induction the statement is always true.
    \end{example}
    \newpage
    We can use induction to solve some of the more basic questions of calculus.
    Particularly, what is the area under the curve $f(x)=x$ and what is the
    area under the curve $g(x)=x^{2}$. These questions have classical solutions
    dating to antiquity. Euclid (\textit{c.} 300 B.C.E.) gave the formula for
    the area under $f(x)=x$ and Archimedes' (\textit{c.} 250 B.C.E.) gave the
    solution to the area under $g(x)=x^{2}$. When approximating these areas
    using Riemann sums (rectangles that fit under the curve) we get the
    following expressions. For the under under $f(x)=x$ in the interval
    $[0,a]$, we get:
    \begin{equation}
        \textrm{Area}(f)\approx\frac{a^{2}}{N^{2}}
            \sum_{n=0}^{N-1}n
    \end{equation}
    and for $g(x)=x^{2}$ on the interval $[0,a]$ we get:
    \begin{equation}
        \textrm{Area}(g)\approx\frac{a^{3}}{N^{3}}
            \sum_{n=0}^{N-1}n^{2}
    \end{equation}
    Where $N$ is the number of rectangles we are using to approximate the area.
    We define the integral to be the limit of these approximations:
    \begin{align}
        \int_{0}^{a}f(x)\;\textrm{d}x
            &=\lim_{N\rightarrow\infty}\frac{a^{2}}{N^{2}}\sum_{n=0}^{N-1}n\\
            \int_{0}^{a}g(x)\;\textrm{d}x
                &=\lim_{N\rightarrow\infty}
                \frac{a^{3}}{N^{3}}\sum_{n=0}^{N-1}n^{2}
    \end{align}
    If we can obtain a formula for these two expressions, we may be able to
    simplify the limit and get a concrete answer.
    \par\hfill\par
    The sum $1+2+\cdots+n$ has two answers, one intuitive, one rigorous with
    induction. Fortunately, these answers agree. Let's add
    $1+2+\cdots+100$. We'll do this using the following table.
    \begin{table}[H]
        \centering
        \begin{tabular}{ccccccccccc}
            &&$1$&$+$&$2$&$+$&$\cdots$&$+$&$99$&$+$&$100$\\
            \hline\\
            $=$&&$1$&$+$&$2$&$+$&$\cdots$&$+$&$49$&$+$&$50$\\
            &$+$&$100$&$+$&$99$&$+$&$\cdots$&$+$&$52$&$+$&$51$\\
            \hline\\
            $=$&&$101$&$+$&$101$&$+$&$\cdots$&$+$&$101$&$+$&$101$\\
            \hline\\
            $=$&&$50$&$\times$&$101$\\
            \hline\\
            $=$&&$5050$
        \end{tabular}
        \caption{Gauss' Sum of 1 to 100}
    \end{table}
    So the answer is half of 100 times $100+1$. This is the general formula:
    \begin{equation}
        \sum_{n=1}^{N}n=\frac{N(N+1)}{2}
    \end{equation}
    Since we want to count from 0 to $N-1$, we need to shift this formula:
    \begin{equation}
        \sum_{n=0}^{N-1}n=\frac{N(N-1)}{2}
    \end{equation}
    Let's prove this is true rigorously with induction. The base case says that
    $0=0$. This is true. Suppose the formula is true for some positive integer
    $N$. Let's prove this implies the formula is also true for $N+1$. We have:
    \begin{align}
        \sum_{n=0}^{N}n
            &=\Big(\sum_{n=0}^{N-1}n\Big)+N\tag{Definition of $\Sigma$}\\
            &=\frac{N(N-1)}{2}+N\tag{Inductive Hypothesis}\\
            &=\frac{N^{2}-N+2N}{2}\tag{Factoring and Common Denominators}\\
            &=\frac{N^{2}+N}{2}\tag{Simplify}\\
            &=\frac{N(N+1)}{2}\tag{Factor}
    \end{align}
    And this is precisely the formula for $N+1$. By induction, the formula
    is true for every positive integer $N$.
    \newpage
    I know of no intuitive method for $1^{2}+2^{2}+3^{2}+\cdots+N^{2}$
    (but would like to), but I have a feeling the answer is a cubic. If you
    add 1 $N$ times you get $N$. If you add 1, 2, $\cdots$, $N$ you get
    roughly $N^{2}$. The pattern seems to be that if you add
    $1^{n}+2^{n}+\cdots+N^{n}$ you should get roughly $N^{n+1}$. So, for
    $1^{2}+2^{2}+\cdots+N^{2}$ I expect the answer to be roughly $N^{3}$.
    The first few sums are 1, 5, 14, 30, and 55. If I fit a cubic to these
    data points, I get the cubic:
    \begin{equation}
        f(N)=\frac{N(N+1)(2N+1)}{6}
    \end{equation}
    Let's prove this. That is, let's prove:
    \begin{equation}
        \sum_{n=1}^{N}n^{2}=\frac{N(N+1)(2N+1)}{6}
    \end{equation}
    The base case $N=1$ says that $1=1$. This is true. Suppose the formula is
    true for $N$. Let's prove that implies it is true for $N+1$. We have:
    \begin{align}
        \sum_{n=1}^{N+1}n^{2}
            &=\Big(\sum_{n=1}^{N}n^{2}\Big)+(N+1)^{2}\\
            &=\frac{N(N+1)(2N+1)}{6}+(N+1)^{2}\\
            &=(N+1)\frac{N(2N+1)+6(N+1)}{6}\\
            &=(N+1)\frac{2N^{2}+7N+6}{6}\\
            &=(N+1)\frac{(N+2)(2N+3)}{6}\\
            &=\frac{(N+1)(N+2)(2N+3)}{6}
    \end{align}
    Using this for integration, we have:
    \begin{align}
        \int_{0}^{a}f(x)\;\textrm{d}x
            &=\lim_{N\rightarrow\infty}\frac{a^{2}}{N^{2}}\sum_{n=0}^{N-1}n
            =\lim_{N\rightarrow\infty}\frac{a^{2}}{N^{2}}\frac{N(N-1)}{2}
            =\frac{a^{2}}{2}\\
        \int_{0}^{a}g(x)\;\textrm{d}x
            &=\lim_{N\rightarrow\infty}
            \frac{a^{3}}{N^{3}}\sum_{n=0}^{N-1}n^{2}
            =\lim_{N\rightarrow\infty}\frac{a^{3}}{N^{3}}
                \frac{N(N-1)(2N-1)}{6}
            =\frac{a^{3}}{3}
    \end{align}
    \newpage
    I, the copyright holder of this work, release it into the public domain.
    This applies worldwide. In some countries this may not be legally possible;
    if so: I grant anyone the right to use this work for any purpose, without
    any conditions, unless such conditions are required by law.
    \par\hfill\par
    The source code used to generate this document is free software and released
    under version 3 of the GNU General Public License.
\end{document}
