%-----------------------------------LICENSE------------------------------------%
%   This file is part of Mathematics-and-Physics.                              %
%                                                                              %
%   Mathematics-and-Physics is free software: you can redistribute it and/or   %
%   modify it under the terms of the GNU General Public License as             %
%   published by the Free Software Foundation, either version 3 of the         %
%   License, or (at your option) any later version.                            %
%                                                                              %
%   Mathematics-and-Physics is distributed in the hope that it will be useful, %
%   but WITHOUT ANY WARRANTY; without even the implied warranty of             %
%   MERCHANTABILITY or FITNESS FOR A PARTICULAR PURPOSE.  See the              %
%   GNU General Public License for more details.                               %
%                                                                              %
%   You should have received a copy of the GNU General Public License along    %
%   with Mathematics-and-Physics.  If not, see <https://www.gnu.org/licenses/>.%
%------------------------------------------------------------------------------%
\documentclass{article}
\usepackage{graphicx}                           % Needed for figures.
\usepackage{hyperref}                           % Hyperlinks for figures.
\hypersetup{colorlinks=true, linkcolor=blue}    % Colors for hyperref.

\title{Domains of Functions}
\author{Ryan Maguire}
\date{Fall 2021}

% No indent and no paragraph skip.
\setlength{\parindent}{0em}
\setlength{\parskip}{0em}

\begin{document}
    \maketitle
    The square root of a non-negative real number $x$ is the unique
    non-negative real number $y$ such that $x=y^{2}$. That such a number exists
    is not trivial, but requires concepts we don't currently have. We
    label this value $y=\sqrt{x}$, and the $\sqrt{\empty}$ symbol is called the
    \textit{radical}. By the very nature of this definition, we do not allow
    the input to be negative. Indeed, if we did, we could solve the equation
    $x^{2}+1=0$ via $\sqrt{-1}$, but there is no real number such that
    $x^{2}=-1$. Remember from the graph of the parabola that $x^{2}$ is
    always greater than or equal to zero. See
    Fig.~\ref{fig:x_squared} for a visual as to why $x^{2}$ is never negative,
    and Fig.~\ref{fig:x_squared_plus_one} for why $x^{2}+1=0$ has no real
    solution.
    \par\hfill\par
    Consider the following expression:
    \begin{equation}
        f(x)=\frac{\sqrt{2x}}{x-\sqrt{1-x}}
    \end{equation}
    There are several cases where $f(x)$ could fail to be well-defined. Firstly,
    we have a division and must ensure the denominator is non-zero. The
    expression $x-\sqrt{1-x}=0$ occurs when $x=\sqrt{1-x}$. Squaring both
    sides, we must avoid $x^{2}=1-x$, which is the same thing as avoiding
    $x^{2}+x-1=0$. Using the quadratic formula (with $a=1$, $b=1$, and
    $c=-1$), the solutions to this are:
    \begin{equation}
        x=\frac{-1\pm\sqrt{5}}{2}
    \end{equation}
    The numerator $\sqrt{2x}$ creates the restriction that $2x\geq{0}$, which
    is equivalent to requiring $x\geq{0}$. Because of this, we can rid ourselves
    of the requirement that $x\ne\frac{-1-\sqrt{5}}{2}$ since this value is
    negative and $x\geq{0}$ already excludes it. Thus far we need
    $x\geq{0}$ and $x\ne\frac{-1+\sqrt{5}}{2}$. The denominator also has the
    expression $\sqrt{1-x}$. We need the input to this to be non-negative, and
    thus require $1-x\geq{0}$. This implies $1\geq{x}$. All together we need
    $0\leq{x}\leq{1}$ and $x\ne\frac{-1+\sqrt{5}}{2}$. In set theory notation,
    the domain of $f$ is:
    \begin{equation}
        D=\Big[0,\frac{-1+\sqrt{5}}{2}\Big)\bigcup
        \Big(\frac{-1+\sqrt{5}}{2},1\Big]
    \end{equation}
    The function $f(x)=\sqrt{2x}/(x-\sqrt{1-x})$ is plotted in
    Fig.~\ref{fig:sqrt_2x_by_x_minus_sqrt_one_minus_x}.
    \begin{figure}
        \centering
        \includegraphics{../../images/x_squared.pdf}
        \caption{The graph of $x^{2}$}
        \label{fig:x_squared}
    \end{figure}
    \begin{figure}
        \centering
        \includegraphics{../../images/x_squared_plus_one.pdf}
        \caption{The graph of $x^{2}+1$}
        \label{fig:x_squared_plus_one}
    \end{figure}
    \begin{figure}
        \centering
        \includegraphics{../../images/sqrt_2x_by_x_minus_sqrt_one_minus_x.pdf}
        \caption{The graph of $f(x)$}
        \label{fig:sqrt_2x_by_x_minus_sqrt_one_minus_x}
    \end{figure}
    \newpage
    I, the copyright holder of this work, release it into the public domain.
    This applies worldwide. In some countries this may not be legally possible;
    if so: I grant anyone the right to use this work for any purpose, without
    any conditions, unless such conditions are required by law.
    \par\hfill\par
    The source code used to generate this document is free software and released
    under version 3 of the GNU General Public License.
\end{document}
