%-----------------------------------LICENSE------------------------------------%
%   This file is part of Mathematics-and-Physics.                              %
%                                                                              %
%   Mathematics-and-Physics is free software: you can redistribute it and/or   %
%   modify it under the terms of the GNU General Public License as             %
%   published by the Free Software Foundation, either version 3 of the         %
%   License, or (at your option) any later version.                            %
%                                                                              %
%   Mathematics-and-Physics is distributed in the hope that it will be useful, %
%   but WITHOUT ANY WARRANTY; without even the implied warranty of             %
%   MERCHANTABILITY or FITNESS FOR A PARTICULAR PURPOSE.  See the              %
%   GNU General Public License for more details.                               %
%                                                                              %
%   You should have received a copy of the GNU General Public License along    %
%   with Mathematics-and-Physics.  If not, see <https://www.gnu.org/licenses/>.%
%------------------------------------------------------------------------------%
\documentclass{article}
\usepackage{graphicx}                           % Needed for figures.
\usepackage{hyperref}                           % Hyperlinks for figures.
\hypersetup{colorlinks=true, linkcolor=blue}    % Colors for hyperref.

\title{Domains of Functions}
\author{Ryan Maguire}
\date{Fall 2021}

% No indent and no paragraph skip.
\setlength{\parindent}{0em}
\setlength{\parskip}{0em}

\begin{document}
    \maketitle
    When an expression involves logarithmic functions such as the natural
    log we need to be careful that the input is positive. The log base
    $b$ (with $b>1$) of a real number $x$ returns the number $y$ such that
    $x=b^{y}$. Since $b^{y}$ can never be negative, $\log_{b}(x)$ has no
    meaning for negative values of $x$. For 0, we can say
    $\log_{b}(0)=-\infty$ since $b^{-\infty}=0$ (more precisely, $b^{x}$ tends
    to zero as $x$ approaches $-\infty$), but $-\infty$ is not a real
    number and so 0 must be excluded from the input of $\log_{b}$.
    A plot of $2^{x}$ is given in Fig.~\ref{fig:two_to_the_x} indicating that
    $2^{x}$ is never negative and tends to $0$ as $x$ tends to $-\infty$.
    \par\hfill\par
    Consider the expression below:
    \begin{equation}
        f(x)=\frac{1}{x\ln(x)}
    \end{equation}
    There are a few restrictions on $x$ for $f(x)$ to be well-defined. Firstly,
    we have a division so we need the denominator to be non-zero. We must
    avoid $x\ln(x)=0$. If $x\ln(x)=0$, then either $x=0$ or $\ln(x)=0$. For
    any $b>1$, $\log_{b}(x)=0$ is true precisely when $x=1$. If we want
    $1=b^{y}$, we use that fact that for any non-zero real number $b$ it is
    true that $b^{0}=1$. This gives us $\log_{b}(1)=0$. So, in particular,
    $\ln(1)=0$. To avoid a division by zero in our expression we need
    $x\ne{0}$ and $x\ne{1}$. There is another restriction. We need that
    $\ln(x)$ is well-defined as well. This occurs when $x>0$. So, in total
    we need $x\ne{0}$, $x\ne{1}$, and $x>0$. The requirement $x>0$ excludes
    0 so we can rid ourselves of the first requirement, and need
    $x>0$ and $x\ne{1}$. We can write the domain of $f$ as:
    \begin{equation}
        D=(0,1)\cup(1,\infty)
    \end{equation}
    The function is plotted in Fig.~\ref{fig:one_by_x_lnx}.
    \begin{figure}
        \centering
        \includegraphics{../../images/two_to_the_x.pdf}
        \caption{The function $2^{x}$}
        \label{fig:two_to_the_x}
    \end{figure}
    \begin{figure}
        \centering
        \includegraphics{../../images/one_by_x_times_ln_x.pdf}
        \caption{The function $f(x)=1/x\ln(x)$}
        \label{fig:one_by_x_lnx}
    \end{figure}
    \newpage
    I, the copyright holder of this work, release it into the public domain.
    This applies worldwide. In some countries this may not be legally possible;
    if so: I grant anyone the right to use this work for any purpose, without
    any conditions, unless such conditions are required by law.
    \par\hfill\par
    The source code used to generate this document is free software and released
    under version 3 of the GNU General Public License.
\end{document}
