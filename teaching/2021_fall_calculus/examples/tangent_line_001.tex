%-----------------------------------LICENSE------------------------------------%
%   This file is part of Mathematics-and-Physics.                              %
%                                                                              %
%   Mathematics-and-Physics is free software: you can redistribute it and/or   %
%   modify it under the terms of the GNU General Public License as             %
%   published by the Free Software Foundation, either version 3 of the         %
%   License, or (at your option) any later version.                            %
%                                                                              %
%   Mathematics-and-Physics is distributed in the hope that it will be useful, %
%   but WITHOUT ANY WARRANTY; without even the implied warranty of             %
%   MERCHANTABILITY or FITNESS FOR A PARTICULAR PURPOSE.  See the              %
%   GNU General Public License for more details.                               %
%                                                                              %
%   You should have received a copy of the GNU General Public License along    %
%   with Mathematics-and-Physics.  If not, see <https://www.gnu.org/licenses/>.%
%------------------------------------------------------------------------------%
\documentclass{article}
\usepackage{graphicx}                           % Needed for figures.
\usepackage{amsmath}                            % Needed for align.
\usepackage{amssymb}                            % Needed for mathbb.
\usepackage{amsthm}                             % For the theorem environment.

\newtheoremstyle{normal}
    {\topsep}               % Amount of space above the theorem.
    {\topsep}               % Amount of space below the theorem.
    {}                      % Font used for body of theorem.
    {}                      % Measure of space to indent.
    {\bfseries}             % Font of the header of the theorem.
    {}                      % Punctuation between head and body.
    {.5em}                  % Space after theorem head.
    {}

\theoremstyle{normal}
\newtheorem{definition}{Definition}

\title{Differentiation}
\author{Ryan Maguire}
\date{Fall 2021}

% No indent and no paragraph skip.
\setlength{\parindent}{0em}
\setlength{\parskip}{0em}

\begin{document}
    \maketitle
    Let's find the equation of the tangent line of
    $f(x)=2x-3x^{3}+x^{5}$ at $x_{0}=1$. The difference quotient for any real
    number $x\in\mathbb{R}$ is:
    \begin{equation}
        \frac{f(x+h)-f(x)}{h}
    \end{equation}
    This is the slope of the \textit{secant line} passing through the
    points $(x,f(x))$ and $(x+h,f(x+h))$. Using $f(x)=2x-3x^{3}+x^{5}$ we get:
    \begin{equation}
        \frac{2(x+h)-3(x+h)^{3}+(x+h)^{5}-\big(2x-3x^{3}+x^{5}\big)}{h}
    \end{equation}
    As $h$ approaches zero, this secant line better approximates the
    \textit{tangent} line. This is shown in
    Fig.~\ref{fig:sec_line}. In fact, the limit as $h$ tends to zero
    \textit{is} the tangent line. The limit as $h$ tends to zero is also the
    definition of the derivative:
    \begin{equation}
        \frac{\textrm{d}f}{\textrm{d}x}(x)
            =f'(x)
            =\lim_{h\rightarrow{0}}\frac{f(x+h)-f(x)}{h}
    \end{equation}
    The notations $\frac{\textrm{d}f}{\textrm{d}x}(x)$ and $f'(x)$ are
    equivalent. In physics one often sees $\dot{f}(x)$ (read aloud as
    \textit{f dot of x}), and this too means the derivative of $f$ at $x$.
    \par\hfill\par
    Let's use the \textit{sum rule} for differentiation, which says that if
    $g_{0}$ and $g_{1}$ are differentiable functions, then:
    \begin{equation}
        \frac{\textrm{d}}{\textrm{d}x}\big(g_{0}(x)+g_{1}(x)\big)
        =\frac{\textrm{d}g_{0}}{\textrm{d}x}(x)+
            \frac{\textrm{d}g_{1}}{\textrm{d}x}(x)
    \end{equation}
    Applying this to $f$, we have:
    \begin{equation}
        \frac{\textrm{d}f}{\textrm{d}x}(x)
        =\frac{\textrm{d}(2x)}{\textrm{d}x}+
        \frac{\textrm{d}(-3x^{3})}{\textrm{d}x}+
        \frac{\textrm{d}(x^{5})}{\textrm{d}x}
    \end{equation}
    Next we use the fact that constants can be factored out of the derivative.
    This gives us:
    \begin{equation}
        \frac{\textrm{d}f}{\textrm{d}x}(x)
        =2\frac{\textrm{d}(x)}{\textrm{d}x}-
        3\frac{\textrm{d}(x^{3})}{\textrm{d}x}+
        \frac{\textrm{d}(x^{5})}{\textrm{d}x}
    \end{equation}
    To wrap this up, we apply the \textit{power rule}. This says, for a
    function of the form $g(x)=x^{n}$, the derivative can be computed as:
    $g'(x)=nx^{n-1}$. That is:
    \begin{equation}
        \frac{\textrm{d}(x^{n})}{\textrm{d}x}
        =nx^{n-1}
    \end{equation}
    Using this, the derivative of $f$ becomes:
    \begin{align}
        \frac{\textrm{d}f}{\textrm{d}x}(x)
        &=2\frac{\textrm{d}(x)}{\textrm{d}x}-
            3\frac{\textrm{d}(x^{3})}{\textrm{d}x}+
            \frac{\textrm{d}(x^{5})}{\textrm{d}x}\\
        &=2-3(3x^{2})+5x^{4}\\
        &=2-9x^{2}+5x^{4}
    \end{align}
    Since we now know that $f'(x)=2-9x^{2}+5x^{4}$, we can compute the slope of
    the tangent line of $f$ at $x_{0}=1$ by evaluating $f'$ at 1. We get:
    \begin{equation}
        f'(1)=2-9(1)^{2}+5(1)^{4}=2-9+5=-2
    \end{equation}
    So the slope of at $x_{0}=1$ is $-2$. The tangent line has the formula:
    \begin{equation}
        y_{T}=m(x-x_{0})+y_{0}
    \end{equation}
    We know the slope is $m=f'(x_{0})=f'(1)=-2$, so we now have:
    \begin{equation}
        y_{T}=-2(x-x_{0})+y_{0}
    \end{equation}
    When we plug in $x=x_{0}$ we see that the right hand side becomes $y_{0}$.
    We want the tangent line of $f$ at $x_{0}$ to have both the same
    \textit{slope} as $f$ at $x_{0}$, and the same \textit{height}. That is,
    we want $y_{T}$ and $f$ to meet at $x=x_{0}$. To do this, we see that we
    need $y_{0}=f(x_{0})$. Since we chose $x_{0}=1$, we can compute this:
    \begin{equation}
        y_{0}=f(x_{0})=f(1)=2(1)-3(1)^{3}+(1)^{5}=2-3+1=0
    \end{equation}
    So $y_{0}=0$, and thus the tangent line is:
    \begin{equation}
        y_{T}=-2(x-x_{0})
    \end{equation}
    This is plotted in Fig.~\ref{fig:tan_line}.
    \begin{figure}
        \centering
        \resizebox{\textwidth}{!}{%
            \includegraphics{../../images/secant_line_001.pdf}
        }
        \caption{Secant Line for $f$}
        \label{fig:sec_line}
    \end{figure}
    \begin{figure}
        \centering
        \resizebox{!}{0.9\textheight}{%
            \includegraphics{../../images/tangent_line_001.pdf}
        }
        \caption{Tangent Line for $f$}
        \label{fig:tan_line}
    \end{figure}
    \newpage
    I, the copyright holder of this work, release it into the public domain.
    This applies worldwide. In some countries this may not be legally possible;
    if so: I grant anyone the right to use this work for any purpose, without
    any conditions, unless such conditions are required by law.
    \par\hfill\par
    The source code used to generate this document is free software and released
    under version 3 of the GNU General Public License.
\end{document}
