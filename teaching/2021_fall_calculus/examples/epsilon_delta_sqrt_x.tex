%-----------------------------------LICENSE------------------------------------%
%   This file is part of Mathematics-and-Physics.                              %
%                                                                              %
%   Mathematics-and-Physics is free software: you can redistribute it and/or   %
%   modify it under the terms of the GNU General Public License as             %
%   published by the Free Software Foundation, either version 3 of the         %
%   License, or (at your option) any later version.                            %
%                                                                              %
%   Mathematics-and-Physics is distributed in the hope that it will be useful, %
%   but WITHOUT ANY WARRANTY; without even the implied warranty of             %
%   MERCHANTABILITY or FITNESS FOR A PARTICULAR PURPOSE.  See the              %
%   GNU General Public License for more details.                               %
%                                                                              %
%   You should have received a copy of the GNU General Public License along    %
%   with Mathematics-and-Physics.  If not, see <https://www.gnu.org/licenses/>.%
%------------------------------------------------------------------------------%
\documentclass{article}
\usepackage{amsmath}                            % Needed for align.
\usepackage{amssymb}                            % Needed for mathbb.
\usepackage{amsthm}                             % For the theorem environment.

\newtheoremstyle{normal}
    {\topsep}               % Amount of space above the theorem.
    {\topsep}               % Amount of space below the theorem.
    {}                      % Font used for body of theorem.
    {}                      % Measure of space to indent.
    {\bfseries}             % Font of the header of the theorem.
    {}                      % Punctuation between head and body.
    {.5em}                  % Space after theorem head.
    {}

\theoremstyle{normal}
\newtheorem{definition}{Definition}
\theoremstyle{plain}
\newtheorem{theorem}{Theorem}

\title{$\varepsilon-\delta$ Continuity - Square Roots}
\author{Ryan Maguire}
\date{\today}

% No indent and no paragraph skip.
\setlength{\parindent}{0em}
\setlength{\parskip}{0em}

\begin{document}
    \maketitle
    The definition of continuity is as follows:
    \begin{definition}
        A real-valued function that is continuous at a point
        $x_{0}\in\mathbb{R}$ is a function $f:\mathbb{R}\rightarrow\mathbb{R}$
        such that for all $\varepsilon>0$ there exists a $\delta>0$ such that
        for all $x\in\mathbb{R}$ with $|x-x_{0}|<\delta$ it is true that
        $|f(x)-f(x_{0})|<\varepsilon$.
    \end{definition}
    Let's prove $f(x)=\sqrt{x}$ is continuous at every point
    $x_{0}\in[0,\infty)$. First, let's handle $x_{0}=0$ separately. We want
    $|x-0|<\delta$ implies $|\sqrt{x}-\sqrt{0}|<\varepsilon$. In other
    words, we want $|x|<\delta$ implies $|\sqrt{x}|<\varepsilon$. Choosing
    $\delta=\varepsilon^{2}$, if $x<\delta$ (we can drop the absolute value
    sign since $x\in[0,\infty)$, so $x$ is never negative), then
    $x<\varepsilon^{2}$, and therefore $\sqrt{x}<\varepsilon$. Now, for
    $x_{0}>0$.
    \begin{equation}
        \textbf{Want:}\quad
        |x-x_{0}|<\delta\Rightarrow
        |f(x)-f(x_{0})|<\varepsilon
    \end{equation}
    Substituting the formula for $f$:
    \begin{equation}
        \textbf{Want:}\quad
        |x-x_{0}|<\delta\Rightarrow
        |\sqrt{x}-\sqrt{x_{0}}|<\varepsilon
    \end{equation}
    Using the trick of \textit{conjugates} for square roots, if
    $\sqrt{x}+\sqrt{x_{0}}\ne{0}$ we can write:
    \begin{align}
        \sqrt{x}-\sqrt{x_{0}}
        &=(\sqrt{x}-\sqrt{x}_{0})
            \frac{\sqrt{x}+\sqrt{x_{0}}}{\sqrt{x}+\sqrt{x_{0}}}\\
        &=\frac{(\sqrt{x}-\sqrt{x_{0}})(\sqrt{x}+\sqrt{x_{0}})}
              {\sqrt{x}+\sqrt{x_{0}}}\\
        &=\frac{x-x_{0}}{\sqrt{x}+\sqrt{x_{0}}}
    \end{align}
    Since $x_{0}>0$, we have $\sqrt{x}+\sqrt{x_{0}}>0$, and so this
    trick is valid. Next we note for every $x\geq{0}$ it is true that
    $\sqrt{x}\geq{0}$, and therefore $\sqrt{x}+\sqrt{x_{0}}\geq\sqrt{x_{0}}$.
    But then:
    \begin{equation}
        \frac{1}{\sqrt{x}+\sqrt{x_{0}}}\leq\frac{1}{\sqrt{x_{0}}}
    \end{equation}
    We update our wish-list:
    \begin{equation}
        \textbf{Want:}\quad
        |x-x_{0}|<\delta
        \Rightarrow
        \frac{|x-x_{0}|}{\sqrt{x_{0}}}<\varepsilon
    \end{equation}
    Since we only care about $|x-x_{0}|<\delta$, we have:
    \begin{equation}
        \frac{|x-x_{0}|}{\sqrt{x_{0}}}<\frac{\delta}{\sqrt{x_{0}}}
    \end{equation}
    We update our wish-list one last time:
    \begin{equation}
        \textbf{Want:}\quad
        |x-x_{0}|<\delta
        \Rightarrow
        \frac{\delta}{\sqrt{x_{0}}}\leq\varepsilon
    \end{equation}
    Choosing $\delta=\varepsilon\sqrt{x_{0}}$ fulfills everything on our list.
    \par\hfill\par
    Now that we have a candidate for $\delta$, let's show that it works.
    Let $\varepsilon>0$. Choose $\delta=\varepsilon\sqrt{x_{0}}$. If
    $|x-x_{0}|<\delta$, then:
    \begin{align}
        |x-x_{0}|&<\varepsilon\sqrt{x_{0}}
            \tag{Definition of $\delta$}\\
        \Rightarrow
            \frac{|x-x_{0}|}{\sqrt{x_{0}}}&<\varepsilon
                \tag{Division by a Positive Number}\\
        \Rightarrow
            \frac{|x-x_{0}|}{\sqrt{x}+\sqrt{x_{0}}}&<\varepsilon
                \tag{%
                    Since
                    $\frac{1}{\sqrt{x}+\sqrt{x_{0}}}\leq\frac{1}{\sqrt{x_{0}}}$%
                }\\
        \Rightarrow
            |\sqrt{x}-\sqrt{x_{0}}|&<\varepsilon
                \tag{Conjugate the Expression}\\
        \Rightarrow
            |f(x)-f(x_{0})|&<\varepsilon
                \tag{Definition of $f$}
    \end{align}
    So given $x_{0}>0$ and any $\varepsilon>0$ there is a $\delta>0$ such
    that for all $x\in(0,\,\infty)$ with $|x-x_{0}|<\delta$ it is true that
    $|f(x)-f(x_{0})|<\varepsilon$.
    \par\hfill\par
    Before ending, let's briefly discuss \textit{uniform} continuity. Functions
    such as $f(x)=ax+b$ with $x\in\mathbb{R}$ or $f(x)=1/x$ with
    $x\in[1,\infty)$ have the property that, given $\varepsilon>0$, one can
    choose a $\delta>0$ that is \textit{independent} of $x_{0}$, and such that
    $|x-x_{0}|<\delta$ implies $|f(x)-f(x_{0})|<\varepsilon$. For
    $f(x)=ax+b$, $a\ne{0}$, we can choose $\delta=\varepsilon/|a|$, and for
    $f(x)=1/x$ with $x\in[1,\infty)$ we can choose
    $\delta=\varepsilon$. Other functions, like $f(x)=x^{2}$ with
    $x\in\mathbb{R}$ or $f(x)=1/x$ with $x\in(0,\infty)$ do not have this
    property. The choice of $\delta$ depends not only on $\varepsilon$, but on
    the point of interest $x_{0}$. For $x^{2}$ we get formulas like
    $\delta=\min(x_{0}/2,5\varepsilon/2x_{0})$ and for $f(x)=1/x$ we got
    $\delta=\min(x_{0}/2,\varepsilon{x}_{0}^{2}/2)$. That is, the formula
    for $\delta$ depends on $\varepsilon$ and $x_{0}$. This is perfectly fine
    within the definition of continuity. Functions where $\delta$ does
    \textit{not} need to depend on $x_{0}$ are called
    \textit{uniformly continuous}. Intuitively, these functions have the
    property that they don't get too \textit{steep}. The function
    $x^{2}$ gets steeper and steeper as $x$ increases, and so it is
    \textit{not} uniformly continuous. Examining the function $f(x)=1/x$,
    when we consider $x\in[1,\infty)$, the steepest the function gets is at
    $x=1$. We find a $\delta$ that works at this value, and because the
    function is steepest there, this $\delta$ works for every other
    $x\in[1,\infty)$. When we look at $f(x)=1/x$ on $(0,\infty)$, there is no
    \textit{steepest point}. The function gets steeper and steeper as $x$ gets
    closer to zero, and this is why $\delta$ depends on both the point of
    interest and $\varepsilon$.
    \par\hfill\par
    With the above discussion in mind, is $f(x)=\sqrt{x}$ uniformly continuous
    on $x\in[0,\infty)$? We might say \textit{no} because the formula we got
    is $\delta=\varepsilon\sqrt{x_{0}}$ and this depends on $x_{0}$. On the
    other hand, the \textit{intuition} behind uniform continuity requires the
    function to never get too \textit{steep}. The steepest $f(x)=\sqrt{x}$
    gets is at $x_{0}=0$, but we demonstrated that the $\varepsilon-\delta$
    problem works at $x_{0}=0$. So, is $\sqrt{x}$ uniformly continuous?
    \par\hfill\par
    The answer is yes, but our method of searching for a $\delta$ did not
    yield anything fruitful. Let's try again. We need the following
    inequality: $|\sqrt{b}-\sqrt{a}|\leq\sqrt{|b-a|}$. This is proved in two
    steps.
    \begin{theorem}
        If $a$ and $b$ are real numbers, and if $a\geq{0}$ and $b\geq{0}$,
        then:
        \begin{equation}
            \sqrt{a+b}\leq\sqrt{a}+\sqrt{b}
        \end{equation}
    \end{theorem}
    \begin{proof}
        Since $a\geq{0}$ and $b\geq{0}$, we have $0\leq2\sqrt{a}\sqrt{b}$.
        But then:
        \begin{align}
            a+b&\leq{a}+2\sqrt{a}\sqrt{b}+b\\
            &=(\sqrt{a}+\sqrt{b})^{2}
        \end{align}
        And therefore $a+b\leq(\sqrt{a}+\sqrt{b})^{2}$. And if
        $x$ and $y$ are real numbers with $x\geq{0}$, $y\geq{0}$ and
        $x^{2}\leq{y}^{2}$, then it is true that $x\leq{y}$. Using this, since
        $a+b\leq(\sqrt{a}+\sqrt{b})^{2}$, we have that
        $\sqrt{a+b}\leq\sqrt{a}+\sqrt{b}$.
    \end{proof}
    \begin{theorem}
        If $a$ and $b$ are real numbers, and if $a\geq{0}$ and $b\geq{0}$,
        then:
        \begin{equation}
            |\sqrt{b}-\sqrt{a}|
            \leq\sqrt{|b-a|}
        \end{equation}
    \end{theorem}
    \begin{proof}
        For simplicity, let's assume $b>a$. If $a>b$ we just need to
        mirror our argument, and if $a=b$ this simply says $0\leq{0}$, which is
        true. So assume $a<b$. We now want to prove
        $\sqrt{b}-\sqrt{a}\leq\sqrt{b-a}$. We'll use the previous result
        that $\sqrt{a+b}\leq\sqrt{a}+\sqrt{b}$.
        \begin{align}
            \sqrt{b}-\sqrt{a}
            &=(\sqrt{b}-\sqrt{a})\frac{\sqrt{a}+\sqrt{b}}{\sqrt{a}+\sqrt{b}}\\
            &=\frac{b-a}{\sqrt{a}+\sqrt{b}}\\
            &\leq\frac{b-a}{\sqrt{b+a}}\\
            &\leq\frac{b-a}{\sqrt{b-a}}\\
            &=\sqrt{b-a}
        \end{align}
        And therefore $\sqrt{b}-\sqrt{a}\leq\sqrt{b-a}$.
    \end{proof}
    Let's use this.
    \begin{equation}
        \textbf{Want:}\quad
        |x-x_{0}|<\delta
        \Rightarrow
        |\sqrt{x}-\sqrt{x_{0}}|<\varepsilon
    \end{equation}
    But $|\sqrt{x}-\sqrt{x_{0}}|\leq\sqrt{|x-x_{0}|}$, so we update our
    wish-list:
    \begin{equation}
        \textbf{Want:}\quad
        |x-x_{0}|<\delta
        \Rightarrow
        \sqrt{|x-x_{0}|}<\varepsilon
    \end{equation}
    Since we only care about $|x-x_{0}|<\delta$, we can update our wishlist
    once again:
    \begin{equation}
        \textbf{Want:}\quad
        |x-x_{0}|<\delta
        \Rightarrow
        \sqrt{\delta}\leq\varepsilon
    \end{equation}
    We can now choose $\delta=\varepsilon^{2}$. Note, this is the same
    $\delta$ we chose for the case $x_{0}=0$. This is where $f(x)=\sqrt{x}$ is
    steepest, and since this $\delta$ works at $x_{0}=0$, it will work
    everywhere else.
    \par\hfill\par
    Let's prove this. Let $\varepsilon>0$. Choose $\delta=\varepsilon^{2}$.
    If $|x-x_{0}|<\delta$, then:
    \begin{align}
        |x-x_{0}|&<\varepsilon^{2}
            \tag{Definition of $\delta$}\\
        \Rightarrow
            \sqrt{|x-x_{0}|}&<\varepsilon
                \tag{Since $\sqrt{\hspace{1em}}$ is an Increasing Function}\\
        \Rightarrow
            |\sqrt{x}-\sqrt{x_{0}}|&<\varepsilon
                \tag{Since $|\sqrt{x}-\sqrt{x_{0}}|\leq\sqrt{|x-x_{0}|}$}\\
        \Rightarrow
            |f(x)-f(x_{0})|&<\varepsilon
                \tag{Definition of $f(x)$}
    \end{align}
    And hence the square root function is \textit{uniformly} continuous.
    \newpage
    I, the copyright holder of this work, release it into the public domain.
    This applies worldwide. In some countries this may not be legally possible;
    if so: I grant anyone the right to use this work for any purpose, without
    any conditions, unless such conditions are required by law.
    \par\hfill\par
    The source code used to generate this document is free software and released
    under version 3 of the GNU General Public License.
\end{document}
