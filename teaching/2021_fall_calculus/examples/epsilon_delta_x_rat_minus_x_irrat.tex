%-----------------------------------LICENSE------------------------------------%
%   This file is part of Mathematics-and-Physics.                              %
%                                                                              %
%   Mathematics-and-Physics is free software: you can redistribute it and/or   %
%   modify it under the terms of the GNU General Public License as             %
%   published by the Free Software Foundation, either version 3 of the         %
%   License, or (at your option) any later version.                            %
%                                                                              %
%   Mathematics-and-Physics is distributed in the hope that it will be useful, %
%   but WITHOUT ANY WARRANTY; without even the implied warranty of             %
%   MERCHANTABILITY or FITNESS FOR A PARTICULAR PURPOSE.  See the              %
%   GNU General Public License for more details.                               %
%                                                                              %
%   You should have received a copy of the GNU General Public License along    %
%   with Mathematics-and-Physics.  If not, see <https://www.gnu.org/licenses/>.%
%------------------------------------------------------------------------------%
\documentclass{article}
\usepackage{amsmath}                            % Needed for align.
\usepackage{amssymb}                            % Needed for mathbb.
\usepackage{amsthm}                             % For the theorem environment.
\usepackage{graphicx}                           % Needed for figures.
\usepackage{hyperref}                           % Hyperlinks for figures.
\hypersetup{colorlinks=true, linkcolor=blue}    % Colors for hyperref.
\graphicspath{{../../../images/}}

\newtheoremstyle{normal}
    {\topsep}               % Amount of space above the theorem.
    {\topsep}               % Amount of space below the theorem.
    {}                      % Font used for body of theorem.
    {}                      % Measure of space to indent.
    {\bfseries}             % Font of the header of the theorem.
    {}                      % Punctuation between head and body.
    {.5em}                  % Space after theorem head.
    {}

\theoremstyle{normal}
\newtheorem{definition}{Definition}

\title{$\varepsilon-\delta$ Continuity - Continuous At Only One Point}
\author{Ryan Maguire}
\date{\today}

% No indent and no paragraph skip.
\setlength{\parindent}{0em}
\setlength{\parskip}{0em}

\begin{document}
    \maketitle
    The definition of continuity is as follows:
    \begin{definition}
        A real-valued function that is continuous at a point
        $x_{0}\in\mathbb{R}$ is a function $f:\mathbb{R}\rightarrow\mathbb{R}$
        such that for all $\varepsilon>0$ there exists a $\delta>0$ such that
        for all $x\in\mathbb{R}$ with $|x-x_{0}|<\delta$ it is true that
        $|f(x)-f(x_{0})|<\varepsilon$.
    \end{definition}
    Let's consider the following bizarre function:
    \begin{equation}
        f(x)=
        \begin{cases}
            x&x\in\mathbb{Q}\\
            -x&x\notin\mathbb{Q}
        \end{cases}
    \end{equation}
    The set $\mathbb{Q}$ is the set of \textit{rational numbers} of the
    form $x=p/q$ for integer $p$ and $q$ with $q\ne{0}$, so if
    $x\notin\mathbb{Q}$ (read \textit{$x$ is not in $\mathbb{Q}$}), then
    $x$ is irrational. So we have $f(x)=x$ for rational numbers and
    $f(x)=-x$ for irrational numbers. Two properties of the real number to
    note. First, between any two real numbers $a$ and $b$ with $a<b$ there is a
    rational number $x$ such that $a<x<b$. Second, for any two real numbers
    $a$ and $b$ with $a<b$ there is an irrational number $y$ such that $a<y<b$.
    Intuitively, the rationals and irrationals are densely distributed across
    the real number line. Using this, we have that if $x_{0}\ne{0}$, then
    $f$ is \textit{not} continuous at $x_{0}$. To see this, since the rational
    values give $f(x)=x$ and irrational numbers give $f(x)=-x$, in the region
    around $x_{0}$ there are jumps that are about $2x_{0}$ in height. But what
    would happen if we chose $x_{0}=0$?
    \par\hfill\par
    Claim: $f$ is continuous at $0$. Let's prove this. Like every
    $\varepsilon-\delta$ proof, we start with the statement
    \textit{let $\varepsilon>0$}. Choose $\delta=\varepsilon$. If
    $|x-0|<\delta$, then $|x|<\delta$ since $|x-0|=|x|$. But, since $f(0)=0$,
    and since $f(x)=\pm{x}$, depending on $x$, we have:
    \begin{equation}
        |f(x)-f(0)|=|f(x)|=|\pm{x}|=|x|<\delta=\varepsilon
    \end{equation}
    And therefore $|f(x)-f(0)|<\varepsilon$. That is, $f$ is continuous at 0.
    \par\hfill\par
    Now ask yourself, how could we prove that $f$ is continuous at $x=0$
    without the use of the $\varepsilon-\delta$ definition? There are other
    definitions of continuity (via \textit{sequences} and via
    \textit{open sets}) that are equivalent to the $\varepsilon-\delta$ but
    these are considerably more advanced and are reserved for courses like
    topology or metric spaces. So how, using what we know, would we
    prove $f$ is continuous at $x=0$? The answer is, without
    $\varepsilon-\delta$, we most likely can't. This function is so pathological
    that intuition doesn't help all that much and we need to resort to a
    rigorous definition that we can then apply to the problem. The graph
    of $f$ is shown in Fig.~\ref{fig:graph_of_f}. This is a rough sketch of
    the function. The function itself is impossible to to draw since it has
    infinitely many jumps and computers can't render that. But this picture
    suffices for intuition.
    \begin{figure}
        \centering
        \includegraphics{x_rat_minus_x_irrat.pdf}
        \caption{The graph of $f$}
        \label{fig:graph_of_f}
    \end{figure}
    \newpage
    I, the copyright holder of this work, release it into the public domain.
    This applies worldwide. In some countries this may not be legally possible;
    if so: I grant anyone the right to use this work for any purpose, without
    any conditions, unless such conditions are required by law.
    \par\hfill\par
    The source code used to generate this document is free software and released
    under version 3 of the GNU General Public License.
\end{document}
