%-----------------------------------LICENSE------------------------------------%
%   This file is part of Mathematics-and-Physics.                              %
%                                                                              %
%   Mathematics-and-Physics is free software: you can redistribute it and/or   %
%   modify it under the terms of the GNU General Public License as             %
%   published by the Free Software Foundation, either version 3 of the         %
%   License, or (at your option) any later version.                            %
%                                                                              %
%   Mathematics-and-Physics is distributed in the hope that it will be useful, %
%   but WITHOUT ANY WARRANTY; without even the implied warranty of             %
%   MERCHANTABILITY or FITNESS FOR A PARTICULAR PURPOSE.  See the              %
%   GNU General Public License for more details.                               %
%                                                                              %
%   You should have received a copy of the GNU General Public License along    %
%   with Mathematics-and-Physics.  If not, see <https://www.gnu.org/licenses/>.%
%------------------------------------------------------------------------------%
\documentclass{article}
\usepackage{amssymb}                            % Needed for mathbb.
\usepackage{graphicx}                           % Needed for figures.
\usepackage{hyperref}                           % Hyperlinks for figures.
\hypersetup{colorlinks=true, linkcolor=blue}    % Colors for hyperref.
\graphicspath{{../../../images/}}

\title{Inverse Functions - Example 1}
\author{Ryan Maguire}
\date{\today}

% No indent and no paragraph skip.
\setlength{\parindent}{0em}
\setlength{\parskip}{0em}

\begin{document}
    \maketitle
    If $f:A\rightarrow{B}$ is a function that takes in elements from $A$ and
    returns elements in $B$, it is possible for there to be another function
    $g:B\rightarrow{A}$ that takes in elements from $B$ and returns elements
    in $A$ with the property that for all $a\in{A}$ it is true that
    $g(f(a))=a$ and for all $b\in{B}$ it is true that $f(g(b))=b$. Such a
    function is called the \textit{inverse} of $f$, and is often denoted
    $g=f^{-1}$. There are two conditions for an inverse to exist. First,
    if $a_{0}$ and $a_{1}$ are elements of $A$, and if $f(a_{0})=f(a_{1})$,
    then $a_{0}=a_{1}$. To see why this is required, suppose we have a function
    like $f(x)=x^{2}$, and suppose we ask what should the inverse function do
    at the value 1. Should $f^{-1}(1)=1$ or should it be $-1$? Both have the
    property that they square to 1 since $1^{2}=1$ and $(-1)^{2}=1$. To
    prevent the inverse from being ambiguous, $f$ must be
    \textit{one-to-one}, also known as \textit{injective}. This is precisely
    the statement that $f(a_{0})=f(a_{1})$ implies $a_{0}=a_{1}$. Secondly,
    every element $b\in{B}$ must have some other element $a\in{A}$ such that
    $f(a)=b$. Again consider $f(x)=x^{2}$. What should the inverse of $-1$ be?
    There is no real number such that $x^{2}=-1$, so we have an undefined
    value for our inverse function. Functions satisfying this second property
    are said to be \textit{onto}, but also called \textit{surjective}. So, for
    an inverse to exist we need the function to be
    \textit{injective} and \textit{surjective}. These two properties occur
    together so frequently in mathematics that they are given a new name. A
    \textit{bijective} function is a function that is both \textit{injective}
    and \textit{surjective}. The function $f(x)=x^{2}$, when viewed as a
    function $f:\mathbb{R}\rightarrow\mathbb{R}$, is neither injective nor
    surjective.
    \par\hfill\par
    It is often possible to restrict the domain and target of the function
    so that it then becomes bijective. For example, if we only consider
    $f:\mathbb{R}_{\geq{0}}\rightarrow\mathbb{R}_{\geq{0}}$, where I've used
    the notation $\mathbb{R}_{\geq{0}}$ to mean all real numbers greater than
    or equal to zero, then $f(x)=x^{2}$ \textit{is} bijective. Every
    non-negative real number squares to a unique number, and every non-negative
    real number is the square of a unique non-negative real number. This allows
    us to define the square root $f^{-1}(x)=\sqrt{x}$ which is the inverse
    of our original function $f(x)=x^{2}$. Both $f$ and $f^{-1}$ are
    shown in Fig.~\ref{fig:x2_and_sqrt_x}. A crucial feature to note is that
    $\sqrt{x}$ is the \textit{reflection} of $x^{2}$ across the line $y=x$.
    This is true of inverse functions and gives us a means of plotting them,
    even if we can't find nice formulas.
    \begin{figure}
        \centering
        \includegraphics{x_squared_and_sqrt_x.pdf}
        \caption{The graph of $x^{2}$ and $\sqrt{x}$}
        \label{fig:x2_and_sqrt_x}
    \end{figure}
    \newpage
    I, the copyright holder of this work, release it into the public domain.
    This applies worldwide. In some countries this may not be legally possible;
    if so: I grant anyone the right to use this work for any purpose, without
    any conditions, unless such conditions are required by law.
    \par\hfill\par
    The source code used to generate this document is free software and released
    under version 3 of the GNU General Public License.
\end{document}
