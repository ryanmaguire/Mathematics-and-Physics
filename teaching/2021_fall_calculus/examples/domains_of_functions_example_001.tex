%-----------------------------------LICENSE------------------------------------%
%   This file is part of Mathematics-and-Physics.                              %
%                                                                              %
%   Mathematics-and-Physics is free software: you can redistribute it and/or   %
%   modify it under the terms of the GNU General Public License as             %
%   published by the Free Software Foundation, either version 3 of the         %
%   License, or (at your option) any later version.                            %
%                                                                              %
%   Mathematics-and-Physics is distributed in the hope that it will be useful, %
%   but WITHOUT ANY WARRANTY; without even the implied warranty of             %
%   MERCHANTABILITY or FITNESS FOR A PARTICULAR PURPOSE.  See the              %
%   GNU General Public License for more details.                               %
%                                                                              %
%   You should have received a copy of the GNU General Public License along    %
%   with Mathematics-and-Physics.  If not, see <https://www.gnu.org/licenses/>.%
%------------------------------------------------------------------------------%
\documentclass{article}
\usepackage{graphicx}                           % Needed for figures.
\usepackage{hyperref}                           % Hyperlinks for figures.
\hypersetup{colorlinks=true, linkcolor=blue}    % Colors for hyperref.
\graphicspath{{../../../images/}}

\title{Domains of Functions - Example 1}
\author{Ryan Maguire}
\date{\today}

% No indent and no paragraph skip.
\setlength{\parindent}{0em}
\setlength{\parskip}{0em}

\begin{document}
    \maketitle
    A result known back to the times of Euclid (300 B.C.E) states that if
    $a$ and $b$ are real numbers, and if $ab=0$, then either $a=0$, or
    $b=0$ (or both). The proof is quite simple. If $a$ is not zero, then
    we can divide both sides of $ab=0$ by $a$, and we obtain $b=0$. Similarly,
    if $b$ is not zero, then we can divide both sides of $ab=0$ by $b$ and
    obtain $a=0$. Hence, if $ab=0$, then at least one of these must be zero.
    We can use this result to find for which real numbers a formula may be
    undefined.
    \par\hfill\par
    Let's consider the following expression:
    \begin{equation}
        f(x)=\frac{1}{x(1-x^{2})}
    \end{equation}
    For what subset of the real numbers is this formula well-defined? The only
    problem we could encounter is a division by zero. The denominator of the
    expression is $x(1-x^{2})$, so we need to exclude real numbers where this
    evaluates to zero. Using the statement from the first paragraph, if
    $x(1-x^{2})=0$, then either $x=0$, or $1-x^{2}=0$. The expression
    $1-x^{2}=0$ has two solutions, $1$ and $-1$. We can see this by factoring,
    obtaining $1-x^{2}=(1+x)(1-x)$. So, in total, there are 3 real numbers
    where this formula is not well-defined: 0, 1, and $-1$. Using the notation
    from set theory, we can write the domain of $f$ via:
    \begin{equation}
        D=(-\infty,-1)\cup(-1,0)\cup(0,1)\cup(1,\infty)
    \end{equation}
    The function is plotted in Fig~\ref{fig:graph_of_f}.
    \begin{figure}
        \centering
        \includegraphics{one_by_x_times_one_minus_x_squared.pdf}
        \caption{Graph of the function $f$}
        \label{fig:graph_of_f}
    \end{figure}
    \newpage
    I, the copyright holder of this work, release it into the public domain.
    This applies worldwide. In some countries this may not be legally possible;
    if so: I grant anyone the right to use this work for any purpose, without
    any conditions, unless such conditions are required by law.
    \par\hfill\par
    The source code used to generate this document is free software and released
    under version 3 of the GNU General Public License.
\end{document}
