%-----------------------------------LICENSE------------------------------------%
%   This file is part of Mathematics-and-Physics.                              %
%                                                                              %
%   Mathematics-and-Physics is free software: you can redistribute it and/or   %
%   modify it under the terms of the GNU General Public License as             %
%   published by the Free Software Foundation, either version 3 of the         %
%   License, or (at your option) any later version.                            %
%                                                                              %
%   Mathematics-and-Physics is distributed in the hope that it will be useful, %
%   but WITHOUT ANY WARRANTY; without even the implied warranty of             %
%   MERCHANTABILITY or FITNESS FOR A PARTICULAR PURPOSE.  See the              %
%   GNU General Public License for more details.                               %
%                                                                              %
%   You should have received a copy of the GNU General Public License along    %
%   with Mathematics-and-Physics.  If not, see <https://www.gnu.org/licenses/>.%
%------------------------------------------------------------------------------%
\documentclass{article}

% Needed for align.
\usepackage{amsmath}
\usepackage{amsthm}                             % For the theorem environment.
\usepackage{hyperref}                           % Hyperlinks for figures.
\newtheoremstyle{normal}
    {\topsep}               % Amount of space above the theorem.
    {\topsep}               % Amount of space below the theorem.
    {}                      % Font used for body of theorem.
    {}                      % Measure of space to indent.
    {\bfseries}             % Font of the header of the theorem.
    {}                      % Punctuation between head and body.
    {.5em}                  % Space after theorem head.
    {}

\theoremstyle{plain}
\newtheorem{theorem}{Theorem}

\title{L'H\^{o}pital's Rule}
\author{Ryan Maguire}
\date{\today}

% No indent and no paragraph skip.
\setlength{\parindent}{0em}
\setlength{\parskip}{0em}

\begin{document}
    \maketitle
    Often one comes across two functions, both of which evaluate
    to zero at a given point, and wishes to find the limit of the \textit{ratio}
    of these functions. For an example that is common in optics, consider
    $f(x)=\sin(x)$ and $g(x)=x$. The fraction $f(x)/g(x)$ is the
    $\textrm{sinc}$ function
    \begin{equation}
        \textrm{sinc}(x)=
        \begin{cases}
            \frac{\sin(x)}{x}&x\ne{0}\\
            1&x=0
        \end{cases}
    \end{equation}
    Defining $\textrm{sinc}(0)=1$ is done to make the function continuous at
    the origin. If we know the \textit{Maclaurin expansion} of $\sin$ we may
    observe the behavior of $\textrm{sinc}$ near the origin by simplifying this
    expansion. For $\sin(x)$ we have:
    \begin{align}
        \sin(x)
            &=\sum_{n=0}^{\infty}(-1)^{n}\frac{x^{2n+1}}{(2n+1)!}\\
            &=x-\frac{x^{3}}{6}+\frac{x^{5}}{120}-\frac{x^{7}}{5040}+\cdots
    \end{align}
    the $\textrm{sinc}$ function is then:
    \begin{align}
        \textrm{sinc}(x)
            &=\frac{1}{x}\sum_{n=0}^{\infty}(-1)^{n}\frac{x^{2n+1}}{(2n+1)!}\\
            &=\sum_{n=0}^{\infty}(-1)^{n}\frac{x^{2n}}{(2n+1)!}\\
            &=1-\frac{x^{2}}{6}+\frac{x^{4}}{120}-\frac{x^{6}}{5040}+\cdots
    \end{align}
    from this we may observe that as $x$ tends to zero, $\textrm{sinc}(x)$
    converges to 1. All of this requires knowledge of Taylor and Maclaurin
    series and we may be without those at this point in calculus. Another way
    of seeing this is using the \textit{small angle approximation}. This says
    $\sin(x)\approx{x}$ for very small $x$. The $\textrm{sinc}$ function then
    behaves like:
    \begin{equation}
        \textrm{sinc}(x)\approx\frac{x}{x}=1
    \end{equation}
    This is convincing enough for $\textrm{sinc}$, but other functions may not
    have such convenient expressions. The general plan of attack is to invoke
    \textit{L'H\^{o}pital's rule}.
    \begin{theorem}[L'H\^{o}pital's Rule]
        If $f$ and $g$ are functions on some open interval $(a,\,b)$ about a
        point $x_{0}\in(a,\,b)$, if $f(x_{0})=0$ and $g(x_{0})=0$, if
        $g(x)\ne{0}$ for all $x\ne{x}_{0}$, if
        $f$ and $g$ are differentiable at $x_{0}$ with
        $g'(x_{0})\ne{0}$, and if $f'(x_{0})/g'(x_{0})=L$, then:
        \begin{equation}
            \lim_{x\rightarrow{x_{0}}}\frac{f(x)}{g(x)}=L
        \end{equation}
    \end{theorem}
    \begin{proof}
        Since $g(x)\ne{0}$ for all $x\ne{x}_{0}$ we may divide by $g(x)$ freely.
        We have for all $x\ne{x}_{0}$:
        \begin{align}
            \frac{f(x)}{g(x)}
                &=\frac{f(x)-f(x_{0})}{g(x)-g(x_{0})}
                    \tag{Since $f(x_{0})=g(x_{0})=0$}\\
                &=\frac{f(x)-f(x_{0})}{g(x)-g(x_{0})}
                    \frac{x-x_{0}}{x-x_{0}}
                        \tag{Multiplication by 1}\\
                &=\frac{f(x)-f(x_{0})}{x-x_{0}}
                    \frac{x-x_{0}}{g(x)-g(x_{0})}
                        \tag{Simplify}
        \end{align}
        when we take limits we obtain:
        \begin{align}
            \lim_{x\rightarrow{0}}\frac{f(x)}{g(x)}
                &=\lim_{x\rightarrow{0}}\frac{f(x)-f(x_{0})}{x-x_{0}}
                    \frac{x-x_{0}}{g(x)-g(x_{0})}
                        \tag{Substitution}\\
                &=\lim_{x\rightarrow{0}}\frac{f(x)-f(x_{0})}{x-x_{0}}
                    \lim_{x\rightarrow{0}}\frac{x-x_{0}}{g(x)-g(x_{0})}
                        \tag{Product Rule of Limits}\\
                &=\frac{f'(x_{0})}{g'(x_{0})}
                    \tag{Differentiability of $f$ and $g$ at $x_{0}$}\\
                &=L
                    \tag{Hypothesis}
        \end{align}
        and hence the limit of the ratios is $L$.
    \end{proof}
    A common way of modifying this theorem is by using a formula for the the
    derivatives of $f$ and $g$.
    \begin{theorem}
        If $f$ and $g$ are continuously differentiable on an open interval
        $(a,\,b)$, if $x_{0}\in(a,\,b)$, if $f(x_{0})=0$ and $g(x_{0})=0$,
        if $g(x)\ne{0}$ for all $x\ne{x}_{0}$, if $g'(x_{0})\ne{0}$, then:
        \begin{equation}
            \lim_{x\rightarrow{x}_{0}}\frac{f(x)}{g(x)}
                =\lim_{x\rightarrow{x}_{0}}\frac{f'(x)}{g'(x)}
        \end{equation}
    \end{theorem}
    \begin{proof}
        The hypotheses here satisfy the criterion for the previous theorem, and
        hence:
        \begin{equation}
            \lim_{x\rightarrow{x}_{0}}\frac{f(x)}{g(x)}
                =\frac{f'(x_{0})}{g'(x_{0})}
        \end{equation}
        but $f$ and $g$ are \textit{continuously} differentiable, meaning
        $f'$ and $g'$ are continuous functions of $x$. Hence:
        \begin{equation}
            \lim_{x\rightarrow{x}_{0}}\frac{f'(x)}{g'(x)}
                =\frac{f'(x_{0})}{g'(x_{0})}
        \end{equation}
        by comparing these two equations we obtain:
        \begin{equation}
            \lim_{x\rightarrow{x}_{0}}\frac{f(x)}{g(x)}
                =\lim_{x\rightarrow{x}_{0}}\frac{f'(x)}{g'(x)}
        \end{equation}
        completing the proof.
    \end{proof}
    We can use this to find the limit of $\textrm{sinc}(x)$ as $x$ approaches
    zero.
    \begin{align}
        \lim_{x\rightarrow{0}}\textrm{sinc(x)}
            &=\lim_{x\rightarrow{0}}\frac{\sin(x)}{x}
                \tag{Definition of $\textrm{sinc}$}\\
            &=\lim_{x\rightarrow{0}}
                \frac{\cos(x)}{1}
                    \tag{L'H\^{o}pital's Rule}\\
            &=1
                \tag{Evaluate the limit}
    \end{align}
    A word of warning, the criterion that $f$ and $g$ are \textit{continuously}
    differentiable is needed, and cannot be relaxed to just differentiable.
    Consider:
    \begin{equation}
        f(x)=
        \begin{cases}
            x^{2}\sin\big(\frac{1}{x}\big)&x\ne{0}\\
            0&x=0
        \end{cases}
    \end{equation}
    and let $g(x)=x$. Application of the squeeze theorem tells us that $f$ is
    continuous, and moreover it is differentiable everywhere. The difficulty
    in this situation is that $f$ is not \textit{continuously} differentiable,
    the derivative $f'(x)$ has a discontinuity at $x=0$. The squeeze theorem
    tells us that $f'(x)=0$ since $-x^{2}\leq{f}(x)\leq{x}^{2}$. Application of
    derivative rules gives us a formula for all other $x$:
    \begin{equation}
        f'(x)=2x\sin\Big(\frac{1}{x}\Big)-\cos\Big(\frac{1}{x}\Big)
    \end{equation}
    The $\cos(1/x)$ term oscillates rapidly near the origin causing $f'$ to
    have a discontinuity there. Application of the second version of
    L'H\^{o}pital's rule will do us no good. The first version is still fine,
    we have:
    \begin{equation}
        \lim_{x\rightarrow{0}}\frac{f(x)}{g(x)}
            =\frac{f'(0)}{g'(0)}
            =\frac{0}{1}
            =0
    \end{equation}
    One could also prove this using the squeeze theorem since
    $-x\leq{f}(x)/g(x)\leq{x}$, and hence $f(x)/g(x)$ converges to zero as
    $x$ approaches the origin.
\end{document}
