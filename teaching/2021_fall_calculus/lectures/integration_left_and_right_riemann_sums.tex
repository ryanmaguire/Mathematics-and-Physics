%-----------------------------------LICENSE------------------------------------%
%   This file is part of Mathematics-and-Physics.                              %
%                                                                              %
%   Mathematics-and-Physics is free software: you can redistribute it and/or   %
%   modify it under the terms of the GNU General Public License as             %
%   published by the Free Software Foundation, either version 3 of the         %
%   License, or (at your option) any later version.                            %
%                                                                              %
%   Mathematics-and-Physics is distributed in the hope that it will be useful, %
%   but WITHOUT ANY WARRANTY; without even the implied warranty of             %
%   MERCHANTABILITY or FITNESS FOR A PARTICULAR PURPOSE.  See the              %
%   GNU General Public License for more details.                               %
%                                                                              %
%   You should have received a copy of the GNU General Public License along    %
%   with Mathematics-and-Physics.  If not, see <https://www.gnu.org/licenses/>.%
%------------------------------------------------------------------------------%
\documentclass{article}
\usepackage{graphicx}   % Needed for figures.
\usepackage{amsmath}    % Needed for align.
\usepackage{amssymb}    % Needed for mathbb.
\usepackage{amsthm}     % For the theorem environment.
\graphicspath{{../../../images/}}

\title{Left and Right Riemann Sums}
\author{Ryan Maguire}
\date{\today}

% No indent and no paragraph skip.
\setlength{\parindent}{0em}
\setlength{\parskip}{0em}

\begin{document}
    \maketitle
    For continuous functions $f:[a,b]\rightarrow\mathbb{R}$ it doesn't matter
    \textit{how} you partition $[a,b]$, as long as $\Delta{x}_{n}$ tends to
    zero. Because of this there are two useful methods of computing integrals:
    the left and right Riemann sums. The Riemann sum of a partition
    $a=x_{0}<x_{1}<\cdots<x_{N}=b$ is the sum:
    \begin{equation}
        R_{L}(f,x_{n})=\sum_{n=0}^{N-1}f(x_{n})\Delta{x}_{n}
    \end{equation}
    The advantage this has over lower and upper sums is we do not need to
    calculate the minimum and maximum of $f(x)$ in the interval
    $[x_{n},x_{n+1}]$.
    \begin{figure}
        \centering
        \includegraphics{left_sum_of_continuous_function.pdf}
        \caption{Left Sum of a Continuous Function}
    \end{figure}
    The right hand sum is similarly defined.
    \begin{equation}
        R_{R}(f,x_{n})=\sum_{n=0}^{N-1}f(x_{n+1})\Delta{x}_{n}
    \end{equation}
    \begin{figure}
        \centering
        \includegraphics{right_sum_of_continuous_function.pdf}
        \caption{Right Sum of a Continuous Function}
    \end{figure}
    Let's explicitly use left Riemann sums to compute, for any $a>0$, the
    values:
    \begin{align}
        F(a)&=\int_{0}^{a}x\;\textrm{d}x\\
        G(a)&=\int_{0}^{a}x^{2}\;\textrm{d}x
    \end{align}
    Here we'll let $f(x)=x$ and $g(x)=x^{2}$.
    Our partition is $x_{n}=\frac{an}{N}$ where $0\leq{n}\leq{N}$. This makes
    $\Delta{x}_{n}$ a constant, $\Delta{x}_{n}=\frac{a}{N}$. The sum is:
    \begin{equation}
        R_{L}(f,x_{n})=\sum_{n=0}^{N-1}\frac{an}{N}\frac{a}{N}
            =\frac{a^{2}}{N^{2}}\sum_{n=0}^{N-1}n
            =\frac{a^{2}}{N^{2}}\frac{N(N-1)}{2}
    \end{equation}
    We proved via induction the formula for this summation, which is
    $\frac{N(N-1)}{2}$. As $N$ tends to infinity, $\Delta{x}_{n}$ tends to zero,
    meaning:
    \begin{equation}
        \int_{0}^{a}x\;\textrm{d}x
            =\frac{a^{2}}{2}\lim_{N\rightarrow\infty}\frac{N(N-1)}{N^{2}}
            =\frac{a^{2}}{2}
    \end{equation}
    We do the same thing for $g$, obtaining:
    \begin{equation}
        R_{L}(g,x_{n})=\sum_{n=0}^{N-1}\frac{a^{2}n^{2}}{N^{2}}\frac{a}{N}
            =\frac{a^{3}}{N^{3}}\sum_{n=0}^{N-1}n^{2}
            =\frac{a^{3}}{N^{3}}\frac{N(N-1)(2N-1)}{6}
    \end{equation}
    Taking the limit as $N$ goes to infinity, we get:
    \begin{equation}
        \int_{0}^{a}x^{2}\;\textrm{d}x=\frac{a^{3}}{3}
    \end{equation}
    \newpage
    I, the copyright holder of this work, release it into the public domain.
    This applies worldwide. In some countries this may not be legally possible;
    if so: I grant anyone the right to use this work for any purpose, without
    any conditions, unless such conditions are required by law.
    \par\hfill\par
    The source code used to generate this document is free software and released
    under version 3 of the GNU General Public License.
\end{document}
