%-----------------------------------LICENSE------------------------------------%
%   This file is part of Mathematics-and-Physics.                              %
%                                                                              %
%   Mathematics-and-Physics is free software: you can redistribute it and/or   %
%   modify it under the terms of the GNU General Public License as             %
%   published by the Free Software Foundation, either version 3 of the         %
%   License, or (at your option) any later version.                            %
%                                                                              %
%   Mathematics-and-Physics is distributed in the hope that it will be useful, %
%   but WITHOUT ANY WARRANTY; without even the implied warranty of             %
%   MERCHANTABILITY or FITNESS FOR A PARTICULAR PURPOSE.  See the              %
%   GNU General Public License for more details.                               %
%                                                                              %
%   You should have received a copy of the GNU General Public License along    %
%   with Mathematics-and-Physics.  If not, see <https://www.gnu.org/licenses/>.%
%------------------------------------------------------------------------------%
\documentclass{article}

\usepackage{amssymb} % mathbb font.
\usepackage{amsmath} % Needed for align.
\usepackage{amsthm}  % For the theorem environment.

\theoremstyle{plain}
\newtheorem{theorem}{Theorem}

\title{Chain Rule}
\author{Ryan Maguire}
\date{\today}

% No indent and no paragraph skip.
\setlength{\parindent}{0em}
\setlength{\parskip}{0em}

\begin{document}
    \maketitle
    The chain rule tells us how to differentiate \textit{compositions} of
    differentiable functions. We can use the following for intuition:
    \begin{align}
        \frac{g\big(f(x+h)\big)-g\big(f(x)\big)}{h}
            &=\frac{g\big(f(x+h)\big)-g\big(f(x)\big)}{f(x+h)-f(x)}
            \cdot\frac{f(x+h)-f(x)}{h}\\
            &\approx
                g'\big(f(x)\big)\cdot{f}'(x)
    \end{align}
    taking limits as $h\rightarrow{0}$ can make this precise but there are a
    lot of subleties in this argument. Firstly, the definition of derivative
    involves $h$ in the denominator, and have a formula in terms of $f$. We
    need to show that this is valid.
    \begin{theorem}
        If $f$ is continuous at a point $x_{0}$, if $g$ is differentiable at
        $f(x_{0})$, and if $f(x)\ne{f}(x_{0})$ for all $x$ in some small
        interval around $x_{0}$ (excluding $x_{0}$ itself),
        then:
        \begin{equation}
            g'\big(f(x_{0})\big)=
                \lim_{x\rightarrow{x}_{0}}
                \frac{g\big(f(x)\big)-g\big(f(x_{0})\big)}{f(x)-f(x_{0})}
        \end{equation}
    \end{theorem}
    \begin{proof}
        Since $f$ is continuous at $x_{0}$ we have that:
        \begin{equation}
            \lim_{x\rightarrow{x}_{0}}f(x)=f(x_{0})
        \end{equation}
        Hence if we set $h(x)=f(x)-f(x_{0})$, then $h(x)$ is zero
        only at $x=x_{0}$ by hypothesis, and the limit of $h(x)$ as $x$
        approaches $x_{0}$ is zero. We may thus divide by $h$ and take
        limits. We get:
        \begin{align}
            g'\big(f(x_{0})\big)
            &=\lim_{h\rightarrow{0}}
                \frac{g\big(f(x_{0})+h\big)-g\big(f(x_{0})\big)}{h}\\
            &=\lim_{x\rightarrow{x}_{0}}
                \frac{g\big(f(x)\big)-g\big(f(x_{0})\big)}{f(x)-f(x_{0})}
        \end{align}
        completing the proof.
    \end{proof}
    \begin{theorem}
        If $f$ is continuous at a point $x_{0}$, if $g$ is differentiable at
        $f(x_{0})$, and if $f(x)\ne{f}(x_{0})$ for all $x$ in some small
        interval around $x_{0}$ (excluding $x_{0}$ itself),
        then:
        \begin{equation}
            g'\big(f(x_{0})\big)=
                \lim_{h\rightarrow{0}}
                \frac{g\big(f(x_{0}+h)\big)-g\big(f(x_{0})\big)}{f(x_{0}+h)-f(x_{0})}
        \end{equation}
    \end{theorem}
    \begin{proof}
        By the previous theorem we have:
        \begin{align}
            g'\big(f(x_{0})\big)
            &=\lim_{x\rightarrow{x}_{0}}
                \frac{g\big(f(x)\big)-g\big(f(x_{0})\big)}{f(x)-f(x_{0})}
        \end{align}
        By setting $x=x_{0}+h$ we may replace the limit with:
        \begin{equation}
            g'\big(f(x_{0})\big)=
                \lim_{h\rightarrow{0}}
                \frac{g\big(f(x_{0}+h)\big)-g\big(f(x_{0})\big)}{f(x_{0}+h)-f(x_{0})}
        \end{equation}
        as desired.
    \end{proof}
    Now what if $f$ has points $x$ that are arbitrarily close to $x_{0}$ with
    $f(x)=f(x_{0})$? The above argument won't work since we'll have divisions
    by zero. Fortunately if this is the case, then $(g\circ{f})'(x_{0})=0$.
    Let's prove this.
    \begin{theorem}
        If $f$ is continuous at $x_{0}$, if $g$ is differentiable at $f(x_{0})$,
        and if there are points $x$ that are arbitrarily close to $x_{0}$ such
        that $f(x)=f(x_{0})$, then $(g\circ{f})'(x_{0})=0$
    \end{theorem}
    \begin{proof}
        Since there are points $x$ that are arbitrarily close to $x_{0}$ with
        $f(x)=f(x_{0})$ we can pick a sequence $x_{1},\,x_{2},\,\dots$ that
        converges to $x_{0}$ with the property that $f(x_{n})=f(x_{0})$ for all
        $n\in\mathbb{N}$ and $x_{n}\ne{x}_{0}$ for all $n\ne{0}$. Then we have:
        \begin{equation}
            \lim_{n\rightarrow\infty}
                \frac{g\big(f(x_{n})\big)-g\big(f(x_{0})\big)}{x_{n}-x_{0}}
            =\lim_{n\rightarrow\infty}
                \frac{0}{x_{n}-x_{0}}
            =0
        \end{equation}
        By the uniqueness of limits, we have:
        \begin{equation}
            (g\circ{f})'(x_{0})
            =\lim_{n\rightarrow\infty}
                \frac{g\big(f(x_{n})\big)-g\big(f(x_{0})\big)}{x_{n}-x_{0}}
        \end{equation}
        and hence $(g\circ{f})'(x_{0})=0$.
    \end{proof}
    We can now combine the previous results to prove the chain rule.
    \begin{theorem}[Chain Rule]
        If $g$ and $f$ are differentiable functions, then:
        \begin{equation}
            (g\circ{f})'(x_{0})=g'\big(f(x_{0})\big)\cdot{f}'(x_{0})
        \end{equation}
    \end{theorem}
    \begin{proof}
        We split the proof into two cases. First, the case where there are
        points $x$ arbitrarily close to $x_{0}$ with $f(x)=f(x_{0})$. This
        implies $f'(x_{0})=0$, and it also implies
        $(g\circ{f})'(x_{0})=0$, so our formula merely says $0=0$,
        which is true. Next, the case where there is some interval around
        $x_{0}$ such that $f(x)\ne{f}(x_{0})$ for all $x\ne{x}_{0}$. We have
        by the previous theorems:
        \begin{align}
            (g\circ{f})'(x_{0})
            &=\lim_{h\rightarrow{0}}
            \frac{g\big(f(x_{0}+h)\big)-g\big(f(x_{0})\big)}{h}\\
            &=\lim_{h\rightarrow{0}}
            \frac{g\big(f(x_{0}+h)\big)-g\big(f(x_{0})\big)}{f(x_{0}+h)-f(x_{0})}
            \cdot\frac{f(x_{0}+h)-f(x_{0})}{h}\\
            &=
            \lim_{h\rightarrow{0}}
            \frac{g\big(f(x_{0}+h)\big)-g\big(f(x_{0})\big)}{f(x_{0}+h)-f(x_{0})}
            \cdot\lim_{h\rightarrow{0}}\frac{f(x_{0}+h)-f(x_{0})}{h}\\
            &=g'\big(f(x_{0})\big)\cdot{f}'(x_{0})
        \end{align}
        as desired.
    \end{proof}
    Now let's use it. First, an example where we can perform the calculuation
    in two ways to verify our result. Let $f(x)=x+1$ and $g(x)=x^2$. The
    composition is $(g\circ{f})(x)=(x+1)^{2}$. We could use some elementary
    algebra and see that this is equal to $x^{2}+2x+1$, and differentiating this
    yields $(g\circ{f})'(x)=2x+2$. Let's use the chain rule. We have
    $f'(x)=1$ and $g'(x)=2x$. The chain rule then says:
    \begin{align}
        (g\circ{f})'(x)
            &=g'\big(f(x)\big)\cdot{f}'(x)
                \tag{Chain Rule}\\
            &=g'(x+1)\cdot{1}
                \tag{Substitute $f$ and $f'$}\\
            &=2(x+1)
                \tag{Substitute $g'$}\\
            &=2x+2
                \tag{Simplify}
    \end{align}
    Now let's tackle a problem that would be quite formidable without the
    chain rule. Define $f(x)=1-x^{2}$ and $g(x)=\sqrt{x}$. The composition
    is $(g\circ{f})(x)=\sqrt{1-x^{2}}$. The derivative of $1-x^{2}$ is $-2x$
    and $\sqrt{x}$ can be obtained from the generalized power rule (which
    requires the chain rule to prove) or directly from the limit definition.
    We have:
    \begin{align}
        \lim_{h\rightarrow{0}}
        \frac{\sqrt{x+h}-\sqrt{x}}{h}
        &=\lim_{h\rightarrow{0}}
        \frac{\sqrt{x+h}-\sqrt{x}}{h}\cdot
        \frac{\sqrt{x+h}+\sqrt{x}}{\sqrt{x+h}+\sqrt{x}}\\
        &=\lim_{h\rightarrow{0}}
        \frac{x+h-x}{h(\sqrt{x+h}+\sqrt{x})}\\
        &=\lim_{h\rightarrow{0}}
        \frac{h}{h(\sqrt{x+h}+\sqrt{x}}\\
        &=\lim_{h\rightarrow{0}}\frac{1}{\sqrt{x+h}+\sqrt{x}}\\
        &=\frac{1}{2\sqrt{x}}
    \end{align}
    Let's use this to compute $(g\circ{f})'(x)$. We get:
    \begin{align}
        (g\circ{f})'(x)
        &=g'\big(f(x)\big)\cdot{f}'(x)
            \tag{Chain Rule}\\
        &=g'(1-x^{2})\cdot(-2x)
            \tag{Substitute $f$ and $f'$}\\
        &=\frac{1}{2\sqrt{1-x^{2}}}\cdot(-2x)
            \tag{Substitute $g'$}\\
        &=\frac{-2x}{2\sqrt{1-x^{2}}}
            \tag{Multiply}\\
        &=\frac{-x}{\sqrt{1-x^{2}}}
            \tag{Simplify}
    \end{align}
\end{document}
