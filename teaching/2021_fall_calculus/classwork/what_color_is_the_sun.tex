%-----------------------------------LICENSE------------------------------------%
%   This file is part of Mathematics-and-Physics.                              %
%                                                                              %
%   Mathematics-and-Physics is free software: you can redistribute it and/or   %
%   modify it under the terms of the GNU General Public License as             %
%   published by the Free Software Foundation, either version 3 of the         %
%   License, or (at your option) any later version.                            %
%                                                                              %
%   Mathematics-and-Physics is distributed in the hope that it will be useful, %
%   but WITHOUT ANY WARRANTY; without even the implied warranty of             %
%   MERCHANTABILITY or FITNESS FOR A PARTICULAR PURPOSE.  See the              %
%   GNU General Public License for more details.                               %
%                                                                              %
%   You should have received a copy of the GNU General Public License along    %
%   with Mathematics-and-Physics.  If not, see <https://www.gnu.org/licenses/>.%
%------------------------------------------------------------------------------%
\documentclass{article}
\usepackage{graphicx}                           % Needed for figures.
\usepackage{amsmath}                            % Needed for align.

\title{What Color is the Sun?}
\author{Ryan Maguire}
\date{Fall 2021}

% No indent and no paragraph skip.
\setlength{\parindent}{0em}
\setlength{\parskip}{0em}

\begin{document}
    \maketitle
    I think most of us are convinced that (in the year 2021 C.E.) that the sun
    is yellow. If you're reading this in 4 billion years, congratulations! The
    sun has probably swallowed the Earth and is now redder.
    \par\hfill\par
    Let's use calculus to derive the fact that the sun is yellow. Planck's
    law says that the spectrum of a black-body (which the sun is) is given by
    the following horrendous formula:
    \begin{equation}
        u(\lambda)=
            \frac{2hc^{2}}{\lambda^{5}}
                \frac{1}{\exp\big(\frac{hc}{\lambda{k}T})-1}
    \end{equation}
    Fortunately, most things here are constant. $h$, $c$, and $k$ are universal
    constants known as Planck's constant, the speed of light, and
    Boltzmann's constant, respectively. $T$ is the temperature of the sun,
    and $\lambda$ is the wavelength of light given off. $u(\lambda)$ measures
    how much of a certain wavelength is emmitted. So if we want to know what
    the temperature of the sun is, we are simply asking where does $u(\lambda)$
    hit its \textit{absolute maximum}. We need to differentiate $u$ and solve
    for zero. Becuase $u$ is so messy, the work is very tedious, but it only
    uses the rules of differentiation. The chain rule, the quotient rule, and
    the power rule.
    \par\hfill\par
    At the end of our messy calculation, we'd get the following:
    \begin{equation}
        \lambda_{max}=\frac{b}{T}
    \end{equation}
    where $b$ is a constant called
    \textit{Wien's displacement constant}. This formula is known as
    \textit{Wien's law}. The value $b$ is roughly:
    \begin{equation}
        b=2.897771955\times{10}^{-3}\textrm{mK}
    \end{equation}
    mK means \textit{meters-Kelvin}. The sun has a surface temperature of
    5,772K, so plugging this into our formula we get:
    \begin{equation}
        \lambda_{max}=\frac{2.897771955\times{10}^{-3}}{5,772}\textrm{meters}
            =5.0203949\times{10}^{-7}\textrm{meters}
    \end{equation}
    Perfect yellow is $5.6\times{10}^{-7}$ meters, so our sun slightly greener
    than true yellow. But still, pretty yellow.
    \par\hfill\par
    As a final question, what star is hotter, a blue star, or a red star?
    Our intuition on Earth says that red is hotter. Red is the color of fire,
    blue is the color of water, so red should be hotter. Wein's law says
    otherwise. If we have a very large temperature, we get a very small
    wavelength, and blue has a smaller wavelength than red. So the hottest
    stars in the sky are the blue and blue-white ones.
    \newpage
    I, the copyright holder of this work, release it into the public domain.
    This applies worldwide. In some countries this may not be legally possible;
    if so: I grant anyone the right to use this work for any purpose, without
    any conditions, unless such conditions are required by law.
    \par\hfill\par
    The source code used to generate this document is free software and released
    under version 3 of the GNU General Public License.
\end{document}
