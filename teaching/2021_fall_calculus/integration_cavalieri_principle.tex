%-----------------------------------LICENSE------------------------------------%
%   This file is part of Mathematics-and-Physics.                              %
%                                                                              %
%   Mathematics-and-Physics is free software: you can redistribute it and/or   %
%   modify it under the terms of the GNU General Public License as             %
%   published by the Free Software Foundation, either version 3 of the         %
%   License, or (at your option) any later version.                            %
%                                                                              %
%   Mathematics-and-Physics is distributed in the hope that it will be useful, %
%   but WITHOUT ANY WARRANTY; without even the implied warranty of             %
%   MERCHANTABILITY or FITNESS FOR A PARTICULAR PURPOSE.  See the              %
%   GNU General Public License for more details.                               %
%                                                                              %
%   You should have received a copy of the GNU General Public License along    %
%   with Mathematics-and-Physics.  If not, see <https://www.gnu.org/licenses/>.%
%------------------------------------------------------------------------------%
\documentclass{article}
\usepackage{graphicx}                           % Needed for figures.
\usepackage{amsmath}                            % Needed for align.
\usepackage{amssymb}                            % Needed for mathbb.
\usepackage{amsthm}                             % For the theorem environment.
\usepackage{float}

\newtheoremstyle{normal}
    {\topsep}               % Amount of space above the theorem.
    {\topsep}               % Amount of space below the theorem.
    {}                      % Font used for body of theorem.
    {}                      % Measure of space to indent.
    {\bfseries}             % Font of the header of the theorem.
    {}                      % Punctuation between head and body.
    {.5em}                  % Space after theorem head.
    {}

\theoremstyle{normal}
\newtheorem{definition}{Definition}
\newtheorem{notation}{Notation}
\newtheorem{example}{Example}

\theoremstyle{plain}
\newtheorem{theorem}{Theorem}
\newcommand{\ceil}[2][]{#1\lceil#2#1\rceil}

\title{Cavalieri's Principle}
\author{Math 3}
\date{Fall 2021}

% No indent and no paragraph skip.
\setlength{\parindent}{0em}
\setlength{\parskip}{0em}

\begin{document}
    \maketitle
    Cavalieri's principle states the following:
    \begin{center}
        If two plane regions are bounded between two lines, and if for every
        horizontal line the intersection of this line with the first plane
        region has the same length as the intersection of this line with the
        second plane region, then both regions have the same area.
    \end{center}
    \begin{figure}[H]
        \centering
        \includegraphics{../../images/cavalieri_principle.pdf}
        \caption{Visual for Cavalieri's Principle}
    \end{figure}
    \begin{enumerate}
        \item
        Suppose we have a stack of coins where every coin is directly on top of
        the previous one. If we perturb the coins a bit so they do not sit
        evenly on top of each other, does the total volume of all of the coins
        change?
        \item
        Let's prove Cavalieri's principle with calculus. If
        $f,g:[a,b]\rightarrow\mathbb{R}$ are continuous functions, how can we
        express the \textit{area between} $f$ \textit{and} $g$?
        \item
        If $h:[a,b]\rightarrow\mathbb{R}$ is continuous, and if
        $F,G:[a,b]\rightarrow\mathbb{R}$ are defined by
        $F(x)=f(x)+h(x)$ and $G(x)=g(x)+h(x)$, what is the area between $F$
        and $G$?
        \item
        Prove your formula using the rules of integration.
    \end{enumerate}
    \begin{figure}[H]
        \centering
        \includegraphics{../../images/cavalieri_principle_continuous_function.pdf}
        \caption{Visual for the Area Between Curves}
    \end{figure}
    I, the copyright holder of this work, release it into the public domain.
    This applies worldwide. In some countries this may not be legally possible;
    if so: I grant anyone the right to use this work for any purpose, without
    any conditions, unless such conditions are required by law.
    \par\hfill\par
    The source code used to generate this document is free software and released
    under version 3 of the GNU General Public License.
\end{document}