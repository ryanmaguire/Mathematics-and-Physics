%-----------------------------------LICENSE------------------------------------%
%   This file is part of Mathematics-and-Physics.                              %
%                                                                              %
%   Mathematics-and-Physics is free software: you can redistribute it and/or   %
%   modify it under the terms of the GNU General Public License as             %
%   published by the Free Software Foundation, either version 3 of the         %
%   License, or (at your option) any later version.                            %
%                                                                              %
%   Mathematics-and-Physics is distributed in the hope that it will be useful, %
%   but WITHOUT ANY WARRANTY; without even the implied warranty of             %
%   MERCHANTABILITY or FITNESS FOR A PARTICULAR PURPOSE.  See the              %
%   GNU General Public License for more details.                               %
%                                                                              %
%   You should have received a copy of the GNU General Public License along    %
%   with Mathematics-and-Physics.  If not, see <https://www.gnu.org/licenses/>.%
%------------------------------------------------------------------------------%
\documentclass{article}
\usepackage{graphicx}                           % Needed for figures.
\usepackage{amsmath}                            % Needed for align.
\usepackage{amssymb}                            % Needed for mathbb.

\title{Escape Velocity}
\author{Math 3}
\date{Fall 2021}

% No indent and no paragraph skip.
\setlength{\parindent}{0em}
\setlength{\parskip}{0em}

\begin{document}
    \maketitle
    \begin{enumerate}
        \item   We want $v_{1}=0$, the speed at infinity. Since the potential is
                $U=-GMm/r$, the limit at infinity for this is zero. So,
                $U_{1}+V_{1}=0$ at infinity. By conservation of energy,
                $U_{0}+V_{0}=0$, so $GMm/r=\frac{1}{2}mv_{0}^{2}$. Cancel little
                $m$, we get $v_{0}^{2}=2GM/r$. So:
                \begin{equation}
                  v_{0}=\sqrt{\frac{2GM}{R}}
                \end{equation}
        \item   The limit of $U$ is still zero at infinity, so we'd have:
                \begin{equation}
                    v_{1}^{2}=v_{0}^{2}-\frac{2GM}{R}
                \end{equation}
                Taking square roots of both sides gives us the speed at
                infinity:
                \begin{equation}
                    v_{1}=\sqrt{v_{0}^{2}-\frac{2GM}{R}}
                \end{equation}
        \item   Just plug in the values for $\sqrt{2GM/R}$. You get
                11,186 meters per second, or about 25,000 miles per hour.
        \item   This is the definition of a black hole.
        \item   Now solve for $r$ instead. We have:
                \begin{equation}
                    c=\sqrt{\frac{2GM}{r}}
                \end{equation}
                Where $c$ is the speed of light, $G$ and $M$ are the same. The
                radius Earth would need to be is:
                \begin{equation}
                    r=\frac{2GM}{c^{2}}
                \end{equation}
                Plugging in the numbers, we get about a centimeter. So, smaller
                than a pea.
    \end{enumerate}
    \newpage
    I, the copyright holder of this work, release it into the public domain.
    This applies worldwide. In some countries this may not be legally possible;
    if so: I grant anyone the right to use this work for any purpose, without
    any conditions, unless such conditions are required by law.
    \par\hfill\par
    The source code used to generate this document is free software and released
    under version 3 of the GNU General Public License.
\end{document}
