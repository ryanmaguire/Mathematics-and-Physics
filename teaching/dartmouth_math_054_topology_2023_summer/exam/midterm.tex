%-----------------------------------LICENSE------------------------------------%
%   This file is part of Mathematics-and-Physics.                              %
%                                                                              %
%   Mathematics-and-Physics is free software: you can redistribute it and/or   %
%   modify it under the terms of the GNU General Public License as             %
%   published by the Free Software Foundation, either version 3 of the         %
%   License, or (at your option) any later version.                            %
%                                                                              %
%   Mathematics-and-Physics is distributed in the hope that it will be useful, %
%   but WITHOUT ANY WARRANTY; without even the implied warranty of             %
%   MERCHANTABILITY or FITNESS FOR A PARTICULAR PURPOSE.  See the              %
%   GNU General Public License for more details.                               %
%                                                                              %
%   You should have received a copy of the GNU General Public License along    %
%   with Mathematics-and-Physics.  If not, see <https://www.gnu.org/licenses/>.%
%------------------------------------------------------------------------------%
\documentclass{article}
\usepackage{graphicx}                           % Needed for figures.
\usepackage{amsmath}                            % Needed for align.
\usepackage{amssymb}                            % Needed for mathbb.
\usepackage{amsthm}                             % For the theorem environment.

%------------------------Theorem Styles-------------------------%

% Define theorem style for default spacing and normal font.
\newtheoremstyle{normal}
    {\topsep}               % Amount of space above the theorem.
    {\topsep}               % Amount of space below the theorem.
    {}                      % Font used for body of theorem.
    {}                      % Measure of space to indent.
    {\bfseries}             % Font of the header of the theorem.
    {}                      % Punctuation between head and body.
    {.5em}                  % Space after theorem head.
    {}

% Define default environments.
\theoremstyle{normal}
\newtheorem{problem}{Problem}

\title{Point-Set Topology: Midterm}
\date{Summer 2023}

% No indent and no paragraph skip.
\setlength{\parindent}{0em}
\setlength{\parskip}{0em}

\begin{document}
    \maketitle
    \begin{problem}[\textbf{Set Theory and Logic}]
        \par\hfill\par
        A Boolean algebra is an algebraic system consisting of a set $A$
        with two operations $\star$ and $\circ$. It models the logical
        operators $\land$ and $\lor$, as well as set arithmetic with
        $\cap$ and $\cup$. One of the properties of a Boolean algebra is the
        \textit{absorption laws}:
        \begin{align}
            a\star(a\circ{b})&=a\\
            a\circ(a\star{b})&=a
        \end{align}
        Of equal importance are the \textit{idempotence laws}:
        \begin{align}
            a\star{a}&=a\\
            a\circ{a}&=a
        \end{align}
        Lastly there must be neutral elements
        $e_{\star}$ and $e_{\circ}$ with the property that:
        \begin{align}
            a\star{e}_{\star}&=a\\
            a\circ{e}_{\circ}&=a
        \end{align}
        Let $X$ be a set. Set $A=\mathcal{P}(X)$, $\star=\cap$, and
        $\circ=\cup$.
        \begin{itemize}
            \item (4 Points) Prove that the absorption laws are satisfied.
            \item (3 Points) Prove that the idempotent laws are satisfied.
            \item (3 Points) Prove that there are neutral elements
                $e_{\star}$ and $e_{\circ}$ in $A$. What are they?
        \end{itemize}
    \end{problem}
    \begin{problem}[\textbf{Metric Spaces}]
        \par\hfill\par
        A pseudo-metric on a set $X$ is a function
        $\rho:X\times{X}\rightarrow\mathbb{R}$ such that:
        \begin{align}
            \rho(x,\,y)&\geq{0}\tag{Positivity}\\
            \rho(x,\,x)&=0\tag{Definite}\\
            \rho(x,\,y)&=\rho(y,\,x)\tag{Symmetry}\\
            \rho(x,\,z)&\leq\rho(x,\,y)+\rho(y,\,z)
                \tag{Triangle-Inequality}
        \end{align}
        Note $\rho(x,\,y)=0$ need not imply $x=y$. Open balls can be defined
        just like metric spaces:
        \begin{equation}
            B_{r}^{(X,\,\rho)}(x)=\{\,y\in{X}\;|\;\rho(x,\,y)<r\,\}
        \end{equation}
        And openness can be defined just like metric spaces.
        \begin{itemize}
            \item (5 Points) Let $\tau_{\rho}$ be the set of all
                pseudo-metrically open subsets of $X$. Prove $\tau_{\rho}$ is
                a topology on $X$.
            \item (5 Points) Prove that if $(X,\,\tau_{\rho})$ is Hausdorff,
                then $\rho$ is actually a metric, not just a pseudo-metric.
        \end{itemize}
    \end{problem}
    \begin{problem}[\textbf{Compactness}]
        \par\hfill\par
        For a metric space, compactness means every sequence has a convergent
        subsequence. It also means every open cover has a finite subcover, or
        that the space is complete and totally bounded. You may use any of
        these.
        \par\hfill\par
        For a topological space compact means every open cover has a finite
        subcover. Sequences are not enough to describe general compactness, and
        topological spaces do not have a notion of completeness or being
        totally bounded. So only open covers work in the more general setting.
        \begin{itemize}
            \item (5 Points) Let $(X,\,d)$ be a compact metric space. Prove
                that every infinite subset $A\subseteq{X}$ has an accumulation
                point. That is, a point $x\in{X}$ such that for all
                $\varepsilon>0$ there is a $y\in{A}$, $y\ne{x}$, such that
                $d(x,\,y)<\varepsilon$.
            \item (5 Points) Let $(X,\,\leq)$ be a totally ordered set, and let
                $\tau_{<}$ be the order topology induced by the total order.
                That is, $\tau_{<}$ is the topology generated by open intervals,
                left-rays, and right-rays. Prove that if $(X,\,\tau_{<})$ is
                compact (every open cover has a finite subcover), then
                $(X,\,\leq)$ has a supremum (a greatest element).
        \end{itemize}
    \end{problem}
    \begin{problem}[\textbf{Quotient Spaces}]
        \par\hfill\par
        Let $(X,\,\tau)$ be a topological space, and $R$ an equivalence
        relation on $X$.
        \begin{itemize}
            \item (5 Points) Prove that $(X/R,\,\tau_{X/R})$ is Fr\'{e}chet
                if and only if for all $x\in{X}$ the equivalence class
                $[x]\subseteq{X}$ is a closed subset.
            \item (5 Points) Show by example that $(X/R,\,\tau_{X/R})$ need not
                be a Fr\'{e}chet space, even if $(X,\,\tau)$ is.
        \end{itemize}
    \end{problem}
\end{document}
