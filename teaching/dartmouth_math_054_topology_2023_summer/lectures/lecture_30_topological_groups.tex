%-----------------------------------LICENSE------------------------------------%
%   This file is part of Mathematics-and-Physics.                              %
%                                                                              %
%   Mathematics-and-Physics is free software: you can redistribute it and/or   %
%   modify it under the terms of the GNU General Public License as             %
%   published by the Free Software Foundation, either version 3 of the         %
%   License, or (at your option) any later version.                            %
%                                                                              %
%   Mathematics-and-Physics is distributed in the hope that it will be useful, %
%   but WITHOUT ANY WARRANTY; without even the implied warranty of             %
%   MERCHANTABILITY or FITNESS FOR A PARTICULAR PURPOSE.  See the              %
%   GNU General Public License for more details.                               %
%                                                                              %
%   You should have received a copy of the GNU General Public License along    %
%   with Mathematics-and-Physics.  If not, see <https://www.gnu.org/licenses/>.%
%------------------------------------------------------------------------------%
\documentclass{article}
\usepackage{amsmath}                            % Needed for align.
\usepackage{amssymb}                            % Needed for mathbb.
\usepackage{amsthm}                             % For the theorem environment.

%------------------------Theorem Styles-------------------------%
\theoremstyle{plain}
\newtheorem{theorem}{Theorem}[section]

% Define theorem style for default spacing and normal font.
\newtheoremstyle{normal}
    {\topsep}               % Amount of space above the theorem.
    {\topsep}               % Amount of space below the theorem.
    {}                      % Font used for body of theorem.
    {}                      % Measure of space to indent.
    {\bfseries}             % Font of the header of the theorem.
    {}                      % Punctuation between head and body.
    {.5em}                  % Space after theorem head.
    {}

% Define default environments.
\theoremstyle{normal}
\newtheorem{examplex}{Example}[section]
\newtheorem{definitionx}{Definition}[section]

\newenvironment{example}{%
    \pushQED{\qed}\renewcommand{\qedsymbol}{$\blacksquare$}\examplex%
}{%
    \popQED\endexamplex%
}

\newenvironment{definition}{%
    \pushQED{\qed}\renewcommand{\qedsymbol}{$\blacksquare$}\definitionx%
}{%
    \popQED\enddefinitionx%
}

\title{Point-Set Topology: Lecture 30}
\author{Ryan Maguire}
\date{\today}

% No indent and no paragraph skip.
\setlength{\parindent}{0em}
\setlength{\parskip}{0em}

\begin{document}
    \maketitle
    \section{Topological Groups}
        Now we introduce some topology into our algebra.
        \begin{definition}[\textbf{Topological Group}]
            A topological group is an ordered triple $(X,\,\tau,\,*)$ where
            $(X,\,\tau)$ is a topological space and $(X,\,*)$ is a group
            such that the functions $m:X\times{X}\rightarrow{X}$ and
            $\eta:X\rightarrow{X}$ defined by:
            \begin{align}
                m(x,\,y)&=x*y\\
                \eta(x)&=x^{-1}
            \end{align}
            are continuous (here $X\times{X}$ is given the product topology).
            That is, the group operations are continuous functions.
        \end{definition}
        \begin{example}
            The real line with addition is a topological group. The addition
            of real numbers is indeed a continuous operation, and the inverse
            operation is negation: $x\mapsto{-x}$.
        \end{example}
        \begin{example}
            More generally, $\mathbb{R}^{n}$ as a vector space with vector
            addition becomes a topological group when endowed with the standard
            Euclidean topology.
        \end{example}
        \begin{example}
            The circle $\mathbb{S}^{1}$ with the subspace topology and the
            \textit{rotation operation} is a topological group. That is,
            Given points $e^{i\theta},\,e^{i\phi}\in\mathbb{S}^{1}$, we define
            $e^{i\theta}*e^{i\phi}=e^{i(\theta+\phi)}$. This operation is
            continuous with respect to the subspace topology the circle
            inherits from $\mathbb{R}^{2}$, giving us a topological group.
        \end{example}
        \begin{example}
            The integers $\mathbb{Z}$ with the subspace topology from
            $\mathbb{R}$ and addition form a topological group. Note that the
            subspace topology on $\mathbb{Z}$ is also the discrete topology.
        \end{example}
        \begin{example}
            More generally, if $(X,\,*)$ is any group, then
            $(X,\,\mathcal{P}(X),\,*)$ is a topological group. The product
            topology on $X\times{X}$ is also the discrete topology, and hence
            \textit{any} function $f:X\times{X}\rightarrow{X}$ is continuous.
            Similarly, any function $g:X\rightarrow{X}$ is continuous. So
            in particular the multiplication and inversion operations are
            continuous and $(X,\,\mathcal{P}(X),\,*)$ is a topological group.
        \end{example}
        \begin{example}
            Going the other way, if $(X,\,*)$ is any group, and if
            $\tau=\{\,\emptyset,\,X\,\}$ is the indiscrete topology, then
            $(X,\,\tau,\,*)$ is a topological group. Since $\tau$ is the
            indiscrete topology, any function into $X$ is continuous,
            so in particular multiplication and inversion are continuous.
        \end{example}
        \begin{example}
            If $X=\mathbb{R}$ and $\tau=\{\,\emptyset,\,\mathbb{R}\,\}$ is the
            indiscrete topology, then by the previous example
            $(\mathbb{R},\,\tau,\,+)$ is a topological group. Note that it is
            a non-Hausdorff topological group. Topological groups need not
            satisfy any of the separation properties.
        \end{example}
        \begin{theorem}
            If $(X,\,\tau)$ is a topological space, and if $(X,\,*)$ is a
            group, then $(X,\,\tau,\,*)$ is a topological group if and only
            if the function $f:X\times{X}\rightarrow{X}$ defined by
            $f(x,\,y)=x*y^{-1}$ is continuous.
        \end{theorem}
        \begin{proof}
            The function $f(x,\,y)=x*y^{-1}$ can be seen as a
            combination of multiplication and inversion. If $(X,\,\tau,\,*)$ is
            a topological group, then this function is continuous. In the
            other direction, if this function is continuous, then setting
            $x=e$, the identity, we have $f(e,\,y)=e*y^{-1}=y^{-1}$, and this
            is a continuous function of $y$, meaning inversion is continuous.
            But then $x*y=x*(y^{-1})^{-1}=f(x,\,y^{-1})$, which is
            the composition of continuous functions, so multiplication is
            continuous. Hence $(X,\,\tau,\,*)$ is a topological group.
        \end{proof}
        \begin{theorem}
            If $(X,\,\tau,\,*)$ is a topological group, and if $a\in{X}$,
            and if $L_{a}:X\rightarrow{X}$ is left-translation of $X$ by $a$,
            then $L_{a}$ is a homeomorphism.
        \end{theorem}
        \begin{proof}
            We have already proved that left-translation in a group is
            bijective. Let us show that it is continuous. But
            $L_{a}(x)=a*x$ is the restriction of
            $m:X\times{X}\rightarrow{X}$, defined by $m(x,\,y)=x*y$, to the
            subset $\{\,a\,\}\times{X}$. But the restriction of a continuous
            function to subspace is continuous, and hence
            $L_{a}:X\rightarrow{X}$ is continuous. The inverse function
            is given by $L_{a}^{-1}=L_{a^{-1}}$ since:
            \begin{align}
                (L_{a}\circ{L}_{a^{-1}})(x)
                &=L_{a}\big(L_{a^{-1}}(x)\big)\tag{Definition of Composition}\\
                &=L_{a}(a^{-1}*x)\tag{Definition of $L_{a^{-1}}$}\\
                &=a*(a^{-1}*x)\tag{Definition of $L_{a}$}\\
                &=(a*a^{-1})*x\tag{Associativity}\\
                &=e*x\tag{Inverse}\\
                &=x\tag{Identity}
            \end{align}
            And hence $L_{a}\circ{L}_{a^{-1}}$ is the identity function.
            Similarly, $L_{a^{-1}}\circ{L}_{a}$ is the identity. So the
            inverse of left-translation is another left-translation, which is
            continuous. Hence $L_{a}$ is a homeomorphism.
        \end{proof}
        Two immediate results are often of equal use.
        \begin{theorem}
            If $(X,\,\tau,\,*)$ is a topological group, if $a\in{X}$, and if
            $L_{a}:X\rightarrow{X}$ is left-translation by $a$, then
            $L_{a}$ is an open map.
        \end{theorem}
        \begin{proof}
            Left-translation is a homeomorphism, so it is also an open map.
        \end{proof}
        \begin{theorem}
            If $(X,\,\tau,\,*)$ is a topological group, if $a\in{X}$, and if
            $L_{a}:X\rightarrow{X}$ is left-translation by $a$, then
            $L_{a}$ is an closed map.
        \end{theorem}
        \begin{proof}
            Left-translation is a homeomorphism, so it is also an closed map.
        \end{proof}
        \begin{theorem}
            If $(X,\,\tau,\,*)$ is a topological group, if $a\in{X}$, and if
            $R_{a}:X\rightarrow{X}$ denotes right-translation by $a$, then
            $R_{a}$ is a homeomorphism. In particular it is both an open map
            and a closed map.
        \end{theorem}
        \begin{proof}
            The proof is similar to that for left-translation.
        \end{proof}
        Conjugation is also a homeomorphism, but it is also a group isomorphism.
        Functions that are both continuous and homomorphisms are one of the
        main objects of study in topological groups.
        \begin{definition}[\textbf{Topological Group Homomorphism}]
            A topological group homomorphism from a topological group
            $(X,\,\tau_{X},\,*_{X})$ to a topological group
            $(Y,\,\tau_{Y},\,*_{Y})$ is a function $\varphi:X\rightarrow{Y}$
            such that $\varphi$ is continuous with respect to the topologies
            and also a group homomorphism with respect to the group operations.
        \end{definition}
        For groups, a bijective homomorphism automatically yields a group
        isomorphism since the inverse function will be a group homomorphism.
        Continuity lacks such niceties. We must be careful in our defining of
        topological group isomorphisms.
        \begin{definition}[\textbf{Topological Group Isomorphism}]
            A topological group isomorphism from a topological group
            $(X,\,\tau_{X},\,*_{X})$ to another topological group
            $(Y,\,\tau_{Y},\,*_{Y})$ is a function $\varphi:X\rightarrow{Y}$
            such that $\varphi$ is a homeomorphism with respect to the
            topologies and also a group isomorphism with respect to
            the group operations.
        \end{definition}
        Note we did \textbf{not} just define this as a
        \textit{continuous group isomorphism} since we would like the
        inverse function to be continuous as well. Hence we defined this as a
        \textit{homeomorphic group isomorphism}.
        \begin{theorem}
            If $(X,\,\tau,\,*)$ is a topological group and if $g\in{X}$, then
            $\textrm{conj}_{g}$ is a topological group
            isomorphism, where $\textrm{conj}_{g}:X\rightarrow{X}$ is the
            conjugation function.
        \end{theorem}
        \begin{proof}
            We have already proven that conjugation is a group isomorphism.
            It is also a homeomorphism since:
            \begin{align}
                \textrm{conj}_{g}(a)
                &=g*a*g^{-1}\tag{Definition of $\textrm{conj}_{g}$}\\
                &=g*(a*g^{-1})\tag{Associativity}\\
                &=L_{g}(a*g^{-1})\tag{Definition of $L_{g}$}\\
                &=L_{g}\big(R_{g^{-1}}(a)\big)\tag{Definition of $R_{g^{-1}}$}\\
                &=(L_{g}\circ{R}_{g^{-1}})(a)\tag{Definition of Composition}
            \end{align}
            And hence conjugation is the composition of a left-translation and
            a right-translation, which is the composition of homeomorphisms,
            which is therefore a homeomorphism.
        \end{proof}
        Left-translations, right-translations, and conjugations allow us to
        show that topological groups have a lot of nice properties. For
        one thing, topological groups are \textit{homogeneous}.
        \begin{definition}[\textbf{Homogeneous Topological Space}]
            A homogeneous topological space is a topological space $(X,\,\tau)$
            such that for all $x,y\in{X}$ there is a homeomorphism
            $f:X\rightarrow{X}$ such that $f(x)=y$.
        \end{definition}
        This means that every point in the space looks like every other point.
        Euclidean spaces are examples of homogeneous topological spaces, as
        are all connected topological manifolds. Topological groups are also
        homogeneous.
        \begin{theorem}
            If $(X,\,\tau,\,*)$ is a topological group, then $(X,\,\tau)$ is a
            homogeneous topological space.
        \end{theorem}
        \begin{proof}
            For let $a,b\in{X}$ and define
            $f:X\rightarrow{X}$ by $f(x)=b*x*a^{-1}$. Then $f$ is a
            homeomorphism, being the composition of left and right-translations.
            But also:
            \begin{equation}
                f(a)=b*a*a^{-1}=b*e=b
            \end{equation}
            And hence $(X,\,\tau)$ is a homogeneous topological space.
        \end{proof}
        Homogeneity can be used to prove a lot of nice properties. For one,
        if there are nice \textit{local} topological properties around the
        origin, then these properties might become \textit{global} by using
        translations. Let's motivate this by example.
        \begin{definition}[\textbf{Kolmogorov Topological Space}]
            A Kolmogorov topological space is a topological space
            $(X,\,\tau)$ such that for all $x,y\in{X}$ there is an open set
            $\mathcal{U}\in\tau$ such that either $x\in\mathcal{U}$ and
            $y\notin\mathcal{U}$, or $x\notin\mathcal{U}$ and $y\in\mathcal{U}$.
        \end{definition}
        This is the weakest of the separation properties, weaker than the
        Hausdorff and Frech\'{e}t properties. A Kolmogorov space need not be
        Hausdorff or Frech\'{e}t.
        \begin{example}
            Define $\tau$ on $\mathbb{N}$ to be:
            \begin{equation}
                \tau=\{\,
                    \mathbb{Z}_{n}\subseteq\mathbb{Z}\;|\;n\in\mathbb{N}\,
                \}\cup\{\,\mathbb{N}\,\}
            \end{equation}
            This is a topology since the sets are nested, so the union and
            intersection properties are satisfied. It is Kolmogorov. Given
            $n,m\in\mathbb{N}$, $n\ne{m}$, choose
            $\mathcal{U}=\mathbb{Z}_{k+1}$ where $k=\textrm{min}(m,\,n)$. Then
            $\mathcal{U}$ is open can contains one one of $m$ and $n$, but not
            both. The space is not Frech\'{e}t since no point can be separated
            from $0$.
        \end{example}
        What's remarkable is that Kolmorogov topological groups are Frech\'{e}t,
        Hausdorff, regular, and completely regular. We'll prove the first two
        of the assertions. First we'll need a little lemma.
        \begin{theorem}
            If $(G,\,*)$ is a group, if $A\subseteq{G}$, if $x,y\in{G}$, if
            $x\in{A}$ and $y\notin{A}$, and if
            $B=L_{x}\big[R_{y}[A^{-1}]\big]$, where:
            \begin{equation}
                A^{-1}=\{\,a^{-1}\in{G}\;|\;a\in{A}\,\}
            \end{equation}
            then $x\notin{B}$ and $y\in{B}$.
        \end{theorem}
        \begin{proof}
            First, $y\in{B}$ since $x\in{A}$, and hence
            $x^{-1}\in{A}^{-1}$, so $L_{x}\big(R_{y}(x^{-1})\big)\in{B}$.
            But:
            \begin{equation}
                L_{x}\big(R_{y}(x^{-1})\big)=x*x^{-1}*y=e*y=y
            \end{equation}
            and therefore $y\in{B}$. Second, $x\notin{B}$. For if
            $x\in{B}$, then $x=x*a^{-1}*y$ for some $a\in{A}$. But then by the
            cancellation law, $a^{-1}*y=e$. But inverses are unique, meaning
            $y=a$. But then $y\in{A}$, which is a contradiction. So
            $x\notin{B}$ and $y\in{B}$, completing the proof.
        \end{proof}
        \begin{theorem}
            If $(X,\,\tau,\,*)$ is a topological group, and if
            $\eta:X\rightarrow{X}$ is defined by $\eta(x)=x^{-1}$, then
            $\eta$ is a homeomorphism.
        \end{theorem}
        \begin{proof}
            $\eta$ is continuous since $(X,\,\tau,\,*)$ is a topological
            group. It is also bijective. It is injective since
            $x^{-1}=y^{-1}$ implies $x=y$ since inverses are unique. It is
            surjective since $(X,\,*)$ is a group, and hence every element has
            an inverse. Lastly, the inverse is continuous since the inverse of
            $\eta$ is $\eta$. That is:
            \begin{align}
                \eta^{2}(x)
                &=(\eta\circ\eta)(x)\\
                &=\eta\big(\eta(x)\big)\\
                &=\eta(x^{-1})\\
                &=(x^{-1})^{-1}\\
                &=x
            \end{align}
            And hence $\eta=\eta^{-1}$. But then $\eta^{-1}$ is continuous,
            so $\eta$ is a homeomorphism.
        \end{proof}
        \begin{theorem}
            If $(X,\,\tau,\,*)$ is a topological group, and if
            $(X,\,\tau)$ is a Kolmogorov topological space, then it is a
            Frech\'{e}t topological space.
        \end{theorem}
        \begin{proof}
            For let $x,y\in{X}$. Since $(X,\,\tau)$ is a Kolmogorov space, there
            is a $\mathcal{U}\in\tau$ such that either $x\in\mathcal{U}$ and
            $y\notin\mathcal{U}$, or $x\notin\mathcal{U}$ and
            $y\in\mathcal{U}$. Suppose the former (the proof is symmetric).
            Let $\mathcal{V}=(L_{x}\circ{R}_{y})[\mathcal{U}^{-1}]$ where:
            \begin{equation}
                \mathcal{U}^{-1}=\{\,a^{-1}\in{X}\;|\;a\in\mathcal{U}\,\}
            \end{equation}
            Then $\mathcal{U}^{-1}$ is the image of the inversion function
            $\eta:X\rightarrow{X}$ defined by $\eta(x)=x^{-1}$. Since
            $\eta$ is a homeomorphism, it is an open mapping. Since
            $\mathcal{U}$ is open, $\mathcal{U}^{-1}$ is also open. Hence
            $\mathcal{V}$ is also open since left-translations and
            right-translations are open mappings as well. By a previous theorem,
            since $x\in\mathcal{U}$ and $y\notin\mathcal{U}$, we have
            $x\notin\mathcal{V}$ and $y\in\mathcal{V}$. Hence
            $(X,\,\tau)$ is a Frech\'{e}t space.
        \end{proof}
        The Hausdorff property will be implied as well.
        \begin{theorem}
            If $(X,\,\tau)$ is a topological space, then it is Hausdorff if and
            only if the set:
            \begin{equation}
                \Delta=\{\,(x,\,x)\in{X}\times{X}\;|\;x\in{X}\,\}
            \end{equation}
            is closed with respect to the product topology.
        \end{theorem}
        \begin{proof}
            First, suppose $\Delta$ is closed with respect to the product
            topology. Let's show that $(X,\,\tau)$ is Hausdorff. Let
            $x,y\in{X}$ with $x\ne{y}$. Since $x\ne{y}$ we have that
            $(x,\,y)\notin\Delta$, meaning
            $(x,\,y)\in{X}\times{X}\setminus\Delta$. But since
            $\Delta\subseteq{X}\times{X}$ is closed,
            $X\times{X}\setminus\Delta$ is open. Then from the definition
            of the product topology, there must be open sets
            $\mathcal{U},\,\mathcal{V}\in\tau$ such that
            $\mathcal{U}\times\mathcal{V}\subseteq{X}\times{X}\setminus\Delta$
            and $(x,\,y)\in\mathcal{U}\times\mathcal{V}$. But also
            $\mathcal{U}\cap\mathcal{V}=\emptyset$. For if
            $z\in\mathcal{U}\cap\mathcal{V}$, then
            $(z,\,z)\in{X}\times{X}\setminus\Delta$, which is a contradiction
            since $(z,\,z)\in\Delta$. Hence $(X,\,\tau)$ is Hausdorff.
            In the other direction, suppose $(X,\,\tau)$ is a Hausdorff space
            and let's show that $\Delta$ is closed. To do this we prove that
            $X\times{X}\setminus\Delta$ is open. Let
            $(x,\,y)\in{X}\times{X}\setminus\Delta$. Then $x\ne{y}$. But
            $(X,\,\tau)$ is Hausdorff so there are open sets
            $\mathcal{U},\mathcal{V}\in\tau$ such that
            $x\in\mathcal{U}$, $y\in\mathcal{V}$, and
            $\mathcal{U}\cap\mathcal{V}=\emptyset$. But then
            $\mathcal{U}\times\mathcal{V}$ is an open subset of
            $X\times{X}$ that contains $(x,\,y)$, and since the sets are
            disjoint the product is contained entirely inside of
            $X\times{X}\setminus\Delta$. Hence, $\Delta$ is closed.
        \end{proof}
        \begin{theorem}
            If $(X,\,\tau,\,*)$ is a topological group, and if $(X,\,\tau)$
            is a Kolmogorov space, then it is Hausdorff.
        \end{theorem}
        \begin{proof}
            We have proven that Kolmogorov topological groups are Frech\'{e}t,
            and hence $\{\,e\,\}$ is closed, where $e$ is the unique identity
            element of $(X,\,*)$. But since $(X,\,\tau,\,*)$ is a
            topological group, the function $f:X\times{X}\rightarrow{X}$
            defined by $f(x,\,y)=x*y^{-1}$ is continuous. But the set:
            \begin{equation}
                \Delta=\{\,(x,\,x)\in{X}\times{X}\;|\;x\in{X}\,\}
            \end{equation}
            is the pre-image of the set $\{\,e\,\}$ by $f$. Since $f$ is
            continuous and $\{\,e\,\}$ is closed, $\Delta$ is closed as well.
            Since the diagonal is closed, $(X,\,\tau)$ is Hausdorff.
        \end{proof}
        The idea around Kolmogorov spaces allow us to define the notion of
        points being \textit{topologically distinguishable}. Two points are
        topologically indistinguishable if they belong to all of the same open
        sets. That is, the topology can't tell them apart. This yields an
        equivalence relation on the points in a topological space, and the
        resulting quotient is the \textit{Kolmogorov quotient}. It always yields
        a Kolmogorov topological space. What's more, a topological space is a
        Kolmogorov space if and only if it is homeomorphic to it's
        Kolmogorov quotient.
        \par\hfill\par
        In the context of groups there's a bit more to it. If $(X,\,\tau,\,*)$
        is a topological group, and if $e\in{X}$ is the unique identity, then
        $\textrm{Cl}_{\tau}(\{\,e\,\})$ is a closed normal subgroup. That is,
        it is a closed subset as far as the topology is concerned, but it is
        also a normal subgroup. The points in the closure of $e$ are precisely
        the ones that are topologically indistinguishable from the identity.
        The equivalence relation induced by being
        topologically indistinguishable
        partitions the space into the cosets of $\textrm{Cl}_{\tau}(\{\,e\,\})$
        (viewed as a normal subgroup). The quotient space is hence also a
        topological group, and since it is Kolmogorov, it is automatically
        Frech\'{e}t and Hausdorff. So every topological group has a canonical
        Hausdorff topological group associated to it. Because of this many
        authors require topological groups to have, as part of the definition,
        the Kolmogorov or Frech\'{e}t or Hausdorff properties.
\end{document}
