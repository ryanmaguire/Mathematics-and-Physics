%-----------------------------------LICENSE------------------------------------%
%   This file is part of Mathematics-and-Physics.                              %
%                                                                              %
%   Mathematics-and-Physics is free software: you can redistribute it and/or   %
%   modify it under the terms of the GNU General Public License as             %
%   published by the Free Software Foundation, either version 3 of the         %
%   License, or (at your option) any later version.                            %
%                                                                              %
%   Mathematics-and-Physics is distributed in the hope that it will be useful, %
%   but WITHOUT ANY WARRANTY; without even the implied warranty of             %
%   MERCHANTABILITY or FITNESS FOR A PARTICULAR PURPOSE.  See the              %
%   GNU General Public License for more details.                               %
%                                                                              %
%   You should have received a copy of the GNU General Public License along    %
%   with Mathematics-and-Physics.  If not, see <https://www.gnu.org/licenses/>.%
%------------------------------------------------------------------------------%
\documentclass{article}
\usepackage{graphicx}                           % Needed for figures.
\usepackage{amsmath}                            % Needed for align.
\usepackage{amssymb}                            % Needed for mathbb.
\usepackage{amsthm}                             % For the theorem environment.

%------------------------Theorem Styles-------------------------%

% Define theorem style for default spacing and normal font.
\newtheoremstyle{normal}
    {\topsep}               % Amount of space above the theorem.
    {\topsep}               % Amount of space below the theorem.
    {}                      % Font used for body of theorem.
    {}                      % Measure of space to indent.
    {\bfseries}             % Font of the header of the theorem.
    {}                      % Punctuation between head and body.
    {.5em}                  % Space after theorem head.
    {}

% Define default environments.
\theoremstyle{normal}
\newtheorem{problem}{Problem}

\title{Point-Set Topology: Homework 3}
\date{Summer 2023}

% No indent and no paragraph skip.
\setlength{\parindent}{0em}
\setlength{\parskip}{0em}

\begin{document}
    \maketitle
    \begin{problem}
        \textbf{(Separability)}
        \par\hfill\par
        A separable topological space is a space $(X,\,\tau)$ such that there
        is a countable subset $A\subseteq{X}$ such that
        $\textrm{Cl}_{\tau}(A)=X$.
        A metric space is separable if and only if it is second-countable. This
        feature is special to metric spaces. Take $\mathbb{R}$ with the
        standard topology, and equip $\mathbb{R}/\mathbb{Z}$ with the
        quotient topology. Intuitively this is infinite many circles all
        touching at $0$. It is not first-countable, and hence not
        second-countable, even though $\mathbb{R}$ is. It is still separable.
        \begin{itemize}
            \item (6 Points) Let $(X,\,\tau)$ be a separable topological space.
                Let $R$ be any equivalence relation on $X$. Prove that
                $(X/R,\,\tau_{X/R})$ is separable. That is, separability is
                a topological property preserved by quotients.
        \end{itemize}
    \end{problem}
    \begin{problem}
        \textbf{(Embeddings)}
        \par\hfill\par
        The bug-eyed line is a quotient space of
        $X=\mathbb{R}\times\{\,0,\,1\,\}$ where $\mathbb{R}$ has the standard
        Euclidean topology and $\{\,0,\,1\,\}$ has the discrete topology. $X$
        is given the product topology. We identity
        $(x,\,0)$ with $(x,\,1)$ for all $x\ne{0}$ and then take the quotient
        of $X$ under this relation. This idea is shown in
        Fig.~\ref{fig:bug_eyed_line}
        \begin{figure}
            \centering
            \includegraphics{../../../images/bug_eyed_line_001.pdf}
            \caption{The Bug-Eyed Line Construction}
            \label{fig:bug_eyed_line}
        \end{figure}
        \begin{itemize}
            \item (6 Points) Prove that it is impossible to embed
            the bug-eyed line into $\mathbb{R}^{n}$ for all $n\in\mathbb{N}$.
        \end{itemize}
    \end{problem}
    \begin{problem}
        \textbf{(Quotients)}
        \par\hfill\par
        Let $X=\mathbb{R}/\mathbb{Q}$, equipped with the quotient topology where
        $\mathbb{R}$ carries the usual Euclidean topology.
        \begin{itemize}
            \item (4 Points) Is this space Hausdorff? Is it Fr\'{e}chet?
        \end{itemize}
    \end{problem}
    \begin{problem}
        \textbf{(Products)}
        \par\hfill\par
        Consider topological spaces $(X,\,\tau_{X})$, $(Y,\,\tau_{Y})$, and
        $(Z,\,\tau_{Z})$. Equip $X\times{Y}$ with the product topology
        $\tau_{X\times{Y}}$.
        \begin{itemize}
            \item (6 Points) Prove that a function
                $f:Z\rightarrow{X}\times{Y}$ is continuous if and only if the
                component functions $\textrm{proj}_{X}\circ{f}:Z\rightarrow{X}$
                and $\textrm{proj}_{Y}\circ{f}:Z\rightarrow{Y}$ are continuous.
        \end{itemize}
    \end{problem}
\end{document}
