%-----------------------------------LICENSE------------------------------------%
%   This file is part of Mathematics-and-Physics.                              %
%                                                                              %
%   Mathematics-and-Physics is free software: you can redistribute it and/or   %
%   modify it under the terms of the GNU General Public License as             %
%   published by the Free Software Foundation, either version 3 of the         %
%   License, or (at your option) any later version.                            %
%                                                                              %
%   Mathematics-and-Physics is distributed in the hope that it will be useful, %
%   but WITHOUT ANY WARRANTY; without even the implied warranty of             %
%   MERCHANTABILITY or FITNESS FOR A PARTICULAR PURPOSE.  See the              %
%   GNU General Public License for more details.                               %
%                                                                              %
%   You should have received a copy of the GNU General Public License along    %
%   with Mathematics-and-Physics.  If not, see <https://www.gnu.org/licenses/>.%
%------------------------------------------------------------------------------%
\documentclass{article}
\usepackage{amsmath}                            % Needed for align.
\usepackage{amssymb}                            % Needed for mathbb.
\usepackage{amsthm}                             % For the theorem environment.

%------------------------Theorem Styles-------------------------%

% Define theorem style for default spacing and normal font.
\newtheoremstyle{normal}
    {\topsep}               % Amount of space above the theorem.
    {\topsep}               % Amount of space below the theorem.
    {}                      % Font used for body of theorem.
    {}                      % Measure of space to indent.
    {\bfseries}             % Font of the header of the theorem.
    {}                      % Punctuation between head and body.
    {.5em}                  % Space after theorem head.
    {}

% Define default environments.
\theoremstyle{normal}
\newtheorem{problem}{Problem}

\title{Point-Set Topology: Homework 4}
\date{Summer 2023}

% No indent and no paragraph skip.
\setlength{\parindent}{0em}
\setlength{\parskip}{0em}

\begin{document}
    \maketitle
    \begin{problem}
        \textbf{(Comparing Topologies)}
        \par\hfill\par
        Let $X$ be a set, and $\tau$ and $\tau'$ topologies on $X$ with
        $\tau\subseteq\tau'$.
        \begin{enumerate}
            \item (2 Pts) Prove that if $(X,\,\tau)$ is Fr\'{e}chet,
                then $(X,\,\tau')$ is.
            \item (2 Pts) Prove that if $(X,\,\tau)$ is Hausdorff, then
                $(X,\,\tau')$ is.
            \item (3 Pts) If $(X,\,\tau)$ is regular, does this imply
                $(X,\,\tau')$ is? Prove this or provide a counterexample.
            \item (3 Pts) If $(X,\,\tau)$ is normal, does this imply
                $(X,\,\tau')$ is? Prove this or provide a counterexample.
        \end{enumerate}
    \end{problem}
    \begin{problem}
        \textbf{(The Sorgenfrey Line)}
        \par\hfill\par
        The Sorgenfrey line is the real line $\mathbb{R}$ with the left-interval
        topology. That is, the topology is generated by sets of the form
        $[a,\,b)$ for $a,b\in\mathbb{R}$.
        \begin{enumerate}
            \item (3 Pts) A zero dimensional space is a topological space
                $(X,\,\tau)$ such that there is a basis $\mathcal{B}$ of sets
                that are both open and closed. Prove that the Sorgenfrey line
                is zero dimensional. (Hint: Prove $[a,\,b)$ is open and closed
                in this topology).
            \item (5 Pts) Prove that a zero dimensional topological space is
                completely regular.
            \item (2 Pts) Letting $\tau_{\mathbb{R}}$ and $\tau_{S}$ be the
                Euclidean and Sorgenfrey topologies, respectively, on
                $\mathbb{R}$, prove that $\tau_{\mathbb{R}}\subseteq\tau_{S}$.
                [Hint: What is a basis of $\tau_{\mathbb{R}}$? Are these
                elements open in $\tau_{S}$?]
            \item (8 Pts) Prove that the Sorgenfrey line is Lindel\"{o}f. That
                is, every open cover has a countable subcover. Do this by
                showing that any open cover of the Sorgenfrey line by basis
                elements (sets of the form $[a,\,b)$) has a countable subcover.
                [Hint: Sets of the form $(a,\,b)$ are open in the Euclidean
                topology. Since the Euclidean topology is second-countable,
                every subspace is Lindel\"{o}f. Prove that, if
                $\mathcal{O}$ is an open cover of the Sorgenfrey line of basis
                elements $[a,\,b)$, then the sets of the form $(a,\,b)$
                cover all but a countable subset of $\mathbb{R}$].
        \end{enumerate}
    \end{problem}
    \begin{problem}
        \textbf{(The Sorgenfrey Plane)}
        \par\hfill\par
        A regular Lindel\"{o}f space is paracompact, meaning the Sorgenfrey
        line is a paracompact Hausdorff space. By Dieudonn\'{e}'s theorem it
        is therefore normal. Here you will prove that normality is not
        preserved by products.
        \begin{enumerate}
            \item (3 Pts) Prove that the product of two Hausdorff spaces is
                Hausdorff.
            \item (3 Pts) Prove that the product of two regular spaces is
                regular.
            \item (5 Pts) The Sorgenfrey Plane is the product of the Sorgenfrey
                line with itself. The anti-diagonal:
                \begin{equation}
                    \Delta=\{(x,\,-x)\in\mathbb{R}^{2}\;|\;x\in\mathbb{R}\,\}
                \end{equation}
                has the strange property that it is a discrete subspace. That
                is, the subspace topology of $\Delta$ is the power set of
                $\Delta$. Prove this.
            \item (5 Pts) The set $K$ of points in $\Delta$ with rational
                coordinates:
                \begin{equation}
                    K=\{\,(x,\,-x)\in\mathbb{R}^{2}\;|\;x\in\mathbb{Q}\,\}
                \end{equation}
                is a closed subset of a closed subspace, so it too is closed.
                As is the set $\Delta\setminus{K}$. These sets cannot be
                separated by open sets, showing us that the Sorgenfrey plane
                is not normal. Prove that the Sorgenfrey plane is not
                paracompact, not Lindel{o}f, and not second-countable.
            \item (4 Pts) Prove that a closed subspace of a separable space
                need not be separable.
        \end{enumerate}
    \end{problem}
    \begin{problem}
        \textbf{(Manifolds)}
        \par\hfill\par
        A locally-Euclidean topological space is a topological space
        $(X,\,\tau)$ such that for all $x\in{X}$ there is an open set
        $\mathcal{U}\in\tau$ with $x\in\mathcal{U}$ and a continuous injective
        open mapping $\varphi:\mathcal{U}\rightarrow\mathbb{R}^{n}$ for some
        $n\in\mathbb{N}$.
        A topological manifold is a topological space $(X,\,\tau)$ that is:
        \begin{enumerate}
            \item Hausdorff.
            \item Second-Countable.
            \item Locally-Euclidean.
        \end{enumerate}
        The classic examples are Euclidean spaces $\mathbb{R}^{n}$, the
        circle, sphere, and higher-dimensional analogues $\mathbb{S}^{n}$,
        the Klein bottle, and the projective spaces $\mathbb{RP}^{n}$. The
        Hausdorff and second-countable conditions are not redundant. The
        bug-eyed line is second-countable and locally-Euclidean, but
        non-Hausdorff. The long line is Hausdorff and locally-Euclidean, but
        not second-countable.
        \par\hfill\par
        You may use the fact that a locally-compact Hausdorff space is regular,
        should you find it useful.
        \begin{enumerate}
            \item (5 Pts) Recall that locally compact means for all
                $x$ there is an open set $\mathcal{U}$ and a compact subset
                $K$ such that $x\in\mathcal{U}$ and $\mathcal{U}\subseteq{K}$.
                Prove that a topological manifold is locally compact.
            \item (3 Pts) Prove that a topological manifold is $\sigma$-compact.
            \item (3 Pts) Prove that a topological manifold is paracompact.
                [Hint: Use your theorems. Not much work needed here.]
            \item (3 Pts) Prove that a topological manifold is metrizable.
                [Hint: Again, appeal to theorems. Piece together the puzzle,
                not much effort should be applied here.]
        \end{enumerate}
    \end{problem}
\end{document}
