%-----------------------------------LICENSE------------------------------------%
%   This file is part of Mathematics-and-Physics.                              %
%                                                                              %
%   Mathematics-and-Physics is free software: you can redistribute it and/or   %
%   modify it under the terms of the GNU General Public License as             %
%   published by the Free Software Foundation, either version 3 of the         %
%   License, or (at your option) any later version.                            %
%                                                                              %
%   Mathematics-and-Physics is distributed in the hope that it will be useful, %
%   but WITHOUT ANY WARRANTY; without even the implied warranty of             %
%   MERCHANTABILITY or FITNESS FOR A PARTICULAR PURPOSE.  See the              %
%   GNU General Public License for more details.                               %
%                                                                              %
%   You should have received a copy of the GNU General Public License along    %
%   with Mathematics-and-Physics.  If not, see <https://www.gnu.org/licenses/>.%
%------------------------------------------------------------------------------%
\documentclass{article}
\usepackage{graphicx}                           % Needed for figures.
\usepackage{amsmath}                            % Needed for align.
\usepackage{amssymb}                            % Needed for mathbb.
\usepackage{amsthm}                             % For the theorem environment.

%------------------------Theorem Styles-------------------------%

% Define theorem style for default spacing and normal font.
\newtheoremstyle{normal}
    {\topsep}               % Amount of space above the theorem.
    {\topsep}               % Amount of space below the theorem.
    {}                      % Font used for body of theorem.
    {}                      % Measure of space to indent.
    {\bfseries}             % Font of the header of the theorem.
    {}                      % Punctuation between head and body.
    {.5em}                  % Space after theorem head.
    {}

% Define default environments.
\theoremstyle{normal}
\newtheorem{problem}{Problem}

\title{Point-Set Topology: Homework 2}
\date{Summer 2023}

% No indent and no paragraph skip.
\setlength{\parindent}{0em}
\setlength{\parskip}{0em}

\begin{document}
    \maketitle
    \begin{problem}
        \textbf{(Subspaces)}
        \par\hfill\par
        The inclusion mapping of a subset $A\subseteq{X}$ into $X$ is the
        function $\iota_{A}:A\rightarrow{X}$ defined by $\iota_{A}(x)=x$.
        \begin{itemize}
            \item (1 Point) Prove that if $(X,\,d)$ is a metric space, and if
                $(A,\,d_{A})$ is a metric subspace, then $\iota_{A}$ is
                continuous.
            \item (3 Points) Suppose $(X,\,d_{X})$ and $(Y,\,d_{Y})$ are
                metric spaces and let $A\subseteq{X}$ be a subset and
                $d_{A}$ be the subspace metric. Prove that $f:Y\rightarrow{A}$
                is continuous \textit{if and only if}
                $\iota_{A}\circ{f}:Y\rightarrow{X}$ is continuous.
            \item (2 Points)
                Prove that if $f:X\rightarrow{Y}$ is a continuous function
                from a metric space $(X,\,d_{X})$ to a metric space
                $(Y,\,d_{Y})$, and if $(A,\,d_{A})$ is a subspace of
                $(X,\,d_{X})$, then the restriction $f|_{A}:A\rightarrow{Y}$,
                defined by $f|_{A}(x)=f(x)$, is continuous.
            \item (4 Points) Prove that if $f:X\rightarrow{Y}$ is a
                homeomorphism, and if $A\subseteq{X}$, then
                $f|_{A}:A\rightarrow{f}[A]$ is a homeomorphism.
        \end{itemize}
    \end{problem}
    \begin{problem}
        \textbf{(Continuity)}
        \par\hfill\par
        We have proven the equivalence of three definitions of continuity.
        The definition is that $f$ maps convergent sequences to convergent
        sequences. The calculus $\varepsilon-\delta$ statement is equivalent to
        this, as is the fact that the pre-image of open sets is open. Continuity
        can be described by forward images as well.
        \begin{itemize}
            \item (6 Points) Let $(X,\,d_{X})$ and $(Y,\,d_{Y})$ be metric
                spaces. Prove that $f:X\rightarrow{Y}$ is continuous if and only
                if for all $x\in{X}$ and for all open subsets
                $\mathcal{V}\subseteq{Y}$ with $f(x)\in\mathcal{V}$ there is an
                open subset $\mathcal{U}\subseteq{X}$ such that
                $x\in\mathcal{U}$ and $f[\mathcal{U}]\subseteq\mathcal{V}$.
        \end{itemize}
    \end{problem}
    \begin{problem}
        \textbf{(Compact Spaces)}
        \par\hfill\par
        For metric spaces there are many equivalent ways of defining
        compactness. Your job is to prove some of these equivalences.
        \begin{itemize}
            \item (4 Points) Prove that $(X,\,d)$ is compact if and only if
                for every sequence of closed non-empty nested sets, the
                intersection is non-empty. That is, if
                $\mathcal{C}:\mathbb{N}\rightarrow\mathcal{P}(X)$ is a sequence
                of closed sets such that $\mathcal{C}_{n}\ne\emptyset$ and
                $\mathcal{C}_{n+1}\subseteq\mathcal{C}_{n}$, then
                $\bigcap_{n\in\mathbb{N}}\mathcal{C}_{n}$ is non-empty.
            \item (2 Points) Prove that $(X,\,d)$ is compact if and only if for
                every sequence of nested proper open subsets, the union is not
                the whole space. That is, if
                $\mathcal{U}:\mathbb{N}\rightarrow\mathcal{P}(X)$ is a sequence
                of open sets such that $\mathcal{U}_{n}\ne{X}$ and
                $\mathcal{U}_{n}\subseteq\mathcal{U}_{n+1}$, then
                $\bigcup_{n\in\mathbb{N}}\mathcal{U}_{n}$ is not equal to $X$.
                [Hint: What is the complement of an open set? Can the previous
                part of the problem help?]
        \end{itemize}
    \end{problem}
    \begin{problem}
        \textbf{(Calculus)}
        \par\hfill\par
        With our tools from metric space theory, one of the harder theorems
        from calculus becomes quite simple.
        \begin{itemize}
            \item (4 Points) Prove that if $(X,\,d_{X})$ is compact, if
                $(Y,\,d_{Y})$ is a metric space, and if $f:X\rightarrow{Y}$ is
                continuous, then $f[X]\subseteq{Y}$ is a compact subspace.
            \item (4 Points) The extreme value theorem states that if
                $f:[a,\,b]\rightarrow\mathbb{R}$ is continuous, then there is
                $c_{\textrm{min}},c_{\textrm{max}}\in[a,\,b]$ such that
                $f(c_{\textrm{min}})\leq{f}(x)\leq{f}(c_{\textrm{max}})$ for
                all $x\in[a,\,b]$. Let's take that up a notch. Prove that if
                $(X,\,d)$ is compact, and if $f:X\rightarrow\mathbb{R}$ is
                continuous, then there are points $c_{\textrm{min}}$ and
                $c_{\textrm{max}}$ such that
                $f(c_{\textrm{min}})\leq{f}(x)\leq{f}(c_{\textrm{max}})$ for
                all $x\in{X}$. [Hint: The previous part is enormously helpful.]
        \end{itemize}
    \end{problem}
    \begin{problem}
        \textbf{(Product Spaces)}
        \par\hfill\par
        (6 Points) Given metric spaces $(X,\,d_{X})$
        and $(Y,\,d_{Y})$, prove that all three product metrics are
        topologically equivalent:
        \begin{align}
            d_{1}\big((x_{0},\,y_{0}),\,(x_{1},\,y_{1})\big)
                &=d_{X}(x_{0},\,x_{1})+d_{Y}(y_{0},\,y_{1})\\
            d_{2}\big((x_{0},\,y_{0}),\,(x_{1},\,y_{1})\big)
                &=\sqrt{d_{X}(x_{0},\,x_{1})^{2}+d_{Y}(y_{0},\,y_{1})^{2}}\\
            d_{\infty}\big((x_{0},\,y_{0}),\,(x_{1},\,y_{1})\big)
                &=\textrm{max}\big(%
                    d_{X}(x_{0},\,x_{1}),\,d_{Y}(y_{0},\,y_{1})
                \big)
        \end{align}
    \end{problem}
\end{document}
