%-----------------------------------LICENSE------------------------------------%
%   This file is part of Mathematics-and-Physics.                              %
%                                                                              %
%   Mathematics-and-Physics is free software: you can redistribute it and/or   %
%   modify it under the terms of the GNU General Public License as             %
%   published by the Free Software Foundation, either version 3 of the         %
%   License, or (at your option) any later version.                            %
%                                                                              %
%   Mathematics-and-Physics is distributed in the hope that it will be useful, %
%   but WITHOUT ANY WARRANTY; without even the implied warranty of             %
%   MERCHANTABILITY or FITNESS FOR A PARTICULAR PURPOSE.  See the              %
%   GNU General Public License for more details.                               %
%                                                                              %
%   You should have received a copy of the GNU General Public License along    %
%   with Mathematics-and-Physics.  If not, see <https://www.gnu.org/licenses/>.%
%------------------------------------------------------------------------------%
\documentclass{article}
\usepackage{graphicx}                           % Needed for figures.
\usepackage{amsmath}                            % Needed for align.
\usepackage{amssymb}                            % Needed for mathbb.
\usepackage{amsthm}                             % For the theorem environment.
\usepackage[font=scriptsize,
            labelformat=simple,
            labelsep=colon]{subcaption} % Subfigure captions.
\usepackage[font={scriptsize},
            hypcap=true,
            labelsep=colon]{caption}    % Figure captions.
\usepackage{hyperref}
\hypersetup{colorlinks=true, linkcolor=blue}
\renewcommand\thesubfigure{\arabic{figure}.\arabic{subfigure}}

% Define theorem style for default spacing and normal font.
\newtheoremstyle{normal}
    {\topsep}               % Amount of space above the theorem.
    {\topsep}               % Amount of space below the theorem.
    {}                      % Font used for body of theorem.
    {}                      % Measure of space to indent.
    {\bfseries}             % Font of the header of the theorem.
    {}                      % Punctuation between head and body.
    {.5em}                  % Space after theorem head.
    {}

% Define default environments.
\theoremstyle{normal}
\newtheorem{examplex}{Example}

\newenvironment{example}{%
    \pushQED{\qed}\renewcommand{\qedsymbol}{$\blacksquare$}\examplex%
}{%
    \popQED\endexamplex%
}

\title{Advanced Calculus: Tutorial 1}
\author{Ryan Maguire}
\date{\today}

% No indent and no paragraph skip.
\setlength{\parindent}{0em}
\setlength{\parskip}{0em}

\begin{document}
    \maketitle
    The following notes are from my tutorials for advanced calculus during
    the spring 2021 semester at Dartmouth college. The nature of tutorial
    sessions makes the handwritten notes very scattered as students ask
    questions about different aspects of the course. These types notes
    reflect this scattering of information as well.
    \begin{example}[\textbf{Basics of Linear Functions}]
        A linear function $f:\mathbb{R}^{n}\rightarrow\mathbb{R}^{m}$ is a
        function that respects vector addition. Formally this means that if
        $\mathbf{x}_{1},\mathbf{x}_{2}\in\mathbb{R}$ are vectors, then
        \begin{equation}
            f(\mathbf{x}_{1}+\mathbf{x}_{2})
            =f(\mathbf{x}_{1})+f(\mathbf{x}_{2})
        \end{equation}
        Moreover it respects scalar multiplication. That is, if
        $a\in\mathbb{R}$ and $\mathbf{x}\in\mathbb{R}^{n}$, then:
        \begin{equation}
            f(a\cdot\mathbf{x})=a\cdot{f}(\mathbf{x})
        \end{equation}
        The multiplication symbol is often omitted in the notation, and this
        may be rewritten as:
        \begin{equation}
            f(a\mathbf{x})=a\,f(\mathbf{x})
        \end{equation}
        Another way to put this uses bases. If
        $\{\,\mathbf{e}_{1},\,\cdots,\,\mathbf{e}_{n}\,\}$ is a basis for
        $\mathbb{R}^{n}$ and if $\mathbf{x}\in\mathbb{R}^{n}$ is given by:
        \begin{equation}
            \mathbf{x}=\sum_{k=1}^{n}a_{k}\mathbf{e}_{k}
        \end{equation}
        where $a_{k}\in\mathbb{R}$ are scalars, then:
        \begin{equation}
            f(\mathbf{x})
            =f\Big(\sum_{k=1}^{n}a_{k}\mathbf{e}_{k}\Big)
            =\sum_{k=1}^{n}a_{k}\,f(\mathbf{e}_{k})
        \end{equation}
        Linear functions $f:\mathbb{R}\rightarrow\mathbb{R}$ are just functions
        of the form $f(x)=ax$. We can prove this using elementary calculus.
        Let $a=f(1)$, the value of $f$ at $x=1$. We have:
        \begin{align}
            \frac{f(x+h)-f(x)}{h}
            &=\frac{f(x)+f(h)-f(x)}{h}
            \tag{Linearity}\\
            &=\frac{f(h)}{h}
            \tag{Simplify}\\
            &=\frac{h\,f(1)}{h}
            \tag{Linearity}\\
            &=\frac{ah}{h}
            \tag{Definition of $a$}\\
            &=a
        \end{align}
        If we take limits we see that $f'(x)=a$ for all $x\in\mathbb{R}$.
        Integration tells us $f(x)=ax+b$ for some $b\in\mathbb{R}$. We now show
        that $b=0$. We have:
        \begin{equation}
            f(0)=f(0\cdot{0})=0\cdot{f}(0)=0
        \end{equation}
        And hence $f(0)=0$. Since $f(x)=ax+b$, this tells us that $b=0$. So the
        equation for $f$ is $f(x)=ax$.
    \end{example}
    \begin{example}[\textbf{Basis of a Vector Space}]
        A basis for $\mathbb{R}^{n}$ is a set of
        \textit{linearly independent vectors}
        $\{\,\mathbf{e}_{1},\,\cdots,\,\mathbf{e}_{n}\,\}$ that
        \textit{span} the entirety of $\mathbb{R}^{n}$. That is, every
        vector $\mathbf{x}\in\mathbb{R}^{n}$ can be uniquely written as:
        \begin{equation}
            \mathbf{x}=\sum_{k=1}^{n}a_{k}\mathbf{e}_{k}
        \end{equation}
        for (unique) scalars $a_{1},\,\dots,\,a_{n}$. As a simple example,
        consider $\mathbb{R}^{2}$ and define
        $\mathbf{e}_{1}=(1,\,0)$ and $\mathbf{e}_{2}=(0,\,1)$. Then given a
        vector $\mathbf{x}=(x_{1},\,x_{2})$ we can write
        $\mathbf{x}=x_{1}\mathbf{e}_{1}+x_{2}\mathbf{e}_{2}$.
    \end{example}
    \begin{example}[\textbf{Linear Functions Defined Using Bases}]
        A linear function is uniquely determined by its behavior on a basis.
        That is, if $f:\mathbb{R}^{n}\rightarrow\mathbb{R}^{m}$ is a linear
        function, and if we know the value of $f(\mathbf{e}_{k})$ for all
        $\mathbf{e}_{k}$ in a basis set
        $\{\,\mathbf{e}_{1},\,\cdots,\,\mathbf{e}_{n}\,\}$, then we can compute
        $f(\mathbf{x})$ for any $\mathbf{x}\in\mathbb{R}^{n}$. That is, if we
        know the vectors $\mathbf{v}_{k}\in\mathbb{R}^{m}$ such that
        $f(\mathbf{e}_{k})=\mathbf{v}_{k}$, we know all there is to know about
        the function. For let $\mathbf{x}\in\mathbb{R}^{n}$. Then there are
        (unique) scalars $a_{k}\in\mathbb{R}$ such that:
        \begin{equation}
            \mathbf{x}=\sum_{k=1}^{n}a_{k}\mathbf{e}_{k}
        \end{equation}
        This is the definition of a basis. Since $f$ is linear, and since we
        know $f(\mathbf{e}_{k})=\mathbf{v}_{k}$ for each $k$, we have:
        \begin{equation}
            f(\mathbf{x})
            =f\Big(\sum_{k=1}^{n}a_{k}\mathbf{e}_{k}\Big)
            =\sum_{k=1}^{n}a_{k}\,f(\mathbf{e}_{k})
            =\sum_{k=1}^{n}a_{k}\mathbf{v}_{k}
        \end{equation}
        allowing us to compute $f(\mathbf{x})$ for any arbitrary vector
        $\mathbf{x}\in\mathbb{R}^{n}$.
    \end{example}
    \begin{example}[\textbf{Reflections}]
        A reflection $f:\mathbb{R}^{n}\rightarrow\mathbb{R}^{n}$ (note the
        domain and co-domain are the same here) about a \textbf{non-zero}
        vector $\mathbf{v}\in\mathbb{R}^{n}$ is the linear function defined by:
        \begin{equation}
            f(\mathbf{x})
            =\mathbf{x}-2
            \frac{\langle\,\mathbf{x}\,|\,\mathbf{v}\,\rangle}
                 {\langle\,\mathbf{v}\,|\,\mathbf{v}\,\rangle}
            \mathbf{v}
        \end{equation}
        where $\langle\,\mathbf{x}\,|\,\mathbf{v}\,\rangle$ denotes the
        standard Euclidean inner product, also called the Euclidean
        \textit{dot product}, between vectors. To be explicit, if
        $\mathbf{x}=(x_{1},\,\dots,\,x_{n})$ and
        $\mathbf{y}=(y_{1},\,\dots,\,y_{n})$, then we have:
        \begin{equation}
            \langle\,\mathbf{x}\,|\,\mathbf{y}\,\rangle
            =\sum_{k=1}^{n}x_{k}y_{k}
        \end{equation}
        Let's derive this. A reflection should negate any vector parallel to
        the vector we are reflecting. This is $\mathbf{v}$ in our example.
        That is, $f(\mathbf{x})=-\mathbf{x}$ for any vector $\mathbf{x}$ that
        is a scalar multiple of $\mathbf{v}$. Secondly, any vector that is
        perpendicular to $\mathbf{v}$ should be untouched. Think of the plane
        $\mathbb{R}^{2}$ reflecting the $x$ axis. The positive $x$ axis is
        flipped to the negative $x$ axis, and the $y$ axis remains stationary.
        \par\hfill\par
        The set of all vectors perpendicular to $\mathbf{v}$ forms a
        \textit{vector subspace} of $\mathbb{R}^{n}$. Moreover, a basis
        $\{\,\mathbf{e}_{1},\,\cdots,\,\mathbf{e}_{n-1}\,\}$ for this subspace,
        together with $\mathbf{v}$, forms a basis for all of $\mathbb{R}^{n}$.
        Let's write this as $\{\,\mathbf{e}_{1},\,\cdots,\,\mathbf{e}_{n}\,\}$
        where we define $\mathbf{e}_{n}=\mathbf{v}$. Our previous conditions
        now translate to a formula for the basis elements:
        \begin{equation}
            f(\mathbf{e}_{k})=
            \begin{cases}
                \mathbf{e}_{k}&k\ne{n}\\
                -\mathbf{e}_{k}&k=n
            \end{cases}
        \end{equation}
        Given $\mathbf{x}\in\mathbb{R}$ we may write:
        \begin{equation}
            f(\mathbf{x}=\sum_{k=1}^{n}a_{k}\mathbf{e}_{k}
        \end{equation}
        for appropriate scalars $a_{k}$. Since $f$ is linear, and since we know
        how it works on a basis, we have:
        \begin{equation}
            f(\mathbf{x})=\sum_{k=1}^{n}a_{k}\,f(\mathbf{e}_{k})
            =\Big(\sum_{k=1}^{n-1}a_{k}\mathbf{e}_{k}\Big)-a_{n}\mathbf{e}_{n}
        \end{equation}
        Very close to the original formula for $\mathbf{x}$. Let us add back
        the minus component (and then subtract it out so as to not change the
        formula). We get:
        \begin{align}
            f(\mathbf{x})
            &=\Big(
                \sum_{k=1}^{n-1}a_{k}\mathbf{e}_{k}
            \Big)-a_{n}\mathbf{e}_{n}\\
            &=\Big(
                \sum_{k=1}^{n-1}a_{k}\mathbf{e}_{k}
            \Big)+(a_{n}\mathbf{e}_{n}-a_{n}\mathbf{e}_{n})
            -a_{n}\mathbf{e}_{n}\\
            &=\Big(
                \sum_{k=1}^{n}a_{k}\mathbf{e}_{k}
            \Big)-2a_{n}\mathbf{e}_{n}\\
            &=\mathbf{x}-2a_{n}\mathbf{e}_{n}
        \end{align}
        We'll complete our formula if we can rewrite $2a_{n}\mathbf{e}_{n}$ in
        a different manner. Remember that $\mathbf{e}_{n}$ is
        orthogonal to $\mathbf{e}_{k}$ for all $k\ne{n}$. From the linearity of
        the inner product we get:
        \begin{align}
            \langle\,\mathbf{x}\,|\,\mathbf{e}_{n}\,\rangle
            &=
            \langle\,
                \sum_{k=1}^{n}a_{k}\mathbf{e}_{k}\,|\,\mathbf{e}_{k}
            \rangle\\
            &=
            \sum_{k=1}^{n}\langle\,
                a_{k}\mathbf{e}_{k}\,|\,\mathbf{e}_{k}
            \rangle\\
            &=\sum_{k=1}^{n}a_{k}\langle\,
                \mathbf{e}_{k}\,|\,\mathbf{e}_{k}
            \rangle\\
            &=a_{n}\langle\,\mathbf{e}_{n}\,|\,\mathbf{e}_{n}\,\rangle
        \end{align}
        We hence obtain:
        \begin{equation}
            a_{n}=\frac{\langle\,\mathbf{x}\,|\,\mathbf{e}_{n}\,\rangle}
                       {\langle\,\mathbf{e}_{n}\,|\,\mathbf{e}_{n}\,\rangle}
        \end{equation}
        Our previous formula then becomes:
        \begin{equation}
            f(\mathbf{x})
            =\mathbf{x}-2
            \frac{\langle\,\mathbf{x}\,|\,\mathbf{e}_{n}\,\rangle}
                 {\langle\,\mathbf{e}_{n}\,|\,\mathbf{e}_{n}\,\rangle}
            \mathbf{e}_{n}
        \end{equation}
        Recalling that we set $\mathbf{v}=\mathbf{e}_{n}$, we obtain:
        \begin{equation}
            f(\mathbf{x})
            =\mathbf{x}-2
            \frac{\langle\,\mathbf{x}\,|\,\mathbf{v}\,\rangle}
                 {\langle\,\mathbf{v}\,|\,\mathbf{v}\,\rangle}
            \mathbf{v}
        \end{equation}
        Precisely as claimed.
    \end{example}
    \begin{example}[\textbf{Reflections in the Plane}]
    \end{example}
\end{document}
