%-----------------------------------LICENSE------------------------------------%
%   This file is part of Mathematics-and-Physics.                              %
%                                                                              %
%   Mathematics-and-Physics is free software: you can redistribute it and/or   %
%   modify it under the terms of the GNU General Public License as             %
%   published by the Free Software Foundation, either version 3 of the         %
%   License, or (at your option) any later version.                            %
%                                                                              %
%   Mathematics-and-Physics is distributed in the hope that it will be useful, %
%   but WITHOUT ANY WARRANTY; without even the implied warranty of             %
%   MERCHANTABILITY or FITNESS FOR A PARTICULAR PURPOSE.  See the              %
%   GNU General Public License for more details.                               %
%                                                                              %
%   You should have received a copy of the GNU General Public License along    %
%   with Mathematics-and-Physics.  If not, see <https://www.gnu.org/licenses/>.%
%------------------------------------------------------------------------------%
\documentclass{article}
\usepackage{graphicx} % Needed for figures.
\usepackage{amsmath}  % Needed for align.
\usepackage{amssymb}  % Needed for mathbb.
\usepackage{amsthm}   % For the theorem environment.
\usepackage{hyperref}
\hypersetup{colorlinks=true, linkcolor=blue}
\newtheoremstyle{normal}
    {\topsep}   % Amount of space above the theorem.
    {\topsep}   % Amount of space above the theorem.
    {}          % Font used for body of theorem.
    {}          % Measure of space to indent.
    {\bfseries} % Font of the header of the theorem.
    {}          % Punctuation between head and body.
    {0.5em}     % Space after theorem head.
    {}
\theoremstyle{normal}
\newtheorem{examplex}{Example}
\newenvironment{example}{%
    \pushQED{\qed}\renewcommand{\qedsymbol}{$\blacksquare$}\examplex%
}{%
    \popQED\endexamplex%
}

\title{Advanced Calculus: Tutorial 2}
\author{Ryan Maguire}
\date{\today}
\setlength{\parindent}{0em}
\setlength{\parskip}{0em}

\begin{document}
    \maketitle
    \begin{abstract}
        The following notes are from my tutorials for advanced calculus during
        the spring 2021 semester at Dartmouth college. The nature of tutorial
        sessions makes the handwritten notes very scattered as students ask
        questions about different aspects of the course. These typed notes
        reflect this scattering of information as well.
    \end{abstract}
    \begin{example}[\textbf{Derivatives and Inner Products}]
        Suppose $\gamma_{0},\gamma_{1}:\mathbb{R}\rightarrow\mathbb{R}^{n}$
        are smooth curves in $\mathbb{R}^{n}$. That is, if we write
        $\gamma_{0}(t)=\big(x_{1}(t),\,\cdots,\,x_{n}(t)\big)$, then smooth
        means $x_{k}:\mathbb{R}\rightarrow\mathbb{R}$ is a smooth
        real-valued function for each $k$. Similarly for $\gamma_{1}$.
        We can consider the real-valued function
        $f:\mathbb{R}\rightarrow\mathbb{R}$ defined by:
        \begin{equation}
            f(t)=\langle\,\gamma_{0}(t)\,|\,\gamma_{1}(t)\rangle
        \end{equation}
        Since this function takes in a real number and then returns a real
        number we can ask if it is differentiable. The difference quotient is:
        \begin{equation}
            \frac{f(t+h)-f(t)}{h}
            =\frac{%
                \langle\,\gamma_{0}(t+h)\,|\,\gamma_{1}(t+h)\rangle-%
                \langle\,\gamma_{0}(t)\,|\,\gamma_{1}(t)\rangle%
            }{h}
        \end{equation}
        Using the bilinearity of the inner product we can simplify this.
        \begin{align}
            \frac{f(t+h)-f(t)}{h}
            &=\frac{%
                \langle\,\gamma_{0}(t+h)\,|\,\gamma_{1}(t+h)\rangle-%
                \langle\,\gamma_{0}(t)\,|\,\gamma_{1}(t)\rangle%
            }{h}\\
            &=\frac{%
                \langle\,\gamma_{0}(t+h)\,|\,\gamma_{1}(t+h)\rangle-%
                \langle\,\gamma_{0}(t+h)\,|\,\gamma_{1}(t)\rangle%
            }{h}\nonumber\\
            &\hspace{2em}+\frac{%
                \langle\,\gamma_{0}(t+h)\,|\,\gamma_{1}(t)\rangle-%
                \langle\,\gamma_{0}(t)\,|\,\gamma_{1}(t)\rangle%
            }{h}\\
            &=
            \frac{%
                \langle\,\gamma_{0}(t+h)\,|\,%
                    \gamma_{1}(t+h)-\gamma_{1}(t)\rangle%
            }{h}\nonumber\\
            &\hspace{2em}+\frac{%
                \langle\,\gamma_{0}(t+h)-\gamma_{0}(t)\,|\,%
                    \gamma_{1}(t)\rangle%
            }{h}\\
            &=
            \langle\,\gamma_{0}(t+h)\,|\,
                \frac{\gamma_{1}(t+h)-\gamma_{1}(t)}{h}\,\rangle\nonumber\\
            &\hspace{2em}+
            \langle\,\frac{\gamma_{0}(t+h)-\gamma_{0}(t)}{h}\,|\,
                \gamma_{1}(t)\,\rangle
        \end{align}
        If we take limits we get:
        \begin{align}
            &\lim_{h\rightarrow{0}}\frac{f(t+h)-f(t)}{h}\nonumber\\
            &=\lim_{h\rightarrow{0}}\Big(
            \langle\,\gamma_{0}(t+h)\,|\,
                \frac{\gamma_{1}(t+h)-\gamma_{1}(t)}{h}\,\rangle+
            \langle\,\frac{\gamma_{0}(t+h)-\gamma_{0}(t)}{h}\,|\,
                \gamma_{1}(t)\,\rangle\Big)
        \end{align}
        Since inner products are continuous, we can bring the limit into the
        inside of the angle brackets.
        \begin{align}
            &\lim_{h\rightarrow{0}}\frac{f(t+h)-f(t)}{h}\nonumber\\
            &=\lim_{h\rightarrow{0}}\Big(
            \langle\,\gamma_{0}(t+h)\,|\,
                \frac{\gamma_{1}(t+h)-\gamma_{1}(t)}{h}\,\rangle+
            \langle\,\frac{\gamma_{0}(t+h)-\gamma_{0}(t)}{h}\,|\,
                \gamma_{1}(t)\,\rangle\Big)\\
            &=\langle\,
                \lim_{h\rightarrow{0}}\gamma_{0}(t+h)\,|\,\lim_{h\rightarrow{0}}
                \frac{\gamma_{1}(t+h)-\gamma_{1}(t)}{h}\,\rangle\nonumber\\
            &\hspace{2em}+
            \langle\,\lim_{h\rightarrow{0}}
                \frac{\gamma_{0}(t+h)-\gamma_{0}(t)}{h}\,|\,
                \gamma_{1}(t)\,\rangle\\
            &=\langle\,\gamma_{0}(t)\,|\,\dot{\gamma}_{1}(t)\,\rangle
                +\langle\,\dot{\gamma}_{0}(t)\,|\,\gamma_{1}(t)\,\rangle
        \end{align}
        where $\dot{\gamma}_{0}(t)$ and $\dot{\gamma}_{1}(t)$ denote the
        derivatives of $\gamma_{0}$ and $\gamma_{1}$, respectively.
        This is the \textbf{product rule} for derivatives of inner products.
    \end{example}
    \begin{example}[\textbf{Displacements and Angles}]
        Suppose $\gamma:\mathbb{R}\rightarrow\mathbb{R}^{n}$ is a smooth
        curve. Suppose $\mathbf{p}\in\mathbb{R}^{n}$
        is a fixed point. The \textit{relative position vector}
        from $\mathbf{p}$ to $\gamma(t)$ is the vector:
        \begin{equation}
            \mathbf{R}(t)=\gamma(t)-\mathbf{p}
        \end{equation}
        Associated to this is the \textit{displacement}, which is the length
        of the relative position vector. We write:
        \begin{equation}
            \ell(t)=||\mathbf{R}(t)||
            =\sqrt{\langle\,\mathbf{R}(t)\,|\,\mathbf{R}(t)\,\rangle}
            =\sqrt{\mathbf{R}(t)\cdot\mathbf{R}(t)}
        \end{equation}
        The rate of change of the displacement is the time-derivative of this
        function. We get:
        \begin{align}
            \dot{\ell}(t)
            &=\frac{\textrm{d}}{\textrm{d}t}\Big(
                \sqrt{\langle\,\mathbf{R}(t)\,|\,\mathbf{R}(t)\,\rangle}
            \Big)\\
            &=
            \frac{1}{2\sqrt{\langle\,\mathbf{R}(t)\,|\,\mathbf{R}(t)\,\rangle}}
            \frac{\textrm{d}}{\textrm{d}t}\Big(
                \langle\,\mathbf{R}(t)\,|\,\mathbf{R}(t)\,\rangle
            \Big)\\
            &=
            \frac{1}{2\sqrt{\langle\,\mathbf{R}(t)\,|\,\mathbf{R}(t)\,\rangle}}
            \Big(
                \langle\,\dot{\mathbf{R}}(t)\,|\,\mathbf{R}(t)\,\rangle
                +
                \langle\,\mathbf{R}(t)\,|\,\dot{\mathbf{R}}(t)\,\rangle
            \Big)\\
            &=\frac{\langle\,\dot{\mathbf{R}}(t)\,|\,\mathbf{R}(t)\,\rangle}
                {\sqrt{\langle\,\mathbf{R}(t)\,|\,\mathbf{R}(t)\,\rangle}}\\
            &=\frac{\langle\,\dot{\mathbf{R}}(t)\,|\,\mathbf{R}(t)\,\rangle}
                {||\mathbf{R}(t)||}
        \end{align}
        Since $\mathbf{R}(t)=\gamma(t)-\mathbf{p}$, we have that
        $\dot{\mathbf{R}}(t)=\dot{\gamma}(t)$. We obtain:
        \begin{align}
            \dot{\ell}(t)
            &=
            \frac{\langle\,\dot{\gamma}(t)\,|\,\gamma(t)-\mathbf{p}\,\rangle}
                {||\gamma(t)-\mathbf{p}||}\\
            &=
            \frac{\langle\,\dot{\gamma}(t)\,|\,\gamma(t)-\mathbf{p}\,\rangle}
                {||\mathbf{R}(t)||}
        \end{align}
        We can derive this explicitly using the summation formula for the
        Euclidean inner product. Writing
        $\mathbf{p}=(p_{1},\,\cdots,\,p_{n})$, we have:
        \begin{align}
            \frac{\textrm{d}}{\textrm{d}t}
            \sqrt{\langle\,\mathbf{R}(t)\,|\,\mathbf{R}(t)\,\rangle}
            &=\frac{\textrm{d}}{\textrm{d}t}
            \sqrt{\sum_{k=1}^{n}\big(\gamma_{k}(t)-p_{k}\big)^{2}}\\
            &=
            \frac{1}{2||\mathbf{R}(t)||}
            \frac{\textrm{d}}{\textrm{d}t}\Big(
                \sum_{k=1}^{n}\big(\gamma_{k}(t)-p_{k}\big)^{2}
            \Big)\\
            &=
            \frac{1}{2||\mathbf{R}(t)||}
            \sum_{k=1}^{n}
            2\big(\gamma_{k}(t)-p_{k}\big)\dot{\gamma}_{k}(t)\\
            &=
            \frac{1}{||\mathbf{R}(t)||}
            \sum_{k=1}^{n}
            \big(\gamma_{k}(t)-p_{k}\big)\dot{\gamma}_{k}(t)\\
            &=
            \frac{\langle\,\dot{\gamma}(t)\,|\,\gamma(t)-\mathbf{p}\,\rangle}
                {||\mathbf{R}(t)||}
        \end{align}
        Now suppose $\gamma$ describes a particle traveling at unit speed.
        That is, $||\dot{\gamma}(t)||=1$ for all $t\in\mathbb{R}$. With this
        assumption we can freely multiply and divide any expression by
        $||\dot{\gamma}(t)||$ without changing it. Let's do this for
        $\dot{\ell}(t)$. We get:
        \begin{equation}
            \dot{\ell}(t)
            =\frac{\langle\,\dot{\gamma}(t)\,|\,\gamma(t)-\mathbf{p}\,\rangle}
                {||\dot{\gamma}(t)||\,||\gamma(t)-\mathbf{p}||}
        \end{equation}
        This is the ratio of an inner product and a product of norms, which
        describes the angle between the vectors. That is:
        \begin{equation}
            \cos\big(
                \angle(\dot{\gamma}(t),\,\gamma(t)-\mathbf{p})
            \big)=\dot{\ell}(t)
        \end{equation}
        If we label
        $\theta(t)=\angle(\dot{\gamma}(t),\,\gamma(t)-\mathbf{p})$, then we
        can write this as:
        \begin{equation}
            \dot{\ell}(t)
            =\cos\big(\theta(t)\big)
        \end{equation}
        That is, the time-derivative of the displacement is described by
        the cosine of the angle made with
        $\mathbf{R}(t)$ and $\dot{\mathbf{R}}(t)$. This tells us that
        $\gamma(t)$ is closest (or furthest) to $\mathbf{p}$ when
        $\theta(t)=\frac{\pi}{2}$. That is, the extrema of the curve relative
        to $\mathbf{p}$ occur when the displacement and the change in the
        displacement are perpendicular.
    \end{example}
\end{document}
