\documentclass{beamer}
\usepackage{amsmath}

\title{Problems in Riemannian Geometry}
\author{Ryan Maguire}
\date{May 2022}
\usenavigationsymbolstemplate{}
\setbeamertemplate{footline}[frame number]
\begin{document}
    \maketitle
    \begin{frame}{Problem 1}
        What is a smooth manifold?
        \begin{itemize}
            \item A smooth manifold is a locally Euclidean, Hausdorff, second
                countable topological space with a collection of
                smoothly compatible charts $(\mathcal{U}_{\alpha},\varphi_{\alpha})$
                meaning that for all non-disjoint $\mathcal{U}_{\alpha}$ and
                $\mathcal{U}_{\beta}$ the transition functions
                $\varphi_{\alpha}\circ\varphi_{\beta}^{-1}$ and
                $\varphi_{\beta}\circ\varphi_{\alpha}^{-1}$, which are functions
                between open subsets of $\mathbb{R}^{N}$, are smooth.
        \end{itemize}
    \end{frame}
    \begin{frame}{Problem 2}
        What is an oriented manifold?
        \begin{itemize}
            \item An oriented manifold is a smooth manifold $(M,\mathcal{A})$
            with a subatlas $\mathcal{O}\subseteq\mathcal{A}$ such that for all
            $(\mathcal{U},\varphi),(\mathcal{V},\psi)\in\mathcal{O}$, the
            determinant of the Jacobian $\varphi\circ\psi^{-1}$ is positive.
        \end{itemize}
    \end{frame}
    \begin{frame}{Problem 3}
        What is the relation between normal and tangent bundle orientation for
        a submanifold of an oriented manifold.
        \begin{itemize}
            \item Given an embedded submanifold $M$ and an orientable manifold
                $N$ one may consider the normal bundle $N(M)$ of $M$ which
                consists of all points $(p,v)\in{T}N$ such that
                $v$ is orthogonal to all vectors in $T_{p}M$. The coordinates of
                $T_{p}N$ may be written as $v\oplus{v}^{\perp}$ with
                $v\in{T}_{p}M$ and $v^{\perp}$ perpendicular to $T_{p}M$, so
                in particular $(p,v^{\perp})$ is in the normal bundle. Because of
                this the normal bundle is orientable if and only if the tangent
                bundle is orientable.
        \end{itemize}
    \end{frame}
    \begin{frame}{Problem 4a}
        What is a properly discontinuous group action of a manifold?
        \begin{itemize}
            \item A group action on a smooth manifold $(M,\mathcal{A})$ by a group
                $G$ is a function $\theta:G\times{M}\rightarrow{M}$ such that for
                all $g\in{G}$ the function $f:M\rightarrow{M}$ defined by
                $f(p)=\theta(g,p)$ is a diffeomorphism, and such that for all
                $g\in{G}$ and $p\in{M}$ we get:
            \begin{align}
                \theta(e,p)&=p\\
                \theta(g_{0}*g_{1},p)&=\theta\Big(g_{0},\theta(g_{1},p)\Big)
            \end{align}
            where $e$ is the identity. A properly discontinuous group action is
            one such that for all $p\in{M}$ there is an open subset
            $\mathcal{U}\subseteq{M}$ with $p\in\mathcal{U}$ such that for all
            $g\in{G}$ with $g\ne{e}$ we have
            $\mathcal{U}\cap\theta[\{g\}\times\mathcal{U}]=\emptyset$.
        \end{itemize}
    \end{frame}
    \begin{frame}{Problem 4b}
        Discuss the quotient of a manifold by a properly discontinuous group
        action. Provide an example.
        \begin{itemize}
            \item The space of equivalent classes under a properly discontinuous
                group action of a manifold is a manifold itself with the
                quotient topology. The action being properly discontinuous
                means there is always an open neighborhood
                $\tilde{\mathcal{U}}$ in the quotient topology such that
                the pre-image under the quotient map is a collection of disjoint
                open sets $\mathcal{U}$, each of which being homeomorphic to
                $\mathbb{R}^{N}$. Since $f(p)=\theta(g,p)$ is a diffeomorphism
                for fixed $g\in{G}$, the transition maps between two such sets
                $\mathcal{U}$ are smooth. Since these $\tilde{\mathcal{U}}$
                cover the quotient space, the quotient is a smooth manifold.
                The Hausdorff and second-countable conditions can also be
                verified by the properly discontinuous property. If we take
                $\mathbb{S}^{2}$ and $G=\{-1,1\}$ under multiplication, the
                group action $\theta(g,\mathbf{p})=g\mathbf{p}$ is properly
                discontinuous. The resulting quotient is the real projective
                space $\mathbb{RP}^{2}$.
        \end{itemize}
    \end{frame}
    \begin{frame}{Problem 5}
        Discuss the pre-image of a regular value. Show that the sphere
        $\mathbb{S}^{N}$ is a orientable manifold.
        \begin{itemize}
            \item A regular point of a smooth function
                $f:M\rightarrow{N}$ between smooth manifolds is a point
                $p\in{M}$ such that the differential push-forward
                $\textrm{d}f_{p}:T_{p}M\rightarrow{T}_{f(p)}N$ is surjective.
                A regular value is a point $q\in{f[M]}$ such that for all
                $p\in{f}^{-1}[\{q\}]$ it is true that $q$ is a regular point.
                If $f:M\rightarrow{N}$ is a smooth function and $q\in{N}$ is a
                regular value, then $f^{-1}[\{q\}]$ is a smooth submanifold of
                $M$. The sphere $\mathbb{S}^{N}$ is the pre-image of 1 under
                the mapping $f:\mathbb{R}^{N+1}\rightarrow\mathbb{R}$ defined
                by $f(\mathbf{x})=||\mathbf{x}||^{2}$. The differential
                pushforward for all $\mathbf{x}\in{f}^{-1}[\{1\}]$ is:
                \begin{equation}
                    \textrm{d}f_{\mathbf{x}}(v)(\phi)
                        =v(\phi\circ{f})
                        =\sum_{n=0}^{N}a_{n}\frac{\partial}{\partial{x}_{n}}
                            \Big(\phi\circ{f}\Big)
                \end{equation}
                where $\phi:\mathbb{R}\rightarrow\mathbb{R}$ is smooth.
        \end{itemize}
    \end{frame}
    \begin{frame}{Problem 5 (cont.)}
        \begin{itemize}
            \item Since $||\mathbf{x}||=1$, one of the components $x_{n}$ is
                non-zero, meaning $a_{n}$ is non-zero, so
                $\textrm{d}f_{\mathbf{x}}$ is not the zero map. Since
                $\mathbb{R}$ is one dimensional, $\textrm{d}f_{\mathbf{x}}$ is
                surjective. Hence, $\mathbb{S}^{N}$ is the pre-image of a
                regular value, making it a smooth submanifold. It is orientable
                because it can be covered by two charts via stereographic
                projection.
        \end{itemize}
    \end{frame}
    \begin{frame}{Problem 6a}
        What is a vector field?
        \begin{itemize}
            \item A vector field is a smooth section $X:M\rightarrow{TM}$ of a
                smooth manifold to its tangent bundle. That is, a smooth
                function $X:M\rightarrow{TM}$ such that for all $p\in{M}$ we
                have $p=\textrm{proj}_{M}\big(f(p)\big)$ where
                $\textrm{proj}_{M}\big((x,v)\big)=x$ for all
                $(x,v)\in{TM}$. Equivalently it is an assigment to each
                $p\in{M}$ a tangent vector $X_{p}\in{T}_{p}M$ such that, given
                local coordinates $(\mathcal{U},\varphi)$, for all smooth
                functions $f\in{C}^{\infty}(M,\mathbb{R})$ the function
                $F:\mathcal{U}\rightarrow\mathbb{R}$ given by:
                \begin{equation}
                    F(p)=\sum_{n=0}^{N}a_{n}(p)
                        \frac{\partial{f}}{\partial\varphi_{n}}
                \end{equation}
                is smooth, where the $a_{n}(p)$ are what define $X_{p}$:
                \begin{equation}
                    X_{p}=\sum_{n=0}^{N-1}a_{n}(p)
                        \frac{\partial}{\partial\varphi_{n}}
                \end{equation}
        \end{itemize}
    \end{frame}
    \begin{frame}{Problem 6b}
        What is the Lie bracket of two vector fields?
        \begin{itemize}
            \item Given two vector fields $X,Y$, we may compose them. That is,
                since a vector field is something that takes in smooth functions
                $f\in{C}^{\infty}(M,\mathbb{R})$ and returns smooth functions
                $Yf\in{C}^{\infty}(M,\mathbb{R})$ we may compose
                $XYf=X(Yf)$. This need not be a vector field since it may
                involve higher order terms. However, $XY-YX$ will be a
                vector field since the higher order terms cancel, leaving only
                first order derivatives. This is the Lie bracket of $X$ with
                respect to $Y$:
                \begin{equation}
                    [X,Y]=XY-YX
                \end{equation}
        \end{itemize}
    \end{frame}
    \begin{frame}{Problem 7}
        What is a Lie group? Give an example.
        \begin{itemize}
            \item A Lie group is an ordered triple $(G,\mathcal{A},*)$ such
                that $(G,\mathcal{A})$ is a smooth manifold and
                $(G,*)$ is a group where the binary operation
                $*:G\times{G}\rightarrow{G}$ is a smooth function (with respect
                to the product smooth structure on $G\times{G}$) and such that
                $\nu:G\rightarrow{G}$ defined by $\nu(g)=g^{-1}$ is also smooth.
                The real numbers $\mathbb{R}$ with addition are a Lie group.
                The $N$ dimensional Euclidean space $\mathbb{R}^{N}$ with
                vector addition is a Lie group. Every finite group is a Lie
                group with the discrete topology (Zero dimensional Lie group).
                The unit circle $\mathbb{S}^{1}$ with
                $e^{i\theta}*e^{i\varphi}=e^{i(\theta+\varphi)}$ is a compact
                1 dimensional Lie group.
        \end{itemize}
    \end{frame}
    \begin{frame}{Problem 8}
        Discuss the Lie algebra of a Lie group.
        \begin{itemize}
            \item The Lie algebra of a Lie group is obtained by considering the
                tangent space of the identity $e\in{G}$ of the Lie group. The
                collection of all Left-Invariant vector fields $X$, each being
                uniquely determined by a single tangent vector
                $v\in{T}_{e}G$, is a Lie algebra under the Lie bracket
                operation. It suffices so show the Lie bracket of Left-Invarient
                vector fields is Left-Invariant. We have:
                \begin{align}
                    \textrm{d}L_{g}([X,Y])(f)
                        &=[X,Y](f\circ{L}_{g})\\
                        &=X\big(Y(f\circ{L}_{g})\big)-Y\big(X(f\circ{L}_{g})\big)\\
                        &=XYf-YXf=[X,Y](f)
                \end{align}
                Where the middle equality is because $X$ and $Y$ are left
                invariant.
        \end{itemize}
    \end{frame}
    \begin{frame}{Problem 9a}
        What is a Riemannian metric? Semi-Riemannian? Examples?
        \begin{itemize}
            \item A Riemannian metric is a function $g$ on $M$ such that for all
                $p\in{M}$ $g_{p}$ is a symmetric bilinear form that is
                positive-definite on $T_{p}M$ and such that for all smooth
                vector fields $X,Y$ on $M$, the function $f:M\rightarrow\mathbb{R}$
                defined by:
                \begin{equation}
                    g_{p}(X_{p},Y_{p})
                \end{equation}
                is smooth. A semi-Riemannian metric is a similar function but
                instead of positive-definite we only require the function $g$
                be \textit{non-degenerate}. That is, given $p\in{M}$ and any
                non-zero $v\in{T}_{p}M$, there is a tangent vector
                $w\in{T}_{p}M$ such that $g_{p}(v,w)\ne{0}$. Examples are
                $\mathbb{R}^{N}$ with the standard dot product, the Minkowski
                spacetime $M^{N,1}=\mathbb{R}^{N+1}$ with the metric
                $g=\sum\textrm{d}x_{n}^{2}-\textrm{d}t^{2}$.
        \end{itemize}
    \end{frame}
    \begin{frame}{Problem 9b}
        What is the induced Riemannian metric? Can Semi-Riemannian metric
        by induced by smooth embeddings?
        \begin{itemize}
            \item Given a smooth embedding $f:X\rightarrow{Y}$ of a smooth
            manifold $(X,\mathcal{A}_{X})$ into a Riemannian manifold
            $(Y,\mathcal{A}_{Y},g)$, there is an induced metric $\tilde{g}$ on
            $(X,\mathcal{A}_{Y})$ defined by:
            \begin{equation}
                \tilde{g}_{x}(u,v)=
                    g_{f(x)}\big(\textrm{d}f_{x}(u),\textrm{d}f_{x}(v)\big)
            \end{equation}
            That is, we pullback the metric from $Y$ to $X$. This will be
            positive-definite, bilinear, and symmetric. Moreover, since the
            pushforward of a smooth vector field is a smooth vector field,
            $\tilde{g}$ is smooth since $g$ is smooth. This trick may not
            work for Semi-Riemannian manifold. The induced ``metric`` may be
            degenerate. Take the Minkowsky space $M^{1,1}$. The embedded
            submanifold $\{(x,y)\in\mathbb{R}^{2}\;|\;x=y\}$ has the problem
            that the pullback metric is always zero.
        \end{itemize}
    \end{frame}
    \begin{frame}{Problem 10}
        What is a Left-Invariant metric on a Lie Group?
        \begin{itemize}
            \item A Left-Invariant metric on a Lie group is a metric $g$ such
                that left translation is an isometry of $G$ to itself.
        \end{itemize}
    \end{frame}
    \begin{frame}{Problem 11}
        What is a Levi-Civita connection of a Riemannian manifold?
        \begin{itemize}
            \item A Levi-Civita connection is a function
                $\nabla:\mathfrak{X}(M)\times\mathfrak{X}(M)\rightarrow\mathfrak{X}(M)$
                where $\mathfrak{X}(M)$ is the set of smooth vector fields on $M$
                such that:
                \begin{enumerate}
                    \item $\nabla$ is bilinear.
                    \item $\nabla$ is $C^{\infty}(M,\mathbb{R})$ linear in the
                        first coordinate.
                        \begin{equation}
                            \nabla_{fX}Y=f\nabla_{X}Y
                        \end{equation}
                    \item $\nabla$ is Liebnizean in the second coordinate:
                        \begin{equation}
                            \nabla_{X}fY=(Xf)Y+f\nabla_{X}Y
                        \end{equation}
                    \item $\nabla$ is torsion free ($\nabla_{X}Y-\nabla_{Y}X=[X,Y]$).
                    \item $\nabla$ is compatible with $g$:
                        \begin{equation}
                            Xg(Y,Z)=g(\nabla_{X}Y,Z)+g(Y,\nabla_{X}Z)
                        \end{equation}
                \end{enumerate}
        \end{itemize}
    \end{frame}
    \begin{frame}{Problem 12}
        What are the Christoffel Symbols?
        \begin{itemize}
            \item Given a smooth manifold $(M,\mathcal{A})$ and an affine
                connection $\nabla$ (not necessarily Levi-Civita connection),
                in a chart $(\mathcal{U},\varphi)$ we may look at the coordinate
                vector fields $\frac{\partial}{\partial\varphi_{n}}$. The
                connection takes in two vectors fields and returns a vector
                field. Such a vector field must be spanned by the coordinate
                vector fields, so in particular if we input
                $\frac{\partial}{\partial\varphi_{m}}$ and
                $\frac{\partial}{\partial\varphi_{n}}$ we get:
                \begin{equation}
                    \nabla_{\frac{\partial}{\partial\varphi_{m}}}
                        \frac{\partial}{\partial\varphi_{n}}
                    =\sum_{k=0}^{N-1}a_{k}\frac{\partial}{\partial\varphi_{k}}
                \end{equation}
                The $a_{k}$ depend on the coordinate vector fields, so we may
                write:
                \begin{equation}
                    \nabla_{\frac{\partial}{\partial\varphi_{m}}}
                        \frac{\partial}{\partial\varphi_{n}}
                    =\sum_{k=0}^{N-1}\Gamma_{m,n}^{k}
                        \frac{\partial}{\partial\varphi_{k}}
                \end{equation}
                and call $\Gamma_{m,n}^{k}$ the Christoffel symbols of
                $\nabla$ in the chart $(\mathcal{U},\varphi)$.
        \end{itemize}
    \end{frame}
\end{document}
