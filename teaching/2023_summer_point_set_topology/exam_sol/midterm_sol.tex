%-----------------------------------LICENSE------------------------------------%
%   This file is part of Mathematics-and-Physics.                              %
%                                                                              %
%   Mathematics-and-Physics is free software: you can redistribute it and/or   %
%   modify it under the terms of the GNU General Public License as             %
%   published by the Free Software Foundation, either version 3 of the         %
%   License, or (at your option) any later version.                            %
%                                                                              %
%   Mathematics-and-Physics is distributed in the hope that it will be useful, %
%   but WITHOUT ANY WARRANTY; without even the implied warranty of             %
%   MERCHANTABILITY or FITNESS FOR A PARTICULAR PURPOSE.  See the              %
%   GNU General Public License for more details.                               %
%                                                                              %
%   You should have received a copy of the GNU General Public License along    %
%   with Mathematics-and-Physics.  If not, see <https://www.gnu.org/licenses/>.%
%------------------------------------------------------------------------------%
\documentclass{article}
\usepackage{graphicx}                           % Needed for figures.
\usepackage{amsmath}                            % Needed for align.
\usepackage{amssymb}                            % Needed for mathbb.
\usepackage{amsthm}                             % For the theorem environment.
\usepackage{xcolor}                             % Coloring text.

%------------------------Theorem Styles-------------------------%

% Define theorem style for default spacing and normal font.
\newtheoremstyle{normal}
    {\topsep}               % Amount of space above the theorem.
    {\topsep}               % Amount of space below the theorem.
    {}                      % Font used for body of theorem.
    {}                      % Measure of space to indent.
    {\bfseries}             % Font of the header of the theorem.
    {}                      % Punctuation between head and body.
    {.5em}                  % Space after theorem head.
    {}

% Define default environments.
\theoremstyle{normal}
\newtheorem{problem}{Problem}

\title{Point-Set Topology: Midterm}
\date{Summer 2023}

% No indent and no paragraph skip.
\setlength{\parindent}{0em}
\setlength{\parskip}{0em}

\begin{document}
    \maketitle
    \color{blue}
    \begin{problem}[\textbf{Set Theory and Logic}]
        \par\hfill\par
        A Boolean algebra is an algebraic system consisting of a set $A$
        with two operations $\star$ and $\circ$. It models the logical
        operators $\land$ and $\lor$, as well as set arithmetic with
        $\cap$ and $\cup$. One of the properties of a Boolean algebra is the
        \textit{absorption laws}:
        \begin{align}
            a\star(a\circ{b})&=a\\
            a\circ(a\star{b})&=a
        \end{align}
        Of equal importance are the \textit{idempotence laws}:
        \begin{align}
            a\star{a}&=a\\
            a\circ{a}&=a
        \end{align}
        Lastly there must be neutral elements
        $e_{\star}$ and $e_{\circ}$ with the property that:
        \begin{align}
            a\star{e}_{\star}&=a\\
            a\circ{e}_{\circ}&=a
        \end{align}
        Let $X$ be a set. Set $A=\mathcal{P}(X)$, $\star=\cap$, and
        $\circ=\cup$.
        \begin{itemize}
            \item (4 Points) Prove that the absorption laws are satisfied.
            \item (3 Points) Prove that the idempotent laws are satisfied.
            \item (3 Points) Prove that there are neutral elements
                $e_{\star}$ and $e_{\circ}$ in $A$. What are they?
        \end{itemize}
    \end{problem}
    \color{black}
    \begin{proof}[Solution]
        Let's start with the absorption laws. Suppose
        $A,B\subseteq{X}$ and let $x\in{A}\cup(A\cap{B})$. Then
        $x\in{A}$ or $x\in{A}\cap{B}$. If $x\in{A}$ we're done, otherwise
        suppose $x\in{A}\cap{B}$. Then $x\in{A}$ and $x\in{B}$, and hence
        $x\in{A}$. That is, $x\in{A}\cup(A\cap{B})$ implies $x\in{A}$.
        Reversing this, suppose $x\in{A}$. Then, for any set $C$,
        $x\in{A}\cup{C}$ is true, and hence
        $x\in{A}\cup(A\cap{B})$ is true. We have proven that
        $A\subseteq{A}\cup(A\cap{B})$ and that
        $A\cup(A\cap{B})\subseteq{A}$, and therefore the sets are equal.
        \par\hfill\par
        We may similarly prove that $A\cap(A\cup{B})=A$.
        For any set $C$, if $x\in{A}\cap{C}$, then $x\in{A}$. So in particular,
        if $x\in{A}\cap(A\cup{B})$, then $x\in{A}$, and hence
        $A\cap(A\cup{B})\subseteq{A}$. Reversing this, let $x\in{A}$.
        Then $x\in{A}$ or $x\in{B}$ is true, since $x\in{A}$. But then
        $x\in{A}$ and $x\in{A}\cup{B}$ is true, and hence
        $x\in{A}\cap(A\cup{B})$. Thus, $A\subseteq{A}\cap(A\cup{B})$. The two
        sets are therefore equal.
        \par\hfill\par
        The idempotence laws follow from our use of the words \textit{and} and
        \textit{or}. If $x\in{A}$ is true, then $x\in{A}$ \textit{and} $x\in{A}$
        is true. Hence $A\subseteq{A}\cap{A}$. If $x\in{A}\cap{A}$, then
        $x\in{A}$ \textit{and} $x\in{A}$, and so we might rightly conclude
        that $x\in{A}$. Hence $A\cap{A}\subseteq{A}$. Therefore
        $A=A\cap{A}$. For unions, $A\subseteq{A}\cup{B}$ for any set $B$,
        so we may conclude $A\subseteq{A}\cup{A}$. In the other direction,
        let $x\in{A}\cup{A}$. Then $x\in{A}$ \textit{or} $x\in{A}$. Since
        $A=A$, we may conclude that $x\in{A}$ is a true statement. That is,
        $A\cup{A}\subseteq{A}$. So the idempotence laws are satisfied.
        \par\hfill\par
        Lastly, the identities. $\emptyset$ acts as the identity for unions.
        Given any $A\subseteq{X}$ we have $A\cup\emptyset=A$. The identity for
        intersections is the entire set $X$. Given $A\subseteq{X}$ we have
        $A\cap{X}=A$. So the neutral elements do indeed exist in set arithmetic.
    \end{proof}
    \newpage
    \color{blue}
    \begin{problem}[\textbf{Metric Spaces}]
        \par\hfill\par
        A pseudo-metric on a set $X$ is a function
        $\rho:X\times{X}\rightarrow\mathbb{R}$ such that:
        \begin{align}
            \rho(x,\,y)&\geq{0}\tag{Positivity}\\
            \rho(x,\,x)&=0\tag{Definite}\\
            \rho(x,\,y)&=\rho(y,\,x)\tag{Symmetry}\\
            \rho(x,\,z)&\leq\rho(x,\,y)+\rho(y,\,z)
                \tag{Triangle-Inequality}
        \end{align}
        Note $\rho(x,\,y)=0$ need not imply $x=y$. Open balls can be defined
        just like metric spaces:
        \begin{equation}
            B_{r}^{(X,\,\rho)}(x)=\{\,y\in{X}\;|\;\rho(x,\,y)<r\,\}
        \end{equation}
        And openness can be defined just like metric spaces.
        \begin{itemize}
            \item (5 Points) Let $\tau_{\rho}$ be the set of all
                pseudo-metrically open subsets of $X$. Prove $\tau_{\rho}$ is
                a topology on $X$.
            \item (5 Points) Prove that if $(X,\,\tau_{\rho})$ is Hausdorff,
                then $\rho$ is actually a metric, not just a pseudo-metric.
        \end{itemize}
    \end{problem}
    \color{black}
    \begin{proof}[Solution]
        We must prove the empty set is open, as is the whole space. Furthermore
        that $\tau_{\rho}$ is closed to arbitrary unions and the intersection
        of two elements.
        \begin{itemize}
            \item The empty set is open. If not then there exists an
                $x\in\emptyset$ such that for all $r>0$ the open ball
                $B_{r}^{(X,\,\rho)}(x)$ is not a subset of $\emptyset$. But
                $x\in\emptyset$ is a false statement, so we have a
                contradiction. The empty set is therefore open.
            \item The entire space is open. Given $x\in{X}$ choose
                $r=1$. The set $B_{r}^{(X,\,\rho})(x)$ is, by definition, a
                subset of $X$, proving that for any $x\in{X}$ there is an open
                ball about the point contained entirely inside of $X$. This is
                precisely the criterion for being open, so $X$ is open.
            \item The arbitrary union of open sets is open. Let
                $\mathcal{O}$ be a collection of open sets. If every element of
                $\mathcal{O}$ is empty, or if $\mathcal{O}$ is empty itself,
                the union $\bigcup\mathcal{O}$ is empty, and we've proven this
                is open. Suppose $x\in\bigcup\mathcal{O}$. Then there is some
                $\mathcal{U}\in\mathcal{O}$ such that $x\in\mathcal{U}$, by
                definition of union. But $\mathcal{U}$ is open, so there is
                some $r>0$ such that
                $B_{r}^{(X,\,\rho)}(x)\subseteq\mathcal{U}$. But
                $\mathcal{U}\subseteq\bigcup\mathcal{O}$ since
                $\mathcal{U}\in\mathcal{O}$, and hence
                $B_{r}^{(X,\,\rho)}(x)\subseteq\bigcup\mathcal{O}$. So
                $\bigcup\mathcal{O}$ is open.
            \item The intersection of two open sets is open. If the intersection
                is empty, we are done. Otherwise let $\mathcal{U}$ and
                $\mathcal{V}$ be open sets and let
                $x\in\mathcal{U}\cap\mathcal{V}$. Since $x\in\mathcal{U}$ and
                $\mathcal{U}$ is open, there is an $r_{0}>0$ such that
                $B_{r_{0}}^{(X,\,\rho)}(x)\subseteq\mathcal{U}$. Since
                $\mathcal{V}$ is open, there is an $r_{1}>0$ such that
                $B_{r_{1}}^{(X,\,\rho)}(x)\subseteq\mathcal{V}$. Let
                $r=\textrm{min}(r_{0},\,r_{1})$. Then
                $B_{r}^{(X,\,\rho)}(x)\subseteq\mathcal{U}\cap\mathcal{V}$, and
                hence $\mathcal{U}\cap\mathcal{V}$ is open.
        \end{itemize}
        The difference between a metric and pseudo-metric played no role here.
        The proof that a metric induces a topology is identical to the proof
        that a pseudo-metric induces a topology.
        \par\hfill\par
        Suppose the topology $\tau_{\rho}$ induced by the pseudo-metric
        $\rho$ is Hausdorff. That is, suppose $(X,\,\tau_{\rho})$ is a
        Hausdorff topological space. Let's prove that $\rho$ is a metric. The
        only feature lacking is that $\rho(x,\,y)=0$ implies $x=y$. Suppose not.
        That is, suppose $x,y\in{X}$ are distinct points, $x\ne{y}$, and yet
        $\rho(x,\,y)=0$. Since $x$ and $y$ are distinct, and since
        $(X,\,\tau_{\rho})$ is a Hausdorff space, there are disjoint open sets
        $\mathcal{U}$ and $\mathcal{V}$ separating $x$ and $y$. That is, open
        sets $\mathcal{U},\mathcal{V}\in\tau_{\rho}$ such that
        $x\in\mathcal{U}$, $y\in\mathcal{V}$, and
        $\mathcal{U}\cap\mathcal{V}=\emptyset$. But if $x\in\mathcal{U}$ and
        $\mathcal{U}$ is open, there is an $r>0$ such that
        $B_{r}^{(X,\,\rho)}(x)\subseteq\mathcal{U}$. That is, an $r>0$ such
        that for all $z\in{X}$, $d(x,\,z)<r$ implies $z\in\mathcal{U}$. But
        $d(x,\,y)=0$ and $0<r$, since $r$ is positive. Therefore
        $y\in\mathcal{U}$. But $y\in\mathcal{V}$ and
        $\mathcal{U}\cap\mathcal{V}=\emptyset$, a contradiction. Hence
        $\rho$ is a valid metric on $X$.
    \end{proof}
    \newpage
    \color{blue}
    \begin{problem}[\textbf{Compactness}]
        \par\hfill\par
        For a metric space, compactness means every sequence has a convergent
        subsequence. It also means every open cover has a finite subcover, or
        that the space is complete and totally bounded. You may use any of
        these.
        \par\hfill\par
        For a topological space compact means every open cover has a finite
        subcover. Sequences are not enough to describe general compactness, and
        topological spaces do not have a notion of completeness or being
        totally bounded. So only open covers work in the more general setting.
        \begin{itemize}
            \item (5 Points) Let $(X,\,d)$ be a compact metric space. Prove
                that every infinite subset $A\subseteq{X}$ has an accumulation
                point. That is, a point $x\in{X}$ such that for all
                $\varepsilon>0$ there is a $y\in{A}$, $y\ne{x}$, such that
                $d(x,\,y)<\varepsilon$.
            \item (5 Points) Let $(X,\,\leq)$ be a totally ordered set, and let
                $\tau_{<}$ be the order topology induced by the total order.
                That is, $\tau_{<}$ is the topology generated by open intervals,
                left-rays, and right-rays. Prove that if $(X,\,\tau_{<})$ is
                compact (every open cover has a finite subcover), then
                $(X,\,\leq)$ has a supremum (a greatest element).
        \end{itemize}
    \end{problem}
    \color{black}
    \begin{proof}[Solution]
        Given an infinite set $B\subseteq{X}$ there is always a countably
        infinitd subset $A\subseteq{B}$ (assuming the axiom of countable choice,
        which we said at the beginning of the term is not too contraversial).
        Since $A$ is countably infinite, there is a bijection
        $a:\mathbb{N}\rightarrow{A}$ (countable means there is a surjection
        $a:\mathbb{N}\rightarrow{A}$, countably infinite means there is also a
        bijection). That is, $a:\mathbb{R}\rightarrow{X}$ is a sequence of
        points in our metric space $(X,\,d)$. But $(X,\,d)$ is compact, so
        there is a convergent subsequence $a_{k}$. Let $x\in{X}$ be the
        limit. Let $\varepsilon>0$. Since $a_{k_{n}}\rightarrow{x}$ there is
        an $N_{0}$ such that $n>N_{0}$ implies $d(x,\,a_{k_{n}})<\varepsilon$.
        If $x=a_{n}$ for some $n\in\mathbb{N}$, let $N_{1}=n$. Otherwise set
        $N_{1}=0$. Choose $N=\textrm{max}(N_{0},\,N_{1})+1$. Then
        $d(x,\,a_{k_{N}})<\varepsilon$, since $N>N_{1}$, and also
        $x\ne{a}_{k_{N}}$ since $a:\mathbb{N}\rightarrow{A}$ is a bijection and
        $N\ne{N}_{1}$. Hence $x\in{X}$ is such that for all $\varepsilon>0$
        there is a $y\in{A}$, $y\ne{x}$, such that $d(x,\,y)<\varepsilon$. So
        $(X,\,d)$ is limit point compact (every infinite subset has an
        accumulation point).
        \par\hfill\par
        For the second part, suppose not. Suppose $(X,\,\leq)$ is a totally
        ordered set with no greatest element. That is, for all
        $x\in{X}$ there is a $y\in{X}$ such that $x<y$. The set
        $\mathcal{O}$ defined by:
        \begin{equation}
            \mathcal{O}=\{\,(-\infty,\,a)\subseteq{X}\;|\;a\in{X}\,\}
        \end{equation}
        is then an open cover. Each element is open since left-rays are open
        in the order topology. Moreover, it is an open cover. Given
        $x\in{X}$, chose $y\in{X}$ such that $x<y$. Such a $y$ exists by
        hypothesis since $X$ has no greatest element. Then
        $(-\infty,\,y)\in\mathcal{O}$ and $x\in(-\infty,\,y)$. But
        $(X,\,\tau_{<})$ is compact, so there is a finite subcover
        $\Delta$. The elements of $\Delta$ may be written:
        \begin{equation}
            \mathcal{U}_{n}=(-\infty,\,a_{n})
        \end{equation}
        for some $a_{n}\in{X}$. Let $y$ be the greatest element of all the
        $a_{n}$. Since there are only finitely many such elements, a greatest
        value does indeed exist. Then $y\notin\mathcal{U}_{n}$ for all
        $\mathcal{U}_{n}\in\Delta$. For if $y\in\mathcal{U}_{n}$, then
        $y<a_{n}$, but $y$ was chosen to be the greatest of all of the $a_{n}$.
        This contradicts the fact that $\Delta$ is an open cover of
        $(X,\,\tau_{<})$. So $(X,\,\leq)$ has a greatest element.
    \end{proof}
    \newpage
    \color{blue}
    \begin{problem}[\textbf{Quotient Spaces}]
        \par\hfill\par
        Let $(X,\,\tau)$ be a topological space, and $R$ an equivalence
        relation on $X$.
        \begin{itemize}
            \item (5 Points) Prove that $(X/R,\,\tau_{X/R})$ is Fr\'{e}chet
                if and only if for all $x\in{X}$ the equivalence class
                $[x]\subseteq{X}$ is a closed subset.
            \item (5 Points) Show by example that $(X/R,\,\tau_{X/R})$ need not
                be a Fr\'{e}chet space, even if $(X,\,\tau)$ is.
        \end{itemize}
    \end{problem}
    \color{black}
    \begin{proof}[Solution]
        First we note that a topological space has the Fr\'{e}chet property
        \textit{if and only if} all of the singleton sets are closed. This was
        proved in class and so may be used freely.
        \par\hfill\par
        Suppose $(X/R,\,\tau_{X/R})$ is a Fr\'{e}chet space. Then all of the
        singleton sets are closed. That is, for all $[x]$ the set
        $\{\,[x]\,\}$ is closed as a subset of $X/R$. But the quotient map is
        continuous, so the pre-image of a closed set is closed, meaning
        $q^{-1}[\{\,[x]\,\}]$ is closed. But by the definition of the quotient
        map, this is precisely the set $[x]\subseteq{X}$. Hence equivalence
        classes are closed subsets of $X$.
        \par\hfill\par
        In the other direction, suppose equivalence classes are closed. The
        quotient topology $\tau_{X/R}$ is the final topology with respect to
        the quotient map $q:X\rightarrow{X}/R$. That is a set is open if and
        only if it's pre-image is open. Since closed sets are the complement of
        open sets, a set is closed in $X/R$ if and only if it's pre-image is
        closed. The singleton set $\{\,[x]\,\}$ has the property that
        $q^{-1}[\{\,[x]\,\}]=[x]$. But $[x]\subseteq{X}$ is assumed to be
        closed in $X$, and therefore $\{\,[x]\,\}$ is closed in $X/R$. That is,
        all of the singleton sets in the quotient space are closed, meaning
        $(X/R,\,\tau_{X/R})$ is a Fr\'{e}chet topological space.
        \par\hfill\par
        If we consider $\mathbb{R}$ with the usual topology, it is metrizable,
        hence Hausdorff, and therefore the real line is also a Fr\'{e}chet
        space. The quotient space $\mathbb{R}/\mathbb{Q}$ is not Fr\'{e}chet
        since $\mathbb{Q}\subseteq\mathbb{R}$ is not a closed subset. So the
        quotient of a Fr\'{e}chet topological space need not be Fr\'{e}chet.
    \end{proof}
    \newpage
    \color{blue}
\end{document}
