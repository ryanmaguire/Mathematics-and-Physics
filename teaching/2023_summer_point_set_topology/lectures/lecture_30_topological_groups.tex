%-----------------------------------LICENSE------------------------------------%
%   This file is part of Mathematics-and-Physics.                              %
%                                                                              %
%   Mathematics-and-Physics is free software: you can redistribute it and/or   %
%   modify it under the terms of the GNU General Public License as             %
%   published by the Free Software Foundation, either version 3 of the         %
%   License, or (at your option) any later version.                            %
%                                                                              %
%   Mathematics-and-Physics is distributed in the hope that it will be useful, %
%   but WITHOUT ANY WARRANTY; without even the implied warranty of             %
%   MERCHANTABILITY or FITNESS FOR A PARTICULAR PURPOSE.  See the              %
%   GNU General Public License for more details.                               %
%                                                                              %
%   You should have received a copy of the GNU General Public License along    %
%   with Mathematics-and-Physics.  If not, see <https://www.gnu.org/licenses/>.%
%------------------------------------------------------------------------------%
\documentclass{article}
\usepackage{amsmath}                            % Needed for align.
\usepackage{amssymb}                            % Needed for mathbb.
\usepackage{amsthm}                             % For the theorem environment.

%------------------------Theorem Styles-------------------------%
\theoremstyle{plain}
\newtheorem{theorem}{Theorem}[section]

% Define theorem style for default spacing and normal font.
\newtheoremstyle{normal}
    {\topsep}               % Amount of space above the theorem.
    {\topsep}               % Amount of space below the theorem.
    {}                      % Font used for body of theorem.
    {}                      % Measure of space to indent.
    {\bfseries}             % Font of the header of the theorem.
    {}                      % Punctuation between head and body.
    {.5em}                  % Space after theorem head.
    {}

% Define default environments.
\theoremstyle{normal}
\newtheorem{examplex}{Example}[section]
\newtheorem{definitionx}{Definition}[section]

\newenvironment{example}{%
    \pushQED{\qed}\renewcommand{\qedsymbol}{$\blacksquare$}\examplex%
}{%
    \popQED\endexamplex%
}

\newenvironment{definition}{%
    \pushQED{\qed}\renewcommand{\qedsymbol}{$\blacksquare$}\definitionx%
}{%
    \popQED\enddefinitionx%
}

\title{Point-Set Topology: Lecture 30}
\author{Ryan Maguire}
\date{\today}

% No indent and no paragraph skip.
\setlength{\parindent}{0em}
\setlength{\parskip}{0em}

\begin{document}
    \maketitle
    \section{Topological Groups}
        Now we introduce some topology into our algebra.
        \begin{definition}[\textbf{Topological Group}]
            A topological group is an ordered triple $(X,\,\tau,\,*)$ where
            $(X,\,\tau)$ is a topological space and $(X,\,*)$ is a group
            such that the functions $m:X\times{X}\rightarrow{X}$ and
            $\eta:X\rightarrow{X}$ defined by:
            \begin{align}
                m(x,\,y)&=x*y\\
                \eta(x)=x^{-1}
            \end{align}
            are continuous (here $X\times{X}$ is given the product topology).
            That is, the group operations are continuous functions.
        \end{definition}
        \begin{example}
            The real line with addition is a topological group. The addition
            of real numbers is indeed a continuous operation.
        \end{example}
        \begin{example}
            More generally, $\mathbb{R}^{n}$ as a vector space with vector
            addition becomes a topological group when endowed with the standard
            Euclidean topology.
        \end{example}
        \begin{example}
            The circle $\mathbb{S}^{1}$ with the subspace topology and the
            \textit{rotation operation} is a topological group. That is,
            Given points $e^{i\theta},\,e^{i\phi}\in\mathbb{S}^{1}$, we define
            $e^{i\theta}*e^{i\phi}=e^{i(\theta+\phi)}$. This operation is
            continuous with respect to the subspace topology the circle
            inherits from $\mathbb{R}^{2}$, giving us a topological group.
        \end{example}
        \begin{example}
            The integers $\mathbb{Z}$ with the subspace topology from
            $\mathbb{R}$ and addition form a topological group. Note that the
            subspace topology on $\mathbb{Z}$ is also the discrete topology.
        \end{example}
        \begin{example}
            More generally, if $(X,\,*)$ is any group, then
            $(X,\,\mathcal{P}(X),\,*)$ is a topological group. The product
            topology on $X\times{X}$ is also the discrete topology, and hence
            \textit{any} function $f:X\times{X}\rightarrow{X}$ is continuous.
            Similarly, any function $g:X\rightarrow{X}$ is continuous. So
            in particular the multiplication and inversion operations are
            continuous and $(X,\,\mathcal{P}(X),\,*)$ is a topological group.
        \end{example}
        \begin{example}
            Going the other way, if $(X,\,*)$ is any group, and if
            $\tau=\{\,\emptyset,\,X\,\}$ is the indiscrete topology, then
            $(X,\,\tau,\,*)$ is a topological group. Since $\tau$ is the
            indiscrete topology, any function into $X$ is continuous,
            so in particular multiplication and inversion are continuous.
        \end{example}
        \begin{example}
            If $X=\mathbb{R}$ and $\tau=\{\,\emptyset,\,\mathbb{R}\,\}$ is the
            indiscrete topology, then by the previous example
            $(\mathbb{R},\,\tau,\,+)$ is a topological group. Note that it is
            a non-Hausdorff topological group. Topological groups need not
            satisfy any of the separation properties.
        \end{example}
        \begin{theorem}
            If $(X,\,\tau)$ is a topological space, and if $(X,\,*)$ is a
            group, then $(X,\,\tau,\,*)$ is a topological group if and only
            if the function $f:X\times{X}\rightarrow{X}$ defined by
            $f\big((x,\,y)\big)=x*y^{-1}$ is continuous.
        \end{theorem}
        \begin{proof}
            The function $f(x,\,y)=x*y^{-1}$ can be seen as a
            combination of multiplication and inversion. If $(X,\,\tau,\,*)$ is
            a topological group, then this function is continuous. In the
            other direction, if this function is continuous, then setting
            $x=e$, the identity, we have $f(e,\,y)=e*y^{-1}=y^{-1}$, and this
            is a continuous function of $y$, meaning inversion is continuous.
            But then $x*y=x*(y^{-1})^{-1}=f(x,\,y^{-1})$, which is
            the composition of continuous functions, so multiplication is
            continuous. Hence $(X,\,\tau,\,*)$ is a topological group.
        \end{proof}
        \begin{theorem}
            If $(X,\,\tau,\,*)$ is a topological group, and if $a\in{X}$,
            and if $L_{a}:X\rightarrow{X}$ is left-translation of $X$ by $a$,
            then $L_{a}$ is a homeomorphism.
        \end{theorem}
        \begin{proof}
            We have already proved that left-translation in a group is
            bijective. Let us show that it is continuous. But
            $L_{a}(x)=a*x$ is the restriction of
            $m:X\times{X}\rightarrow{X}$, defined by $m(x,\,y)=x*x$, to the
            subset $\{\,a\,\}\times{X}$. But the restriction of a continuous
            function to subspace is continuous, and hence
            $L_{a}:X\rightarrow{X}$ is continuous. The inverse function
            is given by $L_{a}^{-1}=L_{a^{-1}}$ since:
            \begin{align}
                (L_{a}\circ{L}_{a^{-1}})(x)
                &=L_{a}\big(L_{a^{-1}}(x)\big)\tag{Definition of Composition}\\
                &=L_{a}(a^{-1}*x)\tag{Definition of $L_{a^{-1}}$}\\
                &=a*(a^{-1}*x)\tag{Definition of $L_{a}$}\\
                &=(a*a^{-1})*x\tag{Associativity}\\
                &=e*x\tag{Inverse}\\
                &=x\tag{Identity}
            \end{align}
            And hence $L_{a}\circ{L}_{a^{-1}}$ is the identity function.
            Similarly, $L_{a^{-1}}\circ{L}_{a}$ is the identity. So the
            inverse of left-translation is another left-translation, which is
            continuous. Hence $L_{a}$ is a homeomorphism.
        \end{proof}
        Two immediate results are often of equal use.
        \begin{theorem}
            If $(X,\,\tau,\,*)$ is a topological group, if $a\in{X}$, and if
            $L_{a}:X\rightarrow{X}$ is left-translation by $a$, then
            $L_{a}$ is an open map.
        \end{theorem}
        \begin{proof}
            Left-translation is a homeomorphism, so it is also an open map.
        \end{proof}
        \begin{theorem}
            If $(X,\,\tau,\,*)$ is a topological group, if $a\in{X}$, and if
            $L_{a}:X\rightarrow{X}$ is left-translation by $a$, then
            $L_{a}$ is an closed map.
        \end{theorem}
        \begin{proof}
            Left-translation is a homeomorphism, so it is also an closed map.
        \end{proof}
    \section{Kolmogorov Quotients}
    \section{Major Results}
\end{document}
