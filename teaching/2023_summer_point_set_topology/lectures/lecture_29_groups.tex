%-----------------------------------LICENSE------------------------------------%
%   This file is part of Mathematics-and-Physics.                              %
%                                                                              %
%   Mathematics-and-Physics is free software: you can redistribute it and/or   %
%   modify it under the terms of the GNU General Public License as             %
%   published by the Free Software Foundation, either version 3 of the         %
%   License, or (at your option) any later version.                            %
%                                                                              %
%   Mathematics-and-Physics is distributed in the hope that it will be useful, %
%   but WITHOUT ANY WARRANTY; without even the implied warranty of             %
%   MERCHANTABILITY or FITNESS FOR A PARTICULAR PURPOSE.  See the              %
%   GNU General Public License for more details.                               %
%                                                                              %
%   You should have received a copy of the GNU General Public License along    %
%   with Mathematics-and-Physics.  If not, see <https://www.gnu.org/licenses/>.%
%------------------------------------------------------------------------------%
\documentclass{article}
\usepackage{amsmath}                            % Needed for align.
\usepackage{amssymb}                            % Needed for mathbb.
\usepackage{amsthm}                             % For the theorem environment.

%------------------------Theorem Styles-------------------------%
\theoremstyle{plain}
\newtheorem{theorem}{Theorem}[section]

% Define theorem style for default spacing and normal font.
\newtheoremstyle{normal}
    {\topsep}               % Amount of space above the theorem.
    {\topsep}               % Amount of space below the theorem.
    {}                      % Font used for body of theorem.
    {}                      % Measure of space to indent.
    {\bfseries}             % Font of the header of the theorem.
    {}                      % Punctuation between head and body.
    {.5em}                  % Space after theorem head.
    {}

% Define default environments.
\theoremstyle{normal}
\newtheorem{examplex}{Example}[section]
\newtheorem{definitionx}{Definition}[section]

\newenvironment{example}{%
    \pushQED{\qed}\renewcommand{\qedsymbol}{$\blacksquare$}\examplex%
}{%
    \popQED\endexamplex%
}

\newenvironment{definition}{%
    \pushQED{\qed}\renewcommand{\qedsymbol}{$\blacksquare$}\definitionx%
}{%
    \popQED\enddefinitionx%
}

\title{Point-Set Topology: Lecture 29}
\author{Ryan Maguire}
\date{\today}

% No indent and no paragraph skip.
\setlength{\parindent}{0em}
\setlength{\parskip}{0em}

\begin{document}
    \maketitle
    In algebraic topology one uses algebraic structures like groups and
    vector spaces to solve topological problems. This is usually done by
    \textit{associating} some algebraic object with a given topological space,
    and showing that homotopy equivalences or homeomorphisms preserve
    the nature of this algebraic device. We won't be exploring this route, it
    deserves its own course. Instead we'll dive into
    \textit{topological algebra}. Here we reverse the idea, attaching a
    topology to algebraic structures. In particular we'll discuss
    \textit{topological groups}, perhaps the simplest object of study in
    topological algebra. There are a few motivating examples for the study of
    topological groups.
    \begin{itemize}
        \item A \textit{real topological vector space} is a real vector space
            $(V,\,+)$ together with a topology $\tau$ that makes vector
            addition $(v,\,w)\mapsto{v+w}$ and scalar multiplication
            $(a,\,v)\mapsto{a}v$ continuous operations. The additive nature of
            vector addition yields an Abelian group, meaning every topological
            vector space has a canonical topological group associated to it.
        \item A \textit{Banach space} is a normed vector space (usually over
            $\mathbb{R}$ or $\mathbb{C}$) such that the metric induced by the
            norm yields a \textit{complete} metric space. Banach spaces are,
            in particular, topological vector spaces and the addition of
            vectors forms a topological group.
        \item A Lie group is a smooth manifold with a group operation that is
            \textit{smooth}. Lie groups are, in particular, topological groups.
    \end{itemize}
    There is no pre-requesite for algebra in this course, so we'll take the
    time to develop the basics of group theory.
    \section{Group Theory}
        There are a few competing views on how best to describe groups. Some
        say it is the study of symmetry. One may also view groups as
        combinatorial objects. I'll take the approach of
        \textit{generalized arithmetic}. The addition of integers and the
        multiplication of matrices provide two motivating examples of groups.
        Many of the theorems involving these two arithmetics need only a few
        common traits. Groups generalize these traits to abstract objects.
        \begin{definition}[\textbf{Group}]
            A group is an ordered pair $(G,\,*)$ where $G$ is a set and
            $*:G\times{G}\rightarrow{G}$ is a function (called a
            \textit{binary operation}) that is \textit{associative}, has an
            \textit{identity}, and contains \textit{inverse elements}. That is:
            \begin{enumerate}
                \item For $a,\,b,\,c\in{G}$ it is true that
                    $a*(b*c)=(a*b)*c$ (Associativity)
                \item There is an $e\in{G}$ such that $a*e=e*a=a$ for all
                    $a\in{G}$ (Identity)
                \item For all $a\in{G}$ there is a $b\in{G}$ such that
                    $a*b=b*a=e$ (Inverses)
            \end{enumerate}
        \end{definition}
        \begin{example}
            The integers $\mathbb{Z}$ with addition form a group. That is,
            $(\mathbb{Z},\,+)$ is a group. It is perhaps one of the simplest
            group structures. Addition is associative, zero serves as an
            identity, and for all $n\in\mathbb{Z}$, $-n$ is an additive inverse.
        \end{example}
        \begin{example}
            The real numbers with addition $(\mathbb{R},\,+)$ also form a group.
            Note that we may identify $(\mathbb{Z},\,+)$ as a \textit{subgroup}
            (defined soon) of the real numbers.
        \end{example}
        \begin{example}
            The positive real numbers $\mathbb{R}^{+}$, together with ordinary
            multiplication, form a group. That is,
            $(\mathbb{R}^{+},\,\cdot\,)$ is a group. Multiplication is indeed
            associative, and 1 serves as the identity. Since we have excluded
            0 from the set, for all $x\in\mathbb{R}^{+}$, $\frac{1}{x}$ is the
            inverse of $x$.
        \end{example}
        \begin{example}
            Let $GL_{n}(\mathbb{R})$ be the set of all $n\times{n}$ matrices
            of real numbers with non-zero determinant. That is, all matrices
            $A\in\mathbb{R}^{n\times{n}}$ such that $A^{-1}$ exists. This
            set, together with matrix multiplication, forms a group. This is
            the \textit{general linear group} of order $n$. Matrix
            multiplication is associative, and the identity matrix $I_{n}$
            serves as the identity. Since the set consists solely of invertible
            matrices, inverses exist in $GL_{n}(\mathbb{R})$. Note that, unlike
            the previous examples, matrix multiplication is \textit{not}
            commutative. That is, given $A,B\in{GL}_{n}(\mathbb{R})$, it is
            possible for $AB\ne{BA}$ to be true.
        \end{example}
        \textit{Commutative} groups are useful enough to deserve a name.
        \begin{definition}[\textbf{Abelian Group}]
            An Abelian group is a group $(G,\,*)$ such that for all $a,b\in{G}$
            it is true that $a*b=b*a$.
        \end{definition}
        We now take the time to explore the basic properties that all groups
        have in common. None of these are \textit{deep} theorems, and come
        straight from the definition.
        \begin{theorem}
            If $(G,\,*)$ is a group, and if $e,e'\in{G}$ are identities, then
            $e=e'$
        \end{theorem}
        \begin{proof}
            Since $e$ and $e'$ are identities we have:
            \begin{equation}
                e=e*e'=e'
            \end{equation}
            and hence $e=e'$.
        \end{proof}
        \begin{theorem}
            If $(G,\,*)$ is a group, if $a\in{G}$, and if $b$ and $b'$ are
            inverses of $a$, then $b=b'$.
        \end{theorem}
        \begin{proof}
            For let $e\in{G}$ be the unique identity. Then we have:
            \begin{align}
                b&=b*e\tag{Identity}\\
                &=b*(a*b')\tag{Inverse}\\
                &=(b*a)*b'\tag{Associativity}\\
                &=e*b'\tag{Inverse}\\
                &=b'\tag{Identity}
            \end{align}
            and hence $b=b'$.
        \end{proof}
        Because of this, given $a\in{G}$ we denote by $a^{-1}$ the
        \textit{unique} inverse of $a$.
        \begin{theorem}
            If $(G,\,*)$ is a group, and if $a,b\in{G}$, then
            $(a*b)^{-1}=b^{-1}*a^{-1}$.
        \end{theorem}
        \begin{proof}
            Since inverses are unique, we need only prove that
            $b^{-1}*a^{-1}$ is indeed an inverse of $a*b$. We have:
            \begin{align}
                (a*b)*(b^{-1}*a^{-1})&=
                    a*\big((b*b^{-1})*a^{-1}\big)\tag{Associativity}\\
                    &=a*(e*a^{-1})\tag{Inverse}\\
                    &=a*a^{-1}\tag{Identity}\\
                    &=e\tag{Inverse}
            \end{align}
            by the uniqueness of inverses, $(a*b)^{-1}=b^{-1}*a^{-1}$.
        \end{proof}
        \begin{theorem}
            If $(G,\,*)$ is a group, and if $a\in{G}$, then
            $(a^{-1})^{-1}=a$.
        \end{theorem}
        \begin{proof}
            We have that:
            \begin{align}
                a&=a*e\tag{Identity}\\
                &=a*\big(a^{-1}*(a^{-1})^{-1}\big)\tag{Inverse}\\
                &=(a*a^{-1})*(a^{-1})^{-1}\tag{Associativity}\\
                &=e*(a^{-1})^{-1}\tag{Inverse}\\
                &=(a^{-1})^{-1}\tag{Identity}
            \end{align}
            and hence $a=(a^{-1})^{-1}$.
        \end{proof}
        \begin{theorem}[\textbf{Left-Cancellation Law}]
            If $(G,\,*)$ is a group, if $a,b,c\in{G}$, and if
            $a*b=a*c$, then $b=c$.
        \end{theorem}
        \begin{proof}
            Letting $e\in{G}$ be the unique inverse, if $a*b=a*c$, then we have:
            \begin{align}
                b&=e*b\tag{Identity}\\
                &=(a^{-1}*a)*b\tag{Inverse}\\
                &=a^{-1}*(a*b)\tag{Associativity}\\
                &=a^{-1}*(a*c)\tag{Hypothesis}\\
                &=(a^{-1}*a)*c\tag{Associativity}\\
                &=e*c\tag{Inverse}\\
                &=c\tag{Identity}
            \end{align}
            so we conclude that $b=c$.
        \end{proof}
        We can mirror this argument to prove the right-cancellation law.
        \begin{theorem}[\textbf{Right-Cancellation Law}]
            If $(G,\,*)$ is a group, if $a,b,c\in{G}$, and if $b*a=c*a$, then
            $b=c$.
        \end{theorem}
        \begin{proof}
            Letting $e\in{G}$ be the unique inverse, if $b*a=c*a$, then we have:
            \begin{align}
                b&=b*e\tag{Identity}\\
                &=b*(a*a^{-1})\tag{Inverse}\\
                &=(b*a)*a^{-1}\tag{Associativity}\\
                &=(c*a)*a^{-1}\tag{Hypothesis}\\
                &=c*(a*a^{-1})\tag{Associativity}\\
                &=c*e\tag{Inverse}\\
                &=c\tag{Identity}
            \end{align}
            and hence $b=c$.
        \end{proof}
        \begin{theorem}
            If $(G,\,*)$ is a group, if $a,b\in{G}$, and if
            $a=a*b$, then $b=e$ where $e\in{G}$ is the unique identity element.
        \end{theorem}
        \begin{proof}
            We have that:
            \begin{align}
                a*b&=a\tag{Hypothesis}\\
                &=a*e\tag{Identity}
            \end{align}
            and hence $a*b=a*e$. By the left-cancellation law, $b=e$.
        \end{proof}
        \begin{theorem}
            If $(G,\,*)$ is a group, if $a,b\in{G}$, and if
            $a=b*a$, then $b=e$ where $e\in{G}$ is the unique identity element.
        \end{theorem}
        \begin{proof}
            We have that:
            \begin{align}
                b*a&=a\tag{Hypothesis}\\
                &=e*a\tag{Identity}
            \end{align}
            and hence $b*a=e*a$. By the right-cancellation law, $b=e$.
        \end{proof}
        \begin{definition}[\textbf{Subgroup}]
            A subgroup of a group $(G,\,*)$ is a subset $H\subseteq{G}$ such
            that the restriction of $*$ to $H\times{H}$ yields a group. That
            is, $(H,\,*_{H})$ is a group.
        \end{definition}
        \begin{theorem}
            If $(G,\,*)$ is a group, if $e\in{G}$ is the unique identity, and
            if $H\subseteq{G}$ is a subgroup, then $e\in{H}$ and $e$ is the
            identity of $(H,\,*_{H})$.
        \end{theorem}
        \begin{proof}
            Since $(H,\,*_{H})$ is a group, there is some identity element
            $e_{H}\in{H}$. But then $e_{H}*_{H}e_{H}=e_{H}$. But
            $*_{H}$ is just the restriction of $*$ to $H$, so
            $e_{H}*e_{H}=e_{H}$. By the previous theorem, $e_{H}=e$, so
            $e\in{H}$ and $e$ is the identity of $(H,\,*_{H})$.
        \end{proof}
        \begin{theorem}
            If $(G,\,*)$ is a group, if $H\subseteq{G}$ is a subgroup, and
            if $a\in{H}$, then $a^{-1}\in{H}$.
        \end{theorem}
        \begin{proof}
            Since $(H,\,*_{H})$ is a group, and since $a\in{H}$, there is an
            inverse element $a_{H}^{-1}$ such that
            $a*a_{H}^{-1}=a_{H}^{-1}*a=e_{H}$. But by the previous theorem,
            $e_{H}=e$, so $a*a_{H}^{-1}=a_{H}^{-1}*a=e$. But $(G,\,*)$ is a
            group and inverses are unique, so $a_{H}^{-1}=a^{-1}$. Hence
            $a^{-1}\in{H}$.
        \end{proof}
        \begin{theorem}
            If $(G,\,*)$ is a group, and if $H\subseteq{G}$, then $H$ is a
            subgroup if and only if $H$ is non-empty, for all $a\in{H}$
            it is true that $a^{-1}\in{H}$, and for all $a,b\in{H}$ it is
            true that $a*b\in{H}$.
        \end{theorem}
        \begin{proof}
            By the previous two theorems, if $H\subseteq{G}$ is a subgroup,
            then it is closed to group multiplication and inversion. It also
            non-empty since $e\in{H}$. In the reverse direction, suppose
            $H\subseteq{G}$ is non-empty and
            closed to inversion and multiplication. We need
            only show that $H$ has an identity element (it has inverses, and
            the group operation is associative, so the restriction to $H$ is
            associative as well). Since $H$ is non-empty, there is some
            $a\in{H}$. But $H$ is closed to inversion, so $a^{-1}\in{H}$. But
            $H$ is also closed under multiplication, meaning $a*a^{-1}\in{H}$.
            But $a*a^{-1}=e$, and hence $e\in{H}$. That is, $H$ is a subgroup.
        \end{proof}
        For topology and geometry, two of the most important operations that
        come from groups are \textit{left-translation} and
        \textit{right-translation}. These are defined as follows.
        \begin{definition}[\textbf{Left-Translation of a Group}]
            Left-translation of a group $(G,\,*)$ be an element $a\in{G}$ is
            the function $L_{a}:G\rightarrow{G}$ defined by:
            \begin{equation}
                L_{a}(x)=a*x
            \end{equation}
            That is, each element is translated on the left by $a$.
        \end{definition}
        Right-translation is similarly defined.
        \begin{definition}[\textbf{Right-Translation of a Group}]
            Right-translation of a group $(G,\,*)$ be an element $a\in{G}$ is
            the function $R_{a}:G\rightarrow{G}$ defined by:
            \begin{equation}
                R_{a}(x)=x*a
            \end{equation}
            That is, each element is translated on the right by $a$.
        \end{definition}
        Left-translation and right-translation by the same element may yield
        different functions if $(G,\,*)$ is not Abelian (commutative).
        Regardless, translation is always a bijection.
        \begin{theorem}
            If $(G,\,*)$ is a group, if $a\in{G}$, and if
            $L_{a}:G\rightarrow{G}$ is left-translation by $a$, then
            $L_{a}$ is bijective.
        \end{theorem}
        \begin{proof}
            First, $L_{a}$ is injective. For let $x,y\in{G}$ and suppose
            $L_{a}(x)=L_{a}(y)$. By the definition of left-translation this
            means $a*x=a*y$. By the left-cancellation law, $x=y$, and hence
            $L_{a}$ is injective. It is also surjective. For let
            $y\in{G}$ and let $x=a^{-1}*y$. Then we have:
            \begin{align}
                L_{a}(x)&=a*x\tag{Definition}\\
                &=a*(a^{-1}*y)\tag{Substitution}\\
                &=(a*a^{-1})*y\tag{Associativity}\\
                &=e*y\tag{Inverse}\\
                &=y\tag{Identity}
            \end{align}
            and hence $L_{a}(x)=y$, so $L_{a}$ is surjective. Since $L_{a}$ is
            injective and surjective, it is bijective.
        \end{proof}
        The same result holds for right-translation. The proof is also a mirror
        of left-translation.
        \begin{theorem}
            If $(G,\,*)$ is a group, if $a\in{G}$, and if
            $R_{a}:G\rightarrow{G}$ is right-translation by $a$, then
            $R_{a}$ is bijective.
        \end{theorem}
        \begin{proof}
            Indeed, $R_{a}$ is injective. For let $x,y\in{G}$ and suppose
            $R_{a}(x)=R_{a}(y)$. That is, $x*a=y*a$. By the right-cancellation
            law, $x=y$, and hence $R_{a}$ is injective. It is also surjective.
            For let $y\in{G}$ and let $x=y*a^{-1}$. Then we have:
            \begin{align}
                R_{a}(x)&=x*a\tag{Definition}\\
                &=(y*a^{-1})*a\tag{Substitution}\\
                &=y*(a^{-1}*a)\tag{Associativity}\\
                &=y*e\tag{Inverse}\\
                &=y\tag{Identity}
            \end{align}
            and hence $R_{a}(x)=y$, so $R_{a}$ is surjective. Since $R_{a}$ is
            injective and surjective, it is bijective.
        \end{proof}
        \begin{theorem}
            If $(G,\,*)$ is a group, if $H\subseteq{G}$ is a subgroup, and if
            $a\in{H}$, then $L_{a}[H]=H$. That is, left-translation of a
            subgroup by an element of the subgroup results in the subgroup.
        \end{theorem}
        \begin{proof}
            For let $x\in{H}$ and set $y=a^{-1}*x$. Since $a\in{H}$ and $H$ is
            a subgroup, we have that $a^{-1}\in{H}$. Since $x\in{H}$ and $H$
            is a subgroup we also have that $a^{-1}*x\in{H}$. Hence $y\in{H}$.
            But then:
            \begin{equation}
                L_{a}(y)=a*(a^{-1}*x)=(a^{-1}*a)*x=e*x=x
            \end{equation}
            and therefore $x\in{L}_{a}[H]$. That is, $H\subseteq{L}_{a}[H]$. In
            the reverse direction, let $y\in{L}_{a}[H]$. Then
            $y=L_{a}(x)$ for some $x\in{H}$. That is, $y=a*x$. But then
            $x=a^{-1}*y$. But $a^{-1}\in{H}$ and $y\in{H}$, and hence
            $a^{-1}*y\in{H}$ since $H$ is a subgroup. That is,
            $x\in{H}$, and thus $L_{a}[H]\subseteq{H}$. We conclude that
            $H=L_{a}[H]$.
        \end{proof}
        \begin{theorem}
            If $(G,\,*)$ is a group, if $H\subseteq{G}$ is a subgroup, and if
            $a\in{H}$, then $R_{a}[H]=H$. That is, right-translation of a
            subgroup by an element of the subgroup results in the subgroup.
        \end{theorem}
        \begin{proof}
            The proof is a mirrored mimicry of the previous theorem.
        \end{proof}
        This reverses.
        \begin{theorem}
            If $(G,\,*)$ is a group, if $H\subseteq{G}$, if $H$ is non-empty,
            and if for all $a\in{H}$ it is true that
            $L_{a}[H]=H$, then $H$ is a subgroup.
        \end{theorem}
        \begin{proof}
            First note that $e\in{H}$, where $e\in{G}$ is the unique identity.
            Since $H$ is non-empty there is some $a\in{H}$. But if $a\in{H}$,
            then by hypothesis $L_{a}[H]=H$, and hence $a\in{L}_{a}[H]$.
            That is, there is some $x\in{H}$ such that $L_{a}(x)=a$, and hence
            $a=a*x$. By the left-cancellation law, $x=e$ and $e\in{H}$. So $H$
            contains the identity. It also contains inverses. Since $e\in{H}$
            and $L_{a}[H]=H$ we have that $L_{a}(x)=e$ for some $x\in{H}$.
            That is, $a*x=e$. By the uniqueness of inverses, $x=a^{-1}$ so
            $a^{-1}\in{H}$. Finally, $H$ is
            closed under multiplication since $x*y=L_{x}(y)$ for all
            $x,y\in{H}$, and hence $x*y\in{H}$ since
            $L_{x}[H]=H$. So $H$ is a subgroup.
        \end{proof}
    \section{Homomorphisms}
        Continuous functions are the main functions of study in topology.
        Continuity is solely described by the topologies. For groups we have a
        set and a binary operation, so not much to play with. The main functions
        of study for group theory should then be functions that respect or
        preserve the binary operation in some way. This motivates the definition
        of \textit{group homomorphisms}.
        \begin{definition}[\textbf{Group Homomorphism}]
            A group homomorphism from a group $(G,\,*)$ to a group
            $(G',\,*')$ is a function $\varphi:G\rightarrow{G}'$ such that
            for all $a,b\in{G}$ it is true that
            $\varphi(a*b)=\varphi(a)*'\varphi(b)$.
        \end{definition}
        \begin{example}
            Define $\varphi:\mathbb{Z}\rightarrow\mathbb{Z}_{2}$ by
            $\varphi(n)=n\textrm{ mod }2$. If $\mathbb{Z}$ is given integer
            arithmetic, and if $\mathbb{Z}_{2}$ is equipped with
            \textit{modular arithmetic}, then $\varphi$ is a group homomorphism.
            $\varphi$ sends even integers to 0 and odd integers to 1.
        \end{example}
        Group homomorphisms preserve many aspects of the group.
        \begin{theorem}
            If $(G,\,*)$ and $(G',\,*')$ are groups, if
            $\varphi:G\rightarrow{G}'$ is a group homomorphism, and if
            $e\in{G}$ and $e'\in{G}'$ are the identity elements, then
            $\varphi(e)=\varphi(e')$.
        \end{theorem}
        \begin{proof}
            We have:
            \begin{align}
                \varphi(e)&=\varphi(e*e)\tag{Identity}\\
                &=\varphi(e)*'\varphi(e)\tag{Homomorphism}
            \end{align}
            and hence $\varphi(e)=\varphi(e)*'\varphi(e)$. By the cancellation
            law, $\varphi(e)=e'$.
        \end{proof}
        \begin{theorem}
            If $(G,\,*)$ and $(G',\,*')$ are groups, if
            $\varphi:G\rightarrow{G}'$ is a group homomorphism, and if
            $a\in{G}$, then $\varphi(a^{-1})=\varphi(a)^{-1}$.
        \end{theorem}
        \begin{proof}
            Since inverses are unique, we need only show that
            $\varphi(a^{-1})$ is an inverse for $\varphi(a)$. We have:
            \begin{align}
                \varphi(a)*\varphi(a^{-1})
                &=\varphi(a*a^{-1})\tag{Homomorphism}\\
                &=\varphi(e)\tag{Identity}\\
                &=e'\tag{Previous Theorem}
            \end{align}
            and hence $\varphi(a^{-1})=\varphi(a)^{-1}$.
        \end{proof}
        \begin{theorem}
            If $(G,\,*)$ and $(G',\,*')$ are groups, if
            $\varphi:G\rightarrow{G}'$ is a group homormorphism, and if
            $H\subseteq{G}$ is a subgroup, then
            $\varphi[H]\subseteq{G}'$ is a subgroup.
        \end{theorem}
        \begin{proof}
            Since $H$ is a subgroup, $e\in{H}$ is true, and hence
            $e'\in\varphi[H]$ is also true. Hence $\varphi[H]$ is non-empty.
            If $a'\in\varphi[H]$, then $a'=\varphi(a)$ for some $a\in{H}$.
            But $H$ is a subgroup, so $a^{-1}\in{H}$ and hence
            $\varphi(a^{-1})\in\varphi[H]$. But
            $\varphi(a^{-1})=\varphi(a)^{-1}$, and hence
            $a'^{-1}\in\varphi[H]$. Lastly, if $a',b'\in{H}$, then there are
            $a,b\in{H}$ such that $a'=\varphi(a)$ and $b'=\varphi(b)$. But
            then $a'*b'=\varphi(a)*\varphi(b)=\varphi(a*b)$. Since $H$ is a
            subgroup, $a*b\in{H}$ is true, and hence $a'*b'\in\varphi[H]$.
            Hence $\varphi[H]$ is a subgroup.
        \end{proof}
        Mimicing topology, the \textit{pre-image} of a subgroup is also a
        subgroup.
        \begin{theorem}
            If $(G,\,*)$ and $(G',\,*')$ and groups, if
            $\varphi:G\rightarrow{G}'$ is a group homomorphism,
            and if $H'\subseteq{G}'$ is a subgroup, then
            $\varphi^{-1}[H']\subseteq{G}$ is a subgroup.
        \end{theorem}
        \begin{proof}
            Let $H=\varphi^{-1}[H']$. Firstly, the set is non-empty since
            $e'\in{H}'$, and hence $e\in{H}$. Let's prove $H$ is closed to
            inversions and products. Let
            $a\in{H}$. Then $\varphi(a)\in{H}'$. But $H'$ is a subgroup,
            so $\varphi(a)^{-1}\in{H}'$. But $\varphi(a)^{-1}=\varphi(a^{-1})$
            and hence $a^{-1}\in{H}$. That is, $H$ is closed to inversion.
            Lastly, let $a,b\in{H}$. Then $\varphi(a)\in{H}'$ and
            $\varphi(b)\in{H}'$. But $H'$ is a subgroup, so
            $\varphi(a)*'\varphi(b)\in{H}'$. But
            $\varphi(a)*'\varphi(b)=\varphi(a*b)$ and hence
            $a*b\in{H}$. Thus $H=\varphi^{-1}[H']$ is a subgroup.
        \end{proof}
        Just like \textit{homeomorphisms} told use when two topological spaces
        are the same (topologically), \textit{isomorphisms} tell us when
        two groups are the same (algebraically).
        \begin{definition}[\textbf{Group Isomorphism}]
            A group isomorphism from a group $(G,\,*)$ to a group
            $(G',\,*')$ is a bijective group homomorphism
            $\varphi:G\rightarrow{G}'$ such that
            $\varphi^{-1}:G'\rightarrow{G}$ is also a group homomorphism.
        \end{definition}
        In topology a bijective continuous function need not have a continuous
        inverse. Group theorists do not have such worries.
        \begin{theorem}
            If $(G,\,*)$ and $(G',\,*')$ are groups, and if
            $\varphi:G\rightarrow{G}'$ is a bijective group homomorphism,
            then $\varphi^{-1}$ is a group homomorphism.
        \end{theorem}
        \begin{proof}
            For let $a',b'\in{G}'$. Since $\varphi$ is bijective, there are
            $a,b\in{G}$ such that $a'=\varphi(a)$ and $b'=\varphi(b)$. But then,
            since $\varphi$ is bijective, we also have
            $a=\varphi^{-1}(a')$ and $b=\varphi^{-1}(b')$. Hence:
            \begin{align}
                \varphi^{-1}(a'*'b')
                &=\varphi^{-1}\big(\varphi(a)*'\varphi(b)\big)
                    \tag{Substitution}\\
                &=\varphi^{-1}\big(\varphi(a*b)\big)\tag{Homomorphism}\\
                &=a*b\tag{Inverse Function}\\
                &=\varphi^{-1}(a')*\varphi^{-1}(b')\tag{Substitution}
            \end{align}
            and therefore $\varphi^{-1}$ is a group homomorphism.
        \end{proof}
    \section{Conjugation and Normal Subgroups}
\end{document}
