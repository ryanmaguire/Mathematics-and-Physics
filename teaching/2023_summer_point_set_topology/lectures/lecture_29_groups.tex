%-----------------------------------LICENSE------------------------------------%
%   This file is part of Mathematics-and-Physics.                              %
%                                                                              %
%   Mathematics-and-Physics is free software: you can redistribute it and/or   %
%   modify it under the terms of the GNU General Public License as             %
%   published by the Free Software Foundation, either version 3 of the         %
%   License, or (at your option) any later version.                            %
%                                                                              %
%   Mathematics-and-Physics is distributed in the hope that it will be useful, %
%   but WITHOUT ANY WARRANTY; without even the implied warranty of             %
%   MERCHANTABILITY or FITNESS FOR A PARTICULAR PURPOSE.  See the              %
%   GNU General Public License for more details.                               %
%                                                                              %
%   You should have received a copy of the GNU General Public License along    %
%   with Mathematics-and-Physics.  If not, see <https://www.gnu.org/licenses/>.%
%------------------------------------------------------------------------------%
\documentclass{article}
\usepackage{amsmath}                            % Needed for align.
\usepackage{amssymb}                            % Needed for mathbb.
\usepackage{amsthm}                             % For the theorem environment.

%------------------------Theorem Styles-------------------------%
\theoremstyle{plain}
\newtheorem{theorem}{Theorem}[section]

% Define theorem style for default spacing and normal font.
\newtheoremstyle{normal}
    {\topsep}               % Amount of space above the theorem.
    {\topsep}               % Amount of space below the theorem.
    {}                      % Font used for body of theorem.
    {}                      % Measure of space to indent.
    {\bfseries}             % Font of the header of the theorem.
    {}                      % Punctuation between head and body.
    {.5em}                  % Space after theorem head.
    {}

% Define default environments.
\theoremstyle{normal}
\newtheorem{examplex}{Example}[section]
\newtheorem{definitionx}{Definition}[section]

\newenvironment{example}{%
    \pushQED{\qed}\renewcommand{\qedsymbol}{$\blacksquare$}\examplex%
}{%
    \popQED\endexamplex%
}

\newenvironment{definition}{%
    \pushQED{\qed}\renewcommand{\qedsymbol}{$\blacksquare$}\definitionx%
}{%
    \popQED\enddefinitionx%
}

\title{Point-Set Topology: Lecture 29}
\author{Ryan Maguire}
\date{\today}

% No indent and no paragraph skip.
\setlength{\parindent}{0em}
\setlength{\parskip}{0em}

\begin{document}
    \maketitle
    In algebraic topology one uses algebraic structures like groups and
    vector spaces to solve topological problems. This is usually done by
    \textit{associating} some algebraic object with a given topological space,
    and showing that homotopy equivalences or homeomorphisms preserve
    the nature of this algebraic device. We won't be exploring this route, it
    deserves its own course. Instead we'll dive into
    \textit{topological algebra}. Here we reverse the idea, attaching a
    topology to algebraic structures. In particular we'll discuss
    \textit{topological groups}, perhaps the simplest object of study in
    topological algebra. There are a few motivating examples for the study of
    topological groups.
    \begin{itemize}
        \item A \textit{real topological vector space} is a real vector space
            $(V,\,+)$ together with a topology $\tau$ that makes vector
            addition $(v,\,w)\mapsto{v+w}$ and scalar multiplication
            $(a,\,v)\mapsto{a}v$ continuous operations. The additive nature of
            vector addition yields an Abelian group, meaning every topological
            vector space has a canonical topological group associated to it.
        \item A \textit{Banach space} is a normed vector space (usually over
            $\mathbb{R}$ or $\mathbb{C}$) such that the metric induced by the
            norm yields a \textit{complete} metric space. Banach spaces are,
            in particular, topological vector spaces and the addition of
            vectors forms a topological group.
        \item A Lie group is a smooth manifold with a group operation that is
            \textit{smooth}. Lie groups are, in particular, topological groups.
    \end{itemize}
    There is no pre-requesite for algebra, so we'll take the time to develop
    the basics of group theory.
    \section{Group Theory}
        There are a few competing views on how best to describe groups. Some
        say it is the study of symmetry. One may also view groups as
        combinatorial objects. I'll take the approach of
        \textit{generalized arithmetic}. The addition of integers and the
        multiplication of matrices provide two motivating examples of groups.
        Many of the theorems involving these two arithmetics need only a few
        common traits. Groups generalize these traits to abstract objects.
        \begin{definition}[(\textbf{Group})]
            A group is an ordered pair $(G,\,*)$ such that $G$ is a set and
            $*:G\times{G}\rightarrow{G}$ is a function (called a
            \textit{binary operation}) that is \textit{associative}, has an
            \textit{identity}, and contains \textit{inverse elements}. That is:
            \begin{enumerate}
                \item For $a,\,b,\,c\in{G}$ it is true that
                    $a*(b*c)=(a*b)*c$ (Associativity)
                \item There is an $e\in{G}$ such that $a*e=e*a=a$ for all
                    $a\in{G}$ (Identity)
                \item For all $a\in{G}$ there is a $b\in{G}$ such that
                    $a*b=b*a=e$ (Inverses)
            \end{enumerate}
        \end{definition}
        \begin{theorem}
            If $(G,\,*)$ is a group, and if $e,e'\in{G}$ are identities, then
            $e=e'$
        \end{theorem}
        \begin{proof}
            Since $e$ and $e'$ are identities we have:
            \begin{equation}
                e=e*e'=e'
            \end{equation}
            and hence $e=e'$.
        \end{proof}
        \begin{theorem}
            If $(G,\,*)$ is a group, if $a\in{G}$, and if $b$ and $b'$ are
            inverses of $a$, then $b=b'$.
        \end{theorem}
        \begin{proof}
            For let $e\in{G}$ be the unique identity. Then we have:
            \begin{align}
                b&=b*e\tag{Identity}\\
                &=b*(a*b')\tag{Inverse}\\
                &=(b*a)*b'\tag{Associativity}\\
                &=e*b'\tag{Inverse}\\
                &=b'\tag{Identity}
            \end{align}
            and hence $b=b'$.
        \end{proof}
        Because of this, given $a\in{G}$ we denote by $a^{-1}$ the
        \textit{unique} inverse of $a$.
        \begin{theorem}
            If $(G,\,*)$ is a group, and if $a,b\in{G}$, then
            $(a*b)^{-1}=b^{-1}*a^{-1}$.
        \end{theorem}
        \begin{proof}
            Since inverses are unique, we need only prove that
            $b^{-1}*a^{-1}$ is indeed an inverse of $a*b$. We have:
            \begin{align}
                (a*b)*(b^{-1}*a^{-1})&=
                    a*\big((b*b^{-1})*a^{-1}\big)\tag{Associativity}\\
                    &=a*(e*a^{-1})\tag{Inverse}\\
                    &=a*a^{-1}\tag{Identity}\\
                    &=e\tag{Inverse}
            \end{align}
            by the uniqueness of inverses, $(a*b)^{-1}=b^{-1}*a^{-1}$.
        \end{proof}
        \begin{theorem}
            If $(G,\,*)$ is a group, and if $a\in{G}$, then
            $(a^{-1})^{-1}=a$.
        \end{theorem}
        \begin{proof}
            We have that:
            \begin{align}
                a&=a*e\tag{Identity}\\
                &=a*\big(a^{-1}*(a^{-1})^{-1}\big)\tag{Inverse}\\
                &=(a*a^{-1})*(a^{-1})^{-1}\tag{Associativity}\\
                &=e*(a^{-1})^{-1}\tag{Inverse}\\
                &=(a^{-1})^{-1}\tag{Identity}
            \end{align}
            and hence $a=(a^{-1})^{-1}$.
        \end{proof}
        \begin{theorem}[\textbf{Left-Cancellation Law}]
            If $(G,\,*)$ is a group, if $a,b,c\in{G}$, and if
            $a*b=a*c$, then $b=c$.
        \end{theorem}
        \begin{proof}
            Letting $e\in{G}$ be the unique inverse, if $a*b=a*c$, then we have:
            \begin{align}
                b&=e*b\tag{Identity}\\
                &=(a^{-1}*a)*b\tag{Inverse}\\
                &=a^{-1}*(a*b)\tag{Associativity}\\
                &=a^{-1}*(a*c)\tag{Hypothesis}\\
                &=(a^{-1}*a)*c\tag{Associativity}\\
                &=e*c\tag{Inverse}\\
                &=c\tag{Identity}
            \end{align}
            so we conclude that $b=c$.
        \end{proof}
        We can mirror this argument to prove the right-cancellation law.
        \begin{theorem}[\textbf{Right-Cancellation Law}]
            If $(G,\,*)$ is a group, if $a,b,c\in{G}$, and if $b*a=c*a$, then
            $b=c$.
        \end{theorem}
        \begin{proof}
            Letting $e\in{G}$ be the unique inverse, if $b*a=c*a$, then we have:
            \begin{align}
                b&=b*e\tag{Identity}\\
                &=b*(a*a^{-1})\tag{Inverse}\\
                &=(b*a)*a^{-1}\tag{Associativity}\\
                &=(c*a)*a^{-1}\tag{Hypothesis}\\
                &=c*(a*a^{-1})\tag{Associativity}\\
                &=c*e\tag{Inverse}\\
                &=c\tag{Identity}
            \end{align}
            and hence $b=c$.
        \end{proof}
        \begin{theorem}
            If $(G,\,*)$ is a group, if $a,b\in{G}$, and if
            $a=a*b$, then $b=e$ where $e\in{G}$ is the unique identity element.
        \end{theorem}
        \begin{proof}
            We have that:
            \begin{align}
                a*b&=a\tag{Hypothesis}\\
                &=a*e\tag{Identity}
            \end{align}
            and hence $a*b=a*e$. By the left-cancellation law, $b=e$.
        \end{proof}
        \begin{definition}[\textbf{Subgroup}]
            A subgroup of a group $(G,\,*)$ is a subset $H\subseteq{G}$ such
            that the restriction of $*$ to $H\times{H}$ yields a group. That
            is, $(H,\,*_{H})$ is a group.
        \end{definition}
        \begin{theorem}
            If $(G,\,*)$ is a group, if $e\in{G}$ is the unique identity, and
            if $H\subseteq{G}$ is a subgroup, then $e\in{H}$.
        \end{theorem}
        \begin{proof}
            Since $(H,\,*_{H})$ is a group, there is some identity element
            $e_{H}\in{H}$. But then $e_{H}*_{H}e_{H}=e_{H}$. But
            $*_{H}$ is just the restriction of $*$ to $H$, so
            $e_{H}*e_{H}=e_{H}$. By the previous theorem, $e_{H}=e$, so
            $e\in{H}$.
        \end{proof}
        \begin{theorem}
            If $(G,\,*)$ is a group, if $H\subseteq{G}$ is a subgroup, and
            if $a\in{H}$, then $a^{1}\in{H}$.
        \end{theorem}
        \begin{proof}
            Since $(H,\,*_{H})$ is a group, and since $a\in{H}$, there is an
            inverse element $a_{H}^{-1}$ such that
            $a*a_{H}^{-1}=a_{H}^{-1}*a=e_{H}$. But by the previous theorem,
            $e_{H}=e$, so $a*a_{H}^{-1}=a_{H}^{-1}*a=e$. But $(G,\,*)$ is a
            group and inverses are unique, so $a_{H}^{-1}=a^{-1}$. Hence
            $a^{-1}\in{H}$.
        \end{proof}
        \begin{theorem}
            If $(G,\,*)$ is a group, and if $H\subseteq{G}$, then $H$ is a
            subgroup if and only if $H$ is non-empty and for all $a\in{H}$
            it is true that $a^{-1}\in{H}$, and if for all $a,b\in{H}$ it is
            true that $a*b\in{H}$.
        \end{theorem}
        \begin{proof}
            By the previous two theorems, if $H\subseteq{G}$ is a subgroup,
            then it is closed to group multiplication and inversion. Suppose
            $H\subseteq{G}$ is closed to inversion and multiplication. We need
            only show that $H$ has an identity element (it has inverses, and
            the group operation is associative, so the restriction to $H$ is
            associative as well). Since $H$ is non-empty, there is some
            $a\in{H}$. But $H$ is closed to inversion, so $a^{-1}\in{H}$. But
            $H$ is also closed under multiplication, meaning $a*a^{-1}\in{H}$.
            But $a*a^{-1}=e$, and hence $e\in{H}$. That is, $H$ is a subgroup.
        \end{proof}
    \section{Homomorphisms}
\end{document}
