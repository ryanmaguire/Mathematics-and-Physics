%-----------------------------------LICENSE------------------------------------%
%   This file is part of Mathematics-and-Physics.                              %
%                                                                              %
%   Mathematics-and-Physics is free software: you can redistribute it and/or   %
%   modify it under the terms of the GNU General Public License as             %
%   published by the Free Software Foundation, either version 3 of the         %
%   License, or (at your option) any later version.                            %
%                                                                              %
%   Mathematics-and-Physics is distributed in the hope that it will be useful, %
%   but WITHOUT ANY WARRANTY; without even the implied warranty of             %
%   MERCHANTABILITY or FITNESS FOR A PARTICULAR PURPOSE.  See the              %
%   GNU General Public License for more details.                               %
%                                                                              %
%   You should have received a copy of the GNU General Public License along    %
%   with Mathematics-and-Physics.  If not, see <https://www.gnu.org/licenses/>.%
%------------------------------------------------------------------------------%
\documentclass{article}
\usepackage{graphicx}                           % Needed for figures.
\usepackage{amsmath}                            % Needed for align.
\usepackage{amssymb}                            % Needed for mathbb.
\usepackage{amsthm}                             % For the theorem environment.
\usepackage{hyperref}
\hypersetup{colorlinks=true, linkcolor=blue}

%------------------------Theorem Styles-------------------------%
\theoremstyle{plain}
\newtheorem{theorem}{Theorem}[section]

% Define theorem style for default spacing and normal font.
\newtheoremstyle{normal}
    {\topsep}               % Amount of space above the theorem.
    {\topsep}               % Amount of space below the theorem.
    {}                      % Font used for body of theorem.
    {}                      % Measure of space to indent.
    {\bfseries}             % Font of the header of the theorem.
    {}                      % Punctuation between head and body.
    {.5em}                  % Space after theorem head.
    {}

% Define default environments.
\theoremstyle{normal}
\newtheorem{examplex}{Example}[section]
\newtheorem{definitionx}{Definition}[section]

\newenvironment{example}{%
    \pushQED{\qed}\renewcommand{\qedsymbol}{$\blacksquare$}\examplex%
}{%
    \popQED\endexamplex%
}

\newenvironment{definition}{%
    \pushQED{\qed}\renewcommand{\qedsymbol}{$\blacksquare$}\definitionx%
}{%
    \popQED\enddefinitionx%
}

\title{Point-Set Topology: Lecture 28}
\author{Ryan Maguire}
\date{\today}

% No indent and no paragraph skip.
\setlength{\parindent}{0em}
\setlength{\parskip}{0em}

\begin{document}
    \maketitle
    \section{Topological Manifolds}
        Topological manifolds are spaces that \textit{look} like
        $\mathbb{R}^{n}$, and that are topologically \textit{nice}. They are
        one of the primary motivators for general topology and have widespread
        applications in physics, computer graphics, and other branches of
        mathematics.
        \begin{definition}[\textbf{Locally Euclidean Topological Space}]
            A locally Euclidean topological space is a topological space
            $(X,\,\tau)$ such that for all $x\in{X}$ there is an open set
            $\mathcal{U}\in\tau$ such that $x\in\mathcal{U}$, an
            $n\in\mathbb{N}$, and a continuous injective open mapping
            $\varphi:\mathcal{U}\rightarrow\mathbb{R}^{n}$.
        \end{definition}
        Since $\varphi:\mathcal{U}\rightarrow\mathbb{R}^{n}$ is an injective
        function, it is bijective onto it's image. Hence
        $\varphi:\mathcal{U}\rightarrow\varphi[\mathcal{U}]$ is a continuous
        bijective open mapping, which is therefore a homeomorphism. Another way
        of stating the definition of locally Euclidean spaces is that every
        point has an open set about it that is homeomorphic to an open subset
        of $\mathbb{R}^{n}$ for some $n\in\mathbb{N}$.
        \begin{example}
            Euclidean space $\mathbb{R}^{n}$ with the standard Euclidean
            topology $\tau_{\mathbb{R}^{n}}$ is locally Euclidean. Given a
            point $\mathbf{x}\in\mathbb{R}^{n}$ choose
            $\mathcal{U}=\mathbb{R}^{n}$ and
            $\varphi=\textrm{id}_{\mathbb{R}^{n}}$. That is, the open set
            about $\mathbf{x}$ is all of Euclidean space, and the function
            $\varphi$ is the identity. The identity is always a homeomorphism,
            which is therefore a continuous injective open mapping. This shows
            us that Euclidean space is indeed locally Euclidean.
        \end{example}
        \begin{example}
            If $\mathcal{V}\subseteq\mathbb{R}^{n}$ is an open subset, then
            $(\mathcal{V},\,\tau_{\mathcal{V}})$ is locally Euclidean where
            $\tau_{\mathcal{V}}$ is the subspace topology inherited from the
            standard Euclidean topology $\tau_{\mathbb{R}^{n}}$. Given a point
            $\mathbf{x}\in\mathcal{V}$, choose
            $\mathcal{U}=\mathcal{V}$ and $\varphi=\iota_{\mathcal{V}}$, the
            inclusion mapping. Since $\mathcal{V}$ is open,
            $\iota_{\mathcal{V}}$ is an open mapping, and inclusion mappings
            are always injective and continuous. Thus any open subspace of
            $\mathbb{R}^{n}$ is locally Euclidean.
        \end{example}
        \begin{example}
            The bug-eyed line, which is the quotient of
            $\mathbb{R}\times\mathbb{Z}_{2}$ under the identification
            $(x,\,0)R(x,\,1)$ for all $x\ne{0}$, is locally Euclidean. For
            points away from the double-origin the bug-eyed line looks, locally,
            like the real line. The only cause for concern is the two origins.
            Label $0'=[(0,\,0)]$ and $0''=[(0,\,1)]$. The set
            $(-1,\,1)\times\{\,0\,\}$ is open in the product space
            $\mathbb{R}\times\mathbb{Z}_{2}$ since $\mathbb{Z}_{2}$ is discrete
            and $(-1,\,1)$ is open in $\mathbb{R}$. The set
            $\big((-1,\,0)\cup(0,\,1)\big)\times\{\,1\,\}$ is also open for the
            same reason. Let $\mathcal{U}$ be the union of these two open sets,
            which is therefore open. Note that this set is saturated with
            respect to the quotient map. That is,
            $\mathcal{U}=q^{-1}\big[q[\mathcal{U}]\big]$. But the quotient map
            takes saturated open sets to saturated open sets, so
            $q[\mathcal{U}]$ is open in the quotient topology. This open set
            is homeomorphic to $(-1,\,1)$ and contains $0'$. We can do a
            similar argument for $0''$ showing that the bug-eyed line is
            locally Euclidean. Note, however, that it is not Hausdorff.
        \end{example}
        \begin{example}
            The long-line is locally Euclidean. Given any point in the
            long-line it \textit{locally} looks like $(-1,\,1)$, the open
            unit interval in $\mathbb{R}$. This space is very \textit{large},
            it is not second-countable. Note then that is would be impossible
            to embed the long-line into Euclidean space, no matter how large
            the dimension.
        \end{example}
        Topological manifolds should be nice enough that it is possible to
        embed them into $\mathbb{R}^{n}$. This is not a requirement, however,
        but the previous two examples show us that \textit{locally Euclidean}
        by itself can still yield bizarre examples. This motivates the
        following definition.
        \begin{definition}[\textbf{Topological Manifold}]
            A topological manifold is a topological space $(X,\,\tau)$ that
            is locally Euclidean, second-countable, and Hausdorff.
        \end{definition}
        \begin{example}
            $\mathbb{R}^{n}$ is a topological manifold, with its standard
            topology. It is locally Euclidean, as seen previously, and it is
            also second-countable (being the product of a second-countable
            space) and Hausdorff (since it is metrizable).
        \end{example}
        \begin{example}
            If $\mathcal{V}\subseteq\mathbb{R}^{n}$ is an open subset, then
            $(\mathcal{V},\,\tau_{\mathcal{V}})$ is a topological manifold.
            We've already shown that such a space is locally Euclidean, but it
            is also Hausdorff and second-countable since these properties are
            inherited by subspaces.
        \end{example}
        \begin{example}
            As far as set theory is concerned, a function
            $f:A\rightarrow{B}$ from a set $A$ to a set $B$ is a subset of
            $A\times{B}$ satisfying certain properties. We can use this to
            define locally Euclidean topological spaces by looking at
            continuous functions from $\mathbb{R}^{m}$ to $\mathbb{R}^{n}$ for
            some $m,n\in\mathbb{N}$. Given
            $f:\mathbb{R}^{m}\rightarrow\mathbb{R}^{n}$, continuous,
            $f\subseteq\mathbb{R}^{m}\times\mathbb{R}^{n}$ can be given the
            subspace topology. This makes it a closed subset since $f$ is
            continuous. It is also a locally Euclidean subspace. For given
            $\big(\mathbf{x},\,f(\mathbf{x})\big)\in{f}$, let
            $\mathcal{U}=f$ and define $F:f\rightarrow\mathbb{R}^{m}$ via:
            \begin{equation}
                F\big((\mathbf{x},\,f(\mathbf{x})\big)=\mathbf{x}
            \end{equation}
            This is just the projection of the elements of
            $f\subseteq\mathbb{R}^{m}\times\mathbb{R}^{n}$ onto
            $\mathbb{R}^{m}$. Projections are continuous. Let's show
            $F$ is injective and an open mapping. It is injective since given:
            \begin{equation}
                \big(\mathbf{x}_{0},\,f(\mathbf{x}_{0})\big)
                \ne\big(\mathbf{x}_{1},\,f(\mathbf{x}_{1})\big)
            \end{equation}
            we must have $\mathbf{x}_{0}\ne\mathbf{x}_{1}$ (since if
            $\mathbf{x}_{0}=\mathbf{x}_{1}$, then
            $f(\mathbf{x}_{0})=f(\mathbf{x}_{1})$ by definition of a function).
            So then:
            \begin{equation}
                F\big((\mathbf{x}_{0},\,f(\mathbf{x}_{0})\big)
                \ne{F}\big((\mathbf{x}_{1},\,f(\mathbf{x}_{1})\big)
            \end{equation}
            meaning $F$ is injective. There is a continuous inverse
            $F^{-1}:\mathbb{R}^{m}\rightarrow{f}$ given by:
            \begin{equation}
                F^{-1}(\mathbf{x})
                =\big(\mathbf{x},\,f(\mathbf{x})\big)
            \end{equation}
            Since $f$ is continuous, $F^{-1}$ is continuous since both
            components are continuous. So $F$ is an open mapping and $f$
            is a locally Euclidean subspace of
            $\mathbb{R}^{m}\times\mathbb{R}^{n}$. Since subspaces of
            $\mathbb{R}^{m+n}$ are also Hausdorff and second-countable, this
            shows us that the \textit{graph} of a continuous function
            $f:\mathbb{R}^{m}\rightarrow\mathbb{R}^{n}$ is a topological
            manifold.
        \end{example}
        \begin{example}
            $\mathbb{S}^{1}$ with the subspace topology from $\mathbb{R}^{2}$
            is locally Euclidean. We'll show this in two ways. First, via
            orthographic projection. We split the circle into four parts:
            \begin{align}
                \mathcal{U}_{\textrm{North}}
                &=\{\,(x,\,y)\in\mathbb{S}^{1}\;|\;y>0\,\}\\
                \mathcal{U}_{\textrm{South}}
                &=\{\,(x,\,y)\in\mathbb{S}^{1}\;|\;y<0\,\}\\
                \mathcal{U}_{\textrm{East}}
                &=\{\,(x,\,y)\in\mathbb{S}^{1}\;|\;x>0\,\}\\
                \mathcal{U}_{\textrm{West}}
                &=\{\,(x,\,y)\in\mathbb{S}^{1}\;|\;x<0\,\}
            \end{align}
            See Fig.~\ref{fig:circle_is_locally_euclidean_001}.
            Then we define four functions:
            \begin{align}
                \varphi_{\textrm{North}}:
                \mathcal{U}_{\textrm{North}}&\rightarrow\mathbb{R}&
                \quad\quad
                \varphi_{\textrm{North}}\big((x,\,y)\big)&=x\\
                \varphi_{\textrm{South}}:
                \mathcal{U}_{\textrm{South}}&\rightarrow\mathbb{R}&
                \quad\quad
                \varphi_{\textrm{South}}\big((x,\,y)\big)&=x\\
                \varphi_{\textrm{East}}:
                \mathcal{U}_{\textrm{East}}&\rightarrow\mathbb{R}&
                \quad\quad
                \varphi_{\textrm{East}}\big((x,\,y)\big)&=y\\
                \varphi_{\textrm{West}}:
                \mathcal{U}_{\textrm{West}}&\rightarrow\mathbb{R}&
                \quad\quad
                \varphi_{\textrm{West}}\big((x,\,y)\big)&=y
            \end{align}
            Since these are projection mappings, they are continuous. From
            how the four open sets are defined, each is also injective. To
            show it is an open mapping we just need to find a continuous
            inverse with respect to the image of these sets. Note that for all
            four functions the range is $(-1,\,1)$. We have the following
            inverse functions:
            \begin{align}
                \varphi_{\textrm{North}}^{-1}(x)
                &=\big(x,\,\sqrt{1-x^{2}}\big)\\
                \varphi_{\textrm{South}}^{-1}(x)
                &=\big(x,\,-\sqrt{1-x^{2}}\big)\\
                \varphi_{\textrm{East}}^{-1}(y)
                &=\big(\sqrt{1-y^{2}},\,y\big)\\
                \varphi_{\textrm{West}}^{-1}(y)
                &=\big(-\sqrt{1-y^{2}},\,y\big)
            \end{align}
            each of which is continuous since the square root function is
            continuous. The four sets also cover $\mathbb{S}^{1}$, showing that
            $\mathbb{S}^{1}$ is locally Euclidean. Since
            $\mathbb{R}^{2}$ is second-countable and locally Euclidean,
            $\mathbb{S}^{1}$ is as well. Hence the circle is a topological
            manifold.
        \end{example}
        \begin{figure}
            \centering
            \includegraphics{../../../images/circle_is_locally_euclidean_001.pdf}
            \caption{Cover of $\mathbb{S}^{1}$ with Locally Euclidean Sets}
            \label{fig:circle_is_locally_euclidean_001}
        \end{figure}
        This shows we can cover $\mathbb{S}^{1}$ using four sets each of which
        is homeomorphic to an open subset of $\mathbb{R}$. We can do better,
        only two sets are needed. Place an observer at the north pole
        $N=(0,\,1)$. Given any other point $(x,\,y)$ the line from the
        observer to the point is not parallel to the $x$ axis, meaning
        eventually it must intersect it. Let's solve for when. The line segment
        $\alpha(t)=(1-t)N+t(x,\,y)$ starts at the north pole at time $t=0$ and
        ends at the point $(x,\,y)$ on the circle at time $t=1$. The line
        intersects the $x$ axis when the $y$ component is zero. Thus we wish
        to solve $1-t+ty=0$ for $t$. We get:
        \begin{equation}
            t_{0}=\frac{1}{1-y}
        \end{equation}
        The $x$ coordinate at time $t=t_{0}$ is then:
        \begin{equation}
            \varphi_{N}\big((x,\,y)\big)
            =\frac{x}{1-y}
        \end{equation}
        This is \textit{stereographic projection} about the north pole. It is
        continuous since it is a rational function. It is also bijective with a
        continuous inverse. Given $X\in\mathbb{R}$ we can solve for the value
        $(x,\,y)\in\mathbb{S}^{1}$ that gets mapped to $X$ by reversing the
        previous process. The line
        $\beta(t)=(1-t)N+t(X,\,0)$ starts at the north pole and ends at
        $(X,\,0)$. We wish to solve for the time $t$ when
        $||\beta(t)||_{2}=1$ which corresponds to the moment the line
        intersects the circle. We have:
        \begin{align}
            ||\beta(t)||_{2}
            &=||(1-t)N+t(X,\,0)||_{2}\\
            &=||(1-t)(0,\,1)+t(X,\,0)||_{2}\\
            &=||(tX,\,1-t)||_{2}\\
            &=\sqrt{(tX)^{2}+(1-t)^{2}}
        \end{align}
        Solving for $||\beta(t)||_{2}=1$ is equivalent to solving
        $||\beta(t)||_{2}^{2}=1$ so we need to consider the expression
        $(tX)^{2}+(1-t)^{2}$. We get:
        \begin{align}
            1&=(tX)^{2}+(1-t)^{2}\\
            &=t^{2}X^{2}+1-2t+t^{2}\\
            &=t^{2}(1+X^{2})-2t+1
        \end{align}
        meaning we want to solve for $t^{2}(1+X^{2})-2t=0$. The solution
        $t=0$ corresponds to the North pole, which is not the one we want.
        Dividing through by $t$ we get:
        \begin{equation}
            t_{1}=\frac{2}{1+X^{2}}
        \end{equation}
        The point $(x,\,y)$ corresponds to $\beta(t_{1})$ and is given by:
        \begin{equation}
            \varphi_{N}^{-1}(X)=
            \Big(\frac{2X}{1+X^{2}},\,\frac{-1+X^{2}}{1+X^{2}}\Big)
        \end{equation}
        This function is continuous since it is a rational function in each
        component. Because of this
        $\varphi_{N}:\mathbb{S}^{1}\setminus\{\,(0,\,1)\,\}\rightarrow\mathbb{R}$
        is a homeomorphism. Doing a similar projection about the south pole
        shows that $\mathbb{S}^{1}$ can be covered by two open sets,
        $\mathbb{S}^{1}\setminus\{\,(0,\,1)\,\}$ and
        $\mathbb{S}^{1}\setminus\{\,(0,\,-1)\,\}$, each of which is
        homeomorphic to $\mathbb{R}$.
        \par\hfill\par
        It is impossible to do this with one set. This is because
        $\mathbb{S}^{1}$ is not homeomorphic to an open subset of
        $\mathbb{R}$ since $\mathbb{S}^{1}$ is compact and the only open subset
        of $\mathbb{R}$ that is compact is the empty set, but
        $\mathbb{S}^{1}$ is non-empty. So two is the best we can do.
        \begin{example}
            The sphere $\mathbb{S}^{n}\subseteq\mathbb{R}^{n+1}$ is also
            locally Euclidean for all $n\in\mathbb{N}$. Define
            $\mathcal{U}_{k}^{\pm}\subseteq\mathbb{S}^{n}$ via:
            \begin{align}
                \mathcal{U}_{k}^{+}
                &=\{\,\mathbf{x}\in\mathbb{S}^{n}\;|\;
                    \mathbf{x}_{k}>0\,\}\\
                \mathcal{U}_{k}^{-}
                &=\{\,\mathbf{x}\in\mathbb{S}^{n}\;|\;
                    \mathbf{x}_{k}<0\,\}
            \end{align}
            These $2n+2$ open sets cover $\mathbb{S}^{n}$ and each is
            homeomorphic to an open subset of $\mathbb{R}^{n}$. Define
            $\varphi_{k}^{\pm}:\mathcal{U}_{k}^{\pm}\rightarrow{B}_{1}^{\mathbb{R}^{n}}(\mathbf{0})$
            via:
            \begin{equation}
                \varphi_{k}^{\pm}(\mathbf{x})=
                (\mathbf{x}_{0},\,\dots,\,\mathbf{x}_{k-1},\,
                    \mathbf{x}_{k+1},\,\mathbf{x}_{n})
            \end{equation}
            That is, projecting down that $k^{th}$ axis. This is continuous
            with a continuous inverse
            ${\varphi_{k}^{\pm}}^{-1}:B_{1}^{\mathbb{R}^{n}}(\mathbf{0})\rightarrow\mathcal{U}_{k}^{\pm}$
            given by:
            \begin{equation}
                {\varphi_{k}^{\pm}}^{-1}(\mathbf{x})
                =(\mathbf{x}_{0},\,\dots,\,\mathbf{x}_{k-1},\,
                    \pm\sqrt{1-||\mathbf{x}||_{2}^{2}},\,\mathbf{x}_{k},\,
                    \dots,\,\mathbf{x}_{n-1})
            \end{equation}
            This is also continuous, so $\mathbb{S}^{n}$ is locally Euclidean.
            For reasons similar to the circle, the higher dimensional spheres
            are also topological manifolds.
        \end{example}
        These mappings are called \textit{orthographic projections}. They are
        formed by placing an observer at \textit{infinity} and projecting what
        they see down to the plane. This is shown in
        Fig.~\ref{fig:orthographic_projection_001}
        \begin{figure}
            \centering
            \includegraphics{../../../images/sphere_orthographic_projection_001.pdf}
            \caption{Orthographic Projection of the Sphere}
            \label{fig:orthographic_projection_001}
        \end{figure}
        \begin{definition}[\textbf{Topological Chart}]
            A topological chart of dimension $n$ in a topological space
            $(X,\,\tau)$ about a point $x\in{X}$ is an ordered pair
            $(\mathcal{U},\,\varphi)$ such that $\mathcal{U}\in\tau$,
            $x\in\mathcal{U}$, and
            $\varphi:\mathcal{U}\rightarrow\mathbb{R}^{n}$ is an injective
            continuous open mapping.
        \end{definition}
        Locally Euclidean could equivalently be described as a topological space
        $(X,\,\tau)$ such that for all $x\in{X}$ there is a chart
        $(\mathcal{U},\,\varphi)$ such that $x\in\mathcal{U}$. A collection of
        charts that covers a space is called an \textit{atlas}.
        See Fig.~\ref{fig:chart_in_a_manifold}.
        \begin{figure}
            \centering
            \includegraphics{../../../images/chart_in_a_manifold.pdf}
            \caption{A Chart in a Manifold}
            \label{fig:chart_in_a_manifold}
        \end{figure}
        \begin{definition}[\textbf{Topological Atlas}]
            A topological atlas for a topological space $(X,\,\tau)$ is a
            set $\mathcal{A}$ of topological charts in $(X,\,\tau)$ such that
            for all $x\in{X}$ there is a $(\mathcal{U},\,\varphi)\in\mathcal{A}$
            such that $x\in\mathcal{U}$.
        \end{definition}
        \begin{figure}
            \centering
            \includegraphics{../../../images/sphere_stereographic_projection.pdf}
            \caption{Stereographic Projection for the Sphere}
            \label{fig:sphere_stereographic_projection}
        \end{figure}
        That is, an atlas is a collection of charts whose domains cover the
        space. Think of an actual atlas used for navigating. The pages consist
        of various locations on the globe, but only provides local information.
        To get information that is more global requires piecing some of the
        charts of the atlas together. A locally Euclidean space is a topological
        space $(X,\,\tau)$ such that there exists an atlas $\mathcal{A}$ for it.
        We've shown that $\mathbb{S}^{n}$ can be covered by $2n+2$ charts using
        orthographic projection. We can do better using stereographic projection
        the same way we did for $\mathbb{S}^{1}$. This is shown for
        $\mathbb{S}^{2}$ in Fig.~\ref{fig:sphere_stereographic_projection}.
        \par\hfill\par
        \begin{figure}
            \centering
            \includegraphics{../../../images/sphere_near_sided_projection.pdf}
            \caption{Near-Sided Projection of the Sphere}
            \label{fig:sphere_near_sided_projection}
        \end{figure}
        There are two other types of projections that are useful for geometric
        reasons in covering $\mathbb{S}^{n}$. These are the near-sided and
        far-sided projections. Near-sided projection is shown in
        Fig.~\ref{fig:sphere_near_sided_projection}. The idea is to take an
        observer and place them somewhere on the $z$ axis above the sphere.
        The portion of the sphere that is visible is then projected down to the
        $xy$ plane. Far-sided projection is the opposite. You place the
        observer at the same spot but remove everything that can be seen.
        The result is a hollow semi-sphere. You then unwrap this on to the
        plane to get the projection. This is shown in
        Fig.~\ref{fig:sphere_far_sided_projection}. Stereographic projection
        is then just far-sided projection at the north pole, and orthographic
        projection is near-sided projection at infinity.
        \begin{figure}
            \centering
            \includegraphics{../../../images/sphere_far_sided_projection.pdf}
            \caption{Far-Sided Projection of the Sphere}
            \label{fig:sphere_far_sided_projection}
        \end{figure}
       \begin{example}[\textbf{Real Projective Space}]
            Let $X=\mathbb{R}^{n+1}\setminus\{\,\mathbf{0}\,\}$. Define the
            equivalence relation $R$ on $X$ via $\mathbf{x}R\mathbf{y}$ if and
            only if $\mathbf{y}=\lambda\mathbf{x}$ for some
            $\lambda\in\mathbb{R}\setminus\{\,0\,\}$. $\mathbb{RP}^{n}$ is the
            set $X/R$ and the topology $\tau_{\mathbb{RP}^{n}}$ is the
            quotient topology induced by $R$. As a set this is the set of all
            lines in $\mathbb{R}^{n+1}$ that pass through the origin. That is,
            a point $[\mathbf{x}]\in\mathbb{RP}^{n}$ is the entire line through
            the origin that passes through the point $\mathbf{x}$. Let's start
            with $\mathbb{RP}^{1}$. Any line can be described by an angle
            $0\leq\theta<\pi$. If you vary the line you are on slightly, you
            are just varying this angle. Hopefully it becomes intuitive that
            $\mathbb{RP}^{1}$ is in fact a one dimensional locally Euclidean
            space (it may not be intuitive as to why it is Hausdorff or
            second countable, but we'll get there). A similar thinking applies
            to $\mathbb{RP}^{n}$. Let's be precise. Let
            $\mathcal{U}_{k}\subseteq{X}$ be defined by:
            \begin{equation}
                \mathcal{U}_{k}
                =\{\,\mathbf{x}\in\mathbb{R}^{n+1}\setminus\{\,\mathbf{0}\,\}\;
                    |\;\mathbf{x}_{k}\ne{0}\,\}
            \end{equation}
            This is the complement of the $k^{\textrm{th}}$ axis, which is open
            since the $k^{\textrm{th}}$ axis is closed. It is also saturated
            with respect to the canonical quotient map
            $q:X\rightarrow\mathbb{RP}^{n}$ defined by
            $q(\mathbf{x})=[\mathbf{x}]$. That is,
            $q^{-1}\big[q[\mathcal{U}_{k}]\big]=\mathcal{U}_{k}$. It is always
            the case that
            $\mathcal{U}_{k}\subseteq{q}^{-1}\big[q[\mathcal{U}_{k}]\big]$,
            let's show this reverses for our particular set $\mathcal{U}_{k}$.
            Let $\mathbf{x}\in{q}^{-1}\big[q[\mathcal{U}_{k}]\big]$. Then
            $[\mathbf{x}]\in{q}\big[\mathcal{U}_{k}\big]$ so there is some
            $\mathbf{y}\in\mathcal{U}_{k}$ such that
            $[\mathbf{x}]=[\mathbf{y}]$. But then
            $\mathbf{y}_{k}\ne{0}$ and $\mathbf{x}=\lambda\mathbf{y}$ for some
            $\lambda\in\mathbb{R}\setminus\{\,0\,\}$. But then
            $\mathbf{x}_{k}\ne{0}$, and hence $\mathbf{x}\in\mathcal{U}_{k}$.
            So $\mathcal{U}_{k}$ is saturated. But since $q$ is a quotient map,
            if $\mathcal{U}_{k}$ is open and saturated, the set
            $\tilde{\mathcal{U}}_{k}=q[\mathcal{U}_{k}]$ is open. Define
            $\varphi_{k}:\tilde{\mathcal{U}}_{k}\rightarrow\mathbb{R}^{n}$ via:
            \begin{equation}
                \varphi_{k}\big([\mathbf{x}]\big)
                =\Big(\frac{\mathbf{x}_{0}}{\mathbf{x}_{k}},\,\dots,\,
                    \frac{\mathbf{x}_{k-1}}{\mathbf{x}_{k}},\,
                    \frac{\mathbf{x}_{k+1}}{\mathbf{x}_{k}},\,\dots,\,
                    \frac{\mathbf{x}_{n}}{\mathbf{x}_{k}}\Big)
            \end{equation}
            We have to prove this is well-defined in two regards. First, there
            is no division by zero since $\mathbf{x}\in\mathcal{U}_{k}$ implies
            $\mathbf{x}_{k}\ne{0}$. Second, this is well defined as a function.
            By that I mean if $[\mathbf{x}]=[\mathbf{y}]$, then there is some
            $\lambda\in\mathbb{R}\setminus\{\,0\,\}$ such that
            $\mathbf{y}=\lambda\mathbf{x}$. But then:
            \begin{align}
                \varphi_{k}\big([\mathbf{y}]\big)
                &=\Big(\frac{\mathbf{y}_{0}}{\mathbf{y}_{k}},\,\dots,\,
                    \frac{\mathbf{y}_{k-1}}{\mathbf{y}_{k}},\,
                    \frac{\mathbf{y}_{k+1}}{\mathbf{y}_{k}},\,\dots,\,
                    \frac{\mathbf{y}_{n}}{\mathbf{y}_{k}}\Big)\\
                &=\Big(\frac{\lambda\mathbf{x}_{0}}{\lambda\mathbf{x}_{k}},\,
                    \dots,\,
                    \frac{\lambda\mathbf{x}_{k-1}}{\lambda\mathbf{x}_{k}},\,
                    \frac{\lambda\mathbf{x}_{k+1}}{\lambda\mathbf{x}_{k}},\,
                    \dots,\,
                    \frac{\lambda\mathbf{x}_{n}}{\lambda\mathbf{x}_{k}}\Big)\\
                &=\Big(\frac{\mathbf{x}_{0}}{\mathbf{x}_{k}},\,\dots,\,
                    \frac{\mathbf{x}_{k-1}}{\mathbf{x}_{k}},\,
                    \frac{\mathbf{x}_{k+1}}{\mathbf{x}_{k}},\,\dots,\,
                    \frac{\mathbf{x}_{n}}{\mathbf{x}_{k}}\Big)\\
                &=\varphi_{k}\big([\mathbf{x}]\big)
            \end{align}
            So it is well-defined. It is also continuous. This is one of the
            characteristics of the quotient map. Given a topological space
            $(Y,\,\tau_{Y})$ and a function $f:X/R\rightarrow{Y}$, $f$ is
            continuous if and only if $f\circ{q}:X\rightarrow{Y}$ is
            continuous where $q:X\rightarrow{X}/R$ is the canonical quotient
            map. The composition $\varphi_{k}\circ{q}$ is a rational function,
            which is continuous, so $\varphi_{k}$ is continuous. The inverse
            function is given by:
            \begin{equation}
                \varphi_{k}^{-1}(\mathbf{x})
                =\big[
                    (\mathbf{x}_{0},\,\dots,\,\mathbf{x}_{k-1},\,1,\,
                    \mathbf{x}_{k},\,\dots,\,\mathbf{x}_{n-1})
                \big]
            \end{equation}
            which is continuous since the function
            $f:\mathbb{R}^{n}\rightarrow\mathbb{R}^{n+1}\setminus\{\,\mathbf{0}\,\}$
            defined by:
            \begin{equation}
                f(\mathbf{x})=
                (\mathbf{x}_{0},\,\dots,\,\mathbf{x}_{k-1},\,1,\,
                    \mathbf{x}_{k},\,\dots,\,\mathbf{x}_{n-1})
            \end{equation}
            is continuous, so $\varphi_{k}^{-1}$ is the composition of
            continuous functions. Since the sets
            $\mathcal{U}_{k}$ cover
            $\mathbb{R}^{n+1}\setminus\{\,\mathbf{0}\,\}$, the sets
            $\tilde{\mathcal{U}}_{k}$ also cover $\mathbb{RP}^{n}$. Because of
            this $\mathbb{RP}^{n}$ is locally Euclidean. It is also second
            countable since it can be covered with finitely many open sets
            each of which is homeomorphic to an open subset of $\mathbb{R}^{n}$,
            which is hence second countable. Since $\mathbb{RP}^{n}$ is the
            finite union of second countable open subspaces, it is second
            countable itself. It is also Hausdorff. Given
            $[\mathbf{x}]\ne[\mathbf{y}]$ we have that $\mathbf{y}$ is not
            of the form $\lambda\mathbf{x}$ for any real number, meaning
            $\mathbf{x}$ and $\mathbf{y}$ lie on different lines through the
            origin. Let $\theta$ be defined by:
            \begin{equation}
                \theta=
                \frac{1}{4}\arccos\Big(
                    \frac{\mathbf{x}\cdot\mathbf{y}}
                        {||\mathbf{x}||_{2}\,||\mathbf{y}||_{2}}
                \Big)
            \end{equation}
            $\theta$ is one-fourth the angle made between the lines through
            the origin spanned by $\mathbf{x}$ and $\mathbf{y}$. Let
            $\mathcal{U}$ and $\mathcal{V}$ be defined by:
            \begin{align}
                \mathcal{U}
                &=\big\{\,
                    \mathbf{z}\in\mathbb{R}^{n+1}\setminus\{\,\mathbf{0}\,\}
                        \;|\;\measuredangle(\mathbf{x},\,\mathbf{z})<\theta\,
                    \big\}\\
                \mathcal{V}
                &=\big\{\,
                    \mathbf{z}\in\mathbb{R}^{n+1}\setminus\{\,\mathbf{0}\,\}
                        \;|\;\measuredangle(\mathbf{y},\,\mathbf{z})<\theta\,
                    \big\}
            \end{align}
            Where $\measuredangle(\mathbf{p},\,\mathbf{q})$ is the angle
            between the non-zero vectors $\mathbf{p}$ and $\mathbf{q}$. These
            sets are open cones in
            $\mathbb{R}^{n+1}\setminus\{\,\mathbf{0}\,\}$
            (Fig.~\ref{fig:rpn_is_hausdorff_001}) which are also
            saturated with respect to $q$, and by the choice of $\theta$ they
            are disjoint. But then $\tilde{\mathcal{U}}=q[\mathcal{U}]$ and
            $\tilde{\mathcal{V}}=q[\mathcal{V}]$ are disjoint open subsets of
            $\mathbb{RP}^{n}$ such that
            $[\mathbf{x}]\in\tilde{\mathcal{U}}$ and
            $[\mathbf{y}]\in\tilde{\mathcal{V}}$. Hence $\mathbb{RP}^{n}$ is
            Hausdorff. The real projective space is therefore a topological
            manifold.
        \end{example}
        \begin{figure}
            \centering
            \includegraphics{../../../images/rpn_is_hausdorff_001.pdf}
            \caption{$\mathbb{RP}^{n}$ is Hausdorff}
            \label{fig:rpn_is_hausdorff_001}
        \end{figure}
        The elements of $\mathbb{RP}^{n}$ are equivalence classes of
        $\mathbb{R}^{n+1}\setminus\{\,\mathbf{0}\,\}$. A
        \textit{point} in $\mathbb{RP}^{n}$ is a \textit{line} in
        $\mathbb{R}^{n+1}$ through the origin. It is not immediately clear that
        $\mathbb{RP}^{n}$ can be embedded as a subspace of $\mathbb{R}^{N}$ for
        some $N\in\mathbb{N}$. It indeed can, in fact
        $\mathbb{RP}^{n}$ can be embedded into $\mathbb{R}^{2n}$ for all $n>0$,
        but this is by no means obvious. The case $n=1$ is slightly obvious if
        you really think about what $\mathbb{RP}^{1}$ is
        (it's just a circle $\mathbb{S}^{1}$). The case $\mathbb{RP}^{2}$ is
        less obvious ($\mathbb{RP}^{2}$ is \textbf{not} a sphere). We can not
        embed the real projective plane into $\mathbb{R}^{3}$, unlike the
        sphere. If we try we'll end up with a surface that must intersect
        itself. This is shown in
        Fig.~\ref{fig:real_proj_plane_cross_cap_001}. This representation is
        known as the \textit{cross cap}.
        \begin{figure}
            \centering
            \includegraphics{../../../images/real_proj_plane_cross_cap_001.pdf}
            \caption{The Real Projective Plane}
            \label{fig:real_proj_plane_cross_cap_001}
        \end{figure}
        We can do better than this. The cross cap has a crease in it, and this
        can be removed. David Hilbert, one of the pioneering mathematicians of
        the early $20^{\textrm{th}}$ century, thought it impossible to draw the
        real projective plane in $\mathbb{R}^{3}$ in such a way that it has
        no crease. He asked his student Werney Boy to try and prove this.
        Instead Boy discovered a method of drawing the real projective plane
        in $\mathbb{R}^{3}$ that has no crease (it is still self intersecting).
        This is called the \textit{Boy surface}. It is shown in
        Fig.~\ref{fig:boy_surface_apery_immersion}.
        \begin{figure}
            \centering
            \includegraphics{../../../images/boy_surface_apery_immersion.pdf}
            \caption{The Boy Surface}
            \label{fig:boy_surface_apery_immersion}
        \end{figure}
        Bryant and Kusner discovered a way to do this using somewhat simpler
        functions involving complex variables. The Bryant-Kusner parameteriation
        is shown in Fig.~\ref{fig:boy_surface_bryant_kusner_parameterization}.
        \begin{figure}
            \centering
            \includegraphics{../../../images/boy_surface_bryant_kusner_parameterization.pdf}
            \caption{The Bryant-Kusner Parameterization of the Boy Surface}
            \label{fig:boy_surface_bryant_kusner_parameterization}
        \end{figure}
    \section{Smooth Manifolds}
        General topological spaces do not have a notion of derivative.
        We can speak of continuity, but differentiation in $\mathbb{R}^{n}$
        requires a function to \textit{locally} be approximated by a
        \textit{tangent-hyperplane}. That is, the tangent line for functions
        $f:\mathbb{R}\rightarrow\mathbb{R}$, and tangent plane for functions
        $F:\mathbb{R}^{2}\rightarrow\mathbb{R}$.
        \textit{Topological vector spaces} (vector spaces with a topology that
        makes scalar multiplication and vector addition continuous) have
        enough structure, but this is perhaps too prohibitive. Topological
        manifolds also have enough structure that one can ask questions about
        smoothness. Let $(X,\,\tau)$ be a topological manifold, and let
        $x\in{X}$. Given two charts $(\mathcal{U},\,\varphi)$ and
        $(\mathcal{V},\,\psi)$ that contain $x$, the function
        $\psi\circ\varphi^{-1}$ is a continuous function from
        $\varphi[\mathcal{U}\cap\mathcal{V}]$ to
        $\psi[\mathcal{U}\cap\mathcal{V}]$, both of which are open subsets of
        $\mathbb{R}^{n}$. That is, if we label
        $\mathcal{E}=\varphi[\mathcal{U}\cap\mathcal{V}]$ and
        $\mathcal{W}=\psi[\mathcal{U}\cap\mathcal{V}]$, then
        $\psi\circ\varphi^{-1}:\mathcal{E}\rightarrow\mathcal{W}$ is a
        continuous function from an open subset of Euclidean space to another
        open subset of Euclidean space. It is then perfectly valid to ask if
        this function has partial derivatives, or continuous partial
        derivatives, or if \textit{all partial derivatives of all orders} exist.
        That is, if the function is \textit{smooth}. This motivates the
        following.
        \begin{definition}[\textbf{Smoothly Compatible Charts}]
            Smoothly compatible charts in a topological space $(X,\,\tau)$
            are charts $(\mathcal{U},\,\varphi)$ and $(\mathcal{V},\,\psi)$
            in $(X,\,\tau)$ such that either
            $\mathcal{U}\cap\mathcal{V}=\emptyset$, or the function
            $\psi\circ\varphi^{-1}:\varphi[\mathcal{U}\cap\mathcal{V}]%
             \rightarrow\psi[\mathcal{U}\cap\mathcal{V}]$
            is smooth (as a function from on open subset of $\mathbb{R}^{n}$ to
            another open subset of $\mathbb{R}^{n}$).
        \end{definition}
        \begin{definition}[\textbf{Smooth Atlas}]
            A smooth atlas on a topological space $(X,\,\tau)$ is an atlas
            $\mathcal{A}$ on $(X,\,\tau)$ such that for all
            $(\mathcal{U},\,\varphi),(\mathcal{V},\,\psi)\in\mathcal{A}$ it is
            true that $(\mathcal{U},\,\varphi)$ and $(\mathcal{V},\,\psi)$ are
            \textit{smoothly compatible}.
        \end{definition}
        Recall our analogy with an actual atlas. A smooth atlas says that as
        you transition from one chart to another, this is done
        \textit{smoothly}. That is, suppose you have two maps containing
        sections of Europe. The first map has Paris, the second map has Berlin,
        and they overlap somewhere in between. As you travel from Paris to
        Berlin you start with the first map, since it contains Paris. Eventually
        you'll reach the edge of the first map and need to start using the
        second. When you do this, when you \textit{transition} between maps,
        it would be nice if the second map looked roughly the same as the first
        map in this overlapping region. That is, it would be nice if you could
        \textit{smoothly transition} between maps. This is precisely what the
        smooth compatibility condition does. A smooth atlas allows you to
        navigate anywhere in your space using smooth transitions.
        \begin{definition}[\textbf{Smooth Manifold}]
            A smooth manifold is an ordered triple $(X,\,\tau,\,\mathcal{A})$
            such that $(X,\,\tau)$ is a topological manifold, and
            $\mathcal{A}$ is a smooth atlas for $(X,\,\tau)$.
        \end{definition}
        The examples of topological manifolds that we've so far discussed are
        all smooth manifolds, when the appropriate atlas is chosen. This then
        inspires the following question.
        \begin{equation}
            \textrm{Are all topological manifolds also smooth manifolds?}
        \end{equation}
        Let's provide a definition.
        \begin{definition}[\textbf{Smoothable Manifold}]
            A smoothable manifold is a topological manifold $(X,\,\tau)$ such
            that there exists a smooth atlas $\mathcal{A}$ for $(X,\,\tau)$.
        \end{definition}
        Kervaire showed in 1960 that there are 10-dimensional
        topological manifolds that are not smoothable. That is, there is no
        smooth atlas for the space. Freedman found a compact 4-dimensional
        example in 1982. Kuiper's example, found in 1967, gives an 8-dimensional
        example that has a rather explicit formula. Note that the sphere
        $\mathbb{S}^{2}$ can be described by $x^{2}+y^{2}=1$. Under certain
        conditions, functions $f:X\rightarrow\mathbb{R}^{n}$ can yield
        topological manifolds by considering $f^{-1}[\{\,\mathbf{c}\,\}]$ for
        some constant $\mathbf{c}\in\mathbb{R}^{n}$. For example
        $f:\mathbb{R}^{2}\rightarrow\mathbb{R}$ defined by
        $f(x,\,y)=x^{2}+y^{2}$. The circle is the pre-image
        $f^{-1}[\{\,1\,\}]$. In a similar manner, consider the function
        $f:\mathbb{C}^{5}\rightarrow\mathbb{C}$ defined by:
        \begin{align}
            \nonumber
            f(&z_{1},\,z_{2},\,z_{3},\,z_{4},\,z_{5})
                =\\
                &z_{1}^{5}(1+z_{1})
                +z_{2}^{3}(1+ez_{2}^{3})
                +z_{3}^{2}(1+e^{2}z_{3}^{4})
                +z_{4}^{2}(1+e^{3}z_{4}^{4})
                +z_{5}^{2}(1+e^{4}z_{5}^{4})
        \end{align}
        Where $e$ is the standard Euler constant.
        Note that $\mathbb{C}^{5}$ can be identified with $\mathbb{R}^{10}$
        and $\mathbb{C}$ can be seen as $\mathbb{R}^{2}$. The pre-image of
        $\{\,0\,\}$ is an 8-dimensional manifold that cannot be smoothed.
        \par\hfill\par
        In dimensions 1, 2, and 3, all topological manifolds are smoothable.
        The cube may not be smooth itself, but we can smooth it out into a
        sphere. Similarly, a triangle can be smoothed into a circle. The fact
        that some topological manifolds cannot be smooth becomes very hard to
        imagine. This difficulty is amplified by the fact that the first
        examples of such a phenomenon occur with four dimensional spaces.
    \section{Smooth Functions and Tangent Spaces}
        \begin{figure}
            \centering
            \resizebox{\textwidth}{!}{%
                \includegraphics{%
                    ../../../images/smooth_function_between_manifolds.pdf%
                }
            }
            \caption{Smooth Function Between Manifolds}
            \label{fig:smooth_function_between_manifolds}
        \end{figure}
        Given two smooth manifolds $(X,\,\tau_{X},\,\mathcal{A}_{X})$ and
        $(Y,\,\tau_{Y},\,\mathcal{A}_{Y})$, it is possible to speak of
        \textit{smooth} functions between them. The key here is to locally
        translate the problem back to $\mathbb{R}^{n}$ and ask about smoothness
        there. We use the transition maps to do this. Note that if
        $\phi:X\rightarrow{Y}$ is a function, if $x\in{X}$, and if
        $(\mathcal{U},\,\varphi)\in\mathcal{A}_{X}$ and
        $(\mathcal{V},\,\psi)\in\mathcal{A}_{Y}$ are charts such that
        $x\in\mathcal{U}$ and $f(x)\in\mathcal{V}$, then the function
        $\psi\circ\phi\circ\varphi^{-1}$ is a function from a subset of
        $\mathbb{R}^{m}$ to a subset of $\mathbb{R}^{n}$. If
        $\phi$ is \textit{continuous}, then this composition is a function
        between an open subset of $\mathbb{R}^{m}$ and a subset of
        $\mathbb{R}^{n}$ and we can ask questions about
        smoothness. The visual for this scheme is provided in
        Fig.~\ref{fig:smooth_function_between_manifolds}.
        This motivates the following definition.
        \begin{definition}[\textbf{Smooth Functions Between Smooth Manifolds}]
            A smooth function from a smooth manifold
            $(X,\,\tau_{X},\,\mathcal{A}_{X})$ to a smooth manifold
            $(Y,\,\tau_{Y},\,\mathcal{A}_{Y})$ is a continuous function
            $\phi:X\rightarrow{Y}$ such that for all $x\in{X}$ there are
            charts $(\mathcal{U},\,\varphi)\in\mathcal{A}_{X}$ and
            $(\mathcal{V},\,\psi)\in\mathcal{A}_{Y}$ such that
            $x\in\mathcal{U}$, $f(x)\in\mathcal{V}$, and the composition
            $\psi\circ\phi\circ\varphi^{-1}$ is a smooth function from an open
            subset of $\mathbb{R}^{m}$ to a subset of $\mathbb{R}^{n}$.
        \end{definition}
        Note the definition does not require $m=n$. The two smooth manifolds
        can have different dimensions. You could, for example, smoothly embed
        the circle $\mathbb{S}^{1}$ into the sphere $\mathbb{S}^{2}$ by placing
        it along the equator.
        \par\hfill\par
        Apart from allowing us to perform calculus with manifolds, smooth
        functions also allow us to define the notion of \textit{tangent space}.
        For surfaces in $\mathbb{R}^{3}$ we define the tangent space to a point
        as a plane that contains point and lies \textit{tangential} to the
        surface. That is, it is the best linear approximation to the surface
        near the point. This definition requires an embedding of the surface
        into $\mathbb{R}^{3}$, whereas the general smooth manifold is an
        abstract topological space and has no embedding into $\mathbb{R}^{N}$
        associated with it. It is indeed possible to embed a smooth manifold
        of dimension $n$ into some Euclidean space $\mathbb{R}^{N}$
        ($N=2n+1$ does the trick by Whitney's theorem), but this takes some
        work. Given an $n$ dimensional smooth manifold
        $(X,\,\tau,\,\mathcal{A})$ it would be nice if we can define
        \textit{tangent spaces} without extra assumptions about embeddings.
        We do this using \textit{derivations}.
        \par\hfill\par
        Let $C^{\infty}(X,\,\mathbb{R})$ denote the set of all \textit{smooth}
        functions $f:X\rightarrow\mathbb{R}$ where $\mathbb{R}$ carries the
        standard Euclidean smooth structure. A derivation at a point
        $x\in{X}$ is a function
        $D:C^{\infty}(X,\,\mathbb{R})\rightarrow\mathbb{R}$ such that:
        \begin{align}
            D(af+bg)&=aD(f)+bD(g)\tag{Linearity}\\
            D(fg)&=f(x)D(g)+D(f)g(x)\tag{Liebniz Rule}
        \end{align}
        This second condition is the product rule from calculus. The tangent
        space at $x$ is defined as follows.
        \begin{definition}[\textbf{Tangent Space in a Smooth Manifold}]
            The tangent space to a point $x\in{X}$ in a smooth manifold
            $(X,\,\tau_{X},\,\mathcal{A}_{X})$ is the set $T_{x}X$ defined by:
            \begin{equation}
                T_{x}X
                    =\{\,D:C^{\infty}(X,\,\mathbb{R})\rightarrow\mathbb{R}\;|\;
                    D\textrm{ is a derivation at }x\,\}
            \end{equation}
            That is, the set of all linear and Liebnizean functions at $x$.
        \end{definition}
        The tangent space at $x$ has the structure of a vector space since
        derivations can be added. Since we wish to think of the elements of
        $T_{x}X$ as \textit{tangent vectors} starting at the point $x$, we
        usually denote them by $v\in{T}_{x}X$ or $w\in{T}_{x}X$. We can add
        them by using the following rule. Given a function
        $f\in{C}^{\infty}(X,\,\mathbb{R})$, we define $(v+w)(f)$ via:
        \begin{equation}
            (v+w)(f)=v(f)+w(f)
        \end{equation}
        This addition on the right hand side makes sense since $v(f)$ and $w(f)$
        are real numbers. Scalar multiplication is also defined:
        \begin{equation}
            (a\cdot{v})(f)=a\cdot{v}(f)
        \end{equation}
        where $a\in\mathbb{R}$ is any scalar. Being a vector space, $T_{x}X$ has
        a dimension. It is quite fortunate (and perhaps the reason why we have
        defined $T_{x}X$ this way) that the dimension is precisely the
        dimension of the manifold. This can be proved by finding an explicit
        basis. Pick a chart $(\mathcal{U},\,\varphi)\in\mathcal{A}$ that
        contains $x$ and define the \textit{differentiation operators}
        $\partial\varphi_{k}:C^{\infty}(X,\,\mathbb{R})\rightarrow\mathbb{R}$
        via:
        \begin{equation}
            \partial\varphi_{k}(f)=
            \frac{\partial}{\partial{x}_{k}}
            \Big|_{\mathbf{x}=\varphi(x)}\big(f\circ\varphi^{-1}\big)
        \end{equation}
        Since $\varphi:\mathcal{U}\rightarrow\mathbb{R}^{n}$ and
        $f:X\rightarrow\mathbb{R}$ are smooth, we have that
        $f\circ\varphi^{-1}$ is a smooth function from an open subset of
        Euclidean space to the real numbers. Taking partial derivatives is
        thus well-defined. \textit{Any} tangent vector (i.e., any derivation)
        $v\in{T}_{x}X$ can be written as a linear combination of these
        $n$ partial derivatives. We have thus attached to every point
        $x\in{X}$ a real-vector space $T_{x}X$, mimicing the notion of
        \textit{tangent space} for surfaces in $\mathbb{R}^{3}$. We've also
        accomplished this without embedding the manifold into Euclidean space.
    \section{Riemannian Manifolds}
        Topological manifolds belong to point-set (or \textit{general})
        topology. Smooth manifolds initiate the study of
        \textit{differential topology}. Geometry starts when we can measure
        things like lengths and angles, volumes and areas, and so-on. Smooth
        manifolds have no means of making such measurements. To do so we need
        an \textit{angle measuring device}. In $\mathbb{R}^{n}$ this is given
        by the Euclidean dot product. We define:
        \begin{equation}
            \mathbf{x}\cdot\mathbf{y}=
                \sum_{k=0}^{n-1}x_{k}y_{k}
        \end{equation}
        From this we may define lengths.
        \begin{equation}
            ||\mathbf{x}||=\sqrt{\mathbf{x}\cdot\mathbf{x}}
        \end{equation}
        and from this we obtain angles.
        \begin{equation}
            \angle(\mathbf{x},\,\mathbf{y})=
            \arccos\Big(
                \frac{\mathbf{x}\cdot\mathbf{y}}{||\mathbf{x}||\,||\mathbf{y}||}
            \Big)
        \end{equation}
        Geometry begins with the ability to define a \textit{dot-product}
        on the tangent vectors of a smooth manifold. This is given via a
        \textit{Riemannian metric}. The rest of these notes attempt to define
        this notion.
        \par\hfill\par
        A \textit{vector field} $V$ on a smooth manifold
        $(X,\,\tau,\,\mathcal{A})$ is an assignment of a tangent vector
        $v\in{T}_{x}X$ to every $x\in{X}$. That is, at every point in the space
        we choose an \textit{arrow} that starts at that point. If
        $v\in{T}_{x}X$ is the tangent vector assigned to $x\in{X}$ we denote
        this by $v=V_{x}$.
        \par\hfill\par
        Given such a vector field, if $f\in{C}^{\infty}(X,\,\mathbb{R})$ is a
        smooth function, we obtain another function $Vf:X\rightarrow\mathbb{R}$
        via:
        \begin{equation}
            (Vf)(x)=V_{x}(f)
        \end{equation}
        Remember that $V_{x}$ is a tangent vector at $x$, meaning it is a
        derivation $v:C^{\infty}(X,\,\mathbb{R})\rightarrow\mathbb{R}$. That is,
        it takes in smooth functions and returns a real number. Because of this
        the above equation is well-defined.
        The vector field is said to be \textit{smooth} if $Vf$ is a smooth
        function for all $f\in{C}^{\infty}(X,\,\mathbb{R})$.
        \par\hfill\par
        A \textit{Riemannian metric} is an assigment to every point
        $x\in{X}$ a function $g_{x}:T_{x}X\times{T}_{x}X\rightarrow\mathbb{R}$
        that mimics the definition of inner products. That is:
        \begin{align}
            g_{x}(a_{0}v_{0}+a_{1}v_{1},\,w)
            &=a_{0}g_{x}(v_{0},\,w)+a_{1}g_{x}(v_{1},\,w)\tag{Linearity}\\
            g_{x}(v,\,w)&=g_{x}(w,\,v)\tag{Symmetry}\\
            g_{x}(v,\,v)&\geq{0}\tag{Positivity}\\
            g_{x}(v,\,v)&=0\Leftrightarrow{v}=0\tag{Definiteness}
        \end{align}
        Moreover, this assignment is done \textit{smoothly}. This is,
        for every pair of smooth vector fields $V,W$ on $X$, the function
        $g(V,\,W):X\rightarrow\mathbb{R}$ defined by:
        \begin{equation}
            \big(g(V,\,W)\big)(x)=g_{x}(V_{x},\,W_{x})
        \end{equation}
        is smooth. A \textit{Riemannian manifold} is a smooth manifold together
        with a Riemannian metric.
        \par\hfill\par
        While not every topological manifold is \textit{smoothable}, it is true
        that every smooth manifold can be made into a Riemannian metric. This
        is one of the more important applications of partitions of unity and
        paracompactness. Every smooth manifold is paracompact, and hence every
        open cover has a subordinate partition of unity. Moreover, for smooth
        manifolds the partition of unity can be chosen to consist solely of
        \textit{smooth} functions. This new fact can be used to build a
        Riemannian metric on any smooth manifold.
\end{document}
