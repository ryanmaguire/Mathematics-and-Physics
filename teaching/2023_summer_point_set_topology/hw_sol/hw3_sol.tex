%-----------------------------------LICENSE------------------------------------%
%   This file is part of Mathematics-and-Physics.                              %
%                                                                              %
%   Mathematics-and-Physics is free software: you can redistribute it and/or   %
%   modify it under the terms of the GNU General Public License as             %
%   published by the Free Software Foundation, either version 3 of the         %
%   License, or (at your option) any later version.                            %
%                                                                              %
%   Mathematics-and-Physics is distributed in the hope that it will be useful, %
%   but WITHOUT ANY WARRANTY; without even the implied warranty of             %
%   MERCHANTABILITY or FITNESS FOR A PARTICULAR PURPOSE.  See the              %
%   GNU General Public License for more details.                               %
%                                                                              %
%   You should have received a copy of the GNU General Public License along    %
%   with Mathematics-and-Physics.  If not, see <https://www.gnu.org/licenses/>.%
%------------------------------------------------------------------------------%
\documentclass{article}
\usepackage{graphicx}                           % Needed for figures.
\usepackage{amsmath}                            % Needed for align.
\usepackage{amssymb}                            % Needed for mathbb.
\usepackage{amsthm}                             % For the theorem environment.
\usepackage{xcolor}                             % Coloring text.

%------------------------Theorem Styles-------------------------%

% Define theorem style for default spacing and normal font.
\newtheoremstyle{normal}
    {\topsep}               % Amount of space above the theorem.
    {\topsep}               % Amount of space below the theorem.
    {}                      % Font used for body of theorem.
    {}                      % Measure of space to indent.
    {\bfseries}             % Font of the header of the theorem.
    {}                      % Punctuation between head and body.
    {.5em}                  % Space after theorem head.
    {}

% Define default environments.
\theoremstyle{normal}
\newtheorem{problem}{Problem}

\title{Point-Set Topology: Homework 3}
\date{Summer 2023}

% No indent and no paragraph skip.
\setlength{\parindent}{0em}
\setlength{\parskip}{0em}

\begin{document}
    \maketitle
    \color{blue}
    \begin{problem}
        \textbf{(Separability)}
        \par\hfill\par
        A separable topological space is a space $(X,\,\tau)$ such that there
        is a countable subset $A\subseteq{X}$ such that
        $\textrm{Cl}_{\tau}(A)=X$.
        A metric space is separable if and only if it is second-countable. This
        feature is special to metric spaces. Take $\mathbb{R}$ with the
        standard topology, and equip $\mathbb{R}/\mathbb{Z}$ with the
        quotient topology. Intuitively this is infinite many circles all
        touching at $0$. It is not first-countable, and hence not
        second-countable, even though $\mathbb{R}$ is. It is still separable.
        \begin{itemize}
            \item (6 Points) Let $(X,\,\tau)$ be a separable topological space.
                Let $R$ be any equivalence relation on $X$. Prove that
                $(X/R,\,\tau_{X/R})$ is separable. That is, separability is
                a topological property preserved by quotients.
        \end{itemize}
    \end{problem}
    \color{black}
    \begin{proof}[Solution]
        Since $(X,\,\tau)$ is separable, there is a countable dense subset
        $A$. Let $q:X\rightarrow{X}/R$ be the quotient map, $q(x)=[x]$ where
        $[x]$ is the equivalence class of $x$, and let $B=q[A]$. Since $B$ is
        the image of a countable set, it too is countable. We must prove
        $\textrm{Cl}_{\tau_{X/R}}(B)=X/R$. Let
        $\tilde{\mathcal{C}}\subseteq{X}/R$
        be a closed set containing $B$. But $q$ is continuous, and so the
        pre-image of closed sets is closed, meaning
        $q^{-1}[\tilde{\mathcal{C}}]$ is closed. Let
        $\mathcal{C}=q^{-1}[\tilde{\mathcal{C}}]$. Then $\mathcal{C}$ is a
        closed set that contains $A$ since $q[A]=B$. But if
        $A\subseteq\mathcal{C}$ and $\mathcal{C}$ is closed, then
        $\textrm{Cl}_{\tau}(A)\subseteq\mathcal{C}$. But
        $\textrm{Cl}_{\tau}(A)=X$, and hence $\mathcal{C}=X$. But then, since
        quotient maps are surjective, we have that
        $q[\mathcal{C}]=q[X]=X/R$. But
        $q[\mathcal{C}]\subseteq\tilde{\mathcal{C}}$, by definition of
        $\mathcal{C}$ and $\tilde{\mathcal{C}}$, and hence
        $\tilde{\mathcal{C}}=X/R$. That is, if $\tilde{\mathcal{C}}$ is a
        closed subset of $X/R$ such that $B\subseteq\tilde{\mathcal{C}}$, then
        $\tilde{\mathcal{C}}=X/R$. Hence
        $\textrm{Cl}_{\tau_{X/R}}(B)=X/R$, so $B$ is a countable dense subset
        and $(X/R,\,\tau_{X/R})$ is separable.
    \end{proof}
    \newpage
    \color{blue}
    \begin{problem}
        \textbf{(Embeddings)}
        \par\hfill\par
        The bug-eyed line is a quotient space of
        $X=\mathbb{R}\times\{\,0,\,1\,\}$ where $\mathbb{R}$ has the standard
        Euclidean topology and $\{\,0,\,1\,\}$ has the discrete topology. $X$
        is given the product topology. We identity
        $(x,\,0)$ with $(x,\,1)$ for all $x\ne{0}$ and then take the quotient
        of $X$ under this relation. This idea is shown in
        Fig.~\ref{fig:bug_eyed_line}
        \begin{figure}
            \centering
            \includegraphics{../../../images/bug_eyed_line_001.pdf}
            \caption{The Bug-Eyed Line Construction}
            \label{fig:bug_eyed_line}
        \end{figure}
        \begin{itemize}
            \item (6 Points) Prove that it is impossible to embed
            the bug-eyed line into $\mathbb{R}^{n}$ for all $n\in\mathbb{N}$.
        \end{itemize}
    \end{problem}
    \color{black}
    \begin{proof}[Solution]
        This space is not Hausdorff. For let $q:X\rightarrow{X}/R$ be the
        canonical quotient map, and define $0'=q((0,\,0))$ and
        $0''=q((0,\,1))$. These are the two origins in the buy-eyed line.
        Let $\mathcal{U}\subseteq{X}/R$ be an open set about $0'$ and
        $\mathcal{V}\subseteq{X}/R$ be an open set about $0''$. Then
        $q^{-1}[\mathcal{U}],q^{-1}[\mathcal{V}]\subseteq{X}$ are open
        subsets of $X$ since $q$ is continuous. Moreover, $(0,\,0)$ is an
        element of $q^{-1}[\mathcal{U}]$ and $(0,\,1)$ is an element of
        $q^{-1}[\mathcal{V}]$. But the topology on $X$ is the product topology
        from $\mathbb{R}$ and $\mathbb{Z}_{2}=\{\,0,\,1\,\}$, the latter given
        the discrete topology. So an open subset about
        $(0,\,0)$ must contain all points between $(-\varepsilon,\,0)$ and
        $(\varepsilon,\,0)$. A similar statement can be made for
        $(0,\,1)$. Since $(x,\,0)$ and $(x,\,1)$ are identified by the relation
        for all $x\ne{0}$, projecting these open intervals down to the quotient
        space shows that $\mathcal{U}$ and $\mathcal{V}$ must overlap.
        Hence $(X/R,\,\tau_{X/R})$ is not Hausdorff. But if
        $f:X/R\rightarrow\mathbb{R}^{n}$ is an embedding, then
        $X/R$ is homeomorphic to the subspace $f[X/R]\subseteq\mathbb{R}^{n}$.
        But $\mathbb{R}^{n}$ is Hausdorff, so all of its subspaces are
        Hausdorff. So no such embedding could possibly exist.
        \par\hfill\par
    \end{proof}
    Fig.~\ref{fig:bug_eyed_line_open} provides a visual of the description of
    open sets given in solution. Any open set containing the first origin
    must overlap with any open set containing the second. We imagine the
    bug-eyed line as the real line with an extra origin that is almost
    indistinguishable from the first. This space is Fr\'{e}chet, however.
    The two origins are indeed still closed.
    \begin{figure}
        \centering
        \includegraphics{../../../images/bug_eyed_line_002.pdf}
        \caption{Open Sets in the Bug-Eyed Line}
        \label{fig:bug_eyed_line_open}
    \end{figure}
    \newpage
    \color{blue}
    \begin{problem}
        \textbf{(Quotients)}
        \par\hfill\par
        Let $X=\mathbb{R}/\mathbb{Q}$, equipped with the quotient topology where
        $\mathbb{R}$ carries the usual Euclidean topology.
        \begin{itemize}
            \item (4 Points) Is this space Hausdorff? Is it Fr\'{e}chet?
        \end{itemize}
    \end{problem}
    \color{black}
    \begin{proof}[Solution]
        This space is not Hausdorff. A subset of $\mathbb{R}/\mathbb{Q}$ is
        open if and only if the pre-image is open. This is one of the
        defining characteristics of the quotient map $q$.
        Let $[x],[y]\in\mathbb{R}/\mathbb{Q}$. There are three possibilities.
        Both $x$ and $y$ are irrational, only one of $x$ and $y$ are irrational,
        and both $x$ and $y$ are rational. If $x$ and $y$ are rational, then
        $[x]=[y]$, so we may discard this possibility. Suppose both
        $x$ and $y$ are irrational. Let $\mathcal{U},\mathcal{V}$ be
        open sets about $[x]$ and $[y]$, respectively. Then
        $q^{-1}[\mathcal{U}]$ and $q^{-1}[\mathcal{V}]$ are open sets
        containing $x$ and $y$, respectively. But open sets in $\mathbb{R}$
        are described by the metric. So there is some $\varepsilon_{x}>0$ and
        some $\varepsilon_{y}>0$ such that
        $(x-\varepsilon_{x},\,x+\varepsilon_{x})\subseteq{q}^{-1}[\mathcal{U}]$
        and
        $(y-\varepsilon_{y},\,y+\varepsilon_{y})\subseteq{q}^{-1}[\mathcal{V}]$.
        But there must be a rational number between
        $x+\varepsilon_{x}$ and a rational number between $y+\varepsilon_{y}$.
        But all rationals are identified together by the equivalence relation,
        meaning these rationals map to the same point under $q$. Hence
        $\mathcal{U}$ and $\mathcal{V}$ must overlap, so $[x]$ and $[y]$ can not
        be separated by open sets.
        \par\hfill\par
        If $x$ is irrational and $y$ is rational, the argument is almost
        identical. Any open set about $[x]$ must contain $[y]$ by the previous
        argument, and hence $[x]$ and $[y]$ can not be separated by open sets
        either. So $\mathbb{R}/\mathbb{Q}$ is not Hausdorff.
        \par\hfill\par
        The space is also not Fr\'{e}chet. Given irrational numbers $x$ and
        $y$ we can indeed find open sets for $[x]$ and $[y]$ satisfying the
        Fr\'{e}chet condition. Namely, set
        $\mathcal{U}=\mathbb{R}\setminus\{\,y\,\}$ and
        $\mathcal{V}=\mathbb{R}\setminus\{\,x\,\}$. These sets are open in
        $\mathbb{R}$ and moreover they are saturated with respect to $q$.
        Hence $q[\mathcal{U}]$ and $q[\mathcal{V}]$ are open sets with
        $[x]\in{q}[\mathcal{U}]$, $[x]\notin{q}[\mathcal{V}]$, and
        $[y]\in{q}[\mathcal{V}]$ and $[y]\notin{q}[\mathcal{U}]$. The
        problem arises when we consider $x$ irrational and $y$ rational. When
        exploring the Hausdorff property we saw that any open set containing
        $[x]$ must also contain $[y]$, and hence the Fr\'{e}chet condition
        cannot be satisfied. $\mathbb{R}/\mathbb{Q}$ is neither Hausdorff nor
        Fr\'{e}chet.
    \end{proof}
    \newpage
    \color{blue}
    \begin{problem}
        \textbf{(Products)}
        \par\hfill\par
        Consider topological spaces $(X,\,\tau_{X})$, $(Y,\,\tau_{Y})$, and
        $(Z,\,\tau_{Z})$. Equip $X\times{Y}$ with the product topology
        $\tau_{X\times{Y}}$.
        \begin{itemize}
            \item (6 Points) Prove that a function
                $f:Z\rightarrow{X}\times{Y}$ is continuous if and only if the
                component functions $\textrm{proj}_{X}\circ{f}:Z\rightarrow{X}$
                and $\textrm{proj}_{Y}\circ{f}:Z\rightarrow{Y}$ are continuous.
        \end{itemize}
    \end{problem}
    \color{black}
    \begin{proof}[Solution]
        One direction is easier than the other. If $f$ is continuous, then
        since projections are continuous, $\textrm{proj}_{X}\circ{f}$ and
        $\textrm{proj}_{Y}\circ{f}$ are the compositions of continuous
        functions, which are therefore continuous. That is, if $f$ is continuous,
        then so are the component functions. In the other direction, suppose
        the component functions are continuous. Let $z\in{Z}$ and
        $\mathcal{W}\in\tau_{X\times{Y}}$ be any open set containing
        $f(z)$. Since $\tau_{X\times{Y}}$ has as a basis the set of all open
        rectangles, there must be some open sets $\mathcal{U}\in\tau_{X}$ and
        $\mathcal{V}\in\tau_{Y}$ such that $f(z)\in\mathcal{U}\times\mathcal{V}$
        and $\mathcal{U}\times\mathcal{V}\subseteq\mathcal{W}$. But
        $\textrm{proj}_{X}\circ{f}$ is continuous, and
        $(\textrm{proj}_{X}\circ{f})(z)\in\mathcal{U}$, so there is an open set
        $\mathcal{E}_{X}$ such that $z\in\mathcal{E}_{X}$ and
        $(\textrm{proj}_{X}\circ{f})[\mathcal{E}_{X}]\subseteq\mathcal{U}$.
        Similarly there exists a set $\mathcal{E}_{Y}$ for
        $\textrm{proj}_{Y}\circ{f}$. Let
        $\mathcal{E}=\mathcal{E}_{X}\cap\mathcal{E}_{Y}$. Then $\mathcal{E}$
        is open, being the intersection of two open sets, and
        $z\in\mathcal{E}$. Moreover,
        $f[\mathcal{E}]\subseteq\mathcal{U}\times\mathcal{V}$. For let
        $(x,\,y)\in{f}[\mathcal{E}]$. Then
        $\textrm{proj}_{X}\big((x,\,y)\big)=x$ and hence $x\in\mathcal{U}$, and
        $\textrm{proj}_{Y}\big((x,\,y)\big)=y$ and so $y\in\mathcal{V}$. But
        then $(x,\,y)\in\mathcal{U}\times\mathcal{V}$, meaning
        $f[\mathcal{E}]\subseteq\mathcal{U}\times\mathcal{V}$. But
        $\mathcal{U}\times\mathcal{V}\subseteq\mathcal{W}$, and hence
        $f[\mathcal{E}]\subseteq\mathcal{W}$. That is, for all
        $z\in{Z}$ and for every open set $\mathcal{W}$ containing $f(z)$ there
        is an open set $\mathcal{E}\subseteq{Z}$ such that
        $z\in\mathcal{E}$ and $f[\mathcal{E}]\subseteq\mathcal{W}$. So $f$ is
        continuous.
    \end{proof}
\end{document}
