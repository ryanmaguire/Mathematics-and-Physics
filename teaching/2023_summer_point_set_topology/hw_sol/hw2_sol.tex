%-----------------------------------LICENSE------------------------------------%
%   This file is part of Mathematics-and-Physics.                              %
%                                                                              %
%   Mathematics-and-Physics is free software: you can redistribute it and/or   %
%   modify it under the terms of the GNU General Public License as             %
%   published by the Free Software Foundation, either version 3 of the         %
%   License, or (at your option) any later version.                            %
%                                                                              %
%   Mathematics-and-Physics is distributed in the hope that it will be useful, %
%   but WITHOUT ANY WARRANTY; without even the implied warranty of             %
%   MERCHANTABILITY or FITNESS FOR A PARTICULAR PURPOSE.  See the              %
%   GNU General Public License for more details.                               %
%                                                                              %
%   You should have received a copy of the GNU General Public License along    %
%   with Mathematics-and-Physics.  If not, see <https://www.gnu.org/licenses/>.%
%------------------------------------------------------------------------------%
\documentclass{article}
\usepackage{graphicx}                           % Needed for figures.
\usepackage{amsmath}                            % Needed for align.
\usepackage{amssymb}                            % Needed for mathbb.
\usepackage{amsthm}                             % For the theorem environment.
\usepackage{xcolor}                             % Coloring text.

%------------------------Theorem Styles-------------------------%

% Define theorem style for default spacing and normal font.
\newtheoremstyle{normal}
    {\topsep}               % Amount of space above the theorem.
    {\topsep}               % Amount of space below the theorem.
    {}                      % Font used for body of theorem.
    {}                      % Measure of space to indent.
    {\bfseries}             % Font of the header of the theorem.
    {}                      % Punctuation between head and body.
    {.5em}                  % Space after theorem head.
    {}

% Define default environments.
\theoremstyle{normal}
\newtheorem{problem}{Problem}

\title{Point-Set Topology: Homework 2}
\date{Summer 2023}

% No indent and no paragraph skip.
\setlength{\parindent}{0em}
\setlength{\parskip}{0em}

\begin{document}
    \maketitle
    \color{blue}
    \begin{problem}
        \textbf{(Subspaces)}
        \par\hfill\par
        The inclusion mapping of a subset $A\subseteq{X}$ into $X$ is the
        function $\iota_{A}:A\rightarrow{X}$ defined by $\iota_{A}(x)=x$.
        \begin{itemize}
            \item (1 Point) Prove that if $(X,\,d)$ is a metric space, and if
                $(A,\,d_{A})$ is a metric subspace, then $\iota_{A}$ is
                continuous.
            \item (3 Points) Suppose $(X,\,d_{X})$ and $(Y,\,d_{Y})$ are
                metric spaces and let $A\subseteq{X}$ be a subset and
                $d_{A}$ be the subspace metric. Prove that $f:Y\rightarrow{A}$
                is continuous \textit{if and only if}
                $\iota_{A}\circ{f}:Y\rightarrow{X}$ is continuous.
            \item (2 Points)
                Prove that if $f:X\rightarrow{Y}$ is a continuous function
                from a metric space $(X,\,d_{X})$ to a metric space
                $(Y,\,d_{Y})$, and if $(A,\,d_{A})$ is a subspace of
                $(X,\,d_{X})$, then the restriction $f|_{A}:A\rightarrow{Y}$,
                defined by $f|_{A}(x)=f(x)$, is continuous.
            \item (4 Points) Prove that if $f:X\rightarrow{Y}$ is a
                homeomorphism, and if $A\subseteq{X}$, then
                $f|_{A}:A\rightarrow{f}[A]$ is a homeomorphism.
        \end{itemize}
    \end{problem}
    \color{black}
    \begin{proof}[Solution]
        Let $a:\mathbb{N}\rightarrow{A}$ be a convergent sequence with
        $a_{n}\rightarrow{x}$ for some $x\in{A}$. That is,
        $d_{A}(a_{n},\,x)\rightarrow{0}$. But then:
        \begin{align}
            d\big(\iota_{A}(a_{n}),\,\iota_{A}(x)\big)
            &=d_{A}\big(\iota_{A}(a_{n}),\,\iota_{A}(x)\big)
                \tag{Definition of $d_{A}$}\\
            &=d_{A}(a_{n},\,x)\tag{Definition of $\iota_{A}$}
        \end{align}
        and hence $d(\iota_{A}(a_{n}),\,\iota_{A}(x))\rightarrow{0}$, so
        $\iota_{A}(a_{n})\rightarrow\iota_{A}(x)$,
        meaning $\iota_{A}$ is continuous.
        \par\hfill\par
        Suppose $f:Y\rightarrow{A}$ is continuous. Then since
        $\iota_{A}:A\rightarrow{X}$ is continuous,
        $\iota_{A}\circ{f}$ is the composition of continuous functions, which
        is continuous. In the other direction, suppose
        $\iota_{A}\circ{f}$ is continuous. Let $a:\mathbb{N}\rightarrow{Y}$ be
        a convergent sequence with limit $y\in{Y}$. We must prove that
        $f(a_{n})\rightarrow{f}(y)$. Let $\varepsilon>0$ be given. Since, by
        hypothesis, $\iota_{A}\circ{f}$ is continuous, there is an
        $N\in\mathbb{N}$ such that $n\in\mathbb{N}$ and $n>N$ implies
        $d\big((\iota_{A}\circ{f})(a_{n}),\,(\iota_{A}\circ{f})(y)\big)<\varepsilon$.
        But then, by definition of $d_{A}$ and $\iota_{A}$, we have that $n>N$
        implies $d_{A}\big(f(a_{n}),\,f(y)\big)<\varepsilon$, and therefore
        $f(a_{n})\rightarrow{f}(y)$. That is, $f:Y\rightarrow{A}$ is continuous.
        \par\hfill\par
        For part 3, let $a:\mathbb{N}\rightarrow{A}$ be a convergent sequence
        with limit $x\in{A}$. We must prove
        $f|_{A}(a_{n})\rightarrow{f}|_{A}(x)$. Let $\varepsilon>0$. Since
        $A\subseteq{X}$ and $a:\mathbb{N}\rightarrow{A}$ is a convergent
        sequence in $A$, $a:\mathbb{N}\rightarrow{X}$ is a convergent sequence
        in $X$ as well with the same limit. But $f:X\rightarrow{Y}$ is
        continuous, so $f(a_{n})\rightarrow{f}(x)$. But then, since
        $\varepsilon>0$, there is an $N\in\mathbb{N}$ such that $n\in\mathbb{N}$
        and $n>N$ implies $d_{Y}\big(f(a_{n}),\,f(x)\big)<\varepsilon$. But
        $f|_{A}(a_{n})=f(a_{n})$ and $f|_{A}(x)=f(x)$, so $n>N$ implies
        $d_{Y}\big(f|_{A}(a_{n}),\,f|_{A}(x)\big)<\varepsilon$. Hence
        $f|_{A}(a_{n})\rightarrow{f}|_{A}(x)$ and $f|_{A}$ is continuous.
        \par\hfill\par
        Lastly, we are to prove the restriction of a homeomorphism to a subspace
        yields a homeomorphism to the image. The restriction is continuous by
        the previous problem. Since $f:X\rightarrow{Y}$ is bijective, it is
        injective, and hence $f|_{A}$ is injective as well. Since the co-domain
        is $f[A]$, the function is also surjective. So
        $f|_{A}:A\rightarrow{f}[A]$ is a continuous bijection. We must prove
        the inverse function is continuous. But the inverse function
        $(f|_{A})^{-1}$ is $f^{-1}|_{f[A]}$, the restriction of $f^{-1}$ to
        $f[A]$. But $f^{-1}$ is continuous since $f$ is a homeomorphism. But
        then $f^{-1}|_{f[A]}$ is the restriction of a continuous function to a
        subspace, meaning it is continuous. Therefore
        $(f|_{A})^{-1}$ is continuous, and $f|_{A}:A\rightarrow{f[A]}$ is a
        homeomorphism.
    \end{proof}
    \newpage
    \color{blue}
    \begin{problem}
        \textbf{(Continuity)}
        \par\hfill\par
        We have proven the equivalence of three definitions of continuity.
        The definition is that $f$ maps convergent sequences to convergent
        sequences. The calculus $\varepsilon-\delta$ statement is equivalent to
        this, as is the fact that the pre-image of open sets is open. Continuity
        can be described by forward images as well.
        \begin{itemize}
            \item (6 Points) Let $(X,\,d_{X})$ and $(Y,\,d_{Y})$ be metric
                spaces. Prove that $f:X\rightarrow{Y}$ is continuous if and only
                if for all $x\in{X}$ and for all open subsets
                $\mathcal{V}\subseteq{Y}$ with $f(x)\in\mathcal{V}$ there is an
                open subset $\mathcal{U}\subseteq{X}$ such that
                $x\in\mathcal{U}$ and $f[\mathcal{U}]\subseteq\mathcal{V}$.
        \end{itemize}
    \end{problem}
    \color{black}
    \begin{proof}[Solution]
        It is easiest to prove this using the fact that a function is continuous
        if and only if the pre-image of an open set is open. First suppose
        $f:X\rightarrow{Y}$ is continuous. Let $x\in{X}$ and let
        $\mathcal{V}\subseteq{Y}$ be an open set such that $f(x)\in\mathcal{V}$.
        Since $f$ is continous we have that $\mathcal{U}=f^{-1}[\mathcal{V}]$
        is open. But then $x\in\mathcal{U}$ and
        $f[\mathcal{U}]\subseteq\mathcal{V}$, by definition of images and
        pre-images. Hence a continuous function has the desired property.
        \par\hfill\par
        In the other direction, suppose $f:X\rightarrow{Y}$ is such that for all
        $x\in{X}$ and for all open sets $\mathcal{V}\subseteq{Y}$ with
        $f(x)\in\mathcal{V}$ there exists an open set $\mathcal{U}\subseteq{X}$
        such that $x\in\mathcal{U}$ and $f[\mathcal{U}]\subseteq\mathcal{V}$.
        We must prove that the pre-image of an open set is open in order to
        conclude that $f$ is continuous. Let $\mathcal{V}\subseteq{Y}$ be open.
        If $\mathcal{V}$ is empty we are done, since
        $f^{-1}[\emptyset]=\emptyset$. Similarly, if
        $f^{-1}[\mathcal{V}]=\emptyset$ there is nothing to prove. So suppose
        $\mathcal{V}\subseteq{Y}$ is an open set whose pre-image is not empty.
        Let $x\in{f}^{-1}[\mathcal{V}]$. Then, by definition of pre-image,
        $f(x)\in\mathcal{V}$. By hypothesis there is then open open set
        $\mathcal{U}\subseteq{X}$ such that $x\in\mathcal{U}$ and
        $f[\mathcal{U}]\subseteq\mathcal{V}$. But if $\mathcal{U}$ is open and
        $x\in\mathcal{U}$, then there is an $\varepsilon>0$ such that
        $B_{\varepsilon}^{(X,\,d_{X})}(x)\subseteq\mathcal{U}$. But then
        $B_{\varepsilon}^{(X,\,d_{X})}(x)\subseteq{f}^{-1}[\mathcal{V}]$. That
        is, for all $x\in{f}^{-1}[\mathcal{V}]$ there is an $\varepsilon$ ball
        about $x$ contained entirely inside of $f^{-1}[\mathcal{V}]$, and hence
        $f^{-1}[\mathcal{V}]$ is open. Therefore $f$ is continuous.
    \end{proof}
    \newpage
    \color{blue}
    \begin{problem}
        \textbf{(Compact Spaces)}
        \par\hfill\par
        For metric spaces there are many equivalent ways of defining
        compactness. Your job is to prove some of these equivalences.
        \begin{itemize}
            \item (4 Points) Prove that $(X,\,d)$ is compact if and only if
                for every sequence of closed non-empty nested sets, the
                intersection is non-empty. That is, if
                $\mathcal{C}:\mathbb{N}\rightarrow\mathcal{P}(X)$ is a sequence
                of closed sets such that $\mathcal{C}_{n}\ne\emptyset$ and
                $\mathcal{C}_{n+1}\subseteq\mathcal{C}_{n}$, then
                $\bigcap_{n\in\mathbb{N}}\mathcal{C}_{n}$ is non-empty.
            \item (2 Points) Prove that $(X,\,d)$ is compact if and only if for
                every sequence of nested proper open subsets, the union is not
                the whole space. That is, if
                $\mathcal{U}:\mathbb{N}\rightarrow\mathcal{P}(X)$ is a sequence
                of open sets such that $\mathcal{U}_{n}\ne{X}$ and
                $\mathcal{U}_{n}\subseteq\mathcal{U}_{n+1}$, then
                $\bigcup_{n\in\mathbb{N}}\mathcal{U}_{n}$ is not equal to $X$.
                [Hint: What is the complement of an open set? Can the previous
                part of the problem help?]
        \end{itemize}
    \end{problem}
    \color{black}
    \begin{proof}[Solution]
        There are a few ways to prove this. Let's use sequences first.
        Suppose $(X,\,d)$ is compact and
        $\mathcal{C}:\mathbb{N}\rightarrow\mathcal{P}(X)$ is a sequence of
        nested closed non-empty sets. Then for each
        $n\in\mathbb{N}$, since $\mathcal{C}_{n}$ is non-empty, there is an
        $a_{n}\in\mathcal{C}_{n}$. Since $(X,\,d)$ is compact and
        $a:\mathbb{N}\rightarrow{X}$ is a sequence in $X$, there is a
        convergent subsequence $a_{k}$. Let $x\in{X}$ be the limit. Then
        for all $n\in\mathbb{N}$, $x\in\mathcal{C}_{n}$. To see this, given
        $N\in\mathbb{N}$, for all $n>N$ we have
        $a_{k_{n}}\in\mathcal{C}_{k_{n}}$, and since $k_{n}>N$ this implies
        $a_{k_{n}}\in\mathcal{C}_{N}$ since the sets are nested. So
        $a_{k}$ is convergent sequence that is eventually contained in
        $\mathcal{C}_{n}$, and $\mathcal{C}_{n}$ is closed so it contains its
        limit points, and therefore $x\in\mathcal{C}_{n}$. Since this is true
        for all $n\in\mathbb{N}$, we have that
        $x\in\bigcap_{n\in\mathbb{N}}\mathcal{C}_{n}$. That is, the intersection
        is non-empty.
        \par\hfill\par
        Now suppose $(X,\,d)$ is a metric space with the property that for
        all nested sequences of non-empty closed sets
        $\mathcal{C}:\mathbb{N}\rightarrow\mathcal{P}(X)$ it is true that
        $\bigcap_{n\in\mathbb{N}}\mathcal{C}_{n}$ is non-empty. Let us prove
        that $(X,\,d)$ is compact. Suppose not. Then there is a sequence
        $a:\mathbb{N}\rightarrow{X}$ with no convergent subsequence. Then the
        set:
        \begin{equation}
            A=\{a_{n}\in{X}\;|\;n\in\mathbb{N}\,\}
        \end{equation}
        is closed. For if not then there is a point $x\in{X}$ that is a limit
        point of this set, but is not contained in it. But if $x$ is a
        limit point of $A$ then there is a sequence of points in this set that
        converges to $x$. But then the sequence
        $a:\mathbb{N}\rightarrow{X}$ would have a convergent subsequence,
        contradicting the claim that no such subsequence exists. Hence
        $A=\{\,a_{n}\in{X}\;|\;n\in\mathbb{N}\,\}$ is closed.
        \par\hfill\par
        Alternatively, if you'd rather, we can show $X\setminus{A}$ is open.
        Give $x\in{X}\setminus{A}$ there must be some $\varepsilon>0$ such
        that $B_{\varepsilon}^{(X,\,d)}(x)\cap{A}=\emptyset$. Otherwise, if for
        all $\varepsilon>0$ there is an $n\in\mathbb{N}$ such that
        $d(x,\,a_{n})<\varepsilon$ we could find a subsequence of $a$ that
        converges to $x$, contradicting the claim that $a$ has no such
        subsequences. So about every element of $X\setminus{A}$ we can place an
        $\varepsilon$ ball contained entirely inside of $X\setminus{A}$, and
        hence this set is open. Since $X\setminus{A}$ is open, $A$ is closed.
        \par\hfill\par
        Moreover, $A\setminus\{\,a_{0}\,\}$ is closed. Since no subsequence
        of $a$ converges to $a_{0}$ there must be some
        $\varepsilon>0$ such that
        $B_{\varepsilon}^{(X,\,d)}(a_{0})\cap{A}=\{\,a_{0}\,\}$.
        Since open balls are open, the set
        $A\setminus{B}_{\varepsilon}^{(X,\,d)}(a_{0})$ is the difference of an
        open set from a closed set, which is closed.
        Since $B_{\varepsilon}^{(X,\,d)}(a_{0})\cap{A}=\{\,a_{0}\,\}$ we have
        that
        $A\setminus{B}_{\varepsilon}^{(X,\,d)}(a_{0})=A\setminus\{\,a_{0}\,\}$.
        So $A\setminus\{\,a_{0}\,\}$ is closed. Even more, for all
        $n\in\mathbb{N}$ if we define the set:
        \begin{equation}
            B_{n}=\{\,a_{0},\,\dots,\,a_{n-1}\,\}
        \end{equation}
        then the set $A\setminus{B}_{n}$ is closed.
        Denote this set by $\mathcal{C}_{n}$:
        \begin{equation}
            \mathcal{C}_{n}=A\setminus{B}_{n}
        \end{equation}
        But then
        $\mathcal{C}:\mathbb{N}\rightarrow\mathcal{P}(X)$ is a nested sequence
        of non-empty closed sets, so the intersection is non-empty by
        hypothesis. But:
        \begin{align}
            \bigcap_{n\in\mathbb{N}}\mathcal{C}_{n}
            &=\bigcap_{n\in\mathbb{N}}(A\setminus{B}_{n})\\
            &=A\setminus\bigcup_{n\in\mathbb{N}}B_{n}\\
            &=A\setminus{A}\\
            &=\emptyset
        \end{align}
        A contradiction. Hence $(X,\,d)$ is compact.
        \par\hfill\par
        As mentioned there are many ways to prove this claim. Let's use
        open sets. We proved that a metric space is compact if and only if
        for every open cover $\mathcal{O}$ of $X$ there is a finite subcover
        $\Delta\subseteq\mathcal{O}$. Suppose $(X,\,d)$ is compact and let
        $\mathcal{C}:\mathbb{N}\rightarrow\mathcal{P}(X)$ be a nested sequence
        of closed non-empty sets with the property that
        $\bigcap_{n\in\mathbb{N}}\mathcal{C}_{n}=\emptyset$. Define:
        \begin{equation}
            \mathcal{U}_{n}=X\setminus\mathcal{C}_{n}
        \end{equation}
        Then the set $\mathcal{O}=\{\,\mathcal{U}_{n}\;|\;n\in\mathbb{N}\,\}$ is
        and open cover of $X$ since:
        \begin{align}
            X&=X\setminus\emptyset\\
            &=X\setminus\bigcap_{n\in\mathbb{N}}\mathcal{C}_{n}\\
            &=\bigcup_{n\in\mathbb{N}}(X\setminus\mathcal{C}_{n})\\
            &=\bigcup_{n\in\mathbb{N}}\mathcal{U}_{n}\\
            &=\bigcup\mathcal{O}
        \end{align}
        But $(X,\,d)$ is compact, so there is a finite subcover
        $\Delta=\{\mathcal{U}_{n_{0}},\,\dots,\,\mathcal{U}_{n_{m}}\,\}$.
        Let $\mathcal{U}_{N}$ be the element of $\Delta$ with the largest
        index. Since the sets $\mathcal{C}_{n}$ are nested,
        $\mathcal{C}_{n+1}\subseteq\mathcal{C}_{n}$, so are the sets
        $\mathcal{U}_{n}$. That is,
        $\mathcal{U}_{n}\subseteq\mathcal{U}_{n+1}$. But then
        $\{\,\mathcal{U}_{N}\,\}$ is an open cover of $X$, meaning
        $\mathcal{U}_{N}=X$. But then
        $\mathcal{C}_{N}=X\setminus\mathcal{U}_{N}$ is empty, contradicting
        the fact that all $\mathcal{C}_{n}$ are non-empty.
        \par\hfill\par
        To use open sets for the converse of this statement we need the fact
        that a metric space is compact if and only if it is
        \textit{countably} compact. That is, for every
        \textit{countable} open cover $\mathcal{O}$ of $X$ there is a finite
        subcover $\Delta\subseteq\mathcal{O}$ of $X$. Compactness certainly
        implies countable compactness, since we can consider arbitrary open
        covers $\mathcal{O}$, not just countable ones. Let's prove that
        countably compact metric spaces are compact. If not, then there is a
        sequence $a:\mathbb{N}\rightarrow{X}$ with no convergent
        subsequence. The set $A$ described previously is closed:
        \begin{equation}
            A=\{a_{n}\in{X}\;|\;n\in\mathbb{N}\,\}
        \end{equation}
        and the sets $\mathcal{C}_{n}=A\setminus{B}_{n}$ are also closed.
        Moreover they are nested, $\mathcal{C}_{n+1}\subseteq\mathcal{C}_{n}$,
        and hence the sets:
        \begin{equation}
            \mathcal{U}_{n}=X\setminus\mathcal{C}_{n}
        \end{equation}
        are open and nested, $\mathcal{U}_{n}\subseteq\mathcal{U}_{n+1}$.
        By similar reasoning as before, the collection
        $\mathcal{O}=\{\,\mathcal{U}_{n}\;|\;n\in\mathbb{N}\,\}$ forms an open
        cover of $X$. More than that, it is a \textit{countable} open cover,
        and since $(X,\,d)$ is countably compact there is a finite subcover
        $\Delta\subseteq\mathcal{O}$. Since the sets are nested, we may choose
        $\Delta=\{\,\mathcal{U}_{N}\,\}$ for some $N\in\mathbb{N}$. But then
        $\mathcal{U}_{N}=X$, meaning $\mathcal{C}_{N}=\emptyset$. This implies
        that the set $A$ is finite. But a sequence
        $a:\mathbb{N}\rightarrow{A}$ into a finite set must have a convergent
        subsequence, a contradiction. Hence $(X,\,d)$ is compact.
        \par\hfill\par
        \textbf{Note:} Similar arguments do not hold for topological spaces.
        Countable compactness and compactness can be different.
        \par\hfill\par
        Using this, let's show that a metric space with the nested intersection
        property is countably compact (and hence compact). Suppose not. Then
        there is a countable open cover $\mathcal{O}$ of $(X,\,d)$ with no
        finite subcover. Since it is countable there is a surjection
        $\mathcal{U}:\mathbb{N}\rightarrow\mathcal{O}$. That is, we may list
        the elements as:
        \begin{equation}
            \mathcal{O}=\{\,\mathcal{U}_{0},\,\mathcal{U}_{1},\,\dots,\,\}
        \end{equation}
        Define $\mathcal{V}:\mathbb{N}\rightarrow\tau_{d}$ as follows:
        \begin{equation}
            \mathcal{V}_{N}=\bigcup_{n=0}^{N}\mathcal{U}_{n}
        \end{equation}
        The sets $\mathcal{V}_{n}$ are open, being the union of open sets, and
        nested, $\mathcal{V}_{n}\subseteq\mathcal{V}_{n+1}$. Moreover, since
        $\mathcal{O}$ has no finite subcover, $\mathcal{V}_{n}\ne{X}$ for all
        $n\in\mathbb{N}$. But then $\mathcal{C}_{n}=X\setminus\mathcal{V}_{n}$
        is a sequence of non-empty nested closed sets. But then, by hypothesis,
        $\bigcap_{n\in\mathbb{N}}\mathcal{C}_{n}$ is non-empty. Let
        $x\in\bigcap_{n\in\mathbb{N}}\mathcal{C}_{n}$. Then
        $x\in\mathcal{C}_{n}$ for all $n\in\mathbb{N}$, and hence
        $x\notin\mathcal{V}_{n}$ for all $n\in\mathbb{N}$. But
        $\mathcal{O}$ is an open cover, so $x\in\mathcal{U}_{n}$ for some
        $n\in\mathbb{N}$, which implies $x\in\mathcal{V}_{n}$, a contradiction.
        Hence $(X,\,d)$ is countably compact. Since countably compact metric
        spaces are compact, $(X,\,d)$ is also compact.
        \par\hfill\par
        The second part of the problem comes straight from the first part.
        If every sequence $\mathcal{U}:\mathbb{N}\rightarrow\tau_{d}$ of
        nested open proper subsets of $X$ is such that
        $\bigcup_{n\in\mathbb{N}}\mathcal{U}_{n}\ne{X}$, then by taking
        complements we see that every sequence of nested non-empty closed
        sets $\mathcal{C}:\mathbb{N}\rightarrow\mathcal{P}(X)$ has non-empty
        intersection, and therefore the space is compact. Similarly, if the
        space is compact, then we have the nested intersection property for
        closed sets. By looking at the complement we see that the union of
        nested open proper subsets cannot be the entire space since it cannot
        contain the element common to the intersection of the closed sets.
        So this is yet another equivalent definition of compactness in a
        metric space.
    \end{proof}
    \newpage
    \color{blue}
    \begin{problem}
        \textbf{(Calculus)}
        \par\hfill\par
        With our tools from metric space theory, one of the harder theorems
        from calculus becomes quite simple.
        \begin{itemize}
            \item (4 Points) Prove that if $(X,\,d_{X})$ is compact, if
                $(Y,\,d_{Y})$ is a metric space, and if $f:X\rightarrow{Y}$ is
                continuous, then $f[X]\subseteq{Y}$ is a compact subspace.
            \item (4 Points) The extreme value theorem states that if
                $f:[a,\,b]\rightarrow\mathbb{R}$ is continuous, then there is
                $c_{\textrm{min}},c_{\textrm{max}}\in[a,\,b]$ such that
                $f(c_{\textrm{min}})\leq{f}(x)\leq{f}(c_{\textrm{max}})$ for
                all $x\in[a,\,b]$. Let's take that up a notch. Prove that if
                $(X,\,d)$ is compact, and if $f:X\rightarrow\mathbb{R}$ is
                continuous, then there are points $c_{\textrm{min}}$ and
                $c_{\textrm{max}}$ such that
                $f(c_{\textrm{min}})\leq{f}(x)\leq{f}(c_{\textrm{max}})$ for
                all $x\in{X}$. [Hint: The previous part is enormously helpful.]
        \end{itemize}
    \end{problem}
    \color{black}
    \begin{proof}[Solution]
        Let $b:\mathbb{N}\rightarrow{f}[X]$ be a sequence. Since
        $f:X\rightarrow{f}[X]$ is surjective there is a right-inverse
        $g:f[X]\rightarrow{X}$ with the property that
        $(f\circ{g})(x)=x$. Let $a_{n}=g(b_{n})$. Since $(X,\,d)$ is compact
        there is a convergent subsequence $a_{k}$. Let $x$ be the limit,
        $a_{k_{n}}\rightarrow{x}$. But $f$ is continuous, so
        $f(a_{k_{n}})\rightarrow{f}(x)$. But
        $f(a_{k_{n}})=b_{k_{n}}$, so $b_{k_{n}}\rightarrow{f}(x)$, and hence
        $b$ has a convergent subsequence. Thus, $f[X]$ is a compact subspace
        of $(Y,\,d_{Y})$.
        \par\hfill\par
        To prove the extreme value theorem, note that
        $f[X]\subseteq\mathbb{R}$ is compact, so by Heine-Borel it is closed
        and bounded. Since it is bounded there is a least upper bound
        $y_{\textrm{max}}$ and a greatest lower bound $y_{\textrm{min}}$. Since
        $f[X]$ is closed $y_{\textrm{min}}$ and $y_{\textrm{max}}$ are elements
        for $f[X]$. That is, there are $x_{\textrm{min}},x_{\textrm{max}}\in{X}$
        such that $f(x_{\textrm{min}})=y_{\textrm{min}}$ and
        $f(x_{\textrm{max}})=y_{\textrm{max}}$. But then, since
        $y_{\textrm{min}}$ and $y_{\textrm{max}}$ are the greatest lower bound
        and least upper bound of $f[X]$, respectively, for all $x\in{X}$ we
        have $f(x_{\textrm{min}})\leq{f}(x)$ and
        $f(x)\leq{f}(x_{\textrm{max}})$, which is the desired property.
    \end{proof}
    \newpage
    \color{blue}
    \begin{problem}
        \textbf{(Product Spaces)}
        \par\hfill\par
        (6 Points) Given metric spaces $(X,\,d_{X})$
        and $(Y,\,d_{Y})$, prove that all three product metrics are
        topologically equivalent:
        \begin{align}
            d_{1}\big((x_{0},\,y_{0}),\,(x_{1},\,y_{1})\big)
                &=d_{X}(x_{0},\,x_{1})+d_{Y}(y_{0},\,y_{1})\\
            d_{2}\big((x_{0},\,y_{0}),\,(x_{1},\,y_{1})\big)
                &=\sqrt{d_{X}(x_{0},\,x_{1})^{2}+d_{Y}(y_{0},\,y_{1})^{2}}\\
            d_{\infty}\big((x_{0},\,y_{0}),\,(x_{1},\,y_{1})\big)
                &=\textrm{max}\big(%
                    d_{X}(x_{0},\,x_{1}),\,d_{Y}(y_{0},\,y_{1})
                \big)
        \end{align}
    \end{problem}
    \color{black}
    \begin{proof}[Solution]
        Topological equivalence is an equivalence relation, essentially since
        equality is. If $d_{0}$ is topologically equivalent to $d_{1}$, then
        $\tau_{d_{0}}=\tau_{d_{1}}$, and hence
        $\tau_{d_{1}}=\tau_{d_{0}}$, so $d_{1}$ is topologically equivalent to
        $d_{0}$. Reflexivity and transitivity can similarly be checked.
        So let's prove the metrics $d_{1}$ and $d_{\infty}$ are equivalent, as
        are the metrics $d_{2}$ and $d_{\infty}$. Let's start with
        $d_{1}$ and $d_{\infty}$. Let $r>0$. We must find
        $r'>0$ and $r''>0$ such that:
        \begin{subequations}
            \begin{align}
                B_{r'}^{(X,\,d_{1})}\big((x,\,y)\big)&\subseteq
                    B_{r}^{(X,\,d_{\infty})}\big((x,\,y)\big)\\
                B_{r''}^{(X,\,d_{\infty})}\big((x,\,y)\big)&\subseteq
                    B_{r}^{(X,\,d_{1})}\big((x,\,y)\big)
            \end{align}
        \end{subequations}
        Let $r'=r$ and $r''=r/2$. Then:
        \begin{subequations}
            \begin{align}
                d_{1}\big((x_{0},\,y_{0}),\,(x_{1},\,y_{1})\big)
                &<r'\\
                \Rightarrow
                d_{X}(x_{0},\,x_{1})+d_{Y}(y_{0},\,y_{1})&<r'\\
                \Rightarrow
                \textrm{max}\big(%
                    d_{X}(x_{0},\,x_{1}),\,d_{Y}(y_{0},\,y_{1})
                \big)&<r'\\
                \Rightarrow
                \textrm{max}\big(%
                    d_{X}(x_{0},\,x_{1}),\,d_{Y}(y_{0},\,y_{1})
                \big)&<r
            \end{align}
        \end{subequations}
        And therefore:
        \begin{equation}
            B_{r'}^{(X,\,d_{1})}\big((x,\,y)\big)\subseteq
                B_{r}^{(X,\,d_{\infty})}\big((x,\,y)\big)
        \end{equation}
        In the other direction:
        \begin{subequations}
            \begin{align}
                \textrm{max}\big(%
                    d_{X}(x_{0},\,x_{1}),\,d_{Y}(y_{0},\,y_{1})
                \big)&<r''\\
                \Rightarrow
                2\textrm{max}\big(%
                    d_{X}(x_{0},\,x_{1}),\,d_{Y}(y_{0},\,y_{1})
                \big)&<2r''\\
                \Rightarrow
                d_{X}(x_{0},\,x_{1})+d_{Y}(y_{0},\,y_{1})&<2r''\\
                \Rightarrow
                d_{1}\big((x_{0},\,y_{0}),\,(x_{1},\,y_{1})\big)&<r
            \end{align}
        \end{subequations}
        so we may conclude:
        \begin{equation}
            B_{r''}^{(X,\,d_{\infty})}\big((x,\,y)\big)\subseteq
                B_{r}^{(X,\,d_{1})}\big((x,\,y)\big)
        \end{equation}
        Now to compare $d_{2}$ and $d_{\infty}$. Again, choose
        $r'=r$. We get:
        \begin{subequations}
            \begin{align}
                d_{2}\big((x_{0},\,y_{0}),\,(x_{1},\,y_{1})\big)
                &<r'\\
                \Rightarrow
                \sqrt{d_{X}(x_{0},\,x_{1})^{2}+d_{Y}(y_{0},\,y_{1})^{2}}&<r'\\
                \Rightarrow
                \textrm{max}\big(%
                    d_{X}(x_{0},\,x_{1}),\,d_{Y}(y_{0},\,y_{1})
                \big)&<r'\\
                \Rightarrow
                \textrm{max}\big(%
                    d_{X}(x_{0},\,x_{1}),\,d_{Y}(y_{0},\,y_{1})
                \big)&<r
            \end{align}
        \end{subequations}
        and so:
        \begin{equation}
            B_{r'}^{(X,\,d_{2})}\big((x,\,y)\big)\subseteq
                B_{r}^{(X,\,d_{\infty})}\big((x,\,y)\big)
        \end{equation}
        choosing $r''=r/\sqrt{2}$ we get:
        \begin{subequations}
            \begin{align}
                \textrm{max}\big(%
                    d_{X}(x_{0},\,x_{1}),\,d_{Y}(y_{0},\,y_{1})
                \big)&<r''\\
                \Rightarrow
                \textrm{max}\big(%
                    d_{X}(x_{0},\,x_{1}),\,d_{Y}(y_{0},\,y_{1})
                \big)\sqrt{2}&<r''\sqrt{2}\\
                \Rightarrow
                \sqrt{2\textrm{max}\big(%
                    d_{X}(x_{0},\,x_{1}),\,d_{Y}(y_{0},\,y_{1})
                \big)^{2}}&<r''\sqrt{2}\\
                \Rightarrow
                \sqrt{d_{X}(x_{0},\,x_{1})^{2}+d_{Y}(y_{0},\,y_{1})^{2}}
                &<r''\sqrt{2}\\
                \Rightarrow
                d_{2}\big((x_{0},\,y_{0}),\,(x_{1},\,y_{1})\big)&<r
            \end{align}
        \end{subequations}
        Hence:
        \begin{equation}
            B_{r''}^{(X,\,d_{\infty})}\big((x,\,y)\big)\subseteq
                B_{r}^{(X,\,d_{2})}\big((x,\,y)\big)
        \end{equation}
        so $d_{2}$ and $d_{\infty}$ are topologically equivalent.
    \end{proof}
\end{document}

