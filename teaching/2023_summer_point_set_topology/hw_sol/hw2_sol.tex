%-----------------------------------LICENSE------------------------------------%
%   This file is part of Mathematics-and-Physics.                              %
%                                                                              %
%   Mathematics-and-Physics is free software: you can redistribute it and/or   %
%   modify it under the terms of the GNU General Public License as             %
%   published by the Free Software Foundation, either version 3 of the         %
%   License, or (at your option) any later version.                            %
%                                                                              %
%   Mathematics-and-Physics is distributed in the hope that it will be useful, %
%   but WITHOUT ANY WARRANTY; without even the implied warranty of             %
%   MERCHANTABILITY or FITNESS FOR A PARTICULAR PURPOSE.  See the              %
%   GNU General Public License for more details.                               %
%                                                                              %
%   You should have received a copy of the GNU General Public License along    %
%   with Mathematics-and-Physics.  If not, see <https://www.gnu.org/licenses/>.%
%------------------------------------------------------------------------------%
\documentclass{article}
\usepackage{graphicx}                           % Needed for figures.
\usepackage{amsmath}                            % Needed for align.
\usepackage{amssymb}                            % Needed for mathbb.
\usepackage{amsthm}                             % For the theorem environment.
\usepackage{xcolor}                             % Coloring text.

%------------------------Theorem Styles-------------------------%

% Define theorem style for default spacing and normal font.
\newtheoremstyle{normal}
    {\topsep}               % Amount of space above the theorem.
    {\topsep}               % Amount of space below the theorem.
    {}                      % Font used for body of theorem.
    {}                      % Measure of space to indent.
    {\bfseries}             % Font of the header of the theorem.
    {}                      % Punctuation between head and body.
    {.5em}                  % Space after theorem head.
    {}

% Define default environments.
\theoremstyle{normal}
\newtheorem{problem}{Problem}

\title{Point-Set Topology: Homework 2}
\date{Summer 2023}

% No indent and no paragraph skip.
\setlength{\parindent}{0em}
\setlength{\parskip}{0em}

\begin{document}
    \maketitle
    \color{blue}
    \begin{problem}
        \textbf{(Subspaces)}
        \par\hfill\par
        The inclusion mapping of a subset $A\subseteq{X}$ into $X$ is the
        function $\iota_{A}:A\rightarrow{X}$ defined by $\iota_{A}(x)=x$.
        \begin{itemize}
            \item (1 Point) Prove that if $(X,\,d)$ is a metric space, and if
                $(A,\,d_{A})$ is a metric subspace, then $\iota_{A}$ is
                continuous.
            \item (3 Points) Suppose $(X,\,d_{X})$ and $(Y,\,d_{Y})$ are
                metric spaces and let $A\subseteq{X}$ be a subset and
                $d_{A}$ be the subspace metric. Prove that $f:Y\rightarrow{A}$
                is continuous \textit{if and only if}
                $\iota_{A}\circ{f}:Y\rightarrow{X}$ is continuous.
            \item (2 Points)
                Prove that if $f:X\rightarrow{Y}$ is a continuous function
                from a metric space $(X,\,d_{X})$ to a metric space
                $(Y,\,d_{Y})$, and if $(A,\,d_{A})$ is a subspace of
                $(X,\,d_{X})$, then the restriction $f|_{A}:A\rightarrow{Y}$,
                defined by $f|_{A}(x)=f(x)$, is continuous.
            \item (4 Points) Prove that if $f:X\rightarrow{Y}$ is a
                homeomorphism, and if $A\subseteq{X}$, then
                $f|_{A}:A\rightarrow{f}[A]$ is a homeomorphism.
        \end{itemize}
    \end{problem}
    \color{black}
    \begin{proof}[Solution]
        Let $a:\mathbb{N}\rightarrow{A}$ be a convergent sequence with
        $a_{n}\rightarrow{x}$ for some $x\in{A}$. That is,
        $d_{A}(a_{n},\,x)\rightarrow{0}$. But then:
        \begin{align}
            d\big(\iota_{A}(a_{n}),\,\iota_{A}(x)\big)
            &=d_{A}\big(\iota_{A}(a_{n}),\,\iota_{A}(x)\big)
                \tag{Definition of $d_{A}$}\\
            &=d_{A}(a_{n},\,x)\tag{Definition of $\iota_{A}$}
        \end{align}
        and hence $d(\iota_{A}(a_{n}),\,\iota_{A}(x))\rightarrow{0}$, so
        $\iota_{A}(a_{n})\rightarrow\iota_{A}(x)$,
        meaning $\iota_{A}$ is continuous.
        \par\hfill\par
        Suppose $f:Y\rightarrow{A}$ is continuous. Then since
        $\iota_{A}:A\rightarrow{X}$ is continuous,
        $\iota_{A}\circ{f}$ is the composition of continuous functions, which
        is continuous. In the other direction, suppose
        $\iota_{A}\circ{f}$ is continuous. Let $a:\mathbb{N}\rightarrow{Y}$ be
        a convergent sequence with limit $y\in{Y}$. We must prove that
        $f(a_{n})\rightarrow{f}(y)$. Let $\varepsilon>0$ be given. Since, by
        hypothesis, $\iota_{A}\circ{f}$ is continuous, there is an
        $N\in\mathbb{N}$ such that $n\in\mathbb{N}$ and $n>N$ implies
        $d\big((\iota_{A}\circ{f})(a_{n}),\,(\iota_{A}\circ{f})(y)\big)<\varepsilon$.
        But then, by definition of $d_{A}$ and $\iota_{A}$, we have that $n>N$
        implies $d_{A}\big(f(a_{n}),\,f(y)\big)<\varepsilon$, and therefore
        $f(a_{n})\rightarrow{f}(y)$. That is, $f:Y\rightarrow{A}$ is continuous.
        \par\hfill\par
        For part 3, let $a:\mathbb{N}\rightarrow{A}$ be a convergent sequence
        with limit $x\in{A}$. We must prove
        $f|_{A}(a_{n})\rightarrow{f}|_{A}(x)$. Let $\varepsilon>0$. Since
        $A\subseteq{X}$ and $a:\mathbb{N}\rightarrow{A}$ is a convergent
        sequence in $A$, $a:\mathbb{N}\rightarrow{X}$ is a convergent sequence
        in $X$ as well with the same limit. But $f:X\rightarrow{Y}$ is
        continuous, so $f(a_{n})\rightarrow{f}(x)$. But then, since
        $\varepsilon>0$, there is an $N\in\mathbb{N}$ such that $n\in\mathbb{N}$
        and $n>N$ implies $d_{Y}\big(f(a_{n}),\,f(x)\big)<\varepsilon$. But
        $f|_{A}(a_{n})=f(a_{n})$ and $f|_{A}(x)=f(x)$, so $n>N$ implies
        $d_{Y}\big(f|_{A}(a_{n}),\,f|_{A}(x)\big)<\varepsilon$. Hence
        $f|_{A}(a_{n})\rightarrow{f}|_{A}(x)$ and $f|_{A}$ is continuous.
        \par\hfill\par
        Lastly, we are to prove the restriction of a homeomorphism to a subspace
        yields a homeomorphism to the image. The restriction is continuous by
        the previous problem. Since $f:X\rightarrow{Y}$ is bijective, it is
        injective, and hence $f|_{A}$ is injective as well. Since the co-domain
        is $f[A]$, the function is also surjective. So
        $f|_{A}:A\rightarrow{f}[A]$ is a continuous bijection. We must prove
        the inverse function is continuous. But the inverse function
        $(f|_{A})^{-1}$ is $f^{-1}|_{f[A]}$, the restriction of $f^{-1}$ to
        $f[A]$. But $f^{-1}$ is continuous since $f$ is a homeomorphism. But
        then $f^{-1}|_{f[A]}$ is the restriction of a continuous function to a
        subspace, meaning it is continuous. Therefore
        $(f|_{A})^{-1}$ is continuous, and $f|_{A}:A\rightarrow{f[A]}$ is a
        homeomorphism.
    \end{proof}
    \newpage
    \color{blue}
    \begin{problem}
        \textbf{(Continuity)}
        \par\hfill\par
        We have proven the equivalence of three definitions of continuity.
        The definition is that $f$ maps convergent sequences to convergent
        sequences. The calculus $\varepsilon-\delta$ statement is equivalent to
        this, as is the fact that the pre-image of open sets is open. Continuity
        can be described by forward images as well.
        \begin{itemize}
            \item (6 Points) Let $(X,\,d_{X})$ and $(Y,\,d_{Y})$ be metric
                spaces. Prove that $f:X\rightarrow{Y}$ is continuous if and only
                if for all $x\in{X}$ and for all open subsets
                $\mathcal{V}\subseteq{Y}$ with $f(x)\in\mathcal{V}$ there is an
                open subset $\mathcal{U}\subseteq{X}$ such that
                $x\in\mathcal{U}$ and $f[\mathcal{U}]\subseteq\mathcal{V}$.
        \end{itemize}
    \end{problem}
    \color{black}
    \begin{proof}[Solution]
        It is easiest to prove this using the fact that a function is continuous
        if and only if the pre-image of an open set is open. First suppose
        $f:X\rightarrow{Y}$ is continuous. Let $x\in{X}$ and let
        $\mathcal{V}\subseteq{Y}$ be an open set such that $f(x)\in\mathcal{V}$.
        Since $f$ is continous we have that $\mathcal{U}=f^{-1}[\mathcal{V}]$
        is open. But then $x\in\mathcal{U}$ and
        $f[\mathcal{U}]\subseteq\mathcal{V}$, by definition of images and
        pre-images. Hence a continuous function has the desired property.
        \par\hfill\par
        In the other direction, suppose $f:X\rightarrow{Y}$ is such that for all
        $x\in{X}$ and for all open sets $\mathcal{V}\subseteq{Y}$ with
        $f(x)\in\mathcal{V}$ there exists an open set $\mathcal{U}\subseteq{X}$
        such that $x\in\mathcal{U}$ and $f[\mathcal{U}]\subseteq\mathcal{V}$.
        We must prove that the pre-image of an open set is open in order to
        conclude that $f$ is continuous. Let $\mathcal{V}\subseteq{Y}$ be open.
        If $\mathcal{V}$ is empty we are done, since
        $f^{-1}[\emptyset]=\emptyset$. Similarly, if
        $f^{-1}[\mathcal{V}]=\emptyset$ there is nothing to prove. So suppose
        $\mathcal{V}\subseteq{Y}$ is an open set whose pre-image is not empty.
        Let $x\in{f}^{-1}[\mathcal{V}]$. Then, by definition of pre-image,
        $f(x)\in\mathcal{V}$. By hypothesis there is then open open set
        $\mathcal{U}\subseteq{X}$ such that $x\in\mathcal{U}$ and
        $f[\mathcal{U}]\subseteq\mathcal{V}$. But if $\mathcal{U}$ is open and
        $x\in\mathcal{U}$, then there is an $\varepsilon>0$ such that
        $B_{\varepsilon}^{(X,\,d_{X})}(x)\subseteq\mathcal{U}$. But then
        $B_{\varepsilon}^{(X,\,d_{X})}(x)\subseteq{f}^{-1}[\mathcal{V}]$. That
        is, for all $x\in{f}^{-1}[\mathcal{V}]$ there is an $\varepsilon$ about
        $x$ contained entirely inside of $f^{-1}[\mathcal{V}]$, and hence
        $f^{-1}[\mathcal{V}]$ is open. Therefore $f$ is continuous.
    \end{proof}
    \newpage
    \color{blue}
    \begin{problem}
        \textbf{(Compact Spaces)}
        \par\hfill\par
        For metric spaces there are many equivalent ways of defining
        compactness. Your job is to prove some of these equivalences.
        \begin{itemize}
            \item (4 Points) Prove that $(X,\,d)$ is compact if and only if
                for every sequence of closed non-empty nested sets, the
                intersection is non-empty. That is, if
                $\mathcal{C}:\mathbb{N}\rightarrow\mathcal{P}(X)$ is a sequence
                of closed sets such that $\mathcal{C}_{n}\ne\emptyset$ and
                $\mathcal{C}_{n+1}\subseteq\mathcal{C}_{n}$, then
                $\bigcap_{n\in\mathbb{N}}\mathcal{C}_{n}$ is non-empty.
            \item (2 Points) Prove that $(X,\,d)$ is compact if and only if for
                every sequence of nested proper open subsets, the union is not
                the whole space. That is, if
                $\mathcal{U}:\mathbb{N}\rightarrow\mathcal{P}(X)$ is a sequence
                of open sets such that $\mathcal{U}_{n}\ne{X}$ and
                $\mathcal{U}_{n}\subseteq\mathcal{U}_{n+1}$, then
                $\bigcup_{n\in\mathbb{N}}\mathcal{U}_{n}$ is not equal to $X$.
                [Hint: What is the complement of an open set? Can the previous
                part of the problem help?]
        \end{itemize}
    \end{problem}
    \color{black}
    \begin{proof}[Solution]
    \end{proof}
    \newpage
    \color{blue}
    \begin{problem}
        \textbf{(Calculus)}
        \par\hfill\par
        With our tools from metric space theory, one of the harder theorems
        from calculus becomes quite simple.
        \begin{itemize}
            \item (4 Points) Prove that if $(X,\,d_{X})$ is compact, if
                $(Y,\,d_{Y})$ is a metric space, and if $f:X\rightarrow{Y}$ is
                continuous, then $f[X]\subseteq{Y}$ is a compact subspace.
            \item (4 Points) The extreme value theorem states that if
                $f:[a,\,b]\rightarrow\mathbb{R}$ is continuous, then there is
                $c_{\textrm{min}},c_{\textrm{max}}\in[a,\,b]$ such that
                $f(c_{\textrm{min}})\leq{f}(x)\leq{f}(c_{\textrm{max}})$ for
                all $x\in[a,\,b]$. Let's take that up a notch. Prove that if
                $(X,\,d)$ is compact, and if $f:X\rightarrow\mathbb{R}$ is
                continuous, then there are points $c_{\textrm{min}}$ and
                $c_{\textrm{max}}$ such that
                $f(c_{\textrm{min}})\leq{f}(x)\leq{f}(c_{\textrm{max}})$ for
                all $x\in{X}$. [Hint: The previous part is enormously helpful.]
        \end{itemize}
    \end{problem}
    \color{black}
    \begin{proof}[Solution]
    \end{proof}
    \newpage
    \color{blue}
    \begin{problem}
        \textbf{(Product Spaces)}
        \par\hfill\par
        (6 Points) Given metric spaces $(X,\,d_{X})$
        and $(Y,\,d_{Y})$, prove that all three product metrics are
        topologically equivalent:
        \begin{align}
            d_{1}\big((x_{0},\,y_{0}),\,(x_{1},\,y_{1})\big)
                &=d_{X}(x_{0},\,x_{1})+d_{Y}(y_{0},\,y_{1})\\
            d_{2}\big((x_{0},\,y_{0}),\,(x_{1},\,y_{1})\big)
                &=\sqrt{d_{X}(x_{0},\,x_{1})^{2}+d_{Y}(y_{0},\,y_{1})^{2}}\\
            d_{\infty}\big((x_{0},\,y_{0}),\,(x_{1},\,y_{1})\big)
                &=\textrm{max}\big(%
                    d_{X}(x_{0},\,x_{1}),\,d_{Y}(y_{0},\,y_{1})
                \big)
        \end{align}
    \end{problem}
    \color{black}
    \begin{proof}[Solution]
    \end{proof}
\end{document}
