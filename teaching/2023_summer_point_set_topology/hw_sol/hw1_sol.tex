%-----------------------------------LICENSE------------------------------------%
%   This file is part of Mathematics-and-Physics.                              %
%                                                                              %
%   Mathematics-and-Physics is free software: you can redistribute it and/or   %
%   modify it under the terms of the GNU General Public License as             %
%   published by the Free Software Foundation, either version 3 of the         %
%   License, or (at your option) any later version.                            %
%                                                                              %
%   Mathematics-and-Physics is distributed in the hope that it will be useful, %
%   but WITHOUT ANY WARRANTY; without even the implied warranty of             %
%   MERCHANTABILITY or FITNESS FOR A PARTICULAR PURPOSE.  See the              %
%   GNU General Public License for more details.                               %
%                                                                              %
%   You should have received a copy of the GNU General Public License along    %
%   with Mathematics-and-Physics.  If not, see <https://www.gnu.org/licenses/>.%
%------------------------------------------------------------------------------%
\documentclass{article}
\usepackage{graphicx}                           % Needed for figures.
\usepackage{amsmath}                            % Needed for align.
\usepackage{amssymb}                            % Needed for mathbb.
\usepackage{amsthm}                             % For the theorem environment.
\usepackage{xcolor}                             % Coloring text.
\usepackage{hyperref}
\hypersetup{colorlinks=true, linkcolor=blue}
\newcommand\largeoverset[2]{\overset{\textstyle #1\mathstrut}{#2}}

%------------------------Theorem Styles-------------------------%

% Define theorem style for default spacing and normal font.
\newtheoremstyle{normal}
    {\topsep}               % Amount of space above the theorem.
    {\topsep}               % Amount of space below the theorem.
    {}                      % Font used for body of theorem.
    {}                      % Measure of space to indent.
    {\bfseries}             % Font of the header of the theorem.
    {}                      % Punctuation between head and body.
    {.5em}                  % Space after theorem head.
    {}

% Define default environments.
\theoremstyle{normal}
\newtheorem{problem}{Problem}

\title{Point-Set Topology: Homework 1}
\date{Summer 2023}

% No indent and no paragraph skip.
\setlength{\parindent}{0em}
\setlength{\parskip}{0em}

\begin{document}
    \maketitle
    \color{blue}
    \begin{problem}
        \textbf{(Hilbert Systems)}
        \par\hfill\par
        The Hilbert System is a collection of axioms for how propositional
        logic should behave. It claims the following four statements are true
        and do not need proof. Let $P$, $Q$, and $R$ be propositions
        (statements that are true or false). Then the following are true:
        \begin{align}
            P&\Rightarrow{P}\\
            P&\Rightarrow(Q\Rightarrow{P})\\
            \big(P\Rightarrow(Q\Rightarrow{R})\big)
            &\Rightarrow\big((P\Rightarrow{Q})\Rightarrow(P\Rightarrow{R}\big)\\
            (\neg{P}\Rightarrow\neg{Q})&\Rightarrow(Q\Rightarrow{P})
        \end{align}
        Here $\neg$ is the negation operator. $\neg{P}$ means \textit{not} $P$.
        \begin{itemize}
            \item (8 Points) Give the truth table for each of the four axioms.
                Using this, should we accept the axioms as valid?
            \item (4 Points) The first axiom is redundant. Together with
                \textit{modus ponens} (which is the axiom that if $P$ implies
                $Q$ is true, and if $P$ is true, then $Q$ is true), the second
                and third axiom can be used to prove that the first axiom is
                true. Prove this (partial credit will of course be given).
        \end{itemize}
    \end{problem}
    \color{black}
    \begin{proof}[Solution]
        \begin{table}
            \centering
            \begin{tabular}{c | c}
                $P$&$P\Rightarrow{P}$\\
                \hline
                False&True\\
                True&True
            \end{tabular}
            \caption{Truth Table for Hilbert's First Axiom}
            \label{tab:hilbert_first}
        \end{table}
        The truth table for $P\Rightarrow{P}$ is given in
        Tab.~\ref{tab:hilbert_first}. In general, $P\Rightarrow{Q}$
        is only false when $P$ is true and $Q$ is false. So $P\Rightarrow{P}$
        would only be false when $P$ is true and $P$ is false, a contradiction.
        So $P\Rightarrow{P}$ is always true, unless there exists a contradiction
        in our language (hopefully there doesn't).
        \par\hfill\par
        \begin{table}
            \centering
            \begin{tabular}{c | c | c | c}
                $P$&$Q$&$Q\Rightarrow{P}$&$P\Rightarrow(Q\Rightarrow{P})$\\
                \hline
                False&False&True&True\\
                False&True&False&True\\
                True&False&True&True\\
                True&True&True&True
            \end{tabular}
            \caption{Truth Table for Hilbert's Second Axiom}
            \label{tab:hilbert_second}
        \end{table}
        The truth table for Hilbert's second axiom,
        $P\Rightarrow(Q\Rightarrow{P})$, is given in
        Tab.~\ref{tab:hilbert_second}. Again, $P\Rightarrow{Q}$ is false only
        when $P$ is true and $Q$ is false. So $P\Rightarrow(Q\Rightarrow{P})$
        is false only when $P$ is true and $Q\Rightarrow{P}$ is false.
        Examining $Q\Rightarrow{P}$, this is only false when $Q$ is true and
        $P$ is false. So if $P$ is true, $Q\Rightarrow{P}$ is true, meaning
        $P\Rightarrow(Q\Rightarrow{P})$ is also true.
        \par\hfill\par
        \begin{table}
            \centering
            \resizebox{\textwidth}{!}{%
                \begin{tabular}{c | c | c | c | c | c | c | c | c }
                    $P$&$Q$&$R$&
                    $P\Rightarrow{Q}$&$P\Rightarrow{R}$&$Q\Rightarrow{R}$&
                    $P\Rightarrow(Q\Rightarrow{R})$&
                    $\largeoverset{(P\Rightarrow{Q})\Rightarrow}%
                     {(P\Rightarrow{R})}$&
                    $\largeoverset{\big(P\Rightarrow(Q\Rightarrow{R})\big)%
                     \Rightarrow}{\big((P\Rightarrow{Q})%
                     \Rightarrow(P\Rightarrow{R})\big)}$\\
                    \hline
                    False&False&False&True&True&True&True&True&True\\
                    False&False&True&True&True&True&True&True&True\\
                    False&True&False&True&True&False&True&True&True\\
                    False&True&True&True&True&True&True&True&True\\
                    True&False&False&False&False&True&True&True&True\\
                    True&False&True&False&True&True&True&True&True\\
                    True&True&False&True&False&False&False&False&True\\
                    True&True&True&True&True&True&True&True&True
                \end{tabular}
            }
            \caption{Truth Table for Hilbert's Third Axiom}
            \label{tab:hilberts_third}
        \end{table}
        The massive truth table for Hilbert's third axiom is given in
        Tab.~\ref{tab:hilberts_third}. The only case to inspect is when
        $P\Rightarrow(Q\Rightarrow{R})$ is true and
        $(P\Rightarrow{Q})\Rightarrow(P\Rightarrow{R})$ is false. In this case
        $P$ must be true, otherwise $P\Rightarrow{Q}$ would true, and
        $P\Rightarrow{R}$ would be true, and hence
        $(P\Rightarrow{Q})\Rightarrow(P\Rightarrow{R})$ would be true. But
        since $P\Rightarrow(Q\Rightarrow{R})$ is true, and since $P$ is true,
        $Q\Rightarrow{R}$ must be true. Now $Q$ must be true, for if not
        $P\Rightarrow{Q}$ would be false, since $P$ is true, and then
        $(P\Rightarrow{Q})\Rightarrow(P\Rightarrow{R})$ would be true. We have
        concluded thus far that $P$ and $Q$ are true, so $P\Rightarrow{Q}$ is
        true. But $(P\Rightarrow{Q})\Rightarrow(P\Rightarrow{R})$ is supposed
        to be false, and thus $P\Rightarrow{R}$ must be false. And since $P$ is
        true, $R$ must be false. That is, $P$ is true, $Q$ is true, and $R$ is
        false. We may thus conclude that $Q\Rightarrow{R}$ is false. But
        $P\Rightarrow(Q\Rightarrow{R})$ is true, and $P$ is true, meaning
        $Q\Rightarrow{R}$ is true, a contradiction. So the scenario that
        $P\Rightarrow(Q\Rightarrow{R})$ is true and
        $(P\Rightarrow{Q})\Rightarrow(P\Rightarrow{R})$ is false never occurs,
        unless we have a contradiction.
        \par\hfill\par
        \begin{table}
            \centering
            \begin{tabular}{c | c | c | c | c | c | c}
                $P$&$Q$&$\neg{P}$&$\neg{Q}$&
                $\neg{P}\Rightarrow\neg{Q}$&
                $Q\Rightarrow{P}$&
                $(\neg{P}\Rightarrow\neg{Q})\Rightarrow(Q\Rightarrow{R})$\\
                \hline
                False&False&True&True&True&True&True\\
                False&True&True&False&False&False&True\\
                True&False&False&True&True&True&True\\
                True&True&False&False&True&True&True
            \end{tabular}
            \caption{Truth Table for Hilbert's Fourth Axiom}
            \label{tab:hilberts_fourth}
        \end{table}
        The final truth table is given in Tab.~\ref{tab:hilberts_fourth}. This
        is the \textit{law of the contrapositive}. Note that
        $\neg{P}\Rightarrow\neg{Q}$ and $Q\Rightarrow{P}$ have identical
        columns. This is because they are equivalent statements. Mathematicians
        often use the law of the contrapositive to prove statements
        $P\Rightarrow{Q}$ when $\neg{Q}\Rightarrow\neg{P}$ is easier.
        \par\hfill\par
        This is not an \textit{independent} system of axioms, the first
        statement can be proved from axioms 2 and 3 (together with
        \textit{modus ponens}). By setting $R=P$ and $Q=(P\Rightarrow{P})$
        we obtain:
        \begin{align}
            \big(P\Rightarrow(Q\Rightarrow{R})\big)
            &\Rightarrow\big((P\Rightarrow{Q})\Rightarrow(P\Rightarrow{R})\big)
                \tag{Axiom 3}\\
            P&\Rightarrow(Q\Rightarrow{P})
                \tag{Axiom 2}\\
            \Big(P\Rightarrow\big((P\Rightarrow{P})\Rightarrow{P}\big)\Big)
            &\Rightarrow\Big(\big(P\Rightarrow(P\Rightarrow{P})\big)
            \Rightarrow(P\Rightarrow{P})\Big)
                \tag{Substitute}\\
            P&\Rightarrow\big((P\Rightarrow{P})\Rightarrow{P}\big)
                \tag{Substitute}\\
            \big(P\Rightarrow(P\Rightarrow{P})\big)&\Rightarrow(P\Rightarrow{P})
                \tag{Modus Ponens}\\
            P\Rightarrow(P\Rightarrow{P})
                \tag{Axiom 2}\\
            P\Rightarrow{P}
                \tag{Modus Ponens}
        \end{align}
        So the negation of the first axiom would be \textit{inconsistent} with
        the others.
    \end{proof}
    \newpage
    \color{blue}
    \begin{problem}
        \textbf{(Disjunction and Conjunction)}
        \par\hfill\par
        The \textit{logical or} and \textit{logical and} are not primitives,
        but rather can be defined with implication and negation.
        It is common to use the
        $\lor$ symbol for \textit{or} and the $\land$ symbol for \textit{and}.
        $P\lor{Q}$ then reads $P$ \textit{or} $Q$, and $P\land{Q}$ reads
        $P$ \textit{and} $Q$. These can be defined as follows:
        \begin{align}
            (P\lor{Q})&\Leftrightarrow(\neg{P}\Rightarrow{Q})\\
            (P\land{Q})&\Leftrightarrow\neg(P\Rightarrow\neg{Q})
        \end{align}
        Where $\Leftrightarrow$ means \textit{is equivalent to} or
        \textit{if and only if}.
        \begin{itemize}
            \item (2 Points) $P\lor{Q}$ is only false when both $P$ and $Q$ are
                false. Explain (with words, no mathematics needed here) when
                $\neg{P}\Rightarrow{Q}$ is false. Create the truth table for
                $\neg{P}\Rightarrow{Q}$ and explain why this is a valid choice
                for the logical or.
            \item (2 Points) $P\land{Q}$ is only true when both $P$ and $Q$ are
                true. Explain why $\neg(P\Rightarrow\neg{Q})$ is a good choice
                for logical and. Construct the truth table for this.
            \item (6 Points) Prove that \textit{or} is commutative. That is,
                $P\lor{Q}$ if and only if $Q\lor{P}$. You must prove:
                \begin{align}
                    (\neg{P}\Rightarrow{Q})&\Rightarrow(\neg{Q}\Rightarrow{P})\\
                    (\neg{Q}\Rightarrow{P})&\Rightarrow(\neg{P}\Rightarrow{Q})
                \end{align}
                Hint: Use your Hilbert system.
        \end{itemize}
    \end{problem}
    \color{black}
    \begin{proof}[Solution]
        $P\Rightarrow{Q}$ is only false when $P$ is true, yet $Q$ is false.
        Introducing negation, $\neg{P}\Rightarrow{Q}$ is false when
        $\neg{P}$ is true and $Q$ is false. $\neg{P}$ being true means $P$ is
        false, and hence $\neg{P}\Rightarrow{Q}$ is false only when $P$ and $Q$
        are both false. This is the same condition for logical or, meaning it is
        a good candidate for the definition of $P\lor{Q}$. The truth table is
        given in Tab.~\ref{tab:truth_tab_disjunction}.
        \par\hfill\par
        Logical \textit{and}, or conjunction, is only true when both
        propositions are true. $P\Rightarrow\neg{Q}$ is only false when $P$
        and $Q$ are both true, meaning $\neg(P\Rightarrow\neg{Q})$ is only
        true when both $P$ and $Q$ are true. The truth table is given in
        Tab.~\ref{tab:truth_tab_conjunction}.
        \par\hfill\par
        We can use the Hilbert system to prove some of the basic laws of logical
        \textit{or} and \textit{and}, such as commutativity, associativity, and
        much more. To prove commutativity we wish to show that:
        \begin{equation}
            P\lor{Q}\Leftrightarrow{Q}\lor{P}
        \end{equation}
        which is equivalent to two implications:
        \begin{align}
            P\lor{Q}&\Rightarrow{Q}\lor{P}\\
            Q\lor{P}&\Rightarrow{P}\lor{Q}
        \end{align}
        Since we have defined disjunction using implication and negation, we
        can expand the $\lor$ symbol out and get the following implications:
        \begin{align}
            (\neg{P}\Rightarrow{Q})&\Rightarrow(\neg{Q}\Rightarrow{P})\\
            (\neg{Q}\Rightarrow{P})&\Rightarrow(\neg{P}\Rightarrow{Q})
        \end{align}
        Let's prove these two claims. Hilbert's fourth says
        $(\neg{P}\Rightarrow\neg{Q})\Rightarrow(Q\Rightarrow{P})$. Substituting
        $P=P$ and $Q=\neg{Q}$, and invoking the law of double negation
        $(Q\Leftrightarrow\neg\neg{Q})$, we get:
        \begin{equation}
            (\neg{P}\Rightarrow{Q})\Rightarrow(\neg{Q}\Rightarrow{P})
        \end{equation}
        Which is the first desired implication. Returning to Hilbert's fourth
        axiom, $(\neg{P}\Rightarrow\neg{Q})\Rightarrow(Q\Rightarrow{P})$, if
        we substitute $P=Q$ and $Q=\neg{P}$, we get:
        \begin{equation}
            (\neg{Q}\Rightarrow{P})\Rightarrow(\neg{P}\Rightarrow{Q})
        \end{equation}
        which is the second desired implication. Hence
        $P\lor{Q}\Rightarrow{Q}\lor{P}$ and $Q\lor{P}\Rightarrow{P}\lor{Q}$.
        That is, the logical \textit{or} is commutative.
        \begin{table}
            \centering
            \begin{tabular}{c | c | c | c | c}
                $P$&$Q$&$\neg{P}$&$\neg{P}\Rightarrow{Q}$&$P\lor{Q}$\\
                \hline
                False&False&True&False&False\\
                False&True&True&True&True\\
                True&False&False&True&True\\
                True&True&False&True&True
            \end{tabular}
            \caption{Truth Table for Disjunction}
            \label{tab:truth_tab_disjunction}
        \end{table}
        \begin{table}
            \centering
            \begin{tabular}{c | c | c | c | c}
                $P$&$Q$&$\neg{Q}$&$\neg(P\Rightarrow\neg{Q})$&$P\land{Q}$\\
                \hline
                False&False&True&False&False\\
                False&True&False&False&False\\
                True&False&True&False&False\\
                True&True&False&True&True
            \end{tabular}
            \caption{Truth Table for Conjunction}
            \label{tab:truth_tab_conjunction}
        \end{table}
    \end{proof}
    \newpage
    \color{blue}
    \begin{problem}
        \textbf{(Set Arithmetic)}
        \par\hfill\par
        Two sets $A$ and $B$ are equal if and only if $A\subseteq{B}$ and
        $B\subseteq{A}$. We use this often to prove two expressions are equal.
        Remember, $A\subseteq{B}$ if and only if $x\in{A}$ implies $x\in{B}$.
        \begin{itemize}
            \item (3 Points) Prove the distributive law of unions:
                \begin{equation}
                    A\cup(B\cap{C})=(A\cup{B})\cap(A\cup{C})
                \end{equation}
            \item (3 Points) Prove the distributive law of intersections:
                \begin{equation}
                    A\cap(B\cup{C})=(A\cap{B})\cup(A\cap{C})
                \end{equation}
            \item (3 Points) Prove De Morgan's Law of Unions. If
                $A,B\subseteq{X}$, then:
                \begin{equation}
                    X\setminus(A\cup{B})=(X\setminus{A})\cap(X\setminus{B})
                \end{equation}
            \item (3 Points) Prove De Morgan's Law of Intersections. If
                $A,B\subseteq{X}$, then:
                \begin{equation}
                    X\setminus(A\cap{B})=(X\setminus{A})\cup(X\setminus{B})
                \end{equation}
        \end{itemize}
    \end{problem}
    \color{black}
    \begin{proof}[Solution]
        Let's start with $A\cup(B\cap{C})=(A\cup{B})\cap(A\cup{C})$. We do this
        by showing the left-hand side is a subset of the right-hand side, and
        vice-versa. Suppose $x\in{A}\cup(B\cap{C})$. Then, by definition of
        union, $x\in{A}$ or $x\in{B}\cap{C}$. Then $x\in{A}$ or $x\in{B}$ and
        $x\in{C}$. But if $x\in{A}$, then $x\in{A}$ or $x\in{B}$, by definition
        of \textit{or}. Similarly if $x\in{A}$, then $x\in{A}$ or $x\in{C}$.
        Hence if $x\in{A}$ then $x\in{A}$ or $B$ and $x\in{A}$ or $C$, so
        $x\in(A\cup{B})\cap(A\cup{C})$. If $x\in{B}\cap{C}$, then $x\in{B}$ and
        $C$ by definition of intersection. Again by our use of the word
        \textit{or}, $x\in{A}$ or $x\in{B}$ and $x\in{A}$ or $x\in{C}$, so
        $x\in(A\cup{B})\cap(A\cup{C})$, meaning
        $A\cup(B\cap{C})\subseteq(A\cup{B})\cap(A\cup{C})$. In the other
        direction, if $x\in(A\cup{B})\cap(A\cup{C})$, then
        $x\in{A}\cup{B}$ and $x\in{A}\cup{C}$. So either $x\in{A}$ is true,
        or $x\in{B}$ and $x\in{C}$ is true, meaning
        $x\in{A}\cup(B\cap{C})$, and hence
        $(A\cup{B})\cap(A\cup{C})\subseteq{A}\cup(B\cap{C})$. Thus,
        $A\cup(B\cap{C})=(A\cup{B})\cap(A\cup{C})$.
        \par\hfill\par
        For the next equality we need to show that
        $A\cap(B\cup{C})\subseteq(A\cap{B})\cup(A\cap{C})$ and
        $(A\cap{B})\cup(A\cap{C})\subseteq{A}\cap(B\cup{C})$.
        Suppose $x\in{A}\cap(B\cup{C})$. Then $x\in{A}$ and
        $x\in{B}\cup{C}$, by the definition of intersection. But then
        $x\in{A}$ and $x\in{B}$, or $x\in{A}$ and $x\in{C}$, by definition of
        union. Hence $x\in{A}\cap{B}$ or $x\in{A}\cap{C}$, meaning
        $x\in(A\cap{B})\cup(A\cap{C})$. In the other direction, if
        $x\in(A\cap{B})\cup(A\cap{C})$, then $x\in{A}\cap{B}$ or
        $x\in{A}\cap{C}$. But then $x\in{A}$ and $x\in{B}$, or
        $x\in{A}$ and $x\in{C}$, by definition of intersection. This means
        $x\in{A}$ and either $x\in{B}$ or $x\in{C}$, meaning
        $x\in{A}\cap(B\cup{C})$.
        \par\hfill\par
        Now for the De Morgan laws. Suppose $x\in{X}\setminus(A\cup{B})$.
        Then $x\notin{A}\cup{B}$, which means $x\notin{A}$ and
        $x\notin{B}$. That is, $x\in(X\setminus{A})\cap(X\setminus{B})$,
        so $X\setminus(A\cup{B})\subseteq(X\setminus{A})\cap(X\setminus{B})$.
        Now suppose $x\in(X\setminus{A})\cap(X\setminus{B})$. Then
        $x\in{X}\setminus{A}$ and $x\in{X}\setminus{B}$. That is,
        $x\notin{A}$ and $x\notin{B}$, and hence $x\notin{A}\cup{B}$. But if
        $x\notin{A}\cup{B}$ and $x\in{X}$, then $x\in{X}\setminus(A\cup{B})$.
        Therefore
        $(X\setminus{A})\cap(X\setminus{B})\subseteq{X}\setminus(A\cup{B})$,
        and thus $(X\setminus{A})\cap(X\setminus{B})={X}\setminus(A\cup{B})$.
        \par\hfill\par
        The other De Morgan law is dealt with similarly. Suppose
        $x\in{X}\setminus(A\cap{B})$. Then $x\notin{A}\cap{B}$, meaning
        $x\notin{A}$ or $x\notin{B}$. But then
        $x\in(X\setminus{A})\cup(X\setminus{B})$, implying
        $X\setminus(A\cap{B})\subseteq(X\setminus{A})\cup(X\setminus{B})$.
        Now suppose $x\in(X\setminus{A})\cup(X\setminus{B})$. Then
        $x\notin{A}$ or $x\notin{B}$. But then $x\notin{A}\cap{B}$, and
        therefore $x\in{X}\setminus(A\cap{B})$.
        We conclude that
        $X\setminus(A\cap{B})=(X\setminus{A})\cup(X\setminus{B})$.
    \end{proof}
    \newpage
    \color{blue}
    \begin{problem}
        \textbf{(The Cantor-Schroeder-Bernstein Theorem)}
        \par\hfill\par
        There are two versions of the Cantor-Schroeder-Bernstein theorem. The
        first says that if $A$ and $B$ are sets, and if $f:A\rightarrow{B}$ and
        $g:B\rightarrow{A}$ are injective, then there is a bijection
        $h:A\rightarrow{B}$. The second states that if
        $A$ and $B$ are sets, and if $f:A\rightarrow{B}$ and $g:B\rightarrow{A}$
        are surjective, then there is a bijection $h:A\rightarrow{B}$.
        \begin{itemize}
            \item (3 Points) Prove that if $f:A\rightarrow{B}$ is an injective
                function, then there is a surjection $g:B\rightarrow{A}$.
            \item (3 Points) Prove that if $f:A\rightarrow{B}$ is a surjective
                function, then there is an injection $g:B\rightarrow{A}$.
            \item (4 Points) Prove that the truth of the first
                Cantor-Schroeder-Bernstein theorem implies the validity of the
                second, and vice-versa.
        \end{itemize}
    \end{problem}
    \color{black}
    \begin{proof}[Solution]
        The problem is \textit{vacuous} if $A$ or $B$ are empty, so first
        suppose they are not. Since $A$ is non-empty, pick some $x\in{A}$.
        Given an injective function $f:A\rightarrow{B}$, define
        $g:B\rightarrow{A}$ as follows:
        \begin{equation}
            g(b)=
            \begin{cases}
                a\in{A}\textrm{ such that }f(a)=b\\
                x\textrm{ if such an element does not exist}
            \end{cases}
        \end{equation}
        There's no axiom of choice needed here, since $f$ is injective we have
        a well-defined function. $g$ is surjective. Given $a\in{A}$ let
        $b=f(a)$. Then $g(b)=a$ by definition.
        \par\hfill\par
        Given a surjection $f:A\rightarrow{B}$, for each
        $b\in{B}$ pick some $a_{b}\in{A}$ such that
        $f(a_{b})=b$ and define $g(b)=a_{b}$. This \textit{choosing} does invoke
        the axiom of choice, but very subtly. The function $g$ is an injective
        function from $B$ to $A$ since
        $g(b_{0})=g(b_{1})$ implies $a_{b_{0}}=a_{b_{1}}$, which means:
        \begin{equation}
            b_{0}=f(a_{b_{0}})=f(a_{b_{1}})=b_{1}
        \end{equation}
        and hence $b_{0}=b_{1}$, so $g$ is injective.
        \par\hfill\par
        Now suppose the first Cantor-Schroeder-Bernstein theorem is true.
        That is, if there exists injective functions $f:A\rightarrow{B}$ and
        $g:B\rightarrow{A}$, then there is a bijection $h:A\rightarrow{B}$.
        Let's use this to prove the second Cantor-Schroeder-Bernstein theorem.
        Suppose $\tilde{f}:A\rightarrow{B}$ and $\tilde{g}:B\rightarrow{A}$ are
        surjective functions. By the previous part of the problem there then
        exists injective functions $f:A\rightarrow{B}$ and $g:B\rightarrow{A}$.
        But then by the first Cantor-Schroeder-Bernstein theorem there is a
        bijection $h:A\rightarrow{B}$. Hence the existence of surjective
        functions $\tilde{f}:A\rightarrow{B}$ and $\tilde{g}:B\rightarrow{B}$
        implies the existence of a bijection $h:A\rightarrow{B}$.
        \par\hfill\par
        In the other direction, suppose the second Cantor-Schroeder-Bernstein
        theorem holds. That is, if $f:A\rightarrow{B}$ and $g:B\rightarrow{A}$
        are surjective functions, then there is a bijection $h:A\rightarrow{B}$.
        Given injective function $\tilde{f}:A\rightarrow{B}$ and
        $\tilde{g}:B\rightarrow{A}$, by the previous part of the problem there
        are surjective functions $f:A\rightarrow{B}$ and $g:B\rightarrow{A}$.
        Then by the second Cantor-Schroeder-Bernstein theorem there is a
        bijection $h:A\rightarrow{B}$. Hence the existence of injective
        functions $\tilde{f}:A\rightarrow{B}$ and $\tilde{g}:B\rightarrow{A}$
        implies the existence of a bijection $h:A\rightarrow{B}$.
        \par\hfill\par
        Since the first Cantor-Schroeder-Bernstein theorem implies the second,
        and vice-versa, all you need to do is prove \textit{one} of them, and
        the second immediately follows. Most texts usually prove the first one,
        that if there exists injective function $f:A\rightarrow{B}$ and
        $g:B\rightarrow{A}$, then there is a bijection $h:A\rightarrow{B}$.
    \end{proof}
    \newpage
    \color{blue}
    \begin{problem}
        \textbf{(Induced Metrics)}
        \par\hfill\par
        A \textit{norm} on $\mathbb{R}^{n}$ is a function that assigns a
        \textit{length} to each point. That is, a function
        $||\cdot||:\mathbb{R}^{n}\rightarrow\mathbb{R}$ such that for all
        points $\mathbf{x},\mathbf{y}\in\mathbb{R}^{n}$ and all real numbers
        $a\in\mathbb{R}$ we have:
        \begin{align}
            ||\mathbf{x}||&\geq{0}&\tag{Positivity}\\
            ||\mathbf{x}||&=0
                \Rightarrow\mathbf{x}=\mathbf{0}\tag{Definiteness}\\
            ||a\mathbf{x}||&=|a|\cdot||\mathbf{x}||
                \tag{Homogeneity}\\
            ||\mathbf{x}+\mathbf{y}||&\leq||\mathbf{x}||+||\mathbf{y}||
                \tag{Triangle-Inequality}
        \end{align}
        The metric induced by a norm is:
        \begin{equation}
            d(\mathbf{x},\,\mathbf{y})=||\mathbf{x}-\mathbf{y}||
        \end{equation}
        \begin{itemize}
            \item (4 Points) Prove that the induced metric is a metric on
                $\mathbb{R}^{n}$.
            \item (6 Points) A convex set is a set $A\subseteq\mathbb{R}^{n}$
                such that for all $\mathbf{x},\mathbf{y}\in{A}$ and for all
                $0\leq{t}\leq{1}$ it is true that
                $t\mathbf{x}+(1-t)\mathbf{y}\in{A}$. Prove that open balls
                centered about the origin are convex when the metric comes from
                a norm.
        \end{itemize}
    \end{problem}
    \color{black}
    \begin{proof}[Solution]
        This function is indeed a metric. It is positive-definite since
        $||\mathbf{x}||$ is non-negative and:
        \begin{align}
            d(\mathbf{x},\,\mathbf{y})&=0\\
            \Leftrightarrow||\mathbf{x}-\mathbf{y}||&=0\\
            \Leftrightarrow\mathbf{x}-\mathbf{y}&=\mathbf{0}\\
            \Leftrightarrow\mathbf{x}&=\mathbf{y}
        \end{align}
        It is symmetric since:
        \begin{align}
            d(\mathbf{x},\,\mathbf{y})&=||\mathbf{x}-\mathbf{y}||\\
                &=||(-1)(\mathbf{y}-\mathbf{x})||\\
                &=|-1|\;||\mathbf{y}-\mathbf{x}||\\
                &=||\mathbf{y}-\mathbf{x}||\\
                &=d(\mathbf{y},\,\mathbf{x})
        \end{align}
        Lastly, the triangle inequality is satisfied. We have, for any
        $\mathbf{x},\mathbf{y},\mathbf{z}\in\mathbb{R}^{n}$:
        \begin{align}
            d(\mathbf{x},\,\mathbf{y})&=||\mathbf{x}-\mathbf{y}||\\
                &=||\mathbf{x}+\mathbf{0}-\mathbf{y}||\\
                &=||\mathbf{x}+(-\mathbf{z}+\mathbf{z})-\mathbf{y}||\\
                &=||(\mathbf{x}-\mathbf{z})+(\mathbf{z}-\mathbf{y})||\\
                &\leq||\mathbf{x}-\mathbf{z}||+||\mathbf{z}-\mathbf{y}||\\
                &=||\mathbf{x}-\mathbf{z}||+|\mathbf{y}-\mathbf{z}||\\
                &=d(\mathbf{x},\,\mathbf{z})+d(\mathbf{y},\,\mathbf{z})
        \end{align}
        So $d$ is a metric.
        \par\hfill\par
        The balls in a normed vector space are convex. Since translation
        $T:\mathbb{R}^{n}\rightarrow\mathbb{R}^{n}$, defined by
        $T(\mathbf{x})=\mathbf{x}+\mathbf{a}$ for some fixed
        $\mathbf{a}\in\mathbb{R}^{n}$, is a global isometry, and since
        $||a\mathbf{x}||=|a|\;||\mathbf{x}||$ for all real $a\in\mathbb{R}$,
        we can consider open balls of radius 1 centered at the origin. Let
        $\mathbf{x},\,\mathbf{y}\in\mathbb{R}^{n}$ be such that
        $d(\mathbf{x},\,\mathbf{0})<1$ and $d(\mathbf{y},\,\mathbf{0})<1$.
        In other words, suppose $||\mathbf{x}||<1$ and $||\mathbf{y}||<1$.
        Then for any $0\leq{t}\leq{1}$ we have:
        \begin{align}
            ||t\mathbf{x}+(1-t)\mathbf{y}||
                &\leq||t\mathbf{x}||+||(1-t)\mathbf{y}||
                    \tag{Triangle Inequality}\\
                &=|t|\;||\mathbf{x}||+|1-t|\;||\mathbf{y}||
                    \tag{Factoring Scalars}\\
                &<|t|+|1-t|
                    \tag{Since $||\mathbf{x}||<1$ and $||\mathbf{y}||<1$}\\
                &=t+1-t\tag{Since $0\leq{t}\leq{1}$}\\
                &=1\tag{Simplify}
        \end{align}
        And hence this point lies in the unit ball as well.
    \end{proof}
    \newpage
    \color{blue}
    \begin{problem}
        \textbf{(Connected Subsets)}
        \par\hfill\par
        (4 Points)
        A connected subset of a metric space $(X,d)$ is a subset
        $A\subseteq{X}$ such that it is impossible to write
        $A=\mathcal{U}\cup\mathcal{V}$ where $\mathcal{U}$ and $\mathcal{V}$
        are disjoint non-empty open sets. Give an example that shows that
        open balls do not need to be connected.
    \end{problem}
    \color{black}
    \begin{proof}[Solution]
        There are many examples, but the easiest is probably the discrete
        metric on a two point set. Let $X=\{\,0,\,1\,\}$ and $d$ be the
        discrete metric on $X$. The ball of radius 2 centered about zero is
        disconnected since it can be written as the union of the ball of radius
        1 about 0 and the ball of radius 1 about 1, two non-empty disjoint
        open sets. That is, the set $\{\,0,\,1\,\}$, which is the open ball of
        radius 2 centered at 0, can be written as the union of
        $\{\,0\,\}$ and $\{\,1\,\}$, which are the open balls of radius 1
        centered about 0 and 1, respectively.
    \end{proof}
\end{document}
