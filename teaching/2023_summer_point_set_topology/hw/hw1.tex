%-----------------------------------LICENSE------------------------------------%
%   This file is part of Mathematics-and-Physics.                              %
%                                                                              %
%   Mathematics-and-Physics is free software: you can redistribute it and/or   %
%   modify it under the terms of the GNU General Public License as             %
%   published by the Free Software Foundation, either version 3 of the         %
%   License, or (at your option) any later version.                            %
%                                                                              %
%   Mathematics-and-Physics is distributed in the hope that it will be useful, %
%   but WITHOUT ANY WARRANTY; without even the implied warranty of             %
%   MERCHANTABILITY or FITNESS FOR A PARTICULAR PURPOSE.  See the              %
%   GNU General Public License for more details.                               %
%                                                                              %
%   You should have received a copy of the GNU General Public License along    %
%   with Mathematics-and-Physics.  If not, see <https://www.gnu.org/licenses/>.%
%------------------------------------------------------------------------------%
\documentclass{article}
\usepackage{graphicx}                           % Needed for figures.
\usepackage{amsmath}                            % Needed for align.
\usepackage{amssymb}                            % Needed for mathbb.
\usepackage{amsthm}                             % For the theorem environment.
\usepackage{float}
\usepackage{hyperref}
\hypersetup{
    colorlinks=true,
    linkcolor=blue,
    filecolor=magenta,
    urlcolor=Cerulean,
    citecolor=SkyBlue
}

%------------------------Theorem Styles-------------------------%

% Define theorem style for default spacing and normal font.
\newtheoremstyle{normal}
    {\topsep}               % Amount of space above the theorem.
    {\topsep}               % Amount of space below the theorem.
    {}                      % Font used for body of theorem.
    {}                      % Measure of space to indent.
    {\bfseries}             % Font of the header of the theorem.
    {}                      % Punctuation between head and body.
    {.5em}                  % Space after theorem head.
    {}

% Define default environments.
\theoremstyle{normal}
\newtheorem{problem}{Problem}

\title{Point-Set Topology: Homework 1}
\date{Summer 2023}

% No indent and no paragraph skip.
\setlength{\parindent}{0em}
\setlength{\parskip}{0em}

\begin{document}
    \maketitle
    \begin{problem}
        \textbf{(Hilbert Systems)}
        \par\hfill\par
        The Hilbert System is a collection of axioms for how propositional
        logic should behave. It claims the following four statements are true
        and do not need proof. Let $P$, $Q$, and $R$ be propositions
        (statements that are true or false). Then the following are true:
        \begin{align}
            P&\Rightarrow{P}\\
            P&\Rightarrow(Q\Rightarrow{P})\\
            \big(P\Rightarrow(Q\Rightarrow{R})\big)
            &\Rightarrow\big((P\Rightarrow{Q})\Rightarrow(P\Rightarrow{R}\big)\\
            (\neg{P}\Rightarrow\neg{Q})&\Rightarrow(Q\Rightarrow{P})
        \end{align}
        Here $\neg$ is the negation operator. $\neg{P}$ means \textit{not} $P$.
        \begin{itemize}
            \item (8 Points) Give the truth table for each of the four axioms.
                Using this, should we accept the axioms as valid?
            \item (4 Points) The first axiom is redundant. Together with
                \textit{modus ponens} (which is the axiom that if $P$ implies
                $Q$ is true, and if $P$ is true, then $Q$ is true), the second
                and third axiom can be used to prove that the first axiom is
                true. Prove this (partial credit will of course be given).
        \end{itemize}
    \end{problem}
    \begin{problem}
        \textbf{(Disjunction and Conjunction)}
        \par\hfill\par
        The \textit{logical or} and \textit{logical and} are not primitives,
        but rather can be defined with implication and negation.
        It is common to use the
        $\lor$ symbol for \textit{or} and the $\land$ symbol for \textit{and}.
        $P\lor{Q}$ then reads $P$ \textit{or} $Q$, and $P\land{Q}$ reads
        $P$ \textit{and} $Q$. These can be defined as follows:
        \begin{align}
            (P\lor{Q})&\Leftrightarrow(\neg{P}\Rightarrow{Q})\\
            (P\land{Q})&\Leftrightarrow\neg(P\Rightarrow\neg{Q})
        \end{align}
        Where $\Leftrightarrow$ means \textit{is equivalent to} or
        \textit{if and only if}.
        \begin{itemize}
            \item (2 Points) $P\lor{Q}$ is only false when both $P$ and $Q$ are
                false. Explain (with words, no mathematics needed here) when
                $\neg{P}\Rightarrow{Q}$ is false. Create the truth table for
                $\neg{P}\Rightarrow{Q}$ and explain why this is a valid choice
                for the logical or.
            \item (2 Points) $P\land{Q}$ is only true when both $P$ and $Q$ are
                true. Explain why $\neg(P\Rightarrow\neg{Q})$ is a good choice
                for logical and. Construct the truth table for this.
            \item (6 Points) Prove that \textit{or} is commutative. That is,
                $P\lor{Q}$ if and only if $Q\lor{P}$. You must prove:
                \begin{align}
                    (\neg{P}\Rightarrow{Q})&\Rightarrow(\neg{Q}\Rightarrow{P})\\
                    (\neg{Q}\Rightarrow{P})&\Rightarrow(\neg{P}\Rightarrow{Q})
                \end{align}
                Hint: Use your Hilbert system.
        \end{itemize}
    \end{problem}
    \begin{problem}
        \textbf{(Set Arithmetic)}
        \par\hfill\par
        Two sets $A$ and $B$ are equal if and only if $A\subseteq{B}$ and
        $B\subseteq{A}$. We use this often to prove two expressions are equal.
        Remember, $A\subseteq{B}$ if and only if $x\in{A}$ implies $x\in{B}$.
        \begin{itemize}
            \item (3 Points) Prove the distributive law of unions:
                \begin{equation}
                    A\cup(B\cap{C})=(A\cup{B})\cap(A\cup{C})
                \end{equation}
            \item (3 Points) Prove the distributive law of intersections:
                \begin{equation}
                    A\cap(B\cup{C})=(A\cap{B})\cup(A\cap{C})
                \end{equation}
            \item (3 Points) Prove De Morgan's Law of Unions. If
                $A,B\subseteq{X}$, then:
                \begin{equation}
                    X\setminus(A\cup{B})=(X\setminus{A})\cap(X\setminus{B})
                \end{equation}
            \item (3 Points) Prove De Morgan's Law of Intersections. If
                $A,B\subseteq{X}$, then:
                \begin{equation}
                    X\setminus(A\cap{B})=(X\setminus{A})\cup(X\setminus{B})
                \end{equation}
        \end{itemize}
    \end{problem}
    \begin{problem}
        \textbf{(The Cantor-Schroeder-Bernstein Theorem)}
        \par\hfill\par
        There are two versions of the Cantor-Schroeder-Bernstein theorem. The
        first says that if $A$ and $B$ are sets, and if $f:A\rightarrow{B}$ and
        $g:B\rightarrow{A}$ are injective, then there is a bijection
        $h:A\rightarrow{B}$. The second states that if
        $A$ and $B$ are sets, and if $f:A\rightarrow{B}$ and $g:B\rightarrow{A}$
        are surjective, then there is a bijection $h:A\rightarrow{B}$.
        \begin{itemize}
            \item (3 Points) Prove that if $f:A\rightarrow{B}$ is an injective
                function, then there is a surjection $g:B\rightarrow{A}$.
            \item (3 Points) Prove that if $f:A\rightarrow{B}$ is a surjective
                function, then there is an injection $g:B\rightarrow{A}$.
            \item (4 Points) Prove that the truth of the first
                Cantor-Schroeder-Bernstein theorem implies the validity of the
                second, and vice-versa.
        \end{itemize}
    \end{problem}
    \begin{problem}
        \textbf{(Induced Metrics)}
        \par\hfill\par
        A \textit{norm} on $\mathbb{R}^{n}$ is a function that assigns a
        \textit{length} to each point. That is, a function
        $||\cdot||:\mathbb{R}^{n}\rightarrow\mathbb{R}$ such that for all
        points $\mathbf{x},\mathbf{y}\in\mathbb{R}^{n}$ and all real numbers
        $a\in\mathbb{R}$ we have:
        \begin{align}
            ||\mathbf{x}||&\geq{0}&\tag{Positivity}\\
            ||\mathbf{x}||&=0
                \Rightarrow\mathbf{x}=\mathbf{0}\tag{Definiteness}\\
            ||a\mathbf{x}||&=|a|\cdot||\mathbf{x}||
                \tag{Homogeneity}\\
            ||\mathbf{x}+\mathbf{y}||&\leq||\mathbf{x}||+||\mathbf{y}||
                \tag{Triangle-Inequality}
        \end{align}
        The metric induced by a norm is:
        \begin{equation}
            d(\mathbf{x},\,\mathbf{y})=||\mathbf{x}-\mathbf{y}||
        \end{equation}
        \begin{itemize}
            \item (4 Points) Prove that the induced metric is a metric on
                $\mathbb{R}^{n}$.
            \item (6 Points) A convex set is a set $A\subseteq\mathbb{R}^{n}$
                such that for all $\mathbf{x},\mathbf{y}\in{A}$ and for all
                $0\leq{t}\leq{1}$ it is true that
                $t\mathbf{x}+(1-t)\mathbf{y}\in{A}$. Prove that open balls
                centered about the origin are convex when the metric comes from
                a norm.
        \end{itemize}
    \end{problem}
    \begin{problem}
        \textbf{(Connected Subsets)}
        \par\hfill\par
        (4 Points)
        A connected subset of a metric space $(X,d)$ is a subset
        $A\subseteq{X}$ such that it is impossible to write
        $A=\mathcal{U}\cup\mathcal{V}$ where $\mathcal{U}$ and $\mathcal{V}$
        are disjoint non-empty open sets. Give an example that shows that
        open balls do not need to be connected.
    \end{problem}
\end{document}
