%-----------------------------------LICENSE------------------------------------%
%   This file is part of Mathematics-and-Physics.                              %
%                                                                              %
%   Mathematics-and-Physics is free software: you can redistribute it and/or   %
%   modify it under the terms of the GNU General Public License as             %
%   published by the Free Software Foundation, either version 3 of the         %
%   License, or (at your option) any later version.                            %
%                                                                              %
%   Mathematics-and-Physics is distributed in the hope that it will be useful, %
%   but WITHOUT ANY WARRANTY; without even the implied warranty of             %
%   MERCHANTABILITY or FITNESS FOR A PARTICULAR PURPOSE.  See the              %
%   GNU General Public License for more details.                               %
%                                                                              %
%   You should have received a copy of the GNU General Public License along    %
%   with Mathematics-and-Physics.  If not, see <https://www.gnu.org/licenses/>.%
%------------------------------------------------------------------------------%
\documentclass{article}
\usepackage{graphicx}                           % Needed for figures.
\usepackage{amsmath}                            % Needed for align.
\usepackage{amssymb}                            % Needed for mathbb.
\usepackage{amsthm}                             % For the theorem environment.
\usepackage{hyperref}
\hypersetup{colorlinks=true, linkcolor=blue}

%------------------------Theorem Styles-------------------------%

% Define theorem style for default spacing and normal font.
\newtheoremstyle{normal}
    {\topsep}               % Amount of space above the theorem.
    {\topsep}               % Amount of space below the theorem.
    {}                      % Font used for body of theorem.
    {}                      % Measure of space to indent.
    {\bfseries}             % Font of the header of the theorem.
    {}                      % Punctuation between head and body.
    {.5em}                  % Space after theorem head.
    {}

% Define default environments.
\theoremstyle{normal}
\newtheorem{problem}{Problem}

\title{Point-Set Topology: Homework 4}
\date{Summer 2022}

% No indent and no paragraph skip.
\setlength{\parindent}{0em}
\setlength{\parskip}{0em}

\begin{document}
    \maketitle
    \begin{problem}
        Locally compact has two meanings, unfortunately. And they are not
        equivalent. To avoid ambiguity, some authors call one notion
        \textit{locally compact} and another notion
        \textit{strongly locally compact}.
        \begin{itemize}
            \item Locally Compact Topological Space: A locally compact
                topological space is a topological space $(X,\,\tau)$ such that
                for all $x\in{X}$ there is an open set $\mathcal{U}\in\tau$
                and a compact subset $K\subseteq{X}$ such that
                $x\in\mathcal{U}$ and $\mathcal{U}\subseteq{K}$.
            \item Strongly Locally Compact Topological Space: A strongly
                locally compact topological space is a topological space
                $(X,\,\tau)$ such that for all $x\in{X}$ there is a
                neighborhood basis $\mathcal{B}$ of $x$ such that for all
                $\mathcal{U}\in\mathcal{B}$, $\textrm{Cl}_{\tau}(\mathcal{U})$
                is compact.
        \end{itemize}
        Your task is to show there is no ambiguity in a Hausdorff space.
        \begin{itemize}
            \item (1 Point) Prove strongly locally compact implies
                locally compact (No Hausdorffness needed).
            \item (3 Points) Prove that if $(X,\,\tau)$ is Hausdorff, then it
                is locally compact if and only if it is strongly locally
                compact.
        \end{itemize}
        The only space I know of that is locally compact but not strongly
        locally compact is the one point compactification of $\mathbb{Q}$.
        This is compact, since it is a compactification, and hence locally
        compact, but not strongly locally compact. It is not Hausdorff, however.
    \end{problem}
    \begin{problem}
        You've tackled Baire's first category theorem. Every completely
        metrizable space (a space that comes from a complete metric) is a
        Baire space. That is, the intersection of countably
        many open and dense subsets is still dense. You will now prove Baire's
        second category theorem, a locally compact Hausdorff space is a
        Baire space.
        \begin{itemize}
            \item (2 Points)
                Prove that if $(X,\,\tau)$ is a topological space, and if for
                all $n\in\mathbb{N}$, $\mathcal{C}_{n}\subseteq{X}$ is a
                non-empty closed compact subset such that
                $\mathcal{C}_{n+1}\subseteq\mathcal{C}_{n}$, then
                $\bigcap_{n\in\mathbb{N}}\mathcal{C}_{n}$ is non-empty.
                [Hint: Contradiction works well here. If it is empty, can you
                cover $\mathcal{C}_{0}$ with certain sets? Does this open cover
                have a finite subcover?]
            \item (2 Points)
                Prove that if $(X,\,\tau)$ is locally compact and Hausdorff, if
                $x\in{X}$, and if $\mathcal{U}\in\tau$ is such that
                $x\in\mathcal{U}$, then there is a compact $K\subseteq{X}$ and
                an open $\mathcal{V}\in\tau$ such that
                $x\in\mathcal{V}$, $\mathcal{V}\subseteq{K}$, and
                $K\subseteq\mathcal{U}$. [Hint: Locally compact Hausdorff
                implies regular.]
        \end{itemize}
        From here, the proof is a mimicry of the idea for completely metrizable
        spaces. Given $\mathcal{U}_{n}$, $n\in\mathbb{N}$, open and dense,
        and $\mathcal{W}\in\tau$ non-empty, we construct nested open sets
        $\mathcal{V}_{n}\subseteq\mathcal{W}\cap\bigcap_{k=0}^{n}\mathcal{U}_{k}$
        and compact nested non-empty sets $K_{n}$ such that
        $K_{n+1}\subseteq\mathcal{V}_{n}$. Using the intersection property of
        the $K_{n}$, we conclude $\mathcal{W}\cap\bigcap_{n\in\mathbb{N}}\mathcal{U}_{k}$
        is non-empty, meaning $\bigcap_{n\in\mathbb{N}}\mathcal{U}_{n}$ is
        dense. Note, the first Baire category theorem is not stronger than the
        second Baire category theorem, and vice-versa. The Paris plane is
        completely metrizable, but not locally compact. The long line is locally
        compact and Hausdorff, but not paracompact, and hence not metrizable,
        and hence not competely metrizable. Both theorems have separate
        applications that make them equally useful.
    \end{problem}
    \begin{problem}
        Let $(X,\,\tau)$ be a topological space. Prove that if
        $\mathcal{A}\subseteq\mathcal{P}(X)$ is locally finite (that is,
        every point $x\in{X}$ has an open set $\mathcal{U}\in\tau$ such that
        $x\in\mathcal{U}$ and $\mathcal{U}$ has non-empty intersection with
        only finitely many elements of $\mathcal{A}$), then the following are
        true:
        \begin{itemize}
            \item (2 Points) The set $\mathcal{A}'$ defined by:
                \begin{equation}
                    \mathcal{A}'=
                    \{\,\textrm{Cl}_{\tau}(A)\;|\;A\in\mathcal{A}\,\}
                \end{equation}
                is locally finite as well.
            \item (2 Points)
                \begin{equation}
                    \textrm{Cl}_{\tau}\Big(\bigcup_{A\in\mathcal{A}}A\Big)
                    =\bigcup_{A\in\mathcal{A}}\big(\textrm{Cl}_{\tau}(A)\big)
                \end{equation}
        \end{itemize}
        Note there is no requirement that $\mathcal{A}$ covers $X$, nor is there
        a requirement that $\mathcal{A}$ consists of open sets. The only
        requirement is that the collection of sets is locally finite.
    \end{problem}
    \begin{problem}
        Some definitions from class.
        \begin{itemize}
            \item Basis: A basis for a topological space $(X,\,\tau)$ is a
                subset $\mathcal{B}\subseteq\tau$ such that $\mathcal{B}$ is an
                open cover of $X$, and such that $\mathcal{B}$ generates $\tau$
                and for all $\mathcal{U},\mathcal{V}\in\mathcal{B}$ and for all
                $x\in\mathcal{U}\cap\mathcal{V}$ there is a
                $\mathcal{W}\in\mathcal{B}$ such that $x\in\mathcal{W}$ and
                $\mathcal{W}\subseteq\mathcal{U}\cap\mathcal{V}$.
            \item Locally Finite Collection: A locally finite collection of
                sets in a topological space $(X,\,\tau)$ is a set
                $\mathcal{A}\subseteq\mathcal{P}(X)$ such that for all $x\in{X}$
                there is a $\mathcal{U}\in\tau$ such that $x\in\mathcal{U}$ and
                only finitely many elements of $\mathcal{A}$ have non-empty
                intersection with $\mathcal{U}$.
            \item $\sigma$ Locally Finite Collection: A $\sigma$ locally finite
                collection of sets in a topological space $(X,\,\tau)$ is a set
                $\mathcal{A}\subseteq\mathcal{P}(X)$ such that there exists
                countably many sets $\mathcal{A}_{n}$, each of which is locally
                finite in $(X,\,\tau)$, such that
                $\mathcal{A}=\bigcup_{n\in\mathbb{N}}\mathcal{A}_{n}$.
            \item $\sigma$ Locally Finite Basis: A $\sigma$ locally finite
                basis of a topological space $(X,\,\tau)$ is a basis
                $\mathcal{B}$ of $\tau$ such that $\mathcal{B}$ is $\sigma$
                locally finite.
            \item Locally Metrizable: A locally metrizable topological space
                is a topological space $(X,\,\tau)$ such that for all $x\in{X}$
                there is a $\mathcal{U}\in\tau$ such that $x\in\mathcal{U}$ and
                $(\mathcal{U},\,\tau_{\mathcal{U}})$ is metrizable, where
                $\tau_{\mathcal{U}}$ is the subspace topology.
        \end{itemize}
        From class, the Nagata-Smirnov theorem says $(X,\,\tau)$ is metrizable
        if and only if it is Hausdorff, regular,
        and has a $\sigma$ locally finite basis. You may use this freely.
        \textbf{Smirnov's Theorem}: $(X,\,\tau)$ is metrizable if and only if
        it is Hausdorff, paracompact, and locally metrizable. One direction has
        already been proved. Metrizable implies Hausdorff, metrizable definitely
        implies locally metrizable (for each $x\in{X}$ pick $\mathcal{U}=X$),
        and metrizable implies paracompact by Stone's theorem (from class).
        Prove the other direction. Let $(X,\,\tau)$ be Hausdorff, locally
        metrizable, and paracompact. Prove it is metrizable.
        \begin{itemize}
            \item (1 Point) Why is $(X,\,\tau)$ regular?
            \item (1 Point) From being locally metrizable, there is an open
                cover $\mathcal{O}\subseteq\tau$ such that for all
                $\mathcal{U}\in\mathcal{O}$,
                $(\mathcal{U},\,\tau_{\mathcal{U}})$ is metrizable. Why is there
                a locally finite open refinement $\mathcal{X}$ of $\mathcal{O}$
                that still covers $X$?
            \item (2 Points)
                Since the elements of $\mathcal{X}$ are subsets of elements
                of $\mathcal{O}$, the elements of $\mathcal{X}$ are also
                metrizable (subspaces of metrizable spaces are metrizable).
                So for all $\mathcal{U}\in\mathcal{X}$ there is a metric
                $d_{\mathcal{U}}$ that induces the subspace topology
                $\tau_{\mathcal{U}}$. Given $x\in\mathcal{U}$ and
                $\varepsilon>0$, the open ball
                $B_{\varepsilon}^{(\mathcal{U},\,d_{\mathcal{U}})}(x)$ is
                open in $\mathcal{U}$. Why is it open in $X$?
            \item (1 Point)
                For all $q\in\mathbb{Q}^{+}$ let $\mathcal{A}_{q}$ be the set
                of all open balls of radius $q$ centered about all points
                $x\in\mathcal{U}$ for all $\mathcal{U}\in\mathcal{X}$. This is
                an open cover of $X$ for all $q\in\mathbb{Q}^{+}$. Again, for
                each $q\in\mathbb{Q}^{+}$ can you find a locally finite open
                refinement $\mathcal{Y}_{q}$ of $\mathcal{A}_{q}$ that covers
                $X$?
            \item (1 Point)
                Explain why $\mathcal{Y}=\bigcup_{q\in\mathbb{Q}^{+}}\mathcal{Y}_{q}$
                is a $\sigma$ locally finite open cover.
            \item (2 Points)
                We want to show $\mathcal{Y}$ is a basis for $\tau$.
                Given $x\in{X}$ and $\mathcal{V}\in\tau$ with $x\in\mathcal{V}$,
                since $\mathcal{X}$ is locally finite, there are only finitely
                many sets $\mathcal{U}_{0},\,\dots,\,\mathcal{U}_{n}$ in
                $\mathcal{X}$ that contain $x$. So
                $\mathcal{V}\cap\mathcal{U}_{k}$ is an open subset of
                $\mathcal{U}_{k}$ for all $k\in\mathbb{Z}_{n+1}$ that contains
                $x$, so there is an $\varepsilon_{k}>0$ such that
                $B_{\varepsilon_{k}}^{(\mathcal{U}_{k},\,d_{\mathcal{U}_{k}})}(x)\subseteq\mathcal{V}\cap\mathcal{U}_{k}$.
                Let $q\in\mathbb{Q}^{+}$ be less than
                $\textrm{min}\{\,\varepsilon_{k}\;|\;k\in\mathbb{Z}_{n+1}\,\}/2$.
                Since $\mathcal{Y}_{q}$ covers $X$ there is a set
                $\mathcal{W}\in\mathcal{Y}_{q}$ such that $x\in\mathcal{W}$.
                Since $\mathcal{Y}_{q}$ is a refinement of $\mathcal{A}_{q}$
                there is an open ball
                $B_{q}^{(\mathcal{U},\,d_{\mathcal{U}})}(y)\in\mathcal{A}_{q}$
                that contains $\mathcal{W}$. Show that $\mathcal{U}$ is actually
                one of the sets $\mathcal{U}_{0},\,\dots,\,\mathcal{U}_{n}$.
                [Hint: You just need to show that $x\in\mathcal{U}$ is true].
            \item (2 Points)
                Conclude that $\mathcal{W}\subseteq\mathcal{V}$, so
                $\mathcal{Y}$ is a basis. Conclude that $(X,\,\tau)$ is
                metrizable.
        \end{itemize}
    \end{problem}
    \begin{problem}
        (4 Points)
        The real projective space $\mathbb{RP}^{n}$ is the quotient of
        $\mathbb{R}^{n+1}\setminus\{\,\mathbf{0}\,\}$ by the equivalence
        relation $\mathbf{y}R\mathbf{x}$ if and only if
        $\mathbf{y}=\lambda\mathbf{x}$ for some
        $\lambda\in\mathbb{R}\setminus\{\,0\,\}$. Equipped with the quotient
        topology, $\mathbb{RP}^{n}$ is a topological manifold (this was proven
        in class). Show that $\mathbb{RP}^{n}$ is a compact
        topological manifold.
    \end{problem}
    \begin{problem}
        Prove that $(X,\,\tau)$ is a topological manifold if and only if
        it is locally Euclidean, Hausdorff, and $\sigma$ compact.
        \begin{itemize}
            \item (2 Points) Prove a metrizable Lindel\"{o}f space is
                second countable. [Hint: For all $n\in\mathbb{N}$, cover the
                space is $1/(n+1)$ balls. Use Lindel\"{o}f to extract a
                countable subcover $\mathcal{B}_{n}$. Consider the collection
                of all such open sets for all $n\in\mathbb{N}$. Prove this is
                a countable basis.]
            \item (2 Points) $\sigma$ compact implies Lindel\"{o}f. Using
                locally Euclidean, Hausdorff, and $\sigma$ compact, prove
                $(X,\,\tau)$ is compactly exhaustible. From class, a
                compactly exhaustible Hausdorff space is paracompact, hence
                $(X,\,\tau)$ is a locally metrizable (since locally Euclidean),
                Hausdorff, paracompact space, so by Smirnov's theorem it is
                metrizable. The previous part of the problem then shows that
                $(X,\,\tau)$ is second countable.
        \end{itemize}
    \end{problem}
    \begin{problem}
        (12 Points) This can be quite tricky, so I've made the bounty 12 points.
        Hopefully this pleases the masses. Let $(X,\,\tau)$ be locally
        Euclidean (for all $x\in{X}$ there is an open set
        $\mathcal{U}\in\tau$, $x\in\mathcal{U}$, and an injective continuous
        open mapping $f:\mathcal{U}\rightarrow\mathbb{R}^{n}$
        for some $n\in\mathbb{N}$), Hausdorff, and \textbf{connected}
        (this last part is very important). Prove $(X,\,\tau)$ is a
        topological manifold if and only if it is paracompact.
        [Hint: All that is missing is second countability. Prove a
        locally Euclidean, paracompact, connected, Hausdorff space is second
        countable.]
    \end{problem}
    \begin{problem}
        (2 Points) Think of a space $(X,\,\tau)$ that is locally Euclidean,
        Hausdorff, and paracompact, but not a manifold
        (Note: The assumption of connectedness has been dropped).
        [Hint: Give $\mathbb{R}$ a topology that comes from a
        particular metric. Problem 7 says the space better be disconnected.]
    \end{problem}
\end{document}
