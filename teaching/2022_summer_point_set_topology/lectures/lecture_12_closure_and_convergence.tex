%-----------------------------------LICENSE------------------------------------%
%   This file is part of Mathematics-and-Physics.                              %
%                                                                              %
%   Mathematics-and-Physics is free software: you can redistribute it and/or   %
%   modify it under the terms of the GNU General Public License as             %
%   published by the Free Software Foundation, either version 3 of the         %
%   License, or (at your option) any later version.                            %
%                                                                              %
%   Mathematics-and-Physics is distributed in the hope that it will be useful, %
%   but WITHOUT ANY WARRANTY; without even the implied warranty of             %
%   MERCHANTABILITY or FITNESS FOR A PARTICULAR PURPOSE.  See the              %
%   GNU General Public License for more details.                               %
%                                                                              %
%   You should have received a copy of the GNU General Public License along    %
%   with Mathematics-and-Physics.  If not, see <https://www.gnu.org/licenses/>.%
%------------------------------------------------------------------------------%
\documentclass{article}
\usepackage{graphicx}                           % Needed for figures.
\usepackage{amsmath}                            % Needed for align.
\usepackage{amssymb}                            % Needed for mathbb.
\usepackage{amsthm}                             % For the theorem environment.
\usepackage{hyperref}
\hypersetup{colorlinks=true, linkcolor=blue}

%------------------------Theorem Styles-------------------------%
\theoremstyle{plain}
\newtheorem{theorem}{Theorem}[section]

% Define theorem style for default spacing and normal font.
\newtheoremstyle{normal}
    {\topsep}               % Amount of space above the theorem.
    {\topsep}               % Amount of space below the theorem.
    {}                      % Font used for body of theorem.
    {}                      % Measure of space to indent.
    {\bfseries}             % Font of the header of the theorem.
    {}                      % Punctuation between head and body.
    {.5em}                  % Space after theorem head.
    {}

% Define default environments.
\theoremstyle{normal}
\newtheorem{examplex}{Example}[section]
\newtheorem{definitionx}{Definition}[section]

\newenvironment{example}{%
    \pushQED{\qed}\renewcommand{\qedsymbol}{$\blacksquare$}\examplex%
}{%
    \popQED\endexamplex%
}

\newenvironment{definition}{%
    \pushQED{\qed}\renewcommand{\qedsymbol}{$\blacksquare$}\definitionx%
}{%
    \popQED\enddefinitionx%
}

\title{Point-Set Topology: Lecture 12}
\author{Ryan Maguire}
\date{\today}

% No indent and no paragraph skip.
\setlength{\parindent}{0em}
\setlength{\parskip}{0em}

\begin{document}
    \maketitle
    \section{Closure, Interior, and Boundary}
        In previous lectures we used large collections of topologies to
        generate a new one. In particular, we took a collection
        of subsets $\mathcal{B}\subseteq\mathcal{P}(X)$, and looked at the
        set $T$ of all topologies $\tau$ on $X$ such that
        $\mathcal{B}\subseteq\tau$. This set $T$ is non-empty since
        $\mathcal{P}(X)\in{T}$. We then created a new topology via the
        intersection $\bigcap{T}$. This is the \textit{generated} topology.
        We now use a similar idea, but instead of collections of topologies,
        we look at collections of open and closed sets. We've seen some laws
        about open sets, these are the rules dictated by the definition of a
        topology. Using the De Morgan law's we get similar statements about
        closed sets.
        \begin{theorem}
            If $(X,\,\tau)$ is a topological space, then $\emptyset$ is closed.
        \end{theorem}
        \begin{proof}
            Since $X$ is open and $\emptyset=X\setminus{X}$, $\emptyset$ is
            closed.
        \end{proof}
        \begin{theorem}
            If $(X,\,\tau)$ is a topological space, then $X$ is closed.
        \end{theorem}
        \begin{proof}
            Since $\emptyset$ is open and $X=X\setminus\emptyset$, $X$ is
            closed.
        \end{proof}
        \begin{theorem}
            If $(X,\,\tau)$ is a topological space, and if
            $\mathcal{C},\,\mathcal{D}\subseteq{X}$ are closed, then
            $\mathcal{C}\cup\mathcal{D}$ is closed.
        \end{theorem}
        \begin{proof}
            Since $\mathcal{C}$ and $\mathcal{D}$ are closed,
            $X\setminus\mathcal{C}$ and $X\setminus\mathcal{D}$ are open.
            But then:
            \begin{equation}
                X\setminus(\mathcal{C}\cup\mathcal{D})
                =(X\setminus\mathcal{C})\cap(X\setminus\mathcal{D})
            \end{equation}
            which is the intersection of two open sets, which is therefore open,
            so $X\setminus(\mathcal{C}\cup\mathcal{D})$ is open. But then
            $\mathcal{C}\cup\mathcal{D}$ is closed.
        \end{proof}
        \begin{theorem}
            If $(X,\,\tau)$ is a topological space, and if
            $\mathcal{O}\subseteq\mathcal{P}(X)$ is such that for all
            $\mathcal{C}\in\mathcal{O}$ it is true that $\mathcal{C}$ is closed,
            then $\bigcap\mathcal{O}$ is closed.
        \end{theorem}
        \begin{proof}
            If $\mathcal{O}$ is empty, then $\bigcap\mathcal{O}=\emptyset$,
            which is closed. Otherwise we may write:
            \begin{equation}
                \bigcap\mathcal{O}
                =\bigcap_{\mathcal{C}\in\mathcal{O}}\mathcal{C}
                =\bigcap_{\mathcal{C}\in\mathcal{O}}\Big(
                    X\setminus(X\setminus\mathcal{C})
                \Big)
                =X\setminus\bigcup_{\mathcal{C}\in\mathcal{O}}(
                    X\setminus\mathcal{C}
                )
            \end{equation}
            Since all $\mathcal{C}$ are closed, $X\setminus\mathcal{C}$ is open,
            so this union is open, meaning $\bigcap\mathcal{O}$ is the
            complement of an open set and is therefore closed.
        \end{proof}
        \begin{theorem}
            If $(X,\,\tau)$ is a topological space, and if
            $\mathcal{O}\subseteq\mathcal{P}(X)$ is a finite set such that for
            all $\mathcal{C}\in\mathcal{O}$ it is true that $\mathcal{C}$ is
            closed, then $\bigcup\mathcal{O}$ is closed.
        \end{theorem}
        \begin{proof}
            We prove by induction. The base case is true by a previous
            theorem. Suppose the statement is true for all such $\mathcal{O}$
            with $n$ elements. Now, let $\mathcal{O}$ be a set of $n+1$ closed
            sets. That is, we may write
            $\mathcal{O}=\{\,\mathcal{C}_{0},\,\dots,\,\mathcal{C}_{n}\,\}$.
            Define $\mathcal{D}$ via:
            \begin{equation}
                \mathcal{D}=\bigcup_{k=0}^{n-1}\mathcal{C}_{k}
            \end{equation}
            Then $\mathcal{D}$ is the union of $n$ closed sets, and by
            the induction hypothesis it is closed. But then:
            \begin{equation}
                \bigcup\mathcal{O}
                =\bigcup_{k=0}^{n}\mathcal{C}_{k}
                =\mathcal{D}\cup\mathcal{C}_{n}
            \end{equation}
            which is the union of two closed sets, which is closed. Hence,
            by induction, $\bigcup\mathcal{O}$ is closed for any finite
            collection of closed sets.
        \end{proof}
        We use the intersection property to define \textit{closure}. Given
        \textit{any} subset $A\subseteq{X}$ in a topological space $(X,\,\tau)$
        there is at least one closed set containing $A$ since
        $A\subseteq{X}$ and $X$ is closed. The \textit{closure} of $A$ is the
        \textit{smallest} closed set containing $A$. We can be very precise
        about this.
        \begin{definition}[\textbf{Closure of a Set}]
            The closure of a subset $A\subseteq{X}$ in a topological space
            $(X,\,\tau)$ is the set $\textrm{Cl}_{\tau}(A)$ defined by:
            \begin{equation}
                \textrm{Cl}_{\tau}(A)=\bigcap\{\,\mathcal{C}\subseteq{X}\;|\;
                    \mathcal{C}\textrm{ is closed and }A\subseteq\mathcal{C}\,\}
            \end{equation}
            That is, the \textit{smallest} closed set containing $A$.
        \end{definition}
        \begin{theorem}
            If $(X,\,\tau)$ is a topological space and $A\subseteq{X}$, then
            $A\subseteq\textrm{Cl}_{\tau}(A)$.
        \end{theorem}
        \begin{proof}
            Let $\mathcal{O}$ be the set of all closed sets containing $A$.
            This set is non-empty since $X\in\mathcal{O}$. Given any element
            $\mathcal{C}\in\mathcal{O}$ we have $A\subseteq\mathcal{C}$ by
            definition. Hence, $A\subseteq\bigcap\mathcal{O}$. But
            $\textrm{Cl}_{\tau}(A)=\bigcap\mathcal{O}$, completing the proof.
        \end{proof}
        \begin{theorem}
            If $(X,\,\tau)$ is a topological space, and if $A\subseteq{X}$,
            then $\textrm{Cl}_{\tau}(A)$ is closed.
        \end{theorem}
        \begin{proof}
            Since $\textrm{Cl}_{\tau}(A)$ is the intersection of closed sets,
            it is closed.
        \end{proof}
        \begin{theorem}
            If $(X,\,\tau)$ is a topological space, then
            $\mathcal{C}\subseteq{X}$ is closed if and only if
            $\textrm{Cl}_{\tau}(\mathcal{C})=\mathcal{C}$.
        \end{theorem}
        \begin{proof}
            If $\mathcal{C}=\textrm{Cl}_{\tau}(\mathcal{C})$, then
            $\mathcal{C}$ is closed since $\textrm{Cl}_{\tau}(\mathcal{C})$ is
            closed. In the other direction, if
            $\mathcal{C}$ is closed, then $\mathcal{C}$ is a closed set that
            contains $\mathcal{C}$ since $\mathcal{C}\subseteq\mathcal{C}$. But
            then $\textrm{Cl}_{\tau}(\mathcal{C})\subseteq\mathcal{C}$. But
            $\mathcal{C}\subseteq\textrm{Cl}_{\tau}(\mathcal{C})$ is also true,
            so $\mathcal{C}=\textrm{Cl}_{\tau}(\mathcal{C})$.
        \end{proof}
        \begin{theorem}
            If $(X,\,\tau)$ is a topological space, and if $A\subseteq{X}$,
            then:
            \begin{equation}
                \textrm{Cl}_{\tau}\big(\textrm{Cl}_{\tau}(A)\big)
                =\textrm{Cl}_{\tau}(A)
            \end{equation}
        \end{theorem}
        \begin{proof}
            Since $\textrm{Cl}_{\tau}(A)$ is closed, we have that
            $\textrm{Cl}_{\tau}\big(\textrm{Cl}_{\tau}(A)\big)=\textrm{Cl}_{\tau}(A)$
            by the previous theorem.
        \end{proof}
        \begin{theorem}
            If $(X,\,\tau)$ is a topological space, if
            $A,B\subseteq{X}$, then:
            \begin{equation}
                \textrm{Cl}_{\tau}(A\cup{B})
                =\textrm{Cl}_{\tau}(A)\cup\textrm{Cl}_{\tau}(B)
            \end{equation}
        \end{theorem}
        \begin{proof}
            Since $\textrm{Cl}_{\tau}(A)$ and $\textrm{Cl}_{\tau}(B)$ are
            closed, and since
            $A\subseteq\textrm{Cl}_{\tau}(A)$ and
            $B\subseteq\textrm{Cl}_{\tau}(A)$, we have that
            $\textrm{Cl}_{\tau}(A)\cup\textrm{Cl}_{\tau}(B)$ is a closed set
            (since it is the union of two closed sets) that contains
            $A\cup{B}$. Hence
            $\textrm{Cl}_{\tau}(A\cup{B})\subseteq\textrm{Cl}_{\tau}(A)\cup\textrm{Cl}_{\tau}(B)$.
            But a closed set that contains $A\cup{B}$ is also a closed set that
            contains $A$, and a closed set that contains $A\cup{B}$ is also a
            closed set that contains $B$, so
            $\textrm{Cl}_{\tau}(A)\subseteq\textrm{Cl}_{\tau}(A\cup{B})$ and
            $\textrm{Cl}_{\tau}(B)\subseteq\textrm{Cl}_{\tau}(A\cup{B})$. Thus,
            $\textrm{Cl}_{\tau}(A)\cup\textrm{Cl}_{\tau}(B)\subseteq\textrm{Cl}_{\tau}(A\cup{B})$.
            Therefore,
            $\textrm{Cl}_{\tau}(A\cup{B})=\textrm{Cl}_{\tau}(A)\cup\textrm{Cl}_{\tau}(B)$.
        \end{proof}
        \begin{theorem}
            If $(X,\,\tau)$ is a topological space, then
            $\textrm{Cl}_{\tau}(\emptyset)=\emptyset$.
        \end{theorem}
        \begin{proof}
            This follows since $\emptyset$ is closed, and the closure of a
            closed set is itself.
        \end{proof}
        \begin{theorem}
            If $(X,\,\tau)$ is a topological space, then
            $\textrm{Cl}_{\tau}(X)=X$.
        \end{theorem}
        \begin{proof}
            This also follows since $X$ is closed.
        \end{proof}
        \begin{theorem}
            If $(X,\,\tau)$ is a topological space, if $A,B\subseteq{X}$, and
            if $A\subseteq{B}$, then
            $\textrm{Cl}_{\tau}(A)\subseteq\textrm{Cl}_{\tau}(B)$.
        \end{theorem}
        \begin{proof}
            Let $T_{A}$ be the set of all closed subsets of $X$ that contain
            $A$, and similarly define $T_{B}$. Since $A\subseteq{B}$, if
            $\mathcal{C}\subseteq{X}$ is a closed subset such that
            $B\subseteq\mathcal{C}$, since inclusion is transitive we have
            $A\subseteq\mathcal{C}$. That is $\mathcal{C}\in{T}_{B}$ implies
            $\mathcal{C}\in{T}_{A}$, and hence
            $T_{B}\subseteq{T}_{A}$. Intersections are order reversing, and
            hence $\bigcap{T}_{A}\subseteq\bigcap{T}_{B}$, meaning
            $\textrm{Cl}_{\tau}(A)\subseteq\textrm{Cl}_{\tau}(B)$.
        \end{proof}
        \begin{example}
            Take $\mathbb{R}$ with the standard topology. Let $\mathbb{Q}$
            be the set of all rational numbers. The closure of
            $\mathbb{Q}$ is all of $\mathbb{R}$. Every real number can be
            written as a limit point of $\mathbb{Q}$ since we may approximate
            any $x\in\mathbb{R}$ with a convergent sequence of rational numbers.
            Because of this
            $\textrm{Cl}_{\tau_{\mathbb{R}}}(\mathbb{Q})=\mathbb{R}$.
        \end{example}
        \begin{example}
            In the real line $\mathbb{R}$ with the standard topology
            $\tau_{\mathbb{R}}$, given $a,b\in\mathbb{R}$ with $a<b$, the
            closure of $(a,\,b)$ is the set $[a,\,b]$. In a metric space
            you can obtain the closure of a set by adding all of the
            limit points (points that can be approximated via sequences) of
            the set. Since $a$ and $b$ are limit points of $(a,\,b)$, we
            see that $\textrm{Cl}_{\tau_{\mathbb{R}}}\big((a,\,b)\big)=[a,\,b]$.
        \end{example}
        The interior of a set uses similar ideas, but using open sets and
        unions.
        \begin{definition}[\textbf{Interior of a Set}]
            The interior of a subset $A\subseteq{X}$ in a topological space
            $(X,\,\tau)$ is the set $\textrm{Int}_{\tau}(A)$ defined by:
            \begin{equation}
                \textrm{Int}_{\tau}(A)=\bigcup\{\,
                    \mathcal{U}\in\tau\;|\;\mathcal{U}\subseteq{A}\,\}
            \end{equation}
            That is, the \textit{largest} open set that is contained inside of
            $A$.
        \end{definition}
        \begin{theorem}
            If $(X,\,\tau)$ is a topological space and $A\subseteq{X}$,
            then $\textrm{Int}_{\tau}(A)\subseteq{A}$.
        \end{theorem}
        \begin{proof}
            Since $\textrm{Int}_{\tau}(A)$ is the union over open sets that are
            contained in $A$, the union is contained in $A$, meaning
            $\textrm{Int}_{\tau}(A)\subseteq{A}$.
        \end{proof}
        \begin{theorem}
            If $(X,\,\tau)$ is a topological space and $A\subseteq{X}$, then
            $\textrm{Int}_{\tau}(A)$ is open.
        \end{theorem}
        \begin{proof}
            Since $\textrm{Int}_{\tau}(A)$ is the union of open sets, it is
            open.
        \end{proof}
        \begin{theorem}
            If $(X,\,\tau)$ is a topological space, and if
            $\mathcal{U}\subseteq{X}$, then $\mathcal{U}\in\tau$ if and only
            if $\textrm{Int}_{\tau}(\mathcal{U})=\mathcal{U}$.
        \end{theorem}
        \begin{proof}
            If $\mathcal{U}=\textrm{Int}_{\tau}(\mathcal{U})$, then
            $\mathcal{U}$ is equal to an open set, and so is open. In the
            other direction, if $\mathcal{U}$ is open, then
            $\mathcal{U}$ is an open set that is contained in $\mathcal{U}$ since
            $\mathcal{U}\subseteq\mathcal{U}$. But then
            $\mathcal{U}\subseteq\textrm{Int}_{\tau}(\mathcal{U})$. But
            $\textrm{Int}_{\tau}(\mathcal{U})\subseteq\mathcal{U}$, so
            $\mathcal{U}=\textrm{Int}_{\tau}(\mathcal{U})$.
        \end{proof}
        \begin{theorem}
            If $(X,\,\tau)$ is a topological space, and if $A\subseteq{X}$,
            then:
            \begin{equation}
                \textrm{Int}_{\tau}\big(\textrm{Int}_{\tau}(A)\big)
                =\textrm{Int}_{\tau}(A)
            \end{equation}
        \end{theorem}
        \begin{proof}
            Since $\textrm{Int}_{\tau}(A)$ is open, we have that
            $\textrm{Int}_{\tau}\big(\textrm{Int}_{\tau}(A)\big)=\textrm{Int}_{\tau}(A)$
            by the previous theorem.
        \end{proof}
        \begin{theorem}
            If $(X,\,\tau)$ is a topological space, and if
            $A,B\subseteq{X}$, then:
            \begin{equation}
                \textrm{Int}_{\tau}(A\cap{B})
                =\textrm{Int}_{\tau}(A)\cap\textrm{Int}_{\tau}(B)
            \end{equation}
        \end{theorem}
        \begin{proof}
            Since $\textrm{Int}_{\tau}(A\cap{B})$ is an open set that is
            contained inside of $A$, we have
            $\textrm{Int}_{\tau}(A\cap{B})\subseteq\textrm{Int}_{\tau}(A)$. But
            $\textrm{Int}_{\tau}(A\cap{B})$ is also an open set contained inside
            of $B$, so
            $\textrm{Int}_{\tau}(A\cap{B})\subseteq\textrm{Int}_{\tau}(B)$.
            But then
            $\textrm{Int}_{\tau}(A\cap{B})\subseteq\textrm{Int}_{\tau}(A)\cap\textrm{Int}_{\tau}(B)$.
            Since $\textrm{Int}_{\tau}(A)$ and $\textrm{Int}_{\tau}(B)$ are
            open, $\textrm{Int}_{\tau}(A)\cap\textrm{Int}_{\tau}(B)$ is open.
            But this is an open set that is contains inside of
            $A\cap{B}$, meaning
            $\textrm{Int}_{\tau}(A)\cap\textrm{Int}_{\tau}(B)\subseteq\textrm{Int}_{\tau}(A\cap{B})$.
            Hence,
            $\textrm{Int}_{\tau}(A\cap{B})=\textrm{Int}_{\tau}(A)\cap\textrm{Int}_{\tau}(B)$.
        \end{proof}
        \begin{theorem}
            If $(X,\,\tau)$ is a topological space, then
            $\textrm{Int}_{\tau}(\emptyset)=\emptyset$.
        \end{theorem}
        \begin{proof}
            Since $\emptyset$ is open, it is equal to its interior.
        \end{proof}
        \begin{theorem}
            If $(X,\,\tau)$ is a topological space, then
            $\textrm{Int}_{\tau}(X)=X$.
        \end{theorem}
        \begin{proof}
            This also follows from the fact that $X$ is open.
        \end{proof}
        \begin{theorem}
            If $(X,\,\tau)$ is a topological space, if $A,B\subseteq{X}$, and
            if $A\subseteq{B}$, then
            $\textrm{Int}_{\tau}(A)\subseteq\textrm{Int}_{\tau}(B)$.
        \end{theorem}
        \begin{proof}
            If $x\in\textrm{Int}_{\tau}(A)$, then there is an open set
            $\mathcal{U}\subseteq{A}$ such that $x\in\mathcal{U}$. But since
            $A\subseteq{B}$ we have $\mathcal{U}\subseteq{B}$, and since
            $\mathcal{U}$ is open it is true that
            $\mathcal{U}\subseteq\textrm{Int}_{\tau}(B)$. Therefore
            $x\in\textrm{Int}_{\tau}(B)$, so
            $\textrm{Int}_{\tau}(A)\subseteq\textrm{Int}_{\tau}(B)$.
        \end{proof}
        \begin{example}
            Let $A=\mathbb{Q}$ as a subset of the standard topology on
            $\mathbb{R}$. The interior
            $\textrm{Int}_{\tau_{\mathbb{R}}}(\mathbb{Q})$ is \textit{empty}.
            The only open subset $\mathcal{U}\subseteq\mathbb{Q}$ is the empty
            set. Given any $x\in\mathbb{Q}$ and any positive $\varepsilon>0$
            there are points $y\in\mathbb{R}$ such that $y$ is irrational
            and $|x-y|<\varepsilon$. So there are no open balls centered about
            any rational points that contain only rational numbers, meaning
            $\textrm{Int}_{\tau_{\mathbb{R}}}(\mathbb{Q})=\emptyset$.
        \end{example}
        \begin{example}
            Given the standard topology on $\mathbb{R}$, $\tau_{\mathbb{R}}$,
            the interior of the closed interval $[a,\,b]$ with
            $a<b$ is the open interval $(a,\,b)$. This is the largest open
            subset of $[a,\,b]$.
        \end{example}
        \begin{example}
            Let $X=[0,\,1]$ and $\tau_{X}$ be the topology induced by the
            subspace metric. That is, given the standard metric
            $d(x,\,y)=|x-y|$ on $\mathbb{R}$, we create the subspace metric
            $d_{X}(x,\,y)=d(x,\,y)$ for all $x,y\in{X}$. This induces a topology
            on $[0,\,1]$. What is the interior of $[0,\,1]$ with respect
            to the topology $\tau_{X}$? It is tempting to say the interior
            is $(0,\,1)$, but this is \textbf{false}. Do not confuse the
            topology $\tau_{\mathbb{R}}$ with the topology
            $\tau_{X}$. In $\mathbb{R}$, the interior of $[0,\,1]$ is indeed
            $(0,\,1)$. In $\tau_{X}$ the interior of $[0,\,1]$ is
            $[0,\,1]$. We are not considering $X=[0,\,1]$ as a subset of the
            real line anymore, but rather as it's own topological space
            $(X,\,\tau_{X})$. In this topological space the number 2 does not
            exist, nor does $-1$. The entire space $X$ is open in $\tau_{X}$,
            so $\textrm{Int}_{\tau_{X}}(X)=X$.
        \end{example}
        \begin{definition}[\textbf{Topological Boundary}]
            The boundary of a subset $A\subseteq{X}$ in a topological space
            $(X,\,\tau)$ is the set
            $\partial_{\tau}(A)=\textrm{Cl}_{\tau}(A)\setminus\textrm{Int}_{\tau}(A)$.
        \end{definition}
        Boundaries are always closed. This is because the set difference of an
        open set from a closed set is always closed.
        \begin{theorem}
            If $(X,\,\tau)$ is a topological space, if $\mathcal{U}\in\tau$,
            and if $\mathcal{C}$ is closed, then
            $\mathcal{C}\setminus\mathcal{U}$ is closed.
        \end{theorem}
        \begin{proof}
            Since $\mathcal{U},\mathcal{C}\subseteq{X}$, we can use the
            following fact from set theory. If $A$, $B$, and $C$ are sets,
            and if $A,B\subseteq{C}$, then:
            \begin{equation}
                A\setminus{B}=A\cap(C\setminus{B})
            \end{equation}
            We have (with $A=\mathcal{C}$, $B=\mathcal{U}$, and
            $C=X$):
            \begin{equation}
                \mathcal{C}\setminus\mathcal{U}
                =\mathcal{C}\cap(X\setminus\mathcal{U})
            \end{equation}
            But $\mathcal{U}$ is open, so $X\setminus\mathcal{U}$ is closed.
            But $\mathcal{C}$ is closed, so this is the intersection of two
            closed sets, which is closed. Therefore,
            $\mathcal{C}\setminus\mathcal{U}$ is closed.
        \end{proof}
        \begin{theorem}
            If $(X,\,\tau)$ is a topological space, and if $A\subseteq{X}$,
            then $\partial_{\tau}(A)$ is closed.
        \end{theorem}
        \begin{proof}
            Applying the previous theorem, since $\textrm{Cl}_{\tau}(A)$ is
            closed and $\textrm{Int}_{\tau}(A)$ is open,
            $\partial_{\tau}(A)=\textrm{Cl}_{\tau}(A)\setminus\textrm{Int}_{\tau}(A)$
            is closed.
        \end{proof}
        \begin{figure}
            \centering
            \includegraphics{../../../images/interior_boundary_and_closure_001.pdf}
            \caption{Interior, Closure, and Boundary}
            \label{fig:interior_boundary_and_closure_001}
        \end{figure}
    \section{Sequences and Convergence}
        Convergence in a metric space required the metric, but we can alter the
        definition to only use open sets. In a metric space $(X,\,d)$, we said
        $a:\mathbb{N}\rightarrow{X}$ converges to $x\in{X}$, written
        $a_{n}\rightarrow{x}$, if for all $\varepsilon>0$ there is an
        $N\in\mathbb{N}$ such that $n\in\mathbb{N}$ and $n>N$ implies
        $d(x,\,a_{n})<\varepsilon$. Worded differently, the sequence is
        \textit{eventually} contained inside the $\varepsilon$ ball centered
        at $x$ for all $\varepsilon>0$. $\varepsilon$ balls are, in particular,
        open sets, so we can say that $a_{n}\rightarrow{x}$ if for every
        open set $\mathcal{U}\subseteq{X}$ such that $x\in\mathcal{U}$, there
        is an $N\in\mathbb{N}$ such that $n\in\mathbb{N}$ and $n>N$ implies
        $a_{n}\in\mathcal{U}$. This final definition, which is equivalent to
        the metric one, relies only on open sets and can be phrased in a
        topological space.
        \begin{definition}[\textbf{Convergent Sequence in a Topological Space}]
            A convergent sequence in a topological space $(X,\,\tau)$ is a
            sequence $a:\mathbb{N}\rightarrow{X}$ such that there is an
            $x\in{X}$ such that for all $\mathcal{U}\in\tau$ with
            $x\in\mathcal{U}$ there is an $N\in\mathbb{N}$ such that for all
            $n\in\mathbb{N}$ with $n>N$ it is true that $a_{n}\in\mathcal{U}$.
            We write $a_{n}\rightarrow{x}$.
        \end{definition}
        The first theorem we proved in a metric space was that limits are
        unique, meaning we can say \textit{the} limit of the sequence. This
        is not true in a general topological space. This is a very important
        distinction. In metric spaces we used sequences to define continuity.
        We can still use this in topological space. Given
        $(X,\,\tau_{X})$ and $(Y,\,\tau_{Y})$, we can require a function
        $f:X\rightarrow{Y}$ to be such that if $a:\mathbb{N}\rightarrow{X}$
        is a convergent sequence such that $a_{n}\rightarrow{x}$ with
        $x\in{X}$, then $f(a_{n})$ is a convergent sequence in $Y$ and
        $f(a_{n})\rightarrow{f}(x)$. This type of function does indeed get a
        name, it's called a \textit{sequentially continuous} function. It is
        inadequate to describe general continuity in a general topological
        space. The following examples should show why.
        \begin{example}
            Equip $\mathbb{R}$ with the indiscrete topology,
            $\tau=\{\,\emptyset,\,\mathbb{R}\,\}$. Let
            $a:\mathbb{N}\rightarrow\mathbb{R}$ be \textit{any} sequence.
            Then for all $x\in\mathbb{R}$, the sequence $a$ converges to
            $x$. Let's prove this. To show $a_{n}\rightarrow{x}$ we need to
            show that for all $\mathcal{U}\in\tau$ with $x\in\mathcal{U}$ there
            is an $N\in\mathbb{N}$ such that $n\in\mathbb{N}$ and $n>N$ implies
            $a_{n}\in\mathcal{U}$. But the only open sets in $\tau$ are
            $\emptyset$ and $\mathbb{R}$. So if $x\in\mathcal{U}$, then
            $\mathcal{U}\ne\emptyset$, so $\mathcal{U}=\mathbb{R}$. Pick
            $N=0$. Then for all $n\in\mathbb{N}$ with $n>N$, since
            $a:\mathbb{N}\rightarrow\mathbb{R}$ is a sequence in $\mathbb{R}$,
            we have that $a_{n}\in\mathbb{R}$. This shows that
            $a_{n}\rightarrow{x}$ regardless of $x\in\mathbb{R}$.
        \end{example}
        \begin{example}
            There's nothing special about $\mathbb{R}$ for the previous example,
            the set is just more concrete for visualization. If $X$ is a set,
            and $\tau=\{\,\emptyset,\,X\,\}$ is the indiscrete topology on $X$,
            then given any sequence $a:\mathbb{N}\rightarrow{X}$, and any
            $x\in{X}$, it is true that $a_{n}\rightarrow{x}$.
        \end{example}
        \begin{example}
            Let $X=\mathbb{N}$ and $\tau$ be the set of all $\mathbb{Z}_{n}$,
            $n\in\mathbb{N}$, together with $\mathbb{N}$. That is:
            \begin{equation}
                \tau=\{\,\mathbb{Z}_{n}\;|\;n\in\mathbb{N}\,\}
                    \cup\{\,\mathbb{N}\,\}
            \end{equation}
            $(\mathbb{N},\,\tau)$ is a topological space.
            Let $a:\mathbb{N}\rightarrow\mathbb{N}$ be the sequence:
            \begin{equation}
                a_{n}=
                \begin{cases}
                    1&n\textrm{ is even}\\
                    2&n\textrm{ is odd}
                \end{cases}
            \end{equation}
            Does $a_{n}$ converge to 0? No, let
            $\mathcal{U}=\mathbb{Z}_{1}=\{\,0\,\}$. Then $a_{n}$ is never
            contained in the set $\mathcal{U}$, so $a_{n}$ can not converge
            to 0. Does $a_{n}$ converge to 1? Also no. Let
            $\mathcal{V}=\mathbb{Z}_{2}=\{\,0,\,1\,\}$. Infinitely many
            $a_{n}$ are such that $a_{n}\notin\mathcal{V}$. In particular, for
            all odd integers $n\in\mathbb{N}$, $a_{n}\notin\mathcal{V}$. Does
            $a_{n}$ converge to 2? Any open set the contains 2 also contains 1,
            so given any $\mathcal{U}\in\tau$ with $2\in\mathcal{U}$, choose
            $N=0$. For all $n>N$ we have $a_{n}\in\mathcal{U}$. So
            $a_{n}\rightarrow{2}$. Also, $a_{n}\rightarrow{3}$ and
            $a_{n}\rightarrow{4}$. Moreover, for every integer
            $k>1$, $a_{n}\rightarrow{k}$ is a true statement.
        \end{example}
        \begin{example}
            Let $X$ be a set and $\tau=\mathcal{P}(X)$ be the discrete topology.
            Let $a:\mathbb{N}\rightarrow{X}$ be a sequence. Then given
            $x\in{X}$, $a_{n}\rightarrow{x}$ if and only if there is an
            $N\in\mathbb{N}$ such that for all $n>N$ we have
            $a_{n}=x$. To see this, choose $\mathcal{U}=\{\,x\,\}$. This set
            is open since it is a subset of $X$ and $\tau$ is the discrete
            topology. Applying the definition of convergence to this set
            shows that $a_{n}$ is eventually a constant.
        \end{example}
        \begin{example}
            Let $X=\mathbb{R}$ and $\tau_{C}$ be the countable complement
            topology. If $a:\mathbb{N}\rightarrow\mathbb{R}$ converges, then
            $a_{n}$ is eventually a constant. We can show that if
            $a_{n}$ converges to $x$, then eventually $a_{n}=x$ for at least
            one $n\in\mathbb{N}$. Define $A\subseteq\mathbb{R}$ via:
            \begin{equation}
                A=\{\,a_{n}\in\mathbb{R}\;|\;n\in\mathbb{N}\,\}
            \end{equation}
            This is a countable subset, so $\mathbb{R}\setminus{A}$ is open
            in the countable complement topology. If
            $a_{n}\ne{x}$ for all $n\in\mathbb{N}$, then
            $x\in\mathbb{R}\setminus{A}$. But this is an open set that contains
            $x$ and never contains any of the $a_{n}$, meaning $a_{n}$ can't
            possible converge to $x$. So if $a_{n}\rightarrow{x}$, then
            $a_{n}=x$ for at least one integer $n\in\mathbb{N}$. Now, we can
            show $a_{n}=x$ for all sufficiently large $n$. Define $B$ by:
            \begin{equation}
                B=\{\,a_{n}\in\mathbb{R}\;|\;n\in\mathbb{N}
                    \textrm{ and }a_{n}\ne{x}\,\}
            \end{equation}
            This set is also countable, so $\mathbb{R}\setminus{B}$ is open.
            Applying the definition of convergence shows that
            $a_{n}=x$ for all large $n\in\mathbb{N}$. Contrast this with
            convergence in the standard topology on $\mathbb{R}$. The
            sequence $a_{n}=\frac{1}{n+1}$ converges to zero but is never
            equal to zero. The countable complement topology does not have
            such sequences.
        \end{example}
        Uniqueness of limits is given by the Hausdorff property. All of the
        bizarre examples we've discussed so far involved non-Hausdorff spaces.
        \begin{theorem}
            If $(X,\,\tau)$ is a Hausdorff topological space, if
            $a:\mathbb{N}\rightarrow{X}$ is a convergent sequence, and if
            $x,y\in{X}$ are such that $a_{n}\rightarrow{x}$ and
            $a_{n}\rightarrow{y}$, then $x=y$.
        \end{theorem}
        \begin{proof}
            Suppose not. Since $x\ne{y}$ and $(X,\,\tau)$ is Hausdorff, there
            are open sets $\mathcal{U},\mathcal{V}\in\tau$ such that
            $x\in\mathcal{U}$, $y\in\mathcal{V}$, and
            $\mathcal{U}\cap\mathcal{V}=\emptyset$. But
            $a_{n}\rightarrow{x}$, so there is an $N_{0}\in\mathbb{N}$ such
            that for all $n\in\mathbb{N}$ with $n>N_{0}$ it is true that
            $a_{n}\in\mathcal{U}$. But also $a_{n}\rightarrow{y}$ so there
            is an $N_{1}\in\mathbb{N}$ such that $n\in\mathbb{N}$ and
            $n>N_{1}$ implies $a_{n}\in\mathcal{V}$. Let
            $N=\textrm{max}(N_{0},\,N_{1})$. Then for all $n\in\mathbb{N}$
            with $n>N$ we have $a_{n}\in\mathcal{U}$ and $a_{n}\in\mathcal{V}$.
            But $\mathcal{U}\cap\mathcal{V}=\emptyset$, which is a
            contradiction. Hence, $x=y$.
        \end{proof}
        \begin{example}
            The converse of this theorem is not true. It is possible for
            sequences to be unique, but the space to not be Hausdorff. The
            countable complement topology on $\mathbb{R}$ is an example.
        \end{example}
        Sequences were sufficient to describe open sets in metric spaces.
        We used the metric, but then proved that a set $\mathcal{U}$ in a
        metric space $(X,\,d)$ is open if and only if for every sequence
        $a:\mathbb{N}\rightarrow{X}$ that converges to some $x\in\mathcal{U}$,
        there is an $N\in\mathbb{N}$ such that $n>N$ implies
        $a_{n}\in\mathcal{U}$. We take this and use it to define
        \textit{sequentially open} subsets.
        \begin{definition}[\textbf{Sequentially Open Subset}]
            A sequentially open subset in a topological space $(X,\,\tau)$ is
            a set $\mathcal{U}\subseteq{X}$ such that for every sequence
            $a:\mathbb{N}\rightarrow{X}$ that converges to a point
            $x\in\mathcal{U}$ there exists an $N\in\mathbb{N}$ such that
            for all $n\in\mathbb{N}$ with $n>N$ it is true that
            $a_{n}\in\mathcal{U}$.
        \end{definition}
        This is insufficient for topological spaces. We need to use the
        topology to define openness, not just sequences.
        The countable complement topology on $\mathbb{R}$ gives us an example.
        Every subset of $\mathbb{R}$ is sequentially open in the countable
        complement topology since $a:\mathbb{N}\rightarrow\mathbb{R}$
        converges if and only if it is eventually constant. However, not every
        subset of $\mathbb{R}$ is open with the countable complement topology.
        \par\hfill\par
        Open always implies sequentially open, almost by definition.
        \begin{theorem}
            If $(X,\,\tau)$ is a topological space, and if $\mathcal{U}\in\tau$,
            then $\mathcal{U}$ is sequentially open.
        \end{theorem}
        \begin{proof}
            For let $a:\mathbb{N}\rightarrow{X}$ be a sequence that converges
            to $x\in\mathcal{U}$. Then, since $\mathcal{U}$ is open, by the
            definition of convergence there is an $N\in\mathbb{N}$ such that
            $n\in\mathbb{N}$ and $n>N$ implies $a_{n}\in\mathcal{U}$. Hence,
            $\mathcal{U}$ is sequentially open.
        \end{proof}
        \begin{definition}[\textbf{Sequential Topological Space}]
            A sequential topological space is a topological space $(X,\,\tau)$
            such that for all $\mathcal{U}\subseteq{X}$, $\mathcal{U}$ is open
            if and only if $\mathcal{U}$ is sequentially open.
        \end{definition}
        Sequential spaces are spaces where sequences are enough. Enough for
        just about everything. These are spaces where sequences can describe
        open sets, closed sets, and continuity. It is fortunate that most spaces
        one encounters are sequential.
    \section{Continuity}
        Sequences are not sufficient to describe continuity, since they are
        not sufficient to describe open sets. In the theory of metric spaces
        we proved that a function is continuous if and only if the
        pre-image of an open set is open. This only requires the topology,
        meaning it is perfect to describe continuity in the general topological
        setting.
        \begin{definition}[\textbf{Continuous Function Between Topological Spaces}]
            A continuous function from a topological space $(X,\,\tau_{X})$ to
            a topological space $(Y,\,\tau_{Y})$ is a function
            $f:X\rightarrow{Y}$ such that for all $\mathcal{V}\in\tau_{Y}$ it
            is true that $f^{-1}[\mathcal{V}]\in\tau_{X}$. That is, the
            pre-image of an open set is open.
        \end{definition}
        \begin{example}
            If $Y$ is a set, $\tau_{Y}=\{\,\emptyset,\,Y\,\}$ is the indiscrete
            topology, and if $(X,\,\tau_{X})$ is any topological space,
            then any function $f:X\rightarrow{Y}$ is continuous. We need to
            check for every open set $\mathcal{V}\in\tau_{Y}$ that the
            pre-image $f^{-1}[\mathcal{V}]$ is open. There are only two
            candidates to check. We have $f^{-1}[\emptyset]=\emptyset$ and
            $f^{-1}[Y]=X$, both of which are open sets in $\tau_{X}$.
            Hence, $f$ is continuous.
        \end{example}
        \begin{example}
            If $X$ is a set, $\tau_{X}=\mathcal{P}(X)$ is the discrete topology,
            and if $(Y,\,\tau_{Y})$ is any topological space, then for any
            function $f:X\rightarrow{Y}$ it is true that $f$ is continuous.
            Given any open subset $\mathcal{V}\in\tau_{Y}$, the pre-image
            is a subset of $X$, so $f^{-1}[\mathcal{V}]\in\mathcal{P}(X)$.
            That is, $f^{-1}[\mathcal{V}]\in\tau_{X}$, so
            $f^{-1}[\mathcal{V}]$ is open, and $f$ is continuous.
        \end{example}
        \begin{theorem}
            If $(X,\,\tau_{X})$ and $(Y,\,\tau_{Y})$ are topological spaces,
            then $f:X\rightarrow{Y}$ is continuous if and only if for every
            closed subset $\mathcal{D}\subseteq{Y}$, the pre-image
            $f^{-1}[\mathcal{D}]$ is closed in $X$.
        \end{theorem}
        \begin{proof}
            Suppose $f$ is continuous, and let $\mathcal{D}$ be closed.
            Then:
            \begin{equation}
                f^{-1}[Y\setminus\mathcal{D}]
                =f^{-1}[Y]\setminus{f}^{-1}[\mathcal{D}]
                =X\setminus{f}^{-1}[\mathcal{D}]
            \end{equation}
            But $Y\setminus\mathcal{D}$ is open since $\mathcal{D}$ is closed,
            so $X\setminus{f}^{-1}[\mathcal{D}]$ is open. But then
            $f^{-1}[\mathcal{D}]$ is closed. Now, suppose the pre-image of
            closed sets are closed. Given $\mathcal{V}\in\tau_{Y}$, we have:
            \begin{equation}
                f^{-1}[\mathcal{V}]
                =f^{-1}[Y\setminus(Y\setminus\mathcal{V})]
                =f^{-1}[Y]\setminus{f}^{-1}[Y\setminus\mathcal{V}]
                =X\setminus{f}^{-1}[Y\setminus\mathcal{V}]
            \end{equation}
            But $\mathcal{V}$ is open, so $Y\setminus\mathcal{V}$ is closed.
            By assumption $f^{-1}[Y\setminus\mathcal{V}]$ is closed, so
            $X\setminus{f}^{-1}[Y\setminus\mathcal{V}]$ is the complement of
            a closed set, and hence is open. That is, the pre-image of an open
            set is open, so $f$ is continuous.
        \end{proof}
        As mentioned, sequences are not enough to describe continuity. We give
        a new definition to functions that map convergent sequences to
        convergent sequences.
        \begin{definition}[\textbf{Sequentially Continuous Function}]
            A sequentially continuous function from a topological
            space $(X,\,\tau_{X})$ to a topological space $(Y,\,\tau_{Y})$
            is a function such that for every convergent sequence
            $a:\mathbb{N}\rightarrow{X}$ with $x\in{X}$ such that
            $a_{n}\rightarrow{x}$, it is true that $f(a_{n})\rightarrow{f}(x)$.
        \end{definition}
        There's no requirement that limits be unique in either $(X,\,\tau_{X})$
        nor $(Y,\,\tau_{Y})$. The definition does not need such a notion.
        Continuity always implies sequential continuity.
        \begin{theorem}
            If $(X,\,\tau_{X})$ and $(Y,\,\tau_{Y})$ are topological spaces,
            and if $f:X\rightarrow{Y}$ is a continuous function, then
            $f$ is sequentially continuous.
        \end{theorem}
        \begin{proof}
            Suppose not. Then there is a sequence
            $a:\mathbb{N}\rightarrow{X}$ and an $x\in{X}$ such that
            $a_{n}\rightarrow{x}$ but $f(a_{n})\not\rightarrow{f}(x)$.
            But if $f(a_{n})\not\rightarrow{f}(x)$, then by the definition of
            convergence, there is an open set $\mathcal{V}\in\tau_{Y}$ with
            $f(x)\in\mathcal{V}$ such
            that for all $N\in\mathbb{N}$ there is an $n\in\mathbb{N}$ with
            $n>N$ but $f(a_{n})\notin\mathcal{V}$. But $\mathcal{V}$ is open,
            and $f$ is continuous, so $f^{-1}[\mathcal{V}]$ is open.
            But since $f(x)\in\mathcal{V}$ it is true that
            $x\in{f}^{-1}[\mathcal{V}]$ by the definition of pre-image.
            But since $f^{-1}[\mathcal{V}]$ is open,
            $x\in{f}^{-1}[\mathcal{V}]$, and $a_{n}\rightarrow{x}$, there is
            an $N\in\mathbb{N}$ such that for all $n\in\mathbb{N}$ with
            $n>N$ it is true that $a_{n}\in{f}^{-1}[\mathcal{V}]$. But then
            $f(a_{n})\in\mathcal{V}$ for all $n>N$, which is a contradiction.
            Hence, $f$ is sequentially continuous.
        \end{proof}
        \begin{example}
            The converse does not reverse, in general. Let
            $X=Y=\mathbb{R}$, let $\tau_{X}=\tau_{C}$, the countable complement
            topology, and let $\tau_{Y}=\tau_{\mathbb{R}}$ be the
            standard topology. Define $f:\mathbb{R}\rightarrow\mathbb{R}$
            by $f(x)=x$. Then $f$ is \textit{not} continuous, but it is
            sequentially continuous. It is not continuous since
            $(0,\,1)$ is open in $\tau_{\mathbb{R}}$, but
            $f^{-1}[(0,\,1)]=(0,\,1)$, and $(0,\,1)$ is not open in
            $\tau_{C}$. $f$ is sequentially continuous. If
            $a:\mathbb{N}\rightarrow\mathbb{R}$ converges to
            $x\in\mathbb{R}$ with respect to $\tau_{C}$, then there is an
            $N\in\mathbb{N}$ such that for all $n\in\mathbb{N}$ with $n>N$
            we have $a_{n}=x$. But then $f(a_{n})=x$ for all
            $n>N$, and therefore $f(a_{n})\rightarrow{f}(x)$.
        \end{example}
        \begin{theorem}
            If $(X,\,\tau_{X})$ is a sequential topological space, if
            $(Y,\,\tau_{Y})$ is a topological space, and if
            $f:X\rightarrow{Y}$ is a function, then $f$ is continuous if and
            only if $f$ is sequentially continuous.
        \end{theorem}
        \begin{proof}
            Continuity implies sequential continuity in every setting.
            Let's go the other way. Suppose $f:X\rightarrow{Y}$ is
            sequentially continuous. Suppose $f$ is \textit{not} continuous.
            Then there is a $\mathcal{V}\in\tau_{Y}$ such that
            $f^{-1}[\mathcal{V}]\notin\tau_{X}$. That is, there is an open
            set in $Y$ whose pre-image is not open in $X$. But
            $(X,\,\tau_{X})$ is sequential, so if $f^{-1}[\mathcal{V}]$ is not
            open, then it is not sequentially open. But if
            $f^{-1}[\mathcal{V}]$ is not sequentially open, then
            there is a sequence $a:\mathbb{N}\rightarrow{X}$ and an
            $x\in{f}^{-1}[\mathcal{V}]$ such that
            $a_{n}\rightarrow{x}$, but for all $N\in\mathbb{N}$ there is an
            $n\in\mathbb{N}$ with $n>N$ such that
            $f(a_{n})\notin{f}^{-1}[\mathcal{V}]$. But $f$ is sequentially
            continuous, so if $a_{n}\rightarrow{x}$, then
            $f(a_{n})\rightarrow{f}(x)$. But then $\mathcal{V}$ is an open set
            containing $f(x)$ and $f(a_{n})\rightarrow{f}(x)$, so there is an
            $N\in\mathbb{N}$ such that for all $n\in\mathbb{N}$ with $n>N$
            it is true that $f(a_{n})\in\mathcal{V}$. But then for all
            $n>N$ it is true that $a_{n}\in{f}^{-1}[\mathcal{V}]$, which is
            a contradiction. Hence, $f$ is continuous.
        \end{proof}
\end{document}
