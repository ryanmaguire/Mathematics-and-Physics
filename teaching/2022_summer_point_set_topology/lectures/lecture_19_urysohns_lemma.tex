%-----------------------------------LICENSE------------------------------------%
%   This file is part of Mathematics-and-Physics.                              %
%                                                                              %
%   Mathematics-and-Physics is free software: you can redistribute it and/or   %
%   modify it under the terms of the GNU General Public License as             %
%   published by the Free Software Foundation, either version 3 of the         %
%   License, or (at your option) any later version.                            %
%                                                                              %
%   Mathematics-and-Physics is distributed in the hope that it will be useful, %
%   but WITHOUT ANY WARRANTY; without even the implied warranty of             %
%   MERCHANTABILITY or FITNESS FOR A PARTICULAR PURPOSE.  See the              %
%   GNU General Public License for more details.                               %
%                                                                              %
%   You should have received a copy of the GNU General Public License along    %
%   with Mathematics-and-Physics.  If not, see <https://www.gnu.org/licenses/>.%
%------------------------------------------------------------------------------%
\documentclass{article}
\usepackage{graphicx}                           % Needed for figures.
\usepackage{amsmath}                            % Needed for align.
\usepackage{amssymb}                            % Needed for mathbb.
\usepackage{amsthm}                             % For the theorem environment.
\usepackage{float}
\usepackage{tabularx, booktabs}
\usepackage[font=scriptsize,
            labelformat=simple,
            labelsep=colon]{subcaption} % Subfigure captions.
\usepackage[font={scriptsize},
            hypcap=true,
            labelsep=colon]{caption}    % Figure captions.
\usepackage{hyperref}
\hypersetup{
    colorlinks=true,
    linkcolor=blue,
    filecolor=magenta,
    urlcolor=Cerulean,
    citecolor=SkyBlue
}

%------------------------Theorem Styles-------------------------%
\theoremstyle{plain}
\newtheorem{theorem}{Theorem}[section]

% Define theorem style for default spacing and normal font.
\newtheoremstyle{normal}
    {\topsep}               % Amount of space above the theorem.
    {\topsep}               % Amount of space below the theorem.
    {}                      % Font used for body of theorem.
    {}                      % Measure of space to indent.
    {\bfseries}             % Font of the header of the theorem.
    {}                      % Punctuation between head and body.
    {.5em}                  % Space after theorem head.
    {}

% Define default environments.
\theoremstyle{normal}
\newtheorem{examplex}{Example}[section]
\newtheorem{definitionx}{Definition}[section]

\newenvironment{example}{%
    \pushQED{\qed}\renewcommand{\qedsymbol}{$\blacksquare$}\examplex%
}{%
    \popQED\endexamplex%
}

\newenvironment{definition}{%
    \pushQED{\qed}\renewcommand{\qedsymbol}{$\blacksquare$}\definitionx%
}{%
    \popQED\enddefinitionx%
}

\title{Point-Set Topology: Lecture 20}
\author{Ryan Maguire}
\date{Summer 2022}

% No indent and no paragraph skip.
\setlength{\parindent}{0em}
\setlength{\parskip}{0em}

\begin{document}
    \maketitle
    \section{Urysohn's Lemma and Metrization Theorem}
        We now come to one of the major theorems of point-set topology, the
        so-called \textit{Urysohn Lemma}. The theorem deals with normal spaces
        and is used in the proof of one of the first metrization theorems.
        \begin{theorem}[\textbf{Urysohn's Lemma}]
            If $(X,\,\tau)$ is a normal topological space, and if
            $\mathcal{C},\mathcal{D}\subseteq{X}$ are disjoint closed subsets,
            then there is a continuous function
            $f:X\rightarrow[0,\,1]$, where $[0,\,1]$ has the subspace topology,
            such that $f[\mathcal{C}]=\{\,0\,\}$ and
            $f[\mathcal{D}]=\{\,1\,\}$. That is,
            $\mathcal{C}\subseteq{f}^{-1}[\{\,0\,\}]$ and
            $\mathcal{D}\subseteq{f}^{-1}[\{\,1\,\}]$.
        \end{theorem}
        \begin{proof}
            Let $A=\mathbb{Q}\cap[0,\,1]$, the set of all rational numbers
            between $0$ and $1$, inclusive. Since $\mathbb{Q}$ is countable,
            $A$ is countable as well. Moreover, $A$ is countably infinite since
            it is not finite. Let $a:\mathbb{N}\rightarrow{A}$ be a bijection
            such that $a_{0}=0$ and $a_{1}=1$. We will now define open sets
            $\mathcal{U}_{a_{n}}$ such that whenever $a_{m}<a_{n}$ is true
            we have:
            \begin{equation}
                \textrm{Cl}_{\tau}(\mathcal{U}_{a_{m}})
                \subseteq\mathcal{U}_{a_{n}}
            \end{equation}
            To start, define:
            \begin{equation}
                \mathcal{U}_{1}=X\setminus\mathcal{D}
            \end{equation}
            Since $(X,\,\tau)$ is normal and
            $\mathcal{C}\subseteq\mathcal{U}_{1}$ there is an open
            set $\mathcal{U}_{0}$ such that
            $\mathcal{C}\subseteq\mathcal{U}_{0}$ and
            $\textrm{Cl}_{\tau}(\mathcal{U}_{0})\subseteq\mathcal{U}_{1}$.
            Define $\mathcal{A}_{N}$ via:
            \begin{equation}
                \mathcal{A}_{N}=\{\,a_{n}\in{A}\;|\;n\in\mathbb{Z}_{N}\,\}
            \end{equation}
            That is, the first $N$ rational numbers given by the bijection
            $a:\mathbb{N}\rightarrow{A}$. We define $\mathcal{U}_{a_{n}}$
            recursively. Suppose $\mathcal{U}_{a_{n}}\in\tau$ has been defined
            for all $n\in\mathbb{Z}_{N}$ such that $a_{m}<a_{n}$ implies
            $\textrm{Cl}_{\tau}(\mathcal{U}_{a_{m}})\subseteq\mathcal{U}_{a_{n}}$.
            Since $\mathcal{A}_{N+1}$ is a subset of $\mathbb{Q}$, which is
            totally ordered, it is ordered as well. But it is also finite, and
            since $a_{N}\ne{0}$ and $a_{N}\ne{1}$, there are
            $a_{m},a_{n}\in\mathcal{A}_{N+1}$ such that
            $a_{m}<a_{N}$ and $a_{N}<a_{n}$ where $a_{m}$ is the largest
            such value and $a_{n}$ is the smallest such value. But by the
            recursive definition
            $\textrm{Cl}_{\tau}(\mathcal{U}_{a_{m}})\subseteq\mathcal{U}_{a_{n}}$.
            But $\textrm{Cl}_{\tau}(\mathcal{U}_{a_{m}})$ is closed and
            $\mathcal{U}_{a_{n}}$ is open, so since $(X,\,\tau)$ is normal there
            is $\mathcal{U}_{a_{N}}\in\tau$ such that
            $\textrm{Cl}_{\tau}(\mathcal{U}_{a_{m}})\subseteq\mathcal{U}_{a_{N}}$
            and $\mathcal{U}_{a_{N}}\subseteq\mathcal{U}_{a_{n}}$. By the
            principle of induction, such a set exists for all $a_{n}$. That is,
            we now have for all rational numbers $p,q\in{A}$ with $p<q$, the
            following:
            \begin{equation}
                \textrm{Cl}_{\tau}(\mathcal{U}_{p})\subseteq\mathcal{U}_{q}
            \end{equation}
            We extend this to all rationals as follows. Given
            $p\in\mathbb{Q}$, $p\not\in{A}$, define:
            \begin{equation}
                \mathcal{U}_{p}=
                \begin{cases}
                    X&p>1\\
                    \emptyset&p<0
                \end{cases}
            \end{equation}
            Define $F:X\rightarrow\mathcal{P}(\mathbb{Q})$ via:
            \begin{equation}
                F(x)=\{\,p\in\mathbb{Q}\;|\;x\in\mathcal{U}_{p}\,\}
            \end{equation}
            Define $f:X\rightarrow[0,\,1]$ via:
            \begin{equation}
                f(x)=\textrm{inf}\big(F(x)\big)
            \end{equation}
            First, since $A$ is bounded below by $0$, $f(x)$ is well-defined
            for all $x\in{X}$. We now need to show that $f$ is continuous,
            $f[\mathcal{C}]=\{\,0\,\}$, and $f[\mathcal{D}]=\{\,1\,\}$.
            First, $f[\mathcal{C}]=\{\,0\,\}$. If $x\in\mathcal{C}$, then
            $f(x)\in\mathcal{U}_{0}$ by definition of $\mathcal{U}_{0}$ (see
            above). Hence $0$ is the smallest value $p\in\mathbb{Q}$ such that
            $f(x)\in\mathcal{U}_{p}$, and hence $f(x)=0$. Next,
            $f[\mathcal{D}]=\{\,1\,\}$. By definition, for all $p\in\mathbb{Q}$
            with $p\leq{1}$, $f(x)\notin\mathcal{U}_{p}$. Hence
            $f(x)=\textrm{inf}\big((1,\,\infty)\big)=1$, so
            $f[\mathcal{D}]=\{\,1\,\}$. Lastly, we must prove $f$ is continuous.
            This follows from the fact that $A$ is a dense subset of $[0,\,1]$.
            First, if $p\in\mathbb{Q}$ and
            $x\in\textrm{Cl}_{\tau}(\mathcal{U}_{p})$, then $f(x)\leq{p}$.
            This is true since for all $q\in\mathbb{Q}$ with $p<q$ we have
            $\mathcal{U}_{p}\subseteq\mathcal{U}_{q}$, and hence:
            \begin{equation}
                f(x)
                =\textrm{inf}\big\{\,r\in\mathbb{Q}\;|\;
                    f(x)\in\mathcal{U}_{r}\,\big\}
                \leq{p}
            \end{equation} 
            so $f(x)\leq{p}$. Next, if $x\notin\mathcal{U}_{p}$, then
            $f(x)\geq{p}$. Since $x\notin\mathcal{U}_{p}$, the only values
            $q\in\mathbb{Q}$ with $x\in\mathcal{U}_{q}$ must be greater than
            $p$, and hence $f(x)\geq{p}$. To conclude, a function
            is continuous if and only if for all $x\in{X}$ and all open
            $\mathcal{V}$ with $f(x)\in\mathcal{V}$ there is an open
            $\mathcal{U}\subseteq{X}$ such that $x\in\mathcal{U}$ and
            $f[\mathcal{U}]\subseteq\mathcal{V}$. Let $x\in{X}$ and
            $\mathcal{V}\subseteq\mathbb{R}$ be an open set such that
            $f(x)\in\mathcal{V}$. But $\mathcal{V}$ is open so there is an
            $\epsilon>0$ such that $|y-f(x)|<\varepsilon$ implies
            $y\in\mathcal{V}$. Let $c=f(x)-\varepsilon/2$ and
            $d=f(x)+\varepsilon/2$. Let $p$ and $q$ be rational numbers such
            that $c<p<f(x)<q<d$. Define $\mathcal{U}$ via:
            \begin{equation}
                \mathcal{U}=\mathcal{U}_{q}
                    \setminus\textrm{Cl}_{\tau}(\mathcal{U}_{p})
            \end{equation}
            Then $\mathcal{U}$ is the difference of a closed set from an open
            set, and is hence open. By the above observation, for all
            $x_{0}\in\mathcal{U}$ we have $p\leq{f}(x)\leq{f}(q_{0})$, and hence
            $f[\mathcal{U}]\subseteq\mathcal{V}$. But also
            $x\in\mathcal{U}$ since $p<f(x)<q$. So $f$ is continuous.
        \end{proof}
        We get some use out of this immediately via Urysohn's metrization
        theorem.
        \begin{theorem}[\textbf{Urysohn's Metrization Theorem}]
            If $(X,\,\tau)$ is a regular second countable Hausdorff topological
            space, then it is metrizable.
        \end{theorem}
        \begin{proof}
            Since $(X,\,\tau)$ is regular and second countable, it is normal.
            But also since $(X,\,\tau)$ is second countable there is a countable
            basis $\mathcal{B}$ for $\tau$. Let
            $\mathcal{U}:\mathbb{N}\rightarrow\mathcal{B}$ be a surjection so
            that we may list the elements as:
            \begin{equation}
                \mathcal{B}=
                \{\,\mathcal{U}_{0},\,\dots,\,\mathcal{U}_{n},\,\dots\,\}
            \end{equation}
            For all $m,n\in\mathbb{N}$ with
            $\textrm{Cl}_{\tau}(\mathcal{U}_{m})\subseteq\mathcal{U}_{n}$,
            by Urysohn's lemma there is a continuous function
            $g_{m,n}:X\rightarrow[0,\,1]$ such that:
            \begin{equation}
                g_{m,n}[\textrm{Cl}_{\tau}(\mathcal{U}_{m})]=\{\,1\,\}
                \textrm{ and }
                g_{m,n}[X\setminus\mathcal{U}_{n}]=\{\,0\,\}
            \end{equation}
            The set of all such $\mathcal{U}_{m}$ cover $X$ since $\mathcal{B}$
            is a basis and the set of all such functions is countable since
            the elements are indexed by $\mathbb{N}\times\mathbb{N}$. Relabel
            these functions as $f_{n}:X\rightarrow[0,\,1]$ for all
            $n\in\mathbb{N}$. Define the function
            $F:X\rightarrow\mathbb{R}^{\infty}$ via:
            \begin{equation}
                F(x)=\big(f_{0}(x),\,\dots,\,f_{n}(x),\,\dots)
            \end{equation}
            Since $\mathbb{R}^{\infty}$ has the product topology, and since each
            component function $f_{n}$ is continuous, $F$ is continuous. $F$
            is injective since given $x,y\in{X}$ with $x\ne{y}$ one can find
            a basis element $\mathcal{U}_{n}$ such that $x\in\mathcal{U}_{n}$
            and $y\notin\mathcal{U}_{n}$, since $(X,\,\tau)$ is Hausdorff, but
            then there is a function $f_{n}$ such that $f_{n}(x)=1$ and
            $f_{n}(y)=0$, hence $F(x)\ne{F}(y)$ since one of the components
            is different. Lastly, we must show $F$ is a homeomorphism between
            $(X,\,\tau)$ and $(F[X],\,\tau_{\mathbb{R}^{\infty}_{F[X]}})$.
            Since $F:X\rightarrow\mathbb{R}^{\infty}$ is injective,
            $F:X\rightarrow{F}[X]$ is bijective. To show
            $F:X\rightarrow{F}[X]$ is a homeomorphism, since $F$ is continuous,
            all that's left to show is that $F$ is an open mapping.
            Let $\mathcal{U}\subseteq{X}$ be open, and given
            $y\in{f}[\mathcal{U}]$, let $x\in\mathcal{U}$ be such that
            $F(x)=y$. Since $x\in{X}$ there is an $n\in\mathbb{N}$ such that
            $f_{n}(x)>0$ and $f_{n}[X\setminus\mathcal{U}]=\{\,0\,\}$.
            Let $\mathcal{V}=\textrm{proj}_{n}^{-1}\big[(0,\,\infty)\big]$.
            Then, since projections are continuous and $(0,\,\infty)$ is open,
            $\mathcal{V}\subseteq\mathbb{R}^{\infty}$ is open. But then
            $f[X]\cap\mathcal{V}$ is open in $f[X]$ by definition of the
            subspace topology. But then $y\in{f}[X]\cap\mathcal{V}$, since:
            \begin{equation}
                \textrm{proj}_{n}(y)
                =\textrm{proj}_{n}\big(f(x)\big)
                =f_{n}(x)
            \end{equation}
            and $f_{n}(x)>0$, so $y\in\mathcal{V}$ by definition of
            $\mathcal{V}$. Lastly,
            $f[X]\cap\mathcal{V}\subseteq{f}[\mathcal{U}]$. For given
            $y\in{f}[X]\cap\mathcal{V}$, since $y\in{f}[X]$, there is some
            $x\in{X}$ such that $f(x)=y$. But if
            $y\in\mathcal{V}$, then $\textrm{proj}_{n}(y)>0$. But
            $f_{n}$ is the zero function outside of $\mathcal{U}$, and hence
            $y\in{f}[\mathcal{U}]$. Since $f[\mathcal{U}]$ can be written as
            the union of all such $\mathcal{V}$, $f[\mathcal{U}]$ is open.
            That is, $f$ is an open mapping with respect to the subspace
            topology on $f[X]$. Therefore $f:X\rightarrow\mathbb{R}^{\infty}$
            is a topological embedding, meaning $(X,\,\tau)$ is homeomorphic
            to a subspace of a metrizable space, and is therefore metrizable.
        \end{proof}
        The last theorem to show is the Tietze extension theorem. It is
        logically equivalent to Urysohn's lemma.
        \begin{theorem}[\textbf{Tietze Extension Theorem}]
            If $(X,\,\tau)$ is a normal topological space, if
            $\mathcal{C}\subseteq{X}$ is closed, and if
            $f:\mathcal{C}\rightarrow\mathbb{R}$ is continuous, then there is
            a continuous function $\tilde{f}:X\rightarrow\mathbb{R}$ such that
            $\tilde{f}|_{A}=f$, and if $f$ is bounded, then $\tilde{f}$ is
            bounded as well with the same bounds.
        \end{theorem}
        The proof is a bit lengthy, but I'd like to point out what this theorem
        does \textit{not} say. It does not say
        $f:\mathcal{C}\rightarrow{Y}$ can be extended to all of $X$ where
        $(Y,\,\tau_{Y})$ is any topological space. This is false. One need look
        no further than the Euclidean plane.
        $\mathbb{S}^{1}$ is a closed subset of the Euclidean plane
        $\mathbb{R}^{2}$. The identity function
        $f:\mathbb{S}^{1}\rightarrow\mathbb{S}^{1}$ is continuous. However,
        there is \textbf{no} extension of this function to all of
        $\mathbb{R}^{2}$. To map $\mathbb{R}^{2}$ to $\mathbb{S}^{1}$ while
        keeping $\mathbb{S}^{1}$ fixed means, intuitively, we'd need to tear the
        plane at some point. This is not continuous. Imagine you had a lump of
        dough in the shape of a disk. How would you push the inside
        of the lump of dough to the outside to make a circle? You'd need to
        press your fingers through the dough and make a hole.
        The Tietze extension theorem is only applicable when the co-domain is
        $\mathbb{R}$.
\end{document}
