%-----------------------------------LICENSE------------------------------------%
%   This file is part of Mathematics-and-Physics.                              %
%                                                                              %
%   Mathematics-and-Physics is free software: you can redistribute it and/or   %
%   modify it under the terms of the GNU General Public License as             %
%   published by the Free Software Foundation, either version 3 of the         %
%   License, or (at your option) any later version.                            %
%                                                                              %
%   Mathematics-and-Physics is distributed in the hope that it will be useful, %
%   but WITHOUT ANY WARRANTY; without even the implied warranty of             %
%   MERCHANTABILITY or FITNESS FOR A PARTICULAR PURPOSE.  See the              %
%   GNU General Public License for more details.                               %
%                                                                              %
%   You should have received a copy of the GNU General Public License along    %
%   with Mathematics-and-Physics.  If not, see <https://www.gnu.org/licenses/>.%
%------------------------------------------------------------------------------%
\documentclass{article}
\usepackage{graphicx}                           % Needed for figures.
\usepackage{amsmath}                            % Needed for align.
\usepackage{amssymb}                            % Needed for mathbb.
\usepackage{amsthm}                             % For the theorem environment.
\usepackage{float}
\usepackage{hyperref}
\hypersetup{
    colorlinks=true,
    linkcolor=blue,
    filecolor=magenta,
    urlcolor=Cerulean,
    citecolor=SkyBlue
}

%------------------------Theorem Styles-------------------------%
\theoremstyle{plain}
\newtheorem{theorem}{Theorem}[section]

% Define theorem style for default spacing and normal font.
\newtheoremstyle{normal}
    {\topsep}               % Amount of space above the theorem.
    {\topsep}               % Amount of space below the theorem.
    {}                      % Font used for body of theorem.
    {}                      % Measure of space to indent.
    {\bfseries}             % Font of the header of the theorem.
    {}                      % Punctuation between head and body.
    {.5em}                  % Space after theorem head.
    {}

% Define default environments.
\theoremstyle{normal}
\newtheorem{examplex}{Example}[section]
\newtheorem{definitionx}{Definition}[section]
\newtheorem{notationx}{Notation}[section]
\newtheorem{axiomx}{Axiom}[section]

\newenvironment{example}{%
    \pushQED{\qed}\renewcommand{\qedsymbol}{$\blacksquare$}\examplex%
}{%
    \popQED\endexamplex%
}

\newenvironment{definition}{%
    \pushQED{\qed}\renewcommand{\qedsymbol}{$\blacksquare$}\definitionx%
}{%
    \popQED\enddefinitionx%
}

\title{Point-Set Topology: Lecture 14}
\author{Ryan Maguire}
\date{Summer 2022}

% No indent and no paragraph skip.
\setlength{\parindent}{0em}
\setlength{\parskip}{0em}

\begin{document}
    \maketitle
    \section{Homeomorphisms and Open Mappings}
        Thus far we've discussed conditions for when continuity can be
        described by sequences. It is worthwhile studying the general notion
        of continuity as well. As a reminder, given two topological spaces
        $(X,\,\tau_{X})$ and $(Y,\,\tau_{Y})$, a continuous function from $X$
        to $Y$ is a function $f:X\rightarrow{Y}$ such that for all
        $\mathcal{V}\in\tau_{Y}$ it is true that
        $f^{-1}[\mathcal{V}]\in\tau_{X}$. That is, the pre-image of an open set
        is open. It was proved this is equivalent to the pre-image of a closed
        set being closed using some of the set-theoretic laws of pre-image and
        set difference. \textit{Homeomorphism} is a stronger notion. It tells
        us when two topological spaces are the same.
        \begin{definition}[\textbf{Homeomorphism}]
            A homeomorphism from a topological space $(X,\,\tau_{X})$ to a
            topological space $(Y,\,\tau_{Y})$ is a bijective continuous
            function $f:X\rightarrow{Y}$ such that $f^{-1}$ is continuous.
        \end{definition}
        \begin{example}
            Let $(X,\,\tau)$ be any topological space, and
            $f:X\rightarrow{X}$ be the identity function $f(x)=x$. Then $f$ is
            a homeomorphism. It is certainly a bijection, but it is also
            continuous. Given $\mathcal{U}\in\tau$ we have
            $f^{-1}[\mathcal{U}]=\mathcal{U}$, which is an element of $\tau$.
            Given $\mathcal{V}\in\tau$ we have
            $\big(f^{-1}\big)^{-1}[\mathcal{V}]=f[\mathcal{V}]=\mathcal{V}$,
            which is in $\tau$ (Note:
            $\big(f^{-1}\big)^{-1}[\mathcal{V}]=f[\mathcal{V}]=\mathcal{V}$
            is true since $f$ is a bijection). This shows $f$ is a
            homeomorphism.
        \end{example}
        \begin{example}
            Take $X=Y=\mathbb{R}$ and give both of these the standard Euclidean
            topology $\tau_{\mathbb{R}}$. The function
            $f:\mathbb{R}\rightarrow\mathbb{R}$ defined by $f(x)=x^{3}$ is
            a homeomorphism. It is bijective, continuous, and the inverse is
            given by $f^{-1}(x)=\sqrt[3]{x}=x^{1/3}$, which is also continuous.
        \end{example}
        \begin{example}
            Any continuous bijective function
            $f:\mathbb{R}\rightarrow\mathbb{R}$ is a homeomorphism with respect
            to the standard topology. This is \textbf{not} true for general
            topological spaces. \textbf{It is not true that a continuous}
            \textbf{bijection must have a continuous inverse}. The real line is
            special in this regard. This property comes from the fact that the
            real line has a complete total ordered (via the $<$ symbol). If
            $f:\mathbb{R}\rightarrow\mathbb{R}$ is a continuous bijection, it
            must be strictly increasing or strictly decreasing. If not, if
            there are $a<b<c$ with $f(a)<f(b)$ and $f(c)<f(b)$, or
            $f(b)<f(a)$ and $f(b)<f(c)$, then by the intermediate value theorem
            there must be values $x_{0}\in(a,\,b)$ and $x_{1}\in(b,\,c)$ such
            that $f(x_{0})=f(x_{1})$, violating the fact that $f$ is a
            bijection. Using this you can then show that $f^{-1}$ is also
            continuous.
        \end{example}
        \begin{example}
            Let $X=[0,\,1)$ and $Y=\mathbb{S}^{1}\subseteq\mathbb{R}^{2}$, the
            unit circle. Both of these are metric subspaces of the Euclidean
            spaces $\mathbb{R}$ and $\mathbb{R}^{2}$, respectively, meaning they
            are metric spaces in their own right, and hence topological spaces
            with the induced topology from the subspace metric. The function
            $f:[0,\,1)\rightarrow\mathbb{S}^{1}$ defined by
            $f(t)=\big(\cos(2\pi{t}),\,\sin(2\pi{t})\big)$ is a continuous
            bijection, but it is \textit{not} a homeomorphism. To go from the
            circle to the interval requires \textit{tearing} the circle at a
            point, and this operation is not continuous.
        \end{example}
        We can be more precice in proving that $[0,\,1)$ and $\mathbb{S}^{1}$
        do not have a homeomorphism between them. The idea of
        \textit{compactness} from metric spaces is a notion that is preserved
        by homeomorphisms.
        \begin{theorem}
            If $(X,\,d_{X})$ and $(Y,\,d_{Y})$ are metric spaces, if
            $f:X\rightarrow{Y}$ is a homeomorphism, and if $(X,\,d_{X})$ is
            compact, then $(Y,\,d_{Y})$ is compact.
        \end{theorem}
        \begin{proof}
            For let $b:\mathbb{N}\rightarrow{Y}$ be a sequence. Let
            $a:\mathbb{N}\rightarrow{X}$ be defined by
            $a_{n}=f^{-1}(b_{n})$ (this is well-defined since $f$ is a bijection
            and hence has an inverse). Since $(X,\,d_{X})$ is compact, there
            is a convergent subsequence $a_{k}$. Let $x\in{X}$ be the limit,
            $a_{k_{n}}\rightarrow{x}$. Then, since $f$ is continuous, we have
            $f(a_{k_{n}})\rightarrow{f}(x)$. But $f(a_{k_{n}})=b_{k_{n}}$,
            and hence $b_{k}$ is a convergent subsequence in $Y$, so
            $(Y,\,d_{Y})$ is compact.
        \end{proof}
        The circle $\mathbb{S}^{1}$ is compact by the Heine-Borel theorem since
        it is a closed and bounded subset of $\mathbb{R}^{2}$. The half-open
        interval $[0,\,1)$ is not compact, again by Heine-Borel, since it is
        not closed. Since homeomorphisms preserve compactness, there can be
        no homeomorphism between $[0,\,1)$ and $\mathbb{S}^{1}$.
        \par\hfill\par
        Homeomorphisms give a notion of \textit{equivalence} between topological
        spaces. There is no \textit{set of all topological spaces}, just like
        there is no set of all sets, so it is meaningless to say there is an
        \textit{equivalence relation on topological spaces}. Still, the
        following few theorems highlight what is meant by saying homeomorphisms
        tell us which spaces are equivalent.
        \begin{theorem}
            If $(X,\,\tau)$ is a topological space, then there is a
            homeomorphism $f:X\rightarrow{X}$.
        \end{theorem}
        \begin{proof}
            Define $f:X\rightarrow{Y}$ by $f(x)=x$. Then $f$ is bijective and
            continuous since $f^{-1}[\mathcal{V}]=\mathcal{V}$ for all
            $\mathcal{V}\in\tau$. The inverse is also continuous since
            $\big(f^{-1}\big)^{-1}[\mathcal{U}]=f[\mathcal{U}]=\mathcal{U}$ for
            all $\mathcal{U}\in\tau$, showing us that $f^{-1}$ is continuous.
            So $f$ is a homeomorphism.
        \end{proof}
        \begin{theorem}
            If $(X,\,\tau_{X})$ and $(Y,\,\tau_{Y})$ are topological spaces,
            and if $f:X\rightarrow{Y}$ is a homeomorphism, then there is a
            homeomorphism $g:Y\rightarrow{X}$.
        \end{theorem}
        \begin{proof}
            Define $g:Y\rightarrow{X}$ via $g(y)=f^{-1}(y)$. Since $f$ is a
            homeomorphism it is bijective, meaning $g$ is well-defined. But
            since $f$ is bijective, $f^{-1}$ is bijective, so $g$ is bijective.
            Since $f^{-1}$ is continuous, $g$ is continuous. Lastly, since
            $(f^{-1})^{-1}=f$, and $f$ is continuous, it is true that
            $g^{-1}$ is continuous. So $g:Y\rightarrow{X}$ is a homeomorphism.
        \end{proof}
        To prove \textit{transitivity}, we first need the following theorem.
        \begin{theorem}
            If $(X,\,\tau_{X})$, $(Y,\,\tau_{Y})$, and $(Z,\,\tau_{Z})$ are
            topological spaces, if $f:X\rightarrow{Y}$ is a continuous
            function, and if $g:Y\rightarrow{Z}$ is a continuous function,
            then $g\circ{f}:X\rightarrow{Z}$ is continuous.
        \end{theorem}
        \begin{proof}
            Let $\mathcal{W}\in\tau_{Z}$. Since $g$ is continuous,
            $g^{-1}[\mathcal{W}]\in\tau_{Y}$. But since $f$ is continuous,
            $f^{-1}\big[g^{-1}[\mathcal{W}]\big]\in\tau_{X}$. But
            $(g\circ{f})^{-1}[\mathcal{W}]=f^{-1}\big[g^{-1}[\mathcal{W}]\big]$,
            so $g\circ{f}$ is continuous.
        \end{proof}
        \begin{theorem}
            If $(X,\,\tau_{X})$, $(Y,\,\tau_{Y})$, and $(Z,\,\tau_{Z})$ are
            topological spaces, if $f:X\rightarrow{Y}$ is a homeomorphism, and
            if $g:Y\rightarrow{Z}$ is a homeomorphism, then
            $g\circ{f}:X\rightarrow{Z}$ is a homeomorphism.
        \end{theorem}
        \begin{proof}
            The composition of bijections is a bijection, so $g\circ{f}$ is
            bijective. The composition of continuous functions is continuous,
            so $g\circ{f}$ is continuous. Lastly,
            $(g\circ{f})^{-1}=f^{-1}\circ{g}^{-1}$, which is the composition of
            continuous functions since $f^{-1}$ and $g^{-1}$ are continuous,
            so $(g\circ{f})^{-1}$ is continuous. That is, $g\circ{f}$ is a
            homeomorphism.
        \end{proof}
        Homeomorphisms are just rebalellings of topological spaces.
        Given $(X,\,\tau_{X})$ and $(Y,\,\tau_{Y})$, and a homeomorphism
        $f:X\rightarrow{Y}$, given $x\in{X}$ we \textit{relabel} this as
        $f(x)=y\in{Y}$. Given $\mathcal{U}\in\tau_{X}$ we
        \textit{relabel} this as $f[\mathcal{U}]=\mathcal{V}\in\tau_{Y}$.
        The real line $\mathbb{R}$ and the imaginary line
        $i\mathbb{R}$, which is the set of all complex numbers of the form
        $iy$ with $y\in\mathbb{R}$, are topologically the same. They're just
        a line. We took a real number $r\in\mathbb{R}$ and relabelled it as
        $ir\in{i}\mathbb{R}$, but this doesn't really change anything.
        When we talk about the \textit{initial} and \textit{final} topologies
        in a few pages, this statement will be made clear.
        \clearpage
        Homeomorphisms preserve \textit{topological properties}.
        \begin{theorem}
            If $(X,\,\tau_{X})$ is a Hausdorff topological space, if
            $(Y,\,\tau_{Y})$ is a topological space, and if $f:X\rightarrow{Y}$
            is a homeomorphism, then $(Y,\,\tau_{Y})$ is Hausdorff.
        \end{theorem}
        \begin{proof}
            Let $y_{0},\,y_{1}\in{Y}$ with $y_{0}\ne{y}_{1}$. Since $f$ is a
            homeomorphism, it is bijective, so there exists unique
            $x_{0},x_{1}\in{X}$ such that $f(x_{0})=y_{0}$ and
            $f(x_{1})=y_{1}$. Since $y_{0}\ne{y}_{1}$, we have
            $x_{0}\ne{x}_{1}$. But $(X,\,\tau_{X})$ is Hausdorff, so there
            exists $\mathcal{U},\mathcal{V}\in\tau_{X}$ such that
            $x\in\mathcal{U}$, $y\in\mathcal{V}$, and
            $\mathcal{U}\cap\mathcal{V}=\emptyset$. Let
            $\tilde{\mathcal{U}}=f[\mathcal{U}]$ and
            $\tilde{\mathcal{V}}=f[\mathcal{V}]$. Then, since $f$ is a
            homeomorphism, it is bijective, and hence
            $\tilde{\mathcal{U}}=(f^{-1})^{-1}[\mathcal{U}]$ and
            $\tilde{\mathcal{V}}=(f^{-1})^{-1}[\mathcal{V}]$, the pre-image of
            open sets by a continuous function since $f^{-1}$ is continuous,
            and hence $\tilde{\mathcal{U}},\tilde{\mathcal{V}}\in\tau_{Y}$. But
            $y_{0}\in\tilde{\mathcal{U}}$, $y_{1}\in\tilde{\mathcal{V}}$, and
            $\tilde{\mathcal{U}}\cap\tilde{\mathcal{V}}=\emptyset$ since:
            \begin{align}
                \tilde{\mathcal{U}}\cap\tilde{\mathcal{V}}
                &=f[\mathcal{U}]\cap{f}[\mathcal{V}]\tag{Substitution}\\
                &=f[\mathcal{U}\cap\mathcal{V}]\tag{Since $f$ is bijective}\\
                &=f[\emptyset]
                    \tag{Since $\mathcal{U}$ and $\mathcal{V}$ are disjoint}\\
                &=\emptyset\tag{The image of the empty set is empty}
            \end{align}
            Hence $(Y,\,\tau_{Y})$ is Hausdorff.
        \end{proof}
        If the target space $(Y,\,\tau_{Y})$ is Hausdorff, and if
        $f:X\rightarrow{Y}$ is continuous and injective, you can then prove that
        $(X,\,\tau_{X})$ is Hausdorff as well. You do not need $f$ to be a
        homeomorphism in this direction.
        \begin{theorem}
            If $(X,\,\tau_{X})$ is a topological space, if $(Y,\,\tau_{Y})$ is a
            Hausdorff topological space, and if $f:X\rightarrow{Y}$ is a
            continuous injective function, then $(X,\,\tau_{X})$ is Hausdorff. 
        \end{theorem}
        \begin{proof}
            Let $x_{0},x_{1}\in{X}$, $x_{0}\ne{x}_{1}$. Let
            $y_{0}=f(x_{0})$ and $y_{1}=f(x_{1})$. Then since $f$ is injective,
            $y_{0}\ne{y}_{1}$. But $(Y,\,\tau_{Y})$ is Hausdorff, so there
            exists $\mathcal{U},\mathcal{V}\in\tau_{Y}$ such that
            $y_{0}\in\mathcal{U}$, $y_{1}\in\mathcal{V}$, and
            $\mathcal{U}\cap\mathcal{V}=\emptyset$. But $f$ is continuous,
            so $f^{-1}[\mathcal{U}]\in\tau_{X}$ and
            $f^{-1}[\mathcal{V}]\in\tau_{Y}$. But then
            $x\in{f}^{-1}[\mathcal{U}]$, $y\in{f}^{-1}[\mathcal{V}]$, and:
            \begin{equation}
                f^{-1}[\mathcal{U}]\cap{f}^{-1}[\mathcal{V}]
                =f^{-1}[\mathcal{U}\cap\mathcal{V}]
                =f^{-1}[\emptyset]
                =\emptyset
            \end{equation}
            so $(X,\,\tau_{X})$ is Hausdorff.
        \end{proof}
        Many of the theorems about homeomorphisms use the fact that since
        $f$ is bijective, $(f^{-1})^{-1}[\mathcal{U}]=\mathcal{U}$. So in
        particular, if $\mathcal{U}\subseteq{X}$ is open, then
        $f[\mathcal{U}]\subseteq{Y}$ is open. Functions with this property
        are called \textit{open mappings}.
        \begin{definition}[\textbf{Open Mapping}]
            An open mapping from a topological space $(X,\,\tau_{Y})$ to a
            topological space $(Y,\,\tau_{Y})$ is a function
            $f:X\rightarrow{Y}$ such that for all $\mathcal{U}\in\tau_{X}$ it is
            true that $f[\mathcal{U}]\in\tau_{Y}$.
        \end{definition}
        Open mappings do not need to be continuous and continuous functions do
        not need to be open mappings.
        \begin{example}
            Let $f:\mathbb{R}\rightarrow\mathbb{R}$ be defined by
            $f(x)=x^{2}$, with the standard Euclidean topology on $\mathbb{R}$.
            This is continuous, since it is a polynomial, but not an open
            mapping. The forward image of $(-1,\,1)$ is
            $f\big[(-1,\,1)\big]=[0,\,1)$, which is not open.
        \end{example}
        \begin{example}
            Let $X=Y=\mathbb{R}$, $\tau_{X}=\{\,\emptyset,\,\mathbb{R}\,\}$, and
            $\tau_{Y}=\mathcal{P}(\mathbb{R})$, and
            $f:\mathbb{R}\rightarrow\mathbb{R}$ be defined by $f(x)=x$. Then
            $f$ is \textit{not} continuous. The pre-image of $\{\,0\,\}$ is
            $\{\,0\,\}$. $\{\,0\,\}$ is an element of $\mathcal{P}(\mathbb{R})$,
            but not $\{\,\emptyset,\,\mathbb{R}\,\}$, so $f$ is not continuous.
            $f$ is an open mapping. There are only two elements of $\tau_{X}$
            to check. We have $f[\emptyset]=\emptyset$ and
            $f[\mathbb{R}]=\mathbb{R}$, both of which are elements of
            $\tau_{Y}$, so $f$ is an open mapping.
        \end{example}
        \begin{theorem}
            If $(X,\,\tau_{X})$, $(Y,\,\tau_{Y})$, and $(Z,\,\tau_{Z})$ are
            topological spaces, if $f:X\rightarrow{Y}$ and $g:Y\rightarrow{Z}$
            are open mappings, then $g\circ{f}$ is an open mapping.
        \end{theorem}
        \begin{proof}
            Given $\mathcal{U}\in\tau_{X}$, we have:
            \begin{equation}
                \big(g\circ{f}\big)[\mathcal{U}]
                =g\big[f[\mathcal{U}]\big]
            \end{equation}
            but since $f$ is an open mapping, $f[\mathcal{U}]\in\tau_{Y}$. But
            since $g$ is an open mapping,
            $g\big[f[\mathcal{U}]\big]\in\tau_{Z}$. Therefore $g\circ{f}$ is an
            open mapping.
        \end{proof}
        \begin{theorem}
            If $(X,\,\tau_{X})$ and $(Y,\,\tau_{Y})$ are topological spaces, and
            if $f:X\rightarrow{Y}$ is a function, then $f$ is a homeomorphism
            if and only if $f$ is continuous, bijective, and an open mapping.
        \end{theorem}
        \begin{proof}
            If $f$ is a homeomorphism it is continuous and bijective. It is
            also an open mapping since if $\mathcal{U}\in\tau_{X}$, then:
            \begin{equation}
                f[\mathcal{U}]
                =(f^{-1})^{-1}[\mathcal{U}]
            \end{equation}
            but $f^{-1}$ is continuous since $f$ is a homeomorphism, so
            $f[\mathcal{U}]$ is the pre-image of an open set under a continuous
            function and is therefore open. That is, $f$ is an open mapping.
            Now suppose $f$ is a continuous bijective open mapping. Let
            $\mathcal{U}\in\tau_{X}$. Then:
            \begin{equation}
                (f^{-1})^{-1}[\mathcal{U}]=f[\mathcal{U}]
            \end{equation}
            But $\mathcal{U}$ is an open mapping, so
            $f[\mathcal{U}]\in\tau_{Y}$. Therefore $f^{-1}$ is continuous and
            $f$ is a homeomorphism.
        \end{proof}
        Homeomorphisms preserve the notion of \textit{sequential} as well.
        First, the following theorem about sequentially open sets and continuous
        functions.
        \begin{theorem}
            If $(X,\,\tau_{X})$ and $(Y,\,\tau_{Y})$ are topological spaces,
            if $f:X\rightarrow{Y}$ is continuous, and if
            $\mathcal{V}\subseteq{Y}$ is sequentially open, then
            $f^{-1}[\mathcal{V}]$ is sequentially open.
        \end{theorem}
        \begin{proof}
            Suppose not. Then there is an $x\in{f}^{-1}[\mathcal{V}]$ and a
            sequence $a:\mathbb{N}\rightarrow{X}$ such that
            $a_{n}\rightarrow{x}$ and for all $N\in\mathbb{N}$ there is an
            $n\in\mathbb{N}$ such that $n>N$ and
            $a_{n}\notin{f}^{-1}[\mathcal{V}]$. But $f$ is continuous, so it
            is sequentially continuous, and therefore
            $f(a_{n})\rightarrow{f}(x)$. But since $x\in{f}^{-1}[\mathcal{V}]$
            we have $f(x)\in\mathcal{V}$. But $\mathcal{V}$ is sequentially
            open, so if $f(a_{n})\rightarrow{f}(x)$, then there is an
            $N\in\mathbb{N}$ such that for all $n\in\mathbb{N}$ with $n>N$
            we have $f(a_{n})\in\mathcal{V}$. But then for all $n>N$,
            $a_{n}\in{f}^{-1}[\mathcal{V}]$, a contradiction. So
            $f^{-1}[\mathcal{V}]$ is sequentially open.
        \end{proof}
        \begin{theorem}
            If $(X,\,\tau_{X})$ is a sequential topological space, if
            $(Y,\,\tau_{Y})$ is a topological space, and if $f:X\rightarrow{Y}$
            is a homeomorphism, then $(Y,\,\tau_{Y})$ is sequential.
        \end{theorem}
        \begin{proof}
            Let $\mathcal{V}\subseteq{Y}$ be sequentially open. Since $f$ is
            a homeomorphism, $f$ is continuous, so $f^{-1}[\mathcal{V}]$
            is sequentially open. But $(X,\,\tau_{X})$ is sequential, so
            if $f^{-1}[\mathcal{V}]$ is sequentially open, then it is open.
            But since $f$ is a homeomorphism, it is an open mapping, meaning
            $f\big[f^{-1}[\mathcal{V}]\big]\in\tau_{Y}$. But
            $f$ is bijective, so $f\big[f^{-1}[\mathcal{V}]\big]=\mathcal{V}$.
            That is, $\mathcal{V}$ is open, and $(Y,\,\tau_{Y})$ is
            sequential.
        \end{proof}
        \begin{theorem}
            If $(X,\,\tau_{X})$ is a second-countable topological space, if
            $(Y,\,\tau_{Y})$ is a topological space, and if $f:X\rightarrow{Y}$
            is a homeomorphism, then $(Y,\,\tau_{Y})$ is second-countable.
        \end{theorem}
        \begin{proof}
            Since $(X,\,\tau_{X})$ is second-countable, there is a
            countable basis $\mathcal{B}$. Define
            $\tilde{\mathcal{B}}$ by:
            \begin{equation}
                \tilde{\mathcal{B}}=
                \{\,f[\mathcal{U}]\;|\;\mathcal{U}\in\mathcal{B}\,\}
            \end{equation}
            But $f$ is a homeomorphism, so it is an open mapping, meaning for
            all $\mathcal{U}\in\tau_{X}$, $f[\mathcal{U}]$ is open, so
            $\tilde{\mathcal{B}}\subseteq\tau_{Y}$. Moreover since $\mathcal{B}$
            is countable, so is $\tilde{\mathcal{B}}$. Let's show
            $\tilde{\mathcal{B}}$ is a basis.
            Given an open set $\mathcal{V}\in\tau_{Y}$, let
            $\mathcal{U}=f^{-1}[\mathcal{V}]$. Then $\mathcal{U}\in\tau_{X}$
            since $f$ is continuous, and since $\mathcal{B}$ is a basis there
            is some $\mathcal{O}\subseteq\mathcal{B}$ such that
            $\bigcup\mathcal{O}=\mathcal{U}$. Define $\tilde{\mathcal{O}}$ via:
            \begin{equation}
                \tilde{\mathcal{O}}
                =\{\,f[\mathcal{W}]\;|\;\mathcal{W}\in\mathcal{O}\,\}
            \end{equation}
            Then $\tilde{\mathcal{O}}\subseteq\tilde{\mathcal{B}}$, by the
            definition of $\tilde{\mathcal{B}}$, and since $f$ is a bijection
            we may conclude that
            $\bigcup\tilde{\mathcal{O}}=f[\mathcal{U}]=\mathcal{V}$. Hence
            $\tilde{\mathcal{B}}$ is a countable basis for $\tau_{Y}$.
        \end{proof}
        Similar to open mappings, one often studies \textit{closed mappings}.
        Closed mappings arise quite frequently in functional analysis and
        geometry.
        \begin{definition}[\textbf{Closed Mappings}]
            A closed mapping from a topological space $(X,\,\tau_{X})$ to a
            topological space $(Y,\,\tau_{Y})$ is a function
            $f:X\rightarrow{Y}$ such that for every closed
            $\mathcal{C}\subseteq{X}$ it is true that
            $f[\mathcal{C}]\subseteq{Y}$ is closed.
        \end{definition}
        \begin{example}
            The function $f:\mathbb{R}\rightarrow\mathbb{R}$ defined by
            $f(x)=x^{2}$ is closed but not open.
        \end{example}
        \begin{example}
            Give $X=Y=\mathbb{R}$, $\tau_{X}=\{\,\emptyset,\,\mathbb{R}\,\}$,
            and $\tau_{Y}=\mathcal{P}(\mathbb{R})$, the identity function
            $f(x)=x$ is closed (and open) but \textit{not} continuous.
        \end{example}
        Open mappings, closed mappings, and continuous functions
        are three distinct notions.
        \begin{theorem}
            If $(X,\,\tau_{X})$ and $(Y,\,\tau_{Y})$ are topological spaces,
            and if $f:X\rightarrow{Y}$ is a bijective open mapping, then it is
            a closed mapping.
        \end{theorem}
        \begin{proof}
            Let $\mathcal{C}\subseteq{X}$ be closed. Then
            $X\setminus\mathcal{C}$ is open. But then:
            \begin{equation}
                f[\mathcal{C}]=
                f[X\setminus(X\setminus\mathcal{C})]
            \end{equation}
            But $f$ is bijective, so:
            \begin{equation}
                f[X\setminus(X\setminus\mathcal{C})]
                =f[X]\setminus{f}[X\setminus\mathcal{C}]
            \end{equation}
            But $f$ is an open mapping, so $f[X\setminus\mathcal{C}]$ is open.
            But $f$ is bijective, so $f[X]=Y$, and therefore
            $f[X]\setminus{f}[X\setminus\mathcal{C}]$ is the complement of an
            open set, which is therefore closed. Thus, $f$ is a closed mapping.
        \end{proof}
        Without bijectivity, this statement fails. The function
        $f:\mathbb{R}\rightarrow\mathbb{R}$ defined by $f(x)=x^{2}$ is a
        continuous closed mapping that is \textit{not} an open mapping. The
        reason being that $f$ is not bijective.
        \par\hfill\par
        Homeomorphisms can also be described via closed mappings.
        \begin{theorem}
            If $(X,\,\tau_{X})$ and $(Y,\,\tau_{Y})$ are topological spaces,
            and if $f:X\rightarrow{Y}$ is a function, then $f$ is a
            homeomorphism if and only if $f$ is continuous, bijective, and a
            closed mapping.
        \end{theorem}
        \begin{proof}
            $f$ is a homeomorphism if and only if $f$ is continuous, bijective,
            and open. If $f$ is bijective, then $f$ is open if and only if
            $f$ is closed, so $f$ is a homeomorphism if and only if $f$ is
            continuous, bijective, and a closed mapping. 
        \end{proof}
    \section{Subspaces}
        In the study of metric spaces, given such a space $(X,\,d)$ and a subset
        $A\subseteq{X}$, we could restrict the metric
        $d:X\times{X}\rightarrow\mathbb{R}$ to
        $d_{A}:A\times{A}\rightarrow\mathbb{R}$, making $(A,\,d_{A})$ a metric
        space. This gave us a metric topology on $A$, and hence allowed us to
        think of $A$ as a topological space, with the topology stemming from
        the metric topology on $A$. We then proved that a subset $\mathcal{U}$
        of $A$ is open with respect to this subspace topology if and only if
        there is an open subset $\mathcal{V}\subseteq{X}$
        (that is open with respect to the metric topology on $X$) such that
        $\mathcal{U}=A\cap\mathcal{V}$. In the general topological setting we
        lack a metric, but this theorem allows us to define subspaces solely
        in terms of open sets.
        \begin{definition}[\textbf{Subspace Topology}]
            The subspace topology of a subset $A\subseteq{X}$ with respect to
            a topological space $(X,\,\tau)$ is the set $\tau_{A}$ defined by:
            \begin{equation}
                \tau_{A}=\{\,\mathcal{U}\subseteq{A}\;|\;\textrm{there exists }
                    \mathcal{V}\in\tau\textrm{ such that }
                    \mathcal{U}=A\cap\mathcal{V}\,\}
            \end{equation}
        \end{definition}
        I am calling this a topology, but \textit{proof by definition} is
        generally a bad practice. Let's prove the subspace topology is indeed
        a topology on $A$.
        \begin{theorem}
            If $(X,\,\tau)$ is a topological space, if $A\subseteq{X}$, and if
            $\tau_{A}$ is the subspace topology on $A$, then $\tau_{A}$ is a
            topology on $A$.
        \end{theorem}
        \begin{proof}
            We must prove the four properties of a topology. First,
            $\emptyset\in\tau_{A}$ since $\emptyset\in\tau$ and
            $\emptyset=\emptyset\cap{A}$. Next, $A\in\tau_{A}$ since
            $X\in\tau$ and since $A\subseteq{X}$ we have $A=A\cap{X}$.
            If $\mathcal{U},\mathcal{V}\in\tau_{A}$ then there exists
            $\tilde{\mathcal{U}},\tilde{\mathcal{V}}\in\tau$ such that
            $\mathcal{U}=A\cap\tilde{\mathcal{U}}$ and
            $\mathcal{V}=A\cap\tilde{\mathcal{V}}$. But then:
            \begin{equation}
                \mathcal{U}\cap\mathcal{V}
                =\big(A\cap\tilde{\mathcal{U}}\big)\cap
                \big(A\cap\tilde{\mathcal{V}}\big)
                =A\cap\big(\tilde{\mathcal{U}}\cap\tilde{\mathcal{V}}\big)
            \end{equation}
            but $\tau$ is a topology, so if
            $\tilde{\mathcal{U}},\tilde{\mathcal{V}}\in\tau$, then
            $\tilde{\mathcal{U}}\cap\tilde{\mathcal{V}}\in\tau$. Therefore,
            $\mathcal{U}\cap\mathcal{V}\in\tau_{A}$. Lastly, if
            $\mathcal{O}\subseteq\tau_{A}$, then for all
            $\mathcal{U}\in\mathcal{O}$ there is a
            $\tilde{\mathcal{U}}\in\tau$ such that
            $\mathcal{U}=A\cap\tilde{\mathcal{U}}$. Let $\tilde{\mathcal{O}}$
            be the collection of all such $\tilde{\mathcal{U}}$ for all
            $\mathcal{U}\in\mathcal{O}$. Then:
            \begin{align}
                \bigcup\mathcal{O}
                &=\bigcup_{\mathcal{U}\in\mathcal{O}}\mathcal{U}\\
                &=\bigcup_{\tilde{\mathcal{U}}\in\tilde{\mathcal{O}}}
                    \big(A\cap\tilde{\mathcal{U}}\big)\\
                &=A\cap\bigcup_{\tilde{\mathcal{U}}\in\tilde{\mathcal{O}}}
                    \tilde{\mathcal{U}}\\
                &=A\cap\bigcup\tilde{\mathcal{O}}
            \end{align}
            But $\tilde{\mathcal{O}}\subseteq\tau$ and $\tau$ is a topology,
            so $\bigcup\tilde{\mathcal{O}}\in\tau$, and hence
            $\bigcup\mathcal{O}\in\tau_{A}$. That is, $\tau_{A}$ is a
            topology on $A$.
        \end{proof}
        \begin{example}
            The familiar examples of topological subspaces are just metric
            subspaces. The circle $\mathbb{S}^{1}$ is a subspace of
            $\mathbb{R}^{2}$. The closed interval $[a,\,b]\subseteq\mathbb{R}$
            and the open interval $(a,\,b)\subseteq\mathbb{R}$ are
            subspaces of $\mathbb{R}$ with their respective subspace topologies.
            The unit sphere $\mathbb{S}^{2}$ lives in $\mathbb{R}^{3}$ as a
            subspace as well.
        \end{example}
        \begin{definition}[\textbf{Inclusion Map}]
            The inclusion map of a subset $A\subseteq{X}$ into the set $X$
            is the function $\iota:A\rightarrow{X}$ defined by $\iota(x)=x$.
        \end{definition}
        \begin{theorem}
            If $(X,\,\tau)$ is a topological space, if $A\subseteq{X}$, if
            $\tau_{A}$ is the subspace topology on $A$, and if
            $\iota:A\rightarrow{X}$ is the inclusion map, then $\iota$ is
            continuous.
        \end{theorem}
        \begin{proof}
            Let $\mathcal{V}\in\tau$. Then by the definition of pre-image,
            $\iota^{-1}[\mathcal{V}]$ is the set of all $x\in{A}$ such that
            $\iota(x)\in\mathcal{V}$. But $\iota(x)=x$, so
            $\iota^{-1}[\mathcal{V}]$ is the set of all $x\in{A}$ such that
            $x\in\mathcal{V}$. That is, the set of all elements in
            $\mathcal{V}\cap{A}$. But $\mathcal{V}\cap{A}$ is open in $A$ since
            $\mathcal{V}$ is open in $X$. Thus, $\iota$ is continuous.
        \end{proof}
        \begin{theorem}
            If $(X,\,\tau)$ is a topological space, and if $\tau_{A}'$ is a
            topology on $A$ such that the inclusion map $\iota:A\rightarrow{X}$
            is continuous, then $\tau_{A}\subseteq\tau_{A}'$ where
            $\tau_{A}$ is the subspace topology.
        \end{theorem}
        \begin{proof}
            Let $\mathcal{U}\in\tau_{A}$. Since $\tau_{A}$ is the subspace
            topology, there is a $\mathcal{V}\in\tau$ such that
            $\mathcal{U}=A\cap\mathcal{V}$. But then
            $\iota^{-1}[\mathcal{V}]=\mathcal{U}$. Since $\iota$ is continuous
            with respect to $\tau_{A}'$ and $\tau$ it must be true that
            $\mathcal{U}\in\tau_{A}'$. Hence, $\tau_{A}\subseteq\tau_{A}'$.
        \end{proof}
        The subspace topology is the \textit{smallest} topology that makes
        the inclusion map continuous. This is a way of defining the subspace
        topology altogether, the intersection of all topologies on $A$ that
        make $\iota$ continuous.
        \par\hfill\par
        This idea of defining a topology in terms of a function that you want
        to be continuous is common. There are two directions. If we have a
        topological space $(X,\,\tau)$, a set $Y$, and a function
        $f:X\rightarrow{Y}$, then the \textit{final topology} on $Y$ with
        respect to $f$ is the largest topology $\tau_{f}$ that makes $f$
        continuous. If $X$ is a set, $(Y,\,\tau)$ is a topological space, and
        $f:X\rightarrow{Y}$ is a function, then the \textit{initial topology}
        is the smallest topology $\tau_{f}$ on $X$ that makes $f$ continuous.
        \begin{definition}[\textbf{Final Topology}]
            The final topology on a set $Y$ with respect to a topological
            space $(X,\,\tau)$ and a function $f:X\rightarrow{Y}$ is the
            set $\tau_{f}$ defined by:
            \begin{equation}
                \tau_{f}=
                \{\,\mathcal{V}\subseteq{Y}\;|\;f^{-1}[\mathcal{V}]\in\tau\,\}
            \end{equation}
            That is, the set of all subsets of $Y$ whose pre-image is open in
            $X$.
        \end{definition}
        Again, avoiding proof by definition, the final topology is a topology.
        \begin{theorem}
            If $(X,\,\tau)$ is a topological space, if $Y$ is a set, if
            $f:X\rightarrow{Y}$ is a function, and if $\tau_{f}$ is the final
            topology, then $\tau_{f}$ is a topology on $Y$.
        \end{theorem}
        \begin{proof}
            First, $\emptyset\in\tau_{f}$ since $f^{-1}[\emptyset]=\emptyset$
            and $\emptyset\in\tau$. Next, $Y\in\tau_{f}$ since $f^{-1}[Y]=X$
            and $X\in\tau$, since $\tau$ is a topology. Suppose
            $\mathcal{U},\mathcal{V}\in\tau_{f}$. Then:
            \begin{equation}
                f^{-1}[\mathcal{U}\cap\mathcal{V}]
                =f^{-1}[\mathcal{U}]\cap{f}^{-1}[\mathcal{V}]
            \end{equation}
            But if $\mathcal{U},\mathcal{V}\in\tau_{f}$, then
            $f^{-1}[\mathcal{U}]\in\tau$ and $f^{-1}[\mathcal{V}]\in\tau$. But
            $\tau$ is a topology on $X$, so
            $f^{-1}[\mathcal{U}]\cap{f}^{-1}[\mathcal{V}]\in\tau$. Hence,
            $\mathcal{U}\cap\mathcal{V}\in\tau_{f}$. Lastly, let
            $\mathcal{O}\subseteq\tau_{f}$. Then for all
            $\mathcal{U}\in\mathcal{O}$ it is true that
            $f^{-1}[\mathcal{U}]\in\tau$. But then:
            \begin{align}
                f^{-1}\Big[\bigcup\mathcal{O}\Big]
                &=f^{-1}\Big[
                    \bigcup_{\mathcal{U}\in\mathcal{O}}\mathcal{U}
                \Big]\\
                &=\bigcup_{\mathcal{U}\in\mathcal{O}}f^{-1}[\mathcal{U}]
            \end{align}
            But $\tau$ is a topology, so this final union is an element of
            $\tau$. Hence, $\bigcup\mathcal{O}\in\tau_{f}$ and $\tau_{f}$ is
            a topology.
        \end{proof}
        \begin{theorem}
            If $(X,\,\tau)$ is a topological space, if $Y$ is a set, and if
            $f:X\rightarrow{Y}$ is a function, then $f$ is continuous with
            respect to $\tau$ and the final topology $\tau_{f}$.
        \end{theorem}
        \begin{proof}
            Given $\mathcal{V}\in\tau_{f}$, by definition of the final topology
            we have $f^{-1}[\mathcal{V}]\in\tau$, so $f$ is continuous.
        \end{proof}
        \begin{theorem}
            If $(X,\,\tau)$ is a topological space, if $Y$ is a set, if
            $f:X\rightarrow{Y}$ is a function, if $\tau_{f}$ is the final
            topology, and if $\tau_{Y}$ is a topology such that
            $f$ is continuous with respect to $\tau$ and $\tau_{Y}$, then
            $\tau_{Y}\subseteq\tau_{f}$.
        \end{theorem}
        \begin{proof}
            For let $\mathcal{V}\in\tau_{Y}$. Since $f$ is continuous with
            respect to $\tau$ and $\tau_{Y}$ it is true that
            $f^{-1}[\mathcal{V}]\in\tau$. But $\tau_{f}$ is the final topology
            which is the set of all such $\mathcal{U}\subseteq{Y}$ such that
            $f^{-1}[\mathcal{U}]\in\tau$, and therefore
            $\mathcal{V}\in\tau_{f}$. That is, $\tau_{Y}\subseteq\tau_{f}$.
        \end{proof}
        This is what was meant by the claim that the final topology is the
        \textit{largest} topology that makes $f$ continuous.
        \begin{definition}[\textbf{Initial Topology}]
            The initial topology on a set $X$ with respect to a topological
            space $(Y,\,\tau)$ and a function $f:X\rightarrow{Y}$ is the
            set $\tau_{f}$ defined by:
            \begin{equation}
                \tau_{f}=\{\,f^{-1}[\mathcal{V}]\;|\;\mathcal{V}\in\tau\,\}
            \end{equation}
            That is, the set of all pre-images of open subsets of $Y$.
        \end{definition}
        The initial topology is, in fact, a topology.
        \begin{theorem}
            If $X$ is a set, if $(Y,\,\tau)$ is a topological space, and if
            $f:X\rightarrow{Y}$ is a function, then the initial topology
            $\tau_{f}$ is a topology on $X$.
        \end{theorem}
        \begin{proof}
            We have that $\emptyset=f^{-1}[\emptyset]$, so
            $\emptyset\in\tau_{f}$. We also have $X=f^{-1}[Y]$, so
            $X\in\tau_{f}$. If $\mathcal{U},\mathcal{V}\in\tau_{f}$, then
            there are $\tilde{\mathcal{U}},\tilde{\mathcal{V}}\in\tau$ such
            that $\mathcal{U}=f^{-1}[\tilde{\mathcal{U}}]$ and
            $\mathcal{V}=f^{-1}[\tilde{\mathcal{V}}]$. But then:
            \begin{equation}
                \mathcal{U}\cap\mathcal{V}
                =f^{-1}[\tilde{\mathcal{U}}]\cap{f}^{-1}[\tilde{\mathcal{V}}]
                =f^{-1}[\tilde{\mathcal{U}}\cap\tilde{\mathcal{V}}]
            \end{equation}
            But $\tau$ is a topology, so
            $\tilde{\mathcal{U}}\cap\tilde{\mathcal{V}}\in\tau$, and hence
            $\mathcal{U}\cap\mathcal{V}\in\tau_{f}$. Lastly, if
            $\mathcal{O}\subseteq\tau_{f}$, then for all
            $\mathcal{U}\in\mathcal{O}$ there is a $\tilde{\mathcal{U}}\in\tau$
            such that $\mathcal{U}=f^{-1}[\tilde{\mathcal{U}}]$. Let
            $\tilde{\mathcal{O}}\subseteq\tau$ be the set of all such
            $\tilde{\mathcal{U}}$. Then:
            \begin{align}
                \bigcup\mathcal{O}
                &=\bigcup_{\mathcal{U}\in\mathcal{O}}\mathcal{U}\\
                &=\bigcup_{\tilde{\mathcal{U}}\in\tilde{\mathcal{O}}}
                    f^{-1}[\tilde{\mathcal{U}}]\\
                &=f^{-1}\Big[
                    \bigcup_{\tilde{\mathcal{U}}\in\tilde{\mathcal{O}}}
                    \tilde{\mathcal{U}}
                \Big]\\
                &=f^{-1}\Big[
                    \bigcup\tilde{\mathcal{O}}
                \Big]
            \end{align}
            But $\tau$ is a topology so $\bigcup\tilde{\mathcal{O}}\in\tau$
            and therefore $\bigcup\mathcal{O}\in\tau_{f}$. So $\tau_{f}$
            is a topology.
        \end{proof}
        \begin{theorem}
            If $X$ is a set, if $(Y,\,\tau)$ is a topological space, if
            $f:X\rightarrow{Y}$ is a function, and if $\tau_{f}$ is the initial
            topology, then $f$ is continuous.
        \end{theorem}
        \begin{proof}
            Let $\mathcal{V}\in\tau$. By the definition of the initial topology,
            $f^{-1}[\mathcal{V}]\in\tau_{f}$, so $f$ is continuous.
        \end{proof}
        \begin{theorem}
            If $X$ is a set, if $(Y,\,\tau)$ is a topological space, if
            $f:X\rightarrow{Y}$ is a function, if $\tau_{f}$ is the initial
            topology with respect of $f$ and $(Y,\,\tau)$, and if
            $\tau_{X}$ is a topology such that $f$ is continuous with respect
            to $\tau_{X}$ and $\tau$, then $\tau_{f}\subseteq\tau_{X}$.
        \end{theorem}
        \begin{proof}
            Let $\mathcal{U}\in\tau_{f}$. Then by the definition of the initial
            topology, there is a $\tilde{\mathcal{U}}\in\tau$ such that
            $f^{-1}[\tilde{\mathcal{U}}]=\mathcal{U}$. But $f$ is continuous
            with respect to $\tau_{X}$ and $\tau$, so if
            $\tilde{\mathcal{U}}\in\tau$, then
            $f^{-1}[\tilde{\mathcal{U}}]\in\tau_{X}$. That is,
            $\mathcal{U}\in\tau_{X}$ and therefore $\tau_{f}\subseteq\tau_{X}$.
        \end{proof}
        The intial topology is therefore the \textit{smallest} topology that
        makes $f$ continuous.
        \begin{theorem}
            If $(X,\,\tau)$ is a topological space, if $A\subseteq{X}$, if
            $\iota:A\rightarrow{X}$ is the inclusion mapping, and if
            $\tau_{\iota}$ is the initial topology with respect to
            $(X,\,\tau)$ and $\iota$, then $\tau_{\iota}=\tau_{A}$ where
            $\tau_{A}$ is the subspace topology.
        \end{theorem}
        \begin{proof}
            $\tau_{A}$ is a topology that makes $\iota$ continuous, and hence
            by the previous theorem, $\tau_{\iota}\subseteq\tau_{A}$. Suppose
            $\mathcal{U}\in\tau_{A}$. Then there is an open set
            $\mathcal{V}\in\tau$ such that
            $\mathcal{U}=\mathcal{V}\cap{A}$. By the definition of the
            inclusion mapping, $\iota^{-1}[\mathcal{V}]=\mathcal{V}\cap{A}$,
            so $\iota^{-1}[\mathcal{V}]=\mathcal{U}$ and therefore
            $\mathcal{U}\in\tau_{\iota}$. That is,
            $\tau_{A}\subseteq\tau_{\iota}$. Therefore, $\tau_{A}=\tau_{\iota}$.
        \end{proof}
        A few pages ago it was stated that homeomorphisms are just relabellings
        of spaces. With the initial and final topology we can make this
        precise.
        \begin{theorem}
            If $(X,\,\tau)$ is a topological space, if $Y$ is a set, and if
            $f:X\rightarrow{Y}$ is a bijection, then there is a unique
            topology $\tau_{Y}$ on $Y$ such that $f$ is a homeomorphism.
        \end{theorem}
        \begin{proof}
            Let $\tau_{Y}$ be the final topology from $f$. This makes $f$
            continuous. Given any topology $\tau_{Y}'$ that makes $f$
            continuous, since $\tau_{Y}$ is the final topology, we have
            $\tau_{Y}'\subseteq\tau_{Y}$. But then $\tau_{Y}$ is also the
            initial topology with respect to $f^{-1}$, making $f^{-1}$
            continuous. Then given any topology $\tau_{Y}''$ that makes
            $f^{-1}$ continuous, we have $\tau_{Y}\subseteq\tau_{Y}''$. Hence
            any topology $\tau_{Y}'''$ that makes $f$ and $f^{-1}$ continuous
            must have $\tau_{Y}\subseteq\tau_{Y}'''$ and
            $\tau_{Y}'''\subseteq\tau_{Y}$, so $\tau_{Y}=\tau_{Y}'''$. That is,
            $\tau_{Y}$ is the unique topology that makes $f$ a homeomorphism.
        \end{proof}
        The subspace topology preserves many (but not all) properties of the
        ambient space.
        \begin{theorem}
            If $(X,\,\tau)$ is a Hausdorff topological space, if
            $A\subseteq{X}$, and if $\tau_{A}$ is the subspace topology, then
            $(A,\,\tau_{A})$ is a Hausdorff topological space.
        \end{theorem}
        \begin{proof}
            Let $x,y\in{A}$ with $x\ne{y}$. Since $A\subseteq{X}$ we have
            $x,y\in{X}$. But $x\ne{y}$ and $(X,\,\tau)$ is Hausdorff, so there
            exist $\mathcal{U},\mathcal{V}\in\tau$ such that $x\in\mathcal{U}$,
            $y\in\mathcal{V}$, and $\mathcal{U}\cap\mathcal{V}=\emptyset$.
            Let $\tilde{\mathcal{U}}=A\cap\mathcal{U}$ and
            $\tilde{\mathcal{V}}=A\cap\mathcal{V}$. Since $x\in{A}$ and
            $x\in\mathcal{U}$, we have $x\in{A}\cap\mathcal{U}$. Since
            $y\in{A}$ and $y\in\mathcal{V}$, it is also true that
            $y\in{A}\cap\mathcal{V}$. So $x\in\tilde{\mathcal{U}}$ and
            $y\in\tilde{\mathcal{V}}$. But also:
            \begin{equation}
                \tilde{\mathcal{U}}\cap\tilde{\mathcal{V}}
                =\big(A\cap\mathcal{U}\big)\cap\big(A\cap\mathcal{V}\big)
                =A\cap\big(\mathcal{U}\cap\mathcal{V}\big)
                =A\cap\emptyset
                =\emptyset
            \end{equation}
            so $\tilde{\mathcal{U}}$ and $\tilde{\mathcal{V}}$ are open sets
            such that $x\in\tilde{\mathcal{U}}$, $y\in\tilde{\mathcal{V}}$, and
            $\tilde{\mathcal{U}}\cap\tilde{\mathcal{V}}=\emptyset$. That is,
            $(A,\,\tau_{A})$ is Hausdorff.
        \end{proof}
        \begin{theorem}
            If $(X,\,\tau)$ is a second-countable topological space, if
            $A\subseteq{X}$, and if $\tau_{A}$ is the subspace topology, then
            $(A,\,\tau_{A})$ is second-countable.
        \end{theorem}
        \begin{proof}
            Since $(X,\,\tau)$ is second-countable, there is a countable basis
            $\mathcal{B}\subseteq\tau$. That is, there is a surjection
            $\mathcal{U}:\mathbb{N}\rightarrow\mathcal{B}$ so that we may list
            the elements as:
            \begin{equation}
                \mathcal{B}=
                \{\,\mathcal{U}_{0},\,\dots,\,\mathcal{U}_{n},\,\dots\,\}
            \end{equation}
            Let $\tilde{\mathcal{B}}\subseteq\tau_{A}$ be defined by:
            \begin{equation}
                \tilde{\mathcal{B}}
                =\{\,A\cap\mathcal{U}_{n}\;|\;n\in\mathbb{N}\,\}
            \end{equation}
            $\tilde{\mathcal{B}}$ is countable since the elements are indexed
            by the natural numbers. We now must show that $\tilde{\mathcal{B}}$
            is a basis for $(A,\,\tau_{A})$. That is, given
            $\tilde{\mathcal{U}}\in\tau_{A}$ we must find
            $\tilde{\mathcal{O}}\subseteq\tilde{B}$ such that
            $\bigcup\tilde{\mathcal{O}}=\tilde{\mathcal{U}}$. Since
            $\tilde{\mathcal{U}}\in\tau_{A}$, by definition of the subspace
            topology there is some $\mathcal{V}\in\tau$ such that
            $\tilde{\mathcal{U}}=A\cap\mathcal{V}$. But $\mathcal{B}$ is a
            basis for $\tau$ so there is $\mathcal{O}\subseteq\mathcal{B}$
            such that $\bigcup\mathcal{O}=\mathcal{V}$. Define
            $\tilde{\mathcal{O}}$ via:
            \begin{equation}
                \tilde{\mathcal{O}}=
                \{\,A\cap\mathcal{W}\;|\;\mathcal{W}\in\mathcal{O}\,\}
            \end{equation}
            Then by definition of $\tilde{\mathcal{B}}$ we have
            $\tilde{\mathcal{O}}\subseteq\tilde{\mathcal{B}}$. Since the
            elements $\mathcal{W}\in\mathcal{O}$ are subsets of $\mathcal{V}$,
            and since $\tilde{\mathcal{U}}=A\cap\mathcal{V}$, we have
            $A\cap\mathcal{W}\subseteq\tilde{\mathcal{U}}$ for all
            $\mathcal{W}\in\mathcal{O}$, and hence
            $\bigcup\tilde{\mathcal{O}}\subseteq\tilde{\mathcal{U}}$. Reversing
            this, let $x\in\tilde{\mathcal{U}}$. Then $x\in{A}\cap\mathcal{V}$,
            and hence $x\in\mathcal{V}$. But $\bigcup\mathcal{O}=\mathcal{V}$
            so there is some $\mathcal{W}\in\mathcal{O}$ such that
            $x\in\mathcal{W}$. But then $x\in{A}$ and $x\in\mathcal{W}$, and
            hence $x\in{A}\cap\mathcal{W}$. But
            $A\cap\mathcal{W}\in\tilde{\mathcal{O}}$, so
            $x\in\bigcup\tilde{\mathcal{O}}$. Hence,
            $\tilde{\mathcal{U}}\subseteq\bigcup\tilde{\mathcal{O}}$, meaning
            $\tilde{\mathcal{U}}=\bigcup\tilde{\mathcal{O}}$. So
            $\tilde{\mathcal{B}}$ is a countable basis of $\tau_{A}$ and
            $(A,\,\tau_{A})$ is second-countable.
        \end{proof}
        \begin{theorem}
            If $(X,\,\tau)$ is a first-countable topological space, if
            $A\subseteq{X}$, and if $\tau_{A}$ is the subspace topology, then
            $(A,\,\tau_{A})$ is a first-countable topological space.
        \end{theorem}
        \begin{proof}
            Let $x\in{A}$. Since $A\subseteq{X}$ we have that $x\in{X}$. But
            $(X,\,\tau)$ is first countable, so there is a countable
            neighborhood basis $\mathcal{B}$ of $x$. Since $\mathcal{B}$ is
            countable, there is a surjection
            $\mathcal{U}:\mathbb{N}\rightarrow\mathcal{B}$ so that we may write
            the elements as:
            \begin{equation}
                \mathcal{B}=
                \{\,\mathcal{U}_{0},\,\dots,\,\mathcal{U}_{n},\,\dots\,\}
            \end{equation}
            Define $\tilde{\mathcal{B}}$ as:
            \begin{equation}
                \tilde{\mathcal{B}}
                =\{\,A\cap\mathcal{U}_{n}\;|\;n\in\mathbb{N}\,\}
            \end{equation}
            The set $\tilde{\mathcal{B}}$ is countable. We must now show it is
            a neighborhood basis of $x$. First, for all
            $\mathcal{V}\in\tilde{\mathcal{B}}$ we have $x\in\mathcal{V}$.
            For if $\mathcal{V}=\tilde{\mathcal{B}}$ we can write
            $\mathcal{V}=A\cap\mathcal{U}_{n}$ for some $n\in\mathbb{N}$. But
            $\mathcal{B}$ is a neighborhood basis for $x$ and
            $\mathcal{U}_{n}\in\mathcal{B}$, so $x\in\mathcal{U}_{n}$. But
            $x\in{A}$, and hence $x\in{A}\cap\mathcal{U}_{n}$. So every element
            of $\tilde{\mathcal{B}}$ contains $x$. If
            $\tilde{\mathcal{V}}\in\tau_{A}$ is such that
            $x\in\tilde{\mathcal{V}}$,
            then there is a $\mathcal{V}\in\tau$ such that
            $\tilde{\mathcal{V}}=A\cap\mathcal{V}$. But $\mathcal{B}$ is a
            neighborhood basis of $x$, so there is a
            $\mathcal{U}_{n}\in\mathcal{B}$ such that
            $x\in\mathcal{U}_{n}$ and $\mathcal{U}_{n}\subseteq\mathcal{V}$.
            But then $x\in{A}\cap\mathcal{U}_{n}$ and
            $A\cap\mathcal{U}_{n}\subseteq{A}\cap\mathcal{V}=\tilde{\mathcal{V}}$.
            But $A\cap\mathcal{U}_{n}$ is an element of $\tilde{\mathcal{B}}$,
            showing us that $(A,\,\tau_{A})$ is first-countable.
        \end{proof}
        Subspaces of sequential spaces do \textbf{not} need to be sequential.
        Spaces where every subspace is sequential are given a name.
        \begin{definition}[\textbf{Fr\'{e}chet-Urysohn Topological Space}]
            A Fr\'{e}chet-Urysohn topological space is a topological space
            $(X,\,\tau)$ such that for all $A\subseteq{X}$ it is true that
            $(A,\,\tau_{A})$ is a sequential topological space where $\tau_{A}$
            is the subspace topology.
        \end{definition}
        \begin{theorem}
            If $(X,\,\tau)$ is a first-countable topological space, then it
            is a Fr\'{e}chet-Urysohn topological space.
        \end{theorem}
        \begin{proof}
            Since first-countable spaces are sequential, and every subspace of
            a first-countable space is first-countable, every subspace of a
            first-countable space is also sequential, and hence
            $(X,\,\tau)$ is a Fr\'{e}chet-Urysohn space.
        \end{proof}
        \begin{theorem}
            If $(X,\,\tau)$ is a second-countable topological space, then it
            is a Fr\'{e}chet-Urysohn space.
        \end{theorem}
        \begin{proof}
            Since second-countable spaces are first-countable, this follows
            from the previous theorem.
        \end{proof}
        \begin{theorem}
            If $(X,\,\tau)$ is a metrizable topological space, then it is
            a Fr\'{e}chet-Urysohn space.
        \end{theorem}
        \begin{proof}
            Since metrizable spaces are first-countable, this follows from a
            previous theorem.
        \end{proof}
        The easiest space to describe that is sequential but \textit{not}
        Fr\'{e}chet-Urysohn requires the product topology, which we'll get to
        soon enough.
        \par\hfill\par
        Subspaces give rise to the notion of \textit{topological embeddings},
        which are another special type of function commonly studied in
        topology, analysis, and geometry.
        \begin{definition}[\textbf{Topological Embedding}]
            A topological embedding of a topological space $(X,\,\tau_{X})$ to
            a topological space $(Y,\,\tau_{Y})$ is a function
            $f:X\rightarrow{Y}$ such that $f:X\rightarrow{f}[X]$ is a
            homeomorphism between $X$ and $f[X]$ with respect to the
            subspace topology $\tau_{Y_{f[X]}}$.
        \end{definition}
        Topological embeddings allow us to think of a topological space
        $(X,\,\tau_{X})$ as just a \textit{subspace} of $(Y,\,\tau_{Y})$ with
        the subspace topology.
        \begin{theorem}
            If $(X,\,\tau)$ is a topological space, if $A\subseteq{X}$, if
            $\tau_{A}$ is the subspace topology, and if $\iota$ is the
            inclusion mapping, then $\iota:A\rightarrow{X}$ is a topological
            embedding.
        \end{theorem}
        \begin{proof}
            The image of $A$ is $\iota[A]=A$. $\iota:A\rightarrow\iota[A]$ is
            then just the identity function $\iota:A\rightarrow{A}$ with
            $\iota(x)=x$, which is a homeomorphism.
        \end{proof}
\end{document}
