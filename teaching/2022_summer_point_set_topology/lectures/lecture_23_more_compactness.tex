%-----------------------------------LICENSE------------------------------------%
%   This file is part of Mathematics-and-Physics.                              %
%                                                                              %
%   Mathematics-and-Physics is free software: you can redistribute it and/or   %
%   modify it under the terms of the GNU General Public License as             %
%   published by the Free Software Foundation, either version 3 of the         %
%   License, or (at your option) any later version.                            %
%                                                                              %
%   Mathematics-and-Physics is distributed in the hope that it will be useful, %
%   but WITHOUT ANY WARRANTY; without even the implied warranty of             %
%   MERCHANTABILITY or FITNESS FOR A PARTICULAR PURPOSE.  See the              %
%   GNU General Public License for more details.                               %
%                                                                              %
%   You should have received a copy of the GNU General Public License along    %
%   with Mathematics-and-Physics.  If not, see <https://www.gnu.org/licenses/>.%
%------------------------------------------------------------------------------%
\documentclass{article}
\usepackage{graphicx}                           % Needed for figures.
\usepackage{amsmath}                            % Needed for align.
\usepackage{amssymb}                            % Needed for mathbb.
\usepackage{amsthm}                             % For the theorem environment.
\usepackage{float}
\usepackage{tabularx, booktabs}
\usepackage{mathrsfs}
\usepackage[font=scriptsize,
            labelformat=simple,
            labelsep=colon]{subcaption} % Subfigure captions.
\usepackage[font={scriptsize},
            hypcap=true,
            labelsep=colon]{caption}    % Figure captions.
\usepackage{hyperref}
\hypersetup{
    colorlinks=true,
    linkcolor=blue,
    filecolor=magenta,
    urlcolor=Cerulean,
    citecolor=SkyBlue
}

%------------------------Theorem Styles-------------------------%
\theoremstyle{plain}
\newtheorem{theorem}{Theorem}[section]

% Define theorem style for default spacing and normal font.
\newtheoremstyle{normal}
    {\topsep}               % Amount of space above the theorem.
    {\topsep}               % Amount of space below the theorem.
    {}                      % Font used for body of theorem.
    {}                      % Measure of space to indent.
    {\bfseries}             % Font of the header of the theorem.
    {}                      % Punctuation between head and body.
    {.5em}                  % Space after theorem head.
    {}

% Define default environments.
\theoremstyle{normal}
\newtheorem{examplex}{Example}[section]
\newtheorem{definitionx}{Definition}[section]

\newenvironment{example}{%
    \pushQED{\qed}\renewcommand{\qedsymbol}{$\blacksquare$}\examplex%
}{%
    \popQED\endexamplex%
}

\newenvironment{definition}{%
    \pushQED{\qed}\renewcommand{\qedsymbol}{$\blacksquare$}\definitionx%
}{%
    \popQED\enddefinitionx%
}

\title{Point-Set Topology: Lecture 23}
\author{Ryan Maguire}
\date{Summer 2022}

% No indent and no paragraph skip.
\setlength{\parindent}{0em}
\setlength{\parskip}{0em}

\begin{document}
    \maketitle
    \section{Other Ideas for Compactness}
        Compact and sequentially compact are two of the most desirable
        properties used in analysis, geometry, and topology. There are several
        weaker notions that have found there way into several branches of
        mathematics. These properties are weaker, but are satisfied by many
        more spaces. \textit{Paracompactness}, for example, is a particularly
        weak property that has enormous use in manifold theory and geometry,
        and every metric space is paracompact (even though your average metric
        space is not compact).
        \begin{definition}[\textbf{Limit Point Compact Topogical Space}]
            A limit point compact topological space is a topological space
            $(X,\,\tau)$ such that for all infinite subsets $A\subseteq{X}$
            there exists a point $x\in{X}$ such that for all
            $\mathcal{U}\in\tau$ with $x\in\mathcal{U}$, there is a
            $y\in{A}$ such that $y\ne{x}$ and $y\in\mathcal{U}$
        \end{definition}
        Limit point compact was the original defining property of compactness
        when mathematicians were first thinking about the topology of the
        real line. Unlike compactness and sequential compactness, where one
        can't really say one idea is \textit{stronger} than the other, limit
        point compactness is a weaker notion. For the real line, however,
        limit point compact is equivalent to compactness
        (this is a corollary of the Bolzano-Weierstrass theorem).
        \begin{theorem}
            If $(X,\,\tau)$ is a sequentially compact topological space, then
            it is limit point compact.
        \end{theorem}
        \begin{proof}
            For if not, then there is an infinite set $A\subseteq{X}$ such that
            for all $x\in{X}$ there is a $\mathcal{U}\in\tau$ such that
            $x\in\mathcal{U}$ and for $y\in{A}$ either $y=x$ or
            $y\notin\mathcal{U}$. But if $A$ is infinite, there is a countably
            infinite subset $B\subseteq{A}$. Let $a:\mathbb{N}\rightarrow{B}$
            be a bijection. But $(X,\,\tau)$ is sequentially compact, so there
            is a convergent subsequence $a_{k}$. Let $x\in{X}$ be the limit.
            Then there
            is a $\mathcal{U}\in\tau$ such that $x\in\mathcal{U}$ and for
            all $y\in{A}$ either $y=x$ or $y\notin\mathcal{U}$. But
            $a_{k_{n}}\rightarrow{x}$ and $x\in\mathcal{U}$, so there is an
            $N\in\mathbb{N}$ such that for all $n\in\mathbb{N}$ with $n>N$ we
            have $a_{k_{n}}\in\mathcal{U}$. But $a:\mathbb{N}\rightarrow{B}$
            is bijective and $k:\mathbb{N}\rightarrow\mathbb{N}$ is strictly
            increasing, so $a_{k}$ is injective, meaning for all
            $n>N$, $a_{k_{n}}$ are distinct elements of $B$, and hence $A$,
            that are contained in $\mathcal{U}$, which is a contradiction.
            So $(X,\,\tau)$ is limit point compact.
        \end{proof}
        Compact also implies limit point compact, but a weaker notion than
        compact also implies limit point compact. This weaker notion is
        occasionally useful.
        \begin{definition}[\textbf{Countably Compact Topological Space}]
            A countably compact topological space is a topological space
            $(X,\,\tau)$ such that for all countable open covers
            $\mathcal{O}\subseteq\tau$ there is a finite subset
            $\Delta\subseteq\mathcal{O}$ such that $\Delta$ is an open cover.
        \end{definition}
        \begin{theorem}
            If $(X,\,\tau)$ is compact, then it is countably compact.
        \end{theorem}
        \begin{proof}
            Any countable open cover is indeed an open cover, and since
            $(X,\,\tau)$ is compact, there must be a finite subcover.
        \end{proof}
        \begin{theorem}
            If $(X,\,\tau)$ is countably compact, then it is limit point
            compact.
        \end{theorem}
        \begin{proof}
            If not there is an infinite subset $A\subseteq{X}$ such that for
            all $x\in{X}$ there is a $\mathcal{U}\in\tau$ such that for all
            $y\in{A}$ either $y=x$ or $y\notin\mathcal{U}$. But since
            $A$ is infinite, there is a countably infinite subset
            $B\subseteq{A}$. But then for all $x\in{X}$ there is a
            $\mathcal{U}\in\tau$ such that for all $y\in{B}$ either
            $y=x$ or $y\notin\mathcal{U}$. But then
            $\textrm{Cl}_{\tau}(B)=B$. For suppose not and let
            $y\in\textrm{Cl}_{\tau}(B)\setminus{B}$. Then there is an open
            subset $\mathcal{U}\in\tau$ such that $y\in\mathcal{U}$ and for
            all $z\in{B}$ either $z=y$ or $z\notin\mathcal{B}$. But
            $y\notin{B}$, so $z\ne{y}$, and hence
            $\mathcal{U}\cap{B}=\emptyset$. But then, since $\mathcal{U}$ is
            open, $\textrm{Cl}_{\tau}(B)\subseteq{X}\setminus\mathcal{U}$ and
            $y\notin{X}\setminus\mathcal{U}$, a contradiction. So
            $\textrm{Cl}_{\tau}(B)=B$ and $B$ is closed. For all
            $n\in\mathbb{N}$ let $\mathcal{U}_{n+1}\in\tau$ be such that
            $\mathcal{U}_{n+1}\cap{B}=\{\,a_{n+1}\,\}$. Let
            $\mathcal{U}_{0}=X\setminus\{\,B\,\}$. Then:
            \begin{equation}
                \mathcal{O}=
                \{\,\mathcal{U}_{n}\;|\;n\in\mathbb{N}\,\}
            \end{equation}
            is a countable open cover of $(X,\,\tau)$. But since $(X,\,\tau)$
            is countably compact, there is a finite subcover $\Delta$. But
            then there is a $\mathcal{U}_{n}$ such that infinitely many
            elements of $B$ are contained inside $\mathcal{U}_{n}$, which is
            a contradiction. So $(X,\,\tau)$ is limit point compact.
        \end{proof}
        \begin{theorem}
            If $(X,\,\tau)$ is compact, then it is limit point compact.
        \end{theorem}
        \begin{proof}
            If $(X,\,\tau)$ is compact, then it is countably compact, and hence
            $(X,\,\tau)$ is limit point compact.
        \end{proof}
        \begin{theorem}
            If $(X,\,\tau)$ if a limit point compact Fr\"{e}chet topological
            space, then it is countably compact.
        \end{theorem}
        \begin{proof}
            Suppose not, and let $\mathcal{O}\subseteq{X}$ be a countably
            infinite open cover that has no finite subcover. Since
            $\mathcal{O}$ is countably infinite there is a bijection
            $\mathcal{U}:\mathbb{N}\rightarrow\mathcal{O}$. Define
            $\mathcal{V}_{n}$ to be:
            \begin{equation}
                \mathcal{V}_{n}=\bigcup_{k=0}^{n}\mathcal{U}_{k}
            \end{equation}
            Since $\mathcal{O}$ has no finite subcover, $\mathcal{V}_{n}\ne{X}$
            for all $n\in\mathbb{N}$. But then $X\setminus\mathcal{V}_{n}$ must
            be infinite for each $n\in\mathbb{N}$, so we can pick a bijective
            sequence $a:\mathbb{N}\rightarrow{X}$ such that
            $a_{n}\notin\mathcal{V}_{n}$ for all $n\in\mathbb{N}$. But then
            the set:
            \begin{equation}
                A=\{\,a_{n}\in{X}\;|\;n\in\mathbb{N}\,\}
            \end{equation}
            is infinite, and since $(X,\,\tau)$ is limit point compact, there
            is an $x\in{X}$ such that for all $\mathcal{W}\in\tau$ with
            $x\in\mathcal{W}$, there is a $y\in{A}$ such that $x\ne{y}$ and
            $y\in\mathcal{W}$. But since $\mathcal{O}$ covers $X$ there is a
            $\mathcal{U}_{N}$ such that $x\in\mathcal{U}_{N}$. But $(X,\,\tau)$
            is Fr\'{e}chet, so for all $n\in\mathbb{N}$, $\{\,a_{n}\,\}$ is
            closed. But the finite union of closed sets is closed, so the set:
            \begin{equation}
                \mathcal{C}_{n}=\bigcup_{k=0}^{n}\{\,a_{k}\,\}
            \end{equation}
            is closed. But then $X\setminus\mathcal{C}_{N}$ is an open
            subset such that $x\in{X}\setminus\mathcal{C}_{N}$. But then
            $x\in\mathcal{V}_{N}\cap(X\setminus\mathcal{C}_{N})$, and this
            is the intersection of two open sets, which is therefore open. But
            if $x\in\mathcal{U}_{N}\cap(X\setminus\mathcal{C}_{N})$, then
            there is an $a_{n}\in{A}$ such that $a_{n}\ne{x}$ and
            $a_{n}\in\mathcal{U}_{N}\cap(X\setminus\mathcal{C}_{N})$. But
            then $n>N$ since if $n\leq{N}$, then $a_{n}\in\mathcal{C}_{N}$, and
            hence $a_{n}\notin{X}\setminus\mathcal{C}_{N}$. But then
            $n>N$ is such that $a_{n}\in\mathcal{U}_{N}$, a contradiction,
            so $(X,\,\tau)$ is countably compact.
        \end{proof}
        Sequentially compact is a nice property, and it is quite a shame
        compactness does not imply it in general. It is also a shame
        sequential compactness does not imply compact. If we add
        \textit{sequential} to our hypothesis, we can get one direction to
        work. First, a little lemma.
        \begin{theorem}
            If $(X,\,\tau)$ is countably compact, and if
            $\mathcal{C}\subseteq{X}$ is closed, then
            $(\mathcal{C},\,\tau_{\mathcal{C}})$ is countably compact.
        \end{theorem}
        \begin{proof}
            The proof is a mimicry of the idea for compact spaces. Given
            a countable open cover of $\mathcal{C}$, by adding
            $X\setminus\mathcal{C}$ we obtain a countable open cover of $X$
            since $\mathcal{C}$ is closed, so $X\setminus\mathcal{C}$ is open,
            and adding one more set to a countable collection is still
            countable. But since $(X,\,\tau)$ is countably compact there is a
            finite subcover. Restricting this finite subcover to $\mathcal{C}$
            shows that $(\mathcal{C},\,\tau_{\mathcal{C}})$ is countably
            compact. 
        \end{proof}
        \begin{theorem}
            If $(X,\,\tau)$ is a sequential countably compact topological space,
            then it is sequentially compact.
        \end{theorem}
        \begin{proof}
            For if not, then there is a sequence $a:\mathbb{N}\rightarrow{X}$
            with no convergent subsequence. Let $A\subseteq{X}$ be defined by:
            \begin{equation}
                A=\bigcup_{n\in\mathbb{N}}
                    \textrm{Cl}_{\tau}\big(\{\,a_{n}\,\}\big)
            \end{equation}
            Then $A$ is sequentially closed. For if
            $b:\mathbb{N}\rightarrow{A}$ is a sequence that converges to
            $y\in{X}$, either there is an $m\in\mathbb{N}$ such that
            $y\in\textrm{Cl}_{\tau}(\{\,a_{m}\,\})$ or not. If there is
            such an $m$, then $y\in{A}$. If not, then by choosing
            $a_{k}$ and $b_{\ell}$ to be subsequences such that
            $b_{\ell_{n}}\in\textrm{Cl}_{\tau}(\{\,a_{k_{n}}\,\})$, we have
            found a convergent subsequence $a_{k_{n}}\rightarrow{y}$, which is
            a contradiction. Hence $A$ is sequentially closed. But
            $(X,\,\tau)$ is sequential, so $A$ is closed. But then
            $(A,\,\tau_{A})$ is countably compact, where $\tau_{A}$ is the
            subspace topology. But countably compact implies limit point
            compact, so there is a point $x\in{A}$ such that for all
            $\mathcal{U}\in\tau_{A}$ with $x\in\mathcal{U}$, there is a
            $a_{n}$ such that $a_{n}\ne{x}$ and $a_{n}\in\mathcal{U}$. But
            $x$ must be in only finitely many sets of the form
            $\textrm{Cl}_{\tau}\big(\{\,a_{n}\,\}\big)$, otherwise
            $a$ would have a convergent subsequence converging to $x$.
            But then there is an $N\in\mathbb{N}$ such that for all
            $n>N$ we have $x\notin\textrm{Cl}_{\tau}\big(\{\,a_{n}\,\}\big)$.
            But then, defining:
            \begin{equation}
                B=\bigcup_{n=N+1}^{\infty}
                    \textrm{Cl}_{\tau}\big(\{\,a_{n}\,\}\big)
            \end{equation}
            we see that $B$ is closed, by the previous argument, but $B$
            does not contain the point $x$, which is a contradiction since
            $x$ is still a limit point of $B$. So $(X,\,\tau)$ is
            sequentially compact.
        \end{proof}
        A short corollary of this is often used when sequential compactness is
        desired.
        \begin{theorem}
            If $(X,\,\tau)$ is compact and first countable, then it is
            sequentially compact.
        \end{theorem}
        \begin{proof}
            Compact implies countably compact, and first countable implies
            sequential. So $(X,\,\tau)$ is countably compact and sequential,
            and therefore sequentially compact.
        \end{proof}
        While sequentially compact does not imply compact, there is a partial
        result. Sequentially compact always implies countably compact, and
        often enough countably compact is sufficient.
        \begin{theorem}
            If $(X,\,\tau)$ is sequentially compact, then it is
            countably compact.
        \end{theorem}
        \begin{proof}
            If not there is a countably infinite open cover
            $\mathcal{O}\subseteq\tau$ with no finite subcover. But then,
            since $\mathcal{O}$ is countably infinite, there is a bijection
            $\mathcal{U}:\mathbb{N}\rightarrow\mathcal{O}$ so that we may
            list the elements as:
            \begin{equation}
                \mathcal{O}=
                \{\,\mathcal{U}_{0},\,\dots,\,\mathcal{U}_{n},\,\dots\,\}
            \end{equation}
            But $\mathcal{U}_{n}\ne{X}$ for all $n\in\mathbb{N}$, otherwise
            $\Delta=\{\,\mathcal{U}_{n}\,\}$ is a finite subcover. So
            $X\setminus\mathcal{U}_{n}\ne\emptyset$ for all $n\in\mathbb{N}$.
            Moreover, the set $\mathcal{V}_{n}$ defined by:
            \begin{equation}
                \mathcal{V}_{n}=\bigcup_{k=0}^{n}\mathcal{U}_{n}
            \end{equation}
            is such that $\mathcal{V}_{n}\ne{X}$, otherwise $\mathcal{O}$ has a
            finite subcover. Define $a:\mathbb{N}\rightarrow{X}$ via
            $a_{n}\in{X}\setminus\mathcal{V}_{n}$ for all $n\in\mathbb{N}$.
            But $(X,\,\tau)$ is sequentially compact, so there is a
            convergent subsequence $a_{k}$ with limit $x\in{X}$. But since
            $\mathcal{O}$ covers $X$, there is a $\mathcal{U}_{N}$ such
            that $x\in\mathcal{U}_{N}$. But then for all
            $n>N$ we have $a_{k_{n}}\not\in\mathcal{U}_{N}$, which is a
            contradiction since $a_{k_{n}}\rightarrow{x}$. So
            $(X,\,\tau)$ is countably compact.
        \end{proof}
        One way to weaken compactness is by lessening open covers to
        countable open covers. The other way is by lessening finite subcover
        to countable subcover. This idea has proven quite useful in many
        applications in analysis.
        \begin{definition}[\textbf{Lindel\"{o}f Topological Space}]
            A Lindel\"{o}f topological space is a topological space
            $(X,\,\tau)$ such that for every open cover
            $\mathcal{O}\subseteq\tau$ there is a countable subcover
            $\Delta\subseteq\mathcal{O}$.
        \end{definition}
        \begin{theorem}
            If $(X,\,\tau)$ is a topological space, then it is countably
            compact and Lindel\"{o}f if and only if it is compact.
        \end{theorem}
        \begin{proof}
            Compact implies countably compact, and it also implies Lindel\"{o}f
            since every open has a finite open subcover, which is definitely a
            countable open subcover. Going the other, if we are given
            $\mathcal{O}\subseteq\tau$ an open cover, since
            $(X,\,\tau)$ is Lindel\"{o}f there is a countable subcover
            $\tilde{\Delta}\subseteq\mathcal{O}$. But $(X,\,\tau)$ is countably
            compact, so if $\tilde{\Delta}$ is a countable open cover, then
            there is a finite subcover $\Delta\subseteq\tilde{\Delta}$. But
            then $\Delta\subseteq\mathcal{O}$ is a finite subcover, so
            $(X,\,\tau)$ is compact.
        \end{proof}
        \begin{theorem}
            If $(X,\,\tau)$ is second countable, then it is Lindel\"{o}f.
        \end{theorem}
        \begin{proof}
            If not, there is an open cover $\mathcal{O}\subseteq\tau$ with no
            countable subcover. Since $(X,\,\tau)$ is second countable there is
            a countable basis $\mathcal{B}$. Let
            $\mathcal{U}:\mathbb{N}\rightarrow\mathcal{B}$ be a surjection:
            \begin{equation}
                \mathcal{B}=
                \{\,\mathcal{U}_{0},\,\dots,\,\mathcal{U}_{n},\,\dots\,\}
            \end{equation}
            Define $A\subseteq\mathbb{N}$ via:
            \begin{equation}
                A=\{\,n\in\mathbb{N}\;|\;\textrm{there exists }
                    \mathcal{V}\in\mathcal{O}\textrm{ such that }
                    \mathcal{U}_{n}\subseteq\mathcal{V}\,\}
            \end{equation}
            $A$ is non-empty since $\mathcal{B}$ is a basis, and hence for
            all $\mathcal{V}\in\mathcal{O}$ there is some
            $\mathcal{U}_{n}\in\mathcal{B}$ such that
            $\mathcal{U}_{n}\subseteq\mathcal{V}$. Since $\mathcal{O}$ is not
            countable, it is certainly not finite, and hence not empty, showing
            that $A$ is non-empty as well. Since $A\subseteq\mathbb{N}$ it is
            countable as well. By the axiom of choice we can find
            a function $\mathcal{V}:A\rightarrow\mathcal{O}$ such that for all
            $n\in{A}$, $\mathcal{U}_{n}\subseteq\mathcal{V}_{n}$. But then the
            set $\Delta\subseteq\mathcal{O}$ defined by:
            \begin{equation}
                \Delta=\{\,\mathcal{V}_{n}\;|\;n\in\mathbb{N}\,\}
            \end{equation}
            is a countable open cover of $(X,\,\tau)$. For given
            $x\in{X}$, since $\mathcal{O}$ is an open cover there is a
            $\mathcal{W}\in\mathcal{O}$ such that $x\in\mathcal{W}$. But
            $\mathcal{B}$ is a basis, so there is a
            $\mathcal{U}_{n}\in\mathcal{B}$ such that $x\in\mathcal{U}_{n}$ and
            $\mathcal{U}_{n}\subseteq\mathcal{W}$. But then
            $\mathcal{U}_{n}\subseteq\mathcal{V}_{n}$, so
            $x\in\mathcal{V}_{n}$, showing that $\Delta$ is a countable
            subcover of $\mathcal{O}$, which is a contradiction. Hence,
            $(X,\,\tau)$ is Lindel\"{o}f.
        \end{proof}
        \begin{theorem}
            If $(X,\,\tau)$ is second countable and sequentially compact,
            then it is compact.
        \end{theorem}
        \begin{proof}
            Second countable implies Lindel\"{o}f and sequentially compact
            implies countably compact, so $(X,\,\tau)$ is a countably compact
            Lindel\"{o}f space, and is therefore compact.
        \end{proof}
        \begin{theorem}
            If $(X,\,\tau)$ is compact, and if $f:X\rightarrow\mathbb{R}$ is
            continuous with respect to the standard topology
            $\tau_{\mathbb{R}}$, then $f$ is bounded.
        \end{theorem}
        \begin{proof}
            For if not, if $f$ is unbounded, then for all $n\in\mathbb{N}$,
            since $f$ is continuous
            $f^{-1}\big[(-n,\,n)]$ is an open subset of $X$ and the set:
            \begin{equation}
                \mathcal{O}=\Big\{\,f^{1}\big[(-n,\,n)\big]\;|\;n\in\mathbb{N}
                    \Big\}
            \end{equation}
            is an open cover of $X$ that has no finite subcover since $f$ is
            unbounded, which is a contradiction since $f$ is compact. Hence,
            $f$ is bounded.
        \end{proof}
        We can make this a lot easier using some theorems about compactness, and
        get a much stronger result.
        \begin{theorem}[\textbf{Extreme Value Theorem}]
            If $(X,\,\tau)$ is compact, and if $f:X\rightarrow\mathbb{R}$ is
            continuous with respect to the standard topology
            $\tau_{\mathbb{R}}$, then there are points
            $x_{\textrm{min}},x_{\textrm{max}}\in{X}$ such that for all
            $x\in{X}$, $f(x_{\textrm{min}})\leq{f}(x)\leq{f}(x_{\textrm{max}})$.
        \end{theorem}
        \begin{proof}
            Since $f$ is continuous and $(X,\,\tau)$ is compact,
            $f[X]\subseteq\mathbb{R}$ is compact. But then by the Heine-Borel
            theorem, $f[X]$ is closed and bounded. Since it is bounded, there
            exists $m,M\in\mathbb{R}$ such that $m$ is the infimum, and
            $M$ is the supremum. Since $f[X]$ is closed, $m,M\in{f}[X]$, meaning
            there are $x_{\textrm{min}},x_{\textrm{max}}$ such that
            $f(x_{\textrm{min}})=m$ and $f(x_{\textrm{max}})=M$. Since
            $m$ and $M$ are the bounds of $f[X]$, for all
            $x\in{X}$ we have
            $f(x_{\textrm{min}})\leq{f}(x)\leq{f}(x_{\textrm{max}})$.
        \end{proof}
        This idea is useful enough that it gets a name.
        \begin{definition}[\textbf{Pseudocompact Topological Space}]
            A pseudocompact space is a topological space $(X,\,\tau)$ such that
            every continuous function $f:X\rightarrow\mathbb{R}$ is bounded.
        \end{definition}
        The extreme value theorem shows that compact implies pseudocompact.
        So does sequentially compact. The proof is identical if we know that
        the continuous image of a sequentially compact space is sequentially
        compact. Let's prove this.
        \begin{theorem}
            If $(X,\,\tau_{X})$ is sequentially compact, if
            $(Y,\,\tau_{Y})$ is a topological space, and if $f:X\rightarrow{Y}$
            is continuous, then $(f[X],\,\tau_{Y_{f[X]}})$ is sequentially
            compact.
        \end{theorem}
        \begin{proof}
            For suppose not and let $b:\mathbb{N}\rightarrow{f}[X]$ be a
            sequence with no converent subsequence. Since $b_{n}\in{f}[X]$, by
            the axiom of choice we can find a sequence
            $a:\mathbb{N}\rightarrow{X}$ such that $f(a_{n})=b_{n}$ for all
            $n\in\mathbb{N}$. But $(X,\,\tau_{X})$ is sequentially compact,
            so there is a convergent subsequence $a_{k}$ with limit
            $x\in{X}$. That is, $a_{k_{n}}\rightarrow{x}$. But $f$ is
            continuous, so $f(a_{k_{n}})\rightarrow{f}(x)$. But then
            $b_{k_{n}}\rightarrow{f}(x)$, meaning $b_{k}$ is a converegent
            subsequence, which is a contradiction. So
            $(f[X],\,\tau_{Y_{f[X]}})$ is sequentially compact.
        \end{proof}
        \begin{theorem}
            If $(X,\,\tau)$ is sequentially compact, then it is pseudocompact.
        \end{theorem}
        \begin{proof}
            For if $f:X\rightarrow\mathbb{R}$ is continuous, then
            $f[X]\subseteq\mathbb{R}$ is sequentially compact, and since
            $(\mathbb{R},\,\tau_{\mathbb{R}})$ is metrizable, sequentially
            compact implies compact, meaning $f[X]$ is closed and bounded by
            the Heine-Borel theorem. Hence, $(X,\,\tau)$ is pseudocompact.
        \end{proof}
        \begin{theorem}
            If $(X,\,\tau)$ is metrizable, then $(X,\,\tau)$ is compact if and
            only if it is pseudocompact.
        \end{theorem}
        \begin{proof}
            Compact always implies pseudocompact. Since $(X,\,\tau)$ is
            metrizable, to prove pseudocompact implies compact it is sufficient
            to prove that $(X,\,\tau)$ is sequentially compact
            (since compact and sequentially compact are equivalent in
            metrizable spaces). Suppose $(X,\,\tau)$ is not sequentially
            compact, and let $d$ be a metric that induces $\tau$. Then there is
            a sequence $a:\mathbb{N}\rightarrow{X}$ with no convergent
            subsequence. Then the set $A\subseteq{X}$ defined by:
            \begin{equation}
                A=\{\,a_{n}\in{X}\;|\;n\in\mathbb{N}\,\}
            \end{equation}
            is closed. Not only that, but $(A,\,\tau_{A})$ as a subspace is
            discrete. For given $x\in{X}$ there is some $\varepsilon_{x}>0$ such
            that the $\varepsilon_{x}$ ball around $x$ contains at most one element
            of $A$ (otherwise we could obtain a convergent subsequence tending
            to $x$). In particular, we can apply this to every $a_{n}\in{A}$
            meaning there is an open subset in $A$ that contains only $a_{n}$.
            Using this, define $f:A\rightarrow\mathbb{R}$ via
            $f(a_{n})=n$. Then, since $A$ is closed, and since $f$ is
            continuous since $A$ is a discrete space, by the Tietze extension
            theorem there is a
            continuous function $\tilde{f}:X\rightarrow\mathbb{R}$ such that
            $\tilde{f}|_{A}=f$. But $\tilde{f}$ is not bounded, which is a
            contradiction since $(X,\,\tau)$ is pseudocompact. Hence,
            $(X,\,\tau)$ is sequentially compact, and therefore compact
            since $(X,\,\tau)$ is metrizable.
        \end{proof}
        The real line has the property that it can be written as the union
        of countably many compact sets. Namely, given $n\in\mathbb{N}$, define
        $\mathcal{C}_{n}=[-n,\,n]$. Then each $\mathcal{C}_{n}$ is a compact
        subset and $\bigcup_{n}\mathcal{C}_{n}=\mathbb{R}$, so $\mathbb{R}$ is
        the union of countably many compact sets. This gets a name. 
        \begin{definition}[\textbf{$\sigma$ Compact Topological Space}]
            A $\sigma$ compact topological space is a topological space
            $(X,\,\tau)$ such that there exists a countable set
            $\mathcal{O}$ such that for all $\mathcal{C}\in\mathcal{O}$,
            $\mathcal{C}\subseteq{X}$ is compact, and such that
            $X=\bigcup\mathcal{O}$.
        \end{definition}
        The real line, complex plane (or Euclidean plane), and Euclidean space
        are all Lindel\"{o}f spaces, even though they are not compact. This has
        some use in analysis. It's a lot easier to see that
        $\mathbb{R}^{n}$ is $\sigma$ compact since closed balls of radius
        $n$ for all $n\in\mathbb{N}$ create a countable collection of
        compact subsets that cover the space. Fortunately, $\sigma$ compact
        implies the Lindel\"{o}f property.
        \begin{theorem}
            If $(X,\,\tau)$ is $\sigma$ compact, then it is Lindel\"{o}f.
        \end{theorem}
        \begin{proof}
            For if not, then there is an open cover $\mathcal{O}\subseteq\tau$
            with no countable subcover. But $(X,\,\tau)$ is $\sigma$ compact
            so there is a countable set $\mathcal{B}$ of compact subsets of $X$
            such that $X=\bigcup\mathcal{B}$. Since $\mathcal{B}$ is countable
            there is a surjection $A:\mathbb{N}\rightarrow\mathcal{B}$.
            But then for all $n\in\mathbb{N}$, $(A_{n},\,\tau_{A_{n}})$ is
            compact, by hypothesis, and $\mathcal{O}$ covers $A_{n}$. So
            there is a finite subcover $\Delta_{n}$. But then, since
            $\Delta_{n}$ is finite for all $n\in\mathbb{N}$, the set
            $\Delta\subseteq\mathcal{O}$ defined by:
            \begin{equation}
                \Delta=\bigcup_{n\in\mathbb{N}}\Delta_{n}
            \end{equation}
            is countable. But $\Delta$ is a cover of $X$ since given
            $x\in{X}$ there is an $n\in\mathbb{N}$ such that $x\in{A}_{n}$,
            and hence $x\in\bigcup\Delta_{n}$, but $\Delta_{n}\subseteq\Delta$,
            so $x\in\bigcup\Delta$. So $\Delta$ is a countable subcover of
            $\mathcal{O}$, a contradiction. Hence, $(X,\,\tau)$ is
            Lindel\"{o}f.
        \end{proof}
        There is one more property that is ever-so-slightly stronger than
        $\sigma$ compact, but has found quite a lot of use in the theory of
        manifolds and Riemannian geometry. The idea of being
        \textit{compactly exhaustible}.
        \begin{definition}[\textbf{Compactly Exhaustible Topological Space}]
            A compactly exhaustible topological space is a topological space
            $(X,\,\tau)$ such that there is a sequence
            $A:\mathbb{N}\rightarrow\mathcal{P}(X)$ such that for all
            $n\in\mathbb{N}$ it is true that $A_{n}$ is a compact subset,
            $A_{n}\subseteq\textrm{Int}_{\tau}(A_{n+1})$, and such that
            $X=\bigcup_{n\in\mathbb{N}}A_{n}$.
        \end{definition}
        Euclidean space is compactly exhaustible (every manifold is).
        Take $\mathcal{C}_{n}$ to be the closed ball of radius $n$. Then, just
        like with $\sigma$ compact, these sets are all compact and cover
        $\mathbb{R}^{n}$, but also the closed ball of radius $n$ fits entirely
        inside the open ball of radius $n+1$ showing that
        $\mathcal{C}_{n}\subseteq\textrm{Int}_{\tau}(\mathcal{C}_{n+1})$.
        Now, for the result. Compactly exhaustible implies $\sigma$ compact. 
        \begin{theorem}
            If $(X,\,\tau)$ is compactly exhaustible, then it is
            $\sigma$ compact.
        \end{theorem}
        \begin{proof}
            Let $A:\mathbb{N}\rightarrow\mathcal{P}(X)$ be a sequence such that
            $A_{n}$ is compact, $A_{n}\subseteq\textrm{Int}_{\tau}(A_{n+1})$,
            and $\bigcup_{n\in\mathbb{N}}A_{n}=X$. Then the set:
            \begin{equation}
                \mathcal{O}=
                \{\,A_{n}\;|\;n\in\mathbb{N}\,\}
            \end{equation}
            is a countable collection of subsets that are compact and cover
            $X$, so $(X,\,\tau)$ is $\sigma$ compact.
        \end{proof}
\end{document}
