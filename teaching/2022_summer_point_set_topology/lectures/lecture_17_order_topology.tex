%-----------------------------------LICENSE------------------------------------%
%   This file is part of Mathematics-and-Physics.                              %
%                                                                              %
%   Mathematics-and-Physics is free software: you can redistribute it and/or   %
%   modify it under the terms of the GNU General Public License as             %
%   published by the Free Software Foundation, either version 3 of the         %
%   License, or (at your option) any later version.                            %
%                                                                              %
%   Mathematics-and-Physics is distributed in the hope that it will be useful, %
%   but WITHOUT ANY WARRANTY; without even the implied warranty of             %
%   MERCHANTABILITY or FITNESS FOR A PARTICULAR PURPOSE.  See the              %
%   GNU General Public License for more details.                               %
%                                                                              %
%   You should have received a copy of the GNU General Public License along    %
%   with Mathematics-and-Physics.  If not, see <https://www.gnu.org/licenses/>.%
%------------------------------------------------------------------------------%
\documentclass{article}
\usepackage{graphicx}                           % Needed for figures.
\usepackage{amsmath}                            % Needed for align.
\usepackage{amssymb}                            % Needed for mathbb.
\usepackage{amsthm}                             % For the theorem environment.
\usepackage{float}
\usepackage[font=scriptsize,
            labelformat=simple,
            labelsep=colon]{subcaption} % Subfigure captions.
\usepackage[font={scriptsize},
            hypcap=true,
            labelsep=colon]{caption}    % Figure captions.
\usepackage{hyperref}
\hypersetup{
    colorlinks=true,
    linkcolor=blue,
    filecolor=magenta,
    urlcolor=Cerulean,
    citecolor=SkyBlue
}

%------------------------Theorem Styles-------------------------%
\theoremstyle{plain}
\newtheorem{theorem}{Theorem}[section]

% Define theorem style for default spacing and normal font.
\newtheoremstyle{normal}
    {\topsep}               % Amount of space above the theorem.
    {\topsep}               % Amount of space below the theorem.
    {}                      % Font used for body of theorem.
    {}                      % Measure of space to indent.
    {\bfseries}             % Font of the header of the theorem.
    {}                      % Punctuation between head and body.
    {.5em}                  % Space after theorem head.
    {}

% Define default environments.
\theoremstyle{normal}
\newtheorem{examplex}{Example}[section]
\newtheorem{definitionx}{Definition}[section]

\newenvironment{example}{%
    \pushQED{\qed}\renewcommand{\qedsymbol}{$\blacksquare$}\examplex%
}{%
    \popQED\endexamplex%
}

\newenvironment{definition}{%
    \pushQED{\qed}\renewcommand{\qedsymbol}{$\blacksquare$}\definitionx%
}{%
    \popQED\enddefinitionx%
}

\title{Point-Set Topology: Lecture 17}
\author{Ryan Maguire}
\date{Summer 2022}

% No indent and no paragraph skip.
\setlength{\parindent}{0em}
\setlength{\parskip}{0em}

\begin{document}
    \maketitle
    \section{Order Topology and Orderable Spaces}
        A total order on a set $X$ is a relation $\leq$ that is reflexive
        ($a\leq{a}$), anti-symmetric
        ($a\leq{b}$ and $b\leq{a}$ implies $a=b$), transitive
        ($a\leq{b}$ and $b\leq{c}$ implies $a\leq{c}$), and total
        (either $a\leq{b}$ or $b\leq{a}$ for all $a$ and $b$). This induces
        another relation $<$ on $X$ defined by $a<b$ if and only if $a\leq{b}$
        and $a\ne{b}$. The primary example is
        \textit{less than or equal to} on $\mathbb{R}$, and the induced relation
        is \textit{less than}. Given a total order on a set $X$ it is possible
        to use this to induce a topology $\tau_{<}$ on $X$ that has some very
        nice properties. Many topological properties of the real line stem from
        the fact that the standard topology $\tau_{\mathbb{R}}$ and the
        order topology $\tau_{<}$ are the same (where $<$ is the usual
        \textit{less than} relation). First, some notation. Given a totally
        ordered set $(X,\,<)$, and $a,b\in{X}$, we write:
        \begin{align}
            (a,\,b)&=\{\,c\in{X}\;|\;a<c\textrm{ and }c<b\,\}\\
            [a,\,b)&=\{\,c\in{X}\;|\;a\leq{c}\textrm{ and }c<b\,\}\\
            (a,\,b]&=\{\,c\in{X}\;|\;a<c\textrm{ and }c\leq{b}\,\}\\
            (-\infty,\,a)&=\{\,c\in{X}\;|\;c<a\,\}\\
            (a,\,\infty)&=\{\,c\in{X}\;|\;a<c\,\}\\
            (-\infty,\,a]&=\{\,c\in{X}\;|\;c\leq{a}\,\}\\
            [a,\,\infty)&=\{\,c\in{X}\;|\;a\leq{c}\,\}
        \end{align}
        Note, we're not saying $\infty$ is a thing, or an element of $X$, this
        is just notation. Just like how $(0,\,\infty)$ is the set of all
        positive numbers in $\mathbb{R}$, even though $\infty$ is not a number.
        We use this to define the order topology.
        \begin{definition}[\textbf{Order Topology}]
            The order topology on a totally ordered set $(X,\,<)$ is the
            topology $\tau_{<}$ generated by the set $\mathcal{B}$ defined by:
            \begin{equation}
                \mathcal{B}=\{\,(a,\,b)\;|\;a,b\in{X}\,\}
                    \cup\{\,(a,\,\infty)\;|\;a\in{X}\,\}
                    \cup\{\,(-\infty,\,a)\;|\;a\in{X}\,\}
            \end{equation}
            That is, the set of all open intervals, open right-rays, and
            open left-rays.
        \end{definition}
        \begin{example}
            The real line $\mathbb{R}$ with the standard Euclidean topology is
            also the order topology induced by the \textit{less than} relation.
            The Euclidean metric on $\mathbb{R}$ yields a basis consisting of
            open intervals, which is precisely the order topologies basis.
        \end{example}
        \begin{definition}[\textit{Linearly Orderable Topological Space}]
            A linearly orderable topological space is a topological space
            $(X,\,\tau)$ such that there exists a total order $\leq$ on $X$
            such that $\tau=\tau_{<}$ where $\tau_{<}$ is the order topology.
        \end{definition}
        Like metrizable spaces, linearly orderable spaces are very nice,
        topologically.
        \begin{theorem}
            If $(X,\,\tau)$ is a linearly orderable topological space, then
            it is Hausdorff.
        \end{theorem}
        \begin{proof}
            Let $a,b\in{X}$, $a\ne{b}$, and let $<$ be the order that induces
            $\tau$. Since $<$ comes from a total order, either $a<b$ or $b<a$.
            Suppose $a<b$ (the proof is symmetric). If there are no elements
            $c\in{X}$ such that $c\in(a,\,b)$, then let
            $\mathcal{U}=(-\infty,\,b)$ and $\mathcal{V}=(a,\,\infty)$.
            Then $a\in\mathcal{U}$ and $b\in\mathcal{V}$ since $a<b$. But
            also $\mathcal{U}$ and $\mathcal{V}$ are open by the definition
            of the order topology. Moreover,
            $\mathcal{U}\cap\mathcal{V}=\emptyset$ since there are no elements
            $c$ such that $a<c$ and $c<b$. If there is an element
            $c\in(a,\,b)$, let $\mathcal{U}=(-\infty,\,c)$ and
            $\mathcal{V}=(c,\,\infty)$. Then $a\in\mathcal{U}$,
            $b\in\mathcal{V}$, and $\mathcal{U}$ and $\mathcal{V}$ are open.
            Also $\mathcal{U}\cap\mathcal{V}=\emptyset$ since you can't have
            $x<c$ and $c<x$ simultaneously. Hence, $(X,\,\tau)$ is Hausdorff.
        \end{proof}
        \begin{example}[\textbf{Subspace Order}]
            Given a totally ordered space $(X,\,\leq)$ with the order topology
            $\tau_{<}$, if we have $A\subseteq{X}$ there are two topologies we
            can place on $A$. First, the subspace topology from $\tau_{<}$.
            Second, $\leq$ restricts to a total order on $A$, label this
            $\leq_{A}$. We get a topology $\tau_{<_{A}}$ via this subspace
            order. It does \textbf{not} need to be the case that the subspace
            topology and the suborder topologies are the same. Let
            $A\subseteq\mathbb{R}$ be defined by:
            \begin{equation}
                A=\big\{\,x\in\mathbb{R}\;|\;
                    x=-1\textrm{ or }x=\frac{1}{n+1}
                    \textrm{ for some }n\in\mathbb{N}\,\big\}
            \end{equation}
            In the subspace topology $-1$ is isolated, the set
            $\{\,-1\,\}$ is open since:
            \begin{equation}
                A\cap\big(-\frac{3}{2},\,-\frac{1}{2}\big)=\{\,-1\,\}
            \end{equation}
            However, with the order induced
            from $\mathbb{R}$, the subspace order does not have
            $\{\,1\,\}$ as an open set. Any open set containing $-1$ must also
            contain some $\frac{1}{N}$, and hence also contain every
            $\frac{1}{n}$ for all $n>N$.
        \end{example}
        \begin{example}[\textbf{Lexicographic Plane}]
            We can define a total order on $\mathbb{R}^{2}$. Given
            $(x_{0},\,y_{0})$ and $(x_{1},\,y_{1})$, define
            $(x_{0},\,y_{0})\leq(x_{1},\,y_{1})$ if and only if either
            $x_{0}\leq{x}_{1}$ or, $x_{0}=x_{1}$ and $y_{0}\leq{y}_{1}$. That
            is, first examine the $x$ axis and compare these. If they're
            identical, move on to the $y$ axis. This is also called the
            \textit{dictionary order} since it mimics how words are ordered in
            a dictionary. First, you compare the first letter, then the second,
            and so on. This order does \textbf{not} give the standard topology
            on $\mathbb{R}^{2}$, but it does give a good example to test ideas
            on. The lexicographic plane often serves as a counterexample to
            many plausible conjectures in topology.
        \end{example}
    \section{The Long Line}
        This next example is a bit involved, but I hope you'll stick
        around for the ride. It is one of my favorite spaces. The axiom of
        choice tells us that the well-ordering theorem is true. That is,
        every set $X$ has a well-order $\leq$ which is a total order such
        that every non-empty subset $A\subseteq{X}$ has a \textit{smallest}
        element. There is only one well-order that I know of that arises
        naturally in mathematics, and that is the natural order on
        $\mathbb{N}$. $\mathbb{Z}$, with the standard order, is not
        well-ordered since $\mathbb{Z}$ has no least element
        (there is no \textit{negative infinity} integer). What about
        $\mathbb{R}^{+}$, all positive numbers? Also no, since there is
        no smallest positive real number. How about
        $\mathbb{R}_{\geq{0}}$, all positive numbers and zero? Also no.
        This set does have a smallest number, it is zero, but
        $\mathbb{R}^{+}\subseteq\mathbb{R}_{\geq{0}}$ is a non-empty subset
        that has no smallest element.
        \par\hfill\par
        Well-orders are quite special. If $(X,\,\leq)$ is well-ordered and
        $x\in{X}$, either $x$ is the largest element or there is a
        \textit{next largest element}. The
        set $[x,\,\infty)$ is such that $x$ is the smallest element. Removing
        it, considering $(x,\,\infty)$, since $x$ is not the largest element
        (well-ordered sets don't need to have a largest element, but it is
        possible for such an element to exist) this subset is
        non-empty, so there is a least element. This least element is the
        next largest element after $x$. So, knowing this, how could one possibly
        order the real numbers in a way that gives a well-order? Well, there's
        a reason the well-ordering theorem is equivalent to the axiom of choice,
        there's no constructive way to do it. But pretend, for a moment, that
        we accept the axiom of choice and the well-ordering theorem and let
        $\prec$ be a well-order on $\mathbb{R}$. Consider the sentence
        $P(x)$ \textit{there are uncountably many elements less than} $x$.
        The reals are uncountable, so the set of real numbers satisfying this
        relation is non-empty. So there is a \textit{least} element
        $\alpha$. This is the \textit{first} number that has uncountably many
        elements less than it. So every element $x\prec\alpha$ has only
        countably many elements less than $x$. This is bizarre. As noted,
        there is always a $+1$ element, a next largest element, in a well-order.
        $\alpha$ shows there does not need to be a \textit{next smallest}, or a
        $-1$ before $\alpha$.
        \par\hfill\par
        Let $\omega$ be the set of all $x\in\mathbb{R}$
        such that $x\prec\alpha$. $\omega$ is the
        \textit{first-uncountable ordinal}. With the order $\prec$ restricted
        to $\omega$, $\omega$ also becomes well-ordered. Let
        $X=\omega\times[0,\,1)$, where $[0,\,1)$ is the set of numbers between
        $0$ and $1$ with the \textbf{standard order}, including $0$ but
        excluding $1$. We can equip $\omega\times[0,\,1)$ with the
        lexicographic ordering, saying
        $(x_{0},\,y_{0})\leq(x_{1},\,y_{1})$ if and only if
        $x_{0}\preceq{x}_{1}$ or, $x_{0}=x_{1}$ and $y_{0}\leq{y}_{1}$.
        Equipping $X$ with this order topology gives us the
        \textit{long ray}. The real ray $\mathbb{R}_{\geq{0}}$ can be thought
        of as stringing along countably many copies of $[0,\,1)$ in a row.
        The long ray does this with \textit{uncountably} many copies of
        $[0,\,1)$. The long ray is extremely long. The long line is obtained by
        taking two copies of the long ray and gluing the endpoints together.
        Topologically, the long line is very pleasant and a nightmare. It has
        many very nice topological properties, but also serves as a great
        utensil for counterexamples to some very plausible claims.
    \section{Other Ordered Spaces}
        There are four more topologies a total order can give us.
        \begin{definition}[\textbf{Lower Limit Topology}]
            The lower limit topology on a totally ordered set $(X,\,\leq)$ is
            the topology $\tau$ generated by:
            \begin{equation}
                \mathcal{B}=\{\,[a,\,b)\subseteq{X}\;|\;a,b\in{X}\,\}
            \end{equation}
            That is, the topology generated by half-open intervals
            closed on the left.
        \end{definition}
        \begin{definition}[\textbf{Upper Limit Topology}]
            The upper limit topology on a totally ordered set $(X,\,\leq)$ is
            the topology $\tau$ generated by:
            \begin{equation}
                \mathcal{B}=\{\,(a,\,b]\subseteq{X}\;|\;a,b\in{X}\,\}
            \end{equation}
            That is, the topology generated by half-open intervals
            closed on the right.
        \end{definition}
        \begin{definition}[\textbf{Right Ray Topology}]
            The right ray topology on a totally ordered set $(X,\,\leq)$ is
            the topology $\tau$ generated by:
            \begin{equation}
                \mathcal{B}=\{\,(a,\,\infty)\subseteq{X}\;|\;a\in{X}\,\}
            \end{equation}
            That is, the topology generated by rays that go off to the right.
        \end{definition}
        \begin{definition}[\textbf{Left Ray Topology}]
            The left ray topology on a totally ordered set $(X,\,\leq)$ is
            the topology $\tau$ generated by:
            \begin{equation}
                \mathcal{B}=\{\,(-\infty,\,a)\subseteq{X}\;|\;a\in{X}\,\}
            \end{equation}
            That is, the topology generated by rays that go off to the left.
        \end{definition}
        All of these give us plenty of examples of spaces, but we will be
        particularly concerned with the lower limit topology on $\mathbb{R}$.
        This spaces is so frequently discussed in counterexamples that it is
        given a name (it was the first known example of a normal spaces whose
        product is not normal. We'll get to this next lecture).
        \begin{definition}[\textbf{The Sorgenfrey Line}]
            The Sorgenfrey Line is the topological space
            $(\mathbb{R},\,\tau_{S})$ where $\tau_{S}$ is the lower limit
            topology induced by the standard order $\leq$ on $\mathbb{R}$.
        \end{definition}
        Recall that two topologies on a set $X$ do not need to be comparable.
        The Sorgenfrey line, however, is comparable to the Euclidean line.
        \begin{theorem}
            If $\tau_{\mathbb{R}}$ is the standard topology on $\mathbb{R}$,
            and if $\tau_{S}$ is the Sorgenfrey topology, then
            $\tau_{\mathbb{R}}\subseteq\tau_{S}$.
        \end{theorem}
        \begin{proof}
            It suffices to show that basis elements $(a,\,b)$ in the standard
            topology are open in the Sorgenfrey line. Let
            $a<b$ and $r=\frac{b-a}{2}$. Define
            $\mathcal{U}_{n}$ to be:
            \begin{equation}
                \mathcal{U}_{n}=[a+\frac{r}{n+1},\,b)
            \end{equation}
            Then $\mathcal{U}_{n}\in\tau_{S}$ since the Sorgenfrey topology is
            the lower limit topology on $\mathbb{R}$. But
            $\bigcup_{n}\mathcal{U}_{n}=(a,\,b)$, and the union of open sets
            is open, so $(a,\,b)\in\tau_{S}$. Hence
            $\tau_{\mathbb{R}}\subseteq\tau_{S}$.
        \end{proof}
        \begin{theorem}
            The Sorgenfrey line is Hausdorff.
        \end{theorem}
        \begin{proof}
            Since $(\mathbb{R},\,\tau_{\mathbb{R}})$ is Hausdorff and
            $\tau_{\mathbb{R}}\subseteq\tau_{S}$, we have that
            $(\mathbb{R},\,\tau_{S})$ is Hausdorff.
        \end{proof}
\end{document}
