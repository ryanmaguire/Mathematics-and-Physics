%-----------------------------------LICENSE------------------------------------%
%   This file is part of Mathematics-and-Physics.                              %
%                                                                              %
%   Mathematics-and-Physics is free software: you can redistribute it and/or   %
%   modify it under the terms of the GNU General Public License as             %
%   published by the Free Software Foundation, either version 3 of the         %
%   License, or (at your option) any later version.                            %
%                                                                              %
%   Mathematics-and-Physics is distributed in the hope that it will be useful, %
%   but WITHOUT ANY WARRANTY; without even the implied warranty of             %
%   MERCHANTABILITY or FITNESS FOR A PARTICULAR PURPOSE.  See the              %
%   GNU General Public License for more details.                               %
%                                                                              %
%   You should have received a copy of the GNU General Public License along    %
%   with Mathematics-and-Physics.  If not, see <https://www.gnu.org/licenses/>.%
%------------------------------------------------------------------------------%
\documentclass{article}
\usepackage{graphicx}                           % Needed for figures.
\usepackage{amsmath}                            % Needed for align.
\usepackage{amssymb}                            % Needed for mathbb.
\usepackage{amsthm}                             % For the theorem environment.
\usepackage{hyperref}
\hypersetup{colorlinks=true, linkcolor=blue}

%------------------------Theorem Styles-------------------------%
\theoremstyle{plain}
\newtheorem{theorem}{Theorem}[section]

% Define theorem style for default spacing and normal font.
\newtheoremstyle{normal}
    {\topsep}               % Amount of space above the theorem.
    {\topsep}               % Amount of space below the theorem.
    {}                      % Font used for body of theorem.
    {}                      % Measure of space to indent.
    {\bfseries}             % Font of the header of the theorem.
    {}                      % Punctuation between head and body.
    {.5em}                  % Space after theorem head.
    {}

% Define default environments.
\theoremstyle{normal}
\newtheorem{examplex}{Example}[section]
\newtheorem{definitionx}{Definition}[section]

\newenvironment{example}{%
    \pushQED{\qed}\renewcommand{\qedsymbol}{$\blacksquare$}\examplex%
}{%
    \popQED\endexamplex%
}

\newenvironment{definition}{%
    \pushQED{\qed}\renewcommand{\qedsymbol}{$\blacksquare$}\definitionx%
}{%
    \popQED\enddefinitionx%
}

\title{Point-Set Topology: Lecture 15}
\author{Ryan Maguire}
\date{\today}

% No indent and no paragraph skip.
\setlength{\parindent}{0em}
\setlength{\parskip}{0em}

\begin{document}
    \maketitle
    \section{Induced Equivalence Relation}
        Given a relation $R$ on a set $A$, it is possible for $R$ to be very
        dull. It does not need to be reflexive, symmetric, or transitive. We
        can always transform $R$ into a reflexive relation by adding in
        $aRa$ for all $a\in{A}$. We can then make it symmetric by adding
        $bRa$ for all $a,b\in{A}$ such that $aRb$. Lastly, we can make it
        transitive by enlarging the relation as well. This idea is the
        \textit{induced equivalence relation} from $R$.
        \begin{theorem}
            If $A$ is a set, and if
            $\mathcal{R}\subseteq\mathcal{P}(A\times{A})$ is a non-empty set
            such that for all $R\in\mathcal{R}$ it is true that $R$ is an
            equivalence relation on $A$, then $\bigcap\mathcal{R}$ is an
            equivalence relation on $A$.
        \end{theorem}
        \begin{proof}
            Given $a\in{A}$, since for all $R\in\mathcal{R}$ it is true that
            $R$ is an equivalence relation, we have $aRa$.
            Hence, $(a,\,a)\in\bigcap\mathcal{R}$. That is,
            $a\big(\bigcap\mathcal{R}\big)a$. If $a,b\in{A}$ are such that
            $(a,\,b)\in\bigcap\mathcal{R}$, then for all $R\in\mathcal{R}$ we
            have $aRb$. But since $R$ is an equivalence relation this implies
            $bRa$. So $bRa$ for all $R\in\mathcal{R}$ and therefore
            $(b,\,a)\in\bigcap\mathcal{R}$. Lastly, if $a,b,c\in{A}$ are such
            that $(a,\,b)\in\bigcap\mathcal{R}$ and
            $(b,\,c)\in\bigcap\mathcal{R}$, then for all $R\in\mathcal{R}$ we
            have $aRb$ and $bRc$. But $R$ is an equivalence relation, so then
            $aRc$. But then $(a,\,c)\in\bigcap\mathcal{R}$, so
            $\bigcap\mathcal{R}$ is transitive. Hence, $\bigcap\mathcal{R}$ is
            an equivalence relation.
        \end{proof}
        \begin{theorem}
            If $A$ is a set, if $R$ is a relation on $A$, and if
            $\mathcal{R}$ is the set of all equivalence relations
            $R'$ on $A$ such that $R\subseteq{R}'$, then
            $\bigcap\mathcal{R}$ is an equivalence relation on $A$ such that
            $R\subseteq\bigcap\mathcal{R}$.
        \end{theorem}
        \begin{proof}
            First, $\mathcal{R}$ is non-empty since $A\times{A}$ is an
            equivalence relation on $A$. It is the relation that says $a$ is
            related to $b$ for all $a,b\in{R}$. That is, the relation that says
            everything is related to everything else. But $R$ is a relation on
            $A$, so by definition $R\subseteq{A}\times{A}$. Hence $\mathcal{R}$
            is a non-empty set of equivalence relations on $A$, so
            $\bigcap\mathcal{R}$ is an equivalence relation on $A$ by the
            previous theorem. But for all $R'\in\mathcal{R}$ it is true that
            $R\subseteq{R}'$, so $R\subseteq\bigcap\mathcal{R}$.
        \end{proof}
        \begin{definition}[\textbf{Induced Equivalence Relation}]
            The induced equivalence relation on a set $A$ by a relation $R$ is
            the equivalence relation $\bigcap\mathcal{R}$ where $\mathcal{R}$
            is the set of all equivalence relations $R'$ on $A$ such that
            $R\subseteq{R}'$.
        \end{definition}
        \begin{example}
            Let $A$ be a set and $R=\emptyset$, the empty relation. This
            relation says nothing is related, not even $a\in{A}$ is related to
            itself. The induced equivalence relation is the diagonal
            $\Delta_{A}=\{\,(a,\,a)\;|\;a\in{A}\,\}$. The only thing we need to
            add to make $R$ an equivalence relation is reflexivity.
        \end{example}
        \begin{theorem}
            If $A$ is a set, and if $R$ is an equivalence relation on $A$,
            then the induced equivalence relation $R'$ is equal to $R$.
        \end{theorem}
        \begin{proof}
            Let $\mathcal{R}$ be the set of all equivalence relations $R''$ on
            $A$ such that $R\subseteq{R}''$. But $R$ is an equivalence relation
            on $A$, and $R\subseteq{R}$, so $R\in\mathcal{R}$. Hence
            $\bigcap\mathcal{R}\subseteq{R}$. But also
            $R\subseteq\bigcap\mathcal{R}$. So $R=\bigcap\mathcal{R}$. But
            $R'=\bigcap\mathcal{R}$ since $R'$ is the induced equivalence
            relation, so $R=R'$.
        \end{proof}
        \begin{definition}[\textbf{Induced Equivalence Relation by a Subset}]
            The induced equivalence relation of a subset $A\subseteq{X}$ of a
            set $X$ is the induced equivalence relation $R_{A}$ induced by the
            relation $R$ on $X$ defined by:
            \begin{equation}
                R=\{\,(a,\,b)\in{X}\times{X}\;|
                    \;a\in{A}\textrm{ and }b\in{A}\,\}
            \end{equation}
            That is, the equivalence relation induced by saying that everything
            in $A$ is related to everything else in $A$.
        \end{definition}
    \section{Quotient Topology}
        Given a set $X$ and an equivalence relation $R$ on $X$, we may form
        the quotient set $X/R$ which is the set of all equivalence classes
        of $X$ under $R$. Intuitively, we are taking points in $X$ and
        \textit{gluing} them together in the quotient set. If $X$ had a
        topology, it seems like it should be possible to give a topology to
        the quotient since gluing things together certainly seems like a
        topological operation. This is indeed possible, and quotient spaces are
        very common in topology since they provide a plethora of spaces one
        can ponder and construct. We define the quotient topology via the
        quotient map. In set theory, there is a canonical quotient function
        $q:X\rightarrow{X}/R$ given by $q(x)=[x]$ for all $x\in{X}$, where
        $[x]\in{X}/R$ is the equivalence class of $x$. This is something like
        projecting points $x$ in $X$ to the point in $X/R$ where $x$ was glued
        to, the equivalence class $[x]$. This gluing operation should be
        continuous. We define the quotient topology on $X/R$ via the
        \textit{final topology} on $X/R$ which makes $q$ continuous.
        \begin{definition}[\textbf{Quotient Topology}]
            The quotient topology on the quotient set $X/R$ of a set $X$ under
            an equivalence relation $R$ with respect to a topological space
            $(X,\,\tau)$ is the set $\tau_{X/R}$ defined as the final topology
            with respect to the quotient map $q:X\rightarrow{X}/R$ defined
            by $q(x)=[x]$, and with respect to the topology $\tau$ on $X$.
        \end{definition}
        \begin{theorem}
            If $(X,\,\tau)$ is a topological space, if $R$ is an equivalence
            relation on $X$, and if $\tau_{X/R}$ is the quotient topology on
            $X/R$, then $(X/R,\,\tau_{X/R})$ is a topological space.
        \end{theorem}
        \begin{proof}
            The quotient topology is the final topology with respect to
            $(X,\,\tau)$ and the quotient map $q:X\rightarrow{X}/R$ defined by
            $q(x)=[x]$. But the final topology for any function
            $f:X\rightarrow{X}/R$ with respect to $(X,\,\tau)$ is a topology on
            $X/R$, so in particular $\tau_{X/R}$ is a topology.
        \end{proof}
        Since $\tau_{X/R}$ is the final topology with respect to the quotient
        mapping $q$, a subset $\mathcal{U}\subseteq{X}/R$ is open
        \textit{if and only if} $q^{-1}[\mathcal{U}]$ is open.
        Please note continuity alone is not sufficient enough to say that
        $q^{-1}[\mathcal{U}]$ being open implies $\mathcal{U}$ is open. The
        implication goes one way. If $\mathcal{U}$ is open, and if $q$ is
        continuous, then $q^{-1}[\mathcal{U}]$ is open. For a general continuous
        function $f:X\rightarrow{Y}$ with topologies $\tau_{X}$ and $\tau_{Y}$,
        given $\mathcal{V}\subseteq{Y}$ and $f^{-1}[\mathcal{V}]\in\tau_{X}$, it
        is not necessarily true that we can conclude that
        $\mathcal{V}\in\tau_{Y}$. Let $X=Y=\mathbb{R}$, and
        $\tau_{X}=\tau_{Y}=\tau_{\mathbb{R}}$, the standard topology on
        $\mathbb{R}$. Let $f(x)=1$. Since it is a constant function, it is
        continuous. But $f^{-1}[\{\,1\,\}]=\mathbb{R}$, which is open, however
        $\{\,1\,\}$ is not open in $\mathbb{R}$.
        The quotient map, with the quotient topology, is very special in this
        regard. $\mathcal{U}\subseteq{X}/R$ is open \textit{if and only if}
        $q^{-1}[\mathcal{U}]$ is open in $X$. This fact is used constantly in
        the proofs of various claims about quotient spaces.
        \begin{definition}[\textbf{Saturated Subset}]
            A saturated set with respect to a function $f:X\rightarrow{Y}$
            between sets $X$ and $Y$ is a set $A\subseteq{X}$ such that
            $f^{-1}\big[f[A]\big]=A$.
        \end{definition}
        Not every subset is saturated (unless the function is injective).
        We can always conclude that $A\subseteq{f}^{-1}\big[f[A]\big]$, however.
        \begin{theorem}
            If $X$ and $Y$ are sets, if $f:X\rightarrow{Y}$, and if
            $A\subseteq{X}$, then $A\subseteq{f}^{-1}\big[f[A]\big]$.
        \end{theorem}
        \begin{proof}
            Given $x\in{A}$ it is true that $f(x)\in{f}[A]$ by the definition
            of image. But then $x$ is an element of $X$ such that
            $f(x)\in{f}[A]$, and hence $x\in{f}^{-1}\big[f[A]\big]$ by the
            definition of pre-image. So $A\subseteq{f}^{-1}\big[f[A]\big]$.
        \end{proof}
        \begin{theorem}
            If $X$ and $Y$ are sets, if $f:X\rightarrow{Y}$ is an injective
            function, and if $A\subseteq{X}$, then $A=f^{-1}\big[f[A]\big]$.
        \end{theorem}
        \begin{proof}
            We have proven that $A\subseteq{f}^{-1}\big[f[A]\big]$. Let's go
            the other way. Let $x\in{f}^{-1}\big[f[A]\big]$. Then there is
            a $y\in{f}[A]$ such that $f(x)=y$. But $y\in{f}[A]$, so there is an
            element $x_{0}\in{A}$ such that $f(x_{0})=y$. But $f$ is injective,
            so $x=x_{0}$. Therefore, $x\in{A}$. That is,
            $f^{-1}\big[f[A]\big]\subseteq{A}$, so
            $A=f^{-1}\big[f[A]\big]$.
        \end{proof}
        Lacking injectivity, we can make no such conclusion.
        \begin{example}
            Let $X=Y=\mathbb{R}$, and let $A=\mathbb{R}_{\geq{0}}$. Define
            $f(x)=x^{2}$. Then
            $f^{-1}\big[f[A]\big]=\mathbb{R}$, but $A\ne\mathbb{R}$.
        \end{example}
        \begin{theorem}
            If $X$ and $Y$ are sets, if $f:X\rightarrow{Y}$ is a function,
            and if $B\subseteq{Y}$, then $A=f^{-1}[B]$ is a saturated subset
            of $X$.
        \end{theorem}
        \begin{proof}
            We have proven that $A\subseteq{f}^{-1}\big[f[A]\big]$. Going the
            other way, let $x\in{f}^{-1}\big[f[A]\big]$. Then
            $f(x)\in{f}[A]$ by the definition of pre-image. But
            $f[A]=f\big[f^{-1}[B]\big]$ by the definition of $A$. So if
            $f(x)\in{f}[A]$, then $f(x)\in{f}\big[f^{-1}[B]\big]$. But then
            $f(x)\in{B}$. But if $f(x)\in{B}$, then $x\in{f}^{-1}[B]$. Therefore
            $x\in{A}$, so $f^{-1}\big[f[A]\big]\subseteq{A}$, and hence
            $f^{-1}\big[f[A]\big]=A$. That is, $A$ is saturated.
        \end{proof}
        \begin{theorem}
            If $(X,\,\tau)$ is a topological space, if $R$ is an equivalence
            relation on $X$, and if $\tau_{X/R}$ is the quotient topology,
            then the quotient map $q:X\rightarrow{X}/R$ is a continuous
            surjective function such that for all saturated
            $\mathcal{U}\in\tau$, it is true that $q[\mathcal{U}]\in\tau_{X/R}$.
        \end{theorem}
        \begin{proof}
            Since $q$ is the final topology with respect to $(X,\,\tau)$ and
            $q$, $q$ is continuous. $q$ is also surjective, since given
            $[x]\in{X}/R$ we have $q(x)=[x]$. Lastly, if $\mathcal{U}\in\tau$
            is saturated, then
            $q^{-1}\big[q[\mathcal{U}]\big]=\mathcal{U}$. But then
            $q[\mathcal{U}]$ is a set in $X/R$ such that the pre-image is an
            open subset of $X$, and since $\tau_{X/R}$ is the quotient topology,
            it must be true that $q[\mathcal{U}]$ is open. Hence the image of
            a saturated open set is open.
        \end{proof}
        This theorem motivates the more general idea of a quotient map between
        different topological spaces.
        \begin{definition}[\textbf{Quotient Map}]
            A quotient map from a topological space $(X,\,\tau_{X})$ to a
            topological space $(Y,\,\tau_{Y})$ is a continuous surjective
            function $f:X\rightarrow{Y}$ such that for every saturated
            set $\mathcal{U}\in\tau_{X}$ it is true that
            $f[\mathcal{U}]\in\tau_{Y}$.
        \end{definition}
        \begin{theorem}
            If $(X,\,\tau_{X})$ and $(Y,\,\tau_{Y})$ are topological spaces,
            if $f:X\rightarrow{Y}$ is a quotient map, then there is an
            equivalence relation $R$ on $X$ such that $(Y,\,\tau_{Y})$ is
            homeomorphic to $(X/R,\,\tau_{X/R})$ where $\tau_{X/R}$ is the
            quotient topology.
        \end{theorem}
        \begin{proof}
            Let $R$ be the relation on $X$ defined by $aRb$ if and only if
            $f(a)=f(b)$. $R$ is an equivalence relation. $aRa$ since
            $f(a)=f(a)$. $R$ is symmetric since $aRb$ implies $f(a)=f(b)$,
            and hence $f(b)=f(a)$, so $bRa$. Lastly, $R$ is transitive. If
            $aRb$ and $bRc$, then $f(a)=f(b)$ and $f(b)=f(c)$. By the
            transitivity of equality, $f(a)=f(c)$ and hence $aRc$.
            Define $g:X/R\rightarrow{Y}$ via $g([a])=f(a)$. This is
            well-defined. If $[a]=[b]$, then $aRb$, and hence $f(a)=f(b)$. Thus
            $g([a])=f(a)=f(b)=g([b])$. We now must prove that $g$ is a
            homeomorphism. First, it is bijective. It is injective since if
            $g([a])=g([b])$, then $f(a)=f(b)$, and hence $aRb$, so $[a]=[b]$.
            It is surjective since given $y\in{Y}$, since $f$ is a quotient map
            it is surjective, so there is an $x\in{X}$ such that $f(x)=y$. But
            then $g([x])=f(x)=y$, so $g$ is surjective. Therefore, $g$ is
            bijective. Next, to prove $g$ is a continuous open map. Given
            $\mathcal{V}\in\tau_{Y}$, $g^{-1}[\mathcal{V}]$ is the set of
            all $[x]\in{X}/R$ such that $g([x])\in\mathcal{V}$. But
            $g([x])\in\mathcal{V}$ if and only if $f(x)\in\mathcal{V}$, and
            $f(x)\in\mathcal{V}$ if and only if $x\in{f}^{-1}[\mathcal{V}]$. But
            $f$ is a quotient map, so it is continuous, and hence
            $f^{-1}[\mathcal{V}]$ is open. But
            $f^{-1}[\mathcal{V}]=q^{-1}\big[g^{-1}[\mathcal{V}]\big]$ where
            $q:X\rightarrow{X}/R$ is the quotient map $q(x)=[x]$. But
            $\tau_{X/R}$ is the quotient topology, so if
            $q^{-1}\big[g^{-1}[\mathcal{V}]\big]$ is open, then
            $g^{-1}[\mathcal{V}]$ is open, so $g$ is continuous. Lastly,
            $g$ is an open mapping. Given $\mathcal{U}\in\tau_{X/R}$,
            $q^{-1}[\mathcal{U}]$ is a saturated open subset of $X$. But $f$ is
            a quotient map, so then $f\big[q^{-1}[\mathcal{U}]\big]$ is open.
            But $f\big[q^{-1}[\mathcal{U}]\big]=g[\mathcal{U}]$, so
            $g[\mathcal{U}]$ is open. Therefore $g$ is a homeomorphism.
        \end{proof}
        Quotient spaces $(Y,\,\tau_{Y})$ are just topological spaces that can
        be thought of as quotients by some equivalence relation of another
        topological $(X,\,\tau_{X})$. This mimics how topological embeddings
        give us topological spaces that can be thought of as subspaces of some
        other topological space.
    \section{Quotient of a Subspace}
        The most common way to create a quotient of a topological space
        $(X,\,\tau)$ is to take a subset $A\subseteq{X}$, give the equivalence
        relation $R_{A}$ on $X$ that is induced by $A$, and consider
        $X/R_{A}$ with the quotient topology $\tau_{X/R_{A}}$. The notation for
        this is quite unfortunate, we write $X/A$ and $\tau_{X/A}$. The reason
        this is unfortunate is because we now have competing notation with
        algebraists. Given $\mathbb{R}$ with the standard topology, and
        $\mathbb{Z}\subseteq\mathbb{R}$, algebraists will tell you that
        $\mathbb{R}/\mathbb{Z}$ is a circle. Topologists will tell you this is
        actually infinitely many circles. The reason for the competing notions
        is that, to algebraists, $\mathbb{R}$ is a \textit{group},
        $\mathbb{Z}\subseteq\mathbb{R}$ is a \textit{subgroup}, and
        $\mathbb{R}/\mathbb{Z}$ is a \textit{quotient group}, which is indeed
        the same thing as a circle, as far as groups are concerned. For
        topologists, $\mathbb{R}$ is a \textit{topological space},
        $\mathbb{Z}\subseteq\mathbb{R}$ is a \textit{topological subspace},
        and $\mathbb{R}/\mathbb{Z}$ is a \textit{quotient space}, which we will
        see later looks like infinitely many circles all touching at one point.
        \begin{figure}
            \centering
            \includegraphics{../../../images/quotient_interval_to_circle.pdf}
            \caption{Quotient of an Interval to a Circle}
            \label{fig:quotient_interval_to_circle}
        \end{figure}
        \begin{example}
            Let $X=[0,\,1]$, $\tau_{X}$ the subspace topology from $\mathbb{R}$,
            and $A=\{\,0,\,1\,\}$. The quotient space $X/A$ is formed by taking
            the endpoints of $X$ and gluing them together. The result is a
            circle. A visual for is given in
            Fig.~\ref{fig:quotient_interval_to_circle}. Note, the quotient space
            is not exactly a circle, it is just homeomorphic to it. The points
            in $\mathbb{S}^{1}$ are points in the plane $\mathbb{R}^{2}$. Points
            in $X/A$ are \textit{equivalence classes of} $X$, which means points
            in $X/A$ are subsets $[x]\subseteq{X}$ for $x\in{X}$. The
            homeomorphism goes as follows. Define:
            \begin{equation}
                f([t])=\big(\cos(2\pi{t}),\,\sin(2\pi{t})\big)
            \end{equation}
            $[0]=[1]$ since the equivalence relation glues $0$ to $1$, meaning
            this function is indeed bijective, and it is also continuous with
            a continuous inverse.
        \end{example}
        \begin{example}
            Let $X\subseteq\mathbb{R}^{2}$ be the closed unit disk:
            \begin{equation}
                X=\{\,\mathbf{x}\in\mathbb{R}^{2}\;|\;
                    ||\mathbf{x}||_{2}\leq{1}\,\}
            \end{equation}
            This includes the \textit{boundary}, the points that are precisely
            1 unit away from the origin. Let $A\subseteq{X}$ be the unit
            circle, $A=\mathbb{S}^{1}$:
            \begin{equation}
                A=\{\,\mathbf{x}\in\mathbb{R}^{2}\;|\;
                    ||\mathbf{x}||_{2}=1\,\}
            \end{equation}
            Equip both $X$ and $A$ with the subspace topologies from the
            Euclidean plane. The quotient space $X/A$ is a sphere. We are
            taking the points on the boundary and gluing them to a single point.
            This is shown in Fig.~\ref{fig:quotient_disk_to_sphere}. Again,
            $X/A$ is not exactly the sphere in $\mathbb{R}^{3}$, it is just
            homeomorphic to it. The sphere contains points in $\mathbb{R}^{3}$
            whereas $X/A$ contains subsets of $X$, the equivalence classes of
            $X$ under the relation that glues $A$ to a single point.
            Topologically, however, there is little point in differentiating
            between $X/A$ and the sphere $\mathbb{S}^{2}$ since they are
            homeomorphic.
        \end{example}
        \begin{figure}
            \centering
            \resizebox{\textwidth}{!}{%
                \includegraphics{../../../images/quotient_disk_to_sphere.pdf}%
            }
            \caption{Quotient of a Disk to a Sphere}
            \label{fig:quotient_disk_to_sphere}
        \end{figure}
        Consider the square $[0,\,1]\times[0,\,1]$. Identity
        $(x,\,0)$ with $(x,\,1)$ for all $x\in[0,\,1]$, and also
        $(0,\,y)$ with $(1,\,y)$ for all $y\in[0,\,1]$. This identification
        is shown in Fig.~\ref{fig:square_representation_of_torus}. The quotient
        of the square under this identication is a torus, which is a hollow
        donut.
        \begin{figure}
            \centering
            \includegraphics{../../../images/square_representation_of_torus.pdf}
            \caption{Square Representation of a Torus}
            \label{fig:square_representation_of_torus}
        \end{figure}
        \par\hfill\par
        By gluing the top edge to the bottom edge we obtain a cylinder. The
        left and right edges become circles in the process, and we now have to
        glue these circles together with matching orientations. By doing this
        we obtain a torus. This is shown in
        Fig.~\ref{fig:square_to_torus}.
        \begin{figure}
            \centering
            \includegraphics{../../../images/square_to_torus.pdf}
            \caption{Quotient of a Square to a Torus}
            \label{fig:square_to_torus}
        \end{figure}
        \begin{figure}
            \centering
            \includegraphics{../../../images/square_torus_pacman.pdf}
            \caption{Pac-Man's World}
            \label{fig:square_torus_pacman}
        \end{figure}
        \par\hfill\par
        The torus is the world the Pac-Man lives on
        (See Fig.~\ref{fig:square_torus_pacman}).
        Does Pac-Man see his own back? We can tile the plane with squares, so
        let's take a copy of Fig.~\ref{fig:square_representation_of_torus}
        and use it to cover the page, ensuring that the orientation of the
        arrows match when we glue adjacent squares together. The result
        is Fig.~\ref{fig:plane_torus_pacman}. In this figure the two Pac-Men are
        given different colors so we can differentiate them. This idea of tiling
        the plane with the square representation of the torus will be very
        important later. It shows that the plane is the
        \textit{universal cover} of the torus, a concept that is fundamental
        to algebraic topology and the theory of manifolds.
        \begin{figure}
            \centering
            \resizebox{\textwidth}{!}{%
                \includegraphics{../../../images/plane_torus_pacman.pdf}%
            }
            \caption{Tiling the Plane with Pac-Man's World}
            \label{fig:plane_torus_pacman}
        \end{figure}
        \par\hfill\par
        Now, let's do a different identification on the square. Let's identify
        $(0,\,y)$ with $(1,\,1-y)$. That is, we are taking the square and
        gluing the left and right sides together, but with a twist. The
        result is the M\"{o}bius strip, this construction is shown in
        Fig.~\ref{fig:square_to_mobius_strip}. A 3D drawing is shown in
        Fig.~\ref{fig:mobius_strip}.
        \begin{figure}
            \centering
            \includegraphics{../../../images/square_to_mobius_strip.pdf}
            \caption{Square Representation of the M\"{o}bius Strip}
            \label{fig:square_to_mobius_strip}
        \end{figure}
        \begin{figure}
            \centering
            \includegraphics{../../../images/mobius_strip.pdf}
            \caption{A M\"{o}bius Strip}
            \label{fig:mobius_strip}
        \end{figure}
        \par\hfill\par
        And now let's go nuts. Let identify
        $(x,\,0)$ with $(1-x,\,1)$, like in the M\"{o}bius band, but also
        $(0,\,y)$ with $(1,\,y)$, like in the torus. The result is something
        that's like a torus, but also like a M\"{o}bius band. The square
        representation is given in Fig.~\ref{fig:square_to_klein_bottle}.
        This object is called the \textit{Klein bottle}.
        \begin{figure}
            \centering
            \includegraphics{../../../images/square_to_klein_bottle.pdf}
            \caption{Square Representation of a Klein Bottle}
            \label{fig:square_to_klein_bottle}
        \end{figure}
        \par\hfill\par
        Let's imagine Pac-Man lived on a Klein bottle, instead of a torus. This
        is shown in Fig.~\ref{fig:square_klein_bottle_pacman}. We know that
        Pac-Man will see his back due to the torus-like identification made with
        the left and right edges, but will Pac-Man also see his face? If we
        play the same game as before, taking copies of
        Fig.~\ref{fig:square_klein_bottle_pacman} and attach them to tile the
        plane in a consistent manner, we can see that Pac-Man does indeed see
        this face. This is shown in Fig.~\ref{fig:plane_klein_bottle_pacman}.
        \begin{figure}
            \centering
            \includegraphics{../../../images/square_klein_bottle_pacman.pdf}
            \caption{Pac-Man in a Klein Bottle}
            \label{fig:square_klein_bottle_pacman}
        \end{figure}
        \begin{figure}
            \centering
            \resizebox{\textwidth}{!}{%
                \includegraphics{../../../images/plane_klein_bottle_pacman.pdf}%
            }
            \caption{Tiling the Plane with the Klein Bottle}
            \label{fig:plane_klein_bottle_pacman}
        \end{figure}
        \par\hfill\par
        This shows the plane is also the universal cover of the Klein bottle
        as well. This idea actually allows us to \textit{immerse} (which is a
        weaker notion than embed) the Klein bottle into $\mathbb{R}^{3}$.
        It is impossible to embed the Klein bottle into $\mathbb{R}^{3}$ since
        you will need the object to pass through itself, which is not an
        embedding. This is given in Fig.~\ref{fig:klein_bottle}.
        \begin{figure}
            \centering
            \includegraphics{../../../images/klein_bottle.pdf}
            \caption{The Klein Bottle in $\mathbb{R}^{3}$}
            \label{fig:klein_bottle}
        \end{figure}
        \par\hfill\par
        Let's end with the \textit{real projective plane}. This is denoted
        $\mathbb{RP}^{2}$. Take the square and identify
        $(0,\,y)$ with $(1,\,1-y)$, and $(x,\,0)$ with $(1-x,\,1)$. That is,
        do the M\"{o}bius twist for both top and bottom, and left and right.
        This is shown in Fig.~\ref{fig:square_to_real_proj_plane}.
        \begin{figure}
            \centering
            \includegraphics{../../../images/square_to_real_proj_plane}
            \caption{Square Representation of $\mathbb{RP}^{2}$}
            \label{fig:square_to_real_proj_plane}
        \end{figure}
        Can we tile the plane with this object? Let's try.
        Fig.~\ref{fig:attempted_tile_of_plane_with_rp2} seems to do it, but
        there's a cheat. We are not just using copies of
        Fig.~\ref{fig:square_to_real_proj_plane}, rather we are using copies of
        Fig.~\ref{fig:square_to_real_proj_plane} and its mirror. This is why
        the horizontal arrows converge to the center, and the vertical arrows
        diverge. Using only copies of
        Fig.~\ref{fig:square_to_real_proj_plane} (no mirrors), it is not
        possible to tile the plane in a way that the arrows match. The reason
        being that the real projective plane does \textbf{not} have the plane
        as its universal cover. The universal cover of $\mathbb{RP}^{2}$ is
        $\mathbb{S}^{2}$, the sphere. This can be described using a
        quotient. On $\mathbb{S}^{2}$, define
        $\mathbf{x}R\mathbf{y}$ if and only if $\mathbf{y}=-\mathbf{x}$ or
        $\mathbf{y}=\mathbf{x}$. This is
        the \textit{antepodal} identification. We are gluing opposite ends of
        the sphere together. For example, the north pole is glued to the south
        pole. The result of the quotient $\mathbb{S}^{2}/R$ is the real
        projective plane.
        \begin{figure}
            \centering
            \includegraphics{../../../images/attempted_tile_of_plane_with_rp2.pdf}
            \caption{Fake Tiling of the Plane with the $\mathbb{RP}^{2}$ Square}
            \label{fig:attempted_tile_of_plane_with_rp2}
        \end{figure}
        \par\hfill\par
        Like the Klein bottle, it is not possible to embed
        $\mathbb{RP}^{2}$ into $\mathbb{R}^{3}$. We can draw $\mathbb{RP}^{2}$
        if we allow the object to intersect itself. There are many ways to
        do this, but two nice drawings are given by the so-called
        \textit{cross-cap} and the
        \textit{Bryant-Kusner parameterization}. These are given in
        Figs.~\ref{fig:real_proj_plane_cross_cap_001} and
        \ref{fig:boy_surface_bryant_kusner_parameterization},
        respectively.
        \begin{figure}
            \centering
            \includegraphics{../../../images/real_proj_plane_cross_cap_001.pdf}
            \caption{The Cross-Cap $\mathbb{RP}^{2}$}
            \label{fig:real_proj_plane_cross_cap_001}
        \end{figure}
        \begin{figure}
            \centering
            \includegraphics{../../../images/boy_surface_bryant_kusner_parameterization.pdf}
            \caption{Bryan-Kusner Parameterization of $\mathbb{RP}^{2}$}
            \label{fig:boy_surface_bryant_kusner_parameterization}
        \end{figure}
    \section{Properties of Quotients}
        Most topological properties are \textbf{not} preserved by quotients.
        For example, just because $(X,\,\tau)$ is Hausdorff, doesn't mean all of
        it's quotient spaces are. The easiest example to describe is the
        \textit{bug-eyed line}, also known as the line with two origins.
        Take $X\subseteq\mathbb{R}^{2}$ to be:
        \begin{figure}
            \centering
            \includegraphics{../../../images/bug_eyed_line_001.pdf}
            \caption{The Bug-Eyed Line}
            \label{fig:bug_eyed_line_001}
        \end{figure}
        \begin{equation}
            X=\{\,(x,\,y)\in\mathbb{R}^{2}\;|\;y=-1\textrm{ or }y=1\,\}
        \end{equation}
        Equip this with the subspace topology from $\mathbb{R}^{2}$.
        Define $R$ to be the equivalence relation induced by identifying
        $(x,\,-1)$ with $(x,\,1)$ for all $x\ne{0}$. Do not identify
        $(0,\,-1)$ and $(0,\,1)$ together, keep them separate. Consider the
        quotient space $X/R$. First note that since $X$ is a subspace of
        $\mathbb{R}^{2}$, which is Hausdorff, $(X,\,\tau_{X})$ is
        Hausdorff as well ($\tau_{X}$ being the subspace topology). The
        visual is given in Fig.~\ref{fig:bug_eyed_line_001}. The top part
        is $X$, the center is $X/R$, and the bottom is how we intuitively try
        to think of $X/R$, though realizing the middle picture is slightly
        more accurate. What do open subsets around the two origins look like?
        A subset of the bug-eyed line is open if and only if the pre-image of
        the set is an open subset of the subspace $X\subseteq\mathbb{R}^{2}$
        via the quotient map $q:X\rightarrow{X}/R$.
        Using this we see that open sets can \textit{look like} open intervals.
        In particular, if we look at the \textit{top} origin, we can put an
        open interval around it that does not include the bottom origin.
        Similarly we can put an open interval around the
        \textit{bottom} interval that does not include the top. This is done
        in Fig.~\ref{fig:bug_eyed_line_002}. The open sets
        $\mathcal{U}$ and $\mathcal{V}$ in this figure help show that
        the bug-eyed line is a Fr\'{e}chet topological space, but it is not
        Hausdorff. Any open set $\mathcal{U}$ that contains the top origin must
        overlap with any open set $\mathcal{V}$ that contains the bottom
        origin (again, see Fig.~\ref{fig:bug_eyed_line_002}). Quotients of
        Hausdorff spaces do not need to be Hausdorff.
        \begin{figure}
            \centering
            \includegraphics{../../../images/bug_eyed_line_002.pdf}
            \caption{Open Subsets in the Bug-Eyed Line}
            \label{fig:bug_eyed_line_002}
        \end{figure}
        \par\hfill\par
        Quotients do no need to preserve first or second-countable, either.
        Give $\mathbb{R}$ the standard topology, and consider the quotient space
        $\mathbb{R}/\mathbb{Z}$. This is \textbf{not} the same as the quotient
        group in abstract algebra, where $\mathbb{R}/\mathbb{Z}$ is just a
        circle, this is a topological quotient. We are taking all of the
        integers and gluing them to $0$. The result, intuitively, is infinitely
        many circles that are all touching at $0$. $\mathbb{R}$ is
        second-countable, and hence first-countable, but $\mathbb{R}/\mathbb{Z}$
        is neither. To show this, I'll demonstrate that $[0]$, the equivalence
        class of $0$, has no countable neighborhood basis. For let
        $\tau_{\mathbb{R}}$ be the standard topology and
        $\tau_{\mathbb{R}/\mathbb{Z}}$ the quotient topology on
        $\mathbb{R}/\mathbb{Z}$, and let
        $\mathcal{B}\subseteq\tau_{\mathbb{R}/\mathbb{Z}}$ be any countable
        collection such that $[0]\in\mathcal{U}$ for all
        $\mathcal{U}\in\mathcal{B}$. Since $\mathcal{B}$ is countable, there
        is a surjection $\mathcal{U}:\mathbb{N}\rightarrow\mathcal{B}$ so
        that we may list the elements of $\mathcal{B}$ as:
        \begin{equation}
            \mathcal{B}=
            \{\,\mathcal{U}_{0},\,\dots,\,\mathcal{U}_{n},\,\dots\,\}
        \end{equation}
        Let $q:\mathbb{R}\rightarrow\mathbb{R}/\mathbb{Z}$ be the quotient
        map $q(x)=[x]$. Since each $\mathcal{U}_{n}$ is open,
        $q^{-1}[\mathcal{U}_{n}]$ is open. But $[0]\in\mathcal{U}_{0}$, so
        $0\in{q}^{-1}[\mathcal{U}_{0}]$. But since $q^{-1}[\mathcal{U}_{0}]$ is
        open, there is an $0<\varepsilon_{0}<1/2$ such that
        $|y|<\varepsilon_{0}$ implies $y\in{q}^{-1}[\mathcal{U}_{0}]$. Let
        $\mathcal{V}_{1}$ be the $\varepsilon_{0}/2$ ball centered at 0. Now,
        since $[0]\in\mathcal{U}_{1}$ and $[0]=[1]$, we have that
        $[1]\in\mathcal{U}_{1}$. But then $1\in{q}^{-1}[\mathcal{U}_{1}]$. Since
        $1\in{q}^{-1}[\mathcal{U}_{1}]$ and
        $q^{-1}[\mathcal{U}_{1}]$ is open, there is a $0<\varepsilon_{1}<1/2$
        such that $|1-y|<\varepsilon_{1}$ implies
        $y\in{q}^{-1}[\mathcal{U}_{1}]$. Let $\mathcal{V}_{1}$ be the
        $\varepsilon_{1}/2$ ball centered at $1$. Inductively, since
        $[0]=[n]$ and $[0]\in\mathcal{U}_{n}$, we have
        $[n]\in\mathcal{U}_{n}$ for all $n\in\mathbb{N}$. But then
        $n\in{q}^{-1}[\mathcal{U}_{n}]$. But $q^{-1}[\mathcal{U}_{n}]$ is open,
        so there is a $0<\varepsilon_{n}<1/2$ such that $|y-n|<\varepsilon_{n}$
        implies $y\in{q}^{-1}[\mathcal{U}_{n}]$. Let
        $\mathcal{V}_{n}$ be the $\varepsilon_{n}/2$ ball centered about $n$.
        Let
        $\mathcal{W}=\bigcup_{n=0}^{\infty}\mathcal{V}_{n}\cup(-\infty,\,-1/2)$. Since
        $\mathcal{W}$ is the union of open subsets of $\mathbb{R}$, it is open.
        This set is also saturated, so
        $q[\mathcal{W}]\subseteq\mathbb{R}/\mathbb{Z}$ is open. By construction
        there is no $\mathcal{U}_{n}\in\mathcal{B}$ such that
        $\mathcal{U}_{n}\subseteq{q}[\mathcal{W}]$, even though
        $[0]\in{q}[\mathcal{W}]$. Hence $\mathcal{B}$ is not a neighborhood
        basis for $[0]$ and
        $(\mathbb{R}/\mathbb{Z},\,\tau_{\mathbb{R}/\mathbb{Z}})$ is not
        first-countable, and hence not second-countable either.
        \par\hfill\par
        The sequential property \textit{is} preserved by quotients.
        \begin{theorem}
            If $(X,\,\tau)$ is a sequential topological space, if
            $R$ is an equivalence relation on $X$, and if
            $\tau_{X/R}$ is the quotient topology on $X/R$, then
            $(X/R,\,\tau_{X/R})$ is sequential.
        \end{theorem}
        \begin{proof}
            Suppose not. Then there is a sequentially open subset
            $\mathcal{U}\subseteq{X}/R$ that is not open. But $q$ is a
            quotient map, so if $\mathcal{U}$ is not open, then
            $q^{-1}[\mathcal{U}]$ is not open. But $(X,\,\tau)$ is sequential,
            so if $q^{-1}[\mathcal{U}]$ is not open, then it is not
            sequentially open. But then there is a point
            $x\in{q}^{-1}[\mathcal{U}]$ and a sequence
            $a:\mathbb{N}\rightarrow{X}$ such that $a_{n}\rightarrow{x}$ but
            for all $N\in\mathbb{N}$ there is an $n\in\mathbb{N}$ with $n>N$
            and $a_{n}\notin{q}^{-1}[\mathcal{U}]$. But since
            $a_{n}\rightarrow{x}$ and $q$ is continuous, we have
            $q(a_{n})\rightarrow{q}(x)$. But $q(x)\in\mathcal{U}$ and
            $\mathcal{U}$ is sequentially open, so there is an
            $N\in\mathbb{N}$ such that for all $n\in\mathbb{N}$ with $n>N$
            we have $q(a_{n})\in\mathcal{U}$. But the for all $n>N$ we have
            $a_{n}\in{q}^{-1}[\mathcal{U}]$, a contradiction. So
            $\mathcal{U}$ is open, and $(X/R,\,\tau_{X/R})$ is a
            sequential topological space.
        \end{proof}
        Two more vital properties are preserved by quotients, but we haven't
        gotten to them yet. The quotient of a \textit{connected} space is still
        connected, and the quotient of a \textit{compact} is still compact.
        We'll discuss both ideas for topological spaces in due time.
\end{document}
