%-----------------------------------LICENSE------------------------------------%
%   This file is part of Mathematics-and-Physics.                              %
%                                                                              %
%   Mathematics-and-Physics is free software: you can redistribute it and/or   %
%   modify it under the terms of the GNU General Public License as             %
%   published by the Free Software Foundation, either version 3 of the         %
%   License, or (at your option) any later version.                            %
%                                                                              %
%   Mathematics-and-Physics is distributed in the hope that it will be useful, %
%   but WITHOUT ANY WARRANTY; without even the implied warranty of             %
%   MERCHANTABILITY or FITNESS FOR A PARTICULAR PURPOSE.  See the              %
%   GNU General Public License for more details.                               %
%                                                                              %
%   You should have received a copy of the GNU General Public License along    %
%   with Mathematics-and-Physics.  If not, see <https://www.gnu.org/licenses/>.%
%------------------------------------------------------------------------------%
\documentclass{article}
\usepackage{graphicx}                           % Needed for figures.
\usepackage{amsmath}                            % Needed for align.
\usepackage{amssymb}                            % Needed for mathbb.
\usepackage{amsthm}                             % For the theorem environment.
\usepackage{tabularx, booktabs}
\usepackage{hyperref}
\hypersetup{colorlinks=true, linkcolor=blue}

%------------------------Theorem Styles-------------------------%
\theoremstyle{plain}
\newtheorem{theorem}{Theorem}[section]

% Define theorem style for default spacing and normal font.
\newtheoremstyle{normal}
    {\topsep}               % Amount of space above the theorem.
    {\topsep}               % Amount of space below the theorem.
    {}                      % Font used for body of theorem.
    {}                      % Measure of space to indent.
    {\bfseries}             % Font of the header of the theorem.
    {}                      % Punctuation between head and body.
    {.5em}                  % Space after theorem head.
    {}

% Define default environments.
\theoremstyle{normal}
\newtheorem{examplex}{Example}[section]
\newtheorem{definitionx}{Definition}[section]

\newenvironment{example}{%
    \pushQED{\qed}\renewcommand{\qedsymbol}{$\blacksquare$}\examplex%
}{%
    \popQED\endexamplex%
}

\newenvironment{definition}{%
    \pushQED{\qed}\renewcommand{\qedsymbol}{$\blacksquare$}\definitionx%
}{%
    \popQED\enddefinitionx%
}

\title{Point-Set Topology: Lecture 18}
\author{Ryan Maguire}
\date{\today}

% No indent and no paragraph skip.
\setlength{\parindent}{0em}
\setlength{\parskip}{0em}

\begin{document}
    \maketitle
    \section{Regular and Normal Spaces}
        The Hausdorff condition is a very mild one and most spaces you'll
        encounter are Hausdorff. Metric spaces have far stronger separation
        properties, and these ideas are useful in the general topological
        setting as well. The new ideas are
        \textit{regular}, \textit{normal}, \textit{completely Hausdorff},
        \textit{completely regular}, \textit{completely normal}, and
        \textit{perfectly normal}. In this section we'll discuss all of these
        ideas, show some relations between them, and draw some pictures.
        \begin{definition}[\textbf{Regular Topological Spaces}]
            A regular topological space is a topological space $(X,\,\tau)$ such
            that for all $x\in{X}$ and for all closed subsets
            $\mathcal{C}\subseteq{X}$ such that $x\not\in\mathcal{C}$, there
            are open sets $\mathcal{U},\mathcal{V}\in\tau$ such that
            $x\in\mathcal{U}$, $\mathcal{C}\subseteq\mathcal{V}$, and
            $\mathcal{U}\cap\mathcal{V}=\emptyset$.
        \end{definition}
        \begin{figure}
            \centering
            \includegraphics{../../../images/regular_condition_001.pdf}
            \caption{Regular Topological Space}
            \label{fig:regular_condition_001}
        \end{figure}
        \textbf{Note:} in analysis authors usually define regular to mean
        \textit{regular plus Fr\'{e}chet}. That is, these authors state that
        a regular space is one where all singleton sets $\{\,x\,\}$ are closed,
        and such that all $x\in{X}$ and closed $\mathcal{C}\subseteq{X}$ can
        be separated by disjoint open sets.
        \textbf{We are not adopting this definition}. While many regular spaces
        that are studied happen to also be Hausdorff, there are also regular
        spaces that are not. If the space is Kolmogorov, however, then
        regular implies Fr\'{e}chet. From many weeks ago, a Kolmogorov space
        is a topological space $(X,\,\tau)$ such that for all $x,y\in{X}$ with
        $x\ne{y}$, there is an open set $\mathcal{U}\in\tau$ such that either
        $x\in\mathcal{U}$ and $y\not\in\mathcal{U}$, or
        $x\not\in\mathcal{U}$ and $y\in\mathcal{U}$.
        \begin{theorem}
            If $(X,\,\tau)$ is a topological space, then it is a Kolmogorov
            space if and only if for all $x,y\in{X}$ with $x\ne{y}$ there
            exists $\mathcal{C}\subseteq{X}$ that is closed such that either
            $x\in\mathcal{C}$ and $y\notin\mathcal{C}$, or
            $x\notin\mathcal{C}$ and $y\in\mathcal{C}$.
        \end{theorem}
        \begin{proof}
            Let $x,y\in{X}$ and $x\ne{y}$. Suppose $(X,\,\tau)$ is a Kolmogorov
            topological space. Then there is an open set $\mathcal{U}\in\tau$
            such that either $x\in\mathcal{U}$ and $y\notin\mathcal{U}$, or
            $x\notin\mathcal{U}$ and $y\in\mathcal{U}$. But then
            $\mathcal{C}=X\setminus\mathcal{U}$ is a closed set such that either
            $x\notin\mathcal{C}$ and $y\in\mathcal{C}$, or
            $x\in\mathcal{C}$ and $y\notin\mathcal{C}$. Going the other way,
            suppose $(X,\,\tau)$ is such that for all $x,y\in{X}$ with
            $x\ne{y}$ there is a closed set $\mathcal{C}\subseteq{X}$ such
            that either $x\in\mathcal{C}$ and $y\notin\mathcal{C}$, or
            $x\notin\mathcal{C}$ and $y\in\mathcal{C}$. Then
            $\mathcal{U}=X\setminus\mathcal{C}$ is an open set such that either
            $x\notin\mathcal{U}$ and $y\in\mathcal{U}$, or
            $x\in\mathcal{U}$ and $y\notin\mathcal{U}$. Hence,
            $(X,\,\tau)$ is a Kolmogorov topological space.
        \end{proof}
        \begin{theorem}
            If $(X,\,\tau)$ is a regular Kolmogorov topological space, then
            it is a Hausdorff topological space.
        \end{theorem}
        \begin{proof}
            Let $x,y\in{X}$, $x\ne{y}$. Since $(X,\,\tau)$ is
            Kolmogorov, there is a closed set $\mathcal{C}\subseteq{X}$ such
            that either $x\in\mathcal{C}$ and $y\notin\mathcal{C}$, or
            $x\notin\mathcal{C}$ and $y\in\mathcal{C}$. Suppose
            $x\notin\mathcal{C}$ and $y\in\mathcal{C}$, the proof is symmetric
            either way. But then $x\in{X}$ and $\mathcal{C}\subseteq{X}$ is a
            closed set such that $x\notin\mathcal{C}$. But $(X,\,\tau)$ is
            regular, so there are open sets $\mathcal{U},\,\mathcal{V}$ such
            that $x\in\mathcal{U}$, $\mathcal{C}\subseteq\mathcal{V}$, and
            $\mathcal{U}\cap\mathcal{V}=\emptyset$. But
            $y\in\mathcal{C}$, so $y\in\mathcal{V}$. But then $\mathcal{U}$
            and $\mathcal{V}$ are disjoint open sets separating $x$ and $y$.
            Hence, $(X,\,\tau)$ is Hausdorff.
        \end{proof}
        \begin{theorem}
            If $(X,\,\tau)$ is a regular Kolmogorov topological space, then
            it is a Fr\'{e}chet topological space.
        \end{theorem}
        \begin{proof}
            Since regular Kolmogorov spaces are Hausdorff, and Hausdorff spaces
            are Fr\'{e}chet, we have that $(X,\,\tau)$ is a Fr\'{e}chet
            topological space.
        \end{proof}
        \begin{theorem}
            If $(X,\,\tau)$ is a regular Fr\'{e}chet topological space, then
            it is Hausdorff.
        \end{theorem}
        \begin{proof}
            Fr\'{e}chet implies Kolmogorov, so $(X,\,\tau)$ is a regular
            Kolmogorov space, which is therefore Hausdorff.
        \end{proof}
        \begin{example}
            The Kolmogorov property is a very mild one, which justifies
            some authors including it in the definition of regular. But, as
            we've defined, regular does not imply Hausdorff by itself.
            The set $\mathbb{Z}_{2}=\{\,0,\,1\,\}$ with the indiscrete
            topology $\tau=\{\,\emptyset,\,\mathbb{Z}_{2}\,\}$ is not Hausdorff,
            not Fr\'{e}chet, and not Kolmogorov, but it is regular.
        \end{example}
        \begin{example}[\textbf{The Double Pointed Reals}]
            The double pointed real space is the topological space
            $\mathbb{R}\times\mathbb{Z}_{2}$, where $\mathbb{R}$ carries the
            standard Euclidean topology, and $\mathbb{Z}_{2}$ carries the
            indiscrete topology. Intuitively, for every real number
            $r\in\mathbb{R}$ you have another redundant copy $r'$ that is
            topologically indistinguishable from $r$, even though $r$ and $r'$
            are techinically different. This space is regular, but it is
            not Hausdorff.
        \end{example}
        \begin{theorem}
            If $(X,\,\tau)$ is a topological space, then it is regular if and
            only if for all $x\in{X}$ and $\mathcal{U}\in\tau$ such that
            $x\in\mathcal{U}$, there is an open set $\mathcal{V}\in\tau$
            such that $x\in\mathcal{V}$ and
            $\textrm{Cl}_{\tau}(\mathcal{V})\subseteq\mathcal{U}$.
        \end{theorem}
        \begin{proof}
            Suppose $(X,\,\tau)$ is regular and let $x\in{X}$ and
            $\mathcal{U}\in\tau$ be such that $x\in\mathcal{U}$. Since
            $\mathcal{U}$ is open, $X\setminus\mathcal{U}$ is closed. But
            $x\notin{X}\setminus\mathcal{U}$, so since $(X,\,\tau)$ is regular
            there exists $\mathcal{V},\mathcal{W}\in\tau$ such that
            $x\in\mathcal{V}$, $X\setminus\mathcal{U}\subseteq\mathcal{W}$, and
            $\mathcal{V}\cap\mathcal{W}=\emptyset$. Now to prove that
            $\textrm{Cl}_{\tau}(\mathcal{V})\subseteq\mathcal{U}$. Since
            $\mathcal{W}$ is open and
            $X\setminus\mathcal{U}\subseteq\mathcal{W}$, we have that
            $X\setminus\mathcal{W}$ is closed and
            $X\setminus\mathcal{W}\subseteq\mathcal{U}$. But, since
            $\mathcal{V}\cap\mathcal{W}=\emptyset$, we have that
            $\mathcal{V}\subseteq{X}\setminus\mathcal{W}$. Hence
            $X\setminus\mathcal{W}$ is a closed set that contains $\mathcal{V}$
            and sits inside of $\mathcal{U}$. But then
            $\textrm{Cl}_{\tau}(\mathcal{V})\subseteq{X}\setminus\mathcal{W}$,
            and hence $\textrm{Cl}_{\tau}(\mathcal{V})\subseteq\mathcal{U}$.
            Now, the other direction. Suppose for all
            $x\in{X}$, $\mathcal{U}\in\tau$ such that $x\in\mathcal{U}$, there
            is a $\mathcal{V}\in\tau$ with $x\in\mathcal{V}$ and
            $\textrm{Cl}_{\tau}(\mathcal{V})\subseteq\mathcal{U}$. Let
            $x\in{X}$ and $\mathcal{C}\subseteq{X}$ be closed and such that
            $x\notin\mathcal{C}$. Since $\mathcal{C}$ is closed,
            $X\setminus\mathcal{C}$ is open. But then there is a
            $\mathcal{U}\in\tau$ such that $x\in\mathcal{U}$ and
            $\textrm{Cl}_{\tau}(\mathcal{U})\subseteq{X}\setminus\mathcal{C}$.
            Let $\mathcal{V}=X\setminus\textrm{Cl}_{\tau}(\mathcal{U})$. Then
            $\mathcal{V}$ is open since it is the complement of a closed set.
            But by definition $\mathcal{C}\subseteq\mathcal{V}$ and
            $\mathcal{U}\cap\mathcal{V}=\emptyset$. Hence $\mathcal{U}$ and
            $\mathcal{V}$ are disjoint open sets that separate
            $x$ and $\mathcal{C}$, so $(X,\,\tau)$ is regular.
        \end{proof}
        \begin{definition}[\textbf{Normal Topological Space}]
            A normal topological space is a topological space $(X,\,\tau)$ such
            that for all disjoint closed subsets
            $\mathcal{C},\mathcal{D}\subseteq{X}$, there are open sets
            $\mathcal{U},\mathcal{V}\in\tau$ such that
            $\mathcal{C}\subseteq\mathcal{U}$,
            $\mathcal{D}\subseteq\mathcal{V}$, and
            $\mathcal{U}\cap\mathcal{V}=\emptyset$.
        \end{definition}
        \begin{figure}
            \centering
            \includegraphics{../../../images/normal_condition_001.pdf}
            \caption{Normal Topological Space}
            \label{fig:normal_condition_001}
        \end{figure}
        \begin{theorem}
            If $(X,\,\tau)$ is a second-countable and regular, then it is
            normal.
        \end{theorem}
        \begin{proof}
            Let $\mathcal{C},\mathcal{D}$ be disjoint closed subsets of $X$.
            Since $(X,\,\tau)$ is second-countable there is a countable basis
            $\mathcal{B}$. For all $x\in\mathcal{C}$, since $(X,\,\tau)$ is
            regular, there is an open set $\mathcal{U}_{x}$ such that
            $x\in\mathcal{U}_{x}$ and
            $\mathcal{U}_{x}\cap\mathcal{D}=\emptyset$. But since
            $(X,\,\tau)$ is regular and $x\in\mathcal{U}_{x}$, there is a
            $\tilde{\mathcal{U}}_{x}$ such that $x\in\tilde{\mathcal{U}}_{x}$
            and $\textrm{Cl}_{\tau}(\tilde{\mathcal{U}}_{x})\subseteq\mathcal{U}_{x}$.
            Similarly we can cover $\mathcal{D}$ with sets
            $\mathcal{V}_{y}$ and $\tilde{\mathcal{V}}_{y}$ such that for all
            $y\in\mathcal{D}$ we have
            $y\in\tilde{\mathcal{V}}_{y}$,
            $\textrm{Cl}_{\tau}(\tilde{\mathcal{V}}_{y})\subseteq\mathcal{V}$,
            and $\mathcal{V}_{y}\cap\mathcal{C}=\emptyset$. Let
            $\mathcal{O}_{\mathcal{C}}$ be
            defined by:
            \begin{equation}
                \mathcal{O}_{\mathcal{C}}
                =\{\,\mathcal{W}\in\mathcal{B}\;|\;
                    \mathcal{W}\subseteq\tilde{\mathcal{U}}_{x}
                    \textrm{ for some }x\in\mathcal{C}\,\}
            \end{equation}
            and $\mathcal{O}_{\mathcal{D}}$ defined by:
            \begin{equation}
                \mathcal{O}_{\mathcal{D}}
                =\{\,\mathcal{W}\in\mathcal{B}\;|\;
                    \mathcal{W}\subseteq\tilde{\mathcal{V}}_{y}
                    \textrm{ for some }y\in\mathcal{D}\,\}
            \end{equation}
            Since $\mathcal{B}$ is countable, both
            $\mathcal{O}_{\mathcal{C}}$ and $\mathcal{O}_{\mathcal{D}}$ are
            countable. But $\mathcal{B}$ is a basis, so
            $\mathcal{O}_{\mathcal{C}}$ and $\mathcal{O}_{\mathcal{D}}$ are
            open covers of $\mathcal{C}$ and $\mathcal{D}$, respectively.
            Let $\mathcal{U}:\mathbb{N}\rightarrow\mathcal{O}_{\mathcal{C}}$
            and $\mathcal{V}:\mathbb{N}\rightarrow\mathcal{O}_{\mathcal{D}}$ be
            surjections. The union over all $\mathcal{U}_{n}$ covers
            $\mathcal{C}$, and the union over $\mathcal{V}_{n}$ covers
            $\mathcal{D}$, but it is possible for these unions to overlap.
            We make them disjoint as follows.
            Define $\mathcal{U}_{n}'$ by:
            \begin{equation}
                \mathcal{U}_{n}'
                =\mathcal{U}_{n}
                \setminus\bigcup_{k=0}^{n}\textrm{Cl}_{\tau}(\mathcal{V}_{k})
            \end{equation}
            and $\mathcal{V}_{n}'$ via:
            \begin{equation}
                \mathcal{V}_{n}'
                =\mathcal{V}_{n}
                \setminus\bigcup_{k=0}^{n}\textrm{Cl}_{\tau}(\mathcal{U}_{k})
            \end{equation}
            Then $\mathcal{U}_{n}'$ and $\mathcal{V}_{n}'$ are the difference
            of a closed set from an open set, and hence are all open.
            But now $\bigcup_{n}\mathcal{U}_{n}'$ and
            $\bigcup_{n}\mathcal{V}_{n}'$ are disjoint open sets that cover
            $\mathcal{C}$ and $\mathcal{D}$, respectively. Hence,
            $(X,\,\tau)$ is normal.
        \end{proof}
        In a just world, there would be three types of separation properties
        and two adjectives for these properties. There would be the
        three separation properties Hausdorff, regular, and normal.
        The three properties with the adjective \textit{completely}. And the
        three properties with the adjective \textit{perfectly}. The adjective
        \textit{completely} should mean something similar for all three
        properties, and the adjective \textit{perfectly} should mean something
        similar for all three properties as well. This is not the case, and
        life is not art, unfortunately.
        \par\hfill\par
        In a just world, \textit{completely} should mean every subspace has
        the property, \textit{perfectly} should mean the separation property
        can replace open sets with continuous functions. There is a reason this
        is not done. If it were, Hausdorff and completely Hausdorff would mean
        the same thing, regular and completely regular would mean the same
        thing, and only normal and completely normal would be different ideas.
        Let's prove this. We've already done the Hausdorff case when we studied
        subspaces, but let's do it again. Why not.
        \begin{theorem}
            If $(X,\,\tau)$ is a Hausdorff topological space, if
            $A\subseteq{X}$, and if $\tau_{A}$ is the subspace topology, then
            $(A,\,\tau_{A})$ is a Hausdorff topological space.
        \end{theorem}
        \begin{proof}
            Let $x,y\in{A}$ with $x\ne{y}$. Then, since $A\subseteq{X}$, we
            have that $x,y\in{X}$ are distinct points. But $(X,\,\tau)$ is
            Hausdorff, so there are $\mathcal{U},\mathcal{V}\in\tau$ such that
            $x\in\mathcal{U}$, $y\in\mathcal{V}$, and
            $\mathcal{U}\cap\mathcal{V}=\emptyset$. But then
            $\tilde{\mathcal{U}}=A\cap\mathcal{U}$ and
            $\tilde{\mathcal{V}}=A\cap\mathcal{V}$ are open sets in $\tau_{A}$
            by the definition of the subspace topology,
            $x\in\tilde{\mathcal{U}}$, $y\in\tilde{\mathcal{V}}$, and
            $\tilde{\mathcal{U}}\cap\tilde{\mathcal{V}}=\emptyset$. So
            $(A,\,\tau_{A})$ is Hausdorff.
        \end{proof}
        \begin{theorem}
            If $(X,\,\tau)$ is a regular topological space, if
            $A\subseteq{X}$, and if $\tau_{A}$ is the subspace topology,
            then $(A,\,\tau_{A})$ is regular.
        \end{theorem}
        \begin{proof}
            Let $x\in{A}$, $\mathcal{C}\subseteq{A}$ be closed with respect
            to $\tau_{A}$, and $x\notin\mathcal{C}$. Then
            $A\setminus\mathcal{C}$ is open in $\tau_{A}$, so there is a
            $\mathcal{U}\in\tau$ such that
            $A\setminus\mathcal{C}=A\cap\mathcal{U}$. Let
            $\tilde{\mathcal{C}}=\textrm{Cl}_{\tau}(\mathcal{C})$. Note, this
            is closure with respect to $\tau$, not $\tau_{A}$. Since
            $x\in\mathcal{U}$ and $\mathcal{C}\cap\mathcal{U}=\emptyset$, we
            have that $x\not\in\textrm{Cl}_{\tau}(\mathcal{C})$, hence
            $x\notin\tilde{\mathcal{C}}$. But $(X,\,\tau)$ is regular and
            $\tilde{\mathcal{C}}$ is closed, being the closure of $\mathcal{C}$.
            So there are $\mathcal{V},\mathcal{W}\in\tau$ such that
            $x\in\mathcal{V}$, $\tilde{\mathcal{C}}\subseteq\mathcal{W}$, and
            $\mathcal{V}\cap\mathcal{W}=\emptyset$. But then
            $\tilde{\mathcal{V}}=\mathcal{V}\cap{A}$ and
            $\tilde{\mathcal{W}}=\mathcal{W}\cap{A}$ are disjoint open sets in
            the subspace topology that separate $x$ and $\mathcal{C}$.
            Hence, $(A,\,\tau_{A})$ is regular.
        \end{proof}
        There is no identical theorem for normal spaces. A subspace of a
        normal space need not be normal. We give a new name to spaces with this
        property.
        \begin{definition}[\textbf{Completely Normal Topological Space}]
            A completely normal topological space is a topological space
            $(X,\,\tau)$ such that for all $A\subseteq{X}$ it is true that
            $(A,\,\tau_{A})$ is normal, where $\tau_{A}$ is the subspace
            topology.
        \end{definition}
        \begin{theorem}
            If $(X,\,\tau)$ is completely normal, then it is normal.
        \end{theorem}
        \begin{proof}
            If every subspace of $(X,\,\tau)$ is normal, then
            $(X,\,\tau)$ is normal since it is a subspace of itself.
        \end{proof}
        It is now extremely unfortunate that \textit{completely} has a very
        different meaning when placed in front of the words Hausdorff and
        regular.
        \begin{definition}[\textbf{Completely Hausdorff Topological Space}]
            A completely Hausdorff topological space is a topological space
            $(X,\,\tau)$ such that for all $x,y\in{X}$ with $x\ne{y}$ there
            is a continuous function $f:X\rightarrow[0,\,1]$ where $[0,\,1]$
            has the subspace topology, such that $f(x)=0$ and $f(y)=1$.
            That is, $x\in{f}^{-1}[\{\,0\,\}]$ and
            $y\in{f}^{-1}[\{\,1\,\}]$.
        \end{definition}
        \begin{theorem}
            If $(X,\,\tau)$ is a completely Hausdorff topological space, then
            it is a Hausdorff topological space.
        \end{theorem}
        \begin{proof}
            Let $x,y\in{X}$, $x\ne{y}$, and let $f:X\rightarrow[0,\,1]$ be
            a continuous function such that $f(x)=0$ and $f(y)=1$.
            Let $\mathcal{U}=f^{-1}\big[[0,\,\frac{1}{4})\big]$ and
            $\mathcal{V}=f^{-1}\big[(\frac{3}{4},\,1]\big]$. Since
            $[0,\,\frac{1}{4})$ and $(\frac{3}{4},\,1]$ are open in the
            subspace topology, $\mathcal{U}$ and $\mathcal{V}$ are open.
            But also, by definition, $\mathcal{U}\cap\mathcal{V}=\emptyset$.
            Since $f(x)=0$ we have $x\in\mathcal{U}$ and since $f(y)=1$ we
            have $y\in\mathcal{V}$. So $(X,\,\tau)$ is Hausdorff.
        \end{proof}
        \begin{definition}[\textbf{Completely Regular Topological Space}]
            A completely regular topological space is a topological space
            $(X,\,\tau)$ such that for all $x\in{X}$ and all closed
            $\mathcal{C}\subseteq{X}$ with $x\notin\mathcal{C}$ there is a
            continuous function $f:X\rightarrow[0,\,1]$, where $[0,\,1]$ has
            the subspace topology, such that $f(x)=0$ and for all
            $y\in\mathcal{C}$ we have $f(y)=1$. That is,
            $x\in{f}^{-1}[\{\,0\,\}]$ and
            $\mathcal{C}\subseteq{f}^{-1}[\{\,1\,\}]$.
        \end{definition}
        \begin{theorem}
            If $(X,\,\tau)$ is a completely regular topological space, then
            it is regular.
        \end{theorem}
        \begin{proof}
            Let $x\in{X}$, $\mathcal{C}\subseteq{X}$ be closed, and
            $x\notin\mathcal{C}$. Since $(X,\,\tau)$ is completely regular
            there is a continuous function $f:X\rightarrow[0,\,1]$ such that
            $x\in{f}^{-1}[\{\,0\,\}]$ and
            $\mathcal{C}\subseteq{f}^{-1}[\{\,1\,\}]$. Let
            $\mathcal{U}=f^{-1}\big[[0,\,\frac{1}{4})\big]$ and
            $\mathcal{V}=f^{-1}\big[(\frac{3}{4},\,1]\big]$. Since
            $[0,\,\frac{1}{4})$ and $(\frac{3}{4},\,1]$ are open in the
            subspace topology and $f$ is continuous it is true that
            $\mathcal{U}$ and $\mathcal{V}$ are open. But by definition
            $\mathcal{U}\cap\mathcal{V}=\emptyset$. But also
            $x\in\mathcal{U}$ and $\mathcal{C}\subseteq\mathcal{V}$, so
            $(X,\,\tau)$ is regular.
        \end{proof}
        Now, isn't this quite dumb? \textit{Completely} has one meaning for
        Hausdorff and regular, and an entirely different meaning for normal.
        Don't blame me, I didn't make the rules! This idea of separating things
        via continuous functions does have a name for normal spaces, but is
        slightly different.
        \begin{definition}[\textbf{Perfectly Normal Topological Space}]
            A perfectly normal topological space is a topological space
            $(X,\,\tau)$ such that for all disjoint closed sets
            $\mathcal{C},\mathcal{D}\subseteq{X}$ there is a continuous
            function $f:X\rightarrow[0,\,1]$, where $[0,\,1]$ has the subspace
            topology, such that $f^{-1}[\{\,0\,\}]=\mathcal{C}$ and
            $f^{-1}[\{\,1\,\}]=\mathcal{D}$.
        \end{definition}
        Perfectly normal means closed sets can be
        \textit{precisely separated} by a continuous function. Contrast this
        with completely regular where it is only required that, given
        $x$ and a closed set $\mathcal{C}$ with $x\notin\mathcal{C}$, that
        $x\in{f}^{-1}[\{\,0\,\}]$ and
        $\mathcal{C}\subseteq{f}^{-1}[\{\,1\,\}]$. With perfectly normal we
        require equality. There is no notion of this idea for regular and
        Hausdorff spaces (though \textit{perfectly Hausdorff} and
        \textit{perfectly regular} would be the likely candidate names).
        See Tab.~\ref{tab:sep_props} for an outline of the various ideas.
        \begin{table}
            \centering
            \begin{tabularx}{\textwidth}{@{}| l | X | X | X |@{}}
                \hline
                $(X,\,\tau)$&\textbf{Hausdorff}&\textbf{Regular}&\textbf{Normal}\\
                \hline
                $-$&$x,y\in{X}$, $x\ne{y}$, there exists disjoint
                $\mathcal{U},\mathcal{V}\in\tau$ such that
                $x\in\mathcal{U}$ and $y\in\mathcal{V}$.
                &$x\in{X}$, $\mathcal{C}\subseteq{X}$ closed with
                $x\notin\mathcal{C}$, there exist disjoint
                $\mathcal{U},\mathcal{V}\in\tau$ such that $x\in\mathcal{U}$
                and $\mathcal{C}\subseteq\mathcal{V}$.&
                $\mathcal{C},\mathcal{D}\subseteq{X}$ closed and disjoint,
                there exists disjoint $\mathcal{U},\mathcal{V}\in\tau$ such
                that $\mathcal{C}\subseteq\mathcal{U}$ and
                $\mathcal{D}\subseteq\mathcal{V}$.\\
                \hline
                Completely&
                $x,y\in{X}$, $x\ne{y}$, there exists continuous
                $f:X\rightarrow[0,\,1]$ such that $f(x)=0$ and $f(y)=1$.
                &$x\in{X}$, $\mathcal{C}\subseteq{X}$ closed with
                $x\not\in\mathcal{C}$, there exists continuous
                $f:X\rightarrow[0,\,1]$ with $f(x)=0$ and
                $f[\mathcal{C}]=\{\,1\,\}$.
                &$A\subseteq{X}$, then $(A,\,\tau_{A})$ is normal where
                $\tau_{A}$ is the subspace topology.\\
                \hline
                Perfectly&N/A&N/A&
                $\mathcal{C},\mathcal{D}\subseteq{X}$ closed and disjoint,
                there exists continuous $f:X\rightarrow[0,\,1]$ such that
                $\mathcal{C}=f^{-1}[\{\,0\,\}]$ and
                $\mathcal{D}=f^{-1}[\{\,1\,\}]$.
                \\
                \hline
            \end{tabularx}
            \caption{The Various Separation Properties}
            \label{tab:sep_props}
        \end{table}
\end{document}
