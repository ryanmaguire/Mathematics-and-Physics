%-----------------------------------LICENSE------------------------------------%
%   This file is part of Mathematics-and-Physics.                              %
%                                                                              %
%   Mathematics-and-Physics is free software: you can redistribute it and/or   %
%   modify it under the terms of the GNU General Public License as             %
%   published by the Free Software Foundation, either version 3 of the         %
%   License, or (at your option) any later version.                            %
%                                                                              %
%   Mathematics-and-Physics is distributed in the hope that it will be useful, %
%   but WITHOUT ANY WARRANTY; without even the implied warranty of             %
%   MERCHANTABILITY or FITNESS FOR A PARTICULAR PURPOSE.  See the              %
%   GNU General Public License for more details.                               %
%                                                                              %
%   You should have received a copy of the GNU General Public License along    %
%   with Mathematics-and-Physics.  If not, see <https://www.gnu.org/licenses/>.%
%------------------------------------------------------------------------------%
\documentclass{article}
\usepackage{amsmath}                            % Needed for align.
\usepackage{amssymb}                            % Needed for mathbb.
\usepackage{amsthm}                             % For the theorem environment.

%------------------------Theorem Styles-------------------------%
\theoremstyle{plain}
\newtheorem{theorem}{Theorem}[section]

% Define theorem style for default spacing and normal font.
\newtheoremstyle{normal}
    {\topsep}               % Amount of space above the theorem.
    {\topsep}               % Amount of space below the theorem.
    {}                      % Font used for body of theorem.
    {}                      % Measure of space to indent.
    {\bfseries}             % Font of the header of the theorem.
    {}                      % Punctuation between head and body.
    {.5em}                  % Space after theorem head.
    {}

% Define default environments.
\theoremstyle{normal}
\newtheorem{examplex}{Example}[section]
\newtheorem{definitionx}{Definition}[section]

\newenvironment{example}{%
    \pushQED{\qed}\renewcommand{\qedsymbol}{$\blacksquare$}\examplex%
}{%
    \popQED\endexamplex%
}

\newenvironment{definition}{%
    \pushQED{\qed}\renewcommand{\qedsymbol}{$\blacksquare$}\definitionx%
}{%
    \popQED\enddefinitionx%
}

\title{Point-Set Topology: Lecture 24}
\author{Ryan Maguire}
\date{\today}

% No indent and no paragraph skip.
\setlength{\parindent}{0em}
\setlength{\parskip}{0em}

\begin{document}
    \maketitle
    \section{Paracompact Spaces}
        Paracompactness is an idea that seems, at first glance, strange, and is
        nowhere near as strong as compactness. One may then
        wonder why on Earth it deserves study, let alone a name.
        One of the motivations is the Urysohn metrization theorem. This theorem
        goes one way, a second countable regular Hausdorff space is metrizable.
        It does not reverse. The discrete topology $\mathbb{R}$ is metrizable
        but not second countable. It would be nice to have a theorem that gives
        necessary and sufficient conditions for a space to metrizable. The
        Nagata-Smirnov and Smirnov metrization theorems do this. At the heart
        of both theorems is the idea of local finiteness. The Nagata-Smirnov
        theorem requires $\sigma$ locally finite bases, the Smirnov theorem
        uses paracompactness. We take the time to develop these and similar
        ideas. This leads in to the Stone paracompactness theorem and these
        two metrization theorems.
        \begin{definition}[\textbf{Locally Finite Collection}]
            A locally finite collection in a topological space $(X,\,\tau)$
            is a subset $\mathcal{A}\subseteq\mathcal{P}(X)$ such that for all
            $x\in{X}$ there is a $\mathcal{U}\in\tau$ such that
            $x\in\mathcal{U}$ and only finitely many $A\in\mathcal{A}$ are
            such that $A\cap\mathcal{U}\ne\emptyset$.
        \end{definition}
        Note there is no requirement that $\mathcal{A}$ consist of open sets or
        closed sets. There is no requirement that $\mathcal{A}$ covers the
        space either. All that is required is local finiteness.
        \begin{definition}[\textbf{Refinement of a Collection}]
            A refinement of a collection $\mathcal{A}\subseteq\mathcal{P}(X)$
            in a topological space $(X,\,\tau)$ is a set
            $\tilde{\mathcal{A}}\subseteq\mathcal{P}(X)$ such that for all
            $\tilde{A}\in\tilde{\mathcal{A}}$ there is an $A\in\mathcal{A}$
            such that $\tilde{A}\subseteq{A}$.
        \end{definition}
        The idea behind a refinement is that we take sets in $\mathcal{A}$ and
        \textit{shrink} them, in some sense.
        Again, in general, there is no requirement that $\mathcal{A}$ or
        $\tilde{\mathcal{A}}$ consist of open or closed sets.
        \textit{Open refinements} are refinements consisting of open sets, and
        \textit{closed refinements} consist of closed sets. This idea is used
        to define paracompactness.
        \begin{definition}[\textbf{Paracompact Topological Space}]
            A paracompact topological space is a topological space $(X,\,\tau)$
            such that for all open covers $\mathcal{O}\subseteq\tau$ of $X$
            there exists a locally finite open refinement
            $\mathcal{X}\subseteq\tau$ of $\mathcal{O}$
            that is an open cover of $(X,\,\tau)$.
        \end{definition}
        This is a very weak notion, many familiar spaces are paracompact, and
        yet it has enormous use in manifold theory and the study of metric
        spaces. In particular because every manifold and every metrizable space
        is paracompact.
        \begin{theorem}
            If $(X,\,\tau)$ is a compact topological space, then it is
            paracompact.
        \end{theorem}
        \begin{proof}
            In a compact space every open cover has a finite subcover, which is
            certainly a locally finite open refinement of the cover.
        \end{proof}
        Far weaker than compactness, $\sigma$ compact plus locally compact
        Hausdorff implies paracompact. We've discussed local compactness in
        metric spaces, and the definition has very little difference in
        topological spaces.
        \begin{definition}[\textbf{Locally Compact Topological Space}]
            A locally compact topological space is a topological space
            $(X,\,\tau)$ such that for all $x\in{X}$ there is an open set
            $\mathcal{U}\in\tau$ and a compact set $K\subseteq{X}$ such that
            $x\in\mathcal{U}$ and $\mathcal{U}\subseteq{K}$.
        \end{definition}
        Before proving locally compact $\sigma$ compact spaces are paracompact,
        we'll need a little lemma.
        \begin{theorem}
            If $(X,\,\tau)$ is locally compact and Hausdorff and if $x\in{X}$,
            then there is a $\mathcal{U}\in\tau$ such that $x\in\mathcal{U}$
            and $\textrm{Cl}_{\tau}(\mathcal{U})$ is compact.
        \end{theorem}
        \begin{proof}
            Since $(X,\,\tau)$ is locally compact there is a
            $\mathcal{U}\in\tau$ and a compact $K\subseteq{X}$ such that
            $x\in\mathcal{U}$ and $\mathcal{U}\subseteq{K}$. But $(X,\,\tau)$
            is Hausdorff, so $K$ is closed. But then
            $\textrm{Cl}_{\tau}(\mathcal{U})\subseteq{K}$. But then
            $\textrm{Cl}_{\tau}(\mathcal{U})$ is a closed subset of a compact
            space, which is therefore compact.
        \end{proof}
        \begin{theorem}
            If $(X,\,\tau)$ is $\sigma$ compact, locally compact, and Hausdorff,
            then it is compactly exhaustible.
        \end{theorem}
        \begin{proof}
            Since $(X,\,\tau)$ is $\sigma$ compact there are countably many
            sets $\mathcal{C}_{n}$, $n\in\mathbb{N}$, each of which is compact
            and such that they cover the space. For all $n\in\mathbb{N}$ and
            for all $x\in\mathcal{C}_{n}$, since $(X,\,\tau)$ is locally compact
            and Hausdorff, there is a $\mathcal{V}_{x,n}\in\tau$ such that
            $x\in\mathcal{V}_{x,n}$ and $\textrm{Cl}_{\tau}(\mathcal{V}_{x,n})$
            is compact. But these sets cover $\mathcal{C}_{n}$, which is
            compact, so we can do it with finitely many,
            $\mathcal{V}_{0,n},\,\dots,\,\mathcal{V}_{N,n}$. Since this is
            a finite collection we have:
            \begin{equation}
                \textrm{Cl}_{\tau}\Big(\bigcup_{k=0}^{N}\mathcal{V}_{k,n}\Big)
                =\bigcup_{k=0}^{N}\textrm{Cl}_{\tau}(\mathcal{V}_{k,n})
            \end{equation}
            Define $\mathcal{U}_{n}$ via:
            \begin{equation}
                \mathcal{U}_{n}=\bigcup_{k=0}^{N}\mathcal{V}_{k,n}
            \end{equation}
            Then $\mathcal{U}_{n}$ is open and by the previous equation
            $\textrm{Cl}_{n}(\mathcal{U}_{n})$ is the finite union of compact
            sets, which is therefore compact. Recursively define
            $\mathcal{W}_{n}$ as follows. Set $\mathcal{W}_{0}=\mathcal{U}_{0}$.
            Let $\mathcal{W}_{n}\in\tau$ be such that
            $\bigcup_{k=0}^{n}\mathcal{U}_{k}\subseteq\mathcal{W}_{n}$,
            $\mathcal{W}_{n-1}\subseteq\mathcal{W}_{n}$, and
            such that $\textrm{Cl}_{\tau}(\mathcal{W}_{n})$ is compact.
            Define $\mathcal{W}_{n+1}$ as follows. Since
            $\textrm{Cl}_{\tau}(\mathcal{W}_{n})$ and
            $\textrm{Cl}_{\tau}(\mathcal{U}_{n+1})$ are compact, so is the
            union. Thus, by the previous argument, we can cover it in finitely
            many open sets $\mathcal{V}_{0},\,\dots,\,\mathcal{V}_{N}$, each
            of which has compact closure. Define:
            \begin{equation}
                \mathcal{W}_{n+1}=\bigcup_{k=0}^{N}\mathcal{V}_{k}
            \end{equation}
            Then $\mathcal{W}_{n+1}$ is open and:
            \begin{equation}
                \textrm{Cl}_{\tau}(\mathcal{W}_{n+1})
                =\textrm{Cl}_{\tau}\Big(\bigcup_{k=0}^{N}\mathcal{V}_{k}\Big)
                =\bigcup_{k=0}^{N}\textrm{Cl}_{\tau}(\mathcal{V}_{k})
            \end{equation}
            which is the finite union of compact sets, so it is compact.
            But moreover, from the construction, since the $\mathcal{V}_{k}$
            cover $\textrm{Cl}_{\tau}(\mathcal{W}_{n})$, we have
            $\textrm{Cl}_{\tau}(\mathcal{W}_{n})\subseteq\mathcal{W}_{n+1}$.
            Define:
            \begin{equation}
                K_{n}=\textrm{Cl}_{\tau}(\mathcal{W}_{n})
            \end{equation}
            Then $K_{n}$ is compact and
            $K_{n}\subseteq\textrm{Int}_{\tau}(K_{n+1})$ since
            $\mathcal{W}_{n+1}\subseteq\textrm{Int}_{\tau}(K_{n+1})$. Morever
            $\bigcup_{n\in\mathbb{N}}K_{n}=X$ since
            $\mathcal{C}_{n}\subseteq\mathcal{U}_{n}$,
            $\mathcal{U}_{n}\subseteq\mathcal{W}_{n}$, and
            $\mathcal{W}_{n}\subseteq{K}_{n}$. Since the $\mathcal{C}_{n}$
            cover $X$, so do the $K_{n}$. Hence, $(X,\,\tau)$ is
            compactly exhaustible.
        \end{proof}
        \begin{theorem}
            If $(X,\,\tau)$ is compactly exhaustible and Hausdorff,
            then it is paracompact.
        \end{theorem}
        \begin{proof}
            Let $K:\mathbb{N}\rightarrow\mathcal{P}(X)$ be such that
            for all $n\in\mathbb{N}$ $K_{n}$ is compact,
            $K_{n}\subseteq\textrm{Int}_{\tau}(K_{n+1})$, and
            $\bigcup_{n\in\mathbb{N}}K_{n}=X$. Note that, since
            $\textrm{Int}_{\tau}(K_{n})\subseteq{K}_{n+1}$ is open, and
            $K_{n+1}$ is compact, $K_{n+1}\setminus\textrm{Int}_{\tau}(K_{n})$
            is compact. Let $\mathcal{O}$ be an open cover. We must find a
            locally finite open refinement $\mathcal{X}$ of $\mathcal{O}$.
            But $\mathcal{O}$ covers $X$, so it covers
            $K_{n+1}\setminus\textrm{Int}_{\tau}(K_{n})$. By compactness there
            are finitely many $\mathcal{V}_{0},\,\dots,\,\mathcal{V}_{n_{N}}$
            that cover $K_{n+1}\setminus\textrm{Int}_{\tau}(K_{n})$.
            Define $\Delta_{n}$ via:
            \begin{equation}
                \Delta_{n}=
                \Big\{\,\mathcal{V}_{k}\cap\big(\textrm{Int}_{\tau}(K_{n+2})
                    \setminus{K}_{n-1}\big)\;|\;0\leq{k}\leq{N}_{n}\Big\}
            \end{equation}
            (define $K_{-1}=\emptyset$ for the case $n=0$).
            But $(X,\,\tau)$ is Hausdorff, so each $K_{n}$ is closed, hence
            $\textrm{Int}_{\tau}(K_{n+2})\setminus{K}_{n-1}$ is open, meaning
            all elements of $\Delta$ are open. The set
            $\mathcal{X}=\bigcup_{n\in\mathbb{N}}\Delta_{n}$ is a locally finite
            open refinement. It is an open refinement, the elements are open
            and are contained as subsets of the elements of $\mathcal{O}$ by
            construction. It is also an open cover since it covers each
            $K_{n}$, and the $K_{n}$ cover $X$. Lastly, it is locally finite.
            Every element of $X$ is contained in some
            $\textrm{Int}_{\tau}(K_{n+1})\setminus{K}_{n-1}$ for some $n$,
            and $\mathcal{X}$ has only finitely many elements with non-empty
            intersection with this set, the elements of $\Delta_{n}$. So
            $\mathcal{X}$ is a locally finite open refinement of $\mathcal{O}$
            that covers $X$, so $(X,\,\tau)$ is paracompact.
        \end{proof}
        \begin{theorem}
            If $(X,\,\tau)$ is paracompact, and if $\mathcal{C}\subseteq{X}$ is
            closed, then $(\mathcal{C},\,\tau_{\mathcal{C}})$ is paracompact
            where $\tau_{\mathcal{C}}$ is the subspace topology.
        \end{theorem}
        \begin{proof}
            The proof is a mimicry of the idea for compact spaces. Given an
            open cover $\mathcal{O}$ of $\mathcal{C}$, we extend it to an open
            cover $\tilde{\mathcal{O}}$ of $X$ via
            $\tilde{\mathcal{O}}=\mathcal{O}\cup\{\,X\setminus\mathcal{C}\,\}$.
            Using the paracompactness of $(X,\,\tau)$
            we get a locally finite open
            refinement $\tilde{\mathcal{X}}$ that covers $X$. We restrict these
            sets to $\mathcal{C}$ to obtain a locally finite open refinement
            $\mathcal{X}$ of $\mathcal{O}$ that covers $\mathcal{C}$.
        \end{proof}
        \begin{theorem}
            If $(X,\,\tau)$ is a topological space, and if
            $\mathcal{A}\subseteq\mathcal{P}(X)$ is locally finite, then the
            set:
            \begin{equation}
                \mathcal{A}'=\{\,\textrm{Cl}_{\tau}(A)\;|\;A\in\mathcal{A}\,\}
            \end{equation}
            is locally finite as well
        \end{theorem}
        \begin{proof}
            Homework.
        \end{proof}
        \begin{theorem}
            If $(X,\,\tau)$ is a topological space, and if
            $\mathcal{A}\subseteq\mathcal{P}(X)$ is locally finite, then:
            \begin{equation}
                \textrm{Cl}_{\tau}\Big(\bigcup_{A\in\mathcal{A}}A\big)
                =\bigcup_{A\in\mathcal{A}}\textrm{Cl}_{\tau}(A)
            \end{equation}
        \end{theorem}
        \begin{proof}
            Also homework.
        \end{proof}
        \begin{theorem}
            If $(X,\,\tau)$ is paracompact and Hausdorff, then it is regular.
        \end{theorem}
        \begin{proof}
            For let $x\in{X}$, $\mathcal{C}\subseteq{X}$ be closed, and
            $x\notin\mathcal{C}$. Since $(X,\,\tau)$ is Hausdorff, for all
            $y\in\mathcal{C}$ there are $\mathcal{U}_{y},\mathcal{V}_{y}\in\tau$
            such that $x\in\mathcal{U}_{y}$, $y\in\mathcal{V}_{y}$, and
            $\mathcal{U}_{y}\cap\mathcal{V}_{y}=\emptyset$. But then:
            \begin{equation}
                \mathcal{O}=\{\,\mathcal{V}_{y}\;|\;y\in\mathcal{C}\,\}
            \end{equation}
            is an open cover of $\mathcal{C}$. But $(X,\,\tau)$ is paracompact
            and $\mathcal{C}$ is closed, so there is a locally finite open
            refinement $\mathcal{X}$ that covers $\mathcal{C}$. But by the
            definition of $\mathcal{O}$, since $\mathcal{X}$ is a refinement of
            $\mathcal{O}$, for all $\mathcal{W}\in\mathcal{X}$ there is a
            $\mathcal{V}_{y}\in\mathcal{O}$ such that
            $\mathcal{W}\subseteq\mathcal{V}_{y}$. Hence all elements of
            $\mathcal{X}$ are subsets of $\mathcal{V}_{y}$ for some
            $y\in\mathcal{C}$. But then, since $x\in\mathcal{U}_{y}$ and
            $\mathcal{U}_{y}\cap\mathcal{V}_{y}=\emptyset$, we have
            $x\notin\textrm{Cl}_{\tau}(\mathcal{V}_{y})$. But
            $\mathcal{X}$ is locally finite, hence:
            \begin{equation}
                \textrm{Cl}_{\tau}\Big(\bigcup_{A\in\mathcal{X}}A\Big)
                =\bigcup_{A\in\mathcal{X}}\textrm{Cl}_{\tau}(A)
            \end{equation} 
            And hence
            $x\notin\textrm{Cl}_{\tau}\big(\bigcup_{A\in\mathcal{X}}A\big)$.
            Let
            $\mathcal{U}=X\setminus\textrm{Cl}_{\tau}\Big(\bigcup\mathcal{X}\Big)$
            and $\mathcal{V}=\bigcup\mathcal{X}$. Then $\mathcal{U}$ and
            $\mathcal{V}$ are open and disjoint, $x\in\mathcal{U}$, and
            $\mathcal{C}\subseteq\mathcal{V}$. Hence, $(X,\,\tau)$ is regular.
        \end{proof}
        \begin{theorem}[\textbf{Dieudonne's Theorem}]
            If $(X,\,\tau)$ is paracompact and Hausdorff, then it is normal.
        \end{theorem}
        \begin{proof}
            We apply the same idea as before. Since $(X,\,\tau)$ is paracompact
            and Hausdorff, it is regular. Given two closed disjoint sets
            $\mathcal{C},\mathcal{D}\subseteq{X}$ for all
            $x\in\mathcal{C}$ we find $\mathcal{U}_{x},\mathcal{V}_{x}\in\tau$
            such that $x\in\mathcal{U}_{x}$,
            $\mathcal{D}\subseteq\mathcal{V}_{x}$, and
            $\mathcal{U}_{x}\cap\mathcal{V}_{x}=\emptyset$. We use
            paracompactness and apply a similar argument to the previous theorem
            to prove normality.
        \end{proof}
        We now take the steps towards proving the two main metrization theorems.
        The Nagata-Smirnov theorem, and the Smirnov theorem. The Smirnov theorem
        uses paracompactness to characterize metrizable spaces, the
        Nagata-Smirnov theorem uses a very similar idea as the Urysohn
        metrization theorem. Urysohn's theorem said a regular Hausdorff space
        that is second countable is metrizable. All metrizable spaces are
        regular and Hausdorff, so this can not be omitted, however the
        second countability can be weakened. The idea that is required for
        metrizability is $\sigma$ locally finite bases. Related to this
        concept is the notion of a $\sigma$ locally finite open cover.
        \begin{definition}[\textbf{$\sigma$ Locally Finite Open Cover}]
            A $\sigma$ locally finite open cover of a topological space
            $(X,\,\tau)$ is an open cover $\mathcal{O}\subseteq\tau$ such that
            there exists countably many locally finite collections
            $\Delta_{n}\subseteq\tau$ such that
            $\mathcal{O}=\bigcup_{n\in\mathbb{N}}\Delta_{n}$.
        \end{definition}
        Now for some results related to this notion, paracompactness, and
        metrization.
        \begin{theorem}
            If $(X,\,\tau)$ is metrizable, and if $\mathcal{O}\subseteq\tau$ is
            an open cover, then there is a $\sigma$ locally finite open cover
            $\mathcal{X}$ that is a refinement of $\mathcal{O}$.
        \end{theorem}
        \begin{proof}
            This theorem is part of the proof that metrizable spaces are
            paracompact (which is Stone's paracompactness theorem). The
            original proof is very long but can be shortened by using the
            well ordering theorem. This makes the theorem less-than-intuitive,
            but pages and pages shorter. The curious reader should consult
            A. H. Stone's original papers for the more straight-forward but
            longer and more involved proof.
            \par\hfill\par
            Since $\mathcal{O}$ is a set, there is a well-order on it
            $\prec$. Since $(X,\,\tau)$ is metrizable there is a metric
            $d$ that induces the topology $\tau$. For all $n\in\mathbb{N}$
            and for all $\mathcal{U}\in\mathcal{O}$ define:
            \begin{equation}
                S_{n}(\mathcal{U})=
                \{\,x\in{X}\;|\;
                    {B}_{\frac{1}{n+1}}^{(X,\,d)}(x)\subseteq\mathcal{U}\,\}
            \end{equation}
            That is, the set of all points in $\mathcal{U}$ that can be
            surrounded with a ball of radius $1/(n+1)$ that is completely
            contained in $\mathcal{U}$. Define $T_{n}(\mathcal{U})$ via:
            \begin{equation}
                T_{n}(\mathcal{U})=
                S_{n}(\mathcal{U})\setminus\bigcup
                    \big\{\,\mathcal{V}\in\mathcal{O}\;|\;
                        \mathcal{V}\prec\mathcal{U}\,\big\}
            \end{equation}
            We have used the well-order. We take the union of all sets that are
            \textit{less than} $\mathcal{U}$ with respect to the well-order
            $\prec$ in the last part of this equation. For all
            $x\in{T}_{n}(\mathcal{U})$ and $y\in{T}_{n}(\mathcal{V})$, with
            $\mathcal{U},\mathcal{V}\in\mathcal{O}$ distinct, we have from this
            construction that $d(x,\,y)\geq\frac{1}{n+1}$. The sets
            $S_{n}(\mathcal{U})$ are closed, and $T_{n}(\mathcal{U})$
            is the difference of an open set from a closed set, which is
            therefore closed.
            We want open sets. Define $E_{n}(\mathcal{U})$ via:
            \begin{equation}
                E_{n}(\mathcal{U})=
                \bigcup_{x\in{T}_{n}(\mathcal{U})}
                B_{\frac{1}{3n+3}}^{(X,\,\tau)}(x)
            \end{equation}
            Then $E_{n}(\mathcal{U})$ is open, being the union of open balls.
            By choosing the radii to be $\frac{1}{3n+3}$, we ensure that
            $E_{n}(\mathcal{U})$ and $E_{n}(\mathcal{V})$ are disjoint
            whenever $\mathcal{U},\mathcal{V}\in\tau$ are distinct. We use
            this to obtain our $\sigma$ locally finite cover that is a
            refinement of $\mathcal{O}$. First, define:
            \begin{equation}
                \mathcal{A}_{n}=
                \{\,E_{n}(\mathcal{U})\;|\;\mathcal{U}\in\mathcal{O}\,\}
            \end{equation}
            And define $\mathcal{X}=\bigcup_{n\in\mathbb{N}}\mathcal{A}_{n}$.
            From the construction, given $\mathcal{U}\in\mathcal{O}$, we have
            $E_{n}(\mathcal{U})\subseteq\mathcal{U}$, and hence
            $\mathcal{X}$ is a refinement of $\mathcal{O}$. Each
            $\mathcal{A}_{n}$ is locally finite, the ball of radius
            $\frac{1}{6n+6}$ centered at $x$ intersects only one element of
            $\mathcal{A}_{n}$. Lastly, $\mathcal{X}$ covers $X$. Given
            $x\in{X}$, since $\mathcal{O}$ covers $X$, there is a
            $\mathcal{U}\in\mathcal{O}$ such that $x\in\mathcal{U}$. But
            $d$ induces $\tau$, so there is an $r>0$ such that
            the $r$ ball centered at $x$ is contained in $\mathcal{U}$. Choose
            $N\in\mathbb{N}$ so that $\frac{1}{N+1}<r$. Then
            $x\in{E}_{N}(\mathcal{U})$, showing that there is an element of
            $\mathcal{X}$ that contains $x$. So $\mathcal{X}$ is a
            $\sigma$ locally finite open cover that is a refinement of
            $\mathcal{X}$.
        \end{proof}
        This idea comes from Mary Ellen Rudin, one of the great topologists of
        the second half of the $20^{\textrm{th}}$ century. Stone's
        paracompactness theorem now comes in two more steps. These
        theorems are a bit of work, but we get two pretty results out of
        them. First, Stone's paracompactness theorem, but also the fact that
        every regular Lindel\"{o}f space is paracompact, essentially for free.
        \begin{theorem}
            If $(X,\,\tau)$ is regular, if $\mathcal{O}\subseteq\tau$ is an
            open cover, and if $\mathcal{X}$ is a $\sigma$ locally finite open
            cover that is a refinement of $\mathcal{O}$, then there is a
            locally finite cover (but not necessarily an open one)
            $\Delta$ that is a refinement of $\mathcal{O}$.
        \end{theorem}
        \begin{proof}
            Since $\mathcal{X}$ is $\sigma$ locally finite and an open cover
            there are countably many $\mathcal{A}_{n}$, each of which is
            locally finite, such that
            $\mathcal{X}=\bigcup_{n\in\mathbb{N}}\mathcal{A}_{n}$. Let
            $\mathcal{U}_{n}\in\tau$ be defined by:
            \begin{equation}
                \mathcal{U}_{n}=\bigcup\mathcal{A}_{n}
            \end{equation}
            Since the elements of $\mathcal{A}_{n}$ are open subsets,
            $\mathcal{U}_{n}$ is open. For all
            $\mathcal{V}\in\mathcal{A}_{n}$, define:
            \begin{equation}
                S_{n}(\mathcal{V})=
                \mathcal{V}\setminus\bigcup_{k=0}^{n-1}\mathcal{U}_{k}
            \end{equation}
            $S_{n}(\mathcal{V})$ is the set difference of an open set from
            an open set, so it may be open, closed, both, or neither. We simply
            don't know. Define $\Delta_{n}$ to be:
            \begin{equation}
                \Delta_{n}
                =\{\,S_{n}(\mathcal{V})\;|\;\mathcal{V}\in\mathcal{A}_{n}\,\}
            \end{equation}
            Let $\Delta=\bigcup_{n\in\mathbb{N}}\Delta_{n}$. We must now show
            $\Delta$ is locally finite and covers $X$. First, $\Delta$ is a
            cover of $X$. Given $x\in{X}$, since the sets $\mathcal{U}_{n}$ cover
            $X$, there is some $n\in\mathbb{N}$ such that $x\in\mathcal{U}_{n}$.
            By the well-ordering property of the natural numbers there is a
            least such number $N\in\mathbb{N}$ such that $x\in\mathcal{U}_{N}$.
            Then for all $n<N$ we have $x\notin\mathcal{U}_{n}$, and hence
            by definition $x\in{S}_{N}(\mathcal{U}_{N})$. So
            $\Delta$ covers $X$. Next, it is locally finite. Given
            $x\in{X}$, since each $\mathcal{A}_{n}$ is locally finite we have
            that there is a $\mathcal{V}_{n}\in\tau$ such that
            $x\in\mathcal{V}_{n}$ and $\mathcal{V}_{n}$ intersects only
            finitely many elements of $\mathcal{A}_{n}$. Again, let
            $N\in\mathbb{N}$ be the least integer such that
            $x\in\mathcal{U}_{N}$. Then for all $n>N$ we have
            $\mathcal{U}_{N}$ has empty intersection with the elements of
            $\Delta_{n}$, by definition of $S_{n}$. So the set:
            \begin{equation}
                \tilde{\mathcal{U}}
                =\mathcal{U}_{N}\cap\bigcap_{k=0}^{N}\mathcal{V}_{k}
            \end{equation}
            is an open set that contains $x$ and is such that only finitely
            many elements of $\Delta$ have non-empty intersection with
            $\tilde{\mathcal{U}}$. Hence $\Delta$ is locally finite.
        \end{proof}
        This would be much stronger if the set $\Delta$ consists of open sets.
        In this construction it almost certainly does not. The difference of an
        open set from an open set is rarely open in usual spaces. For example,
        in $\mathbb{R}$, given $a<c<b<d$, $(a,\,b)\setminus(c,\,d)=(a,\,c]$,
        which is neither closed nor open. We need to modify this idea.
        \begin{theorem}
            If $(X,\,\tau)$ is a regular topological space such that for all
            open covers $\mathcal{O}$ of $X$ there is a locally finite
            refinement $\Delta$ of $\mathcal{O}$ that covers $X$, then every
            open cover $\mathcal{O}$ has a locally finite open refinement
            $\mathcal{X}$ that covers $X$.
        \end{theorem}
        \begin{proof}
            Let $\mathcal{O}$ be an open cover and let $\Delta$ be a locally
            finite refinement (it does not need to be an open one). Since
            $\Delta$ is locally finite, for all $x\in{X}$ there is a
            $\mathcal{U}_{x}\in\tau$ such that $\mathcal{U}_{x}$ has non-empty
            intersection with only finitely many elements of $\Delta$. The set:
            \begin{equation}
                \mathcal{A}=
                \{\,\mathcal{U}_{x}\;|\;x\in{X}\,\}
            \end{equation}
            is thus an open cover of $X$. Let $\Delta'$ be a locally finite
            refinement of $\mathcal{A}$, which exists by hypothesis. Define
            \begin{equation}
                \Delta''
                =\{\,\textrm{Cl}_{\tau}(A)\;|\;A\in\Delta'\,\}
            \end{equation}
            Since $\Delta'$ is locally finite, so is $\Delta''$ and $\Delta''$
            consists of closed sets. For all $A\in\Delta$,
            define $\mathcal{B}_{A}$ via:
            \begin{equation}
                \mathcal{B}_{A}=
                \{\,B\in\Delta''\;|\;B\subseteq{X}\setminus{A}\,\}
            \end{equation}
            and define $\mathcal{V}_{A}$ via:
            \begin{equation}
                \mathcal{V}_{A}=X\setminus\bigcup\mathcal{B}_{A}
            \end{equation}
            Since $\Delta''$ is locally finite, and since the elements of it
            are closed, we have $\bigcup\mathcal{B}_{A}$ is closed, and hence
            $\mathcal{V}_{A}$ is open. Since each element of $\mathcal{B}_{A}$
            is disjoint from $A$, we have that $A\subseteq\mathcal{V}_{A}$.
            Since $\Delta$ is a refinement of $\mathcal{O}$, for all
            $A\in\Delta$ there is a $\mathcal{W}_{A}\in\mathcal{O}$ such that
            $A\subseteq\mathcal{W}_{A}$. Define:
            \begin{equation}
                \mathcal{X}=
                \{\,\mathcal{V}_{A}\cap\mathcal{W}_{A}\;|\;A\in\Delta\,\}
            \end{equation}
            Then $\mathcal{X}$ is a refinement of $\mathcal{O}$ and also an
            open cover of $X$, by definition of $\mathcal{V}_{A}$ and
            $\mathcal{W}_{A}$. We must show it is locally finite.
            For given $x\in{X}$, since $\Delta''$ is
            locally finite, there is a $\mathcal{W}\in\tau$ such that
            $x\in\mathcal{W}$ and only finitely many elements of $\Delta''$
            have non-empty intersection with $\mathcal{W}$. But
            $\Delta''$ covers $X$, so the elements of $\Delta''$ with non-empty
            intersection with $\mathcal{W}$ must also cover $\mathcal{W}$.
            Because of this we need only show that any given element of
            $\Delta''$ intersects only finitely many elements of $\mathcal{X}$.
            Let let $B\in\Delta''$. If $B$ intersects some element of
            $\mathcal{X}$, $\mathcal{V}_{A}\cap\mathcal{W}_{A}$ for some
            $A\in\Delta$, then by definition of $\mathcal{V}_{A}$ we have
            $B\cap(X\setminus{A})=\emptyset$. But then $B\cap{A}\ne\emptyset$.
            But the elements of $\Delta''$ intersect only finitely many elements
            of $\Delta$, $B$ can only intersect finitely many elements of
            $\mathcal{X}$. Hence, $\mathcal{X}$ is a locally finite open
            refinement of $\mathcal{O}$ that covers $X$.
        \end{proof}
        \begin{theorem}
            If $(X,\,\tau)$ is regular and Lindel\"{o}f, then it is
            paracompact.
        \end{theorem}
        \begin{proof}
            Given a cover $\mathcal{O}$, since $(X,\,\tau)$ is Lindel\"{o}f
            there is a countable subcover $\Delta$. Since this is countable, it
            is also a $\sigma$ locally finite open cover which is a refinement
            of $\mathcal{O}$, since it is a subset of it. It is $\sigma$
            locally finite since we can find a surjection
            $\mathcal{U}:\mathbb{N}\rightarrow\Delta$ and letting
            $\Delta_{n}=\{\,\mathcal{U}_{n}\,\}$, we have
            $\Delta=\bigcup_{n\in\mathbb{N}}\Delta_{n}$. Each $\Delta_{n}$ is
            locally finite since it is, well, finite. By a previous theorem,
            since $(X,\,\tau)$ is regular, there is therefore a locally finite
            open refinement of $\Delta$ that covers $X$. So $(X,\,\tau)$ is
            paracompact.
        \end{proof}
        \begin{theorem}[\textbf{Stone's Paracompactness Theorem}]
            If $(X\,\tau)$ is metrizable, then it is paracompact.
        \end{theorem}
        \begin{proof}
            Since $(X,\,\tau)$ is metrizable, every open cover $\mathcal{O}$
            has a $\sigma$ locally finite open cover $\mathcal{X}$ that is a
            refinement of $\mathcal{O}$. By a previous theorem, since
            metrizable spaces are regular, there is a locally finite open
            refinement of this that is an open cover, and hence $(X,\,\tau)$
            is paracompact.
        \end{proof}
\end{document}
