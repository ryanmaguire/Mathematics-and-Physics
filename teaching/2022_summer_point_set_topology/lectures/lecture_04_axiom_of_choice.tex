%-----------------------------------LICENSE------------------------------------%
%   This file is part of Mathematics-and-Physics.                              %
%                                                                              %
%   Mathematics-and-Physics is free software: you can redistribute it and/or   %
%   modify it under the terms of the GNU General Public License as             %
%   published by the Free Software Foundation, either version 3 of the         %
%   License, or (at your option) any later version.                            %
%                                                                              %
%   Mathematics-and-Physics is distributed in the hope that it will be useful, %
%   but WITHOUT ANY WARRANTY; without even the implied warranty of             %
%   MERCHANTABILITY or FITNESS FOR A PARTICULAR PURPOSE.  See the              %
%   GNU General Public License for more details.                               %
%                                                                              %
%   You should have received a copy of the GNU General Public License along    %
%   with Mathematics-and-Physics.  If not, see <https://www.gnu.org/licenses/>.%
%------------------------------------------------------------------------------%
\documentclass{article}
\usepackage{graphicx}                           % Needed for figures.
\usepackage{amsmath}                            % Needed for align.
\usepackage{amssymb}                            % Needed for mathbb.
\usepackage{amsthm}                             % For the theorem environment.
\usepackage{float}
\usepackage{hyperref}
\hypersetup{
    colorlinks=true,
    linkcolor=blue,
    filecolor=magenta,
    urlcolor=Cerulean,
    citecolor=SkyBlue
}

%------------------------Theorem Styles-------------------------%
\theoremstyle{plain}
\newtheorem{theorem}{Theorem}[section]

% Define theorem style for default spacing and normal font.
\newtheoremstyle{normal}
    {\topsep}               % Amount of space above the theorem.
    {\topsep}               % Amount of space below the theorem.
    {}                      % Font used for body of theorem.
    {}                      % Measure of space to indent.
    {\bfseries}             % Font of the header of the theorem.
    {}                      % Punctuation between head and body.
    {.5em}                  % Space after theorem head.
    {}

% Define default environments.
\theoremstyle{normal}
\newtheorem{examplex}{Example}[section]
\newtheorem{definitionx}{Definition}[section]
\newtheorem{axiomx}{Axiom}[section]

\newenvironment{example}{%
    \pushQED{\qed}\renewcommand{\qedsymbol}{$\blacksquare$}\examplex%
}{%
    \popQED\endexamplex%
}

\newenvironment{definition}{%
    \pushQED{\qed}\renewcommand{\qedsymbol}{$\blacksquare$}\definitionx%
}{%
    \popQED\enddefinitionx%
}

\newenvironment{axiom}{%
    \pushQED{\qed}\renewcommand{\qedsymbol}{$\blacksquare$}\axiomx%
}{%
    \popQED\endaxiomx%
}

\title{Point-Set Topology: Lecture 4}
\author{Ryan Maguire}
\date{\today}

% No indent and no paragraph skip.
\setlength{\parindent}{0em}
\setlength{\parskip}{0em}

\begin{document}
    \maketitle
    \section{Generalized Operations on Sets}
        In topology, set theory, analysis, etc., we often need to consider more
        than just two sets at a time. We may have a set $X$ and a set
        $\mathcal{O}$ consisting of subsets $A\subseteq{X}$. The union over
        such a collection exists by the axiom of the union. The definition of
        this set is:
        \begin{definition}[\textbf{Union of a Collection}]
            The over a set $\mathcal{O}$ is the set $\bigcup\mathcal{O}$
            defined by:
            \begin{equation}
                \bigcup\mathcal{O}=
                    \{\,x\;|\;\textrm{there exists }A\in\mathcal{O}
                        \textrm{ such that }x\in{A}\,\}
            \end{equation}
            That is, the set of all elements contained in some set
            $A\in\mathcal{O}$.
        \end{definition}
        The intersection does not need an axiom, and can be defined via the
        union set and the axiom schema of specification.
        \begin{definition}[\textbf{Intersection of a Collection}]
            The intersection over a set $\mathcal{O}$ is the set
            $\bigcap\mathcal{O}$ defined by:
            \begin{equation}
                \bigcap\mathcal{O}=\{\,x\in\bigcup\mathcal{O}\;|\;x\in{A}
                    \textrm{ for all }A\in\mathcal{O}\;\}
            \end{equation}
            That is, the set of all elements common to all $A\in\mathcal{O}$.
        \end{definition}
        If the set $\mathcal{O}$ is indexed by the natural numbers,
        say:
        \begin{equation}
            \mathcal{O}=\{\,A_{0},\,A_{1},\,A_{2},\,\dots\,\}
        \end{equation}
        We write the intersection and union as follows:
        \begin{align}
            \bigcup\mathcal{O}&=\bigcup_{n=0}^{\infty}A_{n}\\
            \bigcap\mathcal{O}&=\bigcap_{n=0}^{\infty}A_{n}
        \end{align}
        We may also write:
        \begin{align}
            \bigcup\mathcal{O}&=\bigcup_{A\in\mathcal{O}}A\\
            \bigcap\mathcal{O}&=\bigcap_{A\in\mathcal{O}}A
        \end{align}
        If $\mathcal{O}$ is indexed by some \textit{indexing set} $I$
        (for example, the real numbers), where $A_{\alpha}\in\mathcal{O}$
        for all $\alpha\in{I}$, we write:
        \begin{align}
            \bigcup\mathcal{O}&=\bigcup_{\alpha\in{I}}A_{\alpha}\\
            \bigcap\mathcal{O}&=\bigcap_{\alpha\in{I}}A_{\alpha}
        \end{align}
        The \textit{generalized distributive laws} then say:
        \begin{align}
            A\cap\bigcup_{B\in\mathcal{O}}B
                &=\bigcup_{B\in\mathcal{O}}(A\cap{B})\\
            A\cup\bigcap_{B\in\mathcal{O}}B
                &=\bigcap_{B\in\mathcal{O}}(A\cup{B})
        \end{align}
        The \textit{generalized De Morgan's Laws} read:
        \begin{align}
            X\setminus\bigcup_{A\in\mathcal{O}}A
                &=\bigcap_{A\in\mathcal{O}}(X\setminus{A})\\
            X\setminus\bigcap_{A\in\mathcal{O}}A
                &=\bigcup_{A\in\mathcal{O}}(X\setminus{A})
        \end{align}
    \section{Product Sets and the Axiom of Choice}
        My goal, now, is to convince you there is another way of thinking of
        Cartesian products. A \textit{better} way. The axioms of set theory
        tell us there exists a \textit{set of all functions} from a set $A$ to
        a set $B$. That is, given sets $A$
        and $B$, there is a set $\mathcal{F}(A,\,B)$ such that for all $f$,
        $f\in\mathcal{F}(A,\,B)$ if and only if $f:A\rightarrow{B}$ is a
        function from $A$ to $B$. This can be proven using the axiom of the
        power set and of specification, but we won't bother. Let us redefine the
        Cartesian product of a set $A$ with a set $B$ to be the set of all
        functions $f:\mathbb{Z}_{2}\rightarrow{A}\cup{B}$ with the property
        that $f(0)\in{A}$ and $f(1)\in{B}$. Take $\mathbb{R}^{2}$ for example.
        An element of $\mathbb{R}^{2}$ is then a function
        $\mathbf{x}:\mathbb{Z}_{2}\rightarrow\mathbb{R}\cup\mathbb{R}$,
        which we may simply write
        $\mathbf{x}:\mathbb{Z}_{2}\rightarrow\mathbb{R}$, such that
        $\mathbf{x}(0)\in\mathbb{R}$ and $\mathbf{x}(1)\in\mathbb{R}$.
        Let's rewrite $\mathbf{x}(0)=x_{0}$ and $\mathbf{x}(1)=x_{1}$.
        Then the \textit{function} $\mathbf{x}$ is really just the
        \textit{vector} $\mathbf{x}=(x_{0},\,x_{1})$. Why might we do this?
        It allows us to more easily define the product of three sets.
        $A\times(B\times{C})$ and $(A\times{B})\times{C}$ are different sets.
        $A\times(B\times{C})$ consists of elements of the form
        $\big(a,\,(b,\,c)\big)$, whereas elements of
        $(A\times{B})\times{C}$ are of the form
        $\big((a,\,b),\,c\big)$. Very nit-picky, but alas, this is how the
        Cartesian product was defined. Instead, we can define the product over
        three sets $A$, $B$, and $C$ to be the set of all functions
        $f:\mathbb{Z}_{3}\rightarrow{A}\cup{B}\cup{C}$ with the property that
        $f(0)\in{A}$, $f(1)\in{B}$, and $f(2)\in{C}$. We can say that this
        function is an \textit{ordered triple} and write
        $f=\big(f(0),\,f(1),\,f(1)\big)$, which is more in line with our usual
        notion of higher order products. The main benefit is that we may define
        \textit{arbitrary} products. If we have a set $\mathcal{O}$, we may
        define the \textit{product set}.
        \begin{definition}[\textbf{Product Set}]
            The product set over a set $\mathcal{O}$ is the set
            $\prod\mathcal{O}$ defined by:
            \begin{equation}
                \prod\mathcal{O}=\{\,f:\mathcal{O}\rightarrow\bigcup\mathcal{O}
                    \;|\;f(A)\in{A}\textrm{ for all }A\in\mathcal{O}\,\}
            \end{equation}
        \end{definition}
        Intuitively, this is the set of all \textit{generalized ordered tuples}
        $(a_{0},\,a_{1},\,a_{2},\,\dots)$. The problem with this notation is
        it seems to imply the product is over a countable set, but it need not
        be. We can have \textit{uncountably long} ordered tuples. The ordered
        tuple notation is just for convenience. The definition given above is
        rigorous. If $\mathcal{O}$ is indexed by the natural numbers, we can
        write:
        \begin{equation}
            \prod\mathcal{O}=\prod_{n=0}^{\infty}A_{n}
                =\{\,f:\mathcal{O}\rightarrow\bigcup_{n=0}^{\infty}A_{n}\;|\;
                    f(A_{n})\in{A}_{n}\textrm{ for all }n\in\mathbb{N}\,\}
        \end{equation}
        We may also write:
        \begin{equation}
            \prod\mathcal{O}=\prod_{A\in\mathcal{O}}A=
                \{\,f:\mathcal{O}\rightarrow\bigcup_{A\in\mathcal{O}}A\;|\;
                    f(A)\in{A}\textrm{ for all }A\in\mathcal{O}\,\}
        \end{equation}
        Lastly, if $\mathcal{O}$ is indexed by some indexing set $I$, we can
        write:
        \begin{equation}
            \prod\mathcal{O}=\prod_{\alpha\in{I}}A_{\alpha}=
                \{\,f:\mathcal{O}\rightarrow\bigcup_{\alpha\in{I}}A_{\alpha}
                    \;|\;f(A_{\alpha})\in{A}_{\alpha}\textrm{ for all }
                        \alpha\in{I}\;\}
        \end{equation}
        Now for some controversy. If $\mathcal{O}$ is a non-empty set such that
        for all $A\in\mathcal{O}$ it is true that $A$ is not empty, is
        $\prod\mathcal{O}$ empty? Let's suppose $\mathcal{O}=\{\,A,\,B\,\}$.
        Suppose $a\in{A}$ and $b\in{B}$. The product $\prod\mathcal{O}$
        then consists of the element
        $f:\mathcal{O}\rightarrow{A}\cup{B}$ such that
        $f(A)=a$ and $f(B)=b$, so in particular,
        $\prod\mathcal{O}$ is not empty. Hurray! Indeed, if $\mathcal{O}$ is any
        \textit{finite} set, $\prod\mathcal{O}$ is non-empty, by a similar
        argument. What if $\mathcal{O}$ is countable? We can write each set as
        $A_{n}$ for some $n\in\mathbb{N}$. Since each $A_{n}$ is non-empty,
        there is an element $a_{n}\in{A}_{n}$. We can then define the function
        $f:\mathcal{O}\rightarrow\bigcup_{n=0}^{\infty}A_{n}$ by
        $f(A_{n})=a_{n}$. But hold on! What axiom allows us to define such a
        function? In the finite case, the axioms of set theory can prove the
        function given previously exists, but what about this infinite case?
        As it turns out, the existence of this function can \textit{not} be
        proven using the axioms of set theory that we've so far discussed
        (union, pairing, power set, specification, infinity, extensionality,
        and the other two we have not mentioned). The claim is
        \textit{independent} of these axioms. It is impossible to prove it is
        true and it is impossible to prove it is false. Contrast this with
        Euclidean geometry. The axioms are:
        \begin{enumerate}
            \item Given two points, there is a line segment between them.
            \item Given a line segment, there is a line extending it on either side.
            \item Given a point and a length, there is a circle with the point as
                its center and the length as its radius.
            \item All right angles are equal in size.
            \item If two straight lines fall on a third line, and if the
                sum of the angles the two straight lines make with the third
                sum to less than $\pi$, then the two straight lines intersect.
        \end{enumerate}
        As mentioned a few classes ago, the fifth axiom is equivalent to
        Playfair's axiom.
        \begin{enumerate}
            \item[5a.)] Given a straight lines and point not on that line,
                there is a unique parallel line passing through the point.
        \end{enumerate}
        For 2100 years many mathematicians (several quite rudely) attempted to
        prove the fifth axiom is redundant. That is, to prove the fifth axiom
        can be proven using the first four. In the 1800's C.E. it was shown that
        the fifth axiom is \textit{independent} of the first four, meaning it
        cannot be proven or disproven. The \textit{proofs} other mathematicans
        had that the fifth axiom can be proven simply introduced new
        assumptions (Khayyam introduced the quadrilateral axiom, Playfair
        introduced his axiom, and so on). There are two \textit{models} of
        geometry that obey the first four but not the fifth. In
        hyperbolic geometry, given a point and a line, there are infinitely
        many parallel lines passing through the point. In spherical geometry
        there are no parallel lines.%
        \footnote{%
            Not too important, but spherical geometry does not obey the third
            axiom as it is worded here. It obeys the axiom that
            \textit{given two points, there is a circle with one point as the}
            \textit{center, and the other lying on the circle}.
        }
        \par\hfill\par
        Just like Euclid's fifth axiom can not be proven from the other four,
        the axiom of choice can not be proven from the other axioms of set
        theory. Now you may say, \textit{oh come on, obviously that set is not}
        \textit{empty. In fact, that set is probably huge!} The
        \textit{construction} given above certainly seems convincing, and this
        is why most accept the \textit{axiom of countable choice}.
        \begin{axiom}[\textbf{Axiom of Countable Choice}]
            If $\mathcal{O}$ is a countable non-empty set such that for all
            $A\in\mathcal{O}$ it is true that $A$ is non-empty, then
            $\prod\mathcal{O}$ is non-empty. 
        \end{axiom}
        Richard Dedekind gave another definition of infinite. He says that
        an infinite set is a set $A$ such that there is a proper subset
        $B\subsetneq{A}$ with $\textrm{Card}(A)=\textrm{Card}(B)$. This is
        now called \textit{Dedekind infinite}. A set is
        \textit{Dedekind finite} if it is not Dedekind infinite. The question
        is: Is $A$ finite if and only if $A$ is Dedekind finite? When you
        think about this question, it may eventually become
        \textit{obvious} that these are the same definition. You may say
        \textit{if I through a single element away, I end up with a smaller set}
        \textit{and hence the set must be finite!} Without the axiom of
        countable choice, it is not possible to prove this statement.
        \par\hfill\par
        Let's use the axiom of countable choice to prove that every subset
        of a countable set is countable.
        \begin{theorem}
            If $X$ is a countable set, and $A\subseteq{X}$, then $A$ is
            countable.
        \end{theorem}
        \begin{proof}
            If $X$ is finite, then $A$ is a subset of a finite set, and hence
            finite. Suppose $X$ is countably infinite. If $A$ is finite, we're
            done, so suppose not. Since $A$ is infinite, for all
            $N\in\mathbb{N}$, there exist finite sets $B_{n}\subseteq{A}$,
            $n\in\mathbb{Z}_{N+1}$, such that $\textrm{Card}(B_{n})=n+1$
            and $B_{n}\subseteq{B}_{n+1}$ (if not, $A$ must be finite).
            Let $C_{n}=B_{n+1}\setminus{B}_{n}$. Let
            $\mathcal{O}$ be the set of all $C_{n}$ for all $n\in\mathbb{N}$.
            By the axiom of countable choice, the product set
            $\prod\mathcal{O}$ is not empty ($C_{n}$ is never empty since
            $\textrm{Card}(B_{n})=n+1$). Let $f\in\prod\mathcal{O}$.
            Then $f(C_{n})\in{C}_{n}$ for all $n\in\mathbb{N}$. This function
            is injective. For if not, suppose $m<n$ and
            $f(C_{m})=f(C_{n})$. Then $f(C_{m})\in{C}_{n}$. But
            $C_{n}=B_{n+1}\setminus{B}_{n}$ and
            $C_{m}\subseteq{B}_{n}$ since $m<n$, a
            contradiction. Hence, $f(C_{n})\ne{f}(C_{m})$. Define the injection
            $g:\mathbb{N}\rightarrow{A}$ by $g(n)=f(C_{n})$. We can define an
            injection $h:A\rightarrow\mathbb{N}$ as follows. Since $X$ is
            countable, there is a bijection $F:X\rightarrow\mathbb{N}$.
            By restricting $F$ to $A$ we get an injection from
            $A$ to $\mathbb{N}$. By the Cantor-Schr\"{o}eder-Bernstein theorem,
            there is a bijection $h$ from $A$ to $\mathbb{N}$, and hence
            $A$ is countable.
        \end{proof}
        It would be quite strange if there was a set with size between
        finite and countable, which is one of the reasons the axiom of
        countable choice is accepted without much contraversy. The general
        axiom of choice is not so embraced.
        \begin{axiom}[\textbf{Axiom of Choice}]
            If $\mathcal{O}$ is a non-empty set such that for all
            $A\in\mathcal{O}$ it is true that $A$ is non-empty, then
            $\prod\mathcal{O}$ is non-empty.
        \end{axiom}
        Again, you might be screaming
        \textit{of course that set is non-empty, it's probably uncountable!}
        This is impossible to prove, however. So, we may accept it as true
        and we may accept it as false. For this course, we will accept it as
        true. Here are a few equivalent statements that make it
        \textit{obvious} that the statement is true.
        \begin{theorem}
            If $f:A\rightarrow{B}$ is surjective, then there is a
            right inverse, a function $g:B\rightarrow{A}$ such that
            $(f\circ{g})(b)=b$ for all $b\in{B}$.
        \end{theorem}
        \begin{proof}
            Since $f$ is surjective, for all $b\in{B}$, there is an
            $a\in{A}$ such that $f(a)=b$. \textit{Choose} $g(b)=a$.
        \end{proof}
        We are not allowed to do this \textit{choosing} without the axiom of
        choice. Indeed, if you think this proof should be valid, you are
        accepting the axiom of choice. This statement is
        \textit{equivalent} to the axiom of choice.
        \begin{theorem}
            If $A$ and $B$ are sets, either there is a surjection
            $f:A\rightarrow{B}$ or a surjection $g:B\rightarrow{A}$.
        \end{theorem}
        Again, this is equivalent to choice. There are some algebraic
        equivalents.
        \begin{itemize}
            \item Every vector space has a basis.
            \item Every set has a group structure.
            \item Every ring has a maximal ideal.
        \end{itemize}
        In this course we will prove two topological facts that are also
        equivalent to the axiom of choice.
        \begin{itemize}
            \item The product of compact sets is compact.
            \item The product of connected sets is connected.
        \end{itemize}
        Lastly, an important theorem about quotient sets.
        \begin{theorem}
            If $A$ is a set, and $R$ is an equivalence relation on $A$, then
            there is an injective function $f:A/R\rightarrow{A}$ such that
            $f(\mathcal{U})\in\mathcal{U}$ for all $\mathcal{U}\in{A}/R$. Given
            $\mathcal{U}\in{A}/R$ and $f(\mathcal{U})=x$, we write
            $\mathcal{U}=[x]$ and call this a \textit{representative} for
            $\mathcal{U}$.
        \end{theorem}
        \begin{proof}
            For all $a\in{A}$, $[a]$ is non-empty set since $a\in[a]$. By the
            axiom of choice, there is a function
            $f:A/R\rightarrow\bigcup{A}/R$ such that for all
            $\mathcal{U}\in{A}/R$, $f(\mathcal{U})\in\mathcal{U}$. But
            $\bigcup{A}/R=A$ (Remember, $A/R$ is a set of subsets of $A$, the
            equivalence classes of $A$), so $f$ is such a function from
            $A/R$ to $A$. It is indeed injective. If
            $\mathcal{U}$ and $\mathcal{V}$ are equivalence classes,
            then $f(\mathcal{U})=f(\mathcal{V})$ implies
            $\mathcal{U}=\mathcal{V}$ since
            $f(\mathcal{U})\in\mathcal{V}$ and $f(\mathcal{V})\in\mathcal{U}$
            must be true. Hence, $f$ is injective.
        \end{proof}
        This is another intuitively \textit{obvious} claim.
        \textit{Of course I can pick a representative of an equivalence class!}
        \par\hfill\par
        As you study algebra and topology more, you may become convinced that
        these statements \textit{must} be true, and then you will become a
        defender of the axiom of choice. So why would anyone say it isn't true?
        \begin{theorem}[\textbf{Banach-Tarski Paradox}]
            It is possible to take a sphere, cut it into five disjoint subsets,
            and using only rotation and translation, put the pieces back
            together into two new spheres, both being the same size as the
            original. 
        \end{theorem}
        The proof relies on the axiom of choice. Reading this statement over
        a few times you may think \textit{that's clearly false}.
        \par\hfill\par
        The big theorem that is \textit{clearly false} we will need to use a
        few times. It is also equivalent to the axiom of choice.
        First, a few definitions.
        \begin{definition}[\textbf{Partial Order}]
            A partial order on a set $A$ is a relation $R$ such that
            $R$ is reflexive, transitive, and if $aRb$ and $bRa$, then
            $a=b$.
        \end{definition}
        The two canonical examples are less than or equal to $(\leq)$ and
        inclusion. $a\leq{a}$ is true. $a\leq{b}$ and $b\leq{c}$ implies
        $a\leq{c}$ is also true. Lastly, if $a\leq{b}$ and $b\leq{a}$, then
        $a=b$. Inclusion has the same properties.
        $A\subseteq{A}$, $A\subseteq{B}$ and $B\subseteq{C}$ implies
        $A\subseteq{C}$, and $A\subseteq{B}$ and $B\subseteq{A}$ implies
        $A=B$. Inclusion is perhaps a \textit{better} example because it is
        not a total order.
        \begin{definition}[\textbf{Total Order}]
            A total order on a set $A$ is a partial order $R$ such that for all
            $a,b\in{A}$, either $aRb$ is true, or $bRa$ is true (or both).
        \end{definition}
        $\leq$ is a total order on $\mathbb{R}$, but $\subseteq$ is
        not a total order. Let $A=\{\,1,\,2,\,3\,\}$ and
        $B=\{\,2,\,3,\,4\,\}$. Then $A\not\subseteq{B}$ and
        $B\not\subseteq{A}$, hence $\subseteq$ is not a total order. It is
        still a partial order. There is a special type of total order called
        \textit{well-orderings}.
        \begin{definition}[\textbf{Well-Ordering}]
            A well-ordering on a set $A$ is a total order $\leq$ on $A$ such
            that for every non-empty subset $B\subseteq{A}$ there is a
            least element $a\in{B}$. That is, an element $a\in{B}$ such that
            for all other $b\in{B}$ it is true that $a\leq{b}$.
        \end{definition}
        The natural numbers have this property. The standard ordering on the
        real numbers does \textit{not} have this property. Let
        $A=(0,\,1)$. There is no least element. If you pick a number
        $0<r<1$, I can pick $\frac{r}{2}$ and obtain a smaller number. It may
        seem impossible to order the real numbers in a way that gives a
        well-ordering, but the axiom of choice says it is possible.
        \begin{theorem}
            If $A$ is a set, then there exists a well-ordering $\leq$ on $A$.
        \end{theorem}
        We will talk more about the well-ordering theorem when we discuss the
        \textit{order topology} and the \textit{long line}.
\end{document}
