%-----------------------------------LICENSE------------------------------------%
%   This file is part of Mathematics-and-Physics.                              %
%                                                                              %
%   Mathematics-and-Physics is free software: you can redistribute it and/or   %
%   modify it under the terms of the GNU General Public License as             %
%   published by the Free Software Foundation, either version 3 of the         %
%   License, or (at your option) any later version.                            %
%                                                                              %
%   Mathematics-and-Physics is distributed in the hope that it will be useful, %
%   but WITHOUT ANY WARRANTY; without even the implied warranty of             %
%   MERCHANTABILITY or FITNESS FOR A PARTICULAR PURPOSE.  See the              %
%   GNU General Public License for more details.                               %
%                                                                              %
%   You should have received a copy of the GNU General Public License along    %
%   with Mathematics-and-Physics.  If not, see <https://www.gnu.org/licenses/>.%
%------------------------------------------------------------------------------%
\documentclass{article}
\usepackage{graphicx}                           % Needed for figures.
\usepackage{amsmath}                            % Needed for align.
\usepackage{amssymb}                            % Needed for mathbb.
\usepackage{amsthm}                             % For the theorem environment.
\usepackage{float}
\usepackage{hyperref}
\hypersetup{
    colorlinks=true,
    linkcolor=blue,
    filecolor=magenta,
    urlcolor=Cerulean,
    citecolor=SkyBlue
}

%------------------------Theorem Styles-------------------------%
\theoremstyle{plain}
\newtheorem{theorem}{Theorem}[section]

% Define theorem style for default spacing and normal font.
\newtheoremstyle{normal}
    {\topsep}               % Amount of space above the theorem.
    {\topsep}               % Amount of space below the theorem.
    {}                      % Font used for body of theorem.
    {}                      % Measure of space to indent.
    {\bfseries}             % Font of the header of the theorem.
    {}                      % Punctuation between head and body.
    {.5em}                  % Space after theorem head.
    {}

% Define default environments.
\theoremstyle{normal}
\newtheorem{examplex}{Example}[section]
\newtheorem{definitionx}{Definition}[section]
\newtheorem{notationx}{Notation}[section]
\newtheorem{axiomx}{Axiom}[section]

\newenvironment{example}{%
    \pushQED{\qed}\renewcommand{\qedsymbol}{$\blacksquare$}\examplex%
}{%
    \popQED\endexamplex%
}

\newenvironment{definition}{%
    \pushQED{\qed}\renewcommand{\qedsymbol}{$\blacksquare$}\definitionx%
}{%
    \popQED\enddefinitionx%
}

\title{Point-Set Topology: Lecture 13}
\author{Ryan Maguire}
\date{Summer 2022}

% No indent and no paragraph skip.
\setlength{\parindent}{0em}
\setlength{\parskip}{0em}

\begin{document}
    \maketitle
    \section{First Countable}
        So far we've seen that sequential spaces are the nice ones
        (and Hausdorff spaces. Sequential \textit{and} Hausdorff is just
        the bee's knees). We also have no tools for determining if a space is
        sequential or not. One of the most mild conditions one can impose on a
        topological space is that it be \textit{first-countable}. Most of the
        topological spaces studied are first-countable, with some crucial
        exceptions (the Zariski topology on $\mathbb{R}$ is not
        first-countable). We now proceed to describe this notion, give examples,
        prove some theorems, and show how first-countable spaces relate to
        sequential spaces.
        \begin{definition}[\textbf{Neighborhood Basis}]
            A neighborhood basis in a topological space $(X,\,\tau)$ of a
            point $x\in{X}$ is a subset $\mathcal{B}\subseteq\tau$ such that
            for all $\mathcal{U}\in\mathcal{B}$ it is true that
            $x\in\mathcal{U}$, and for all $\mathcal{V}\in\tau$ such that
            $x\in\mathcal{V}$, there is a $\mathcal{U}\in\mathcal{B}$ with
            $\mathcal{U}\subseteq\mathcal{V}$.
        \end{definition}
        First-countable spaces are defined in terms of neighborhood bases.
        \begin{definition}[\textbf{First-Countable Topological Space}]
            A first-countable topological space is a topological space
            $(X,\,\tau)$ such that for all $x\in{X}$ there exists a countable
            subset $\mathcal{B}\subseteq\tau$ such that
            $\mathcal{B}$ is a neighborhood basis for $x$.
        \end{definition}
        \begin{example}
            The real line with the standard topology is first-countable.
            Given $x\in\mathbb{R}$, define
            $\mathcal{U}_{n}=(x-\frac{1}{n+1},\,x+\frac{1}{n+1})$. The set
            $\mathcal{B}=\{\,\mathcal{U}_{n}\;|\;n\in\mathbb{N}\,\}$ forms a
            neighborhood basis for $x$. Given any open
            $\mathcal{V}\subseteq\mathbb{R}$ with $x\in\mathcal{V}$, there is
            an $\varepsilon>0$ such that for all $y\in\mathbb{R}$ with
            $|x-y|<\varepsilon$, it is true that $y\in\mathcal{V}$. Choosing
            $n$ so that $\frac{1}{n+1}<\varepsilon$ shows that
            $\mathcal{U}_{n}\subseteq\mathcal{V}$.
        \end{example}
        \begin{example}
            All Euclidean spaces are first-countable as well. We can apply a
            similar trick in $\mathbb{R}^{n}$ by surrounding
            $\mathbf{x}\in\mathbb{R}^{n}$ with countably many open balls of
            radius $\frac{1}{n+1}$.
        \end{example}
        The ideas behind these two examples only require the existence of
        open sets that get \textit{smaller}, in a sense. Metric spaces have
        such a notion, \textit{open balls}.
        \begin{theorem}
            If $(X,\,\tau)$ is a metrizable topological space, then
            it is first-countable.
        \end{theorem}
        \begin{proof}
            Since $(X,\,\tau)$ is metrizable, there is a metric $d$ on $X$ such
            that $\tau=\tau_{d}$, where $\tau_{d}$ is the metric topology. Let
            $x\in{X}$ and define:
            \begin{equation}
                \mathcal{B}=\{\,B_{\frac{1}{n+1}}^{(X,\,d)}(x)\;|\;
                    n\in\mathbb{N}\,\}
            \end{equation}
            $\mathcal{B}$ is a countable neighborhood basis of $x$. First, since
            $\frac{1}{n+1}>0$ for all $n\in\mathbb{N}$, these open balls are
            non-empty and all contain the point $x$. Next, the set is countable
            since the elements are indexed by the natural numbers. Lastly,
            given any $\mathcal{U}\in\tau_{d}$ with $x\in\mathcal{U}$, by the
            definition of the metric topology there is an
            $\varepsilon>0$ such that:
            \begin{equation}
                B_{\varepsilon}^{(X,\,d)}(x)\subseteq\mathcal{U}
            \end{equation} 
            Choose $n\in\mathbb{N}$ such that $\frac{1}{n+1}<\varepsilon$. Then:
            \begin{equation}
                B_{\frac{1}{n+1}}^{(X,\,d)}(x)
                \subseteq{B}_{\varepsilon}^{(X,\,d)}(x)
                \subseteq\mathcal{U}
            \end{equation}
            hence $\mathcal{B}$ is a countable neighborhood basis of $x$,
            so $(X,\,\tau)$ is first-countable.
        \end{proof}
        First-countable always implies sequential. It is often easier to prove a
        particular space is first-countable rather than prove it is sequential,
        so the following theorem has many uses in topology.
        \begin{theorem}
            If $(X,\,\tau)$ is a first-countable topological space, then it is
            sequential.
        \end{theorem}
        \begin{proof}
            Suppose not. Then there is $\mathcal{U}\subseteq{X}$ that is
            sequentially open but not open. Since $\mathcal{U}$ is not open,
            there is an $x\in\mathcal{U}$ such that for all $\mathcal{V}\in\tau$
            with $x\in\mathcal{V}$, it is not true that
            $\mathcal{V}\subseteq\mathcal{U}$ (otherwise we could write
            $\mathcal{U}$ as the union of all such $\mathcal{V}_{x}$, which is
            the union of open sets, and hence open, but $\mathcal{U}$ is not
            open). Since $(X,\,\tau)$ is first countable, there is a countable
            neighborhood basis $\mathcal{B}$ of $x$. Since $\mathcal{B}$ is
            countable, there is a surjection
            $\mathcal{V}:\mathbb{N}\rightarrow\mathcal{B}$. That is, we can
            list the elements of $\mathcal{B}$ as:
            \begin{equation}
                \mathcal{B}=
                \{\,\mathcal{V}_{0},\,\dots,\,\mathcal{V}_{n},\,\dots\,\}
            \end{equation}
            Define $\mathcal{W}_{n}$ via:
            \begin{equation}
                \mathcal{W}_{n}=\bigcap_{k=0}^{n}\mathcal{V}_{k}
            \end{equation}
            For all $n\in\mathbb{N}$, $\mathcal{W}_{n}$ is open since it is
            the finite intersection of open sets. But $x\in\mathcal{W}_{n}$,
            meaning there is an $a_{n}\in\mathcal{W}_{n}$ such that
            $a_{n}\notin\mathcal{U}$ (again, since $\mathcal{U}$ is not open
            and $x\in\mathcal{U}$ was chosen so that there are no open subsets
            of $X$ that contain $x$ and fit inside $\mathcal{U}$). We must
            show $a_{n}\rightarrow{x}$. Let
            $\mathcal{E}\in\tau$ be an open set with $x\in\mathcal{E}$. Since
            $\mathcal{B}$ is a neighborhood basis, there is an
            $N\in\mathbb{N}$ such that $\mathcal{V}_{N}\subseteq\mathcal{E}$.
            But then for all $n>N$, by the definition of $\mathcal{W}_{n}$,
            we have that $\mathcal{W}_{n}\subseteq\mathcal{V}_{N}$ and hence
            $\mathcal{W}_{n}\subseteq\mathcal{E}$. But then for all $n>N$ we
            have $a_{n}\in\mathcal{E}$. Hence, $a_{n}\rightarrow{x}$. But
            $\mathcal{U}$ is sequentially open and $x\in\mathcal{U}$, so if
            $a_{n}\rightarrow{x}$, then there is an $N\in\mathbb{N}$ such that
            for all $n\in\mathbb{N}$ with $n>N$ we have $a_{n}\in\mathcal{U}$.
            But by definition of the sequence, for all $n\in\mathbb{N}$,
            $a_{n}\notin\mathcal{U}$, a contradiction. Hence,
            $\mathcal{U}$ is open and $(X,\,\tau)$ is sequential.
        \end{proof}
    \section{Second Countable}
        Second-countable is a much stronger notion than first-countable. All of
        the spaces the human brain can hope to visualize are second-countable.
        Spaces we can envision are subsets of $\mathbb{R}^{n}$ for
        some $n\in\mathbb{N}$ (in fact, probably just 0, 1, 2, 3, and
        \textit{maybe} 4 if you're really good). In other words, metric
        subspaces of Euclidean
        spaces. Any metric subspace of $\mathbb{R}^{n}$ is second-countable,
        so if you are to try and picture a topological space (without lying to
        yourself), the space better be second-countable. With that, I give a
        definition.
        \begin{definition}[\textbf{Second-Countable Topological Space}]
            A second-countab-\\
            le topological space is a topological
            space $(X,\,\tau)$ such that there exists a countable basis
            $\mathcal{B}$ for the topology $\tau$.
        \end{definition}
        Second-countable spaces are \textit{not too big}.
        \begin{example}
            The real line, with the standard topology, is second-countable.
            Let $\mathcal{B}$ be the set of all intervals
            $(a,\,b)$ with $a,b\in\mathbb{Q}$ and $a<b$. This set has the
            cardinality of $\mathbb{Q}\times\mathbb{Q}$, which is countable.
            It is also a basis, essentially because the rationals are dense
            in $\mathbb{R}$. So $\mathcal{B}$ is a countable basis for the
            real line, meaning $\mathbb{R}$ is second-countable.
        \end{example}
        \begin{example}
            All Euclidean spaces are second-countable. We can apply a similar
            trick, take all points $\mathbf{x}\in\mathbb{R}^{n}$ where every
            coordinate in $\mathbf{x}$ is rational. That is, if
            $\mathbf{x}=(x_{0},\,\dots,\,x_{n-1})$, require that
            $x_{k}\in\mathbb{Q}$ for all $k\in\mathbb{Z}_{n}$. About each point
            collect all open balls with rational radii. This set has cardinality
            $\mathbb{Q}^{n+1}$, which again is countable
            (essentially by induction). This also forms a basis, showing that
            $\mathbb{R}^{n}$ is second-countable.
        \end{example}
        \begin{example}
            Not every metrizable space is second-countable. Equip
            $\mathbb{R}$ with the discrete topology $\mathcal{P}(\mathbb{R})$.
            Since all of the singletons $\{\,x\,\}$ are open, any basis
            $\mathcal{B}$ must include a copy of
            $\{\,x\,\}$ for each $x\in\mathbb{R}$, meaning $\mathcal{B}$ cannot
            be countable since $\mathbb{R}$ is uncountable. The space is
            metrizable, however, since it comes from the discrete metric.
        \end{example}
        \begin{theorem}
            If $(X,\,\tau)$ is second-countable, then it is first-countable.
        \end{theorem}
        \begin{proof}
            Since $(X,\,\tau)$ is second-countable, there is a countable
            basis $\mathcal{B}$. Given $x\in{X}$, define
            $\mathcal{B}_{x}$ via:
            \begin{equation}
                \mathcal{B}_{x}=\{\,\mathcal{U}\in\mathcal{B}\;|\;
                    x\in\mathcal{U}\,\}
            \end{equation}
            Since $\mathcal{B}_{x}$ is a subset of a countable set, it is
            countable. It is also a neighborhood basis of $x$. By definition
            every element of $\mathcal{B}_{x}$ contains $x$. Suppose
            $\mathcal{V}\in\tau$ is an open set such that
            $x\in\mathcal{V}$. Since $\mathcal{B}$ is a basis, there is a
            $\mathcal{U}\in\mathcal{B}$ such that
            $x\in\mathcal{U}$ and $\mathcal{U}\subseteq\mathcal{V}$. But since
            $x\in\mathcal{U}$ it is true that $\mathcal{U}\in\mathcal{B}_{x}$.
            But then $\mathcal{U}$ is an element of $\mathcal{B}_{x}$ such that
            $\mathcal{U}\subseteq\mathcal{V}$. Hence, $(X,\,\tau)$ is
            first-countable.
        \end{proof}
        The converse does not hold. We've shown that all metrizable spaces are
        first-countable, but the discrete topology on $\mathbb{R}$ is a
        metrizable space that is not second-countable. That is,
        $\big(\mathbb{R},\,\mathcal{P}(\mathbb{R})\big)$ is a first-countable
        space that is not second-countable.
    \section{Separable}
        Separable spaces are important in analysis and geometry.
        The real line has the property that there is a countable subset
        $\mathbb{Q}$ that can approximate every point on the line. We take this
        idea to motivate the term separable.
        \begin{definition}[\textbf{Separable Topological Space}]
            A separable topological space is a topological space $(X,\,\tau)$
            such that there exists a countable dense subset $A\subseteq{X}$.
            That is, $A$ is countable and $\textrm{Cl}_{\tau}(A)=X$.
        \end{definition}
        \begin{example}
            The real line is separable, taking $A=\mathbb{Q}$ gives us a
            countable dense subset. $\mathbb{R}^{n}$ is also separable, setting
            $A=\mathbb{Q}^{n}$ shows there is a countable dense subset of
            $\mathbb{R}^{n}$.
        \end{example}
        The two previous examples of separable spaces are also second-countable.
        This is not a coincidence, every second-countable space is
        separable.
        \begin{theorem}
            If $(X,\,\tau)$ is second-countable, then it is separable.
        \end{theorem}
        \begin{proof}
            Since $(X,\,\tau)$ is second-countable, there exists a countable
            basis $\mathcal{B}'$ for $\tau$. Let
            $\mathcal{B}\subseteq\mathcal{B}'$ be the set
            of all non-empty elements of $\mathcal{B}'$. Since
            $\mathcal{B}'$ is a countable basis, so is $\mathcal{B}$
            (we're only removing the empty set). $\mathcal{B}$ now has the
            property that for all $\mathcal{V}\in\mathcal{B}$,
            $\mathcal{V}$ is non-empty. Since $\mathcal{B}$ is countable, there
            is a surjection $\mathcal{U}:\mathbb{N}\rightarrow\mathcal{B}$. That
            is, we may list the elements as:
            \begin{equation}
                \mathcal{B}=
                \{\,\mathcal{U}_{0},\,\dots,\,\mathcal{U}_{n},\,\dots\,\}
            \end{equation}
            Since $\mathcal{U}_{n}$ is non-empty for all $n\in\mathbb{N}$, by
            the axiom of countable choice we can find a sequence
            $a:\mathbb{N}\rightarrow{X}$ such that $a_{n}\in\mathcal{U}_{n}$
            for all $n\in\mathbb{N}$. Define $A$ by:
            \begin{equation}
                A=\{\,a_{n}\in{X}\;|\;n\in\mathbb{N}\,\}
            \end{equation}
            $A$ is countable since
            it is index by the natural numbers. It is also dense. For suppose
            not. A set $B$ is dense if and only if for every non-empty open
            subset $\mathcal{V}\in\tau$ the intersection $B\cap\mathcal{V}$ is
            non-empty. Since we are supposing $A$ is not dense, there must be a
            non-empty open subset $\mathcal{V}$ such that
            $A\cap\mathcal{V}$ is empty. But since $\mathcal{V}$ is open and
            $\mathcal{B}$ is a basis, there is a $\mathcal{U}_{n}\in\mathcal{B}$
            such that $\mathcal{U}_{n}\subseteq\mathcal{V}$. But
            $a_{n}\in\mathcal{U}_{n}$, so $a_{n}\in\mathcal{V}$. But
            $a_{n}\in{A}$, which is a contradiction since
            $A\cap\mathcal{V}=\emptyset$. Hence, $A$ is dense.
        \end{proof}
        \begin{example}[\textbf{The Particular Point Topology}]
            This theorem does not reverse. There are topological spaces that
            are separable but not second-countable. The easiest example to
            describe is the \textit{particular point topology} on
            $\mathbb{R}$. Define $\mathcal{U}\subseteq\mathbb{R}$ to be open in
            $\tau$ if and only if $\mathcal{U}=\emptyset$ or $0\in\mathcal{U}$.
            This space is separable since $A=\{\,0\,\}$ is a dense subset and it
            is certainly countable since it is finite. $A$ is dense since the
            only closed set containing $\{\,0\,\}$ is $\mathbb{R}$.
            Given any closed set $\mathcal{C}$, $\mathbb{R}\setminus\mathcal{C}$
            is open, meaning either $\mathbb{R}\setminus\mathcal{C}$ is
            empty, or $0\in\mathbb{R}\setminus\mathcal{C}$. So if
            $\mathcal{C}$ is a closed set with $0\in\mathcal{C}$, it must be
            true that $\mathbb{R}\setminus\mathcal{C}=\emptyset$. Hence, the
            only closed set contain $0$ is $\mathbb{R}$. So,
            $\textrm{Cl}_{\tau}\big(\{\,0\,\}\big)=\mathbb{R}$. This space is
            not second-countable. Every set of the form
            $\{\,0,\,x\,\}$ for all $x\in\mathbb{R}$ is open, so a basis must
            contain a copy of each of these. The cardinality of such a basis
            must then be at least as big as $\mathbb{R}$,
            which is uncountable, meaning $(\mathbb{R},\,\tau)$ is not
            second-countable.
        \end{example}
        \begin{theorem}
            If $(X,\,\tau)$ is metrizable and separable, then it is
            second-countable.
        \end{theorem}
        \begin{proof}
            Since $(X,\,\tau)$ is metrizable, there is a metric $d$ on $X$
            such that $\tau=\tau_{d}$, where $\tau_{d}$ is the metric topology
            from $d$. Since $(X,\,\tau)$ is separable, there is a countable
            dense subset $A\subseteq{X}$. Define $\mathcal{B}$ via:
            \begin{equation}
                \mathcal{B}=
                \{\,B_{r}^{(X,\,d)}(x)\;|\;
                    x\in{A}\textrm{ and }q\in\mathbb{Q}^{+}\,\}
            \end{equation}
            That is, $\mathcal{B}$ is the set of all balls of rational radii
            centered at all points in $A$. Since $A$ and $\mathbb{Q}$ are
            countable, the set $\mathcal{B}$ is countable as well.
            $\mathcal{B}$ is a basis. It covers $X$, for given $y\in{X}$,
            pick any $x\in{A}$ and $r\in\mathbb{Q}^{+}$ such that
            $r>d(x,\,y)$. Then $y\in{B}_{r}^{(X,\,d)}(x)$, which is an element
            of $\mathcal{B}$, so $\mathcal{B}$ is an open cover of $(X,\,\tau)$.
            Suppose $\mathcal{U},\mathcal{V}\in\tau$ and
            $x\in\mathcal{U}\cap\mathcal{V}$. Since $\mathcal{U}\cap\mathcal{V}$
            is open and $x\in\mathcal{U}\cap\mathcal{V}$ there is an
            $r>0$ such that
            $B_{r}^{(X,\,d)}(x)\subseteq\mathcal{U}\cap\mathcal{V}$. But since
            $r>0$, $r/4>0$ as well, so $B_{r/4}^{(X,\,d)}(x)$ is open. But
            $A$ is dense, so there is a $y\in{A}$ such that
            $y\in{B}_{r/4}^{(X,\,d)}(x)$. Choose
            $\varepsilon\in\mathbb{Q}^{+}$ such that $r/4<\varepsilon<r/2$.
            Then $B_{\varepsilon}^{(X,\,d)}(y)\subseteq{B}_{r}^{(X,\,d)}(x)$,
            so $B_{\varepsilon}^{(X,\,d)}(y)\subseteq\mathcal{U}\cap\mathcal{V}$
            but also $x\in{B}_{\varepsilon}^{(X,\,d)}(y)$. That is, we've found
            an element of $\mathcal{B}$ that contains $x$ and fits inside of
            $\mathcal{U}\cap\mathcal{V}$. Hence, $\mathcal{B}$ is a countable
            basis. Therefore, $(X,\,\tau)$ is second countable.
        \end{proof}
        The trick \textit{seems} to hint that a separable first-countable space
        should be second-countable, but this is \textbf{false}. The metric
        topology was very helpful in this proof in some subtle ways.
        \begin{example}
            The particular point topology on $\mathbb{R}$ is separable
            (as we've already seen) and first-countable, but not
            second-countable (again, we saw this in a previous example). To
            show that it is first-countable, pick any $x\in\mathbb{R}$. The
            set $\mathcal{B}=\big\{\,\{\,0,\,x\,\}\,\big\}$ is a neighborhood
            basis of $x$. Given any open $\mathcal{U}\subseteq\mathbb{R}$ that
            contains $x$, it must also contain $0$ by the definition of the
            particular point topology. Hence
            $\{\,0,\,x\,\}\subseteq\mathcal{U}$. But also
            $\{\,0,\,x\,\}$ is open. This shows $\mathcal{B}$ is a
            neighborhood basis of $x$, and it is finite, hence countable.
        \end{example}
\end{document}
