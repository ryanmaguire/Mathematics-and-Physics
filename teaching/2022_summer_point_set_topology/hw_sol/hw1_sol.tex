%-----------------------------------LICENSE------------------------------------%
%   This file is part of Mathematics-and-Physics.                              %
%                                                                              %
%   Mathematics-and-Physics is free software: you can redistribute it and/or   %
%   modify it under the terms of the GNU General Public License as             %
%   published by the Free Software Foundation, either version 3 of the         %
%   License, or (at your option) any later version.                            %
%                                                                              %
%   Mathematics-and-Physics is distributed in the hope that it will be useful, %
%   but WITHOUT ANY WARRANTY; without even the implied warranty of             %
%   MERCHANTABILITY or FITNESS FOR A PARTICULAR PURPOSE.  See the              %
%   GNU General Public License for more details.                               %
%                                                                              %
%   You should have received a copy of the GNU General Public License along    %
%   with Mathematics-and-Physics.  If not, see <https://www.gnu.org/licenses/>.%
%------------------------------------------------------------------------------%
\documentclass{article}
\usepackage{graphicx}                           % Needed for figures.
\usepackage{amsmath}                            % Needed for align.
\usepackage{amssymb}                            % Needed for mathbb.
\usepackage{amsthm}                             % For the theorem environment.
\usepackage{float}
\usepackage{hyperref}
\hypersetup{
    colorlinks=true,
    linkcolor=blue,
    filecolor=magenta,
    urlcolor=Cerulean,
    citecolor=SkyBlue
}

%------------------------Theorem Styles-------------------------%

% Define theorem style for default spacing and normal font.
\newtheoremstyle{normal}
    {\topsep}               % Amount of space above the theorem.
    {\topsep}               % Amount of space below the theorem.
    {}                      % Font used for body of theorem.
    {}                      % Measure of space to indent.
    {\bfseries}             % Font of the header of the theorem.
    {}                      % Punctuation between head and body.
    {.5em}                  % Space after theorem head.
    {}

% Define default environments.
\theoremstyle{normal}
\newtheorem{problem}{Problem}

\title{Point-Set Topology: Homework 1}
\date{Summer 2022}

% No indent and no paragraph skip.
\setlength{\parindent}{0em}
\setlength{\parskip}{0em}

\begin{document}
    \maketitle
    \color{blue}
    \begin{problem}
        This problem explores the use of the word \textit{or} in mathematics.
        You will prove the \textit{principle of explosion}. The principle says
        that if $P$ is a statement that is both true and false, then for any
        statement $Q$, $Q$ is true. A system that contains a sentence that is
        both true and false is called \textit{inconsistent}. The principle of
        explosion shows that inconsistent systems are very boring.
        \begin{itemize}
            \item (2 Points) Let $P$ be a statement that is true, and let $Q$
                be any other claim. Since $P$ is true, what can you conclude
                about $P$ \textit{or} $Q$?
            \item (2 Points) Suppose $P$ is also false. Using what you've
                concluded about $P$ \textit{or} $Q$, what can you prove about
                $Q$?
        \end{itemize}
    \end{problem}
    \color{black}
    \begin{proof}[Solution.]
        The truth table for logical or, also called \textit{disjunction},
        which is represented by $\lor$ is given in
        Tab.~\ref{tab:logical_or}. We see that
        $P$ \textit{or} $Q$ is false only when both $P$ and $Q$ are both
        false simultaneously. Since we have a statement $P$ and we are told
        $P$ is true, regardless of the statement $Q$ we may conclude that
        $P$ \textit{or} $Q$ is a true sentence. Using the notation of
        mathematical logic, we have that $P\lor{Q}$ is true.
        \par\hfill\par
        Now we suppose also that $P$ is false. Again appealing to the truth
        table, since $P\lor{Q}$ is a true statement, and since $P$ is a false
        statement, $Q$ must also be a true statement (otherwise
        $P\lor{Q}$ would be false since both $P$ and $Q$ are false). Therefore
        if $P$ is a statement that is both true and false, and if $Q$ is any
        statement, then $Q$ is true.
    \end{proof}
    \begin{table}
        \centering
        \begin{tabular}{ c | c | c }
            $P$&$Q$&$P\lor{Q}$\\
            \hline
            False&False&False\\
            False&True&True\\
            True&False&True\\
            True&True&True
        \end{tabular}
        \caption{Truth Table for Logical Or}
        \label{tab:logical_or}
    \end{table}
    \clearpage
    \color{blue}
    \begin{problem}
        Here you will explore more set theory. You will prove the
        \textit{axiom of unrestricted comprehension} is inconsistent. The axiom
        allows you to construct a set arbitrarily using any sentence. That is,
        if $P(x)$ is a sentence, you may collect all $x$ such that $P(x)$ is
        true. You could write:
        \begin{equation}
            A=\{\,x\;|\;P(x)\,\}\nonumber
        \end{equation}
        \begin{itemize}
            \item (1 Point) Let $P(x)$ be the sentence $x$ \textit{is a set}.
                Let $A$ be the set $A=\{\,x\;|\;P(x)\,\}$. That is:
                \begin{equation}
                    A=\{\,x\;|\;x\textrm{ is a set}\,\}\nonumber
                \end{equation}
                Describe the set $A$. Is it true that $A\in{A}$?
            \item (2 Points) Let $B$ be the set
                \begin{equation}
                    B=\{\,x\in{A}\;|\;x\notin{x}\,\}\nonumber
                \end{equation}
                Prove that $B\in{B}$. (Hint: Suppose $B\notin{B}$ and arrive
                at a contradiction).
            \item (2 Points) Now prove that $B\notin{B}$.
                (Hint: Same as before. Suppose $B\in{B}$ and arrive at a
                contradiction).
            \item (1 Point) Using the previous problem, why should we not accept
                the axiom of unrestricted comprehension as true?
        \end{itemize}
    \end{problem}
    \color{black}
    \begin{proof}[Solution]
        The set $A$ is, in plain English, the \textit{set of all sets}. Since it
        is itself a set, by definition, it must be true that $A\in{A}$.
        \par\hfill\par
        The set $B$ is the set of all sets that do not contain themselves.
        I believe Russell called these \textit{proper} sets, but I can't quite
        recall. To prove $B\in{B}$ we suppose not. Then $B\notin{B}$. But if
        $B\notin{B}$, then by the definition of $B$, $B\in{B}$, which is a
        contradiction. We conclude that $B\in{B}$.
        \par\hfill\par
        Now we prove $B\notin{B}$. Again, we prove by contradiction. Suppose
        $B\in{B}$. But by definition of $B$, $B$ is the set of all sets that
        do not contain themselves, so $B$ can't be an element of $B$, a
        contradiction. Hence, $B\notin{B}$.
        \par\hfill\par
        This shows the axiom of unrestricted comprehension is invalid. We cannot
        be allowed to form sets via any sentence, we must restrict ourselves
        to applying sentences to sets we already know exist. The principle of
        explosion, as demonstrated in the first problem, shows that if we
        allow the axiom of unrestricted comprehension, then \textit{every}
        statement is both true and false.
    \end{proof}
    \clearpage
    \color{blue}
    \begin{problem}
        The axiom of infinity tells us $\mathbb{N}=\{\,0,\,1,\,2,\,\dots\,\}$
        exists. It does not tell us
        $\mathbb{Z}=\{\,\dots,\,-2,\,-1,\,0,\,1,\,2,\,\dots\,\}$ exists, but we
        can construct it. For this problem we can assume addition $(+)$ for
        natural numbers exists.
        \begin{itemize}
            \item (2 Points) Consider the set $\mathbb{N}\times\mathbb{N}$.
                Define $R$ to be the relation $(a,b)R(c,d)$ if and only if
                $a+d=b+c$. Prove this is an equivalence relation.
            \item (1 Point) Describe the equivalence class of $(a,b)$
                geometrically. (Hint: $\mathbb{N}\times\mathbb{N}$ is a lattice
                of points in the plane. Describe the equivalence class of
                $(a,b)$ using this lattice).
            \item (2 Points) For equivalence classes $[(a,b)]$ and $[(c,d)]$,
                define $[(a,b)]+[(c,d)]=[(a+c,b+d)]$. Prove this is
                well-defined.
            \item (1 Point) With this construction we now write
                (for convenience) $0=[(0,0)]$, $n=[(n,0)]$, and $-n=[(0,n)]$.
                Justifiy this notation (Does $[(0,0)]$ behave like 0? Does
                $[(0,n)]$ act like $-n$?)
        \end{itemize}
    \end{problem}
    \color{black}
    \begin{proof}[Solution.]
        $R$ is reflexive. Given $(a,\,b)\in\mathbb{N}\times\mathbb{N}$, we have
        $a+b=b+a$ by the commutative law of addition, and therefore
        $(a,\,b)R(a,\,b)$.
        \par\hfill\par
        $R$ is also symmetric. If $(a,\,b)R(c,\,d)$, then
        $a+d=b+c$. But equality is symmetric, so $b+c=a+d$. But addition is
        commutative, so $c+b=d+a$. That is, $(c,\,d)R(a,\,b)$.
        \par\hfill\par
        Lastly, $R$ is transitive. If $(a,\,b)R(c,\,d)$, then $a+d=b+c$.
        If $(c,\,d)R(e,\,f)$, then $c+f=d+e$. But then:
        \begin{align}
            (a+f)+(c+d)
            &=(a+d)+(c+f)\tag{Associativity and Commutativity}\\
            &=(b+c)+(c+f)\tag{Substitution}\\
            &=(b+c)+(d+e)\tag{Substitution}\\
            &=(b+e)+(c+d)\tag{Associativity and Commutativity}\\
            \Rightarrow(a+f)+(c+d)&=(b+e)+(c+d)
            \tag{Transitivity of Equality}
        \end{align}
        By the cancellative property of addition, $a+f=b+e$. That is,
        $(a,\,b)R(e,\,f)$. Since $R$ is reflexive, symmetric, and transtive,
        $R$ is an equivalence relation.
        \par\hfill\par
        Next, consider the equivalence class of $(0,\,0)$, the set
        $[(0,\,0)]$. This is the set of all
        $(m,\,n)\in\mathbb{N}\times\mathbb{N}$ such that
        $0+n=0+m$, which is the same as $n=m$. This is the \textit{diagonal}
        through the Cartesian plane, the set of points $(n,\,n)$ in the lattice
        $\mathbb{N}\times\mathbb{N}$. What about $[(1,\,0)]$? This is the set
        of points $(m,\,n)$ such that $1+n=0+m$, so $m=1+n$. This is the set
        of points that lie on the line of slope 1 passing through the point
        $(1,\,0)$. In general, $[(x_{0},\,y_{0})]$ is the set of all $(x,\,y)$
        with $y+x_{0}=x+y_{0}$. This property is precisely the property of a
        line with slope 1 passing through the point $(x_{0},\,y_{0})$. The
        equivalence class of $[(m,\,n)]$ is a straight line through the
        lattice $\mathbb{N}\times\mathbb{N}$ of slope 1 containing
        $(m,\,n)$. This is depicted in
        Fig.~\ref{fig:grothendique_construction_of_integers_001}.
        \par\hfill\par
        Addition of equivalence classes is well-defined here. Let
        $[(a,\,b)]=[(x,\,y)]$ and $[(c,\,d)]=[(z,\,w)]$. We need to show
        that $[(a+c,\,b+d)]=[(x+z,\,y+w)]$. Since $[(a,\,b)]=[(x,\,y)]$, and
        since $R$ is an equivalence relation, we know that
        $(a,\,b)R(x,\,y)$. That is, $a+y=b+x$. Similarly, $(c,\,d)R(z,\,w)$, so
        $c+w=d+z$. Then:
        \begin{align}
            a+c+y+w
            &=(a+y)+(c+w)\tag{Associativity and Commutativity}\\
            &=(b+x)+(c+w)\tag{Substitution}\\
            &=(b+x)+(d+z)\tag{Substitution}\\
            &=b+d+x+z\tag{Associativity and Commutativity}\\
            \Rightarrow
            a+c+y+w
            &=b+d+x+z\tag{Transitivity of Equality}
        \end{align}
        So $(a+c,\,b+d)R(x+z,\,y+w)$, and hence
        $[(a+c,\,b+d)]=[(x+z,\,y+w)]$, so addition is well-defined.
        \par\hfill\par
        Lastly, $[(0,\,0)]$ behaves like the additive identity. This is the
        arithmetic property of \textit{zero}. Given $[(m,\,n)]$, we have
        $[(0,\,0)]+[(m,\,n)]=[(0+m,\,0+n)]=[(m,\,n)]$, so addition by
        $[(0,\,0)]$ does not change anything. The \textit{number}
        $[(0,\,n)]$ also behaves like the negative of $[(n,\,0)]$ since
        $[(n,\,0)]+[(0,\,n)]=[(n,\,n)]=[(0,\,0)]$.
    \end{proof}
    \begin{figure}
        \centering
        \includegraphics{../../../images/grothendique_construction_of_integers_001.pdf}
        \caption{Equivalence Class of Points}
        \label{fig:grothendique_construction_of_integers_001}
    \end{figure}
    \clearpage
    \color{blue}
    \begin{problem}
        Now that we have constructed $\mathbb{Z}$, let's construct $\mathbb{Q}$,
        the set of rational numbers. Consider the set
        $\mathbb{Z}\times(\mathbb{Z}\setminus\{\,0\,\})$. Define the relation
        $R$ by $(a,b)R(c,d)$ if and only if $ad=bc$. (We are assuming we have
        already constructed multiplication for integers $n\in\mathbb{Z}$).
        \begin{itemize}
            \item (2 Points) Prove $R$ is an equivalence relation.
            \item (2 Points) For equivalence classes $[(a,b)]$ and $[(c,d)]$,
                define $[(a,b)]+[(c,d)]=[(ad+bc,bd)]$ (This is
                \textit{cross-multiplying}). Prove this is well-defined.
        \end{itemize}
        With this we write $[(a,b)]=\frac{a}{b}$.
    \end{problem}
    \color{black}
    \begin{proof}[Solution.]
        $R$ is reflexive. Given $(a,\,b)$ we have $ab=ba$ by the commutative
        property of multiplication, so $(a,\,b)R(a,\,b)$.
        \par\hfill\par
        $R$ is symmetric. If $(a,\,b)R(c,\,d)$, then $ad=bc$. But then
        $bc=ad$ since equality is symmetric. But then
        $cb=da$ since multiplication is commutative. Therefore,
        $(c,\,d)R(a,\,b)$.
        \par\hfill\par
        Lastly, $R$ is transitive. If $(a,\,b)R(c,\,d)$, then
        $ad=bc$. If $(c,\,d)R(e,\,f)$, then $cf=de$. Using this we have:
        \begin{align}
            (af)d
            &=(ad)f\tag{Associativity and Commutativity}\\
            &=(bc)f\tag{Substitution}\\
            &=b(cf)\tag{Associativity}\\
            &=b(de)\tag{Substitution}\\
            &=(be)d\tag{Associativity and Commutativity}
        \end{align}
        So $afd=bed$. But $d\in\mathbb{Z}\setminus\{\,0\,\}$, so $d$ is
        non-zero, and hence by the cancellation law of multiplication,
        $af=be$. That is, $(a,\,b)R(e,\,f)$, so $R$ is transitive. Since
        $R$ is reflexive, symmetric, and transitive, it is an equivalence
        relation.
        \par\hfill\par
        Addition of equivalence classes is well-defined here. Given
        $[(a,\,b)]=[(x,\,y)]$ and $[(c,\,d)]=[(z,\,w)]$, we have, since $R$ is
        an equivalence relation, that $(a,\,b)R(x,\,y)$ and
        $(c,\,d)R(z,\,w)$. That is, $ay=bx$ and $cw=dz$. We need to show that
        $[(ad+bc,\,bd)]=[(xw+yz,\,yw)]$, meaning we need to show that
        $(ad+bc)yw=(xw+yz)bd$. We have:
        \begin{align}
            (ad+bc)yw
            &=(ad)(yw)+(bc)(yw)\tag{Distributivity}\\
            &=(ay)(dw)+(bc)(yw)\tag{Associativity and Commutativity}\\
            &=(bx)(dw)+(bc)(yw)\tag{Substitution}\\
            &=(bx)(dw)+(cw)(by)\tag{Associativity and Commutativity}\\
            &=(bx)(dw)+(dz)(by)\tag{Substitution}\\
            &=(xw)(bd)+(dz)(by)\tag{Associativity and Commutativity}\\
            &=(xw)(bd)+(yz)(bd)\tag{Associativity and Commutativity}\\
            &=(xw+yz)bd\tag{Distributivity}
        \end{align}
        So $[(ad+bc,\,bd)]=[(xw+yz,\,yw)]$, and hence addition is well-defined.
    \end{proof}
    \clearpage
    \color{blue}
    \begin{problem}
        Prove Cantor's theorem. If $A$ is a set, and $\mathcal{P}(A)$ is the
        power set of $A$, then there is no surjection
        $f:A\rightarrow\mathcal{P}(A)$.
        \begin{itemize}
            \item (1 Point) Suppose there is a surjection
                $f:A\rightarrow\mathcal{P}(A)$. Consider the set
                $B=\{\,x\in{A}\;|\;x\notin{f}(x)\,\}$. Describe in words what
                the set $B$ contains.
            \item (2 Points) Since $B\subseteq{A}$, and since $f$ is surjective,
                there is an element $a\in{A}$ such that $f(a)=B$. Show that
                this is a contradiction.
            \item (1 Point) Construct an injective function
                $g:A\rightarrow\mathcal{P}(A)$. (Hint: Given $a\in{A}$, what's
                an ``obvious'' subset we can send $a$ to?)
        \end{itemize}
    \end{problem}
    \color{black}
    \begin{proof}[Solution]
        The set $B$ is the set of all elements $a\in{A}$ such that the image of
        $a$ under $f$ does not contain $a$. That is, since $f$ takes elements
        of $A$ and returns subsets of $A$, it is possible for $a$ to be
        an element of $f(a)$. Consider $\mathbb{Z}_{3}$ and define
        $f:\mathbb{Z}_{3}\rightarrow\mathcal{P}(\mathbb{Z}_{3})$ by
        $f(0)=\{\,1,\,2\,\}$, $f(1)=\{\,0,\,1\,\}$, and
        $f(2)=\emptyset$. Then $0\notin{f}(0)$ since $f(0)$ only contains
        $1$ and $2$. We see that $1\in{f}(1)$ since $f(1)$ is a set containing
        $1$. Lastly, $2\notin{f}(2)$ since $f(2)$ is the empty set. The set
        $B$ consisting of all elements $a\in\mathbb{Z}_{3}$ such that
        $a\notin{f}(a)$ is $\{\,0,\,2\,\}$. Note that there is no element
        $n\in\mathbb{Z}_{3}$ such that $f(n)=B$.
        \par\hfill\par
        Suppose $f:A\rightarrow\mathcal{P}(A)$ is a surjection and $B$ is
        defined as above. Since $f$ is surjective, there is an $a\in{A}$
        such that $f(a)=B$. Suppose $a\in{B}$. Then $a\in{f}(a)$ since
        $f(a)=B$, a contradiction since $B$ is the set of all $x\in{A}$ such
        that $x\notin{f}(x)$. So $a\notin{B}$. But if $a\notin{B}$, then
        $a\notin{f}(a)$ since $f(a)=B$. But if $a\notin{f}(a)$, then
        $a\in{B}$ since $B$ is the set of all $x\in{A}$ such that
        $x\notin{f}(x)$. This is a contradiction meaning $a$ does not exist, so
        $f$ is not a surjection.
        \par\hfill\par
        There is an injective function $f:A\rightarrow\mathcal{P}(A)$.
        Define $f(a)=\{\,a\,\}$ for all $a\in{A}$. If
        $f(a)=f(b)$, then $\{\,a\,\}=\{\,b\,\}$, meaning $a=b$, so $f$
        is injective.
    \end{proof}
    \clearpage
    \color{blue}
    \begin{problem}
        We proved in class that a function
        $f:X\rightarrow{Y}$ from a metric space $(X,d_{X})$ to a metric space
        $(Y,d_{Y})$ is continuous if and only if for every open subset
        $\mathcal{V}\subseteq{Y}$, the pre-image $f^{-1}[\mathcal{V}]$ is also
        open. You will now prove the equivalence of the third definition of
        continuity.
        \begin{itemize}
            \item (3 Points) Prove that if $f$ is continuous, then
                for all $\varepsilon>0$, and for all $x\in{X}$, there is a
                $\delta>0$ such that if $x_{0}\in{X}$ and
                $d_{X}(x,x_{0})<\delta$,
                then $d_{Y}\big(f(x),f(x_{0})\big)<\varepsilon$
                (Hint: Suppose not. Then there is an $\varepsilon>0$ such that
                for each $n\in\mathbb{N}$, $n>0$,
                there is a point $a_{n}\in{X}$ with $d_{X}(x,a_{n})<\frac{1}{n}$
                and $d_{Y}\big(f(x),f(a_{n})\big)\geq\varepsilon$. What is
                $\lim_{n\rightarrow\infty}a_{n}$? What is
                $\lim_{n\rightarrow\infty}f(a_{n})$? Is there a contradiction?)
            \item (3 Points) Prove that if $f:X\rightarrow{Y}$ is a function
                such that for all $\varepsilon>0$ and for all $x\in{X}$, there
                is a $\delta>0$ such that for all $x_{0}\in{X}$,
                $d_{X}(x,x_{0})<\delta$ implies
                $d_{Y}\big(f(x),f(x_{0})\big)<\varepsilon$, then $f$ is
                continuous. (Hint: Let $a_{n}\rightarrow{x}$ by a convergent
                sequence. What does this property say about
                $\lim_{n\rightarrow\infty}f(a_{n})$?)
        \end{itemize}
    \end{problem}
    \color{black}
    \begin{proof}[Solution.]
        Suppose $f:X\rightarrow{Y}$ is continuous. Suppose there is an
        $\varepsilon>0$ and an $x\in{X}$ such that for all $\delta>0$ there
        is an $x_{0}\in{X}$ such that $d_{X}(x,\,x_{0})<\delta$
        and $d_{Y}\big(f(x),\,f(x_{0})\big)\geq\varepsilon$.
        Then, in particular, for all $n\in\mathbb{N}$ there is an
        $a_{n}\in{X}$ such that $d_{X}(x,\,a_{n})<\frac{1}{n+1}$, but
        $d_{Y}\big(f(x),\,f(a_{n})\big)\geq\varepsilon$. But then
        $d_{X}(x,\,a_{n})\rightarrow{0}$, and so $a_{n}\rightarrow{x}$. But $f$
        is continuous, so if $a_{n}\rightarrow{x}$, then
        $f(a_{n})\rightarrow{f}(x)$. That is, by the definition of convergence,
        there is an $N\in\mathbb{N}$ such that $n\in\mathbb{N}$ and $n>N$
        implies $d_{Y}\big(f(x),\,f(a_{n})\big)<\varepsilon$. But for all
        $n\in\mathbb{N}$ we have $d_{Y}\big(f(x),\,f(a_{n})\big)\geq\varepsilon$
        which is a contradiction. Hence, for all $\varepsilon>0$ and for all
        $x\in{X}$ there is a $\delta>0$ such that for all $x_{0}\in{X}$ it is
        true that $d_{X}(x,\,x_{0})<\delta$ implies
        $d_{Y}\big(f(x),\,f(x_{0})\big)<\varepsilon$.
        \par\hfill\par
        Now suppose $f$ has the property that for all $x\in{X}$ and for
        all $\varepsilon>0$ there is a $\delta>0$ such that for all
        $x_{0}\in{X}$, $d_{X}(x,\,x_{0})<\delta$ implies
        $d_{Y}\big(f(x),\,f(x_{0})\big)<\varepsilon$. Suppose $f$ is not
        continuous. Then there is a convergent sequence
        $a:\mathbb{N}\rightarrow{X}$ and an $x\in{X}$ such that
        $a_{n}\rightarrow{x}$, but $f(a_{n})\not\rightarrow{f}(x)$.
        But then there is an $\varepsilon>0$ such that for all
        $N\in\mathbb{N}$ there is an $n\in\mathbb{N}$ with $n>N$ such that
        $d_{Y}\big(f(x),\,f(a_{n})\big)\geq\varepsilon$. But by the property
        of $f$, since $\varepsilon>0$ there is a $\delta>0$ such that
        for all $x_{0}\in{X}$, $d_{X}(x,\,x_{0})<\delta$ implies
        $d_{Y}\big(f(x),\,f(x_{0})\big)$. But $a_{n}\rightarrow{x}$ and
        $\delta>0$, so there is an $N\in\mathbb{N}$ such that $n\in\mathbb{N}$
        and $n>N$ implies $d_{X}(x,\,a_{n})<\delta$. But then for all
        $n>N$ we have $d_{Y}\big(f(x),\,f(a_{n})\big)<\varepsilon$, a
        contradiction. Hence, $f$ is continuous.
    \end{proof}
    \clearpage
    \color{blue}
    \begin{problem}
        A locally compact metric space is a metric space $(X,\,d)$ where for all
        $x\in{X}$ there is a compact set $K$ and an open set $\mathcal{U}$
        such that $x\in\mathcal{U}$ and $\mathcal{U}\subseteq{K}$
        (See Fig.~\ref{fig:locally_compact_metric_space_001}).
        \begin{itemize}
            \item (2 Points) Construct a metric space that is \textit{not}
                locally compact. Explain why it is not locally compact.
                (Hint: Il est utile de penser \'{a} la France).
        \end{itemize}
    \end{problem}
    \begin{figure}[H]
        \centering
        \includegraphics{../../../images/locally_compact_example_001.pdf}
        \caption{Diagram for Locally Compact Metric Spaces}
        \label{fig:locally_compact_metric_space_001}
    \end{figure}
    \color{black}
    Let $X=\mathbb{R}^{2}$ and
    $d_{P}:\mathbb{R}^{2}\times\mathbb{R}^{2}\rightarrow\mathbb{R}$
    be the Paris metric:
    \begin{equation}
        d_{P}(\mathbf{x},\,\mathbf{y})=
        \begin{cases}
            ||\mathbf{x}-\mathbf{x}||_{2}&\mathbf{y}=\lambda\mathbf{x}
                \textrm{ for some }\lambda\in\mathbb{R}\\
            ||\mathbf{x}||_{2}+||\mathbf{y}||_{2}&\textrm{otherwise}
        \end{cases}
    \end{equation}
    For every point $\mathbf{x}\ne\mathbf{0}$, the point $\mathbf{x}$ has the
    local compactness property. Choosing $r=||\mathbf{x}||_{2}/2$, the
    open ball of radius $r$ centered at $\mathbf{x}$ in the Paris plane is
    essentially an \textit{open interval}, topologically similar to the open
    unit interval $(0,\,1)$. By adding the end points of this interval,
    that is, the points two points $\mathbf{y}$ lying on the line between the
    origin and $\mathbf{x}$ such that $||\mathbf{x}-\mathbf{y}||_{2}=r$, then
    we get a set that topologically looks like the closed interval
    $[0,\,1]$, which is compact by the Heine-Borel theorem. So we have found
    an open set $\mathcal{U}$ and a compact set $K$ such that
    $\mathbf{x}\in\mathcal{U}$ and $\mathcal{U}\subseteq{K}$. If we are going
    to prove the Paris plane is not locally compact, we're going to have to
    look at the origin. Suppose the Paris plane is locally compact. Then there
    is an open set $\mathcal{U}\subseteq\mathbb{R}^{2}$ and a compact set
    $K\subseteq\mathbb{R}^{2}$ such that $\mathbf{0}\in\mathcal{U}$ and
    $\mathcal{U}\subseteq{K}$. But since $\mathbf{0}\in\mathcal{U}$ and
    $\mathcal{U}$ is open, there is an $\varepsilon>0$ such that
    $B_{\varepsilon}^{(\mathbb{R}^{2},\,d_{P})}(\mathbf{0})\subseteq\mathcal{U}$.
    Since $\mathcal{U}\subseteq{K}$, we have
    $B_{\varepsilon}^{(\mathbb{R}^{2},\,d_{P})}(\mathbf{0})\subseteq{K}$.
    Define $a:\mathbb{N}\rightarrow{K}$ by:
    \begin{equation}
        a_{n}=\frac{\varepsilon}{2}\Big(
            \cos\big(\frac{\pi}{n+1}\big),\,\sin\big(\frac{\pi}{n+1}\big)
        \Big)
    \end{equation}
    This function is injective since $f(n)=\frac{\pi}{n+1}$ is injective, and
    $\big(\cos(t),\,\sin(t)\big)$ is injective for $t\in[0,\,\pi]$. Moreover,
    since every $n\in\mathbb{N}$ corresponds to a different angle, no
    two points $a_{n}$ and $a_{m}$ lie on the same line through the origin.
    But then, for all $n,m\in\mathbb{N}$, we have:
    \begin{equation}
        d_{P}(a_{m},\,a_{n})
        =||a_{m}||_{2}+||a_{n}||_{2}
        =\frac{\varepsilon}{2}+\frac{\varepsilon}{2}
        =\varepsilon
    \end{equation}
    That is, $a:\mathbb{N}\rightarrow\mathbb{R}^{2}$ is not a Cauchy sequence
    since the distance betwen distinct points does not converge to zero, but
    rather is a constant. Moreover, no subsequence can converge since for
    any subsequence $a_{k}$, since $k$ must be strictly increasing, we also
    have $d_{P}(a_{k_{m}},\,a_{k_{n}})=\varepsilon$ for all $m\ne{n}$.
    We have constructed a sequence that has no convergent subsequence.
    But $a:\mathbb{N}\rightarrow{K}$ is a sequence in a compact set, so it must
    have a convergent subsequence, which is a contradiction. Therefore,
    $(\mathbb{R}^{2},\,d_{P})$ is \textit{not} locally compact.
    See Fig.~\ref{fig:paris_not_locally_compact_001}.
    \begin{figure}
        \centering
        \includegraphics{../../../images/paris_not_locally_compact_001.pdf}
        \caption{Non-Convergent Sequence in the Paris Plane}
        \label{fig:paris_not_locally_compact_001}
    \end{figure}
\end{document}
