%-----------------------------------LICENSE------------------------------------%
%   This file is part of Mathematics-and-Physics.                              %
%                                                                              %
%   Mathematics-and-Physics is free software: you can redistribute it and/or   %
%   modify it under the terms of the GNU General Public License as             %
%   published by the Free Software Foundation, either version 3 of the         %
%   License, or (at your option) any later version.                            %
%                                                                              %
%   Mathematics-and-Physics is distributed in the hope that it will be useful, %
%   but WITHOUT ANY WARRANTY; without even the implied warranty of             %
%   MERCHANTABILITY or FITNESS FOR A PARTICULAR PURPOSE.  See the              %
%   GNU General Public License for more details.                               %
%                                                                              %
%   You should have received a copy of the GNU General Public License along    %
%   with Mathematics-and-Physics.  If not, see <https://www.gnu.org/licenses/>.%
%------------------------------------------------------------------------------%
\documentclass{article}
\usepackage{graphicx}                           % Needed for figures.
\usepackage{amsmath}                            % Needed for align.
\usepackage{amssymb}                            % Needed for mathbb.
\usepackage{amsthm}                             % For the theorem environment.
\usepackage{float}
\usepackage{hyperref}
\hypersetup{
    colorlinks=true,
    linkcolor=blue,
    filecolor=magenta,
    urlcolor=Cerulean,
    citecolor=SkyBlue
}

%------------------------Theorem Styles-------------------------%
\theoremstyle{plain}
\newtheorem{theorem}{Theorem}

% Define theorem style for default spacing and normal font.
\newtheoremstyle{normal}
    {\topsep}               % Amount of space above the theorem.
    {\topsep}               % Amount of space below the theorem.
    {}                      % Font used for body of theorem.
    {}                      % Measure of space to indent.
    {\bfseries}             % Font of the header of the theorem.
    {}                      % Punctuation between head and body.
    {.5em}                  % Space after theorem head.
    {}

% Define default environments.
\theoremstyle{normal}
\newtheorem{definitionx}{Definition}

\newenvironment{example}{%
    \pushQED{\qed}\renewcommand{\qedsymbol}{$\blacksquare$}\examplex%
}{%
    \popQED\endexamplex%
}

\newenvironment{definition}{%
    \pushQED{\qed}\renewcommand{\qedsymbol}{$\blacksquare$}\definitionx%
}{%
    \popQED\enddefinitionx%
}

\title{Completing a Metric Space}
\author{Ryan Maguire}
\date{\today}

% No indent and no paragraph skip.
\setlength{\parindent}{0em}
\setlength{\parskip}{0em}

\begin{document}
    \maketitle
    \begin{definition}[\textbf{Dense Subspace in a Metric Space}]
        A dense subspace in a metric space $(X,\,d)$ is a metric subspace
        $(A,\,d_{A})$, $A\subseteq{X}$, such that for every $x\in{X}$ there
        is a sequence $a:\mathbb{N}\rightarrow{A}$ such that
        $a_{n}\rightarrow{x}$. That is, a sequence in $A$ that converges to
        $x$ in $(X,\,d)$.
    \end{definition}
    The motivating examples are the rational and irrational numbers in the real
    line. Every real number can be approximated arbitrarily well by a rational
    number, and every real number can also be approximated by
    an irrational number.
    \begin{definition}[\textbf{Completion of a Metric Space}]
        A completion of a metric space $(X,\,d)$ is a complete metric space
        $(\tilde{X},\,\tilde{d})$ such that there is an isometry
        $f:X\rightarrow\tilde{X}$ where $f[X]\subseteq\tilde{X}$ is a
        dense subspace.
    \end{definition}
    We will prove in these notes that every metric space $(X,\,d)$ has a
    completion, and that this completion is essentially unique.
    \begin{theorem}
        If $(X,\,d)$ is a metric space, and if $a,b:\mathbb{N}\rightarrow{X}$
        are Cauchy sequences, then the sequence
        $r:\mathbb{N}\rightarrow\mathbb{R}$ defined by
        $r_{n}=d(a_{n},\,b_{n})$ is bounded.
    \end{theorem}
    \begin{proof}
        Let $\varepsilon=1$. Since $a$ and $b$ are Cauchy, there are
        $N_{0},N_{1}\in\mathbb{N}$ such that $m,n>N_{0}$ implies
        $d(a_{m},\,a_{n})<\varepsilon$ and $m,n>N_{1}$ implies
        $d(b_{m},\,b_{n})<\varepsilon$. Let $N=\textrm{max}(N_{0},\,N_{1})$
        and let
        $M=\textrm{max}\big(d(a_{0},\,b_{0}),\,\dots,\,d(a_{N+1},\,b_{N+1})\big)+2$.
        $M$ is a bound for $r$. For given any $n\in\mathbb{N}$, if $n\leq{N}$
        we have:
        \begin{equation}
            r_{n}=d(a_{n},\,b_{n})\leq\textrm{max}
                \big(d(a_{0},\,b_{0}),\,\dots,\,d(a_{N+1},\,b_{N+1})\big)
                <M
        \end{equation}
        by definition of $M$. If $n>N$ we get:
        \begin{align}
            r_{n}
            &=d(a_{n},\,b_{n})\\
            &\leq{d}(a_{n},\,a_{N+1})+d(a_{N+1},\,b_{N+1})
                +d(b_{N+1},\,b_{n})\\
            &<\varepsilon+
                \textrm{max}\big(d(a_{0},\,b_{0}),\,\dots,\,d(a_{N+1},\,b_{N+1})\big)+
                \varepsilon\\
            &=2+\textrm{max}\big(d(a_{0},\,b_{0}),\,\dots,\,d(a_{N+1},\,b_{N+1})\big)\\
            &=M
        \end{align}
        So $r_{n}$ is bounded between $0$ and $M$, and so is bounded.
    \end{proof}
    \begin{theorem}[\textbf{The Trapezoid Inequality}]
        If $(X,\,d)$ is a metric space, and if
        $a,b,c,d\in{X}$, then $|d(a,\,c)-d(b,\,d)|\leq{d}(a,\,b)+d(c,\,d)$.
    \end{theorem}
    \begin{proof}
        There are two cases, $d(a,\,c)\geq{d}(b,\,d)$ and
        $d(a,\,c)\leq{d}(b,\,d)$. Suppose $d(a,\,c)\geq{d}(b,\,d)$. The
        argument is symmetric in the other case. Then:
        \begin{equation}
            |d(a,\,c)-d(b,\,d)|=d(a,\,c)-d(b,\,d)
        \end{equation}
        By the triangle inequality,
        $d(b,\,c)\leq{d}(b,\,d)+d(c,\,d)$, and therefore:
        \begin{equation}
            d(b,\,c)-d(b,\,d)\leq{d}(c,\,d)
        \end{equation}
        Using this we then have:
        \begin{align}
            |d(a,\,c)-d(b,\,d)|
            &=d(a,\,c)-d(b,\,d)\\
            &\leq{d}(a,\,b)+d(b,\,c)-d(b,\,d)\\
            &\leq{d}(a,\,b)+d(c,\,d)
        \end{align}
        Completing the proof.
    \end{proof}
    The trapezoid inequality gets its name from
    Fig.~\ref{fig:trapezoid_inequality_001}.
    \begin{figure}
        \centering
        \includegraphics{../../../images/trapezoid_inequality_001.pdf}
        \caption{The Trapezoid Inequality}
        \label{fig:trapezoid_inequality_001}
    \end{figure}
    \begin{theorem}
        If $(X,\,d)$ is a metric space, if $a,b:\mathbb{N}\rightarrow{X}$ are
        Cauchy sequences, and if $r:\mathbb{N}\rightarrow\mathbb{R}$ is defined
        by $r_{n}=d(a_{n},\,b_{n})$, then $r$ is a convergent sequence in
        $\mathbb{R}$.
    \end{theorem}
    \begin{proof}
        Let $\varepsilon>0$. Since $a$ and $b$ are
        Cauchy, there is an $N_{0},N_{1}\in\mathbb{N}$ such that
        $m,n>N_{0}$ implies $d(a_{m},\,a_{n})<\varepsilon/4$ and
        $m,n>N_{1}$ implies $d(b_{m},\,b_{m})<\varepsilon/4$. Let
        $N=\textrm{max}(N_{0},\,N_{1})+1$. Then by the trapezoid inequality,
        $m,n>N$ implies:
        \begin{align}
            |r_{m}-r_{n}|
            &=|r_{m}-r_{N}+r_{N}-r_{n}|\\
            &\leq|r_{m}-r_{N}|+|r_{N}-r_{n}|\\
            &=|d(a_{m},\,b_{m})-d(a_{N},\,b_{N})|
                +|d(a_{N},\,b_{N})-d(a_{n},\,b_{n})|\\
            &\leq{d}(a_{N},\,a_{m})+d(b_{N},\,b_{m})
                +d(a_{N},\,a_{n})+d(b_{N},\,b_{n})\\
            &<\frac{\varepsilon}{4}+\frac{\varepsilon}{4}
                +\frac{\varepsilon}{4}+\frac{\varepsilon}{4}\\
            &=\varepsilon
        \end{align}
        so $r$ is a Cauchy sequence. But $(\mathbb{R},\,|\cdot|)$ is complete,
        so Cauchy sequences converge. Hence, $r$ is a convergent sequence.
    \end{proof}
    \begin{theorem}
        If $(X,\,d)$ is a metric space, if $\mathcal{A}$ is the set of all
        Cauchy sequences $a:\mathbb{N}\rightarrow{X}$, and if
        $R$ is the relation defined on $\mathcal{A}$ by
        $aRb$ if and only if $d(a_{n},\,b_{n})\rightarrow{0}$, then
        $R$ is an equivalence relation on $\mathcal{A}$.
    \end{theorem}
    \begin{proof}
        $R$ is reflexive since $d(a_{n},\,a_{n})=0$, so $aRa$. $R$ is
        symmetric since $aRb$ implies $d(a_{n},\,b_{n})\rightarrow{0}$,
        but $d(a_{n},\,b_{n})=d(b_{n},\,a_{n})$, so
        $d(b_{n},\,a_{n})\rightarrow{0}$, and hence $bRa$. Lastly, it is
        transitive. If $aRb$ and $bRc$, then
        $d(a_{n},\,c_{n})\leq{d}(a_{n},\,b_{n})+d(b_{n},\,c_{n})$, and both of
        these latter two expressions tend to zero since $aRb$ and $bRc$, so
        $d(a_{n},\,c_{n})\rightarrow{0}$. That is, $aRc$. So $R$ is
        reflexive, symmetric, and transitive, and is therefore an
        equivalence relation.
    \end{proof}
    \begin{theorem}
        If $(X,\,d)$ is a metric space, if $\mathcal{A}$ is the set of all
        Cauchy sequences $a:\mathbb{N}\rightarrow{X}$, if
        $R$ is equivalence relation $aRb$ if and only if
        $d(a_{n},\,b_{n})\rightarrow{0}$, if $\tilde{X}=\mathcal{A}/R$, and
        if $\tilde{d}$ is defined by the formula:
        \begin{equation}
            \tilde{d}\big([a],\,[b]\big)=\lim_{n\rightarrow{\infty}}
                d(a_{n},\,b_{n})
        \end{equation}
        then $\tilde{d}$ is a well-defined function
        $\tilde{d}:\tilde{X}\times\tilde{X}\rightarrow\mathbb{R}$.
    \end{theorem}
    \begin{proof}
        Let $a,b,c,d\in\mathcal{A}$ be Cauchy sequences with
        $aRc$ and $bRd$. Then $d(a_{n},\,c_{n})\rightarrow{0}$ and
        $d(b_{n},\,d_{n})\rightarrow{0}$. But then:
        \begin{align}
            \tilde{d}\big([c],\,[d]\big)
            &=\lim_{n\rightarrow\infty}d(c_{n},\,d_{n})\\
            &\leq\lim_{n\rightarrow\infty}\big(d(a_{n},\,c_{n})
                +d(a_{n},\,d_{n}\big)\\
            &\leq\lim_{n\rightarrow\infty}\big(
                d(a_{n},\,c_{n})+d(a_{n},\,b_{n})+d(b_{n},\,d_{n})\big)\\
            &=\lim_{n\rightarrow\infty}d(a_{n},\,c_{n})+
                \lim_{n\rightarrow\infty}d(a_{n},\,b_{n})+
                \lim_{n\rightarrow\infty}d(b_{n},\,d_{n})\\
            &=0+\lim_{n\rightarrow\infty}d(a_{n},\,b_{n})+0\\
            &=\lim_{n\rightarrow\infty}d(a_{n},\,b_{n})\\
            &=\tilde{d}\big([a],\,[b]\big)
        \end{align}
        so $\tilde{d}$ is well-defined.
    \end{proof}
    \begin{theorem}
        If $(X,\,d)$ is a metric space, if $\mathcal{A}$ is the set of all
        Cauchy sequences $a:\mathbb{N}\rightarrow{X}$, and if
        $R$ is the equivalence relation $aRb$ if and only if
        $d(a_{n},\,b_{n})\rightarrow{0}$, if $\tilde{X}=\mathcal{A}/R$, and
        if $\tilde{d}$ is the function:
        \begin{equation}
            \tilde{d}\big([a],\,[b]\big)=\lim_{n\rightarrow\infty}
                d(a_{n},\,b_{n})
        \end{equation}
        then $(\tilde{X},\,\tilde{d})$ is a complete metric space.
    \end{theorem}
    \begin{proof}
        $\tilde{d}$ is indeed a metric. Since $(X,\,d)$ is a metric space,
        $\tilde{d}$ is non-negative since $d$ is non-negative. Also:
        \begin{align}
            &\tilde{d}\big([a],\,[b]\big)=0\\
            \Leftrightarrow&{d}(a_{n},\,b_{n})\rightarrow{0}\\
            \Leftrightarrow&{a}Rb\\
            \Leftrightarrow&[a]=[b]
        \end{align}
        so $\tilde{d}$ is positive-definite. It is symmetric since:
        \begin{align}
            \tilde{d}\big([a],\,[b]\big)
            &=\lim_{n\rightarrow\infty}d(a_{n},\,b_{n})\\
            &=\lim_{n\rightarrow\infty}d(b_{n},\,a_{n})\\
            &=\tilde{d}\big([b],\,[a]\big)
        \end{align}
        Lastly, it satisfies the triangle inequality. Given $[a]$, $[b]$, and
        $[c]$, we have:
        \begin{align}
            \tilde{d}\big([a],\,[b]\big)
            &=\lim_{n\rightarrow\infty}d(a_{n},\,b_{n})\\
            &\leq\lim_{n\rightarrow\infty}
                \big(d(a_{n},\,c_{n})+d(c_{n},\,b_{n}\big)\\
            &=\lim_{n\rightarrow\infty}d(a_{n},\,c_{n})+
                \lim_{n\rightarrow\infty}d(c_{n},\,b_{n})\\
            &=\tilde{d}\big([a],\,[c]\big)+\tilde{d}\big([c],\,[b]\big)
        \end{align}
% TODO: Fix this part, it isn't correct.
        It is also complete. Let $\mathbf{x}:\mathbb{N}\rightarrow\tilde{X}$
        be a Cauchy sequence. That is, $\mathbf{x}$ is a
        \textit{sequence of equivalence classes of Cauchy sequences}. For every
        $n\in\mathbb{N}$ there is a Cauchy sequence
        $x^{n}:\mathbb{N}\rightarrow{X}$ such that $\mathbf{x}_{n}=[x^{n}]$.
        Since $x^{n}$ is a Cauchy sequence, there is an $N_{n}\in\mathbb{N}$
        such that $k,\ell>N_{n}$ implies
        $d(x_{k}^{n},\,x_{\ell}^{n})<\frac{1}{n+1}$. Define
        $a:\mathbb{N}\rightarrow{X}$ by $a_{n}=x_{N_{n}}^{n}$. We now must
        show that $a$ is a Cauchy sequence and that
        $\mathbf{x}_{n}\rightarrow[a]$. Let $\varepsilon>0$. Let
        $M_{0}$ be such that $N_{M_{0}}+1>3/\varepsilon$. Since
        $\mathbf{x}$ is Cauchy there is an $M_{1}\in\mathbb{N}$ such that
        $k,\ell\in\mathbb{N}$ and $k,\ell>M_{1}$ implies
        $\tilde{d}(\mathbf{x}_{k},\,\mathbf{x}_{\ell})<\varepsilon/3$.
        That is:
        \begin{equation}
            \tilde{d}(\mathbf{x}_{k},\,\mathbf{x}_{\ell})
            =\lim_{n\rightarrow\infty}d(x_{n}^{k},\,x_{n}^{\ell})
            <\frac{\varepsilon}{3}
        \end{equation}
        Let $M=\textrm{max}(M_{0},\,M_{1})+1$. Then $m,n>M$ implies:
        \begin{align}
            d(a_{m},\,a_{n})
            &=d(x_{N_{m}}^{m},\,x_{N_{n}}^{n})\\
            &\leq{d}(x_{N_{m}}^{m},\,x_{M}^{m})
                +d(x_{M}^{m},\,x_{M}^{n})
                +d(x_{M}^{n},\,x_{N_{n}}^{n})\\
            &<\frac{\varepsilon}{3}
                +\frac{\varepsilon}{3}+\frac{\varepsilon}{3}\\
            &=\varepsilon
        \end{align}
        So $a$ is a Cauchy sequence. Now we must show that
        $\mathbf{x}_{n}\rightarrow[a]$. We have:
        \begin{align}
            \tilde{d}([a],\,\mathbf{x}_{n})
            &=\lim_{m\rightarrow\infty}d(a_{m},\,x_{m}^{n})\\
            &=\lim_{m\rightarrow\infty}d(x_{N_{m}}^{m},\,x_{m}^{n})
        \end{align}
        and this converges to zero as $n$ tends to infinity. So,
        $\mathbf{x}_{n}\rightarrow[a]$ and thus $(\tilde{X},\,\tilde{d})$
        is complete.
    \end{proof}
    \begin{theorem}
        If $(X,\,d)$ is a metric space, if $\mathcal{A}$ is the set of all
        Cauchy sequences $a:\mathbb{N}\rightarrow{X}$, and if
        $R$ is the equivalence relation $aRb$ if and only if
        $d(a_{n},\,b_{n})\rightarrow{0}$, if $\tilde{X}=\mathcal{A}/R$, and
        if $\tilde{d}$ is the function:
        \begin{equation}
            \tilde{d}\big([a],\,[b]\big)=\lim_{n\rightarrow\infty}
                d(a_{n},\,b_{n})
        \end{equation}
        then there is an isometry $f:X\rightarrow\tilde{X}$ into the
        complete metric space $(\tilde{X},\,\tilde{d})$ such that
        $f[X]\subseteq\tilde{X}$ is a dense subspace.
    \end{theorem}
    \begin{proof}
        Given $x\in{X}$, define $g:X\rightarrow\mathcal{A}$ via
        $g(x)=a$ where $a:\mathbb{N}\rightarrow{X}$ is the sequence
        $a_{n}=x$. Since $a$ is a constant sequence, it is a Cauchy sequence.
        Let $f:X\rightarrow\tilde{X}$ be defined by
        $f(x)=[g(x)]$. $f$ is an isometry. For if $x,y\in{X}$, then:
        \begin{align}
            \tilde{d}\big(f(x),\,f(y)\big)
            &=\tilde{d}\big([g(x)],\,[g(y)]\big)\\
            &=\lim_{n\rightarrow\infty}d\big(g(x)_{n},\,g(y)_{n}\big)\\
            &=\lim_{n\rightarrow\infty}d(x,\,y)\\
            &=d(x,\,y)
        \end{align}
        and hence, $f$ is an isometry. Moreover, $f[X]$ is a dense subset of
        $\tilde{X}$. Let $[a]\in\tilde{X}$ where $a\in\mathcal{A}$
        is a Cauchy sequence. Define the
        sequence $\mathbf{x}:\mathbb{N}\rightarrow{f}[X]$ via
        $\mathbf{x}_{n}=f(a_{n})$. Then:
% TODO: This also needs a better explanation.
        \begin{align}
            \lim_{n\rightarrow\infty}\tilde{d}([a],\,\mathbf{x}_{n})
            &=\lim_{n\rightarrow\infty}
                \lim_{m\rightarrow\infty}d(a_{m},\,a_{n})\\
            &=0
        \end{align}
        so $\mathbf{x}$ is a sequence in $f[X]$ that converges to
        $[a]$ in $(\tilde{X},\,\tilde{d})$, and hence $f[X]$ is dense.
    \end{proof}
    This is essentially the unique metric space that completes
    $(X,\,d)$. If $(Y,\,d_{Y})$ is another complete metric space such that
    there exists an isometry $f:X\rightarrow{Y}$ such that
    $f[X]\subseteq{Y}$ is a dense subspace, then there is a global
    isometry between $(Y,\,d_{Y})$ and $(\tilde{X},\,\tilde{d})$.
% TODO: Prove this claim.
\end{document}
