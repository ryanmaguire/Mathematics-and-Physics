%-----------------------------------LICENSE------------------------------------%
%   This file is part of Mathematics-and-Physics.                              %
%                                                                              %
%   Mathematics-and-Physics is free software: you can redistribute it and/or   %
%   modify it under the terms of the GNU General Public License as             %
%   published by the Free Software Foundation, either version 3 of the         %
%   License, or (at your option) any later version.                            %
%                                                                              %
%   Mathematics-and-Physics is distributed in the hope that it will be useful, %
%   but WITHOUT ANY WARRANTY; without even the implied warranty of             %
%   MERCHANTABILITY or FITNESS FOR A PARTICULAR PURPOSE.  See the              %
%   GNU General Public License for more details.                               %
%                                                                              %
%   You should have received a copy of the GNU General Public License along    %
%   with Mathematics-and-Physics.  If not, see <https://www.gnu.org/licenses/>.%
%------------------------------------------------------------------------------%
\documentclass{article}
\usepackage{graphicx}                           % Needed for figures.
\usepackage{amsmath}                            % Needed for align.
\usepackage{amssymb}                            % Needed for mathbb.
\usepackage{amsthm}                             % For the theorem environment.
\usepackage{imakeidx}
\usepackage{hyperref}
\hypersetup{
    colorlinks=true,
    linkcolor=blue,
    filecolor=magenta,
    urlcolor=Cerulean,
    citecolor=SkyBlue
}

%------------------------Theorem Styles-------------------------%
\theoremstyle{plain}
\newtheorem{theorem}{Theorem}[section]

% Define theorem style for default spacing and normal font.
\newtheoremstyle{normal}
    {\topsep}               % Amount of space above the theorem.
    {\topsep}               % Amount of space below the theorem.
    {}                      % Font used for body of theorem.
    {}                      % Measure of space to indent.
    {\bfseries}             % Font of the header of the theorem.
    {}                      % Punctuation between head and body.
    {.5em}                  % Space after theorem head.
    {}

% Define default environments.
\theoremstyle{normal}
\newtheorem{examplex}{Example}[section]
\newtheorem{definitionx}{Definition}[section]

\newenvironment{example}{%
    \pushQED{\qed}\renewcommand{\qedsymbol}{$\blacksquare$}\examplex%
}{%
    \popQED\endexamplex%
}

\newenvironment{definition}{%
    \pushQED{\qed}\renewcommand{\qedsymbol}{$\blacksquare$}\definitionx%
}{%
    \popQED\enddefinitionx%
}

\title{Planar Topology: Lecture 1 - Point-Set Topology}
\author{Ryan Maguire}
\date{Spring 2023}

% No indent and no paragraph skip.
\setlength{\parindent}{0em}
\setlength{\parskip}{0em}
\makeindex[intoc]
\begin{document}
    \maketitle
    \tableofcontents
    \listoffigures
    \section{Basic Definitions}
        Topological spaces form the study of \textit{generalized geometry}
        where things may be stretched and squeezed and still thought of as the
        \textit{same} object. The definition comes from the study of metric
        spaces.
        \begin{definition}[Metric Space]
            A metric space is an ordered pair $(X,\,d)$ where $X$ is a set and
            $d:X\times{X}\rightarrow\mathbb{R}$ is a function such that for all
            $x,\,y,\,z\in{X}$:
            \begin{align}
                d(x,\,y)&\geq{0}\tag{Positivity}\\
                d(x,\,x)&=0\tag{Definiteness}\\
                d(x,\,y)&=d(y,\,x)\tag{Symmetry}\\
                d(x,\,z)&\leq{d}(x,\,y)+d(y,\,z)\tag{Triangle Inequality}
            \end{align}
            The function $d$ is called a \textit{metric} on $X$. The elements of
            $X$ are usually referred to as \textit{points}.\index{Metric Space}
        \end{definition}
        \begin{example}
            The quintessential example is the standard metric on the real line.
            Equip $\mathbb{R}$ with the function
            $d:\mathbb{R}\times\mathbb{R}\rightarrow\mathbb{R}$ defined by:
            \begin{equation}
                d(x,\,y)=|x-y|
            \end{equation}
            That is, $d$ is defined by the absolute value function. From real
            analysis we know the absolute value function satisfies the triangle
            inequality (the proof is not hard, either). Positive-definiteness
            and symmetry are almost immediate from the definition as well.
            This is the \textit{distance} between two real numbers on the
            real line.
        \end{example}
        \begin{example}
            The Pythagoras theorem gives us a \textit{distance} formula on
            $\mathbb{R}^{N}$. Given two points
            $\mathbf{x},\,\mathbf{y}\in\mathbb{R}^{N}$ we may define:
            \begin{equation}
                d(\mathbf{x},\mathbf{y})
                =\sqrt{\sum_{n=0}^{N-1}(x_{n}-y_{n})^{2}}
            \end{equation}
            where $x_{n}$ and $y_{n}$ are the $n^{th}$ components of
            $\mathbf{x}$ and $\mathbf{y}$, respectively. This is the
            \textit{Euclidean} metric on $\mathbb{R}^{N}$, also called the
            \textit{standard} metric.
        \end{example}
        \begin{example}
            You can place different metrics on the same set. The
            \textit{Manhattan} metric on $\mathbb{R}^{N}$ is defined by:
            \begin{equation}
                d(\mathbf{x},\,\mathbf{y})
                =\sum_{n=0}^{N-1}|x_{n}-y_{n}|
            \end{equation}
            That this is a metric can be proved by induction. The base case
            $N=1$ is the standard metric on $\mathbb{R}$.
        \end{example}
        \begin{example}
            A \textit{norm} on a (real or complex) vector space $V$ is a
            function $||\cdot||:V\rightarrow\mathbb{R}$ such that for all
            $\mathbf{x},\,\mathbf{y}\in{V}$ and (either real or complex)
            scalars $a$:
            \begin{align}
                ||\mathbf{x}||&\geq{0}
                    \tag{Positivity}\\
                ||\mathbf{x}||&=0\Leftrightarrow\mathbf{x}=\mathbf{0}
                    \tag{Definiteness}\\
                ||a\mathbf{x}||&=|a|\cdot||\mathbf{x}||
                    \tag{Factoring Scalars}\\
                ||\mathbf{x}+\mathbf{y}||&\leq||\mathbf{x}||+||\mathbf{y}||
                    \tag{Triangle Inequality}
            \end{align}
            Norms define metrics (the \textit{induced} metric):
            \begin{equation}
                d(\mathbf{x},\,\mathbf{y})=||\mathbf{x}-\mathbf{y}||
            \end{equation}
            The Euclidean metric on $\mathbb{R}^{N}$ comes from the Euclidean
            norm on $\mathbb{R}^{N}$, which is the usual Pythagorean length
            of vectors in $N$-space.%
            \index{Norm}
        \end{example}
        \begin{example}
            An inner product on a (real) vector space $V$ is a function
            $\langle{\cdot|\cdot}\rangle:V\times{V}\rightarrow\mathbb{R}$
            such that for all $\mathbf{x},\,\mathbf{y},\,\mathbf{z}\in{V}$ and
            $a,b\in\mathbb{R}$:
            \begin{align}
                \langle{\mathbf{x}|\mathbf{y}}\rangle
                    &=\langle{\mathbf{y}|\mathbf{x}}\rangle
                        \tag{Symmetry}\\
                \langle{a\mathbf{x}+b\mathbf{y}|\mathbf{z}}\rangle
                    &=a\langle{\mathbf{x}|\mathbf{z}}\rangle
                        +b\langle{\mathbf{y}|\mathbf{z}}\rangle
                            \tag{Linearity}\\
                \mathbf{x}\ne\mathbf{0}\Rightarrow
                    \langle{\mathbf{x}|\mathbf{x}}\rangle&>0
                        \tag{Positive-Definiteness}
            \end{align}
            Inner-products define norms (the \textit{induced} norm):
            \begin{equation}
                ||\mathbf{x}||=\sqrt{\langle{\mathbf{x}|\mathbf{x}}\rangle}
            \end{equation}
            The Euclidean norm on $\mathbb{R}^{N}$ is the norm induced by the
            Euclidean dot product:
            \begin{equation}
                \langle{\mathbf{x}|\mathbf{y}}\rangle
                =\mathbf{x}\cdot\mathbf{y}
                =\sum_{n=0}^{N-1}x_{n}y_{n}
            \end{equation}
            The metric induced by the induced norm is the metric
            \textit{induced} by the inner product.%
            \index{Inner Product}
        \end{example}
        Topological spaces generalize metric spaces by axiomatizing the
        properties of \textit{open} subsets. In a metric space this is very
        pictorial.
        \begin{definition}[Open Subsets (Metric Space)]
            An open subset in a metric $(X,\,d)$ is a subset
            $\mathcal{U}\subseteq{X}$ such that for all $x\in\mathcal{U}$ there
            is an $\varepsilon>0$ such that for all $y\in\mathcal{U}$ with
            $d(x,\,y)<\varepsilon$ it is true that $y\in\mathcal{U}$. That is,
            the \textit{open ball} of radius $\varepsilon$ centered about $x$
            fits entirely inside $\mathcal{U}$
            (Fig.~\ref{fig:open_set_in_a_metric_space_001}).%
            \index{Open Subset}\index{Open Subset!in a Metric Space}
        \end{definition}
        \begin{figure}
            \centering
            \includegraphics{../../../images/open_set_in_a_metric_space_001.pdf}
            \caption{Open Subset in a Metric Space}
            \label{fig:open_set_in_a_metric_space_001}
        \end{figure}
        \begin{theorem}
            \index{Theorem!Open Sets in a Metric Space}%
            The collection $\tau$ of all open subsets in a metric space
            $(X,\,d)$ satisfies the following:
            \begin{align}
                &\emptyset\in\tau\tag{The Empty Set is Open}\\
                &X\in\tau\tag{The Entire Space is Open}\\
                &\mathcal{O}\subseteq\tau\Rightarrow\bigcup\mathcal{O}\in\tau
                    \tag{The Union of Open Sets is Open}\\
                &\mathcal{U},\mathcal{V}\in\tau\Rightarrow
                    \mathcal{U}\cap\mathcal{V}\in\tau
                        \tag{The Intersection of Two Open Sets is Open}
            \end{align}
        \end{theorem}
        The definition of a topological space is a mimicry of this theorem.
        \begin{definition}[Topological Space]
            A topological space is an ordered pair $(X,\,\tau)$ such that
            $X$ is a set and $\tau\subseteq\mathcal{P}(X)$
            (the power set of $X$) is such that:
            \begin{align}
                &\emptyset\in\tau\tag{The Empty Set is Open}\\
                &X\in\tau\tag{The Entire Space is Open}\\
                &\mathcal{O}\subseteq\tau\Rightarrow\bigcup\mathcal{O}\in\tau
                    \tag{The Union of Open Sets is Open}\\
                &\mathcal{U},\mathcal{V}\in\tau\Rightarrow
                    \mathcal{U}\cap\mathcal{V}\in\tau
                        \tag{The Intersection of Two Open Sets is Open}
            \end{align}
            The sets $\mathcal{U}\in\tau$ are called \textit{open} and
            $\tau$ is called a \textit{topology} on $X$.%
            \index{Topological Space}\index{Topological Space!Definition}%
            \index{Open Subset!in a Topological Space}
        \end{definition}
        \textit{Closed} sets are the complement of open sets.%
        \index{Closed Subset}
        \begin{example}
            If $X$ is any set then $\{\,\emptyset,\,X\,\}$ is a topology on $X$.
            This is the \textit{chaotic} or \textit{indiscrete} topology, also
            called the \textit{trivial} topology. It states that the only open
            subsets of $X$ are the empty set and the whole space.%
            \index{Indiscrete Topology}
        \end{example}
        \begin{example}
            If $X$ is any set, then $\mathcal{P}(X)$ is a topology on $X$.
            This is the \textit{discrete} topology. It states that
            \textit{every} subset is open.\index{Discrete Topology}
        \end{example}
        \textit{Metrizable}%
        \index{Metrizable Space}\index{Topological Space!Metrizable}
        spaces are those where the topology is induced by a
        metric. Not every topological space is induced by a metric. The trivial
        topology on a set with at least two distinct points is an example of a
        non-metrizable topological space. We can prove this by noting there is a
        topological property that this space lacks that all metric spaces have.
        The easiest such property is the \textit{Hausdorff} one.
        \begin{definition}[Hausdorff Topological Space]
            A Hausdorff topological space is a topological space
            $(X,\,\tau)$ such that for all distinct $x,y\in{X}$ there are open
            sets $\mathcal{U},\,\mathcal{V}\in\tau$ such that
            $x\in\mathcal{U}$, $y\in\mathcal{V}$, and
            $\mathcal{U}\cap\mathcal{V}=\emptyset$
            (Fig.~\ref{fig:hausdorff_condition_001}).%
            \index{Hausdorff Topological Space}%
            \index{Topological Space!Hausdorff Property}
        \end{definition}
        \begin{figure}
            \centering
            \includegraphics{../../../images/hausdorff_condition_001.pdf}
            \caption{The Hausdorff Property}
            \label{fig:hausdorff_condition_001}
        \end{figure}
        \begin{theorem}
            Metrizable spaces are Hausdorff.%
            \index{Theorem!Metric Spaces are Hausdorff}
        \end{theorem}
        \begin{proof}
            Given a metric space $(X,\,d)$ with distinct
            $x,\,y\in{X}$, we have $d(x,\,y)>0$.
            Let $\varepsilon=\frac{d(x,\,y)}{2}$, and
            $\mathcal{U}$ and $\mathcal{V}$ be the $\varepsilon$ balls about
            $x$ and $y$, respectively. Then $x\in\mathcal{U}$,
            $y\in\mathcal{V}$, and from the triangle inequality we get
            $\mathcal{U}\cap\mathcal{V}=\emptyset$.
        \end{proof}
        Given the trivial topology on a set with at least two distinct points
        $x,y\in{X}$, we see that this space is not Hausdorff since these points
        can not be separated by open sets. Hence this space is not induced by
        a metric.
        \subsection{Subspaces}
            Given a topological space $(X,\,\tau)$ and a subset $A\subseteq{X}$
            we can get a new space by considering the
            \textit{subspace topology}.
            \begin{definition}[Subspace Topology]
                The subspace topology of a subset $A\subseteq{X}$ with respect
                to a topological space $(X,\,\tau)$ is the set
                $\tau_{A}$ defined by:
                \begin{equation}
                    \tau_{A}=\{\,A\cap\mathcal{U}\;|\;\mathcal{U}\in\tau\,\}
                \end{equation}
                That is, the set of all intersections of $A$
                with the open subsets of $X$.%
                \index{Subspace Topology}%
                \index{Topological Space!Subspaces}
            \end{definition}
            \begin{theorem}
                The subspace topology is indeed a topology.%
                \index{Theorem!of Subspace Topologies}
            \end{theorem}
            \begin{proof}
                Given a topological space $(X,\,\tau)$ and $A\subseteq{X}$,
                we have $A=A\cap{X}$, and since $X\in\tau$, it is true that
                $A\in\tau_{A}$. Similarly since $\emptyset\in\tau$ and
                $\emptyset=A\cap\emptyset$, we obtain $\emptyset\in\tau_{A}$.
                For unions we invoke the distributive law. A collection of open
                subsets of $\tau_{A}$ are of the form $\mathcal{U}\cap{A}$.
                Taking their union we get:
                \begin{equation}
                    =\bigcup_{\mathcal{U}}(\mathcal{U}\cap{A})
                    =\Big(\bigcup_{\mathcal{U}}\mathcal{U}\Big)\cap{A}
                \end{equation}
                since $\tau$ is a topology, $\bigcup_{\mathcal{U}}\mathcal{U}$
                is open in $\tau$, and hence this final set is open
                in $\tau_{A}$. Lastly, given $\mathcal{U}\cap{A}$ and
                $\mathcal{V}\cap{A}$, we have:
                \begin{equation}
                    \big(\mathcal{U}\cap{A}\big)\cap\big(\mathcal{V}\cap{A}\big)
                    =\big(\mathcal{U}\cap\mathcal{V}\big)\cap{A}
                \end{equation}
                Since $\tau$ is a topology, $\mathcal{U}\cap\mathcal{V}$ is
                open in $\tau$, and hence
                $\big(\mathcal{U}\cap\mathcal{V}\big)\cap{A}$ is open in
                $\tau_{A}$. So $\tau_{A}$ is a topology on $A$.
            \end{proof}
            \begin{example}
                The unit $N$-sphere $\mathbb{S}^{N}$ is the subset of
                $\mathbb{R}^{N+1}$ defined by:
                \begin{equation}
                    \mathbb{S}^{N}
                    =\{\,\mathbf{x}\in\mathbb{R}^{N+1}\;|\;||\mathbf{x}||=1\,\}
                \end{equation}
                That is, the set of all points of unit length. The standard
                topology is the subspace topology induced by the standard
                topology on $\mathbb{R}^{N+1}$, which is induced by the
                Euclidean metric.
            \end{example}
    \section{Continuity}
        From calculus we know how to describe continuity.
        Minor perturbations in the domain result in small changes in the range.
        To be precise, given a function $f:\mathbb{R}\rightarrow\mathbb{R}$ and
        a point $x_{0}\in\mathbb{R}$, we'll claim that $f$ is continuous
        here if for all $\varepsilon>0$ there is a $\delta>0$ such that for all
        $|x-x_{0}|<\delta$ we have $|f(x)-f(x_{0})|<\varepsilon$.
        \par\hfill\par
        This definition is adapted to metric spaces immediately.
        \begin{definition}[Continuous Function (Metric Space)]
            A continuous function from a metric space $(X,\,d_{X})$ to a
            metric space $(Y,\,d_{Y})$ is a function
            $f:X\rightarrow{Y}$ such that for all $x_{0}\in{X}$ and for all
            $\varepsilon>0$ there is a $\delta>0$ such that for all
            $x\in{X}$ with $d_{X}(x,\,x_{0})<\delta$ we have
            $d_{Y}\big(f(x),\,f(x_{0})\big)<\varepsilon$.%
            \index{Continuity}\index{Continuity!in a Metric Space}
        \end{definition}
        This makes use of real numbers and metrics, neither of which are
        available in the general topological setting. We instead use open sets
        to define continuity. This is motivated by the following.
        \begin{theorem}
            If $(X,\,d_{X})$ and $(Y,\,d_{Y})$ are metric spaces, and if
            $f:X\rightarrow{Y}$ is a function, then $f$ is continuous if and
            only if for all open $\mathcal{V}\subseteq{Y}$, the pre-image
            $f^{-1}[\mathcal{V}]\subseteq{X}$ is open.%
            \index{Theorem!Equivalence of Continuity}
        \end{theorem}
        \begin{proof}
            Suppose $\mathcal{V}\subseteq{Y}$ is open and $f$ is continuous.
            If $f^{-1}[\mathcal{V}]=\emptyset$ we are done since the empty set
            is open. If not, let $x\in{f}^{-1}[\mathcal{V}]$. Since
            $\mathcal{V}$ is open, there is an $\varepsilon>0$ such that the
            $\varepsilon$ ball about $y=f(x)$ is contained inside $\mathcal{V}$.
            But $f$ is continuous, so there is a $\delta>0$ such that for all
            $x_{0}\in{X}$ with $d_{X}(x,\,x_{0})<\delta$ we have
            $d_{Y}\big(f(x),\,f(x_{0})\big)<\varepsilon$. But this implies
            $x_{0}\in{f}^{-1}[\mathcal{V}]$. That is, the $\delta$ ball about
            $x$ is a subset of $f^{-1}[\mathcal{V}]$, and hence this set is
            open.
            \par\hfill\par
            In the other direction, let $x\in{X}$ and $\varepsilon>0$ be given.
            Let $\mathcal{V}$ be the $\epsilon$ ball about
            $y=f(x)$. Since this is open, $f^{-1}[\mathcal{V}]$ is open. But
            then there is a $\delta>0$ such that the $\delta$ ball about
            $x$ is contained within $f^{-1}[\mathcal{V}]$. But then for all
            $x_{0}\in{X}$ such that $d_{X}(x,\,x_{0})<\delta$, we have
            $d_{Y}\big(f(x),\,f(x_{0})\big)<\varepsilon$.
            Hence $f$ is continuous.
        \end{proof}
        Topological spaces do have a notion of open sets, meaning we can take
        this theorem and turn it into a definition.
        \begin{definition}[Continuous Function (Topological Space)]
            A continuous function from a topological space $(X,\,\tau_{X})$ to
            a topological space $(Y,\,\tau_{Y})$ is a function
            $f:X\rightarrow{Y}$ such that for all $\mathcal{V}\in\tau_{Y}$ it
            is true that $f^{-1}[\mathcal{V}]\in\tau_{X}$.%
            \index{Continuity!in a Topological Space}
        \end{definition}
        \begin{example}
            If $(X,\,\tau)$ is any topological space, if
            $(Y,\,\tau_{Y})$ is the indiscrete topological space on $Y$
            $(\tau_{Y}=\{\,\emptyset,\,Y\,\})$, and if $f:X\rightarrow{Y}$ is
            \textit{any} function, then $f$ is continuous. The only open sets
            to check are $\emptyset$ and $Y$. But $f^{-1}[\emptyset]=\emptyset$,
            which is open, and $f^{-1}[Y]=X$, which is also open. So $f$ is
            continuous.\index{Indiscrete Topology}
        \end{example}
        \begin{example}
            If $(Y,\,\tau)$ is any topological space, if
            $(X,\,\mathcal{P}(X))$ is the discrete topological space on $X$,
            and if $f:X\rightarrow{Y}$ is \textit{any} function, then $f$ is
            continuous. Regardless of the open sets $\mathcal{V}\in\tau$, we
            have $f^{-1}[\mathcal{V}]\subseteq{X}\in\mathcal{P}(X)$, which is
            open, so $f$ is continuous.\index{Discrete Topology}
        \end{example}
        \subsection{Category Theory}
            A small rant, I'm not a fan of category theory. Any subject
            that requires \textit{proper classes} is, to me, a fiction.
            Nevertheless the language can at times be helpful, pedagogically.
            The discrete and indiscrete topologies are such examples where
            this language can be useful. A category $\mathbf{C}$ is a
            \textit{thing} (almost never a set, maybe not even class,
            pending on who you ask) consisting of:%
            \index{Category}\index{Category!Definition}
            \begin{itemize}
                \item A class $\textrm{obj}(\mathbf{C})$
                    (perhaps proper) of \textit{objects}.
                \item A class $\textrm{hom}(\mathbf{C})$
                    (again, perhaps proper)
                    of \textit{arrows} between objects.
                \item A class function
                    $\textrm{dom}:\textrm{hom}(\mathbf{C})\rightarrow\textrm{obj}(\mathbf{C})$
                    called the \textit{domain}.
                \item A class function
                    $\textrm{cod}:\textrm{hom}(\mathbf{C})\rightarrow\textrm{obj}(\mathbf{C})$
                    called the \textit{codomain}.
                \item A \textit{composition operator}
                    $\textrm{hom}(A,\,B)\times\textrm{hom}(B,\,C)\rightarrow\textrm{hom}(A,\,C)$
                    for all three objects $A,\,B,\,C$. Here
                    $\textrm{hom}(A,\,B)$ denotes the subclass of
                    $\textrm{hom}(\mathbf{C})$ of arrows $f$ such that
                    $\textrm{dom}(f)=A$ and $\textrm{cod}(f)=B$.
                \item Associativity holds:
                    $(f\circ{g})\circ{h}=f\circ(g\circ{h})$
                    where $\circ$ is the composition operator.
                \item For all objects $X$ there is an identity arrow
                    $\textrm{id}_{X}:X\rightarrow{X}$ such that for every arrow
                    $f$ with $\textrm{dom}(f)=X$ we have
                    $f\circ\textrm{id}_{X}=f$ and for every arrow
                    $g$ with $\textrm{cod}(f)=X$ we have
                    $\textrm{id}_{X}\circ{g}=g$.
            \end{itemize}
            Given two categories $\mathbf{C}$ and $\mathbf{D}$ a
            \textit{functor} $F$ is a \textit{thing} (often called a mapping,
            but not in the sense of set theory) such that:%
            \index{Functor}\index{Functor!Definition}
            \begin{itemize}
                \item For every object $X$ in $\textrm{obj}(\mathbf{C})$
                    there is an object $F(X)$ in $\textrm{obj}(\mathbf{D})$.
                \item For every arrow $f$ in $\textrm{hom}(\mathbf{C})$ there
                    is an arrow $F(f)$ in $\textrm{hom}(\mathbf{D})$ such that
                    $\textrm{dom}(F(f))=F(\textrm{dom}(f))$ and
                    $\textrm{cod}(F(f))=F(\textrm{cod}(f))$.
                \item For each $X$ in $\textrm{obj}(\mathbf{C})$ the
                    identity arrow is preserved,
                    $F(\textrm{id}_{X})=\textrm{id}_{F(X)}$.
                \item For all $f$ and $g$ in $\textrm{hom}(\mathbf{C})$,
                    composition is preserved, $F(f\circ{g})=F(f)\circ{F}(g)$.
            \end{itemize}
            A \textit{small} category%
            \index{Category!Small}\index{Small Category}
            is a category $\mathbf{C}$ where
            $\textrm{obj}(\mathbf{C})$ and $\textrm{hom}(\mathbf{C})$ are sets
            (not proper classes). The study of small categories can be done in
            ZFC entirely without proper classes. Groupoids, which appear in
            topology, geometry, and analysis, are small categories in which all
            arrows are invertible (have reverse arrows).
            \textit{Locally small} categories are categories $\mathbf{C}$ where
            for all objects $A,\,B$ in $\textrm{obj}(\mathbf{C})$ the
            subclass $\textrm{hom}(A,\,B)$ is a set.%
            \begin{example}
                The category \textbf{Set} has as objects the (proper class) of
                all sets.\footnote{%
                    See Russell's paradox for why this is not a set.%
                }
                The \textit{arrows} are just functions. Every set $X$ has an
                identity function $\textrm{id}_{X}:X\rightarrow{X}$, and the
                composition of functions is associative. \textbf{Set} is
                locally small. Given two sets $A,\,B$, the set of all functions
                $\mathcal{F}(A,\,B)$ from $A$ to $B$ is provably a set within
                the framework of ZFC.\footnote{%
                    Using the ordered pair definition of function,
                    $\mathcal{F}(A,\,B)$ is a
                    subset of $\mathcal{P}\big(\mathcal{P}(A\times{B})\big)$.
                }
            \end{example}
            \begin{example}
                The category \textbf{Top} has as objects the (proper class)%
                \footnote{%
                    Every set has a corresponding topological space, the
                    discrete topology. Intuitively there are as many sets as there
                    are topological spaces, so the collection of all topological
                    spaces is not a set.%
                }
                of all topological spaces. The arrows are continuous functions.
                This category is also locally small, given $(X,\,\tau_{X})$ and
                $(Y,\,\tau_{Y})$, the collection
                $C\big((X,\,\tau_{X}),\,(Y,\,\tau_{Y})\big)$ of all continuous
                functions $f:X\rightarrow{Y}$ is a set, being a subset of the set
                $\mathcal{F}(X,\,Y)$ of all functions $f:X\rightarrow{Y}$.
            \end{example}
            \begin{example}
                The category \textbf{Grp} has as objects the (proper class)%
                \footnote{%
                    For every set, there is a group. This is equivalent to axiom
                    of choice.%
                }
                of all groups. The arrows are group homomorphisms. \textbf{Grp} is
                locally small since the collection of all group homomorphisms
                $\varphi:G\rightarrow{H}$ is a subset of $\mathcal{F}(G,\,H)$.
            \end{example}
            I know of no examples of categories that are not locally small.
            After some digging I found the category of \textit{spans}, but I
            don't know what these are. Some claim there is a category
            \textbf{Cat} of all categories, but that just sounds like Russell's
            paradox waiting to happen.
            \par\hfill\par
            In a locally small category, the subclasses $\textrm{hom}(A,\,B)$
            are sets, so we call them \textit{homsets}.\index{Homsets}
            Given two locally small
            categories $\mathbf{C}$ and $\mathbf{D}$, two objects
            $X,\,Y$ in $\textrm{obj}(\mathbf{C})$, and a functor
            $F:\mathbf{C}\rightarrow\mathbf{D}$, we get a function
            (an actual function from set theory)
            $F_{X,\,Y}:\textrm{hom}(X,\,Y)\rightarrow\textrm{hom}\big(F(X),\,F(X)\big)$.
            $F$ is called \textit{faithful}%
            \index{Faithful Functor}\index{Functor!Faithful}
            if $F_{X,\,Y}$ is injective for all
            objects $X$ and $Y$. It is called \textit{full}%
            \index{Full Functor}\index{Functor!Full}
            if $F_{X,\,Y}$ is surjective for all objects $X$ and $Y$.
            \begin{example}
                The functor $F:\mathbf{Top}\rightarrow\mathbf{Set}$ defined by
                $F\big((X,\,\tau)\big)=X$ for objects and $F(f)=f$ for arrows
                (a continuous function is a function, after all) is faithful.
                Given two topological spaces $(X,\,\tau_{X})$ and $(Y,\,\tau_{Y})$
                the function
                $F_{X,\,Y}:C\big((X,\,\tau_{X}),\,(Y,\,\tau_{Y})\big)\rightarrow\mathcal{F}(X,\,Y)$,
                which is $F_{X,\,Y}(f)=f$, is injective, but in general it is not
                surjective. It is not surjective since there may be functions
                $f:X\rightarrow{Y}$ that are not continuous. $F$ is called the
                \textit{forgetful} functor.%
                \index{Forgetful Functor}\index{Functor!Forgetful}
                It can be similarly defined for \textbf{Grp}.
            \end{example}
            A concrete category%
            \index{Concrete Category}\index{Category!Concrete}
            is a locally small category $\mathbf{C}$ with a faithful
            functor $U:\mathbf{C}\rightarrow\mathbf{Set}$. \textbf{Top} and
            \textbf{Grp} with the forgetful functor form concrete categories.
            A \textit{free object}\index{Free Object}
            from a set $X$ in a concrete category
            $(\mathbf{C},\,U)$ is an object $F(X)$ in $\textrm{obj}(\mathbf{C})$
            with an injective function $\alpha:X\rightarrow{U}\big(F(X)\big)$
            with the following property. Given any object $Y$ in
            $\textrm{obj}(\mathbf{C})$ and any function $f:X\rightarrow{U}(Y)$
            there is a unique arrow $\tilde{f}:F(X)\rightarrow{Y}$ such that
            $\tilde{f}=U(f)\circ\alpha$. In other words, the diagram
            in Fig.~\ref{fig:commutative_diagram_free_object_in_topology} is
            commutative.
            \par\hfill\par
            Let's rephrase this in the language of topology. We have a set $X$
            and we want a topological space $F(X)$ with an injective
            function $\alpha:X\rightarrow{U}\big(F(X)\big)$ such that for
            \textit{any} topological space $(Y,\,\tau_{Y})$ and \textit{any}
            function $f:X\rightarrow{Y}$ there is a unique continuous function
            $\tilde{f}:F(X)\rightarrow(Y,\,\tau_{Y})$ such that
            $f=U(\tilde{f})\circ\alpha$.
            \par\hfill\par
            We've seen this already. Define $U\big((X,\,\tau)\big)=X$
            (the forgetful functor),
            $F(X)=\big(X,\,\mathcal{P}(X)\big)$ (the discrete topology),
            and let $\alpha:X\rightarrow{X}$ be the identity
            $\alpha=\textrm{id}_{X}$. Given any topological space
            $(Y,\,\tau_{Y})$ and any function $f:X\rightarrow{Y}$, the
            unique continuous function is $\tilde{f}=f$.
            \begin{figure}
                \centering
                \includegraphics{../../../images/commutative_diagram_free_object_in_topology.pdf}
                \caption{Free Object in Topology}
                \label{fig:commutative_diagram_free_object_in_topology}
            \end{figure}
            \par\hfill\par
            To summarize, the discrete topology%
            \index{Discrete Topology}
            is the \textit{free object}\index{Free Object}
            in the category \textbf{Top}. Thus the discrete topology is
            analogous to the free group in algebra. The underlying set $X$
            acts as a \textit{basis} for the topological space, just like
            generators act as a basis for free group. This is also similar to
            bases which are used to generate vector spaces.
            \par\hfill\par
            The \textit{cofree} object\index{Cofree Object}
            flips all the arrows. Given a set
            $X$ we want a topological space $C(X)$ and an injective function
            $\alpha:X\rightarrow{U}\big(C(X)\big)$ such that for any space
            $(Y,\,\tau_{Y})$ and any function $f:Y\rightarrow{X}$, there is a
            unique continuous function
            $\tilde{f}:(Y,\,\tau_{Y})\rightarrow{C}(X)$ such that
            $f=\alpha\circ{U}(\tilde{f})$.
            \par\hfill\par
            We've seen this too. Let $U$ be the forgetful functor,
            and define $C(X)=(X,\,\{\,\emptyset,\,X\,\})$.%
            \index{Indiscrete Topology}
            The injective
            function $\alpha:X\rightarrow{X}$ is once again the identity,
            $\alpha=\textrm{id}_{X}$. Then given any topological space
            $(Y,\,\tau_{Y})$ and any function $f:Y\rightarrow{X}$, the
            unique continuous function that does the trick is
            $\tilde{f}=f$ (See
            Fig.~\ref{fig:commutative_diagram_cofree_object_in_topology}).
            That is, the cofree object in \textbf{Top} is the indiscrete
            topology.\index{Cofree Object}
            \begin{figure}
                \centering
                \includegraphics{../../../images/commutative_diagram_cofree_object_in_topology.pdf}
                \caption{Cofree Object in Topology}
                \label{fig:commutative_diagram_cofree_object_in_topology}
            \end{figure}
        \subsection{Homeomorphisms}
            The arrows in category theory are called \textit{morphisms}.%
            \index{Morphisms}
            Given a category $\mathbf{C}$, two objects
            $A,\,B$ in $\textrm{obj}(\mathbf{C})$, and an arrow
            $f:A\rightarrow{B}$, an inverse is an arrow
            $g:B\rightarrow{A}$ such that the composition operations
            $f\circ{g}$ and $g\circ{f}$ yield the identity morphism.
            \textit{Isomorphisms}\index{Isomorphisms}
            are morphisms that have inverses.
            \begin{example}
                In \textbf{Set} the morphisms are just functions. The
                invertible functions are precisely those that are bijective.
                That is, the isomorphisms in \textbf{Set} are bijections.
            \end{example}
            \begin{example}
                In \textbf{Grp} the morphisms are group homomorphisms. A
                bijective group homorphism automatically yields a group
                homomorphism for the inverse. The isomorphisms in
                \textbf{Grp} are bijective group homomorphisms, also called
                group isomorphisms.
            \end{example}
            \begin{example}
                In $\mathbf{Vec}_{\mathbb{R}}$, the category of real vector
                spaces, the morphisms are linear transformations. Bijective
                linear transformations have linear inverses, so the
                isomorphisms in $\mathbf{Vec}_{\mathbb{R}}$ are just bijective
                linear transformations.
            \end{example}
            In topology the morphisms are continuous functions. Unlike the
            aforementioned algebraic structures, bijective continuous functions
            need not yield continuous inverses.
            \begin{example}
                Consider the circle $\mathbb{S}^{1}$ and half-open interval
                $[0,\,1)$, both with their standard subspace topologies.
                The function $\varphi:[0,\,1)\rightarrow\mathbb{S}^{1}$ defined
                by:
                \begin{equation}
                    \varphi(t)=\big(\cos(2\pi{t}),\,\sin(2\pi{t})\big)
                \end{equation}
                is continuous (since it is continuous in both components) and
                bijective, but the inverse is not continuous. The inverse
                function creates a \textit{rip} at the point
                $(1,\,0)$ (you can prove this using a $\varepsilon-\delta$
                argument since these are metric spaces).%
                \index{Continuity}
            \end{example}
            The isomorphisms in topology%
            \index{Isomorphisms!in Topology}
            are continuous bijective functions
            with continuous inverses. These are given a name.
            \begin{definition}[Homeomorphism]
                A homeomorphism from a topological space $(X,\,\tau_{X})$ to a
                topological space $(Y,\,\tau_{Y})$ is a continuous bijective
                function $f:X\rightarrow{Y}$ such that
                $f^{-1}:Y\rightarrow{X}$ is continuous.%
                \index{Homeomorphism!Definition}\index{Homeomorphism}
            \end{definition}
            \begin{example}
                Equip $\mathbb{R}$ with the Euclidean topology. The function
                $f(x)=x^{3}$ is continuous and bijective, and the inverse
                $f^{-1}(x)=x^{1/3}$ is continuous as well, meaning $f$ is a
                homeomorphism. A homeomorphism from a space to itself is
                sometimes called an \textit{autohomeomorphism}.%
                \index{Homeomorphism!Autohomeomorphism}
            \end{example}
            \begin{definition}[Open Mapping]
                An open mapping from a topological space $(X,\,\tau_{X})$ to a
                topological space $(Y,\,\tau_{Y})$ is a function
                $f:X\rightarrow{Y}$ such that for all $\mathcal{U}\in\tau_{X}$
                it is true that $f[\mathcal{U}]\in\tau_{Y}$.%
                \index{Open Mapping}
            \end{definition}
            Open mappings need not be continuous, and continuous functions do
            not need to be open mappings. When we do have both, we're not far
            from a homeomoprhism.
            \begin{theorem}
                If $(X,\,\tau_{X})$ and $(Y,\,\tau_{Y})$ are topological spaces,
                and if $f:X\rightarrow{Y}$ is a function, then $f$ is a
                homeomorphism if and only if it is a continuous bijective open
                mapping.\index{Theorem!Equivalence of Homeomorphism}
            \end{theorem}
        \subsection{Product Spaces}
            \begin{figure}
                \centering
                \includegraphics{../../../images/commutative_diagram_product_object_in_topology.pdf}
                \caption{Categorical Diagram for Products in Topology}
                \label{fig:commutative_diagram_product_object_in_topology}
            \end{figure}
            We need to discuss product spaces in order to talk about homotopy.
            Products in topology are defined in one of two ways. The first,
            and perhaps more natural, way is
            by \textit{generating} the topology. Given
            two spaces $(X,\,\tau_{X})$ and $(Y,\,\tau_{Y})$, we form the
            product space $(X\times{Y},\,\tau_{X\times{Y}})$ by looking at
            the products of open sets in $\tau_{X}$ and $\tau_{Y}$. The
            collection of all such sets is not usually a topology, so we define
            $\tau_{X\times{Y}}$ to be the \textit{smallest} topology such that
            $\mathcal{U}\times\mathcal{V}\in\tau_{X\times{Y}}$ for all
            $\mathcal{U}\in\tau_{X}$ and $\mathcal{V}\in\tau_{Y}$.%
            \index{Product Topology}\index{Topological Space!Product}
            \begin{definition}[Product Topological Space]
                The product of two topological spaces $(X,\,\tau_{X})$ and
                $(Y,\,\tau_{Y})$ is the space $(X\times{Y},\,\tau_{X\times{Y}})$
                where $\tau_{X\times{Y}}$ is the smallest topology that
                contains the set:
                \begin{equation}
                    \tilde{\tau}_{X\times{Y}}
                    =\{\,\mathcal{U}\times\mathcal{V}\;|\;
                        \mathcal{U}\in\tau_{X}\textrm{ and }
                        \mathcal{V}\in\tau_{Y}\,\}
                \end{equation}
                That is, the topology \textit{generated} by the product of
                open sets.\index{Product Topology!Definition}%
                \index{Product Topology}
            \end{definition}
            This, to me, is the more intuitive definition. It generalizes to
            finite products by induction. Infinite products have the issue of
            choosing the \textit{product} or the \textit{box} topologies,%
            \index{Box Topology}\index{Product Topology!Box}
            see my notes for Math 54. We'll be mostly concerned with finite
            products in this course.
            \par\hfill\par
            The alternate definition is categorical. The use is that it has the
            same definition for the product of groups, vector spaces, sets,
            etc. For the product of sets we have the canonical projections
            $\pi_{1}:X_{1}\times{X}_{2}\rightarrow{X}_{1}$ defined by
            $\pi_{1}(x_{1},\,x_{2})=x_{1}$, and similarly for $\pi_{2}$. The
            product topological space is defined by the object in
            \textbf{Top} with the following property. Given \textit{any}
            topological space $(Y,\,\tau_{Y})$, and any two continuous functions
            $f_{1}:Y\rightarrow{X}_{1}$ and $f_{2}:Y\rightarrow{X}_{2}$, there
            is a unique continuous function $f:Y\rightarrow{X}_{1}\times{X}_{2}$
            that makes
            Fig.~\ref{fig:commutative_diagram_product_object_in_topology}
            commute.
            \par\hfill\par
            This says that to check the continuity of
            $f:Y\rightarrow{X}_{1}\times{X}_{2}$ it is sufficient to check that
            the components of $f$ are continuous.
            \begin{example}
                The function $f:\mathbb{R}\rightarrow\mathbb{R}^{3}$ defined
                by $f(t)=(t,\,t^{2}\exp(t),\,\cos(t^{2}))$ is
                continuous. Why? We will not be looking at the pre-image of
                open sets, and even $\varepsilon-\delta$ proofs look tedious
                here. But each of the components are continuous, so we
                automatically know that function itself is continuous.
            \end{example}
            One way of thinking of product spaces is by taking a copy of
            $X_{1}$ and attaching it to every point of $X_{2}$ (or vice-versa).
            This is most easily visualized with the torus,
            $\mathbb{T}=\mathbb{S}^{1}\times\mathbb{S}^{1}$ (See
            Fig.~\ref{fig:../../../images/torus_skeleton_product_space.pdf}).
            \begin{figure}
                \centering
                \includegraphics{../../../images/torus_skeleton_product_space.pdf}
                \caption{Torus as a Product of Circles}
                \label{fig:../../../images/torus_skeleton_product_space.pdf}
            \end{figure}
        \subsection{Homotopy and Homotopy Equivalence}
            A notion weaker than homeomorphism, but equally useful, is
            homotopy equivalence.\index{Homotopy Equivalence}
            It is defined in terms of homotopy\index{Homotopy}, which is
            the idea of stretching continuous functions in a topological space.
            \begin{definition}[Homotopy]
                A homotopy between continuous function
                $f,g:X\rightarrow{Y}$ from a topological space $(X,\,\tau_{X})$
                to a topological space $(Y,\,\tau_{Y})$ is a continuous
                functions $H:X\times[0,\,1]:\rightarrow{Y}$
                (with respect to the subspace topology on $[0,\,1]$ and the
                product topology on $X\times[0,\,1]$) such that
                $H(x,\,0)=f(x)$ and $H(x,\,1)=g(x)$ for all $x\in{X}$.%
                \index{Homotopy}\index{Homotopy!Definition}
            \end{definition}
            This is shown pictorially in Fig.~\ref{fig:homotopy_basic_example}.
            \begin{figure}
                \centering
                \includegraphics{../../../images/homotopy_basic_example.pdf}
                \caption{Pictorial Representation of Homotopy}
                \label{fig:homotopy_basic_example}
            \end{figure}
            \begin{example}
                Given any two continuous functions
                $f,g:\mathbb{R}^{M}\rightarrow\mathbb{R}^{N}$, the
                \textit{straight-line homotopy}%
                \index{Homotopy!Straight-Line}\index{Straight-Line Homotopy}
                is the function
                $H:\mathbb{R}^{M}\times[0,\,1]\rightarrow\mathbb{R}^{N}$
                defined by:
                \begin{equation}
                    H(\mathbf{x},\,t)=
                    (1-t)\,f(\mathbf{x})+t\,g(\mathbf{x})
                \end{equation}
                This is continuous, being the sum of continuous functions.
                Plugging in $t=0$ we get $H(\mathbf{x},\,0)=f(\mathbf{x})$,
                and $t=1$ yields $H(\mathbf{x},\,1)=g(\mathbf{x})$. So
                $f$ and $g$ are homotopic, and $H$ is such a homotopy.
            \end{example}
            \begin{figure}
                \centering
                \includegraphics{../../../images/homotopy_on_unit_interval.pdf}
                \caption{Straight-Line Homotopy}
                \label{fig:homotopy_on_unit_interval}
            \end{figure}
            The straight line homotopy is shown in
            Fig.~\ref{fig:homotopy_on_unit_interval} for two curves in a
            subspace $Y\subseteq\mathbb{R}^{2}$.
            \par\hfill\par
            Homotopy is used to define \textit{homotopy equivalence}%
            \index{Homotopy Equivalence}\index{Homotopy!Equivalence}
            which is a weaker form of equivalence for topological spaces.
            It is defined using \textit{homotopy inverses}.%
            \index{Homotopy Inverse}\index{Homotopy!Inverse}
            \begin{definition}[Homotopy Inverse]
                A homotopy inverse of a continuous function
                $f:X\rightarrow{Y}$ from a topological space $(X,\,\tau_{X})$
                to a space $(Y,\,\tau_{Y})$ is a continuous function
                $g:Y\rightarrow{X}$ such that $g\circ{f}$ is homotopic to
                $\textrm{id}_{X}$ and $f\circ{g}$ is homotopic to
                $\textrm{id}_{X}$.\index{Homotopy Inverse}
            \end{definition}
            \begin{definition}[Homotopy Equivalence]
                A homotopy equivalence from a topological space
                $(X,\,\tau_{X})$ to a topological space $(Y,\,\tau_{Y})$ is a
                continuous function $f:X\rightarrow{Y}$ such that there exists
                a homotopy inverse $g:Y\rightarrow{X}$ for $f$.%
                \index{Homotopy Equivalence}
            \end{definition}
            \begin{theorem}
                A homeomorphism is a homotopy equivalence.%
                \index{Theorem!Homeomorphism implies Homotopy Equivalence}
            \end{theorem}
            \begin{proof}
                A homeomorphism $f:X\rightarrow{Y}$ has the property that
                $f^{-1}$ is continuous and hence
                $f\circ{f}^{-1}$ is \textit{equal} to $\textrm{id}_{X}$, not
                just homotopy equivalent. Similarly,
                $f^{-1}\circ{f}=\textrm{id}_{Y}$.
                So $f$ is a homotopy equivalence.
            \end{proof}
            This theorem does not reverse.
            \begin{example}
                $\mathbb{R}^{N}$ is homotopy equivalent to a one-point
                space $\{\,0\,\}$ (There is only one topology on a one-point
                space, $\tau=\big\{\,\emptyset,\,\{\,0\,\}\,\big\}$). The
                homotopy equivalence $f:\mathbb{R}^{N}\rightarrow\{\,0\,\}$
                is the only function that exists, $f(\mathbf{x})=0$.
                The homotopy inverse $g:\{\,0\,\}\rightarrow\mathbb{R}^{N}$
                can be any function you like, but let's pick
                $g(0)=\mathbf{0}$ to make things simple. The composition
                $f\circ{g}$ is the identity on $\{\,0\,\}$, so it is certainly
                homotopic to the identity. Going the other way,
                $g\circ{f}$ is homotopic to the identity on
                $\mathbb{R}^{N}$. Define $H$ by:
                \begin{equation}
                    H(\mathbf{x},\,t)=t\mathbf{x}
                \end{equation}
                $H(\mathbf{x},\,0)$ is the function $g\circ{f}$, and
                $H(\mathbf{x},\,1)$ is the identity. This is the
                \textit{straight-line} homotopy we've seen before.%
                \index{Straight-Line Homotopy}\index{Homotopy!Straight-Line}
            \end{example}
            $\mathbb{R}^{N}$ is an example of a \textit{contractible} space.%
            \index{Contractible Topological Space}
            \begin{definition}[Contractible Topological Space]
                A contractible topological space is a space
                $(X,\,\tau)$ that is homotopy equivalent to a one-point space.%
                \index{Contractible Space}
            \end{definition}
            We saw that every continuous function from $\mathbb{R}^{M}$ to
            $\mathbb{R}^{N}$ is homotopy equivalent via the straight-line
            homotopy. The real culprit behind this is contractibility.
            \begin{theorem}
                If $(X,\,\tau_{X})$ is a topological space, if $(Y,\,\tau_{Y})$
                is contractible, and if $f,g:X\rightarrow{Y}$ are continuous,
                then they are homotopy equivalent.
            \end{theorem}
            \begin{proof}
                We can exclude the case of $X$ or $Y$ being empty as trivial.
                Let $y\in{Y}$. There is only one one-point space, so
                $\{\,y\,\}$ with the subspace topology is a one point space.
                Since $(Y,\,\tau_{Y})$ is contractible, there is a homotopy
                equivalence $\alpha:Y\rightarrow\{\,y\,\}$ and a homotopy
                inverse $\beta:\{\,y\,\}\rightarrow{Y}$. Let
                $H$ be a homotopy between
                $\textrm{id}_{Y}$ and $\beta\circ\alpha$. Then
                $G:X\times[0,\,1]\rightarrow{Y}$ defined by:
                \begin{equation}
                    G(x,\,t)=
                    \begin{cases}
                        H\big(f(x),\,2t\big),&0\leq{t}\leq\frac{1}{2}\\
                        H\big(g(x),\,2-2t\big),&\frac{1}{2}\leq{t}\leq{1}
                    \end{cases}
                \end{equation}
                is continuous by the pasting-lemma since
                $H\big(f(x),\,1\big)=H\big(g(x),\,1\big)=\beta(y)$.
                $G$ is thus a homotopy between $f$ and $g$.
            \end{proof}
            You can get a different category out of topology. The objects are
            still topological spaces, but the arrows are
            \textit{equivalent classes} of continuous functions under the
            equivalence relation of homotopic. Isomorphisms are then
            equivalence classes of homotopy equivalences. Two topological
            spaces are then considered the \textit{same} if they are
            homotopy equivalent. This allows a lot more squooshing of the
            space ($\mathbb{R}^{N}$ is homotopy equivalent to a single point
            but certainly not homeomorphic).
    \section{Compactness and Connectedness}
        I think that was more than enough category theory for a while. Let's
        return to simpler point-set language and review some more topological
        ideas.
        \subsection{Compact Spaces}
            \begin{definition}[Compact Topological Space]
                A compact topological space is a topological space $(X,\,\tau)$
                such that for every open over (a set $\mathcal{O}\subseteq\tau$
                such that $\bigcup\mathcal{O}=X$) there is a finite subset
                $\Delta\subseteq\mathcal{O}$ that covers $X$.%
                \index{Compact Topological Space}
            \end{definition}
            \begin{theorem}[Heine-Borel Theorem]
                A subset $\mathcal{C}\subseteq\mathbb{R}^{N}$ is compact if and
                only if $\mathcal{C}$ is closed and bounded (with respect to
                the Euclidean metric).\index{Theorem!Heine-Borel}
            \end{theorem}
            \begin{theorem}[Generalized Heine-Borel Theorem]
                A subset $\mathcal{C}\subseteq{X}$ of a metric space is compact
                if and only if it is closed and totally-bounded.%
                \index{Theorem!Generalized Heine-Borel}
            \end{theorem}
            Both of these theorems were proved in detail in Math 54. We'll make
            frequent use of them.
        \subsection{Connected Spaces}
            There are several notions of a space being \textit{connected}, and
            in a course on planar topology it is essential to note the
            differences. The simplest notion of connectedness uses open sets
            and describes how to \textit{disconnect} a
            space.\index{Topological Space!Disconnected}%
            \index{Disconnected Topological Space}
            \begin{definition}[Disconnected Topological Space]
                A disconnected topological space is a topological space
                $(X,\,\tau)$ such that there exist two disjoint non-empty open
                subsets $\mathcal{U},\mathcal{V}\subseteq\tau$ such that
                $\mathcal{U}\cup\mathcal{V}=X$.%
                \index{Topological Space!Disconnected}
            \end{definition}
            \begin{definition}[Connected Topological Space]
                A connected topological space is a topological space that is
                not disconnected.\index{Topological Space!Connected}
            \end{definition}
            \begin{figure}
                \centering
                \includegraphics{../../../images/disconnected_space_001.pdf}
                \caption{A Disconnected Topological Space}
                \label{fig:disconnected_space_001}
            \end{figure}
            Stronger than connectedness is the notion of
            \textit{path} connected.\index{Path Connected Space}%
            \index{Topological Space!Path Connected}
            \begin{definition}[Path Connected Topological Space]
                A path connected topological space is a topological space
                $(X,\,\tau)$ such that for all $x,y\in{X}$ there is a
                continuous path $\gamma:[0,\,1]\rightarrow{X}$ such that
                $\gamma(0)=x$ and $\gamma(1)=y$.
            \end{definition}
            \begin{theorem}
                If $(X,\,\tau)$ is a path connected topological space,
                then it is connected.%
                \index{Theorem!Path Connected implies Connected}
            \end{theorem}
            \begin{proof}
                Suppose not. Then there are two non-empty disjoint open subsets
                $\mathcal{U},\mathcal{V}\subseteq{X}$ such that
                $\mathcal{U}\cup\mathcal{V}=X$. Since they are non-empty, there
                are points $x\in\mathcal{U}$ and $y\in\mathcal{V}$. But
                $(X,\,\tau)$ is path connected, so there is a continuous path
                $\gamma:[0,\,1]\rightarrow{X}$ such that $\gamma(0)=x$
                and $\gamma(1)=y$. Since $\gamma$ is continuous, and since
                $\mathcal{U}$ and $\mathcal{V}$ are open,
                $\gamma^{-1}[\mathcal{U}]$ and $\gamma^{-1}[\mathcal{V}]$ are
                open. But these subsets are also disjoint and non-empty, meaning
                $[0,\,1]$ is disconnected, but it is not, a contradiction. So
                $(X,\,\tau)$ is connected.
            \end{proof}
            One more stronger notion.
            \begin{definition}[Arc Connected Topological Space]
                An arc connected topological space is a topological space
                $(X,\,\tau)$ such that for all $x,y\in{X}$ there is an
                injective continuous path $\gamma:[0,\,1]\rightarrow{X}$ such
                that $\gamma(0)=x$ and $\gamma(1)=y$.
                \index{Arc Connected Topological Space}%
                \index{Topological Space!Arc Connected}
            \end{definition}
            Every path connected Hausdorff space is arc connected. This will
            take a bit of work to prove but many of the central ideas of
            planar topology are involved. We'll be talking a lot about arc
            connected spaces when we discuss space filling curves.
    \section{More Topological Properties}
        Lastly, a very brief review of some more properties covered in Math 54.
        \subsection{Separation Properties}
            \begin{figure}
                \centering
                \includegraphics{../../../images/frechet_condition_001.pdf}
                \caption{The Fr\'{e}chet Condition for Topological Spaces}
                \label{fig:frechet_condition_001}
            \end{figure}
            \begin{figure}
                \centering
                \includegraphics{../../../images/regular_condition_001.pdf}
                \caption{The Regular Condition for Topological Spaces}
                \label{fig:regular_condition_001}
            \end{figure}
            \begin{figure}
                \centering
                \includegraphics{../../../images/normal_condition_001.pdf}
                \caption{The Normal Condition for Topological Spaces}
                \label{fig:normal_condition_001}
            \end{figure}
            \begin{definition}[Fr\'{e}chet Topological Space]
                A Fr\'{e}chet topological space, also called a
                $T_{1}$ space, is a topological space $(X,\,\tau)$ such that
                for all distinct $x,y\in{X}$ there are open sets
                $\mathcal{U},\mathcal{V}\in\tau$ such that $x\in\mathcal{U}$,
                $x\notin\mathcal{V}$, and $ y\in\mathcal{V}$,
                $y\notin\mathcal{U}$ (Fig.~\ref{fig:frechet_condition_001}).%
                \index{Fr\'{e}chet Topological Space}%
                \index{Topological Space!Fr\'{e}chet}
            \end{definition}
            \begin{definition}[Regular Topological Space]
                A regular topological space is a space $(X,\,\tau)$ such that
                for all $x\in{X}$ and for all closed $\mathcal{C}\subseteq{X}$
                with $x\notin\mathcal{C}$ there exist disjoint open subsets
                $\mathcal{U},\mathcal{V}\in\tau$ such that
                $x\in\mathcal{U}$ and $\mathcal{C}\subseteq\mathcal{V}$
                (Fig.~\ref{fig:regular_condition_001}).%
                \index{Regular Topological Space}%
                \index{Topological Space!Regular}
            \end{definition}
            \begin{definition}[Normal Topological Space]
                A normal topological space is a space $(X,\,\tau)$ such that
                for all disjoint closed sets
                $\mathcal{C},\mathcal{D}\subseteq{X}$ there exist disjoint
                open subsets $\mathcal{U},\mathcal{V}\in\tau$ such that
                $\mathcal{C}\subseteq\mathcal{U}$ and
                $\mathcal{D}\subseteq\mathcal{V}$
                (Fig.~\ref{fig:normal_condition_001}).%
                \index{Normal Topological Space}%
                \index{Topological Space!Normal}
            \end{definition}
        \subsection{Countability Properties}
            \begin{definition}[Base (Topology)]
                A base for a topology $\tau$ on a set $X$ is a subset
                $\mathcal{B}\subseteq\tau$ such that $\bigcup\mathcal{B}=X$
                ($\mathcal{B}$ is an open cover), and for all
                $\mathcal{U},\mathcal{V}\in\mathcal{B}$, and for all
                $x\in\mathcal{U}\cap\mathcal{V}$, there is a
                $\mathcal{W}\in\mathcal{B}$ such that
                $x\in\mathcal{W}$ and
                $\mathcal{W}\subseteq\mathcal{U}\cap\mathcal{V}$.%
                \index{Base (Topology)}
            \end{definition}
            \begin{example}
                The set of all open intervals $(a,\,b)\subseteq\mathbb{R}$
                form a basis for the Euclidean topology on the real line.
            \end{example}
            \begin{definition}[Second Countability]
                A second countable topological space is a topological space
                $(X,\,\tau)$ such that there exists a countable base
                $\mathcal{B}$ for $\tau$.%
                \index{Second Countable Topological Space}
            \end{definition}
            \begin{example}
                The set of all open intervals $(p,\,q)$ with
                \textit{rational endpoint}, $p,q\in\mathbb{Q}$, forms a
                countable basis for the Euclidean topology on the real line.
                This shows that $\mathbb{R}$ is second countable.
            \end{example}
            \begin{theorem}[Urysohn's Metrization Theorem]
                If $(X,\,\tau)$ is a regular Hausdorff topological space that
                is second countable, then it is metrizable. That is, there is
                some metric $d$ on $X$ that induces $\tau$.%
                \index{Theorem!Urysohn's}
            \end{theorem}
    \printindex
\end{document}
