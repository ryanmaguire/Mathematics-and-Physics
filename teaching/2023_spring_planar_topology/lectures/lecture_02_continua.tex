%-----------------------------------LICENSE------------------------------------%
%   This file is part of Mathematics-and-Physics.                              %
%                                                                              %
%   Mathematics-and-Physics is free software: you can redistribute it and/or   %
%   modify it under the terms of the GNU General Public License as             %
%   published by the Free Software Foundation, either version 3 of the         %
%   License, or (at your option) any later version.                            %
%                                                                              %
%   Mathematics-and-Physics is distributed in the hope that it will be useful, %
%   but WITHOUT ANY WARRANTY; without even the implied warranty of             %
%   MERCHANTABILITY or FITNESS FOR A PARTICULAR PURPOSE.  See the              %
%   GNU General Public License for more details.                               %
%                                                                              %
%   You should have received a copy of the GNU General Public License along    %
%   with Mathematics-and-Physics.  If not, see <https://www.gnu.org/licenses/>.%
%------------------------------------------------------------------------------%
\documentclass{article}
\usepackage{graphicx}                           % Needed for figures.
\usepackage{amsmath}                            % Needed for align.
\usepackage{amssymb}                            % Needed for mathbb.
\usepackage{amsthm}                             % For the theorem environment.
\usepackage{imakeidx}
\usepackage{hyperref}
\hypersetup{
    colorlinks=true,
    linkcolor=blue,
    filecolor=magenta,
    urlcolor=Cerulean,
    citecolor=SkyBlue
}

%------------------------Theorem Styles-------------------------%
\theoremstyle{plain}
\newtheorem{theorem}{Theorem}[section]

% Define theorem style for default spacing and normal font.
\newtheoremstyle{normal}
    {\topsep}               % Amount of space above the theorem.
    {\topsep}               % Amount of space below the theorem.
    {}                      % Font used for body of theorem.
    {}                      % Measure of space to indent.
    {\bfseries}             % Font of the header of the theorem.
    {}                      % Punctuation between head and body.
    {.5em}                  % Space after theorem head.
    {}

% Define default environments.
\theoremstyle{normal}
\newtheorem{examplex}{Example}[section]
\newtheorem{definitionx}{Definition}[section]

\newenvironment{example}{%
    \pushQED{\qed}\renewcommand{\qedsymbol}{$\blacksquare$}\examplex%
}{%
    \popQED\endexamplex%
}

\newenvironment{definition}{%
    \pushQED{\qed}\renewcommand{\qedsymbol}{$\blacksquare$}\definitionx%
}{%
    \popQED\enddefinitionx%
}

\title{Planar Topology: Lecture 2 - Continua}
\author{Ryan Maguire}
\date{Spring 2023}

% No indent and no paragraph skip.
\setlength{\parindent}{0em}
\setlength{\parskip}{0em}
\makeindex[intoc]
\begin{document}
    \maketitle
    \tableofcontents
    \listoffigures
    \section{Motivation}
        Georg Cantor (C.E. 1845 - 1918)\index{Cantor, Georg}
        demonstrated that every topological
        manifold (Hausdorff, second countable, and locally Euclidean, think of
        surfaces in $\mathbb{R}^{3}$) of positive dimension (i.e. excluding
        the empty manifold and zero dimensional singleton sets) can be put in
        bijection with the unit interval $[0,\,1]$. The function that does this
        likely will not be continuous, and almost certainly will not be a
        homeomorphism, but it shows that the closed interval has the same
        \textit{size}, in some sense, as any other non-trivial
        topological manifold.
        Giuseppe Peano (C.E. 1858 - 1932)\index{Peano, Giuseppe}, motivated by
        this discovery, proved that the unit interval can be
        \textit{continuously} mapped in a surjective fashion onto the unit
        square $[0,\,1]^{2}$. The result is a
        \textit{space-filling curve}\index{Space-Filling Curve}, a curve
        that fills a higher dimensional region. Because of Peano's works,
        space-filling curves that curve a two-dimensional region are called
        \textit{Peano Curves}\index{Peano Curve}.
        \par\hfill\par
        These results naturally lead to the following question.
        \begin{equation}
            \textrm{Which spaces are the continuous image of the unit interval?}
        \end{equation}
        The answer is unbelievable surprising. \textit{Most} reasonable spaces
        are just the continuous image of $[0,\,1]$. This is the
        \textit{Hahn-Mazurkiewicz Theorem}.\index{Theorem!Hahn-Mazurkiewicz}
        \begin{theorem}[Hahn-Mazurkiewicz]
            If $(X,\,\tau)$ is a Hausdorff topological space, then there is a
            continuous surjective function $f:[0,\,1]\rightarrow{X}$ if and
            only if $(X,\,\tau)$ is compact, connected, locally connected, and
            second countable.\index{Theorem!Hahn-Mazurkiewicz}
        \end{theorem}
        One direction is nearly trivial. Since $[0,\,1]$ is compact and
        connected, if $(X,\,\tau)$ is the continuous image of $[0,\,1]$ it
        inherits these properties. Second countable and locally connected are
        a little harder, but not too difficult. The
        reverse direction is so unbelievable one may question if it is true.
        We will spend a fair amount of time working to prove this result and
        many of its corollaries, including that Hausdorff path connected spaces
        are arc connected. Another freebie is that the continuous image of
        $[0,\,1]$ is metrizable (Hausdorff and compact implies regular,
        and second countable regular Hausdorff spaces are metrizable by
        Urysohn's Metrization Theorem\index{Theorem!Urysohn's}).
    \printindex
\end{document}
