%-----------------------------------LICENSE------------------------------------%
%   This file is part of Mathematics-and-Physics.                              %
%                                                                              %
%   Mathematics-and-Physics is free software: you can redistribute it and/or   %
%   modify it under the terms of the GNU General Public License as             %
%   published by the Free Software Foundation, either version 3 of the         %
%   License, or (at your option) any later version.                            %
%                                                                              %
%   Mathematics-and-Physics is distributed in the hope that it will be useful, %
%   but WITHOUT ANY WARRANTY; without even the implied warranty of             %
%   MERCHANTABILITY or FITNESS FOR A PARTICULAR PURPOSE.  See the              %
%   GNU General Public License for more details.                               %
%                                                                              %
%   You should have received a copy of the GNU General Public License along    %
%   with Mathematics-and-Physics.  If not, see <https://www.gnu.org/licenses/>.%
%------------------------------------------------------------------------------%
\documentclass{article}
\usepackage{graphicx}                           % Needed for figures.
\usepackage{amsmath}                            % Needed for align.
\usepackage{amssymb}                            % Needed for mathbb.
\usepackage{amsthm}                             % For the theorem environment.
\usepackage{imakeidx}
\usepackage{minted}
\usepackage{hyperref}
\hypersetup{
    colorlinks=true,
    linkcolor=blue,
    filecolor=magenta,
    urlcolor=Cerulean,
    citecolor=SkyBlue
}

%------------------------Theorem Styles-------------------------%
\theoremstyle{plain}
\newtheorem{theorem}{Theorem}[section]

% Define theorem style for default spacing and normal font.
\newtheoremstyle{normal}
    {\topsep}               % Amount of space above the theorem.
    {\topsep}               % Amount of space below the theorem.
    {}                      % Font used for body of theorem.
    {}                      % Measure of space to indent.
    {\bfseries}             % Font of the header of the theorem.
    {}                      % Punctuation between head and body.
    {.5em}                  % Space after theorem head.
    {}

% Define default environments.
\theoremstyle{normal}
\newtheorem{examplex}{Example}[section]
\newtheorem{definitionx}{Definition}[section]

\newenvironment{example}{%
    \pushQED{\qed}\renewcommand{\qedsymbol}{$\blacksquare$}\examplex%
}{%
    \popQED\endexamplex%
}

\newenvironment{definition}{%
    \pushQED{\qed}\renewcommand{\qedsymbol}{$\blacksquare$}\definitionx%
}{%
    \popQED\enddefinitionx%
}

\title{Planar Topology: Lecture 2 - Continua}
\author{Ryan Maguire}
\date{Spring 2023}

% No indent and no paragraph skip.
\setlength{\parindent}{0em}
\setlength{\parskip}{0em}
\makeindex[intoc]
\begin{document}
    \maketitle
    \tableofcontents
    \listoffigures
    \section{Motivation}
        Georg Cantor (C.E. 1845 - 1918)\index{Cantor, Georg}
        demonstrated that every connected topological
        manifold (Hausdorff, second countable, and locally Euclidean, think of
        surfaces in $\mathbb{R}^{3}$) of positive dimension (i.e. excluding
        the empty manifold and zero dimensional singleton sets) can be put in
        bijection with the unit interval $[0,\,1]$. The function that does this
        likely will not be continuous, but it shows that the closed interval
        has the same \textit{size}, in some sense, as any other non-trivial
        topological manifold.
        Giuseppe Peano (C.E. 1858 - 1932)\index{Peano, Giuseppe}, motivated by
        this discovery, proved that the unit interval can be
        \textit{continuously} mapped in a surjective fashion onto the unit
        square $[0,\,1]^{2}$. The result is a
        \textit{space-filling curve}\index{Space-Filling Curve}, a curve
        that fills a higher dimensional region. Because of Peano's works,
        space-filling curves that fill a two-dimensional region are called
        \textit{Peano Curves}\index{Peano Curve}.
        \par\hfill\par
        These results naturally lead to the following question.
        \begin{equation}
            \textrm{Which spaces are the continuous image of the unit interval?}
        \end{equation}
        The answer is surprising. \textit{Most} reasonable spaces
        are just the continuous image of $[0,\,1]$. This is the
        \textit{Hahn-Mazurkiewicz Theorem}.\index{Theorem!Hahn-Mazurkiewicz}
        \begin{theorem}[Hahn-Mazurkiewicz]
            If $(X,\,\tau)$ is a Hausdorff topological space, then there is a
            continuous surjective function $f:[0,\,1]\rightarrow{X}$ if and
            only if $(X,\,\tau)$ is compact, connected, locally connected, and
            second countable.\index{Theorem!Hahn-Mazurkiewicz}
        \end{theorem}
        One direction is nearly trivial. Since $[0,\,1]$ is compact and
        connected, if $(X,\,\tau)$ is the continuous image of $[0,\,1]$ it
        inherits these properties. Second countable and locally connected are
        a little harder, but not too difficult. The
        reverse direction is so unbelievable one may question if it is true.
        We will spend a fair amount of time working to prove this result and
        many of its corollaries, including that Hausdorff path connected spaces
        are arc connected.
    \section{Peano's Curve}
        \begin{figure}
            \centering
            \includegraphics{../../../images/peano_curve_001.pdf}
            \caption{Three Iterations of the Peano Curve}
            \label{fig:peano_curve_001}
        \end{figure}
        The Peano curve is described as the limit of a sequence of continuous
        functions $P_{n}:[0,\,1]\rightarrow[0,\,1]^{2}$. The curves
        $P_{n}$ are actually piecewise linear (polygonal). We start with the
        square $S_{0}=[0,\,1]^{2}$ and the center of this square $C_{0}$.
        If $S_{n}$ has been given, for each square $s$ in $S_{n}$ divide
        $s$ into nine other equally sized squares. Replace the center $c$ of $s$
        with a sequence of the centers of the nine newly created squares.
        This is obtained by grouping the smaller squares into three columns
        and ordering the centers contiguously within each column (that is,
        no jumping over squares in this ordering). We then
        order the columns from one side of the square so that the
        distance between consecutive pairs of points is given by the side
        length of the smaller squares. In total there are four ways of
        doing this.
        \begin{enumerate}
            \item Order everything bottom-to-top. That is, the left three
                centers are bottom-to-top, as are the middle centers, and the
                right centers.
            \item Zig-zag starting with bottom-to-top. That is, the left three
                centers are bottom-to-top, the middle three centers are
                top-to-bottom, and the right centers are again bottom-to-top.
            \item Zig-zag starting top-to-bottom. This is the mirror of the
                previous one, the left three centers are ordered top-to-bottom,
                the middle three are bottom-to-top, and the right centers are
                top-to-bottom.
            \item Order everything top-to-bottom. This is the mirror of the
                first idea. The left three centers go top-to-bottom, as do the
                middle and right columns.
        \end{enumerate}
        The ordering chosen for the next iteration $S_{n+1}$ is
        taken to be the one such that the distance between the first point of
        the ordering and the previous iteration in $P_{n}$ equals the side
        length of the small squares. We create $P_{n+1}$ be connecting the
        dots in the order given for all of the squares. The number of line
        segments in $P_{n}$ is hence given by $9^{n}$, which grows very fast.
        The Peano curve is the limit $P=\lim_{n\rightarrow\infty}P_{n}$.
        \par\hfill\par
        The classical description of the Peano curve, given above, is often
        easier to grasp using recursion. This is done efficiently using the
        standard Python module \texttt{turtle}, which helps one draw simple
        shapes and patterns. This is done below.
        \begin{minted}[gobble=10]{python}
            import turtle

            def peano(level, length):
                if level == 0:
                    turtle.forward(length)
                else:
                    angle = 90
                    peano(level-1, length/3)
                    turtle.right(angle)
                    peano(level-1, length/3)

                    for _ in range(2):
                        for _ in range(3):
                            turtle.left(angle)
                            peano(level-1, length/3)
                        angle = -angle

                    turtle.left(angle)
                    peano(level-1, length/3)
        \end{minted}
        I adapted this to the asymptote vector graphics language to create
        Fig.~\ref{fig:peano_curve_001}. The rainbow gradient is given so one
        can understand how the curve progresses. The starting point is purple
        and the end of the curve is red. Note the function is \textit{not}
        injective, and it can't be. Since $[0,\,1]$ is compact, and
        since $[0,\,1]^{2}$ is Hausdorff (it's a metric space), any
        continuous bijection $f:[0,\,1]\rightarrow[0,\,1]^{2}$ would
        automatically be a homeomorphism, but these are not homeomorphic
        spaces.
        \par\hfill\par
        Many modifications of Peano's construction have been made since, and
        perhaps most popular is the one presented by
        David Hilbert (C.E. 1862 - 1943) in 1891. The
        \textit{Hilbert curve}\index{Hilbert Curve}\index{Hilbert, David}
        is similar to the Peano curve, but the $n^{th}$ iteration is injective
        for all $n\in\mathbb{N}$ (the limit is still not surjective). Again,
        it is perhaps easiest to describe using recursion.
        \begin{minted}[gobble=10]{python}
            import turtle

            def hilbert(level, angle, length):
                if level == 0:
                    return

                else:
                    turtle.right(angle)
                    hilbert(level-1, -angle, length)
             
                    turtle.forward(length)
                    turtle.left(angle)
                    hilbert(level-1, angle, length)
             
                    turtle.forward(length)
                    hilbert(level-1, angle, length)

                    turtle.left(angle)
                    turtle.forward(length)
                    hilbert(level-1, -angle, length)

                    turtle.right(angle)
        \end{minted}
        Using the previous scheme we get another asymptote figure. See
        Fig.~\ref{fig:hilber_curve_001}.
        \begin{figure}
            \centering
            \includegraphics{../../../images/hilbert_curve_001.pdf}
            \caption{Four Iterations of the Hilbert Curve}
            \label{fig:hilber_curve_001}
        \end{figure}
        Motivated by these ideas, \textit{Peano spaces}\index{Peano Space}
        are defined as follows.
        \begin{definition}[Peano Space]
            A Peano topological space is a topological space $(X,\,\tau)$
            that is Hausdorff, compact, connected, locally connected, and
            second countable.
        \end{definition}
        \begin{example}
            Every compact connected manifold $(M,\,\tau)$ is a Peano space.
            The Hausdorff property and second countability are part of the
            definition of manifold. Compact and connected are included in the
            hypothesis as well. Being locally connected comes fro the locally
            Euclidean property (every point $p\in{M}$ is locally like
            $\mathbb{R}^{n}$ for some $n\in\mathbb{N}$).
        \end{example}
        \begin{example}
            The wedge of 2 circles is the quotient of
            $\mathbb{S}^{1}\sqcup\mathbb{S}^{1}$ by identifying the points
            $(1,\,0)$ on both circles. This embeds into the plane as a
            \textit{figure eight} (Fig.~\ref{fig:wedge_of_two_circles_001}).
            It is not a manifold since the \textit{center} point is locally
            like an \textbf{X} and not $\mathbb{R}^{n}$ for any
            $n\in\mathbb{N}$. Nevertheless it is still a Peano space.
        \end{example}
        \begin{figure}
            \centering
            \includegraphics{../../../images/wedge_of_two_circles_001.pdf}
            \caption{Wedge of Two Circles}
            \label{fig:wedge_of_two_circles_001}
        \end{figure}
        \begin{example}
            Every locally connected, connected, and compact metric space is a
            Peano space. Metrizable implies Hausdorff, and
            metrizable plus Lindel\"{o}f (every \textit{countable} open cover
            has a finite subcover) implies second countable (see Math 54 notes),
            hence compact metrizable spaces are second countable.
        \end{example}
        \begin{example}
            The Hilbert cube is the product space:
            \begin{equation}
                H=\prod_{n\in\mathbb{N}}\Big[0,\,\frac{1}{n+1}\Big]
            \end{equation}
            with the product topology. It is compact, being the product of
            compact spaces, Hausdorff (again, being the product of Hausdorff
            spaces), and connected (see previous reasons). It is locally
            connected as well, and since it is metrizable with metric:
            \begin{equation}
                d(a,\,b)=
                \sum_{n\in\mathbb{N}}\frac{|a_{n}-b_{n}|}{2^{n}}
            \end{equation}
            it is automatically second countable. It is thus a Peano
            space. It is not locally Euclidean, but some consider
            it to be an \textit{infinite dimensional} manifold.
        \end{example}
        \begin{example}
            Topological graphs are 1 dimensional CW complexes. Planar
            topological graphs are topological graphs that can be embedded into
            $\mathbb{R}^{2}$. Let's try to describe this. We take a discrete
            subset of $\mathbb{R}^{2}$. Let $S_{0}\subseteq\mathbb{R}^{2}$
            (Question: Why is $S_{0}$ countable?). Between any two points
            $x,y\in{S}_{0}$ you may connect a continuous curve between them,
            so long as this curve doesn't intersect another curve you may have
            already drawn. You may also draw two or three or any finite number
            of curves between the same to points, and you can draw a curve
            from a point to itself.
            The resulting collection of curves and points is given the subspace
            topology from $\mathbb{R}^{2}$. If the set $S_{0}$ of points and the
            set $S_{1}$ of curves is finite, the graph is compact. If the graph
            is connected, it is then also a Peano space.
            \par\hfill\par
            A general 1 dimensional CW complex is formed in a similar manner.
            You take a discrete topological space
            $\big(S_{0},\,\tau_{0}\big)$ and another space
            $(S_{1},\,\tau_{1})$ that is the disjoint union of copies of the
            unit interval $[0,\,1]$. Given such a unit interval
            $I_{\alpha}\subseteq{S}_{1}$ you define an \textit{attaching map}
            $\varphi_{\alpha}:\{\,0,\,1\,\}\rightarrow{S}_{0}$ that maps the
            endpoints of the interval to (either one or two) points in $S_{0}$.
            We equip this new space with the quotient topology that comes from
            these gluings and the result is a 1 dimensional CW complex.
            The wedge of two circles that we saw in
            Fig.~\ref{fig:wedge_of_two_circles_001} can be seen as a 1
            dimensional CW complex formed by attaching two copies of the
            unit interval to a single point.
        \end{example}
        \begin{example}
            The Hawaiian earings are an example of a Peano space that
            \textit{looks} like a 1 dimensional CW complex, but is not.
            You attach circles of radius $r_{n}=\frac{1}{n+1}$ for all
            $n\in\mathbb{N}$ to the origin
            (See Fig.~\ref{fig:hawaiian_earrings}) and endow this with the
            subspace topology. This is a Peano space. It is compact since it is
            closed and bounded. It is second countable and Hausdorff, being a
            subspace of $\mathbb{R}^{2}$, and it is connected (it is path
            connected since every point connects to the origin along the
            circles). Lastly it is locally connected (the only point that may
            be of concern is the origin, but as you zoom in you will still find
            a connected neighborhood). This may \textit{look} like a 1
            dimensional CW complex, but it is not. It is tempting to call this
            the wedge of \textit{countably infinite} many circles, but this is
            a very different space. The quotient topology given to a CW complex
            can't have a point like the origin in the Hawaiian earrings.
            Nevertheless, the Hahn-Mazurkiewicz theorem still applies. This
            space is the continuous image of the closed unit interval.
        \end{example}
        \begin{figure}
            \centering
            \includegraphics{../../../images/hawaiian_earrings.pdf}
            \caption{The Hawaiian Earrings}
            \label{fig:hawaiian_earrings}
        \end{figure}
        Weaker than Peano spaces are the idea of \textit{continua}. We'll be
        talking about these for a while.
        \begin{definition}[Weak Topological Continuum]
            A weak topological continuum is a topological space $(X,\,\tau)$
            that is compact, connected, non-empty, and Hausdorff.%
            \index{Weak Topological Continuum}%
            \index{Topological Continuum!Weak}
        \end{definition}
        \begin{example}
            Every Peano space is a weak topological continuum by definition.
            A Peano space can be concisely stated to be spaces that are
            weak topological continua, and also locally connected and second
            countable.
        \end{example}
        \begin{definition}[Topological Continuum]
            A topological continuum is a topological space $(X,\,\tau)$ that
            is non-empty, compact, connected, and metrizable.%
            \index{Topological Continuum}
        \end{definition}
        The difference between a weak continuum and a continuum is that the
        Hausdorff condition has been upgraded to metrizability. Weak continua
        are sometimes called \textit{non-metric continua}.
        \begin{example}
            The closed long line is the one point compactification of the
            long line. It is Hausdorff, connected, and locally connected.
            It is also compact, being the one point compactification of a
            space. It is not metrizable since it contains a subspace that is
            not metrizable (it contains the long line as a subspace). It is
            hence a \textit{weak} topological continuum, but not a topological
            continuum.
        \end{example}
        \begin{example}
            The lexicographic unit square is the set $[0,\,1]^{2}$ equipped
            with the order topology induced by the lexicographic ordering.
            Given $(x_{0},\,y_{0})$ and $(x_{1},\,y_{1})$ we write
            $(x_{0},\,y_{0})\leq(x_{1},\,y_{1})$ if and only if
            $x_{0}\leq{x}_{1}$ or $x_{0}=x_{1}$ and $y_{0}\leq{y}_{1}$. That
            is, we examine the first component and if they are equal we then
            look to the second one. This space is not metrizable. The reason
            is slightly hard, we don't have the ingredients yet, but intuitively
            it's \textit{too big}. The precise reason is that a compact
            connected metrizable space that is locally connected must
            be path connected. The lexicographic square is not path connected,
            but it is compact, connected, and locally connected, so it must not
            be metrizable.
        \end{example}
        Another strange space that is a non-metrizable continuum is the
        \textit{Alexandroff square}, but it is difficult to describe. Suffice
        it to say that \textit{most} reasonable spaces that are weak
        topological continua are also topological continua.
        \begin{definition}[Nondegenerate Topological Continuum]
            A nondegenerate topological continuum is a topological continuum
            $(X,\,\tau)$ such that $\textrm{Card}(X)>1$. That is, a
            topological continuum that has more than one point.%
            \index{Topological Continuum!Nondegenerate}%
            \index{Nondegenerate Topological Continuum}
        \end{definition}
        This just means the space isn't too boring. Note that if a
        topological continuum (weak or not) is finite, it is automatically a
        discrete space. This is because the only Hausdorff topology on a finite
        collection of points is the power set. But a discrete space is
        always disconnected if the space has more than one point. Hence
        a continuum is finite if and only if it has one point. So
        nondegenerate continua are precisely those that aren't finite.
        \begin{example}
            There is only one \textit{degenerate} continuum, the singleton
            space $X=\{,0\,\}$ and $\tau=\{\,\emptyset,\,X\,\}$. This is one
            of the rare cases where the discrete and indiscrete topologies on
            $X$ agree (the other case is $X=\emptyset$).
        \end{example}
        \begin{definition}[Subcontinuum of a Topological Continuum]
            A subcontinuum of a topological continuum $(X,\,\tau)$ is a
            subspace $(A,\,\tau_{A})$, $A\subseteq{X}$, such that
            $(A,\,\tau_{A})$ is a topological continuum.
        \end{definition}
        \begin{example}
            The unit disk $\mathbb{D}^{2}\subseteq\mathbb{R}^{2}$ defined by:
            \begin{equation}
                \mathbb{D}^{2}=
                \{\,\mathbf{x}\in\mathbb{R}^{2}\;|\;
                    ||\mathbf{x}||_{2}\leq{1}\,\}
            \end{equation}
            where $||\cdot||_{2}$ denotes the standard $L^{2}$ norm, also called
            the \textit{Euclidean} norm. With the subspace topology from
            $\mathbb{R}^{2}$, the disk $\mathbb{D}^{2}$ becomes a topological
            continuum. It is a subspace of a metrizable space, so it too is
            metrizable. It is compact since it is closed and bounded. Lastly
            it is path connected since every point is connected to the origin,
            and hence $\mathbb{D}^{2}$ is connected. The circle $\mathbb{S}^{1}$
            is a subcontinuum of the disk. It too is a non-empty compact
            connected metric space and it lives as a subspace of the disk.
        \end{example}
    \printindex
\end{document}
