%-----------------------------------LICENSE------------------------------------%
%   This file is part of Mathematics-and-Physics.                              %
%                                                                              %
%   Mathematics-and-Physics is free software: you can redistribute it and/or   %
%   modify it under the terms of the GNU General Public License as             %
%   published by the Free Software Foundation, either version 3 of the         %
%   License, or (at your option) any later version.                            %
%                                                                              %
%   Mathematics-and-Physics is distributed in the hope that it will be useful, %
%   but WITHOUT ANY WARRANTY; without even the implied warranty of             %
%   MERCHANTABILITY or FITNESS FOR A PARTICULAR PURPOSE.  See the              %
%   GNU General Public License for more details.                               %
%                                                                              %
%   You should have received a copy of the GNU General Public License along    %
%   with Mathematics-and-Physics.  If not, see <https://www.gnu.org/licenses/>.%
%------------------------------------------------------------------------------%
\documentclass{article}
\usepackage{graphicx}                           % Needed for figures.
\usepackage{amsmath}                            % Needed for align.
\usepackage{amssymb}                            % Needed for mathbb.
\usepackage{amsthm}                             % For the theorem environment.
\usepackage{imakeidx}
\usepackage{minted}
\usepackage{hyperref}
\hypersetup{
    colorlinks=true,
    linkcolor=blue,
    filecolor=magenta,
    urlcolor=Cerulean,
    citecolor=SkyBlue
}

%------------------------Theorem Styles-------------------------%
\theoremstyle{plain}
\newtheorem{theorem}{Theorem}[section]

% Define theorem style for default spacing and normal font.
\newtheoremstyle{normal}
    {\topsep}               % Amount of space above the theorem.
    {\topsep}               % Amount of space below the theorem.
    {}                      % Font used for body of theorem.
    {}                      % Measure of space to indent.
    {\bfseries}             % Font of the header of the theorem.
    {}                      % Punctuation between head and body.
    {.5em}                  % Space after theorem head.
    {}

% Define default environments.
\theoremstyle{normal}
\newtheorem{examplex}{Example}[section]
\newtheorem{definitionx}{Definition}[section]

\newenvironment{example}{%
    \pushQED{\qed}\renewcommand{\qedsymbol}{$\blacksquare$}\examplex%
}{%
    \popQED\endexamplex%
}

\newenvironment{definition}{%
    \pushQED{\qed}\renewcommand{\qedsymbol}{$\blacksquare$}\definitionx%
}{%
    \popQED\enddefinitionx%
}

\title{Planar Topology: Lecture 2 - Continua}
\author{Ryan Maguire}
\date{Spring 2023}

% No indent and no paragraph skip.
\setlength{\parindent}{0em}
\setlength{\parskip}{0em}
\makeindex[intoc]
\begin{document}
    \maketitle
    \tableofcontents
    \listoffigures
    \section{Motivation}
        Georg Cantor (C.E. 1845 - 1918)\index{Cantor, Georg}
        demonstrated that every connected topological
        manifold (Hausdorff, second countable, and locally Euclidean, think of
        surfaces in $\mathbb{R}^{3}$) of positive dimension (i.e. excluding
        the empty manifold and zero dimensional singleton sets) can be put in
        bijection with the unit interval $[0,\,1]$. The function that does this
        likely will not be continuous, and almost certainly will not be a
        homeomorphism, but it shows that the closed interval has the same
        \textit{size}, in some sense, as any other non-trivial
        topological manifold.
        Giuseppe Peano (C.E. 1858 - 1932)\index{Peano, Giuseppe}, motivated by
        this discovery, proved that the unit interval can be
        \textit{continuously} mapped in a surjective fashion onto the unit
        square $[0,\,1]^{2}$. The result is a
        \textit{space-filling curve}\index{Space-Filling Curve}, a curve
        that fills a higher dimensional region. Because of Peano's works,
        space-filling curves that fill a two-dimensional region are called
        \textit{Peano Curves}\index{Peano Curve}.
        \par\hfill\par
        These results naturally lead to the following question.
        \begin{equation}
            \textrm{Which spaces are the continuous image of the unit interval?}
        \end{equation}
        The answer is surprising. \textit{Most} reasonable spaces
        are just the continuous image of $[0,\,1]$. This is the
        \textit{Hahn-Mazurkiewicz Theorem}.\index{Theorem!Hahn-Mazurkiewicz}
        \begin{theorem}[Hahn-Mazurkiewicz]
            If $(X,\,\tau)$ is a Hausdorff topological space, then there is a
            continuous surjective function $f:[0,\,1]\rightarrow{X}$ if and
            only if $(X,\,\tau)$ is compact, connected, locally connected, and
            second countable.\index{Theorem!Hahn-Mazurkiewicz}
        \end{theorem}
        One direction is nearly trivial. Since $[0,\,1]$ is compact and
        connected, if $(X,\,\tau)$ is the continuous image of $[0,\,1]$ it
        inherits these properties. Second countable and locally connected are
        a little harder, but not too difficult. The
        reverse direction is so unbelievable one may question if it is true.
        We will spend a fair amount of time working to prove this result and
        many of its corollaries, including that Hausdorff path connected spaces
        are arc connected.
    \section{Peano's Curve}
        \begin{figure}
            \centering
            \includegraphics{../../../images/peano_curve_001.pdf}
            \caption{Three Iterations of the Peano Curve}
            \label{fig:peano_curve_001}
        \end{figure}
        The Peano curve is described as the limit for a sequence of continuous
        function $P_{n}:[0,\,1]\rightarrow[0,\,1]^{2}$. The curves
        $P_{n}$ are actually piecewise linear (polygonal). We start with the
        square $S_{0}=[0,\,1]^{2}$ and the center of this square $C_{0}$.
        If $S_{n}$ has been given, for each square $s$ in $S_{n}$ divide
        $s$ into nine other equally sized squares. Replace the center $c$ of $s$
        with a sequence of the centers of the nine newly created squares.
        This is obtained by grouping the smaller squares into three columns
        and ordering the centers contiguously within each column (that is,
        no jumping over squares in this ordering). We then
        order the columns from one side of the square so that the
        distance between consecutive pairs of points is given by the side
        length of the smaller squares. In total there are four ways to do this.
        \begin{enumerate}
            \item Order everything bottom-to-top. That is, the left three
                centers are bottom-to-top, as are the middle centers, and the
                right centers.
            \item Zig-zag starting with bottom-to-top. That is, the left three
                centers are bottom-to-top, the middle three centers are
                top-to-bottom, and the right centers are again bottom-to-top.
            \item Zig-zag starting top-to-bottom. This is the mirror of the
                previous one, the left three centers are ordered top-to-bottom,
                the middle three are bottom-to-top, and the right centers are
                top-to-bottom.
            \item Order everything top-to-bottom. This is the mirror of the
                first idea. The left three centers go top-to-bottom, as do the
                middle and right columns.
        \end{enumerate}
        The ordering chosen for the next iteration $S_{n+1}$ is
        taken to be the one such that the distance between the first point of
        the ordering and the previous iteration in $P_{n}$ equals the side
        length of the small squares. We create $P_{n+1}$ be connecting the
        dots in the order given for all of the squares. The number of line
        segments in $P_{n}$ is hence given by $9^{n}$, which grows very fast.
        The Peano curve is the limit $P=\lim_{n\rightarrow\infty}P_{n}$.
        \par\hfill\par
        The classical description of the Peano curve, given above, is often
        easier to grasp using recursion. This is done efficiently using the
        standard Python module \texttt{turtle}, which helps one draw simple
        shapes and patterns. This is done below.
        \begin{minted}[gobble=10]{python}
            import turtle

            def peano(level, length):
                if level == 0:
                    turtle.forward(length)

                else:
                    angle = 90
                    peano(level-1, length/3)
                    turtle.right(angle)
                    peano(level-1, length/3)

                    for _ in range(2):
                        for _ in range(3):
                            turtle.left(angle)
                            peano(level-1, length/3)

                        angle = -angle

                    turtle.left(angle)
                    peano(level-1, length/3)
        \end{minted}
        I adapted this to the asymptote vector graphics language to create
        Fig.~\ref{fig:peano_curve_001}. The rainbow gradient is given so one
        can understand how the curve progresses. The starting point is purple
        and the end of the curve is red. Note the function is \textit{not}
        injective, and it can't be. Since $[0,\,1]$ is compact, and
        since $[0,\,1]^{2}$ is Hausdorff (it's a metric space), any
        continuous bijection $f:[0,\,1]\rightarrow[0,\,1]^{2}$ would
        automatically be a homeomorphism, but these are not homeomorphic
        spaces.
        \par\hfill\par
        Many modifications of Peano's construction have been made since, and
        perhaps most popular is the one presented by
        David Hilbert (C.E. 1862 - 1943) in 1891. The
        \textit{Hilbert curve}\index{Hilbert Curve}\index{Hilbert, David}
        is similar to the Peano curve, but the $n^{th}$ iteration is injective
        for all $n\in\mathbb{N}$ (the limit is still not surjective). Again,
        it is perhaps easiest to describe using recursion.
        \begin{minted}[gobble=10]{python}
            import turtle

            def hilbert(level, angle, length):
                if level == 0:
                    return

                else:
                    turtle.right(angle)
                    hilbert(level-1, -angle, length)
             
                    turtle.forward(length)
                    turtle.left(angle)
                    hilbert(level-1, angle, length)
             
                    turtle.forward(length)
                    hilbert(level-1, angle, length)

                    turtle.left(angle)
                    turtle.forward(length)
                    hilbert(level-1, -angle, length)

                    turtle.right(angle)
        \end{minted}
    \printindex
\end{document}
