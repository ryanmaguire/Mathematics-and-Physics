%-----------------------------------LICENSE------------------------------------%
%   This file is part of Mathematics-and-Physics.                              %
%                                                                              %
%   Mathematics-and-Physics is free software: you can redistribute it and/or   %
%   modify it under the terms of the GNU General Public License as             %
%   published by the Free Software Foundation, either version 3 of the         %
%   License, or (at your option) any later version.                            %
%                                                                              %
%   Mathematics-and-Physics is distributed in the hope that it will be useful, %
%   but WITHOUT ANY WARRANTY; without even the implied warranty of             %
%   MERCHANTABILITY or FITNESS FOR A PARTICULAR PURPOSE.  See the              %
%   GNU General Public License for more details.                               %
%                                                                              %
%   You should have received a copy of the GNU General Public License along    %
%   with Mathematics-and-Physics.  If not, see <https://www.gnu.org/licenses/>.%
%------------------------------------------------------------------------------%
\documentclass{article}
\usepackage{xcolor}

% No indent and no paragraph skip.
\setlength{\parindent}{0em}
\setlength{\parskip}{0em}

\begin{document}
    \LARGE
    \textbf{Planar Topology}
    \hrule\par\hfill\par
    \normalsize
    \textbf{Math 87 (Spring 2023)}\\
    Ryan Maguire\\
    \color{gray}{Ryan.J.Maguire.GR@dartmouth.edu}
    \par\vspace{0.5cm}
    \color{black}
    Class Time: Th 1:05 - 3:15, Kemeny 121\\
    Office Hours: MW 3:30 - 5:00, Kemeny 241\\
    \par\vspace{0.5cm}
    \textbf{Course Description}
    \par\hfill\par
    Point-set topology covers things like topological and metric spaces,
    discussing topics like separation and countability properties, and perhaps
    manifolds if time permits. Algebraic topology deals with homology,
    cohomology, and homotopy theory. Differential topology extends the
    topological ideas of manifolds to a new setting in which one can conduct
    calculus and study things like curvature. There are several topics in
    between that have been somewhat forgotten but are used daily by working
    mathematicians, even if they do not know it.
    \par\hfill\par
    In this course we’ll talk about continua theory and planar topology, and to
    some extent the general study of topological manifolds. Many of the great
    theorems of topology such as the Jordan curve theorem, Schoenflies theorem,
    and Brouwer fixed-point theorems have classical proofs that explore
    different ideas than the homological proofs one finds in an algebraic
    topology textbook. We’ll discuss these ideas and go through these proofs.
    \par\hfill\par
    \textbf{Course Objectives}
    \par\hfill\par
    By the end of the course one should have an understanding of continua
    theory, planar topology, topological manifolds, and homotopy groups. If
    there’s time we may dive into some homological ideas.
    \par\hfill\par
    \textbf{Prerequesites}
    \par\hfill\par
    The only prerequisite is a course in point-set topology (Math 54). We’ll
    briefly review these ideas before the main subject, but still one should
    have seen the material in-depth before hand.
    \par\hfill\par
    \textbf{Textbook}
    \par\hfill\par
    There is \textbf{no required textbook}. I will be handing out notes
    that will cover everything needed in the class.
    For those who become really fascinated with topology, the following are
    \textit{recommended}, but not required.
    \begin{itemize}
        \item \textit{General Topology} by Stephen Willard
        \item \textit{General Topology} by John L. Kelley
        \item \textit{Introduction to Topological Manifolds} by John Lee
    \end{itemize}
    \par\hfill\par
    \textbf{Grading}
    \par\hfill\par
    This is a pass/fail course that is project based. Students will meet with
    me to discuss progress and at the end of the term they will present to the
    class. Depending on the topic chosen, either a poster or slides will
    suffice.
    \par\hfill\par
    \textbf{Order of Topics}
    \par\hfill\par
    This being somewhat of a topics course, the material is likely to change.
    Here’s my current idea of the semester.
    \begin{enumerate}
        \item Review of Point-Set Topology
            \begin{itemize}
                \item The Hausdorff property.
                \item Continuity, homeomorphisms, homotopy, and
                    homotopy equivalence.
                \item Metric spaces and countability properties.
            \end{itemize}
        \item Topological Continuum
            \begin{itemize}
                \item Definitions.
                \item Examples (Lakes of Wada, Warsaw Circle, etc.)
                \item Basic Properties.
            \end{itemize}
        \item Planar Topology
            \begin{itemize}
                \item The Kuratowski Graph Theorem.
                \item Kuratowski's Indecomposibility Theorem.
                \item Jordan Curves and Jordan Arcs.
            \end{itemize}
        \item The Jordan Curve Theorem
            \begin{itemize}
                \item Proof of the Theorem
                \item The Schoenflies Theorem
                \item The Jordan Schoenflies Theorem
            \end{itemize}
        \item More Planar Topology
            \begin{itemize}
                \item The Denjoy-Riesz Theorem
                \item Osgood Curves, Lebesgue Measure, and Julia Sets.
                \item Janiszewski’s Theorem.
            \end{itemize}
        \item Partial Results in Higher Dimensions
            \begin{itemize}
                \item The Jordan-Bouer Theorem (Generalized Jordan Theorem).
                \item The Alexander Horned Sphere.
                \item Antoine's Necklace
                \item The Brouwer Fixed Point Theorem
            \end{itemize}
        \item Topological Manifolds and Topological Groups
            \begin{itemize}
                \item Basic definitions and properties.
                \item Examples.
                \item Applications of Topological Groups.
                \item Lie Groups and Smooth Manifolds.
            \end{itemize}
        \item Homotopy Groups
            \begin{itemize}
                \item The Fundamental Group.
                \item Homotopy Groups of Spheres.
                \item The Hopf Fibration.
            \end{itemize}
        \item Duality
            \begin{itemize}
                \item Alexander Duality
                \item Generalizations of the Jordan Curve Theorem.
            \end{itemize}
    \end{enumerate}
    \textbf{Disabilities}
    \par\hfill\par
    Students with disabilities who may need disability-related academic
    adjustments and services for this course are encouraged to see me privately
    as early in the term as possible. Students requiring disability-related
    academic adjustments and services must consult the Student Accessibility
    Services office (Carson Hall, Suite 125, 646-9900). Once SAS has authorized
    services, students must show the originally signed SAS Services and Consent
    Form and/or a letter on SAS letterhead to their professor. As a first step,
    if students have questions about whether they qualify to receive academic
    adjustments and services, they should contact the SAS office. All inquiries
    and discussions will remain confidential.
    \par\hfill\par
    \textbf{Stress and Mental Well-Being}
    \par\hfill\par
    The academic environment at Dartmouth is challenging, our terms are
    intensive, and classes are not the only demanding part of your life. There
    are a number of resources available to you on campus to support your
    wellness, including your undergraduate dean, Counseling and Human
    Development, and the Student Wellness Center.
    \par\hfill\par
    \textbf{Religious Observances}
    \par\hfill\par
    Some students may wish to take part in religious observances that occur
    during this academic term. If you have a religious observance that
    conflicts with your participation in the course, please meet with me before
    the end of the second week of the term to discuss appropriate
    accommodations.
\end{document}
