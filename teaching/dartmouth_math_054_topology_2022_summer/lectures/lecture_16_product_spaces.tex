%-----------------------------------LICENSE------------------------------------%
%   This file is part of Mathematics-and-Physics.                              %
%                                                                              %
%   Mathematics-and-Physics is free software: you can redistribute it and/or   %
%   modify it under the terms of the GNU General Public License as             %
%   published by the Free Software Foundation, either version 3 of the         %
%   License, or (at your option) any later version.                            %
%                                                                              %
%   Mathematics-and-Physics is distributed in the hope that it will be useful, %
%   but WITHOUT ANY WARRANTY; without even the implied warranty of             %
%   MERCHANTABILITY or FITNESS FOR A PARTICULAR PURPOSE.  See the              %
%   GNU General Public License for more details.                               %
%                                                                              %
%   You should have received a copy of the GNU General Public License along    %
%   with Mathematics-and-Physics.  If not, see <https://www.gnu.org/licenses/>.%
%------------------------------------------------------------------------------%
\documentclass{article}
\usepackage{graphicx}                           % Needed for figures.
\usepackage{amsmath}                            % Needed for align.
\usepackage{amssymb}                            % Needed for mathbb.
\usepackage{amsthm}                             % For the theorem environment.
\usepackage{float}
\usepackage[font=scriptsize,
            labelformat=simple,
            labelsep=colon]{subcaption} % Subfigure captions.
\usepackage[font={scriptsize},
            hypcap=true,
            labelsep=colon]{caption}    % Figure captions.
\usepackage{hyperref}
\hypersetup{
    colorlinks=true,
    linkcolor=blue,
    filecolor=magenta,
    urlcolor=Cerulean,
    citecolor=SkyBlue
}

%------------------------Theorem Styles-------------------------%
\theoremstyle{plain}
\newtheorem{theorem}{Theorem}[section]

% Define theorem style for default spacing and normal font.
\newtheoremstyle{normal}
    {\topsep}               % Amount of space above the theorem.
    {\topsep}               % Amount of space below the theorem.
    {}                      % Font used for body of theorem.
    {}                      % Measure of space to indent.
    {\bfseries}             % Font of the header of the theorem.
    {}                      % Punctuation between head and body.
    {.5em}                  % Space after theorem head.
    {}

% Define default environments.
\theoremstyle{normal}
\newtheorem{examplex}{Example}[section]
\newtheorem{definitionx}{Definition}[section]

\newenvironment{example}{%
    \pushQED{\qed}\renewcommand{\qedsymbol}{$\blacksquare$}\examplex%
}{%
    \popQED\endexamplex%
}

\newenvironment{definition}{%
    \pushQED{\qed}\renewcommand{\qedsymbol}{$\blacksquare$}\definitionx%
}{%
    \popQED\enddefinitionx%
}

\title{Point-Set Topology: Lecture 16}
\author{Ryan Maguire}
\date{\today}

% No indent and no paragraph skip.
\setlength{\parindent}{0em}
\setlength{\parskip}{0em}

\begin{document}
    \maketitle
    \section{Finite Products}
        Continuing with our trend of building new topological spaces, thus far
        we have subspaces and quotients. If $(X,\,\tau_{X})$ and
        $(Y,\,\tau_{Y})$ are topological spaces, it is possible to put a
        topology on the Cartesian product $X\times{Y}$ in a way that respects
        the topologies $\tau_{X}$ and $\tau_{Y}$. It is somewhat natural to hope
        that the set $\tilde{\tau}_{X\times{Y}}$ defined by:
        \begin{equation}
            \tilde{\tau}_{X\times{Y}}=
            \{\,\mathcal{U}\times\mathcal{V}\subseteq{X}\times{Y}\;|\;
                \mathcal{U}\in\tau_{X}\textrm{ and }\mathcal{V}\in\tau_{Y}\,\}
        \end{equation}
        would be a topology on $X\times{Y}$, but it usually is not. Consider the
        real line $\mathbb{R}$ with the standard topology $\tau_{\mathbb{R}}$.
        This has a basis $\mathcal{B}$ consisting of open intervals of the
        form $(a,\,b)$ for all $a,b\in\mathbb{R}$. The product
        $\mathbb{R}\times\mathbb{R}$ should just be the Euclidean plane
        $\mathbb{R}^{2}$, but open sets of the form $(a,\,b)\times(c,\,d)$ are
        \textit{open rectangles}. The product of more general open subsets
        $\mathcal{U},\mathcal{V}\subseteq\mathbb{R}$ could not possibly form an
        open disk in the plane, even though we want an open disk to be, well,
        \textit{open}.
        \par\hfill\par
        The set $\tilde{\tau}_{X\times{Y}}$ is
        nearly a topology. The empty set is contained in it since
        $\emptyset=\emptyset\times\emptyset$. The entire Cartesian product is
        an element since $X\in\tau_{X}$ and $Y\in\tau_{Y}$, hence
        $X\times{Y}\in\tilde{\tau}_{X\times{Y}}$. It is also closed under
        intersections since:
        \begin{equation}
            \big(\mathcal{U}_{0}\times\mathcal{V}_{0}\big)\cap
            \big(\mathcal{U}_{1}\times\mathcal{V}_{1}\big)
            =\big(\mathcal{U}_{0}\cap\mathcal{U}_{1}\big)\times
            \big(\mathcal{V}_{0}\cap\mathcal{V}_{1}\big)
        \end{equation}
        and this is an element of $\tilde{\tau}_{X\times{Y}}$. What fails is
        the union property. Again, think of $\mathbb{R}\times\mathbb{R}$. The
        union of two rectangles does not need to be a rectangle.
        \begin{figure}
            \centering
            \begin{subfigure}[b]{0.49\textwidth}
                \centering
                \includegraphics{../../../images/open_rectangle_r2.pdf}
                \caption{An Open Rectangle in $\mathbb{R}^{2}$}
            \end{subfigure}
            \hfill
            \begin{subfigure}[b]{0.49\textwidth}
                \centering
                \includegraphics{../../../images/open_not_rectangle_r2.pdf}
                \caption{The Union of Rectangles in $\mathbb{R}^{2}$}
                \label{fig:open_not_rectangle_r2}
            \end{subfigure}
            \caption{Open Subsets of the Plane}
            \label{fig:open_subsets_of_plane}
        \end{figure}
        Moreover, open
        subsets of $\mathbb{R}^{2}$ such as the open unit disk can not be
        written in the form $\mathcal{U}\times\mathcal{V}$ for open subsets
        $\mathcal{U},\mathcal{V}\subseteq\mathbb{R}$. To ensure the product
        topology is indeed a topology, we need to take the topology
        \textit{generated} from $\tilde{\tau}_{X\times{Y}}$.
        \begin{definition}[\textbf{Product Topology of Two Topologies}]
            The product topology of two topologies $\tau_{X}$ and $\tau_{Y}$
            on sets $X$ and $Y$, respectively, is the topology on $X\times{Y}$
            generated by the set $\mathcal{B}$ defined by:
            \begin{equation}
                \mathcal{B}=
                \{\,\mathcal{U}\times\mathcal{V}\subseteq{X}\times{Y}\;|\;
                    \mathcal{U}\in\tau_{X}\textrm{ and }
                    \mathcal{V}\in\tau_{Y}\,\}
            \end{equation}
            This is denoted $\tau_{X\times{Y}}$.
        \end{definition}
        \begin{theorem}
            If $(X,\,\tau_{X})$ and $(Y,\,\tau_{Y})$ are topological spaces, and
            if $\tau_{X\times{Y}}$ is the product topology of $\tau_{X}$ and
            $\tau_{Y}$, then $(X\times{Y},\,\tau_{X\times{Y}})$ is a topological
            space.
        \end{theorem}
        \begin{proof}
            The product topology $\tau_{X\times{Y}}$ is a generated topology,
            by definition, which is hence a topology, so
            $(X\times{Y},\,\tau_{X\times{Y}})$ is a topological space.
        \end{proof}
        The product of two topological spaces better give us the right
        topologies on familiar spaces, otherwise its useless.
        \begin{theorem}
            If $\tau_{\mathbb{R}\times\mathbb{R}}$ is the product topology of
            $\tau_{\mathbb{R}}$ and $\tau_{\mathbb{R}}$, and if
            $\tau_{\mathbb{R}^{2}}$ is the standard Euclidean topology on
            $\mathbb{R}^{2}$, then
            $(\mathbb{R}^{2},\,\tau_{\mathbb{R}^{2}})$ and
            $(\mathbb{R}\times\mathbb{R},\,\tau_{\mathbb{R}\times\mathbb{R}})$
            are homeomorphic.
        \end{theorem}
        \begin{proof}
            We simply must prove the topologies $\tau_{\mathbb{R}^{2}}$ and
            $\tau_{\mathbb{R}\times\mathbb{R}}$ are the same set, meaning the
            identity function
            $\textrm{id}:\mathbb{R}^{2}\rightarrow\mathbb{R}\times\mathbb{R}$
            is a homeomorphism. The standard topology on $\mathbb{R}^{2}$
            is generated by the Euclidean metric:
            \begin{equation}
                d(\mathbf{x},\,\mathbf{y})=||\mathbf{x}-\mathbf{y}||_{2}
            \end{equation}
            The topology of $\mathbb{R}\times\mathbb{R}$ is generated by open
            rectangles, which in turn can be generated by open squares, and
            open squares are the open balls in the max metric:
            \begin{equation}
                d_{\textrm{max}}(\mathbf{x},\,\mathbf{y})
                =\textrm{max}(|x_{0}-y_{0}|,\,|x_{1}-y_{1}|)
            \end{equation}
            But the Euclidean metric and the max metric are topologically
            equivalent metrics, meaning they produce the same topologies.
            So, we're done.
        \end{proof}
        \begin{definition}[\textbf{Projection Maps}]
            The projection map of the Cartesian product $X\times{Y}$ onto
            $X$ is the function $\textrm{proj}_{X}:X\times{Y}\rightarrow{X}$
            defined by $\textrm{proj}_{X}\big((x,\,y)\big)=x$. The projection
            map of $X\times{Y}$ onto $Y$ is defined by
            $\textrm{proj}_{Y}:X\times{Y}\rightarrow{Y}$,
            $\textrm{proj}_{Y}\big((x,\,y)\big)=y$.
        \end{definition}
        Projections are continuous.
        \begin{theorem}
            If $(X,\,\tau_{X})$ and $(Y,\,\tau_{Y})$ are topological spaces,
            and if $(X\times{Y},\,\tau_{X\times{Y}})$ is the product space,
            then $\textrm{proj}_{X}:X\times{Y}\rightarrow{X}$ and
            $\textrm{proj}_{Y}:X\times{Y}\rightarrow{Y}$ are continuous.
        \end{theorem}
        \begin{proof}
            Let $\mathcal{U}\in\tau_{X}$ and $\mathcal{V}\in\tau_{Y}$. Then by
            the definition of the projection map,
            $\textrm{proj}_{X}^{-1}[\mathcal{U}]=\mathcal{U}\times{Y}$, and
            $\mathcal{U}\times{Y}\in\tau_{X\times{Y}}$, so
            $\textrm{proj}_{X}$ is continuous. Similarly,
            $\textrm{proj}_{Y}^{-1}[\mathcal{V}]=X\times\mathcal{V}$, and
            $X\times\mathcal{V}\in\tau_{X\times{Y}}$, so
            $\textrm{proj}_{Y}$ is continuous.
        \end{proof}
        \begin{theorem}
            If $(X,\,\tau_{X})$ and $(Y,\,\tau_{Y})$ are topological spaces,
            and if $(X\times{Y},\,\tau_{X\times{Y}})$ is the product space,
            then $\textrm{proj}_{X}:X\times{Y}\rightarrow{X}$ and
            $\textrm{proj}_{X}:X\times{Y}\rightarrow{Y}$ are open maps.
        \end{theorem}
        \begin{proof}
            Let $\mathcal{W}\in\tau_{X\times{Y}}$. Since $\tau_{X\times{Y}}$
            is generated by the basis $\mathcal{B}$ defined by:
            \begin{equation}
                \mathcal{B}=\{\,
                    \mathcal{U}\times\mathcal{V}\;|\;
                    \mathcal{U}\in\tau_{X}\textrm{ and }
                    \mathcal{V}\in\tau_{Y}\,\}
            \end{equation}
            there is some collection $\mathcal{O}\subseteq\mathcal{B}$ such
            that $\mathcal{W}=\bigcup\mathcal{O}$. But for each
            $\mathcal{U}\times\mathcal{V}\in\mathcal{O}$ we have
            $\textrm{proj}_{X}[\mathcal{U}\times\mathcal{V}]=\mathcal{U}$ and
            $\textrm{proj}_{Y}[\mathcal{U}\times\mathcal{V}]=\mathcal{V}$.
            So then $\textrm{proj}_{X}[\mathcal{W}]$ is the union of open sets
            in $X$, and $\textrm{proj}_{Y}[\mathcal{W}]$ is the union of open
            sets in $Y$, so $\textrm{proj}_{X}[\mathcal{W}]\in\tau_{X}$ and
            $\textrm{proj}_{Y}[\mathcal{W}]\in\tau_{Y}$. That is,
            $\textrm{proj}_{X}$ and $\textrm{proj}_{Y}$ are open maps.
        \end{proof}
        Unlike quotients, which preserve very few properties, products preserve
        quite a lot of properties.
        \begin{theorem}
            If $(X,\,\tau_{X})$ and $(Y,\,\tau_{Y})$ are
            Fr\'{e}chet topological spaces, then
            $(X\times{Y},\,\tau_{X\times{Y}})$ is a Fr\'{e}chet topological
            space.
        \end{theorem}
        \begin{proof}
            Let $(x_{0},\,y_{0}),(x_{1},\,y_{1})\in{X}\times{Y}$ with
            $(x_{0},\,y_{0})\ne(x_{1},\,y_{1})$. Then either
            $x_{0}\ne{x}_{1}$ or $y_{0}\ne{y}_{1}$. Suppose $x_{0}\ne{x}_{1}$.
            The proof is symmetric if $y_{0}\ne{y}_{1}$. Since
            $(X,\,\tau_{X})$ is a Fr\'{e}chet topological space, and
            $x_{0}\ne{x}_{1}$, there exists $\mathcal{U},\mathcal{V}\in\tau_{X}$
            such that $x_{0}\in\mathcal{U}$, $x_{0}\notin\mathcal{V}$,
            $x_{1}\in\mathcal{V}$, and $x_{1}\notin\mathcal{U}$. But then
            $\mathcal{U}\times{Y}$ and $\mathcal{V}\times{Y}$
            are open sets such that
            $(x_{0},\,y_{0})\in\mathcal{U}\times{Y}$,
            $(x_{0},\,y_{0})\notin\mathcal{V}\times{Y}$,
            $(x_{1},\,y_{1})\in\mathcal{V}\times{Y}$, and
            $(x_{1},\,y_{1})\notin\mathcal{U}\times{Y}$. Hence,
            $(X\times{Y},\,\tau_{X\times{Y}})$ is a Fr\'{e}chet topological
            space.
        \end{proof}
        \begin{theorem}
            If $(X,\,\tau_{X})$ and $(Y,\,\tau_{Y})$ are Hausdorff topological
            spaces, then $(X\times{Y},\,\tau_{X\times{Y}})$ is a Hausdorff
            topological space.
        \end{theorem}
        \begin{proof}
            Let $(x_{0},\,y_{0}),(x_{1},\,y_{1})\in{X}\times{Y}$ with
            $(x_{0},\,y_{0})\ne(x_{1},\,y_{1})$. Then either
            $x_{0}\ne{x}_{1}$ or $y_{0}\ne{y}_{1}$. Suppose
            $x_{0}\ne{x}_{1}$, the proof is symmetric if
            $y_{0}\ne{y}_{1}$. But $(X,\,\tau_{X})$ is Hausdorff, so there are
            opens sets $\mathcal{U},\mathcal{V}\in\tau_{X}$ such that
            $x_{0}\in\mathcal{U}$, $x_{1}\in\mathcal{V}$, and
            $\mathcal{U}\cap\mathcal{V}=\emptyset$. But then
            $\mathcal{U}\times{Y}$ and $\mathcal{V}\times{Y}$ are disjoint open
            sets such that $(x_{0},\,y_{0})\in\mathcal{U}\times{Y}$ and
            $(x_{1},\,y_{1})\in\mathcal{V}\times{Y}$. Therefore
            $(X\times{Y},\,\tau_{X\times{Y}})$ is a Hausdorff topological space.
        \end{proof}
        \begin{theorem}
            If $(X,\,\tau_{X})$ and $(Y,\,\tau_{Y})$ are first-countable
            topological spaces, then $(X\times{Y},\,\tau_{X\times{Y}})$ is
            first-countable.
        \end{theorem}
        \begin{proof}
            Let $(x,\,y)\in{X}\times{Y}$. Since $x\in{X}$ and $(X,\,\tau_{X})$
            is first-countable, there is a countable neighborhood basis
            $\mathcal{B}_{X}\subseteq\tau_{X}$ of $x$. Since $y\in{Y}$ and
            $(Y,\,\tau_{Y})$ is first-countable, there is a countable
            neighborhood basis $\mathcal{B}_{Y}\subseteq\tau_{Y}$ of $y$.
            Let $\mathcal{B}\subseteq\tau_{X\times{Y}}$ be defined as:
            \begin{equation}
                \mathcal{B}=
                \{\,\mathcal{U}\times\mathcal{V}\in\tau_{X\times{Y}}\;|\;
                \mathcal{U}\in\mathcal{B}_{X}\textrm{ and }
                \mathcal{V}\in\mathcal{B}_{Y}\,\}
            \end{equation}
            Since $\mathcal{B}_{X}$ and $\mathcal{B}_{Y}$ are countable, so
            is $\mathcal{B}$. Since $x\in\mathcal{U}$ for all
            $\mathcal{U}\in\mathcal{B}_{X}$ and $y\in\mathcal{V}$ for all
            $\mathcal{V}\in\mathcal{B}_{Y}$, we have that
            $(x,\,y)\in\mathcal{U}\times\mathcal{V}$ for all
            $\mathcal{U}\times\mathcal{V}\in\mathcal{B}$. We now need to show
            that $\mathcal{B}$ is a neighborhood basis of $(x,\,y)$. Let
            $\mathcal{W}\in\tau_{X\times{Y}}$ be a set containing $(x,\,y)$.
            Then, by the definition of the product topology, there is a subset
            $\mathcal{O}\subseteq\tau_{X\times{Y}}$ such that
            $\mathcal{W}=\bigcup\mathcal{O}$ and all elements of
            $\mathcal{O}$ are of the form
            $\mathcal{U}\times\mathcal{V}$ with $\mathcal{U}\in\tau_{X}$ and
            $\mathcal{V}\in\tau_{Y}$. But $(x,\,y)\in\mathcal{W}$, so there
            is an element $\mathcal{U}\times\mathcal{V}\in\mathcal{O}$
            with $(x,\,y)\in\mathcal{U}\times\mathcal{V}$. Then
            $x\in\mathcal{U}$ and $y\in\mathcal{V}$. But
            $\mathcal{B}_{X}$ is a neighborhood basis of $x$, so there is a
            $\tilde{\mathcal{U}}\in\mathcal{B}_{X}$ such that
            $\tilde{\mathcal{U}}\subseteq\mathcal{U}$. But $\mathcal{B}_{Y}$
            is a neighborhood basis of $y$ so there is a
            $\tilde{\mathcal{V}}\in\mathcal{B}_{Y}$ such that
            $\tilde{\mathcal{V}}\subseteq\mathcal{V}$. But then
            $\tilde{\mathcal{U}}\times\tilde{\mathcal{V}}$ is an element of
            $\mathcal{B}$ such that
            $\tilde{\mathcal{U}}\times\tilde{\mathcal{V}}\subseteq\mathcal{U}\times\mathcal{V}$,
            and since $\mathcal{U}\times\mathcal{V}\subseteq\mathcal{W}$, we
            have $\tilde{\mathcal{U}}\times\tilde{\mathcal{V}}\subseteq\mathcal{W}$.
            Hence, $\mathcal{B}$ is a countable neighborhood basis of
            $(x,\,y)$ and $(X\times{Y},\,\tau_{X\times{Y}})$ is
            first-countable.
        \end{proof}
        \begin{theorem}
            If $(X,\,\tau_{X})$ and $(Y,\,\tau_{Y})$ are second-countable
            topological spaces, then $(X\times{Y},\,\tau_{X\times{Y}})$ is
            second-countable.
        \end{theorem}
        \begin{proof}
            Since $(X,\,\tau_{X})$ is second-countable, there is a countable
            basis $\mathcal{B}_{X}$ for $\tau_{X}$. Since $(Y,\,\tau_{Y})$ is
            second-countable, there is a countable basis $\mathcal{B}_{Y}$ for
            $\tau_{Y}$. Let $\mathcal{B}$ be defined by:
            \begin{equation}
                \mathcal{B}=
                    \{\,\mathcal{U}\times\mathcal{V}\;|\;
                    \mathcal{U}\in\mathcal{B}_{X}\textrm{ and }
                    \mathcal{V}\in\mathcal{B}_{Y}\,\}
            \end{equation}
            Then $\mathcal{B}$ has a cardinality bounded by
            $\mathbb{N}\times\mathbb{N}$, which is countable. We now need to
            prove $\mathcal{B}$ is a countable basis of $\tau_{X\times{Y}}$.
            To do this it suffices to show that
            any open set $\mathcal{W}\in\tau_{X\times{Y}}$ can be written
            as $\mathcal{W}=\bigcup\mathcal{O}$ for some set
            $\mathcal{O}\subseteq\mathcal{B}$. Define $\mathcal{O}$ via:
            \begin{equation}
                \mathcal{O}=
                \{\,\mathcal{U}\times\mathcal{V}\in\mathcal{B}\;|\;
                    \mathcal{U}\times\mathcal{V}\subseteq\mathcal{W}\,\}
            \end{equation}
            Then by definition of $\mathcal{O}$ we have that
            $\bigcup\mathcal{O}\subseteq\mathcal{W}$. Let's reverse this.
            Let $(x,\,y)\in\mathcal{W}$. Then, since projection maps are open
            maps, $x\in\textrm{proj}_{X}(\mathcal{W})$, which is an open set.
            Since $\mathcal{B}_{X}$ is a basis there is some
            $\mathcal{U}\in\mathcal{B}_{X}$ such that $x\in\mathcal{U}$ and
            $\mathcal{U}\subseteq\textrm{proj}_{X}(\mathcal{W})$. Similarly
            there is some $\mathcal{V}\in\mathcal{B}_{Y}$ such that
            $y\in\mathcal{V}$ and
            $\mathcal{V}\subseteq\textrm{proj}_{Y}(\mathcal{W})$. But then
            $(x,\,y)\in\mathcal{U}\times\mathcal{V}$ and
            $\mathcal{U}\times\mathcal{V}\subseteq\mathcal{W}$. Morever, since
            $\mathcal{U}\in\mathcal{B}_{X}$ and $\mathcal{V}\in\mathcal{B}_{Y}$,
            we have that $\mathcal{U}\times\mathcal{V}\in\mathcal{B}$. But
            then $\mathcal{U}\times\mathcal{V}\in\mathcal{O}$, and hence
            $(x,\,y)\in\bigcup\mathcal{O}$. That is,
            $\mathcal{W}\subseteq\bigcup\mathcal{O}$, and therefore
            $\mathcal{W}=\bigcup\mathcal{O}$. So $\mathcal{B}$ is a countable
            basis.
        \end{proof}
        The way to often think of product spaces
        $(X\times{Y},\,\tau_{X\times{Y}})$ is to take a copy of
        $X$ and attach it to each $y\in{Y}$. Similarly, you could think of
        attaching a copy of $Y$ to each $x\in{X}$. This mode of thinking makes
        it easier to visualize certain product spaces.
        \begin{example}
            The plane $\mathbb{R}^{2}$ can be thought of as attaching a
            copy of $\mathbb{R}$ to each real number $x\in\mathbb{R}$.
            That is, think of $\mathbb{R}$ as a horizontal line, and at
            each $x\in\mathbb{R}$ attach a copy of $\mathbb{R}$ that is
            directed vertically. The result is
            $\mathbb{R}^{2}$.
        \end{example}
        \begin{example}
            Letting $\mathbb{S}^{1}$ denote the unit circle in the plane,
            $\mathbb{R}\times\mathbb{S}^{1}$ is a \textit{cylinder}. At every
            point $\mathbf{x}\in\mathbb{S}^{1}$, attach a copy of the real
            line that is directed upwards out of the plane. The result is a
            cylinder in $\mathbb{R}^{3}$.
        \end{example}
        \begin{example}
            The space $\mathbb{S}^{1}\times\mathbb{S}^{1}$ is the torus, and
            this is denoted $\mathbb{T}^{2}$. As a set it lives as a subset of
            $\mathbb{R}^{4}$ since $\mathbb{S}^{1}\subseteq\mathbb{R}^{2}$,
            and hence $\mathbb{S}^{1}\times\mathbb{S}^{1}\subseteq\mathbb{R}^{4}$.
            However, it is far easier to visualize this space as a subset of
            $\mathbb{R}^{3}$ using our intuition of \textit{attaching} spaces
            to points. At every point on the first circle $\mathbb{S}^{1}$
            we attach a copy of the second circle $\mathbb{S}^{1}$. The result
            is a \textit{circle of circles}, which is the torus, and this can
            be embedded into $\mathbb{R}^{3}$. This is done in
            Fig.~\ref{fig:torus_skeleton_product_space}.
        \end{example}
        \begin{figure}
            \centering
            \includegraphics{../../../images/torus_skeleton_product_space.pdf}
            \caption{The Torus $\mathbb{T}^{2}$}
            \label{fig:torus_skeleton_product_space}
        \end{figure}
        Products also preserve the property of being metrizable. We've seen
        this before when we constructed several equivalent metrics on
        $\mathbb{R}^{2}$ using the standard metric on $\mathbb{R}$.
        \begin{theorem}
            If $(X,\,\tau_{X})$ and $(Y,\,\tau_{Y})$ are metrizable topological
            spaces, then $(X\times{Y},\,\tau_{X\times{Y}})$ is metrizable.
        \end{theorem}
        \begin{proof}
            Since $(X,\,\tau_{X})$ is metrizable, there is a metric $d_{X}$
            on $X$ that induces $\tau_{X}$. Since $(Y,\,\tau_{Y})$ is
            metrizable, there is a metric $d_{Y}$ on $Y$ that induces
            $\tau_{Y}$. Let $d_{X\times{Y}}$ be defined by:
            \begin{equation}
                d_{X\times{Y}}\big((x_{0},\,y_{0}),\,(x_{1},\,y_{1})\big)
                =d_{X}(x_{0},\,x_{1})+d_{Y}(y_{0},\,y_{1})
            \end{equation}
            Then $d_{X\times{Y}}$ is a metric. It is positive-definite,
            symmetric, and satisfies the triangle inequality since
            $d_{X}$ and $d_{Y}$ do. It also induces $\tau_{X\times{Y}}$.
            A set is open with the metric $d_{X\times{Y}}$ if and only if
            it is the union of open balls with the
            $d_{X\times{Y}}$ metric. Since sets of the form
            $\mathcal{U}\times\mathcal{V}\in\tau_{X\times{Y}}$ with
            $\mathcal{U}\in\tau_{X}$ and $\mathcal{V}\in\tau_{Y}$ form a basis
            for $\tau_{X\times{Y}}$, it suffices to show that all
            $\mathcal{U}\times\mathcal{V}$ are the union of open balls.
            For all $(x,\,y)\in\mathcal{U}\times\mathcal{V}$, since
            $d_{X}$ induces $\tau_{X}$, there is an $r_{x}>0$ such that
            $B_{r_{x}}^{(X,\,d_{X})}(x)\subseteq\mathcal{U}$. Since
            $d_{Y}$ induces $\tau_{Y}$, there is an $r_{y}>0$ such that
            $B_{r_{y}}^{(Y,\,d_{Y})}(y)\subseteq\mathcal{V}$. Let
            $r_{(x,\,y)}=\textrm{min}(r_{x},\,r_{y})$. Then
            $B_{r_{(x,\,y)}}^{(X\times{Y},\,d_{X\times{Y}})}\big((x,\,y)\big)$
            is a subset of $\mathcal{U}\times\mathcal{V}$. For let
            $(a,\,b)\in{B}_{r_{(x,\,y)}}^{(X\times{Y},\,d_{X\times{Y}})}\big((x,\,y)\big)$.
            But then:
            \begin{equation}
                d_{X}(a,\,x)
                \leq{d}_{X}(a,\,x)+d_{Y}(b,\,y)
                =d_{X\times{Y}}\big((a,\,x),\,(b,\,y)\big)
                <r_{(x,\,y)}\leq{r}_{x}
            \end{equation}
            so $a\in{B}_{r_{x}}^{(X,\,d_{X})}(x)$, and hence
            $a\in\mathcal{U}$. Similarly:
            \begin{equation}
                d_{Y}(b,\,y)
                \leq{d}_{X}(a,\,x)+d_{Y}(b,\,y)
                =d_{X\times{Y}}\big((a,\,x),\,(b,\,y)\big)
                <r_{(x,\,y)}\leq{r}_{y}
            \end{equation}
            so $b\in{B}_{r_{y}}^{(Y,\,d_{Y})}(y)$, and hence
            $b\in\mathcal{V}$. So
            $(a,\,b)\in\mathcal{U}\times\mathcal{V}$. Let
            $\mathcal{W}_{(x,\,y)}$ be the $r_{(x,\,y)}$ ball in
            $(X\times{Y},\,d_{X\times{Y}})$ centered at $(x,\,y)$ for all
            $(x,\,y)\in\mathcal{U}\times\mathcal{V}$. Then, since
            $(x,\,y)\in\mathcal{W}_{(x,\,y)}$, we have:
            \begin{equation}
                \mathcal{U}\times\mathcal{V}
                \subseteq
                \bigcup_{(x,\,y)\in\mathcal{U}\times\mathcal{V}}
                    \mathcal{W}_{(x,\,y)}
            \end{equation}
            But also
            $\mathcal{W}_{(x,\,y)}\subseteq\mathcal{U}\times\mathcal{V}$
            for all $(x,\,y)\in\mathcal{U}\times\mathcal{V}$, so:
            \begin{equation}
                \bigcup_{(x,\,y)\in\mathcal{U}\times\mathcal{V}}
                    \mathcal{W}_{(x,\,y)}
                \subseteq
                \mathcal{U}\times\mathcal{V}
            \end{equation}
            Hence $\mathcal{U}\times\mathcal{V}$ is the union of open balls
            in the $d_{X\times{Y}}$ metric. That is, an open set in
            $\tau_{X\times{Y}}$ is the union of open balls, meaning
            $\tau_{X\times{Y}}\subseteq\tau_{d_{X\times{Y}}}$. We must reverse
            this. This is, we must show that the union of open balls is open
            with respect to $\tau_{X\times{Y}}$. To do this it suffices to show
            that open balls with the $d_{X\times{Y}}$ metric are open in
            the topology $\tau_{X\times{Y}}$, since then the union of open
            balls would be the union of open sets, which is therefore open.
            So let $(x,\,y)\in{X}\times{Y}$ and $r>0$. Let
            $(a,\,b)\in{B}_{r}^{(X\times{Y},\,d_{X\times{Y}})}\big((x,\,y)\big)$.
            Let $r_{(a,\,b)}=\frac{1}{2}\big(r-d_{X\times{Y}}\big((a,\,x),\,(b,\,y)\big)$.
            Let $\mathcal{U}_{a}=B_{r_{(a,\,b)}}^{(X,\,d_{X})}(x)$ and
            $\mathcal{V}_{b}=B_{r_{(a,\,b)}}^{(Y,\,d_{Y})}(y)$. Then
            $\mathcal{U}_{a}\times\mathcal{V}_{b}\subseteq{B}_{r}^{(X\times{Y},\,d_{X\times{Y}})}\big((x,\,y)\big)$.
            For if $(x_{0},\,y_{0})\in\mathcal{U}_{a}\times\mathcal{V}_{b}$,
            then:
            \begin{align}
                d_{X\times{Y}}&\big((x_{0},\,y_{0}),\,(x,\,y)\big)
                \nonumber\\
                &\leq{d}_{X\times{Y}}\big((x_{0},\,y_{0}),\,(a,\,b)\big)
                    +d_{X\times{Y}}\big((a,\,b),\,(x,\,y)\big)\\
                &={d}_{X}(x_{0},\,a)+d_{Y}(y_{0},\,b)+
                    d_{X\times{Y}}\big((a,\,b),\,(x,\,y)\big)\\
                &<\frac{1}{2}\Big(
                    r-d_{X\times{Y}}\big((a,\,b),\,(x,\,y)\big)
                \Big)+\frac{1}{2}\Big(
                    r-d_{X\times{Y}}\big((a,\,b),\,(x,\,y)\big)
                \Big)\nonumber\\
                &\quad+d_{X\times{Y}}\big((a,\,b),\,(x,\,y)\big)\\
                &=r
            \end{align}
            So $B_{r}^{(X\times{Y},\,\tau_{X\times{Y}})}\big((x,\,y)\big)$
            can be written as the union of all such
            $\mathcal{U}_{a}\times\mathcal{V}_{b}$ for all
            $(a,\,b)$ in the set, meaning open balls with the
            $d_{X\times{Y}}$ metric are open. So a set is open in
            $\tau_{X\times{Y}}$ if and only if it is open with respect to
            $d_{X\times{Y}}$, so $d_{X\times{Y}}$ induces the topology and
            $(X\times{Y},\,\tau_{X\times{Y}})$ is metrizable.
        \end{proof}
        Products can be performed for any finite collection of topological
        spaces. We replace $X\times{Y}$ with
        $\prod_{n\in\mathbb{Z}_{N}}X_{n}$, given a collection of
        $N\in\mathbb{N}$ topological spaces $(X_{n},\,\tau_{n})$. The
        topology is generated by sets of the form
        $\prod_{n\in\mathbb{Z}_{N}}\mathcal{U}_{n}$ where
        $\mathcal{U}_{n}\in\tau_{n}$ for all $n\in\mathbb{Z}_{N}$. All of the
        previous theorems still hold for finite products, and the proofs are
        done by induction. (Try it yourself, I can't prove everything for you!)
        \begin{itemize}
            \item The finite product of Fr\'{e}chet spaces is Fr\'{e}chet.
            \item The finite product of Hausdorff spaces is Hausdorff.
            \item The finite product of first-countable spaces is
                first-countable.
            \item The finite product of second-countable spaces is
                second-countable.
            \item The finite product of metrizable spaces is metrizable.
        \end{itemize}
        The product of sequential spaces does \textbf{not} need to be
        sequential.
    \section{Infinite Products}
        When we go from the finite world to the infinite things get a bit
        problematic. First, how do we even topologize an infinite product?
        There are two ways: the \textit{obvious} way, and the correct one.
        It took me a long time to realize that the obvious way is not the
        correct one. I've a few examples up my sleeves, so hopefully you'll
        realize sooner than I did. The obvious way is the
        \textit{box topology}. Given a set $I$ such that for all
        $\alpha\in{I}$ we have that $(X_{\alpha},\,\tau_{\alpha})$ is a
        topological space, we can form the following basis
        $\mathcal{B}_{\textrm{Box}}$ for the product:
        \begin{equation}
            \mathcal{B}_{\textrm{Box}}
            =\Big\{\,\prod_{\alpha\in{I}}\mathcal{U}_{\alpha}\;|\;
                \mathcal{U}_{\alpha}\in\tau_{\alpha}
                \textrm{ for all }\alpha\in{I}\,\Big\}
        \end{equation}
        This should definitely be considered the obvious way. We stole our
        idea for finite products and just generated a topology using this.
        This idea is horrible, unfortunately. The set
        $\mathbb{R}^{\infty}=\prod_{n=0}^{\infty}\mathbb{R}$ is the set of
        all sequences in $\mathbb{R}$. The function
        $f:\mathbb{R}\rightarrow\mathbb{R}^{\infty}$ defined by
        $f(x)=a:\mathbb{N}\rightarrow\mathbb{R}$ where $a_{n}=x$ for all
        $n\in\mathbb{N}$ certainly seems like a simple enough function.
        Intuitively, this is the function:
        \begin{equation}
            f(x)=(x,\,x,\,x,\,\dots,\,x,\,\dots)
        \end{equation}
        Note that in each component the function is indeed continuous.
        That is, $f_{n}:\mathbb{R}\rightarrow\mathbb{R}$ defined by
        $f_{n}(x)=f(x)_{n}$ is just $f_{n}(x)=x$, which is continuous.
        With respect to the box topology, $f$ is \textit{nowhere continuous}.
        Talk about aweful! This is one of the simplest functions one could
        describe from $\mathbb{R}$ to $\mathbb{R}^{\infty}$ and yet the
        box topology says it's everywhere discontinuous.
        \par\hfill\par
        If you were given a function $f:\mathbb{R}\rightarrow\mathbb{R}^{3}$
        from calculus like:
        \begin{equation}
            f(t)=\big(t^{2}+1,\,\sin(t)e^{t},\,t^{3}-t\big)
        \end{equation}
        would you bother checking that the pre-image of an open set is open
        to determine $f$ is continuous? Of course not, you'd note that in
        the $x$ coordinate we have $x(t)=t^{2}+1$, which is a polynomial, so
        it is continuous. In the $y$ coordinate you have
        $y(t)=\sin(t)e^{t}$, the product of continuous functions, so continuous.
        In the $z$ coordinate you have $z(t)=t^{3}-t$, another polynomial.
        Since $f$ is continuous in all of its components, you'd rightly
        conclude that $f$ is continuous. This is the way continuous functions
        should work with infinite products as well, but the box topology
        lacks such a feature. The problem is the box topology is way to big.
        We need to restrict which sets we consider open if we want a nice
        topology on the product. Let's try the following.
        Define $\mathcal{B}_{\textrm{Prod}}$ as:
        \begin{equation}
            \mathcal{B}_{\textrm{Prod}}
            =\Big\{\,\prod_{\alpha\in{I}}\mathcal{U}_{\alpha}\;|\;
                \mathcal{U}_{\alpha}\in\tau_{\alpha}\textrm{ and }
                \mathcal{U}_{\alpha}=X_{\alpha}
                \textrm{ for all but finitely many }
                \alpha\in{I}\,\Big\}
        \end{equation}
        Let $\tau_{\textrm{Box}}$ and $\tau_{\textrm{Prod}}$ be the
        topologies generated from $\mathcal{B}_{\textrm{Box}}$ and
        $\mathcal{B}_{\textrm{Prod}}$, respectively. Hopefully from the
        definition it is
        clear that $\tau_{\textrm{Prod}}\subseteq\tau_{\textrm{Box}}$. The
        product topology is formed in a similar manner to the box topology, but
        with a major restriction on which sets we use to generate our
        topology.
        \par\hfill\par
        The product topology is precisely the topology that makes it so that
        a function $f:Y\rightarrow\prod_{\alpha\in{I}}X_{\alpha}$,
        with respect to a topological space $(Y,\,\tau_{Y})$, is continuous
        if and only if $f_{\alpha}:Y\rightarrow{X}_{\alpha}$, the component
        function, is continuous for all $\alpha\in{I}$. You will prove this
        in your homework.
        \par\hfill\par
        Note that the box topology and the product topology are the same for
        finite products. It's only in the infinite world where things differ.
        The product topology also has the following nice feature.
        \begin{theorem}
            If $\mathcal{X}$ is a countable set of metrizable spaces
            $(X_{n},\,\tau_{n})$, if $\tau_{\prod}$ is the product topology
            on $\prod_{n\in\mathbb{N}}X_{n}$, then
            $(\prod_{n\in\mathbb{N}}X_{n},\,\tau_{\prod})$ is metrizable.
        \end{theorem}
        \begin{proof}
            For each space $(X_{n},\,\tau_{n})$ there is a metric $d_{n}$ that
            induces the topology. These metrics may be unbounded, so define
            $\rho_{n}$ to be the topologically equivalence metric given by:
            \begin{equation}
                \rho_{n}(x,\,y)=
                \frac{d_{n}(x,\,y)}{1+d_{n}(x,\,y)}
            \end{equation}
            Define $d_{\prod}$ by:
            \begin{equation}
                d_{\prod}\big(a,\,b)
                =\sum_{n=0}^{\infty}\frac{\rho_{n}(a_{n},\,b_{n})}{2^{n}}
            \end{equation}
            (Remember, $a\in\prod_{n\in\mathbb{N}}X_{n}$ is a sequence
            $a:\mathbb{N}\rightarrow\bigcup_{n\in\mathbb{N}}X_{n}$ such that
            $a_{n}\in{X}_{n}$ for all $n\in\mathbb{N}$). This sum converges
            since each $\rho_{n}$ is bounded by 1, so we have a valid function
            on $\prod_{n\in\mathbb{N}}X_{n}$. It is also metric. It is
            positive-definite, symmetric, and satisfies the triangle inequality
            since all of the $\rho_{n}$ do. The product topology has a subbasis
            of open sets of the form:
            \begin{equation}
                \tilde{\mathcal{U}}=\prod_{n=0}^{\infty}\mathcal{U}_{n}
            \end{equation}
            where $\mathcal{U}_{n}\in\tau_{n}$ for all $n\in\mathbb{N}$, and
            $\mathcal{U}_{n}=X_{n}$ for all but \textbf{one} $n\in\mathbb{N}$
            (this is a subbasis, not a basis). So we need to just show that
            these subbasis elements are open with respect to
            $d_{\prod}$. Given $a\in\tilde{\mathcal{U}}$,
            $a_{n}\in\mathcal{U}_{n}\in\tau_{n}$, since
            $\rho_{n}$ induces $\tau_{n}$, there is an $r'>0$ such that
            $B_{r}^{(X_{n},\,\rho_{n})}(a_{n})\subseteq\mathcal{U}_{n}$.
            Let $r=r'/2^{n}$. But then
            $B_{r}^{(\prod_{n}X_{n},\,d_{\prod})}(a)\subseteq\tilde{\mathcal{U}}$
            since given $b\in{B}_{r}^{(\prod_{n}X_{n},\,d_{\prod})}(a)$,
            we have:
            \begin{equation}
                \frac{1}{2^{n}}\rho_{n}(a_{n},\,b_{n})
                \leq\sum_{k=0}^{\infty}\frac{\rho_{k}(a_{k},\,b_{k})}{2^{k}}
                <r=\frac{r'}{2^{n}}
            \end{equation}
            and hence $\rho_{n}(a_{n},\,b_{n})<r'$, so
            $b_{n}\in\mathcal{U}_{n}$, and therefore $b\in\tilde{\mathcal{U}}$.
            Next, to show open balls are open. Let
            $r>0$ and $a$ an element of the product set and choose
            $N\in\mathbb{N}$ such that $1/2^{N}<r/2$. Let
            $\mathcal{U}_{n}$ be the $r/4$ ball centered at $a_{n}$ for all
            $n\in\mathbb{Z}_{N}$, and $\mathcal{U}_{n}=X_{n}$ for all
            $n\geq{N}$. Then
            $\prod_{n\in\mathbb{N}}\mathcal{U}_{n}\in\tau_{\prod}$ by the
            definition of the product topology. But also
            $\prod_{n\in\mathbb{N}}\mathcal{U}_{n}\subseteq{B}_{r}^{(\prod_{n}X_{n},\,d_{\prod})}(a)$.
            For if $b\in\prod_{n\in\mathbb{N}}\mathcal{U}_{n}$, then:
            \begin{align}
                d_{\prod}(a,\,b)
                &=\sum_{n=0}^{\infty}\frac{\rho_{n}(a_{n},\,b_{n})}{2^{n}}\\
                &=\sum_{n=0}^{N-1}\frac{\rho_{n}(a_{n},\,b_{n})}{2^{n}}
                    +\sum_{n=N}^{\infty}\frac{\rho_{n}(a_{n},\,b_{n})}{2^{n}}\\
                &<\sum_{n=0}^{N-1}\frac{r}{4}\frac{1}{2^{n}}
                +\sum_{n=N}^{\infty}\frac{1}{2^{n}}\\
                &<\frac{r}{2}+\frac{r}{2}\\
                &=r
            \end{align}
            So we can find an open set containing $a$ that fits entirely
            inside of the $r$ ball centered at $a$. This can be modified for
            all elements of the $r$ ball centered at $a$, meaning this set can
            be written as the union of open sets, which is therefore open.
            So open balls with respect to $d_{\prod}$ are open in
            $\tau_{\prod}$, and open sets in $\tau_{\prod}$ are open with
            respect to $d_{\prod}$. Hence $d_{\prod}$ induces the topology
            and $(\prod_{n\in\mathbb{N}}X_{n},\,\tau_{\prod})$ is
            metrizable.
        \end{proof}
        The claim is not true for uncountable products. The product of
        uncountably many metrizable spaces need not be first-countable, and
        hence cannot possibly be metrizable.
    \section{Homotopy and Homotopy Equivalence}
        Homeomorphism is the main notion of \textit{sameness} for topological
        spaces. If $(X,\,\tau_{X})$ and $(Y,\,\tau_{Y})$ are homeomorphic,
        then topologically they are indistinguishable and may as well be
        regarded as the same topological space. There is another notion of
        \textit{same} that is far weaker, but also very intuitive and
        pictorial. This idea is described by \textit{homotopies}. Homotopy
        is motivated by curves in the plane. Suppose we have
        $f:[0,\,1]\rightarrow\mathbb{R}$ and $g:[0,\,1]\rightarrow\mathbb{R}$
        defined by $f(x)=\exp(x)$ and $g(x)=x^{3}$. We visualize these functions
        as curves $\alpha,\beta:[0,\,1]\rightarrow\mathbb{R}^{2}$ defined by
        $\alpha(t)=\big(t,\,f(t)\big)$ and $\beta(t)=\big(t,\,g(t)\big)$. A
        homotopy from the curve $\alpha$ to the curve $\beta$ is a way of
        \textit{continuously} deforming $\alpha$ into $\beta$. In the plane this
        can be done by dragging the point $\alpha(t)$ to the point $\beta(t)$
        along the straight line between them for all $t\in[0,\,1]$. This is
        shown in Fig.~\ref{fig:homotopy_straight_line_cubed_to_exp_2d}.
        \begin{figure}
            \centering
            \includegraphics{../../../images/homotopy_straight_line_cubed_to_exp_2d.pdf}
            \caption{Homotopy Between Curves}
            \label{fig:homotopy_straight_line_cubed_to_exp_2d}
        \end{figure}
        We use this to motivate homotopies in general. It should be a way of
        continuously deforming one function into another.
        \begin{definition}[\textbf{Homotopy}]
            A homotopy from a continuous function $f_{0}:X\rightarrow{Y}$ to
            a continuous function $f_{1}:X\rightarrow{Y}$ between topological
            spaces $(X,\,\tau_{X})$ and $(Y,\,\tau_{Y})$ is a continuous
            function $H:X\times[0,\,1]\rightarrow{Y}$, where $[0,\,1]$ has the
            subspace topology from $\mathbb{R}$ and $X\times[0,\,1]$ has the
            product topology, such that for all $x\in{X}$ we have
            $H(x,\,0)=f_{0}(x)$ and $H(x,\,1)=f_{1}(x)$.
        \end{definition}
        Some spaces, such as Euclidean spaces, are \textit{too nice} and have
        the property that all continuous functions are homotopic to one another.
        \begin{theorem}
            If $(X,\,\tau)$ is a topological space, if $\tau_{\mathbb{R}^{n}}$
            is the standard Euclidean topology on $\mathbb{R}^{n}$, and if
            $f_{0},f_{1}:X\rightarrow\mathbb{R}^{n}$ are continuous functions,
            then there is a homotopy $H:X\times[0,\,1]\rightarrow\mathbb{R}^{n}$
            between $f_{0}$ and $f_{1}$.
        \end{theorem}
        \begin{proof}
            Define $H:X\times[0,\,1]\rightarrow\mathbb{R}^{n}$ via:
            \begin{equation}
                H(x,\,t)=(1-t)f_{0}(x)+tf_{1}(x)
            \end{equation}
            Since multiplication and addition is continuous in $\mathbb{R}^{n}$,
            and since $f_{0}$ and $f_{1}$ are continuous, $H$ is continuous.
            Moreover, for all $x\in{X}$ we have $H(x,\,0)=f_{0}(x)$ and
            $H(x,\,1)=f_{1}(x)$, so $H$ is a homotopy between $f_{0}$ and
            $f_{1}$.
        \end{proof}
        Homotopic is an equivalence relation on the set of all continuous
        functions between $(X,\,\tau_{X})$ and $(Y,\,\tau_{Y})$.
        \begin{theorem}
            If $(X,\,\tau_{X})$ and $(Y,\,\tau_{Y})$ are topological spaces,
            and if $f:X\rightarrow{Y}$ is continuous, then $f$ is homotopic to
            itself.
        \end{theorem}
        \begin{proof}
            Let $H:X\times[0,\,1]\rightarrow{Y}$ be defined by
            $H(x,\,t)=f(x)$. Then, since $f$ is continuous, so is $H$. However
            $H(x,\,0)=f(x)$ and $H(x,\,1)=f(x)$, so $H$ is a homotopy from $f$
            to itself.
        \end{proof}
        \begin{theorem}
            If $(X,\,\tau_{X})$ and $(Y,\,\tau_{Y})$ are topological spaces,
            and if $f_{0},f_{1}:X\rightarrow{Y}$ are continuous functions such
            that $f_{0}$ is homotopic to $f_{1}$, then $f_{1}$ is homotopic to
            $f_{0}$.
        \end{theorem}
        \begin{proof}
            Since $f_{0}$ is homotopic to $f_{1}$ there is a homotopy
            $H:X\times[0,\,1]\rightarrow{Y}$ such that $H(x,\,0)=f_{0}(x)$ and
            $H(x,\,1)=f_{1}(x)$ for all $x\in{X}$. Define
            $G:X\times[0,\,1]\rightarrow{Y}$ via:
            \begin{equation}
                G(x,\,t)=H(x,\,1-t)
            \end{equation}
            Since $h:[0,\,1]\rightarrow[0,\,1]$ defined by $h(t)=1-t$ is
            continuous, and since $H$ is continuous, $G$ is continuous as well.
            But $G(x,\,0)=H(x,\,1)=f_{1}(x)$ and $G(x,\,1)=H(x,\,0)=f_{0}(x)$.
            So $G$ is a homotopy from $f_{1}$ to $f_{0}$.
        \end{proof}
        Transitivity requires the pasting lemma, a fundamental result about
        building continuous functions by gluing two functions together.
        \begin{theorem}[\textbf{The Pasting Lemma}]
            If $(X,\,\tau_{X})$ and $(Y,\,\tau_{Y})$ are topological spaces,
            if $A,B\subseteq{X}$ are closed subsets, if $X=A\cup{B}$, and if
            $f_{0}:A\rightarrow{Y}$ and $f_{1}:B\rightarrow{Y}$ are continuous
            functions with the subspace topologies on $A$ and $B$ such that
            for all $x\in{A}\cap{B}$ it is true that $f_{0}(x)=f_{1}(x)$, then
            the function $f:X\rightarrow{Y}$ defined by:
            \begin{equation}
                f(x)=
                \begin{cases}
                    f_{0}(x)&x\in{A}\\
                    f_{1}(x)&x\in{B}
                \end{cases}
            \end{equation}
            is continuous.
        \end{theorem}
        \begin{proof}
            First, $f$ is a function. It is well-defined since on
            $A\cap{B}$ the functions $f_{0}$ and $f_{1}$ agree. Second, for
            all $x\in{X}$ there is a $y\in{Y}$ such that $f(x)=y$ since
            $A\cup{B}=X$, so both $A$ and $B$ cover $X$. Now to show it is
            continuous. Let $\mathcal{D}\subseteq{Y}$ be closed. Then since
            $f_{0}$ is continuous, $f_{0}^{-1}[\mathcal{D}]$ is closed. But
            $f_{1}$ is also continuous, so $f_{1}^{-1}[\mathcal{D}]$ is
            closed. But then:
            \begin{equation}
                f^{-1}[\mathcal{D}]
                =f_{0}^{-1}[\mathcal{D}]\cup{f}_{1}^{-1}[\mathcal{D}]
            \end{equation}
            Hence $f^{-1}[\mathcal{D}]$ is the union of two closed sets,
            which is closed. Therefore, $f$ is continuous.
        \end{proof}
        \begin{theorem}
            If $(X,\,\tau_{X})$ and $(Y,\,\tau_{Y})$ are topological spaces,
            if $f_{0},f_{1},f_{2}:X\rightarrow{Y}$ are continuous, if
            $f_{0}$ is homotopic to $f_{1}$, and if $f_{1}$ is homotopic to
            $f_{2}$, then $f_{0}$ is homotopic to $f_{2}$.
        \end{theorem}
        \begin{proof}
            Since $f_{0}$ is homotopic to $f_{1}$, there is a homotopy
            $H:X\times[0,\,1]\rightarrow{Y}$ such that $H(x,\,0)=f_{0}(x)$ and
            $H(x,\,1)=f_{1}(x)$. Since $f_{1}$ and $f_{2}$ are homotopic, there
            is a homotopy $G:X\times[0,\,1]\rightarrow{Y}$ such that
            $G(x,\,0)=f_{1}(x)$ and $G(x,\,1)=f_{2}(x)$. Define
            $F:X\times[0,\,1]\rightarrow{Y}$ via:
            \begin{equation}
                F(x,\,t)=
                \begin{cases}
                    H(x,\,2t)&0\leq{t}\leq\frac{1}{2}\\
                    G(x,\,2t-1)&\frac{1}{2}\leq{t}\leq{1}
                \end{cases}
            \end{equation}
            This is well-defined since
            $F(t,\,\tfrac{1}{2})=H(x,\,1)=G(x,\,0)=f_{1}(x)$ for all
            $x\in{X}$. It is also continuous by the pasting lemma, since both
            $H$ and $G$ are continuous. But also
            $F(x,\,0)=H(x,\,0)=f_{0}(x)$ and $F(x,\,1)=G(x,\,1)=f_{2}(x)$, so
            $F$ is a homotopy between $f_{0}$ and $f_{2}$.
        \end{proof}
        \begin{figure}
            \centering
            \includegraphics{../../../images/homotopy_basic_example.pdf}
            \caption{Homotopy Between Continuous Functions}
            \label{fig:homotopy_basic_example}
        \end{figure}
        For the more general picture with $(X,\,\tau_{X})$ and $(Y,\,\tau_{Y})$
        being arbitrary topological spaces, we use
        Fig.~\ref{fig:homotopy_basic_example} for guiding intuition. In the
        case $(X,\,\tau_{X})=([0,\,1],\,\tau_{[0,\,1]})$, the closed unit
        interval with the subspace topology, we again think of curves in the
        space $(Y,\,\tau_{Y})$. See
        Fig.~\ref{fig:homotopy_on_unit_interval}
        \begin{figure}[H]
            \centering
            \includegraphics{../../../images/homotopy_on_unit_interval.pdf}
            \caption{Homotopy Between Curves}
            \label{fig:homotopy_on_unit_interval}
        \end{figure}
        Think of the circle $\mathbb{S}^{1}$ with the subspace topology from
        $\mathbb{R}^{2}$. Given two continuous functions
        $f_{0},f_{1}:\mathbb{S}^{1}\rightarrow\mathbb{S}^{1}$, do you think it
        must be true that $f_{0}$ and $f_{1}$ are homotopic, like was the case
        with $\mathbb{R}^{n}$? Let's alter the question slightly. Consider the
        functions $f_{0},f_{1}:[0,\,1]\rightarrow\mathbb{S}^{1}$ defined by
        $f_{0}(t)=\big(\cos(\pi{t}),\,\sin(\pi{t})\big)$ and
        $f_{1}(t)=\big(\cos(\pi{t}),\,-\sin(\pi(t))\big)$. These functions
        start and end at the same points on the circle. Can you deform
        $f_{0}$ into $f_{1}$ while keeping the endpoints fixed and
        staying inside the circle? If you could leave the circle, the problem
        would be easy, just do the straight line homotopy
        $H(s,\,t)=(1-t)f_{0}(s)+tf_{1}(s)$, but that is not the question. You
        may not change the endpoints and you can't leave the circle. Hopefully
        this seems impossible, and because this is impossible it is not true
        that all functions $f_{0},f_{1}:\mathbb{S}^{1}\rightarrow\mathbb{S}^{1}$
        are homotopic.
        \begin{figure}
            \centering
            \includegraphics{../../../images/two_curves_on_circle_001.pdf}
            \caption{Two Curves on $\mathbb{S}^{1}$}
            \label{fig:two_curves_on_circle_001}
        \end{figure}
        \par\hfill\par
        The feature Euclidean space has is that it is
        \textit{contractible}, it can be shrunk down continuously to a point.
        The circle has a large hole in it and cannot be collapsed to a point.
        To make this precise, now is the time to talk about
        \textit{homotopy equivalences}. First, one more definition.
        \begin{definition}[\textbf{Homotopy Inverse}]
            A homotopy inverse for a continuous function $f:X\rightarrow{Y}$
            from a topological space $(X,\,\tau_{X})$ to a topological space
            $(Y,\,\tau_{Y})$ is a continuous function $g:Y\rightarrow{X}$ such
            that $g\circ{f}:X\rightarrow{X}$ is homotopic to the identity
            function $\textrm{id}_{X}$ and $f\circ{g}:Y\rightarrow{Y}$ is
            homotopic to the identity function $\textrm{id}_{Y}$.
        \end{definition}
        \begin{definition}[\textbf{Homotopy Equivalence Topological Spaces}]
            Homotopy equivalent topological spaces are topological spaces
            $(X,\,\tau_{X})$ and $(Y,\,\tau_{Y})$ such that there is a
            continuous function $f:X\rightarrow{Y}$ that has a homotopy inverse
            $g:Y\rightarrow{X}$. $f$ and $g$ are called
            \textit{homotopy equivalences}.
        \end{definition}
        Homotopy equivalent is a new notion of \textit{sameness} for topological
        spaces, but it is far weaker than homeomorphic. It is also extremely
        visual and intuitive, once you get the idea. In homeomorphisms you are
        allowed to continuously and bijectively move your space around. With
        homotopy equivalence you are allowed to do a lot more. You can squeeze
        points together, stretch points out, you just can't tear your space.
        Homeomorphisms are, in particular, homotopy equivalences.
        \begin{theorem}
            If $(X,\,\tau_{X})$ and $(Y,\,\tau_{Y})$ are homeomorphic
            topological spaces, then they are homotopy equivalent.
        \end{theorem}
        \begin{proof}
            Since $(X,\,\tau_{X})$ and $(Y,\,\tau_{Y})$ are homeomorphic, there
            is a homeomorphism $f:X\rightarrow{Y}$. But then $f$ is continuous,
            bijective, and $f^{-1}$ is continuous. But then $f^{-1}$ is a
            homotopy inverse of $f$ since it is continuous and
            $f\circ{f}^{-1}=\textrm{id}_{Y}$ and
            $f^{-1}\circ{f}=\textrm{id}_{X}$. But any continuous function is
            homotopic to itself, so if $f\circ{f}^{-1}=\textrm{id}_{Y}$, then
            $f\circ{f}^{-1}$ is homotopic to $\textrm{id}_{Y}$. Similarly
            $f^{-1}\circ{f}$ is homotopic to $\textrm{id}_{X}$, and therefore
            $f$ and $f^{-1}$ are homotopy inverses of each other, meaning
            $(X,\,\tau_{X})$ and $(Y,\,\tau_{Y})$ are homotopy equivalent.
        \end{proof}
        This theorem does not reverse.
        \begin{theorem}
            $\mathbb{R}^{n}$, with the standard topology, is homotopy equivalent
            to $\{\,0\,\}$ with the subspace topology.
        \end{theorem}
        \begin{proof}
            Define $f:\mathbb{R}^{n}\rightarrow\{\,0\,\}$ via
            $f(\mathbf{x})=0$. This is a constant function, so it is continuous.
            Let $g:\{\,0\,\}\rightarrow\mathbb{R}^{n}$ be defined by
            $g(0)=\mathbf{0}$. Since $g$ is a constant function, it is
            continuous. But $(g\circ{f})(\mathbf{x})=\mathbf{0}$, and this
            is homotopic to $\textrm{id}_{\mathbb{R}^{n}}$ with the homotopy
            $H(\mathbf{x},\,t)=t\mathbf{x}$. Also,
            $(f\circ{g})(0)=0$, so $f\circ{g}=\textrm{id}_{\{\,0\,\}}$, so
            $f\circ{g}$ is certainly homotopic to the identity since it is
            equal to it. Hence $f$ and $g$ are homotopy inverses of each other.
        \end{proof}
        This idea gets a name.
        \begin{definition}[\textbf{Contractible Topological Space}]
            A contractible topological space is a topological space
            $(X,\,\tau)$ that is homotopy equivalent to a single point
            $\{\,0\,\}$.
        \end{definition}
        $\mathbb{R}^{n}$ is not homeomorphic to a point, homeomorphisms must
        be bijective. This shows homotopy equivalent is much weaker. But
        even if $(X,\,\tau_{X})$ and $(Y,\,\tau_{Y})$ are topological spaces
        where $X$ and $Y$ have the same cardinality, it is possible for these
        spaces to be homotopy equivalent but not homeomorphic.
        \par\hfill\par
        Let $X=\mathbb{R}^{2}\setminus\{\,\mathbf{0}\,\}$ be the punctured
        plane with the subspace topology. This has the same cardinality as
        $\mathbb{S}^{1}$ since both have the same cardinality as $\mathbb{R}$.
        They are not homeomorphic. The circle is compact by Heine-Borel, the
        punctured plane is not compact (also by Heine-Borel). Define
        $f:\mathbb{S}^{1}\rightarrow\mathbb{R}^{2}\setminus\{\,\mathbf{0}\,\}$
        to be the inclusion map, $f(\mathbf{x})=\mathbf{x}$. Define
        $g:\mathbf{R}^{2}\setminus\{\,\mathbf{0}\,\}\rightarrow\mathbb{S}^{1}$
        via $g(\mathbf{x})=\mathbf{x}/||\mathbf{x}||_{2}$, the normalization
        map. Since $\mathbf{0}\not\in\mathbb{R}^{2}\setminus\{\,\mathbf{0}\,\}$
        this function is well-defined and continuous. We have
        $g\circ{f}$ is the identity function on $\mathbb{S}^{1}$, so it is
        homotopic to it. $f\circ{g}$ is the function sending
        $\mathbf{x}\ne\mathbf{0}$ to $\mathbf{x}/||\mathbf{x}||_{2}$. This
        is homotopic to the identity on
        $\mathbb{R}^{2}\setminus\{\,\mathbf{0}\,\}$, define $H$ via:
        \begin{equation}
            H(\mathbf{x},\,t)
            =(1-t)\frac{\mathbf{x}}{||\mathbf{x}||_{2}}+t\mathbf{x}
        \end{equation}
        which is a homotopy between $f\circ{g}$ and the identity.
        \begin{figure}
            \centering
            \includegraphics{../../../images/retraction_punc_plane_to_circle.pdf}
            \caption{Homotopy Equivalence from the Punctured Plane to $\mathbb{S}^{1}$}
            \label{fig:retraction_punc_plane_to_circle}
        \end{figure}
        \par\hfill\par
        Let's modify our constraints. What if we have \textit{compact} subsets
        of the plane? Could compact subsets of the same cardinality be
        homotopy equivalent but not homeomorphic? Consider
        $X\subseteq\mathbb{R}^{2}$ defined by:
        \begin{equation}
            X=
            \big\{\,\mathbf{x}\in\mathbb{R}^{2}\;|\;
                \frac{1}{2}\leq||\mathbf{x}||_{2}\leq{1}\,\}
        \end{equation}
        This is the \textit{closed annulus} in the plane. By Heine-Borel it
        is compact, and it too has the same cardinality as $\mathbb{S}^{1}$.
        The circle and the closed annulus are also homotopy equivalent, but
        not homeomorphic. To see this, intuively, if we remove two points from
        $\mathbb{S}^{1}$ we end up with two pieces. If we remove two pieces
        from $X$ we still have one piece. The two spaces are homotopy
        equivalent, the same functions used with the punctured plane work.
        This is shown in Fig.~\ref{fig:homotopy_circle}.
        \begin{figure}[H]
            \centering
            \includegraphics{../../../images/homotopy_circle.pdf}
            \caption{Homotopy Equivalence Between $\mathbb{S}^{1}$ and an Annulus}
            \label{fig:homotopy_circle}
        \end{figure}
        \par\hfill\par
        The annulus looks \textit{two dimensional}, the circle is one
        dimensional (whatever this means). You modify your question.
        If both subsets are compact and have the same dimension, does homotopy
        equivalence imply homeomorphic? Great question! This is one of the
        most famous conjectures of topology, the Poincar\'{e} conjecture.
        If $(X,\,\tau)$ is a three dimensional manifold (locally the space
        looks just like $\mathbb{R}^{3}$) that is compact and homotopy
        equivalent to $\mathbb{S}^{3}$, the three dimensional sphere that lives
        as a subspace of $\mathbb{R}^{4}$, is $(X,\,\tau)$ homeomorphic to
        $\mathbb{S}^{3}$? The answer is yes, but this took about 100 years to
        solve.
\end{document}
