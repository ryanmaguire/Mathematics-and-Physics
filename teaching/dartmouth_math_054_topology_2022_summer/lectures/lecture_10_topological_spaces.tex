%-----------------------------------LICENSE------------------------------------%
%   This file is part of Mathematics-and-Physics.                              %
%                                                                              %
%   Mathematics-and-Physics is free software: you can redistribute it and/or   %
%   modify it under the terms of the GNU General Public License as             %
%   published by the Free Software Foundation, either version 3 of the         %
%   License, or (at your option) any later version.                            %
%                                                                              %
%   Mathematics-and-Physics is distributed in the hope that it will be useful, %
%   but WITHOUT ANY WARRANTY; without even the implied warranty of             %
%   MERCHANTABILITY or FITNESS FOR A PARTICULAR PURPOSE.  See the              %
%   GNU General Public License for more details.                               %
%                                                                              %
%   You should have received a copy of the GNU General Public License along    %
%   with Mathematics-and-Physics.  If not, see <https://www.gnu.org/licenses/>.%
%------------------------------------------------------------------------------%
\documentclass{article}
\usepackage{graphicx}                           % Needed for figures.
\usepackage{amsmath}                            % Needed for align.
\usepackage{amssymb}                            % Needed for mathbb.
\usepackage{amsthm}                             % For the theorem environment.
\usepackage{hyperref}
\hypersetup{colorlinks=true, linkcolor=blue}

%------------------------Theorem Styles-------------------------%
\theoremstyle{plain}
\newtheorem{theorem}{Theorem}[section]

% Define theorem style for default spacing and normal font.
\newtheoremstyle{normal}
    {\topsep}               % Amount of space above the theorem.
    {\topsep}               % Amount of space below the theorem.
    {}                      % Font used for body of theorem.
    {}                      % Measure of space to indent.
    {\bfseries}             % Font of the header of the theorem.
    {}                      % Punctuation between head and body.
    {.5em}                  % Space after theorem head.
    {}

% Define default environments.
\theoremstyle{normal}
\newtheorem{examplex}{Example}[section]
\newtheorem{definitionx}{Definition}[section]

\newenvironment{example}{%
    \pushQED{\qed}\renewcommand{\qedsymbol}{$\blacksquare$}\examplex%
}{%
    \popQED\endexamplex%
}

\newenvironment{definition}{%
    \pushQED{\qed}\renewcommand{\qedsymbol}{$\blacksquare$}\definitionx%
}{%
    \popQED\enddefinitionx%
}

\title{Point-Set Topology: Lecture 10}
\author{Ryan Maguire}
\date{\today}

% No indent and no paragraph skip.
\setlength{\parindent}{0em}
\setlength{\parskip}{0em}

\begin{document}
    \maketitle
    \section{Topological Spaces}
        \begin{definition}[\textbf{Topology on a Set}]
            A topology on a set $X$ is a subset $\tau\subseteq\mathcal{P}(X)$
            such that:
            \begin{enumerate}
                \item $\emptyset\in\tau$
                \item $X\in\tau$
                \item For every $\mathcal{O}\subseteq\tau$ it is true that
                    $\bigcup\mathcal{O}\in\tau$
                \item For all $\mathcal{U},\mathcal{V}\in\tau$ it is true that
                    $\mathcal{U}\cap\mathcal{V}\in\tau$.
            \end{enumerate}
            That is, $\tau$ contains the empty set and the whole set, it is
            closed under arbitrary unions, and closed under the
            intersection of two elements.
    \end{definition}
    \begin{theorem}
        If $X$ is a set, if $\tau$ is a topology on $X$, and if
        $\mathcal{O}\subseteq\tau$ is finite, then $\bigcap\mathcal{O}\in\tau$.
    \end{theorem}
    \begin{proof}
        We prove by induction on the size of $\mathcal{O}$. The base case
        is true by the definition of a topology. Suppose the claim is true
        for $n\in\mathbb{N}$ and let $\mathcal{O}\subseteq\tau$ have
        $n+1$ elements. Label them $\mathcal{U}_{0},\,\dots,\,\mathcal{U}_{n}$.
        Then, by induction, the set $\mathcal{V}$ defined by:
        \begin{equation}
            \mathcal{V}=\bigcap_{k=0}^{n-1}\mathcal{U}_{k}
        \end{equation}
        is in $\tau$. But then
        $\bigcap\mathcal{O}=\mathcal{V}\cap\mathcal{U}_{n}$,
        the intersection of two elements of $\tau$, which is
        an element of $\tau$. Hence, $\bigcap\mathcal{O}\in\tau$.
    \end{proof}
    \begin{definition}[\textbf{Topological Space}]
        A topological space is an ordered pair $(X,\,\tau)$ where $X$ is a set
        and $\tau$ is a topology on $X$.
    \end{definition}
    \begin{example}
        If $X$ is a set, then $\mathcal{P}(X)$, the power set of $X$, is a
        topology on $X$. The power set is trivially closed under arbitrary
        unions and finite intersections, and moreover
        $\emptyset\in\mathcal{P}(X)$ and $X\in\mathcal{P}(X)$. This is
        the \textit{discrete topology} on $X$.
    \end{example}
    \begin{example}
        If $X$ is a set, then the set $\tau=\{\,\emptyset,\,X\,\}$ is a topology
        on $X$. This has several names, the \textit{chaotic topology}, the
        \textit{trivial topology}, and the \textit{indiscrete topology}.
    \end{example}
    \begin{example}
        Take $X=\{\,0,\,1,\,2\,\}$ and
        $\tau=\big\{\,\emptyset,\,\{\,0\,\},\,\{\,0,\,1,\,2\,\}\big\}$. The set
        $\tau$ is a topology on $X$. The sets are all nested since
        $\emptyset\subseteq\{\,0\,\}\subseteq\{\,0,\,1,\,2\,\}$, so it is closed
        under unions and intersections.
    \end{example}
    In metric spaces we used the metric $d$ to define openness. Here, we use
    the topology.
    \begin{definition}[\textbf{Open Set in a Topological Space}]
        An open set in a topological space $(X,\,\tau)$ is an element
        $\mathcal{U}\in\tau$.
    \end{definition}
    In the metric setting we were able to use sequences to define limit points
    and closed sets. This gave us a theorem that closed sets are just the
    complements of open sets. Since we lack a metric, we take this and use it
    to \textit{define} closed sets in a topological space.
    \begin{definition}[\textbf{Closed Set in a Topological Space}]
        A closed set in a topological space $(X,\,\tau)$ is a set
        $\mathcal{C}\subseteq{X}$ such that $X\setminus\mathcal{C}$ is open.
        That is, a set $\mathcal{C}$ such that
        $X\setminus\mathcal{C}\in\tau$.
    \end{definition}
    \begin{theorem}
        If $(X,\,\tau)$ is a topological space, then $\mathcal{U}\subseteq{X}$
        is open if and only if $X\setminus\mathcal{U}$ is closed.
    \end{theorem}
    \begin{proof}
        Suppose $\mathcal{U}$ is open. Then by definition,
        $X\setminus\mathcal{U}$ is closed. Now, suppose
        $X\setminus\mathcal{U}$ is closed. Then by the definition of closed
        sets, $X\setminus(X\setminus\mathcal{U})$ is open. But by the double
        complement law, $\mathcal{U}=X\setminus(X\setminus{U})$, and hence
        $\mathcal{U}$ is open.
    \end{proof}
    \begin{example}
        Do not confuse \textit{open} for \textit{not-closed} and do not confuse
        \textit{not-open} for \textit{closed}. These are distinct notions.
        It is possible to be open, closed, both, or neither. Let $X$ be a set
        and let $\tau=\mathcal{P}(X)$, the discrete topology on $X$. Then
        \textit{every} subset is open, and by the previous theorem,
        \textit{every} subset is closed. Now let
        $\tau=\{\,\emptyset,\,X\,\}$. Then every non-empty proper
        subset is \textit{not} open, and hence every non-empty proper subset is
        \textit{not} closed. These two examples show it is possible for
        $\mathcal{U}\subseteq{X}$ to be both open and closed, and for
        $\mathcal{V}\subseteq{X}$ to be neither open nor closed. It completely
        depends on the topology $\tau$.
    \end{example}
    Topological spaces are direct generalizations of metric spaces. Every
    metric space is also a topological space.
    \begin{theorem}
        If $(X,\,d)$ is a metric space, and if $\tau_{d}$ is the metric
        topology, then $(X,\,\tau_{d})$ is a topological space.
    \end{theorem}
    \begin{proof}
        We have proven that $\emptyset$ and $X$ are metrically open, so
        $\emptyset,X\in\tau_{d}$. Moreover, the intersection of two
        open sets in a metric space is open, as is the union of arbitrarily
        many open sets. That is, $\tau_{d}$ is a topology on $X$, and
        $(X,\,\tau_{d})$ is a topological space.
    \end{proof}
    \begin{example}
        The standard topology on the real line $\mathbb{R}$, the metric topology
        induced by the metric $d(x,\,y)=|x-y|$, denoted $\tau_{\mathbb{R}}$,
        is such that every proper non-empty subset
        $\mathcal{U}\subseteq\mathbb{R}$ is either open, closed, or neither,
        but \textit{not} both. That is, the only subsets of $\mathbb{R}$ that
        are both open and closed are $\emptyset$ and $\mathbb{R}$. This isn't
        too easy to show, it is essentially the statement that the real line
        is \textit{connected}, but we don't have the vocabulary to prove such
        a claim yet. Still, it is worth knowing this property when trying to
        intuitively understand open and closed sets.
    \end{example}
    A natural question is whether or not all topological spaces arise from
    metrics. This is false. The indiscrete topology on a set $X$ containing
    at least two points can't come from a metric. For suppose $X$ is a set
    with $a,b\in{X}$ and $a\ne{b}$. Suppose $d$ is any metric. Setting
    $\varepsilon={d}(a,\,b)/2$, the open ball around $a$ of radius
    $\varepsilon$ is a metrically open subset that contains $a$ but does not
    contain $b$. But in the indiscrete topology $\tau=\{\,\emptyset,\,X\,\}$,
    the only open sets are the empty set (which $B_{\varepsilon}^{(X,\,d)}(a)$
    is not empty since $\varepsilon>0$ and hence
    $a\in{B}_{\varepsilon}^{(X,\,d)}(a)$) and the whole space $X$
    (and $B_{\varepsilon}^{(X,\,d)}(a)\ne{X}$ since
    $b\notin{B}_{\varepsilon}^{(X,\,d)}(a)$). So the indiscrete topology cannot
    come from a metric.
    \par\hfill\par
    One of the problems with the indiscrete topology is that it lacks the
    \textit{Hausdorff property}. In Felix Hausdorff's original definition of
    topological spaces he required points in the space to be able to be
    separated by disjoint open sets. That is, given $x,y\in{X}$ with
    $x\ne{y}$, Hausdorff required there to be open sets $\mathcal{U}$ and
    $\mathcal{V}$ such that $x\in\mathcal{U}$, $y\in\mathcal{V}$, and
    $\mathcal{U}\cap\mathcal{V}=\emptyset$. We take Hausdorff's property and
    use it to define \textit{Hausdorff topological spaces}
    (also see Fig.~\ref{fig:hausdorff_condition_001}).
    \begin{definition}[\textbf{Hausdorff Topological Space}]
        A Hausdorff topological space is a topological space $(X,\,\tau)$ such
        that for all $x,y\in{X}$, $x\ne{y}$, there exists
        $\mathcal{U},\mathcal{V}\in\tau$
        such that $x\in\mathcal{U}$, $y\in\mathcal{V}$, and
        $\mathcal{U}\cap\mathcal{V}=\emptyset$.
    \end{definition}
    There are reasons we take the more general definition as \textit{the}
    definition of a topological space. There are certain operations that can be
    performed on a topological space (such as glueing points together) that can
    take a Hausdorff space $(X,\,\tau)$ and transform it into a
    non-Hausdroff space (but it'll still be a topological space). Also many
    non-Hausdorff topological spaces have found their way into the mathematical
    world with applications to physics and geometry. In algebraic geometry,
    the \textit{Zariski topology} is non-Hausdorff, and in general relativity,
    the space of light rays in a spacetime can be non-Hausdorff, depending on
    the topology of the spacetime.
    \par\hfill\par
    \begin{figure}
        \centering
        \includegraphics{../../../images/hausdorff_condition_001.pdf}
        \caption{The Hausdorff Condition}
        \label{fig:hausdorff_condition_001}
    \end{figure}
    Metric spaces have the Hausdorff property, but since we've moved on to
    topology, it is better to speak of \textit{metrizable spaces}. These are
    topological spaces where the topology $\tau$ comes from some metric $d$.
    \begin{definition}[\textbf{Metrizable Topological Space}]
        A metrizable topological space is a topological space $(X,\,\tau)$ such
        that there exists a metric $d$ on $X$ such that $\tau=\tau_{d}$,
        where $\tau_{d}$ is the metric topology from $d$.
    \end{definition}
    Every metric space we've examined is an example of a metrizable topological
    space. In particular, the real line, Cartesian or complex plane,
    Euclidean space, Paris plane, London plane, discrete metric spaces,
    all of that. We've also found a non-metrizable space, the indiscrete
    topology on a set containing at least two points. Some of the central
    theorems of point-set topology tell us when a topological space is
    metrizable. Hausdorff alone is not enough. We will see plenty of Hausdorff
    spaces that can not come from a metric. The converse is true, however. Every
    metrizable space is Hausdorff.
    \begin{theorem}
        If $(X,\,\tau)$ is a metrizable topological space, then it is
        Hausdorff.
    \end{theorem}
    \begin{proof}
        Since $(X,\,\tau)$ is metrizable, there is a metric $d$ such that
        $\tau=\tau_{d}$, where $\tau_{d}$ is the metric topology from $d$.
        Let $x,y\in{X}$ with $x\ne{y}$. To prove $(X,\,\tau)$ is Hausdorff we
        must find disjoint open sets $\mathcal{U}$ and $\mathcal{V}$ such that
        $x\in\mathcal{U}$ and $y\in\mathcal{V}$. Let
        $\varepsilon=d(x,\,y)/2$ and define
        $\mathcal{U}=B_{\varepsilon}^{(X,\,d)}(x)$ and
        $\mathcal{V}=B_{\varepsilon}^{(X,\,d)}(y)$. Since $x\ne{y}$ and $d$
        is a metric, it is true that $d(x,\,y)>0$ and therefore $\varepsilon$
        is positive. But then $\mathcal{U}$ and $\mathcal{V}$ are open sets
        with $x\in\mathcal{U}$ and $y\in\mathcal{V}$, since open balls are open.
        Suppose $z\in\mathcal{U}\cap\mathcal{V}$. Then:
        \begin{equation}
            d(x,\,y)
            \leq{d}(x,\,z)+d(z,\,y)
            <\varepsilon+\varepsilon
            =d(x,\,y)
        \end{equation}
        And thus $d(x,\,y)<d(x,\,y)$, a contradiction. So
        $z\notin\mathcal{U}\cap\mathcal{V}$, and therefore
        $\mathcal{U}\cap\mathcal{V}=\emptyset$. Thus, $(X,\,\tau)$ is
        Hausdorff.
    \end{proof}
    See Fig.~\ref{fig:metrizable_is_hausdorff} for the idea behind the proof.
    With this we can rigorously prove that the indiscrete topology on a set
    $X$ that has at least two distinct points $a,b\in{X}$ cannot possibly
    come from a metric. This is done by observing that, if
    $\tau=\{\,\emptyset,\,X\,\}$, then $(X,\,\tau)$ is
    \textit{non-Hausdorff}, and since it is non-Hausdorff, it can't
    be metrizable since metrizable topological spaces are Hausdorff. Given
    $a\in{X}$, the only open set in $\tau$ that contains $a$ is
    $X$ (since $a\notin\emptyset$). But $b\in{X}$ as well, so
    there are not two open sets $\mathcal{U}$ and $\mathcal{V}$ with
    $a\in\mathcal{U}$, $b\in\mathcal{V}$, and
    $\mathcal{U}\cap\mathcal{V}=\emptyset$. That is, $(X,\,\tau)$ is not
    Hausdorff, meaning it is not metrizable. Topological spaces are
    far more general than metric spaces.
    \begin{figure}
        \centering
        \includegraphics{../../../images/metrizable_is_hausdorff.pdf}
        \caption{Metrizable Implies Hausdorff}
        \label{fig:metrizable_is_hausdorff}
    \end{figure}
    \begin{example}
        Let's intuitively try to imagine what the indiscrete
        topology on $\mathbb{R}$ \textit{looks} like. The indiscrete topology
        $\tau=\{\,\emptyset,\,\mathbb{R}\,\}$ says the only non-empty open set
        is the entire real line $\mathbb{R}$. That is, all of the points in the
        real line are very very close to each other, essentially
        indistinguishable. Topologically it is impossible to tell
        two points in the line apart. This is hard to imagine, and your brain
        probably pictures a single point. But the space $(\mathbb{R},\,\tau)$
        is not the same as a single point, the set $\mathbb{R}$ still has
        infinitely many points. This is hard to imagine, and fortunately such
        bizarre topological spaces like $(\mathbb{R},\,\tau)$ rarely find their
        way into applications. Spaces like this usually serve as counterexamples
        to claims (such as the claim that every topological space comes from a
        metric. The topological space $(\mathbb{R},\,\tau)$ is a
        counterexample).
    \end{example}
    \begin{example}
        Now ponder the discrete topology on $\mathbb{R}$,
        $\tau=\mathcal{P}(\mathbb{R})$. Here, \textit{every} subset is open,
        so in particular given $x,y\in\mathbb{R}$, the sets
        $\{\,x\,\}$ and $\{\,y\,\}$ are open. The way to picture this space is
        as a bunch of scattered points with empty space between them.
    \end{example}
    There is a weaker notion than Hausdorff that has found it's way into
    physics and geometry, the notion of a
    \textit{Fr\'{e}chet topological space}.
    \begin{definition}[\textbf{Fr\'{e}chet Topological Space}]
        A Fr\'{e}chet topological space is a topological space $(X,\,\tau)$
        such that for all $x,y\in{X}$, $x\ne{y}$, there exists
        $\mathcal{U},\mathcal{V}\in\tau$ such that $x\in\mathcal{U}$,
        $y\in\mathcal{V}$, and $x\notin\mathcal{V}$, $y\notin\mathcal{U}$.
    \end{definition}
    See Fig.~\ref{fig:metrizable_is_hausdorff} for a visual.
    This modifies the Hausdorff condition. Instead of requiring
    $\mathcal{U}$ and $\mathcal{V}$ to be disjoint (as in the Hausdroff case),
    we only require $\mathcal{U}$ to include $x$ and not $y$, and
    $\mathcal{V}$ to include $y$ and not $x$. Every Hausdorff space is,
    in particular, Fr\'{e}chet.
    \begin{theorem}
        If $(X,\,\tau)$ is a Hausdorff topological space, then it is a
        Fr\'{e}chet topological space.
    \end{theorem}
    \begin{proof}
        Let $x,y\in{X}$, $x\ne{y}$. Since $(X,\,\tau)$ is Hausdorff, there
        exists open sets $\mathcal{U},\mathcal{V}\in\tau$ such that
        $x\in\mathcal{U}$, $y\in\mathcal{V}$, and
        $\mathcal{U}\cap\mathcal{V}=\emptyset$. But since $x\in\mathcal{U}$,
        it must be true that $x\notin\mathcal{V}$ since
        $\mathcal{U}\cap\mathcal{V}=\emptyset$. But since
        $y\in\mathcal{V}$, it must be true that $y\notin\mathcal{U}$ since
        $\mathcal{U}\cap\mathcal{V}=\emptyset$. So $\mathcal{U}$ and
        $\mathcal{V}$ are such that $x\in\mathcal{U}$,
        $y\in\mathcal{V}$, and $x\notin\mathcal{V}$, $y\notin\mathcal{U}$.
        So $(X,\,\tau)$ is a Fr\'{e}chet topological space.
    \end{proof}
    A word of warning. There are three types of spaces that are called
    \textit{Fr\'{e}chet} spaces. Maurice Ren\'{e} Fr\'{e}chet was a very
    prolific mathematician. In the study of topological vector spaces one
    studies \textit{Fr\'{e}chet spaces}. In general topology one studies
    \textit{Fr\'{e}chet topological spaces} and
    \textit{Fr\'{e}chet-Urysohn topological spaces}. Fr\'{e}chet also invented
    the idea of metric spaces, but fortunately the mathematical community has
    universally adopted the term \textit{metric space}, rather than name another
    type of space after Fr\'{e}chet. To reduce confusion,
    Fr\'{e}chet topological spaces are often called $T_{1}$ spaces, and
    Hausdorff topological spaces are sometimes called $T_{2}$. It is far
    more common to just say \textit{Hausdorff}, and this $T_{n}$ notation seems
    to be mostly historical.
    \begin{figure}
        \centering
        \includegraphics{../../../images/frechet_condition_001.pdf}
        \caption{The Fr\'{e}chet Condition}
        \label{fig:frechet_condition_001}
    \end{figure}
    \begin{example}[\textbf{Finite Complement Topology}]
        Not every Fr\'{e}chet topological space is Hausdorff. The space of
        light rays in a spacetime, as alluded to earlier, need not be
        Hausdorff, but light rays always form a Fr\'{e}chet topological
        space. A simpler example is the \textit{finite complement topology}
        on the real line. Define a set $\mathcal{U}\subseteq\mathbb{R}$ to be
        open if and only if $\mathbb{R}\setminus\mathcal{U}$ is a finite set,
        or if $\mathcal{U}=\emptyset$.
        The collection $\tau$ of all such sets is a topology on $\mathbb{R}$.
        Since $\mathbb{R}\setminus\mathbb{R}=\emptyset$, which is finite,
        we see that $\mathbb{R}\in\tau$. The empty set was intentionally
        included, so $\emptyset\in\tau$. If $\mathcal{U},\mathcal{V}\in\tau$,
        then:
        \begin{equation}
            \mathbb{R}\setminus(\mathcal{U}\cap\mathcal{V})=
                (\mathbb{R}\setminus\mathcal{U})\cup
                (\mathbb{R}\setminus\mathcal{U})
        \end{equation}
        by the De Morgan law. Since this is the union of two finite sets, it
        is finite, and hence $\tau$ is closed under the intersection of two
        elements. Lastly, if $\mathcal{O}\subseteq\tau$, then:
        \begin{equation}
            \mathbb{R}\setminus\bigcup_{\mathcal{U}\in\mathcal{O}}\mathcal{U}
            =\bigcap_{\mathcal{U}\in\mathcal{O}}\mathbb{R}\setminus\mathcal{U}
        \end{equation}
        If $\mathcal{O}$ is empty, then this intersection is $\mathbb{R}$,
        and $\mathbb{R}$ is open. If $\mathcal{O}$ is non-empty, then there is
        some $\mathcal{V}\in\mathcal{O}$. But then, by the definition of
        intersection:
        \begin{equation}
            \bigcap_{\mathcal{U}\in\mathcal{O}}\mathbb{R}\setminus\mathcal{U}
            \subseteq\mathbb{R}\setminus\mathcal{V}
        \end{equation}
        which is a finite set, so the intersection is a subset of a finite set,
        and is therefore finite itself. This shows $\bigcup\mathcal{O}\in\tau$,
        so $\tau$ is a topology. $(\mathbb{R},\,\tau)$ is not Hausdorff. Given
        any non-empty $\mathcal{U},\mathcal{V}\in\tau$, since
        $\mathbb{R}\setminus\mathcal{U}$ and $\mathbb{R}\setminus\mathcal{V}$
        are finite, and since $\mathbb{R}$ is infinite,
        $\mathcal{U}\cap\mathcal{V}$ is non-empty. $(\mathbb{R},\,\tau)$ is a
        Fr\'{e}chet topological space, however. Given $x\ne{y}$, let
        $\mathcal{U}=\mathbb{R}\setminus\{\,y\,\}$ and
        $\mathcal{V}=\mathbb{R}\setminus\{\,x\,\}$.
        Then $x\in\mathcal{U}$ since $x\ne{y}$, but $y\notin\mathcal{U}$.
        Also, $y\in\mathcal{V}$, but $x\notin\mathcal{V}$. But
        $\mathcal{U}$ and $\mathcal{V}$ are open since their complements contain
        one point, and are hence finite. This shows $(\mathbb{R},\,\tau)$ is a
        Fr\'{e}chet topological space.
    \end{example}
    \begin{example}[\textbf{Countable Complement Topology}]
        Define $\tau$ on $\mathbb{R}$ to be the set of all
        $\mathcal{U}\subseteq\mathbb{R}$ such that
        $\mathbb{R}\setminus\mathcal{U}$ is countable
        (that is, finite or countably infinite), or $\mathcal{U}$ is the
        empty set. This is a topology on $\mathbb{R}$. The empty set and
        $\mathbb{R}$ are elements of $\tau$ since
        $\mathbb{R}\setminus\mathbb{R}=\emptyset$ is countable, and
        $\emptyset$ was deliberately included. Given
        $\mathcal{U},\mathcal{V}\in\tau$, the intersection is
        also included since:
        \begin{equation}
            \mathbb{R}\setminus(\mathcal{U}\cap\mathcal{V})
            =(\mathbb{R}\setminus\mathcal{U})\cup
            (\mathbb{R}\setminus\mathcal{V})
        \end{equation}
        which is the union of countable sets, and is therefore countable.
        By a similar argument to the finite complement topology, the
        union of any collection $\mathcal{O}\subseteq\tau$ is also an
        element of $\tau$. The space $(\mathbb{R},\,\tau)$ is not Hausdorff,
        but it is a Fr\'{e}chet topological space. It is not Hausdorff since
        for any non-empty $\mathcal{U},\mathcal{V}\in\tau$, the complements
        $\mathbb{R}\setminus\mathcal{U}$ and $\mathbb{R}\setminus\mathcal{V}$
        are countable. The real numbers are uncountable, meaning there must be
        an element common to both $\mathcal{U}$ and $\mathcal{V}$, showing
        $(\mathbb{R},\,\tau)$ cannot be Hausdorff. To show it is Fr\'{e}chet
        we use the same construction as before. Given $x,y\in\mathbb{R}$,
        $x\ne{y}$, we define $\mathcal{U}=\mathbb{R}\setminus\{\,y\,\}$ and
        $\mathcal{V}=\mathbb{R}\setminus\{\,x\,\}$, showing
        $(\mathbb{R},\,\tau)$ has the Fr\'{e}chet property.
    \end{example}
    The construction used in both of these examples relies on the fact that
    $\mathbb{R}\setminus\{\,x\,\}$ is an \textit{open} subset for any real
    number $x\in\mathbb{R}$ in both the finite and countable complement
    topologies. To phrase this differently, the proof requires that
    $\{\,x\,\}$ is a \textit{closed} set. This is what our intuition tells us.
    Singleton sets should always be closed. In a metric space $(X,\,d)$, given
    $x\in{X}$, the set $\{\,x\,\}$ is indeed closed since the only sequence
    $a:\mathbb{N}\rightarrow\{\,x\,\}$ is the constant sequence
    $a_{n}=x$, and this does indeed converge to $x$, showing that
    $\{\,x\,\}$ has all of its limit points. This property does not exist for
    all topological spaces. The real line $\mathbb{R}$ with the indiscrete
    topology $\tau=\{\,\emptyset,\,\mathbb{R}\,\}$ lacks this feature. The
    \textit{only} closed sets are $\emptyset$ and $\mathbb{R}$, so in
    particular, given $x\in\mathbb{R}$, it is \textit{not} true that
    $\{\,x\,\}$ is closed. Fr\'{e}chet topological spaces (and hence Hausdorff
    topological spaces) do not lack this feature.
    \begin{theorem}
        If $(X,\,\tau)$ is a Fr\'{e}chet topological space, and if $x\in{X}$,
        then the set $\{\,x\,\}$ is closed.
    \end{theorem}
    \begin{proof}
        For all $y\in{X}$, $y\ne{x}$, there is an open set
        $\mathcal{U}_{y}\in\tau$ such that $x\notin\mathcal{U}_{y}$ and
        $y\in\mathcal{U}_{y}$. Define $\mathcal{O}$ by:
        \begin{equation}
            \mathcal{O}=\{\,\mathcal{U}_{y}\in\tau\;|\;
                y\in{X}\textrm{ and }y\ne{x}\,\}
        \end{equation}
        Since $\mathcal{O}$ is a collection of open sets,
        $\bigcup\mathcal{O}$ is open. Since for all $y\in{X}$ with
        $y\ne{x}$ the set $\mathcal{U}_{y}$ is contained in $\mathcal{O}$,
        we have that $y\in\bigcup\mathcal{O}$. But also for every open set
        $\mathcal{U}$ in $\mathcal{O}$, $x\notin\mathcal{U}$ by construction,
        and hence $x\notin\bigcup\mathcal{O}$. Therefore
        $\bigcup\mathcal{O}=X\setminus\{\,x\,\}$. But then
        $\{\,x\,\}$ is the complement of an open set, and is therefore closed.
    \end{proof}
    The converse of this theorem is true as well, and this is what we used to
    show that the finite complement and countable complement topologies on
    $\mathbb{R}$ are Fr\'{e}chet topological spaces.
    \begin{theorem}
        If $(X,\,\tau)$ is a topological space, and if for all $x\in{X}$ it is
        true that $\{\,x\,\}$ is closed, then $(X,\,\tau)$ is a
        Fr\'{e}chet topological space.
    \end{theorem}
    \begin{proof}
        For given $x,y\in{X}$, $x\ne{y}$, define
        $\mathcal{U}=X\setminus\{\,y\,\}$ and $\mathcal{V}=X\setminus\{\,x\,\}$.
        Then $x\in\mathcal{U}$ since $x\ne{y}$, and
        $y\in\mathcal{V}$ since $y\ne{x}$. But also
        $x\notin\mathcal{V}$ and $y\notin\mathcal{U}$. Since
        $\{\,x\,\}$ and $\{\,y\,\}$ are closed, by hypothesis, the sets
        $\mathcal{U}$ and $\mathcal{V}$ are the complements of closed sets
        and are therefore open. Thus, $(X,\,\tau)$ is a Fr\'{e}chet topological
        space.
    \end{proof}
    \begin{example}[\textbf{Standard Topology on $\mathbb{R}$}]
        Let $\tau_{\mathbb{R}}$, the \textit{standard topology} on
        $\mathbb{R}$, be defined as the set of all
        $\mathcal{U}\subseteq\mathbb{R}$ such that $x\in\mathcal{U}$ implies
        there is an $\varepsilon>0$ such that for all $y\in\mathbb{R}$ with
        $|x-y|<\varepsilon$, it is true that $y\in\mathcal{U}$. Then
        $(\mathbb{R},\,\tau_{\mathbb{R}})$ is a topological space, and moreover
        it is a \textit{metrizable} topological space since it stems from the
        standard Euclidean metric $d(x,\,y)=|x-y|$ defined on $\mathbb{R}$.
        Since this is metrizable, $(\mathbb{R},\,\tau_{\mathbb{R}})$ is
        a Hausdorff topological space and a Fr\'{e}chet topological space.
    \end{example}
    Let $\tau_{F}$ denote the finite complement topology on $\mathbb{R}$,
    $\tau_{C}$ the countable complement topology on $\mathbb{R}$, and
    $\tau_{\mathbb{R}}$ the standard topology on $\mathbb{R}$. Since a finite
    set is countable, we instantly have that $\tau_{F}\subseteq\tau_{C}$.
    But also, since $(\mathbb{R},\,\tau_{\mathbb{R}})$ is a metrizable,
    and therefore Hausdorff, and thus a Fr\'{e}chet topological space,
    points $\{\,x\,\}$ with $x\in\mathbb{R}$ are closed in
    $\tau_{\mathbb{R}}$. Since the finite union of closed sets is still closed,
    we see that all finite subsets of $\mathbb{R}$ are closed in
    $\tau_{\mathbb{R}}$, and therefore sets whose complement is finite are
    open in $\tau_{\mathbb{R}}$. But then
    $\tau_{F}\subseteq\tau_{\mathbb{R}}$. There is no comparison between
    $\tau_{C}$ and $\tau_{\mathbb{R}}$. The set
    $(0,\,1)$ is open in $\tau_{\mathbb{R}}$ but not $\tau_{C}$ since the
    complement of $(0,\,1)$ is $(-\infty,\,0]\cup[1,\,\infty)$, and this is
    not countable. Moreover, the set of all irrational numbers
    $\mathbb{R}\setminus\mathbb{Q}$ is open in $\tau_{F}$ since the
    complement is the rational numbers $\mathbb{Q}$ and this is countable.
    The irrationals are not open in $\tau_{\mathbb{R}}$ since for any
    irrational $x$ and for any $\varepsilon>0$ there is a rational number
    $y$ with $|x-y|<\varepsilon$. This notion of comparing topologies allows
    us to generate new topologies. In particular, if we are given a collection
    of topologies on a set $X$, we can construct a new topology by looking at
    the intersection over this collection.
    \begin{theorem}
        If $X$ is a set, and if $T\subseteq\mathcal{P}\big(\mathcal{P}(X)\big)$
        is a non-empty set such that for all $\tau\in{T}$ it is true that
        $\tau$ is a topology on $X$, then $\bigcap{T}$ is a topology on $X$.
    \end{theorem}
    \begin{proof}
        Since for all $\tau\in{T}$, $\tau$ is a topology, we have that
        $\emptyset\in\tau$ and $X\in\tau$, and since $T$ is non-empty, we
        conclude that $\emptyset\in\bigcap{T}$ and $X\in\bigcap{T}$. If
        $\mathcal{U},\mathcal{V}\in\bigcap{T}$, then for all $\tau\in{T}$, it is
        true that $\mathcal{U},\mathcal{V}\in\tau$. But all $\tau\in{T}$ are
        topologies, so $\mathcal{U}\cap\mathcal{V}\in\tau$ for all $\tau$, so
        $\mathcal{U}\cap\mathcal{V}\in\bigcap{T}$. If
        $\mathcal{O}\subseteq\bigcap{T}$, then for all $\tau\in{T}$ we have
        $\mathcal{O}\subseteq\tau$. But then, since $\tau$ is a topology,
        $\bigcup\mathcal{O}\in\tau$. Since this is true for all $\tau\in{T}$ we
        have that $\bigcup\mathcal{O}\in\bigcap{T}$.
        So $\bigcap{T}$ is a topology.
    \end{proof}
    Given a set $X$, we use this theorem to define the topology
    \textit{generated} by any subset $B\subseteq\mathcal{P}(X)$. We collect the
    \textit{smallest} topology that contains $B$ as a subset. There is always
    a topology that contains $B$ as a subset since $\mathcal{P}(X)$ is a
    topology. We write $T$ as the set of all topologies $\tau$ on $X$ with
    $B\subseteq\tau$ and then write $\tau(B)=\bigcap{T}$.
    \begin{definition}[\textbf{Generated Topology}]
        The topology on a set $X$ generated by a collection
        $B\subseteq\mathcal{P}(X)$ is the set $\tau(B)$ defined by:
        \begin{equation}
            \tau(B)=\bigcap{T}
        \end{equation}
        where $T$ is the set of all topologies $\tau$ on $X$ such that
        $B\subseteq\tau$.
    \end{definition}
    What we are doing is taking a collection of subsets
    $B\subseteq\mathcal{P}(X)$ that we \textit{want} to be open. However,
    this collection might not be a topology itself. We may need to add more
    sets to ensure we have the intersection and union properties that topologies
    enjoy. $\tau(B)$, the generated topology, does precisely this.
\end{document}
