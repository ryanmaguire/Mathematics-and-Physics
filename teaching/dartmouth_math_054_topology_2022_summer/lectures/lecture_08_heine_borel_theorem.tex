%-----------------------------------LICENSE------------------------------------%
%   This file is part of Mathematics-and-Physics.                              %
%                                                                              %
%   Mathematics-and-Physics is free software: you can redistribute it and/or   %
%   modify it under the terms of the GNU General Public License as             %
%   published by the Free Software Foundation, either version 3 of the         %
%   License, or (at your option) any later version.                            %
%                                                                              %
%   Mathematics-and-Physics is distributed in the hope that it will be useful, %
%   but WITHOUT ANY WARRANTY; without even the implied warranty of             %
%   MERCHANTABILITY or FITNESS FOR A PARTICULAR PURPOSE.  See the              %
%   GNU General Public License for more details.                               %
%                                                                              %
%   You should have received a copy of the GNU General Public License along    %
%   with Mathematics-and-Physics.  If not, see <https://www.gnu.org/licenses/>.%
%------------------------------------------------------------------------------%
\documentclass{article}
\usepackage{graphicx}                           % Needed for figures.
\usepackage{amsmath}                            % Needed for align.
\usepackage{amssymb}                            % Needed for mathbb.
\usepackage{amsthm}                             % For the theorem environment.
\theoremstyle{plain}
\newtheorem{theorem}{Theorem}[section]

\title{Point-Set Topology: Lecture 8}
\author{Ryan Maguire}
\date{Summer 2022}

% No indent and no paragraph skip.
\setlength{\parindent}{0em}
\setlength{\parskip}{0em}

\begin{document}
    \maketitle
    \section{Theorems on Compactness}
        \begin{theorem}
            If $(X,\,d)$ is a metric space and
            $a:\mathbb{N}\rightarrow\mathbb{R}$ is a convergent sequence,
            then $a$ is a Cauchy sequence.
        \end{theorem}
        \begin{proof}
            Since $a:\mathbb{N}\rightarrow{X}$ converges, there is an
            $x\in{X}$ such that $a_{n}\rightarrow{x}$. Let
            $\varepsilon>0$. Since $a_{n}\rightarrow{x}$ there is an
            $N\in\mathbb{N}$ such that $n>N$ implies
            $d(x,\,a_{n})<\frac{\varepsilon}{2}$. But then $n,m>N$ implies:
            \begin{equation}
                d(a_{m},\,a_{n})\leq{d}(x,\,a_{m})+d(x,\,a_{n})
                    <\frac{\varepsilon}{2}+\frac{\varepsilon}{2}
                    =\varepsilon
            \end{equation}
            and therefore $a$ is a Cauchy sequence.
        \end{proof}
        Without completeness, a metric space $(X,\,d)$ can have non-convergent
        Cauchy sequences. But given a Cauchy sequence with a convergent
        subsequence, the entire sequence must then convergent. The intuition is
        that a Cauchy sequence is a sequence where all of the points start to
        get really close together as the indices increase. Since there is a
        convergent subsequence, there is some point $x\in{X}$ where
        \textit{some} of the points in the sequence start to get really close
        to. But since all of the points get close together at higher and higher
        indices, all of the points must also get closer to $x$, and hence the
        entire sequence converges to $x$. Let's prove this.
        \begin{theorem}
            If $(X,\,d)$ is a metric space, if $a:\mathbb{N}\rightarrow{X}$
            is a Cauchy sequence, and if $a_{k}$ is a convergent subsequence,
            then $a$ is a convergent sequence.
        \end{theorem}
        \begin{proof}
            Since $a_{k}$ is a convergent sequence, there is an
            $x\in{X}$ such that $a_{k_{n}}\rightarrow{x}$. Let
            $\varepsilon>0$. Since $a_{k_{n}}\rightarrow{x}$, there is an
            $N_{0}\in\mathbb{N}$ such that $n>N_{0}$ implies
            $d(x,\,a_{k_{n}})<\frac{\varepsilon}{2}$. Since $a$ is a Cauchy
            sequence there is an $N_{1}\in\mathbb{N}$ such that
            $n,m>N_{1}$ implies $d(a_{m},\,a_{n})<\frac{\varepsilon}{2}$.
            Let $N=\textrm{max}(k_{N_{0}},\,N_{1})$.
            Then since $k$ is strictly increasing,
            $m>N$ implies $k_{m}>N_{0}$ and $k_{m}>N_{1}$.
            But then for all $n,m>N$:
            \begin{equation}
                d(x,\,a_{n})\leq{d}(a_{n},\,a_{k_{m}})+d(x,\,a_{k_{m}})
                    <\frac{\varepsilon}{2}+\frac{\varepsilon}{2}=\varepsilon
            \end{equation}
            and therefore $a_{n}\rightarrow{x}$. That is, $a$ is a convergent
            sequence.
        \end{proof}
        Completeness and closedness are related for metric space. Given a
        complete metric space $(X,\,d)$, the only complete subspaces are the
        closed ones. In particular, since $(\mathbb{R},\,|\cdot|)$, the standard
        metric on the real line, is complete, we see that the open unit interval
        is \textit{not} complete. The sequence
        $a:\mathbb{N}\rightarrow(0,\,1)$ defined by $a_{n}=\frac{1}{n+1}$
        is a Cauchy sequence, but it does not converge. We want to say it
        ``converges'' to zero, but zero is not an element of this subspace.
        \begin{theorem}
            If $(X,\,d)$ is a complete metric space, and if
            $A\subseteq{X}$, then $(A,\,d_{A})$ is a complete metric space if
            and only if $A$ is closed.
        \end{theorem}
        \begin{proof}
            Suppose $(A,\,d_{A})$ is a complete metric space and let
            $a:\mathbb{N}\rightarrow{A}$ be a sequence that converges in $X$.
            But convergent sequences are Cauchy sequences, and $(A,\,d_{A})$ is
            complete, and therefore Cauchy sequences converge. But then the
            limit of $a$ is contained in $A$, and therefore $A$ is closed.
            Now, suppose $A\subseteq{X}$ is closed. Let
            $a:\mathbb{N}\rightarrow{A}$ be a Cauchy sequence. Then, since
            $A\subseteq{X}$, $a:\mathbb{N}\rightarrow{X}$ is a Cauchy sequence
            in $X$. But $(X,\,d)$ is complete, and therefore $a$ converges.
            That is, there is an $x\in{X}$ such that $a_{n}\rightarrow{x}$.
            But $A$ is closed and therefore contains all of its limit points,
            so $x\in{A}$. Hence Cauchy sequences in $A$ converge in $A$, and
            therefore $(A,\,d_{A})$ is complete.
        \end{proof}
        This theorem is the baby version of the same idea for compactness.
        \begin{theorem}
            If $(X,\,d)$ is a compact metric space, and $A\subseteq{X}$, then
            $(A,\,d_{A})$ is compact if and only if $A$ is closed.
        \end{theorem}
        \begin{proof}
            Suppose $(A,\,d_{A})$ is compact and $x\in{X}$ a limit point of
            $A$. Then there is a sequence $a:\mathbb{N}\rightarrow{A}$ such
            that $a_{n}$ converges to $x$ in $X$. But $(A,\,d_{A})$ is compact,
            so there is a convergent subsequence $a_{k}$ with limit in $A$.
            But limits are unique, so $a_{k_{n}}\rightarrow{x}$, and therefore
            $x\in{A}$. Now suppose $A$ is closed. Let
            $a:\mathbb{N}\rightarrow{A}$ be a sequence. Then, since
            $A\subseteq{X}$, $a:\mathbb{N}\rightarrow{X}$ is a sequence in $X$.
            But $(X,\,d)$ is compact, so there is a convergent subsequence
            $a_{k}$ with limit $x\in{X}$. But $A$ is closed and hence contains
            all of its limit points, and therefore $x\in{A}$. But then
            $a_{k}$ is a convergent subsequence of $a$ in $A$. Therefore,
            $(A,\,d_{A})$ is compact.
        \end{proof}
        And lastly, it should be noted that compactness is far stronger than
        completeness. Let's prove this.
        \begin{theorem}
            If $(X,\,d)$ is a compact metric space, then $(X,\,d)$ is complete.
        \end{theorem}
        \begin{proof}
            Let $a:\mathbb{N}\rightarrow{X}$ be a Cauchy sequence. Since
            $(X,\,d)$ is compact, there is a convergent subsequence
            $a_{k}$. But a Cauchy sequence with a convergent subsequence is
            convergent, and therefore $a$ is convergent. Thus, $(X,\,d)$ is
            complete.
        \end{proof}
        \begin{theorem}[\textbf{Heine-Borel Theorem}]
            If $A\subseteq\mathbb{R}^{N}$ and
            $d:\mathbb{R}^{N}\times\mathbb{R}^{N}\rightarrow\mathbb{R}$ is the
            standard Euclidean metric,
            $d(\mathbf{x},\,\mathbf{y})=||\mathbf{x}-\mathbf{y}||_{2}$, then
            $(A,\,d_{A})$ is compact if and only if $A$ is closed and bounded.
        \end{theorem}
        \begin{proof}
            Suppose $(A,\,d_{A})$ is compact. We have proved that compact
            subspaces of any metric space are closed, so in particular
            $A$ is a closed subset of $\mathbb{R}^{N}$. Suppose $A$ is not
            bounded. Then for all $n\in\mathbb{N}$ there is an
            $a_{n}\in{A}$ with $||a_{n}||_{2}>n$, otherwise
            $A$ is bounded. The sequence $a:\mathbb{N}\rightarrow{A}$ can be
            chosen so that $||a_{m}||_{2}<||a_{n}||_{2}$ whenever $m<n$, while
            diverging off to infinity, and hence contains no
            convergent subsequences. This contradicts the assumption that
            $(A,\,d_{A})$ is compact. Hence, $A$ is bounded. Now, suppose
            $(A,\,d_{A})$ is closed and bounded. Let
            $\mathbf{x}:\mathbb{N}\rightarrow{A}$ be any sequence. Denote
            $\mathbf{x}_{n}\in{A}$ via the tuple:
            \begin{equation}
                \mathbf{x}_{n}=\big(x_{n}^{0},\,x_{n}^{1},\,\dots,\,
                    x_{n}^{N-1}\big)
            \end{equation}
            The sequence $x^{0}:\mathbb{N}\rightarrow\mathbb{R}$ defined by
            setting $x_{n}^{0}$ equal to the zeroth component of
            $\mathbf{x}_{n}$ is bounded since $A$ is bounded. By
            the Bolzano-Weierstrass theorem there is a
            convergent subsequence $x_{k}^{0}$. That is, there is some
            real number $r_{0}\in\mathbb{R}$ such that
            $x_{k_{n}}^{0}\rightarrow{r}_{0}$. But then
            $x_{k}^{1}$ is a (not necessarily convergent) subsequence of the
            sequence $x^{1}:\mathbb{N}\rightarrow\mathbb{R}$, the sequence
            defined by setting $x_{n}^{1}$ equal to the first component of
            $\mathbf{x}_{n}$. But then $x_{k}^{1}$ is a bounded sequence of
            real numbers and hence by the Bolzano-Weierstrass theorem, there
            is a convergent subsequence $x_{k_{k'}}^{1}$. That is, there is
            some $r_{1}\in\mathbb{R}$ such that
            $x_{k_{k'n}}^{1}\rightarrow{r}_{1}$. But
            $x_{k_{k'}}^{0}$ is a subsequence of $x_{k}^{0}$, and hence
            $x_{k_{k'}}^{0}$ is a subsequence of a convergent subsequence.
            But subsequences of convergent sequences converge and they converge
            to the same limit. That is, we now have that
            $x_{k_{k'}}^{0}$ and $x_{k_{k'}}^{1}$ are convergent sequences
            converging to $r_{0}$ and $r_{1}$, respectively.
            Continuing inductively, we obtain subseqence
            $k''$, $k'''$, up to $k^{(N-1)}$ such that
            $x_{k\dots{k}^{(N-1)}}^{m}$ is a convergent sequence for all
            $m\in\mathbb{Z}_{N}$ where $k\dots{k}^{(N-1)}$ denotes the sequence
            obtain by repeated composition:
            \begin{equation}
                k\dots{k}^{N-1}(n)=k\bigg(k'\Big(\cdots\big(k^{(N-1)}(n)\big)\Big)\cdots\bigg)
            \end{equation}
            But then
            $\mathbf{x}_{k\dots{k}^{(N-1)}}$ is a sequence in $A$ that converges
            to the $\mathbf{y}\in\mathbb{R}^{N}$ defined by
            $\mathbf{y}=(r_{0},\,r_{1},\,\dots,\,r_{N-1})$. But
            $A\subseteq\mathbb{R}^{N}$ is closed and hence contains all of its
            limit points, so $\mathbf{y}\in{A}$. That is, $\mathbf{x}$ has a
            convergent subsequence. Therefore $(A,\,d_{A})$ is compact.
        \end{proof}
        Do not attempt to apply this result to general metric spaces. The
        Heine-Borel theorem is specific to Euclidean spaces with the Euclidean
        metric (or topologically equivalent metrics such as the Manhattan and
        maximum metrics). The discrete metric on any set is bounded. In
        particular, the discrete metric on $\mathbb{R}$ is bounded. Moreover,
        since every subset of a discrete metric space is open, every subset of
        a discrete metric space is closed (being the complement of an open set).
        Any infinite subset of $\mathbb{R}$ with the discrete metric is
        \textit{not} compact. In the discrete metric space $(X,\,d)$ a sequence
        $a:\mathbb{N}\rightarrow{X}$ converges to $x\in{X}$ if and only if
        there is some $N\in\mathbb{N}$ such that $n>N$ implies
        $a_{n}=x$. That is, convergent sequences are eventually constant.
        (To see this, apply the definition of convergence to the positive
        number $\varepsilon=\frac{1}{2}$). So given an infinite subset $A$ of
        $\mathbb{R}$ we can find an injective sequence
        $a:\mathbb{N}\rightarrow{A}$. $A$ is indeed closed and bounded, but
        $a$ has no convergent subsequence. So $A$ is not compact with the
        discrete metric.
\end{document}
