%-----------------------------------LICENSE------------------------------------%
%   This file is part of Mathematics-and-Physics.                              %
%                                                                              %
%   Mathematics-and-Physics is free software: you can redistribute it and/or   %
%   modify it under the terms of the GNU General Public License as             %
%   published by the Free Software Foundation, either version 3 of the         %
%   License, or (at your option) any later version.                            %
%                                                                              %
%   Mathematics-and-Physics is distributed in the hope that it will be useful, %
%   but WITHOUT ANY WARRANTY; without even the implied warranty of             %
%   MERCHANTABILITY or FITNESS FOR A PARTICULAR PURPOSE.  See the              %
%   GNU General Public License for more details.                               %
%                                                                              %
%   You should have received a copy of the GNU General Public License along    %
%   with Mathematics-and-Physics.  If not, see <https://www.gnu.org/licenses/>.%
%------------------------------------------------------------------------------%
\documentclass{article}
\usepackage{graphicx}                           % Needed for figures.
\usepackage{amsmath}                            % Needed for align.
\usepackage{amssymb}                            % Needed for mathbb.
\usepackage{amsthm}                             % For the theorem environment.
\usepackage{float}
\usepackage{tabularx, booktabs}
\usepackage{mathrsfs}
\usepackage[font=scriptsize,
            labelformat=simple,
            labelsep=colon]{subcaption} % Subfigure captions.
\usepackage[font={scriptsize},
            hypcap=true,
            labelsep=colon]{caption}    % Figure captions.
\usepackage{hyperref}
\hypersetup{
    colorlinks=true,
    linkcolor=blue,
    filecolor=magenta,
    urlcolor=Cerulean,
    citecolor=SkyBlue
}

%------------------------Theorem Styles-------------------------%
\theoremstyle{plain}
\newtheorem{theorem}{Theorem}[section]

% Define theorem style for default spacing and normal font.
\newtheoremstyle{normal}
    {\topsep}               % Amount of space above the theorem.
    {\topsep}               % Amount of space below the theorem.
    {}                      % Font used for body of theorem.
    {}                      % Measure of space to indent.
    {\bfseries}             % Font of the header of the theorem.
    {}                      % Punctuation between head and body.
    {.5em}                  % Space after theorem head.
    {}

% Define default environments.
\theoremstyle{normal}
\newtheorem{examplex}{Example}[section]
\newtheorem{definitionx}{Definition}[section]

\newenvironment{example}{%
    \pushQED{\qed}\renewcommand{\qedsymbol}{$\blacksquare$}\examplex%
}{%
    \popQED\endexamplex%
}

\newenvironment{definition}{%
    \pushQED{\qed}\renewcommand{\qedsymbol}{$\blacksquare$}\definitionx%
}{%
    \popQED\enddefinitionx%
}

\title{Point-Set Topology: Lecture 29}
\author{Ryan Maguire}
\date{Summer 2022}

% No indent and no paragraph skip.
\setlength{\parindent}{0em}
\setlength{\parskip}{0em}

\begin{document}
    \maketitle
    \section{Locally Euclidean Topological Spaces}
        We now reach the final topic of the course, \textit{manifolds}. More
        general than manifolds, we start with locally Euclidean spaces.
        \begin{definition}[\textbf{Locally Euclidean Topological Space}]
            A locally Euclidean topological space is a topological space
            $(X,\,\tau)$ such that for all $x\in{X}$ there is a
            $\mathcal{U}\in\tau$ an $n\in\mathbb{N}$, and an injective
            continuous open mapping
            $\varphi:\mathcal{U}\rightarrow\mathbb{R}^{n}$ such that
            $x\in\mathcal{U}$.
        \end{definition}
        \begin{example}
            For all $n\in\mathbb{N}$ the Euclidean space $\mathbb{R}^{n}$ with
            the Euclidean topology is locally Euclidean. For all
            $\mathbf{x}\in\mathbb{R}^{n}$ choose
            $\mathcal{U}=\mathbb{R}^{n}$ and
            $f:\mathbb{R}^{n}\rightarrow\mathbb{R}^{n}$ to be the identity
            mapping $f(\mathbf{x})=\textrm{id}_{\mathbb{R}^{n}}(\mathbf{x})=\mathbf{x}$.
        \end{example}
        \begin{example}
            If $\mathcal{U}\subseteq\mathbb{R}^{n}$ is an open subset with
            respect to the Euclidean topology, then
            $(\mathcal{U},\,\tau_{\mathbb{R}^{n}_{\mathcal{U}}})$ is locally
            Euclidean. Given $x\in\mathcal{U}$ define
            $f:\mathcal{U}\rightarrow\mathbb{R}^{n}$ via
            $f=\textrm{id}_{\mathbb{R}^{n}}|_{\mathcal{U}}$. Then
            $f$ is injective and continuous, and since $\mathcal{U}$ is open,
            $f$ is also an open mapping.
        \end{example}
        \begin{example}
            The solution to $y^{2}-x^{2}=0$ in the plane is not locally
            Euclidean. This forms an \textbf{X}, $y=\pm|x|$. Every point except
            the origin is locally Euclidean, locally looking like
            $\mathbb{R}$. The origin is where this goes wrong. No matter how
            much you zoom in it still locally looks like an \textbf{X}. This is
            certainly not locally like $\mathbb{R}$, but it's also not 2
            dimensional. Similarly, it's not $n$ dimensional for any
            $n\in\mathbb{N}$. Thus this subspace of $\mathbb{R}^{2}$ is not
            locally Euclidean.
        \end{example}
        This example tells us that closed subspaces of locally Euclidean spaces
        do not need to be locally Euclidean.
        \begin{example}
            The bug-eyed line is locally Euclidean, second countable, but not
            Hausdorff. Every point other than the two origins is locally like
            $\mathbb{R}$. The two origins are also locally like $\mathbb{R}$.
            See Fig.~\ref{fig:bug_eyed_line}.
        \end{example}
        \begin{figure}
            \centering
            \includegraphics{../../../images/bug_eyed_line_003.pdf}
            \caption{The Bug-Eyed Line}
            \label{fig:bug_eyed_line}
        \end{figure}
        \begin{example}
            The branching line is another example of a non-Hausdorff space that
            is locally Euclidean. The construction is similar to the bug-eyed
            line. Take $X\subseteq\mathbb{R}^{2}$ to be the set of all points
            of the form $(x,\,y)\in\mathbb{R}^{2}$ such that $x\in\mathbb{R}$
            and $y=\pm{1}$. Define $(x_{0},\,y_{0})R(x_{1},\,y_{1})$ if and
            only if $x_{0}=x_{1}$ and $x_{0}<0$. The branching line is the
            quotient $X/R$ with the quotient topology
            (See Fig.~\ref{fig:branching_line_001}). Like the bug-eyed line,
            it too is locally Euclidean, see Fig.~\ref{fig:branching_line_002}.
        \end{example}
        \begin{figure}
            \centering
            \includegraphics{../../../images/branching_line_001.pdf}
            \caption{The Branching Line Construction}
            \label{fig:branching_line_001}
        \end{figure}
        \begin{figure}
            \centering
            \includegraphics{../../../images/branching_line_002.pdf}
            \caption{The Branching Line is Locally Euclidean}
            \label{fig:branching_line_002}
        \end{figure}
        \begin{example}
            The long line is locally Euclidean and Hausdorff, but not
            second countable. It is also not paracompact.
        \end{example}
        \begin{example}
            As far as set theory is concerned, a function
            $f:A\rightarrow{B}$ from a set $A$ to a set $B$ is a subset of
            $A\times{B}$ satisfying certain properties. We can use this to
            define locally Euclidean topological spaces by looking at
            continuous functions from $\mathbb{R}^{m}$ to $\mathbb{R}^{n}$ for
            some $m,n\in\mathbb{N}$. Given
            $f:\mathbb{R}^{m}\rightarrow\mathbb{R}^{n}$, continuous,
            $f\subseteq\mathbb{R}^{m}\times\mathbb{R}^{n}$ can be given the
            subspace topology. This makes it a closed subset since $f$ is
            continuous. It is also a locally Euclidean subspace. For given
            $\big(\mathbf{x},\,f(\mathbf{x})\big)\in{f}$, let
            $\mathcal{U}=f$ and define $F:f\rightarrow\mathbb{R}^{m}$ via:
            \begin{equation}
                F\big((\mathbf{x},\,f(\mathbf{x})\big)=\mathbf{x}
            \end{equation}
            This is just the projection of the elements of
            $f\subseteq\mathbb{R}^{m}\times\mathbb{R}^{n}$ onto
            $\mathbb{R}^{m}$. Projections are continuous. Let's show
            $F$ is injective and an open mapping. It is injective since given:
            \begin{equation}
                \big(\mathbf{x}_{0},\,f(\mathbf{x}_{0})\big)
                \ne\big(\mathbf{x}_{1},\,f(\mathbf{x}_{1})\big)
            \end{equation}
            we must have $\mathbf{x}_{0}\ne\mathbf{x}_{1}$ (since if
            $\mathbf{x}_{0}=\mathbf{x}_{1}$, then
            $f(\mathbf{x}_{0})=f(\mathbf{x}_{1})$ by definition of a function).
            So then:
            \begin{equation}
                F\big((\mathbf{x}_{0},\,f(\mathbf{x}_{0})\big)
                \ne{F}\big((\mathbf{x}_{1},\,f(\mathbf{x}_{1})\big)
            \end{equation}
            meaning $F$ is injective. There is a continuous inverse
            $F^{-1}:\mathbb{R}^{m}\rightarrow{f}$ given by:
            \begin{equation}
                F^{-1}(\mathbf{x})
                =\big(\mathbf{x},\,f(\mathbf{x})\big)
            \end{equation}
            Since $f$ is continuous, $F^{-1}$ is continuous since both
            components are continuous. So $F$ is an open mapping and $f$
            is a locally Euclidean subspace of
            $\mathbb{R}^{m}\times\mathbb{R}^{n}$.
        \end{example}
        \begin{example}
            $\mathbb{S}^{1}$ with the subspace topology from $\mathbb{R}^{2}$
            is locally Euclidean. We'll show this in two ways. First, via
            orthographic projection. We split the circle into four parts:
            \begin{align}
                \mathcal{U}_{\textrm{North}}
                &=\{\,(x,\,y)\in\mathbb{S}^{1}\;|\;y>0\,\}\\
                \mathcal{U}_{\textrm{South}}
                &=\{\,(x,\,y)\in\mathbb{S}^{1}\;|\;y<0\,\}\\
                \mathcal{U}_{\textrm{East}}
                &=\{\,(x,\,y)\in\mathbb{S}^{1}\;|\;x>0\,\}\\
                \mathcal{U}_{\textrm{West}}
                &=\{\,(x,\,y)\in\mathbb{S}^{1}\;|\;x<0\,\}
            \end{align}
            See Fig.~\ref{fig:circle_is_locally_euclidean_001}.
            Then we define four functions:
            \begin{align}
                \varphi_{\textrm{North}}:
                \mathcal{U}_{\textrm{North}}\rightarrow\mathbb{R}
                \quad\quad
                \varphi_{\textrm{North}}\big((x,\,y)\big)&=x\\
                \varphi_{\textrm{South}}:
                \mathcal{U}_{\textrm{South}}\rightarrow\mathbb{R}
                \quad\quad
                \varphi_{\textrm{South}}\big((x,\,y)\big)&=x\\
                \varphi_{\textrm{East}}:
                \mathcal{U}_{\textrm{East}}\rightarrow\mathbb{R}
                \quad\quad
                \varphi_{\textrm{East}}\big((x,\,y)\big)&=y\\
                \varphi_{\textrm{West}}:
                \mathcal{U}_{\textrm{West}}\rightarrow\mathbb{R}
                \quad\quad
                \varphi_{\textrm{West}}\big((x,\,y)\big)&=y
            \end{align}
            Since these are projection mappings, they are continuous. From
            how the four open sets are defined, each is also injective. To
            show it is an open mapping we just need to find a continuous
            inverse with respect to the image of these sets. Note that for all
            four functions the range of $(-1,\,1)$. We have the following
            inverse functions:
            \begin{align}
                \varphi_{\textrm{North}}^{-1}(x)
                &=\big(x,\,\sqrt{1-x^{2}}\big)\\
                \varphi_{\textrm{South}}^{-1}(x)
                &=\big(x,\,-\sqrt{1-x^{2}}\big)\\
                \varphi_{\textrm{East}}^{-1}(y)
                &=\big(\sqrt{1-y^{2}},\,y\big)\\
                \varphi_{\textrm{West}}^{-1}(y)
                &=\big(-\sqrt{1-y^{2}},\,y\big)
            \end{align}
            each of which is continuous since the square root function is
            continuous. The four sets also cover $\mathbb{S}^{1}$, showing that
            $\mathbb{S}^{1}$ is locally Euclidean.
        \end{example}
        \begin{figure}
            \centering
            \includegraphics{../../../images/circle_is_locally_euclidean_001.pdf}
            \caption{Cover of $\mathbb{S}^{1}$ with Locally Euclidean Sets}
            \label{fig:circle_is_locally_euclidean_001}
        \end{figure}
        This shows we can cover $\mathbb{S}^{1}$ using four sets each of which
        is homeomorphic to an open subset of $\mathbb{R}$. We can do better,
        only two sets are needed. Place an observer at the north pole
        $N=(0,\,1)$. Given any other point $(x,\,y)$ the line from the
        observer to the point is not parallel to the $x$ axis, meaning
        eventually it must intersect it. Let's solve for when. The line segment
        $\alpha(t)=(1-t)N+t(x,\,y)$ starts at the north pole at time $t=0$ and
        ends at the point $(x,\,y)$ on the circle at time $t=1$. The line
        intersects the $x$ axis when the $y$ component is zero. Thus we wish
        to solve $1-t+ty=0$ for $t$. We get:
        \begin{equation}
            t_{0}=\frac{1}{1-y}
        \end{equation}
        The $x$ coordinate at time $t=t_{0}$ is then:
        \begin{equation}
            \varphi_{N}\big((x,\,y)\big)
            =\frac{x}{1-y}
        \end{equation}
        This is \textit{stereographic projection} about the north pole. It is
        continuous since it is a rational function. It is also bijective with a
        continuous inverse. Given $X\in\mathbb{R}$ we can solve for the value
        $(x,\,y)\in\mathbb{S}^{1}$ that gets mapped to $X$ by reversing the
        previous process. The line
        $\beta(t)=(1-t)N+t(X,\,0)$ starts at the north pole and ends at
        $(X,\,0)$. We wish to solve for the time $t$ when
        $||\beta(t)||_{2}=1$ which corresponds to the moment the line
        intersects the circle. We have:
        \begin{align}
            ||\beta(t)||_{2}
            &=||(1-t)N+t(X,\,0)||_{2}\\
            &=||(1-t)(0,\,1)+t(X,\,0)||_{2}\\
            &=||(tX,\,1-t)||_{2}\\
            &=\sqrt{(tX)^{2}+(1-t)^{2}}
        \end{align}
        Solving for $||\beta(t)||_{2}=1$ is equivalent to solving
        $||\beta(t)||_{2}^{2}=1$ so we need to consider the expression
        $(tX)^{2}+(1-t)^{2}$. We get:
        \begin{align}
            1&=(tX)^{2}+(1-t)^{2}\\
            &=t^{2}X^{2}+1-2t+t^{2}\\
            &=t^{2}(1+X^{2})-2t+1
        \end{align}
        meaning we want to solve for $t^{2}(1+X^{2})-2t=0$. The solution
        $t=0$ corresponds to the North pole, which is not the one we want.
        Dividing through by $t$ we get:
        \begin{equation}
            t_{1}=\frac{2}{1+X^{2}}
        \end{equation}
        The point $(x,\,y)$ corresponds to $\beta(t_{1})$ and is given by:
        \begin{equation}
            \varphi_{N}^{-1}(X)=
            \Big(\frac{2X}{1+X^{2}},\,\frac{-1+X^{2}}{1+X^{2}}\Big)
        \end{equation}
        This function is continuous since it is a rational function in each
        component. Because of this
        $\varphi_{N}:\mathbb{S}^{1}\setminus\{\,(0,\,1)\,\}\rightarrow\mathbb{R}$
        is a homeomorphism. Doing a similar projection about the south pole
        shows that $\mathbb{S}^{1}$ can be covered by two open sets,
        $\mathbb{S}^{1}\setminus\{\,(0,\,1)\,\}$ and
        $\mathbb{S}^{1}\setminus\{\,(0,\,-1)\,\}$, each of which is
        homeomorphic to $\mathbb{R}$.
        \par\hfill\par
        It is impossible to do this with one set. This is because
        $\mathbb{S}^{1}$ is not homeomorphic to an open subset of
        $\mathbb{R}$ since $\mathbb{S}^{1}$ is compact and the only open subset
        of $\mathbb{R}$ that is compact is the empty set, but
        $\mathbb{S}^{1}$ is non-empty. So two is the best we can do.
        \begin{example}
            The sphere $\mathbb{S}^{n}\subseteq\mathbb{R}^{n+1}$ is also
            locally Euclidean for all $n\in\mathbb{N}$. Define
            $\mathcal{U}_{k}^{\pm}\subseteq\mathbb{S}^{n}$ via:
            \begin{align}
                \mathcal{U}_{k}^{+}
                &=\{\,\mathbf{x}\in\mathbb{S}^{n}\;|\;
                    \mathbf{x}_{k}>0\,\}\\
                \mathcal{U}_{k}^{-}
                &=\{\,\mathbf{x}\in\mathbb{S}^{n}\;|\;
                    \mathbf{x}_{k}<0\,\}
            \end{align}
            These $2n+2$ open sets cover $\mathbb{S}^{n}$ and each is
            homeomorphic to an open subset of $\mathbb{R}^{n}$. Define
            $\varphi_{k}^{\pm}:\mathcal{U}_{k}^{\pm}\rightarrow{B}_{1}^{\mathbb{R}^{n}}(\mathbf{0})$
            via:
            \begin{equation}
                \varphi_{k}^{\pm}(\mathbf{x})=
                (\mathbf{x}_{0},\,\dots,\,\mathbf{x}_{k-1},\,
                    \mathbf{x}_{k+1},\,\mathbf{x}_{n})
            \end{equation}
            That is, projecting down that $k^{th}$ axis. This is continuous
            with a continuous inverse
            ${\varphi_{k}^{\pm}}^{-1}:B_{1}^{\mathbb{R}^{n}}(\mathbf{0})\rightarrow\mathcal{U}_{k}^{\pm}$
            given by:
            \begin{equation}
                {\varphi_{k}^{\pm}}^{-1}(\mathbf{x})
                =(\mathbf{x}_{0},\,\dots,\,\mathbf{x}_{k-1},\,
                    \pm\sqrt{1-||\mathbf{x}||_{2}^{2}},\,\mathbf{x}_{k},\,
                    \dots,\,\mathbf{x}_{n-1})
            \end{equation}
            This is also continuous, so $\mathbb{S}^{n}$ is locally Euclidean.
        \end{example}
        These mappings are called \textit{orthographic projections}. They are
        formed by placing an observer at \textit{infinity} and projecting what
        they see down to the plane. This is shown in
        Fig.~\ref{fig:orthographic_projection_001}
        \begin{figure}
            \centering
            \includegraphics{../../../images/sphere_orthographic_projection_001.pdf}
            \caption{Orthographic Projection of the Sphere}
            \label{fig:orthographic_projection_001}
        \end{figure}
        \begin{definition}[\textbf{Topological Chart}]
            A topological chart of dimension $n$ in a topological space
            $(X,\,\tau)$ about a point $x\in{X}$ is an ordered pair
            $(\mathcal{U},\,\varphi)$ such that $\mathcal{U}\in\tau$,
            $x\in\mathcal{U}$, and
            $\varphi:\mathcal{U}\rightarrow\mathbb{R}^{n}$ is an injective
            continuous open mapping.
        \end{definition}
        Locally Euclidean could equivalently be described as a topological space
        $(X,\,\tau)$ such that for all $x\in{X}$ there is a chart
        $(\mathcal{U},\,\varphi)$ such that $x\in\mathcal{U}$. A collection of
        charts that covers a space is called an \textit{atlas}.
        \begin{definition}[\textbf{Topological Atlas}]
            A topological atlas for a topological space $(X,\,\tau)$ is a
            set $\mathcal{A}$ of topological charts in $(X,\,\tau)$ such that
            for all $x\in{X}$ there is a $(\mathcal{U},\,\varphi)\in\mathcal{A}$
            such that $x\in\mathcal{U}$.
        \end{definition}
        \begin{figure}
            \centering
            \includegraphics{../../../images/sphere_stereographic_projection.pdf}
            \caption{Stereographic Projection for the Sphere}
            \label{fig:sphere_stereographic_projection}
        \end{figure}
        That is, an atlas is a collection of charts whose domains cover the
        space. Think of an actual atlas used for navigating. The pages consist
        of various locations on the globe, but only provides local information.
        To get information that is more global requires piecing some of the
        charts of the atlas together. A locally Euclidean space is a topological
        space $(X,\,\tau)$ such that there exists an atlas $\mathcal{A}$ for it.
        We've shown that $\mathbb{S}^{n}$ can be covered by $2n+2$ charts using
        orthographic projection. We can do better using stereographic projection
        the same way we did for $\mathbb{S}^{1}$. This is shown for
        $\mathbb{S}^{2}$ in Fig.~\ref{fig:sphere_stereographic_projection}.
        \par\hfill\par
        \begin{figure}
            \centering
            \includegraphics{../../../images/sphere_near_sided_projection.pdf}
            \caption{Near-Sided Projection of the Sphere}
            \label{fig:sphere_near_sided_projection}
        \end{figure}
        There are two other types of projections that are useful for geometric
        reasons in covering $\mathbb{S}^{n}$. These are the near-sided and
        far-sided projections. Near-sided projection is shown in
        Fig.~\ref{fig:sphere_near_sided_projection}. The idea is to take an
        observer and place them somewhere on the $z$ axis above the sphere.
        The portion of the sphere that is visible is then projected down to the
        $xy$ plane. Far-sided projection is the opposite. You place the
        observer at the same spot but remove everything that can be seen.
        The result is a hollow semi-sphere. You then unwrap this on to the
        plane to get the projection. This is shown in
        Fig.~\ref{fig:sphere_far_sided_projection}. Stereographic projection
        is then just far-sided projection at the north pole, and orthographic
        projection is far-sided projection at infinity.
        \begin{figure}
            \centering
            \includegraphics{../../../images/sphere_far_sided_projection.pdf}
            \caption{Far-Sided Projection of the Sphere}
            \label{fig:sphere_far_sided_projection}
        \end{figure}
\end{document}
