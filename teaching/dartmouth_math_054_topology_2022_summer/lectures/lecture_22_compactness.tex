%-----------------------------------LICENSE------------------------------------%
%   This file is part of Mathematics-and-Physics.                              %
%                                                                              %
%   Mathematics-and-Physics is free software: you can redistribute it and/or   %
%   modify it under the terms of the GNU General Public License as             %
%   published by the Free Software Foundation, either version 3 of the         %
%   License, or (at your option) any later version.                            %
%                                                                              %
%   Mathematics-and-Physics is distributed in the hope that it will be useful, %
%   but WITHOUT ANY WARRANTY; without even the implied warranty of             %
%   MERCHANTABILITY or FITNESS FOR A PARTICULAR PURPOSE.  See the              %
%   GNU General Public License for more details.                               %
%                                                                              %
%   You should have received a copy of the GNU General Public License along    %
%   with Mathematics-and-Physics.  If not, see <https://www.gnu.org/licenses/>.%
%------------------------------------------------------------------------------%
\documentclass{article}
\usepackage{amsmath}                            % Needed for align.
\usepackage{amssymb}                            % Needed for mathbb.
\usepackage{amsthm}                             % For the theorem environment.
\usepackage{mathrsfs}

%------------------------Theorem Styles-------------------------%
\theoremstyle{plain}
\newtheorem{theorem}{Theorem}[section]

% Define theorem style for default spacing and normal font.
\newtheoremstyle{normal}
    {\topsep}               % Amount of space above the theorem.
    {\topsep}               % Amount of space below the theorem.
    {}                      % Font used for body of theorem.
    {}                      % Measure of space to indent.
    {\bfseries}             % Font of the header of the theorem.
    {}                      % Punctuation between head and body.
    {.5em}                  % Space after theorem head.
    {}

% Define default environments.
\theoremstyle{normal}
\newtheorem{examplex}{Example}[section]
\newtheorem{definitionx}{Definition}[section]

\newenvironment{example}{%
    \pushQED{\qed}\renewcommand{\qedsymbol}{$\blacksquare$}\examplex%
}{%
    \popQED\endexamplex%
}

\newenvironment{definition}{%
    \pushQED{\qed}\renewcommand{\qedsymbol}{$\blacksquare$}\definitionx%
}{%
    \popQED\enddefinitionx%
}

\title{Point-Set Topology: Lecture 22}
\author{Ryan Maguire}
\date{\today}

% No indent and no paragraph skip.
\setlength{\parindent}{0em}
\setlength{\parskip}{0em}

\begin{document}
    \maketitle
    \section{Compactness}
        For a course in point-set topology, if you understand the general
        notions (topological spaces, continuity, Hausdorffness, sequentialness),
        the basis properties (first and second countable), creating new spaces
        (products, subspaces, quotients), the separation ideas
        (regular and normal), connectedness, and compactness, then you have an
        absolutely solid understanding of general topology. We've covered all
        of these ideas except compactness, which we've only discussed in the
        context of metric spaces (or \textit{metrizable} spaces). We now take
        the time to study compactness in the general topological setting.
        \par\hfill\par
        In a metric space we proved several theorems about compactness,
        primarily the Bolzano-Weierstrass, Heine-Borel, generalized
        Heine-Borel, and equivalence of compactness theorems. This told us
        that compactness can be described by sequences and by open sets. In the
        topological setting it is more natural to define compactness via open
        sets.
        \begin{definition}[\textbf{Compact Topological Space}]
            A compact topological space is a topological space $(X,\,\tau)$
            such that for all open covers $\mathcal{O}\subseteq\tau$ there is
            a finite subset $\Delta\subseteq\mathcal{O}$ such that $\Delta$ is
            open cover.
        \end{definition}
        We have spent a lot of time on compactness in the setting of metric
        spaces. Let's not waste that time, and copy over some of the theorems
        but rephrase them for \textit{metrizable} spaces.
        \begin{theorem}
            If $(X,\,\tau)$ is a metrizable topological space, then it is
            compact if and only if for all metrics $d$ on $X$ that induce
            $\tau$, $(X,\,d)$ is a compact metric space.
        \end{theorem}
        \begin{proof}
            By the equivalence of compactness theorem, any metric $d$ that
            induces $\tau$ has the property that any open cover of open sets
            in the metric space $(X,\,d)$ has a finite open subcover, which is
            precisely the definition of compactness in the topological setting.
        \end{proof}
        \begin{theorem}
            If $(X,\,\tau)$ is a metrizable topological space, then it is
            compact if and only if for every metric $d$ on $X$ that induces
            $\tau$, $(X,\,d)$ is complete and totally bounded.
        \end{theorem}
        \begin{proof}
            This follows from the previous theorem and the generalized
            Heine-Borel theorem.
        \end{proof}
        \begin{theorem}
            If $(\mathbb{R},\,\tau_{\mathbb{R}})$ is the standard Euclidean
            line, and if $A\subseteq\mathbb{R}$, then
            $(A,\,\tau_{\mathbb{R}_{A}})$ is compact if and only if $A$ is
            closed and bounded.
        \end{theorem}
        \begin{proof}
            The Euclidean topology on $\mathbb{R}$,
            $\tau_{\mathbb{R}}$, is induced by the Euclidean metric
            $d(x,\,y)=|x-y|$. The result then follows from the Heine-Borel
            theorem.
        \end{proof}
        \begin{theorem}
            If $(\mathbb{R}^{n},\,\tau_{\mathbb{R}^{n}})$ is Euclidean space,
            and if $A\subseteq\mathbb{R}^{n}$, then
            $(A,\,\tau_{\mathbb{R}^{n}_{A}})$ is compact if and only if $A$ is
            closed and bounded.
        \end{theorem}
        \begin{proof}
            This too follows from the Heine-Borel theorem.
        \end{proof}
        This now gives us plenty of familiar spaces that are compact.
        Lacking a metrizable space, there are still plenty of pleasing
        properties compact topologies yield.
        \begin{theorem}
            If $(X,\,\tau)$ is a compact topological space, and if
            $\mathcal{C}\subseteq{X}$ is closed, then
            $(\mathcal{C},\,\tau_{\mathcal{C}})$ is compact where
            $\tau_{\mathcal{C}}$ is the subspace topology.
        \end{theorem}
        \begin{proof}
            For if not then there is an open cover $\mathcal{O}_{\mathcal{C}}$
            of $(\mathcal{C},\,\tau_{\mathcal{C}})$ with no finite subcover.
            But by the definition of the subspace topology, for all
            $\mathcal{U}\in\mathcal{O}_{\mathcal{C}}$
            there is an open set $\tilde{\mathcal{U}}\in\tau$ such that
            $\mathcal{U}=\mathcal{C}\cap\tilde{\mathcal{U}}$. Let
            $\mathcal{O}_{X}\subseteq\tau$ be defined by:
            \begin{equation}
                \mathcal{O}_{X}
                =\{\,\tilde{\mathcal{U}}\in\tau\;|\;
                    \mathcal{U}\in\mathcal{O}_{\mathcal{C}}\,\}
            \end{equation}
            $\mathcal{O}_{X}$ covers $\mathcal{C}$ with elements of $\tau$, but
            it need not cover all of $X$. However, since $\mathcal{C}$ is
            closed, $X\setminus\mathcal{C}$ is open. Let
            $\mathcal{O}\subseteq\tau$ be defined by:
            \begin{equation}
                \mathcal{O}=\mathcal{O}_{X}\cup\{\,X\setminus\mathcal{C}\,\}
            \end{equation}
            Then $\mathcal{O}\subseteq\tau$ is an open cover of $X$, and since
            $(X,\,\tau)$ is compact there is a finite subcover $\Delta$.
            Define $\Delta_{X}=\Delta\setminus\{\,X\setminus\mathcal{C}\,\}$.
            Then, by definition of $\mathcal{O}$ and $\mathcal{O}_{X}$,
            $\Delta_{X}\subseteq\mathcal{O}_{X}$. But also $\Delta_{X}$ is
            finite. But since $\Delta$ covers $X$ and $X\setminus\mathcal{C}$
            is disjoint from $\mathcal{C}$, $\Delta_{X}$ must cover
            $\mathcal{C}$ as well. Define $\Delta_{\mathcal{C}}$ via:
            \begin{equation}
                \Delta_{\mathcal{C}}=
                \{\,\mathcal{U}\cap\mathcal{C}\;|\;\mathcal{U}\in\Delta_{X}\,\}
            \end{equation}
            By definition of $\mathcal{O}_{X}$ and $\Delta_{X}$ we have that
            $\Delta_{\mathcal{C}}\subseteq\mathcal{O}_{\mathcal{C}}$. But since
            $\Delta_{X}$ covers $\mathcal{C}$, so does $\Delta_{\mathcal{C}}$.
            But then $\Delta_{\mathcal{C}}\subseteq\mathcal{O}_{\mathcal{C}}$
            is a finite subset that still covers $\mathcal{C}$, a contradiction.
            Hence, $(\mathcal{C},\,\tau_{\mathcal{C}})$ is compact.
        \end{proof}
        This theorem does not need to reverse, in general. That is, compact
        subsets don't need to be closed (but in metric spaces they are). Give
        $\mathbb{R}$ the indiscrete topology
        $\tau=\{\,\emptyset,\,\mathbb{R}\,\}$. Then every subset
        $A\subseteq\mathbb{R}$ is compact since the only open covers possible
        are finite (they have at most two subsets). However only
        $\emptyset$ and $\mathbb{R}$ are closed. If we add the Hausdorff
        condition, then compact subspaces are closed.
        \par\hfill\par
        Before proving this, it was quite annoying dealing with the subspace
        topology in the previous theorem. It feels unnecessary. With compact
        subspaces, if we can cover the subspace with sets that are open in the
        \textit{ambient space}, then there is a finite subcover of this as
        well. Let's prove this.
        \begin{theorem}
            If $(X,\,\tau)$ is a topological space, if $A\subseteq{X}$, and
            if $(A,\,\tau_{A})$ is compact, where $\tau_{A}$ is the subspace
            topology, then for all $\mathcal{O}\subseteq\tau$ such that
            $A\subseteq\bigcup\mathcal{O}$, there is a finite subset
            $\Delta\subseteq\mathcal{O}$ such that $A\subseteq\bigcup\Delta$.
        \end{theorem}
        \begin{proof}
            If not, then there is a subset $\mathcal{O}\subseteq\tau$ such
            that $A\subseteq\bigcup\mathcal{O}$ but with no finite subset that
            still covers $A$. Let $\tilde{\mathcal{O}}$ be defined by:
            \begin{equation}
                \tilde{\mathcal{O}}
                =\{\,\mathcal{U}\cap{A}\;|\;A\in\mathcal{O}\,\}
            \end{equation}
            By definition of the subspace topology,
            $\tilde{\mathcal{O}}\subseteq\tau_{A}$. But
            since $A\subseteq\bigcup\mathcal{O}$, we have
            $A=\bigcup\tilde{\mathcal{O}}$. But $(A,\,\tau_{A})$ is compact,
            so there is a finite subcover
            $\tilde{\Delta}\subseteq\tilde{\mathcal{O}}$. Since it is finite we
            may label it:
            \begin{equation}
                \tilde{\Delta}=
                \{\,\mathcal{U}_{0}\cap{A},\,\dots,\,\mathcal{U}_{n}\cap{A}\,\}
            \end{equation}
            Define $\Delta\subseteq\mathcal{O}$ via:
            \begin{equation}
                \Delta=
                \{\,\mathcal{U}_{0},\,\dots,\,\mathcal{U}_{n}\,\}
            \end{equation}
            Then we have:
            \begin{equation}
                A=\bigcup_{k=0}^{n}\big(\mathcal{U}_{k}\cap{A}\big)
                =\Big(\bigcup_{k=0}^{n}\mathcal{U}_{n}\Big)\cap{A}
                \subseteq\bigcup_{k=0}^{n}\mathcal{U}_{n}
            \end{equation}
            So $\Delta\subseteq\mathcal{O}$ is a finite subset of $\mathcal{O}$
            that covers $A$, which is a contradiction. Hence, for any
            $\mathcal{O}\subseteq\tau$ such that $A\subseteq\bigcup\mathcal{O}$,
            there is a finite subset $\Delta\subseteq\mathcal{O}$ such that
            $A\subseteq\bigcup\Delta$.
        \end{proof}
        Now when dealing with compact subspaces we can restrict our attention
        to open sets in the ambient space, which is often easier.
        \begin{theorem}
            If $(X,\,\tau)$ is a Hausdorff topological space, if
            $A\subseteq{X}$, and if $(A,\,\tau_{A})$ is compact, where
            $\tau_{A}$ is the subspace topology, then $A$ is closed. 
        \end{theorem}
        \begin{proof}
            If $A=\emptyset$, then there is nothing to prove since $\emptyset$
            is closed. Suppose $A\ne\emptyset$. If $A$ is not closed, then
            $X\setminus{A}$ is not open, and hence there is an
            $x\in{X}\setminus{A}$ such that for all $\mathcal{U}\in\tau$ with
            $x\in\mathcal{U}$ it is not true that
            $\mathcal{U}\subseteq{X}\setminus{A}$ (otherwise we could write
            $X\setminus{A}$ as the union of all such $\mathcal{U}$ for all
            $x\in{X}\setminus{A}$, showing that $X\setminus{A}$ is the union of
            open sets, which is therefore open). But then for all $y\in{A}$,
            since $(X,\,\tau)$ is Hausdorff, there
            exist open sets $\mathcal{U}_{y},\mathcal{V}_{y}$ such that
            $x\in\mathcal{U}_{y}$, $y\in\mathcal{V}_{y}$, and
            $\mathcal{U}_{y}\cap\mathcal{V}_{y}=\emptyset$. But the collection
            of all such $\mathcal{V}_{y}$ cover $A$, and since
            $(A,\,\tau_{A})$ is compact, there is a finite subcover. Label the
            elements of the finite subcover as
            $\mathcal{V}_{0},\,\dots,\,\mathcal{V}_{n}$. Label the corresponding
            open sets around $x$ as $\mathcal{U}_{0},\,\dots,\,\mathcal{U}_{n}$.
            Define $\tilde{\mathcal{U}}$ via:
            \begin{equation}
                \tilde{\mathcal{U}}=\bigcap_{k=0}^{n}\mathcal{U}_{k}
            \end{equation}
            Then $\tilde{\mathcal{U}}$ is open, being the intersection of
            finitely many open sets, and
            $x\in\tilde{\mathcal{U}}$ since $x\in\mathcal{U}_{k}$
            for all $k$. But $\tilde{\mathcal{U}}$ is disjoint from
            $A$. For if $y\in{A}$ and $y\in\tilde{\mathcal{U}}$, since
            $\mathcal{V}_{0},\,\dots,\,\mathcal{V}_{n}$ cover $A$ there is some
            $0\leq{k}\leq{n}$ such that $y\in\mathcal{V}_{k}$. But then:
            \begin{equation}
                \tilde{\mathcal{U}}\cap\mathcal{V}_{k}
                \subseteq\mathcal{U}_{k}\cap\mathcal{V}_{k}
                =\emptyset
            \end{equation}
            a contradiction, so $\tilde{\mathcal{U}}$ and $A$ are disjoint.
            But then $\tilde{\mathcal{U}}$ is an open set such that
            $x\in\tilde{\mathcal{U}}$ and
            $\tilde{\mathcal{U}}\subseteq{X}\setminus{A}$, which is a
            contradiction. Hence, $A$ is closed.
        \end{proof}
        \begin{theorem}
            If $(X,\,\tau_{X})$ is a compact topological space, if
            $(Y,\,\tau_{Y})$ is a topological space, and if
            $f:X\rightarrow{Y}$ is continuous, then
            $(f[X],\,\tau_{Y_{f[X]}})$ is compact where
            $\tau_{Y_{f[X]}}$ is the subspace topology.
        \end{theorem}
        \begin{proof}
            Suppose not and let $\mathcal{O}$ be an open cover of $f[X]$ with
            no finite subcover. Then for all
            $\mathcal{V}\in\mathcal{O}$, by the definition of the subspace
            topology, there is an open $\tilde{\mathcal{V}}\in\tau_{Y}$ such
            that $\mathcal{V}=\tilde{\mathcal{V}}\cap{f}[X]$.
            Define $\tilde{\mathcal{O}}$ via:
            \begin{equation}
                \tilde{\mathcal{O}}
                =\{\,\tilde{\mathcal{V}}\;|\;\mathcal{V}\in\mathcal{O}\,\}
            \end{equation}
            Then $\tilde{\mathcal{O}}$ is a collection of open sets in $Y$ that
            cover $f[X]$. Since $f$ is continuous, for all
            $\tilde{\mathcal{V}}\in\tilde{\mathcal{O}}$ the set
            $f^{-1}[\tilde{\mathcal{V}}]$ is open in $X$. But
            $\tilde{\mathcal{O}}$ covers $f[X]$, and hence the set:
            \begin{equation}
                \mathscr{O}
                =\{\,f^{-1}[\tilde{\mathcal{V}}]\;|\;
                    \tilde{\mathcal{V}}\in\tilde{\mathcal{O}}\,\}
            \end{equation}
            is an open cover of $(X,\,\tau_{X})$. But $(X,\,\tau_{X})$ is
            compact, so there is a finite subcover
            $\mathscr{D}\subseteq\mathscr{O}$. Form the set
            $\tilde{\Delta}\subseteq\tilde{\mathcal{O}}$ by choosing a single
            element $\tilde{\mathcal{V}}\in\tilde{\mathcal{O}}$ for each
            $\mathcal{U}\in\mathscr{D}$ such that
            $\mathcal{U}=f^{-1}[\tilde{\mathcal{V}}]$. Then
            $\tilde{\Delta}\subseteq\tilde{\mathcal{O}}$ is a finite subset that
            covers $f[X]$. But then the set $\Delta$ of sets of the form
            $\tilde{\mathcal{V}}\cap{f}[X]$ for all
            $\tilde{\mathcal{V}}\in\tilde{\Delta}$ is a finite subset of
            $\mathcal{O}$ that covers $f[X]$, a contradiction.
            Hence, $(f[X],\,\tau_{Y_{f[X]}})$ is compact.
        \end{proof}
        \begin{theorem}
            if $(X,\,\tau_{X})$ is a compact topological space, if
            $(Y,\,\tau_{Y})$ is a Hausdorff topological space, and if
            $f:X\rightarrow{Y}$ is continuous and bijective, then $f$ is a
            homeomorphism.
        \end{theorem}
        \begin{proof}
            It suffices to show that $f$ is a closed mapping since $f$ is a
            homeomorphism if and only if it is bijective, continuous, and a
            closed mapping. Since $f$ is bijective and continuous by
            hypothesis, we need only show it is also a closed mapping.
            Let $\mathcal{C}\subseteq{X}$ be closed. But $(X,\,\tau_{X})$ is
            compact and $\mathcal{C}$ is closed, so
            $(\mathcal{C},\,\tau_{X_{\mathcal{C}}})$ is compact. But then, since
            $f$ is continuous, $f[\mathcal{C}]\subseteq{Y}$ is a compact
            subspace. But $(Y,\,\tau_{Y})$ is Hausdorff, so $f[\mathcal{C}]$ is
            closed. Hence, $f$ is a closed mapping, so it is a homeomorphism.
        \end{proof}
        \begin{theorem}
            If $(X,\,\tau)$ is a compact Hausdorff space, then it is regular.
        \end{theorem}
        \begin{proof}
            Let $x\in{X}$, $\mathcal{C}\subseteq{X}$ be closed, and
            $x\notin\mathcal{C}$. We must find open subsets
            $\mathcal{U},\mathcal{V}\in\tau$ such that $x\in\mathcal{U}$,
            $\mathcal{C}\subseteq\mathcal{V}$, and
            $\mathcal{U}\cap\mathcal{V}=\emptyset$. Since $x\notin\mathcal{C}$,
            for all $y\in\mathcal{C}$ we have $x\ne{y}$. But $(X,\,\tau)$ is
            Hausdorff so for all $y\in\mathcal{C}$ there are open sets
            $\mathcal{U}_{y},\,\mathcal{V}_{y}$ such that $x\in\mathcal{U}_{y}$,
            $y\in\mathcal{V}_{y}$, and
            $\mathcal{U}_{y}\cap\mathcal{V}_{y}=\emptyset$. But then the
            set:
            \begin{equation}
                \mathcal{O}=\{\,\mathcal{V}_{y}\;|\;y\in\mathcal{C}\,\}
            \end{equation}
            is a collection of open sets that cover $\mathcal{C}$. But
            $(X,\,\tau)$ is compact, and $\mathcal{C}$ is a closed subset,
            meaning there is a finite subset $\Delta\subseteq\mathcal{O}$ that
            covers $\mathcal{C}$. Label the sets as:
            \begin{equation}
                \Delta=\{\,\mathcal{V}_{0},\,\dots,\,\mathcal{V}_{N}\,\}
            \end{equation}
            Label the corresponding open sets around $x$ similarly:
            \begin{equation}
                \Lambda=\{\,\mathcal{U}_{0},\,\dots,\,\mathcal{U}_{N}\,\}
            \end{equation}
            Define:
            \begin{equation}
                \mathcal{W}=\bigcap\Lambda=\bigcap_{k=0}^{N}\mathcal{U}_{k}
            \end{equation}
            Then $\mathcal{W}$ is open, being the intersection of finitely
            many open sets, and $x\in\mathcal{W}$ since
            $x\in\mathcal{U}_{k}$ for each $k$. Furthermore, define:
            \begin{equation}
                \mathcal{E}=\bigcup\Delta=\bigcup_{k=0}^{N}\mathcal{V}_{k}
            \end{equation}
            Since $\Delta$ covers $\mathcal{C}$, we have
            $\mathcal{C}\subseteq\mathcal{E}$. Morever, $\mathcal{E}$ is open,
            being the union of open sets. So $x\in\mathcal{W}$ and
            $\mathcal{C}\subseteq\mathcal{E}$, and both $\mathcal{W}$ and
            $\mathcal{E}$ are open. We conclude by showing that
            $\mathcal{W}\cap\mathcal{E}=\emptyset$.
            By definition of $\mathcal{W}$ we have that
            $\mathcal{W}\subseteq\mathcal{U}_{k}$ for each $k$.
            But $\mathcal{U}_{k}\cap\mathcal{V}_{k}=\emptyset$. So
            $\mathcal{W}\cap\mathcal{V}_{k}=\emptyset$ for all $k$, and hence
            $\mathcal{W}\cap\mathcal{E}=\emptyset$. So $(X,\,\tau)$ is regular.
        \end{proof}
        \begin{theorem}
            If $(X,\,\tau)$ is a compact Hausdorff space, then it is normal.
        \end{theorem}
        \begin{proof}
            Let $\mathcal{C}$ and $\mathcal{D}$ be closed disjoint subsets of
            $X$. If one of them is empty, we may choose $\mathcal{U}=\emptyset$
            and $\mathcal{V}=X$. So suppose neither are empty. By the previous
            theorem, a compact Hausdorff space is regular. Hence for all
            $x\in\mathcal{C}$, since $x\notin\mathcal{D}$, we have that there
            are open disjoint sets $\mathcal{U}_{x},\mathcal{V}_{x}\in\tau$
            such that $x\in\mathcal{U}_{x}$,
            $\mathcal{C}\subseteq\mathcal{V}_{x}$, and
            $\mathcal{U}_{x}\cap\mathcal{V}_{x}=\emptyset$. The collection:
            \begin{equation}
                \mathcal{O}
                    =\{\,\mathcal{U}_{x}\;|\;x\in\mathcal{C}\,\}\quad
            \end{equation}
            is a collection of open sets that cover $\mathcal{C}$. Since
            $(X,\,\tau)$ is compact and $\mathcal{C}\subseteq{X}$ is closed
            there is a finite subcover $\Delta\subseteq\mathcal{O}$ of
            $\mathcal{C}$. Label the elements as:
            \begin{equation}
                \Delta=\{\,\mathcal{U}_{0},\,\dots,\,\mathcal{U}_{N}\,\}
            \end{equation}
            label the corresponding open sets around $\mathcal{D}$ as well:
            \begin{equation}
                \Lambda=\{\,\mathcal{V}_{0},\,\dots,\,\mathcal{V}_{N}\,\}
            \end{equation}
            Define:
            \begin{equation}
                \mathcal{W}=\bigcap\Lambda=\bigcap_{k=0}^{N}\mathcal{V}_{k}
            \end{equation}
            Then $\mathcal{W}$ is open, being the intersection of finitely
            many open sets, and $\mathcal{D}\subseteq\mathcal{W}$ since
            $\mathcal{D}\subseteq\mathcal{V}_{k}$ for each $k$.
            Furthermore, define:
            \begin{equation}
                \mathcal{E}=\bigcup\Delta=\bigcup_{k=0}^{N}\mathcal{U}_{k}
            \end{equation}
            Then $\mathcal{E}$ is open, being the union of open sets, and
            $\mathcal{C}\subseteq\mathcal{E}$ since $\Delta$ is an open cover
            of $\mathcal{C}$. Finally,
            $\mathcal{W}\cap\mathcal{E}=\emptyset$ since
            $\mathcal{U}_{k}\cap\mathcal{V}_{k}=\emptyset$ for all $k$, and
            hence $\mathcal{W}\cap\mathcal{V}_{k}=\emptyset$ as well, meaning
            $\mathcal{W}\cap\mathcal{E}=\emptyset$. Therefore $(X,\,\tau)$
            is normal.
        \end{proof}
        \begin{definition}[\textbf{Sequentially Compact Topological Space}]
            A sequentially compact topological space is a topological space
            $(X,\,\tau)$ such that for every sequence
            $a:\mathbb{N}\rightarrow{X}$ there is a convergent subsequence
            $a_{k}$.
        \end{definition}
        \begin{theorem}
            If $(X,\,\tau)$ is metrizable, then it is compact if and only if
            it is sequentially compact.
        \end{theorem}
        \begin{proof}
            This follows from the equivalence of compactness theorem.
        \end{proof}
        \textit{Metrizable}, \textit{sequentially compact}, and
        \textit{compact} are three properties such that none implies the other.
        This is a good counterexample to the bad practice many students often
        make in logic. If $P$, $Q$, and $R$ are statements, and if
        $P\land{Q}\Leftrightarrow{P}\land{Q}$, is it true that
        $Q\Leftrightarrow{R}$? That is, can you \textit{divide} by $P$?
        These three topological properties provide a
        counterexample. Metrizable and compact if and only if metrizable and
        sequentially compact. Let's show none of these statements, by
        themselves, are logically equivalent or imply any of the others.
        \begin{itemize}
            \item Compact and not metrizable: The indiscrete topology on
                $\mathbb{R}$. The only open covers possible are finite to begin
                with, so the space is compact. It is not metrizable since it
                is not Hausdorff.
            \item Sequentially compact and not metrizable: The
                Sierpinski space $(\mathbb{Z}_{2},\,\tau)$ where
                $\tau=\big\{\,\emptyset,\,\{\,0\,\},\,\mathbb{Z}_{2}\,\big\}$.
                Any sequence $a:\mathbb{N}\rightarrow\mathbb{Z}_{2}$ must have
                a convergent subsequence since either infinitely many indices
                $n\in\mathbb{N}$ are such that $a_{n}=0$ or infinitely many are
                such that $a_{n}=1$ (since $\mathbb{N}$ is infinite). Hence
                there must be a constant subsequence, which is a convergent
                one. The space is not metrizable since it is not Hausdorff.
            \item Metrizable and not compact: The discrete topology on
                $\mathbb{R}$. The cover consisting of all single points
                $\{\,x\,\}$ for $x\in\mathbb{R}$ has no finite subcover. It
                doesn't even have a countable subcover.
            \item Metrizable and not sequentially compact: The standard topology
                on $\mathbb{R}$. The sequence
                $a:\mathbb{N}\rightarrow\mathbb{R}$ defined by $a_{n}=n$ has no
                convergent subsequence.
            \item Compact and not sequentially compact: The product space
                $\prod_{r\in[0,\,1]}[0,\,1]$. This is compact, with the product
                topology, by the Tychonoff theorem, something we'll get to
                soon. It is not sequentially compact. The product is
                uncountable so the space is not first countable, and intuitively
                sequences are not enough to describe the space.
            \item Sequentially compact and not compact: The long line. It is not
                compact, take as your open cover open intervals about the
                center that get larger and larger and exhaust the space. The
                open sets in this cover are nested, and no finite collection of
                such intervals cover the space. It is sequentially compact,
                which is hard to imagine. For simplicity, let's just use the
                long ray, which is the product of the first uncountable
                ordinal $\omega^{1}$ with $[0,\,1)$ equipped with the
                lexicographic order topology. Given a sequence $a$ in the long
                ray, $a_{n}=(\alpha_{n},\,x_{n})$ where $\alpha_{n}$ is an
                element of the first uncountable ordinal and $x_{n}\in[0,\,1)$.
                But the first uncountable ordinal is uncountable and
                $\mathbb{N}$ is countable, so this sequence cannot exhaust all
                of $\omega^{1}$. Because of this the elements will be
                contained in a small subset of the long ray, a subset that
                \textit{looks} like the closed unit interval $[0,\,1]$,
                topologically speaking. By using an argument similar to
                the proof of Bolzano's theorem (which is used to prove the
                Heine-Borel theorem for $\mathbb{R}$), we can conclude that
                any such sequence must have a convergent subsequence. So the
                long ray is sequentially compact.
        \end{itemize}
\end{document}
