%-----------------------------------LICENSE------------------------------------%
%   This file is part of Mathematics-and-Physics.                              %
%                                                                              %
%   Mathematics-and-Physics is free software: you can redistribute it and/or   %
%   modify it under the terms of the GNU General Public License as             %
%   published by the Free Software Foundation, either version 3 of the         %
%   License, or (at your option) any later version.                            %
%                                                                              %
%   Mathematics-and-Physics is distributed in the hope that it will be useful, %
%   but WITHOUT ANY WARRANTY; without even the implied warranty of             %
%   MERCHANTABILITY or FITNESS FOR A PARTICULAR PURPOSE.  See the              %
%   GNU General Public License for more details.                               %
%                                                                              %
%   You should have received a copy of the GNU General Public License along    %
%   with Mathematics-and-Physics.  If not, see <https://www.gnu.org/licenses/>.%
%------------------------------------------------------------------------------%
\documentclass{article}
\usepackage{graphicx}                           % Needed for figures.
\usepackage{amsmath}                            % Needed for align.
\usepackage{amssymb}                            % Needed for mathbb.
\usepackage{amsthm}                             % For the theorem environment.
\usepackage{float}
\usepackage{tabularx, booktabs}
\usepackage[font=scriptsize,
            labelformat=simple,
            labelsep=colon]{subcaption} % Subfigure captions.
\usepackage[font={scriptsize},
            hypcap=true,
            labelsep=colon]{caption}    % Figure captions.
\usepackage{hyperref}
\hypersetup{
    colorlinks=true,
    linkcolor=blue,
    filecolor=magenta,
    urlcolor=Cerulean,
    citecolor=SkyBlue
}

%------------------------Theorem Styles-------------------------%
\theoremstyle{plain}
\newtheorem{theorem}{Theorem}[section]

% Define theorem style for default spacing and normal font.
\newtheoremstyle{normal}
    {\topsep}               % Amount of space above the theorem.
    {\topsep}               % Amount of space below the theorem.
    {}                      % Font used for body of theorem.
    {}                      % Measure of space to indent.
    {\bfseries}             % Font of the header of the theorem.
    {}                      % Punctuation between head and body.
    {.5em}                  % Space after theorem head.
    {}

% Define default environments.
\theoremstyle{normal}
\newtheorem{examplex}{Example}[section]
\newtheorem{definitionx}{Definition}[section]

\newenvironment{example}{%
    \pushQED{\qed}\renewcommand{\qedsymbol}{$\blacksquare$}\examplex%
}{%
    \popQED\endexamplex%
}

\newenvironment{definition}{%
    \pushQED{\qed}\renewcommand{\qedsymbol}{$\blacksquare$}\definitionx%
}{%
    \popQED\enddefinitionx%
}

\title{Point-Set Topology: Lecture 21}
\author{Ryan Maguire}
\date{Summer 2022}

% No indent and no paragraph skip.
\setlength{\parindent}{0em}
\setlength{\parskip}{0em}

\begin{document}
    \maketitle
    \section{Connected and Path Connected Components}
        A topological space $(X,\,\tau)$ does not need to be connected, but it
        can always be divided into connected parts. These parts are called the
        \textit{connected components} of the space.
        \begin{definition}[\textbf{Connected Component}]
            The connected component of a point $x\in{X}$ in a topological space
            $(X,\,\tau)$ is the set $C\subseteq{X}$ defined by:
            \begin{equation}
                C=\bigcup\{\,A\subseteq{X}\;|\;
                    x\in{A}\textrm{ and }A\textrm{ is connected}\,\}
            \end{equation}
            That is, the \textit{largest} connected subset of $X$ containing
            $x$.
        \end{definition}
        \begin{theorem}
            If $(X,\,\tau)$ is a topological space, if $x\in{X}$, and if
            $C\subseteq{X}$ is the connected component of $x$, then
            $(C,\,\tau_{C})$ is a connected topological space where
            $\tau_{C}$ is the subspace topology.
        \end{theorem}
        \begin{proof}
            For if not then there are disjoint non-empty open subsets
            $\mathcal{U},\mathcal{V}$ such that $C=\mathcal{U}\cup\mathcal{V}$.
            But since $x\in{C}$ either $x\in\mathcal{U}$ or $x\in\mathcal{V}$.
            Suppose $x\in\mathcal{U}$. Let $\mathcal{O}$ be defined by:
            \begin{equation}
                \mathcal{O}=\{\,A\subseteq{X}\;|\;
                    x\in{A}\textrm{ and }A\textrm{ is connected}\,\}
            \end{equation}
            Then by definition of connected components $C=\bigcup\mathcal{O}$.
            Suppose there is some $A\in\mathcal{O}$ such that
            $\mathcal{V}\cap{A}\ne\emptyset$. But then
            $A\cap\mathcal{V}$ and $A\cap\mathcal{U}$ are non-empty open
            subsets of $A$ with respect to the subspace topology $\tau_{A}$.
            But $A\cap\mathcal{V}$ and $A\cap\mathcal{U}$ are disjoint since
            $\mathcal{U}$ and $\mathcal{V}$ are, and hence $(A,\,\tau_{A})$ can
            be separated by open sets, which is a contradiction since
            $A\in\mathcal{O}$ and hence $(A,\,\tau_{A})$ is connected. We have
            thus shown that $\mathcal{V}\cap{A}=\emptyset$ for every
            $A\in\mathcal{O}$, and hence
            $\mathcal{V}\cap\bigcup\mathcal{O}=\emptyset$. But
            $\mathcal{V}\subseteq{C}$ and $C=\bigcup\mathcal{O}$, meaning
            $\mathcal{V}=\emptyset$, which is a contradiction since
            $\mathcal{V}$ is non-empty. So $(C,\,\tau_{C})$ is connected.
        \end{proof}
        There is no analogous theorem for intersections. The intersection
        of two connected sets does not need to be connected. See
        Fig.~\ref{fig:intersection_of_connected_can_be_disconnected_001}.
        \begin{figure}
            \centering
            \includegraphics{../../../images/intersection_of_connected_can_be_disconnected_001.pdf}
            \caption{The Intersection of Connected Sets}
            \label{fig:intersection_of_connected_can_be_disconnected_001}
        \end{figure}
        Even if we are given infinitely many connected sets
        $\mathcal{U}_{n}$ that are all nested,
        $\mathcal{U}_{n+1}\subseteq\mathcal{U}_{n}$, the intersection
        $\bigcap_{n\in\mathbb{N}}\mathcal{U}_{n}$ need not be connected.
        Take $\mathcal{U}_{n}$ to be all points $(x,\,y)$ such that
        $y=0$, $y=1$, or $0<y<1$ and $x\geq{n}$. Each $\mathcal{U}_{n}$ is
        connected, and $\mathcal{U}_{n+1}\subseteq\mathcal{U}_{n}$, but
        $\bigcap_{n}\mathcal{U}_{n}$ is two separated lines, the lines
        $y=0$ and $y=1$, which is not connected.
        \par\hfill\par
        As another example, take $\mathcal{C}_{n}$ to be the closed box
        $[-1,\,1]\times[-n,\,n]$ and let
        $\mathcal{U}_{n}=\mathbb{R}^{2}\setminus\mathcal{C}_{n}$. Then all of
        the sets $\mathcal{U}_{n}$ are open, and are nested and connected,
        however $\bigcap\mathcal{U}_{n}$ splits the plane in half.
        \par\hfill\par
        Connected components are closed. To show this we'll need to following
        theorem about subspaces.
        \begin{theorem}
            If $(X,\,\tau)$ is a topological space, if $\mathcal{C}\subseteq{X}$
            is closed, and if $A\subseteq\mathcal{C}$ is closed with respect
            to the subspace topology $\tau_{\mathcal{C}}$, then $A$ is closed
            with respect to $\tau$.
        \end{theorem}
        \begin{proof}
            Since $A\subseteq\mathcal{C}$ is closed with respect to
            $\tau_{\mathcal{C}}$, $\mathcal{C}\setminus{A}$ is open. But
            if $\mathcal{C}\setminus{A}$ is open, then by the definition of the
            subspace topology there is an open set $\mathcal{U}\in\tau$ such
            that $\mathcal{C}\setminus{A}=\mathcal{C}\cap\mathcal{U}$.
            But then:
            \begin{equation}
                A=\mathcal{C}\setminus(\mathcal{C}\cap\mathcal{U})
                    =(\mathcal{C}\setminus\mathcal{C})
                        \cup(\mathcal{C}\setminus\mathcal{U})
                    =\emptyset\cup(\mathcal{C}\setminus\mathcal{U})
                    =\mathcal{C}\setminus\mathcal{U}
            \end{equation}
            But $\mathcal{C}\subseteq{X}$ is closed, so
            $\mathcal{C}\setminus\mathcal{U}$ is the difference of an open set
            from a closed set which is therefore closed. Therefore $A$ is
            closed with respect to $\tau$.
        \end{proof}
        \begin{theorem}
            If $(X,\,\tau)$ is a topological space, if $A\subseteq{X}$, if
            $\tau_{A}$ is the subspace topology, and if $(A,\,\tau_{A})$ is
            connected, then
            $(\textrm{Cl}_{\tau}(A),\,\tau_{\textrm{Cl}_{\tau}(A)})$ is
            connected.
        \end{theorem}
        \begin{proof}
            Suppose not. Then there are disjoint closed subsets
            $\mathcal{C},\mathcal{D}\subseteq\textrm{Cl}_{\tau}(A)$ such that
            $\mathcal{C}\cup\mathcal{D}=\textrm{Cl}_{\tau}(C)$. But
            $\textrm{Cl}_{\tau}(A)$ is closed with respect to $\tau$, so
            $\mathcal{C}$ and $\mathcal{D}$ are also closed with respect
            to $\tau$. But then
            $\mathcal{U}=X\setminus\mathcal{C}$ and
            $\mathcal{V}=X\setminus\mathcal{D}$ are disjoint open sets in
            $\tau$. But then $\mathcal{U}\cap{A}$ and
            $\mathcal{V}\cap{A}$ are disjoint non-empty open subsets with
            respect to $\tau_{A}$ that separate $A$, a contradiction since
            $(A,\,\tau_{A})$ is connected. Hence,
            $(\textrm{Cl}_{\tau}(A),\,\tau_{\textrm{Cl}_{\tau}(A)})$ is
            connected.
        \end{proof}
        \begin{theorem}
            If $(X,\,\tau)$ is a topological space, and if $C\subseteq{X}$ is
            the connected component of $x\in{X}$, then it is closed.
        \end{theorem}
        \begin{proof}
            We have that $C\subseteq\textrm{Cl}_{\tau}(C)$ by the definition of
            closure. But since connected components are connected we have that
            $\textrm{Cl}_{\tau}(C)$ is connected. But
            $x\in\textrm{Cl}_{\tau}(C)$ so, by the definition of connected
            components, $\textrm{Cl}_{\tau}(C)\subseteq{C}$. Hence
            $C=\textrm{Cl}_{\tau}(C)$ meaning $C$ is closed.
        \end{proof}
        The idea of connected components allows us to define totally
        disconnected spaces.
        \begin{definition}[\textbf{Totally Disconnected Topological Space}]
            A totally disconnected topological space is a topological space
            $(X,\,\tau)$ such that for all $x\in{X}$ the connected component of
            $x$ is the set $C=\{\,x\,\}$. That is, the connected components of
            the space are singleton sets.
        \end{definition}
        \begin{theorem}
            If $\tau_{\mathbb{Q}}$ is the subspace topology of $\mathbb{Q}$
            with respect to the standard Euclidean topology $\tau_{\mathbb{R}}$,
            then $(\mathbb{Q},\,\tau_{\mathbb{Q}})$ is totally disconnected.
        \end{theorem}
        \begin{proof}
            For let $x\in\mathbb{Q}$ and let $C$ be the connected component of
            $x$. Suppose $y\in\mathbb{Q}$ is such that $y\in{C}$ with $y\ne{x}$.
            Suppose $x<y$ (the idea is symmetric either way).
            Since $x,y\in\mathbb{Q}$, and since $\mathbb{Q}\subseteq\mathbb{R}$,
            it is true that $x,y\in\mathbb{R}$. Let
            $z\in\mathbb{R}\setminus\mathbb{Q}$ be such that $x<z$ and $z<y$.
            This is possible since $x$ and $y$ are real numbers and the
            irrational numbers are dense in $\mathbb{R}$. Let
            $\mathcal{U}=\mathbb{Q}\cap(-\infty,\,z)$ and
            $\mathcal{V}=\mathbb{Q}\cap(z,\,\infty)$. Then $\mathcal{U}$ and
            $\mathcal{V}$ are open in the subspace topology. But moreover
            $\mathcal{U}$ and $\mathcal{V}$ separate $x$ and $y$, meaning
            $y\notin{C}$. Hence $C$ is just a singleton set.
        \end{proof}
        \begin{theorem}
            If $(X,\,\tau)$ is a totally disconnected topological space, then
            it is a Fr\'{e}chet topological space.
        \end{theorem}
        \begin{proof}
            For all $x\in{X}$, since $(X,\,\tau)$ is totally disconnected, the
            connected component of $x$ is $C=\{\,x\,\}$. But connected
            components are closed, hence $\{\,x\,\}$ is closed. But this is true
            for all $x\in{X}$, hence $(X,\,\tau)$ is a Fr\"{e}chet topological
            space.
        \end{proof}
        \begin{example}[\textbf{The Rational Bug-Eyed Line}]
            Let $X=\mathbb{Q}\times\{\,0,\,1\,\}$, and give this the product
            topology $\tau$ where $\mathbb{Q}$ has the subspace topology from
            $\mathbb{R}$ and $\mathbb{Z}_{2}$ has the discrete topology. Define
            the equivalence relation $R$ on $X$ to be the equivalence relation
            induced by $(x,\,0)R(x,\,1)$ for all $x\in\mathbb{Q}$ except for
            $x=0$. The quotient space $(X/R,\,\tau_{X/R})$ is like the
            bug-eyed line, but has only rational points (and the two origins).
            This space is totally disconnected, but not Hausdorff.
        \end{example}
        Path connected components can be similarly defined.
        \begin{definition}[\textbf{Path Connected Component}]
            The path connected component of a point $x\in{X}$ in a topological
            space $(X,\,\tau)$ is the set $C\subseteq{X}$ defined by:
            \begin{equation}
                C=\big\{\,y\in{X}\;|\;\textrm{There is a path from }x
                    \textrm{ to }y\,\big\}
            \end{equation}
            That is, the set of all points that can be connected by a path
            to $x$.
        \end{definition}
        \begin{theorem}
            If $(X,\,\tau)$ is a topological space, if $x\in{X}$, and if
            $C$ is the path-connected component of $x$, then
            $(C,\,\tau_{C})$ is connected.
        \end{theorem}
        \begin{proof}
            By definition of path connected component, $(C,\,\tau_{C})$ forms
            a path connected space, and path connected spaces are connected.
        \end{proof}
    \section{Locally Connected and Locally Path Connected}
        Some warning. In topology the word \textit{locally} has two possible
        meanings. Given some \textit{property}, locally could mean:
        \begin{enumerate}
            \item For every point $x$ in the space there is some open or closed
                set $A$ containing the point such that $A$ has the desired
                property.
            \item There exists a basis of open sets $\mathcal{B}$ such that
                every element of $\mathcal{B}$ has the desired property.
        \end{enumerate}
        For example, a \textit{locally metrizable} space $(X,\,\tau)$ is a
        topological space such that for all $x\in{X}$ there is an open set
        $\mathcal{U}\in\tau$ such that $x\in\mathcal{U}$ and
        $(\mathcal{U},\,\tau_{\mathcal{U}})$ is a metrizable topological space.
        \par\hfill\par
        Locally compact means that for all $x\in{X}$ there is an open set
        $\mathcal{U}$ and a compact subspace $(K,\,\tau_{X})$ such that
        $x\in\mathcal{U}$ and $\mathcal{U}\subseteq{K}$.
        \par\hfill\par
        With the first use of the word \textit{locally}, if a space has the
        property \textit{globally}, then it automatically has it locally.
        That is, metrizable spaces are locally metrizable, and compact spaces
        are locally compact. With the second use of the word we are not so
        lucky. For connectedness, \textit{locally} connected and
        \textit{locally} path connected mean we can describe the topology using
        connected or path connected open subsets.
        \par\hfill\par
        The weird connected examples that are not path connected fail to be
        \textit{locally connected} and \textit{locally path connected}. These
        two properties have some very useful results associated with them, so
        we take the time to study these ideas.
        \begin{definition}[\textbf{Locally Connected Topological Space}]
            A locally connected topological space is a topological space
            $(X,\,\tau)$ such that there is a basis $\mathcal{B}$ for $\tau$
            such that for all $\mathcal{U}\in\mathcal{B}$ the subspace
            $(\mathcal{U},\,\tau_{\mathcal{U}})$ is connected.
        \end{definition}
        \begin{example}
            Neither the infinite broom nor the topologist's sine curve are
            locally connected, though both are connected. For the topologist's
            sine curve, any sufficiently small (small makes sense, this is a
            metric space since its a subset of $\mathbb{R}^{2}$) open set
            containing $(0,\,0)$ must contain disconnected pieces of the graph
            of $f(x)=\sin(1/x)$ as well, meaning there can be no basis of
            connected subsets. A similar argument holds for the infinite broom.
        \end{example}
        In any space connected components are closed. In locally connected
        spaces they are also open.
        \begin{theorem}
            If $(X,\,\tau)$ is a locally connected topological space, if
            $x\in{X}$, and if $C$ is the connected component of $x$, then
            $C\in\tau$.
        \end{theorem}
        \begin{proof}
            Since $(X,\,\tau)$ is locally connected there is a basis of
            connected subsets $\mathcal{B}$. Given $y\in{C}$, since
            $\mathcal{B}$ is a basis it is an open cover, so there is a
            $\mathcal{U}_{y}\in\mathcal{B}$ such that $y\in\mathcal{U}_{y}$.
            But then $C$ and $\mathcal{U}_{y}$ are connected subsets that both
            contain $y$, so $C\cup\mathcal{U}_{y}$ is connected. But $C$ is
            the connected component of $x$, so since $C\cup\mathcal{U}_{y}$ is
            connected and $x\in{C}\cup\mathcal{U}_{y}$ it must be true that
            $C\cup\mathcal{U}_{y}\subseteq{C}$. But
            $C\subseteq{C}\cup\mathcal{U}_{y}$ by definition of unions, so
            $C=C\cup\mathcal{U}_{y}$ and therefore
            $\mathcal{U}_{y}\subseteq{C}$. But $C$ can be written as the
            union of all such $\mathcal{U}_{y}$ for all $y\in{C}$, meaning
            $C$ is the union of open sets, which is therefore open.
        \end{proof}
        \begin{theorem}
            If $(X,\,\tau)$ is a totally disconnected and locally connected
            topological space, then $\tau=\mathcal{P}(X)$.
        \end{theorem}
        \begin{proof}
            Since $(X,\,\tau)$ is totally disconnected, connected components
            are singleton sets. But since $(X,\,\tau)$ is locally connected,
            connected components are open. Hence for all $x\in{X}$ the set
            $\{\,x\,\}$ is open, meaning all subsets of $X$ are open. That is,
            $\tau=\mathcal{P}(X)$.
        \end{proof}
        There is a path connected analogue to locally connected spaces.
        \begin{definition}[\textbf{Locally Path Connected Topological Space}]
            A locally path connected topological space is a topological space
            $(X,\,\tau)$ such that there is a basis $\mathcal{B}$ for $\tau$
            such that for all $\mathcal{U}$ the subspace
            $(\mathcal{U},\,\tau_{\mathcal{U}})$ is path connected.
        \end{definition}
        \begin{theorem}
            If $(X,\,\tau)$ is locally path connected, if $x\in{X}$, and if
            $C\subseteq{X}$ is the path connected component of $x$, then $C$
            is open.
        \end{theorem}
        \begin{proof}
            Since $(X,\,\tau)$ is locally path connected there is a basis
            $\mathcal{B}$ of open path connected subspaces. But since
            $\mathcal{B}$ is a basis it is an open cover, so given $y\in{C}$
            there is a $\mathcal{U}_{y}\in\mathcal{B}$ such that
            $y\in\mathcal{U}_{y}$. But $C$ and $\mathcal{U}_{y}$ are path
            connected and $y\in\mathcal{U}_{y}\cap{C}$, so
            $\mathcal{U}_{y}\cup{C}$ is path connected. Since $C$ is the path
            connected component of $x$, $\mathcal{U}_{y}\cup{C}\subseteq{C}$,
            and hence $\mathcal{U}_{y}\subseteq{C}$. But then $C$ can be written
            as the union of all such $\mathcal{U}_{y}$ for all $y\in{C}$,
            meaning $C$ is the union of open sets, and is therefore open.
        \end{proof}
        \begin{theorem}
            If $(X,\,\tau)$ is locally path connected, if $x\in{X}$, and if
            $C\subseteq{X}$ is the path connected component of $x$, then $C$
            is closed.
        \end{theorem}
        \begin{proof}
            Suppose not. Then $C\ne\textrm{Cl}_{\tau}(C)$. Let
            $y\in\textrm{Cl}_{\tau}(C)\setminus{C}$. Since $(X,\,\tau)$ is
            locally path connected there is a basis $\mathcal{B}$ of open path
            connected subspaces. But if $\mathcal{B}$ is a basis, since
            $y\in{X}$ there is a $\mathcal{U}\in\mathcal{B}$ such that
            $y\in\mathcal{U}$. But if $y\in\textrm{Cl}_{\tau}(X)$ and
            $y\in\mathcal{U}$, then $\mathcal{U}\cap{C}$ is non-empty
            (from the definition of closure). But $\mathcal{U}$ and $C$ are path
            connected subspaces with non-empty intersection, so
            $\mathcal{U}\cup{C}$ is path connected. But $C$ is the path
            connected component of $x$, so $\mathcal{U}\cup{C}\subseteq{C}$,
            and hence $\mathcal{U}\subseteq{C}$. But then $y\in{C}$,
            a contradiction. Hence, $C$ is closed.
        \end{proof}
        \begin{theorem}
            If $(X,\,\tau)$ is connected and locally path connected, then it is
            path connected.
        \end{theorem}
        \begin{proof}
            Suppose not. Then there are $x,y\in{X}$ such that there is no
            continuous function $f:[0,\,1]\rightarrow{X}$ such that
            $f(0)=x$ and $f(1)=y$. Let $C$ be the path connected component of
            $x$. Then $y\notin{C}$, and hence $C$ is a proper subset of $X$.
            But $x\in{C}$, so $C$ is non-empty. But since $C$ is the path
            connected component of $x$, and since $(X,\,\tau)$ is locally path
            connected, $C$ is both open and closed. But then $C$ is a non-empty
            proper subset of $X$ that is both open and closed, and therefore
            $(X,\,\tau)$ is disconnected, which is a contradiction since
            $(X,\,\tau)$ is connected. Hence, $(X,\,\tau)$ is path connected.
        \end{proof}
    \section{Arc Connected}
        Arc connected is a slightly stronger notion that many might think is
        the same thing as path connected. In Hausdorff spaces, the notions are
        the same. First, a definition.
        \begin{definition}[\textbf{Arc Connected Topological Space}]
            An arc connected topological space is a topological space
            $(X,\,\tau)$ such that for all $x,y\in{X}$ there is an injective
            continuous function $f:[0,\,1]\rightarrow{X}$ such that $f(0)=x$
            and $f(1)=y$.
        \end{definition}
        The only difference between path connected and arc connected is the
        introduction of the word \textit{injective}. The following theorem takes
        a lot of effort, and we don't have time for it, but I still want to
        present it. The proof is omitted.
        \begin{theorem}
            If $(X,\,\tau)$ is a Hausdorff path connected topological space,
            then it is arc connected.
        \end{theorem}
        \begin{example}
            Without the Hausdorff condition you can have path connected spaces
            that are not arc connected. The bug-eyed line is an example.
            There is a path from the first zero $0_{0}$ to the second zero
            $0_{1}$, namely:
            \begin{equation}
                f(x)=
                \begin{cases}
                    0_{0}&t=0\\
                    t-t^{2}&0<t<1\\
                    0_{1}&t=1
                \end{cases}
            \end{equation}
            That is, you start at the first zero, walk out a bit to some
            positive real number, and then walk back to the second zero.
            This is not injective, you crossed over a bunch of real numbers
            twice. It's impossible to do this injectively. This space
            is not Hausdorff, so there is no violation of the previous theorem.
        \end{example}
\end{document}
