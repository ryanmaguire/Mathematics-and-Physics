%-----------------------------------LICENSE------------------------------------%
%   This file is part of Mathematics-and-Physics.                              %
%                                                                              %
%   Mathematics-and-Physics is free software: you can redistribute it and/or   %
%   modify it under the terms of the GNU General Public License as             %
%   published by the Free Software Foundation, either version 3 of the         %
%   License, or (at your option) any later version.                            %
%                                                                              %
%   Mathematics-and-Physics is distributed in the hope that it will be useful, %
%   but WITHOUT ANY WARRANTY; without even the implied warranty of             %
%   MERCHANTABILITY or FITNESS FOR A PARTICULAR PURPOSE.  See the              %
%   GNU General Public License for more details.                               %
%                                                                              %
%   You should have received a copy of the GNU General Public License along    %
%   with Mathematics-and-Physics.  If not, see <https://www.gnu.org/licenses/>.%
%------------------------------------------------------------------------------%
\documentclass{article}
\usepackage{graphicx}                           % Needed for figures.
\usepackage{amsmath}                            % Needed for align.
\usepackage{amssymb}                            % Needed for mathbb.
\usepackage{amsthm}                             % For the theorem environment.
\usepackage{float}
\usepackage{tabularx, booktabs}
\usepackage{mathrsfs}
\usepackage[font=scriptsize,
            labelformat=simple,
            labelsep=colon]{subcaption} % Subfigure captions.
\usepackage[font={scriptsize},
            hypcap=true,
            labelsep=colon]{caption}    % Figure captions.
\usepackage{hyperref}
\hypersetup{
    colorlinks=true,
    linkcolor=blue,
    filecolor=magenta,
    urlcolor=Cerulean,
    citecolor=SkyBlue
}

%------------------------Theorem Styles-------------------------%
\theoremstyle{plain}
\newtheorem{theorem}{Theorem}[section]

% Define theorem style for default spacing and normal font.
\newtheoremstyle{normal}
    {\topsep}               % Amount of space above the theorem.
    {\topsep}               % Amount of space below the theorem.
    {}                      % Font used for body of theorem.
    {}                      % Measure of space to indent.
    {\bfseries}             % Font of the header of the theorem.
    {}                      % Punctuation between head and body.
    {.5em}                  % Space after theorem head.
    {}

% Define default environments.
\theoremstyle{normal}
\newtheorem{examplex}{Example}[section]
\newtheorem{definitionx}{Definition}[section]

\newenvironment{example}{%
    \pushQED{\qed}\renewcommand{\qedsymbol}{$\blacksquare$}\examplex%
}{%
    \popQED\endexamplex%
}

\newenvironment{definition}{%
    \pushQED{\qed}\renewcommand{\qedsymbol}{$\blacksquare$}\definitionx%
}{%
    \popQED\enddefinitionx%
}

\title{Point-Set Topology: Lecture 30}
\author{Ryan Maguire}
\date{Summer 2022}

% No indent and no paragraph skip.
\setlength{\parindent}{0em}
\setlength{\parskip}{0em}

\begin{document}
    \maketitle
    \section{The Topology of Manifolds}
        In these notes we will prove some facts about the basic topology of
        manifolds. For convenience, these results are summarized below.
        \par\hfill\par
        Topological manifolds are:
        \begin{itemize}
            \item Locally compact.
            \item Locally metrizable.
            \item Regular
            \item Metrizable.
            \item Paracompact.
            \item Lindel\"{o}f.
            \item Have a countable basis of precompact coordinate balls.
            \item $\sigma$ compact.
            \item Compactly exhaustible.
            \item Locally connected.
            \item Locally path-connected.
            \item Connected if and only if path connected.
        \end{itemize}
        Our goal is to prove all of these claims. But we won't add a condition
        if it's not needed. For example, locally Euclidean implies locally
        compact, there is no need to add the Hausdorff and second countability
        requirements.
        \begin{theorem}
            If $(X,\,\tau)$ is a topological space, then it is locally Euclidean
            if and only if for all $x\in{X}$ there is a $\mathcal{U}\in\tau$
            such that $x\in\mathcal{U}$ and
            $(\mathcal{U},\,\tau_{\mathcal{U}})$ is homeomorphic to
            $B_{1}^{\mathbb{R}^{n}}(\mathbf{0})$ for some $n\in\mathbb{N}$. 
        \end{theorem}
        \begin{proof}
            If $(X,\,\tau)$ is locally Euclidean, then for all $x\in{X}$ there
            is a $\mathcal{U}\in\tau$ such that $x\in\mathcal{U}$, and there
            is an $n\in\mathbb{N}$ and a continuous injective open mapping
            $f:\mathcal{U}\rightarrow\mathbb{R}^{n}$. But since $f$ is an
            open mapping, $f[\mathcal{U}]\subseteq\mathbb{R}^{n}$ is open.
            But $f(x)\in{f}[\mathcal{U}]$ so there is an $\varepsilon>0$ such
            that $||\mathbf{y}-f(x)||_{2}<\varepsilon$ implies
            $\mathbf{y}\in{f}[\mathcal{U}]$. Let $\tilde{\mathcal{V}}$ be the
            $\varepsilon$ centered at $f(x)$ and
            $\mathcal{V}=f^{-1}[\tilde{\mathcal{V}}]$. Then
            $\mathcal{V}\subseteq\mathcal{U}$ is open since $f$ is continuous.
            Moreover
            $f|_{\mathcal{V}}:\mathcal{V}\rightarrow\tilde{\mathcal{V}}$ is
            a continuous bijective open mapping, which is therefore a
            homeomorphism. But $\tilde{\mathcal{V}}$ is an open ball in
            $\mathbb{R}^{n}$ of non-zero radius, and such a ball is homeomorphic
            to $B_{1}^{\mathbb{R}^{n}}(\mathbf{0})$ via:
            \begin{equation}
                g(\mathbf{x})=\frac{1}{\varepsilon}(\mathbf{x}-\mathbf{x}_{0})
            \end{equation}
            where $\varepsilon>0$ is the radius and $\mathbf{x}_{0}$ is the
            center. Since homeomorphic is a transitive notion,
            $\mathcal{V}$ is homeomorphic to
            $B_{1}^{\mathbb{R}^{n}}(\mathbf{0})$ and $x\in\mathcal{V}$. In the
            other direction, if $x\in{X}$ implies there is an open set
            $\mathcal{U}\in\tau$ such that $x\in\mathcal{U}$ and
            $(\mathcal{U},\,\tau_{\mathcal{U}})$ is homeomorphic to the open
            ball in $\mathbb{R}^{n}$, then there is a homeomorphism
            $f:\mathcal{U}\rightarrow{B}_{1}^{\mathbb{R}^{n}}(\mathbf{0})$. But
            then, in particular, $f:\mathcal{U}\rightarrow\mathbb{R}^{n}$ is an
            injective continuous open mapping. Hence, $(X,\,\tau)$ is locally
            Euclidean.
        \end{proof}
        \begin{theorem}
            If $(X,\,\tau)$ is a topological space, then it is locally
            Euclidean if and only if for all $x\in{X}$ there is a
            $\mathcal{U}\in\tau$ such that $x\in\mathcal{U}$ and
            $(\mathcal{U},\,\tau_{\mathcal{U}})$ is homeomorphic to
            $\mathbb{R}^{n}$.
        \end{theorem}
        \begin{proof}
            By the previous theorem $(X,\,\tau)$ is locally Euclidean if and
            only if for all $x\in{X}$ there is a $\mathcal{U}\in\tau$ such that
            $x\in\mathcal{U}$ and $(\mathcal{U},\,\tau_{\mathcal{U}})$ is
            homeomorphic to $B_{1}^{\mathbb{R}^{n}}(\mathbf{0})$. But
            the open unit ball in $\mathbb{R}^{n}$ is homeomorphic to
            $\mathbb{R}^{n}$ via:
            \begin{equation}
                f(\mathbf{x})=
                \frac{\mathbf{x}}{1-||\mathbf{x}||_{2}}
            \end{equation}
            and hence $(X,\,\tau)$ is locally Euclidean if and only if for all
            $x\in{X}$ there is a $\mathcal{U}\in\tau$ such that
            $x\in\mathcal{U}$ and $(\mathcal{U},\,\tau_{\mathcal{U}})$ is
            homeomorphic to $\mathbb{R}^{n}$.
        \end{proof}
        \begin{definition}[\textbf{Coordinate Ball}]
            A coordinate ball in a topological space $(X,\,\tau)$ is an open
            set $\mathcal{U}\in\tau$ that is homeomorphic to $\mathbb{R}^{n}$
            (or, equivalently, homeomorphic to the unit ball
            $B_{1}^{\mathbb{R}^{n}}(\mathbf{0})$).
        \end{definition}
        \begin{theorem}
            If $(X,\,\tau)$ is locally Euclidean, then there is a basis
            $\mathcal{B}$ of $\tau$ consisting only of coordinate balls.
        \end{theorem}
        \begin{proof}
            Let $\mathcal{B}$ be the set of all coordinate balls in
            $(X,\,\tau)$. By the previous theorems this set is an open cover of
            $X$. It is also a basis. For given $\mathcal{U}\in\tau$ and
            $x\in\mathcal{U}$, since $x\in{X}$ there is a coordinate ball
            $\mathcal{V}\in\tau$ such that $x\in\mathcal{V}$. But then there
            is a homeomorphism $f:\mathcal{V}\rightarrow\mathbb{R}^{n}$ for some
            $n\in\mathbb{N}$. Since $\mathcal{U}$ is open,
            $\mathcal{U}\cap\mathcal{V}\subseteq\mathcal{V}$ is open. But then,
            since $f$ is a homeomorphism, $f[\mathcal{U}\cap\mathcal{V}]$ is
            an open subset of $\mathbb{R}^{n}$. And since
            $x\in\mathcal{U}\cap\mathcal{V}$ we have
            $f(x)\in{f}[\mathcal{U}\cap\mathcal{V}]$. So there is an
            $\varepsilon>0$ such that the open ball of radius $\varepsilon$
            centered at $f(x)$ sits inside of $f[\mathcal{U}\cap\mathcal{V}]$.
            Label this open ball as $\tilde{\mathcal{W}}$. But then
            $\mathcal{W}=f^{-1}[\tilde{\mathcal{W}}]$ is a coordinate ball in
            $(X,\,\tau)$, so $\mathcal{W}\in\mathcal{B}$, and it is such that
            $x\in\mathcal{W}$ and $\mathcal{W}\subseteq\mathcal{V}$. Therefore
            $\mathcal{B}$ is a basis.
        \end{proof}
        \begin{theorem}
            If $(X,\,\tau)$ is locally Euclidean and Lindel\"{o}f, then
            there is a countable basis $\mathcal{B}$ of coordinate balls for
            $\tau$. In particular, $(X,\,\tau)$ is second countable.
        \end{theorem}
        \begin{proof}
            Let $\mathcal{B}$ be the set of all coordinate ball in
            $(X,\,\tau)$. By the previous theorem this is a basis, and hence an
            open cover of $X$. But $(X,\,\tau)$ is Lindel\"{o}f so there is a
            countable subcover $\Delta\subseteq\mathcal{B}$. Then for all
            $\mathcal{U}\in\Delta$ we have that $\mathcal{U}$ is a coordinate
            ball, so homeomorphic to $\mathbb{R}^{n}$. But $\mathbb{R}^{n}$ has
            a countable basis consisting of open balls, meaning
            $\mathcal{U}\in\Delta$ has a countable basis of coordinate balls.
            Let $\mathcal{B}_{\mathcal{U}}$ be such a basis for $\mathcal{U}$
            and define:
            \begin{equation}
                \mathcal{B}=\bigcup_{\mathcal{U}\in\Delta}B_{\mathcal{U}}
            \end{equation}
            This is the countable union (since $\Delta$ is countable) of
            countable sets (since each $\mathcal{B}_{\mathcal{U}}$ is
            countable) and hence $\mathcal{B}$ is countable. It is also a
            basis since each $\mathcal{B}_{\mathcal{U}}$ is a basis for
            $\mathcal{U}$ and the set of all $\mathcal{U}\in\Delta$ cover
            $X$. Hence $\mathcal{B}$ is a countable basis of coordinate balls.
            Since it is a countable basis, $(X,\,\tau)$ is second countable.
        \end{proof}
        \begin{definition}[\textbf{Precompact}]
            A precompact subset of a topological space $(X,\,\tau)$ is a subset
            $A\subseteq{X}$ such that $\textrm{Cl}_{\tau}(A)$ is compact.
        \end{definition}
        \begin{theorem}
            If $(X,\,\tau)$ is locally Euclidean and Hausdorff, then for all
            $x\in{X}$ there is a precompact coordinate ball $\mathcal{U}\in\tau$
            such that $x\in\mathcal{U}$.
        \end{theorem}
        \begin{proof}
            Since $(X,\,\tau)$ is locally Euclidean, then is a coordinate
            ball $\mathcal{V}\in\tau$ such that $x\in\mathcal{V}$. Let
            $f:\mathcal{V}\rightarrow\mathbb{R}^{n}$ be a homeomorphism such
            that $f(x)=\mathbf{0}$ (we can always do this by translating).
            Let $\tilde{\mathcal{U}}=B_{1}^{\mathbb{R}^{n}}(\mathbf{0})$ and
            $\mathcal{U}=f^{-1}[\tilde{\mathcal{U}}]$. Then, since $f$ is a
            homeomorphism it is continuous, so
            $\mathcal{U}\subseteq\mathcal{V}$ is an open subset. But
            $\textrm{Cl}_{\tau_{\mathbb{R}^{n}}}(\tilde{\mathcal{V}})$ is
            compact, since it is the closed unit ball in $\mathbb{R}^{n}$.
            So
            $f^{-1}[\textrm{Cl}_{\tau_{\mathbb{R}^{n}}}(\tilde{\mathcal{V}})]$
            is a compact subset of $\mathcal{U}$ since $f$ is a homeomorphisn,
            and since $(X,\,\tau)$ is Hausdorff, this set is a closed subset as
            well. But then:
            \begin{align}
                f^{-1}[\textrm{Cl}_{\tau_{\mathbb{R}^{n}}}(\tilde{\mathcal{V}})]
                &=\textrm{Cl}_{\tau_{\mathcal{U}}}(\mathcal{V})\\
                &=\textrm{Cl}_{\tau}(\mathcal{V})
            \end{align}
            and hence $\mathcal{V}\subseteq{X}$ is a precompact coordinate
            ball that contains $x$.
        \end{proof}
        Note in the previous theorem we considered the closure with respect to
        three different topologies. First, the standard Euclidean topology in
        $\mathbb{R}^{n}$. Next, the subspace topology of the open set
        $\mathcal{U}$. Lastly, the topology on $X$. Since $X$ is Hausdorff,
        $\textrm{Cl}_{\tau_{\mathcal{U}}}(\mathcal{V})$ being compact implies it
        is closed. Because of this the closure with respect to the subspace
        topology $\tau_{\mathcal{U}}$ is the same as the closure with respect
        to the ambient topology $\tau$, meaning $\mathcal{V}$ is precompact
        in $(X,\,\tau)$. The Hausdorff property is essential. Without it the
        closure with respect to the subpace $\mathcal{U}$ and the entire space
        $X$ may be very different, and $\textrm{Cl}_{\tau}(X)$ might fail to be
        compact.
        \begin{theorem}
            If $(X,\,\tau)$ is locally Euclidean and Hausdorff, then there is a
            basis $\mathcal{B}$ of precompact coordinate balls.
        \end{theorem}
        \begin{proof}
            Let $\mathcal{B}$ be the set of all precompact coordinate balls in
            $(X,\,\tau)$. By the previous theorem $\mathcal{B}$ is an open
            cover. It is also a basis. Given $\mathcal{U}\in\tau$ and
            $x\in\mathcal{U}$ we can find a precompact coordinate ball
            $\mathcal{V}\in\mathcal{B}$ such that $x\in\mathcal{V}$. That is,
            there is a homeomorphism $f:\mathcal{V}\rightarrow\mathbb{R}^{n}$.
            Since $x\in\mathcal{U}\cap\mathcal{V}$, and since $f$ is a
            homeomorphism, $f[\mathcal{U}\cap\mathcal{V}]$ is an open subset of
            $\mathbb{R}^{n}$ and $f(x)\in{f}[\mathcal{U}\cap\mathcal{V}]$. So
            there is an $\varepsilon>0$ such that the $\varepsilon$ ball about
            $f(x)$ is a subset of $f[\mathcal{U}\cap\mathcal{V}]$. Let
            $\tilde{\mathcal{W}}$ be the $\varepsilon$ ball centered about
            $f(x)$. Then $\mathcal{W}=f^{-1}[\tilde{\mathcal{W}}]$ is a
            coordinate ball that sits inside of $\mathcal{U}\cap\mathcal{V}$,
            and since $\mathcal{V}$ is precompact, $\mathcal{W}$ is precompact
            as well since
            $\textrm{Cl}_{\tau}(\mathcal{W})\subseteq\textrm{Cl}_{\tau}(\mathcal{V})$.
            Hence $\mathcal{W}$ is a precompact coordinate ball such that
            $x\in\mathcal{W}$ and $\mathcal{W}\subseteq\mathcal{U}$. Therefore
            $\mathcal{B}$ is a basis for $\tau$.
        \end{proof}
        \begin{theorem}
            If $(X,\,\tau)$ is a topological manifold, then it is Lindel\"{o}f.
        \end{theorem}
        \begin{proof}
            By definition topological manifolds are second countable. But
            second countable spaces are Lindel\"{o}f, and hence
            $(X,\,\tau)$ is Lindel\"{o}f.
        \end{proof}
        \begin{theorem}
            If $(X,\,\tau)$ is a topological manifold, then there exists a
            countable basis $\mathcal{B}$ of $\tau$ consisting of precompact
            coordinate balls.
        \end{theorem}
        \begin{proof}
            The proof is now a combination of a few previous ideas. We just
            proved there is a basis $\mathcal{B}$ of precompact coordinate
            balls. Since topological manifolds are Lindel\"{o}f, there is a
            countable subcover $\Delta\subseteq\mathcal{B}$. So countably many
            precompact coordinate balls cover $X$, each of which has a countable
            basis of precompact coordinate balls. The union of all of these
            bases for every element of $\Delta$ is a basis for $\tau$, and
            since it is the countable union of countable sets, this union is
            countable itself. So $\tau$ has a countable basis of
            precompact coordinate balls.
        \end{proof}
        \begin{theorem}
            If $(X,\,\tau)$ is locally Euclidean, then it is locally metrizable.
        \end{theorem}
        \begin{proof}
            Every point $x\in{X}$ has an open set $\mathcal{U}\in\tau$ such
            that $x\in\mathcal{U}$ and $\mathcal{U}$ is homeomorphic to
            $\mathbb{R}^{n}$ for some $n\in\mathbb{N}$. But $\mathbb{R}^{n}$
            is metrizable, the topology being induced by the Euclidean metric,
            so $\mathcal{U}$ is a metrizable subspace of $(X,\,\tau)$ that
            contains $x$. Hence, $(X,\,\tau)$ is locally metrizable.
        \end{proof}
        \begin{theorem}
            If $(X,\,\tau)$ is locally Euclidean, then it is locally compact.
        \end{theorem}
        \begin{proof}
            Let $x\in{X}$. Since $(X,\,\tau)$ is locally Euclidean, there is
            a coordinate ball $\mathcal{U}\in\tau$ such that $x\in\mathcal{U}$
            and $\mathcal{U}$ is homeomorphic to $\mathbb{R}^{n}$. Let
            $f:\mathbb{R}^{n}\rightarrow\mathcal{U}$ be a homeomorphism. Then
            in particular $f:\mathbb{R}^{n}\rightarrow{X}$ is an injective
            continuous open mapping. Let $\tilde{\mathcal{V}}$ be the open
            ball of radius $1$ centered about $f(x)$ and let $\tilde{K}$ be the
            closed ball of radius $1$ centered at $f(x)$. Then
            $\tilde{\mathcal{V}}\subseteq\tilde{K}$. By the Heine-Borel theorem
            $\tilde{K}$ is compact. But then, since $f$ is continuous,
            $K=f[\tilde{K}]\subseteq{X}$ is compact. Let
            $\mathcal{V}=f[\tilde{\mathcal{V}}]$. Since $\tilde{\mathcal{V}}$ is
            open and $f$ is an open mapping, $\mathcal{V}\subseteq\mathcal{U}$
            is open. But $x\in\mathcal{V}$ and $\mathcal{V}\subseteq{K}$.
            Hence, $(X,\,\tau)$ is locally compact.
        \end{proof}
        \begin{theorem}
            If $(X,\,\tau)$ is locally Euclidean and Hausdorff, then it is
            regular.
        \end{theorem}
        \begin{proof}
            Since $(X,\,\tau)$ is locally Euclidean, it is locally compact.
            But locally compact Hausdorff spaces are regular, and hence
            $(X,\,\tau)$ is regular.
        \end{proof}
        \begin{theorem}
            If $(X,\,\tau)$ is a topological manifold, then it is metrizable.
        \end{theorem}
        \begin{proof}
            Since $(X,\,\tau)$ is a topological manifold, it is Hausdorff and
            second countable by definition. But topological manifolds are
            also locally Euclidean and locally Euclidean Hausdorff spaces are
            regular. Hence $(X,\,\tau)$ is regular, Hausdorff, and second
            countable, so by Urysohn's metrization theorem it is metrizable.
        \end{proof}
        \begin{theorem}
            If $(X,\,\tau)$ is a topological manifold, then it is paracompact.
        \end{theorem}
        \begin{proof}
            Since $(X,\,\tau)$ is a topological manifold, it is metrizable.
            But by Stone's theorem metrizable spaces are paracompact. Hence,
            $(X,\,\tau)$ is paracompact.
        \end{proof}
        \begin{theorem}
            If $(X,\,\tau)$ is locally Euclidean, Hausdorff, and paracompact,
            then it is metrizable.
        \end{theorem}
        \begin{proof}
            Since locally Euclidean space are locally metrizable,
            $(X,\,\tau)$ is a locally metrizable Hausdorff space that is
            paracompact. By Smirnov's theorem, $(X,\,\tau)$ is metrizable.
        \end{proof}
        \begin{theorem}
            If $(X,\,\tau)$ is a topological manifold, then it is
            $\sigma$ compact.
        \end{theorem}
        \begin{proof}
            Since $(X,\,\tau)$ is a topological manifold, there is a countable
            basis $\mathcal{B}$ of precompact coordinate balls. Let
            $\mathcal{U}:\mathbb{N}\rightarrow\mathcal{B}$ be a surjection.
            Then, since $\mathcal{B}$ is a basis, we have
            $X=\bigcup_{n\in\mathbb{N}}\mathcal{U}_{n}$. But also, since all
            elements of $\mathcal{B}$ are precompact, for all $n\in\mathbb{N}$
            it is true that $\textrm{Cl}_{\tau}(\mathcal{U}_{n})$ is
            compact. Let $K_{n}=\textrm{Cl}_{\tau}(\mathcal{U}_{n})$. Then
            each $K_{n}$ is compact and $\bigcup_{n\in\mathbb{N}}K_{n}=X$.
            Hence $(X,\,\tau)$ is $\sigma$ compact.
        \end{proof}
        \begin{theorem}
            If $(X,\,\tau)$ is a topological manifold, it is compactly
            exhaustible.
        \end{theorem}
        \begin{proof}
            Since $(X,\,\tau)$ is a topological manifold, it is $\sigma$
            compact. But topological manifolds are also Hausdorff and locally
            compact. But locally compact Hausdorff spaces that are
            $\sigma$ compact are compactly exhaustible. Hence,
            $(X,\,\tau)$ is compactly exhaustible.
        \end{proof}
        We've already stated this theorem, but let's prove it again.
        \begin{theorem}
            If $(X,\,\tau)$ is a topological manifold, then it is paracompact.
        \end{theorem}
        \begin{proof}
            Since $(X,\,\tau)$ is a topological manifold, it is compactly
            exhaustible. But compactly exhaustible Hausdorff spaces are
            paracompact. Hence, $(X,\,\tau)$ is paracompact.
        \end{proof}
        The connectedness theorems come from the locally Euclidean property.
        \begin{theorem}
            If $(X,\,\tau)$ is locally Euclidean, then it is locally connected.
        \end{theorem}
        \begin{proof}
            Since $(X,\,\tau)$ is locally Euclidean it has a basis of
            coordinate balls. But coordinate balls are connected, so
            $(X,\,\tau)$ is locally connected.
        \end{proof}
        \begin{theorem}
            If $(X,\,\tau)$ is locally Euclidean, then it is locally
            path connected.
        \end{theorem}
        \begin{proof}
            Since $(X,\,\tau)$ is locally Euclidean it has a basis of
            coordinate balls. But coordinate balls are path connected, so
            $(X,\,\tau)$ is locally path connected.
        \end{proof}
        \begin{theorem}
            If $(X,\,\tau)$ is locally Euclidean, then it is connected if and
            only if it is path connected.
        \end{theorem}
        \begin{proof}
            Path connected always implies connected. If $(X,\,\tau)$ is
            locally Euclidean and connected, then since locally Euclidean
            spaces are locally path connected, $(X,\,\tau)$ must be
            path connected.
        \end{proof}
\end{document}
