%-----------------------------------LICENSE------------------------------------%
%   This file is part of Mathematics-and-Physics.                              %
%                                                                              %
%   Mathematics-and-Physics is free software: you can redistribute it and/or   %
%   modify it under the terms of the GNU General Public License as             %
%   published by the Free Software Foundation, either version 3 of the         %
%   License, or (at your option) any later version.                            %
%                                                                              %
%   Mathematics-and-Physics is distributed in the hope that it will be useful, %
%   but WITHOUT ANY WARRANTY; without even the implied warranty of             %
%   MERCHANTABILITY or FITNESS FOR A PARTICULAR PURPOSE.  See the              %
%   GNU General Public License for more details.                               %
%                                                                              %
%   You should have received a copy of the GNU General Public License along    %
%   with Mathematics-and-Physics.  If not, see <https://www.gnu.org/licenses/>.%
%------------------------------------------------------------------------------%
\documentclass{article}
\usepackage{graphicx}                           % Needed for figures.
\usepackage{amsmath}                            % Needed for align.
\usepackage{amssymb}                            % Needed for mathbb.
\usepackage{amsthm}                             % For the theorem environment.
\usepackage{float}
\usepackage{hyperref}
\hypersetup{
    colorlinks=true,
    linkcolor=blue,
    filecolor=magenta,
    urlcolor=Cerulean,
    citecolor=SkyBlue
}

%------------------------Theorem Styles-------------------------%
\theoremstyle{plain}
\newtheorem{theorem}{Theorem}[section]

% Define theorem style for default spacing and normal font.
\newtheoremstyle{normal}
    {\topsep}               % Amount of space above the theorem.
    {\topsep}               % Amount of space below the theorem.
    {}                      % Font used for body of theorem.
    {}                      % Measure of space to indent.
    {\bfseries}             % Font of the header of the theorem.
    {}                      % Punctuation between head and body.
    {.5em}                  % Space after theorem head.
    {}

% Define default environments.
\theoremstyle{normal}
\newtheorem{examplex}{Example}[section]
\newtheorem{definitionx}{Definition}[section]
\newtheorem{notationx}{Notation}[section]

\newenvironment{example}{%
    \pushQED{\qed}\renewcommand{\qedsymbol}{$\blacksquare$}\examplex%
}{%
    \popQED\endexamplex%
}

\newenvironment{definition}{%
    \pushQED{\qed}\renewcommand{\qedsymbol}{$\blacksquare$}\definitionx%
}{%
    \popQED\enddefinitionx%
}

\title{Geometric Series}
\author{Ryan Maguire}
\date{\today}

% No indent and no paragraph skip.
\setlength{\parindent}{0em}
\setlength{\parskip}{0em}

\begin{document}
    \maketitle
    Given a real number $r\in\mathbb{R}$ with $|r|<1$ it can be shown that
    $r^{n}$ tends to zero as $n$ increases to infinity. This fact will be used
    freely throughout. Consider the series of partial sums of $r^{n}$:
    \begin{equation}
        S_{N}=\sum_{n=0}^{N}r^{n}
    \end{equation}
    We can rewrite this in a smaller closed-form. Multiplying both sides
    by $1-r$ we get:
    \begin{align}
        S_{N}(1-r)
        &=(1-r)\sum_{n=0}^{N}r^{n}\\
        &=\sum_{n=0}^{N}r^{n}(1-r)\\
        &=\sum_{n=0}^{N}\big(r^{n}-r^{n+1}\big)\\
        &=\sum_{n=0}^{N}r^{n}-\sum_{n=0}^{N}r^{n+1}\\
        &=1+\sum_{n=1}^{N}r^{n}-\sum_{n=0}^{N}r^{n+1}\\
        &=1+\sum_{n=1}^{N}r^{n}-r^{N+1}-\sum_{n=0}^{N-1}r^{n+1}\\
        &=(1-r^{N+1})+\sum_{n=1}^{N}r^{n}-\sum_{n=0}^{N-1}r^{n+1}\\
        &=(1-r^{N+1})+\sum_{n=1}^{N}r^{n}-\sum_{n=1}^{N}r^{n}\\
        &=1-r^{N+1}
    \end{align}
    Dividing both sides by $1-r$ gives us:
    \begin{equation}
        S_{N}=\frac{1-r^{N+1}}{1-r}
    \end{equation}
    Since $r^{N+1}$ tends to zero as $N$ increases to infinity, we get the
    final result:
    \begin{equation}
        \sum_{n=0}^{\infty}=\frac{1}{1-r}
    \end{equation}
    Let's examine the tail end.
    \begin{align}
        \sum_{n=N+1}^{\infty}r^{n}
        &=\sum_{n=0}^{\infty}r^{n}-\sum_{n=0}^{N}r^{n}\\
        &=\frac{1}{1-r}-\sum_{n=0}^{N}r^{n}\\
        &=\frac{1}{1-r}-\frac{1-r^{N+1}}{1-r}\\
        &=\frac{r^{N+1}}{1-r}
    \end{align}
    Since $r^{N+1}$ gets smaller as $N$ gets larger we see that the tail
    end converges to zero (as it must for the series to converge).
    \par\hfill\par
    We can prove the partial sum formula using induction as well. The base
    case $N=0$ says:
    \begin{equation}
        \sum_{n=0}^{0}r^{n}=r^{0}=1=\frac{1-r^{0+1}}{1-r}=\frac{1-r}{1-r}
    \end{equation}
    which is true. Suppose the formula holds for $N\in\mathbb{N}$. Then:
    \begin{align}
        \sum_{n=0}^{N+1}r^{n}
        &=r^{N+1}+\sum_{n=0}^{N}r^{n}\\
        &=r^{N+1}+\frac{1-r^{N+1}}{1-r}\\
        &=\frac{r^{N+1}(1-r)+1-r^{N+1}}{1-r}\\
        &=\frac{r^{N+1}-r^{N+2}+1-r^{N+1}}{1-r}\\
        &=\frac{1-r^{N+2}}{1-r}
    \end{align}
    and so the formula holds for $N+1$ as well.
\end{document}
