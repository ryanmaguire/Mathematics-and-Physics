%-----------------------------------LICENSE------------------------------------%
%   This file is part of Mathematics-and-Physics.                              %
%                                                                              %
%   Mathematics-and-Physics is free software: you can redistribute it and/or   %
%   modify it under the terms of the GNU General Public License as             %
%   published by the Free Software Foundation, either version 3 of the         %
%   License, or (at your option) any later version.                            %
%                                                                              %
%   Mathematics-and-Physics is distributed in the hope that it will be useful, %
%   but WITHOUT ANY WARRANTY; without even the implied warranty of             %
%   MERCHANTABILITY or FITNESS FOR A PARTICULAR PURPOSE.  See the              %
%   GNU General Public License for more details.                               %
%                                                                              %
%   You should have received a copy of the GNU General Public License along    %
%   with Mathematics-and-Physics.  If not, see <https://www.gnu.org/licenses/>.%
%------------------------------------------------------------------------------%
\documentclass{article}
\usepackage{graphicx}                           % Needed for figures.
\usepackage{amsmath}                            % Needed for align.
\usepackage{amssymb}                            % Needed for mathbb.
\usepackage{amsthm}                             % For the theorem environment.
\usepackage{float}
\usepackage{hyperref}
\hypersetup{
    colorlinks=true,
    linkcolor=blue,
    filecolor=magenta,
    urlcolor=Cerulean,
    citecolor=SkyBlue
}

%------------------------Theorem Styles-------------------------%
\theoremstyle{plain}
\newtheorem{theorem}{Theorem}

% Define theorem style for default spacing and normal font.
\newtheoremstyle{normal}
    {\topsep}               % Amount of space above the theorem.
    {\topsep}               % Amount of space below the theorem.
    {}                      % Font used for body of theorem.
    {}                      % Measure of space to indent.
    {\bfseries}             % Font of the header of the theorem.
    {}                      % Punctuation between head and body.
    {.5em}                  % Space after theorem head.
    {}

% Define default environments.
\theoremstyle{normal}
\newtheorem{examplex}{Example}

\newenvironment{example}{%
    \pushQED{\qed}\renewcommand{\qedsymbol}{$\blacksquare$}\examplex%
}{%
    \popQED\endexamplex%
}

\newenvironment{definition}{%
    \pushQED{\qed}\renewcommand{\qedsymbol}{$\blacksquare$}\definitionx%
}{%
    \popQED\enddefinitionx%
}

\title{Mathematical Induction}
\author{Ryan Maguire}
\date{Summer 2022}

% No indent and no paragraph skip.
\setlength{\parindent}{0em}
\setlength{\parskip}{0em}

\begin{document}
    \maketitle
    Peano arithmetic is a weaker system of axioms that is sufficient to prove a
    lot of the elementary properties of the natural numbers. The system of
    set theory we have been working with, Zermelo-Fraenkel set theory with the
    the axiom of choice (ZFC) contains Peano arithmetic as a \textit{subset}
    (via the axiom of infinity, and a few others). This system postulates the
    following:
    \begin{enumerate}
        \item There is a set $\mathbb{N}$ with $0\in\mathbb{N}$.
        \item There is a function $\sigma:\mathbb{N}\rightarrow\mathbb{N}$.
        \item If $m,n\in\mathbb{N}$ and $\sigma(m)=\sigma(n)$, then $m=n$.
        \item $\sigma(n)=0$ is always false.
        \item If $A\subseteq\mathbb{N}$ is such that $0\in{A}$ and
            $n\in{A}$ implies $\sigma(n)\in{A}$, then $A=\mathbb{N}$.
    \end{enumerate}
    This is constructable in ZFC, the function $\sigma$ is the
    \textit{plus one} function, $\sigma(n)=n+1$. This last property is
    crucial. It says if $A\subseteq\mathbb{N}$ is such that
    $0\in{A}$ and $n\in{A}$ implies $n+1\in{A}$, then $A=\mathbb{N}$. The
    intuition goes like this. $0\in{A}$ is true. But $0\in{A}$ implies
    $0+1\in{A}$, so $1\in{A}$. But $1\in{A}$ implies $1+1\in{A}$, so
    $2\in{A}$. But $2\in{A}$ implies $2+1\in{A}$, so $3\in{A}$. And so on.
    This fact is also provable in ZFC (it is an \textit{axiom} in Peano
    arithmetic). Containment $\in$ defines an order on the integers.
    Remember, in ZFC, that $0=\emptyset$, $1=\{\,0\,\}$,
    $2=\{\,0,\,1\,\}$, and so on. We write $m<n$ if and only if $m\in{n}$.
    This defines a \textit{well-order}. Every non-empty subset of $\mathbb{N}$
    has a \textit{smallest} element. This well-ordering property gives us
    property 5 in Peano arithmetic.
    \par\hfill\par
    Let $P$ be a predicate on the natural numbers. That is,
    for every $n\in\mathbb{N}$, $P(n)$ is a sentence which we may say is
    true or false. Suppose $P(0)$ is true, and the truth of $P(n)$ implies
    the truth of $P(n+1)$. Then $P(n)$ is true for all $n\in\mathbb{N}$. Why?
    Well, $P(1)$ is true since $P(0)$ is true, and $P(0)$ being true implies
    $P(0+1)$ is true. Then $P(2)$ is true since $P(1)$ is true and $P(1)$ being
    true implies $P(2)$ is true, and so on. We can prove this rigorously.
    \begin{theorem}[\textbf{Principle of Mathematical Induction}]
        If $P$ is a predicate on the natural numbers such that $P(0)$ is true
        and for all $n\in\mathbb{N}$ the truth of $P(n)$ implies the truth of
        $P(n+1)$, then $P(n)$ is true for all $n\in\mathbb{N}$.
    \end{theorem}
    \begin{proof}
        Suppose not. Then there is some $m\in\mathbb{N}$ such that
        $P(m)$ is false. Define $A$ via:
        \begin{equation}
            A=\{\,k\in\mathbb{N}\;|\;P(k)\textrm{ is false.}\,\}
        \end{equation}
        Since $m\in\mathbb{N}$, $A$ is non-empty. So there is a least element
        $N\in{A}$. But $N\ne{0}$ since $P(0)$ is true by hypothesis. Since
        $N\ne{0}$ and $N\in\mathbb{N}$, we can write $N=n+1$ for some
        $n\in\mathbb{N}$. But then $n<N$, and since $N$ is the least element
        of $A$, $P(n)$ must be true. But the truth of $P(n)$ implies the
        truth of $P(n+1)$, so $P(n+1)$ is true. But $n+1=N$, so $P(N)$ is true,
        a contradiction. Hence, $P(n)$ is true for all $n\in\mathbb{N}$.
    \end{proof}
    Let's use this.
    \begin{example}
        Consider the partial sums $S_{N}$ for $N\in\mathbb{N}$ defined by:
        \begin{equation}
            S_{N}=\sum_{n=0}^{N}n
        \end{equation}
        We can provide a closed-form formula for this via the principle of
        induction. We want to prove that:
        \begin{equation}
            S_{N}=\frac{N(N+1)}{2}
        \end{equation}
        That is, $P(N)$ is the predicate on the natural numbers that
        $S_{N}=N(N+1)/2$. We prove $P(N)$ is true for all $N\in\mathbb{N}$
        via induction. The case $N=0$ says
        $0=0(0+1)/2$, which is true. Suppose the statement is true for
        some $N\in\mathbb{N}$. We must prove this implies the statement is
        true for $N+1$. We compute:
        \begin{align}
            S_{N+1}
            &=\sum_{n=0}^{N+1}n\\
            &=N+1+\sum_{n=0}^{N}n\\
            &=N+1+S_{N}
        \end{align}
        But $P(N)$ is true, by hypothesis, so $S_{N}=N(N+1)/2$. We get:
        \begin{align}
            S_{N+1}
            &=N+1+S_{N}\\
            &=N+1+\frac{N(N+1)}{2}\\
            &=\frac{2N+2+N^{2}+N}{2}\\
            &=\frac{N^{2}+3N+2}{2}\\
            &=\frac{(N+1)(N+2)}{2}
        \end{align}
        and therefore $P(N+1)$ is true. Since the formula is true for
        $N=0$ and $P(N)$ implies $P(N+1)$, we see that by the principle of
        induction $P(N)$ is true for all $N\in\mathbb{N}$.    
    \end{example}
\end{document}
