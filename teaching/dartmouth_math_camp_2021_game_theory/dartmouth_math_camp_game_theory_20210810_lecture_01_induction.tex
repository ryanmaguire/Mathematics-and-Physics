%-----------------------------------LICENSE------------------------------------%
%   This file is part of Mathematics-and-Physics.                              %
%                                                                              %
%   Mathematics-and-Physics is free software: you can redistribute it and/or   %
%   modify it under the terms of the GNU General Public License as             %
%   published by the Free Software Foundation, either version 3 of the         %
%   License, or (at your option) any later version.                            %
%                                                                              %
%   Mathematics-and-Physics is distributed in the hope that it will be useful, %
%   but WITHOUT ANY WARRANTY; without even the implied warranty of             %
%   MERCHANTABILITY or FITNESS FOR A PARTICULAR PURPOSE.  See the              %
%   GNU General Public License for more details.                               %
%                                                                              %
%   You should have received a copy of the GNU General Public License along    %
%   with Mathematics-and-Physics.  If not, see <https://www.gnu.org/licenses/>.%
%------------------------------------------------------------------------------%
%   Author:     Ryan Maguire                                                   %
%   Date:       2021                                                           %
%------------------------------------------------------------------------------%
\documentclass{beamer}
\usepackage{caption}
\usepackage{subcaption}
\usepackage{amsmath}
\title{Math Camp Day 1}
\author{Ben Adenbaum, Juanita Duque-Rosero, Ryan Maguire, Kameron McCombs}
\date{August 2021}
\usenavigationsymbolstemplate{}
\setbeamertemplate{footline}[frame number]
\begin{document}
    \maketitle
    \begin{frame}{Outline}
        \begin{itemize}
            \item How do proofs work?
            \item What is \textit{not} a proof?
            \item What is proof by contradiction?
            \item What is mathematical induction?
        \end{itemize}
    \end{frame}
    \begin{frame}{Euclidean Geometry}
        Let's start with probably the most important
        textbook in the entire history of mathematics,%
        \footnote{%
            Possibly tied with Newton's \textit{Principia} for the most
            important text in the history of science.%
        }
        Euclid's \textit{Elements}. It is a foundational textbook for
        Euclidean geometry, which is the type of geometry one
        learns in high school.
    \end{frame}
    \begin{frame}{Euclidean Geometry}
        Euclid's book starts with 5 \textit{mathematical assumptions}. These are claims which Euclid's
        says are true and do not need any real justification or proof.%
        \footnote{The fancy word for this is \textit{axiom}. We'll just say assumption.}
        They are:
        \begin{enumerate}
            \item
                \textit{To draw a straight line from any point}
                \textit{to any point.}
            \item
                \textit{To produce a finite straight line continuously in}
                \textit{a straight line.}
            \item
                \textit{To describe a circle with any centre and}
                \textit{distance.}
            \item
                \textit{That all right angles are equal to one another.}
            \item
                \textit{That if a straight line falling on two straight}
                \textit{lines make the interior angles on the same side}
                \textit{less than two right angles, the two straight}
                \textit{lines, if produced indefinitely, meet on that}
                \textit{side on which are the angles less than two right}
                \textit{angles.}
        \end{enumerate}
    \end{frame}
    \begin{frame}{Euclidean Geometry}
        Only the fifth one is not obvious, and that is probably because it is very wordy.
        If we draw what it is describing, it becomes clear.
        \begin{figure}
            \centering
            \includegraphics{figs/euclids_fifth_axiom.eps}
            \caption{Euclid's Fifth Axiom}
            \label{fig:euclids_fifth_axiom}
        \end{figure}
    \end{frame}
    \begin{frame}{Euclidean Geometry}
        Let's now prove the first theorem of Euclid's book. It says that
        if you are given two points, it is possible to construct an
        \textit{equilateral triangle} from these two.
        \begin{figure}
            \centering
            \includegraphics{figs/euclid_book_1_proposition_1_euclids_solution.eps}
            \caption{Euclid 1.1}
            \label{fig:euclid_book_1_proposition_1_euclids_solution}
        \end{figure}
    \end{frame}
    \begin{frame}{Euclidean Geometry}
        Let's analyze what is needed for this proof. We need to draw
        line segments (assumption 1) and draw circles (assumption 3).
        So note that assumptions 2, 4, and 5 are never used. What's not
        too clear is why the point $C$ exists? What if we changed assumptions
        2, 4, and 5? Would the theorem still be true? Since our argument doesn't
        rely on these assumptions, it should!
    \end{frame}
    \begin{frame}{Euclidean Geometry}
        If we modify assumption 5, we encounter \textit{hyperbolic geometry}. Since
        assumption 5 is so very long, for 2000 years mathematicians tried to show it was
        unnecessary. The closest one came was from the great Persian mathematician
        Omar Khayyam, but he accidentally introduced another assumption. In the
        1800's the Scottish mathematician John Playfair showed that assumption 5 is
        \textit{independent} of the other 4. This means you can discard assumption 5
        or modify it, and you won't have a contradiction. Playfair showed assumption 5
        is equivalent to the following sentence:
        \begin{center}
            Given a line and a point not on that line, there is a unique line that is
            parallel to that line.
        \end{center}
    \end{frame}
    \begin{frame}{Euclidean Geometry}
        There are 2 ways to modify Playfair's assumption. We can say there are
        \textit{infinitely} many parallel lines. This is hyperbolic geometry.
        Euclid's first theorem holds in Hyperbolic geometry.
        \begin{figure}
            \centering
            \includegraphics{figs/Euclid_Book_I_Prop_I_Hyperbolic_Poincare_Disk.pdf}
            \caption{Euclid 1.1 in Hyperbolic Geometry}
            \label{fig:Euclid_Book_I_Prop_I_Hyperbolic_Poincare_Disk}
        \end{figure}
    \end{frame}
    \begin{frame}{Euclidean Geometry}
        The other ways is we say there are \textit{no} parallel lines. This is
        \textit{spherical} geometry. Euclid's first theorem does \textit{not} hold here.
        \begin{figure}
            \centering
            \includegraphics{figs/Euclid_Book_I_Prop_I_Spherical_Alt.pdf}
            \caption{Euclid 1.1 in Spherical Geometry}
            \label{fig:Euclid_Book_I_Prop_I_Spherical_Alt}
        \end{figure}
    \end{frame}
    \begin{frame}{Euclidean Geometry}
        When we move the points further away, the picture starts to look strange.
        \begin{figure}
            \centering
            \includegraphics{figs/Euclid_Book_I_Prop_I_Spherical.pdf}
            \caption{Euclid 1.1 in Spherical Geometry}
            \label{fig:Euclid_Book_I_Prop_I_Spherical}
        \end{figure}
    \end{frame}
    \begin{frame}{Euclidean Geometry}
        And when the points are really far apart, the theorem is no longer valid.
        \begin{figure}
            \centering
            \includegraphics{figs/Euclid_Book_I_Prop_I_Spherical_No_Intersect.pdf}
            \caption{Euclid 1.1 fails on the sphere}
            \label{fig:Euclid_Book_I_Prop_I_Spherical_No_Intersect}
        \end{figure}
    \end{frame}
    \begin{frame}{What's a proof}
        The take away is to be careful with assumptions, and be sure not to introduce
        other assumptions implicitly. This leads us to our definition of a proof.
        \begin{definition}[Proof]
            A proof of a theorem is a logical argument that combines a set of valid assumptions
            with other previously proved theorems to validate the truth of the claim.
        \end{definition}
        We'll talk about two types of proofs today, \textit{contradiction} and \textit{induction}.
    \end{frame}
    \begin{frame}{What's not a proof}
        Before introducing two valid types of proof, let's show two \textit{invalid} types.
        First is proof by trickery. We'll show $1=2$.
        \begin{align}
            1.)\quad&
            \text{Let $a$ and $b$ be non-zero numbers with $a=b$}
            \tag{Hypothesis}\\
            2.)\quad
            &\text{If $a=b$, then $a^{2}=ab$}
            \tag{Multiply both sides by $a$}\\
            3.)\quad
            &\text{If $a^{2}=ab$, then $a^{2}-b^{2}=ab-b^{2}$}
            \tag{Subtract $b^{2}$}\\
            4.)\quad
            &\text{But $a^{2}-b^{2}=(a-b)(a+b)$}
            \tag{Factoring}\\
            5.)\quad
            &\text{And $ab-b^{2}=(a-b)b$}
            \tag{Factoring}\\
            6.)\quad
            &\text{Therefore $(a-b)(a+b)=(a-b)b$}
            \tag{Laws of equality}\\
            7.)\quad
            &\text{Therefore $a+b=b$}
            \tag{Divide by $a-b$}\\
            8.)\quad
            &\text{But $a=b$, so $a+b=b+b=2b$}
            \tag{Hypothesis}\\
            9.)\quad
            &\text{Hence $2b=b$}
            \tag{Laws of equality}\\
            10.)\quad
            &\textit{But $b\ne{0}$, and therefore $2=1$}
            \tag{Divide by $b$}
        \end{align}
    \end{frame}
    \begin{frame}{What's not a proof}
        The problem is step 7 (divide by $a-b$). Since $a=b$, $a-b=0$ and we can't divide by zero.
        So this is not a proof, more of a sleight-of-hand. Next is proof by picture. We'll show
        that $31.5=32.5$.
        \begin{figure}
            \centering
            \includegraphics{figs/missing_square_puzzle_001.pdf}
            \caption{Missing Square Puzzle}
            \label{fig:missing_square_puzzle_001}
        \end{figure}
    \end{frame}
    \begin{frame}{What's not a proof}
        The problem is the first picture is not a triangle. It has four sides.
        The red and green triangles are not similar, so the \textit{hypotenuse} actually
        two lines.
        \begin{figure}
            \centering
            \includegraphics{figs/missing_square_puzzle_002.pdf}
            \caption{Missing Square Puzzle}
            \label{fig:missing_square_puzzle_002}
        \end{figure}
    \end{frame}
    \begin{frame}{What's not a proof}
        See?
        \begin{figure}
            \centering
            \includegraphics{figs/missing_square_puzzle_003.pdf}
            \caption{Missing Square Puzzle}
            \label{fig:missing_square_puzzle_003}
        \end{figure}
    \end{frame}
    \begin{frame}{What's not a proof}
        Don't get the idea that pictures are bad in mathematics. You just need to be careful.
        Below is a valid proof of Pythagoras' theorem using only pictures.
        \begin{figure}
            \centering
            \includegraphics{figs/pythagoras_theorem_proof.pdf}
            \caption{Pythagoras Theorem}
            \label{fig:pythagoras_theorem_proof}
        \end{figure}
    \end{frame}
    \begin{frame}{Proof by Contradiction}
        Proof by contradiction works based on the
        \textit{law of the excluded middle}. Given a claim, either it is
        true or its negation is true. A proof by contradiction works by showing the
        negation of the claim is impossible, and hence the original claim must be true.
    \end{frame}
    \begin{frame}{Proof by Contradiction}
        Here's an example you may know, $\sqrt{2}$ is \textit{irrational}. It cannot be written as
        $\sqrt{2}=\frac{p}{q}$ for integers $p$ and $q$. Let's prove this. We suppose it is, and then
        look for a contradiction. Suppose $\sqrt{2}=\frac{p}{q}$, and suppose we reduce $p$ and $q$
        as far as possible (Ex. $\frac{8}{6}=\frac{4}{3}$). Squaring, we get
        $2q^{2}=p^{2}$. This means $p$ is even (why?), so we can write $p=2k$. So we have
        $2q^{2}=(2k)^{2}=4k^{2}$, and hence $q^{2}=2k^{2}$. This means $q$ is even. But then
        $p$ and $q$ can be reduced further, a contradiction!
    \end{frame}
    \begin{frame}{Proof by Induction}
        We'll use proof by contradiction to prove the validity of
        proof by \textit{induction}. If for every non-negative integer
        we have a claim $P_{n}$, if $P_{0}$ is true, and if $P_{n}$ being true
        implies $P_{n+1}$ is true, then $P_{n}$ is true for all $n$. Intuitively this makes
        sense. $P_{0}$ is true. $P_{0}$ being true implies $P_{1}$ is true. But $P_{1}$ being
        true implies $P_{2}$ is true. But $P_{2}$ being true implies $P_{3}$ is true, and so on.
        We can prove this more rigorously.
    \end{frame}
    \begin{frame}{Proof by Induction}
        One of the assumptions of the non-negative integer is that if you have a
        collection of them, then there is a smallest number in your collection.
        This assumption is reasonable. You can ask if 0 is in the list. If yes, that's
        the smallest, if not move on to 1, and so on. Eventually you'll find a number that
        is in your collection, and that's the smallest. This is called the
        \textit{Well-Ordering} principle of the integers.
    \end{frame}
    \begin{frame}{Proof by Induction}
        \begin{theorem}
            If $P_{n}$ is a claim for each non-negative integer $n$, if
            $P_{0}$ is true, and if $P_{n}$ being true implies $P_{n+1}$ is true,
            then $P_{n}$ is always true.
        \end{theorem}
        \begin{proof}
            Suppose not. Since $P_{0}$ is true, and since $P_{n}$ is not always true,
            there must be some least integer $N$ such that $P_{N}$ is not true. But then
            $P_{N-1}$ is true. But $P_{N-1}$ being true implies that
            $P_{(N-1)+1}=P_{N}$ is true, so $P_{N}$ is true, a contradiction! Hence,
            $P_{n}$ is always true.
        \end{proof}
    \end{frame}
    \begin{frame}{Proof by Induction}
        To use induction, we prove the base case $P_{0}$. We then prove the
        \textit{inductive} step $P_{n}$ being true imples $P_{n+1}$ is true.
        Here's a simple example of using this.
        $1+1+\cdots+1$, where there are $n$ 1's, is equal to $n$. The base case is
        $1=1$, which is true. The inductive step is, if
        $1+1+\cdots+1=n$ ($n$ 1's), then $1+1+\cdots+1+1=n+1$ ($n+1$ 1's). To prove this,
        we do:
        \begin{align}
            1+1+\cdots+1+1&=(1+1+\cdots+1)+1\\
                &=(n)+1\\
                &=n+1
        \end{align}
    \end{frame}
    \begin{frame}{Proof by Induction}
        A slightly harder example, but very useful for computers.
        If $b$ is a non-negative integer and $b>1$, and if $N$ is a non-negative integer,
        then there is a unique way to write:
        \begin{equation}
            N=a_{0} + a_{1}\cdot{b} + a_{2}\cdot{b}^{2}+\cdots+a_{m}\cdot{b}^{m}
        \end{equation}
        for some non-negative integer $m$ and with each $a_{n}<b$. In the case $b=2$ this is called
        \textit{binary}, but the case of $b=8$ and $b=16$ are equally useful for computers, and
        these are called \textit{octal} and \textit{hexidecimal}.
    \end{frame}
    \begin{frame}{Proof by Induction}
        Some examples:
        \begin{align}
            7&=4+2+1\\
                &=111_{2}\\
            47&=32+15\\
                &=2\cdot{16}+15\\
                &=2F_{16}
        \end{align}
        Some explanation of $2F_{16}$. Hexidecimal has 16 symbols, whereas decimal has 10.
        The symbols are 0, 1, 2, 3, 4, 5, 6, 7, 8, 9, $A$, $B$, $C$, $D$, $E$, $F$.
    \end{frame}
    \begin{frame}{Proof by Induction}
        Let's prove this. We do so by induction. The base case
        is $N=0$, which is just 0 in every system. Suppose $N$ can be written in base $b$.
        Let $m$ be the largest integer such that $b^{m}\leq{N+1}$. But then
        $N+1-b^{m}$ is a non-negative integer less than or equal to $N$, so by induction we
        can write $N+1-b^{m}$ in base $b$. To write $N+1$ in base $b$, we simply add $b^{m}$ back.
    \end{frame}
    \begin{frame}{Proof by Induction}
        We'll end induction with a group activity. We're going to prove the following:
        \begin{equation}
            1+2+\cdots+n=\frac{n(n+1)}{2}
        \end{equation}
        To prove this you must:
        \begin{enumerate}
            \item Prove the base case.
            \item Prove the inductive step.
        \end{enumerate}
    \end{frame}
    \begin{frame}{Proof by Induction}
        The base case is $n=1$, so we get:
        \begin{equation}
            \frac{1(1+1)}{2}=\frac{1\cdot{2}}{2}=1
        \end{equation}
        and $1=1$ is true. The inductive step is:
        \begin{align}
            1+2+\cdots{+}n+n+1
                &=(1+2+\cdots{+}n)+n+1\\
                &=\frac{n(n+1)}{2}+n+1\\
                &=\frac{n^{2}+n+2n+2}{2}\\
                &=\frac{n^{2}+3n+2}{2}\\
                &=\frac{(n+1)(n+2)}{2}
        \end{align}
    \end{frame}
\end{document}
