%-----------------------------------LICENSE------------------------------------%
%   This file is part of Mathematics-and-Physics.                              %
%                                                                              %
%   Mathematics-and-Physics is free software: you can redistribute it and/or   %
%   modify it under the terms of the GNU General Public License as             %
%   published by the Free Software Foundation, either version 3 of the         %
%   License, or (at your option) any later version.                            %
%                                                                              %
%   Mathematics-and-Physics is distributed in the hope that it will be useful, %
%   but WITHOUT ANY WARRANTY; without even the implied warranty of             %
%   MERCHANTABILITY or FITNESS FOR A PARTICULAR PURPOSE.  See the              %
%   GNU General Public License for more details.                               %
%                                                                              %
%   You should have received a copy of the GNU General Public License along    %
%   with Mathematics-and-Physics.  If not, see <https://www.gnu.org/licenses/>.%
%------------------------------------------------------------------------------%
\documentclass{article}
\usepackage{graphicx}                           % Needed for figures.
\usepackage{amsmath}                            % Needed for align.
\usepackage{amssymb}                            % Needed for mathbb.
\usepackage{amsthm}                             % For the theorem environment.
\usepackage{float}

\newtheoremstyle{normal}
    {\topsep}               % Amount of space above the theorem.
    {\topsep}               % Amount of space below the theorem.
    {}                      % Font used for body of theorem.
    {}                      % Measure of space to indent.
    {\bfseries}             % Font of the header of the theorem.
    {}                      % Punctuation between head and body.
    {.5em}                  % Space after theorem head.
    {}

\theoremstyle{normal}
\newtheorem{definition}{Definition}
\newtheorem{notation}{Notation}
\newtheorem{example}{Example}

\theoremstyle{plain}
\newtheorem{theorem}{Theorem}
\newcommand{\ceil}[2][]{#1\lceil#2#1\rceil}

\title{Velocity and Displacement}
\author{Math 3}
\date{Fall 2021}

% No indent and no paragraph skip.
\setlength{\parindent}{0em}
\setlength{\parskip}{0em}

\begin{document}
    \maketitle
    If $r:[t_{0},t_{1}]\rightarrow\mathbb{R}$ describes the position of a
    particle in an interval of time $[t_{1},t_{1}]$, and if the velocity is
    defined by $v(t)=r'(t)$, how can we relate the total displacement of the
    particle to the velocity? The total dispacement is
    $\Delta{r}=r(t_{1})-r(t_{0})$.
    \par\hfill\par
    The acceleration of gravity on the surface of the Earth is roughly
    $-9.81$ meters per second per second. Let's round this to $-10$ meters per
    second per second. The position of a particle that is dropped from 5
    meters is then:
    \begin{equation}
        r(t)=-5t^{2}+5
    \end{equation}
    The velocity of the particle is the derivative with respect to time.
    \begin{equation}
        v(t)=-10t
    \end{equation}
    Split the interval $[0,1]$ into a partition and numerically integrate
    the following:
    \begin{equation}
        D=\int_{0}^{1}v(t)\;\textrm{d}t
    \end{equation}
    \begin{enumerate}
        \item
        What physical quantity does $D$ represent?
        \item
        Numerically, what value did you get for the integral?
        \item
        Compare this with $r(1)-r(0)$. What can you conclude?
    \end{enumerate}
    \newpage
    I, the copyright holder of this work, release it into the public domain.
    This applies worldwide. In some countries this may not be legally possible;
    if so: I grant anyone the right to use this work for any purpose, without
    any conditions, unless such conditions are required by law.
    \par\hfill\par
    The source code used to generate this document is free software and released
    under version 3 of the GNU General Public License.
\end{document}