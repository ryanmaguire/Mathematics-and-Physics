%-----------------------------------LICENSE------------------------------------%
%   This file is part of Mathematics-and-Physics.                              %
%                                                                              %
%   Mathematics-and-Physics is free software: you can redistribute it and/or   %
%   modify it under the terms of the GNU General Public License as             %
%   published by the Free Software Foundation, either version 3 of the         %
%   License, or (at your option) any later version.                            %
%                                                                              %
%   Mathematics-and-Physics is distributed in the hope that it will be useful, %
%   but WITHOUT ANY WARRANTY; without even the implied warranty of             %
%   MERCHANTABILITY or FITNESS FOR A PARTICULAR PURPOSE.  See the              %
%   GNU General Public License for more details.                               %
%                                                                              %
%   You should have received a copy of the GNU General Public License along    %
%   with Mathematics-and-Physics.  If not, see <https://www.gnu.org/licenses/>.%
%------------------------------------------------------------------------------%
\documentclass{article}
\usepackage{graphicx}                           % Needed for figures.
\usepackage{amsmath}                            % Needed for align.
\usepackage{amssymb}                            % Needed for mathbb.
\usepackage{amsthm}                             % For the theorem environment.
\usepackage{float}

\newtheoremstyle{normal}
    {\topsep}               % Amount of space above the theorem.
    {\topsep}               % Amount of space below the theorem.
    {}                      % Font used for body of theorem.
    {}                      % Measure of space to indent.
    {\bfseries}             % Font of the header of the theorem.
    {}                      % Punctuation between head and body.
    {.5em}                  % Space after theorem head.
    {}

\theoremstyle{normal}
\newtheorem{definition}{Definition}
\newtheorem{notation}{Notation}
\newtheorem{example}{Example}

\theoremstyle{plain}
\newtheorem{theorem}{Theorem}
\newcommand{\ceil}[2][]{#1\lceil#2#1\rceil}

\title{First Fundamental Theorem of Calculus}
\author{Ryan Maguire}
\date{Fall 2021}

% No indent and no paragraph skip.
\setlength{\parindent}{0em}
\setlength{\parskip}{0em}

\begin{document}
    \maketitle
    The fundamental theorem of calculus relates integration and differentiation.
    It can be used to rigorously find the anti-derivative of various functions,
    and to explicitly solve the area under certain curves. The proof isn't too
    bad either, but first let's get some intuition as to \textit{why} it is
    true. The theorem states that if $f:[a,b]\rightarrow\mathbb{R}$ is a
    continuous function, and if $F:[a,b]\rightarrow\mathbb{R}$ is defined by:
    \begin{equation}
        F(x)=\int_{a}^{x}f(t)\;\textrm{d}t
    \end{equation}
    Then $F$ is differentiable, and:
    \begin{equation}
        \frac{\textrm{d}F}{\textrm{d}x}(x)=f(x)
    \end{equation}
    Written more concisely:
    \begin{equation}
        \frac{\textrm{d}}{\textrm{d}x}\int_{a}^{x}f(t)\;\textrm{d}t=f(x)
    \end{equation}
    Intuitively this makes sense, we just need to understand the picture.
    $\textrm{d}\int_{a}^{x}f(t)\textrm{d}x$ asks how much more
    area do we get if we increment $x$ infinitesimally so. That is, if we
    increment from $x$ to $x+\textrm{d}x$ for an infinitesimally small
    $\textrm{d}x$, how much more area do we have? Well, the height at the point
    $x$ times the displacement, or $f(x)\textrm{d}x$. The value
    $\frac{\textrm{d}}{\textrm{d}x}\int_{a}^{x}f(t)\textrm{d}t$ then asks for
    the ratio of this change with respect to the increment $\textrm{d}x$.
    Informally we have $f(x)\textrm{d}x/\textrm{d}x$ and the $\textrm{d}x$
    cancel, leaving $f(x)$. So, the \textit{rate of change} of the area under
    $f$ as a function of $x$ is simply the height of $f$, or $f(x)$. This
    geometrically makes sense.
    \begin{figure}[H]
        \centering
        \includegraphics{../../images/fundamental_theorem_of_calculus_1.pdf}
        \caption{Intuition for the Fundamental Theorem of Calculus}
    \end{figure}
    Unfortunately, all of this is very
    \textit{imprecise} and may have passed as a proof in the 1600's, but not
    in modern times. We need to be more rigorous. We start with the
    \textit{mean value theorem for integrals}.
    \begin{theorem}
        If $f:[a,b]\rightarrow\mathbb{R}$ is a continuous function, $a<b$,
        then there is a point $c\in[a,b]$ such that:
        \begin{equation}
            f(c)=\frac{1}{b-a}\int_{a}^{b}f(x)\;\textrm{d}x
        \end{equation}
    \end{theorem}
    \begin{proof}
        Note that $\int_{a}^{b}f(x)\textrm{d}x$ is just a number. Label this
        number $A$ (for Area). Define $g:[a,b]\rightarrow\mathbb{R}$ to be:
        \begin{equation}
            g(x)=f(x)(b-a)-A
        \end{equation}
        Since $f$ is continuous, $g$ is continuous. But since $f$ is continuous,
        by the extreme value theorem there are points $x_{\textrm{min}}$ and
        $x_{\textrm{max}}$ in $[a,b]$ where $f(x)$ achieves its minimum and
        maximum, respectively. Then $g(x_{\textrm{min}})$ will be
        non-positive since the upper Riemann sum is always greater than or
        equal $f(x_{\textrm{min}})(b-a)$, and $g(x_{\textrm{max}})$ will be
        non-negative. By the intermediate value theorem, since $g$ is
        continuous, there is some point $c\in[a,b]$ where $g(c)=0$. But then:
        \begin{equation}
            f(c)(b-a)-\int_{a}^{b}f(x)\;\textrm{d}x=0
        \end{equation}
        Which completes the proof.
    \end{proof}
    \begin{figure}[H]
        \centering
        \includegraphics{../../images/mean_value_theorem_for_integrals.pdf}
        \caption{Intuition for the Fundamental Theorem of Calculus}
    \end{figure}
    We can now prove the fundamental theorem of calculus. We have:
    \begin{align}
        \frac{\textrm{d}}{\textrm{d}x}\int_{a}^{x}f(t)\;\textrm{d}t
            &=\lim_{h\rightarrow{0}}\frac{1}{h}\Big(
                \int_{a}^{x+h}f(t)\;\textrm{d}t-
                \int_{a}^{x}f(t)\;\textrm{d}t\Big)\\
            &=\lim_{h\rightarrow{0}}\frac{1}{h}
                \int_{x}^{x+h}f(t)\;\textrm{d}t\\
            &=\lim_{h\rightarrow{0}}\frac{1}{h}f(c_{h})(x+h-x)\\
            &=\lim_{h\rightarrow{0}}\frac{1}{h}f(c_{h})h\\
            &=\lim_{h\rightarrow{0}}f(c_{h})
    \end{align}
    Here, $c_{h}$ is a value between $x$ and $x+h$ such that:
    \begin{equation}
        f(c_{h})=\frac{1}{x+h-x}\int_{x}^{x+h}f(t)\;\textrm{d}t
    \end{equation}
    which exists by the mean value theorem for integrals.
    Since $c_{h}\in[x,x+h]$, as $h$ tends to zero, $c_{h}$ tends to x.
    But $f$ is continuous, and therefore:
    \begin{equation}
        \lim_{h\rightarrow{0}}f(c_{h})=f(x)
    \end{equation}
    Which completes the proof.
    \newpage
    I, the copyright holder of this work, release it into the public domain.
    This applies worldwide. In some countries this may not be legally possible;
    if so: I grant anyone the right to use this work for any purpose, without
    any conditions, unless such conditions are required by law.
    \par\hfill\par
    The source code used to generate this document is free software and released
    under version 3 of the GNU General Public License.
\end{document}