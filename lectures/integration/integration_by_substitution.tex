%-----------------------------------LICENSE------------------------------------%
%   This file is part of Mathematics-and-Physics.                              %
%                                                                              %
%   Mathematics-and-Physics is free software: you can redistribute it and/or   %
%   modify it under the terms of the GNU General Public License as             %
%   published by the Free Software Foundation, either version 3 of the         %
%   License, or (at your option) any later version.                            %
%                                                                              %
%   Mathematics-and-Physics is distributed in the hope that it will be useful, %
%   but WITHOUT ANY WARRANTY; without even the implied warranty of             %
%   MERCHANTABILITY or FITNESS FOR A PARTICULAR PURPOSE.  See the              %
%   GNU General Public License for more details.                               %
%                                                                              %
%   You should have received a copy of the GNU General Public License along    %
%   with Mathematics-and-Physics.  If not, see <https://www.gnu.org/licenses/>.%
%------------------------------------------------------------------------------%
\documentclass{article}
\usepackage{graphicx}                           % Needed for figures.
\usepackage{amsmath}                            % Needed for align.
\usepackage{amssymb}                            % Needed for mathbb.
\usepackage{amsthm}                             % For the theorem environment.
\usepackage{float}

\newtheoremstyle{normal}
    {\topsep}               % Amount of space above the theorem.
    {\topsep}               % Amount of space below the theorem.
    {}                      % Font used for body of theorem.
    {}                      % Measure of space to indent.
    {\bfseries}             % Font of the header of the theorem.
    {}                      % Punctuation between head and body.
    {.5em}                  % Space after theorem head.
    {}

\theoremstyle{normal}
\newtheorem{definition}{Definition}
\newtheorem{notation}{Notation}
\newtheorem{example}{Example}

\theoremstyle{plain}
\newtheorem{theorem}{Theorem}
\newcommand{\ceil}[2][]{#1\lceil#2#1\rceil}

\title{Integration by Substitution}
\author{Ryan Maguire}
\date{Fall 2021}

% No indent and no paragraph skip.
\setlength{\parindent}{0em}
\setlength{\parskip}{0em}

\begin{document}
    \maketitle
    The main techniques of evaluating integrals merely combine the rules for
    differentiation (in reverse) via the fundamental theorem of calculus. Let's
    look at the chain rule. It says if we have two differentiable functions
    $f$ and $g$, then:
    \begin{equation}
        (g\circ{f})'(x)=g'\big(f(x)\big)f'(x)
    \end{equation}
    Let's integrate this. The fundamental theorem of calculus says:
    \begin{equation}
        \int_{a}^{b}(g\circ{f})'(x)\;\textrm{d}x
            =(g\circ{f})(b)-(g\circ{f})(a)
    \end{equation}
    But we have a formula for $(g\circ{f})'(x)$ above. Combining this, we have:
    \begin{equation}
        \int_{a}^{b}g'\big(f(x)\big)f'(x)\;\textrm{d}x
            =g\big(f(b)\big)-g\big(f(a)\big)
    \end{equation}
    This method of integration is usually called $u$ substitution. Why?
    Well, let's set $u=f(x)$. We then have:
    \begin{equation}
        \int_{a}^{b}g'\big(f(x)\big)f'(x)\;\textrm{d}x
            =\int_{a}^{b}g'(u)\frac{\textrm{d}u}{\textrm{d}x}\;\textrm{d}x
            =g\big(f(b)\big)-g\big(f(a)\big)
    \end{equation}
    But $g(f(b))-g(f(a))$ is also equal to the following:
    \begin{equation}
        g(f(b))-g(f(a))=\int_{f(a)}^{f(b)}g'(u)\;\textrm{d}u
    \end{equation}
    Again, this is from the fundamental theorem of calculus. This is all
    rigorously justified by the chain rule and the fundamental theorem of
    calculus. Let's forget rigor for a second and direct our attention to the
    previous equation:
    \begin{equation}
        \int_{a}^{b}g'(u)\frac{\textrm{d}u}{\textrm{d}x}\;\textrm{d}x
            =g\big(f(b)\big)-g\big(f(a)\big)
    \end{equation}
    It's almost as if the $\frac{\textrm{d}u}{\textrm{d}x}\textrm{d}x$ cancels
    and simplifies to $\textrm{d}u$ and the limits change from $a$ to $b$ and
    become $f(a)$ to $f(b)$. This is a great way to remember this, even if it
    isn't a rigorous statement. We know the thinking is justified by other
    means.
    \par\hfill\par
    To use the method of substitution, when trying to integrate a function
    $f$ we try to break it into a product $g(u)u'$. For example, let's
    integrate $2x/(1+x^{2})$.
    \begin{align}
        \int_{0}^{1}\frac{2x}{1+x^{2}}\;\textrm{d}x
            &=\int_{0}^{1}\frac{1}{1+x^{2}}\frac{\textrm{d}}{\textrm{d}x}
                \big(1+x^{2}\big)\;\textrm{d}x\\
            &=\int_{0}^{1}\frac{\textrm{d}}{\textrm{d}x}
                \ln(1+x^{2})\;\textrm{d}x\\
            &=\ln(1+x^{2})\big|_{0}^{1}\\
            &=\ln(1+1^{2})-ln(1+0^{2})\\
            &=\ln(2)-\ln(1)\\
            &=\ln(1)
    \end{align}
    Recognizing that $2x/(1+x^{2})$ is the derivative of $\ln(1+x^{2})$ is hard
    so instead we use $u$ substitution. I see that
    $2x$ is the derivative of $1+x^{2}$. So I set $u=1+x^{2}$. Then
    $\textrm{d}u=2x\;\textrm{d}x$. So:
    \begin{align}
        \int_{0}^{1}\frac{2x}{1+x^{2}}\;\textrm{d}x
            &=\int_{0}^{1}\frac{2x\;\textrm{d}x}{1+x^{2}}\\
            &=\int_{?}^{?}\frac{\textrm{d}u}{u}
    \end{align}
    What do the limits of integration become when we perform a $u$ substitution?
    This comes from the chain rule. The limits go from being $a$ to $b$ to
    being $u(a)$ to $u(b)$. We have $u=1+x^{2}$ so $u(0)=1$ and $u(1)=2$.
    Our integral is then:
    \begin{equation}
        \int_{1}^{2}\frac{1}{u}\;\textrm{d}u=\ln(u)\big|_{1}^{2}
            =\ln(2)-\ln(1)
    \end{equation}
    Which is precisely what we got before.
    \newpage
    I, the copyright holder of this work, release it into the public domain.
    This applies worldwide. In some countries this may not be legally possible;
    if so: I grant anyone the right to use this work for any purpose, without
    any conditions, unless such conditions are required by law.
    \par\hfill\par
    The source code used to generate this document is free software and released
    under version 3 of the GNU General Public License.
\end{document}