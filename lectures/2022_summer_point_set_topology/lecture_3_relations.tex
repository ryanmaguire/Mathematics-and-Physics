%-----------------------------------LICENSE------------------------------------%
%   This file is part of Mathematics-and-Physics.                              %
%                                                                              %
%   Mathematics-and-Physics is free software: you can redistribute it and/or   %
%   modify it under the terms of the GNU General Public License as             %
%   published by the Free Software Foundation, either version 3 of the         %
%   License, or (at your option) any later version.                            %
%                                                                              %
%   Mathematics-and-Physics is distributed in the hope that it will be useful, %
%   but WITHOUT ANY WARRANTY; without even the implied warranty of             %
%   MERCHANTABILITY or FITNESS FOR A PARTICULAR PURPOSE.  See the              %
%   GNU General Public License for more details.                               %
%                                                                              %
%   You should have received a copy of the GNU General Public License along    %
%   with Mathematics-and-Physics.  If not, see <https://www.gnu.org/licenses/>.%
%------------------------------------------------------------------------------%
\documentclass{article}
\usepackage{graphicx}                           % Needed for figures.
\usepackage{amsmath}                            % Needed for align.
\usepackage{amssymb}                            % Needed for mathbb.
\usepackage{amsthm}                             % For the theorem environment.
\usepackage{float}
\usepackage{hyperref}
\hypersetup{
    colorlinks=true,
    linkcolor=blue,
    filecolor=magenta,
    urlcolor=Cerulean,
    citecolor=SkyBlue
}

%------------------------Theorem Styles-------------------------%
\theoremstyle{plain}
\newtheorem{theorem}{Theorem}[section]

% Define theorem style for default spacing and normal font.
\newtheoremstyle{normal}
    {\topsep}               % Amount of space above the theorem.
    {\topsep}               % Amount of space below the theorem.
    {}                      % Font used for body of theorem.
    {}                      % Measure of space to indent.
    {\bfseries}             % Font of the header of the theorem.
    {}                      % Punctuation between head and body.
    {.5em}                  % Space after theorem head.
    {}

% Define default environments.
\theoremstyle{normal}
\newtheorem{examplex}{Example}[section]
\newtheorem{definitionx}{Definition}[section]
\newtheorem{notationx}{Notation}[section]

\newenvironment{example}{%
    \pushQED{\qed}\renewcommand{\qedsymbol}{$\blacksquare$}\examplex%
}{%
    \popQED\endexamplex%
}

\newenvironment{definition}{%
    \pushQED{\qed}\renewcommand{\qedsymbol}{$\blacksquare$}\definitionx%
}{%
    \popQED\enddefinitionx%
}

\newenvironment{notation}{%
    \pushQED{\qed}\renewcommand{\qedsymbol}{$\blacksquare$}\notationx%
}{%
    \popQED\endnotationx%
}

\title{Point-Set Topology: Lecture 3}
\author{Ryan Maguire}
\date{Summer 2022}

% No indent and no paragraph skip.
\setlength{\parindent}{0em}
\setlength{\parskip}{0em}

\begin{document}
    \maketitle
    \section{More Cardinality}
        In the last lecture we showed
        $\textrm{Card}(\mathbb{N})=\textrm{Card}(\mathbb{Z})=\textrm{Card}(\mathbb{Q})$.
        A \textit{countably infinite} set is a set that can be put into a
        bijection with $\mathbb{N}$. A \textit{countable} set is a set that
        is either countably infinite or finite. An \textit{uncountable} set is
        a set that is infinite but not countable. We now arrive at our first
        uncountable set, the real numbers $\mathbb{R}$. Suppose they are
        countable. Then there is a bijection
        $f:\mathbb{N}\rightarrow\mathbb{R}$. For simplicity, let us assume there
        is a bijection $f:\mathbb{N}\rightarrow(0,1)$. Then we can write out
        this bijection with a list.
        \begin{align}
            f(0)&=0.d_{0,0}d_{0,1}d_{0,2}d_{0,3}d_{0,4}d_{0,5}\dots\\
            f(1)&=0.d_{1,0}d_{1,1}d_{1,2}d_{1,3}d_{1,4}d_{1,5}\dots\\
            f(2)&=0.d_{2,0}d_{2,1}d_{2,2}d_{2,3}d_{2,4}d_{2,5}\dots\\
            f(3)&=0.d_{3,0}d_{3,1}d_{3,2}d_{3,3}d_{3,4}d_{3,5}\dots\\
            f(4)&=0.d_{4,0}d_{4,1}d_{4,2}d_{4,3}d_{4,4}d_{4,5}\dots\\
            f(5)&=0.d_{5,0}d_{5,1}d_{5,2}d_{5,3}d_{5,4}d_{5,5}\dots
        \end{align}
        where $d_{m,n}$ is the decimal of the $m^{th}$ number in the
        $n^{th}$ decimal place. Since the bijection is between
        $\mathbb{N}$ and $(0,1)$, the interger part of each $f(n)$ is zero.
        We now show that $f$ is not a bijection by giving a new number that is
        not on the list. Define $r\in(0,1)$ as follows:
        \begin{equation}
            r=0.r_{0}r_{1}r_{2}r_{3}r_{4}r_{5}\dots
        \end{equation}
        where
        \begin{equation}
            r_{n}=
            \begin{cases}
                d_{n,n}+1&d_{n,n}\ne{9}\\
                0&d_{n,n}=9
            \end{cases}
        \end{equation}
        This number is not equal to $f(n)$ for any $n$. It is not
        $f(0)$ since $r_{0}$ and $d_{0,0}$ are different. It is not
        $f(1)$ since $r_{1}$ and $d_{1,1}$ differ. Similarly, it is not
        $f(n)$ since $r_{n}$ and $d_{n,n}$ are not the same decimal. So
        $r$ is not on our list, meaning $f(n)\ne{r}$ for any $n\in\mathbb{N}$,
        contradicting the fact that $f$ is a bijection.
        \par\hfill\par
        There are small gaps here, meaning this is a \textit{sketch of proof}
        and not a full proof. The argument does not take into account the fact
        that $0.1=0.0\overline{9}$, for example, but this can be corrected.
        \begin{theorem}[\textbf{Cantor's Theorem}]
            If $A$ is a set, then there is an injective function
            $f:A\rightarrow\mathcal{P}(A)$, where $\mathcal{P}(A)$ is the power
            set of $A$, but there exists no surjection, and hence no bijection.
        \end{theorem}
        \begin{proof}
            Homework.
        \end{proof}
        There is a bijection from $\mathcal{P}(\mathbb{N})$ to
        $\mathbb{R}$. Again, a sketch of proof is given. We'll construct a
        surjection $f:\mathcal{P}(\mathbb{N})\rightarrow[0,1]$, the closed
        unit interval. Given a set $A\subseteq\mathbb{N}$, construct the number
        $r\in[0,1]$ using binary as follows:
        \begin{equation}
            f(A)=r=0.r_{0}r_{1}r_{2}\dots
        \end{equation}
        where $r_{n}=1$ if $n\in{A}$ and $r_{n}=0$ if $n\notin{A}$. For
        example, if $A=\emptyset$, then
        $f(\emptyset)=0.000\dots=0$. If $A=\mathbb{N}$, then
        $f(\mathbb{N})=0.111\dots=1$. If $A=\{\,0,\,2,\,4,\,\dots\,\}$, then
        $f(A)=0.101010\dots$. If $A=\{\,1,\,2,\,3\,\}$, then
        $f(A)=0.01110000\dots$. Since every number $r\in[0,1]$ can be written
        in binary form in such a way, $f$ is a surjection. We can reverse this
        process as well, but again the issue of $1$ vs. $0.\overline{9}$ arises
        and needs correcting. It is possible to do this, but not currently worth
        our time investigating.
        \par\hfill\par
        You may now ask \textit{this is a bijection from the natural numbers to}
        \textit{the closed unit interval}. \textit{What about} $\mathbb{R}$?
        We can construct a bijection $g:[0,1]\rightarrow(0,1)$, the closed
        unit interval to the open unit interval, via:
        \begin{equation}
            g(x)=
            \begin{cases}
                \frac{1}{2}&x=0\\
                \frac{x}{4}&x=\frac{1}{2^{n}}\textrm{ for some }n\in\mathbb{N}\\
                x&\textrm{otherwise}
            \end{cases}
        \end{equation}
        The graph is shown in Fig.~\ref{fig:bijection_closed_to_open_interval}.
        We will eventually prove that there is no \textit{continuous} bijection
        $f:[0,1]\rightarrow(0,1)$. For those interested, try to find a
        bijection $f:[0,1]\rightarrow(0,1)$ that has only
        \textit{finitely many} discontinuities.
        \par\hfill\par
        Using this bijection $g$, we need a bijection from $(0,1)$ to
        $\mathbb{R}$. This is given by:
        \begin{equation}
            h(x)=\frac{2x-1}{x(1-x)}
        \end{equation}
        By composing $h\circ{g}\circ{f}$ we get a bijection from
        $\mathcal{P}(\mathbb{N})$ to $\mathbb{R}$. This means that cardinality
        is \textit{transitive}.
        \begin{theorem}
            If $\textrm{Card}(A)=\textrm{Card}(B)$ and
            $\textrm{Card}(B)=\textrm{Card}(C)$, then
            $\textrm{Card}(A)=\textrm{Card}(C)$.
        \end{theorem}
        \begin{proof}
            Since $A$ and $B$ are of the same cardinality, there is a
            bijection $f:A\rightarrow{B}$. Similarly, there is a bijection
            $g:B\rightarrow{C}$. By composing we get a bijection
            $g\circ{f}:A\rightarrow{C}$, meaning
            $\textrm{Card}(A)=\textrm{Card}(C)$.
        \end{proof}
        \begin{figure}
            \centering
            \includegraphics{../../images/bijection_closed_to_open_interval.pdf}
            \caption{Bijection from $[0,1]$ to $(0,1)$}
            \label{fig:bijection_closed_to_open_interval}
        \end{figure}
        \begin{figure}
            \centering
            \includegraphics{../../images/2x_minus_1_over_x_times_1_minus_x.pdf}
            \caption{Bijection from $(0,1)$ to $\mathbb{R}$}
            \label{fig:2x_minus_1_over_x_times_1_minus_x}
        \end{figure}
    \section{Relations}
        Relations are ways of saying certain elements of a set are related to
        each other. There are many relations you use daily in mathematics.
        Equality $(=)$, less than $(<)$, greater than $(>)$, less than or equal
        $(\leq)$, and greater than or equal $(\geq)$. We've also seen relations
        on sets such as \textit{inclusion} $(\subseteq)$ and
        \textit{proper inclusion} $(\subsetneq)$. Cardinality can also be
        thought of as a type of relation on sets. The most general definition
        of a relation is as follows.
        \begin{definition}[\textbf{Relation}]
            A relation on a set $A$ is a subset $R\subseteq{A}\times{A}$.
        \end{definition}
        If $(a,b)\in{R}$ we write this as $aRb$.
        \begin{example}
            Suppose we know what \textit{less than} means for real numbers.
            We can define $<$ to be the set:
            \begin{equation}
                <\;=\{\,(a,b)\in\mathbb{R}\times\mathbb{R}\;|\;
                    a\textrm{ is less than }b\,\}
            \end{equation}
            Rather than writing $(a,b)\in{<}$, we write
            $a<b$. It's weird to think of the symbol $<$ as a set, and in
            practice we don't. We think of it as a way of relating elements
            in $\mathbb{R}$. Similarly, for a set $A$ and a relation $R$,
            you should think of $R$ as a way of relating elements.
        \end{example}
        \begin{example}
            The natural numbers can be given a precise construction. We write
            $0=\emptyset$, $1=\{\,0\,\}$, $2=\{\,0,\,1\,\}$,
            $3=\{\,0,\,1,\,2\,\}$, and so on. We can now define
            $<$ on $\mathbb{N}$ as follows:
            \begin{equation}
                <\;=\{\,(m,n)\in\mathbb{N}\times\mathbb{N}\;|\;m\in{n}\,\}
            \end{equation}
            This is bizarre, but makes precise what inequality means.
            Since $3=\{\,0,\,1,\,2\,\}$, we see that $1\in{3}$, meaning
            we can write $1<3$. This is in agreement with the way we intuitively
            think of the \textit{less than} relation.
        \end{example}
        \begin{example}
            If $X$ is a set, and $R\subseteq\mathcal{P}(X)\times\mathcal{P}(X)$
            is defined by:
            \begin{equation}
                R=\{\,(A,B)\in\mathcal{P}(X)\times\mathcal{P}(X)\;|\;
                    A\subseteq{B}\,\}
            \end{equation}
            then $R$ is the \textit{inclusion} relation on the set of all
            subsets of $X$.
        \end{example}
        Since the definition of relation is so general
        (\textit{any} subset of $A\times{A}$), it is often the case the we
        restrict our attention to special types of relations.
        \begin{definition}[\textbf{Reflexive Relation}]
            A reflexive relation on a set $A$ is a relation $R$ such that
            for all $a\in{A}$, $aRa$. That is, for all $a\in{A}$, $a$ is related
            to itself by $R$.
        \end{definition}
        \begin{example}
            Equality $(=)$ is reflexive, as is inclusion $(\subseteq)$.
        \end{example}
        \begin{example}
            Proper inclusion $(\subsetneq)$ is not reflexive, neither is
            less than $(<)$ nor greater than $(>)$.
        \end{example}
        Given a set $A$, the \textit{diagonal} of $A\times{A}$ is the set
        of all ordered pairs $(a,a)$ for all $a\in{A}$. If we look at
        $\mathbb{R}\times\mathbb{R}$, the diagonal is the line $y=x$ in the plane,
        hence the name. A reflexive relation is a relation $R$ the contains
        the diagonal.
        \begin{definition}[\textbf{Symmetric Relation}]
            A symmetric relation on a set $A$ is a relation $R$ such that
            for all $a,b\in{A}$, $aRb$ if and only if $bRa$.
        \end{definition}
        \begin{example}
            Equality is symmetric. $a=b$ implies $b=a$.
        \end{example}
        \begin{example}
            Containment $(\in)$ is not symmetric. It is a theorem of set theory
            that $x\in{y}$ implies $y\notin{x}$. The importance of this claim
            is that it helps us avoid Russell's paradox, one of the reasons for
            developing set theory in the first place.
        \end{example}
        \begin{example}
            Inclusion is a relation that is reflexive but not symmetric. It is
            possible for $A\subseteq{B}$ but $B\not\subseteq{A}$. For example,
            $A=\mathbb{Q}$ and $B=\mathbb{R}$.
        \end{example}
        \begin{definition}[\textbf{Transitive Relation}]
            A transitive relation on a set $A$ is a relation $R$ such that
            for all $a,b,c\in{A}$, $aRb$ and $bRc$ implies $aRc$.
        \end{definition}
        \begin{example}
            Equality is transitive. This is one of the assumptions dating back
            to Euclid. If $a=b$ and $b=c$, then $a=c$.
        \end{example}
        \begin{example}
            Inequality is also transitive. If $a<b$ and $b<c$, then
            $a<c$.
        \end{example}
        \begin{example}
            Inclusion is transitive. If $A\subseteq{B}$ and $B\subseteq{C}$,
            then $A\subseteq{C}$.
        \end{example}
        \begin{example}
            Containment does not need to be transitive. Let
            $a=\emptyset$, $b=\{\,\emptyset\,\}$, and
            $c=\{\,\{\,\emptyset\,\}\,\}$. Then $a\in{b}$, $b\in{c}$,
            but $a\notin{c}$.
        \end{example}
        \begin{definition}[\textbf{Equivalence Relation}]
            An equivalence relation on a set $A$ is a relation $R$ that is
            reflexive, symmetric, and transitive.
        \end{definition}
        Equivalence relations allow us to define equivalence classes.
        \begin{definition}[\textbf{Equivalence Class}]
            The equivalence class of an element $a\in{A}$ with respect to an
            equivalence relation $R$ is the set $[a]$ defined by:
            \begin{equation}
                [a]=\{\,b\in{A}\;|\;aRb\,\}
            \end{equation}
        \end{definition}
        \begin{theorem}
            If $A$ is a set, if $R$ is an equivalence relation, and if
            $a,b\in{A}$, then $[a]=[b]$ if and only if $aRb$.
        \end{theorem}
        \begin{proof}
            Since $R$ is reflexive, $a\in[a]$ and $b\in[b]$. If $aRb$,
            then $b\in[a]$, by definition. But $R$ is symmetric,
            so $bRa$ and hence $a\in[b]$. If $c\in[a]$ then $aRc$. But
            $bRa$, and since $R$ is transitive, $bRc$. Therefore
            $c\in[b]$. Similarly, $c\in[b]$ implies $c\in[a]$. That is,
            $[a]$ and $[b]$ consist of precisely the same elements, so
            $[a]=[b]$. In the other direction, if $[a]=[b]$, then by definition
            $a\in[b]$ and $b\in[a]$, and hence $aRb$ and $bRa$.
        \end{proof}
        \begin{definition}[\textbf{Quotient Set}]
            The quotient set of a set $A$ with respect to an equivalence
            relation $R$ is the set $A/R$ defined by:
            \begin{equation}
                A/R=\{\,A\in\mathcal{P}(A)\;|\;A=[a]\textrm{ for some }a\in{A}\;\}
            \end{equation}
            That is, $A/R$ is the set of all equivalence classes of $A$ with
            respect to $R$.
        \end{definition}
        The notation $A/R$ is just notation. We are not \textit{dividing} sets.
        Intuitively, we are forming a new set by taking all of the elements
        $b\in{A}$ such that $b\in[a]$ and \textit{gluing} them to $a$, creating
        a single element. This will be very important in topology when we talk
        about \textit{quotient spaces}.
        \begin{example}
            We can think of a fraction $\frac{a}{b}$ as an ordered pair
            $(a,b)\in\mathbb{Z}\times(\mathbb{Z}\setminus\{\,0\,\})$. We do not
            want $\frac{1}{2}$ and $\frac{2}{4}$ to be different elements, so
            we need to \textit{glue} some elements of this product together.
            That is, $\mathbb{Z}\times(\mathbb{Z}\setminus\{\,0\,\})$ is a set
            of points in the plane $\mathbb{R}^{2}$ and the points
            $(1,\,2)$ and $(2,\,4)$ are different. We ask
            \textit{how can we say} $\frac{a}{b}=\frac{c}{d}$
            \textit{using only integers}? We are trying to define what a
            rational number is, so it would be circular to use the notation
            $\frac{a}{b}$ in our argument. We obtain the answer via
            cross-multiplying. We know that $\frac{a}{b}=\frac{c}{d}$ is true
            (essentially by definition) when $ad=bc$. This allows us to define
            an equivalence relation on
            $\mathbb{Z}\times(\mathbb{Z}\setminus\{\,0\,\})$. Let
            $A=\mathbb{Z}\times(\mathbb{Z}\setminus\{\,0\,\})$ and define
            \begin{equation}
                R=\{\,\big((a,\,b),\,(c,\,d)\,\big)\in{A}\times{A}\;|\;
                    ad=bc\,\}
            \end{equation}
            The quotient set $A/R$ is the set of \textit{rational numbers}.
            The equivalence classes $[(1,\,2)]$ and $[(2,\,4)]$ are the same
            since $1\cdot{4}=2\cdot{2}$. That is, we have glued together
            $(1,\,2)$ and $(2,\,4)$ to form a single object, the fraction
            $\frac{1}{2}$. We write $[(a,\,b)]=\frac{a}{b}$ for convenience.
        \end{example}
\end{document}
