%-----------------------------------LICENSE------------------------------------%
%   This file is part of Mathematics-and-Physics.                              %
%                                                                              %
%   Mathematics-and-Physics is free software: you can redistribute it and/or   %
%   modify it under the terms of the GNU General Public License as             %
%   published by the Free Software Foundation, either version 3 of the         %
%   License, or (at your option) any later version.                            %
%                                                                              %
%   Mathematics-and-Physics is distributed in the hope that it will be useful, %
%   but WITHOUT ANY WARRANTY; without even the implied warranty of             %
%   MERCHANTABILITY or FITNESS FOR A PARTICULAR PURPOSE.  See the              %
%   GNU General Public License for more details.                               %
%                                                                              %
%   You should have received a copy of the GNU General Public License along    %
%   with Mathematics-and-Physics.  If not, see <https://www.gnu.org/licenses/>.%
%------------------------------------------------------------------------------%
\documentclass{article}
\usepackage{graphicx}                           % Needed for figures.
\usepackage{amsmath}                            % Needed for align.
\usepackage{amssymb}                            % Needed for mathbb.
\usepackage{amsthm}                             % For the theorem environment.
\usepackage{float}
\usepackage{hyperref}
\hypersetup{
    colorlinks=true,
    linkcolor=blue,
    filecolor=magenta,
    urlcolor=Cerulean,
    citecolor=SkyBlue
}

%------------------------Theorem Styles-------------------------%

% Define theorem style for default spacing and normal font.
\newtheoremstyle{normal}
    {\topsep}               % Amount of space above the theorem.
    {\topsep}               % Amount of space below the theorem.
    {}                      % Font used for body of theorem.
    {}                      % Measure of space to indent.
    {\bfseries}             % Font of the header of the theorem.
    {}                      % Punctuation between head and body.
    {.5em}                  % Space after theorem head.
    {}

% Define default environments.
\theoremstyle{normal}
\newtheorem{problem}{Problem}

\title{Point-Set Topology: Homework 1}
\date{Summer 2022}

% No indent and no paragraph skip.
\setlength{\parindent}{0em}
\setlength{\parskip}{0em}

\begin{document}
    \maketitle
    \begin{problem}
        This problem explores the use of the word \textit{or} in mathematics.
        You will prove the \textit{principle of explosion}. The principle says
        that if $P$ is a statement that is both true and false, then for any
        statement $Q$, $Q$ is true. A system that contains a sentence that is
        both true and false is called \textit{inconsistent}. The principle of
        explosion shows that inconsistent systems are very boring.
        \begin{itemize}
            \item (2 Points) Let $P$ be a statement that is true, and let $Q$
                be any other claim. Since $P$ is true, what can you conclude
                about $P$ \textit{or} $Q$?
            \item (2 Points) Suppose $P$ is also false. Using what you've
                concluded about $P$ \textit{or} $Q$, what can you prove about
                $Q$?
        \end{itemize}
    \end{problem}
    \begin{problem}
        Here you will explore more set theory. You will prove the
        \textit{axiom of unrestricted comprehension} is inconsistent. The axiom
        allows you to construct a set arbitrarily using any sentence. That is,
        if $P(x)$ is a sentence, you may collect all $x$ such that $P(x)$ is
        true. You could write:
        \begin{equation}
            A=\{\,x\;|\;P(x)\,\}\nonumber
        \end{equation}
        \begin{itemize}
            \item (1 Point) Let $P(x)$ be the sentence $x$ \textit{is a set}.
                Let $A$ be the set $A=\{\,x\;|\;P(x)\,\}$. That is:
                \begin{equation}
                    A=\{\,x\;|\;x\textrm{ is a set}\,\}\nonumber
                \end{equation}
                Describe the set $A$. Is it true that $A\in{A}$?
            \item (2 Points) Let $B$ be the set
                \begin{equation}
                    B=\{\,x\in{A}\;|\;x\notin{x}\,\}\nonumber
                \end{equation}
                Prove that $B\in{B}$. (Hint: Suppose $B\notin{B}$ and arrive
                at a contradiction).
            \item (2 Points) Now prove that $B\notin{B}$.
                (Hint: Same as before. Suppose $B\in{B}$ and arrive at a
                contradiction).
            \item (1 Point) Using the previous problem, why should we not accept
                the axiom of unrestricted comprehension as true?
        \end{itemize}
    \end{problem}
    \begin{problem}
        The axiom of infinity tells us $\mathbb{N}=\{\,0,\,1,\,2,\,\dots\,\}$
        exists. It does not tell us
        $\mathbb{Z}=\{\,\dots,\,-2,\,-1,\,0,\,1,\,2,\,\dots\,\}$ exists, but we
        can construct it. For this problem we can assume addition $(+)$ for
        natural numbers exists.
        \begin{itemize}
            \item (2 Points) Consider the set $\mathbb{N}\times\mathbb{N}$.
                Define $R$ to be the relation $(a,b)R(c,d)$ if and only if
                $a+d=b+c$. Prove this is an equivalence relation.
            \item (1 Point) Describe the equivalence class of $(a,b)$
                geometrically. (Hint: $\mathbb{N}\times\mathbb{N}$ is a lattice
                of points in the plane. Describe the equivalence class of
                $(a,b)$ using this lattice).
            \item (2 Points) For equivalence classes $[(a,b)]$ and $[(c,d)]$,
                define $[(a,b)]+[(c,d)]=[(a+c,b+d)]$. Prove this is
                well-defined.
            \item (1 Point) With this construction we now write
                (for convenience) $0=[(0,0)]$, $n=[(n,0)]$, and $-n=[(0,n)]$.
                Justifiy this notation (Does $[(0,0)]$ behave like 0? Does
                $[(0,n)]$ act like $-n$?)
        \end{itemize}
    \end{problem}
    \begin{problem}
        Now that we have constructed $\mathbb{Z}$, let's construct $\mathbb{Q}$,
        the set of rational numbers. Consider the set
        $\mathbb{Z}\times(\mathbb{Z}\setminus\{\,0\,\})$. Define the relation
        $R$ by $(a,b)R(c,d)$ if and only if $ad=bc$. (We are assuming we have
        already constructed multiplication for integers $n\in\mathbb{Z}$).
        \begin{itemize}
            \item (2 Points) Prove $R$ is an equivalence relation.
            \item (2 Points) For equivalence clases $[(a,b)]$ and $[(c,d)]$,
                define $[(a,b)]+[(c,d)]=[(ad+bc,bd)]$ (This is
                \textit{cross-multiplying}). Prove this is well-defined.
        \end{itemize}
        With this we write $[(a,b)]=\frac{a}{b}$.
    \end{problem}
    \begin{problem}
        Prove Cantor's theorem. If $A$ is a set, and $\mathcal{P}(A)$ is the
        power set of $A$, then there is no surjection
        $f:A\rightarrow\mathcal{P}(A)$.
        \begin{itemize}
            \item (1 Point) Suppose there is a surjection
                $f:A\rightarrow\mathcal{P}(A)$. Consider the set
                $B=\{\,x\in{A}\;|\;x\notin{f}(x)\,\}$. Describe in words what
                the set $B$ contains.
            \item (2 Points) Since $B\subseteq{A}$, and since $f$ is surjective,
                there is an element $a\in{A}$ such that $f(a)=B$. Show that
                this is a contradiction.
            \item (1 Point) Construct an injective function
                $g:A\rightarrow\mathcal{P}(A)$. (Hint: Given $a\in{A}$, what's
                an ``obvious'' subset we can send $a$ to?) 
        \end{itemize}
    \end{problem}
    \begin{problem}
        We proved in class that a function
        $f:X\rightarrow{Y}$ from a metric space $(X,d_{X})$ to a metric space
        $(Y,d_{Y})$ is continuous if and only if for every open subset
        $\mathcal{V}\subseteq{Y}$, the pre-image $f^{-1}[\mathcal{V}]$ is also
        open. You will now prove the equivalence of the third definition of
        continuity.
        \begin{itemize}
            \item (3 Points) Prove that if $f$ is continuous, then
                for all $\varepsilon>0$, and for all $x\in{X}$, there is a
                $\delta>0$ such that if $x_{0}\in{X}$ and
                $d_{X}(x,x_{0})<\delta$,
                then $d_{Y}\big(f(x),f(x_{0})\big)<\varepsilon$
                (Hint: Suppose not. Then there is an $\varepsilon>0$ such that
                for each $n\in\mathbb{N}$, $n>0$,
                there is a point $a_{n}\in{X}$ with $d_{X}(x,a_{n})<\frac{1}{n}$
                and $d_{Y}\big(f(x),f(a_{n})\big)\geq\varepsilon$. What is
                $\lim_{n\rightarrow\infty}a_{n}$? What is
                $\lim_{n\rightarrow\infty}f(a_{n})$? Is there a contradiction?)
            \item (3 Points) Prove that if $f:X\rightarrow{Y}$ is a function
                such that for all $\varepsilon>0$ and for all $x\in{X}$, there
                is a $\delta>0$ such that for all $x_{0}\in{X}$,
                $d_{X}(x,x_{0})<\delta$ implies
                $d_{Y}\big(f(x),f(x_{0})\big)<\varepsilon$, then $f$ is
                continuous. (Hint: Let $a_{n}\rightarrow{x}$ by a convergent
                sequence. What does this property say about
                $\lim_{n\rightarrow\infty}f(a_{n})$?)
        \end{itemize}
    \end{problem}
    \begin{problem}
        A locally compact metric space is a metric space $(X,\,d)$ where for all
        $x\in{X}$ there is a compact set $K$ and an open set $\mathcal{U}$
        such that $x\in\mathcal{U}$ and $\mathcal{U}\subseteq{K}$
        (See Fig.~\ref{fig:locally_compact_metric_space_001}).
        \begin{itemize}
            \item (2 Points) Construct a metric space that is \textit{not}
                locally compact. Explain why it is not locally compact.
                (Hint: Il est utile de penser \'{a} la France).
        \end{itemize}
    \end{problem}
    \begin{figure}[H]
        \centering
        \includegraphics{../../images/locally_compact_example_001.pdf}
        \caption{Diagram for Locally Compact Metric Spaces}
        \label{fig:locally_compact_metric_space_001}
    \end{figure}
\end{document}
