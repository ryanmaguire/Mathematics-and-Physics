%-----------------------------------LICENSE------------------------------------%
%   This file is part of Mathematics-and-Physics.                              %
%                                                                              %
%   Mathematics-and-Physics is free software: you can redistribute it and/or   %
%   modify it under the terms of the GNU General Public License as             %
%   published by the Free Software Foundation, either version 3 of the         %
%   License, or (at your option) any later version.                            %
%                                                                              %
%   Mathematics-and-Physics is distributed in the hope that it will be useful, %
%   but WITHOUT ANY WARRANTY; without even the implied warranty of             %
%   MERCHANTABILITY or FITNESS FOR A PARTICULAR PURPOSE.  See the              %
%   GNU General Public License for more details.                               %
%                                                                              %
%   You should have received a copy of the GNU General Public License along    %
%   with Mathematics-and-Physics.  If not, see <https://www.gnu.org/licenses/>.%
%------------------------------------------------------------------------------%
\documentclass{article}
\usepackage{graphicx}                           % Needed for figures.
\usepackage{amsmath}                            % Needed for align.
\usepackage{amssymb}                            % Needed for mathbb.
\usepackage{amsthm}                             % For the theorem environment.
\usepackage{float}
\usepackage{hyperref}
\hypersetup{
    colorlinks=true,
    linkcolor=blue,
    filecolor=magenta,
    urlcolor=Cerulean,
    citecolor=SkyBlue
}

%------------------------Theorem Styles-------------------------%
\theoremstyle{plain}
\newtheorem{theorem}{Theorem}[section]

% Define theorem style for default spacing and normal font.
\newtheoremstyle{normal}
    {\topsep}               % Amount of space above the theorem.
    {\topsep}               % Amount of space below the theorem.
    {}                      % Font used for body of theorem.
    {}                      % Measure of space to indent.
    {\bfseries}             % Font of the header of the theorem.
    {}                      % Punctuation between head and body.
    {.5em}                  % Space after theorem head.
    {}

% Define default environments.
\theoremstyle{normal}
\newtheorem{examplex}{Example}[section]
\newtheorem{definitionx}{Definition}[section]
\newtheorem{notationx}{Notation}[section]
\newtheorem{axiomx}{Axiom}[section]

\newenvironment{example}{%
    \pushQED{\qed}\renewcommand{\qedsymbol}{$\blacksquare$}\examplex%
}{%
    \popQED\endexamplex%
}

\newenvironment{definition}{%
    \pushQED{\qed}\renewcommand{\qedsymbol}{$\blacksquare$}\definitionx%
}{%
    \popQED\enddefinitionx%
}

\title{Point-Set Topology: Lecture 7}
\author{Ryan Maguire}
\date{Summer 2022}

% No indent and no paragraph skip.
\setlength{\parindent}{0em}
\setlength{\parskip}{0em}

\begin{document}
    \maketitle
    \section{Metric Topology}
        \begin{definition}[\textbf{Metric Topology}]
            The metric topology on a metric space $(X,\,d)$ is the
            set $\tau_{d}\subseteq\mathcal{P}(X)$ such that for all
            $\mathcal{U}$, $\mathcal{U}\in\tau_{d}$ if and only if
            $\mathcal{U}$ is open in $(X,\,d)$. That is, $\tau_{d}$ is the
            set of all open subsets of $(X,\,d)$.
        \end{definition}
        We have seen in previous theorems that the metric topology $\tau_{d}$
        of a metric space $(X,\,d)$ has several properties. First,
        $\emptyset\in\tau_{d}$ and $X\in\tau_{d}$. That is, the empty set is
        open and the whole space is open. Secondly, given any subset
        $\mathcal{O}\subseteq\tau_{d}$, the union $\bigcup\mathcal{O}$ is an
        element of $\tau_{d}$. That is, the arbitrary union of open sets is
        open. Lastly, if $\mathcal{U},\mathcal{V}\in\tau_{d}$, then
        $\mathcal{U}\cap\mathcal{V}\in\tau_{d}$. That is, the finite
        intersection of open sets is open. We will take these properties and
        use them to define a topological space. A topological space is a set
        $X$ and a subset $\tau\subseteq\mathcal{P}(X)$ with the four properties
        mentioned previously. This will be made clear in later lectures, for now
        we want to discuss which properties of a metric space are
        \textit{topological} and which properties are geometric, or metric
        properties.
        \begin{definition}[\textbf{Topologically Equivalent Metrics}]
            Topologically equivalent metrics on a set $X$ are metrics
            $d_{0}$ and $d_{1}$ with equal metric topologies,
            $\tau_{d_{0}}=\tau_{d_{1}}$.
        \end{definition}
        To provide examples of equivalent metrics, it is best to use
        the following theorem.
        \begin{theorem}
            If $X$ is a set, and $d_{0}$ and $d_{1}$ are metrics on $X$,
            then $d_{0}$ and $d_{1}$ are topologically equivalent if and only
            if for all $x\in{X}$ and $r>0$, there is an $r_{0}>0$ and an
            $r_{1}>0$ such that
            $B_{r_{0}}^{(X,\,d_{0})}(x)\subseteq{B}_{r}^{(X,\,d_{1})}(x)$ and
            $B_{r_{1}}^{(X,\,d_{1})}(x)\subseteq{B}_{r}^{(X,\,d_{0})}(x)$.
            That is, the open balls can be \textit{nested} inside of each other.
        \end{theorem}
        \begin{proof}
            If $\tau_{d_{0}}=\tau_{d_{1}}$, then
            $B_{r}^{(X,\,d_{0})}(x)$ is open in
            $\tau_{d_{1}}$ meaning there is an $r_{1}>0$ such that
            $B_{r_{1}}^{(X,\,d_{1})}(x)\subseteq{B}_{r}^{(X,\,d_{0})}(x)$.
            Similarly, if $\tau_{d_{0}}=\tau_{d_{1}}$, then
            $B_{r}^{(X,\,d_{1})}(x)$ is open in
            $\tau_{d_{0}}$ meaning there is an $r_{0}>0$ such that
            $B_{r_{0}}^{(X,\,d_{0})}(x)\subseteq{B}_{r}^{(X,\,d_{0})}(x)$.
            In the other direction, suppose $\tau_{d_{0}}$ and
            $\tau_{d_{1}}$ are such that open balls can be nested.
            Let $\mathcal{V}\in\tau_{d_{1}}$. For all
            $x\in\mathcal{V}$, since $\mathcal{V}$ is open, there is an
            $r>0$ such that $B_{r}^{(X,\,d_{1})}(x)\subseteq\mathcal{V}$.
            But then there is an $r_{0}>0$ such that
            $B_{r_{0}}^{(X,\,d_{0})}(x)\subseteq{B}_{r}^{(X,\,d_{1})}(x)$, and
            hence
            $B_{r_{0}}^{(X,\,d_{0})}(x)\subseteq\mathcal{V}$, so
            $\mathcal{V}\in\tau_{d_{0}}$. Similarly, if
            $\mathcal{U}\in\tau_{d_{0}}$, then for all $x\in\mathcal{U}$, since
            $\mathcal{U}$ is open, there is an $r>0$ such that
            $B_{r}^{(X,\,d_{0})}(x)\subseteq\mathcal{U}$. But then there is
            an $r_{1}>0$ such that
            $B_{r_{1}}^{(X,\,d_{1})}(x)\subseteq{B}_{r}^{(X,\,d_{0})}(x)$, and
            hence $B_{r_{1}}^{(X,\,d_{1})}(x)\subseteq\mathcal{U}$. That is,
            $\mathcal{U}\in\tau_{d_{1}}$. Therefore,
            $\tau_{d_{0}}=\tau_{d_{1}}$.
        \end{proof}
        \begin{example}
            Let $(\mathbb{R}^{2},\,d_{E})$ be the Euclidean metric space on the
            plane, and $(\mathbb{R}^{2},\,d_{M})$ be the Manhattan metric space.
            Open balls in the Euclidean metric are open disks and open balls
            in the Manhattan metric are open diamonds. We can nest one inside
            of the other, showing that the Euclidean and Manhattan metrics are
            topologically equivalent.
            See Figs.~\ref{fig:manhattan_top_equiv_euclidean_001}
            and \ref{fig:manhattan_top_equiv_euclidean_002}.
        \end{example}
        \begin{figure}
            \centering
            \includegraphics{../../images/manhattan_top_equiv_euclidean_001.pdf}
            \caption{Manhattan Open Sets Nested in Euclidean Open Sets}
            \label{fig:manhattan_top_equiv_euclidean_001}
        \end{figure}
        \begin{figure}
            \centering
            \includegraphics{../../images/manhattan_top_equiv_euclidean_002.pdf}
            \caption{Euclidean Open Sets Nested in Manhattan Open Sets}
            \label{fig:manhattan_top_equiv_euclidean_002}
        \end{figure}
        \begin{example}
            Let $(\mathbb{R}^{2},\,d_{E})$ be the Euclidean plane and
            $(\mathbb{R}^{2},\,d_{\textrm{max}})$ be the maximum metric space
            on the plane (the chess board metric). Open balls in the Euclidean
            plane are open disks and open balls in the max metric are open
            squares. We can nest on inside the other, meaning the Euclidean
            metric and the maximum metric are equivalent on $\mathbb{R}^{2}$.
        \end{example}
        \begin{figure}
            \centering
            \includegraphics{../../images/max_top_equiv_euclidean_001.pdf}
            \caption{Max Open Sets Nested in Euclidean Open Sets}
            \label{fig:max_top_equiv_euclidean_001}
        \end{figure}
        \begin{figure}
            \centering
            \includegraphics{../../images/max_top_equiv_euclidean_002.pdf}
            \caption{Euclidean Open Sets Nested in Max Open Sets}
            \label{fig:max_top_equiv_euclidean_002}
        \end{figure}
        \begin{example}
            The Euclidean metric and the Paris metric on
            $\mathbb{R}^{2}$ are not equivalent. Given a point
            $\mathbf{x}\in\mathbb{R}^{2}$, $\mathbf{x}\ne\mathbf{0}$,
            choose $r=||\mathbf{x}||_{2}/2$, where $||\mathbf{x}||_{2}$ is the
            standard Euclidean length of the vector $\mathbf{x}$. The open
            ball centered at $\mathbf{x}$ with radius $r$ is an open
            line segment going from $\mathbf{x}-(r,\,r)$ to
            $\mathbf{x}+(r,\,r)$. Open line segments are not open in the
            Euclidean metric, so the Paris metric is different, topologically,
            than the Euclidean metric.
        \end{example}
        \begin{example}
            The London metric and the Paris metric are not equivalent. Given
            $\mathbf{x}\in\mathbb{R}^{2}$, $\mathbf{x}\ne\mathbf{0}$, choose
            $r=||\mathbf{x}||_{2}/2$. The open ball of radius $r$ centered about
            $\mathbf{x}$ in the Paris metric, as described before, is an open
            line segment in the plane. The open ball centered about
            $\mathbf{x}$ of radius $r$ in the London metric is just the
            point $\{\,\mathbf{x}\,\}$. For all other points
            $\mathbf{y}\ne\mathbf{x}$, the distance in the London metric is:
            \begin{equation}
                d_{L}(\mathbf{x},\,\mathbf{y})=
                    ||\mathbf{x}||_{2}+||\mathbf{y}||_{2}
                    =2r+||\mathbf{y}||_{2}>r
            \end{equation}
            Meaning $\mathbf{y}$ is not in the ball of radius $r$ centered about
            $\mathbf{x}$ in the London metric. So, the ball of radius $r$
            centered about $\mathbf{x}$ is just $\{\,\mathbf{x}\,\}$. Single
            points are not open in the Paris metric, showing the two metrics
            are not topologically equivalent.
        \end{example}
        \begin{example}
            The London metric is not topologically equivalent to the discrete
            metric. It is true that for every point $\mathbf{x}\ne\mathbf{0}$,
            the point $\{\,\mathbf{x}\,\}$ is open in the London metric, which
            certainly seems similar to the discrete metric, the set
            $\{\,\mathbf{0}\,\}$ is not open. An open ball about
            $\mathbf{0}$ in the London metric is an open disk, so
            $\{\,\mathbf{0}\,\}$ is not open. However, $\{\,\mathbf{0}\,\}$
            is open in the discrete metric.
        \end{example}
        Topological properties are those that are detected by the
        topologies of the metric space. Convergence, continuity, open and
        closed, are all notions that are topological. As we will see
        repeatedly throughout the course, \textit{homeomorphisms} are functions
        that preserve topological properties.
        \begin{definition}[\textbf{Homeomorphism}]
            A homeomorphism from a metric space $(X,\,d_{X})$ to a metric space
            $(Y,\,d_{Y})$ is a bijective continuous function $f:X\rightarrow{Y}$
            such that $f^{-1}$ is continuous.
        \end{definition}
        Global isometries are functions that preserve all metric properties.
        Global isometries are, in particular, homeomorphisms.
        \begin{theorem}
            If $(X,\,d_{X})$ and $(Y,\,d_{Y})$ are metric spaces, and if
            $f:X\rightarrow{Y}$ is a global isometry, then $f$ is a
            homeomorphism.
        \end{theorem}
        \begin{proof}
            We have proven that isometries are continuous. All we need to do
            now is prove that if $f:X\rightarrow{Y}$ is a global isometry,
            then $f^{-1}$ is also an isometry. Let $y_{0},y_{1}\in{Y}$.
            Since $f$ is a global isometry, it is bijective, and hence there
            are $x_{0},x_{1}\in{X}$ with $f(x_{0})=y_{0}$ and $f(x_{1})=y_{1}$.
            But then, since $f$ is an isometry, we have:
            \begin{align}
                d_{Y}(y_{0},\,y_{1})
                    &=d_{Y}\big(f(x_{0}),\,f(x_{1})\big)\\
                    &=d_{X}(x_{0},\,x_{1})\\
                    &=d_{Y}\big(f^{-1}(x_{0}),\,f^{-1}(x_{1})\big)
            \end{align}
            and hence $f^{-1}$ is an isometry. But then $f$ and $f^{-1}$ are
            continuous, so $f$ is a homeomorphism.
        \end{proof}
        Just like not every continuous function is an isometry, not every
        homeomorphism is a global isometry. Homeomorphism is a weaker notion,
        but also a far more general notion with more applications. Think about
        the real line. The only isometries are translations
        $f(x)=x+a$, reflections $f(x)=-x$, and glide reflections,
        $f(x)=-x+a$. Most of the functions used in calculus and physics are not
        isometries, but are usually continuous.
    \section{Completeness}
        \begin{definition}[\textbf{Cauchy Sequences}]
            A Cauchy sequence in a metric space $(X,\,d)$ is a sequence
            $a:\mathbb{N}\rightarrow{X}$ such that for all $\varepsilon>0$ there
            is an $N\in\mathbb{N}$ with for all $m,n\in\mathbb{N}$ with
            $m>N$ and $n>N$, it is true that $d(a_{n},\,a_{m})<\varepsilon$.
        \end{definition}
        Cauchy sequences are sequences where the points $a_{n}$ start to get
        closer and closer together as $n$ increases. Convergent sequences are,
        in particular, Cauchy sequences.
        \begin{theorem}
            If $(X,\,d)$ is a metric space, and if $a:\mathbb{N}\rightarrow{X}$
            is a convergent sequence, then $a$ is a Cauchy sequence.
        \end{theorem}
        \begin{proof}
            Let $\varepsilon>0$. Since $a:\mathbb{N}\rightarrow{X}$ is
            convergent, there is an $x\in{X}$ with $a_{n}\rightarrow{x}$. But
            then there is an $N\in\mathbb{N}$ such that for all $n\in\mathbb{N}$
            with $n>N$ it is true that $d(a_{n},\,x)<\frac{\varepsilon}{2}$.
            But then for $m>N$ and $n>N$ we have:
            \begin{equation}
                d(a_{m},\,a_{n})\leq
                d(a_{m},\,x)+d(a_{n},\,x)<
                \frac{\varepsilon}{2}+\frac{\varepsilon}{2}
                =\varepsilon
            \end{equation}
            and therefore $a$ is a Cauchy sequence.
        \end{proof}
        Since the points $a_{n}$ are getting closer and closer together it is
        natural to ask if the converse of this theorem is true as well. That is,
        if $a:\mathbb{N}\rightarrow{X}$ is a Cauchy sequence in a metric space
        $(X,\,d)$, is $a$ also a convergent sequence?
        \begin{example}
            Define $a:\mathbb{N}\rightarrow\mathbb{Q}$ by
            $a_{0}=1$, $a_{1}=1.4$, $a_{2}=1.41$, $a_{3}=1.414$, and
            $a_{n}$ is the first $n$ decimals of $\sqrt{2}$. This is a
            Cauchy sequence, given $m<n$,
            $d(a_{m},\,a_{n})$ is less than $10^{-n}$, which can be made
            arbitrarily small. It does not converge in $\mathbb{Q}$, the
            \textit{limit} we want to say this converges to is
            $\sqrt{2}$, but $\sqrt{2}$ is not a rational number.
        \end{example}
        The problem with $\mathbb{Q}$ is it has a lot of holes, these are the
        irrational numbers.
        \begin{definition}[\textbf{Complete Metric Space}]
            A complete metric space is a metric space $(X,\,d)$ such that for
            every Cauchy sequence $a:\mathbb{N}\rightarrow{X}$, it is true
            that $a$ is a convergent sequence.
        \end{definition}
        The real numbers are complete, with the usual metric
        $d(x,\,y)=|x-y|$. Given a bounded set $A\subseteq\mathbb{R}$, meaning
        there is an $M\in\mathbb{R}$ such that for all $x\in{A}$ it is true
        that $|x|<M$, the real numbers have the property that $A$ has a
        \textit{least upper bound} and a \textit{greatest lower bound}. That is,
        numbers $r$ and $s$ such that $r$ is a lower bound, all $x\in{A}$
        are such that $r\leq{x}$, and $s$ is an upper bound, all $x\in{A}$
        are such that $x\leq{s}$, but moreover $r$ is the \textit{largest}
        possible lower bound, and $s$ is the \textit{smallest} possible upper
        bound. The rationals do not have this property. Given the set
        $A=\{\,x\in\mathbb{Q}\;|\;x^{2}<2\,\}$, there is no least upper bound.
        If you give me a rational number that is an upper bound for $A$, I can
        find a smaller rational number that is also an upper bound. The
        \textit{least} upper bound of this set is $\sqrt{2}$, but again, this
        is not a rational number.
\end{document}
