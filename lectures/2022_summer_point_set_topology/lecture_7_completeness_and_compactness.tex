%-----------------------------------LICENSE------------------------------------%
%   This file is part of Mathematics-and-Physics.                              %
%                                                                              %
%   Mathematics-and-Physics is free software: you can redistribute it and/or   %
%   modify it under the terms of the GNU General Public License as             %
%   published by the Free Software Foundation, either version 3 of the         %
%   License, or (at your option) any later version.                            %
%                                                                              %
%   Mathematics-and-Physics is distributed in the hope that it will be useful, %
%   but WITHOUT ANY WARRANTY; without even the implied warranty of             %
%   MERCHANTABILITY or FITNESS FOR A PARTICULAR PURPOSE.  See the              %
%   GNU General Public License for more details.                               %
%                                                                              %
%   You should have received a copy of the GNU General Public License along    %
%   with Mathematics-and-Physics.  If not, see <https://www.gnu.org/licenses/>.%
%------------------------------------------------------------------------------%
\documentclass{article}
\usepackage{graphicx}                           % Needed for figures.
\usepackage{amsmath}                            % Needed for align.
\usepackage{amssymb}                            % Needed for mathbb.
\usepackage{amsthm}                             % For the theorem environment.
\usepackage{float}
\usepackage{hyperref}
\hypersetup{
    colorlinks=true,
    linkcolor=blue,
    filecolor=magenta,
    urlcolor=Cerulean,
    citecolor=SkyBlue
}

%------------------------Theorem Styles-------------------------%
\theoremstyle{plain}
\newtheorem{theorem}{Theorem}[section]

% Define theorem style for default spacing and normal font.
\newtheoremstyle{normal}
    {\topsep}               % Amount of space above the theorem.
    {\topsep}               % Amount of space below the theorem.
    {}                      % Font used for body of theorem.
    {}                      % Measure of space to indent.
    {\bfseries}             % Font of the header of the theorem.
    {}                      % Punctuation between head and body.
    {.5em}                  % Space after theorem head.
    {}

% Define default environments.
\theoremstyle{normal}
\newtheorem{examplex}{Example}[section]
\newtheorem{definitionx}{Definition}[section]
\newtheorem{notationx}{Notation}[section]
\newtheorem{axiomx}{Axiom}[section]

\newenvironment{example}{%
    \pushQED{\qed}\renewcommand{\qedsymbol}{$\blacksquare$}\examplex%
}{%
    \popQED\endexamplex%
}

\newenvironment{definition}{%
    \pushQED{\qed}\renewcommand{\qedsymbol}{$\blacksquare$}\definitionx%
}{%
    \popQED\enddefinitionx%
}

\title{Point-Set Topology: Lecture 7}
\author{Ryan Maguire}
\date{Summer 2022}

% No indent and no paragraph skip.
\setlength{\parindent}{0em}
\setlength{\parskip}{0em}

\begin{document}
    \maketitle
    \section{Metric Topology}
        \begin{definition}[\textbf{Metric Topology}]
            The metric topology on a metric space $(X,\,d)$ is the
            set $\tau_{d}\subseteq\mathcal{P}(X)$ such that for all
            $\mathcal{U}$, $\mathcal{U}\in\tau_{d}$ if and only if
            $\mathcal{U}$ is open in $(X,\,d)$. That is, $\tau_{d}$ is the
            set of all open subsets of $(X,\,d)$.
        \end{definition}
        We have seen in previous theorems that the metric topology $\tau_{d}$
        of a metric space $(X,\,d)$ has several properties. First,
        $\emptyset\in\tau_{d}$ and $X\in\tau_{d}$. That is, the empty set is
        open and the whole space is open. Secondly, given any subset
        $\mathcal{O}\subseteq\tau_{d}$, the union $\bigcup\mathcal{O}$ is an
        element of $\tau_{d}$. That is, the arbitrary union of open sets is
        open. Lastly, if $\mathcal{U},\mathcal{V}\in\tau_{d}$, then
        $\mathcal{U}\cap\mathcal{V}\in\tau_{d}$. That is, the finite
        intersection of open sets is open. We will take these properties and
        use them to define a topological space. A topological space is a set
        $X$ and a subset $\tau\subseteq\mathcal{P}(X)$ with the four properties
        mentioned previously. This will made clear in later lectures, for now
        we want to discuss which properties of a metric space are
        \textit{topological} and which properties are geometric, or metric
        properties.
        \begin{definition}[\textbf{Equivalent Metrics}]
            Equivalent metrics on a set $X$ are metrics
            $d_{0}$ and $d_{1}$ with equal metric topologies
            $\tau_{d_{0}}$ and $\tau_{d_{1}}$. That is,
            $\tau_{d_{0}}=\tau_{d_{1}}$.
        \end{definition}
        To provide examples of equivalent metrics, it is best to make use of
        the following theorem.
        \begin{theorem}
            If $X$ is a set, and $d_{0}$ and $d_{1}$ are metrics on $X$,
            then $d_{0}$ and $d_{1}$ are topologically equivalent if and only
            if for all $x\in{X}$ and $r>0$, there is an $r_{0}>0$ and an
            $r_{1}>0$ such that
            $B_{r_{0}}^{(X,\,d_{0})}(x)\subseteq{B}_{r}^{(X,\,d_{1})}(x)$ and
            $B_{r_{1}}^{(X,\,d_{1})}(x)\subseteq{B}_{r}^{(X,\,d_{0})}(x)$.
            That is, the open balls can be \textit{nested} inside of each other.
        \end{theorem}
        \begin{proof}
            If $\tau_{d_{0}}=\tau_{d_{1}}$, then
            $B_{r}^{(X,\,d_{0})}(x)$ is open in
            $\tau_{d_{1}}$ meaning there is an $r_{1}>0$ such that
            $B_{r_{1}}^{(X,\,d_{1})}(x)\subseteq{B}_{r}^{(X,\,d_{0})}(x)$.
            Similarly, if $\tau_{d_{0}}=\tau_{d_{1}}$, then
            $B_{r}^{(X,\,d_{1})}(x)$ is open in
            $\tau_{d_{0}}$ meaning there is an $r_{0}>0$ such that
            $B_{r_{0}}^{(X,\,d_{0})}(x)\subseteq{B}_{r}^{(X,\,d_{0})}(x)$.
            In the other direction, suppose $\tau_{d_{0}}$ and
            $\tau_{d_{1}}$ are such that open balls can be nested.
            Let $\mathcal{V}\in\tau_{d_{1}}$. For all
            $x\in\mathcal{V}$, since $\mathcal{V}$ is open, there is an
            $r>0$ such that $B_{r}^{(X,\,d_{1})}(x)\subseteq\mathcal{V}$.
            But then there is an $r_{0}>0$ such that
            $B_{r_{0}}^{(X,\,d_{0})}(x)\subseteq{B}_{r}^{(X,\,d_{1})}(x)$, and
            hence
            $B_{r_{0}}^{(X,\,d_{0})}(x)\subseteq\mathcal{V}$, so
            $\mathcal{V}\in\tau_{d_{0}}$. Similarly, if
            $\mathcal{U}\in\tau_{d_{0}}$, then for all $x\in\mathcal{U}$, since
            $\mathcal{U}$ is open, there is an $r>0$ such that
            $B_{r}^{(X,\,d_{0})}(x)\subseteq\mathcal{U}$. But then there is
            an $r_{1}>0$ such that
            $B_{r_{1}}^{(X,\,d_{1})}(x)\subseteq{B}_{r}^{(X,\,d_{0})}(x)$, and
            hence $B_{r_{1}}^{(X,\,d_{1})}(x)\subseteq\mathcal{U}$. That is,
            $\mathcal{U}\in\tau_{d_{1}}$. Therefore,
            $\tau_{d_{0}}=\tau_{d_{1}}$.
        \end{proof}
        \begin{example}
            Let $(\mathbb{R}^{2},\,d_{E})$ be the Euclidean metric space on the
            plane, and $(\mathbb{R}^{2},\,d_{M})$ be the Manhattan metric space.
        \end{example}
\end{document}
