%-----------------------------------LICENSE------------------------------------%
%   This file is part of Mathematics-and-Physics.                              %
%                                                                              %
%   Mathematics-and-Physics is free software: you can redistribute it and/or   %
%   modify it under the terms of the GNU General Public License as             %
%   published by the Free Software Foundation, either version 3 of the         %
%   License, or (at your option) any later version.                            %
%                                                                              %
%   Mathematics-and-Physics is distributed in the hope that it will be useful, %
%   but WITHOUT ANY WARRANTY; without even the implied warranty of             %
%   MERCHANTABILITY or FITNESS FOR A PARTICULAR PURPOSE.  See the              %
%   GNU General Public License for more details.                               %
%                                                                              %
%   You should have received a copy of the GNU General Public License along    %
%   with Mathematics-and-Physics.  If not, see <https://www.gnu.org/licenses/>.%
%------------------------------------------------------------------------------%
\documentclass{article}
\usepackage{graphicx}                           % Needed for figures.
\usepackage{amsmath}                            % Needed for align.
\usepackage{amssymb}                            % Needed for mathbb.
\usepackage{amsthm}                             % For the theorem environment.
\usepackage{float}
\usepackage{tabularx, booktabs}
\usepackage[font=scriptsize,
            labelformat=simple,
            labelsep=colon]{subcaption} % Subfigure captions.
\usepackage[font={scriptsize},
            hypcap=true,
            labelsep=colon]{caption}    % Figure captions.
\usepackage{hyperref}
\hypersetup{
    colorlinks=true,
    linkcolor=blue,
    filecolor=magenta,
    urlcolor=Cerulean,
    citecolor=SkyBlue
}

%------------------------Theorem Styles-------------------------%
\theoremstyle{plain}
\newtheorem{theorem}{Theorem}[section]

% Define theorem style for default spacing and normal font.
\newtheoremstyle{normal}
    {\topsep}               % Amount of space above the theorem.
    {\topsep}               % Amount of space below the theorem.
    {}                      % Font used for body of theorem.
    {}                      % Measure of space to indent.
    {\bfseries}             % Font of the header of the theorem.
    {}                      % Punctuation between head and body.
    {.5em}                  % Space after theorem head.
    {}

% Define default environments.
\theoremstyle{normal}
\newtheorem{examplex}{Example}[section]
\newtheorem{definitionx}{Definition}[section]

\newenvironment{example}{%
    \pushQED{\qed}\renewcommand{\qedsymbol}{$\blacksquare$}\examplex%
}{%
    \popQED\endexamplex%
}

\newenvironment{definition}{%
    \pushQED{\qed}\renewcommand{\qedsymbol}{$\blacksquare$}\definitionx%
}{%
    \popQED\enddefinitionx%
}

\title{Point-Set Topology: Lecture 20}
\author{Ryan Maguire}
\date{Summer 2022}

% No indent and no paragraph skip.
\setlength{\parindent}{0em}
\setlength{\parskip}{0em}

\begin{document}
    \maketitle
    \section{Connectedness}
        \begin{figure}
            \centering
            \includegraphics{../../images/disconnected_space_001.pdf}
            \caption{A Disconnected Topological Space}
            \label{fig:disconnected_space_001}
        \end{figure}
        Connectedness is one of the fundamental notions in topology. Intuitively
        a connected space is one that is in \textit{one piece}. It can be hard
        to make precise what one means by this, but it can be easier to describe
        what \textit{disconnected} is. For intuition we use the plane. Two
        isolated discs in the plane should not be considered as a connected
        subspace since it is definitely not one piece
        (Fig.~\ref{fig:disconnected_space_001}). We use this to motivate
        disconnected spaces.
        \begin{definition}[\textbf{Disconnected Topological Space}]
            A disconnected topological space is a topological space $(X,\,\tau)$
            such that there are non-empty open subsets
            $\mathcal{U},\mathcal{V}\in\tau$ such that
            $\mathcal{U}\cap\mathcal{V}=\emptyset$ and
            $\mathcal{U}\cup\mathcal{V}=X$.
        \end{definition}
        \begin{example}
            The discrete topology on $\mathbb{Z}_{2}$ is disconnected. This
            space is two isolated points. To be precise, the set
            $\mathcal{U}=\{\,0\,\}$ is open and non-empty, the set
            $\mathcal{V}=\{\,1\,\}$ is open and non-empty, and these two sets
            satisfy $\mathcal{U}\cap\mathcal{V}=\emptyset$ and
            $\mathcal{U}\cup\mathcal{V}=\mathbb{Z}_{2}$.
        \end{example}
        \begin{example}
            If $X$ is any set containing at least two points, and if $\tau$ is
            the discrete topology, then $(X,\,\tau)$ is disconnected. Let
            $x\in{X}$ be one point and define $\mathcal{U}=\{\,x\,\}$. Since
            $\tau$ is the discrete topology $\mathcal{U}$ is open and non-empty.
            Let $\mathcal{V}=X\setminus\mathcal{U}$. Again, since $\tau$ is the
            discrete topology, $\mathcal{V}$ is open and since $X$ has at least
            two points it is also non-empty. But then $\mathcal{U}$ and
            $\mathcal{V}$ are non-empty open subsets such that
            $\mathcal{U}\cap\mathcal{V}=\emptyset$ and
            $\mathcal{U}\cup\mathcal{V}=X$, showing that $(X,\,\tau)$ is
            disconnected.
        \end{example}
        Connected is just \textit{not disconnected}.
        \begin{definition}[\textbf{Connected Topological Space}]
            A connected topological space is a topological space $(X,\,\tau)$
            that is not disconnected.
        \end{definition}
        Some familiar spaces like $\mathbb{R}$ and $\mathbb{R}^{2}$ are
        connected, but it takes a bit of work to show this. The spaces that are
        easy to show are connected straight from the definition have
        less-than-useful topologies.
        \begin{example}
            If $X$ is any set and $\tau$ is the indiscrete topology, then
            $(X,\,\tau)$ is connected. There are no two disjoint open sets
            $\mathcal{U},\mathcal{V}$ that are non-empty and cover $X$ since
            the only open sets are $\emptyset$ and $X$. So $(X,\,\tau)$ is
            connected.
        \end{example}
        \begin{example}
            The particular point topology on $\mathbb{R}$ defines a set
            $\mathcal{U}$ to be open if and only if $0\in\mathcal{U}$ or
            $\mathcal{U}=\emptyset$. Hence any two non-empty open sets that
            cover $\mathbb{R}$ must have $0$ in common, meaning we cannot
            separate the space into two disjoint non-empty open sets, so the
            particular point space is connected. Intuitively, every point is
            \textit{connected} to zero. 
        \end{example}
        \begin{example}
            The excluded point topology on $\mathbb{R}$ defines a set
            $\mathcal{U}$ to be open if and only if $0\notin\mathcal{U}$ or
            $\mathcal{U}=\mathbb{R}$. Because of this if $\mathcal{U}$ and
            $\mathcal{V}$ are open sets that cover $\mathbb{R}$, one of these
            sets must be $\mathbb{R}$. Because of this it is impossible to
            separate the space using disjoint non-empty open sets.
        \end{example}
        \begin{example}
            The finite complement topology on $\mathbb{R}$ is open. Given any
            non-empty open subsets $\mathcal{U},\mathcal{V}$, the intersection
            can not be empty since $\mathbb{R}\setminus\mathcal{U}$ and
            $\mathbb{R}\setminus\mathcal{V}$ are both finite, meaning
            $\mathcal{U}\cap\mathcal{V}$ is infinite (since $\mathbb{R}$ is
            infinite). So the finite complement topology on $\mathbb{R}$ is
            connected.
        \end{example}
        \begin{example}
            For similar reasons, the countable complement topology on
            $\mathbb{R}$ yields a connected space. Any two non-empty open
            subsets must have non-empty intersection since $\mathbb{R}$ is
            uncountable and the complements of two non-empty open subsets is
            countable (and hence so is the union of their complements).
        \end{example}
        \begin{example}
            The rationals $\mathbb{Q}$ with the subspace topology from
            $\mathbb{R}$ are disconnected. Let $\mathcal{U}$ be all positive
            rational numbers $x$ such that $x^{2}>2$. Let $\mathcal{V}$ be
            all rational numbers $x$ such that either $x<0$ or $x^{2}<2$. There
            is no rational number whose square is 2, so $\mathcal{U}$ and
            $\mathcal{V}$ are non-empty disjoint open subsets whose union is
            the entirety of $\mathbb{Q}$. So the rationals are disconnected. 
        \end{example}
        \begin{example}
            Let $X\subseteq\mathbb{R}$ be defined by
            $X=(-\infty,\,0)\cup(0,\,\infty)$. This is the real line with the
            origin removed. Equipping this with the subspace topology yields a
            disconnected space. Setting $\mathcal{U}=(-\infty,\,0)$ and
            $\mathcal{V}=(0,\,\infty)$ shows why.
        \end{example}
        When first discussing open and closed sets many students have trouble
        realizing that \textit{open} does not mean \textit{not closed}, and
        \textit{closed} does not mean \textit{not open}. It is possible for a
        subset to be open and not closed, closed and not open, neither open nor
        closed, and both open and closed. This last part is particular hard to
        grasp since $\mathbb{R}$ has no subsets $\mathcal{U}\subseteq\mathbb{R}$
        that are both open and closed with the exception of
        $\mathcal{U}=\mathbb{R}$ and $\mathcal{U}=\emptyset$. This is because
        the real line is \textit{connected} and connected subsets have no
        proper non-empty subsets that are both open and closed. Let's prove
        this.
        \begin{theorem}
            If $(X,\,\tau)$ is a topological space, then it is disconnected
            if and only if there is a non-empty open proper subset
            $\mathcal{U}\subsetneq{X}$ that is also closed.
        \end{theorem}
        \begin{proof}
            If $(X,\,\tau)$ is disconnected there exists non-empty disjoint
            open sets $\mathcal{U}$ and $\mathcal{V}$ whose union is $X$. But
            then $X\setminus\mathcal{U}=\mathcal{V}$, and $\mathcal{V}$ is
            open, so $\mathcal{U}$ is closed. But $\mathcal{U}$ is also open,
            so $\mathcal{U}$ is a non-empty proper subset of $X$ that is also
            closed. Now suppose there is a subset $\mathcal{U}\subsetneq{X}$
            that is non-empty and both open and closed. Since $\mathcal{U}$ is
            closed, $\mathcal{V}=X\setminus\mathcal{U}$ is open. But since
            $\mathcal{U}$ is proper, $\mathcal{V}$ is non-empty. But then
            $\mathcal{U}$ and $\mathcal{V}$ are disjoint non-empty open subsets
            whose union is $X$, so $(X,\,\tau)$ is disconnected.
        \end{proof}
        \begin{theorem}
            If $(X,\,\tau)$ is a topological space, then it is disconnected if
            and only if there are non-empty disjoint closed subsets
            $\mathcal{C},\mathcal{D}\subseteq{X}$ such that
            $\mathcal{C}\cup\mathcal{D}=\emptyset$.
        \end{theorem}
        \begin{proof}
            If $(X,\,\tau)$ is disconnected there are non-empty disjoint open
            subsets $\mathcal{U},\mathcal{V}$ such that
            $\mathcal{U}\cup\mathcal{V}=X$. But then
            $\mathcal{V}=X\setminus\mathcal{U}$ and
            $\mathcal{U}=X\setminus\mathcal{V}$ are closed non-empty disjoint
            subsets whose union is $X$. Now, suppose there are non-empty
            disjoint closed subsets $\mathcal{C},\mathcal{D}\subseteq{X}$ such
            that $\mathcal{C}\cup\mathcal{D}=X$. But then
            $\mathcal{U}=X\setminus\mathcal{C}$ and
            $\mathcal{V}=X\setminus\mathcal{D}$ are open non-empty disjoint
            sets whose union is $X$, so $(X,\,\tau)$ is disconnected.
        \end{proof}
    \section{The Connected Subsets of the Real Line}
    \section{Path-Connectedness}
\end{document}
