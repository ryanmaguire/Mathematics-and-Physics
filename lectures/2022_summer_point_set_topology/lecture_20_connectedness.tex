%-----------------------------------LICENSE------------------------------------%
%   This file is part of Mathematics-and-Physics.                              %
%                                                                              %
%   Mathematics-and-Physics is free software: you can redistribute it and/or   %
%   modify it under the terms of the GNU General Public License as             %
%   published by the Free Software Foundation, either version 3 of the         %
%   License, or (at your option) any later version.                            %
%                                                                              %
%   Mathematics-and-Physics is distributed in the hope that it will be useful, %
%   but WITHOUT ANY WARRANTY; without even the implied warranty of             %
%   MERCHANTABILITY or FITNESS FOR A PARTICULAR PURPOSE.  See the              %
%   GNU General Public License for more details.                               %
%                                                                              %
%   You should have received a copy of the GNU General Public License along    %
%   with Mathematics-and-Physics.  If not, see <https://www.gnu.org/licenses/>.%
%------------------------------------------------------------------------------%
\documentclass{article}
\usepackage{graphicx}                           % Needed for figures.
\usepackage{amsmath}                            % Needed for align.
\usepackage{amssymb}                            % Needed for mathbb.
\usepackage{amsthm}                             % For the theorem environment.
\usepackage{float}
\usepackage{tabularx, booktabs}
\usepackage[font=scriptsize,
            labelformat=simple,
            labelsep=colon]{subcaption} % Subfigure captions.
\usepackage[font={scriptsize},
            hypcap=true,
            labelsep=colon]{caption}    % Figure captions.
\usepackage{hyperref}
\hypersetup{
    colorlinks=true,
    linkcolor=blue,
    filecolor=magenta,
    urlcolor=Cerulean,
    citecolor=SkyBlue
}

%------------------------Theorem Styles-------------------------%
\theoremstyle{plain}
\newtheorem{theorem}{Theorem}[section]

% Define theorem style for default spacing and normal font.
\newtheoremstyle{normal}
    {\topsep}               % Amount of space above the theorem.
    {\topsep}               % Amount of space below the theorem.
    {}                      % Font used for body of theorem.
    {}                      % Measure of space to indent.
    {\bfseries}             % Font of the header of the theorem.
    {}                      % Punctuation between head and body.
    {.5em}                  % Space after theorem head.
    {}

% Define default environments.
\theoremstyle{normal}
\newtheorem{examplex}{Example}[section]
\newtheorem{definitionx}{Definition}[section]

\newenvironment{example}{%
    \pushQED{\qed}\renewcommand{\qedsymbol}{$\blacksquare$}\examplex%
}{%
    \popQED\endexamplex%
}

\newenvironment{definition}{%
    \pushQED{\qed}\renewcommand{\qedsymbol}{$\blacksquare$}\definitionx%
}{%
    \popQED\enddefinitionx%
}

\title{Point-Set Topology: Lecture 20}
\author{Ryan Maguire}
\date{Summer 2022}

% No indent and no paragraph skip.
\setlength{\parindent}{0em}
\setlength{\parskip}{0em}

\begin{document}
    \maketitle
    \section{Connectedness}
        \begin{figure}
            \centering
            \includegraphics{../../images/disconnected_space_001.pdf}
            \caption{A Disconnected Topological Space}
            \label{fig:disconnected_space_001}
        \end{figure}
        Connectedness is one of the fundamental notions in topology. Intuitively
        a connected space is one that is in \textit{one piece}. It can be hard
        to make precise what one means by this, but it can be easier to describe
        what \textit{disconnected} is. For intuition we use the plane. Two
        isolated discs in the plane should not be considered as a connected
        subspace since it is definitely not one piece
        (Fig.~\ref{fig:disconnected_space_001}). We use this to motivate
        disconnected spaces.
        \begin{definition}[\textbf{Disconnected Topological Space}]
            A disconnected topological space is a topological space $(X,\,\tau)$
            such that there are non-empty open subsets
            $\mathcal{U},\mathcal{V}\in\tau$ such that
            $\mathcal{U}\cap\mathcal{V}=\emptyset$ and
            $\mathcal{U}\cup\mathcal{V}=X$.
        \end{definition}
        \begin{example}
            The discrete topology on $\mathbb{Z}_{2}$ is disconnected. This
            space is two isolated points. To be precise, the set
            $\mathcal{U}=\{\,0\,\}$ is open and non-empty, the set
            $\mathcal{V}=\{\,1\,\}$ is open and non-empty, and these two sets
            satisfy $\mathcal{U}\cap\mathcal{V}=\emptyset$ and
            $\mathcal{U}\cup\mathcal{V}=\mathbb{Z}_{2}$.
        \end{example}
        \begin{example}
            If $X$ is any set containing at least two points, and if $\tau$ is
            the discrete topology, then $(X,\,\tau)$ is disconnected. Let
            $x\in{X}$ be one point and define $\mathcal{U}=\{\,x\,\}$. Since
            $\tau$ is the discrete topology $\mathcal{U}$ is open and non-empty.
            Let $\mathcal{V}=X\setminus\mathcal{U}$. Again, since $\tau$ is the
            discrete topology, $\mathcal{V}$ is open and since $X$ has at least
            two points it is also non-empty. But then $\mathcal{U}$ and
            $\mathcal{V}$ are non-empty open subsets such that
            $\mathcal{U}\cap\mathcal{V}=\emptyset$ and
            $\mathcal{U}\cup\mathcal{V}=X$, showing that $(X,\,\tau)$ is
            disconnected.
        \end{example}
        Connected is just \textit{not disconnected}.
        \begin{definition}[\textbf{Connected Topological Space}]
            A connected topological space is a topological space $(X,\,\tau)$
            that is not disconnected.
        \end{definition}
        Some familiar spaces like $\mathbb{R}$ and $\mathbb{R}^{2}$ are
        connected, but it takes a bit of work to show this. The spaces that are
        easy to show are connected straight from the definition have
        less-than-useful topologies.
        \begin{example}
            If $X$ is any set and $\tau$ is the indiscrete topology, then
            $(X,\,\tau)$ is connected. There are no two disjoint open sets
            $\mathcal{U},\mathcal{V}$ that are non-empty and cover $X$ since
            the only open sets are $\emptyset$ and $X$. So $(X,\,\tau)$ is
            connected.
        \end{example}
        \begin{example}
            The particular point topology on $\mathbb{R}$ defines a set
            $\mathcal{U}$ to be open if and only if $0\in\mathcal{U}$ or
            $\mathcal{U}=\emptyset$. Hence any two non-empty open sets that
            cover $\mathbb{R}$ must have $0$ in common, meaning we cannot
            separate the space into two disjoint non-empty open sets, so the
            particular point space is connected. Intuitively, every point is
            \textit{connected} to zero. 
        \end{example}
        \begin{example}
            The excluded point topology on $\mathbb{R}$ defines a set
            $\mathcal{U}$ to be open if and only if $0\notin\mathcal{U}$ or
            $\mathcal{U}=\mathbb{R}$. Because of this if $\mathcal{U}$ and
            $\mathcal{V}$ are open sets that cover $\mathbb{R}$, one of these
            sets must be $\mathbb{R}$. So it is impossible to
            separate the space using disjoint non-empty open sets.
        \end{example}
        \begin{example}
            The finite complement topology on $\mathbb{R}$ is connected.
            Given any
            non-empty open subsets $\mathcal{U},\mathcal{V}$, the intersection
            can not be empty since $\mathbb{R}\setminus\mathcal{U}$ and
            $\mathbb{R}\setminus\mathcal{V}$ are both finite, meaning
            $\mathcal{U}\cap\mathcal{V}$ is infinite (since $\mathbb{R}$ is
            infinite). So the finite complement topology on $\mathbb{R}$ is
            connected.
        \end{example}
        \begin{example}
            For similar reasons, the countable complement topology on
            $\mathbb{R}$ yields a connected space. Any two non-empty open
            subsets must have non-empty intersection since $\mathbb{R}$ is
            uncountable and the complements of two non-empty open subsets is
            countable (and hence so is the union of their complements).
        \end{example}
        \begin{example}
            The rationals $\mathbb{Q}$ with the subspace topology from
            $\mathbb{R}$ are disconnected. Let $\mathcal{U}$ be all positive
            rational numbers $x$ such that $x^{2}>2$. Let $\mathcal{V}$ be
            all rational numbers $x$ such that either $x<0$ or $x^{2}<2$. There
            is no rational number whose square is 2, so $\mathcal{U}$ and
            $\mathcal{V}$ are non-empty disjoint open subsets whose union is
            the entirety of $\mathbb{Q}$. So the rationals are disconnected. 
        \end{example}
        \begin{example}
            Let $X\subseteq\mathbb{R}$ be defined by
            $X=(-\infty,\,0)\cup(0,\,\infty)$. This is the real line with the
            origin removed. Equipping this with the subspace topology yields a
            disconnected space. Setting $\mathcal{U}=(-\infty,\,0)$ and
            $\mathcal{V}=(0,\,\infty)$ shows why.
        \end{example}
        When first discussing open and closed sets many students have trouble
        realizing that \textit{open} does not mean \textit{not closed}, and
        \textit{closed} does not mean \textit{not open}. It is possible for a
        subset to be open and not closed, closed and not open, neither open nor
        closed, and both open and closed. This last part is particularly hard to
        grasp since $\mathbb{R}$ has no subsets $\mathcal{U}\subseteq\mathbb{R}$
        that are both open and closed with the exception of
        $\mathcal{U}=\mathbb{R}$ and $\mathcal{U}=\emptyset$. This is because
        the real line is \textit{connected} and connected spaces have no
        proper non-empty subsets that are both open and closed. Let's prove
        this.
        \begin{theorem}
            If $(X,\,\tau)$ is a topological space, then it is disconnected
            if and only if there is a non-empty open proper subset
            $\mathcal{U}\subsetneq{X}$ that is also closed.
        \end{theorem}
        \begin{proof}
            If $(X,\,\tau)$ is disconnected there exists non-empty disjoint
            open sets $\mathcal{U}$ and $\mathcal{V}$ whose union is $X$. But
            then $X\setminus\mathcal{U}=\mathcal{V}$, and $\mathcal{V}$ is
            open, so $\mathcal{U}$ is closed. But $\mathcal{U}$ is also open,
            so $\mathcal{U}$ is a non-empty proper subset of $X$ that is also
            closed. Now suppose there is a proper subset
            $\mathcal{U}\subsetneq{X}$
            that is non-empty and both open and closed. Since $\mathcal{U}$ is
            closed, $\mathcal{V}=X\setminus\mathcal{U}$ is open. But since
            $\mathcal{U}$ is proper, $\mathcal{V}$ is non-empty. But then
            $\mathcal{U}$ and $\mathcal{V}$ are disjoint non-empty open subsets
            whose union is $X$, so $(X,\,\tau)$ is disconnected.
        \end{proof}
        \begin{theorem}
            If $(X,\,\tau)$ is a topological space, then it is disconnected if
            and only if there are non-empty disjoint closed subsets
            $\mathcal{C},\mathcal{D}\subseteq{X}$ such that
            $\mathcal{C}\cup\mathcal{D}=X$.
        \end{theorem}
        \begin{proof}
            If $(X,\,\tau)$ is disconnected there are non-empty disjoint open
            subsets $\mathcal{U},\mathcal{V}$ such that
            $\mathcal{U}\cup\mathcal{V}=X$. But then
            $\mathcal{C}=X\setminus\mathcal{U}$ and
            $\mathcal{D}=X\setminus\mathcal{V}$ are closed non-empty disjoint
            subsets whose union is $X$. Now, suppose there are non-empty
            disjoint closed subsets $\mathcal{C},\mathcal{D}\subseteq{X}$ such
            that $\mathcal{C}\cup\mathcal{D}=X$. But then
            $\mathcal{U}=X\setminus\mathcal{C}$ and
            $\mathcal{V}=X\setminus\mathcal{D}$ are open non-empty disjoint
            sets whose union is $X$, so $(X,\,\tau)$ is disconnected.
        \end{proof}
        One of the most useful theorems of connected spaces is that the
        continuous image of a connected topological space is still connected.
        To prove this requires a small theorem about subspaces.
        \begin{theorem}
            If $(X,\,\tau_{X})$ and $(Y,\,\tau_{Y})$ are topological spaces,
            if $f:X\rightarrow{Y}$ is continuous, and if
            $\mathcal{U}\subseteq{f}[X]$ is open in the subspace topology
            $\tau_{Y_{f[X]}}$, then $f^{-1}[\mathcal{U}]$ is open.
        \end{theorem}
        \begin{proof}
            Since $\mathcal{U}\in\tau_{Y_{f[X]}}$, by the definition of the
            subspace topology there is $\tilde{\mathcal{U}}\in\tau_{Y}$ such
            that $\mathcal{U}=f[X]\cap\tilde{\mathcal{U}}$. But then we have:
            \begin{align}
                f^{-1}[\mathcal{U}]
                &=f^{-1}\big[f[X]\cap\tilde{\mathcal{U}}\big]\\
                &=f^{-1}\big[f[X]\big]\cap{f}^{-1}[\tilde{\mathcal{U}}]\\
                &=X\cap{f}^{-1}[\tilde{\mathcal{U}}]\\
                &=f^{-1}[\tilde{\mathcal{U}}]
            \end{align}
            but $f$ is continuous, so $f^{-1}[\tilde{\mathcal{U}}]$ is open.
            Hence $f^{-1}[\mathcal{U}]$ is open.
        \end{proof}
        \begin{theorem}
            If $(X,\,\tau_{X})$ is a connected topological space, if
            $(Y,\,\tau_{Y})$ is a topological space, and if $f:X\rightarrow{Y}$
            is a continuous function, then $(f[X],\,\tau_{Y_{f[X]}})$ is
            connected where $\tau_{Y_{f[A]}}$ is the subspace topology from
            $\tau_{Y}$.
        \end{theorem}
        \begin{proof}
            Suppose not. Then there are disjoint non-empty open subsets
            $\mathcal{U},\mathcal{V}\in\tau_{Y_{f[X]}}$ such that
            $\mathcal{U}\cup\mathcal{V}=f[X]$. But then, since $\tau_{Y_{f[X]}}$
            is the subspace topology and $f:X\rightarrow{Y}$ is continuous,
            $f^{-1}[\mathcal{U}]$ and $f^{-1}[\mathcal{V}]$ are open non-empty
            subsets of $X$. But:
            \begin{equation}
                f^{-1}[\mathcal{U}]\cup{f}^{-1}[\mathcal{V}]
                =f^{-1}[\mathcal{U}\cup\mathcal{V}]
                =f^{-1}\big[f[X]\big]
                =X
            \end{equation}
            so $f^{-1}[\mathcal{U}]$ and $f^{-1}[\mathcal{V}]$ separate $X$,
            but $(X,\,\tau)$ is connected, a contradiction. Hence,
            $(f[X],\,\tau_{Y_{f[X]}})$ is connected.
        \end{proof}
        This does not say that $(Y,\,\tau_{Y})$ is connected, only that the
        image of a connected space $(X,\,\tau_{X})$ remains connected. If the
        function $f$ happens to be surjective, then we can upgrade this theorem.
        \begin{theorem}
            If $(X,\,\tau_{X})$ is a connected topological space, if
            $(Y,\,\tau_{Y})$ is a topological space, and if $f:X\rightarrow{Y}$
            is a surjective continuous function, then $(Y,\,\tau_{Y})$ is
            connected.
        \end{theorem}
        \begin{proof}
            By the previous theorem $(f[X],\,\tau_{Y_{f[X]}})$ is connected.
            But $f$ is surjective so $f[X]=Y$. And the subspace topology of
            $Y$ with respect to $\tau_{Y}$ is just $\tau_{Y}$. That is,
            $\tau_{Y_{f[X]}}=\tau_{Y}$. So $(Y,\,\tau_{Y})$ is connected.
        \end{proof}
        \begin{theorem}
            If $(X,\,\tau_{X})$ is a connected topological space, if
            $(Y,\,\tau_{Y})$ is a topological space, and if $f:X\rightarrow{Y}$
            is a quotient map, then $(Y,\,\tau_{Y})$ is connected.
        \end{theorem}
        \begin{proof}
            Quotient maps are continuous and surjective, so by the previous
            theorem $(Y,\,\tau_{Y})$ is connected.
        \end{proof}
        \begin{theorem}
            If $(X,\,\tau)$ is a connected topological space, and if $R$ is an
            equivalence relation on $X$, then $(X/R,\,\tau_{X/R})$ is
            connected where $\tau_{X/R}$ is the quotient topology.
        \end{theorem}
        \begin{proof}
            The canonical quotient function $q:X\rightarrow{X}/R$ defined by
            $q(x)=[x]$ is a quotient map. By the previous theorem, since
            $(X,\,\tau)$ is connected, so is $(X/R,\,\tau_{X/R})$.
        \end{proof}
        Connectedness is one of the few properties that quotient maps preserve.
        Remember that quotients do not need to preserve the Hausdorff condition,
        first or second countability, or any separation properties. The three
        main things they preserve are \textit{sequentialness},
        \textit{connectedness}, and \textit{compactness}.
        \par\hfill\par
        Just like how disconnected has a few equivalent definitions, so does
        connected.
        \begin{theorem}
            If $(X,\,\tau)$ is a topological space, then it is connected if and
            only if the only subsets of $X$ with empty topological boundary are
            $\emptyset$ and $X$.
        \end{theorem}
        \begin{proof}
            For suppose $(X,\,\tau)$ is disconnected. Then there is a non-empty
            proper subset $A\subsetneq{X}$ such that $A$ is open and closed.
            But since $A$ is closed, $\textrm{Cl}_{\tau}(A)=A$. But since
            $A$ is open, $\textrm{Int}_{\tau}(A)=A$. But then:
            \begin{equation}
                \partial_{\tau}(A)
                =\textrm{Cl}_{\tau}(A)\setminus\textrm{Int}_{\tau}(A)
                =A\setminus{A}
                =\emptyset
            \end{equation}
            Going the other way, suppose there is a non-empty proper subset
            $A\subsetneq{X}$ with empty boundary,
            $\partial_{\tau}(A)=\emptyset$. But
            $\textrm{Int}_{\tau}(A)\subseteq{A}$ for all $A\subseteq{X}$ and
            $A\subseteq\textrm{Cl}_{\tau}(A)$ as well. Hence
            $\textrm{Int}_{\tau}(A)\subseteq\textrm{Cl}_{\tau}(A)$. But if
            $\partial_{\tau}(A)=\emptyset$ then
            $\textrm{Cl}_{\tau}(A)\setminus\textrm{Int}_{\tau}(A)=\emptyset$,
            and thus $\textrm{Cl}_{\tau}(A)\subseteq\textrm{Int}_{\tau}(A)$.
            But then $A=\textrm{Cl}_{\tau}(A)=\textrm{Int}_{\tau}(A)$, so
            $A$ is both closed and open. But $A$ is non-empty and a proper
            subset of $X$, and therefore $(X,\,\tau)$ is disconnected. So
            $(X,\,\tau)$ is connected if and only if the only subsets of $X$
            with empty boundary are $\emptyset$ and $X$.
        \end{proof}
        \begin{theorem}
            If $(X,\,\tau)$ is a topological space, and if
            $\big(\mathbb{Z}_{2},\,\mathcal{P}(\mathbb{Z}_{2})\big)$ is the
            discrete topological space on $\mathbb{Z}_{2}$, then $(X,\,\tau)$
            is connected if and only if for every continuous function
            $f:X\rightarrow\mathbb{Z}_{2}$ it is true that $f$ is constant.
        \end{theorem}
        \begin{proof}
            Suppose $(X,\,\tau)$ is connected and $f:X\rightarrow\mathbb{Z}_{2}$
            is not constant. Then there is an $x\in{X}$ such that
            $f(x)=0$ and a $y\in{X}$ with $f(y)=1$. But then
            $f^{-1}[\{\,0\,\}]$ and $f^{-1}[\{\,1\,\}]$ are non-empty disjoint
            open sets that cover $X$ which is a contradiction since $(X,\,\tau)$
            is connected. So $f$ must be a constant. Conversely, suppose every
            continuous function $f:X\rightarrow\mathbb{Z}_{2}$ is a constant
            and suppose $(X,\,\tau)$ is disconnected. Then there is a non-empty
            proper subset $A\subsetneq{X}$ that is both open and closed.
            Define $f:X\rightarrow\mathbb{Z}_{2}$ via:
            \begin{equation}
                f(x)=
                \begin{cases}
                    0&x\in{A}\\
                    1&x\notin{A}
                \end{cases}
            \end{equation} 
            Since $A$ is non-empty there is an $x\in{X}$ such that $f(x)=0$ and
            since $A$ is also a proper subset $X\setminus{A}$ is non-empty so
            there is a $y\in{X}$ such that $f(y)=1$. That is, $f$ is not a
            constant function. But $f$ is continuous. The pre-image of
            $\{\,0\,\}$ is $A$, which is open, and the pre-image of $\{\,1\,\}$
            is $X\setminus{A}$ which is also open since $A$ is closed. But by
            hypothesis the only continuous functions from $X$ to
            $\mathbb{Z}_{2}$ are constants, a contradiction. Therefore
            $(X,\,\tau)$ is connected.
        \end{proof}
    \section{The Connected Subsets of the Real Line}
        Now we classify all subsets of the real line and use this to prove
        the intermediate value theorem, one of the fundamental results of
        real analysis and calculus, using just a little bit of topology.
        \begin{theorem}
            If $\tau_{\mathbb{R}}$ is the standard Euclidean topology on
            $\mathbb{R}$, then $(\mathbb{R},\,\tau_{\mathbb{R}})$ is connected.
        \end{theorem}
        \begin{proof}
            Suppose not and let $\mathcal{U}$ and $\mathcal{V}$ be non-empty
            disjoint open sets with $\mathcal{U}\cup\mathcal{V}=\mathbb{R}$.
            Since $\mathcal{U}$ and $\mathcal{V}$ cover $\mathbb{R}$ and are
            disjoint, either $0\in\mathcal{U}$ or $0\in\mathcal{V}$, but not
            both. Suppose
            $0\in\mathcal{U}$ (the idea is symmetric). Since $\mathcal{V}$ is
            non-empty there is some $x\in\mathcal{V}$. Either $x<0$ or
            $0<x$ by trichotomy. Suppose $0<x$ (again, the proof is symmetric).
            Let $E\subseteq\mathbb{R}$ be defined by:
            \begin{equation}
                E=\{\,y\in\mathcal{U}\;|\;0<y\textrm{ and }y<x\,\}
            \end{equation}
            Then $E$ is bounded above by $x$ and hence there is a least
            upper bound $c\in\mathbb{R}$. Since $\mathcal{U}$ and $\mathcal{V}$
            cover $\mathbb{R}$ either $c\in\mathcal{U}$ or $c\in\mathcal{V}$.
            Suppose $c\in\mathcal{U}$. Since $\mathcal{U}$ is open there is an
            $\varepsilon>0$ such that $|y-c|<\varepsilon$ implies
            $y\in\mathcal{U}$. But then $c+\varepsilon/2$ is an element of
            $\mathcal{U}$ that is still bounded by $x$ since $x\in\mathcal{V}$
            and hence $|x-c|\geq\varepsilon$. But then $c+\varepsilon/2$ is an
            element of $E$ that is greater than $c$, a contradiction since $c$
            is the least upper bound of $E$. So $c\notin\mathcal{U}$. But then
            $c\in\mathcal{V}$. But $\mathcal{V}$ is open so there is an
            $\varepsilon>0$ such that $|y-c|<\varepsilon$ implies
            $y\in\mathcal{V}$. But then $c-\varepsilon/2$ is an upper bound for
            $E$ that is less than $c$, a contradiction since $c$ is the least
            upper bound of $E$. So $c\notin\mathcal{V}$. But $\mathcal{U}$ and
            $\mathcal{V}$ cover $\mathbb{R}$, which is a contradiction. Hence
            $(\mathbb{R},\,\tau_{\mathbb{R}})$ is connected.
         \end{proof}
         \begin{theorem}
            If $A\subseteq\mathbb{R}$ is a connected subset with respect to the
            standard topology $\tau_{\mathbb{R}}$, then $A$ is one of the
            following sets:
            \begin{equation}
                A=
                \begin{cases}
                    (-\infty,\,a)\\
                    (-\infty,\,a]\\
                    [a,\,\infty)\\
                    (a,\,\infty)\\
                    (a,\,b)\\
                    [a,\,b)\\
                    (a,\,b]\\
                    [a,\,b]\\
                    \mathbb{R}\\
                    \emptyset
                \end{cases}
            \end{equation}
            for some $a,b\in\mathbb{R}$.
         \end{theorem}
         \begin{proof}
            Let $a,b\in\mathbb{R}\cup\{\,\pm\infty\,\}$ be defined by
            $a=\textrm{inf}(A)$ and $b=\textrm{sup}(A)$. Apply the same
            argument as before with the set of all real numbers between $a$ and
            $b$. Complete the proof by asking if $a$ and $b$ are finite and
            whether or not $a,b\in{A}$. This will give the table of
            possibilities above.
         \end{proof}
         \begin{theorem}[\textbf{Intermediate Value Theorem}]
            If $a,b\in\mathbb{R}$, $a<b$, and if
            $f:[a,\,b]\rightarrow\mathbb{R}$ is continuous with respect to the
            standard topologies on $[a,\,b]$ and $\mathbb{R}$, then for all
            real numbers $d$ between $f(a)$ and $f(b)$ there is a $c\in(a,\,b)$
            such that $f(c)=d$.
         \end{theorem}
         \begin{proof}
            Since $[a,\,b]$ is connected and $f$ is continuous,
            $f\big[[a,\,b]\big]$ is connected as well. But connected
            non-empty subsets of the real line are intervals
            (open, closed, half-open, or infinite). Meaning if
            $f(a)\leq{f}(b)$, then $[f(a),\,f(b)]\subseteq{f}\big[[a,\,b]\big]$
            and if $f(b)\leq{f}(a)$, then
            $[f(b),\,f(a)]\subseteq{f}\big[[a,\,b]\big]$. Either way, for
            all $d$ between $f(a)$ and $f(b)$ there is a $c$ between $a$ and $b$
            such that $f(c)=d$.
         \end{proof}
         \begin{theorem}[\textbf{1D Borsuk-Ulam Theorem}]
            If $f:\mathbb{S}^{1}\rightarrow\mathbb{R}$ is a continuous
            function, then there is a point $\mathbf{x}\in\mathbb{S}^{1}$ such
            that $f(\mathbf{x})=f(-\mathbf{x})$. That is, there are two
            antepodal points on the circle that map to the same real number.
         \end{theorem}
         \begin{proof}
            Define $g:\mathbb{R}\rightarrow\mathbb{S}^{1}$ via
            $g(t)=\big(\cos(2\pi{t}),\,\sin(2\pi{t})\big)$ and
            $h:[0,\,1]\rightarrow\mathbb{R}$ via
            $h(t)=f\big(g(t)\big)-f\big(g(t+\pi)\big)$. If $h(0)=0$, we
            are done since $(1,\,0)$ and $(-1,\,0)$ are then points with
            $f\big((1,\,0)\big)=f\big((-1,\,0)\big)$. Suppose $h(0)$ is positive
            (the idea is symmetric), $h(t)=c>0$. Then:
            \begin{align}
                h\big(\frac{1}{2}\big)
                &=f\Big(\big(\cos(\pi),\,\sin(\pi)\big)\Big)-
                    f\big(\big(\cos(0),\,\sin(0)\big)\Big)\\
                &=f\big((-1,\,0)\big)-f\big((1,\,0)\big)\\
                &=-\Big(f\big((1,\,0)\big)-f\big((-1,\,0)\big)\Big)\\
                &=-h(0)\\
                &=-c
            \end{align}
            and $-c<0$. So, by the intermediate value theorem, there is some
            real number $0<t_{0}<\frac{1}{2}$ such that $h(t_{0})=0$. But then
            $f\big(g(t_{0})\big)=f\big(g(t_{0}+\pi)\big)$, and hence
            $g(t_{0})$ and $g(t_{0}+\pi)$ are antepodal points that are mapped
            to the same real number via $f$. 
         \end{proof}
         This is called the 1D Borsuk-Ulam theorem because the actual
         Borsuk-Ulam theorem is quite stronger. If
         $f:\mathbb{S}^{n}\rightarrow\mathbb{R}^{n}$ is continuous, then there
         are antepodal points $\mathbf{x},-\mathbf{x}\in\mathbb{S}^{n}$ such
         that $f(\mathbf{x})=f(-\mathbf{x})$.
    \section{Path-Connectedness}
        The more intuitive notion of connected, the idea one probably thinks
        of when hearing \textit{connected}, is being able to walk from one
        point to another while staying in the space. This is a stronger notion
        and is called \textit{path connected}.
        \begin{definition}[\textbf{Path Connected Topological Space}]
            A path connected topological space is a topological space
            $(X,\,\tau)$ such that for all $x,y\in{X}$ there is a continuous
            function $f:[0,\,1]\rightarrow{X}$ such that $f(0)=x$ and
            $f(1)=y$, where $[0,\,1]$ has the subspace topology from
            $\mathbb{R}$.
        \end{definition}
        \begin{theorem}
            If $(X,\,\tau)$ is path connected, then it is connected.
        \end{theorem}
        \begin{proof}
            Suppose not. Then there are non-empty disjoint open subsets
            $\mathcal{U},\mathcal{V}$ such that $\mathcal{U}\cup\mathcal{V}=X$.
            But since $\mathcal{U}$ and $\mathcal{V}$ are non-empty, there are
            $x\in\mathcal{U}$ and $y\in\mathcal{V}$. But $(X,\,\tau)$ is
            path connected so there is a continuous function
            $f:[0,\,1]\rightarrow{X}$ such that $f(0)=x$ and $f(1)=y$. But then
            $f^{-1}[\mathcal{U}]$ and $f^{-1}[\mathcal{V}]$ are non-empty
            disjoint open subsets that cover $[0,\,1]$. But $[0,\,1]$ is
            connected, a contradiction. Hence $(X,\,\tau)$ is connected.
        \end{proof}
        \begin{theorem}
            If $n\in\mathbb{N}$ and $\tau_{\mathbb{R}^{n}}$ is the Euclidean
            topology, then $(\mathbb{R}^{n},\,\tau_{\mathbb{R}^{n}})$ is
            path connected.
        \end{theorem}
        \begin{proof}
            Given $\mathbf{x},\mathbf{y}\in\mathbb{R}^{n}$, define
            $\alpha:[0,\,1]\rightarrow\mathbb{R}^{n}$ by
            $\alpha(t)=(1-t)\mathbf{x}+t\mathbf{y}$. Then $\alpha$ is
            continuous, $\alpha(0)=\mathbf{x}$, and $\alpha(1)=\mathbf{y}$.
            Hence, $(\mathbb{R}^{n},\,\tau_{\mathbb{R}^{n}})$ is
            path connected.
        \end{proof}
        \begin{theorem}
            If $(X,\,\tau_{X})$ is a path connected topological space,
            if $(Y,\,\tau_{Y})$ is a topological space, and if
            $f:X\rightarrow{Y}$ is continuous, then
            $(f[X],\,\tau_{Y_{f[X]}})$ is path connected.
        \end{theorem}
        \begin{proof}
            Let $y_{0},y_{1}\in{f}[X]$. Then there are
            $x_{0},x_{1}\in{X}$ such that $f(x_{0})=y_{0}$ and $f(x_{1})=y_{1}$.
            But $(X,\,\tau_{X})$ is path connected, so there is a continuous
            function $\alpha:[0,\,1]\rightarrow{X}$ such that
            $\alpha(0)=x_{0}$ and $\alpha(1)=x_{1}$. But then
            $f\circ\alpha:[0,\,1]\rightarrow{f}[X]$ is the composition of
            continuous functions, which is hence continuous, such that
            $(f\circ\alpha)(0)=f(x_{0})=y_{0}$ and
            $(f\circ\alpha)(1)=f(x_{1})=y_{1}$. So $(f[X],\,\tau_{Y_{f[X]}})$
            is path connected.
        \end{proof}
        \begin{theorem}
            If $(X,\,\tau)$ is a topological space, if
            $A,B\subseteq{X}$ are such that $(A,\,\tau_{A})$ and
            $(B,\,\tau_{B})$ are path connected, where $\tau_{A}$ and
            $\tau_{B}$ are the subspace topologies, and if
            $A\cap{B}\ne\emptyset$, then $(A\cup{B},\,\tau_{A\cup{B}})$ is
            path connected.
        \end{theorem}
        \begin{proof}
            Since $A\cap{B}\ne\emptyset$ there is an $x\in{A}\cap{B}$.
            Let $y_{0},y_{1}\in{A}\cup{B}$. Since $y_{0}\in{A}\cup{B}$, either
            $y_{0}\in{A}$ or $y_{0}\in{B}$. Either way, since $x\in{A}\cap{B}$
            and $(A,\,\tau_{A})$ and $(B,\,\tau_{B})$ are path connected,
            there is a continuous function $f:[0,\,1]\rightarrow{A}\cup{B}$
            such that $f(0)=y_{0}$ and $f(1)=x$. Similarly there is a
            continuous function $g:[0,\,1]\rightarrow{A}\cup{B}$ such that
            $g(0)=x$ and $g(1)=y_{1}$. By the pasting lemma the function
            $h:[0,\,1]\rightarrow{A}\cup{B}$ defined by:
            \begin{equation}
                h(t)=
                \begin{cases}
                    f(2t)&0\leq{t}\leq\frac{1}{2}\\
                    g(2t-1)&\frac{1}{2}\leq{t}\leq{1}
                \end{cases}
            \end{equation}
            is continuous. But then $h(0)=f(0)=y_{0}$ and $h(1)=g(1)=y_{1}$, so
            $h:[0,\,1]\rightarrow{A}\cup{B}$ is a continuous function such that
            $h(0)=y_{0}$ and $h(1)=y_{1}$. So
            $(A\cup{B},\,\tau_{A\cup{B}})$ is path connected. 
        \end{proof}
        Connected need not imply path connected.
        \begin{example}[\textbf{The Infinite Broom}]
            The infinite broom is a subset of $\mathbb{R}^{2}$ defined by
            taking all closed line segments from $(0,\,0)$ to
            $(1,\,\frac{1}{n+1})$ for all $n\in\mathbb{N}$, together with
            the half-open line segment between
            $(\frac{1}{2},\,0)$ and $(1,\,0)$, including $(1,\,0)$ be excluding
            $(\frac{1}{2},\,0)$. See Fig.~\ref{fig:infinite_broom_001}.
            If we did not include this last half-open interval, the space would
            be path connected. But with this half-open interval the space
            is connected but not path connected. There is no path from
            $(0,\,0)$ to $(1,\,0)$. However, since the infinite broom has the
            subspace topology from $\mathbb{R}^{2}$, any open subset that
            contains the half-open interval must contain points from infinitely
            many of the line segments from $(0,\,0)$ to $(1,\,\frac{1}{n+1})$.
            Because of this it is impossible to disconnect the space with two
            non-empty disjoint open subsets, showing that the infinite broom is
            connected.
        \end{example}
        \begin{figure}
            \centering
            \includegraphics{../../images/infinite_broom_001.pdf}
            \caption{The Infinite Broom}
            \label{fig:infinite_broom_001}
        \end{figure}
        \begin{example}[\textbf{The Topologist's Sine Curve}]
            The topologist's sine curve is a subset of $\mathbb{R}^{2}$ defined
            by taking all points of the form
            $\big(x,\,\sin(1/x)\big)$ with $x\in(0,\,1]$ and adding the
            origin $(0,\,0)$ (Fig.~\ref{fig:topologists_sine_curve_001}). For
            reasons similar to the infinite broom, the topologist's sine curve
            is connected but not path connected.
        \end{example}
        \begin{figure}
            \centering
            \includegraphics{../../images/topologists_sine_curve_001.pdf}
            \caption{The Topologist's Sine Curve}
            \label{fig:topologists_sine_curve_001}
        \end{figure}
\end{document}
