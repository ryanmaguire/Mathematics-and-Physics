%-----------------------------------LICENSE------------------------------------%
%   This file is part of Mathematics-and-Physics.                              %
%                                                                              %
%   Mathematics-and-Physics is free software: you can redistribute it and/or   %
%   modify it under the terms of the GNU General Public License as             %
%   published by the Free Software Foundation, either version 3 of the         %
%   License, or (at your option) any later version.                            %
%                                                                              %
%   Mathematics-and-Physics is distributed in the hope that it will be useful, %
%   but WITHOUT ANY WARRANTY; without even the implied warranty of             %
%   MERCHANTABILITY or FITNESS FOR A PARTICULAR PURPOSE.  See the              %
%   GNU General Public License for more details.                               %
%                                                                              %
%   You should have received a copy of the GNU General Public License along    %
%   with Mathematics-and-Physics.  If not, see <https://www.gnu.org/licenses/>.%
%------------------------------------------------------------------------------%
\documentclass{article}
\usepackage{graphicx}                           % Needed for figures.
\usepackage{amsmath}                            % Needed for align.
\usepackage{amssymb}                            % Needed for mathbb.
\usepackage{amsthm}                             % For the theorem environment.
\usepackage{float}
\usepackage{tabularx, booktabs}
\usepackage{mathrsfs}
\usepackage[font=scriptsize,
            labelformat=simple,
            labelsep=colon]{subcaption} % Subfigure captions.
\usepackage[font={scriptsize},
            hypcap=true,
            labelsep=colon]{caption}    % Figure captions.
\usepackage{hyperref}
\hypersetup{
    colorlinks=true,
    linkcolor=blue,
    filecolor=magenta,
    urlcolor=Cerulean,
    citecolor=SkyBlue
}

%------------------------Theorem Styles-------------------------%
\theoremstyle{plain}
\newtheorem{theorem}{Theorem}[section]

% Define theorem style for default spacing and normal font.
\newtheoremstyle{normal}
    {\topsep}               % Amount of space above the theorem.
    {\topsep}               % Amount of space below the theorem.
    {}                      % Font used for body of theorem.
    {}                      % Measure of space to indent.
    {\bfseries}             % Font of the header of the theorem.
    {}                      % Punctuation between head and body.
    {.5em}                  % Space after theorem head.
    {}

% Define default environments.
\theoremstyle{normal}
\newtheorem{examplex}{Example}[section]
\newtheorem{definitionx}{Definition}[section]

\newenvironment{example}{%
    \pushQED{\qed}\renewcommand{\qedsymbol}{$\blacksquare$}\examplex%
}{%
    \popQED\endexamplex%
}

\newenvironment{definition}{%
    \pushQED{\qed}\renewcommand{\qedsymbol}{$\blacksquare$}\definitionx%
}{%
    \popQED\enddefinitionx%
}

\title{Point-Set Topology: Lecture 26}
\author{Ryan Maguire}
\date{Summer 2022}

% No indent and no paragraph skip.
\setlength{\parindent}{0em}
\setlength{\parskip}{0em}

\begin{document}
    \maketitle
    \section{Metrization Theorems}
        So far the only metrization theorem we have is Urysohn's. It makes use
        of normality. We know from homework that metrizable spaces are
        perfectly normal. In improving Urysohn's metrization theorem to one
        that contains necessary and sufficient conditions we need to upgrade
        to perfect normality. From a few weeks ago, a perfectly normal
        topological space is some space $(X,\,\tau)$ such that for all
        disjoint closed subsets $\mathcal{C},\mathcal{D}\subseteq{X}$ there is
        a continuous function $f:X\rightarrow[0,\,1]$ such that
        $\mathcal{C}=f^{-1}[\{\,0\,\}]$ and $\mathcal{D}=f^{-1}[\{\,1\,\}]$.
        Perfectly normal implies normal (we proved this) and it also implies
        completely normal (every subspace is also normal, we haven't proven
        this). The reformulation we want is in terms of $G_{\delta}$ sets.
        \begin{definition}[\textbf{$G_{\delta}$ Set}]
            A $G_{\delta}$ set in a topological space $(X,\,\tau)$ is a set
            $A\subseteq{X}$ such that there is a countable set
            $\mathcal{O}\subseteq\tau$ such that $A=\bigcap\mathcal{O}$.
        \end{definition}
        \begin{theorem}
            If $(X,\,\tau)$ is a topological space, and if $\mathcal{U}\in\tau$,
            then $\mathcal{U}$ is a $G_{\delta}$ set.
        \end{theorem}
        \begin{proof}
            Let $\mathcal{O}=\{\,\mathcal{U}\,\}$. Then $\mathcal{O}$ is
            countable since it is finite, but also
            $\mathcal{U}=\bigcap\mathcal{O}$. So $\mathcal{U}$ is a
            $G_{\delta}$ set.
        \end{proof}
        \begin{theorem}
            If $(X,\,\tau)$ is metrizable, and if $\mathcal{C}\subseteq{X}$ is
            closed, then it is a $G_{\delta}$ set.
        \end{theorem}
        \begin{proof}
            Since $(X,\,\tau)$ is metrizable, there is a metric $d$ that induces
            $\tau$. For all $n\in\mathbb{N}$ define $\mathcal{U}_{n}$ via:
            \begin{equation}
                \mathcal{U}_{n}
                =\bigcup_{x\in\mathcal{C}}B_{\frac{1}{n+1}}^{(X,\,d)}(x)
            \end{equation}
            Then $\mathcal{U}_{n}$ is open, being the union of open sets.
            Also $\mathcal{U}_{n+1}\subseteq\mathcal{U}_{n}$ and
            $\mathcal{C}\subseteq\mathcal{U}_{n}$ for all $n\in\mathbb{N}$.
            Suppose $x\in\bigcap_{n\in\mathbb{N}}\mathcal{U}_{n}$. Then for
            all $n\in\mathbb{N}$ there is an $a_{n}\in\mathcal{C}$ such that
            $d(x,\,a_{n})<\frac{1}{n+1}$. But then $a_{n}\rightarrow{x}$. But
            $(X,\,\tau)$ is metrizable, and hence sequential, so since
            $\mathcal{C}$ is closed, if $a:\mathbb{N}\rightarrow\mathcal{C}$ is
            a convergent sequence with limit $x\in{X}$, then
            $x\in\mathcal{C}$. Hence
            $\bigcap_{n\in\mathbb{N}}\mathcal{U}_{n}=\mathcal{C}$ so
            $\mathcal{C}$ is a $G_{\delta}$ set.
        \end{proof}
        This is part of the idea we wish to capture. We want closed sets to be
        $G_{\delta}$ sets. Topological spaces with this property are given a
        name.
        \begin{definition}[\textbf{$G_{\delta}$ Topological Space}]
            A $G_{\delta}$ topological space is a topological space $(X,\,\tau)$
            such that every closed subset $\mathcal{C}\subseteq{X}$ is a
            $G_{\delta}$ set.
        \end{definition}
        \begin{theorem}
            If $(X,\,\tau_{X})$ and $(Y,\,\tau_{Y})$ are topological spaces,
            if $f:X\rightarrow{Y}$ is continuous, and if $A\subseteq{Y}$ is a
            $G_{\delta}$ set, then $f^{-1}[A]$ is a $G_{\delta}$ set.
        \end{theorem}
        \begin{proof}
            Since $A$ is a $G_{\delta}$ set there is a sequence
            $\mathcal{V}:\mathbb{N}\rightarrow\tau_{Y}$ such that
            $A=\bigcap_{n\in\mathbb{N}}\mathcal{V}_{n}$. But then:
            \begin{equation}
                f^{-1}[A]
                =f^{-1}\Big[\bigcap_{n\in\mathbb{N}}\mathcal{V}_{n}\Big]
                =\bigcap_{n\in\mathbb{N}}f^{-1}[\mathcal{V}_{n}]
            \end{equation}
            But since $f$ is continuous and $\mathcal{V}_{n}\in\tau_{Y}$, we
            have that $f^{-1}[\mathcal{V}_{n}]\in\tau_{X}$, and hence
            $f^{-1}[A]$ is a $G_{\delta}$ set.
        \end{proof}
        \begin{theorem}
            If $(X,\,\tau)$ is a topological space, then it is perfectly normal
            if and only if for all closed $\mathcal{C}\subseteq{X}$ there is a
            continuous function $f:X\rightarrow[0,\,1]$ such that
            $\mathcal{C}=f^{-1}[\{\,0\,\}]$.
        \end{theorem}
        \begin{proof}
            If $(X,\,\tau)$ is perfectly normal, let $\mathcal{D}=\emptyset$.
            Then $\mathcal{C}$ and $\mathcal{D}$ are disjoint closed sets so
            there is a continuous function $f:X\rightarrow[0,\,1]$ such that
            $\mathcal{C}=f^{-1}[\{\,0\,\}]$ and
            $\mathcal{D}=f^{-1}[\{\,1\,\}]$. So in particular,
            $\mathcal{C}=f^{-1}[\{\,0\,\}]$. Now suppose $(X,\,\tau)$ is such
            that for all closed $\mathcal{C}$ there is a continuous function
            $f:X\rightarrow[0,\,1]$ such that
            $\mathcal{C}=f^{-1}[\{\,0\,\}]$. Given $\mathcal{C},\mathcal{D}$
            closed and disjoint, let $f$ be the corresponding function for
            $\mathcal{C}$ and $g$ the function for $\mathcal{D}$. Define:
            \begin{equation}
                h(x)
                =\frac{f(x)}{f(x)+g(x)}
            \end{equation}
            This is well-defined since $\mathcal{C}\cap\mathcal{D}=\emptyset$.
            For if $x\in\mathcal{C}$, then $f(x)=0$ so
            the denominator is $g(x)$ and $g(x)>0$ for all $x\notin\mathcal{D}$.
            If $x\in\mathcal{D}$ then $g(x)=0$, so the denominator is
            $f(x)$ and $f(x)>0$ for all $x\notin\mathcal{C}$. If
            $x\not\mathcal{C}$ and $x\notin\mathcal{D}$ then $f(x)+g(x)>0$.
            It is continuous since it is the quotient of continuous functions.
            Lastly, $h^{-1}[\{\,0\,\}]=\mathcal{C}$ and
            $h^{-1}[\{\,1\,\}]=\mathcal{D}$. For $h(x)=0$ if and only if
            $f(x)=0$ and hence $x\in\mathcal{C}$. Also $h(x)=1$ if and only if
            $f(x)=f(x)+g(x)$ which is true if and only if $g(x)=0$, meaning
            $x\in\mathcal{D}$. So $(X,\,\tau)$ is perfectly normal.
        \end{proof}
        Perfect normality is equivalent to normal plus $G_{\delta}$. To prove
        this requires the topological version of one of the foundational
        theorems of real analysis.
        \begin{theorem}
            If $(X,\,\tau)$ is a topological space, if
            $F:\mathbb{N}\rightarrow{C}(X,\,\mathbb{R})$ is a sequence of
            continuous functions, if $f:X\rightarrow\mathbb{R}$ is
            such that for all $x\in{X}$ it is true that
            $F_{n}(x)\rightarrow{f}(x)$, and if:
            \begin{equation}
                \lim_{n\rightarrow\infty}\sup_{x\in{X}}|F_{n}(x)-f(x)|=0
            \end{equation}
            then $f$ is continuous.
        \end{theorem}
        \begin{proof}
            We use the equivalent definition of continuity that for all
            $x\in{X}$ and for all $\mathcal{V}\in\tau_{\mathbb{R}}$ such that
            $f(x)\in\mathcal{V}$ there is a $\mathcal{U}\in\tau$ such that
            $x\in\mathcal{U}$ and $f[\mathcal{U}]\subseteq\mathcal{V}$.
            Let $x\in{X}$ and $\mathcal{V}\in\tau_{\mathbb{R}}$ be such that
            $f(x)\in\mathcal{V}$. Since $\mathcal{V}$ is open in $\mathbb{R}$
            there is an $\varepsilon>0$ such that
            $(f(x)-\varepsilon,\,f(x)+\varepsilon)\subseteq\mathcal{V}$. But
            then, since $\textrm{sup}|F_{n}(x)-f(x)|\rightarrow{0}$, there is an
            $N\in\mathbb{N}$ such that for all $n\in\mathbb{N}$ with $n\geq{N}$
            and for all $x\in{X}$ we have:
            \begin{equation}
                |F_{n}(x)-f(x)|<\frac{\varepsilon}{3}
            \end{equation}
            But $F_{N}:X\rightarrow[0,\,1]$ is continuous, so there is an open
            set $\mathcal{U}\in\tau$ such that $x\in\mathcal{U}$ and
            $f[\mathcal{U}]\subseteq(f(x)-\varepsilon/3,\,f(x)+\varepsilon/3)$.
            But then, for all $x,y\in\mathcal{U}$, we have:
            \begin{align}
                |f(x)-f(y)|
                &=|f(x)-F_{N}(x)+F_{N}(x)-F_{N}(y)+f_{N}(y)-f(y)|\\
                &\leq|f(x)-f_{N}(x)|+|F_{N}(x)-F_{N}(y)|+|F_{N}(y)-f(y)|\\
                &<\frac{\varepsilon}{3}+\frac{\varepsilon}{3}
                    +\frac{\varepsilon}{3}\\
                &=\varepsilon
            \end{align}
            and hence for all $y\in\mathcal{U}$ we have
            $f(y)\in(f(x)-\varepsilon,\,f(x)+\varepsilon)$, so
            $f[\mathcal{U}]\subseteq(f(x)-\varepsilon,\,f(x)+\varepsilon)$, and
            hence $f[\mathcal{U}]\subseteq\mathcal{V}$. Thus, $f$ is continuous.
        \end{proof}
        \begin{theorem}
            If $(X,\,\tau)$ is a topological space, then it is perfectly normal
            if and only if it is normal and a $G_{\delta}$ space.
        \end{theorem}
        \begin{proof}
            Perfectly normal implies normal, we need only prove it implies
            $G_{\delta}$ as well. Let $\mathcal{C}\subseteq{X}$ be closed.
            Since $(X,\,\tau)$ is perfectly normal, there is a continuous
            function $f:X\rightarrow[0,\,1]$ such that
            $\mathcal{C}=f^{-1}[\{\,0\,\}]$. For all $n\in\mathbb{N}$ define:
            \begin{equation}
                \mathcal{U}_{n}=f^{-1}\big[[0,\,\frac{1}{n+1})\big]
            \end{equation}
            Since $f$ is continuous, and since $[0,\,\frac{1}{n+1})$ is open in
            the subspace topology for $[0,\,1]$, $\mathcal{U}_{n}$ is open.
            Also $\mathcal{C}\subseteq\mathcal{U}_{n}$ for all $n\in\mathbb{N}$.
            Suppose $x\in\bigcap_{n\in\mathbb{N}}\mathcal{U}_{n}$.
            Then for all $n\in\mathbb{N}$, $f(x)\in[0,\,\frac{1}{n+1})$, and
            hence $f(x)=0$. But then $x\in\mathcal{C}$. Thus,
            $\mathcal{C}=\bigcap_{n\in\mathbb{N}}\mathcal{U}_{n}$ so
            $\mathcal{C}$ is a $G_{\delta}$ set. That is, if $(X,\,\tau)$ is
            perfectly normal, then it is a normal $G_{\delta}$ space. In the
            other direction, suppose $(X,\,\tau)$ is a normal $G_{\delta}$
            space. By a previous theorem to prove normality it suffices to
            show that for all closed $\mathcal{C}\subseteq{X}$ there is a
            continuous function $f:X\rightarrow[0,\,1]$ such that
            $\mathcal{C}=f^{-1}[\{\,0\,\}]$. Since $(X,\,\tau)$ is a
            $G_{\delta}$ space and $\mathcal{C}$ is closed, there is a sequence
            $\mathcal{U}:\mathbb{N}\rightarrow\tau$ such that
            $\mathcal{C}=\bigcap_{n\in\mathbb{N}}\mathcal{U}_{n}$. But then
            for all $n\in\mathbb{N}$ we have
            $\mathcal{C}\subseteq\mathcal{U}_{n}$ and hence
            $\mathcal{C}\cap(X\setminus\mathcal{U}_{n})=\emptyset$. But
            $\mathcal{U}_{n}$ is open, so $X\setminus\mathcal{U}_{n}$ is closed.
            By Urysohn's lemma there is a continuous function
            $F_{n}:X\rightarrow[0,\,1]$ such that
            $\mathcal{C}\subseteq{F}_{n}^{-1}[\{\,0\,\}]$ and
            $X\setminus\mathcal{U}_{n}\subseteq{F}_{n}^{-1}[\{\,1\,\}]$. Define:
            \begin{equation}
                f(x)=\sum_{n=0}^{\infty}\frac{F_{n}(x)}{2^{n}}
            \end{equation}
            Then since $0\leq{F}_{n}(x)\leq{1}$ for all $n\in\mathbb{N}$ and all
            $x\in{X}$ the $n^{\textrm{th}}$ term of this sum is bounded by
            $1/2^{n}$, which converges to zero. By the previous theorem
            $f$ is continuous. Since $F_{n}(x)=0$ for all
            $x\in\mathcal{C}$ and all $n\in\mathbb{N}$, we have that
            $f(x)=0$ for all $x\in\mathcal{C}$. Suppose $x\notin\mathcal{C}$.
            Then there is a $\mathcal{U}_{n}$ such that
            $x\notin\mathcal{U}_{n}$, since
            $\mathcal{C}=\bigcap_{n\in\mathbb{N}}\mathcal{U}_{n}$. But then
            $F_{n}(x)>0$, and hence $f(x)>0$. So
            $x\notin{f}^{-1}[\{\,0\,\}]$. That is,
            $\mathcal{C}=f^{-1}[\{\,0\,\}]$, so $(X,\,\tau)$ is perfectly normal.
        \end{proof}
        The key to the Nagata-Smirnov theorem is $\sigma$ locally finite bases.
        From a few lectures ago, a $\sigma$ locally finite cover of a
        topological space $(X,\,\tau)$ is a cover $\mathcal{O}\subseteq\tau$
        such that there exists a sequence
        $\Delta:\mathbb{N}\rightarrow\mathcal{P}(\tau)$ such that for all
        $n\in\mathbb{N}$ it is true that $\Delta_{n}\subseteq\tau$ is a locally
        finite collection, and such that
        $\mathcal{O}=\bigcup_{n\in\mathbb{N}}\Delta_{n}$. $\sigma$ locally
        finite basis just adds the word \textit{basis}.
        \begin{definition}[\textbf{$\sigma$ Locally Finite Basis}]
            A $\sigma$ locally finite basis for a topological space $(X,\,\tau)$
            is a basis $\mathcal{B}$ such that $\mathcal{B}$ is a $\sigma$
            locally finite open cover.
        \end{definition}
        \begin{theorem}
            If $(X,\,\tau)$ is a regular topological space, if
            $\mathcal{B}$ is a $\sigma$ locally finite bases for $\tau$, and
            if $\mathcal{U}\in\tau$, then there is a sequence
            $\mathcal{V}:\mathbb{N}\rightarrow\tau$ such that
            $\mathcal{U}=\bigcup_{n\in\mathbb{N}}\textrm{Cl}_{\tau}(\mathcal{V}_{n})$.
        \end{theorem}
        \begin{proof}
            Since $\mathcal{B}$ is a $\sigma$ locally finite basis, there is a
            sequence $\Delta:\mathbb{N}\rightarrow\mathcal{P}(\tau)$ such that
            for all $n\in\mathbb{N}$ it is true that $\Delta_{n}$ is locally
            finite and such that
            $\mathcal{B}=\bigcup_{n\in\mathbb{N}}\Delta_{n}$. For all
            $n\in\mathbb{N}$ define $\mathcal{A}_{n}$ via:
            \begin{equation}
                \mathcal{A}_{n}=
                \{\,\mathcal{W}\in\mathcal{B}_{n}\;|\;
                    \textrm{Cl}_{\tau}(\mathcal{W})\subseteq\mathcal{U}\,\}
            \end{equation}
            Since $\Delta_{n}$ is locally finite, so is $\mathcal{A}_{n}$.
            Define $\mathcal{V}_{n}$ via:
            \begin{equation}
                \mathcal{V}_{n}=\bigcup\mathcal{A}_{n}
            \end{equation}
            Then $\mathcal{V}_{n}$ is open, being the union of open sets, and
            since $\mathcal{A}_{n}$ is locally finite we have:
            \begin{align}
                \textrm{Cl}_{\tau}(\mathcal{V}_{n})
                &=\textrm{Cl}_{\tau}\Big(\bigcup\mathcal{A}_{n}\Big)\\
                &=\textrm{Cl}_{\tau}\Big(
                    \bigcup_{\mathcal{W}\in\mathcal{A}_{n}}\mathcal{W}\Big)\\
                &=\bigcup_{\mathcal{W}\in\mathcal{A}_{n}}
                    \textrm{Cl}_{\tau}(\mathcal{W})
            \end{align}
            But for all $\mathcal{W}\in\mathcal{A}_{n}$ it is true that
            $\textrm{Cl}_{\tau}(\mathcal{W})\subseteq\mathcal{U}$ by definition
            of $\mathcal{A}_{n}$, and hence
            $\textrm{Cl}_{\tau}(\mathcal{V}_{n})\subseteq\mathcal{U}$. Since
            this is true for all $n\in\mathbb{N}$ we have:
            \begin{equation}
                \bigcup_{n\in\mathbb{N}}\mathcal{V}_{n}\subseteq\mathcal{U}
            \end{equation}
            But $(X,\,\tau)$ is regular and $\mathcal{B}$ is a basis. So given
            $x\in{X}$ there is a $\mathcal{W}\in\mathcal{B}$ such that
            $x\in\mathcal{W}$ and $\mathcal{W}\subseteq\mathcal{U}$. From
            regularity there is a $\mathcal{W}'\in\tau$ such that
            $x\in\mathcal{W}'$ and
            $\textrm{Cl}_{\tau}(\mathcal{W}')\subseteq\mathcal{W}$. But again,
            since $\mathcal{B}$ is a basis, there is a
            $\mathcal{W}''\in\mathcal{B}$ such that $x\in\mathcal{W}''$ and
            $\mathcal{W}''\subseteq\mathcal{W}'$. But then
            $x\in\mathcal{W}''$ and
            $\textrm{Cl}_{\tau}(\mathcal{W}'')\subseteq\mathcal{U}$. Since
            $\mathcal{W}''\in\mathcal{B}$ and
            $\mathcal{B}=\bigcup_{n\in\mathbb{N}}\Delta_{n}$ there is an
            $n\in\mathbb{N}$ such that $\mathcal{W}''\in\Delta_{n}$. But then,
            by definition of $\mathcal{A}_{n}$,
            $\mathcal{W}''\in\mathcal{A}_{n}$ and hence
            $x\in\mathcal{V}_{n}$. From this we obtain:
            \begin{equation}
                \mathcal{U}\subseteq\bigcup_{n\in\mathbb{N}}
                    \textrm{Cl}_{\tau}(\mathcal{V}_{n})
            \end{equation}
            Meaning
            $\mathcal{U}=\bigcup_{n\in\mathbb{N}}\textrm{Cl}_{\tau}(\mathcal{V}_{n})$.
        \end{proof}
        \begin{theorem}
            If $(X,\,\tau)$ is a regular topological space, and if
            $\mathcal{B}$ is a $\sigma$ locally finite basis for $\tau$, then
            $(X,\,\tau)$ is a $G_{\delta}$ space.
        \end{theorem}
        \begin{proof}
            Let $\mathcal{C}\subseteq{X}$ be closed. Then
            $X\setminus\mathcal{C}$ is open. But since $(X,\,\tau)$ is regular
            and $\mathcal{B}$ is a $\sigma$ locally finite basis, there is a
            sequence $\mathcal{U}:\mathbb{N}\rightarrow\tau$ such that
            $X\setminus\mathcal{C}=\bigcup_{n\in\mathbb{N}}\textrm{Cl}_{\tau}(\mathcal{U}_{n})$.
            But then for all $n\in\mathbb{N}$ we have
            $\mathcal{C}\subseteq{X}\setminus\textrm{Cl}_{\tau}(\mathcal{U}_{n})$,
            and hence
            $\mathcal{C}\subseteq\bigcap_{n\in\mathbb{N}}\textrm{Cl}_{\tau}(\mathcal{U}_{n})$.
            Let $x\in\bigcap_{n\in\mathbb{N}}\textrm{Cl}_{\tau}(\mathcal{U}_{n})$
            and suppose $x\notin\mathcal{C}$. Then
            $x\in{X}\setminus\mathcal{C}$, so there is some
            $n\in\mathcal{N}$ such that $x\in\textrm{Cl}_{\tau}(\mathcal{U}_{n})$.
            But then $x\notin{X}\setminus\mathcal{U}_{n}$, which is a
            contradiction since
            $x\in\bigcap_{n\in\mathbb{N}}\textrm{Cl}_{\tau}(\mathcal{U}_{n})$.
            So $x\in\mathcal{C}$, and therefore $\mathcal{C}$ is the countable
            intersection of open sets, meaning it is a $G_{\delta}$ set.
            So all closed subsets of $X$ are $G_{\delta}$ sets, meaning
            $(X,\,\tau)$ is a $G_{\delta}$ space.
        \end{proof}
        \begin{theorem}
            If $(X,\,\tau)$ is a regular topological space, and if
            $\mathcal{B}$ is a $\sigma$ locally finite basis for $\tau$,
            then $(X,\,\tau)$ is perfectly normal.
        \end{theorem}
        \begin{proof}
            Regularity and a $\sigma$ locally finite basis implies $(X,\,\tau)$
            is a $G_{\delta}$ space by the previous theorem. To prove the space
            is perfectly normal we need only prove it is normal, since
            $(X,\,\tau)$ is perfectly normal if and only if it is normal and a
            $G_{\delta}$ space. Let $\mathcal{C},\mathcal{D}\subseteq{X}$ be
            disjoint closed sets. Then $X\setminus\mathcal{C}$ and
            $X\setminus\mathcal{D}$ are disjoint open sets. But since
            $(X,\,\tau)$ is regular and has a $\sigma$ locally finite basis
            there are sequences
            $\mathcal{U},\mathcal{V}:\mathbb{N}\rightarrow\tau$ such that
            $X\setminus\mathcal{C}=\bigcup_{n\in\mathbb{N}}\textrm{Cl}_{\tau}(\mathcal{U}_{n})$
            and
            $X\setminus\mathcal{D}=\bigcup_{n\in\mathbb{N}}\textrm{Cl}_{\tau}(\mathcal{V}_{n})$.
            Define $\tilde{\mathcal{U}}_{n}$ and
            $\tilde{\mathcal{V}}_{n}$ via:
            \begin{align}
                \tilde{\mathcal{U}}_{n}
                &=\mathcal{U}_{n}\setminus\bigcup_{k=0}^{n}
                    \textrm{Cl}_{\tau}(\mathcal{V}_{n})\\
                \tilde{\mathcal{V}}_{n}
                &=\mathcal{V}_{n}\setminus\bigcup_{k=0}^{n}
                    \textrm{Cl}_{\tau}(\mathcal{U}_{n})
            \end{align}
            Each $\tilde{\mathcal{U}}_{n}$ and $\tilde{\mathcal{V}}_{n}$ is
            open since they are the set difference of a finite union of closed
            sets (which is hence closed) from an open set. Since
            $\mathcal{C}$ is disjoint from each
            $\textrm{Cl}_{\tau}(\mathcal{U}_{n})$ we have that
            $\mathcal{C}\subseteq\bigcup_{n\in\mathbb{N}}\tilde{\mathcal{U}}_{n}$.
            Similarly for $\mathcal{D}$ with
            $\bigcup_{n\in\mathbb{N}}\tilde{\mathcal{V}}_{n}$. Moreover, by the
            construction given, $\tilde{\mathcal{U}}_{n}$ and
            $\tilde{\mathcal{V}}_{m}$ are disjoint for all
            $m,n\in\mathbb{N}$. But then
            $\bigcup_{n\in\mathbb{N}}\tilde{\mathcal{U}_{n}}$ and
            $\bigcup_{n\in\mathbb{N}}\tilde{\mathcal{V}_{n}}$ are disjoint
            open sets that cover $\mathcal{D}$ and $\mathcal{C}$, respectively,
            so $(X,\,\tau)$ is normal. But since $(X,\,\tau)$ is also a
            $G_{\delta}$ space, we conclude that $(X,\,\tau)$ is perfectly
            normal.
        \end{proof}
        \begin{theorem}[\textbf{The Nagata-Smirnov Metrization Theorem}]
            If $(X,\,\tau)$ is regular, Hausdorff, and has a $\sigma$ locally
            finite basis $\mathcal{B}$, then it is metrizable.
        \end{theorem}
        \begin{proof}
            
        \end{proof}
\end{document}
