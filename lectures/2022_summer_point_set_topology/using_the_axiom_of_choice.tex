%-----------------------------------LICENSE------------------------------------%
%   This file is part of Mathematics-and-Physics.                              %
%                                                                              %
%   Mathematics-and-Physics is free software: you can redistribute it and/or   %
%   modify it under the terms of the GNU General Public License as             %
%   published by the Free Software Foundation, either version 3 of the         %
%   License, or (at your option) any later version.                            %
%                                                                              %
%   Mathematics-and-Physics is distributed in the hope that it will be useful, %
%   but WITHOUT ANY WARRANTY; without even the implied warranty of             %
%   MERCHANTABILITY or FITNESS FOR A PARTICULAR PURPOSE.  See the              %
%   GNU General Public License for more details.                               %
%                                                                              %
%   You should have received a copy of the GNU General Public License along    %
%   with Mathematics-and-Physics.  If not, see <https://www.gnu.org/licenses/>.%
%------------------------------------------------------------------------------%
\documentclass{article}
\usepackage{graphicx}                           % Needed for figures.
\usepackage{amsmath}                            % Needed for align.
\usepackage{amssymb}                            % Needed for mathbb.
\usepackage{amsthm}                             % For the theorem environment.
\usepackage{float}
\usepackage{hyperref}
\hypersetup{
    colorlinks=true,
    linkcolor=blue,
    filecolor=magenta,
    urlcolor=Cerulean,
    citecolor=SkyBlue
}

%------------------------Theorem Styles-------------------------%

% Define theorem style for default spacing and normal font.
\newtheoremstyle{normal}
    {\topsep}               % Amount of space above the theorem.
    {\topsep}               % Amount of space below the theorem.
    {}                      % Font used for body of theorem.
    {}                      % Measure of space to indent.
    {\bfseries}             % Font of the header of the theorem.
    {}                      % Punctuation between head and body.
    {.5em}                  % Space after theorem head.
    {}

% Define default environments.
\theoremstyle{normal}
\newtheorem{problem}{Problem}

\title{The Axiom of Choice}
\author{Ryan Maguire}
\date{Summer 2022}

% No indent and no paragraph skip.
\setlength{\parindent}{0em}
\setlength{\parskip}{0em}

\begin{document}
    \maketitle
    Throughout our discussion of metric spaces we've repeatedly used the axiom
    of choice, but neglected mentioning it. Let's review Lebesgue's number
    lemma. We wish to show that for any open cover $\mathcal{O}$ of a compact
    metric space $(X,\,d)$ there is a $\delta>0$ such that for all $x\in{X}$
    there is a $\mathcal{U}\in\mathcal{O}$ such that
    $B_{\delta}^{(X,\,d)}(x)\subseteq\mathcal{U}$. We say \textit{suppose not}.
    We are negating our claim. Using quantifier notation, our claim says:
    \begin{equation}
        \forall_{\mathcal{O}\subseteq\tau_{d},\bigcup\mathcal{O}=X}
            \exists_{\delta>0}\forall_{x\in{X}}
            \exists_{\mathcal{U}\in\mathcal{O}}
            \Big(B_{\delta}^{(X,\,d)}(x)\subseteq\mathcal{U}\Big)
    \end{equation}
    the negation of this swaps $\forall$ with $\exists$ and vice-versa.
    \begin{equation}
        \exists_{\mathcal{O}\subseteq\tau_{d},\bigcup\mathcal{O}=X}
            \forall_{\delta>0}\exists_{x\in{X}}
            \forall_{\mathcal{U}\in\mathcal{O}}
            \Big(B_{\delta}^{(X,\,d)}(x)\nsubseteq\mathcal{U}\Big)
    \end{equation}
    That is, there exists an open cover $\mathcal{O}$ of $X$ such that
    for all $\delta>0$ there is an $x\in{X}$ with the property that for every
    $\mathcal{U}\in\mathcal{O}$ the $\delta$ ball centered at $x$ is not
    entirely contained in $\mathcal{U}$. In particular, since the negation
    says this is true \textit{for all} $\delta>0$, it is true for
    $\delta=1$. It is also true for $\delta=\frac{1}{2}$ and
    $\delta=\frac{1}{3}$. We define (using the axiom schema of specification)
    the set $A_{n}$ to be:
    \begin{equation}
        A_{n}=\Big\{\,x\in{X}\;|\;\forall_{\mathcal{U}\in\mathcal{O}}
            \Big(
                B_{\frac{1}{n+1}}^{(X,\,d)}(x)\nsubseteq\mathcal{U}
            \Big)\,
        \Big\}
    \end{equation}
    $A_{n}$ is not empty for all $n\in\mathbb{N}$ since we are assuming the
    negation of the orignal claim is true. That is, we are assuming for all
    $\delta>0$ there is an $x\in{X}$ such that no $\mathcal{U}$ in the open
    cover $\mathcal{O}$ completely contains the $\delta$ ball centered at $x$.
    Applying this to $\delta=\frac{1}{n+1}$ we then see that $A_{n}$ is a
    non-empty set for all $n\in\mathbb{N}$. We then (using the axiom of the
    power set and the axiom schema of specification) collect a new set
    $\mathcal{A}$ defined by:
    \begin{equation}
        \mathcal{A}=\{\,A_{n}\in\mathcal{P}(X)\;|\;n\in\mathbb{N}\,\}
    \end{equation}
    To be extremely pedantic and using the axioms precisely as stated, we
    are defining $\mathcal{A}$ by:
    \begin{equation}
        \mathcal{A}=\bigg\{\,B\in\mathcal{P}(X)\;\Big|\;
            \exists_{n\in\mathbb{N}}\bigg(
                \big(x\in{B}\big)\Leftrightarrow\Big(
                    \forall_{\mathcal{U}\in\mathcal{O}}
                    \big(B_{\frac{1}{n+1}}^{(X,\,d)}(x)\nsubseteq\mathcal{U}\big)
                \Big)
            \bigg)\,
        \bigg\}
    \end{equation}
    This reads, cryptically, that $\mathcal{A}$ is the set of all
    $B$ such that $B=A_{n}$ for some $n\in\mathbb{N}$.
    We then we examine the product set. Since the elements of
    $\mathcal{A}$ are indexed by the natural numbers, we may write:
    \begin{equation}
        \prod\mathcal{A}=\prod_{n=0}^{\infty}A_{n}
            =\big\{\,f:\mathbb{N}\rightarrow\bigcup_{n=0}^{\infty}A_{n}\;|\;
                \forall_{n\in\mathbb{N}}\big(f(n)\in{A}_{n}\big)\,\big\}
    \end{equation}
    Intuitively, an element $f\in\prod\mathcal{A}$ is a
    \textit{countably infinite tuple}:
    \begin{equation}
        f=(a_{0},\,a_{1},\,\dots,\,a_{n},\,\dots)
    \end{equation}
    with the property that
    $a_{n}\in{A}_{n}$ for all $n\in\mathbb{N}$. This is for intuition, not
    rigor. Now we note that $A_{n}$ is non-empty for all $n\in\mathbb{N}$.
    By the axiom of countable choice (which is implied by the full axiom of
    choice), there is an element $a\in\prod\mathcal{A}$. What is this element?
    It is a sequence $a:\mathbb{N}\rightarrow\bigcup_{n=0}^{\infty}A_{n}$
    such that for all $n\in\mathbb{N}$ it is true that $a_{n}\in{A}_{n}$.
    Realizing that $A_{n}\subseteq{X}$ for all $n\in\mathbb{N}$, we see that
    $\bigcup_{n=0}^{\infty}A_{n}\subseteq{X}$, and so $a$ is also a sequence
    $a:\mathbb{N}\rightarrow{X}$ such that $a_{n}\in{A}_{n}$ for all
    $n\in\mathbb{N}$. But what does it mean to be in $A_{n}$? We now go back
    to Lebesgue's number lemma, $a_{n}$ in $A_{n}$ means for every open set
    $\mathcal{U}$ in the open cover $\mathcal{O}$ it is not true that
    the $\frac{1}{n+1}$ ball centered about $a_{n}$ is contained in
    $\mathcal{U}$. That is, $a:\mathbb{N}\rightarrow{X}$ is a sequence such
    that, for all $n\in\mathbb{N}$, and for all $\mathcal{U}\in\mathcal{O}$, we
    have:
    \begin{equation}
        B_{\frac{1}{n+1}}^{(X,\,d)}(a_{n})\nsubseteq\mathcal{U}
    \end{equation}
    In the proof of Lebesgue's number lemma this was entirely swept under the
    rug. \textit{Obviously} I can pick a point $a_{n}$ for each $n$ satisfying
    the negation of our original claim. But just because something is
    \textit{obvious}, does not make it true. The axiom of countable choice can
    not be proven using the other axioms of set theory. Defining sequences
    recursively requires some form of the axiom of choice as well. Since the
    axiom of choice is confusing and controversial it is all too common to
    appeal to more intuitive language that \textit{hides} the axiom of choice,
    but it is important for a mathematician to realize that it is indeed there.
    \par\hfill\par
    We have appealed to the full axiom of choice as well. For the most part we
    were satisfied with the axiom of countable choice, which is a far less
    contraversial axiom, but at times we have used the stronger version.
    Consider the equivalence of compactness theorem. We said consider a Cauchy
    sequence $a:\mathbb{N}\rightarrow{X}$ that does not converge in the metric
    space $(X,\,d)$. To converge means there is an $x\in{X}$ such that for all
    $\varepsilon>0$ there is an $N\in\mathbb{N}$ such that $n\in\mathbb{N}$ and
    $n>N$ implies $d(x,\,a_{n})<\varepsilon$. We denoted this by
    $a_{n}\rightarrow{x}$. Using quantifer notation, convergence means:
    \begin{equation}
        \exists_{x\in{X}}\forall_{\varepsilon>0}
            \exists_{N\in\mathbb{N}}\forall_{n\in\mathbb{N},n>N}
            \big(d(x,\,a_{n})<\varepsilon\big)
    \end{equation}
    The negation of this means for all $x\in{X}$ there is an
    $\varepsilon>0$ such that for all $N\in\mathbb{N}$ there exists an
    $n\in\mathbb{N}$ with $n>N$ such that
    $d(x,\,a_{n})\geq\varepsilon$. That is:
    \begin{equation}
        \forall_{x\in{X}}\exists_{\varepsilon>0}\forall_{N\in\mathbb{N}}
            \exists_{n\in\mathbb{N},n>N}\big(d(x,\,a_{n})\geq\varepsilon\big)
    \end{equation}
    I \textit{conveniently} labelled such an $\varepsilon$ as
    $\varepsilon_{x}$ to denote that $\varepsilon_{x}$ is a positive value that
    makes the property fail for the point $x\in{X}$. Why am I allowed to do
    this? I define the set $A_{x}\subseteq\mathbb{R}$ as follows:
    \begin{equation}
        A_{x}=\big\{\,\varepsilon\in\mathbb{R}^{+}\;|\;
            \forall_{N\in\mathbb{N}}
            \exists_{n\in\mathbb{N},n>N}\big(d(x,\,a_{n})\geq\varepsilon\big)
        \,\big\}
    \end{equation}
    Since we are assuming the sequence does \textit{not} converge, for all
    $x\in{X}$, the set $A_{x}$ is non-empty. We then consider the collection
    of all of these sets $\mathcal{A}$:
    \begin{equation}
        \mathcal{A}=\{\,A_{x}\in\mathcal{P}(\mathbb{R}^{+})\;|\;x\in{X}\,\}
    \end{equation}
    Again, being overly formal, we are writing:
    \begin{equation}
        \mathcal{A}=\Big\{\,B\in\mathcal{P}(\mathbb{R}^{+})\;\big|\;
            \exists_{x\in{X}}\Big(\big(\varepsilon\in{B}\big)\Leftrightarrow
            \forall_{N\in\mathbb{N}}
            \exists_{n\in\mathbb{N},n>N}\big(d(x,\,a_{n})\geq\varepsilon\big)
            \Big)
        \,\Big\}
    \end{equation}
    This is the set of all $B\subseteq\mathbb{R}^{+}$ such that
    $B=A_{x}$ for some $x\in{X}$. We consider the product set. This time,
    unlike in the previous example, since we don't know the cardinality of
    $X$, we may not be able to index the product set over the natural numbers.
    We may still write:
    \begin{equation}
        \prod\mathcal{A}=\prod_{x\in{X}}A_{x}
            =\big\{\,f:X\rightarrow\bigcup_{x\in{X}}A_{x}\;|\;
                \forall_{x\in{X}}\big(f(x)\in{A}_{x}\big)\,\big\}
    \end{equation}
    Since $A_{x}\subseteq\mathbb{R}^{+}$ for all $x\in{X}$, an element
    of $\prod\mathcal{A}$ is a function $f:X\rightarrow\mathbb{R}^{+}$.
    That is, a function from our metric space to the positive real numbers.
    This function $f$ has the special property that for all $x\in{X}$,
    $f(x)$ is an element of $A_{x}$. That is, $f(x)$ is a positive
    number $\varepsilon_{x}=f(x)$ such that for all $N\in\mathbb{N}$ there is
    an $n\in\mathbb{N}$ with $n>N$ such that
    $d(x,\,a_{n})\geq\varepsilon_{x}$. Since none of the elements of
    $\mathcal{A}$ are empty, by the axiom of choice the product is non-empty.
    Let $\varepsilon\in\prod\mathcal{A}$ be an element of the product. That is,
    let $\varepsilon:X\rightarrow\bigcup_{x\in{X}}A_{x}$ be our
    \textit{choice} function. Given $x\in{X}$, instead of writing the image of
    $x$ as $\varepsilon(x)$, let us write it as $\varepsilon_{x}$. Then
    $\varepsilon_{x}$ is a value such that for all $N\in\mathbb{N}$ there is
    an $n\in\mathbb{N}$ with $n>N$ such that
    $d(x,\,a_{n})\geq\varepsilon_{x}$, and now we're back to where we started
    in the proof of the equivalence of compactness. The axiom of choice
    justifies the notation $\varepsilon_{x}$ where we \textit{choose} a value
    $\varepsilon_{x}>0$ for each $x$.
\end{document}
