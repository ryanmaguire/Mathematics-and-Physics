%-----------------------------------LICENSE------------------------------------%
%   This file is part of Mathematics-and-Physics.                              %
%                                                                              %
%   Mathematics-and-Physics is free software: you can redistribute it and/or   %
%   modify it under the terms of the GNU General Public License as             %
%   published by the Free Software Foundation, either version 3 of the         %
%   License, or (at your option) any later version.                            %
%                                                                              %
%   Mathematics-and-Physics is distributed in the hope that it will be useful, %
%   but WITHOUT ANY WARRANTY; without even the implied warranty of             %
%   MERCHANTABILITY or FITNESS FOR A PARTICULAR PURPOSE.  See the              %
%   GNU General Public License for more details.                               %
%                                                                              %
%   You should have received a copy of the GNU General Public License along    %
%   with Mathematics-and-Physics.  If not, see <https://www.gnu.org/licenses/>.%
%------------------------------------------------------------------------------%
\documentclass{article}
\usepackage{graphicx}                           % Needed for figures.
\usepackage{amsmath}                            % Needed for align.
\usepackage{amssymb}                            % Needed for mathbb.
\usepackage{amsthm}                             % For the theorem environment.
\usepackage{float}
\usepackage{hyperref}
\hypersetup{
    colorlinks=true,
    linkcolor=blue,
    filecolor=magenta,
    urlcolor=Cerulean,
    citecolor=SkyBlue
}

%------------------------Theorem Styles-------------------------%
\theoremstyle{plain}
\newtheorem{theorem}{Theorem}[section]

% Define theorem style for default spacing and normal font.
\newtheoremstyle{normal}
    {\topsep}               % Amount of space above the theorem.
    {\topsep}               % Amount of space below the theorem.
    {}                      % Font used for body of theorem.
    {}                      % Measure of space to indent.
    {\bfseries}             % Font of the header of the theorem.
    {}                      % Punctuation between head and body.
    {.5em}                  % Space after theorem head.
    {}

% Define default environments.
\theoremstyle{normal}
\newtheorem{examplex}{Example}[section]
\newtheorem{definitionx}{Definition}[section]
\newtheorem{notationx}{Notation}[section]
\newtheorem{axiomx}{Axiom}[section]

\newenvironment{example}{%
    \pushQED{\qed}\renewcommand{\qedsymbol}{$\blacksquare$}\examplex%
}{%
    \popQED\endexamplex%
}

\newenvironment{definition}{%
    \pushQED{\qed}\renewcommand{\qedsymbol}{$\blacksquare$}\definitionx%
}{%
    \popQED\enddefinitionx%
}

\title{Point-Set Topology: Lecture 8}
\author{Ryan Maguire}
\date{Summer 2022}

% No indent and no paragraph skip.
\setlength{\parindent}{0em}
\setlength{\parskip}{0em}

\begin{document}
    \maketitle
    \section{Theorems on Compactness}
        \begin{theorem}
            If $(X,\,d)$ is a metric space and
            $a:\mathbb{N}\rightarrow\mathbb{R}$ is a convergent sequence,
            then $a$ is a Cauchy sequence.
        \end{theorem}
        \begin{proof}
            Since $a:\mathbb{N}\rightarrow{X}$ converges, there is an
            $x\in{X}$ such that $a_{n}\rightarrow{x}$. Let
            $\varepsilon>0$. Since $a_{n}\rightarrow{x}$ there is an
            $N\in\mathbb{N}$ such that $n>N$ implies
            $d(x,\,a_{n})<\frac{\varepsilon}{2}$. But then $n,m>N$ implies:
            \begin{equation}
                d(a_{m},\,a_{n})\leq{d}(x,\,a_{m})+d(x,\,a_{n})
                    <\frac{\varepsilon}{2}+\frac{\varepsilon}{2}
                    =\varepsilon
            \end{equation}
            and therefore $a$ is a Cauchy sequence.
        \end{proof}
        Without completeness, a metric space $(X,\,d)$ can have non-convergent
        Cauchy sequences. But given a Cauchy sequence with a convergent
        subsequence, the entire sequence must then convergent. The intuition is
        that a Cauchy sequence is a sequence where all of the points start to
        get really close together as the indices increase. Since there is a
        \begin{theorem}
            If $(X,\,d)$ is a metric space, if $a:\mathbb{N}\rightarrow{X}$
            is a Cauchy sequence, and if $a_{k}$ is a convergent subsequence,
            then $a$ is a convergent sequence.
        \end{theorem}
        \begin{proof}
            Since $a_{k}$ is a convergent sequence, there is an
            $x\in{X}$ such that $a_{k_{n}}\rightarrow{x}$. Let
            $\varepsilon>0$. Since $a_{k_{n}}\rightarrow{x}$, there is an
            $N_{0}\in\mathbb{N}$ such that $n>N_{0}$ implies
            $d(x,\,a_{k_{n}})<\frac{\varepsilon}{2}$. Since $a$ is a Cauchy
            sequence there is an $N_{1}\in\mathbb{N}$ such that
            $n,m>N_{1}$ implies $d(a_{m},\,a_{n})<\frac{\varepsilon}{2}$.
            Let $N=\textrm{max}(k_{N_{0}},\,N_{1})$.
            Then since $k$ is strictly increasing,
            $m>N$ implies $k_{m}>N_{0}$ and $k_{m}>N_{1}$.
            But then for all $n,m>N$:
            \begin{equation}
                d(x,\,a_{n})\leq{d}(a_{n},\,a_{k_{m}})+d(x,\,a_{m})
                    <\frac{\varepsilon}{2}+\frac{\varepsilon}{2}=\varepsilon
            \end{equation}
            and therefore $a_{n}\rightarrow{x}$. That is, $a$ is a convergent
            sequence.
        \end{proof}
        Completeness and closedness are related for metric space. Given a
        complete metric space $(X,\,d)$, the only complete subspaces are the
        closed ones. In particular, since $(\mathbb{R},\,|\cdot|)$, the standard
        metric on the real line, is complete, we see that the open unit interval
        is \textit{not} complete. In particular, the sequence
        $a:\mathbb{N}\rightarrow(0,\,1)$ defined by $a_{n}=\frac{1}{n+1}$
        is a Cauchy sequence, but it does not converge. We want to say it
        ``converges'' to zero, but zero is not an element of this subspace.
        \begin{theorem}
            If $(X,\,d)$ is a complete metric space, and if
            $A\subseteq{X}$, then $(A,\,d_{A})$ is a complete metric space if
            and only if $A$ is closed.
        \end{theorem}
        \begin{proof}
            Suppose $(A,\,d_{A})$ is a complete metric space and let
            $a:\mathbb{N}\rightarrow{A}$ be a sequence that converges in $X$.
            But convergent sequences are Cauchy sequences, and $(A,\,d_{A})$ is
            complete, and therefore Cauchy sequences converge. But then the
            limit of $a$ is contained in $A$, and therefore $A$ is closed.
            Now, suppose $A\subseteq{X}$ is closed. Let
            $a:\mathbb{N}\rightarrow{A}$ be a Cauchy sequence. Then, since
            $A\subseteq{X}$, $a:\mathbb{N}\rightarrow{X}$ is a Cauchy sequence
            in $X$. But $(X,\,d)$ is complete, and therefore $a$ converges.
            That is, there is an $x\in{X}$ such that $a_{n}\rightarrow{x}$.
            But $A$ is closed and therefore contains all of its limit points,
            so $x\in{A}$. Hence Cauchy sequences in $A$ converge in $A$, and
            therefore $(A,\,d_{A})$ is complete.
        \end{proof}
        This theorem is the baby version of the same idea for compactness.
        \begin{theorem}
            If $(X,\,d)$ is a compact metric space, and $A\subseteq{X}$, then
            $(A,\,d_{A})$ is compact if and only if $A$ is closed.
        \end{theorem}
        \begin{proof}
            Suppose $(A,\,d_{A})$ is compact and $x\in{X}$ a limit point of
            $A$. Then there is a sequence $a:\mathbb{N}\rightarrow{A}$ such
            that $a_{n}$ converges to $x$ in $X$. But $(A,\,d_{A})$ is compact,
            so there is a convergent subsequence $a_{k}$ with limit in $A$.
            But limits are unique, so $a_{k_{n}}\rightarrow{x}$, and therefore
            $x\in{A}$. Now suppose $A$ is closed. Let
            $a:\mathbb{N}\rightarrow{A}$ be a sequence. Then, since
            $A\subseteq{X}$, $a:\mathbb{N}\rightarrow{X}$ is a sequence in $X$.
            But $(X,\,d)$ is compact, so there is a convergent subsequence
            $a_{k}$ with limit $x\in{X}$. But $A$ is closed and hence contains
            all of its limit points, and therefore $x\in{A}$. But then
            $a_{k}$ is a convergent subsequence of $a$ in $A$. Therefore,
            $(A,\,d_{A})$ is compact.
        \end{proof}
        And lastly, it should be noted that compactness is far stronger than
        completeness. Let's prove this.
        \begin{theorem}
            If $(X,\,d)$ is a compact metric space, then $(X,\,d)$ is complete.
        \end{theorem}
        \begin{proof}
            Let $a:\mathbb{N}\rightarrow{X}$ be a Cauchy sequence. Since
            $(X,\,d)$ is compact, there is a convergent subsequence
            $a_{k}$. But a Cauchy sequence with a convergent subsequence is
            convergent, and therefore $a$ is convergent. Thus, $(X,\,d)$ is
            complete.
        \end{proof}
        \begin{theorem}[\textbf{Heine-Borel Theorem}]
            If $A\subseteq\mathbb{R}^{N}$ and
            $d:\mathbb{R}^{N}\times\mathbb{R}^{N}\rightarrow\mathbb{R}$ is the
            standard Euclidean metric,
            $d(\mathbf{x},\,\mathbf{y})=||\mathbf{x}-\mathbf{y}||_{2}$, then
            $(A,\,d_{A})$ is compact if and only if $A$ is closed and bounded.
        \end{theorem}
        \begin{proof}
            Suppose $(A,\,d_{A})$ is compact. We have proved that compact
            subspaces of any metric space are closed, so in particular
            $A$ is a closed subset of $\mathbb{R}^{N}$. Suppose $A$ is not
            bounded. Then for all $n\in\mathbb{N}$ there is an
            $a_{n}\in{A}$ with $||a_{n}||_{2}>n$, otherwise
            $A$ is bounded. The sequence $a:\mathbb{N}\rightarrow{A}$ can be
            chosen so that $||a_{m}||_{2}<||a_{n}||_{2}$ whenever $m<n$, while
            diverging off to infinity, and hence contains no
            convergent subsequences. This contradicts the assumption that
            $(A,\,d_{A})$ is compact. Hence, $A$ is bounded. Now, suppose
            $(A,\,d_{A})$ is closed and bounded. Let
            $\mathbf{x}:\mathbb{N}\rightarrow{A}$ be any sequence. Denote
            $\mathbf{x}_{n}\in{A}$ via the tuple:
            \begin{equation}
                \mathbf{x}_{n}=\big(x_{n}^{0},\,x_{n}^{1},\,\dots,\,
                    x_{n}^{N-1}\big)
            \end{equation}
            The sequence $x^{0}:\mathbb{N}\rightarrow\mathbb{R}$ defined by
            setting $x_{n}^{0}$ equal to the zeroth component of
            $\mathbf{x}_{n}$ is bounded since $A$ is bounded. By
            the Bolzano-Weierstrass theorem there is a
            convergent subsequence $x_{k}^{0}$. That is, there is some
            real number $r_{0}\in\mathbb{N}$ such that
            $x_{k_{n}}^{0}\rightarrow{r}_{0}$. But then
            $x_{k}^{1}$ is a (not necessarily convergent) subsequence of the
            sequence $x^{1}:\mathbb{N}\rightarrow\mathbb{R}$, the sequence
            defined by setting $x_{n}^{1}$ equal to the first component of
            $\mathbf{x}_{n}$. But then $x_{k}^{1}$ is a bounded sequence of
            real numbers and hence by the Bolzano-Weierstrass theorem, there
            is a convergen subsequence $x_{k_{k'}}^{1}$. But
            $x_{k_{k'}}^{0}$ is a subsequence of $x_{k}^{0}$, and hence
            $x_{k_{k'}}^{0}$ is a subsequence of a convergent subsequence.
            But subsequences of convergent sequences converge and they converge
            to the same limit. That is, we now have that
            $x_{k_{k'}}^{0}$ and $x_{k_{k'}}^{1}$ are convergent sequences.
            Continuing inductively, we obtain subseqence
            $k''$, $k'''$, up to $k^{N-1}$ such that
            $x_{k\dots{k}^{N-1}}^{m}$ is a convergent sequence for all
            $m\in\mathbb{Z}_{N}$. But then
            $\mathbf{x}_{k\dots{k}^{N-1}}$ is a sequence in $A$ that converges
            to some $\mathbf{y}\in\mathbb{R}^{N}$. But
            $A\subseteq\mathbb{R}^{N}$ is closed and hence contains all of its
            limit points, so $\mathbf{y}\in{A}$. That is, $\mathbf{x}$ has a
            convergent subsequence. Therefore $(A,\,d_{A})$ is compact.
        \end{proof}
        Do not attempt to apply this result to general metric spaces. The
        Heine-Borel theorem is specific to Euclidean spaces with the Euclidean
        metric (or topologically equivalent metrics such as the Manhattan and
        maximum metrics). The discrete metric on any set is bounded. In
        particular, the discrete metric on $\mathbb{R}$ is bounded. Moreover,
        since every subset of a discrete metric space is open, every subset of
        a discrete metric space is closed (being the complement of an open set).
        Any infinite subset of $\mathbb{R}$ with the discrete metric is
        \textit{not} compact. In the discrete metric space $(X,\,d)$ a sequence
        $a:\mathbb{N}\rightarrow{X}$ converges to $x\in{X}$ if and only if
        there is some $N\in\mathbb{N}$ such that $n>N$ implies
        $a_{n}=x$. That is, convergent sequences are eventually constant.
        (To see this, apply the definition of convergence to the positive
        number $\varepsilon=\frac{1}{2}$). So given an infinite subset $A$ of
        $\mathbb{R}$ we can find an injective sequence
        $a:\mathbb{N}\rightarrow{A}$. $A$ is indeed closed and bounded, but
        $a$ has no convergent subsequence. So $A$ is not compact with the
        discrete metric.
        \par\hfill\par
        The generalization of the Heine-Borel theorem for general metric spaces
        needs stronger notions than just \textit{closed} and \textit{bounded}.
        If we replace \textit{closed} with \textit{complete} and
        \textit{bounded} with \textit{totally bounded}, we get the
        generalized Heine-Borel theorem.
        \begin{theorem}[\textbf{Generalized Heine-Borel Theorem}]
            If $(X,\,d)$ is a metric space, then it is compact if and only
            if it is complete and totally bounded.
        \end{theorem}
        \begin{proof}
            We have already proven that compact implies complete. Now let's
            show compact implies totally bounded. Suppose not. Then there is
            an $\varepsilon>0$ such that no matter what finite collection of
            points $a_{0},\,\dots,\,a_{n}$ you pick, there is another point
            $a_{n+1}$ where $a_{n+1}\notin{B}_{\varepsilon}^{(X,\,d)}(a_{k})$
            for all $0\leq{k}\leq{n}$. Inductively this defines a sequence
            $a:\mathbb{N}\rightarrow{X}$. But $(X,\,d)$ is compact, so there
            is a convergent subsequence $a_{k}$. But convergent sequences are
            Cauchy sequences, and therefore there there is an
            $N\in\mathbb{N}$ such that $m,n>N$ implies
            $d(a_{k_{n}},\,a_{k_{m}})<\varepsilon$. But by construction
            $d(a_{k_{n}},\,a_{k_{m}})\geq\varepsilon$ for all
            $n,m\in\mathbb{N}$, a contradiction. So $(X,\,d)$ is complete.
            Now, suppose $(X,\,d)$ is closed and totally bounded. Let
            $a:\mathbb{N}\rightarrow{X}$ be any sequence. Since
            $(X,\,d)$ is totally bounded, there are finitely many points
            $b_{0},\,\dots,\,b_{N}$ such that the open balls
            $B_{1}^{(X,\,d)}(b_{k})$ completely cover $X$. Since there are
            infinitely integers and only finitely many open balls, there
            must be a point $b_{k}$ such that infinitely many $n\in\mathbb{N}$
            are such that $a_{n}\in{B}_{1}^{(X,\,d)}(b_{k})$. Let
            $k_{0}\in\mathbb{N}$ be such a value
            with $a_{k_{0}}\in{B}_{0}^{(X,\,d)}(b_{k})$. Again by total
            boundedness, we can cover $B_{1}^{(X,\,d)}(b_{k})$ with finitely
            many open balls $B_{\frac{1}{2}}^{(X,\,d)}(c_{m})$ with points
            $c_{m}\in{B}_{1}^{(X,\,d)}(b_{k})$. Since infinitely many integers
            $n\in\mathbb{N}$ are such that $a_{n}\in{B}_{1}^{(X,\,d)}(b_{k})$
            and there are only finitely many balls covering this set,
            one of these open balls must again be such that there are
            infinitely many integers $n\in\mathbb{N}$ with
            $a_{n}\in{B}_{\frac{1}{2}}^{(X,\,d)}(c_{\ell})$. Let
            $k_{1}$ be such that $k_{1}>k_{0}$ and
            $a_{k_{1}}\in{B}_{\frac{1}{2}}^{(X,\,d)}(c_{\ell})$. Inductively
            we get a sequence $a_{k}$. This sequence is Cauchy since
            $d(a_{k_{m}},\,a_{k_{n}})$ is bounded by $\frac{1}{N+1}$ where
            $N=\textrm{min}(m,\,n)$. But $(X,\,d)$ is complete, so this sequence
            converges. Hence $a$ has a convergent subsequence.
        \end{proof}
        \begin{definition}[\textbf{Open Cover of a Metric Space}]
            An open cover of a metric space $(X,\,d)$ is a subset
            $\mathcal{O}\subseteq\tau_{d}$, where $\tau_{d}$ is the metric
            topology, such that $\bigcup\mathcal{O}=X$. That is, for all
            $x\in{X}$, there is an open set $\mathcal{U}\in\mathcal{O}$ such
            that $x\in\mathcal{U}$.
        \end{definition}
        \begin{theorem}[\textbf{Lebesgue's Number Lemma}]
            If $(X,\,d)$ is compact, and if $\mathcal{O}$ is an open cover of
            $(X,\,d)$, then there is a $\delta>0$ such that for all
            $x\in{X}$ there is a $\mathcal{U}\in\mathcal{O}$ such that
            $B_{\delta}^{(X,\,d)}(x)\subseteq\mathcal{U}$.
        \end{theorem}
        \begin{proof}
            Suppose not. Then for all $n\in\mathbb{N}$ there is an
            $a_{n}$ such that $B_{\frac{1}{n+1}}^{(X,\,d)}(a_{n})$ is not
            contained entirely inside of $\mathcal{U}$ for any
            $\mathcal{U}\in\mathcal{O}$. But $(X,\,d)$ is compact, so there
            is a convergent subsequence $a_{k}$. Let $x\in{X}$ be the limit,
            $a_{k_{n}}\rightarrow{x}$. Since $\mathcal{O}$ is an open cover,
            there is a $\mathcal{U}$ with $x\in\mathcal{U}$. But $\mathcal{U}$
            is open, so there is an $\varepsilon>0$ such that
            $B_{\varepsilon}^{(X,\,d)}(x)\subseteq\mathcal{U}$. But since
            $a_{k_{n}}\rightarrow{x}$ there is an $N_{0}\in\mathbb{N}$ such that
            $k_{n}>N$ implies $d(x,\,a_{k_{n}})<\frac{\varepsilon}{2}$. Let
            $N_{1}\in\mathbb{N}$ be such that $N+1>\frac{2}\varepsilon$ and
            let $N=\textrm{max}(N_{0},\,N_{1})$. Then
            $y\in{B}_{\frac{1}{N+1}}^{(X,\,d)}(a_{N})$ implies:
            \begin{equation}
                d(x,\,y)\leq{d}(x,\,a_{N})+d(a_{N},\,y)
                    <\frac{\varepsilon}{2}+\frac{\varepsilon}{2}=\varepsilon
            \end{equation}
            and hence $y\in{B}_{\varepsilon}^{(X,\,d)}(x)$. But then
            $B_{\frac{1}{N+1}}^{(X,\,d)}(a_{N})\subseteq\mathcal{U}$, which is
            a contradiction, completing the proof.
        \end{proof}
        \begin{theorem}[\textbf{The Equivalence of Compactness Theorem}]
            If $(X,\,d)$ is a metric space, then $(X,\,d)$ is compact if and
            only if for every open cover $\mathcal{O}$ of $(X,\,d)$ there exists
            a finite open cover $\Delta\subseteq\mathcal{O}$.
        \end{theorem}
        \begin{proof}
            If $(X,\,d)$ is compact and $\mathcal{O}$ is an open cover, then
            by the Lebesgue number lemma there is a $\delta>0$ such that for
            all $x\in{X}$ there is a $\mathcal{U}_{x}$ such that
            $B_{\delta}^{(X,\,d)}(x)\subseteq\mathcal{U}_{x}$. Since
            $(X,\,d)$ is compact, it is totally bounded, and hence there are
            finitely many points $x_{0},\,\dots,\,x_{N}$ such that the
            $\delta$ balls centered at $x_{n}$ cover $X$. But then the set:
            \begin{equation}
                \Delta=\{\,\mathcal{U}_{x_{0}},\,\dots,\,\mathcal{U}_{n_{N}}\,\}
            \end{equation}
            is a finite open cover that is a subset of $\mathcal{O}$. In the
            other direction, suppose $(X,\,d)$ is such that every open cover
            $\mathcal{O}$ contains a finite subset $\Delta\subseteq\mathcal{O}$
            that is also an open cover of $(X,\,d)$. Given $\varepsilon>0$,
            create the set $\mathcal{O}$ by:
            \begin{equation}
                \mathcal{O}=\{\,B_{\varepsilon}^{(X,\,d)}(x)\;|\;x\in{X}\,\}
            \end{equation}
            $\mathcal{O}$ is an open cover of $(X,\,d)$, and hence there is a
            finite open cover $\Delta\subseteq\mathcal{O}$. But then $(X,\,d)$
            can be covered by finitely many balls of radius $\varepsilon$, so
            $(X,\,d)$ is totally bounded. Next, suppose $(X,\,d)$ is not
            complete. There there is a Cauchy sequence
            $a:\mathbb{N}\rightarrow{X}$ that does not converge. Then for all
            $x\in{X}$ there is a $\varepsilon_{x}>0$ such that for all
            $N\in\mathbb{N}$ there exists $n>N$ with
            $d(x,\,a_{n})\geq\varepsilon$. Let $\mathcal{O}$ be defined by:
            \begin{equation}
                \mathcal{O}=\{\,B_{\varepsilon_{x}/2}^{(X,\,d)}(x)\;|\;x\in{X}\,\}
            \end{equation}
            $\mathcal{O}$ is an open cover of $(X,\,d)$, so there is a finite
            open cover $\Delta\subseteq\mathcal{O}$. Let
            $x_{0},\,\dots,x_{N}$ be the finite set of points such that:
            \begin{equation}
                X=\bigcup_{n=0}^{N}B_{\varepsilon_{n}}^{(X,\,d)}(x_{n})
            \end{equation}
            where $\varepsilon_{n}=\varepsilon_{x_{n}}$. Let
            $\varepsilon=\frac{1}{2}\textrm{min}(\varepsilon_{0},\,\dots,\,\varepsilon_{N})$.
            Since $a:\mathbb{N}\rightarrow{X}$ is a Cauchy sequence, there is
            an $N\in\mathbb{N}$ such that $n,m>N$ implies
            $d(a_{m},\,a_{n})<\varepsilon$. But then:
            \begin{equation}
                d(x,\,a_{n})\leq{d}(a_{n},\,a_{m})+d(x,\,a_{n})<\varepsilon
            \end{equation}
            a contradiction, so $(X,\,d)$ is complete. But then
            $(X,\,d)$ is complete and totally bounded, and therefore compact.
        \end{proof}
\end{document}
