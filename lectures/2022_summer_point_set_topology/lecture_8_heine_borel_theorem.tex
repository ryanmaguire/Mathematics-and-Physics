%-----------------------------------LICENSE------------------------------------%
%   This file is part of Mathematics-and-Physics.                              %
%                                                                              %
%   Mathematics-and-Physics is free software: you can redistribute it and/or   %
%   modify it under the terms of the GNU General Public License as             %
%   published by the Free Software Foundation, either version 3 of the         %
%   License, or (at your option) any later version.                            %
%                                                                              %
%   Mathematics-and-Physics is distributed in the hope that it will be useful, %
%   but WITHOUT ANY WARRANTY; without even the implied warranty of             %
%   MERCHANTABILITY or FITNESS FOR A PARTICULAR PURPOSE.  See the              %
%   GNU General Public License for more details.                               %
%                                                                              %
%   You should have received a copy of the GNU General Public License along    %
%   with Mathematics-and-Physics.  If not, see <https://www.gnu.org/licenses/>.%
%------------------------------------------------------------------------------%
\documentclass{article}
\usepackage{graphicx}                           % Needed for figures.
\usepackage{amsmath}                            % Needed for align.
\usepackage{amssymb}                            % Needed for mathbb.
\usepackage{amsthm}                             % For the theorem environment.
\usepackage{float}
\usepackage{hyperref}
\hypersetup{
    colorlinks=true,
    linkcolor=blue,
    filecolor=magenta,
    urlcolor=Cerulean,
    citecolor=SkyBlue
}

%------------------------Theorem Styles-------------------------%
\theoremstyle{plain}
\newtheorem{theorem}{Theorem}[section]

% Define theorem style for default spacing and normal font.
\newtheoremstyle{normal}
    {\topsep}               % Amount of space above the theorem.
    {\topsep}               % Amount of space below the theorem.
    {}                      % Font used for body of theorem.
    {}                      % Measure of space to indent.
    {\bfseries}             % Font of the header of the theorem.
    {}                      % Punctuation between head and body.
    {.5em}                  % Space after theorem head.
    {}

% Define default environments.
\theoremstyle{normal}
\newtheorem{examplex}{Example}[section]
\newtheorem{definitionx}{Definition}[section]
\newtheorem{notationx}{Notation}[section]
\newtheorem{axiomx}{Axiom}[section]

\newenvironment{example}{%
    \pushQED{\qed}\renewcommand{\qedsymbol}{$\blacksquare$}\examplex%
}{%
    \popQED\endexamplex%
}

\newenvironment{definition}{%
    \pushQED{\qed}\renewcommand{\qedsymbol}{$\blacksquare$}\definitionx%
}{%
    \popQED\enddefinitionx%
}

\title{Point-Set Topology: Lecture 8}
\author{Ryan Maguire}
\date{Summer 2022}

% No indent and no paragraph skip.
\setlength{\parindent}{0em}
\setlength{\parskip}{0em}

\begin{document}
    \maketitle
    \section{Theorems on Compactness}
        \begin{theorem}
            If $(X,\,d)$ is a metric space and
            $a:\mathbb{N}\rightarrow\mathbb{R}$ is a convergent sequence,
            then $a$ is a Cauchy sequence.
        \end{theorem}
        \begin{proof}
            Since $a:\mathbb{N}\rightarrow{X}$ converges, there is an
            $x\in{X}$ such that $a_{n}\rightarrow{x}$. Let
            $\varepsilon>0$. Since $a_{n}\rightarrow{x}$ there is an
            $N\in\mathbb{N}$ such that $n>N$ implies
            $d(x,\,a_{n})<\frac{\varepsilon}{2}$. But then $n,m>N$ implies:
            \begin{equation}
                d(a_{m},\,a_{n})\leq{d}(x,\,a_{m})+d(x,\,a_{n})
                    <\frac{\varepsilon}{2}+\frac{\varepsilon}{2}
                    =\varepsilon
            \end{equation}
            and therefore $a$ is a Cauchy sequence.
        \end{proof}
        Without completeness, a metric space $(X,\,d)$ can have non-convergent
        Cauchy sequences. But given a Cauchy sequence with a convergent
        subsequence, the entire sequence must then convergent. The intuition is
        that a Cauchy sequence is a sequence where all of the points start to
        get really close together as the indices increase. Since there is a
        \begin{theorem}
            If $(X,\,d)$ is a metric space, if $a:\mathbb{N}\rightarrow{X}$
            is a Cauchy sequence, and if $a_{k}$ is a convergent subsequence,
            then $a$ is a convergent sequence.
        \end{theorem}
        \begin{proof}
            Since $a_{k}$ is a convergent sequence, there is an
            $x\in{X}$ such that $a_{k_{n}}\rightarrow{x}$. Let
            $\varepsilon>0$. Since $a_{k_{n}}\rightarrow{x}$, there is an
            $N_{0}\in\mathbb{N}$ such that $n>N_{0}$ implies
            $d(x,\,a_{k_{n}})<\frac{\varepsilon}{2}$. Since $a$ is a Cauchy
            sequence there is an $N_{1}\in\mathbb{N}$ such that
            $n,m>N_{1}$ implies $d(a_{m},\,a_{n})<\frac{\varepsilon}{2}$.
            Let $N=\textrm{max}(k_{N_{0}},\,k_{N_{1}})$.
            Then since $k$ is strictly increasing,
            $m>N$ implies $k_{m}>N_{0}$. But then, since $a$ is a Cauchy
            sequence, we have for all $n,m>N$:
            \begin{equation}
                d(x,\,a_{n})\leq{d}(a_{n},\,a_{m})+d(x,\,a_{m})
                    <\frac{\varepsilon}{2}+\frac{\varepsilon}{2}=\varepsilon
            \end{equation}
            and therefore $a_{n}\rightarrow{x}$. That is, $a$ is a convergent
            sequence.
        \end{proof}
\end{document}
