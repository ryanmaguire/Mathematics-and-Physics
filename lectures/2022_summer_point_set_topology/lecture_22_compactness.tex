%-----------------------------------LICENSE------------------------------------%
%   This file is part of Mathematics-and-Physics.                              %
%                                                                              %
%   Mathematics-and-Physics is free software: you can redistribute it and/or   %
%   modify it under the terms of the GNU General Public License as             %
%   published by the Free Software Foundation, either version 3 of the         %
%   License, or (at your option) any later version.                            %
%                                                                              %
%   Mathematics-and-Physics is distributed in the hope that it will be useful, %
%   but WITHOUT ANY WARRANTY; without even the implied warranty of             %
%   MERCHANTABILITY or FITNESS FOR A PARTICULAR PURPOSE.  See the              %
%   GNU General Public License for more details.                               %
%                                                                              %
%   You should have received a copy of the GNU General Public License along    %
%   with Mathematics-and-Physics.  If not, see <https://www.gnu.org/licenses/>.%
%------------------------------------------------------------------------------%
\documentclass{article}
\usepackage{graphicx}                           % Needed for figures.
\usepackage{amsmath}                            % Needed for align.
\usepackage{amssymb}                            % Needed for mathbb.
\usepackage{amsthm}                             % For the theorem environment.
\usepackage{float}
\usepackage{tabularx, booktabs}
\usepackage[font=scriptsize,
            labelformat=simple,
            labelsep=colon]{subcaption} % Subfigure captions.
\usepackage[font={scriptsize},
            hypcap=true,
            labelsep=colon]{caption}    % Figure captions.
\usepackage{hyperref}
\hypersetup{
    colorlinks=true,
    linkcolor=blue,
    filecolor=magenta,
    urlcolor=Cerulean,
    citecolor=SkyBlue
}

%------------------------Theorem Styles-------------------------%
\theoremstyle{plain}
\newtheorem{theorem}{Theorem}[section]

% Define theorem style for default spacing and normal font.
\newtheoremstyle{normal}
    {\topsep}               % Amount of space above the theorem.
    {\topsep}               % Amount of space below the theorem.
    {}                      % Font used for body of theorem.
    {}                      % Measure of space to indent.
    {\bfseries}             % Font of the header of the theorem.
    {}                      % Punctuation between head and body.
    {.5em}                  % Space after theorem head.
    {}

% Define default environments.
\theoremstyle{normal}
\newtheorem{examplex}{Example}[section]
\newtheorem{definitionx}{Definition}[section]

\newenvironment{example}{%
    \pushQED{\qed}\renewcommand{\qedsymbol}{$\blacksquare$}\examplex%
}{%
    \popQED\endexamplex%
}

\newenvironment{definition}{%
    \pushQED{\qed}\renewcommand{\qedsymbol}{$\blacksquare$}\definitionx%
}{%
    \popQED\enddefinitionx%
}

\title{Point-Set Topology: Lecture 21}
\author{Ryan Maguire}
\date{Summer 2022}

% No indent and no paragraph skip.
\setlength{\parindent}{0em}
\setlength{\parskip}{0em}

\begin{document}
    \maketitle
    \section{Compactness}
        For a course in point-set topology, if you understand the general
        notions (topological spaces, continuity, Hausdorffness, sequentialness),
        the basis properties (first and second countable), creating new spaces
        (products, subspaces, quotients), the separation ideas
        (regular and normal), connectedness, and compactness, then you have an
        absolutely solid understanding of topology. We've covered all of these
        ideas except compactness, which we've only discussed in the context of
        metric spaces (or \textit{metrizable} spaces). We now take the time to
        study compactness in the general topological setting.
        \par\hfill\par
        In a metric space we proved several theorems about compactness,
        primarily the Bolzano-Weierstrass, Heine-Borel, generalized
        Heine-Borel, and equivalence of compactness theorems. This told us
        that compactness can be described by sequences and by open sets. In the
        topological setting it is more natural to define compactness via open
        sets.
        \begin{definition}[\textbf{Compact Topological Space}]
            A compact topological space is a topological space $(X,\,\tau)$
            such that for all open covers $\mathcal{O}\subseteq\tau$ there is
            a finite subset $\Delta\subseteq\mathcal{O}$ such that $\Delta$ is
            open cover.
        \end{definition}
        We have spent a lot of time on compactness in the setting of metric
        spaces. Let's not waste that time, and copy over some of the theorems
        but rephrase them for \textit{metrizable} spaces.
        \begin{theorem}
            If $(X,\,\tau)$ is a metrizable topological space, then it is
            compact if and only if for all metrics $d$ on $X$ that induce
            $\tau$, $(X,\,d)$ is a compact metric space.
        \end{theorem}
        \begin{proof}
            By the equivalence of compactness theorem, any metric $d$ that
            induces $\tau$ has the property that any open cover of open sets
            in the metric space $(X,\,d)$ has a finite open subcover, which is
            precisely the definition of compactness in the topological setting.
        \end{proof}
        \begin{theorem}
            If $(X,\,\tau)$ is a metrizable topological space, then it is
            compact if and only if for every metric $d$ on $X$ that induces
            $\tau$, $(X,\,d)$ is complete and totally bounded.
        \end{theorem}
        \begin{proof}
            This follows from the previous theorem and the generalized
            Heine-Borel theorem.
        \end{proof}
        \begin{theorem}
            If $(\mathbb{R},\,\tau_{\mathbb{R}})$ is the standard Euclidean
            line, and if $A\subseteq\mathbb{R}$, then
            $(A,\,\tau_{\mathbb{R}_{A}})$ is compact if and only if $A$ is
            closed and bounded.
        \end{theorem}
        \begin{proof}
            The standard Euclidean topology on $\mathbb{R}$,
            $\tau_{\mathbb{R}}$, is induced by the Euclidean metric
            $d(x,\,y)=|x-y|$. The result then follows from the Heine-Borel
            theorem.
        \end{proof}
        This now gives us plenty of familiar spaces that are compact.
        Lacking a metrizable space, there are still plenty of pleasing
        properties compact topologies yield.
        \begin{theorem}
            If $(X,\,\tau)$ is a compact topological space, and if
            $\mathcal{C}\subseteq{X}$ is closed, then
            $(X,\,\tau_{\mathcal{C}})$ is compact where $\tau_{\mathcal{C}}$ is
            the subspace topology.
        \end{theorem}
        \begin{proof}
            Let $\mathcal{O}$ be an open cover of
            $(\mathcal{C},\,\tau_{\mathcal{C}})$. Then, since
            $\mathcal{C}$ is closed, $X\setminus\mathcal{C}$ is open, and
            hence $\mathcal{O}\cup\{\,X\setminus\mathcal{C}\,\}$ is an open
            cover of $(X,\,\tau)$. But $(X,\,\tau)$ is compact so there is a
            finite subset $\Delta\subseteq\mathcal{O}$ that is an open cover
            of $X$. But then $\Delta\setminus\{\,X\setminus\mathcal{C}\,\}$ is
            a finite open cover of $(\mathcal{C},\,\tau_{\mathcal{C}})$, so
            $(\mathcal{C},\,\tau_{\mathcal{C}})$ is compact.
        \end{proof}
        This theorem does not need to reverse, in general. Give
        $\mathbb{R}$ the indiscrete topology
        $\tau=\{\,\emptyset,\,\mathbb{R}\,\}$. Then every subset
        $A\subseteq\mathbb{R}$ is compact since the only open covers possible
        are finite (they have at most two subsets). However only
        $\emptyset$ and $\mathbb{R}$ are closed. If we add the Hausdorff
        condition, then the only compact subspaces are the closed ones.
\end{document}
