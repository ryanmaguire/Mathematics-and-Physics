%-----------------------------------LICENSE------------------------------------%
%   This file is part of Mathematics-and-Physics.                              %
%                                                                              %
%   Mathematics-and-Physics is free software: you can redistribute it and/or   %
%   modify it under the terms of the GNU General Public License as             %
%   published by the Free Software Foundation, either version 3 of the         %
%   License, or (at your option) any later version.                            %
%                                                                              %
%   Mathematics-and-Physics is distributed in the hope that it will be useful, %
%   but WITHOUT ANY WARRANTY; without even the implied warranty of             %
%   MERCHANTABILITY or FITNESS FOR A PARTICULAR PURPOSE.  See the              %
%   GNU General Public License for more details.                               %
%                                                                              %
%   You should have received a copy of the GNU General Public License along    %
%   with Mathematics-and-Physics.  If not, see <https://www.gnu.org/licenses/>.%
%------------------------------------------------------------------------------%
\documentclass{article}
\usepackage{graphicx}                           % Needed for figures.
\usepackage{amsmath}                            % Needed for align.
\usepackage{amssymb}                            % Needed for mathbb.
\usepackage{amsthm}                             % For the theorem environment.
\usepackage{float}
\usepackage{hyperref}
\hypersetup{
    colorlinks=true,
    linkcolor=blue,
    filecolor=magenta,
    urlcolor=Cerulean,
    citecolor=SkyBlue
}

%------------------------Theorem Styles-------------------------%
\theoremstyle{plain}
\newtheorem{theorem}{Theorem}[section]

% Define theorem style for default spacing and normal font.
\newtheoremstyle{normal}
    {\topsep}               % Amount of space above the theorem.
    {\topsep}               % Amount of space below the theorem.
    {}                      % Font used for body of theorem.
    {}                      % Measure of space to indent.
    {\bfseries}             % Font of the header of the theorem.
    {}                      % Punctuation between head and body.
    {.5em}                  % Space after theorem head.
    {}

% Define default environments.
\theoremstyle{normal}
\newtheorem{examplex}{Example}[section]
\newtheorem{definitionx}{Definition}[section]
\newtheorem{notationx}{Notation}[section]
\newtheorem{axiomx}{Axiom}[section]

\newenvironment{example}{%
    \pushQED{\qed}\renewcommand{\qedsymbol}{$\blacksquare$}\examplex%
}{%
    \popQED\endexamplex%
}

\newenvironment{definition}{%
    \pushQED{\qed}\renewcommand{\qedsymbol}{$\blacksquare$}\definitionx%
}{%
    \popQED\enddefinitionx%
}

\title{Point-Set Topology: Lecture 5}
\author{Ryan Maguire}
\date{Summer 2022}

% No indent and no paragraph skip.
\setlength{\parindent}{0em}
\setlength{\parskip}{0em}

\begin{document}
    \maketitle
    \section{Metric Spaces}
        Most of the analysis of the real numbers comes from the absolute value
        function $|\cdot|:\mathbb{R}\rightarrow\mathbb{R}_{\geq{0}}$ defined
        by:
        \begin{equation}
            |x|=
            \begin{cases}
                x&x\geq{0}\\
                -x&x<0
            \end{cases}
        \end{equation}
        The function $d:\mathbb{R}\times\mathbb{R}\rightarrow\mathbb{R}$
        defined by $d(x,\,y)=|x-y|$ acts as a \textit{distance} function. It is
        the length of the line segment between $x$ and $y$ on the real line.
        Many theorems in real analysis repeatedly use three key facts about this
        function. First, it is \textit{positive-definite}. This means that
        $|x-y|$ is always non-negative, and $|x-y|=0$ if and only if $x=y$.
        This is what a distance function should do, assign a non-negative
        real number to two points, the distance between them. Negative distance
        doesn't have any meaning, and zero distance means the points are
        identical. Second, the function $d(x,\,y)=|x-y|$ is symmetric. That is,
        $d(x,\,y)=d(y,\,x)$ for all $x,y\in\mathbb{R}$. Again, this is what
        distance should mean. The distance from Boston to New York is the same
        as the distance from New York to Boston. The last property is very
        geometrical, the \textit{triangle inequality}. It says that for any
        real numbers $x,y,z\in\mathbb{R}$, $|x-z|\leq|x-y|+|y-z|$. That is,
        the distance from $x$ to $z$ is not greater than the distance from
        $x$ to $y$ plus the distance from $y$ to $z$. This is called the
        triangle inequality since it mimics one of the theorems from Euclid's
        elements relating lengths of triangles. Euclid writes that the length
        of any side of a triangle is not greater than the sum of the lengths of
        the other two sides (Fig.~\ref{fig:triangle_inequality_001}). To put
        this into physical terms, the shortest distance between two points in
        the plane is the straight line segment between them. Deviating off of
        this line results in a longer distance.
        \begin{figure}
            \centering
            \includegraphics{../../images/triangle_inequality_001.pdf}
            \caption{Geometric Interpretation of the Triangle Inequality}
            \label{fig:triangle_inequality_001}
        \end{figure}
        We take these three properties and say that this is what
        \textit{distance} means. Metric spaces are sets that have a method of
        assigning distance between points.
        \begin{definition}[\textbf{Metric Space}]
            A metric space is an ordered pair $(X,\,d)$ where $X$ is a set,
            and $d:X\times{X}\rightarrow\mathbb{R}$ is a function such that:
            \begin{itemize}
                \item $d(x,\,y)\geq{0}$ for all $x,y\in{X}$ and $d(x,\,y)=0$ if
                    and only if $x=y$ (Positive Definite)
                \item $d(x,\,y)=d(y,\,x)$ (Symmetry)
                \item $d(x,\,z)\leq{d}(x,\,y)+d(y,\,z)$ (Triangle Inequality) 
            \end{itemize}
            The function $d$ is called a \textit{metric} function.
        \end{definition}
        For those who wish to go on to geometry, a word of warning. In geometry
        one studies \textit{Riemannian metrics}. These are not metric functions,
        they do not measure distances (they measure angles). An author should
        take care to differentiate between the two, calling one a
        \textit{metric}, and the other a \textit{Riemannian metric}. The author
        should especially do this because Riemannian metrics create metrics,
        called the \textit{metric induced by a Riemannian metric}. Unfortunately
        far too many omit the word \textit{Riemannian} and just call the
        function a metric, hoping it is clear from context which is which.
        You then get gibberish sentences like
        \textit{the metric induced by the metric}. We most certainly will not
        get to geometry in this class, so there should be no confusion
        (However, a course in Riemannian manifolds is perfect after this course.
        We will end on topological manifolds).
        \begin{example}
            If $X=\mathbb{R}$ and $d(x,\,y)=|x-y|$, then $(X,\,d)$ is a metric
            space.
        \end{example}
        \begin{example}
            Define $d(\mathbf{x},\,\mathbf{y})$ on $\mathbb{R}^{2}$ via:
            \begin{equation}
                d(\mathbf{x},\,\mathbf{y})=
                \sqrt{(y_{0}-x_{0})^{2}+(y_{1}-x_{1})^{2}}
            \end{equation}
            This is the Euclidean distance. It is positive-definite since
            the inside of the square root is positive-definite and the square
            root function is increasing. It is symmetric since
            $(x-y)^{2}=(y-x)^{2}$. The hard part is the triangle inequality.
            This turns out to be equivalent to showing that
            $||\mathbf{x}+\mathbf{y}||_{2}\leq||\mathbf{x}||_{2}+||\mathbf{y}||_{2}$
            where $||\mathbf{x}||_{2}$ is the \textit{Euclidean length} of
            $\mathbf{x}$:
            \begin{equation}
                ||\mathbf{x}||_{2}=\sqrt{x_{0}^{2}+x_{1}^{2}}
            \end{equation}
            To see why the triangle inequality is equivalent to this,
            substitute $\mathbf{x}-\mathbf{z}$ for $\mathbf{x}$ and
            $\mathbf{z}-\mathbf{y}$ for $\mathbf{y}$. So, we need only prove
            that $||\mathbf{x}+\mathbf{y}||_{2}\leq||\mathbf{x}||_{2}+||\mathbf{y}||_{2}$.
            We have:
            \begin{align}
                ||\mathbf{x}+\mathbf{y}||_{2}^{2}
                &=(x_{0}+y_{0})^{2}+(x_{1}+y_{1})^{2}\\
                &=x_{0}^{2}+x_{1}^{2}+2(x_{0}y_{0}+x_{1}y_{1})+y_{0}^{2}+y_{1}^{2}\\
                &=||\mathbf{x}||_{2}^{2}+2(x_{0}y_{0}+x_{1}y_{1})+||\mathbf{y}||_{2}^{2}\\
                &=||\mathbf{x}||_{2}^{2}+2\,\mathbf{x}\cdot\mathbf{y}+||\mathbf{y}||_{2}^{2}\\
                &=||\mathbf{x}||_{2}^{2}+2\,||\mathbf{x}||_{2}\,||\mathbf{y}||_{2}\,\cos(\theta)+||\mathbf{y}||_{2}^{2}\\
                &\leq||\mathbf{x}||_{2}^{2}+2\,||\mathbf{x}||_{2}\,||\mathbf{y}||_{2}+||\mathbf{y}||_{2}^{2}\\
                &=(||\mathbf{x}||_{2}+||\mathbf{y}||_{2})^{2}
            \end{align}
            where $\theta$ is the angle between the vectors $\mathbf{x}$ and
            $\mathbf{y}$. By taking the square root of the first and last lines
            we obtain
            the inequality. This makes $(\mathbb{R}^{2},\,d)$ a metric space.
            This is called the \textit{two dimensional Euclidean space} or the
            \textit{standard metric on} $\mathbb{R}^{2}$.
        \end{example}
        \begin{figure}
            \centering
            \includegraphics{../../images/metric_space_example_r2_001.pdf}
            \caption{Euclidean Metric on $\mathbb{R}^{2}$}
            \label{fig:metric_space_example_r2_001}
        \end{figure}
        \begin{example}
            Let $X=\mathbb{R}^{2}$ and
            $d(\mathbf{x},\,\mathbf{y})=|x_{0}-y_{0}|+|x_{1}-y_{1}|$. This is
            called the \textit{Manhattan metric} on $\mathbb{R}^{2}$.
            The function $d$ is indeed a metric on $\mathbb{R}^{2}$
            (prove this!). It is called the Manhattan metric since this metric
            is formed by the following sentence:
            \textit{To move from} $\mathbf{x}$ \textit{to} $\mathbf{y}$
            \textit{you must move horizontally (left-right) or vertically}
            \textit{(up-down). You may not move diagonal.}
            This is similar to how you'd move in Manhattan. You
            can't move diagonally because there are buildings in the way,
            but you may move horizontally and vertically along the streets.
        \end{example}
        \begin{figure}
            \centering
            \includegraphics{../../images/metric_space_example_manhattan_r2_001.pdf}
            \caption{Manhattan Metric on $\mathbb{R}^{2}$}
            \label{fig:metric_space_example_manhattan_r2_001}
        \end{figure}
        \begin{example}
            The function $d$ on $\mathbb{R}\times\mathbb{R}$ defined by
            $d(x,\,y)=\textrm{max}(|x|,\,|y|)$ is a metric. This is the
            \textit{maximum} metric.
        \end{example}
        \begin{example}
            The function $d:\mathbb{R}^{2}\times\mathbb{R}^{2}\rightarrow\mathbb{R}$
            defined by:
            \begin{equation}
                d(\mathbf{x},\,\mathbf{y})=
                \textrm{max}(|x_{0}-y_{0}|,\,|x_{1}-y_{1}|)
            \end{equation}
            is a metric. This is the metric that describes how a king can move
            on a chess board. The distance from the position of a king on the
            board to another position is the maximum of the vertical and
            horizontal distances since the king is allowed to move horizontally,
            vertically, and diagonally.
        \end{example}
        \begin{figure}
            \centering
            \includegraphics{../../images/metric_space_example_paris_metric_001.pdf}
            \caption{Paris Metric on $\mathbb{R}^{2}$}
            \label{fig:metric_space_example_paris_metric_001}
        \end{figure}
        \begin{example}
            To those who have been to Paris, perhaps you have visited the
            Arc de Triomphe. It is surrounded by several roads that all lead
            radially inwards towards the monument, somewhat like spokes on
            a wheel. To walk from one street to another requires walking towards
            the Arc de Triomphe and then outwards along the appropriate street.
            This allows us to define the \textit{Paris metric} on
            $\mathbb{R}^{2}$. Define $d$ as follows:
            \begin{equation}
                d(\mathbf{x},\,\mathbf{y})=
                \begin{cases}
                    ||\mathbf{x}-\mathbf{y}||_{2}&
                        \mathbf{y}=\lambda\mathbf{x}\textrm{ for some }
                        \lambda\in\mathbb{R}\\
                    ||\mathbf{x}||_{2}+||\mathbf{y}||_{2}&
                        \textrm{else}
                \end{cases}
            \end{equation}
            That is, if $\mathbf{x}$ and $\mathbf{y}$ lie on the same line
            passing through the origin, their distance is the usual Euclidean
            distance. Otherwise to get from $\mathbf{x}$ to $\mathbf{y}$ you
            must walk radially inwards from $\mathbf{x}$ to the origin, and
            then radially outwards from the origin to $\mathbf{y}$. This makes
            $(\mathbb{R}^{2},\,d)$ a metric spaces.
        \end{example}
        \begin{example}
            There's a Manhattan metric, a Paris metric, and a London metric.
            If you want to get from point $a$ to point $b$ in England, you take
            the train. It always seems that no matter where you're trying to
            go, your train will first make a stop in London. This gives us the
            \textit{London metric}. Define $d$ by:
            \begin{equation}
                d(\mathbf{x},\,\mathbf{y})=
                \begin{cases}
                    0&\mathbf{x}=\mathbf{y}\\
                    ||\mathbf{x}||_{2}+||\mathbf{y}||_{2}&\textrm{otherwise}
                \end{cases}
            \end{equation}
            That is, if $\mathbf{x}=\mathbf{y}$, there's no point getting on
            the train since you're already where you want to be. Otherwise,
            take the train from $\mathbf{x}$ to London (the origin) and then
            from London to $\mathbf{y}$.
        \end{example}
        \begin{example}
            If $X$ is any set and $d:X\times{X}\rightarrow\mathbb{R}$ is defined
            by:
            \begin{equation}
                d(x,\,y)=
                \begin{cases}
                    0&x=y\\
                    1&x\ne{y}
                \end{cases}
            \end{equation}
            Then $(X,\,d)$ is a metric space. This is called the
            \textit{discete} metric.
        \end{example}
        \begin{example}
            Consider $[a,\,b]\subseteq\mathbb{R}$ for some $a,b\in\mathbb{R}$
            with $a<b$ and let $C\big([a,\,b],\,\mathbb{R}\big)$ be the set of
            all continuous functions $f:[a,\,b]\rightarrow\mathbb{R}$. Since
            $[a,\,b]$ is closed and bounded, if
            $f\in{C}\big([a,\,b],\,\mathbb{R}\big)$ then $f$ is continuous and
            by the extreme value theorem $f$ is bounded above and below, so the
            integral of $f$ is finite. Given two function
            $f,g\in{C}\big([a,\,b],\,\mathbb{R}\big)$ we can define
            $d(f,\,g)$ by:
            \begin{equation}
                d(f,\,g)=\int_{a}^{b}|f(x)-g(x)|\,\textrm{d}x
            \end{equation}
            This makes the set of continuous functions into a metric space.
        \end{example}
        Why require three things for a metric space when you can just as
        easily require two. If instead of writing the triangle inequality
        as $d(x,\,z)\leq{d}(x,\,y)+d(y,\,z)$ we write
        $d(x,\,z)\leq{d}(x,\,y)+d(z,\,y)$, we can prove $d$ is symmetric.
        \begin{theorem}
            If $X$ is a set and $d:X\times{X}\rightarrow\mathbb{R}$ is
            positive-definite and for all
            $x,y,z\in{X}$ it is true that
            $d(x,\,z)\leq{d}(x,\,y)+d(z,\,y)$, then $d$ is symmetric and
            $(X,\,d)$ is a metric space.
        \end{theorem}
        \begin{proof}
            Letting $z=x$ we have:
            \begin{equation}
                d(x,\,y)\leq{d}(x,\,x)+d(y,\,x)=d(y,\,x)
            \end{equation}
            Similarly if we let $z=y$ we get:
            \begin{equation}
                d(y,\,x)\leq{d}(y,\,y)+d(x,\,y)=d(x,\,y)
            \end{equation}
            So $d(x,\,y)\leq{d}(y,\,x)$ and $d(y,\,x)\leq{d}(x,\,y)$, so
            $d(x,\,y)=d(y,\,x)$ and hence
            $d$ is symmetric. Thus, $(X,\,d)$ is a metric space.
        \end{proof}
    \section{Open and Closed Sets}
        Most of the important properties of metric spaces can be defined by
        \textit{sequences} and \textit{convergence}.
        \begin{definition}[\textbf{Sequence}]
            A sequence in a set $A$ is a function $a:\mathbb{N}\rightarrow{A}$.
            Instead of writing $a(n)$ for the value of $n\in\mathbb{N}$, we
            write $a_{n}$.
        \end{definition}
        \begin{definition}[\textbf{Convergent Sequence in a Metric Space}]
            A convergent sequence in a metric space $(X,\,d)$ is a sequence
            $a:\mathbb{N}\rightarrow{X}$ such that there exists an $x\in{X}$
            where for all $\varepsilon>0$ there is an $N\in\mathbb{N}$ such that
            $n\in\mathbb{N}$ and $n>N$ implies $d(x,\,a_{n})<\varepsilon$.
            $x$ is called a limit of $a$ and we write $a_{n}\rightarrow{x}$.
        \end{definition}
        \begin{figure}
            \centering
            \includegraphics{../../images/convergent_sequence_in_metric_space_001.pdf}
            \caption{Convergent Sequence in a Metric Space}
            \label{fig:convergent_sequence_in_metric_space_001}
        \end{figure}
        Contrast this with the definition of convergent sequences of real
        numbers. These are sequences $a:\mathbb{N}\rightarrow\mathbb{R}$ where
        there is an $x\in\mathbb{R}$ such that for all $\varepsilon>0$ there
        is an $N\in\mathbb{N}$ where $n\in\mathbb{N}$ and $n>N$ implies
        $|x-a_{n}|<\varepsilon$. We've merely replaced $|x-a_{n}|$ with
        $d(x,\,a_{n})$. Like the real numbers, convergence in a metric space
        is unique.
        \begin{theorem}
            If $(X,\,d)$ is a metric space, if $a:\mathbb{N}\rightarrow{X}$
            is a convergent sequence, and if $x,y\in{X}$ are such that
            $a_{n}\rightarrow{x}$ and $a_{n}\rightarrow{y}$, then
            $x=y$.
        \end{theorem}
        \begin{proof}
            Suppose not. If $x\ne{y}$, then since $d$ is a metric function,
            $d(x,\,y)>0$. Let $\varepsilon=\frac{1}{2}d(x,\,y)$. Then since
            $\varepsilon>0$ and $a_{n}\rightarrow{x}$, there is an
            $N_{0}\in\mathbb{N}$ such that $n\in\mathbb{N}$ and $n>N_{0}$
            implies $d(x,\,a_{n})<\varepsilon$. But since $a_{n}\rightarrow{y}$
            there is an $N_{1}\in\mathbb{N}$ such that $n\in\mathbb{N}$ and
            $n>N_{1}$ implies $d(y,\,a_{n})<\varepsilon$. Let
            $N=\textrm{max}(N_{0},\,N_{1})$. Then $n>N$ implies
            $d(x,\,a_{n})<\varepsilon$ and $d(y,\,a_{n})<\varepsilon$. But by
            the triangle inequality,
            \begin{align}
                d(x,\,y)&\leq{d}(x,\,a_{N+1})+d(y,\,a_{N+1})\\
                &<\varepsilon+\varepsilon\\
                &=\frac{d(x,\,y)}{2}+\frac{d(x,\,y)}{2}\\
                &=d(x,\,y)
            \end{align}
            a contradiction. Thus, $x=y$.
        \end{proof}
        \begin{definition}[\textbf{Open Ball}]
            An open ball of radius $r$ centered about $x\in{X}$ in a metric
            space $(X,\,d)$ is the set:
            \begin{equation}
                B_{r}^{(X,\,d)}(x)=\{\,y\in{X}\;|\;d(x,\,y)<r\,\}
            \end{equation}
            That is, the set of all points $y\in{X}$ that are closer than
            $r$ away from $x$.
        \end{definition}
        \begin{figure}
            \centering
            \includegraphics{../../images/open_set_in_a_metric_space_001.pdf}
            \caption{Open Ball in a Metric Space}
            \label{fig:open_set_in_a_metric_space_001}
        \end{figure}
        \begin{example}
            An open ball in $\mathbb{R}^{2}$ with respect to the Euclidean
            metric is a disk. For simplicity, let $\mathbf{x}=\mathbf{0}$. Then
            $B_{r}^{(\mathbb{R}^{2},\,||\cdot||_{2})}(\mathbf{0})$ is
            the set of all $\mathbf{y}$ such that
            $||\mathbf{y}||_{2}<r$. This is the set of all points
            $\mathbf{y}=(x,\,y)$ such that $\sqrt{x^{2}+y^{2}}<r$. squaring
            we have $x^{2}+y^{2}<r^{2}$. This is the open disk centered at the
            origin of radius $r$.
        \end{example}
        \begin{example}
            In the Manhattan metric, the open ball centered at the origin of
            radius $r$ is a diamond. This is the set of points
            $(x,\,y)$ such that $|x|+|y|<r$.
        \end{example}
        \begin{example}
            Using the chess king metric, or the maximum metric on
            $\mathbb{R}^{2}$, an open ball is a square. Consider an open ball
            centered about the origin $\mathbf{0}$. This is the set of points
            $(x,\,y)$ such that $\max(|x|,\,|y|)<r$. So the points where both
            $|x|<r$ and $|y|<r$. This forms a square in the plane.
        \end{example}
        \begin{example}
            The Paris metric does not have the symmetry of the other metrics.
            An open ball centered about the origin is a disk. Open balls
            centered about other points look quite different. If
            $\mathbf{x}\ne\mathbf{0}$ and if $r>||\mathbf{x}||_{2}$, then
            the open ball centered at $\mathbf{x}$ of radius $r$ consists of
            the disk centered at the origin of radius
            $r-||\mathbf{x}||_{2}$ and the open line segment passing radially
            from the origin through the point $\mathbf{x}$
            of length $r$. If $r<||\mathbf{x}||_{2}$, the open ball or radius
            $r$ centered at $\mathbf{x}$ is an open line segment
            (Fig.~\ref{fig:open_ball_r2_paris_metric_001}).
        \end{example}
        \begin{figure}
            \centering
            \includegraphics{../../images/open_ball_r2_001.pdf}
            \caption{Open Ball in the Euclidean Plane}
            \label{fig:open_ball_r2_001}
        \end{figure}
        \begin{figure}
            \centering
            \includegraphics{../../images/open_ball_r2_manhattan_001.pdf}
            \caption{Open Ball in the Manhattan Metric}
            \label{fig:open_ball_r2_manhattan_001}
        \end{figure}
        \begin{figure}
            \centering
            \includegraphics{../../images/open_ball_r2_max_metric_001.pdf}
            \caption{Open Ball with the Max Metric}
            \label{fig:open_ball_r2_max_metric_001}
        \end{figure}
        \begin{figure}
            \centering
            \includegraphics{../../images/open_ball_r2_paris_metric_001.pdf}
            \caption{Open Balls in the Paris Metric}
            \label{fig:open_ball_r2_paris_metric_001}
        \end{figure}
        \begin{definition}[\textbf{Open Subsets}]
            An open subset of a metric space $(X,\,d)$ is a set
            $\mathcal{U}\subseteq{X}$ such that for all
            $x\in\mathcal{U}$ there is an $\varepsilon>0$ such that
            $B_{\varepsilon}^{(X,d)}(x)\subseteq\mathcal{U}$.
        \end{definition}
        A theorem every student of mathematics must prove in their life
        is that \textit{open balls are open}. It seems almost like the proof
        comes by definition, but there's a bit of work needed.
        \begin{theorem}
            If $(X,\,d)$ is a metric space, if $x\in{X}$, and if
            $r\in\mathbb{R}^{+}$, then $B_{r}^{(X,\,d)}(x)$ is open.
        \end{theorem}
        \begin{proof}
            Let $y\in{B}_{r}^{(X,\,d)}(x)$. Let
            $\varepsilon=r-d(x,\,y)$. Since
            $y\in{B}_{r}^{(X,\,d)}(x)$, $d(x,\,y)<r$ and hence
            $\varepsilon>0$. If
            $z\in{B}_{\varepsilon}^{(X,\,d)}(y)$, then:
            \begin{align}
                d(x,\,z)&\leq{d}(x,\,y)+d(y,\,z)\\
                    &<d(x,\,y)+\varepsilon\\
                    &=d(x,\,y)+r-d(x,\,y)\\
                    &=r
            \end{align}
            and therefore $z\in{B}_{r}^{(X,\,d)}(x)$, and thus
            $B_{r}^{(X,\,d)}(x)$ is open.
        \end{proof}
        \begin{figure}
            \centering
            \includegraphics{../../images/open_balls_are_open.pdf}
            \caption{Open Balls are Open}
            \label{fig:open_balls_are_open}
        \end{figure}
        \begin{theorem}
            If $(X,\,d)$ is a metric space, then $\emptyset$ is open.
        \end{theorem}
        \begin{proof}
            This is true vacuously. There are no elements
            $x\in\emptyset$ such that the definition of open fails, so
            we say $\emptyset$ is open.
        \end{proof}
        \begin{theorem}
            If $(X,\,d)$ is a metric space, then
            $X$ is open.
        \end{theorem}
        \begin{proof}
            Let $x\in{X}$ and $r=1$. Then, by definition,
            $B_{1}^{(X,\,d)}(x)\subseteq{X}$, so $X$ is open.
        \end{proof}
        \begin{theorem}
            If $(X,\,d)$ is a metric space, and if
            $\mathcal{O}\subseteq\mathcal{P}(X)$ is such that for all
            $\mathcal{U}\in\mathcal{O}$ it is true that $\mathcal{U}$ is open,
            then $\bigcup\mathcal{O}$ is open.
        \end{theorem}
        \begin{proof}
            Let $x\in\bigcup\mathcal{O}$. Since $x\in\bigcup\mathcal{O}$
            there is a $\mathcal{U}\in\mathcal{O}$ such that
            $\mathcal{U}\in\mathcal{O}$. Then by the definition of
            $\mathcal{O}$, $\mathcal{U}$ is open. Hence there is an
            $r>0$ such that $B_{r}^{(X,\,d)}(x)\subseteq\mathcal{U}$.
            But then $B_{r}^{(X,\,d)}(x)\subseteq\bigcup\mathcal{O}$, and
            hence $\bigcup\mathcal{O}$ is open.
        \end{proof}
        \begin{theorem}
            If $(X,\,d)$ is a metric space, if $\mathcal{U}$ and
            $\mathcal{V}$ are open subsets, then
            $\mathcal{U}\cap\mathcal{V}$ is open.
        \end{theorem}
        \begin{proof}
            If $\mathcal{U}\cap\mathcal{V}$ is empty we are done.
            Suppose $x\in\mathcal{U}\cap\mathcal{V}$. Since
            $x\in\mathcal{U}$ and $\mathcal{U}$ is open, there is an
            $r_{0}$ such that
            $B_{r_{0}}^{(X,\,d)}(x)\subseteq\mathcal{U}$. Since
            $x\in\mathcal{V}$ and $\mathcal{V}$ is open there is an
            $r_{1}>0$ such that $B_{r_{1}}^{(X,\,d)}(x)\subseteq\mathcal{V}$.
            Let $r=\textrm{min}(r_{0},\,r_{1})$. Then
            $B_{r}^{(X,\,d)}(x)\subseteq\mathcal{U}$ and
            $B_{r}^{(X,\,d)}(x)\subseteq\mathcal{V}$, and hence
            $B_{r}^{(X,\,d)}(x)\subseteq\mathcal{U}\cap\mathcal{V}$, so
            $\mathcal{U}\cap\mathcal{V}$ is open.
        \end{proof}
        Open balls completely characterize open subsets of a metric space.
        To be more precise, I mean the following theorem.
        \begin{theorem}
            If $(X,\,d)$ is a metric space, and if $\mathcal{U}\subseteq{X}$,
            then $\mathcal{U}$ is open if and only if there is a set
            $\mathcal{O}$ such that for all $\mathcal{V}\in\mathcal{O}$ it
            is true that $\mathcal{V}$ is an open ball in $X$, and such that
            $\bigcup\mathcal{O}=\mathcal{U}$.
        \end{theorem}
        \begin{proof}
            If $\mathcal{U}$ is open, then for all $x\in\mathcal{U}$ there
            is an $r_{x}>0$ such that
            $B_{r_{x}}^{(X,\,d)}(x)\subseteq\mathcal{U}$. Let
            $\mathcal{O}$ be defined by
            (using the axiom of choice here. I'm \textit{choosing} $r_{x}$):
            \begin{equation}
                \mathcal{O}=\{\,B_{r_{x}}^{(X,\,d)}(x)\;|\;x\in\mathcal{U}\,\}
            \end{equation}
            Since $x\in{B}_{r_{x}}^{(X,\,d)}(x)$ for all $x\in\mathcal{U}$,
            $\mathcal{U}\subseteq\bigcup\mathcal{O}$. But
            $B_{r_{x}}^{(X,\,d)}(x)\subseteq\mathcal{U}$ and hence
            $\bigcup\mathcal{O}\subseteq\mathcal{U}$. Therefore,
            $\bigcup\mathcal{O}=\mathcal{U}$. In the other direction, if
            $\mathcal{U}$ is of this form, then it is the union of open sets,
            and hence open.
        \end{proof}
        \begin{definition}[\textbf{Limit Point in a Metric Space}]
            A limit point of a subset $A\subseteq{X}$ of a metric space
            $(X,\,d)$ is a point $x\in{X}$ such that there is a sequence
            $a:\mathbb{N}\rightarrow{A}$ such that
            $a_{n}\rightarrow{x}$.
        \end{definition}
        \begin{definition}[\textbf{Closed Set in a Metric Space}]
            A closed subset of a metric space $(X,\,d)$ is a subset
            $\mathcal{C}\subseteq{X}$ such that for all $x\in{X}$ such that
            $x$ is a limit point of $\mathcal{C}$, it is true that
            $x\in\mathcal{C}$.
        \end{definition}
        \begin{theorem}
            If $(X,\,d)$ is a metric space, then $\emptyset$ is closed.
        \end{theorem}
        \begin{proof}
            Again, this is vacuously true. There are no points
            $x\in\emptyset$ that fail to satisfy the criterion.
        \end{proof}
        \begin{theorem}
            If $(X,\,d)$ is a metric space, then $X$ is closed.
        \end{theorem}
        \begin{proof}
            For if $a:\mathbb{N}\rightarrow{X}$ is a convergent sequence,
            then by definition $a_{n}\rightarrow{x}$ for some $x\in{X}$,
            and hence $X$ has all of its limit points.
        \end{proof}
        \begin{theorem}
            If $(X,\,d)$ is a metric space, if
            $\mathcal{O}\subseteq\mathcal{P}(X)$ is such that for all
            $\mathcal{C}\in\mathcal{O}$ it is true that $\mathcal{C}$ is closed
            in $X$, then $\bigcap\mathcal{O}$ is closed.
        \end{theorem}
        \begin{proof}
            If the intersection is empty, we're done. Suppose
            $a:\mathbb{N}\rightarrow\bigcap\mathcal{O}$ is a convergent
            sequence. Then for all $\mathcal{C}\in\mathcal{O}$,
            $a:\mathbb{N}\rightarrow\mathcal{C}$ is a convergent sequence.
            Suppose the limit is $x\in{X}$. But $\mathcal{C}$ is closed, and
            hence $x\in\mathcal{C}$. Since limits are unique, this is the
            same $x$ for all $\mathcal{C}\in\mathcal{O}$, and hence
            $x\in\bigcap\mathcal{O}$, That is, $\bigcap\mathcal{O}$ is closed.
        \end{proof}
        \begin{theorem}
            If $(X,\,d)$ is a metric space, then
            $\mathcal{U}$ is open if and only if $X\setminus\mathcal{U}$ is
            closed.
        \end{theorem}
        \begin{proof}
            Suppose $\mathcal{U}$ is open and let
            $a:\mathbb{N}\rightarrow{X}\setminus\mathcal{U}$ be a convergent
            sequence converging to $x\in{X}$. Suppose
            $x\notin{X}\setminus\mathcal{U}$. Then $x\in\mathcal{U}$, and hence
            there is an $\varepsilon>0$ such that
            $B_{\varepsilon}^{(X,\,d)}(x)\subseteq\mathcal{U}$. But then,
            since $a_{n}\rightarrow{x}$, there is an $N\in\mathbb{N}$ such that
            $n\in\mathbb{N}$ and $n>N$ implies
            $d(x,\,a_{n})<\varepsilon$. That is, $n>N$ implies
            $a_{n}\in{B}_{\varepsilon}^{(X,\,d)}(x)$. But
            $a_{n}\in{X}\setminus\mathcal{U}$ for all $n\in\mathbb{N}$, a
            contradiction. So $X\setminus\mathcal{U}$ is closed. Now,
            suppose $X\setminus\mathcal{U}$ is closed. Suppose
            $\mathcal{U}$ is not open. Then there is an
            $x\in\mathcal{U}$ such that for all $\varepsilon>0$ there is an
            $a\in{X}$ such that $d(x,\,a)<\varepsilon$ but
            $a\notin\mathcal{U}$. In particular, for each
            $n\in\mathbb{N}$ there is an $a_{n}$ such that
            $d(x,\,a_{n})<\frac{1}{n+1}$ but $a_{n}\notin\mathcal{U}$.
            But then $a_{n}\rightarrow{x}$. But $X\setminus\mathcal{U}$
            is closed and $a:\mathbb{N}\rightarrow{X}\setminus\mathcal{U}$
            is a convergent sequence, so the limit is in
            $X\setminus\mathcal{U}$. But $x\in\mathcal{U}$, a contradiction.
            Hence, $\mathcal{U}$ is open.
        \end{proof}
        \begin{theorem}
            If $(X,\,d)$ is a metric space, then
            $\mathcal{C}\subseteq{X}$ is closed if and only if
            $X\setminus\mathcal{C}$ is open.
        \end{theorem}
        \begin{proof}
            Since $X\setminus(X\setminus\mathcal{C})=\mathcal{C}$,
            this follows from the previous theorem.
        \end{proof}
        \begin{theorem}
            If $(X,\,d)$ is a metric space and if $\mathcal{C}$ and
            $\mathcal{D}$ are closed subsets, then
            $\mathcal{C}\cup\mathcal{D}$ is closed.
        \end{theorem}
        \begin{proof}
            This follows from De Morgan's laws.
            $\mathcal{C}$ and $\mathcal{D}$ are closed and if and only if
            $X\setminus\mathcal{C}$ and $X\setminus\mathcal{D}$ are open.
            But:
            \begin{equation}
                X\setminus(\mathcal{C}\cup\mathcal{D})
                    =(X\setminus\mathcal{C})\cap(X\setminus\mathcal{D})
            \end{equation}
            This is the intersection of two open sets, which is open. So
            $\mathcal{C}\cup\mathcal{D}$ is closed.
        \end{proof}
\end{document}
