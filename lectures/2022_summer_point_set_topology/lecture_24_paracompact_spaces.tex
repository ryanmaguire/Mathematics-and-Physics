%-----------------------------------LICENSE------------------------------------%
%   This file is part of Mathematics-and-Physics.                              %
%                                                                              %
%   Mathematics-and-Physics is free software: you can redistribute it and/or   %
%   modify it under the terms of the GNU General Public License as             %
%   published by the Free Software Foundation, either version 3 of the         %
%   License, or (at your option) any later version.                            %
%                                                                              %
%   Mathematics-and-Physics is distributed in the hope that it will be useful, %
%   but WITHOUT ANY WARRANTY; without even the implied warranty of             %
%   MERCHANTABILITY or FITNESS FOR A PARTICULAR PURPOSE.  See the              %
%   GNU General Public License for more details.                               %
%                                                                              %
%   You should have received a copy of the GNU General Public License along    %
%   with Mathematics-and-Physics.  If not, see <https://www.gnu.org/licenses/>.%
%------------------------------------------------------------------------------%
\documentclass{article}
\usepackage{graphicx}                           % Needed for figures.
\usepackage{amsmath}                            % Needed for align.
\usepackage{amssymb}                            % Needed for mathbb.
\usepackage{amsthm}                             % For the theorem environment.
\usepackage{float}
\usepackage{tabularx, booktabs}
\usepackage{mathrsfs}
\usepackage[font=scriptsize,
            labelformat=simple,
            labelsep=colon]{subcaption} % Subfigure captions.
\usepackage[font={scriptsize},
            hypcap=true,
            labelsep=colon]{caption}    % Figure captions.
\usepackage{hyperref}
\hypersetup{
    colorlinks=true,
    linkcolor=blue,
    filecolor=magenta,
    urlcolor=Cerulean,
    citecolor=SkyBlue
}

%------------------------Theorem Styles-------------------------%
\theoremstyle{plain}
\newtheorem{theorem}{Theorem}[section]

% Define theorem style for default spacing and normal font.
\newtheoremstyle{normal}
    {\topsep}               % Amount of space above the theorem.
    {\topsep}               % Amount of space below the theorem.
    {}                      % Font used for body of theorem.
    {}                      % Measure of space to indent.
    {\bfseries}             % Font of the header of the theorem.
    {}                      % Punctuation between head and body.
    {.5em}                  % Space after theorem head.
    {}

% Define default environments.
\theoremstyle{normal}
\newtheorem{examplex}{Example}[section]
\newtheorem{definitionx}{Definition}[section]

\newenvironment{example}{%
    \pushQED{\qed}\renewcommand{\qedsymbol}{$\blacksquare$}\examplex%
}{%
    \popQED\endexamplex%
}

\newenvironment{definition}{%
    \pushQED{\qed}\renewcommand{\qedsymbol}{$\blacksquare$}\definitionx%
}{%
    \popQED\enddefinitionx%
}

\title{Point-Set Topology: Lecture 23}
\author{Ryan Maguire}
\date{Summer 2022}

% No indent and no paragraph skip.
\setlength{\parindent}{0em}
\setlength{\parskip}{0em}

\begin{document}
    \maketitle
    \section{Paracompact Spaces}
        Paracompactness is an idea that is perhaps so strange and seemingly weak
        that one may wonder why on Earth it deserves study, let alone a name.
        One of the motivations is the Urysohn metrization theorem. This theorem
        goes one way, a second countable regular Hausdorff space is metrizable.
        It does not reverse. The discrete topology $\mathbb{R}$ is metrizable
        but not second countable. It would be nice to have a theorem that gives
        necessary and sufficient conditions for a space to metrizable. The
        Nagata-Smirnov and Smirnov metrization theorems do this. At the heart
        of both theorems is the idea of local finiteness. The Nagata-Smirnov
        theorem requires $\sigma$ locally finite bases, the Smirnov theorem
        uses paracompactness. We take the time to develop these and similar
        ideas. At the end of this lecture we'll prove the
        \textit{Stone paracompactness theorem}, which states that every
        metrizable space is paracompact.
        \begin{definition}[\textbf{Locally Finite Collection}]
            A locally finite collection in a topological space $(X,\,\tau)$
            is a subset $\mathcal{A}\subseteq\mathcal{P}(X)$ such that for all
            $x\in{X}$ there is a $\mathcal{U}\in\tau$ such that
            $x\in\mathcal{U}$ and only finitely many $A\in\mathcal{A}$ are
            such that $A\cap\mathcal{U}\ne\emptyset$.
        \end{definition}
        Note there is no requirement that $\mathcal{A}$ consist of open sets or
        closed sets. There is no requirement that $\mathcal{A}$ covers the
        space either. All that is required is local finiteness.
        \begin{definition}[\textbf{Refinement of a Collection}]
            A refinement of a collection $\mathcal{A}\subseteq\mathcal{P}(X)$
            in a topological space $(X,\,\tau)$ is a set
            $\tilde{\mathcal{A}}\subseteq\mathcal{P}(X)$ such that for all
            $\tilde{A}\in\tilde{\mathcal{A}}$ there is an $A\in\mathcal{A}$
            such that $\tilde{A}\subseteq{A}$.
        \end{definition}
        Again, in general, there is no requirement that $\mathcal{A}$ or
        $\tilde{\mathcal{A}}$ consist of open or closed sets.
        \textit{Open refinements} are refinements consisting of open sets, and
        \textit{closed refinements} consist of closed sets. This idea is used
        to define paracompactnes.
        \begin{definition}[\textbf{Paracompact Topological Space}]
            A paracompact topological space is a topological space $(X,\,\tau)$
            such that for all open covers $\mathcal{O}\subseteq\tau$ of $X$
            there exists a locally finite open refinement $\Delta\subseteq\tau$
            that is still an open cover of $(X,\,\tau)$.
        \end{definition}
        This is a very weak notion, many familiar spaces are paracompact, and
        yet it has enormous use in manifold theory and the study of metric
        spaces. In particular because every manifold and every metrizable space
        is paracompact.
        \begin{theorem}
            If $(X,\,\tau)$ is a compact topological space, then it is
            paracompact.
        \end{theorem}
        \begin{proof}
            In a compact space every open cover has a finite subcover, which is
            certainly a locally finite open refinement of the cover.
        \end{proof}
        Far weaker than compactness, $\sigma$ compact plus locally compact
        Hausdorff implies paracompact. We've discussed local compactness in
        metric spaces, and the definition has very little difference in
        topological spaces.
        \begin{definition}[\textbf{Locally Compact Topological Space}]
            A locally compact topological space is a topological space
            $(X,\,\tau)$ such that for all $x\in{X}$ there is an open set
            $\mathcal{U}\in\tau$ and a compact set $K\subseteq{X}$ such that
            $x\in\mathcal{U}$ and $\mathcal{U}\subseteq{K}$.
        \end{definition}
        Before proving locally compact $\sigma$ compact spaces are paracompact,
        we'll need a little lemma.
        \begin{theorem}
            If $(X,\,\tau)$ is locally compact and Hausdorff and if $x\in{X}$,
            then there is a $\mathcal{U}\in\tau$ such that $x\in\mathcal{U}$
            and $\textrm{Cl}_{\tau}(\mathcal{U})$ is compact.
        \end{theorem}
        \begin{proof}
            Since $(X,\,\tau)$ is locally compact there is a
            $\mathcal{U}\in\tau$ and a compact $K\subseteq{X}$ such that
            $x\in\mathcal{U}$ and $\mathcal{U}\subseteq{K}$. But $(X,\,\tau)$
            is Hausdorff, so $K$ is closed. But then
            $\textrm{Cl}_{\tau}(\mathcal{U})\subseteq{K}$. But then
            $\textrm{Cl}_{\tau}(\mathcal{U})$ is a closed subset of a compact
            space, which is therefore compact.
        \end{proof}
        \begin{theorem}
            If $(X,\,\tau)$ is $\sigma$ compact, locally compact, and Hausdorff,
            then it is compactly exhaustible.
        \end{theorem}
        \begin{proof}
            Since $(X,\,\tau)$ is $\sigma$ compact there are countably many
            sets $\mathcal{C}_{n}$, $n\in\mathbb{N}$, each of which is compact
            and such that they cover the space. For all $n\in\mathbb{N}$ and
            for all $x\in\mathcal{C}_{n}$, since $(X,\,\tau)$ is locally compact
            and Hausdorff, there is a $\mathcal{V}_{x,n}\in\tau$ such that
            $x\in\mathcal{V}_{x,n}$ and $\textrm{Cl}_{\tau}(\mathcal{V}_{x,n})$
            is compact. But these sets cover $\mathcal{C}_{n}$, which is
            compact, so we can do it with finitely many,
            $\mathcal{V}_{0,n},\,\dots,\,\mathcal{V}_{N,n}$. Since this is
            a finite collection we have:
            \begin{equation}
                \textrm{Cl}_{\tau}\Big(\bigcup_{k=0}^{N}\mathcal{V}_{k,n}\Big)
                =\bigcup_{k=0}^{N}\textrm{Cl}_{\tau}(\mathcal{V}_{k,n})
            \end{equation}
            Define $\mathcal{U}_{n}$ via:
            \begin{equation}
                \mathcal{U}_{n}=\bigcup_{k=0}^{N}\mathcal{V}_{k,n}
            \end{equation}
            Then $\mathcal{U}_{n}$ is open and by the previous equation
            $\textrm{Cl}_{n}(\mathcal{U}_{n})$ is the finite union of compact
            sets, which is therefore compact. Recursively define
            $\mathcal{W}_{n}$ as follows. Set $\mathcal{W}_{0}=\mathcal{U}_{0}$.
            Let $\mathcal{W}_{n}\in\tau$ be such that
            $\bigcup_{k=0}^{n}\mathcal{U}_{k}\subseteq\mathcal{W}_{n}$ and
            such that $\textrm{Cl}_{\tau}(\mathcal{W}_{n})$ is compact.
            Define $\mathcal{W}_{n+1}$ as follows. Since
            $\textrm{Cl}_{\tau}(\mathcal{W}_{n})$ and
            $\textrm{Cl}_{\tau}(\mathcal{U}_{n+1})$ are compact, so is the
            union. Thus, by the previous argument, we can cover it in finitely
            many open sets $\mathcal{V}_{0},\,\dots,\,\mathcal{V}_{N}$, each
            of which has compact closure. Define:
            \begin{equation}
                \mathcal{W}_{n+1}=\bigcup_{k=0}^{N}\mathcal{V}_{k}
            \end{equation}
            Then $\mathcal{W}_{n+1}$ is open and:
            \begin{equation}
                \textrm{Cl}_{\tau}(\mathcal{W}_{n+1})
                =\textrm{Cl}_{\tau}\Big(\bigcup_{k=0}^{N}\mathcal{V}_{k}\Big)
                =\bigcup_{k=0}^{N}\textrm{Cl}_{\tau}(\mathcal{V}_{k})
            \end{equation}
            which is the finite union of compact sets, so it is compact.
            But moreover, from the construction, since the $\mathcal{V}_{k}$
            cover $\mathcal{W}_{n}$, we have
            $\textrm{Cl}_{\tau}(\mathcal{W}_{n})\subseteq\mathcal{W}_{n+1}$.
            Define:
            \begin{equation}
                K_{n}=\textrm{Cl}_{\tau}(\mathcal{W}_{n})
            \end{equation}
            Then $K_{n}$ is compact and
            $K_{n}\subseteq\textrm{Int}_{\tau}(K_{n+1})$ since
            $\textrm{Int}_{\tau}(K_{n+1})=\mathcal{W}_{n+1}$. Morever
            $\bigcup_{n\in\mathbb{N}}=X$ since
            $\mathcal{C}_{n}\subseteq\mathcal{U}_{n}$,
            $\mathcal{U}_{n}\subseteq\mathcal{W}_{n}$, and
            $\mathcal{W}_{n}\subseteq{K}_{n}$. Since the $\mathcal{C}_{n}$
            cover $X$, so do the $K_{n}$. Hence, $(X,\,\tau)$ is
            compactly exhaustible.
        \end{proof}
        \begin{theorem}
            If $(X,\,\tau)$ is compactly exhaustible and Hausdorff,
            then it is paracompact.
        \end{theorem}
        \begin{proof}
            Let $K:\mathbb{N}\rightarrow\mathcal{P}(X)$ be such that
            for all $n\in\mathbb{N}$ $K_{n}$ is compact,
            $K_{n}\subseteq\textrm{Int}_{\tau}(K_{n+1})$, and
            $\bigcup_{n\in\mathbb{N}}K_{n}=X$. Note that, since
            $\textrm{Int}_{\tau}(K_{n})\subseteq{K}_{n+1}$ is open, and
            $K_{n+1}$ is compact, $K_{n+1}\setminus\textrm{Int}_{\tau}(K_{n})$
            is compact. Let $\mathcal{O}$ be an open cover. We must find a
            locally finite open refinement $\mathcal{X}$ of $\mathcal{O}$.
            But $\mathcal{O}$ covers $X$, so it covers
            $K_{n+1}\setminus\textrm{Int}_{\tau}(K_{n})$. By compactness there
            are finitely many $\mathcal{V}_{0},\,\dots,\,\mathcal{V}_{n_{N}}$
            that cover $K_{n+1}\setminus\textrm{Int}_{\tau}(K_{n})$.
            Define $\Delta_{n}$ via:
            \begin{equation}
                \Delta_{n}=
                \Big\{\,\mathcal{V}_{k}\cap\big(\textrm{Int}_{\tau}(K_{n+2})
                    \setminus{K}_{n-1}\big)\;|\;0\leq{k}\leq{N}_{n}\Big\}
            \end{equation}
            But $(X,\,\tau)$ is Hausdorff, so each $K_{n}$ is closed, hence
            $\textrm{Int}_{\tau}(K_{n+1})\setminus{K}_{n-1}$ is open, meaning
            all elements of $\Delta$ are open. The set
            $\mathcal{X}=\bigcup_{n\in\mathbb{N}}\Delta_{n}$ is a locally finite
            open refinement. It is an open refinement, the elements are open
            and are contained as subsets of the elements of $\mathcal{O}$ by
            construction. It is also an open cover since it covers each
            $K_{n}$, and the $K_{n}$ cover $X$. Lastly, it is locally finite.
            Every element of $X$ is contained in some
            $\textrm{Int}_{\tau}(K_{n+1})\setminus{K}_{n-1}$ for some $n$,
            and $\mathcal{X}$ has only finitely many elements with non-empty
            intersection with this sets, the elements of $\Delta_{n}$. So
            $\mathcal{X}$ is a locally finite open refinement of $\mathcal{O}$
            that covers $X$, so $(X,\,\tau)$ is paracompact.
        \end{proof}
        \begin{theorem}
            If $(X,\,\tau)$ is paracompact, and if $\mathcal{C}\subseteq{X}$ is
            closed, then $(\mathcal{C},\,\tau_{\mathcal{C}})$ is paracompact
            where $\tau_{\mathcal{C}}$ is the subspace topology.
        \end{theorem}
        \begin{proof}
            The proof is a mimicry of the idea for compact spaces. Given an
            open cover $\mathcal{O}$ of $\mathcal{C}$, we extend it to an open
            cover $\tilde{\mathcal{O}}$ of $X$ via
            $\tilde{\mathcal{O}}=\mathcal{O}\cup\{\,X\setminus\mathcal{C}\,\}$.
            Using paracompact of $(X,\,\tau)$ we get a locally finite open
            refinement $\tilde{\mathcal{X}}$ that covers $X$. We restrict these
            sets to $\mathcal{C}$ to obtain a locally finite open refinite
            $\mathcal{X}$ of $\mathcal{O}$ that cover $\mathcal{C}$.
        \end{proof}
        \begin{theorem}
            If $(X,\,\tau)$ is a topological space, and if
            $\mathcal{A}\subseteq\mathcal{P}(X)$ is locally finite, then the
            set:
            \begin{equation}
                \mathcal{A}'=\{\,\textrm{Cl}_{\tau}(A)\;|\;A\in\mathcal{A}\,\}
            \end{equation}
            is locally finite as well
        \end{theorem}
        \begin{proof}
            Homework.
        \end{proof}
        \begin{theorem}
            If $(X,\,\tau)$ is a topological space, and if
            $\mathcal{A}\subseteq\mathcal{P}(X)$ is locally finite, then:
            \begin{equation}
                \textrm{Cl}_{\tau}\Big(\bigcup_{A\in\mathcal{A}}A\big)
                =\bigcup_{A\in\mathcal{A}}\textrm{Cl}_{\tau}(A)
            \end{equation}
        \end{theorem}
        \begin{proof}
            Also homework.
        \end{proof}
        \begin{theorem}
            If $(X,\,\tau)$ is paracompact and Hausdorff, then it is regular.
        \end{theorem}
        \begin{proof}
            For let $x\in{X}$, $\mathcal{C}\subseteq{X}$ be closed, and
            $x\notin\mathcal{C}$. Since $(X,\,\tau)$ is Hausdorff, for all
            $y\in\mathcal{C}$ there are $\mathcal{U}_{y},\mathcal{V}_{y}\in\tau$
            such that $x\in\mathcal{U}_{y}$, $y\in\mathcal{V}_{y}$, and
            $\mathcal{U}_{y}\cap\mathcal{V}_{y}=\emptyset$. But then:
            \begin{equation}
                \mathcal{O}=\{\,\mathcal{V}_{y}\;|\;y\in\mathcal{C}\,\}
                \cup\{\,X\setminus\mathcal{C}\,\}
            \end{equation}
            is an open cover of $X$. But $(X,\,\tau)$ is paracompact, so there
            is a locally finite open refinement $\mathcal{X}$ that covers $X$.
            Let $\Delta\subseteq\mathcal{X}$ be the set of all elements with
            non-empty intersection with $\mathcal{C}$. Since $\mathcal{X}$
            covers $X$, $\Delta$ covers $\mathcal{C}$. But by the definition
            of $\mathcal{O}$, since $\mathcal{X}$ is a refinement of
            $\mathcal{O}$, for all $\mathcal{W}\in\mathcal{X}$ there is a
            $\mathcal{V}_{y}\in\mathcal{O}$ such that
            $\mathcal{W}\subseteq\mathcal{V}_{y}$, or
            $\mathcal{W}\subseteq{X}\setminus\mathcal{C}$. Hence all elements
            of $\Delta$ are subsets of $\mathcal{V}_{y}$ for some
            $y\in\mathcal{C}$. But then, since $x\in\mathcal{U}_{y}$ and
            $\mathcal{U}_{y}\cap\mathcal{V}_{y}=\emptyset$, we have
            $x\notin\textrm{Cl}_{\tau}(\mathcal{V}_{y})$. But
            $\Delta$ is locally finite, being a subset of $\mathcal{X}$, hence:
            \begin{equation}
                \textrm{Cl}_{\tau}\Big(\bigcup_{A\in\Delta}A\Big)
                =\bigcup_{A\in\Delta}\textrm{Cl}_{\tau}(A)
            \end{equation} 
            And hence $x\notin\textrm{Cl}_{\tau}\big(\bigcup_{A\in\Delta}A\big)$.
            Let $\mathcal{U}=X\setminus\textrm{Cl}_{\tau}\Big(\bigcup_{A\in\Delta}A\Big)$
            and $\mathcal{V}=\bigcup\Delta$. Then $\mathcal{U}$ and $\mathcal{V}$
            are open and disjoint, $x\in\mathcal{U}$, and
            $\mathcal{C}\subseteq\mathcal{V}$. Hence, $(X,\,\tau)$ is regular.
        \end{proof}
        \begin{theorem}[\textbf{Dieudonne's Theorem}]
            If $(X,\,\tau)$ is paracompact and Hausdorff, then it is normal.
        \end{theorem}
        \begin{proof}
            We apply the same idea as before. Since $(X,\,\tau)$ is paracompact
            and Hausdorff, it is regular. Given two closed disjoint sets
            $\mathcal{C},\mathcal{D}\subseteq{X}$ for all
            $x\in\mathcal{C}$ we find $\mathcal{U}_{x},\mathcal{V}_{x}\in\tau$
            such that $x\in\mathcal{U}_{x}$,
            $\mathcal{D}\subseteq\mathcal{V}_{x}$, and
            $\mathcal{U}_{x}\cap\mathcal{V}_{x}=\emptyset$. We use
            paracompactness and apply a similar argument to the previous theorem
            to prove normality.
        \end{proof}
        We now take the steps towards proving the two main metrization theorems.
        The Nagata-Smirnov theorem, and the Smirnov theorem. The Smirnov theorem
        uses paracompactness to characterize metrizable spaces, the
        Nagata-Smirnov theorem uses a very similar idea as the Urysohn
        metrization theorem. Urysohn's theorem said a regular Hausdorff space
        that is second countable is metrizable. All metrizable spaces are
        regular and Hausdorff, so this can not be omitted, however the
        second countability can be weakened. The idea that is required for
        metrizability is $\sigma$ locally finite bases.
        \begin{definition}[\textbf{$\sigma$ Locally Finite Open Cover}]
            A $\sigma$ locally finite open cover of a topological space
            $(X,\,\tau)$ is an open cover $\mathcal{O}\subseteq\tau$ such that
            there exists countably many locally finite collections
            $\Delta_{n}\subseteq\tau$ such that
            $\mathcal{O}=\bigcup_{n\in\mathbb{N}}\Delta_{n}$.
        \end{definition}
        This idea is used to prove Stone's paracompactness theorem, which
        states that every metrizable space is paracompact. The following lemma
        is needed for this theorem.
        \begin{theorem}
            If $(X,\,\tau)$ is regular, and if every open cover
            $\mathcal{O}\subseteq\tau$ has a refinement $\mathcal{X}$ that is
            a $\sigma$ locally finite open cover, then $(X,\,\tau)$ is
            paracompact.
        \end{theorem}
        \begin{theorem}
            If $(X,\,\tau)$ is regular and Lindel\"{o}f, then it is
            paracompact.
        \end{theorem}
        \begin{theorem}[\textbf{Stone's Paracompactness Theorem}]
            If $(X\,\tau)$ is metrizable, then it is paracompact.
        \end{theorem}
        \begin{definition}[\textbf{Partition of Unity}]
        \end{definition}
\end{document}
