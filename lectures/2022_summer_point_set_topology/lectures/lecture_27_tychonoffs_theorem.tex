%-----------------------------------LICENSE------------------------------------%
%   This file is part of Mathematics-and-Physics.                              %
%                                                                              %
%   Mathematics-and-Physics is free software: you can redistribute it and/or   %
%   modify it under the terms of the GNU General Public License as             %
%   published by the Free Software Foundation, either version 3 of the         %
%   License, or (at your option) any later version.                            %
%                                                                              %
%   Mathematics-and-Physics is distributed in the hope that it will be useful, %
%   but WITHOUT ANY WARRANTY; without even the implied warranty of             %
%   MERCHANTABILITY or FITNESS FOR A PARTICULAR PURPOSE.  See the              %
%   GNU General Public License for more details.                               %
%                                                                              %
%   You should have received a copy of the GNU General Public License along    %
%   with Mathematics-and-Physics.  If not, see <https://www.gnu.org/licenses/>.%
%------------------------------------------------------------------------------%
\documentclass{article}
\usepackage{graphicx}                           % Needed for figures.
\usepackage{amsmath}                            % Needed for align.
\usepackage{amssymb}                            % Needed for mathbb.
\usepackage{amsthm}                             % For the theorem environment.
\usepackage{float}
\usepackage{tabularx, booktabs}
\usepackage{mathrsfs}
\usepackage[font=scriptsize,
            labelformat=simple,
            labelsep=colon]{subcaption} % Subfigure captions.
\usepackage[font={scriptsize},
            hypcap=true,
            labelsep=colon]{caption}    % Figure captions.
\usepackage{hyperref}
\hypersetup{
    colorlinks=true,
    linkcolor=blue,
    filecolor=magenta,
    urlcolor=Cerulean,
    citecolor=SkyBlue
}

%------------------------Theorem Styles-------------------------%
\theoremstyle{plain}
\newtheorem{theorem}{Theorem}[section]

% Define theorem style for default spacing and normal font.
\newtheoremstyle{normal}
    {\topsep}               % Amount of space above the theorem.
    {\topsep}               % Amount of space below the theorem.
    {}                      % Font used for body of theorem.
    {}                      % Measure of space to indent.
    {\bfseries}             % Font of the header of the theorem.
    {}                      % Punctuation between head and body.
    {.5em}                  % Space after theorem head.
    {}

% Define default environments.
\theoremstyle{normal}
\newtheorem{examplex}{Example}[section]
\newtheorem{definitionx}{Definition}[section]

\newenvironment{example}{%
    \pushQED{\qed}\renewcommand{\qedsymbol}{$\blacksquare$}\examplex%
}{%
    \popQED\endexamplex%
}

\newenvironment{definition}{%
    \pushQED{\qed}\renewcommand{\qedsymbol}{$\blacksquare$}\definitionx%
}{%
    \popQED\enddefinitionx%
}

\title{Point-Set Topology: Lecture 27}
\author{Ryan Maguire}
\date{Summer 2022}

% No indent and no paragraph skip.
\setlength{\parindent}{0em}
\setlength{\parskip}{0em}

\begin{document}
    \maketitle
    \section{Tychonoff's Theorem}
        We conclude our study of compactness and paracompactness with
        Tychonoff's theorem which states that the product of any collection
        $(X_{\alpha},\,\tau_{\alpha})$, $\alpha\in{I}$, of
        compact spaces, equipped with the product topology, is still compact.
        Note this is not true if the product is given the box topology. The
        definition of the product topology is crucial for this theorem. In part
        because the product topology has as a subbasis the collection of all
        sets $\prod_{\alpha\in{I}}\mathcal{U}_{\alpha}$ where
        $\mathcal{U}_{\alpha}=X_{\alpha}$ for all but at most one
        $\alpha\in{I}$.
        \par\hfill\par
        The proof requires, in some form, the axiom of choice.
        Indeed, Tychonoff's theorem is equivalent to the axiom of choice, given
        the other axioms of set theory. The easiest presentation of the
        theorem is a quick corollary of Alexander's subbasis theorem, which
        uses Zorn's lemma in it's proof. Zorn's lemma, which is also equivalent
        to the axiom of choice, says the following:
        \begin{theorem}[\textbf{Zorn's Lemma}]
            If $(X,\,\leq)$ is a partially ordered set, meaning
            $\leq$ is transitive, anti-symmetric, and symmetric, such that
            for every chain $A\subseteq{X}$, which is a subset such that
            $(A,\,\leq_{A})$ is totally ordered, is bounded, then there is a
            maximal element $a\in{X}$ which is an element such that for all
            $b\in{X}$, $a\leq{b}$ implies $b=a$.
        \end{theorem}
        With this, one can prove Alexander's theorem.
        \begin{theorem}[\textbf{Alexander's Subbasis Theorem}]
            If $(X,\,\tau)$ is a topological space, and if
            $\mathcal{B}\subseteq\tau$ is a subbasis of $\tau$ that covers $X$
            such that for all open covers $\mathcal{O}\subseteq\mathcal{B}$
            there is a finite subcover $\Delta\subseteq\mathcal{O}$, then
            $(X,\,\tau)$ is compact.
        \end{theorem}
        Note the formulation of this theorem is not quite the definition of
        compactness. Compactness says you need to look at every open cover and
        find a finite subcover. The theorem says it is sufficient to look at
        open covers that consist only of elements from the subbasis
        $\mathcal{B}$.
        \begin{theorem}[\textbf{Tychonoff's Theorem}]
            If $I$ is a set such that for all $\alpha\in{I}$ the ordered pair
            $(X_{\alpha},\,\tau_{\alpha})$ is a compact topological space,
            then $(\prod_{\alpha\in{I}}X_{\alpha},\,\tau_{\prod})$ is
            compact where $\tau_{\prod}$ is the product topology.
        \end{theorem}
\end{document}
