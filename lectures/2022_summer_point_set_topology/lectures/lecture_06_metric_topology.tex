%-----------------------------------LICENSE------------------------------------%
%   This file is part of Mathematics-and-Physics.                              %
%                                                                              %
%   Mathematics-and-Physics is free software: you can redistribute it and/or   %
%   modify it under the terms of the GNU General Public License as             %
%   published by the Free Software Foundation, either version 3 of the         %
%   License, or (at your option) any later version.                            %
%                                                                              %
%   Mathematics-and-Physics is distributed in the hope that it will be useful, %
%   but WITHOUT ANY WARRANTY; without even the implied warranty of             %
%   MERCHANTABILITY or FITNESS FOR A PARTICULAR PURPOSE.  See the              %
%   GNU General Public License for more details.                               %
%                                                                              %
%   You should have received a copy of the GNU General Public License along    %
%   with Mathematics-and-Physics.  If not, see <https://www.gnu.org/licenses/>.%
%------------------------------------------------------------------------------%
\documentclass{article}
\usepackage{graphicx}                           % Needed for figures.
\usepackage{amsmath}                            % Needed for align.
\usepackage{amssymb}                            % Needed for mathbb.
\usepackage{amsthm}                             % For the theorem environment.
\usepackage{float}
\usepackage{hyperref}
\hypersetup{
    colorlinks=true,
    linkcolor=blue,
    filecolor=magenta,
    urlcolor=Cerulean,
    citecolor=SkyBlue
}

%------------------------Theorem Styles-------------------------%
\theoremstyle{plain}
\newtheorem{theorem}{Theorem}[section]

% Define theorem style for default spacing and normal font.
\newtheoremstyle{normal}
    {\topsep}               % Amount of space above the theorem.
    {\topsep}               % Amount of space below the theorem.
    {}                      % Font used for body of theorem.
    {}                      % Measure of space to indent.
    {\bfseries}             % Font of the header of the theorem.
    {}                      % Punctuation between head and body.
    {.5em}                  % Space after theorem head.
    {}

% Define default environments.
\theoremstyle{normal}
\newtheorem{examplex}{Example}[section]
\newtheorem{definitionx}{Definition}[section]
\newtheorem{notationx}{Notation}[section]
\newtheorem{axiomx}{Axiom}[section]

\newenvironment{example}{%
    \pushQED{\qed}\renewcommand{\qedsymbol}{$\blacksquare$}\examplex%
}{%
    \popQED\endexamplex%
}

\newenvironment{definition}{%
    \pushQED{\qed}\renewcommand{\qedsymbol}{$\blacksquare$}\definitionx%
}{%
    \popQED\enddefinitionx%
}

\title{Point-Set Topology: Lecture 6}
\author{Ryan Maguire}
\date{Summer 2022}

% No indent and no paragraph skip.
\setlength{\parindent}{0em}
\setlength{\parskip}{0em}

\begin{document}
    \maketitle
    \section{More on Open and Closed}
        Sets are not doors. If $A\subseteq{X}$ is a subset of a metric space
        $(X,\,d)$ it does not need to be true that $A$ is either closed or
        open. $A$ can be open, $A$ can be closed, $A$ can be neither open nor
        closed, and $A$ can be both open and closed.
        \begin{example}
            If $(X,\,d)$ is any metric space, then both
            $\emptyset$ and $X$ are open and closed, simultaneously.
        \end{example}
        \begin{example}
            In the real line $\mathbb{R}$ with the standard metric
            $d(x,\,y)=|x-y|$ there are no proper subsets that are both open and
            closed. Open sets are formed by open intervals $(a,\,b)$ and the
            union of such sets. Closed sets are the complements of these sets.
            Examples include closed intervals $[a,\,b]$, single points
            $\{\,x\,\}$, and finite subsets of the real line. There are many
            other examples not of this form. The rationals $\mathbb{Q}$ are
            neither closed nor open. It is not open since $(a,\,b)$ always
            contains irrational numbers for $a<b$, and hence can't be a
            subset of $\mathbb{Q}$. It is not closed since every irrational can
            be approximated by a sequence of rationals.
        \end{example}
        \begin{example}
            Let $X$ be a non-empty set and let $d$ be the discrete metric:
            \begin{equation}
                d(x,\,y)=
                \begin{cases}
                    1&x\ne{y}\\
                    0&x=y
                \end{cases}
            \end{equation}
            Then \textit{every} subset of $X$ is open, and since the complements
            of open sets are closed, every set is also closed. To see that every
            set is open, note that $\{\,x\,\}$ is open for all $x\in{X}$.
            Let $r=\frac{1}{2}$. Then
            $y\in{B}_{r}^{(X,\,d)}(x)$ means $d(x,\,y)<\frac{1}{2}$. But the
            only element that does this is $x$, so $\{\,x\,\}$ is open.
            Given $A\subseteq{X}$ we can write:
            \begin{equation}
                A=\bigcup_{x\in{A}}\{\,x\,\}
            \end{equation}
            $A$ being the union of open sets, and is thus open.
        \end{example}
        \begin{theorem}
            If $(X,\,d)$ is a metric space, then $\mathcal{U}\subseteq{X}$ is
            open if and only if for every convergent sequence
            $a:\mathbb{N}\rightarrow{X}$ with limit $x\in\mathcal{U}$ it is
            true that there is an $N\in\mathbb{N}$ such that for all
            $n\in\mathbb{N}$ with $n>N$ it is true that $a_{n}\in\mathcal{U}$.
        \end{theorem}
        \begin{proof}
            Suppose $\mathcal{U}$ is open and $a:\mathbb{N}\rightarrow{X}$
            converges to $x\in\mathcal{U}$. Since $x\in\mathcal{U}$ and
            $\mathcal{U}$ is open, there is an $\varepsilon>0$ such that
            $B_{\varepsilon}^{(X,\,d)}(x)\subseteq\mathcal{U}$. But
            $a_{n}\rightarrow{x}$ and hence there is an $N\in\mathbb{N}$
            such that $n\in\mathbb{N}$ and $n>N$ implies
            $d(x,\,a_{n})<\varepsilon$. Thus $n>N$ implies
            $a_{n}\in{B}_{\varepsilon}^{(X,\,d)}(x)$, so
            $a_{n}\in\mathcal{U}$. Now, suppose for every convergent sequence
            $a:\mathbb{N}\rightarrow{X}$ with limit $x\in\mathcal{U}$ there is
            an $N\in\mathbb{N}$ with $n\in\mathbb{N}$ and $n>N$ implying
            $a_{n}\in\mathcal{U}$. Suppose $\mathcal{U}$ is not open.
            Then there is an $x\in\mathcal{U}$ such that for all
            $\varepsilon>0$ there is a point $y\in{X}$ such that
            $d(x,\,y)<\varepsilon$ but $y\notin\mathcal{U}$. In particular,
            for all $n\in\mathbb{N}$ there is an $a_{n}$ such that
            $d(x,\,a_{n})<\frac{1}{n+1}$ but $a_{n}\notin\mathcal{U}$.
            Then $a_{n}\rightarrow{x}$. But $x\in\mathcal{U}$ and hence there
            is an $N\in\mathbb{N}$ such that $n\in\mathbb{N}$ and $n>N$ implies
            $a_{n}\in\mathcal{U}$. But $a_{n}\notin\mathcal{U}$, a
            contradiction. Therefore, $\mathcal{U}$ is open.
        \end{proof}
    \section{Subspaces}
        Many new metric spaces are formed by considering metric spaces we've
        already constructed and examining subsets of these spaces.
        \begin{theorem}
            If $(X,\,d)$ is a metric space, if $A\subseteq{X}$, and if
            $d_{A}$ is the restriction of $d$ to $A\times{A}$, then
            $(A,\,d_{A})$ is a metric space.
        \end{theorem}
        \begin{proof}
            For all $a,b,c\in{A}$ it is true that $a,b,c\in{X}$ since
            $A\subseteq{X}$. Hence $d_{A}(a,\,b)=d(a,\,b)$ and so
            $d_{A}$ is positive-definite, symmetric, and satisfies the
            triangle inequality. Thus, $(A,\,d_{A})$ is a metric space.
        \end{proof}
        \begin{definition}[\textbf{Metric Subspace}]
            A metric subspace of a metric space $(X,\,d)$ is the metric space
            $(A,\,d_{A})$ where $A\subseteq{X}$ and $d_{A}$ is the restriction
            of $d$ to $A\times{A}$.
        \end{definition}
        \begin{example}
            Equip $\mathbb{R}^{2}$ with the Euclidean metric and define
            $\mathbb{S}^{1}$ to be the set of all $\mathbf{x}\in\mathbb{R}^{2}$
            such that $||\mathbf{x}||_{2}=1$. This is the unit circle in the
            plane. We turn this into a metric space by equipping it with the
            subspace metric from $\mathbb{R}^{2}$. This is the standard metric
            on $\mathbb{S}^{1}$. This is shown in
            Fig.~\ref{fig:circle_as_subspace_of_r2}.
        \end{example}
        \begin{example}
            The unit sphere is defined similarly as a subspace of
            $\mathbb{R}^{2}$. It is the set of all points
            $\mathbf{x}\in\mathbb{R}^{3}$ such that $||\mathbf{x}||_{2}=1$.
            In general the $n$ dimensional sphere $\mathbb{S}^{n}$ is the
            subset of all points in $\mathbb{R}^{n+1}$ such that
            $||\mathbf{x}||_{2}=1$.
        \end{example}
        \begin{figure}
            \centering
            \includegraphics{../../../images/circle_as_subspace_of_r2.pdf}
            \caption{$\mathbb{S}^{1}$ as a Subspace of $\mathbb{R}^{2}$}
            \label{fig:circle_as_subspace_of_r2}
        \end{figure}
        If $(X,\,d)$ is a metric space, if $(A,\,d_{A})$ is a metric subspace,
        and if $\mathcal{U}\subseteq{A}$ is open with respect to $d_{A}$, it
        need not be true that $\mathcal{U}$ is open with respect to $d$.
        For example, the real line can be viewed as a subset of the plane
        by identifying $x\in\mathbb{R}$ with $(x,\,0)\in\mathbb{R}^{2}$. An
        open subset of the real line is a interval, but open subsets in the
        plane are blobs (\textit{two dimensional}). What is true is the
        following.
        \begin{theorem}
            If $(X,\,d)$ is a metric space, if $A\subseteq{X}$ and
            $(A,\,d_{A})$ is a metric subspace, then
            $\mathcal{U}\subseteq{A}$ is open with respect to $d_{A}$ if and
            only if there is an open subset $\mathcal{V}\subseteq{X}$ with
            respect to $d$ such that
            $\mathcal{U}=\mathcal{V}\cap{A}$.
        \end{theorem}
        \begin{proof}
            If $\mathcal{U}$ is open in $A$, then for all
            $x\in\mathcal{U}$ there is an $r_{x}>0$ such that
            $B_{r_{x}}^{(A,\,d_{A})}(x)\subseteq\mathcal{U}$. Define
            $\mathcal{O}$ by:
            \begin{equation}
                \mathcal{O}=\{\,B_{r_{x}}^{(X,\,d)}(x)\;|\;x\in\mathcal{U}\,\}
            \end{equation}
            Note that these open balls are open balls in $(X,\,d)$, not
            $(A,\,d_{A})$. Let $\mathcal{V}=\bigcup\mathcal{O}$. Since
            $\mathcal{V}$ is the union of open balls, it is open in $X$.
            If $x\in\mathcal{U}$, then $x\in{A}$ since
            $\mathcal{U}\subseteq{A}$, and also
            $x\in\mathcal{V}$ since $x\in{B}_{r_{x}}^{(X,\,d)}(x)$ which is
            a subset of $\bigcup\mathcal{O}$. Hence
            $x\in\mathcal{V}\cap{A}$, and therefore
            $\mathcal{U}\subseteq\mathcal{V}\cap{A}$. Next, suppose
            $x\in\mathcal{V}\cap{A}$. Then $x\in\mathcal{V}$ and hence there
            is a $y\in\mathcal{U}$ such that
            $x\in{B}_{r_{y}}^{(X,\,d)}(y)$. But $y\in{A}$, and hence
            $d(x,\,y)=d_{A}(x,\,y)$, and so
            $x\in{B}_{r_{y}}^{(A,\,d_{A})}(y)$, and thus
            $x\in\mathcal{U}$. That is,
            $\mathcal{V}\cap{A}\subseteq\mathcal{U}$, and therefore
            $\mathcal{U}=\mathcal{V}\cap{A}$. Now suppose
            $\mathcal{U}=\mathcal{V}\cap{A}$ where $\mathcal{V}$ is open in
            $X$. Let $x\in\mathcal{U}$. Then $x\in\mathcal{V}\cap{A}$, and
            hence $x\in\mathcal{V}$. But $\mathcal{V}$ is open, and therefore
            there is an $r>0$ such that
            $B_{r}^{(X,\,d)}(x)\subseteq\mathcal{V}$. Now given
            $y\in{B}_{r}^{(A,\,d_{A})}(x)$, by definition $y\in{A}$ and
            $d(x,\,y)=d_{A}(x,\,y)$. Therefore
            $y\in\mathcal{V}\cap{A}$. But $\mathcal{U}=\mathcal{V}\cap{A}$,
            so $y\in\mathcal{U}$. Since $y\in{B}_{r}^{(A,\,d_{A})}(x)$ is
            arbitrary, we have that
            $B_{r}^{(A,\,d_{A})}(x)\subseteq\mathcal{U}$, so
            $\mathcal{U}$ is open in $A$.
        \end{proof}
        \begin{example}
            An \textit{open arc} in the circle, an arc that does not include
            the endpoints, is open. See Fig.~\ref{fig:open_subset_of_circle_001}.
        \end{example}
        \begin{figure}
            \centering
            \includegraphics{../../../images/open_subset_of_circle_001.pdf}
            \caption{An Open Subset of $\mathbb{S}^{1}$}
            \label{fig:open_subset_of_circle_001}
        \end{figure}
    \section{Continuity}
        Like most branches of mathematics, there's a notion of
        \textit{identical} metric spaces, or \textit{equivalent} metric spaces.
        The only thing that defines a metric is the set $X$ and the metric $d$.
        If we relabel the points in $X$, forming a new set $Y$, but preserve
        the distances, then we haven't really changed the metric space.
        For example, we can define on $\mathbb{R}^{2}$ the metric
        $d_{\mathbb{R}^{2}}(\mathbf{x},\,\mathbf{y})$ to be the Euclidean
        metric,
        $d_{\mathbb{R}^{2}}(\mathbf{\mathbf{x}},\,\mathbf{y})=||\mathbf{x}-\mathbf{y}||_{2}$.
        We can relabel $(x,\,y)=x+iy$ and call this the \textit{complex plane},
        denote it $\mathbb{C}$, and given $z=x_{0}+ix_{1}$ and
        $w=y_{0}+iy_{1}$, we could define $d_{\mathbb{C}}(z,\,w)=|z-w|$ where
        $|\cdot|$ is the complex absolute value. This is
        no different from the Euclidean metric, all we've done is relabel
        everything. We use this to motivate isometries.
        \begin{definition}[\textbf{Metric Space Isometry}]
            A metric space isometry from a metric space
            $(X,\,d_{X})$ to a metric space $(Y,\,d_{Y})$ is a function
            $f:X\rightarrow{Y}$ such that for all $x_{0},x_{1}\in{X}$ it
            is true that:
            \begin{equation}
                d_{X}(x_{0},\,x_{1})=d_{Y}\big(f(x_{0}),\,f(x_{1})\big)
            \end{equation}
            That is, $f$ preserves the metrics.
        \end{definition}
        \begin{example}
            In Euclidean geometry one often ponders isometries of the plane
            to itself. It is a classic result that all isometries
            $f:\mathbb{R}^{2}\rightarrow\mathbb{R}^{2}$ are combinations
            of \textit{translations}, \textit{reflections}, and
            \textit{rotations} (The so-called \textit{glide-reflections}
            are reflections followed by translations). A translation
            is a function $f:\mathbb{R}^{2}\rightarrow\mathbb{R}^{2}$
            defined by $f(\mathbf{x})=\mathbf{x}+\mathbf{y}$ for some
            fixed $\mathbf{y}\in\mathbb{R}^{2}$. Reflections are functions that
            \textit{flip} bases. For example, a reflection about the $y$ axis
            is the function $f(\mathbf{x})=(-x_{0},\,x_{1})$. A reflection
            about the $x$ axis is of the form
            $f(\mathbf{x})=(x_{0},\,-x_{1})$. Reflections about other lines
            through the origin can be defined similarly. Lastly, rotation
            by an angle $\theta$ is the function defined by:
            \begin{align}
                f(\mathbf{x})&=
                \begin{bmatrix}
                    \cos(\theta)&\sin(\theta)\\
                    -\sin(\theta)&\cos(\theta)
                \end{bmatrix}
                \begin{bmatrix}
                    x_{0}\\
                    x_{1}
                \end{bmatrix}\\
                &=(\cos(\theta)x_{0}+\sin(\theta)x_{1},\,
                    -\sin(\theta)x_{0}+\cos(\theta)x_{1})
            \end{align}
            An isometry $f:\mathbb{R}^{2}\rightarrow\mathbb{R}^{2}$ is a
            combination of these operations.
        \end{example}
        \begin{theorem}
            If $(X,\,d_{X})$ and $(Y,\,d_{Y})$ are metric spaces, and if
            $f:X\rightarrow{Y}$ is a metric space isometry, then
            $f$ is injective.
        \end{theorem}
        \begin{proof}
            For if $x_{0}\ne{x}_{1}$, then $d_{X}(x_{0},\,x_{1})>0$, and thus
            $d_{Y}\big(f(x_{0}),\,f(x_{1})\big)=d_{X}(x_{0},\,x_{1})>0$,
            so $f(x_{0})\ne{f}(x_{1})$.
        \end{proof}
        \begin{definition}[\textbf{Global Metric Space Isometry}]
            A global metric space isometry from a metric space
            $(X,\,d_{X})$ to a metric space $(Y,\,d_{Y})$ is a bijective
            metric space isometry $f:X\rightarrow{Y}$.
        \end{definition}
        Global isometries mean $(X,\,d_{X})$ and $(Y,\,d_{Y})$ are, in a sense,
        the \textit{same} metric space. Just like $\mathbb{R}^{2}$ and
        $\mathbb{C}$ can be thought of as the same, so can
        $X$ and $Y$ by identifying $x\in{X}$ with $f(x)\in{Y}$ and
        $y\in{Y}$ with $f^{-1}(y)\in{X}$.
        \par\hfill\par
        There is a much weaker notion than isometries for metric spaces, and
        this notion is far more useful.
        \begin{definition}[\textbf{Continuous Function Between Metric Spaces}]
            A continuous function from a metric space $(X,\,d_{X})$ to a
            metric space $(Y,\,d_{Y})$ is a function
            $f:X\rightarrow{Y}$ such that for every convergent sequence
            $a:\mathbb{N}\rightarrow{X}$ with limit $x\in{X}$ it is true that
            $f(a_{n})\rightarrow{f}(x)$.
        \end{definition}
        Using notation from calculus, this says that:
        \begin{equation}
            \lim_{n\rightarrow\infty}f(a_{n})=
                f\big(\lim_{n\rightarrow\infty}a_{n}\big)=f(x)
        \end{equation}
        \begin{theorem}
            If $(X,\,d_{X})$ and $(Y,\,d_{Y})$ are metric spaces, and if
            $f:X\rightarrow{Y}$ is a function, then $f$ is continuous if and
            only if for all open $\mathcal{V}\subseteq{Y}$ it is true that
            $f^{-1}[\mathcal{V}]\subseteq{X}$ is open.
        \end{theorem}
        \begin{proof}
            Suppose $f$ is continuous and $\mathcal{V}\subseteq{Y}$ is open.
            Let $\mathcal{U}=f^{-1}[\mathcal{V}]$. Suppose
            $a:\mathbb{N}\rightarrow{X}$ is such that
            $a_{n}\rightarrow{x}$ with $x\in\mathcal{U}$. Since
            $a_{n}\rightarrow{x}$ and $f$ is continuous,
            $f(a_{n})\rightarrow{f}(x)$. But since $x\in{f}^{-1}[\mathcal{V}]$
            it is true that $f(x)\in\mathcal{V}$. But $\mathcal{V}$ is open, so
            there exists an $N\in\mathbb{N}$ such that $n\in\mathbb{N}$ and
            $n>N$ implies $f(a_{n})\in\mathcal{V}$. But then for all
            $n>N$, $a_{n}\in\mathcal{U}=f^{-1}[\mathcal{V}]$, so
            $\mathcal{U}$ is open. Now, suppose $f:X\rightarrow{Y}$ is such
            that for all open $\mathcal{V}\subseteq{Y}$ it is true that
            $f^{-1}[\mathcal{V}]$ is open. Let
            $a:\mathbb{N}\rightarrow{X}$ be a convergent sequence that
            converges to $x\in{X}$. Suppose
            $f(a_{n})$ does not converge to $f(x)$. Then there exists an
            $\varepsilon>0$ such that for all $N\in\mathbb{N}$ there is an
            $n\in\mathbb{N}$ with $n>N$ but
            $d_{Y}\big(f(x),\,f(a_{n})\big)\geq\varepsilon$. But
            $B_{\varepsilon}^{(Y,\,d_{Y})}\big(f(x)\big)$ is open, so by
            assumption the pre-image is open. Letting
            $\mathcal{V}=B_{\varepsilon}^{(Y,\,d_{Y})}\big(f(x)\big)$, we have
            that $f^{-1}[\mathcal{V}]$ is open. But $x\in{f}^{-1}[\mathcal{V}]$
            since $f(x)\in\mathcal{V}$. Since $f^{-1}[\mathcal{V}]$ is open
            there is an $N\in\mathbb{N}$ such that $n\in\mathbb{N}$ and
            $n>N$ implies $a_{n}\in{f}^{-1}[\mathcal{V}]$. But then
            $f(a_{n})\in\mathcal{V}$ for all $n>N$ which is a contradiction.
            Therefore, $f(a_{n})\rightarrow{f}(x)$ and $f$ is continuous.
        \end{proof}
\end{document}
