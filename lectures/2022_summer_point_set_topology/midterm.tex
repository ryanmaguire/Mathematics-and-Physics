%-----------------------------------LICENSE------------------------------------%
%   This file is part of Mathematics-and-Physics.                              %
%                                                                              %
%   Mathematics-and-Physics is free software: you can redistribute it and/or   %
%   modify it under the terms of the GNU General Public License as             %
%   published by the Free Software Foundation, either version 3 of the         %
%   License, or (at your option) any later version.                            %
%                                                                              %
%   Mathematics-and-Physics is distributed in the hope that it will be useful, %
%   but WITHOUT ANY WARRANTY; without even the implied warranty of             %
%   MERCHANTABILITY or FITNESS FOR A PARTICULAR PURPOSE.  See the              %
%   GNU General Public License for more details.                               %
%                                                                              %
%   You should have received a copy of the GNU General Public License along    %
%   with Mathematics-and-Physics.  If not, see <https://www.gnu.org/licenses/>.%
%------------------------------------------------------------------------------%
\documentclass{article}
\usepackage{graphicx}                           % Needed for figures.
\usepackage{amsmath}                            % Needed for align.
\usepackage{amssymb}                            % Needed for mathbb.
\usepackage{amsthm}                             % For the theorem environment.
\usepackage{float}
\usepackage{hyperref}
\hypersetup{
    colorlinks=true,
    linkcolor=blue,
    filecolor=magenta,
    urlcolor=Cerulean,
    citecolor=SkyBlue
}

%------------------------Theorem Styles-------------------------%

% Define theorem style for default spacing and normal font.
\newtheoremstyle{normal}
    {\topsep}               % Amount of space above the theorem.
    {\topsep}               % Amount of space below the theorem.
    {}                      % Font used for body of theorem.
    {}                      % Measure of space to indent.
    {\bfseries}             % Font of the header of the theorem.
    {}                      % Punctuation between head and body.
    {.5em}                  % Space after theorem head.
    {}

% Define default environments.
\theoremstyle{normal}
\newtheorem{problem}{Problem}

\title{Point-Set Topology: Midterm}
\date{Summer 2022}

% No indent and no paragraph skip.
\setlength{\parindent}{0em}
\setlength{\parskip}{0em}

\begin{document}
    \maketitle
    \begin{problem}
        (\textbf{Logic})
        \par\hfill\par
        The truth table for a logical connective (such as $\Rightarrow$) that
        combines two propositions $P$ and $Q$ into a single proposition
        (like $P\Rightarrow{Q}$) is a table that exhausts all possibility of
        $P$ and $Q$ being true and false. The truth table for implication is
        given in Tab.~\ref{tab:truth_table_implication}
        \begin{table}[H]
            \centering
            \begin{tabular}{ c | c | c }
                $P$&$Q$&$P\Rightarrow{Q}$\\
                \hline
                False&False&True\\
                False&True&True\\
                True&False&False\\
                True&True&True
            \end{tabular}
            \caption{Truth Table for Implication}
            \label{tab:truth_table_implication}
        \end{table}
        Prove the absorption laws. If $P$ and $Q$ are propositions, then
        $P$ \textit{or} ($P$ \textit{and} $Q$) if and only if $P$. Using
        $\lor$ and $\land$ this says:
        \begin{equation}
            P\lor(P\land{Q})\Leftrightarrow{P}
        \end{equation}
        Also, $P$ \textit{and} ($P$ \textit{or} $Q$) if and only if $P$.
        That is:
        \begin{equation}
            P\land(P\lor{Q})\Leftrightarrow{P}
        \end{equation}
        \begin{itemize}
            \item (1 Point) Construct the truth table for $P\lor{Q}$.
            \item (1 Point) Construct the truth table for $P\land{Q}$.
            \item (1 Point) Construct the truth table for $P\lor(P\land{Q})$.
            \item (1 Point) Construct the truth table for $P\land(P\lor{Q})$.
            \item (1 Point) Compare these with $P$ to prove the absorption laws.
        \end{itemize}
        Prove that implication can be defined by
        \textit{negation} $(\neg)$ and \textit{logical or} $(\lor)$.
        \begin{itemize}
            \item (1 Point) Give the truth table for $\neg{P}$.
            \item (1 Point) Give the truth table for $\neg{P}\lor{Q}$.
            \item (1 Point) Compare this with implication $(\Rightarrow)$.
        \end{itemize}
    \end{problem}
    \clearpage
    \begin{problem}
        (\textbf{Set Theory})
        \par\hfill\par
        Here you will construct the real numbers.
        \begin{itemize}
            \item (2 Points) Show that if $A$ and $B$ are sets, there is a set
                of all functions $f:A\rightarrow{B}$. [Hint: Functions are
                subsets $f\subseteq{A}\times{B}$ with a special property.
                Use the axiom of the power set and the axiom schema of
                specification to construct the set of all functions
                from $A$ to $B$.
            \item (1 Point) Given the rational numbers $\mathbb{Q}$ with the
                standard metric $d(x,\,y)=|x-y|$, state the definition of a
                Cauchy sequence in $\mathbb{Q}$.
            \item (2 Points) Let $A$ be the set of all Cauchy sequences
                $a:\mathbb{N}\rightarrow\mathbb{Q}$ (This set exists by part 1
                of this problem). Define the relation $R$
                on $A$ by $aRb$ if and only if $a_{n}-b_{n}\rightarrow{0}$.
                Prove $R$ is an equivalence relation.
            \item (3 Points) Let $\mathbb{R}=A/R$. Define $+$ on $\mathbb{R}$
                by $[a]+[b]=[c]$ where $c:\mathbb{N}\rightarrow\mathbb{Q}$ is
                the sequence $c_{n}=a_{n}+b_{n}$. Show that
                $c:\mathbb{N}\rightarrow\mathbb{Q}$ is indeed a Cauchy sequence
                and that $+$ is well defined on $\mathbb{R}$.
        \end{itemize}
    \end{problem}
    \clearpage
    \begin{problem}
        (\textbf{Metric Spaces})
        \par\hfill\par
        \begin{itemize}
            \item (1 Point) State the definition of a metric space.
            \item (1 Point) State the definition of a convergent sequence.
            \item (1 Point) State the definition of a continuous function
                from a metric space $(X,\,d_{X})$ to a
                metric space $(Y,\,d_{Y})$.
            \item (3 Points) Prove that if $(X,\,d_{X})$, $(Y,\,d_{Y})$, and
                $(Z,\,d_{Z})$ are metric spaces, if $f:X\rightarrow{Y}$ and
                $g:Y\rightarrow{Z}$ are continuous, then
                $g\circ{f}:X\rightarrow{Z}$ is continuous.
            \item (1 Point) State the definition of a closed subset.
            \item (3 Points) Prove that if $\mathcal{D}\subseteq{Y}$ is closed
                and $f:X\rightarrow{Y}$ is continuous, then
                $f^{-1}[\mathcal{D}]\subseteq{X}$ is closed.
        \end{itemize}
    \end{problem}
    \clearpage
    \begin{problem}
        (\textbf{Compactness})
        \par\hfil\par
        A uniformly continuous function from a metric space
        $(X,\,d_{X})$ to a metric space $(Y,\,d_{Y})$ is a function
        $f:X\rightarrow{Y}$ such that for all $\varepsilon>0$ there exists
        a $\delta>0$ such that for all $x,x_{0}\in{X}$,
        $d_{X}(x,\,x_{0})$ implies $d_{Y}\big(f(x),\,f(x_{0})\big)<\varepsilon$.
        Using cryptic notation, this says:
        \begin{equation}
            \forall_{\varepsilon>0}\exists_{\delta>0}
                \forall_{x\in{X}}\forall_{x_{0}\in{X}}
                    \Big(d_{X}(x,\,x_{0})<\delta
                    \Rightarrow{d}_{Y}\big(f(x),\,f(x_{0})\big)<\varepsilon\Big)
        \end{equation}
        Note, this is \textbf{stronger} than continuity. You proved in HW 1
        that continuity is equivalent to:
        \begin{equation}
            \forall_{\varepsilon>0}\forall_{x\in{X}}\exists_{\delta>0}
                \forall_{x_{0}\in{X}}\Big(d_{X}(x,\,x_{0})<\delta
                    \Rightarrow{d}_{Y}\big(f(x),\,f(x_{0})\big)<\varepsilon\Big)
        \end{equation}
        The definition of uniform continuity \textit{swaps the quantifiers}.
        In continuity, given an $\varepsilon>0$ and an $x\in{X}$, you can
        find a $\delta>0$ that may depend on $\varepsilon$ and $x$,
        $\delta=\delta(\varepsilon,\,x)$, such that
        $d_{X}(x,\,x_{0})<\delta$ implies
        $d_{Y}\big(f(x),\,f(x_{0})\big)<\varepsilon$. With uniform continuity
        you may find a $\delta>0$ that works for all $x\in{X}$,
        $\delta$ only depends on $\varepsilon$, $\delta=\delta(\varepsilon)$.
        The function $f(x)=\frac{1}{x}$ defined on $\mathbb{R}^{+}$ is an
        example of a function that is continuous but not uniformly continuous.
        Given $\varepsilon>0$ and any $x\in\mathbb{R}^{+}$ you can indeed find
        a $\delta>0$ such that $|x-x_{0}|<\delta$ implies
        $|\frac{1}{x}-\frac{1}{x_{0}}|<\varepsilon$. But as $x$ gets smaller and
        smaller, closer to 0, the value of $\delta$ must get smaller too. This
        shows there can be no fixed positive $\delta>0$ that works for all
        $x\in\mathbb{R}^{+}$.
        \par\hfill\par
        In the problem you will prove the \textit{Heine-Cantor theorem}.
        If $(X,\,d_{X})$ is a compact metric space, if $(Y,\,d_{Y})$ is a
        metric space, and if $f:X\rightarrow{Y}$ is continuous, then $f$ is
        uniformly continuous.
        \begin{itemize}
            \item (2 Points)
                Let $\varepsilon>0$. By continuity, for all $x\in{X}$, there
                is a $\delta_{x}>0$ such that $x_{0}\in{X}$ and
                $d_{X}(x,\,x_{0})<\delta_{x}$ implies
                $d_{Y}\big(f(x),\,f(x_{0})\big)<\varepsilon$.
                Let $\mathcal{U}_{x}=B_{\delta_{x}/4}^{(X,\,d_{X})}(x)$ and
                $\mathcal{O}=\{\,\mathcal{U}_{x}\;|\;x\in{X}\,\}$. Show that
                $\mathcal{O}$ is an open cover of $X$.
            \item (2 Points) We proved that $(X,\,d_{X})$ is compact
                if and only if every
                open cover $\mathcal{O}$ has a finite subcover
                $\Delta\subseteq\mathcal{O}$. Write
                $\Delta=\{\,\mathcal{U}_{a_{0}},\,\dots,\,\mathcal{U}_{a_{N}}\,\}$.
                Let $\delta_{n}=\delta_{a_{n}}$ and
                $\delta=\textrm{min}(\frac{\delta_{0}}{4},\,\dots,\,\frac{\delta_{N}}{4})$.
                Show that if $x,x_{0}\in{X}$ and
                $d_{X}(x,\,x_{0})<\delta$, then there is an
                $a_{n}$ such that $x,x_{0}\in{B}_{\delta_{n}}^{(X,\,d)}(a_{n})$
                [Hint: The triangle inequality is always your friend.]
            \item (3 Points) Conclude that $f$ is uniformly continuous.
        \end{itemize}
    \end{problem}
    \clearpage
    \begin{problem}
        (\textbf{Topological Spaces})
        \begin{itemize}
            \item (1 Point) State the definition of a topological space.
            \item (1 Point) State the definition of a
                Hausdorff topological space.
            \item (3 Points) Let $(X,\,d)$ be a metric space and $\tau_{d}$ the
                metric topology. Prove that $(X,\,\tau_{d})$ is a Hausdorff
                topological space.
            \item (2 Points) Let $\tau_{Z}\subseteq\mathcal{P}(\mathbb{R})$ be
                the set of all $\mathcal{U}\subseteq\mathbb{R}$ such that
                there is a polynomial $f:\mathbb{R}\rightarrow\mathbb{R}$
                with $x\in{X}\setminus\mathcal{U}$ if and only if
                $f(x)=0$. Show that $\tau_{Z}$ is a topology. This is the
                \textit{Zariski Topology} on $\mathbb{R}$.
            \item (2 Points) Show that $(\mathbb{R},\,\tau_{Z})$ is not
                a Hausdorff topological space. [Hint: Use part of the
                \textit{fundamental theorem of algebra}. If
                $f:\mathbb{R}\rightarrow\mathbb{R}$ is a non-zero polynomial,
                there are only finitely many real numbers $x\in\mathbb{R}$
                such that $f(x)=0$.]
        \end{itemize}
    \end{problem}
\end{document}
