%-----------------------------------LICENSE------------------------------------%
%   This file is part of Mathematics-and-Physics.                              %
%                                                                              %
%   Mathematics-and-Physics is free software: you can redistribute it and/or   %
%   modify it under the terms of the GNU General Public License as             %
%   published by the Free Software Foundation, either version 3 of the         %
%   License, or (at your option) any later version.                            %
%                                                                              %
%   Mathematics-and-Physics is distributed in the hope that it will be useful, %
%   but WITHOUT ANY WARRANTY; without even the implied warranty of             %
%   MERCHANTABILITY or FITNESS FOR A PARTICULAR PURPOSE.  See the              %
%   GNU General Public License for more details.                               %
%                                                                              %
%   You should have received a copy of the GNU General Public License along    %
%   with Mathematics-and-Physics.  If not, see <https://www.gnu.org/licenses/>.%
%------------------------------------------------------------------------------%
\documentclass{article}
\usepackage{graphicx}                           % Needed for figures.
\usepackage{amsmath}                            % Needed for align.
\usepackage{amssymb}                            % Needed for mathbb.
\usepackage{amsthm}                             % For the theorem environment.
\usepackage[table]{xcolor}
\usepackage{float}
\usepackage{hyperref}
\hypersetup{
    colorlinks=true,
    linkcolor=blue,
    filecolor=magenta,
    urlcolor=Cerulean,
    citecolor=SkyBlue
}

%------------------------Theorem Styles-------------------------%

% Define theorem style for default spacing and normal font.
\newtheoremstyle{normal}
    {\topsep}               % Amount of space above the theorem.
    {\topsep}               % Amount of space below the theorem.
    {}                      % Font used for body of theorem.
    {}                      % Measure of space to indent.
    {\bfseries}             % Font of the header of the theorem.
    {}                      % Punctuation between head and body.
    {.5em}                  % Space after theorem head.
    {}

% Define default environments.
\theoremstyle{normal}
\newtheorem{problem}{Problem}

\title{Point-Set Topology: Midterm}
\date{Summer 2022}

% No indent and no paragraph skip.
\setlength{\parindent}{0em}
\setlength{\parskip}{0em}

\begin{document}
    \maketitle
    \color{blue}
    \begin{problem}[\textbf{Logic}]
        \par\hfill\par\vspace{1em}
        The truth table for a logical connective (such as $\Rightarrow$) that
        combines two propositions $P$ and $Q$ into a single proposition
        (like $P\Rightarrow{Q}$) is a table that exhausts all possibilities of
        $P$ and $Q$ being true and false. The truth table for implication is
        given in Tab.~\ref{tab:truth_table_implication}
        \begin{table}[H]
            \centering
            \begin{tabular}{ c | c | c }
                $P$&$Q$&$P\Rightarrow{Q}$\\
                \hline
                False&False&True\\
                False&True&True\\
                True&False&False\\
                True&True&True
            \end{tabular}
            \caption{Truth Table for Implication}
            \label{tab:truth_table_implication}
        \end{table}
        Prove the absorption laws. If $P$ and $Q$ are propositions, then
        $P$ \textit{or} ($P$ \textit{and} $Q$) if and only if $P$. Using
        $\lor$ and $\land$ this says:
        \begin{equation}
            P\lor(P\land{Q})\Leftrightarrow{P}
        \end{equation}
        Also, $P$ \textit{and} ($P$ \textit{or} $Q$) if and only if $P$.
        That is:
        \begin{equation}
            P\land(P\lor{Q})\Leftrightarrow{P}
        \end{equation}
        \begin{itemize}
            \item (1 Point) Construct the truth table for $P\lor{Q}$.
            \item (1 Point) Construct the truth table for $P\land{Q}$.
            \item (1 Point) Construct the truth table for $P\lor(P\land{Q})$.
            \item (1 Point) Construct the truth table for $P\land(P\lor{Q})$.
            \item (1 Point) Compare these with $P$ to prove the absorption laws.
        \end{itemize}
        Prove that implication can be defined by
        \textit{negation} $(\neg)$ and \textit{logical or} $(\lor)$.
        \begin{itemize}
            \item (1 Point) Give the truth table for $\neg{P}$.
            \item (1 Point) Give the truth table for $(\neg{P})\lor{Q}$.
            \item (1 Point) Compare this with implication $P\Rightarrow{Q}$.
        \end{itemize}
    \end{problem}
    \clearpage
    \color{black}
    \begin{proof}[Solution]
        \par\hfill\par
        \begin{table}[H]
            \centering
            \begin{tabular}{ c | c | c }
                $P$&$Q$&$P\lor{Q}$\\
                \hline
                False&False&False\\
                False&True&True\\
                True&False&True\\
                True&True&True
            \end{tabular}
            \caption{Truth Table for Logical Disjunction $(\lor)$}
            \label{tab:logical_disjunction}
        \end{table}
        \begin{table}[H]
            \centering
            \begin{tabular}{ c | c | c }
                $P$&$Q$&$P\land{Q}$\\
                \hline
                False&False&False\\
                False&True&False\\
                True&False&False\\
                True&True&True
            \end{tabular}
            \caption{Truth Table for Logical Conjunction $(\land)$}
            \label{tab:logical_conjunction}
        \end{table}
        \begin{table}[H]
            \centering
            \begin{tabular}{ c | c | c | c }
                $P$&$Q$&$P\lor{Q}$&$P\land(P\lor{Q})$\\
                \hline
                False&False&False&False\\
                False&True&True&False\\
                True&False&True&True\\
                True&True&True&True
            \end{tabular}
            \caption{Truth Table for the First Absorption Law}
            \label{tab:first_abs_law}
        \end{table}
        \begin{table}[H]
            \centering
            \begin{tabular}{ c | c | c | c }
                $P$&$Q$&$P\land{Q}$&$P\lor(P\land{Q})$\\
                \hline
                False&False&False&False\\
                False&True&False&False\\
                True&False&False&True\\
                True&True&True&True
            \end{tabular}
            \caption{Truth Table for the Second Absorption Law}
            \label{tab:second_abs_law}
        \end{table}
        In both Tab.~\ref{tab:first_abs_law} and Tab.~\ref{tab:second_abs_law}
        the columns for $P$, $P\land(P\lor{Q})$, and $P\lor(P\land{Q})$ are
        identical, meaning $P$ is true if and only if
        $P\land(P\lor{Q})$ is true, if and only if $P\lor(P\land{Q})$ is true.
        This is precisely the absorption laws.
        \begin{table}[H]
            \centering
            \begin{tabular}{ c | c }
                $P$&$\neg{P}$\\
                \hline
                False&True\\
                True&False
            \end{tabular}
            \caption{Truth Table for Negation}
            \label{tab:negation}
        \end{table}
        \begin{table}[H]
            \centering
            \begin{tabular}{ c | c | c | c }
                $P$&$Q$&$\neg{P}$&$(\neg{P})\lor{Q}$\\
                \hline
                False&False&True&True\\
                False&True&True&True\\
                True&False&False&False\\
                True&True&False&True
            \end{tabular}
            \caption{Equivalent Representation of Implication}
            \label{tab:imp_equiv}
        \end{table}
        This is the same table as implication. Usually this is done the other
        way around. In a \textit{Hilbert System} the main primitive notion is
        implication $(\Rightarrow)$, and there are a few axioms for what it
        means. Logical or is then defined as
        $P\lor{Q}\Leftrightarrow(\neg{P})\Rightarrow{Q}$. All of the logical
        symbols are further defined using implication.
    \end{proof}
    \clearpage
    \color{blue}
    \begin{problem}[\textbf{Set Theory}]
        \par\hfill\par\vspace{1em}
        Here you will construct the real numbers.
        \begin{itemize}
            \item (2 Points) Show that if $A$ and $B$ are sets, there is a set
                of all functions $f:A\rightarrow{B}$. [Hint: Functions are
                subsets $f\subseteq{A}\times{B}$ with a special property.
                Use the axiom of the power set and the axiom schema of
                specification to construct the set of all functions
                from $A$ to $B$.]
            \item (1 Point) Given the rational numbers $\mathbb{Q}$ with the
                standard metric $d(x,\,y)=|x-y|$, state the definition of a
                Cauchy sequence in $\mathbb{Q}$.
            \item (2 Points) Let $A$ be the set of all Cauchy sequences
                $a:\mathbb{N}\rightarrow\mathbb{Q}$ (This set exists by part 1
                of this problem). Define the relation $R$
                on $A$ by $aRb$ if and only if $|a_{n}-b_{n}|\rightarrow{0}$.
                Prove $R$ is an equivalence relation.
            \item (3 Points) Let $\mathbb{R}=A/R$. Define $+$ on $\mathbb{R}$
                by $[a]+[b]=[c]$ where $c:\mathbb{N}\rightarrow\mathbb{Q}$ is
                the sequence $c_{n}=a_{n}+b_{n}$. Show that
                $c:\mathbb{N}\rightarrow\mathbb{Q}$ is indeed a Cauchy sequence
                and that $+$ is well defined on $\mathbb{R}$.
        \end{itemize}
    \end{problem}
    \color{black}
    \begin{proof}[Solution]
        By the axiom of the power set, the set $\mathcal{P}(A\times{B})$ exists.
        Let $P(f)$ be the sentence $f$ \textit{is a function from} $A$
        \textit{to} $B$. Let $\mathcal{F}(A,\,B)$ be defined by:
        \begin{equation}
            \mathcal{F}(A,\,B)
            =\{\,f\in\mathcal{P}(A\times{B})\;|\;P(f)\,\}
        \end{equation}
        Then $f\in\mathcal{F}(A,\,B)$ if and only if $f$ is a function from $A$
        to $B$. That is, $\mathcal{F}(A,\,B)$ is the set of all functions
        $f:A\rightarrow{B}$. We can be very cryptic if we so desire:
        \begin{equation}
            \mathcal{F}(A,\,B)
            =\Big\{\,f\in\mathcal{P}(A\times{B})\;|\;
                \forall_{a\in{A}}\exists!_{b\in{B}}\big((a,\,b)\in{f}\big)
                \,\Big\}
        \end{equation}
        Where $\exists!$ is an extension of the $\exists$ qualifier.
        $\exists!$ means there exists a unique element satisfying the
        following proposition.
        \par\hfill\par
        Now, using this idea to construct the real numbers. A Cauchy
        sequence in $\mathbb{Q}$ is a sequence
        $a:\mathbb{N}\rightarrow\mathbb{Q}$ such that for all
        $r>0$, $r\in\mathbb{Q}$, there is an $N\in\mathbb{N}$ such that for
        all $m,n\in\mathbb{N}$ with $m,n>N$ we have
        $|a_{m}-a_{n}|<\varepsilon$. We want to say for all
        $r>0$, $r\in\mathbb{R}$, but we don't have $\mathbb{R}$ yet!
        Now, using the first part of this problem, since a Cauchy sequence
        is a particular function $a:\mathbb{N}\rightarrow\mathbb{Q}$, and
        the set $\mathcal{F}(\mathbb{N},\,\mathbb{Q})$ of all functions
        from $\mathbb{N}$ to $\mathbb{Q}$ exists, we can apply the sentence
        $P(a)$, $a$ \textit{is a Cauchy sequence} to the set
        $\mathcal{F}(\mathbb{N},\,\mathbb{Q})$ and obtain the set $A$ of all
        Cauchy sequences in $\mathbb{Q}$. The relation $R$ on $A$ defined by
        $aRb$ if and only if $|a_{n}-b_{n}|\rightarrow{0}$ is an equivalence
        relation. It is reflexive since $|a_{n}-a_{n}|=0$ for all
        $n\in\mathbb{N}$, so indeed $|a_{n}-a_{n}|\rightarrow{0}$. It is
        symmetric. If $aRb$, then:
        \begin{equation}
            \lim_{n\rightarrow\infty}|b_{n}-a_{n}|
            =\lim_{n\rightarrow\infty}|(-1)(a_{n}-b_{n})|
            =\lim_{n\rightarrow\infty}|a_{n}-b_{n}|
            =0
        \end{equation}
        and hence $bRa$. Lastly, it is transitive. If $aRb$ and $bRc$, then
        $|a_{n}-b_{n}|\rightarrow{0}$ and $|b_{n}-c_{n}|\rightarrow{0}$.
        Let $\varepsilon>0$. There exists $N_{0},N_{1}\in\mathbb{N}$ such that
        $n\in\mathbb{N}$ and $n>N_{0}$ implies
        $|a_{n}-b_{n}|<\varepsilon/2$, and $n\in\mathbb{N}$ and $n>N_{1}$
        implies $|b_{n}-c_{n}|<\varepsilon/2$. Let
        $N=\textrm{max}(N_{0},\,N_{1})$. Then for $n\in\mathbb{N}$ and $n>N$
        we have:
        \begin{equation}
            |a_{n}-c_{n}|
            =|a_{n}-b_{n}+b_{n}-c_{n}|
            \leq|a_{n}-b_{n}|+|b_{n}-c_{n}|
            <\frac{\varepsilon}{2}+\frac{\varepsilon}{2}
            =\varepsilon
        \end{equation}
        and hence $|a_{n}-c_{n}|\rightarrow{0}$. So $R$ is reflexive, symmetric,
        and transitive, and is therefore an equivalence relation.
        \par\hfill\par
        Defining $[a]+[b]=[c]$ where $c:\mathbb{N}\rightarrow\mathbb{Q}$ is the
        sequence $c_{n}=a_{n}+b_{n}$, first $c$ is indeed a Cauchy sequence.
        Let $\varepsilon>0$. Since $a$ is a Cauchy sequence there is an
        $N_{0}\in\mathbb{N}$ such that for all $m,n\in\mathbb{N}$ with
        $m,n>N_{0}$ we have $|a_{m}-a_{n}|<\varepsilon/2$. But $b$ is a Cauchy
        sequence, so there is an $N_{1}\in\mathbb{N}$ such that for all
        $m,n\in\mathbb{N}$ with $m,n>N_{1}$ we have
        $|b_{m}-b_{n}|<\varepsilon/2$. Let $N=\textrm{max}(N_{0},\,N_{1})$.
        Then for all $m,n\in\mathbb{N}$ with $m,n>N$ we have:
        \begin{align}
            |c_{m}-c_{n}|
            &=|(a_{m}+b_{m})-(a_{n}+b_{n})|\\
            &=|a_{m}+b_{m}-a_{n}-b_{n}|\\
            &=|(a_{m}-a_{n})+(b_{m}-b_{n})|\\
            &\leq|a_{m}-a_{n}|+|b_{m}-b_{n}|\\
            &<\frac{\varepsilon}{2}+\frac{\varepsilon}{2}\\
            &=\varepsilon
        \end{align}
        so $c$ is a Cauchy sequence. Addition is well-defined. If
        $a,b,x,y\in{A}$ are Cauchy sequences, and if
        $[a]=[x]$ and $[b]=[y]$, then:
        \begin{align}
            \lim_{n\rightarrow\infty}|(a_{n}+b_{n})-(x_{n}+y_{n})|
            &=\lim_{n\rightarrow\infty}|a_{n}+b_{n}-x_{n}-y_{n}|\\
            &=\lim_{n\rightarrow\infty}|(a_{n}-x_{n})+(b_{n}-y_{n})|\\
            &\leq\lim_{n\rightarrow\infty}
                \big(|a_{n}-x_{n}|+|b_{n}-y_{n}|\big)\\
            &=\lim_{n\rightarrow\infty}|a_{n}-x_{n}|
                +\lim_{n\rightarrow\infty}|b_{n}-y_{n}|\\
            &=0+0\\
            &=0
        \end{align}
        And hence $(a+b)R(x+y)$, meaning $[a]+[b]=[x]+[y]$, so addition is
        well-defined.
    \end{proof}
    \clearpage
    \color{blue}
    \begin{problem}[\textbf{Metric Spaces}]
        \par\hfill\par\vspace{1em}
        \begin{itemize}
            \item (1 Point) State the definition of a metric space.
            \item (1 Point) State the definition of a convergent sequence.
            \item (1 Point) State the definition of a continuous function
                from a metric space $(X,\,d_{X})$ to a
                metric space $(Y,\,d_{Y})$.
            \item (3 Points) Prove that if $(X,\,d_{X})$, $(Y,\,d_{Y})$, and
                $(Z,\,d_{Z})$ are metric spaces, if $f:X\rightarrow{Y}$ and
                $g:Y\rightarrow{Z}$ are continuous, then
                $g\circ{f}:X\rightarrow{Z}$ is continuous.
            \item (1 Point) State the definition of a closed subset.
            \item (3 Points) Prove that if $\mathcal{D}\subseteq{Y}$ is closed
                and $f:X\rightarrow{Y}$ is continuous, then
                $f^{-1}[\mathcal{D}]\subseteq{X}$ is closed.
        \end{itemize}
    \end{problem}
    \color{black}
    \begin{proof}[Solution]
        A metric space is a set $X$ with a metric function
        $d:X\times{X}\rightarrow\mathbb{R}$ that satisfies the following:
        \begin{itemize}
            \item
                Positive-Definiteness
                \par
                $d(x,\,y)\geq{0}$ for all $x,y\in{X}$ and
                $d(x,\,y)=0$ if and only if $x=y$.
            \item
                Symmetry
                \par
                $d(x,\,y)=d(y,\,x)$ for all $x,y\in{X}$.
            \item
                Triange Inequality
                \par
                $d(x,\,z)\leq{d}(x,\,y)+d(y,\,z)$ for all $x,y,z\in{X}$.
        \end{itemize}
        A convergent sequence in a metric space $(X,\,d)$ is a sequence
        $a:\mathbb{N}\rightarrow{X}$ such that there is an $x\in{X}$ such that
        for all $\varepsilon>0$ there is an $N\in\mathbb{N}$ such that
        $n\in\mathbb{N}$ and $n>N$ implies $d(x,\,a_{n})<\varepsilon$.
        \par\hfill\par
        A continuous function between metric spaces is a function that maps
        convergent sequences to convergent sequences. That is, given metric
        spaces $(X,\,d_{X})$ and $(Y,\,d_{Y})$, a continuous function is a
        function $f:X\rightarrow{Y}$ such that for all convergent sequences
        $a:\mathbb{N}\rightarrow{X}$ such that $a_{n}\rightarrow{x}$ for some
        $x\in{X}$, it is true that $f(a_{n})\rightarrow{f}(x)$.
        \par\hfill\par
        Suppose $f:X\rightarrow{Y}$ and $g:Y\rightarrow{Z}$ are continuous.
        Let $a:\mathbb{N}\rightarrow{X}$ be a convergent sequence with
        $a_{n}\rightarrow{x}\in{X}$. Since $f$ is continuous,
        $f(a_{n})\rightarrow{f}(x)$. Since $g$ is continuous and $f(a)$ is a
        convergent sequence, $g\big(f(a_{n})\big)\rightarrow{g}\big(f(x)\big)$.
        But $(g\circ{f})(a_{n})=g\big(f(a_{n})\big)$, so
        $(g\circ{f})(a_{n})\rightarrow(g\circ{f})(x)$. Hence, $g\circ{f}$ is
        continuous.
        \par\hfill\par
        A closed set in a metric space $(X,\,d)$ is a subset
        $\mathcal{C}\subseteq{X}$ such that for every sequence
        $a:\mathbb{N}\rightarrow\mathcal{C}$ such that
        $a_{n}\rightarrow{x}\in{X}$ it is true that $x\in\mathcal{C}$. That
        is, $\mathcal{C}$ has all of its limit points.
        \par\hfill\par
        Let $(X,\,d_{X})$ and $(Y,\,d_{Y})$ be metric spaces and
        $f:X\rightarrow{Y}$ continuous. Let $\mathcal{D}\subseteq{Y}$ be closed
        and $\mathcal{C}=f^{-1}[\mathcal{D}]$. Let
        $a:\mathbb{N}\rightarrow\mathcal{C}$ be a convergent sequence with
        limit $x\in{X}$. Since $f$ is continuous, $f(a_{n})\rightarrow{f}(x)$.
        But $\mathcal{D}$ is closed and $f(a_{n})\in\mathcal{D}$, so
        $f(x)\in\mathcal{D}$. Hence $x\in\mathcal{C}$, so $\mathcal{C}$ is
        closed.
    \end{proof}
    \clearpage
    \color{blue}
    \begin{problem}[\textbf{Compactness}]
        \par\hfil\par\vspace{1em}
        A uniformly continuous function from a metric space
        $(X,\,d_{X})$ to a metric space $(Y,\,d_{Y})$ is a function
        $f:X\rightarrow{Y}$ such that for all $\varepsilon>0$ there exists
        a $\delta>0$ such that for all $x,x_{0}\in{X}$,
        $d_{X}(x,\,x_{0})<\delta$ implies
        $d_{Y}\big(f(x),\,f(x_{0})\big)<\varepsilon$.
        Using cryptic notation, this says:
        \begin{equation}
            \forall_{\varepsilon>0}\exists_{\delta>0}
                \forall_{x\in{X}}\forall_{x_{0}\in{X}}
                    \Big(d_{X}(x,\,x_{0})<\delta
                    \Rightarrow{d}_{Y}\big(f(x),\,f(x_{0})\big)<\varepsilon\Big)
        \end{equation}
        Note, this is \textbf{stronger} than continuity. You proved in HW 1
        that continuity is equivalent to:
        \begin{equation}
            \forall_{\varepsilon>0}\forall_{x\in{X}}\exists_{\delta>0}
                \forall_{x_{0}\in{X}}\Big(d_{X}(x,\,x_{0})<\delta
                    \Rightarrow{d}_{Y}\big(f(x),\,f(x_{0})\big)<\varepsilon\Big)
        \end{equation}
        The definition of uniform continuity \textit{swaps the quantifiers}.
        In continuity, given an $\varepsilon>0$ and an $x\in{X}$, you can
        find a $\delta>0$ that may depend on $\varepsilon$ and $x$,
        $\delta=\delta(\varepsilon,\,x)$, such that
        $d_{X}(x,\,x_{0})<\delta$ implies
        $d_{Y}\big(f(x),\,f(x_{0})\big)<\varepsilon$. With uniform continuity
        you may find a $\delta>0$ that works for all $x\in{X}$,
        $\delta$ only depends on $\varepsilon$, $\delta=\delta(\varepsilon)$.
        The function $f(x)=\frac{1}{x}$ defined on $\mathbb{R}^{+}$ is an
        example of a function that is continuous but not uniformly continuous.
        Given $\varepsilon>0$ and any $x\in\mathbb{R}^{+}$ you can indeed find
        a $\delta>0$ such that $|x-x_{0}|<\delta$ implies
        $|\frac{1}{x}-\frac{1}{x_{0}}|<\varepsilon$. But as $x$ gets smaller and
        smaller, closer to 0, the value of $\delta$ must get smaller too. This
        shows there can be no fixed positive $\delta>0$ that works for all
        $x\in\mathbb{R}^{+}$.
        \par\hfill\par
        In the problem you will prove the \textit{Heine-Cantor theorem}.
        If $(X,\,d_{X})$ is a compact metric space, if $(Y,\,d_{Y})$ is a
        metric space, and if $f:X\rightarrow{Y}$ is continuous, then $f$ is
        uniformly continuous.
        \begin{itemize}
            \item (2 Points)
                Let $\varepsilon>0$. By continuity, for all $x\in{X}$, there
                is a $\delta_{x}>0$ such that $x_{0}\in{X}$ and
                $d_{X}(x,\,x_{0})<\delta_{x}$ implies
                $d_{Y}\big(f(x),\,f(x_{0})\big)<\varepsilon$.
                Let $\mathcal{U}_{x}=B_{\delta_{x}/2}^{(X,\,d_{X})}(x)$ and
                $\mathcal{O}=\{\,\mathcal{U}_{x}\;|\;x\in{X}\,\}$. Show that
                $\mathcal{O}$ is an open cover of $X$.
            \item (2 Points) We proved that $(X,\,d_{X})$ is compact
                if and only if every
                open cover $\mathcal{O}$ has a finite subcover
                $\Delta\subseteq\mathcal{O}$. Write
                $\Delta=\{\,\mathcal{U}_{a_{0}},\,\dots,\,\mathcal{U}_{a_{N}}\,\}$.
                Let
                $\delta=\frac{1}{2}\textrm{min}(\delta_{a_{0}},\,\dots,\,\delta_{a_{N}})$.
                Show that if $x,x_{0}\in{X}$ and
                $d_{X}(x,\,x_{0})<\delta$, then there is an
                $a_{n}$ such that $x,x_{0}\in{B}_{\delta{a}_{n}}^{(X,\,d)}(a_{n})$
                [Hint: The triangle inequality is always your friend.]
            \item (3 Points) Conclude that $f$ is uniformly continuous.
        \end{itemize}
        \textbf{Bonus:} (4 Points) Prove that if $(X,\,d)$ is a compact metric
        space, and $f:X\rightarrow\mathbb{R}$ is continuous (with the standard
        metric on $\mathbb{R}$), then $f$ is bounded. That is, there is an
        $M\in\mathbb{R}$ such that for all $x\in{X}$ we have
        $|f(x)|<M$.
    \end{problem}
    \clearpage
    \color{black}
    \begin{proof}[Solution]
        For all $\mathcal{U}\in\mathcal{O}$, $\mathcal{U}$ is an open ball, and
        hence open. But moreover, for all $x\in{X}$, since $\delta_{x}$ is
        chosen to be positive, we have that $B_{\delta_{x}}^{(X,\,d)}(x)$ is
        non-empty since it contains the point $x$. Since this is true of
        all $x\in{X}$, $\mathcal{O}$ is a collection of open sets that cover
        $X$, and is therefore an open cover.
        \par\hfill\par
        Let $x,x_{0}\in{X}$ be such that $d_{X}(x,\,x_{0})<\delta$. Since
        $\Delta$ is a cover of $X$ there is a $\mathcal{U}_{n}\in\Delta$ such
        that $x\in\mathcal{U}_{n}$. But
        $\mathcal{U}_{n}=B_{\delta_{a_{n}}/2}^{(X,\,d)}(a_{n})$, so
        $d_{X}(x,\,a_{n})<\delta_{a_{n}}/2$. But then, by the triangle
        inequality, we have:
        \begin{equation}
            d_{X}(x_{0},\,a_{n})
            \leq{d}_{X}(x_{0},\,x)+d_{X}(x,\,a_{n})
            <\delta+\frac{\delta_{a_{n}}}{2}
            \leq\frac{\delta_{a_{n}}}{2}+\frac{\delta_{a_{n}}}{2}
            =\delta_{a_{n}}
        \end{equation}
        by the definition of $\delta$. So
        $x,x_{0}\in{B}_{\delta_{a_{n}}}^{(X,\,d)}(a_{n})$.
        \par\hfill\par
        Let $x,x_{0}\in{X}$ with $d_{X}(x,\,x_{0})<\delta$. Then there is an
        $a_{n}$ such that $x,x_{0}\in{B}_{\delta_{a_{n}}}^{(X,\,d)}(a_{n})$. But
        then:
        \begin{equation}
            d_{Y}\big(f(x),\,f(x_{0})\big)
            \leq{d}_{Y}\big(f(x),\,f(a_{n})\big)
                +d_{Y}\big(f(x_{0}),\,f(a_{n})\big)
            <\varepsilon+\varepsilon
            =2\varepsilon
        \end{equation}
        Since $2\varepsilon$ can be made arbitrarily small, and since
        $\delta$ was chosen independent of $x$, $f$ is uniformly continuous.
        \par\hfill\par
        For the bonus, suppose $f$ is not bounded. Then for all $M\in\mathbb{R}$
        there is an $x\in{X}$ such that $|f(x)|\geq{M}$. In particular, for all
        $n\in\mathbb{N}$ there is an $a_{n}\in{X}$ such that
        $|f(a_{n})|\geq{n}$. But then $a:\mathbb{N}\rightarrow{X}$ is a
        sequence in a compact metric space, so there is a convergent
        subsequence $a_{k}$. Let $x\in{X}$ be the limit,
        $a_{k_{n}}\rightarrow{x}$. But $f$ is continuous, so if
        $a_{k_{n}}\rightarrow{x}$, then
        $f(a_{k_{n}})\rightarrow{f}(x)$. Let $N\in\mathbb{N}$ be such that
        $N>|f(x)|+1$. But then for all $n\in\mathbb{N}$ with $n>N$ we have
        $|f(x)-f(a_{k_{n}})|>1$, so $f(a_{k_{n}})$ can't converge to $f(x)$,
        a contradiction. So $f$ is bounded.
    \end{proof}
    \clearpage
    \color{blue}
    \begin{problem}[\textbf{Topological Spaces}]
        \par\hfill\par\vspace{1em}
        You may freely use the following fact. If
        $f:\mathbb{R}\rightarrow\mathbb{R}$ is a non-zero polynomial, then
        there are only finitely many numbers $x\in\mathbb{R}$ such that
        $f(x)=0$.
        \begin{itemize}
            \item (1 Point) State the definition of a topological space.
            \item (1 Point) State the definition of a
                Hausdorff topological space.
            \item (3 Points) Let $(X,\,d)$ be a metric space and $\tau_{d}$ the
                metric topology. Prove that $(X,\,\tau_{d})$ is a Hausdorff
                topological space.
            \item (2 Points) Let $\tau_{Z}\subseteq\mathcal{P}(\mathbb{R})$ be
                the set of all $\mathcal{U}\subseteq\mathbb{R}$ such that
                there is a polynomial $f:\mathbb{R}\rightarrow\mathbb{R}$
                with $x\in\mathbb{R}\setminus\mathcal{U}$ if and only if
                $f(x)=0$. Show that $\tau_{Z}$ is a topology. This is the
                \textit{Zariski Topology} on $\mathbb{R}$.
            \item (2 Points) Show that $(\mathbb{R},\,\tau_{Z})$ is not
                a Hausdorff topological space.
        \end{itemize}
    \end{problem}
    \color{black}
    \begin{proof}[Solution]
        A topological space is a set $X$ with a topology $\tau$, which is a
        subset $\tau\subseteq\mathcal{P}(X)$ satisfying:
        \begin{itemize}
            \item $\emptyset\in\tau$
            \item $X\in\tau$
            \item If $\mathcal{U},\mathcal{V}\in\tau$, then
                $\mathcal{U}\cap\mathcal{V}\in\tau$.
            \item If $\mathcal{O}\subseteq\tau$, then
                $\bigcup\mathcal{O}\in\tau$.
        \end{itemize}
        A Hausdorff topological space is a topological space $(X,\,\tau)$ such
        that for all $x,y\in{X}$ with $x\ne{y}$ there are open sets
        $\mathcal{U},\mathcal{V}\in\tau$ such that $x\in\mathcal{U}$,
        $y\in\mathcal{V}$, and $\mathcal{U}\cap\mathcal{V}=\emptyset$.
        \par\hfill\par
        A metrizable space is Hausdorff. Let $(X,\,\tau)$ be metrizable, with
        metric $d$ inducing the topology $\tau$. Let $x,y\in{X}$ be distinct,
        $x\ne{y}$. Since $d$ is a metric, $d(x,\,y)>0$. Let
        $\varepsilon=\frac{1}{2}d(x,\,y)$. Let
        $\mathcal{U}=B_{\varepsilon}^{(X,\,d)}(x)$ and
        $\mathcal{V}=B_{\varepsilon}^{(X,\,d)}(y)$. Then, since open balls are
        open, $\mathcal{U}$ and $\mathcal{V}$ are elements of $\tau$.
        Suppose $z\in\mathcal{U}\cap\mathcal{V}$. Then:
        \begin{equation}
            d(x,\,y)\leq{d}(x,\,z)+d(z,\,y)
                <\varepsilon+\varepsilon
                =d(x,\,y)
        \end{equation}
        so $d(x,\,y)<d(x,\,y)$, a contradiction, and therefore
        $\mathcal{U}\cap\mathcal{V}=\emptyset$. That is, $(X,\,\tau)$ is
        Hausdorff.
        \par\hfill\par
        The Zariski topology is a topology. The entire set is in it since
        $f(x)=1$ is a polynomial and $f(x)=0$ if and only if $x\in\emptyset$.
        So $\mathbb{R}=\mathbb{R}\setminus\emptyset$ is an element of
        $\tau_{Z}$. Similarly, $f(x)=0$ is a polynomial and $f(x)=0$ for all
        $x\in\mathbb{R}$, hence $\emptyset=\mathbb{R}\setminus\mathbb{R}$ is
        in $\tau_{Z}$. Let $\mathcal{U},\mathcal{V}\in\tau_{Z}$ be open sets.
        Then there are polynomials $f$ and $g$ corresponding to $\mathcal{U}$
        and $\mathcal{V}$, respectively. But the product of polynomials is a
        polynomial, so $h=fg$ is a polynomial. But then $h(x)=0$ if and only if
        $f(x)g(x)=0$. But $f(x)g(x)=0$ if and only if $f(x)=0$ or $g(x)=0$
        (Euclid's theorem). But then $h(x)=0$ if and only if
        $x\in\mathbb{R}\setminus\mathcal{U}$ or
        $x\in\mathbb{R}\setminus\mathcal{V}$. Thus $h(x)=0$ if and only if
        $x\in\mathbb{R}\setminus(\mathcal{U}\cap\mathcal{V})$, so
        $\mathcal{U}\cap\mathcal{V}$ is open. Lastly, let
        $\mathcal{O}\subseteq\tau_{Z}$. If $\mathcal{O}$ is empty, the union
        is empty, and the empty set is an element of $\tau_{Z}$. If
        $\mathbb{R}\in\mathcal{O}$, then $\bigcup\mathcal{O}=\mathbb{R}$, and
        $\mathbb{R}\in\tau_{Z}$. So suppose $\mathcal{O}$ is non-empty and 
        $\mathbb{R}\notin\mathcal{O}$. But then every
        $\mathcal{U}\in\mathcal{O}$ corresponds to a polynomial $f$ where
        $f(x)=0$ for at least some $x\in\mathbb{R}$. Let
        $\mathcal{U}\in\mathcal{O}$ and let $f$ be the corresponding polynomial.
        But then $\mathbb{R}\setminus\mathcal{U}$ is finite since a non-zero
        polynomial has only finitely many zeros. But then:
        \begin{equation}
            \mathbb{R}\setminus\bigcup\mathcal{O}
            \subseteq\mathbb{R}\setminus\mathcal{U}
        \end{equation}
        So $\mathbb{R}\setminus\bigcup\mathcal{O}$ is finite. Let the elements
        be $x_{0},\dots,x_{n}$. Let $h(x)$ be defined by:
        \begin{equation}
            h(x)=\prod_{k=0}^{n}(x-x_{k})
            =(x-x_{0})(x-x_{1})\cdots(x-x_{n})
        \end{equation}
        Then $h(x)=0$ if and only if
        $x\in\mathbb{R}\setminus\bigcup\mathcal{O}$. Hence
        $\bigcup\mathcal{O}$ is open.
        \par\hfill\par
        $(\mathbb{R},\,\tau_{Z})$ is not Hausdorff. Let
        $\mathcal{U},\mathcal{V}$ be non-empty proper open subsets. Then,
        since non-zero polynomials have only finitely many zeros,
        $\mathbb{R}\setminus\mathcal{U}$ and $\mathbb{R}\setminus\mathcal{V}$
        are finite. But then $\mathcal{U}\cap\mathcal{V}$ must be infinite
        since $\mathbb{R}$ is infinite, and hence $(X,\,\tau_{Z})$ can not be
        Hausdorff.
    \end{proof}
\end{document}
