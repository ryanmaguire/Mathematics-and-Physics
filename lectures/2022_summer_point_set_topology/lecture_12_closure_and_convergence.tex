%-----------------------------------LICENSE------------------------------------%
%   This file is part of Mathematics-and-Physics.                              %
%                                                                              %
%   Mathematics-and-Physics is free software: you can redistribute it and/or   %
%   modify it under the terms of the GNU General Public License as             %
%   published by the Free Software Foundation, either version 3 of the         %
%   License, or (at your option) any later version.                            %
%                                                                              %
%   Mathematics-and-Physics is distributed in the hope that it will be useful, %
%   but WITHOUT ANY WARRANTY; without even the implied warranty of             %
%   MERCHANTABILITY or FITNESS FOR A PARTICULAR PURPOSE.  See the              %
%   GNU General Public License for more details.                               %
%                                                                              %
%   You should have received a copy of the GNU General Public License along    %
%   with Mathematics-and-Physics.  If not, see <https://www.gnu.org/licenses/>.%
%------------------------------------------------------------------------------%
\documentclass{article}
\usepackage{graphicx}                           % Needed for figures.
\usepackage{amsmath}                            % Needed for align.
\usepackage{amssymb}                            % Needed for mathbb.
\usepackage{amsthm}                             % For the theorem environment.
\usepackage{float}
\usepackage{hyperref}
\hypersetup{
    colorlinks=true,
    linkcolor=blue,
    filecolor=magenta,
    urlcolor=Cerulean,
    citecolor=SkyBlue
}

%------------------------Theorem Styles-------------------------%
\theoremstyle{plain}
\newtheorem{theorem}{Theorem}[section]

% Define theorem style for default spacing and normal font.
\newtheoremstyle{normal}
    {\topsep}               % Amount of space above the theorem.
    {\topsep}               % Amount of space below the theorem.
    {}                      % Font used for body of theorem.
    {}                      % Measure of space to indent.
    {\bfseries}             % Font of the header of the theorem.
    {}                      % Punctuation between head and body.
    {.5em}                  % Space after theorem head.
    {}

% Define default environments.
\theoremstyle{normal}
\newtheorem{examplex}{Example}[section]
\newtheorem{definitionx}{Definition}[section]
\newtheorem{notationx}{Notation}[section]
\newtheorem{axiomx}{Axiom}[section]

\newenvironment{example}{%
    \pushQED{\qed}\renewcommand{\qedsymbol}{$\blacksquare$}\examplex%
}{%
    \popQED\endexamplex%
}

\newenvironment{definition}{%
    \pushQED{\qed}\renewcommand{\qedsymbol}{$\blacksquare$}\definitionx%
}{%
    \popQED\enddefinitionx%
}

\title{Point-Set Topology: Lecture 12}
\author{Ryan Maguire}
\date{Summer 2022}

% No indent and no paragraph skip.
\setlength{\parindent}{0em}
\setlength{\parskip}{0em}

\begin{document}
    \maketitle
    \section{Closure, Interior, and Boundary}
        In previous lectures we used large collections of topologies to
        generate a new one. In particular, we took a collection
        of subsets $\mathcal{B}\subseteq\mathcal{P}(X)$, and looked at the
        set $T$ of all topologies $\tau$ on $X$ such that
        $\mathcal{B}\subseteq\tau$. This set $T$ is non-empty since
        $\mathcal{P}(X)\in{T}$. We then created a new topology via the
        intersection $\bigcap{T}$. This is the \textit{generated} topology.
        We now use a similar idea, but instead of collections of topologies,
        we look at collections of open and closed sets. We've seen some laws
        about open sets, these are the rules dictated in the definition of a
        topology. Using the De Morgan law's we get similar statements about
        closed sets.
        \begin{theorem}
            If $(X,\,\tau)$ is a topological space, then $\emptyset$ is closed.
        \end{theorem}
        \begin{proof}
            Since $X$ is open and $\emptyset=X\setminus{X}$, $\emptyset$ is
            closed.
        \end{proof}
        \begin{theorem}
            If $(X,\,\tau)$ is a topological space, then $X$ is closed.
        \end{theorem}
        \begin{proof}
            Since $\emptyset$ is open and $X=X\setminus\emptyset$, $X$ is
            closed.
        \end{proof}
        \begin{theorem}
            If $(X,\,\tau)$ is a topological space, and if
            $\mathcal{C},\,\mathcal{D}\subseteq{X}$ are closed, then
            $\mathcal{C}\cup\mathcal{D}$ is closed.
        \end{theorem}
        \begin{proof}
            Since $\mathcal{C}$ and $\mathcal{D}$ are closed,
            $X\setminus\mathcal{C}$ and $X\setminus\mathcal{D}$ are open.
            But then:
            \begin{equation}
                X\setminus(\mathcal{C}\cup\mathcal{D})
                =(X\setminus\mathcal{C})\cap(X\setminus\mathcal{D})
            \end{equation}
            which is the intersection of two open sets, which is therefore open,
            so $X\setminus(\mathcal{C}\cup\mathcal{D})$ is open. But then
            $\mathcal{C}\cup\mathcal{D}$ is closed.
        \end{proof}
        \begin{theorem}
            If $(X,\,\tau)$ is a topological space, and if
            $\mathcal{O}\subseteq\mathcal{P}(X)$ is such that for all
            $\mathcal{C}\in\mathcal{O}$ it is true that $\mathcal{C}$ is closed,
            then $\bigcap\mathcal{O}$ is closed.
        \end{theorem}
        \begin{proof}
            If $\mathcal{O}$ is empty, then $\bigcap\mathcal{O}=\emptyset$,
            which is closed. Otherwise we may write:
            \begin{equation}
                \bigcap\mathcal{O}
                =\bigcap_{\mathcal{C}\in\mathcal{O}}\mathcal{C}
                =\bigcap_{\mathcal{C}\in\mathcal{O}}\Big(
                    X\setminus(X\setminus\mathcal{C})
                \Big)
                =X\setminus\bigcup_{\mathcal{C}\in\mathcal{O}}(
                    X\setminus\mathcal{C}
                )
            \end{equation}
            Since all $\mathcal{C}$ are closed, $X\setminus\mathcal{C}$ is open,
            so this union is open, meaning $\bigcap\mathcal{O}$ is the
            complement of an open set and is therefore closed.
        \end{proof}
        \begin{theorem}
            If $(X,\,\tau)$ is a topological space, and if
            $\mathcal{O}\subseteq\mathcal{P}(X)$ is a finite set such that for
            all $\mathcal{C}\in\mathcal{O}$ it is true that $\mathcal{C}$ is
            closed, then $\bigcup\mathcal{O}$ is closed.
        \end{theorem}
        \begin{proof}
            We prove by induction. The base case is true by a previous
            theorem. Suppose the statement is true for all such $\mathcal{O}$
            with $n$ elements. Now, let $\mathcal{O}$ be a set of $n+1$ closed
            sets. That is, we may write
            $\mathcal{O}=\{\,\mathcal{C}_{0},\,\dots,\,\mathcal{C}_{n}\,\}$.
            Define $\mathcal{D}$ via:
            \begin{equation}
                \mathcal{D}=\bigcup_{k=0}^{n-1}\mathcal{C}_{k}
            \end{equation}
            Then $\mathcal{D}$ is the intersection of $n$ closed sets, and by
            the induction hypothesis it is closed. But then:
            \begin{equation}
                \bigcup\mathcal{O}
                =\bigcup_{k=0}^{n}\mathcal{C}_{k}
                =\mathcal{D}\cup\mathcal{C}_{n}
            \end{equation}
            which is the union of two closed sets, which is closed. Hence,
            by induction, $\bigcup\mathcal{O}$ is closed for any finite
            collection of closed sets.
        \end{proof}
        We use the intersection property to define \textit{closure}. Given
        \textit{any} subset $A\subseteq{X}$ in a topological space $(X,\,\tau)$
        there is at least one closed set containing $A$ since
        $A\subseteq{X}$ and $X$ is closed. The \textit{closure} of $A$ is the
        \textit{smallest} closed set containing $A$. We can be very precise
        about this.
        \begin{definition}[\textbf{Closure of a Set}]
            The closure of a subset $A\subseteq{X}$ in a topological space
            $(X,\,\tau)$ the set $\textrm{Cl}_{\tau}(A)$ defined by:
            \begin{equation}
                \textrm{Cl}_{\tau}(A)=\bigcap\{\,\mathcal{C}\subseteq{X}\;|\;
                    \mathcal{C}\textrm{ is closed and }A\subseteq\mathcal{C}\,\}
            \end{equation}
            That is, the \textit{smallest} closed set containing $A$.
        \end{definition}
        \begin{theorem}
            If $(X,\,\tau)$ is a topological space and $A\subseteq{X}$, then
            $A\subseteq\textrm{Cl}_{\tau}(A)$.
        \end{theorem}
        \begin{proof}
            Let $\mathcal{O}$ be the set of all closed sets containing $A$.
            This set is non-empty since $X\in\mathcal{O}$. Given any element
            $\mathcal{C}\in\mathcal{O}$ of the $A\subseteq\mathcal{C}$ by
            definition. Hence, $A\subseteq\bigcap\mathcal{O}$. But
            $\textrm{Cl}_{\tau}(A)=\bigcap\mathcal{O}$, completing the proof.
        \end{proof}
        \begin{theorem}
            If $(X,\,\tau)$ is a topological space, and if $A\subseteq{X}$,
            then $\textrm{Cl}_{\tau}(A)$ is closed.
        \end{theorem}
        \begin{proof}
            Since $\textrm{Cl}_{\tau}(A)$ is the intersection of closed sets,
            it is closed.
        \end{proof}
        \begin{theorem}
            If $(X,\,\tau)$ is a topological space, then
            $\mathcal{C}\subseteq{X}$ is closed if and only if
            $\textrm{Cl}_{\tau}(\mathcal{C})=\mathcal{C}$.
        \end{theorem}
        \begin{proof}
            If $\mathcal{C}=\textrm{Cl}_{\tau}(\mathcal{C})$, then
            $\mathcal{C}$ is closed since $\textrm{Cl}_{\tau}(\mathcal{C})$ is
            closed. In the other direction, if
            $\mathcal{C}$ is closed, then $\mathcal{C}$ is a closed set that
            contains $\mathcal{C}$ since $\mathcal{C}\subseteq\mathcal{C}$. But
            then $\textrm{Cl}_{\tau}(\mathcal{C})\subseteq\mathcal{C}$. But
            $\mathcal{C}\subseteq\textrm{Cl}_{\tau}(\mathcal{C})$ is also true,
            so $\mathcal{C}=\textrm{Cl}_{\tau}(\mathcal{C})$.
        \end{proof}
        \begin{theorem}
            If $(X,\,\tau)$ is a topological space, and if $A\subseteq{X}$,
            then $\textrm{Cl}_{\tau}\big(\textrm{Cl}_{\tau}(A)\big)=\textrm{Cl}_{\tau}(A)$.
        \end{theorem}
        \begin{proof}
            Since $\textrm{Cl}_{\tau}(A)$ is closed, we have that
            $\textrm{Cl}_{\tau}\big(\textrm{Cl}_{\tau}(A)\big)=\textrm{Cl}_{\tau}(A)$
            by the previous theorem.
        \end{proof}
        \begin{example}
            Take $\mathbb{R}$ with the standard topology. Let $\mathbb{Q}$
            be the set of all rational numbers. The closure of
            $\mathbb{Q}$ is all of $\mathbb{R}$. Every real number can be
            written as a limit point of $\mathbb{Q}$ since we may approximate
            any $x\in\mathbb{R}$ with a convergent sequence of rational numbers.
            Because of this
            $\textrm{Cl}_{\tau_{\mathbb{R}}}(\mathbb{Q})=\mathbb{R}$.
        \end{example}
        \begin{example}
            The closure of the empty set in any topological space $(X,\,\tau)$
            is just the empty set. This is a corrolary of the fact that for
            closed  subsets $\mathcal{C}$ we have that
            $\textrm{Cl}_{\tau}(\mathcal{C})=\mathcal{C}$. The empty set is,
            in particular, a closed subset. For similar reasons,
            $\textrm{Cl}_{\tau}(X)=X$.
        \end{example}
        The interior of a set uses similar ideas, but using open sets and
        unions.
        \begin{definition}[\textbf{Interior of a Set}]
            
        \end{definition}
\end{document}
