%-----------------------------------LICENSE------------------------------------%
%   This file is part of Mathematics-and-Physics.                              %
%                                                                              %
%   Mathematics-and-Physics is free software: you can redistribute it and/or   %
%   modify it under the terms of the GNU General Public License as             %
%   published by the Free Software Foundation, either version 3 of the         %
%   License, or (at your option) any later version.                            %
%                                                                              %
%   Mathematics-and-Physics is distributed in the hope that it will be useful, %
%   but WITHOUT ANY WARRANTY; without even the implied warranty of             %
%   MERCHANTABILITY or FITNESS FOR A PARTICULAR PURPOSE.  See the              %
%   GNU General Public License for more details.                               %
%                                                                              %
%   You should have received a copy of the GNU General Public License along    %
%   with Mathematics-and-Physics.  If not, see <https://www.gnu.org/licenses/>.%
%------------------------------------------------------------------------------%
\documentclass{article}
\usepackage{graphicx}                           % Needed for figures.
\usepackage{amsmath}                            % Needed for align.
\usepackage{amssymb}                            % Needed for mathbb.
\usepackage{amsthm}                             % For the theorem environment.
\usepackage{float}
\usepackage[font=scriptsize,
            labelformat=simple,
            labelsep=colon]{subcaption} % Subfigure captions.
\usepackage[font={scriptsize},
            hypcap=true,
            labelsep=colon]{caption}    % Figure captions.
\usepackage{hyperref}
\hypersetup{
    colorlinks=true,
    linkcolor=blue,
    filecolor=magenta,
    urlcolor=Cerulean,
    citecolor=SkyBlue
}

%------------------------Theorem Styles-------------------------%
\theoremstyle{plain}
\newtheorem{theorem}{Theorem}[section]

% Define theorem style for default spacing and normal font.
\newtheoremstyle{normal}
    {\topsep}               % Amount of space above the theorem.
    {\topsep}               % Amount of space below the theorem.
    {}                      % Font used for body of theorem.
    {}                      % Measure of space to indent.
    {\bfseries}             % Font of the header of the theorem.
    {}                      % Punctuation between head and body.
    {.5em}                  % Space after theorem head.
    {}

% Define default environments.
\theoremstyle{normal}
\newtheorem{examplex}{Example}[section]
\newtheorem{definitionx}{Definition}[section]

\newenvironment{example}{%
    \pushQED{\qed}\renewcommand{\qedsymbol}{$\blacksquare$}\examplex%
}{%
    \popQED\endexamplex%
}

\newenvironment{definition}{%
    \pushQED{\qed}\renewcommand{\qedsymbol}{$\blacksquare$}\definitionx%
}{%
    \popQED\enddefinitionx%
}

\title{Point-Set Topology: Lecture 16}
\author{Ryan Maguire}
\date{Summer 2022}

% No indent and no paragraph skip.
\setlength{\parindent}{0em}
\setlength{\parskip}{0em}

\begin{document}
    \maketitle
    \section{Finite Products}
        Continuing with our trend of building new topological spaces, thus far
        we have subspaces and quotients. If $(X,\,\tau_{X})$ and
        $(Y,\,\tau_{Y})$ are topological spaces, it is possible to put a
        topology on the Cartesian product $X\times{Y}$ in a way that respects
        the topologies $\tau_{X}$ and $\tau_{Y}$. It is somewhat natural to hope
        that the set $\tilde{\tau}_{X\times{Y}}$ defined by:
        \begin{equation}
            \tilde{\tau}_{X\times{Y}}=
            \{\,\mathcal{U}\times\mathcal{V}\subseteq{X}\times{Y}\;|\;
                \mathcal{U}\in\tau_{X}\textrm{ and }\mathcal{V}\in\tau_{Y}\,\}
        \end{equation}
        would be a topology on $X\times{Y}$, but it usually is not. Consider the
        real line $\mathbb{R}$ with the standard topology $\tau_{\mathbb{R}}$.
        This has a basis $\mathcal{B}$ consisting of open intervals of the
        form $(a,\,b)$ for all $a,b\in\mathbb{R}$. The product
        $\mathbb{R}\times\mathbb{R}$ should just be the Euclidean plane
        $\mathbb{R}^{2}$, but open sets of the form $(a,\,b)\times(c,\,d)$ are
        just \textit{open rectangles}. The product of more general open subsets
        $\mathcal{U},\mathcal{V}\subseteq\mathbb{R}$ could not possible form an
        open disk in the plane, even though we want an open disk to be, well,
        \textit{open}.
        \par\hfill\par
        The set $\tilde{\tau}_{X\times{Y}}$ is
        nearly a topology. The empty set is contained in it since
        $\emptyset=\emptyset\times\emptyset$. The entire Cartesian product is
        an element since $X\in\tau_{X}$ and $Y\in\tau_{Y}$, hence
        $X\times{Y}\in\tilde{\tau}_{X\times{Y}}$. It is also closed under
        intersections since:
        \begin{equation}
            \big(\mathcal{U}_{0}\times\mathcal{V}_{0}\big)\cap
            \big(\mathcal{U}_{1}\times\mathcal{V}_{1}\big)
            =\big(\mathcal{U}_{0}\cap\mathcal{U}_{1}\big)\times
            \big(\mathcal{V}_{0}\cap\mathcal{V}_{1}\big)
        \end{equation}
        and this is an element of $\tilde{\tau}_{X\times{Y}}$. What fails is
        the union property. Again, think of $\mathbb{R}\times\mathbb{R}$. The
        union of two rectangles does not need to be a rectangle.
        \begin{figure}
            \centering
            \begin{subfigure}[b]{0.49\textwidth}
                \centering
                \includegraphics{../../images/open_rectangle_r2.pdf}
                \caption{An Open Rectangle in $\mathbb{R}^{2}$}
            \end{subfigure}
            \hfill
            \begin{subfigure}[b]{0.49\textwidth}
                \centering
                \includegraphics{../../images/open_not_rectangle_r2.pdf}
                \caption{The Union of Rectangles in $\mathbb{R}^{2}$}
                \label{fig:open_not_rectangle_r2}
            \end{subfigure}
            \caption{Open Subsets of the Plane}
            \label{fig:open_subsets_of_plane}
        \end{figure}
        Moreover, open
        subsets of $\mathbb{R}^{2}$ such as the open unit disk can not be
        written in the form $\mathcal{U}\times\mathcal{V}$ for open subsets
        $\mathcal{U},\mathcal{V}\subseteq\mathbb{R}$. To ensure the product
        topology is indeed a topology, we need to take the topology
        \textit{generated} from $\tilde{\tau}_{X\times{Y}}$.
        \begin{definition}[\textbf{Product Topology of Two Topologies}]
            The product topology of two topologies $\tau_{X}$ and $\tau_{Y}$
            on sets $X$ and $Y$, respectively, is the topology generated by
            the set $\mathcal{B}$ defined by:
            \begin{equation}
                \mathcal{B}=
                \{\,\mathcal{U}\times\mathcal{V}\subseteq{X}\times{Y}\;|\;
                    \mathcal{U}\in\tau_{X}\textrm{ and }
                    \mathcal{V}\in\tau_{Y}\,\}
            \end{equation}
            This is denoted $\tau_{X\times{Y}}$.
        \end{definition}
        \begin{theorem}
            If $(X,\,\tau_{X})$ and $(Y,\,\tau_{Y})$ are topological spaces, and
            if $\tau_{X\times{Y}}$ is the product topology of $\tau_{X}$ and
            $\tau_{Y}$, then $(X\times{Y},\,\tau_{X\times{Y}})$ is a topological
            space.
        \end{theorem}
        \begin{proof}
            The product topology $\tau_{X\times{Y}}$ is a generated topology,
            by definition, which is hence a topology, so
            $(X\times{Y},\,\tau_{X\times{Y}})$ is a topological space.
        \end{proof}
        The product of two topological spaces better give us the right
        topologies on familiar spaces, otherwise its useless.
        \begin{theorem}
            If $\tau_{\mathbb{R}\times\mathbb{R}}$ is the product topology of
            $\tau_{\mathbb{R}}$ and $\tau_{\mathbb{R}}$, and if
            $\tau_{\mathbb{R}^{2}}$ is the standard Euclidean topology on
            $\mathbb{R}^{2}$, then
            $(\mathbb{R}^{2},\,\tau_{\mathbb{R}^{2}})$ and
            $(\mathbb{R}\times\mathbb{R},\,\tau_{\mathbb{R}\times\mathbb{R}})$
            are homeomorphic.
        \end{theorem}
        \begin{proof}
            We simply must prove the topologies $\tau_{\mathbb{R}^{2}}$ and
            $\tau_{\mathbb{R}\times\mathbb{R}}$ are the same set, meaning the
            identity function
            $\textrm{id}:\mathbb{R}^{2}\rightarrow\mathbb{R}\times\mathbb{R}$
            is a homeomorphism. The standard topology on $\mathbb{R}^{2}$
            is generated by the Euclidean metric:
            \begin{equation}
                d(\mathbf{x},\,\mathbf{y})=||\mathbf{x}-\mathbf{y}||_{2}
            \end{equation}
            The topology of $\mathbb{R}\times\mathbb{R}$ is generated by open
            rectangles, which in turn can be generated by open squares, and
            open squares are the open balls in the max metric:
            \begin{equation}
                d_{\textrm{max}}(\mathbf{x},\,\mathbf{y})
                =\textrm{max}(|x_{0}-y_{0}|,\,|x_{1}-y_{1}|)
            \end{equation}
            But the Euclidean metric and the max metric are topologically
            equivalent metrics, meaning they produce the same topologies.
            So, we're done.
        \end{proof}
        \begin{definition}[\textbf{Projection Maps}]
            The projection map of the Cartesian product $X\times{Y}$ onto
            $X$ is the function $\textrm{proj}_{X}:X\times{Y}\rightarrow{X}$
            defined by $\prod_{X}\big((x,\,y)\big)=x$. The projection map of
            $X\times{Y}$ onto $Y$ is defined by
            $\textrm{proj}_{Y}:X\times{Y}\rightarrow{Y}$,
            $\textrm{prof}_{Y}\big((x,\,y)\big)=y$.
        \end{definition}
        Projections are continuous.
        \begin{theorem}
            If $(X,\,\tau_{X})$ and $(Y,\,\tau_{Y})$ are topological spaces,
            and if $(X\times{Y},\,\tau_{X\times{Y}})$ is the product space,
            then $\textrm{proj}_{X}:X\times{Y}\rightarrow{X}$ and
            $\textrm{proj}_{X}:X\times{Y}\rightarrow{Y}$ are continuous.
        \end{theorem}
        \begin{proof}
            The proof is symmetric for $X$ and $Y$, so we'll prove
            $\textrm{proj}_{X}:X\times{Y}\rightarrow{X}$ is continuous.
            Let $\mathcal{U}\in\tau_{X}$. Then by the definition of the
            projection map,
            $\textrm{proj}^{-1}[\mathcal{U}]=\mathcal{U}\times{Y}$, and
            $\mathcal{U}\times{Y}\in\tau_{X\times{Y}}$, so
            $\textrm{proj}_{X}$ is continuous. Similarly,
            $\textrm{proj}_{Y}$ is continuous.
        \end{proof}
        \begin{theorem}
            If $(X,\,\tau_{X})$ and $(Y,\,\tau_{Y})$ are topological spaces,
            and if $(X\times{Y},\,\tau_{X\times{Y}})$ is the product space,
            then $\textrm{proj}_{X}:X\times{Y}\rightarrow{X}$ and
            $\textrm{proj}_{X}:X\times{Y}\rightarrow{Y}$ are open maps.
        \end{theorem}
        \begin{proof}
            Let $\mathcal{W}\in\tau_{X\times{Y}}$. Since $\tau_{X\times{Y}}$
            is generated by the basis $\mathcal{B}$ defined by:
            \begin{equation}
                \mathcal{B}=\{\,
                    \mathcal{U}\times\mathcal{V}\;|\;
                    \mathcal{U}\in\tau_{X}\textrm{ and }
                    \mathcal{V}\in\tau_{Y}\,\}
            \end{equation}
            there is some collection $\mathcal{O}\subseteq\mathcal{B}$ such
            that $\mathcal{W}=\bigcup\mathcal{O}$. But for each
            $\mathcal{U}\times\mathcal{V}\in\mathcal{O}$ we have
            $\textrm{proj}_{X}[\mathcal{U}\times\mathcal{V}]=\mathcal{U}$ and
            $\textrm{proj}_{Y}[\mathcal{U}\times\mathcal{V}]=\mathcal{V}$.
            So then $\textrm{proj}_{X}[\mathcal{W}]$ is the union of open sets
            in $X$, and $\textrm{proj}_{Y}[\mathcal{W}]$ is the union of open
            sets in $Y$, so $\textrm{proj}_{X}[\mathcal{W}]\in\tau_{X}$ and
            $\textrm{proj}_{Y}[\mathcal{W}]\in\tau_{Y}$. That is,
            $\textrm{proj}_{X}$ and $\textrm{proj}_{Y}$ are open maps.
        \end{proof}
        Unlike quotients, which preserve very few properties, products preserve
        quite a lot of properties.
        \begin{theorem}
            If $(X,\,\tau_{X})$ and $(Y,\,\tau_{Y})$ are Hausdorff topological
            spaces, then $(X\times{Y},\,\tau_{X\times{Y}})$ is a Hausdorff
            topological space.
        \end{theorem}
        \begin{proof}
            Let $(x_{0},\,y_{0}),(x_{1},\,y_{1})\in{X}\times{Y}$ with
            $(x_{0},\,y_{0})\ne(x_{1},\,y_{1})$. Then either
            $x_{0}\ne{x}_{1}$ or $y_{0}\ne{y}_{1}$. Suppose
            $x_{0}\ne{x}_{1}$, the proof is symmetric if
            $y_{0}\ne{y}_{1}$. But $(X,\,\tau_{X})$ is Hausdorff, so there are
            opens sets $\mathcal{U},\mathcal{V}\in\tau_{X}$ such that
            $x_{0}\in\mathcal{U}$, $x_{1}\in\mathcal{V}$, and
            $\mathcal{U}\cap\mathcal{V}=\emptyset$. But then
            $\mathcal{U}\times{Y}$ and $\mathcal{V}\times{Y}$ are disjoint open
            sets such that $(x_{0},\,y_{0})\in\mathcal{U}\times{Y}$ and
            $(x_{1},\,y_{1})\in\mathcal{V}\times{Y}$. Therefore
            $(X\times{Y},\,\tau_{X\times{Y}})$ is a Hausdorff topological space.
        \end{proof}
        \begin{theorem}
            If $(X,\,\tau_{X})$ and $(Y,\,\tau_{Y})$ are second-countable
            topological spaces, then $(X\times{Y},\,\tau_{X\times{Y}})$ is
            second-countable.
        \end{theorem}
        \begin{proof}
            Since $(X,\,\tau_{X})$ is second-countable, there is a countable
            basis $\mathcal{B}_{X}$ for $\tau_{X}$. Since $(Y,\,\tau_{Y})$ is
            second-countable, there is a countable basis $\mathcal{B}_{Y}$ for
            $\tau_{Y}$. Let $\mathcal{B}$ be defined by:
            \begin{equation}
                \mathcal{B}=
                    \{\,\mathcal{U}\times\mathcal{V}\;|\;
                    \mathcal{U}\in\mathcal{B}_{X}\textrm{ and }
                    \mathcal{V}\in\mathcal{B}_{Y}\,\}
            \end{equation}
            Then $\mathcal{B}$ has the cardinality of
            $\mathbb{N}\times\mathbb{N}$, which is countable. We now need to
            prove $\mathcal{B}$ is a countable basis of $\tau_{X\times{Y}}$.
            $\mathcal{B}$ is an open cover. First, every element is the
            product of open sets in $X$ and $Y$, which is open in
            $X\times{Y}$, so $\mathcal{B}\subseteq\tau_{X\times{Y}}$. Second,
            it covers the space. Given $(x,\,y)\in{X}\times{Y}$, since
            $x\in{X}$ and $\mathcal{B}_{X}$ is a basis of $\tau_{X}$, there is
            a $\mathcal{U}\in\mathcal{B}_{X}$ such that $x\in\mathcal{U}$.
            Since $y\in{Y}$ and $\mathcal{B}_{Y}$ is a basis of $\tau_{Y}$,
            there is a $\mathcal{V}\in\mathcal{B}_{Y}$ such that
            $y\in\mathcal{V}$. But then
            $\mathcal{U}\times\mathcal{V}\in\mathcal{B}$, by definition, and
            $(x,\,y)\in\mathcal{U}\times\mathcal{V}$. So $\mathcal{B}$ is an
            open cover of $X\times{Y}$. Now, to show it has the basis property.
            Let $\mathcal{W}\in\tau_{X\times{Y}}$.
        \end{proof}
\end{document}
