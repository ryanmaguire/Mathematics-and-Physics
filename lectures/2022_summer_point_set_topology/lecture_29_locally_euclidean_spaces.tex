%-----------------------------------LICENSE------------------------------------%
%   This file is part of Mathematics-and-Physics.                              %
%                                                                              %
%   Mathematics-and-Physics is free software: you can redistribute it and/or   %
%   modify it under the terms of the GNU General Public License as             %
%   published by the Free Software Foundation, either version 3 of the         %
%   License, or (at your option) any later version.                            %
%                                                                              %
%   Mathematics-and-Physics is distributed in the hope that it will be useful, %
%   but WITHOUT ANY WARRANTY; without even the implied warranty of             %
%   MERCHANTABILITY or FITNESS FOR A PARTICULAR PURPOSE.  See the              %
%   GNU General Public License for more details.                               %
%                                                                              %
%   You should have received a copy of the GNU General Public License along    %
%   with Mathematics-and-Physics.  If not, see <https://www.gnu.org/licenses/>.%
%------------------------------------------------------------------------------%
\documentclass{article}
\usepackage{graphicx}                           % Needed for figures.
\usepackage{amsmath}                            % Needed for align.
\usepackage{amssymb}                            % Needed for mathbb.
\usepackage{amsthm}                             % For the theorem environment.
\usepackage{float}
\usepackage{tabularx, booktabs}
\usepackage{mathrsfs}
\usepackage[font=scriptsize,
            labelformat=simple,
            labelsep=colon]{subcaption} % Subfigure captions.
\usepackage[font={scriptsize},
            hypcap=true,
            labelsep=colon]{caption}    % Figure captions.
\usepackage{hyperref}
\hypersetup{
    colorlinks=true,
    linkcolor=blue,
    filecolor=magenta,
    urlcolor=Cerulean,
    citecolor=SkyBlue
}

%------------------------Theorem Styles-------------------------%
\theoremstyle{plain}
\newtheorem{theorem}{Theorem}[section]

% Define theorem style for default spacing and normal font.
\newtheoremstyle{normal}
    {\topsep}               % Amount of space above the theorem.
    {\topsep}               % Amount of space below the theorem.
    {}                      % Font used for body of theorem.
    {}                      % Measure of space to indent.
    {\bfseries}             % Font of the header of the theorem.
    {}                      % Punctuation between head and body.
    {.5em}                  % Space after theorem head.
    {}

% Define default environments.
\theoremstyle{normal}
\newtheorem{examplex}{Example}[section]
\newtheorem{definitionx}{Definition}[section]

\newenvironment{example}{%
    \pushQED{\qed}\renewcommand{\qedsymbol}{$\blacksquare$}\examplex%
}{%
    \popQED\endexamplex%
}

\newenvironment{definition}{%
    \pushQED{\qed}\renewcommand{\qedsymbol}{$\blacksquare$}\definitionx%
}{%
    \popQED\enddefinitionx%
}

\title{Point-Set Topology: Lecture 27}
\author{Ryan Maguire}
\date{Summer 2022}

% No indent and no paragraph skip.
\setlength{\parindent}{0em}
\setlength{\parskip}{0em}

\begin{document}
    \maketitle
    \section{Locally Euclidean Topological Spaces}
        We now reach the final topic of the course, \textit{manifolds}. More
        general than manifolds, we start with locally Euclidean spaces.
        \begin{definition}[\textbf{Locally Euclidean Topological Space}]
            A locally Euclidean topological space is a topological space
            $(X,\,\tau)$ such that for all $x\in{X}$ there is a
            $\mathcal{U}\in\tau$ an $n\in\mathbb{N}$, and an injective
            continuous open mapping
            $\varphi:\mathcal{U}\rightarrow\mathbb{R}^{n}$ such that
            $x\in\mathcal{U}$.
        \end{definition}
        \begin{example}
            For all $n\in\mathbb{N}$ the Euclidean space $\mathbb{R}^{n}$ with
            the Euclidean topology is locally Euclidean. For all
            $\mathbf{x}\in\mathbb{R}^{n}$ choose
            $\mathcal{U}=\mathbb{R}^{n}$ and
            $f:\mathbb{R}^{n}\rightarrow\mathbb{R}^{n}$ to be the identity
            mapping $f(\mathbf{x})=\textrm{id}_{\mathbb{R}^{n}}(\mathbf{x})=\mathbf{x}$.
        \end{example}
        \begin{example}
            If $\mathcal{U}\subseteq\mathbb{R}^{n}$ is an open subset with
            respect to the Euclidean topology, then
            $(\mathcal{U},\,\tau_{\mathbb{R}^{n}_{\mathcal{U}}})$ is locally
            Euclidean. Given $x\in\mathcal{U}$ define
            $f:\mathcal{U}\rightarrow\mathbb{R}^{n}$ via
            $f=\textrm{id}_{\mathbb{R}^{n}}|_{\mathcal{U}}$. Then
            $f$ is injective and continuous, and since $\mathcal{U}$ is open,
            $f$ is also an open mapping.
        \end{example}
        \begin{example}
            The solution to $y^{2}-x^{2}=0$ in the plane is not locally
            Euclidean. This forms an \textbf{X}, $y=\pm|x|$. Every point except
            the origin is locally Euclidean, locally it looks like
            $\mathbb{R}$. The origin is where this goes wrong. No matter how
            much you zoom in it still locally looks like an \textbf{X}. This is
            certainly not locally like $\mathbb{R}$, but it's also not 2
            dimensional. Similarly, it's not $n$ dimensional for any
            $n\in\mathbb{N}$. This this subspace of $\mathbb{R}^{2}$ is not
            locally Euclidean.
        \end{example}
        This example tells us that closed subspaces of locally Euclidean spaces
        do not need to be locally Euclidean.
        \begin{example}
            The bug-eyed line is locally Euclidean, second countable, but not
            Hausdorff. Every point other than the two origins is locally like
            $\mathbb{R}$. The two origins are also locally like $\mathbb{R}$.
            See Fig.~\ref{fig:bug_eyed_line}.
        \end{example}
        \begin{figure}
            \centering
            \includegraphics{../../images/bug_eyed_line_003.pdf}
            \caption{The Bug-Eyed Line}
            \label{fig:bug_eyed_line}
        \end{figure}
        \begin{example}
            The branching line is another example of a non-Hausdorff space that
            is locally Euclidean. The construction is similar to the bug-eyed
            line. Take $X\subseteq\mathbb{R}^{2}$ to be the set of all points
            of the form $(x,\,y)\in\mathbb{R}^{2}$ such that $x\in\mathbb{R}$
            and $y=\pm{1}$. Define $(x_{0},\,y_{0})R(x_{1},\,y_{1})$ if and
            only if $x_{0}=x_{1}$ and $x_{0}<0$. Let $\tilde{R}$ be the
            equivalence relation induced by $R$. The branching line is the
            quotient $X/\tilde{R}$ with the quotient topology
            (See Fig.~\ref{fig:branching_line_001}). Like the bug-eyed line,
            it too is locally Euclidean, see Fig.~\ref{fig:branching_line_002}.
        \end{example}
        \begin{figure}
            \centering
            \includegraphics{../../images/branching_line_001.pdf}
            \caption{The Branching Line Construction}
            \label{fig:branching_line_001}
        \end{figure}
        \begin{figure}
            \centering
            \includegraphics{../../images/branching_line_002.pdf}
            \caption{The Branching Line is Locally Euclidean}
            \label{fig:branching_line_002}
        \end{figure}
        \begin{example}
            The long line is locally Euclidean and Hausdorff, but not
            second countable. It is also not paracompact.
        \end{example}
        \begin{example}
            As far as set theory is concerned, a function
            $f:A\rightarrow{B}$ from a set $A$ to a set $B$ is a subset of
            $A\times{B}$ satisfying certain properties. We can use this to
            define locally Euclidean topological spaces by looking at
            continuous functions from $\mathbb{R}^{m}$ to $\mathbb{R}^{n}$ for
            some $m,n\in\mathbb{N}$. Given
            $f:\mathbb{R}^{m}\rightarrow\mathbb{R}^{n}$, continuous,
            $f\subseteq\mathbb{R}^{m}\times\mathbb{R}^{n}$ can be given the
            subspace topology. This makes it a closed subset since $f$ is
            continuous. It is also a locally Euclidean subspace. For given
            $\big(\mathbf{x},\,f(\mathbf{x})\big)\in{f}$, let
            $\mathcal{U}=f$ and define $F:f\rightarrow\mathbb{R}^{m}$ via:
            \begin{equation}
                F\big((\mathbf{x},\,f(\mathbf{x})\big)=\mathbf{x}
            \end{equation}
            This is just the projection of the elements of
            $f\subseteq\mathbb{R}^{m}\times\mathbb{R}^{n}$ onto
            $\mathbb{R}^{m}$. Since projections are continuous open mappings,
            we need only show $F$ is injective. But $f$ is a function, so given:
            \begin{equation}
                \big(\mathbf{x}_{0},\,f(\mathbf{x}_{0})\big)
                \ne\big(\mathbf{x}_{1},\,f(\mathbf{x}_{1})\big)
            \end{equation}
            we must have $\mathbf{x}_{0}\ne\mathbf{x}_{1}$ (since if
            $\mathbf{x}_{0}=\mathbf{x}_{1}$, then
            $f(\mathbf{x}_{0})=f(\mathbf{x}_{1})$ by definition of a function).
            So then:
            \begin{equation}
                F\big((\mathbf{x}_{0},\,f(\mathbf{x}_{0})\big)
                \ne{F}\big((\mathbf{x}_{1},\,f(\mathbf{x}_{1})\big)
            \end{equation}
            meaning $F$ is injective. Hence, $f$ is a locally Euclidean subspace
            of $\mathbb{R}^{m}\times\mathbb{R}^{n}$.
        \end{example}
        \begin{example}
            $\mathbb{S}^{1}$ with the subspace topology from $\mathbb{R}^{2}$
            is locally Euclidean. We'll show this in two ways. First, via
            orthographic. We split the circle into four parts:
            \begin{align}
                \mathcal{U}_{\textrm{North}}
                &=\{\,(x,\,y)\in\mathbb{S}^{1}\;|\;y>0\,\}\\
                \mathcal{U}_{\textrm{South}}
                &=\{\,(x,\,y)\in\mathbb{S}^{1}\;|\;y<0\,\}\\
                \mathcal{U}_{\textrm{East}}
                &=\{\,(x,\,y)\in\mathbb{S}^{1}\;|\;x>0\,\}\\
                \mathcal{U}_{\textrm{West}}
                &=\{\,(x,\,y)\in\mathbb{S}^{1}\;|\;x<0\,\}
            \end{align}
            See Fig.~\ref{fig:circle_is_locally_euclidean_001}.
            Then we define four functions:
            \begin{align}
                \varphi_{\textrm{North}}:
                \mathcal{U}_{\textrm{North}}\rightarrow\mathbb{R}
                \quad\quad
                \varphi_{\textrm{North}}\big((x,\,y)\big)&=x\\
                \varphi_{\textrm{North}}:
                \mathcal{U}_{\textrm{South}}\rightarrow\mathbb{R}
                \quad\quad
                \varphi_{\textrm{South}}\big((x,\,y)\big)&=x\\
                \varphi_{\textrm{North}}:
                \mathcal{U}_{\textrm{North}}\rightarrow\mathbb{R}
                \quad\quad
                \varphi_{\textrm{East}}\big((x,\,y)\big)&=y\\
                \varphi_{\textrm{East}}:
                \mathcal{U}_{\textrm{West}}\rightarrow\mathbb{R}
                \quad\quad
                \varphi_{\textrm{West}}\big((x,\,y)\big)&=y
            \end{align}
            Since these are projection mappings, they are continuous. From
            how the four open sets are defined, each is also injective. To
            show it is an open mapping we just need to find a continuous
            inverse with respect to the image of these sets. Note that for all
            four functions the range of $(-1,\,1)$. We have the following
            inverse functions:
            \begin{align}
                \varphi_{\textrm{North}}^{-1}(x)
                &=\big(x,\,\sqrt{1-x^{2}}\big)\\
                \varphi_{\textrm{South}}^{-1}(x)
                &=\big(x,\,-\sqrt{1-x^{2}}\big)\\
                \varphi_{\textrm{East}}^{-1}(y)
                &=\big(\sqrt{1-y^{2}},\,y\big)\\
                \varphi_{\textrm{West}}^{-1}(y)
                &=\big(-\sqrt{1-y^{2}},\,y\big)
            \end{align}
            each of which is continuous since the square root function is
            continuous. The four sets also cover $\mathbb{S}^{1}$, showing that
            $\mathbb{S}^{1}$ is locally Euclidean.circle_is_locally_euclidean_001.pdf
        \end{example}
        \begin{figure}
            \centering
            \includegraphics{../../images/circle_is_locally_euclidean_001.pdf}
            \caption{Cover of $\mathbb{S}^{1}$ with Locally Euclidean Sets}
            \label{fig:circle_is_locally_euclidean_001}
        \end{figure}
        This shows we can cover $\mathbb{S}^{1}$ using four sets each of which
        is homeomorphic to an open subset of $\mathbb{R}^{1}$. We can do better,
        only two sets are needed. Place an observer at the north pole
        $N=(0,\,1)$. Given any other point $(x,\,y)$ the line from the
        observer to the north pole is not parallel
\end{document}
