%-----------------------------------LICENSE------------------------------------%
%   This file is part of Mathematics-and-Physics.                              %
%                                                                              %
%   Mathematics-and-Physics is free software: you can redistribute it and/or   %
%   modify it under the terms of the GNU General Public License as             %
%   published by the Free Software Foundation, either version 3 of the         %
%   License, or (at your option) any later version.                            %
%                                                                              %
%   Mathematics-and-Physics is distributed in the hope that it will be useful, %
%   but WITHOUT ANY WARRANTY; without even the implied warranty of             %
%   MERCHANTABILITY or FITNESS FOR A PARTICULAR PURPOSE.  See the              %
%   GNU General Public License for more details.                               %
%                                                                              %
%   You should have received a copy of the GNU General Public License along    %
%   with Mathematics-and-Physics.  If not, see <https://www.gnu.org/licenses/>.%
%------------------------------------------------------------------------------%
\documentclass{article}
\usepackage{graphicx}                           % Needed for figures.
\usepackage{amsmath}                            % Needed for align.
\usepackage{amssymb}                            % Needed for mathbb.
\usepackage{amsthm}                             % For the theorem environment.
\usepackage{float}
\usepackage{hyperref}
\hypersetup{
    colorlinks=true,
    linkcolor=blue,
    filecolor=magenta,
    urlcolor=Cerulean,
    citecolor=SkyBlue
}

%------------------------Theorem Styles-------------------------%
\theoremstyle{plain}
\newtheorem{theorem}{Theorem}[section]

% Define theorem style for default spacing and normal font.
\newtheoremstyle{normal}
    {\topsep}               % Amount of space above the theorem.
    {\topsep}               % Amount of space below the theorem.
    {}                      % Font used for body of theorem.
    {}                      % Measure of space to indent.
    {\bfseries}             % Font of the header of the theorem.
    {}                      % Punctuation between head and body.
    {.5em}                  % Space after theorem head.
    {}

% Define default environments.
\theoremstyle{normal}
\newtheorem{examplex}{Example}[section]
\newtheorem{definitionx}{Definition}[section]
\newtheorem{notationx}{Notation}[section]

\newenvironment{example}{%
    \pushQED{\qed}\renewcommand{\qedsymbol}{$\blacksquare$}\examplex%
}{%
    \popQED\endexamplex%
}

\newenvironment{definition}{%
    \pushQED{\qed}\renewcommand{\qedsymbol}{$\blacksquare$}\definitionx%
}{%
    \popQED\enddefinitionx%
}

\newenvironment{notation}{%
    \pushQED{\qed}\renewcommand{\qedsymbol}{$\blacksquare$}\notationx%
}{%
    \popQED\endnotationx%
}

\title{Point-Set Topology: Lecture 2}
\author{Ryan Maguire}
\date{Summer 2022}

% No indent and no paragraph skip.
\setlength{\parindent}{0em}
\setlength{\parskip}{0em}

\begin{document}
    \maketitle
    \section{The Cartesian Product and Power Sets}
        \begin{figure}
            \centering
            \includegraphics{../../images/power_set_001.pdf}
            \caption{Power Set of $\{\,1,\,2,\,3\,\}$}
            \label{fig:power_set_001}
        \end{figure}
        The axioms of set theory allow for two more constructions from sets.
        If $A$ is a set, we can think of the set of all \textit{subsets} of
        $A$.
        \begin{definition}[\textbf{Power Sets}]
            The power set of a set $A$ is the set $\mathcal{P}(A)$ defined by:
            \begin{equation}
                \mathcal{P}(A)=\{\,B\;|\;B\subseteq{A}\,\}
            \end{equation}
            That is, the set of all subsets of $A$.
        \end{definition}
        \begin{example}
            Let $A=\emptyset$. The power set of $A$ is
            $\mathcal{P}(A)=\{\,\emptyset\,\}$. Do not confuse this set with
            the empty set. $\{\,\emptyset\,\}$ is the set that contains the
            empty set, and hence is not empty.
        \end{example}
        \begin{example}
            Let $A=\{\,1,\,2\,\}$. The power set $\mathcal{P}(A)$ is:
            \begin{equation}
                \mathcal{P}(A)=\big\{\,\emptyset,\,\{\,1\,\},\,
                    \{\,2\,\},\,\{1,\,2\,\}\,\big\}
            \end{equation}
        \end{example}
        Note that for any set $A$, $\emptyset\subseteq{A}$ is true. So
        $\emptyset\in\mathcal{P}(A)$ is always true. Also, $A\subseteq{A}$
        is true, so $A\in\mathcal{P}(A)$ is also true. We can visualize the
        power set of finite sets via \textit{Hasse Diagrams}
        (Fig.~\ref{fig:power_set_001}).
        \par\hfill\par
        The axioms of set theory also give us the \textit{Cartesian product}.
        Kuratowski, one of the pioneers of point-set topology, tells us how
        we can define \textit{ordered pairs}. Given $a$ and $b$, we write:
        \begin{equation}
            (a,b)=\big\{\,a,\,\{\,a,\,b\,\}\,\big\}
        \end{equation}
        This definition allows us to define ordered pairs using sets. It has
        the property that $(a,b)=(c,d)$ if and only if $a=c$ and $b=d$. In
        particular, if $a$ and $b$ are distinct, then $(a,b)$ and $(b,a)$ are
        different objects. That is, ordered pairs have \textit{order}.
        \begin{definition}[\textbf{Cartesian Product}]
            The Cartesian of a set $A$ with a set $B$ is the set $A\times{B}$
            defined by:
            \begin{equation}
                A\times{B}=\{\,(a,\,b)\;|\;a\in{A}\textrm{ and }b\in{B}\,\}
            \end{equation}
        \end{definition}
        \begin{example}
            If $A=\{\,1,\,2\,\}$ and $B=\{\,a,\,b\,\}$, the Cartesian product
            $A\times{B}$ is the set:
            \begin{equation}
                A\times{B}=\{\,(1,\,a),\,(1,\,b),\,(2,\,a),\,(2,\,b)\,\}
            \end{equation}
            The Cartesian product $B\times{A}$ is slightly different:
            \begin{equation}
                B\times{A}=\{\,(a,\,1),\,(a,\,2),\,(b,\,1),\,(b,\,2)\,\}
            \end{equation}
            In general, if $A$ and $B$ are different sets, then $A\times{B}$
            and $B\times{A}$ are not equal.
        \end{example}
        \begin{example}
            The Euclidean plane $\mathbb{R}^{2}$ is the Cartesian product
            $\mathbb{R}^{2}=\mathbb{R}\times\mathbb{R}$.
        \end{example}
        If we have two subsets $A$ and $B$ of the real line $\mathbb{R}$, we
        can visualize $A\times{B}$ as a subset of $\mathbb{R}^{2}$.
        This is done in Fig.~\ref{fig:cartesian_product_003}. Points in $A$ are
        shown in green, points in $B$ in red, and the Cartesian product is in
        blue.
        \begin{figure}
            \centering
            \includegraphics{../../images/cartesian_product_003.pdf}
            \caption{Cartesian Product of Sets in $\mathbb{R}$}
            \label{fig:cartesian_product_003}
        \end{figure}
        \par\hfill\par
        A function from a set $A$ to a set $B$ is a \textit{rule} that assigns
        to each $a\in{A}$ a unique element $b\in{B}$. We could adopt the word
        function as a primitive, but it is possible to define functions
        precisely using our already developed vocabulary from set theory.
        \begin{definition}[\textbf{Function}]
            A function from a set $A$ to a set $B$ is a subset
            $f\subseteq{A}\times{B}$, denoted $f:A\rightarrow{B}$, such that
            for all $a\in{A}$ there is a unique $b\in{B}$ with $(a,b)\in{f}$.
            We write $b=f(a)$.
        \end{definition}
        Fig.~\ref{fig:function_example_001} shows this definition in action.
        The green denotes $\mathbb{R}^{2}$ and the blue curve that cuts through
        the plane is a subset $f\subseteq\mathbb{R}\times\mathbb{R}$. This is
        our usual notion of \textit{function}, especially functions of a real
        variable.
        \begin{figure}
            \centering
            \includegraphics{../../images/function_example_001.pdf}
            \caption{A Function $f:\mathbb{R}\rightarrow\mathbb{R}$}
            \label{fig:function_example_001}
        \end{figure}
        \par\hfill\par
        Functions can be more abstract and do not need to be represented by
        curves in the plane. Let $A=\{\,1,\,2,\,3,\,4\,\}$ and
        $B=\{\,a,\,b,\,c\,\}$. The diagram in
        Fig.~\ref{fig:function_example_002} depicts a valid function
        $f:A\rightarrow{B}$. To each element in $A$ there is a unique element
        in $B$ it is assigned to. Contrast this with
        Figs.~\ref{fig:non_function_example_001},
        \ref{fig:non_function_example_002}, and \ref{fig:non_function_example_003}.
        \begin{figure}
            \centering
            \includegraphics{../../images/function_example_002.pdf}
            \caption{A Function from $A$ to $B$}
            \label{fig:function_example_002}
        \end{figure}
        \begin{figure}
            \centering
            \includegraphics{../../images/non_function_example_001.pdf}
            \caption{A Non-Function on $\mathbb{R}$}
            \label{fig:non_function_example_001}
        \end{figure}
        \begin{figure}
            \centering
            \includegraphics{../../images/non_function_example_002.pdf}
            \caption{An Abstract Non-Function}
            \label{fig:non_function_example_002}
        \end{figure}
        \begin{figure}
            \centering
            \includegraphics{../../images/non_function_example_003.pdf}
            \caption{Another Non-Function}
            \label{fig:non_function_example_003}
        \end{figure}
        \par\hfill\par
        There are three special types of functions.
        \begin{definition}[\textbf{Injective Function}]
            An injective function from a set $A$ to a set $B$ is a function
            $f:A\rightarrow{B}$ such that for all $x,y\in{A}$,
            $f(x)=f(y)$ if and only if $x=y$.
        \end{definition}
        \begin{example}
            The functions $f(x)=\sqrt{x}$ defined on $\mathbb{R}_{\geq{0}}$,
            $\exp(x)$ defined on $\mathbb{R}$, and
            $\ln(x)$ defined on $\mathbb{R}^{+}$ are all injective.
        \end{example}
        \begin{definition}[\textbf{Surjective Function}]
            
        \end{definition}
\end{document}
