%-----------------------------------LICENSE------------------------------------%
%   This file is part of Mathematics-and-Physics.                              %
%                                                                              %
%   Mathematics-and-Physics is free software: you can redistribute it and/or   %
%   modify it under the terms of the GNU General Public License as             %
%   published by the Free Software Foundation, either version 3 of the         %
%   License, or (at your option) any later version.                            %
%                                                                              %
%   Mathematics-and-Physics is distributed in the hope that it will be useful, %
%   but WITHOUT ANY WARRANTY; without even the implied warranty of             %
%   MERCHANTABILITY or FITNESS FOR A PARTICULAR PURPOSE.  See the              %
%   GNU General Public License for more details.                               %
%                                                                              %
%   You should have received a copy of the GNU General Public License along    %
%   with Mathematics-and-Physics.  If not, see <https://www.gnu.org/licenses/>.%
%------------------------------------------------------------------------------%
\documentclass{article}
\usepackage{graphicx}                           % Needed for figures.
\usepackage{amsmath}                            % Needed for align.
\usepackage{amssymb}                            % Needed for mathbb.
\usepackage{amsthm}                             % For the theorem environment.
\usepackage{float}
\usepackage{hyperref}
\hypersetup{
    colorlinks=true,
    linkcolor=blue,
    filecolor=magenta,
    urlcolor=Cerulean,
    citecolor=SkyBlue
}

%------------------------Theorem Styles-------------------------%

% Define theorem style for default spacing and normal font.
\newtheoremstyle{normal}
    {\topsep}               % Amount of space above the theorem.
    {\topsep}               % Amount of space below the theorem.
    {}                      % Font used for body of theorem.
    {}                      % Measure of space to indent.
    {\bfseries}             % Font of the header of the theorem.
    {}                      % Punctuation between head and body.
    {.5em}                  % Space after theorem head.
    {}

% Define default environments.
\theoremstyle{normal}
\newtheorem{problem}{Problem}

\title{Point-Set Topology: Homework 3}
\date{Summer 2022}

% No indent and no paragraph skip.
\setlength{\parindent}{0em}
\setlength{\parskip}{0em}

\begin{document}
    \maketitle
    \color{blue}
    \begin{problem}
        From class, given a collection of topological spaces
        $(X_{\alpha},\,\tau_{\alpha})$ (for all $\alpha$ in some indexing set
        $I$), the product topology and box topology are formed as follows.
        For the box topology $\tau_{\textrm{Box}}$ we take as a basis:
        \begin{equation}
            \mathcal{B}_{\textrm{Box}}
            =\Big\{\,\prod_{\alpha\in{I}}\mathcal{U}_{\alpha}\;|\;
                \mathcal{U}_{\alpha}\in\tau_{\alpha}
                \textrm{ for all }\alpha\in{I}\,\Big\}
        \end{equation}
        For the product topology $\tau_{\textrm{Prod}}$, we take as a basis:
        \begin{equation}
            \mathcal{B}_{\textrm{Prod}}
            =\Big\{\,\prod_{\alpha\in{I}}\mathcal{U}_{\alpha}\;|\;
                \mathcal{U}_{\alpha}\in\tau_{\alpha},\,
                \mathcal{U}_{\alpha}=X_{\alpha}
                \textrm{ for all but finitely many }\alpha\in{I}\,\Big\}
        \end{equation}
        from this definition it is hopefully clear that
        $\tau_{\textrm{Prod}}\subseteq\tau_{\textrm{Box}}$. Because of this,
        given any topological space $(Y,\,\tau_{Y})$, and a function
        $f:Y\rightarrow\prod_{\alpha\in{I}}X_{\alpha}$, if $f$ is continuous
        with respect to the box topology, then it is continuous with respect to
        the product topology. You will prove the converse is \textbf{false}.
        \begin{itemize}
            \item (3 Points) Prove that if $(X,\,\tau_{X})$ and
                $(Y,\,\tau_{Y})$ are topological spaces, then
                $f:X\rightarrow{Y}$ is continuous if and only if for every
                $x\in{X}$ and for every $\mathcal{V}\in\tau_{Y}$ such that
                $f(x)\in\mathcal{V}$, there is a $\mathcal{U}\in\tau_{X}$ such
                that $x\in\mathcal{U}$ and
                $f[\mathcal{U}]\subseteq\mathcal{V}$. That is, continuity
                can be described by forward images, as well as by pre-images.
            \item (2 Points) Let
                $\mathbb{R}^{\infty}=\prod_{n\in\mathbb{N}}\mathbb{R}$. This is,
                as a set, the set of all sequences
                $a:\mathbb{N}\rightarrow\mathbb{R}$ (review what the product set
                is and convince yourself of this statement). Let
                $f:\mathbb{R}\rightarrow\mathbb{R}^{\infty}$ be defined by
                $f(x)=a$, where $a:\mathbb{N}\rightarrow\mathbb{R}$ is the
                sequence $a_{n}=x$ for all $n\in\mathbb{N}$. Prove $f$ is
                \textit{not} continuous with the box topology.
                [Hint: Pick
                $\mathcal{V}=\prod_{n\in\mathbb{N}}\big(\frac{-1}{n+1},\,\frac{1}{n+1}\big)$.
                Show there is no open set $\mathcal{U}$ containing $0$ such
                that $f[\mathcal{U}]\subseteq\mathcal{V}$].
            \item (3 Points)
                Given the product set $\prod_{\alpha\in{I}}X_{\alpha}$, the
                projection function
                $\pi_{\beta}:\prod_{\alpha\in{I}}X_{\alpha}\rightarrow{X}_{\beta}$
                is the function $\pi_{\beta}(x)=x_{\beta}\in{X}_{\beta}$,
                where $x_{\beta}$ is the $\beta$ component of $x$. We proved in
                class that projections are continuous. You may use this
                freely. Prove that if $(Y,\,\tau_{Y})$ is a topological
                space, and if $\prod_{\alpha\in{I}}X_{\alpha}$ is given the
                product topology, then
                $g:Y\rightarrow\prod_{\alpha\in{I}}X_{\alpha}$ is continuous
                if and only if $\pi_{\alpha}\circ{g}$ is continuous for all
                $\alpha\in{I}$. [Hint: The set of all
                $\prod_{\alpha\in{X}}\mathcal{U}_{\alpha}$ where
                $\mathcal{U}_{\alpha}=X_{\alpha}$ for all but at most
                \textbf{one} $\alpha\in{I}$ is a subbasis for the product
                topology. It then suffices to show the pre-image under $f$ of
                such a set is open in $Y$.]
            \item (2 Points) Using this, prove
                $f:\mathbb{R}\rightarrow\mathbb{R}^{\infty}$ defined above is
                continuous with the product topology.
        \end{itemize}
    \end{problem}
    \color{black}
    \begin{proof}[Solution]
        If $f:X\rightarrow{Y}$ is continuous, given $x\in{X}$ and
        $\mathcal{V}\in\tau_{Y}$ such that $f(x)\in\mathcal{V}$, pick
        $\mathcal{U}=f^{-1}[\mathcal{V}]$. Then $\mathcal{U}$ is open
        since $f$ is continuous, but also $x\in\mathcal{U}$ and
        $f[\mathcal{U}]\subseteq\mathcal{V}$. In the other direction, suppose
        $f:X\rightarrow{Y}$ is such that for all $x\in{X}$ and all
        $\mathcal{V}\in\tau_{Y}$ such that $f(x)\in\mathcal{V}$, there is an
        open set $\mathcal{U}\in\tau_{X}$ such that $x\in\mathcal{U}$ and
        $f[\mathcal{U}]\subseteq\mathcal{V}$. Let $\mathcal{V}\in\tau_{Y}$ be
        open. If $f^{-1}[\mathcal{V}]=\emptyset$, then the pre-image of this
        set is open. Suppose not. Then for all $x\in{f}^{-1}[\mathcal{V}]$ there
        is a $\mathcal{U}_{x}\in\tau_{X}$ such that $x\in\mathcal{U}_{x}$ and
        $f[\mathcal{U}_{x}]\subseteq\mathcal{V}$. But then the set:
        \begin{equation}
            \mathcal{W}=\bigcup_{x\in{f}^{-1}[\mathcal{V}]}\mathcal{U}_{x}
        \end{equation}
        is the union of open sets, which is therefore open. But
        for all $x\in{f}^{-1}[\mathcal{V}]$ we have
        $\mathcal{U}_{x}\subseteq{f}^{-1}[\mathcal{V}]$, and hence
        $\mathcal{W}\subseteq{f}^{-1}[\mathcal{V}]$. But also for all
        $x\in{f}^{-1}[\mathcal{V}]$ we have $x\in\mathcal{U}_{x}$, and hence
        $x\in\mathcal{W}$, and therefore
        $f^{-1}[\mathcal{V}]\subseteq\mathcal{W}$. Thus
        $\mathcal{W}=f^{-1}[\mathcal{V}]$, so $f^{-1}[\mathcal{V}]$ is open.
        Hence $f$ is continuous.
        \par\hfill\par
        The function $f:\mathbb{R}\rightarrow\mathbb{R}^{\infty}$ defined by
        $f(x)=a$ where $a:\mathbb{N}\rightarrow\mathbb{R}$ is the sequence
        $a_{n}=x$ is \textit{not} continuous with respect to the box topology.
        Let
        $\mathcal{V}=\prod_{n\in\mathbb{N}}\big(-\frac{1}{n+1},\,\frac{1}{n+1}\big)$.
        Then $0\in\mathbb{R}$ is such that $f(0)\in\mathcal{V}$ and
        $\mathcal{V}$ is open with respect to the box topology since it is the
        product of open sets. If $f$ is continuous there must be an open set
        $\mathcal{U}\in\tau_{\mathbb{R}}$ such that $0\in\mathcal{U}$ and
        $f[\mathcal{U}]\subseteq\mathcal{V}$. But if
        $\mathcal{U}\subseteq\mathbb{R}$ and $0\in\mathcal{U}$, since the
        standard topology from $\mathbb{R}$ is induced by the Euclidean metric,
        there is an $\varepsilon>0$ such that
        $(-\varepsilon,\,\varepsilon)\subseteq\mathcal{U}$. Let
        $x=\varepsilon/2$. Then $x\in(-\varepsilon,\,\varepsilon)$, so
        $x\in\mathcal{U}$, and hence $f(x)\in\mathcal{V}$. Let $N\in\mathbb{N}$
        be such that $\frac{1}{N+1}<\varepsilon/2$. Then
        $f(x)_{n}=\varepsilon/2>\frac{1}{N+1}$ and hence
        $f(x)_{n}\notin\big(-\frac{1}{N+1},\,\frac{1}{N+1}\big)$, meaning
        $f(x)\notin\mathcal{V}$, a contradiction. So $f$ is not continuous.
        \par\hfill\par
        Next, to show $f:Y\rightarrow\prod_{\alpha\in{I}}X_{\alpha}$ is
        continuous if and only if $\textrm{proj}_{\alpha}\circ{f}$ is continuous
        for all $\alpha\in{I}$, it suffices to prove the pre-image of subbasis
        elements is open. If $\mathcal{U}\in\mathcal{B}$ is simply the entire
        product set, then $f^{-1}[\mathcal{U}]=Y$, which is open. Suppose
        $\mathcal{U}=\prod_{\alpha\in{I}}\mathcal{V}_{\alpha}$ where
        $\mathcal{U}_{\alpha}=X_{\alpha}$ for all $\alpha\in{I}$ except
        $\beta$. Then:
        \begin{align}
            f^{-1}
        \end{align}
    \end{proof}
    \clearpage
    \color{blue}
    \begin{problem}
        Let $(X,\,\tau)$ be a topological space.
        \begin{itemize}
            \item Fr\'{e}chet means for all $x,y\in{X}$ with $x\ne{y}$, there
                exists $\mathcal{U},\mathcal{V}\in\tau$ with $x\in\mathcal{U}$,
                $x\notin\mathcal{V}$, and $y\in\mathcal{V}$,
                $y\notin\mathcal{U}$.
            \item Hausdorff means for all $x,y\in{X}$ with $x\ne{y}$ there
                are $\mathcal{U},\mathcal{V}\in\tau$ such that
                $x\in\mathcal{U}$, $y\in\mathcal{V}$, and
                $\mathcal{U}\cap\mathcal{V}=\emptyset$.
            \item Regular means for all $x\in{X}$ and all closed
                $\mathcal{C}\subseteq{X}$ with $x\notin\mathcal{C}$, there
                exists $\mathcal{U},\mathcal{V}\in\tau$ such that
                $x\in\mathcal{U}$, $\mathcal{C}\subseteq\mathcal{V}$, and
                $\mathcal{U}\cap\mathcal{V}=\emptyset$.
            \item Normal means for all closed
                $\mathcal{C},\mathcal{D}\subseteq{X}$ with
                $\mathcal{C}\cap\mathcal{D}=\emptyset$, there exists
                $\mathcal{U},\mathcal{V}\in\tau$ such that
                $\mathcal{C}\subseteq\mathcal{U}$,
                $\mathcal{D}\subseteq\mathcal{V}$, and
                $\mathcal{U}\cap\mathcal{V}=\emptyset$.
        \end{itemize}
        We proved in class (some time ago) that a topological space $(X,\,\tau)$
        is Fr\'{e}chet if and only if for all $x\in{X}$, the set
        $\{\,x\,\}$ is closed. You may use this freely.
        \begin{itemize}
            \item (2 Points) Prove that if $(X,\,\tau)$ is Fr\'{e}chet and
                regular, then it is Hausdorff.
            \item (2 Points) Prove that if $(X,\,\tau)$ is Fr\'{e}chet and
                normal, then it is regular.
        \end{itemize}
        Authors often assume regular means Hausdorff, and normal means regular.
        You will now prove that unless you explicity require regular to mean
        regular and Hausdorff, these three notions are different.
        \begin{itemize}
            \item (2 Points) Let $X=\mathbb{R}$ and define the topology
                $\tau$ to be all sets $\mathcal{U}\subseteq\mathbb{R}$ such
                that either $\mathcal{U}=\mathbb{R}$ or
                $0\notin\mathcal{U}$. This is the
                \textit{excluded point topology}. Prove $(X,\,\tau)$ is
                normal. [Hint: What do closed sets look like in this space?
                what do \textit{disjoint} closed sets $\mathcal{C}$ and
                $\mathcal{D}$ look like?]
            \item (2 Points) Prove $(X,\,\tau)$ is not regular.
                [Hint: Let $x=1$ and
                $\mathcal{C}=\mathbb{R}\setminus\{\,1\,\}$.
                Is $\mathcal{C}$ closed? Are there any open sets containing
                $\mathcal{C}$ and not $1$?]
            \item (2 Points) Let $X=\mathbb{Z}_{2}$ and
                $\tau=\big\{\,\emptyset,\,\{\,0\,\},\,\mathbb{Z}_{2}\,\big\}$.
                $(X,\,\tau)$ is the \textit{Sierpinski space}. Show
                $(X,\,\tau)$ is not Hausdorff, not regular, but it is
                normal. [Hint: The set has two points, and three open sets
                total. You can just check all combinations.]
            \item (2 Points)
                Let $X=\mathbb{Z}_{2}$, and let $\tau$ be the indiscrete
                topology, $\tau=\{\,\emptyset,\,\mathbb{Z}_{2}\,\}$. Prove
                that $(X,\,\tau)$ is not Hausdorff, but it is regular.
        \end{itemize}
        The last example is a bit involved (you do not need to solve anything.
        Just sit back and enjoy the read). The real line has as a basis all
        open intervals $(a,\,b)$, $a,b\in\mathbb{R}$. The
        \textit{Sorgenfrey line} has as a basis all half-open intervals
        $[a,\,b)$. Give $\mathbb{R}^{2}=\mathbb{R}\times\mathbb{R}$ the product
        topology of the Sorgenfrey line with itself. The anti-diagonal
        $\Delta=\{\,(x,\,-x)\in\mathbb{R}^{2}\;|\;x\in\mathbb{R}\,\}$ and the
        complement $\mathbb{R}^{2}\setminus\Delta$ are both closed, but any
        open sets $\mathcal{U},\mathcal{V}$ with $\Delta\subseteq\mathcal{U}$
        and $\mathbb{R}^{2}\setminus\Delta\subseteq\mathcal{V}$ must overlap,
        $\mathcal{U}\cap\mathcal{V}\ne\emptyset$. This shows that the Sorgenfrey
        plane is \textit{not} normal. It is, however, Hausdorff and regular.
    \end{problem}
    \color{black}
    \clearpage
    \color{blue}
    \begin{problem}
        Let $(X,\,\tau)$ be a metrizable topological space with metric $d$.
        \begin{itemize}
            \item (2 Points)
                Define the distance function as
                $\textrm{dist}:X\times\mathcal{P}(X)\rightarrow\mathbb{R}$ via:
                \begin{equation}
                    \textrm{dist}(x,\,A)=
                    \textrm{inf}\big\{d(x,\,y)\;|\;y\in{A}\,\big\}
                \end{equation}
                where $\textrm{inf}$ means the \textit{infinum}. Prove that if
                $\mathcal{C}$ is closed and $x\notin\mathcal{C}$, then
                $\textrm{dist}(x,\,\mathcal{C})>0$. [Hint: $\mathcal{C}$ closed
                means it has its limit points. If
                $\textrm{dist}(x,\,\mathcal{C})=0$, can you find a sequence
                in $\mathcal{C}$ that converges to $x$? Any contradiction?]
            \item (2 Points) Prove that $(X,\,\tau)$ is regular.
                [Hint: For all $y\in\mathcal{C}$, let
                $\mathcal{U}_{y}=B_{d(x,\,y)/2}^{(X,\,d)}(y)$. Can you find
                an open subset of $X\setminus\mathcal{C}$ that contains
                $x$ and is disjoint from
                $\bigcup_{y\in\mathcal{C}}\mathcal{U}_{y}$?]
            \item (2 Points)
                Prove that $(X,\,\tau)$ is normal. [Hint: Given closed and
                disjoint $\mathcal{C},\mathcal{D}$, use the $\textrm{dist}$
                function to cover $\mathcal{C}$ and $\mathcal{D}$ with open
                balls.]
            \item (2 Points)
                A completely normal topological space is a topological space
                $(Y,\,\tau_{Y})$ such that for every subset $A\subseteq{Y}$ with
                the subspace topology $\tau_{Y_{A}}$, $(A,\,\tau_{Y_{A}})$ is
                normal. Prove that $(X,\,\tau)$ is completely normal.
                [Hint: What are the subspaces of a metric space?]
            \item (2 Points)
                A perfectly normal topological space is a topological space
                $(Y,\,\tau_{Y})$ such that for all non-empty disjoint closed
                sets $\mathcal{C},\mathcal{D}\subseteq{Y}$ there is a continuous
                function $f:Y\rightarrow[0,\,1]$ (with the subspace topology
                from $\mathbb{R}$) such $f^{-1}[\{\,0\,\}]=\mathcal{C}$ and
                $f^{-1}[\{\,1\,\}]=\mathcal{D}$. Prove $(X,\,\tau)$ is
                perfectly normal. [Hint: Consider
                $f(x)=\textrm{dist}(x,\,\mathcal{C})/\big(\textrm{dist}(x,\,\mathcal{C})+\textrm{dist}(x,\,\mathcal{D})\big)$
                Is this well-defined? Is it continuous? Does it do the trick?]
        \end{itemize}
    \end{problem}
    \color{black}
    \clearpage
    \color{blue}
    \begin{problem}
        A compact topological space is a topological space
        $(X,\,\tau)$ such that for all open covers $\mathcal{O}\subseteq\tau$
        of $X$, there is a finite subcover $\Delta\subseteq\mathcal{O}$. A
        locally compact topological space is a topological space
        $(X,\,\tau)$ such that for all $x\in{X}$ there is an open set
        $\mathcal{U}\in\tau$ and a subset $K\subseteq{X}$ such that
        $x\in\mathcal{U}$, $\mathcal{U}\subseteq{K}$, and
        $(K,\,\tau_{K})$ is a compact subspace with the subspace topology.
        \begin{itemize}
            \item (2 Points)
                Prove that a compact Hausdorff space is regular.
                [Hint: From class, all closed subsets of a compact space are
                compact.]
            \item (2 Points)
                Prove that a compact Hausdorff space is normal.
                [Hint: Use the fact that a compact Hausdorff space is regular.]
            \item (\textbf{Bonus}: 2 Points) Prove that locally compact
                Hausdorff implies regular.
        \end{itemize}
    \end{problem}
    \color{black}
\end{document}
