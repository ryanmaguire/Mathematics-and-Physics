%-----------------------------------LICENSE------------------------------------%
%   This file is part of Mathematics-and-Physics.                              %
%                                                                              %
%   Mathematics-and-Physics is free software: you can redistribute it and/or   %
%   modify it under the terms of the GNU General Public License as             %
%   published by the Free Software Foundation, either version 3 of the         %
%   License, or (at your option) any later version.                            %
%                                                                              %
%   Mathematics-and-Physics is distributed in the hope that it will be useful, %
%   but WITHOUT ANY WARRANTY; without even the implied warranty of             %
%   MERCHANTABILITY or FITNESS FOR A PARTICULAR PURPOSE.  See the              %
%   GNU General Public License for more details.                               %
%                                                                              %
%   You should have received a copy of the GNU General Public License along    %
%   with Mathematics-and-Physics.  If not, see <https://www.gnu.org/licenses/>.%
%------------------------------------------------------------------------------%
\documentclass{article}
\usepackage{graphicx}                           % Needed for figures.
\usepackage{amsmath}                            % Needed for align.
\usepackage{amssymb}                            % Needed for mathbb.
\usepackage{amsthm}                             % For the theorem environment.
\usepackage{float}
\usepackage{hyperref}
\hypersetup{
    colorlinks=true,
    linkcolor=blue,
    filecolor=magenta,
    urlcolor=Cerulean,
    citecolor=SkyBlue
}

%------------------------Theorem Styles-------------------------%
\theoremstyle{plain}
\newtheorem{theorem}{Theorem}[section]

% Define theorem style for default spacing and normal font.
\newtheoremstyle{normal}
    {\topsep}               % Amount of space above the theorem.
    {\topsep}               % Amount of space below the theorem.
    {}                      % Font used for body of theorem.
    {}                      % Measure of space to indent.
    {\bfseries}             % Font of the header of the theorem.
    {}                      % Punctuation between head and body.
    {.5em}                  % Space after theorem head.
    {}

% Define default environments.
\theoremstyle{normal}
\newtheorem{examplex}{Example}[section]
\newtheorem{definitionx}{Definition}[section]
\newtheorem{notationx}{Notation}[section]
\newtheorem{axiomx}{Axiom}[section]

\newenvironment{example}{%
    \pushQED{\qed}\renewcommand{\qedsymbol}{$\blacksquare$}\examplex%
}{%
    \popQED\endexamplex%
}

\newenvironment{definition}{%
    \pushQED{\qed}\renewcommand{\qedsymbol}{$\blacksquare$}\definitionx%
}{%
    \popQED\enddefinitionx%
}

\title{Point-Set Topology: Lecture 11}
\author{Ryan Maguire}
\date{Summer 2022}

% No indent and no paragraph skip.
\setlength{\parindent}{0em}
\setlength{\parskip}{0em}

\begin{document}
    \maketitle
    \section{Bases and Subbases}
        The topology on a set $X$ generated by a collection of subsets
        $B\subseteq\mathcal{P}(X)$ gives us the notion of \textit{subbasis}.
        \begin{definition}[\textbf{Subbasis of a Topology}]
            A subbasis of a topology $\tau$ on a set $X$ is a set
            $\mathcal{B}\subseteq\mathcal{P}(X)$ such that
            $\tau=\tau(\mathcal{B})$, where $\tau(\mathcal{B})$
            is the topology generated by $\mathcal{B}$.
        \end{definition}
        A comment on this definition. Some require that $\mathcal{B}$ also
        covers the set $X$. That is, for all $x\in{X}$ there is some
        $\mathcal{U}\in\mathcal{B}$ such that $x\in\mathcal{U}$. To me this is
        slightly superfluous. If you are given a topology $\tau$ and a
        collection $\mathcal{B}$ such that $\tau(\mathcal{B})=\tau$, you can
        get a new collection $\tilde{\mathcal{B}}$ that covers $X$ via
        $\tilde{\mathcal{B}}=\mathcal{B}\cup\{\,X\,\}$. That is, take your
        original collection and just throw the entire set in. Since a topology
        requires the whole space to be open, we see that
        $\tau(\mathcal{B})=\tau(\tilde{\mathcal{B}})$. That is, the topology
        generated by $\mathcal{B}$ is the same as the topology generated by
        $\tilde{\mathcal{B}}$. The benefits of requiring or omitting this
        new constraint are scarce. On the one hand, you can now say that the
        empty set serves as a subbasis of the indiscrete topology since
        $\tau(\emptyset)=\{\,\emptyset,\,X\,\}$. On the other, you may want
        a subbasis to also serve as an open cover in a theorem, and it may be
        nice to not have to explicitly say that the subbasis is an open cover
        every time.
        \par\hfill\par
        With a subbasis you take a collection of subsets of $X$ and declare that
        you \textit{want} these sets to be open. The topology from this subbasis
        $\mathcal{B}$ is the \textit{smallest} topology that contains
        $\mathcal{B}$ as a subset. This is done to define many new topological
        spaces that we can't easily define explicitly using a formula or rule
        for the open sets. 
        \begin{example}
            The standard topology $\tau_{\mathbb{R}}$ on $\mathbb{R}$, the
            metric topology from
            \begin{equation}
                d(x,\,y)=|x-y|
            \end{equation}
            has as a subbasis the
            collection of all open intervals. Let $\mathcal{B}$ be
            defined by:
            \begin{equation}
                \mathcal{B}=\{\,(a,\,b)\subseteq\mathbb{R}\;|\;a,b\in\mathbb{R}
                    \textrm{ and }a<b\,\}
            \end{equation}
            The topology generated by this set is the standard topology on
            $\mathbb{R}$. This collection $\mathcal{B}$ is \textbf{not} a
            topology. It lacks the union criterion. If $a,b,c,d\in\mathbb{R}$
            and $a<b<c<d$, the open intervals
            $(a,\,b)$ and $(c,\,d)$ do not overlap, meaning the union
            $(a,\,b)\cup(c,\,d)$ is two disjoint open intervals, which is not
            itself an open interval. This is shown in
            Fig.~\ref{fig:open_intervals_not_forming_open_interval}.
        \end{example}
        \begin{figure}
            \centering
            \includegraphics{../../images/open_intervals_not_forming_open_interval.pdf}
            \caption{The Union of Open Intervals of the Real Line}
            \label{fig:open_intervals_not_forming_open_interval}
        \end{figure}
        \begin{example}[\textbf{The Countable Extension Topology}]
            Let $\tau_{C}$ by the countable complement topology on
            $\mathbb{R}$ and $\tau_{\mathbb{R}}$ the standard Euclidean
            topology. The countable extension topology is the topology
            $\tau_{E}=\tau(\tau_{C}\cup\tau_{\mathbb{R}})$. That is, the
            topology generated by the union of the countable complement and
            standard topologies. This space is not easy to describe explicitly
            in terms of what the open sets are precisely, but it is easy to
            say what a subbasis is. All open intervals and all sets whose
            complement is countable create a subbasis for this space. This space
            serves as a counterexample to the claim
            \textit{Hausdorff implies metrizable}. The countable complement
            extension topology is Hausdorff since
            $\tau_{\mathbb{R}}\subseteq\tau_{E}$ and $\tau_{\mathbb{R}}$ is
            Hausdorff, but it is not metrizable. This space lacks a lot of the
            properties of metrizable spaces.
            It is not first countable, not regular, not normal, not
            perfectly normal, and not paracompact. We'll discuss all of these
            ideas soon enough.
        \end{example}
        A stronger notion than subbasis is that of a basis. Bases in topological
        spaces are very similar to bases in vector spaces. A basis is a
        collection of open sets that \textit{spans} the topology. Every
        element of the topology can be written as the
        \textit{sum} (union) of elements of the basis.
        \begin{definition}[\textbf{Basis for a Topology}]
            A basis for a topology $\tau$ on a set $X$ is a set
            $\mathcal{B}\subseteq\tau$ such that $\mathcal{B}$ is an
            open cover of $(X,\,\tau)$ and for all
            $\mathcal{U},\mathcal{V}\in\tau$ and for all
            $x\in\mathcal{U}\cap\mathcal{V}$ there is a
            $\mathcal{W}\in\mathcal{B}$ such that $x\in\mathcal{W}$ and
            $\mathcal{W}\subseteq\mathcal{U}\cap\mathcal{V}$.
        \end{definition}
        \begin{figure}
            \centering
            \includegraphics{../../images/basis_condition_001.pdf}
            \caption{Condition for a Basis}
            \label{fig:basis_condition_001}
        \end{figure}
        \begin{theorem}
            If $(X,\,\tau)$ is a topological space and
            $\mathcal{B}\subseteq\tau$, then $\mathcal{B}$ is a basis if and
            only if for all $\mathcal{U}\in\tau$ there is an
            $\mathcal{O}\subseteq\mathcal{B}$ such that
            $\bigcup\mathcal{O}=\mathcal{U}$.
        \end{theorem}
        \begin{proof}
            Suppose $\mathcal{B}\subseteq\tau$ is such that for all
            $\mathcal{W}\in\tau$ there is an $\mathcal{O}\subseteq\mathcal{B}$
            such that $\bigcup\mathcal{O}=\mathcal{W}$. Setting
            $\mathcal{W}=X$ shows that $\mathcal{B}$ is an open cover of
            $(X,\,\tau)$, the first property of a basis. Given
            $\mathcal{U},\mathcal{V}$, and any $x\in\mathcal{U}\cap\mathcal{V}$,
            setting $\mathcal{W}=\mathcal{U}\cap\mathcal{V}$ shows that
            $\mathcal{B}$ has the second property of a basis, so
            $\mathcal{B}$ is a basis. Now, suppose $\mathcal{B}$ is a
            basis and let $\mathcal{U}\in\tau$. Then, setting
            $\mathcal{V}=X$, for all
            $x\in\mathcal{U}\cap\mathcal{V}=\mathcal{U}\cap{X}=\mathcal{U}$
            there is a $\mathcal{W}\in\mathcal{B}$ such that
            $x\in\mathcal{W}$ and
            $\mathcal{W}\subseteq\mathcal{U}\cap\mathcal{V}$. Label this
            set $\mathcal{W}_{x}$ and form the set $\mathcal{O}$ by:
            \begin{equation}
                \mathcal{O}=\{\,\mathcal{W}_{x}\;|\;x\in\mathcal{U}\,\}
            \end{equation}
            For all $x\in\mathcal{U}$ we have $x\in\mathcal{W}_{x}$, so
            $x\in\bigcup\mathcal{O}$. Thus we've shown that
            $\mathcal{U}\subseteq\bigcup\mathcal{O}$. But also for all
            $x\in\mathcal{U}$ we have $\mathcal{W}_{x}\subseteq\mathcal{U}$,
            so $\bigcup\mathcal{O}\subseteq\mathcal{U}$. Therefore,
            $\mathcal{U}=\bigcup\mathcal{O}$.
        \end{proof}
        \begin{theorem}
            If $(X,\,\tau)$ is a topological space, and if
            $\mathcal{B}\subseteq\tau$ is a basis, then for all
            $\mathcal{U}\subseteq{X}$, $\mathcal{U}\in\tau$ if and only if
            there is an $\mathcal{O}\subseteq\mathcal{B}$ such that
            $\mathcal{U}=\bigcup\mathcal{O}$.
        \end{theorem}
        \begin{proof}
            The previous theorem shows that if $\mathcal{B}$ is a basis, then
            we can write $\mathcal{U}\in\tau$ via
            $\mathcal{U}=\bigcup\mathcal{O}$ for some
            $\mathcal{O}\subseteq\mathcal{B}$. In the other direction, if
            $\mathcal{U}=\bigcup\mathcal{O}$ for some
            $\mathcal{O}\subseteq\mathcal{B}$, then since $\mathcal{B}$ is a
            basis it is a subset of $\tau$, and hence $\mathcal{U}$ is the
            union of a collection of open sets which is therefore open.
        \end{proof}
        \begin{figure}
            \centering
            \includegraphics{../../images/open_intervals_form_basis.pdf}
            \caption{Open Intervals in $\mathbb{R}$ Form a Basis}
            \label{fig:open_intervals_form_basis}
        \end{figure}
        \begin{example}
            Open intervals in $\mathbb{R}$ form a basis, as well as a subbasis,
            for the standard topology. Indeed, every basis is also a subbasis
            for any topology $\tau$ on a set $X$.
            Given two open sets $\mathcal{U},\mathcal{V}\subseteq\mathbb{R}$
            and a point $x\in\mathcal{U}\cap\mathcal{V}$, since
            $\mathcal{U}\cap\mathcal{V}$ is open, there is some
            $\varepsilon>0$ such that for all $y\in\mathbb{R}$,
            $|x-y|<\varepsilon$ implies $y\in\mathcal{U}\cap\mathcal{V}$. That
            is, the open interval $(x-\varepsilon,\,x+\varepsilon)$ sits inside
            the set $\mathcal{U}\cap\mathcal{V}$, showing that the collection
            of open intervals forms a basis for the topology of $\mathbb{R}$.
            This is easier to picture if $\mathcal{U}=(a,\,b)$ and
            $\mathcal{V}=(c,\,d)$ with $a<b$, $c<b$, and $c<d$. The
            intersection of $(a,\,b)$ and $(c,\,d)$ is the open interval
            $(c,\,b)$ (See Fig.~\ref{fig:open_intervals_form_basis}). Given
            any point $x\in(c,\,d)$, the interval $(c,\,b)$ is a basis element
            and fits inside of $(a,\,b)\cap(c,\,d)$.
        \end{example}
        \begin{example}
            If $(X,\,\tau)$ is a metrizable space, and if $d$ is a metric on
            $X$ such that $\tau=\tau_{d}$, then the set of all open balls
            in $(X,\,d)$ centered at all points of all radii forms a basis.
            That is, we may define:
            \begin{equation}
                \mathcal{B}=\{\,B_{r}^{(X,\,d)}(x)\subseteq{X}\;|\;
                    x\in{X}\textrm{ and }r>0\,\}
            \end{equation}
            The set $\mathcal{B}$ is a basis for the topology
            $\tau=\tau_{d}$.
        \end{example}
        \begin{example}[\textbf{The Radial Interval Topology}]
            Let $X=\mathbb{R}^{2}$ be the Cartesian plane. The radial interval
            topology is defined on $X$ by giving it the following basis
            $\mathcal{B}$. If $L$ is an open line segment that does not include
            the origin but is contained on a line that passes through the
            origin, then $L\in\mathcal{B}$. If
            $\mathcal{U}\subseteq\mathbb{R}^{2}$ is a collection of open
            line segments through the origin, each of which contains the
            origin, then $\mathcal{U}\in\mathcal{B}$. The set
            $\mathcal{B}$ is a basis for a topology, and this topology
            $\tau_{R}$ is the \textit{radial interval topology} on the
            Cartesian plane. It definitely has the feeling of the Paris plane,
            but it is not. If we let $\tau_{P}$ be the topology of the Paris
            plane, the topology induced by the Paris metric $d_{P}$, then
            $\tau_{P}\subseteq\tau_{R}$. This inclusion does not reverse.
            Take the open interval in the $x$ axis between the points
            $(-1,\,0)$ and $(1,\,0)$ in the plane. This contains the origin and
            is an open interval, so it is included in the basis $\mathcal{B}$,
            and hence is included in the topology $\tau_{R}$. However, open
            balls about the origin
            in the Paris plane are disks, just like in the Euclidean plane.
            So this open interval containing the origin is not open in the
            Paris plane. This example serves as a counterexample to the
            following claim. If $(X,\,\tau)$ is a metrizable topological space,
            and if $(X,\,\tilde{\tau})$ is a topological space such that
            $\tau\subseteq\tilde{\tau}$, then $(X,\,\tilde{\tau})$ is
            metrizable. This is \textbf{false}. Ideas like this occur quite
            often. If $\tau\subseteq\tilde{\tau}$ and $(X,\,\tau)$ is Hausdorff,
            then $(X,\,\tilde{\tau})$ is Hausdorff, this is true. It is
            natural to think a similar claim might hold for metrizable spaces,
            but it does not. The Paris plane is metrizable, it comes from the
            Paris metric. The radial plane is not metrizable, even though
            $\tau_{P}\subseteq\tau_{R}$.
        \end{example}
\end{document}
