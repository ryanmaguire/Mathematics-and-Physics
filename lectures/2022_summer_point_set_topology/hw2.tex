%-----------------------------------LICENSE------------------------------------%
%   This file is part of Mathematics-and-Physics.                              %
%                                                                              %
%   Mathematics-and-Physics is free software: you can redistribute it and/or   %
%   modify it under the terms of the GNU General Public License as             %
%   published by the Free Software Foundation, either version 3 of the         %
%   License, or (at your option) any later version.                            %
%                                                                              %
%   Mathematics-and-Physics is distributed in the hope that it will be useful, %
%   but WITHOUT ANY WARRANTY; without even the implied warranty of             %
%   MERCHANTABILITY or FITNESS FOR A PARTICULAR PURPOSE.  See the              %
%   GNU General Public License for more details.                               %
%                                                                              %
%   You should have received a copy of the GNU General Public License along    %
%   with Mathematics-and-Physics.  If not, see <https://www.gnu.org/licenses/>.%
%------------------------------------------------------------------------------%
\documentclass{article}
\usepackage{graphicx}                           % Needed for figures.
\usepackage{amsmath}                            % Needed for align.
\usepackage{amssymb}                            % Needed for mathbb.
\usepackage{amsthm}                             % For the theorem environment.
\usepackage{float}
\usepackage{hyperref}
\hypersetup{
    colorlinks=true,
    linkcolor=blue,
    filecolor=magenta,
    urlcolor=Cerulean,
    citecolor=SkyBlue
}

%------------------------Theorem Styles-------------------------%

% Define theorem style for default spacing and normal font.
\newtheoremstyle{normal}
    {\topsep}               % Amount of space above the theorem.
    {\topsep}               % Amount of space below the theorem.
    {}                      % Font used for body of theorem.
    {}                      % Measure of space to indent.
    {\bfseries}             % Font of the header of the theorem.
    {}                      % Punctuation between head and body.
    {.5em}                  % Space after theorem head.
    {}

% Define default environments.
\theoremstyle{normal}
\newtheorem{problem}{Problem}

\title{Point-Set Topology: Homework 2}
\date{Summer 2022}

% No indent and no paragraph skip.
\setlength{\parindent}{0em}
\setlength{\parskip}{0em}

\begin{document}
    \maketitle
    \begin{problem}
        A few more notes about metric spaces. A \textit{contraction} on a
        metric space $(X,\,d)$ is a function $f:X\rightarrow{X}$ such that
        for all $x,y\in{X}$ it is true that
        $d(x,\,y)\leq{r}\,d\big(f(x),\,f(y)\big)$ for some fixed $0\leq{r}<1$.
        This means the function $f$ \textit{squeezes} the points together.
        You will prove one of the most celebrated theorems of the theory of
        metric spaces, the \textit{Banach Fixed Point Theorem}.
        If $(X,\,d)$ is a non-empty complete metric space, and if
        $f:X\rightarrow{X}$ is
        a contraction, then there is a unique point $x\in{X}$ such that
        $f(x)=x$. That is, $f$ has a unique \textit{fixed-point}, a point that
        is not changed by $f$.
        \begin{itemize}
            \item (2 Points) Prove that a contraction $f:X\rightarrow{X}$ is
                continuous.
            \item (2 Points) Prove that if $f:X\rightarrow{X}$ has a
                fixed-point $x\in{X}$, then $x$ is the only fixed-point.
                [Hint: What if $y\in{X}$ is another fixed-point? Anything wrong?]
            \item (2 Points) Let $a_{0}\in{X}$ be arbitrary, define $a_{n}$
                inductively via $a_{n+1}=f(a_{n})$. Prove that for all
                $n\in\mathbb{N}$,
                $d(a_{n+1},\,a_{n})\leq{r}^{n}d(a_{1},\,a_{0})$, where
                $0\leq{r}<1$ is a value such that for all
                $x,y\in{X}$ we have $d(x,\,y)\leq{r}\,d\big(f(x),\,f(y)\big)$.
            \item (2 Points) Conclude that $a:\mathbb{N}\rightarrow{X}$ is a
                Cauchy sequence. [Hint: Apply the triangle inequality and use
                the geometric series from calculus].
            \item (2 Points) Since $(X,\,d)$ is complete, the sequence
            converges. Let $x\in{X}$ be such that
            $a_{n}\rightarrow{x}$. Show that $f(x)=x$.
            [Hint: Use the continuity of $f$ that you proved in the first part
            of this problem]
        \end{itemize}
        The first application of this is the
        \textit{Picard-Lindel\"{o}f theorem}, a theorem with widespread use
        in analysis, geometry, and physics. It says if $f(t,\,\mathbf{x})$ is 
        \textit{nice} function (continuous in $t$, Lipschitz continuous in
        $\mathbf{x}$) from some closed rectangle $R$ in
        $\mathbb{R}\times\mathbb{R}^{n}$, if $(t_{0},\,\mathbf{x}_{0})\in{R}$,
        then there is an $\varepsilon>0$ and
        a unique function $\mathbf{x}(t)$ such that:
        \begin{equation}
            \mathbf{x}'(t)=f\big(t,\,\mathbf{x}(t)\big)
        \end{equation}
        satisfying the initial value problem $\mathbf{x}(t_{0})=\mathbf{x}_{0}$
        in the interval $(t_{0}-\varepsilon,\,t_{0}+\varepsilon)$.
        In the single variable case, this implies we may solve
        $\dot{x}(t)=f\big(t,\,x(t)\big)$ for smooth functions $f$. The proof
        constructs the unique solution. Define
        $\phi_{0}(t)=t_{0}$. Inductively define $\phi_{n}(t)$ via:
        \begin{equation}
            \phi_{n+1}(t)=y_{0}+\int_{t_{0}}^{t}f\big(s,\,\phi_{n}(s)\big)\,
                \textrm{d}s
        \end{equation}
        The Banach fixed-point theorem shows, with a bit of work, that
        $\phi_{n}(t)$ converges to a function and that this limit function
        satisfies the initial value problem.
    \end{problem}
    \begin{problem}
        A dense subset of a topological space $(X,\,d)$ is a subset
        $A\subseteq{X}$ such that $\textrm{Cl}_{\tau}(A)=X$. That is, every
        point in $X$ is a limit point of $A$. For example, the rationals
        $\mathbb{Q}$ are a dense subset of the reals $\mathbb{R}$.
        A Baire topological space is a topological space $(X,\,\tau)$ such that
        for any non-empty countable set $\mathcal{O}\subseteq\tau$ with the
        property that $\mathcal{U}\in\mathcal{O}$ implies $\mathcal{U}$ is
        dense, the intersection $\bigcap\mathcal{U}$ is also dense. Here you
        will prove the first of Baire's Category Theorems (Note: The Baire
        category theorem has absolutely nothing to do with category theory.
        The terminology for this theorem came long before category theory was
        initiated). If $(X,\,d)$ is a complete metric space, and if
        $\tau_{d}$ is the metric topology, then $(X,\,\tau_{d})$ is a Baire
        topological space.
        \begin{itemize}
            \item (2 Points) Prove that, for a topological space
                $(Y,\,\tau_{Y})$, $A\subseteq{Y}$ is dense if and only if
                for every open set $\mathcal{U}\subseteq{Y}$, the
                intersection $\mathcal{U}\cap{A}$ is non-empty.
            \item (2 Points) It now suffices to prove that if
                $\mathcal{W}\subseteq{X}$ is open, then
                $\mathcal{W}\cap\bigcap\mathcal{O}$ is non-empty. Show that if
                $\mathcal{V}$ is an open ball,
                $\mathcal{V}=B_{r}^{(X,\,d)}(x)$, then there is an
                $\varepsilon>0$ such that
                $\textrm{Cl}_{\tau}\big(B_{\varepsilon}^{(X,\,d)}(x)\big)\subseteq\mathcal{V}$.
                That is, there is always a \textit{closed ball} inside of an
                open ball.
            \item (2 Points) Since $\mathcal{O}$ is countable, there is a surjective
                sequence $\mathcal{U}:\mathbb{N}\rightarrow\mathcal{O}$. That
                is, we may list the elements of $\mathcal{O}$ as $\mathcal{U}_{0}$,
                $\mathcal{U}_{1}$, and so on. Since $\mathcal{U}_{0}$ is open
                and dense, $\mathcal{U}_{0}\cap\mathcal{W}$ is non-empty. Hence
                there an $a_{0}\in\mathcal{U}_{0}\cap\mathcal{W}$. Since the
                intersection of open sets is open, there is a positive $r_{0}<1$
                such that
                $B_{r_{0}}^{(X,\,d)}(a_{0})\subseteq\mathcal{U}_{0}\cap\mathcal{W}$.
                By the previous part of the problem, there is a positive
                $\varepsilon_{0}<r_{0}$ such that
                $\textrm{Cl}_{\tau_{d}}\big(B_{\varepsilon_{0}}^{(X,\,d)}(a_{0})\big)\subseteq{B}_{r_{0}}^{(X,\,d)}(a_{0})$.
                Recursively we may define $a_{n}$, $r_{n}$, and
                $\varepsilon_{n}$ such that $r_{n}<\frac{1}{n+1}$,
                $\varepsilon_{n}<r_{n}$, and $a_{n}$ is such that:
                \begin{equation}
                    \textrm{Cl}_{\tau_{d}}
                        \big(B_{\varepsilon_{n}}^{(X,\,d)}(a_{n})\big)
                    \subseteq{B}_{r_{n}}^{(X,\,d)}(a_{n})
                    \subseteq\mathcal{W}\cap\bigcap_{k=0}^{n}\mathcal{U}_{n}
                \end{equation}
                Show that $a:\mathbb{N}\rightarrow{X}$ is a Cauchy sequence.
            \item (2 Points) Since $(X,\,d)$ is complete, there is an $x\in{X}$ such that
                $a_{n}\rightarrow{x}$. Show that, for all $n\in\mathbb{N}$ it
                is true that $x\in\mathcal{U}_{n}$.
                [Hint: Since $\textrm{Cl}_{\tau}\big(B_{\varepsilon_{n}}^{(X,\,d)}(a_{n})\big)$
                is closed, it contained all of its limit points. Show that
                $x$ is a limit point of this for all $n$. Conclude that
                $x$ is in $\mathcal{U}_{n}$ since
                $\textrm{Cl}_{\tau}\big(B_{\varepsilon_{n}}^{(X,\,d)}(a_{n})\big)\subseteq\mathcal{U}_{n}$.
            \item (2 Points) Show that $x\in\mathcal{W}$ as well, and therefore
                $x\in\mathcal{W}\cap\bigcap\mathcal{O}$, proving the intersection
                is non-empty, and therefore $\bigcap\mathcal{O}$ is dense.
        \end{itemize}
    \end{problem}
    \begin{problem}
        From class, a Kolmogorov topology on a set $X$ is a topology
        $\tau$ on $X$ such that for all $x,y\in{X}$, there is an open set
        $\mathcal{U}\in\tau$ such that either $x\in\mathcal{U}$ and
        $y\notin\mathcal{U}$, or $x\notin\mathcal{U}$ and $y\in\mathcal{U}$.
        That is, a Kolmogorov topology is a topology where it is always possible
        to tell two points apart using open sets.
        \begin{itemize}
            \item (2 Points)
                There are 8,977,053,873,043 distinct topologies on the set
                $\mathbb{Z}_{10}$, 6,611,065,248,783 Kolmogorov topologies,
                and 4,717,687 topologies that are not homeomorphic. Quite a lot.
                It would be cruel to ask you to find them all. Instead, find all
                distinct topologies on $\mathbb{Z}_{2}$ (there are 4), all
                distinct Kolmogorov topologies (there's 3), all non-homeomorphic
                topologies (3), all non-homeomorphic Kolmogorov topologies
                (2), and all Hausdorff topologies (1). [Hint: This may seem
                like a lot, but it really isn't. Find the 4 topologies on
                $\mathbb{Z}_{2}$. Then examine which are Kolmogorov and which
                are homeomorphic.]
            \item (2 Points)
                On $\mathbb{Z}_{3}$ there are 29 distinct topologies, 19
                distinct Kolmogorov topologies, 9 non-homeomorphic topologies,
                and 5 non-homeomorphic Kolmogorov topologies. Find 2
                non-homeomorphic Kolmogorov topologies.
                [Hint: Hausdorff implies Kolmogorov. Can you find the
                Hausdorff topology?]
        \end{itemize}
    \end{problem}
    \begin{problem}
        (4 Points) Let $(X,\,\tau)$ be a sequential topological space and $R$ an
        equivalence relation on $X$. Prove that the quotient space
        $(X/R,\,\tau_{X/R})$ is sequential as well.
    \end{problem}
    \begin{problem}
        Kazimeirz Kuratowski gave an alternative, but equivalent, definition
        of topology. To him the notion of \textit{closure} was sufficient to
        describe topological spaces. A Kuratowski closure operator on a set
        $X$ is a function $\sigma:\mathcal{P}(A)\rightarrow\mathcal{P}(A)$
        such that:
        \begin{enumerate}
            \item $\sigma(\emptyset)=\emptyset$
            \item $A\subseteq{\sigma}(A)$
            \item $\sigma(A)=\sigma\big(\sigma(A)\big)$
            \item $\sigma(A\cup{B})=\sigma(A)\cup\sigma(B)$
        \end{enumerate}
        A Kuratowski space is an ordered pair $(X,\,\sigma)$ where $X$ is a set
        and $\sigma$ is a Kuratowski closure operator on $X$. We have seen in
        class that, if $(X,\,\tau)$ is a topological space, then
        $\textrm{Cl}_{\tau}$ is a Kuratowski closure operator. Now, let's go
        the other way.
        \begin{itemize}
            \item (2 Points) Show that, given $(X,\,\sigma)$, the set
                $\tau_{\sigma}$ defined by:
                \begin{equation}
                    \tau_{\sigma}=\{\,X\setminus\mathcal{C}\in\mathcal{P}(X)
                        \;|\;\sigma(\mathcal{C})=\mathcal{C}\,\}
                \end{equation}
                is a topology on $X$. (We proved that, in topological spaces,
                $A\subseteq{X}$ being closed is equivalent to
                $\textrm{Cl}_{\tau}(A)=A$. We are
                intuitively defining $\tau_{\sigma}$ as the set of all
                \textit{complements of closed sets}).
            \item (6 Points) If $(X,\,\sigma_{X})$ and $(Y,\,\sigma_{Y})$ are
                Kuratowski spaces, $f:X\rightarrow{Y}$ is continuous if for all
                $A\subseteq{X}$ it is true that
                $f[\sigma_{X}(A)]\subseteq\sigma_{Y}(f[A])$. Show this is
                equivalent to continuity in topology. That is, if
                $(X,\,\tau_{X})$ and $(Y,\,\tau_{Y})$ are topological spaces,
                then $f:X\rightarrow{Y}$ is continuous if and only if
                for all $A\subseteq{X}$ it is true that
                $f[\textrm{Cl}_{\tau_{X}}(A)]\subseteq\textrm{Cl}_{\tau_{Y}}(f[A])$.
                [Hint: We proved $f:X\rightarrow{Y}$ is continuous if and only
                if for all closed $\mathcal{D}\subseteq{Y}$, the pre-image
                $f^{-1}[\mathcal{D}]$ is closed. Use this definition.]
        \end{itemize}
    \end{problem}
\end{document}
