%-----------------------------------LICENSE------------------------------------%
%   This file is part of Mathematics-and-Physics.                              %
%                                                                              %
%   Mathematics-and-Physics is free software: you can redistribute it and/or   %
%   modify it under the terms of the GNU General Public License as             %
%   published by the Free Software Foundation, either version 3 of the         %
%   License, or (at your option) any later version.                            %
%                                                                              %
%   Mathematics-and-Physics is distributed in the hope that it will be useful, %
%   but WITHOUT ANY WARRANTY; without even the implied warranty of             %
%   MERCHANTABILITY or FITNESS FOR A PARTICULAR PURPOSE.  See the              %
%   GNU General Public License for more details.                               %
%                                                                              %
%   You should have received a copy of the GNU General Public License along    %
%   with Mathematics-and-Physics.  If not, see <https://www.gnu.org/licenses/>.%
%------------------------------------------------------------------------------%
\documentclass{article}
\usepackage{graphicx}                           % Needed for figures.
\usepackage{amsmath}                            % Needed for align.
\usepackage{amssymb}                            % Needed for mathbb.
\usepackage{amsthm}                             % For the theorem environment.
\usepackage{float}
\usepackage{tabularx, booktabs}
\usepackage{mathrsfs}
\usepackage[font=scriptsize,
            labelformat=simple,
            labelsep=colon]{subcaption} % Subfigure captions.
\usepackage[font={scriptsize},
            hypcap=true,
            labelsep=colon]{caption}    % Figure captions.
\usepackage{hyperref}
\hypersetup{
    colorlinks=true,
    linkcolor=blue,
    filecolor=magenta,
    urlcolor=Cerulean,
    citecolor=SkyBlue
}

%------------------------Theorem Styles-------------------------%
\theoremstyle{plain}
\newtheorem{theorem}{Theorem}[section]

% Define theorem style for default spacing and normal font.
\newtheoremstyle{normal}
    {\topsep}               % Amount of space above the theorem.
    {\topsep}               % Amount of space below the theorem.
    {}                      % Font used for body of theorem.
    {}                      % Measure of space to indent.
    {\bfseries}             % Font of the header of the theorem.
    {}                      % Punctuation between head and body.
    {.5em}                  % Space after theorem head.
    {}

% Define default environments.
\theoremstyle{normal}
\newtheorem{examplex}{Example}[section]
\newtheorem{definitionx}{Definition}[section]

\newenvironment{example}{%
    \pushQED{\qed}\renewcommand{\qedsymbol}{$\blacksquare$}\examplex%
}{%
    \popQED\endexamplex%
}

\newenvironment{definition}{%
    \pushQED{\qed}\renewcommand{\qedsymbol}{$\blacksquare$}\definitionx%
}{%
    \popQED\enddefinitionx%
}

\title{Point-Set Topology: Lecture 27}
\author{Ryan Maguire}
\date{Summer 2022}

% No indent and no paragraph skip.
\setlength{\parindent}{0em}
\setlength{\parskip}{0em}

\begin{document}
    \maketitle
    \section{Compactifications}
        Compact topological spaces are nice. Unlike the Hausdorff and
        first countable conditions, which most of your every day spaces possess,
        compact is not as common. Compact spaces are just too convenient.
        For example $\mathbb{R}^{n}$ is not compact for all $n>0$ by the
        Heine-Borel theorem. A \textit{compactification} of a topological space
        $(X,\,\tau)$ is a compact space $(\tilde{X},\,\tilde{\tau})$ that
        contains $(X,\,\tau)$ is an embedded subspace. There are two common
        compactifications that are used in topology and analysis, the
        one point compactification, and the Stone-\v{C}ech compactification.
        \par\hfill\par
        The one point compactification, also called the Alexandroff extension
        or the Alexandroff compactification, takes $(X,\,\tau)$ and adds one
        point in a way that makes it compact. This new point is often
        denoted $\infty$, but what if $X$ already had $\infty$ as an element?
        Is there a way to guarantee we are adding a new element to $X$ that is
        not already contained in it? The axioms of set theory tell us if $A$ is
        a set, then $A\notin{A}$, so we can form our new set
        $\tilde{A}$ via $A\cup\{\,A\,\}$.
        \begin{definition}[\textbf{One Point Compactification}]
            The one point compactification of a topological space $(X,\,\tau)$
            is the ordered pair $(\tilde{X},\,\tilde{\tau})$ where
            $\tilde{X}=X\cup\{\,X\,\}$ and $\tilde{\tau}$ is defined by:
            \begin{equation}
                \tilde{\tau}=
                \big\{\,\mathcal{U}\subseteq\tilde{X}\;|\;
                    \mathcal{U}\in\tau\textrm{ or }
                    \mathcal{U}=(X\setminus\mathcal{C})\cup\{\,X\,\},\,
                    \mathcal{C}\textrm{ compact and closed.}\,\big\}
            \end{equation}
            The new element $\{\,X\,\}$ that is added is often denoted
            $\infty$ (if we know the symbol $\infty$ does not already belong to
            $X$). The set $\tilde{\tau}$ is thus the set of all open sets in
            $\tau$ plus all complements of closed compact sets together with
            infinity.
        \end{definition}
        There are several theorems related to the one point compactification
        that we lack the time to go over, so I'll just present them.
        \begin{itemize}
            \item The one point compactification of a topological space is a
                topological space (that is, $\tilde{\tau}$ is a topology on
                $\tilde{X}$).
            \item $(X,\,\tau)$ is a subspace of $(\tilde{X},\,\tilde{\tau})$ and
                the inclusion map $\iota:X\rightarrow\tilde{X}$ is an
                embedding.
            \item The one point compactification of a topological space is
                compact.
            \item If $(X,\,\tau)$ is a non-compact topological space, then it
                is a dense subspace of its one point compactification.
            \item The one point compactification of a topological space
                $(X,\,\tau)$ is Hausdorff if and only if $(X,\,\tau)$ is
                locally compact and Hausdorff. The one point compactification
                of $\mathbb{Q}$ is an example of a non-Hausdorff
                compactification, even though $\mathbb{Q}$ is Hausdorff
                (it is not locally compact, however).
            \item The one point compactification of a topological space
                $(X,\,\tau)$ is Fr\'{e}chet if and only if
                $(X,\,\tau)$ is Fr\'{e}chet.
        \end{itemize}
        The one point compactification makes the following theorem a lot easier.
        \begin{theorem}
            If $(X,\,\tau)$ is locally compact and Hausdorff, then it is
            regular.
        \end{theorem}
        \begin{proof}
            Let $(\tilde{X},\,\tilde{\tau})$ be the one point compactification
            of $(X,\,\tau)$. Since $(X,\,\tau)$ is locally compact and
            Hausdorff, $(\tilde{X},\,\tilde{\tau})$ is compact and Hausdorff.
            But then $(\tilde{X},\,\tilde{\tau})$ is regular. But
            $(X,\,\tau)$ is a subspace of $(\tilde{X},\,\tilde{\tau})$, and
            a subspace of a regular space is regular. Hence,
            $(X,\,\tau)$ is regular.
        \end{proof}
        The Stone-C\v{e}ch compactification is another common tool used in
        analysis and topology, but its description is a lot harder to convey.
        The simplest definition is the \textit{spectrum} of the set
        $C_{b}(X,\,\mathbb{C})$ of bounded continuous functions from a
        topological space $(X,\,\tau)$ into the complex numbers $\mathbb{C}$,
        which is equipped with the standard Euclidean topology of
        $\mathbb{R}^{2}$. A discussion of this idea would require an
        excursion into analysis that we don't have the time for, unfortunately.
        I will mention that the Stone-\v{C}ech compactification allows one to
        describe \textit{completely metrizable} spaces.
        \begin{definition}[\textbf{Completely Metrizable Topological Space}]
            A completely metrizable topological space is a topological space
            $(X,\,\tau)$ such that there is a complete metric $d$ on $X$ that
            induces $\tau$.
        \end{definition}
        \begin{theorem}
            If $(X,\,\tau)$ is a metrizable topological space, then it is
            completely metrizable if and only if it is a $G_{\delta}$ subspace
            of its Stone-\v{C}ech compactification.
        \end{theorem}
        Lastly, Stone spaces are compact totally disconnected Hausdorff
        topological spaces. They arise in the study of the Stone-\v{C}ech
        compactification, but their use is far broader. Every Boolean algebra
        $(B,\,\land,\,\lor)$ is represented by a Stone space, and conversely
        every Stone space represents a Boolean algebra. Boolean algebras
        capture the algebraic structure of logic. Think of propositions
        together with \textit{and} and \textit{or}. Boolean algebras satisfy,
        for all $P,Q,R\in{B}$, the following:
        \begin{align}
            P\land{Q}&=Q\land{P}
                &P\lor{Q}&=Q\lor{P}\\
            P\land(Q\land{R})&=(P\land{Q})\land{R}
                &P\lor(Q\lor{R})&=(P\lor{Q})\lor{R}\\
            P\land(Q\lor{R})&=(P\land{Q})\lor(P\land{R})
                &P\lor(Q\land{R})&=(P\lor{Q})\land(P\lor{R})\\
            P\lor\textrm{False}&=P&P\land\textrm{True}&=P\\
            P\lor\neg{P}&=\textrm{True}&P\land\neg{P}&=\textrm{False}
        \end{align}
        So amazingly topology and logic become the same study.
\end{document}
