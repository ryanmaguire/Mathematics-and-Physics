%-----------------------------------LICENSE------------------------------------%
%   This file is part of Mathematics-and-Physics.                              %
%                                                                              %
%   Mathematics-and-Physics is free software: you can redistribute it and/or   %
%   modify it under the terms of the GNU General Public License as             %
%   published by the Free Software Foundation, either version 3 of the         %
%   License, or (at your option) any later version.                            %
%                                                                              %
%   Mathematics-and-Physics is distributed in the hope that it will be useful, %
%   but WITHOUT ANY WARRANTY; without even the implied warranty of             %
%   MERCHANTABILITY or FITNESS FOR A PARTICULAR PURPOSE.  See the              %
%   GNU General Public License for more details.                               %
%                                                                              %
%   You should have received a copy of the GNU General Public License along    %
%   with Mathematics-and-Physics.  If not, see <https://www.gnu.org/licenses/>.%
%------------------------------------------------------------------------------%
\documentclass{article}
\usepackage{graphicx}                           % Needed for figures.
\usepackage{amsmath}                            % Needed for align.
\usepackage{amssymb}                            % Needed for mathbb.
\usepackage{amsthm}                             % For the theorem environment.
\usepackage{float}
\usepackage{tabularx, booktabs}
\usepackage{mathrsfs}
\usepackage[font=scriptsize,
            labelformat=simple,
            labelsep=colon]{subcaption} % Subfigure captions.
\usepackage[font={scriptsize},
            hypcap=true,
            labelsep=colon]{caption}    % Figure captions.
\usepackage{hyperref}
\hypersetup{
    colorlinks=true,
    linkcolor=blue,
    filecolor=magenta,
    urlcolor=Cerulean,
    citecolor=SkyBlue
}

%------------------------Theorem Styles-------------------------%
\theoremstyle{plain}
\newtheorem{theorem}{Theorem}[section]

% Define theorem style for default spacing and normal font.
\newtheoremstyle{normal}
    {\topsep}               % Amount of space above the theorem.
    {\topsep}               % Amount of space below the theorem.
    {}                      % Font used for body of theorem.
    {}                      % Measure of space to indent.
    {\bfseries}             % Font of the header of the theorem.
    {}                      % Punctuation between head and body.
    {.5em}                  % Space after theorem head.
    {}

% Define default environments.
\theoremstyle{normal}
\newtheorem{examplex}{Example}[section]
\newtheorem{definitionx}{Definition}[section]

\newenvironment{example}{%
    \pushQED{\qed}\renewcommand{\qedsymbol}{$\blacksquare$}\examplex%
}{%
    \popQED\endexamplex%
}

\newenvironment{definition}{%
    \pushQED{\qed}\renewcommand{\qedsymbol}{$\blacksquare$}\definitionx%
}{%
    \popQED\enddefinitionx%
}

\title{Point-Set Topology: Lecture 23}
\author{Ryan Maguire}
\date{Summer 2022}

% No indent and no paragraph skip.
\setlength{\parindent}{0em}
\setlength{\parskip}{0em}

\begin{document}
    \maketitle
    \section{Other Ideas for Compactness}
        Compact and sequentially compact are the two most commonly used
        properties in analysis, geometry, and topology. There are several
        weaker notions that have found there way into several branches of
        mathematics, so now we'll take that time to discuss these ideas.
        \begin{definition}[\textbf{Limit Point Compact Topogical Space}]
            A limit point compact topological space is a topological space
            $(X,\,\tau)$ such that for all infinite subsets $A\subseteq{X}$
            there exists a point $x\in{X}$ such that for all
            $\mathcal{U}\in\tau$ with $x\in\mathcal{U}$, there is a
            $y\in{A}$ such that $y\ne{x}$ and $y\in\mathcal{U}$
        \end{definition}
        Limit point compact was the original defining property of compactness
        when mathematicians were first thinking about the topology of the
        real line. Unlike compactness and sequential compactness, where one
        can't really say one idea is \textit{stronger} than the other, limit
        point compactness is a weaker notion.
        \begin{theorem}
            If $(X,\,\tau)$ is a sequentially compact topological space, then
            it is limit point compact.
        \end{theorem}
        \begin{proof}
            For if not, then there is an infinite set $A\subseteq{X}$ such that
            for all $x\in{X}$ there is a $\mathcal{U}\in\tau$ such that
            $x\in\mathcal{U}$ and for $y\in{A}$ either $y=x$ or
            $y\notin\mathcal{U}$. But if $A$ is infinite, there is a countably
            infinite subset $B\subseteq{A}$. Let $a:\mathbb{N}\rightarrow{B}$
            be a bijection. But $(X,\,\tau)$ is sequentially compact, so there
            is a convergent subsequence $a_{k}$. Let $x\in{X}$. Then there
            is a $\mathcal{U}\in\tau$ such that $x\in\mathcal{U}$ and for
            all $y\in{A}$ either $y=x$ or $y\notin\mathcal{U}$. But
            $a_{k_{n}}\rightarrow{x}$ and $x\in\mathcal{U}$, so there is an
            $N\in\mathbb{N}$ such that for all $n\in\mathbb{N}$ with $n>N$ we
            have $a_{k_{n}}\in\mathcal{U}$. But $a:\mathbb{N}\rightarrow{B}$
            is bijective and $k:\mathbb{N}\rightarrow\mathbb{N}$ is strictly
            increasing, so $a_{k}$ is injective, meaning for all
            $n>N$, $a_{k_{n}}$ are distinct elements of $B$, and hence $A$,
            that are contained in $\mathcal{U}$, which is a contradiction.
            So $(X,\,\tau)$ is limit point compact.
        \end{proof}
        Compact also implies limit point compact, but a weaker notion than
        compact also implies limit point compact. This weaker notion is
        occasionally useful.
        \begin{definition}[\textbf{Countably Compact Topological Space}]
            A countably compact topological space is a topological space
            $(X,\,\tau)$ such that for all countable open covers
            $\mathcal{O}\subseteq\tau$ there is a finite subset
            $\Delta\subseteq\mathcal{O}$ such that $\Delta$ is an open cover.
        \end{definition}
        \begin{theorem}
            If $(X,\,\tau)$ is compact, then it is countably compact.
        \end{theorem}
        \begin{proof}
            Any countable open cover is indeed an open cover, and since
            $(X,\,\tau)$ is compact, there must be a finite subcover.
        \end{proof}
        \begin{theorem}
            If $(X,\,\tau)$ is countably compact, then it is limit point
            compact.
        \end{theorem}
        \begin{proof}
            If not, there is an infinite subsets $A\subseteq{X}$ such that for
            all $x\in{X}$ there is a $\mathcal{U}\in\tau$ such that for all
            $y\in{A}$ either $y=x$ or $y\notin\mathcal{U}$. But since
            $A$ is infinite, there is a countably infinite subset
            $B\subseteq{A}$. But then for all $x\in{X}$ there is a
            $\mathcal{U}\in\tau$ such that for all $y\in{B}$ either
            $y=x$ or $y\notin\mathcal{U}$. But then
            $\textrm{Cl}_{\tau}(B)=B$, so $B$ is closed. For all
            $n\in\mathbb{N}$ let $\mathcal{U}_{n+1}\in\tau$ be such that
            $\mathcal{U}_{n+1}\cap{B}=\{\,a_{n+1}\,\}$. Let
            $\mathcal{U}_{0}=X\setminus\{\,B\,\}$. Then:
            \begin{equation}
                \mathcal{O}=
                \{\,\mathcal{U}_{n}\;|\;n\in\mathbb{N}\,\}
            \end{equation}
            is a countable open cover of $(X,\,\tau)$. But since $(X,\,\tau)$
            is countably compact, there is a finite subcover $\Delta$. But
            then there is a $\mathcal{U}_{n}$ such that infinitely many
            element of $B$ are contained inside $\mathcal{U}_{n}$, which is
            a contradiction. So $(X,\,\tau)$ is limit point compact.
        \end{proof}
        \begin{theorem}
            If $(X,\,\tau)$ is compact, then it is limit point compact.
        \end{theorem}
        \begin{proof}
            Compact implies countably compact which implies limit point compact.
        \end{proof}
        \begin{theorem}
            If $(X,\,\tau)$ if a limit point compact Fr\"{e}chet topological
            space, then it is countably compact.
        \end{theorem}
        Sequentially compact is a nice property, and it is quite a shame
        compactness does not imply it in general. It is also a shame
        sequential compactness does not imply compact. If we add
        \textit{sequential} to our hypothesis, we can get one direction to
        work.
        \begin{theorem}
            If $(X,\,\tau)$ is countably compact, and if
            $\mathcal{C}\subseteq{X}$ is closed, then
            $(\mathcal{C},\,\tau_{\mathcal{C}})$ is countably compact.
        \end{theorem}
        \begin{proof}
            The proof is a mimicry of the idea for compact spaces. Given
            a countable open cover of $\mathcal{C}$, by adding
            $X\setminus\mathcal{C}$ we obtain a countable open cover of $X$
            since $\mathcal{C}$ is closed, so $X\setminus\mathcal{C}$ is open,
            and adding one more set to a countable collection is still
            countable. But since $(X,\,\tau)$ is countably compact there is a
            finite subcover. Restricting this finite subcover to $\mathcal{C}$
            shows that $(\mathcal{C},\,\tau_{\mathcal{C}})$ is countably
            compact. 
        \end{proof}
        \begin{theorem}
            If $(X,\,\tau)$ is a sequential countably compact topological space,
            then it is sequentially compact.
        \end{theorem}
        \begin{proof}
            For if not, then there is a sequence $a:\mathbb{N}\rightarrow{X}$
            with no convergent subsequence. Let $A\subseteq{X}$ be defined by:
            \begin{equation}
                A=\bigcup_{n\in\mathbb{N}}
                    \textrm{Cl}_{\tau}\big(\{\,a_{n}\,\}\big)
            \end{equation}
            Then $A$ is sequentially closed. For if
            $b:\mathbb{N}\rightarrow{A}$ is a sequence that converges to
            $y\in{X}$, either there is an $m\in\mathbb{N}$ such that
            $y\in\textrm{Cl}_{\tau}(\{\,a_{m}\,\})$ or not. If there is
            such an $m$, then $y\in{A}$. If not, then by choosing
            $a_{k}$ and $b_{\ell}$ to be increasing and such that
            $b_{\ell_{n}}\in\textrm{Cl}_{\tau}(\{\,a_{k_{n}}\,\})$, we have
            found a convergent subsequence $a_{k_{n}}\rightarrow{y}$, which is
            a contradiction. Hence $A$ is sequentially closed. But
            $(X,\,\tau)$ is sequential, so $A$ is closed. But then
            $(A,\,\tau_{A})$ is countably compact, where $\tau_{A}$ is the
            subspace topology. But countably compact implies limit point
            compact, so there is a point $x\in{A}$ such that for all
            $\mathcal{U}\in\tau_{A}$ with $x\in\mathcal{U}$, there is a
            $y\in{A}$ such that $y\ne{x}$ and $y\in\mathcal{U}$. But
            $x$ must be in only finitely many sets of the form
            $\textrm{Cl}_{\tau}\big(\{\,a_{n}\,\}\big)$, otherwise
            $a$ would have a convergent subsequence converging to $x$.
            But then there is an $N\in\mathbb{N}$ such that for all
            $n>N$ we have $x\notin\textrm{Cl}_{\tau}\big(\{\,a_{n}\,\}\big)$.
            But then, defining:
            \begin{equation}
                B=\bigcup_{n=N+1}^{\infty}
                    \textrm{Cl}_{\tau}\big(\{\,a_{n}\,\}\big)
            \end{equation}
            we see that $B$ is closed, by the previous argument, but $B$
            does not contain the point $x$, which is a contradiction since
            $x$ is still a limit point of $B$. So $(X,\,\tau)$ is
            sequentially compact.
        \end{proof}
        A short corollary of this is often used when sequential compactness is
        desired.
        \begin{theorem}
            If $(X,\,\tau)$ is compact and first countable, then it is
            sequentially compact.
        \end{theorem}
        \begin{proof}
            Compact implies countably compact, and first countable implies
            sequential. So $(X,\,\tau)$ is countably compact and sequential,
            so it is therefore sequentially compact.
        \end{proof}
        While sequentially compact does not imply compact, there is a partial
        result. Sequentially compact always implies countably compact, and
        often enough countably compact is sufficient.
        \begin{theorem}
            If $(X,\,\tau)$ is sequentially compact, then it is
            countable compact.
        \end{theorem}
        \begin{proof}
            If not there is a countably infinite open cover
            $\mathcal{O}\subseteq\tau$ with no finite subcover. But then,
            since $\mathcal{O}$ is countably infinite, there is a bijection
            $\mathcal{U}:\mathbb{N}\rightarrow\mathcal{O}$ so that we may
            list the elements as:
            \begin{equation}
                \mathcal{O}=
                \{\,\mathcal{U}_{0},\,\dots,\,\mathcal{U}_{n},\,\dots\,\}
            \end{equation}
            But $\mathcal{U}_{n}\ne{X}$ for all $n\in\mathbb{N}$, otherwise
            $\Delta=\{\,\mathcal{U}_{n}\,\}$ is a finite subcover. So
            $X\setminus\mathcal{U}_{n}\ne\emptyset$ for all $n\in\mathbb{N}$.
            Moreover, the set $\mathcal{V}_{n}$ defined by:
            \begin{equation}
                \mathcal{V}_{n}=\bigcup_{k=0}^{n}\mathcal{U}_{n}
            \end{equation}
            is such that $\mathcal{V}_{n}\ne{X}$, otherwise $\mathcal{O}$ has a
            finite subcover. Define $a:\mathbb{N}\rightarrow{X}$ via
            $a_{n}\in{X}\setminus\mathcal{V}_{n}$ for all $n\in\mathbb{N}$.
            But $(X,\,\tau)$ is sequentially compact, so there is a
            convergent subsequence $a_{k}$ with limit $x\in{X}$. But since
            $\mathcal{O}$ covers $X$, there is a $\mathcal{U}_{N}$ such
            that $x\in\mathcal{U}_{N}$. But then for all
            $n>N$ we have $a_{k_{n}}\not\in\mathcal{U}_{n}$, which is a
            contradiction since $a_{k_{n}}\rightarrow{x}$. So
            $(X,\,\tau)$ is countably compact.
        \end{proof}
        One way to weaken compactness is by lessening open covers to
        countable open covers. The other way is by lessening finite subcover
        to countable subcover. This idea has proven quite useful in many
        applications in analysis.
        \begin{definition}[\textbf{Lindel\"{o}f Topological Space}]
            A Lindel\"{o}f topological space is a topological space
            $(X,\,\tau)$ such that for every open cover
            $\mathcal{O}\subseteq\tau$ there is a countable subcover
            $\Delta\subseteq\mathcal{O}$.
        \end{definition}
        \begin{theorem}
            If $(X,\,\tau)$ is a topological space, then it is countably
            compact and Lindel\"{o}f if and only if it is compact.
        \end{theorem}
        \begin{theorem}
            If $(X,\,\tau)$ is second countable, then it is Lindel\"{o}f.
        \end{theorem}
        \begin{theorem}
            If $(X,\,\tau)$ is second countable and sequentially compact,
            then it is compact.
        \end{theorem}
        \begin{theorem}[\textbf{Extreme Value Theorem}]
        \end{theorem}
        \begin{definition}[\textbf{Pseudocompact Topological Space}]
        \end{definition}
        \begin{theorem}
            If $(X,\,\tau)$ is normal and pseudocompact, then it is
            limit point compact.
        \end{theorem}
        \begin{definition}[\textbf{$\sigma$ Compact Topological Space}]
        \end{definition}
        \begin{theorem}
            If $(X,\,\tau)$ is $\sigma$ compact, then it is Lindel\"{o}f.
        \end{theorem}
\end{document}
