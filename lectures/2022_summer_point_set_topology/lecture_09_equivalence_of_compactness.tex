%-----------------------------------LICENSE------------------------------------%
%   This file is part of Mathematics-and-Physics.                              %
%                                                                              %
%   Mathematics-and-Physics is free software: you can redistribute it and/or   %
%   modify it under the terms of the GNU General Public License as             %
%   published by the Free Software Foundation, either version 3 of the         %
%   License, or (at your option) any later version.                            %
%                                                                              %
%   Mathematics-and-Physics is distributed in the hope that it will be useful, %
%   but WITHOUT ANY WARRANTY; without even the implied warranty of             %
%   MERCHANTABILITY or FITNESS FOR A PARTICULAR PURPOSE.  See the              %
%   GNU General Public License for more details.                               %
%                                                                              %
%   You should have received a copy of the GNU General Public License along    %
%   with Mathematics-and-Physics.  If not, see <https://www.gnu.org/licenses/>.%
%------------------------------------------------------------------------------%
\documentclass{article}
\usepackage{graphicx}                           % Needed for figures.
\usepackage{amsmath}                            % Needed for align.
\usepackage{amssymb}                            % Needed for mathbb.
\usepackage{amsthm}                             % For the theorem environment.
\usepackage{float}
\usepackage{hyperref}
\hypersetup{
    colorlinks=true,
    linkcolor=blue,
    filecolor=magenta,
    urlcolor=Cerulean,
    citecolor=SkyBlue
}

%------------------------Theorem Styles-------------------------%
\theoremstyle{plain}
\newtheorem{theorem}{Theorem}[section]

% Define theorem style for default spacing and normal font.
\newtheoremstyle{normal}
    {\topsep}               % Amount of space above the theorem.
    {\topsep}               % Amount of space below the theorem.
    {}                      % Font used for body of theorem.
    {}                      % Measure of space to indent.
    {\bfseries}             % Font of the header of the theorem.
    {}                      % Punctuation between head and body.
    {.5em}                  % Space after theorem head.
    {}

% Define default environments.
\theoremstyle{normal}
\newtheorem{examplex}{Example}[section]
\newtheorem{definitionx}{Definition}[section]
\newtheorem{notationx}{Notation}[section]
\newtheorem{axiomx}{Axiom}[section]

\newenvironment{example}{%
    \pushQED{\qed}\renewcommand{\qedsymbol}{$\blacksquare$}\examplex%
}{%
    \popQED\endexamplex%
}

\newenvironment{definition}{%
    \pushQED{\qed}\renewcommand{\qedsymbol}{$\blacksquare$}\definitionx%
}{%
    \popQED\enddefinitionx%
}

\title{Point-Set Topology: Lecture 9}
\author{Ryan Maguire}
\date{Summer 2022}

% No indent and no paragraph skip.
\setlength{\parindent}{0em}
\setlength{\parskip}{0em}

\begin{document}
    \maketitle
    \section{The Equivalence of Compactness Theorem}
        The generalization of the Heine-Borel theorem for general metric spaces
        needs stronger notions than just \textit{closed} and \textit{bounded}.
        If we replace \textit{closed} with \textit{complete} and
        \textit{bounded} with \textit{totally bounded}, we get the
        generalized Heine-Borel theorem.
        \begin{theorem}[\textbf{Generalized Heine-Borel Theorem}]
            If $(X,\,d)$ is a metric space, then it is compact if and only
            if it is complete and totally bounded.
        \end{theorem}
        \begin{proof}
            We have already proven that compact implies complete. Now let's
            show compact implies totally bounded. Suppose not. Then there is
            an $\varepsilon>0$ such that no matter what finite collection of
            points $a_{0},\,\dots,\,a_{n}$ you pick, there is another point
            $a_{n+1}$ where $a_{n+1}\notin{B}_{\varepsilon}^{(X,\,d)}(a_{k})$
            for all $0\leq{k}\leq{n}$. Inductively this defines a sequence
            $a:\mathbb{N}\rightarrow{X}$. But $(X,\,d)$ is compact, so there
            is a convergent subsequence $a_{k}$. But convergent sequences are
            Cauchy sequences, and therefore there there is an
            $N\in\mathbb{N}$ such that $m,n>N$ implies
            $d(a_{k_{n}},\,a_{k_{m}})<\varepsilon$. But by construction
            $d(a_{k_{n}},\,a_{k_{m}})\geq\varepsilon$ for all distinct
            $n,m\in\mathbb{N}$, a contradiction. So $(X,\,d)$ is complete.
            Now, suppose $(X,\,d)$ is complete and totally bounded. Let
            $a:\mathbb{N}\rightarrow{X}$ be any sequence. Since
            $(X,\,d)$ is totally bounded, there are finitely many points
            $b_{0},\,\dots,\,b_{N}$ such that the open balls
            $B_{1}^{(X,\,d)}(b_{k})$ completely cover $X$. Since there are
            infinitely integers and only finitely many open balls, there
            must be a point $b_{k}$ such that infinitely many $n\in\mathbb{N}$
            are such that $a_{n}\in{B}_{1}^{(X,\,d)}(b_{k})$. Let
            $k_{0}\in\mathbb{N}$ be such a value
            with $a_{k_{0}}\in{B}_{0}^{(X,\,d)}(b_{k})$. Again by total
            boundedness, we can cover $B_{1}^{(X,\,d)}(b_{k})$ with finitely
            many open balls $B_{1/2}^{(X,\,d)}(c_{\ell})$ with points
            $c_{\ell}\in{B}_{1}^{(X,\,d)}(b_{k})$. Since infinitely many integers
            $n\in\mathbb{N}$ are such that $a_{n}\in{B}_{1}^{(X,\,d)}(b_{k})$
            and there are only finitely many balls covering this set,
            one of these open balls must again be such that there are
            infinitely many integers $n\in\mathbb{N}$ with
            $a_{n}\in{B}_{1/2}^{(X,\,d)}(c_{\ell})$. Let
            $k_{1}$ be such that $k_{1}>k_{0}$ and
            $a_{k_{1}}\in{B}_{1/2}^{(X,\,d)}(c_{\ell})$. Inductively
            we get a subsequence $a_{k}$ with the property that
            $a_{k_{n+1}}$ is contained inside the ball of radius
            $\frac{1}{n+1}$ centered at $a_{k_{n}}$. This sequence is
            Cauchy since $d(a_{k_{m}},\,a_{k_{n}})$ is bounded by
            $\frac{1}{N+1}$ where $N=\textrm{min}(m,\,n)$. But $(X,\,d)$ is
            complete, so this sequence
            converges. Hence $a$ has a convergent subsequence.
        \end{proof}
        We can describe this theorem pictorially. Not totally bounded means
        there is an $\varepsilon>0$ such that no finite set of open balls of
        radius $\varepsilon$ completely cover $X$. We get
        Fig.~\ref{fig:not_totally_bounded_space_001}. This sequence
        $a:\mathbb{N}\rightarrow{X}$ has the property that for all
        $n\ne{m}$ we have $d(a_{n},\,a_{m})\geq\varepsilon$, so $a$ cannot
        possible have any convergent subsequences, violating compactness.
        \begin{figure}
            \centering
            \includegraphics{../../images/not_totally_bounded_space_001.pdf}
            \caption{A Non-Totally Bounded Metric Space}
            \label{fig:not_totally_bounded_space_001}
        \end{figure}
        The latter direction, that totally bounded and complete implies
        compact, is pictorial as well. We cover our metric space in open
        balls of radius 1. This is possible since the space is totally bounded.
        We look at our sequence $a:\mathbb{N}\rightarrow{X}$. There must be an
        open ball with infinitely many $a_{n}$ contained inside it since there
        are infinitely many integers and only finitely many open balls.
        This is shown in
        Fig.~\ref{fig:totally_bounded_and_complete_is_compact_001}. Our
        sequence is the points in red, and the red ball is an open ball that
        contains infinitely many of the $a_{n}$. Note, there could be two such
        balls, or three. In this figure there's only one. We then zoom in on
        this open ball and cover it in finitely many balls of radius
        $\frac{1}{2}$ (Fig.~\ref{fig:totally_bounded_and_complete_is_compact_002}).
        \begin{figure}
            \centering
            \includegraphics{../../images/totally_bounded_and_complete_is_compact_001.pdf}
            \caption{A Sequence in a Totally Bounded Metric Space}
            \label{fig:totally_bounded_and_complete_is_compact_001}
        \end{figure}
        \begin{figure}
            \centering
            \includegraphics{../../images/totally_bounded_and_complete_is_compact_002.pdf}
            \caption{Zooming in on a Totally Bounded Metric Space}
            \label{fig:totally_bounded_and_complete_is_compact_002}
        \end{figure}
        \begin{figure}
            \centering
            \includegraphics{../../images/totally_bounded_and_complete_is_compact_003.pdf}
            \caption{Convergent Subsequence in on a Totally Bounded Metric Space}
            \label{fig:totally_bounded_and_complete_is_compact_003}
        \end{figure}
        We continue and obtain a sequence of nested open balls of radius
        $\frac{1}{n+1}$ and a subsequence of points $a_{k_{n}}$ such that
        $a_{k_{n}}$ lies in the $n^{\textrm{th}}$ ball. The subsequence must
        be Cauchy since the distance between two points is bounded by the
        diameter (twice the radius) of the $\frac{1}{n+1}$ balls, which tends to
        zero. Since the space is complete, this subsequence converges.
        \par\hfill\par
        Compactness is a topological property, but we've yet to describe it
        in terms of open sets. We use the generalized Heine-Borel theorem to
        get one step closer to this, but first we need a definition.
        \begin{definition}[\textbf{Open Cover of a Metric Space}]
            An open cover of a metric space $(X,\,d)$ is a subset
            $\mathcal{O}\subseteq\tau_{d}$, where $\tau_{d}$ is the metric
            topology, such that $\bigcup\mathcal{O}=X$. That is, for all
            $x\in{X}$, there is an open set $\mathcal{U}\in\mathcal{O}$ such
            that $x\in\mathcal{U}$.
        \end{definition}
        Open covers are the topological tool needed to define compactness.
        The slogan for compactness is that
        \textit{every open cover has a finite subcover}, a phrase that is
        dependent solely on the notion of open sets. To prove the equivalence
        of compactness theorem, we first need a lemma.
        \begin{theorem}[\textbf{Lebesgue's Number Lemma}]
            If $(X,\,d)$ is compact, and if $\mathcal{O}$ is an open cover of
            $(X,\,d)$, then there is a $\delta>0$ such that for all
            $x\in{X}$ there is a $\mathcal{U}\in\mathcal{O}$ such that
            $B_{\delta}^{(X,\,d)}(x)\subseteq\mathcal{U}$.
        \end{theorem}
        \begin{proof}
            Suppose not. Then for all $n\in\mathbb{N}$ there is an
            $a_{n}$ such that $B_{\frac{1}{n+1}}^{(X,\,d)}(a_{n})$ is not
            contained entirely inside of $\mathcal{U}$ for any
            $\mathcal{U}\in\mathcal{O}$. But $(X,\,d)$ is compact, so there
            is a convergent subsequence $a_{k}$. Let $x\in{X}$ be the limit,
            $a_{k_{n}}\rightarrow{x}$. Since $\mathcal{O}$ is an open cover,
            there is a $\mathcal{U}$ with $x\in\mathcal{U}$. But $\mathcal{U}$
            is open, so there is an $\varepsilon>0$ such that
            $B_{\varepsilon}^{(X,\,d)}(x)\subseteq\mathcal{U}$. But since
            $a_{k_{n}}\rightarrow{x}$ there is an $N_{0}\in\mathbb{N}$ such that
            $k_{n}>N_{0}$ implies $d(x,\,a_{k_{n}})<\frac{\varepsilon}{2}$. Let
            $N_{1}\in\mathbb{N}$ be such that $N_{1}+1>\frac{2}\varepsilon$ and
            let $N=\textrm{max}(N_{0},\,N_{1})+1$. Then
            $y\in{B}_{\frac{1}{N+1}}^{(X,\,d)}(a_{N})$ implies:
            \begin{equation}
                d(x,\,y)\leq{d}(x,\,a_{N})+d(a_{N},\,y)
                    <\frac{\varepsilon}{2}+\frac{\varepsilon}{2}=\varepsilon
            \end{equation}
            and hence $y\in{B}_{\varepsilon}^{(X,\,d)}(x)$. That is,
            $B_{\frac{1}{N+1}}^{(X,\,d)}(a_{N})\subseteq{B}_{\varepsilon}^{(X,\,d)}(x)$.
            But the $\varepsilon$ ball around $x$ is contained inside of
            $\mathcal{U}$, so
            $B_{\frac{1}{N+1}}^{(X,\,d)}(a_{N})\subseteq\mathcal{U}$, which
            is a contradiction, completing the proof.
        \end{proof}
        \begin{theorem}[\textbf{The Equivalence of Compactness Theorem}]
            If $(X,\,d)$ is a metric space, then $(X,\,d)$ is compact if and
            only if for every open cover $\mathcal{O}$ of $(X,\,d)$ there exists
            a finite open cover $\Delta\subseteq\mathcal{O}$.
        \end{theorem}
        \begin{proof}
            If $(X,\,d)$ is compact and $\mathcal{O}$ is an open cover, then
            by the Lebesgue number lemma there is a $\delta>0$ such that for
            all $x\in{X}$ there is a $\mathcal{U}_{x}$ such that
            $B_{\delta}^{(X,\,d)}(x)\subseteq\mathcal{U}_{x}$. Since
            $(X,\,d)$ is compact, it is totally bounded, and hence there are
            finitely many points $x_{0},\,\dots,\,x_{N}$ such that the
            $\delta$ balls centered at $x_{n}$ cover $X$. But then the set:
            \begin{equation}
                \Delta=\{\,\mathcal{U}_{x_{0}},\,\dots,\,\mathcal{U}_{x_{N}}\,\}
            \end{equation}
            is a finite open cover that is a subset of $\mathcal{O}$. In the
            other direction, suppose $(X,\,d)$ is such that every open cover
            $\mathcal{O}$ contains a finite subset $\Delta\subseteq\mathcal{O}$
            that is also an open cover of $(X,\,d)$. Given $\varepsilon>0$,
            create the set $\mathcal{O}$ by:
            \begin{equation}
                \mathcal{O}=\{\,B_{\varepsilon}^{(X,\,d)}(x)\;|\;x\in{X}\,\}
            \end{equation}
            $\mathcal{O}$ is an open cover of $(X,\,d)$, and hence there is a
            finite open cover $\Delta\subseteq\mathcal{O}$. But then $(X,\,d)$
            can be covered by finitely many balls of radius $\varepsilon$, so
            $(X,\,d)$ is totally bounded. Next, suppose $(X,\,d)$ is not
            complete. There there is a Cauchy sequence
            $a:\mathbb{N}\rightarrow{X}$ that does not converge. Then for all
            $x\in{X}$ there is a $\varepsilon_{x}>0$ such that for all
            $N\in\mathbb{N}$ there exists $n>N$ with
            $d(x,\,a_{n})\geq\varepsilon$. Let $\mathcal{O}$ be defined by:
            \begin{equation}
                \mathcal{O}=\{\,B_{\varepsilon_{x}/2}^{(X,\,d)}(x)\;|\;x\in{X}\,\}
            \end{equation}
            $\mathcal{O}$ is an open cover of $(X,\,d)$, so there is a finite
            open cover $\Delta\subseteq\mathcal{O}$. Let
            $x_{0},\,\dots,x_{N}$ be the finite set of points such that:
            \begin{equation}
                X=\bigcup_{n=0}^{N}B_{\varepsilon_{n}}^{(X,\,d)}(x_{n})
            \end{equation}
            where $\varepsilon_{n}=\varepsilon_{x_{n}}$. Let
            $\varepsilon=\textrm{min}(\varepsilon_{0},\,\dots,\,\varepsilon_{N})$.
            Since $a:\mathbb{N}\rightarrow{X}$ is a Cauchy sequence, there is
            an $N\in\mathbb{N}$ such that $n,m>N$ implies
            $d(a_{m},\,a_{n})<\varepsilon$. Since the open balls of
            radius $\varepsilon_{k}/2$ centered at $x_{k}$ cover the metric
            space, there is some $x_{k}$, $0\leq{k}\leq{N}$, such that
            $d(a_{N+1},\,x)<\frac{\varepsilon}{2}$. But then for all
            $n>N$ we have:
            \begin{equation}
                d(x_{k},\,a_{n})\leq{d}(x_{k},\,a_{N+1})+d(a_{n},\,a_{N+1})
                    <\frac{\varepsilon_{n}}{2}+\frac{\varepsilon}{2}
                    \leq\varepsilon_{n}
            \end{equation}
            Which is a contradiction since for all $N\in\mathbb{N}$ there is
            an $n>N$ such that $d(x_{k},\,a_{n})\geq\varepsilon_{n}$ by
            the definition of $\varepsilon_{n}$. Therefore
            $(X,\,d)$ is complete. Since $(X,\,d)$ is complete and totally
            bounded, by the generalized Heine-Borel theorem it is compact.
        \end{proof}
\end{document}
