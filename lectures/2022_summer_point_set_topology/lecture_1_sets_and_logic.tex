%-----------------------------------LICENSE------------------------------------%
%   This file is part of Mathematics-and-Physics.                              %
%                                                                              %
%   Mathematics-and-Physics is free software: you can redistribute it and/or   %
%   modify it under the terms of the GNU General Public License as             %
%   published by the Free Software Foundation, either version 3 of the         %
%   License, or (at your option) any later version.                            %
%                                                                              %
%   Mathematics-and-Physics is distributed in the hope that it will be useful, %
%   but WITHOUT ANY WARRANTY; without even the implied warranty of             %
%   MERCHANTABILITY or FITNESS FOR A PARTICULAR PURPOSE.  See the              %
%   GNU General Public License for more details.                               %
%                                                                              %
%   You should have received a copy of the GNU General Public License along    %
%   with Mathematics-and-Physics.  If not, see <https://www.gnu.org/licenses/>.%
%------------------------------------------------------------------------------%
\documentclass{article}
\usepackage{graphicx}                           % Needed for figures.
\usepackage{amsmath}                            % Needed for align.
\usepackage{amssymb}                            % Needed for mathbb.
\usepackage{amsthm}                             % For the theorem environment.
\usepackage{float}
\usepackage{hyperref}
\hypersetup{
    colorlinks=true,
    linkcolor=blue,
    filecolor=magenta,
    urlcolor=Cerulean,
    citecolor=SkyBlue
}

%------------------------Theorem Styles-------------------------%
\theoremstyle{plain}
\newtheorem{theorem}{Theorem}[section]

% Define theorem style for default spacing and normal font.
\newtheoremstyle{normal}
    {\topsep}               % Amount of space above the theorem.
    {\topsep}               % Amount of space below the theorem.
    {}                      % Font used for body of theorem.
    {}                      % Measure of space to indent.
    {\bfseries}             % Font of the header of the theorem.
    {}                      % Punctuation between head and body.
    {.5em}                  % Space after theorem head.
    {}

% Define default environments.
\theoremstyle{normal}
\newtheorem{examplex}{Example}[section]
\newtheorem{definitionx}{Definition}[section]
\newtheorem{notationx}{Notation}[section]

\newenvironment{example}{%
    \pushQED{\qed}\renewcommand{\qedsymbol}{$\blacksquare$}\examplex%
}{%
    \popQED\endexamplex%
}

\newenvironment{definition}{%
    \pushQED{\qed}\renewcommand{\qedsymbol}{$\blacksquare$}\definitionx%
}{%
    \popQED\enddefinitionx%
}

\newenvironment{notation}{%
    \pushQED{\qed}\renewcommand{\qedsymbol}{$\blacksquare$}\notationx%
}{%
    \popQED\endnotationx%
}

\title{Point-Set Topology: Lecture 1}
\author{Ryan Maguire}
\date{Summer 2022}

% No indent and no paragraph skip.
\setlength{\parindent}{0em}
\setlength{\parskip}{0em}

\begin{document}
    \maketitle
    \section{Motivation}
        Topology is the study of \textit{continuous deformations}. Continuously
        changing a space without tearing or cutting. Ask yourself if you think
        it is possible to continuously morph a square into a circle.
        See Fig.~\ref{fig:homeo_square_to_circle} for a demonstration.
        \begin{figure}[H]
            \centering
            \includegraphics{../../images/homeo_square_to_circle.pdf}
            \caption{Changing a Square to a Circle}
            \label{fig:homeo_square_to_circle}
        \end{figure}
        Now ask yourself if it is possible to change a sphere into a torus
        (Fig.~\ref{fig:torus_next_to_sphere}).
        \begin{figure}[H]
            \centering
            \includegraphics{../../images/torus_next_to_sphere.pdf}
            \caption{A Torus and a Sphere}
            \label{fig:torus_next_to_sphere}
        \end{figure}
        The first problems in topology
        stem from graph theory, such as the bridges of K\"{o}nigsberg
        problem. You are asked to traverse the seven bridges in the city of
        K\"{o}nigsberg, crossing each bridge exactly once
        (Fig.~\ref{fig:bridges_of_konigsberg_001}). Is this possible?
        \par\hfill\par
        We can simplify the problem by treating plots of land as single dots,
        and bridges as curves and line segments, resulting in the graph
        shown in Fig.~\ref{fig:bridges_of_konigsberg_002}. The problem asks you
        to traverse each edge of this graph once and only once. This turns
        out to be impossible. Turning this into a graph problem is the
        start of topology. We may move the edges and vertices of the graph
        around without altering the question. If we cut an edge or remove a
        vertex, we will change the problem. That is, we may change the graph
        \textit{continuously} without affecting anything.
        \begin{figure}
            \centering
            \includegraphics{../../images/bridges_of_konigsberg_001.pdf}
            \caption{The Seven Bridges of K\"{o}nigsberg}
            \label{fig:bridges_of_konigsberg_001}
        \end{figure}
        \begin{figure}
            \centering
            \includegraphics{../../images/bridges_of_konigsberg_002.pdf}
            \caption{The Graph of the Seven Bridges of K\"{o}nigsberg}
            \label{fig:bridges_of_konigsberg_002}
        \end{figure}
        \begin{figure}
            \centering
            \includegraphics{../../images/tetrahedron_001.pdf}
            \caption{A Tetrahedron}
            \label{fig:tetrahedron_001}
        \end{figure}
        \par\hfill\par
        The next discovery that motivates topology is the
        \textit{Euler characteristic}. A polyhedron is a collection of vertices,
        edges, and faces. A convex polyhedron has the property that if it has
        $V$ vertices, $E$ edges, and $F$ faces, then:
        \begin{equation}
            V-E+F=2
        \end{equation}
        The convexity is not actually needed. The formula $V-E+F=2$ is true as
        long as the polyhedron can be \textit{continuously deformed} into a
        sphere. If we have a polyhedron that can be continuously deformed into
        a torus, we get a different number. Examining
        Fig.~\ref{fig:polyhedral_torus_001} we count:
        \begin{equation}
            V-E+F=0
        \end{equation}
        This \textit{alternating sum} is called the
        \textit{Euler characteristic}
        of the polyhedron and it is preserved by continuous deformations.
        \begin{figure}
            \centering
            \includegraphics{../../images/polyhedral_torus_001.pdf}
            \caption{A Polyhedral Torus}
            \label{fig:polyhedral_torus_001}
        \end{figure}
    \section{Sets}
        Sets are the main object dealt with in most branches of mathematics.
        \begin{definition}[\textbf{Set}]
            A set is a collection of objects called elements.
        \end{definition}
        Sets are often denoted using capital Latin letters such as
        $A$, $B$, $C$, $X$, $Y$, and $Z$. Elements of sets are usually denoted
        by lower case Latin letters like $a$, $b$, $c$, $x$, $y$, and $z$. This
        need not always be the case, but is fairly common.
        \begin{example}
            Finite sets are denoted with braces and commas separating the
            elements. For example, if $A$ is the set consisting of the elements
            1, 2, and 3, we write:
            \begin{equation}
                A=\{\,1,\,2,\,3\,\}
            \end{equation}
            Order does not matter for sets, we can also write:
            \begin{equation}
                A=\{\,3,\,1,\,2\,\}
            \end{equation}
            And repetition is also irrelevant:
            \begin{equation}
                A=\{\,1,\,1,\,2,\,3\,\}
            \end{equation}
            All of these mean the set $A$.
        \end{example}
        \begin{notation}[\textbf{Containment}]
            If $A$ is a set and $a$ is an element of $a$, we write
            $a\in{A}$. If $a$ is not an element of $a$, we write
            $a\notin{A}$.
        \end{notation}
        The expression $a\in{A}$ reads aloud as
        $a$ \textit{is in} $A$ or $a$ \textit{is an element of} $A$. Similarly,
        $a\notin{A}$ reads allowed as $a$ \textit{is not in} $A$.
        \begin{example}
            Using $A=\{\,1,\,2,\,3\,\}$, we have that
            $1\in{A}$, $2\in{A}$, and $3\in{A}$ since 1, 2, and 3 are elements
            of the set $A$. However, $4\notin{A}$ since 4 is not an element of
            $A$.
        \end{example}
        Sets are usually constructed with set-builder notation. If we have
        some set $X$ that we know exists, we can take a sentence $P$, and
        apply it to all elements $x\in{X}$.
        This is expressed
        \begin{equation}
            A=\{\,x\in{X}\;|\;P(x)\,\}
        \end{equation}
        which reads $A$ \textit{is the set of all} $x$ \textit{in} $X$
        \textit{such that} $P(x)$ \textit{is true}.
        \begin{example}
            Let $\mathbb{N}$ be the set of natural numbers
            \begin{equation}
                \mathbb{N}=\{\,0,\,1,\,2,\,3,\,4,\,\dots\,\}
            \end{equation}
            Let $P(n)$ be the sentence $n$ \textit{is even}. Consider the
            set
            \begin{equation}
                A=\{\,n\in\mathbb{N}\;|\;P(n)\,\}
            \end{equation}
            This is the set of all natural numbers that are even. We can write
            this explicitly as follows:
            \begin{equation}
                A=\{\,0,\,2,\,4,\,\dots\,\}
            \end{equation}
        \end{example}
        Be careful of abusing set-builder notation. If you have some sentence
        $P(x)$ you can \textbf{not} write the following:
        \begin{equation}
            A=\{\,x\;|\;P(x)\,\}
        \end{equation}
        That is, you may not consider the set of all $x$ such that $P(x)$ is
        true. You may only apply your sentence to some set $X$ you already
        know exists, and then collect all $x\in{X}$ satisfying your sentence
        $P$. Failure to do this results in Russell's paradox. Consider the
        sentence $P(x)=x$ \textit{is a set}. Then
        $A=\{\,x\;|\;P(x)\,\}$ is the \textit{set of all sets}. Is
        $A$ an element of itself? If allowed, it is possible to construct
        a statement that is both true and false, something that should be
        avoided.
        \begin{definition}[\textbf{Subset}]
            A subset of a set $B$ is a set $A$ such that for all $a\in{A}$
            it is true that $a\in{B}$.
        \end{definition}
        Subsets can be denoted by blobs in the plane. See
        Fig.~\ref{fig:subset_001}.
        \begin{figure}
            \centering
            \includegraphics{../../images/subset_001.pdf}
            \caption{Diagram for Subsets}
            \label{fig:subset_001}
        \end{figure}
        \par\hfill\par
        The axioms of set theory provide the existence of four operations on
        sets: union, intersection, set difference, and symmetric difference.
        \begin{definition}[\textbf{Union of Two Sets}]
            The union of two sets $A$ and $B$ is the set $A\cup{B}$ defined
            by:
            \begin{equation}
                A\cup{B}=\{\,x\;|\;x\in{A}\textrm{ or }x\in{B}\,\}
            \end{equation}
        \end{definition}
        The union of two sets can be visualized with a
        \textit{Venn diagram} (Fig.~\ref{fig:venn_diagram_union_001}). It is
        worth detailing what the word \textit{or} means. In English there are
        two uses of the word or, the \textit{inclusive or} and the
        \textit{exclusive or}. Given two sentences $P$ and $Q$, the exclusive
        means $P$ is true or $Q$ is true, but not both. The inclusive or means
        $P$ is true, $Q$ is true, or both are true. In mathematics we always
        use the inclusive or. The exclusive or is denoted \textit{xor}. It is
        a useful concept in computer science and analysis of algorithms, but we
        will not need it in topology.
        \begin{example}
            Let $A=\{\,1,\,2,\,3\,\}$ and $B=\{\,3,\,4,\,5\,\}$. The union
            is then:
            \begin{equation}
                A\cup{B}=\{\,1,\,2,\,3,\,4,\,5\,\}
            \end{equation}
            Remember, sets have no notion of repetition, so including 3 twice
            would be redundant.
        \end{example}
        \begin{figure}
            \centering
            \includegraphics{../../images/venn_diagram_union_001.pdf}
            \caption{Venn Diagram for Union}
            \label{fig:venn_diagram_union_001}
        \end{figure}
        \begin{definition}[\textbf{Intersection of Two Sets}]
            The intersection of two sets $A$ and $B$ is the set $A\cap{B}$
            defined by:
            \begin{equation}
                A\cap{B}=\{\,x\;|\;x\in{A}\textrm{ and }x\in{B}\,\}
            \end{equation}
        \end{definition}
        The word \textit{and} is easier to understand since its mathematical
        use matches the grammatical one. Given two sentences $P$ and $Q$,
        the statement $P$ \textit{and} $Q$ means both $P$ is true and $Q$ is
        true, and both are true simultaneously.
        \begin{example}
            Let $A=\{\,1,\,2,\,3\,\}$ and $B=\{\,3,\,4,\,5\,\}$. The
            intersection is then:
            \begin{equation}
                A\cap{B}=\{\,3\,\}
            \end{equation}
            The only element $A$ and $B$ have in common is 3.
        \end{example}
        Intersection can also be represented by a Venn diagram
        (See Fig.~\ref{fig:venn_diagram_intersection_001}). What if $A$ and $B$
        have no sets in common? Let's consider the following example.
        \begin{example}
            Let $A=\{\,1,\,2,\,3\,\}$ and $B=\{\,4,\,5,\,6\,\}$. The
            intersection $A\cap{B}$ is the set of elements in both $A$ and $B$.
            But $A$ and $B$ have no elements in common, so the intersection
            is \textit{empty}. This is the empty set, denoted $\emptyset$, and
            occasionally written as $\emptyset=\{\,\}$. We have
            $A\cap{B}=\emptyset$.
        \end{example}
        Disjoint sets are sets $A$ and $B$ such that $A\cap{B}=\emptyset$.
        In the previous example, $A$ and $B$ are disjoint.
        \par\hfill\par
        \begin{figure}
            \centering
            \includegraphics{../../images/venn_diagram_intersection_001.pdf}
            \caption{Venn Diagram for Intersection}
            \label{fig:venn_diagram_intersection_001}
        \end{figure}
        The next operation is set difference. It is somewhat like subtraction
        for sets.
        \begin{definition}[\textbf{Set Difference}]
            The set difference of a set $A$ from a set $B$ is the set
            $B\setminus{A}$ defined by:
            \begin{equation}
                B\setminus{A}=\{\,x\in{B}\;|\;x\notin{A}\,\}
            \end{equation}
            That is, the set of all elements in $B$ that are not in $A$.
        \end{definition}
        The Venn diagram for set difference is given in
        Fig.~\ref{fig:venn_diagram_set_difference_001}.
        \begin{figure}
            \centering
            \includegraphics{../../images/venn_diagram_set_difference_001.pdf}
            \caption{Venn Diagram for Set Difference}
            \label{fig:venn_diagram_set_difference_001}
        \end{figure}
        \begin{example}
            Let $A=\{\,1,\,2,\,3\,\}$ and $B=\{\,3,\,4,\,5\,\}$. The set
            difference $B\setminus{A}$ is:
            \begin{equation}
                B\setminus{A}=\{\,4,\,5\,\}
            \end{equation}
            That is, the only element common to $A$ and $B$ is 3, so
            $B\setminus{A}$ is everything in $B$ except for 3.
        \end{example}
        The last operation is \textit{symmetric difference} and it is defined
        in terms of the other three. We will not use it, but for completeness
        the Venn diagram is shown.
        \begin{definition}[\textbf{Symmetric Difference}]
            The symmetric difference of two sets $A$ and $B$ is the set
            $A\ominus{B}$ defined by:
            \begin{equation}
                A\ominus{B}=(A\cup{B})\setminus(A\cap{B})
            \end{equation}
            That is, the set of all elements in either $A$ or $B$, but not both.
        \end{definition}
        Symmetric difference is the set equivalent of the exclusive or.
        The Venn diagram is shown in Fig.~\ref{fig:venn_diagram_symmetric_difference_001}.
        \begin{figure}
            \centering
            \includegraphics{../../images/venn_diagram_symmetric_difference_001.pdf}
            \caption{Venn Diagram for Symmetric Difference}
            \label{fig:venn_diagram_symmetric_difference_001}
        \end{figure}
        \par\hfill\par
        There are several theorems relating union, intersection, and set
        difference. They are proven in the text, but will just be listed here.
        \begin{align}
            A\cup{B}&=B\cup{A}\tag{Commutativity of Unions}\\
            A\cap{B}&=B\cap{A}\tag{Commutativity of Intersections}\\
            A\cup(B\cup{C})&=(A\cup{B})\cup{C}\tag{Associativity of Unions}\\
            A\cap(B\cap{C})&=(A\cap{B})\cap{C}\tag{Associativity of Intersections}\\
            A\cup\emptyset&=A\tag{Identity Law of Unions}\\
            A\subseteq{B}\Rightarrow{A}\cap{B}&=A\tag{Identity Law of Intersections}\\
            A\cup(B\cap{C})&=(A\cup{B})\cap(A\cup{C})\tag{Distributive Law of Unions}\\
            A\cap(B\cup{C})&=(A\cap{B})\cup(A\cap{C})\tag{Distributive Law of Intersections}\\
            X\setminus(A\cup{B})&=(X\setminus{A})\cap(X\setminus{B})\tag{De Morgan's Law of Unions}\\
            X\setminus(A\cap{B})&=(X\setminus{A})\cup(X\setminus{B})\tag{De Morgan's Law of Intersections}
        \end{align}
        Lastly, a brief discussion on \textit{implication}. This is the use
        of the expression \textit{if-then} in mathematics. Consider the sentence
            \begin{center}
                \textit{if I am late to work, then I will be fired}
            \end{center}
            This contains two sentences. $P(x)$ is \textit{x is late for work}
            and $Q(x)$ is \textit{x will be fired}. There are four possibilities
            for the sentence \textit{if P, then Q}. Let's work through them.
            \par\hfill\par
            \textit{I was not late to work, and I was not fire.}. Is the
            statement \textit{if I am late to work, then I will be fired}
            true or false? The situation that triggers the firing didn't happen
            so we certainly cannot conclude that the statement is false. We
            thus say that in this scenario the sentence is true.
            \par\hfill\par
            \textit{I was not late to work, and I was fired}. Is the statement
            false? Of course not, there are plenty of reasons for being fired.
            Perhaps I have a sailor's tongue or sloppy handwriting. In this
            situation we conclude that the sentence is true.
            \par\hfill\par
            \textit{I was late to work, and I was not fired}. In this event, my
            boss is very nice, however the sentence is false. I was late to work
            and yet I was not fired.
            \par\hfill\par
            \textit{I was late to work, and I was fired}. This is perhaps the
            easiest to grasp since it's the scenario one intuitively thinks
            about when hearing \textit{if-then} sentences. In this case the
            sentence is true.
            \par\hfill\par
            The technical name for \textit{if-then} is \textit{implication}. The
            symbol for this is a large arrow
            $\Rightarrow$. The truth table for implication is given in
            Tab.~\ref{tab:truth_table_implication}.
            \begin{table}
                \centering
                \begin{tabular}{c | c | c}
                    $P$&$Q$&$P\Rightarrow{Q}$\\
                    \hline
                    False&False&True\\
                    \hline
                    False&True&True\\
                    \hline
                    True&False&False\\
                    \hline
                    True&True&True
                \end{tabular}
                \caption{Truth Table for Implication}
                \label{tab:truth_table_implication}
            \end{table}
\end{document}
