%-----------------------------------LICENSE------------------------------------%
%   This file is part of Mathematics-and-Physics.                              %
%                                                                              %
%   Mathematics-and-Physics is free software: you can redistribute it and/or   %
%   modify it under the terms of the GNU General Public License as             %
%   published by the Free Software Foundation, either version 3 of the         %
%   License, or (at your option) any later version.                            %
%                                                                              %
%   Mathematics-and-Physics is distributed in the hope that it will be useful, %
%   but WITHOUT ANY WARRANTY; without even the implied warranty of             %
%   MERCHANTABILITY or FITNESS FOR A PARTICULAR PURPOSE.  See the              %
%   GNU General Public License for more details.                               %
%                                                                              %
%   You should have received a copy of the GNU General Public License along    %
%   with Mathematics-and-Physics.  If not, see <https://www.gnu.org/licenses/>.%
%------------------------------------------------------------------------------%
\documentclass{article}
\usepackage{graphicx}                           % Needed for figures.
\usepackage{amsmath}                            % Needed for align.
\usepackage{amssymb}                            % Needed for mathbb.
\usepackage{amsthm}                             % For the theorem environment.
\usepackage{float}
\usepackage{tabularx, booktabs}
\usepackage{mathrsfs}
\usepackage[font=scriptsize,
            labelformat=simple,
            labelsep=colon]{subcaption} % Subfigure captions.
\usepackage[font={scriptsize},
            hypcap=true,
            labelsep=colon]{caption}    % Figure captions.
\usepackage{hyperref}
\hypersetup{
    colorlinks=true,
    linkcolor=blue,
    filecolor=magenta,
    urlcolor=Cerulean,
    citecolor=SkyBlue
}

%------------------------Theorem Styles-------------------------%
\theoremstyle{plain}
\newtheorem{theorem}{Theorem}[section]

% Define theorem style for default spacing and normal font.
\newtheoremstyle{normal}
    {\topsep}               % Amount of space above the theorem.
    {\topsep}               % Amount of space below the theorem.
    {}                      % Font used for body of theorem.
    {}                      % Measure of space to indent.
    {\bfseries}             % Font of the header of the theorem.
    {}                      % Punctuation between head and body.
    {.5em}                  % Space after theorem head.
    {}

% Define default environments.
\theoremstyle{normal}
\newtheorem{examplex}{Example}[section]
\newtheorem{definitionx}{Definition}[section]

\newenvironment{example}{%
    \pushQED{\qed}\renewcommand{\qedsymbol}{$\blacksquare$}\examplex%
}{%
    \popQED\endexamplex%
}

\newenvironment{definition}{%
    \pushQED{\qed}\renewcommand{\qedsymbol}{$\blacksquare$}\definitionx%
}{%
    \popQED\enddefinitionx%
}

\title{Point-Set Topology: Lecture 30}
\author{Ryan Maguire}
\date{Summer 2022}

% No indent and no paragraph skip.
\setlength{\parindent}{0em}
\setlength{\parskip}{0em}

\begin{document}
    \maketitle
    \section{Manifolds}
        So far we've seen several examples of locally Euclidean spaces.
        \begin{itemize}
            \item $\mathbb{R}^{n}$ for all $n\in\mathbb{N}$.
            \item Open subspaces of $\mathbb{R}^{n}$.
            \item The $n$ sphere $\mathbb{S}^{n}$
            \item The graphs of continuous functions
                $f:\mathbb{R}^{m}\rightarrow\mathbb{R}^{n}$ with the subspace
                topology in $\mathbb{R}^{m}\times\mathbb{R}^{n}$.
            \item The bug-eyed line.
            \item The branching line.
            \item The long line.
        \end{itemize}
        The first four are subspaces of Euclidean space, the last three are not.
        We know these last three spaces cannot be embedded into $\mathbb{R}^{n}$
        since the bug-eyed and branching lines are not Hausdorff and all
        subspaces of Hausdorff spaces are Hausdorff, and the long line is not
        second countable. Topological manifolds add the Hausdorff property and
        second countability to ensure nothing too weird can happen with the
        space.
        \begin{definition}[\textbf{Topological Manifold}]
            A topological manifold is a topological space $(X,\,\tau)$ that is
            Hausdorff, second countable, and locally Euclidean. That is, for
            all distinct $x,y\in{X}$ there are open sets
            $\mathcal{U},\mathcal{V}\in\tau$ such that $x\in\mathcal{U}$,
            $y\in\mathcal{V}$, and $\mathcal{U}\cap\mathcal{V}=\emptyset$. There
            is also a countable basis $\mathcal{B}$ for the topology $\tau$. And
            for all $x\in{X}$ there is a topological chart
            $(\mathcal{U},\,\varphi)$ in $(X,\,\tau)$ such that
            $x\in\mathcal{U}$.
        \end{definition}
        The word \textit{topological} is added to manifold since there are
        several types of manifolds. In many cases the adjective is dropped and
        context is required to know which type of manifold is being talked
        about. A few of the other types of manifold are listed below.
        \begin{itemize}
            \item Topological manifold: A type of topological space.
            \item Smooth manifold: A topological manifold with extra structure
                so that it is possible to ask if functions are differentiable
                and to speak of things like tangent vectors and vector fields.
            \item Riemannian manifold: A smooth manifold that is equipped with
                a method of assigning angles between tangent vectors, measuring
                lengths of curves, and the area and volume of subspaces.
            \item Lorentz Manifold: A smooth manifold with, loosely speaking,
                a method of differentiating between time and space. Lorentz
                manifolds fall into the study of spacetime and general
                relativity.
            \item Semi-Riemannian manifold: A generalization of Lorentz and
                Riemannian manifolds. In particular, all Lorentz and all
                Riemannian manifolds are also semi-Riemannian manifolds.
        \end{itemize}
        We will talk about smooth manifolds briefly, since we actually have all
        of the terminology to discuss them. We won't dive too deep into the
        field however since smooth manifolds belong to differential topology.
        Riemannian and Lorentz manifolds belong to geometry, and we won't have
        anything to say about those (we also lack the algebraic terminology
        required to define them).
        \begin{example}
            Euclidean space $\mathbb{R}^{n}$ is a topological manifold. It is
            locally Euclidean from the previous lecture, and it is also
            Hausdorff and second countable.
        \end{example}
        \begin{example}
            Any open subset $\mathcal{U}\subseteq\mathbb{R}^{n}$ with the
            subspace topology is a topological manifold. It is locally Euclidean
            from the previous lecture, and since subspaces of second countable
            Hausdorff spaces are still second countable and Hausdorff,
            $\mathcal{U}$ is a topological manifold.
        \end{example}
        \begin{example}
            The $n$ dimensional sphere $\mathbb{S}^{n}$ is a topological
            manifold. We used orthographic, stereographic, near-sided, and
            far-sided projections last lecture to show $\mathbb{S}^{n}$ is
            locally Euclidean in several different ways. Since the $n$ sphere
            is a subspace of $\mathbb{R}^{n+1}$, it is second countable and
            Hausdorff.
        \end{example}
        \begin{example}
            The long line is not a topological manifold since it is not
            second countable. It is locally Euclidean and Hausdorff, however.
        \end{example}
        \begin{example}
            The bug-eyed line is not a topological manifold. It is second
            countable and locally Euclidean, but it is not Hausdorff.
        \end{example}
        \begin{example}
            Similarly, the branching line is not a topological manifold since it
            is not Hausdorff.
        \end{example}
        I would like to think the real reason for the Hausdorff and second
        countability requirements is so that we can perhaps hope that
        topological manifolds are really just particular subspaces of
        $\mathbb{R}^{n}$. From the definition there is no such requirement and
        topological manifolds can be considered as abstract objects that do not
        live in any ambient Euclidean space. This can be quite useful. The
        spacetime of general relativity is a four dimensional topological
        manifold (it's actually a Lorentz manifold, but let's not dive into
        that). Whether or not it is possible to embed spacetime into some
        higher dimensional Euclidean space or not seems irrelevant to any
        physical problem one might study. For the curious, it is indeed
        possible to embed a four dimensional spacetime into $\mathbb{R}^{8}$.
        What purpose eight dimensional Euclidean space may serve for any
        physics problems is beyond me.
        \par\hfill\par
        We now arrive at our first set of spaces that are not obviously some
        subspace of $\mathbb{R}^{n}$. These are the
        \textit{real projective spaces} and are denoted $\mathbb{RP}^{n}$.
        \begin{example}[\textbf{Real Projective Space}]
            Let $X=\mathbb{R}^{n+1}\setminus\{\,\mathbf{0}\,\}$. Define the
            equivalence relation $R$ on $X$ via $\mathbf{x}R\mathbf{y}$ if and
            only if $\mathbf{y}=\lambda\mathbf{x}$ for some
            $\lambda\in\mathbb{R}\setminus\{\,0\,\}$. $\mathbb{RP}^{n}$ is the
            set $X/R$ and the topology $\tau_{\mathbb{RP}^{n}}$ is the
            quotient topology induced by $R$. As a set this is the set of all
            lines in $\mathbb{R}^{n+1}$ that pass through the origin. That is,
            a point $[\mathbf{x}]\in\mathbb{RP}^{n}$ is the entire line through
            the origin that passes through the point $\mathbf{x}$. Let's start
            with $\mathbb{RP}^{1}$. Any line can be described by an angle
            $0\leq\theta<\pi$. If you vary the line you are on slightly, you
            are just varying this angle. Hopefully it becomes intuitive that
            $\mathbb{RP}^{1}$ is in fact a one dimensional locally Euclidean
            space (it may not be intuitive as to why it is Hausdorff or
            second countable, but we'll get there). A similar thinking applies
            to $\mathbb{RP}^{n}$. Let's be precise. Let
            $\mathcal{U}_{k}\subseteq{X}$ be defined by:
            \begin{equation}
                \mathcal{U}_{k}
                =\{\,\mathbf{x}\in\mathbb{R}^{n+1}\setminus\{\,\mathbf{0}\,\}\;
                    |\;\mathbf{x}_{k}\ne{0}\,\}
            \end{equation}
            This is the complement of the $k^{\textrm{th}}$ axis, which is open
            since the $k^{\textrm{th}}$ axis is closed. It is also saturated
            with respect to the canonical quotient map
            $q:X\rightarrow\mathbb{RP}^{n}$ defined by
            $q(\mathbf{x})=[\mathbf{x}]$. That is,
            $q^{-1}\big[q[\mathcal{U}_{k}]\big]=\mathcal{U}_{k}$. It is always
            the case that
            $\mathcal{U}_{k}\subseteq{q}^{-1}\big[q[\mathcal{U}_{k}]\big]$,
            let's show this reverses for our particular set $\mathcal{U}_{k}$.
            Let $\mathbf{x}\in{q}^{-1}\big[q[\mathcal{U}_{k}]\big]$. Then
            $[\mathbf{x}]\in{q}\big[\mathcal{U}_{k}\big]$ so there is some
            $\mathbf{y}\in\mathcal{U}_{k}$ such that
            $[\mathbf{x}]=[\mathbf{y}]$. But then
            $\mathbf{y}_{k}\ne{0}$ and $\mathbf{x}=\lambda\mathbf{y}$ for some
            $\lambda\in\mathbb{R}\setminus\{\,0\,\}$. But then
            $\mathbf{x}_{k}\ne{0}$, and hence $\mathbf{x}\in\mathcal{U}_{k}$.
            So $\mathcal{U}_{k}$ is saturated. But since $q$ is a quotient map,
            if $\mathcal{U}_{k}$ is open and saturated, the set
            $\tilde{\mathcal{U}}_{k}=q[\mathcal{U}_{k}]$ is open. Define
            $\varphi_{k}:\tilde{\mathcal{U}}_{k}\rightarrow\mathbb{R}^{n}$ via:
            \begin{equation}
                \varphi_{k}\big([\mathbf{x}]\big)
                =\Big(\frac{\mathbf{x}_{0}}{\mathbf{x}_{k}},\,\dots,\,
                    \frac{\mathbf{x}_{k-1}}{\mathbf{x}_{k}},\,
                    \frac{\mathbf{x}_{k+1}}{\mathbf{x}_{k}},\,\dots,\,
                    \frac{\mathbf{x}_{n}}{\mathbf{x}_{k}}\Big)
            \end{equation}
            We have to prove this is well-defined in two regards. First, there
            is no division by zero since $\mathbf{x}\in\mathcal{U}_{k}$ implies
            $\mathbf{x}_{k}\ne{0}$. Second, this is well defined as a function.
            By that I mean if $[\mathbf{x}]=[\mathbf{y}]$, then there is some
            $\lambda\in\mathbb{R}\setminus\{\,0\,\}$ such that
            $\mathbf{y}=\lambda\mathbf{x}$. But then:
            \begin{align}
                \varphi_{k}\big([\mathbf{y}]\big)
                &=\Big(\frac{\mathbf{y}_{0}}{\mathbf{y}_{k}},\,\dots,\,
                    \frac{\mathbf{y}_{k-1}}{\mathbf{y}_{k}},\,
                    \frac{\mathbf{y}_{k+1}}{\mathbf{y}_{k}},\,\dots,\,
                    \frac{\mathbf{y}_{n}}{\mathbf{y}_{k}}\Big)\\
                &=\Big(\frac{\lambda\mathbf{x}_{0}}{\lambda\mathbf{x}_{k}},\,
                    \dots,\,
                    \frac{\lambda\mathbf{x}_{k-1}}{\lambda\mathbf{x}_{k}},\,
                    \frac{\lambda\mathbf{x}_{k+1}}{\lambda\mathbf{x}_{k}},\,
                    \dots,\,
                    \frac{\lambda\mathbf{x}_{n}}{\lambda\mathbf{x}_{k}}\Big)\\
                &=\Big(\frac{\mathbf{x}_{0}}{\mathbf{x}_{k}},\,\dots,\,
                    \frac{\mathbf{x}_{k-1}}{\mathbf{x}_{k}},\,
                    \frac{\mathbf{x}_{k+1}}{\mathbf{x}_{k}},\,\dots,\,
                    \frac{\mathbf{x}_{n}}{\mathbf{x}_{k}}\Big)\\
                &=\varphi_{k}\big([\mathbf{x}]\big)
            \end{align}
            So it is well-defined. It is also continuous. This is one of the
            characteristics of the quotient map. Given a topological space
            $(Y,\,\tau_{Y})$ and a function $f:X/R\rightarrow{Y}$, $f$ is
            continuous if and only if $f\circ{q}:X\rightarrow{Y}$ is
            continuous where $q:X\rightarrow{X}/R$ is the canonical quotient
            map. The composition $\varphi_{k}\circ{q}$ is a rational function,
            which is continuous, so $\varphi_{k}$ is continuous. The inverse
            function is given by:
            \begin{equation}
                \varphi_{k}^{-1}(\mathbf{x})
                =\big[
                    (\mathbf{x}_{0},\,\dots,\,\mathbf{x}_{k-1},\,1,\,
                    \mathbf{x}_{k},\,\dots,\,\mathbf{x}_{n-1})
                \big]
            \end{equation}
            which is continuous since the function
            $f:\mathbb{R}^{n}\rightarrow\mathbb{R}^{n+1}\setminus\{\,\mathbf{0}\,\}$
            defined by:
            \begin{equation}
                f(\mathbf{x})=
                (\mathbf{x}_{0},\,\dots,\,\mathbf{x}_{k-1},\,1,\,
                    \mathbf{x}_{k},\,\dots,\,\mathbf{x}_{n-1})
            \end{equation}
            is continuous, so $\varphi_{k}^{-1}$ is the composition of
            continuous functions. Since the sets
            $\mathcal{U}_{k}$ cover
            $\mathbb{R}^{n+1}\setminus\{\,\mathbf{0}\,\}$, the sets
            $\tilde{\mathcal{U}}_{k}$ also cover $\mathbb{RP}^{n}$. Because of
            this $\mathbb{RP}^{n}$ is locally Euclidean. It is also second
            countable since it can be covered with finitely many open sets
            each of which is homeomorphic to an open subset of $\mathbb{R}^{n}$,
            which is hence second countable. Since $\mathbb{RP}^{n}$ is the
            finite union of second countable open subspaces, it is second
            countable itself. It is also Hausdorff. Given
            $[\mathbf{x}]\ne[\mathbf{y}]$ we have that $\mathbf{y}$ is not
            of the form $\lambda\mathbf{x}$ for any real number, meaning
            $\mathbf{x}$ and $\mathbf{y}$ lie on different lines through the
            origin. Let $\theta$ be defined by:
            \begin{equation}
                \theta=
                \frac{1}{4}\arccos\Big(
                    \frac{\mathbf{x}\cdot\mathbf{y}}
                        {||\mathbf{x}||_{2}\,||\mathbf{y}||_{2}}
                \Big)
            \end{equation}
            $\theta$ is one-fourth the angle made between the lines through
            the origin spanned by $\mathbf{x}$ and $\mathbf{y}$. Let
            $\mathcal{U}$ and $\mathcal{V}$ be defined by:
            \begin{align}
                \mathcal{U}
                &=\big\{\,
                    \mathbf{z}\in\mathbb{R}^{n+1}\setminus\{\,\mathbf{0}\,\}
                        \;|\;\measuredangle(\mathbf{x},\,\mathbf{z})<\theta\,
                    \big\}\\
                \mathcal{V}
                &=\big\{\,
                    \mathbf{z}\in\mathbb{R}^{n+1}\setminus\{\,\mathbf{0}\,\}
                        \;|\;\measuredangle(\mathbf{y},\,\mathbf{z})<\theta\,
                    \big\}
            \end{align}
            Where $\measuredangle(\mathbf{p},\,\mathbf{q})$ is the angle
            between the non-zero vectors $\mathbf{p}$ and $\mathbf{q}$. These
            sets are open cones in
            $\mathbb{R}^{n+1}\setminus\{\,\mathbf{0}\,\}$
            (Fig.~\ref{fig:rpn_is_hausdorff_001}) which are also
            saturated with respect to $q$, and by the choice of $\theta$ they
            are disjoint. But then $\tilde{\mathcal{U}}=q[\mathcal{U}]$ and
            $\tilde{\mathcal{V}}=q[\mathcal{V}]$ are disjoint open subsets of
            $\mathbb{RP}^{n}$ such that
            $[\mathbf{x}]\in\tilde{\mathcal{U}}$ and
            $[\mathbf{y}]\in\tilde{\mathcal{V}}$. Hence $\mathbb{RP}^{n}$ is
            Hausdorff. The real projective space is therefore a topological
            manifold.
        \end{example}
        \begin{figure}
            \centering
            \includegraphics{../../images/rpn_is_hausdorff_001.pdf}
            \caption{$\mathbb{RP}^{n}$ is Hausdorff}
            \label{fig:rpn_is_hausdorff_001}
        \end{figure}
        The elements of $\mathbb{RP}^{n}$ are equivalence classes of
        $\mathbb{R}^{n+1}\setminus\{\,\mathbf{0}\,\}$. A
        \textit{point} in $\mathbb{RP}^{n}$ is a \textit{line} in
        $\mathbb{R}^{n+1}$ through the origin. It is not immediately clear that
        $\mathbb{RP}^{n}$ can be embedded as a subspace of $\mathbb{R}^{N}$ for
        some $N\in\mathbb{N}$. It indeed can, in fact
        $\mathbb{RP}^{n}$ can be embedded into $\mathbb{R}^{2n}$ for all $n>0$,
        but this is by no means obvious. The case $n=1$ is slightly obvious if
        you really think about what $\mathbb{RP}^{1}$ is
        (it's just a circle $\mathbb{S}^{1}$). The case $\mathbb{RP}^{2}$ is
        less obvious ($\mathbb{RP}^{2}$ is \textbf{not} a sphere). We can not
        embed the real projective plane into $\mathbb{R}^{3}$, unlike the
        sphere. If we try we'll end up with a surface that must intersect
        itself. This is shown in
        Fig.~\ref{fig:real_proj_plane_cross_cap_001}. This representation is
        known as the \textit{cross cap}.
        \begin{figure}
            \centering
            \includegraphics{../../images/real_proj_plane_cross_cap_001.pdf}
            \caption{The Real Projective Plane}
            \label{fig:real_proj_plane_cross_cap_001}
        \end{figure}
        We can do better than this. The cross cap has a crease in it, and this
        can be removed. David Hilbert, one of the pioneering mathematicians of
        the early $20^{\textrm{th}}$ century, thought it impossible to draw the
        real projective plane in $\mathbb{R}^{3}$ in such a way that it has
        no crease. He asked his student Werney Boy to try and prove this.
        Instead Boy discovered a method of drawing the real projective plane
        in $\mathbb{R}^{3}$ that has no crease (it is still self intersecting).
        This is called the \textit{Boy surface}. It is shown in
        Fig.~\ref{fig:boy_surface_apery_immersion}.
        \begin{figure}
            \centering
            \includegraphics{../../images/boy_surface_apery_immersion.pdf}
            \caption{The Boy Surface}
            \label{fig:boy_surface_apery_immersion}
        \end{figure}
        Bryant and Kusner discovered a way to do this using somewhat simpler
        functions involving complex variables. The Bryant-Kusner parameteriation
        is shown in Fig.~\ref{fig:boy_surface_bryant_kusner_parameterization}.
        \begin{figure}
            \centering
            \includegraphics{../../images/boy_surface_bryant_kusner_parameterization.pdf}
            \caption{The Bryant-Kusner Parameterization of the Boy Surface}
            \label{fig:boy_surface_bryant_kusner_parameterization}
        \end{figure}
\end{document}
