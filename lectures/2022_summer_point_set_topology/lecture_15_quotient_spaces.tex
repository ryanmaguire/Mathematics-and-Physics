%-----------------------------------LICENSE------------------------------------%
%   This file is part of Mathematics-and-Physics.                              %
%                                                                              %
%   Mathematics-and-Physics is free software: you can redistribute it and/or   %
%   modify it under the terms of the GNU General Public License as             %
%   published by the Free Software Foundation, either version 3 of the         %
%   License, or (at your option) any later version.                            %
%                                                                              %
%   Mathematics-and-Physics is distributed in the hope that it will be useful, %
%   but WITHOUT ANY WARRANTY; without even the implied warranty of             %
%   MERCHANTABILITY or FITNESS FOR A PARTICULAR PURPOSE.  See the              %
%   GNU General Public License for more details.                               %
%                                                                              %
%   You should have received a copy of the GNU General Public License along    %
%   with Mathematics-and-Physics.  If not, see <https://www.gnu.org/licenses/>.%
%------------------------------------------------------------------------------%
\documentclass{article}
\usepackage{graphicx}                           % Needed for figures.
\usepackage{amsmath}                            % Needed for align.
\usepackage{amssymb}                            % Needed for mathbb.
\usepackage{amsthm}                             % For the theorem environment.
\usepackage{float}
\usepackage{hyperref}
\hypersetup{
    colorlinks=true,
    linkcolor=blue,
    filecolor=magenta,
    urlcolor=Cerulean,
    citecolor=SkyBlue
}

%------------------------Theorem Styles-------------------------%
\theoremstyle{plain}
\newtheorem{theorem}{Theorem}[section]

% Define theorem style for default spacing and normal font.
\newtheoremstyle{normal}
    {\topsep}               % Amount of space above the theorem.
    {\topsep}               % Amount of space below the theorem.
    {}                      % Font used for body of theorem.
    {}                      % Measure of space to indent.
    {\bfseries}             % Font of the header of the theorem.
    {}                      % Punctuation between head and body.
    {.5em}                  % Space after theorem head.
    {}

% Define default environments.
\theoremstyle{normal}
\newtheorem{examplex}{Example}[section]
\newtheorem{definitionx}{Definition}[section]
\newtheorem{notationx}{Notation}[section]
\newtheorem{axiomx}{Axiom}[section]

\newenvironment{example}{%
    \pushQED{\qed}\renewcommand{\qedsymbol}{$\blacksquare$}\examplex%
}{%
    \popQED\endexamplex%
}

\newenvironment{definition}{%
    \pushQED{\qed}\renewcommand{\qedsymbol}{$\blacksquare$}\definitionx%
}{%
    \popQED\enddefinitionx%
}

\title{Point-Set Topology: Lecture 15}
\author{Ryan Maguire}
\date{Summer 2022}

% No indent and no paragraph skip.
\setlength{\parindent}{0em}
\setlength{\parskip}{0em}

\begin{document}
    \maketitle
    \section{Induced Equivalence Relation}
        Given a relation $R$ on a set $A$, it is possible for $R$ to be very
        dull. It does not need to be reflexive, symmetric, or transitive. We
        can always transform $R$ into a reflexive relation by adding in
        $aRa$ for all $a\in{A}$. We can then make it symmetric by adding
        $bRa$ for all $a,b\in{A}$ such that $aRb$. Lastly, we can make it
        transitive by enlarging the relation as well. This idea is the
        \textit{induced equivalence relation} from $R$.
        \begin{theorem}
            If $A$ is a set, and if
            $\mathcal{R}\subseteq\mathcal{P}(A\times{A})$ is a non-empty such
            that for all $R\in\mathcal{R}$ it is true that $R$ is an
            equivalence relation on $A$, then $\bigcap\mathcal{R}$ is an
            equivalence relation on $A$.
        \end{theorem}
        \begin{proof}
            Given $a\in{A}$, since $\mathcal{R}$ is non-empty, there is some
            $R\in\mathcal{R}$ such that $a\in{R}$. But $R$ is an equivalence
            relation, so $aRa$. But for all $R'\in\mathcal{R}$, $R'$ is also
            an equivalence relation, so $aR'a$. Hence,
            $(a,\,a)\in\bigcap\mathcal{R}$. That is, $a\bigcap\mathcal{R}a$.
            If $a,b\in\bigcap\mathcal{R}$ are such that
            $(a,\,b)\in\bigcap\mathcal{R}$, then for all $R\in\mathcal{R}$ we
            have $aRb$. But since $R$ is an equivalence relation this implies
            $bRa$. So $bRa$ for all $R\in\mathcal{R}$ and therefore
            $(b,\,a)\in\bigcap\mathcal{R}$. Lastly, if $a,b,c\in{A}$ are such
            that $(a,\,b)\in\bigcap\mathcal{R}$ and
            $(b,\,c)\in\bigcap\mathcal{R}$, then for all $R\in\mathcal{R}$ we
            have $aRb$ and $bRc$. But $R$ is an equivalence relation, so then
            $aRc$. But then $(a,\,c)\in\bigcap\mathcal{R}$, so
            $\bigcap\mathcal{R}$ is transitive. Hence, $\bigcap\mathcal{R}$ is
            an equivalence relation.
        \end{proof}
        \begin{theorem}
            If $A$ is a set, if $R$ is an equivalence relation on $A$, and if
            $\mathcal{R}$ is the set of all equivalences relations
            $R'$ on $R$ such that $R\subseteq{R}'$, then
            $\bigcap\mathcal{R}$ is an equivalence relation on $A$ such that
            $R\subseteq\bigcap\mathcal{R}$.
        \end{theorem}
        \begin{proof}
            Firstly, $\mathcal{R}$ is non-empty since $A\times{A}$ is an
            equivalence relation on $A$, it is the relation that says $a$ is
            related to $b$ for all $a,b\in{R}$. That is, the relation that says
            everything is related to everything else. Hence $\mathcal{R}$ is a
            non-empty set of equivalence relations on $A$, so
            $\bigcap\mathcal{R}$ is an equivalence relation on $A$ by the
            previous theorem. But for all $R'\in\mathcal{R}$ it is true that
            $R\subseteq{R}'$, so $R\subseteq\bigcap\mathcal{R}$.
        \end{proof}
        \begin{definition}[\textbf{Induced Equivalence Relation}]
            The induced equivalence relation on a set $A$ by a relation $R$ is
            the equivalence relation $\bigcap\mathcal{R}$ where $\mathcal{R}$
            is the set of all equivalence relations $R'$ on $A$ such that
            $R\subseteq{R}'$.
        \end{definition}
        \begin{example}
            Let $A$ be a set and $R=\emptyset$, the empty relation. This
            relation says nothing is related, not even $a\in{A}$ is related to
            itself. The induced equivalence relation is the diagonal
            $\Delta_{A}=\{\,(a,\,a)\;|\;a\in{A}\,\}$. The only thing we need to
            add to make $R$ an equivalence relation is reflexivity.
        \end{example}
        \begin{theorem}
            If $A$ is a set, and if $R$ is an equivalence relation on $A$,
            then the induced equivalence relation $R'$ is equal to $R$.
        \end{theorem}
        \begin{proof}
            Let $\mathcal{R}$ be the set of all equivalence relations $R''$ on
            $A$ such that $R\subseteq{R}''$. But $R$ is an equivalence relation
            on $A$, and $R\subseteq{R}$, so $R\in\mathcal{R}$. Hence
            $\bigcap\mathcal{R}\subseteq{R}$. But also
            $R\subseteq\bigcap\mathcal{R}$. So $R=\bigcap\mathcal{R}$. But
            $R'=\bigcap\mathcal{R}$ since $R'$ is the induced equivalence
            relation, so $R=R'$.
        \end{proof}
        \begin{definition}[\textbf{Induced Equivalence Relation by a Subset}]
            The equivalence relation of a subset $A\subseteq{X}$ of a set $X$
            is the induced equivalence relation $R_{A}$ induced by the
            relation $R$ on $A$ defined by:
            \begin{equation}
                R=\{\,(a,\,b)\in{X}\;|\;a\in{A}\textrm{ and }b\in{A}\,\}
            \end{equation}
            That is, the equivalence relation induced by saying that everything
            in $A$ is related to everything else in $A$.
        \end{definition}
    \section{Quotient Topology}
        Given a set $X$ and an equivalence relation $R$ on $X$, we may form
        the quotient set $X/R$ which is the set of all equivalence relations
        of $X$ under $R$. Intuitively, we are taking points in $X$ and
        \textit{gluing} them together in the quotient space. If $X$ had a
        topology, it seems like it should be possible to give a topology to
        the quotient since gluing things together certainly seems like a
        topological operation. This is indeed possible, and quotient spaces are
        very common in topology since they provide a plethora of spaces one
        can ponder and construct. We construct the quotient topology via the
        quotient map. In set theory, there is a canonical quotient function
        $q:X\rightarrow{X}/R$ defined by $q(x)=[x]$ for all $x\in{X}$, where
        $[x]\in{X}/R$ is the equivalence class of $x$. This is something like
        projecting points $x$ in $X$ to the point in $X/R$ where $x$ was glued
        to, the equivalence class $[x]$. This gluing operation should be
        continuous. We define the quotient topology on $X/R$ via the
        \textit{final topology} on $X/R$ which makes $q$ continuous.
        \begin{definition}[\textbf{Quotient Topology}]
            The quotient topology on the quotient set $X/R$ of a set $X$ under
            an equivalence relation $R$ with respect to a topological space
            $(X,\,\tau)$ is the set $\tau_{X/R}$ defined as the final topology
            with respect to the quotient map $q:X\rightarrow{X}/R$ defined
            by $q(x)=[x]$, and with respect to the topology $\tau$ on $X$.
        \end{definition}
        \begin{theorem}
            If $(X,\,\tau)$ is a topological space, if $R$ is an equivalence
            relation on $X$, and if $\tau_{X/R}$ is the quotient topology on
            $X/R$, then $(X/R,\,\tau_{X/R})$ is a topological space.
        \end{theorem}
        \begin{proof}
            The quotient topology is the final topology with respect to
            $(X,\,\tau)$ and the quotient map $q:X\rightarrow{X}/R$ defined by
            $q(x)=[x]$. But the final topology for any function
            $f:X\rightarrow{X}/R$ with respect to $(X,\,\tau)$ is a topology on
            $X/R$, so in particular $\tau_{X/R}$ is a topology.
        \end{proof}
        Since $\tau_{X/R}$ is the final topology with respect to the quotient
        mapping $q$, a subset $\mathcal{U}\subseteq{X}/R$ is open
        \textit{if and only if} $q^{-1}[\mathcal{U}]$ is open.
        \par\hfill\par
        Please note continuity alone is not sufficient enough to say that
        $q^{-1}[\mathcal{U}]$ being open implies $\mathcal{U}$ is open. The
        implication goes one way. If $\mathcal{U}$ is open, and if $q$ is
        continuous, then $q^{-1}[\mathcal{U}]$ is open. For a general continuous
        function $f:X\rightarrow{Y}$ with topology $\tau_{X}$ and $\tau_{Y}$,
        given $\mathcal{V}\subseteq{Y}$ and $f^{-1}[\mathcal{V}]\in\tau_{X}$, it
        is not necessarily true that we can conclude that
        $\mathcal{V}\in\tau_{Y}$. Let $X=Y=\mathbb{R}$, and
        $\tau_{X}=\tau_{Y}=\tau_{\mathbb{R}}$, the standard topology on
        $\mathbb{R}$. Let $f(x)=1$. Since it is a constant function, it is
        continuous. But $f^{-1}[\{\,1\,\}]=\mathbb{R}$, which is open, however
        $\{\,1\,\}$ is not open in $\mathbb{R}$.
        \par\hfill\par
        The quotient map, with the quotient topology, is very special in this
        regard. $\mathcal{U}\subseteq{X}/R$ is open \textit{if and only if}
        $q^{-1}[\mathcal{U}]$ is open in $X$. This fact is used constantly in
        the proofs of various claims about quotient spaces.
    \section{Quotient of Subspaces}
    \section{Examples and Counterexamples}
\end{document}
