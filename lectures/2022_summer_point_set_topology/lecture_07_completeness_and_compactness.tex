%-----------------------------------LICENSE------------------------------------%
%   This file is part of Mathematics-and-Physics.                              %
%                                                                              %
%   Mathematics-and-Physics is free software: you can redistribute it and/or   %
%   modify it under the terms of the GNU General Public License as             %
%   published by the Free Software Foundation, either version 3 of the         %
%   License, or (at your option) any later version.                            %
%                                                                              %
%   Mathematics-and-Physics is distributed in the hope that it will be useful, %
%   but WITHOUT ANY WARRANTY; without even the implied warranty of             %
%   MERCHANTABILITY or FITNESS FOR A PARTICULAR PURPOSE.  See the              %
%   GNU General Public License for more details.                               %
%                                                                              %
%   You should have received a copy of the GNU General Public License along    %
%   with Mathematics-and-Physics.  If not, see <https://www.gnu.org/licenses/>.%
%------------------------------------------------------------------------------%
\documentclass{article}
\usepackage{graphicx}                           % Needed for figures.
\usepackage{amsmath}                            % Needed for align.
\usepackage{amssymb}                            % Needed for mathbb.
\usepackage{amsthm}                             % For the theorem environment.
\usepackage{float}
\usepackage{hyperref}
\hypersetup{
    colorlinks=true,
    linkcolor=blue,
    filecolor=magenta,
    urlcolor=Cerulean,
    citecolor=SkyBlue
}

%------------------------Theorem Styles-------------------------%
\theoremstyle{plain}
\newtheorem{theorem}{Theorem}[section]

% Define theorem style for default spacing and normal font.
\newtheoremstyle{normal}
    {\topsep}               % Amount of space above the theorem.
    {\topsep}               % Amount of space below the theorem.
    {}                      % Font used for body of theorem.
    {}                      % Measure of space to indent.
    {\bfseries}             % Font of the header of the theorem.
    {}                      % Punctuation between head and body.
    {.5em}                  % Space after theorem head.
    {}

% Define default environments.
\theoremstyle{normal}
\newtheorem{examplex}{Example}[section]
\newtheorem{definitionx}{Definition}[section]
\newtheorem{notationx}{Notation}[section]
\newtheorem{axiomx}{Axiom}[section]

\newenvironment{example}{%
    \pushQED{\qed}\renewcommand{\qedsymbol}{$\blacksquare$}\examplex%
}{%
    \popQED\endexamplex%
}

\newenvironment{definition}{%
    \pushQED{\qed}\renewcommand{\qedsymbol}{$\blacksquare$}\definitionx%
}{%
    \popQED\enddefinitionx%
}

\title{Point-Set Topology: Lecture 7}
\author{Ryan Maguire}
\date{Summer 2022}

% No indent and no paragraph skip.
\setlength{\parindent}{0em}
\setlength{\parskip}{0em}

\begin{document}
    \maketitle
    \section{Metric Topology}
        \begin{definition}[\textbf{Metric Topology}]
            The metric topology on a metric space $(X,\,d)$ is the
            set $\tau_{d}\subseteq\mathcal{P}(X)$ such that for all
            $\mathcal{U}$, $\mathcal{U}\in\tau_{d}$ if and only if
            $\mathcal{U}$ is open in $(X,\,d)$. That is, $\tau_{d}$ is the
            set of all open subsets of $(X,\,d)$.
        \end{definition}
        We have seen in previous theorems that the metric topology $\tau_{d}$
        of a metric space $(X,\,d)$ has several properties. First,
        $\emptyset\in\tau_{d}$ and $X\in\tau_{d}$. That is, the empty set is
        open and the whole space is open. Secondly, given any subset
        $\mathcal{O}\subseteq\tau_{d}$, the union $\bigcup\mathcal{O}$ is an
        element of $\tau_{d}$. That is, the arbitrary union of open sets is
        open. Lastly, if $\mathcal{U},\mathcal{V}\in\tau_{d}$, then
        $\mathcal{U}\cap\mathcal{V}\in\tau_{d}$. That is, the finite
        intersection of open sets is open. We will take these properties and
        use them to define a topological space. A topological space is a set
        $X$ and a subset $\tau\subseteq\mathcal{P}(X)$ with the four properties
        mentioned previously. This will be made clear in later lectures, for now
        we want to discuss which properties of a metric space are
        \textit{topological} and which properties are geometric, or metric
        properties.
        \begin{definition}[\textbf{Topologically Equivalent Metrics}]
            Topologically equivalent metrics on a set $X$ are metrics
            $d_{0}$ and $d_{1}$ such that their respective metric topologies
            $\tau_{d_{0}}$ and $\tau_{d_{1}}$ are equal,
            $\tau_{d_{0}}=\tau_{d_{1}}$.
        \end{definition}
        To provide examples of equivalent metrics, it is best to use
        the following theorem.
        \begin{theorem}
            If $X$ is a set, and $d_{0}$ and $d_{1}$ are metrics on $X$,
            then $d_{0}$ and $d_{1}$ are topologically equivalent if and only
            if for all $x\in{X}$ and $r>0$, there is an $r_{0}>0$ and an
            $r_{1}>0$ such that
            $B_{r_{0}}^{(X,\,d_{0})}(x)\subseteq{B}_{r}^{(X,\,d_{1})}(x)$ and
            $B_{r_{1}}^{(X,\,d_{1})}(x)\subseteq{B}_{r}^{(X,\,d_{0})}(x)$.
            That is, the open balls can be \textit{nested} inside of each other.
        \end{theorem}
        \begin{proof}
            If $\tau_{d_{0}}=\tau_{d_{1}}$, then
            $B_{r}^{(X,\,d_{0})}(x)$ is open in
            $\tau_{d_{1}}$ meaning there is an $r_{1}>0$ such that
            $B_{r_{1}}^{(X,\,d_{1})}(x)\subseteq{B}_{r}^{(X,\,d_{0})}(x)$.
            Similarly, if $\tau_{d_{0}}=\tau_{d_{1}}$, then
            $B_{r}^{(X,\,d_{1})}(x)$ is open in
            $\tau_{d_{0}}$ meaning there is an $r_{0}>0$ such that
            $B_{r_{0}}^{(X,\,d_{0})}(x)\subseteq{B}_{r}^{(X,\,d_{0})}(x)$.
            In the other direction, suppose $\tau_{d_{0}}$ and
            $\tau_{d_{1}}$ are such that open balls can be nested.
            Let $\mathcal{V}\in\tau_{d_{1}}$. For all
            $x\in\mathcal{V}$, since $\mathcal{V}$ is open, there is an
            $r>0$ such that $B_{r}^{(X,\,d_{1})}(x)\subseteq\mathcal{V}$.
            But then there is an $r_{0}>0$ such that
            $B_{r_{0}}^{(X,\,d_{0})}(x)\subseteq{B}_{r}^{(X,\,d_{1})}(x)$, and
            hence
            $B_{r_{0}}^{(X,\,d_{0})}(x)\subseteq\mathcal{V}$, so
            $\mathcal{V}\in\tau_{d_{0}}$. Similarly, if
            $\mathcal{U}\in\tau_{d_{0}}$, then for all $x\in\mathcal{U}$, since
            $\mathcal{U}$ is open, there is an $r>0$ such that
            $B_{r}^{(X,\,d_{0})}(x)\subseteq\mathcal{U}$. But then there is
            an $r_{1}>0$ such that
            $B_{r_{1}}^{(X,\,d_{1})}(x)\subseteq{B}_{r}^{(X,\,d_{0})}(x)$, and
            hence $B_{r_{1}}^{(X,\,d_{1})}(x)\subseteq\mathcal{U}$. That is,
            $\mathcal{U}\in\tau_{d_{1}}$. Therefore,
            $\tau_{d_{0}}=\tau_{d_{1}}$.
        \end{proof}
        \begin{example}
            Let $(\mathbb{R}^{2},\,d_{E})$ be the Euclidean metric space on the
            plane, and $(\mathbb{R}^{2},\,d_{M})$ be the Manhattan metric space.
            Open balls in the Euclidean metric are open disks and open balls
            in the Manhattan metric are open diamonds. We can nest one inside
            of the other, showing that the Euclidean and Manhattan metrics are
            topologically equivalent.
            See Figs.~\ref{fig:manhattan_top_equiv_euclidean_001}
            and \ref{fig:manhattan_top_equiv_euclidean_002}.
        \end{example}
        \begin{figure}
            \centering
            \includegraphics{../../images/manhattan_top_equiv_euclidean_001.pdf}
            \caption{Manhattan Open Sets Nested in Euclidean Open Sets}
            \label{fig:manhattan_top_equiv_euclidean_001}
        \end{figure}
        \begin{figure}
            \centering
            \includegraphics{../../images/manhattan_top_equiv_euclidean_002.pdf}
            \caption{Euclidean Open Sets Nested in Manhattan Open Sets}
            \label{fig:manhattan_top_equiv_euclidean_002}
        \end{figure}
        \begin{example}
            Let $(\mathbb{R}^{2},\,d_{E})$ be the Euclidean plane and
            $(\mathbb{R}^{2},\,d_{\textrm{max}})$ be the maximum metric space
            on the plane (the chess board metric). Open balls in the Euclidean
            plane are open disks and open balls in the max metric are open
            squares. We can nest one inside the other, meaning the Euclidean
            metric and the maximum metric are equivalent on $\mathbb{R}^{2}$
            (See Figs.~\ref{fig:max_top_equiv_euclidean_001} and
            \ref{fig:max_top_equiv_euclidean_002}).
        \end{example}
        \begin{figure}
            \centering
            \includegraphics{../../images/max_top_equiv_euclidean_001.pdf}
            \caption{Max Open Sets Nested in Euclidean Open Sets}
            \label{fig:max_top_equiv_euclidean_001}
        \end{figure}
        \begin{figure}
            \centering
            \includegraphics{../../images/max_top_equiv_euclidean_002.pdf}
            \caption{Euclidean Open Sets Nested in Max Open Sets}
            \label{fig:max_top_equiv_euclidean_002}
        \end{figure}
        \begin{example}
            The Euclidean metric and the Paris metric on
            $\mathbb{R}^{2}$ are not equivalent. Given a point
            $\mathbf{x}\in\mathbb{R}^{2}$, $\mathbf{x}\ne\mathbf{0}$,
            choose $r=||\mathbf{x}||_{2}/2$, where $||\mathbf{x}||_{2}$ is the
            standard Euclidean length of the vector $\mathbf{x}$. The open
            ball centered at $\mathbf{x}$ with radius $r$ is an open
            line segment going from $\mathbf{x}-(r,\,r)$ to
            $\mathbf{x}+(r,\,r)$. Open line segments are not open in the
            Euclidean metric, so the Paris metric is topologically different
            than the Euclidean metric.
        \end{example}
        \begin{example}
            The London metric and the Paris metric are not topologically
            equivalent. Given
            $\mathbf{x}\in\mathbb{R}^{2}$, $\mathbf{x}\ne\mathbf{0}$, choose
            $r=||\mathbf{x}||_{2}/2$. The open ball of radius $r$ centered about
            $\mathbf{x}$ in the Paris metric, as described before, is an open
            line segment in the plane. The open ball centered about
            $\mathbf{x}$ of radius $r$ in the London metric is just the
            singleton $\{\,\mathbf{x}\,\}$. To see this, for all other points
            $\mathbf{y}\ne\mathbf{x}$, the distance in the London metric is:
            \begin{equation}
                d_{L}(\mathbf{x},\,\mathbf{y})=
                    ||\mathbf{x}||_{2}+||\mathbf{y}||_{2}
                    =2r+||\mathbf{y}||_{2}>r
            \end{equation}
            Meaning $\mathbf{y}$ is not in the ball of radius $r$ centered about
            $\mathbf{x}$ in the London metric. So, the ball of radius $r$
            centered about $\mathbf{x}$ is just $\{\,\mathbf{x}\,\}$. Single
            points are not open in the Paris metric, showing the two metrics
            are not topologically equivalent.
        \end{example}
        \begin{example}
            The London metric is not topologically equivalent to the discrete
            metric. It is true that for every point $\mathbf{x}\ne\mathbf{0}$,
            the point $\{\,\mathbf{x}\,\}$ is open in the London metric, which
            certainly seems similar to the discrete metric, but the set
            $\{\,\mathbf{0}\,\}$ is not open. An open ball about
            $\mathbf{0}$ in the London metric is an open disk, so
            $\{\,\mathbf{0}\,\}$ is not open. However, $\{\,\mathbf{0}\,\}$
            is open in the discrete metric.
        \end{example}
        Topological properties are those that are detected by the
        topologies of the metric space. Convergence, continuity, open and
        closed, are all notions that are topological. As we will see
        repeatedly throughout the course, \textit{homeomorphisms} are functions
        that preserve topological properties.
        \begin{definition}[\textbf{Homeomorphism}]
            A homeomorphism from a metric space $(X,\,d_{X})$ to a metric space
            $(Y,\,d_{Y})$ is a bijective continuous function $f:X\rightarrow{Y}$
            such that $f^{-1}$ is continuous.
        \end{definition}
        Global isometries are functions that preserve all metric properties.
        Global isometries are, in particular, homeomorphisms.
        \begin{theorem}
            If $(X,\,d_{X})$ and $(Y,\,d_{Y})$ are metric spaces, and if
            $f:X\rightarrow{Y}$ is a global isometry, then $f$ is a
            homeomorphism.
        \end{theorem}
        \begin{proof}
            We have proven that isometries are continuous. All we need to do
            now is prove that if $f:X\rightarrow{Y}$ is a global isometry,
            then $f^{-1}$ is also an isometry. Let $y_{0},y_{1}\in{Y}$.
            Since $f$ is a global isometry, it is bijective, and hence there
            are $x_{0},x_{1}\in{X}$ with $f(x_{0})=y_{0}$ and $f(x_{1})=y_{1}$.
            But then, since $f$ is an isometry, we have:
            \begin{align}
                d_{Y}(y_{0},\,y_{1})
                    &=d_{Y}\big(f(x_{0}),\,f(x_{1})\big)\\
                    &=d_{X}(x_{0},\,x_{1})\\
                    &=d_{X}\big(f^{-1}(x_{0}),\,f^{-1}(x_{1})\big)
            \end{align}
            and hence $f^{-1}$ is an isometry. But then $f$ and $f^{-1}$ are
            continuous, so $f$ is a homeomorphism.
        \end{proof}
        Similar to how not every continuous function is an isometry, not every
        homeomorphism is a global isometry. Homeomorphism is a weaker notion,
        but also a far more general and with more applications. Think about
        the real line. The only isometries are translations
        $f(x)=x+a$, reflections $f(x)=-x$, and glide reflections,
        $f(x)=-x+a$. Most of the functions used in calculus and physics are not
        isometries, but are usually continuous.
    \section{Completeness}
        \begin{definition}[\textbf{Cauchy Sequences}]
            A Cauchy sequence in a metric space $(X,\,d)$ is a sequence
            $a:\mathbb{N}\rightarrow{X}$ such that for all $\varepsilon>0$ there
            is an $N\in\mathbb{N}$ such that given $m,n\in\mathbb{N}$ with
            $m>N$ and $n>N$, it is true that $d(a_{n},\,a_{m})<\varepsilon$.
        \end{definition}
        Cauchy sequences are sequences where the points $a_{n}$ start to get
        closer and closer together as $n$ increases. Convergent sequences are,
        in particular, Cauchy sequences.
        \begin{theorem}
            If $(X,\,d)$ is a metric space, and if $a:\mathbb{N}\rightarrow{X}$
            is a convergent sequence, then $a$ is a Cauchy sequence.
        \end{theorem}
        \begin{proof}
            Let $\varepsilon>0$. Since $a:\mathbb{N}\rightarrow{X}$ is
            convergent, there is an $x\in{X}$ with $a_{n}\rightarrow{x}$. But
            then there is an $N\in\mathbb{N}$ such that for all $n\in\mathbb{N}$
            with $n>N$ it is true that $d(a_{n},\,x)<\frac{\varepsilon}{2}$.
            But then for $m>N$ and $n>N$ we have:
            \begin{equation}
                d(a_{m},\,a_{n})\leq
                d(a_{m},\,x)+d(a_{n},\,x)<
                \frac{\varepsilon}{2}+\frac{\varepsilon}{2}
                =\varepsilon
            \end{equation}
            and therefore $a$ is a Cauchy sequence.
        \end{proof}
        Since the points $a_{n}$ are getting closer and closer together it is
        natural to ask if the converse of this theorem is true as well. That is,
        if $a:\mathbb{N}\rightarrow{X}$ is a Cauchy sequence in a metric space
        $(X,\,d)$, is $a$ also a convergent sequence?
        \begin{example}
            Define $a:\mathbb{N}\rightarrow\mathbb{Q}$ by
            $a_{0}=1$, $a_{1}=1.4$, $a_{2}=1.41$, $a_{3}=1.414$, and
            $a_{n}$ is the first $n$ decimals of $\sqrt{2}$. This is a
            Cauchy sequence, given $m<n$,
            $d(a_{m},\,a_{n})$ is less than $10^{-m}$, which can be made
            arbitrarily small. It does not converge in $\mathbb{Q}$. The
            \textit{limit} we want to say this converges to is
            $\sqrt{2}$, but $\sqrt{2}$ is not a rational number, so in reality
            there is no limit of this sequence.
        \end{example}
        The problem with $\mathbb{Q}$ is it has a lot of holes, these are the
        irrational numbers. If we fill in those holes we get the real numbers.
        This idea gives rise to the notion of \textit{complete} metric spaces.
        \begin{definition}[\textbf{Complete Metric Space}]
            A complete metric space is a metric space $(X,\,d)$ such that for
            every Cauchy sequence $a:\mathbb{N}\rightarrow{X}$, it is true
            that $a$ is a convergent sequence.
        \end{definition}
        The real numbers are complete, with the usual metric
        $d(x,\,y)=|x-y|$. Given a bounded set $A\subseteq\mathbb{R}$, meaning
        there is an $M\in\mathbb{R}$ such that for all $x\in{A}$ it is true
        that $|x|<M$, the real numbers have the property that $A$ has a
        \textit{least upper bound} and a \textit{greatest lower bound}. That is,
        numbers $r$ and $s$ such that $r$ is a lower bound, all $x\in{A}$
        are such that $r\leq{x}$, and $s$ is an upper bound, all $x\in{A}$
        are such that $x\leq{s}$, but moreover $r$ is the \textit{largest}
        possible lower bound, and $s$ is the \textit{smallest} possible upper
        bound. The rationals do not have this property. Given the set
        $A=\{\,x\in\mathbb{Q}\;|\;x^{2}<2\,\}$, there is no least upper bound.
        If you give me a rational number that is an upper bound for $A$, I can
        find a smaller rational number that is also an upper bound. The
        \textit{least} upper bound of this set is $\sqrt{2}$, but again, this
        is not a rational number. This least upper bound property can be
        used to show that the real numbers are complete. First, a Cauchy
        sequence is bounded. Given $a:\mathbb{N}\rightarrow\mathbb{R}$ a
        Cauchy sequence, the points $a_{n}$ are getting really close together,
        so it would be impossible for the sequence to diverge off to infinity.
        Using this we consider the set of all real numbers $r$ where there are
        infinitely many $a_{n}$ less than $r$. Since the $a_{n}$ are bounded,
        this set is non-empty and bounded above, so there is a least upper
        bound. Using a bit of work, you can then show that this Cauchy
        sequence must converge to this least upper bound, and \textit{viola},
        you have proven that the real numbers are complete.
        \par\hfill\par
        Completeness motivates a more topological notion,
        \textit{compactness}. We'll first introduce a few more metric notions
        before heading into this topic.
        \begin{definition}[\textbf{Bounded Metric Space}]
            A bounded metric space is a metric space $(X,\,d)$ such that there
            is an $M>0$ such that for all $x,y\in{X}$ it is true that
            $d(x,\,y)<M$.
        \end{definition}
        \begin{example}
            The real line is not bounded with the standard metric. Given
            any $M>0$, choose $x=0$ and $y=M+1$. Then
            $d(x,\,y)=|x-y|=M+1$ which is greater than $M$.
        \end{example}
        \begin{example}
            The real line with the arctan metric
            $d(x,\,y)=|\textrm{atan}(x)-\textrm{atan}(y)|$, is bounded with
            bound $M=\pi$.
        \end{example}
        \begin{example}
            The circle $\mathbb{S}^{1}$ with the subspace metric from
            $\mathbb{R}^{2}$ is bounded, any number $M>2$ suffices as a bound.
        \end{example}
        Is boundedness a topological property? That is, if $d_{0}$ and $d_{1}$
        are equivalent metrics on $X$, and if $d_{0}$ is unbounded, is
        $d_{1}$ also unbounded?
        \begin{theorem}
            If $(X,\,d)$ is a metric space, then there exists a topologically
            equivalent metric $\rho$ such that $(X,\,\rho)$ is a bounded
            metric space.
        \end{theorem}
        \begin{proof}
            Let $\rho:X\times{X}\rightarrow\mathbb{R}$ be defined by:
            \begin{equation}
                \rho(x,\,y)=\frac{d(x,\,y)}{1+d(x,\,y)}
            \end{equation}
            $\rho$ is a metric on $X$. It is positive-definite since the
            denominator is always positive and the numerator is
            positive-definite. It is symmetric since:
            \begin{align}
                \rho(y,\,x)&=\frac{d(y,\,x)}{1+d(y,\,x)}\\
                    &=\frac{d(x,\,y)}{1+d(x,\,y)}\\
                    &=\rho(x,\,y)
            \end{align}
            Lastly, the triangle inequality. There are two cases.
            Case 1, $d(x,\,y)\leq{d}(x,\,z)$ and
            $d(y,\,z)\leq{d}(x,\,z)$. We get:
            \begin{align}
                \rho(x,\,z)&=\frac{d(x,\,z)}{1+d(x,\,z)}\\
                &\leq\frac{d(x,\,y)+d(y,\,z)}{1+d(x,\,z)}\\
                &=\frac{d(x,\,y)}{1+d(x,\,z)}+
                    \frac{d(y,\,z)}{1+d(x,\,z)}\\
                &\leq\frac{d(x,\,y)}{1+d(x,\,y)}+\frac{d(y,\,z)}{1+d(y,\,z)}\\
                &=\rho(x,\,y)+\rho(y,\,z)
            \end{align}
            Case 2, $d(x,\,z)\leq{d}(x,\,y)$ or $d(x,\,z)\leq{d}(y,\,z)$.
            Since the function $f(x)=\frac{x}{1+x}$ is strictly increasing
            on the set $[0,\infty)$ we get $\rho(x,\,z)\leq\rho(x,\,y)$ or
            $\rho(x,\,z)\leq\rho(y,\,z)$, and hence
            $\rho(x,\,z)\leq\rho(x,\,y)+\rho(y,\,z)$. This metric is
            topologically equivalent. Given $r>0$, let
            $r_{d}=r$. Then if $y\in{B}_{r_{d}}^{(X,\,d)}(x)$, we have:
            \begin{equation}
                \rho(x,\,y)=\frac{d(x,\,y)}{1+d(x,\,y)}<d(x,\,y)<r_{d}=r
            \end{equation}
            and hence $y\in{B}_{r}^{(X,\,\rho)}(x)$. That is,
            $B_{r_{d}}^{(X,\,d)}(x)\subseteq{B}_{r}^{(X,\,\rho)}(x)$. If $r<1$,
            let $r_{\rho}=\frac{r}{1-r}$.
            If $y\in{B}_{r_{\rho}}^{(X,\,\rho)}(x)$, then
            (since $\frac{x}{1-x}$ is strictly increasing for $0<x<1$):
            \begin{equation}
                d(x,\,y)=\frac{\rho(x,\,y)}{1-\rho(x,\,y)}
                    <\frac{r_{\rho}}{1-r_{\rho}}=r
            \end{equation}
            so $y\in{B}_{r}^{(X,\,d)}(x)$, and hence
            $B_{r_{\rho}}^{(X,\,\rho)}(x)\subseteq{B}_{r}^{(X,\,d)}(x)$.
            If $r\geq{1}$, let $r'=\frac{1}{2}$ and
            $r_{\rho}=\frac{r'}{1-r'}$. Since $r'<r$,
            $B_{r'}^{(X,\,d)}(x)\subseteq{B}_{r}^{(X,\,d)}(x)$ and hence
            $B_{r_{\rho}}^{(X,\,\rho)}\subseteq{B}_{r}^{(X,\,d)}(x)$. By
            the theorem at the start of these notes,
            $(X,\,d)$ and $(X,\,\rho)$ are topologically equivalent. Moreover,
            since $\frac{x}{1+x}$ is bounded on $[0,\infty)$,
            $(X,\,\rho)$ is a bounded metric space.
        \end{proof}
        \begin{definition}[\textbf{Totally Bounded Metric Space}]
            A totally bounded metric space is a metric space $(X,\,d)$ such
            that for all $\varepsilon>0$ there are finitely many points
            $a_{0},\,\dots,\,a_{N}$ such that:
            \begin{equation}
                X=\bigcup_{n=0}^{N}B_{\varepsilon}^{(X,\,d)}(a_{n})
            \end{equation}
            That is, the set of $\varepsilon$ balls centered about the points
            $a_{n}$ completely cover the metric space.
        \end{definition}
        \begin{theorem}
            If $(X,\,d)$ is a totally bounded metric space, then
            $(X,\,d)$ is a bounded metric space.
        \end{theorem}
        \begin{proof}
            Let $(X,\,d)$ be totally bounded, and let $\varepsilon=1$. There
            exists finitely many points $a_{0},\,\dots,\,a_{N}$ such that
            \begin{equation}
                X=\bigcup_{n=0}^{N}B_{\varepsilon}^{(X,\,d)}(a_{n})
            \end{equation}
            Let $r$ be the maximum value of $d(a_{n},\,a_{m})$ for all
            $0\leq{m},n\leq{N}$, and let $M=r+2$. Let $x,y\in{X}$ be arbitrary.
            There are two points $a_{m},a_{n}$ such that
            $x\in{B}_{1}^{(X,\,d)}(a_{m})$ and $y\in{B}_{1}^{(X,\,d)}(a_{n})$
            since these $\varepsilon$ balls cover the entirety of $X$.
            But then:
            \begin{equation}
                d(x,\,y)\leq{d}(x,\,a_{m})+d(a_{m},\,a_{n})+d(a_{n},\,y)
                <1+r+1=M
            \end{equation}
            so $M$ is a bound for $(X,\,d)$.
        \end{proof}
        The converse need not be true.
        \begin{example}
            Equip $\mathbb{R}$ with the discrete metric:
            \begin{equation}
                d(x,\,y)=
                \begin{cases}
                    0&x=y\\
                    1&x\ne{y}
                \end{cases}
            \end{equation}
            The metric space $(\mathbb{R},\,d)$ is bounded by 2. It is not
            totally bounded. Given $\varepsilon=\frac{1}{2}$, the only way
            to cover $\mathbb{R}$ with $\varepsilon$ balls is by placing an
            $\varepsilon$ ball about every real number $r\in\mathbb{R}$, so
            we can't possibly cover $\mathbb{R}$ with finitely many
            $\varepsilon$ balls with the discrete metric, even though the
            space is bounded.
        \end{example}
        \begin{theorem}[\textbf{Bolzano's Theorem}]
            If $a:\mathbb{N}\rightarrow\mathbb{R}$ is a sequence, then there
            is monotone subsequence. That is, a subsequence $a_{k}$ such that
            for all $n\in\mathbb{N}$ it is true that
            $a_{k_{n}}\leq{a}_{k_{n+1}}$ (monotone increasing), or such that
            for all $n\in\mathbb{N}$ it is true that
            $a_{k_{n}}\geq{a}_{k_{n+1}}$ (monotone decreasing).
        \end{theorem}
        I'll give a sketch of proof via a picture. For simplicity, suppose
        $a:\mathbb{N}\rightarrow\mathbb{R}^{+}$ is a sequence of positive
        numbers. Place a flashlight infinitely far away at $-\infty$ on the
        $x$ axis. Given the element $a_{n}\in\mathbb{R}$ of the sequence,
        draw a straight line from $(n,\,a_{n})$ to $(n,\,0)$. These act as
        \textit{walls}. With the light shining from behind, some walls will
        be lit and some will not. If there are infinitely many walls that
        recieve some light, then we must have a monotone subsequence. Simply
        go to the next wall that is lit up, and keep doing this to obtain
        your subsequence. If not, there is a \textit{tallest} wall that keeps
        all the other walls in the shade. Start there and go to the next
        tallest wall, and then the next tallest wall, and so on, obtaining a
        monotone subsequence. See
        Fig.~\ref{fig:monotone_subsequence_theorem_001}.
        \begin{figure}
            \centering
            \resizebox{\textwidth}{!}{%
            \includegraphics{../../images/monotone_subsequence_theorem_001.pdf}
            }
            \caption{Sketch of Proof of Bolzano's Theorem}
            \label{fig:monotone_subsequence_theorem_001}
        \end{figure}
        \par\hfill\par
        The full proof of this theorem belongs in a
        course on real analysis, but the idea of the proof is essentially the
        idea discussed above. We will use it to prove the
        \textit{Bolzano-Weierstrass theorem}, a core theorem to real analysis
        that completely motivates the idea of compact metric spaces.
        \begin{theorem}[\textbf{Bolzano-Weierstrass Theorem}]
            If $a:\mathbb{N}\rightarrow\mathbb{R}$ is a bounded sequence,
            then there is a convergence subsequence $a_{k}$.
        \end{theorem}
        \begin{proof}
            Let $a:\mathbb{N}\rightarrow\mathbb{R}$ be a bounded sequence.
            By Bolzano's theorem there is a monotone subsequence $a_{k}$.
            Suppose $a_{k}$ is monotone increasing (the idea is symmetric if
            $a_{k}$ is monotone decreasing). Since $a$ is a bounded sequence,
            $a_{k}$ is also a bounded sequence. Let $x\in\mathbb{R}$ be the
            least upper bound of this subsequence. Let $\varepsilon>0$. Since
            $x$ is the least upper bound, $x-\varepsilon$ is not an upper
            bound of the sequence (otherwise $x$ is not the \textit{least}
            upper bound since $x-\varepsilon$ is smaller). But if
            $x-\varepsilon$ is not an upper bound, then there is an
            $N\in\mathbb{N}$ with $a_{k_{N}}>x-\varepsilon$. But $a_{k}$
            is monotone increasing, so for all $n>N$, $a_{k_{n}}\geq{a}_{k_{N}}$
            and therefore $a_{k_{n}}>x-\varepsilon$. But also
            $a_{k_{n}}<x$ since $x$ is the least upper bound. That is,
            for all $n>N$ we have
            $x-\varepsilon<a_{k_{n}}<x$. Therefore, for $n>N$, we have
            $|x-a_{k_{n}}|<\varepsilon$, and hence $a_{k_{n}}$ converges to $x$.
        \end{proof}
        We take this idea and use it to define compactness.
        \begin{definition}[\textbf{Compact Metric Space}]
            A compact metric space is a metric space $(X,\,d)$ such that
            for every sequence $a:\mathbb{N}\rightarrow{X}$ there is a
            convergent subsequence $a_{k}$.
        \end{definition}
\end{document}
