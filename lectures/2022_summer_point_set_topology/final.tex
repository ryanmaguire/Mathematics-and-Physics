%-----------------------------------LICENSE------------------------------------%
%   This file is part of Mathematics-and-Physics.                              %
%                                                                              %
%   Mathematics-and-Physics is free software: you can redistribute it and/or   %
%   modify it under the terms of the GNU General Public License as             %
%   published by the Free Software Foundation, either version 3 of the         %
%   License, or (at your option) any later version.                            %
%                                                                              %
%   Mathematics-and-Physics is distributed in the hope that it will be useful, %
%   but WITHOUT ANY WARRANTY; without even the implied warranty of             %
%   MERCHANTABILITY or FITNESS FOR A PARTICULAR PURPOSE.  See the              %
%   GNU General Public License for more details.                               %
%                                                                              %
%   You should have received a copy of the GNU General Public License along    %
%   with Mathematics-and-Physics.  If not, see <https://www.gnu.org/licenses/>.%
%------------------------------------------------------------------------------%
\documentclass{article}
\usepackage{graphicx}                           % Needed for figures.
\usepackage{amsmath}                            % Needed for align.
\usepackage{amssymb}                            % Needed for mathbb.
\usepackage{amsthm}                             % For the theorem environment.
\usepackage{float}
\usepackage{hyperref}
\hypersetup{
    colorlinks=true,
    linkcolor=blue,
    filecolor=magenta,
    urlcolor=Cerulean,
    citecolor=SkyBlue
}

%------------------------Theorem Styles-------------------------%

% Define theorem style for default spacing and normal font.
\newtheoremstyle{normal}
    {\topsep}               % Amount of space above the theorem.
    {\topsep}               % Amount of space below the theorem.
    {}                      % Font used for body of theorem.
    {}                      % Measure of space to indent.
    {\bfseries}             % Font of the header of the theorem.
    {}                      % Punctuation between head and body.
    {.5em}                  % Space after theorem head.
    {}

% Define default environments.
\theoremstyle{normal}
\newtheorem{problem}{Problem}

\title{Point-Set Topology: Final}
\date{Summer 2022}

% No indent and no paragraph skip.
\setlength{\parindent}{0em}
\setlength{\parskip}{0em}

\begin{document}
    \maketitle
    This test is open book (any of the four recommended texts from the syllabus)
    and open notes (including my notes from the course site), and untimed.
    Collaboration with your fellow students is \textbf{not} allowed, but you
    can email me any clarifying questions.
    \begin{problem}[\textbf{General Topology}]
        \par\hfill\par\vspace{2em}
        \begin{itemize}
            \item
            (2 Points)
            Prove that a topological space $(X,\,\tau)$ is a
            Fr\'{e}chet topological space if and only if for all $x\in{X}$ the
            set $\{\,x\,\}\subseteq{X}$ is closed.
            \item
            (2 Points)
            Let $(X,\,\tau_{X})$ and $(Y,\,\tau_{Y})$ be topological spaces.
            Prove that $f:X\rightarrow{Y}$ is a homeomorphism if and only if
            it is a bijective continuous open mapping.
            \item
            (4 Points) Let $(X,\,\tau_{X})$ and $(Y,\,\tau_{Y})$ be two
            topological spaces and $C(X,\,Y)$ the set of continuous functions
            between them. Prove the relation \textit{homotopic} on $C(X,\,Y)$
            is an equivalence relation. [Hint: You will need the pasting lemma.
            If $\mathcal{C},\mathcal{D}\subseteq{X}$ are closed subspaces that
            cover $X$, if $f:\mathcal{C}\rightarrow{Y}$ and
            $g:\mathcal{D}\rightarrow{Y}$ are continuous, and if
            $f|_{\mathcal{C}\cap\mathcal{D}}=g|_{\mathcal{C}\cap\mathcal{D}}$,
            then the \textit{gluing} function $h:X\rightarrow{Y}$ defined by:
            \begin{equation}
                h(x)=
                \begin{cases}
                    f(x)&x\in\mathcal{C}\\
                    g(x)&x\in\mathcal{D}
                \end{cases}
            \end{equation}
            is continuous. You do not need to prove the pasting lemma.]
        \end{itemize}
    \end{problem}
    \clearpage
    \begin{problem}[\textbf{Separation Properties}]
        \par\hfill\par\vspace{2em}
        \begin{itemize}
            \item
            (2 Points)
            A completely Hausdorff space is a topological space $(X,\,\tau)$
            such that for all distinct points $x,y\in{X}$ there is a continuous
            function $f:X\rightarrow[0,\,1]$ (where $[0,\,1]$ inherits the
            subspace topology from $\mathbb{R}$) such that $f(x)=0$ and
            $f(y)=1$. Prove that a completely Hausdorff space is Hausdorff.
            \item
            (2 Points)
            This does not reverse, in general. The prime integer topology on
            $\mathbb{N}$ is Hausdorff but not completely Hausdorff. Prove that
            if $(X,\,\tau)$ is Hausdorff and normal, then it is completely
            Hausdorff.
            \item
            (4 Points)
            A $G_{\delta}$ set in a topological space
            $(X,\,\tau)$ is a set $A\subseteq{X}$ such that there is a sequence
            $\mathcal{U}:\mathbb{N}\rightarrow\tau$ of open sets such that
            $A=\bigcap_{n\in\mathbb{N}}\mathcal{U}_{n}$. A $G_{\delta}$ space
            is a topological space $(X,\,\tau)$ such that for all closed
            subsets $\mathcal{C}\subseteq{X}$ it is true that
            $\mathcal{C}$ is a $G_{\delta}$ set. Prove that if
            $(X,\,\tau)$ is a normal $G_{\delta}$ space, then it is perfectly
            normal. [Hint: You will need the uniform convergence theorem. Given
            a sequence of continuous functions $f_{n}:X\rightarrow[0,\,1]$,
            the sum $g:X\rightarrow[0,\,1]$ defined by:
            \begin{equation}
                g(x)=\sum_{n=0}^{\infty}\frac{f_{n}(x)}{2^{n+1}}
            \end{equation}
            is continuous. You can use this freely. We also proved that a space
            is perfectly normal if and only if for all closed
            $\mathcal{C}\subseteq{X}$ there is a continuous function
            $h:X\rightarrow[0,\,1]$ such that $\mathcal{C}=h^{-1}[\{\,0\,\}]$.
            That is, you do not need to consider another closed set
            $\mathcal{D}$ and try to simultaneously force
            $\mathcal{D}=h^{-1}[\{\,1\,\}]$ to be true.
            You may also use this freely.]
        \end{itemize}
    \end{problem}
    \clearpage
    \begin{problem}[\textbf{Compactness}]
        \par\hfill\par\vspace{2em}
        \begin{itemize}
            \item
            (2 Points) Let $(X,\,\tau_{X})$ be compact and
            $(Y,\,\tau_{Y})$ Hausdorff. Prove that a continuous bijection
            $f:X\rightarrow{Y}$ is a homeomorphism.
            \item
            (2 Points)
            Prove that if $(X,\,\tau_{X})$ and $(Y,\,\tau_{Y})$ are
            compact, then $(X\times{Y},\,\tau_{X\times{Y}})$ is compact.
            \item
            (2 Points)
            Let $(X,\,\tau_{X})$ be a compact topological space,
            $(Y,\,\tau_{Y})$ a topological space, and $q:X\rightarrow{Y}$ a
            quotient map. Prove that $(Y,\,\tau_{Y})$ is compact.
        \end{itemize}
    \end{problem}
    \clearpage
    \begin{problem}[\textbf{Connectedness}]
        \par\hfill\par\vspace{2em}
        \begin{itemize}
            \item
            (4 Points)
            Prove that a path connected space $(X,\,\tau)$ is connected.
            \item
            (4 Points)
            Prove the intermediate value theorem. If
            $a,b\in\mathbb{R}$, $a<b$, and if $f:[a,\,b]\rightarrow\mathbb{R}$
            is continuous, $f(a)<f(b)$, then for all
            $y\in\big(f(a),\,f(b)\big)$ there is an $x\in(a,\,b)$ such that
            $f(x)=y$.
        \end{itemize}
    \end{problem}
    \clearpage
    \begin{problem}[\textbf{Paracompactness and Manifolds}]
        \item
        (4 Points)
        Prove Dieudonn\'{e}'s theorem. A paracompact Hausdorff space is normal.
        [Hint: We proved paracompact Hausdorff spaces are regular, and then
        used some \textit{hand-waving} due to lack of time to prove paracompact
        Hausdorff spaces are normal. Your job is to generalize the argument for
        regularity and make a formal proof.]
        \item
        (4 Points)
        The torus $\mathbb{T}^{2}$ is the product of the circle with itself.
        Using stereographic projection we covered $\mathbb{S}^{1}$ with
        two charts, meaning $\mathbb{S}^{1}\times\mathbb{S}^{1}$ can be
        covered with four charts. Show that you can cover $\mathbb{T}^{2}$
        with just three charts (you can actually do it with just two). Show
        that you can \textbf{not} cover it with just one chart.
    \end{problem}
\end{document}
