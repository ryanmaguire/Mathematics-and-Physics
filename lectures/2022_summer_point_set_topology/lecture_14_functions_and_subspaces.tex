%-----------------------------------LICENSE------------------------------------%
%   This file is part of Mathematics-and-Physics.                              %
%                                                                              %
%   Mathematics-and-Physics is free software: you can redistribute it and/or   %
%   modify it under the terms of the GNU General Public License as             %
%   published by the Free Software Foundation, either version 3 of the         %
%   License, or (at your option) any later version.                            %
%                                                                              %
%   Mathematics-and-Physics is distributed in the hope that it will be useful, %
%   but WITHOUT ANY WARRANTY; without even the implied warranty of             %
%   MERCHANTABILITY or FITNESS FOR A PARTICULAR PURPOSE.  See the              %
%   GNU General Public License for more details.                               %
%                                                                              %
%   You should have received a copy of the GNU General Public License along    %
%   with Mathematics-and-Physics.  If not, see <https://www.gnu.org/licenses/>.%
%------------------------------------------------------------------------------%
\documentclass{article}
\usepackage{graphicx}                           % Needed for figures.
\usepackage{amsmath}                            % Needed for align.
\usepackage{amssymb}                            % Needed for mathbb.
\usepackage{amsthm}                             % For the theorem environment.
\usepackage{float}
\usepackage{hyperref}
\hypersetup{
    colorlinks=true,
    linkcolor=blue,
    filecolor=magenta,
    urlcolor=Cerulean,
    citecolor=SkyBlue
}

%------------------------Theorem Styles-------------------------%
\theoremstyle{plain}
\newtheorem{theorem}{Theorem}[section]

% Define theorem style for default spacing and normal font.
\newtheoremstyle{normal}
    {\topsep}               % Amount of space above the theorem.
    {\topsep}               % Amount of space below the theorem.
    {}                      % Font used for body of theorem.
    {}                      % Measure of space to indent.
    {\bfseries}             % Font of the header of the theorem.
    {}                      % Punctuation between head and body.
    {.5em}                  % Space after theorem head.
    {}

% Define default environments.
\theoremstyle{normal}
\newtheorem{examplex}{Example}[section]
\newtheorem{definitionx}{Definition}[section]
\newtheorem{notationx}{Notation}[section]
\newtheorem{axiomx}{Axiom}[section]

\newenvironment{example}{%
    \pushQED{\qed}\renewcommand{\qedsymbol}{$\blacksquare$}\examplex%
}{%
    \popQED\endexamplex%
}

\newenvironment{definition}{%
    \pushQED{\qed}\renewcommand{\qedsymbol}{$\blacksquare$}\definitionx%
}{%
    \popQED\enddefinitionx%
}

\title{Point-Set Topology: Lecture 14}
\author{Ryan Maguire}
\date{Summer 2022}

% No indent and no paragraph skip.
\setlength{\parindent}{0em}
\setlength{\parskip}{0em}

\begin{document}
    \maketitle
    \section{Homeomorphisms and Open Mappings}
        Thus far we've discussed conditions for when continuity can be
        described by sequences. It is worthwhile studying the general notion
        of continuity as well. As a reminder, given two topological space
        $(X,\,\tau_{X})$ and $(Y,\,\tau_{Y})$, a continuous function from $X$
        to $Y$ is a function $f:X\rightarrow{Y}$ such that for all
        $\mathcal{V}\in\tau_{Y}$ it is true that
        $f^{-1}[\mathcal{V}]\in\tau_{X}$. That is, the pre-image of an open set
        is open. It was proved this is equivalent to the pre-image of a closed
        set being closed using some of the set-theoretic laws of pre-iamge and
        set difference. \textit{Homeomorphism} is a stronger notion. It tells
        us when two topological spaces are the same.
        \begin{definition}[\textbf{Homeomorphism}]
            A homeomorphism from a topological space $(X,\,\tau_{X})$ to a
            topological space $(Y,\,\tau_{Y})$ is a bijective continuous
            function $f:X\rightarrow{Y}$ such that $f^{-1}$ is continuous.
        \end{definition}
        \begin{example}
            Let $(X,\,\tau)$ be any topological space, and
            $f:X\rightarrow{X}$ be the identity function $f(x)=x$. Then $f$ is
            a homeomorphism. It is certainly a bijection, but it is also
            continuous. Given $\mathcal{U}\in\tau$ we have
            $f^{-1}[\mathcal{U}]=\mathcal{U}$, which is an element of $\tau$.
            Given $\mathcal{V}\in\tau$ we have
            $\big(f^{-1}\big)^{-1}[\mathcal{V}]=f[\mathcal{V}]=\mathcal{V}$,
            which is in $\tau$ (Note:
            $\big(f^{-1}\big)^{-1}[\mathcal{V}]=f[\mathcal{V}]=\mathcal{V}$
            is true since $f$ is a bijection). This shows $f$ is a
            homeomorphism.
        \end{example}
        \begin{example}
            Take $X=Y=\mathbb{R}$ and given both of these the standard Euclidean
            topology $\tau_{\mathbb{R}}$. The function
            $f:\mathbb{R}\rightarrow\mathbb{R}$ defined by $f(x)=x^{3}$ is
            a homeomorphism. It is bijective, continuous, and the inverse is
            given by $f^{-1}(x)=\sqrt[3]{x}=x^{1/3}$, which is also continuous.
        \end{example}
        \begin{example}
            Any continuous bijective function
            $f:\mathbb{R}\rightarrow\mathbb{R}$ is a homeomorphism with respect
            to the standard topology. This is \textbf{not} true for general
            topological spaces. \textbf{It is not true that a continuous}
            \textbf{bijection must have a continuous inverse}. The real line is
            special in this regard. This property comes from the fact that the
            real line has a complete total ordered (via the $<$ symbol). If
            $f:\mathbb{R}\rightarrow\mathbb{R}$ is a continuous bijection, it
            must be strictly increasing or strictly decreasing. If not, if
            there are $a<b<c$ with $f(a)<f(b)$ and $f(c)<f(b)$, or
            $f(b)<f(a)$ and $f(b)<f(c)$, then by the intermediate value theorem
            there must be values $x_{0}\in(a,\,b)$ and $x_{1}\in(b,\,c)$ such
            that $f(x_{0})=f(x_{1})$, violated the fact that $f$ is a
            bijection. Using this you can then show that $f^{-1}$ is also
            continuous.
        \end{example}
        \begin{example}
            Let $X=[0,\,1)$ and $Y=\mathbb{S}^{1}\subseteq\mathbb{R}^{2}$, the
            unit circle. Both of these are metric subspaces of the Euclidean
            spaces $\mathbb{R}$ and $\mathbb{R}^{2}$, respectively, meaning they
            are metric spaces in their own right, and hence topological spaces
            with the induced topology from the subspace metric. The function
            $f:[0,\,1)\rightarrow\mathbb{S}^{1}$ defined by
            $f(t)=\big(\cos(2\pi{t}),\,\sin(2\pi{t})\big)$ is a continuous
            bijection, but it is \textit{not} a homeomorphism. To go from the
            circle to the interval requires \textit{tearing} the circle at a
            point, and this operation is not continuous.
        \end{example}
        We can be more precice in proving that $[0,\,1)$ and $\mathbb{S}^{1}$
        do not have a homeomorphism between them. The idea of
        \textit{compactness} from metric spaces is a notion that is preserved
        by homeomorphisms.
        \begin{theorem}
            If $(X,\,d_{X})$ and $(Y,\,d_{Y})$ are metric spaces, if
            $f:X\rightarrow{Y}$ is a homeomorphism, and if $(X,\,d_{X})$ is
            compact, then $(Y,\,d_{Y})$ is compact.
        \end{theorem}
        \begin{proof}
            For let $b:\mathbb{N}\rightarrow{Y}$ be a sequence. Let
            $a:\mathbb{N}\rightarrow{X}$ be defined by
            $a_{n}=f^{-1}(b_{n})$ (this is well-defined since $f$ is a bijection
            and hence has an inverse). Since $(X,\,d_{X})$ is compact, there
            is a convergent subsequence $a_{k}$. Let $x\in{X}$ be the limit,
            $a_{k_{n}}\rightarrow{x}$. Then, since $f$ is continuous, we have
            $f(a_{k_{n}})\rightarrow{f}(x)$. But $f(a_{k_{n}})=b_{k_{n}}$,
            and hence $b_{k}$ is a convergent subsequence in $Y$, so
            $(Y,\,d_{Y})$ is compact.
        \end{proof}
        The circle $\mathbb{S}^{1}$ is compact by the Heine-Borel theorem since
        it is a closed and bounded subset of $\mathbb{R}^{2}$. The half-open
        interval $[0,\,1)$ is not compact, again by Heine-Borel, since it is
        not closed. Since homeomorphisms preserve compactness, there can be
        no homeomorphism between $[0,\,1)$ and $\mathbb{S}^{1}$.
    \section{Subspaces}
    \section{Topological Embeddings}
\end{document}
