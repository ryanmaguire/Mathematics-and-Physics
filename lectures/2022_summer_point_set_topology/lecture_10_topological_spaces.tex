%-----------------------------------LICENSE------------------------------------%
%   This file is part of Mathematics-and-Physics.                              %
%                                                                              %
%   Mathematics-and-Physics is free software: you can redistribute it and/or   %
%   modify it under the terms of the GNU General Public License as             %
%   published by the Free Software Foundation, either version 3 of the         %
%   License, or (at your option) any later version.                            %
%                                                                              %
%   Mathematics-and-Physics is distributed in the hope that it will be useful, %
%   but WITHOUT ANY WARRANTY; without even the implied warranty of             %
%   MERCHANTABILITY or FITNESS FOR A PARTICULAR PURPOSE.  See the              %
%   GNU General Public License for more details.                               %
%                                                                              %
%   You should have received a copy of the GNU General Public License along    %
%   with Mathematics-and-Physics.  If not, see <https://www.gnu.org/licenses/>.%
%------------------------------------------------------------------------------%
\documentclass{article}
\usepackage{graphicx}                           % Needed for figures.
\usepackage{amsmath}                            % Needed for align.
\usepackage{amssymb}                            % Needed for mathbb.
\usepackage{amsthm}                             % For the theorem environment.
\usepackage{float}
\usepackage{hyperref}
\hypersetup{
    colorlinks=true,
    linkcolor=blue,
    filecolor=magenta,
    urlcolor=Cerulean,
    citecolor=SkyBlue
}

%------------------------Theorem Styles-------------------------%
\theoremstyle{plain}
\newtheorem{theorem}{Theorem}[section]

% Define theorem style for default spacing and normal font.
\newtheoremstyle{normal}
    {\topsep}               % Amount of space above the theorem.
    {\topsep}               % Amount of space below the theorem.
    {}                      % Font used for body of theorem.
    {}                      % Measure of space to indent.
    {\bfseries}             % Font of the header of the theorem.
    {}                      % Punctuation between head and body.
    {.5em}                  % Space after theorem head.
    {}

% Define default environments.
\theoremstyle{normal}
\newtheorem{examplex}{Example}[section]
\newtheorem{definitionx}{Definition}[section]
\newtheorem{notationx}{Notation}[section]
\newtheorem{axiomx}{Axiom}[section]

\newenvironment{example}{%
    \pushQED{\qed}\renewcommand{\qedsymbol}{$\blacksquare$}\examplex%
}{%
    \popQED\endexamplex%
}

\newenvironment{definition}{%
    \pushQED{\qed}\renewcommand{\qedsymbol}{$\blacksquare$}\definitionx%
}{%
    \popQED\enddefinitionx%
}

\title{Point-Set Topology: Lecture 10}
\author{Ryan Maguire}
\date{Summer 2022}

% No indent and no paragraph skip.
\setlength{\parindent}{0em}
\setlength{\parskip}{0em}

\begin{document}
    \maketitle
    \section{Topological Spaces}
        \begin{definition}[\textbf{Topology on a Set}]
            A topology on a set $X$ is a subset $\tau\subseteq\mathcal{P}(X)$
            such that:
            \begin{enumerate}
                \item $\emptyset\in\tau$
                \item $X\in\tau$
                \item For every $\mathcal{O}\subseteq\tau$ it is true that
                    $\bigcup\mathcal{O}\in\tau$
                \item For all $\mathcal{U},\mathcal{V}\in\tau$ it is true that
                    $\mathcal{U}\cap\mathcal{V}\in\tau$.
            \end{enumerate}
            That is, $\tau$ contains the empty set and the whole set, it is
            closed under arbitrary unions, and closed under the
            intersection of two elements.
    \end{definition}
    \begin{definition}[\textbf{Topological Space}]
        A topological space is an ordered pair $(X,\,\tau)$ where $X$ is a set
        and $\tau$ is a topology on $X$.
    \end{definition}
    \begin{example}
        If $X$ is a set, then $\mathcal{P}(X)$, the power set of $X$, is a
        topology on $X$. The power set is trivially closed under arbitrary
        unions and finite intersections, and moreover
        $\emptyset\in\mathcal{P}(X)$ and $X\in\mathcal{P}(X)$. This is
        the \textit{discrete topology} on $X$.
    \end{example}
    \begin{example}
        If $X$ is a set, then the set $\tau=\{\,\emptyset,\,X\,\}$ is a topology
        on $X$. This has several names, the \textit{chaotic topology}, the
        \textit{trivial topology}, and the \textit{indiscrete topology}.
    \end{example}
    \begin{example}
        Take $X=\{\,0,\,1,\,2\,\}$ and
        $\tau=\big\{\,\emptyset,\,\{\,0\,\},\,\{\,0,\,1,\,2\,\}\big\}$. The set
        $\tau$ is a topology on $X$. The sets are all nested since
        $\emptyset\subseteq\{\,0\,\}\subseteq\{\,0,\,1,\,2\,\}$, so it is closed
        under unions and intersections.
    \end{example}
    In metric spaces we used the metric $d$ to define openness. Here, we use
    the topology.
    \begin{definition}[\textbf{Open Set in a Topological Space}]
        An open set in a topological space $(X,\,\tau)$ is an element
        $\mathcal{U}\in\tau$.
    \end{definition}
    In the metric setting we were able to use sequences to define limit points
    and closed sets. This gave us a theorem that closed sets are just the
    complements of open sets. Since we lack a metric, we take this and use it
    to \textit{define} closed sets in a topological space.
    \begin{definition}[\textbf{Closed Set in a Topological Space}]
        A closed set in a topological space $(X,\,d)$ is a set
        $\mathcal{C}\subseteq{X}$ such that $X\setminus\mathcal{C}$ is open.
    \end{definition}
    Topological spaces are direct generalizations of metric spaces. Every
    metric space is also a topological space.
    \begin{theorem}
        If $(X,\,d)$ is a metric space, and if $\tau_{d}$ is the metric
        topology, then $(X,\,\tau_{d})$ is a topological space.
    \end{theorem}
    \begin{proof}
        We have proven that $\emptyset$ and $X$ are metrically open, so
        $\emptyset,X\in\tau_{d}$. Moreover, the intersection of two
        open sets in a metric space is open, as is the union of arbitrarily
        many open sets. That is, $\tau_{d}$ is a topology on $X$, and
        $(X,\,\tau_{d})$ is a topological space.
    \end{proof}
    A natural question is whether or not all topological spaces arise from
    metrics. This is false. The indiscrete topology on a set $X$ containing
    at least two points can't come from a metric. For suppose $X$ is a set
    with $a,b\in{X}$ and $a\ne{b}$. Suppose $d$ is any metric. Setting
    $\varepsilon={d}(a,\,b)/2$, the open ball around $a$ of radius
    $\varepsilon$ is a metrically open subset that contains $a$ but does not
    contain $b$. But in the indiscrete topology $\tau=\{\,\emptyset,\,X\,\}$,
    the only open sets are the empty set (which $B_{\varepsilon}^{(X,\,d)}(a)$
    is not empty since $\varepsilon>0$ and hence
    $a\in{B}_{\varepsilon}^{(X,\,d)}(a)$) and the whole space $X$
    (and $B_{\varepsilon}^{(X,\,d)}(a)\ne{X}$ since
    $b\notin{B}_{\varepsilon}^{(X,\,d)}(a)$). So the indiscrete topology cannot
    come from a metric.
    \par\hfill\par
    One of the problems with the indiscrete topology is that it lacks the
    \textit{Hausdorff property}. In Felix Hausdorff's original definition of
    topological spaces he required points in the space to be able to be
    separated by disjoint open sets. That is, given $x,y\in{X}$ Hausdorrf
    required there to be open sets $\mathcal{U}$ and $\mathcal{V}$ such that
    $x\in\mathcal{U}$, $y\in\mathcal{V}$, and
    $\mathcal{U}\cap\mathcal{V}=\emptyset$. We take Hausdorff's property and
    use it to define \textit{Hausdorff topological spaces}.
    \begin{definition}[\textbf{Hausdorff Topological Space}]
        A Hausdorff topological space is a topological space $(X,\,\tau)$ such
        that for all $x,y\in{X}$ there exists $\mathcal{U},\mathcal{V}\in\tau$
        such that $x\in\mathcal{U}$, $y\in\mathcal{V}$, and
        $\mathcal{U}\cap\mathcal{V}=\emptyset$.
    \end{definition}
    There are reasons we take the more general definition as \textit{the}
    definition of a topological space. There are certain operations that can be
    performed on a topological space (such as glueing points together) that can
    take a Hausdorff space $(X,\,\tau)$ and transform it into a
    non-Hausdroff space (but it'll still be a topological space). Also many
    non-Hausdorff topological spaces have found there way into the mathematical
    world with applications to physics and geometry.
\end{document}
