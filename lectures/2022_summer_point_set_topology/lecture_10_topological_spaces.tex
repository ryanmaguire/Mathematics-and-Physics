%-----------------------------------LICENSE------------------------------------%
%   This file is part of Mathematics-and-Physics.                              %
%                                                                              %
%   Mathematics-and-Physics is free software: you can redistribute it and/or   %
%   modify it under the terms of the GNU General Public License as             %
%   published by the Free Software Foundation, either version 3 of the         %
%   License, or (at your option) any later version.                            %
%                                                                              %
%   Mathematics-and-Physics is distributed in the hope that it will be useful, %
%   but WITHOUT ANY WARRANTY; without even the implied warranty of             %
%   MERCHANTABILITY or FITNESS FOR A PARTICULAR PURPOSE.  See the              %
%   GNU General Public License for more details.                               %
%                                                                              %
%   You should have received a copy of the GNU General Public License along    %
%   with Mathematics-and-Physics.  If not, see <https://www.gnu.org/licenses/>.%
%------------------------------------------------------------------------------%
\documentclass{article}
\usepackage{graphicx}                           % Needed for figures.
\usepackage{amsmath}                            % Needed for align.
\usepackage{amssymb}                            % Needed for mathbb.
\usepackage{amsthm}                             % For the theorem environment.
\usepackage{float}
\usepackage{hyperref}
\hypersetup{
    colorlinks=true,
    linkcolor=blue,
    filecolor=magenta,
    urlcolor=Cerulean,
    citecolor=SkyBlue
}

%------------------------Theorem Styles-------------------------%
\theoremstyle{plain}
\newtheorem{theorem}{Theorem}[section]

% Define theorem style for default spacing and normal font.
\newtheoremstyle{normal}
    {\topsep}               % Amount of space above the theorem.
    {\topsep}               % Amount of space below the theorem.
    {}                      % Font used for body of theorem.
    {}                      % Measure of space to indent.
    {\bfseries}             % Font of the header of the theorem.
    {}                      % Punctuation between head and body.
    {.5em}                  % Space after theorem head.
    {}

% Define default environments.
\theoremstyle{normal}
\newtheorem{examplex}{Example}[section]
\newtheorem{definitionx}{Definition}[section]
\newtheorem{notationx}{Notation}[section]
\newtheorem{axiomx}{Axiom}[section]

\newenvironment{example}{%
    \pushQED{\qed}\renewcommand{\qedsymbol}{$\blacksquare$}\examplex%
}{%
    \popQED\endexamplex%
}

\newenvironment{definition}{%
    \pushQED{\qed}\renewcommand{\qedsymbol}{$\blacksquare$}\definitionx%
}{%
    \popQED\enddefinitionx%
}

\title{Point-Set Topology: Lecture 10}
\author{Ryan Maguire}
\date{Summer 2022}

% No indent and no paragraph skip.
\setlength{\parindent}{0em}
\setlength{\parskip}{0em}

\begin{document}
    \maketitle
    \section{Topological Spaces}
        \begin{definition}[\textbf{Topology on a Set}]
            A topology on a set $X$ is a subset $\tau\subseteq\mathcal{P}(X)$
            such that:
            \begin{enumerate}
                \item $\emptyset\in\tau$
                \item $X\in\tau$
                \item For every $\mathcal{O}\subseteq\tau$ it is true that
                    $\bigcup\mathcal{O}\in\tau$
                \item For all $\mathcal{U},\mathcal{V}\in\tau$ it is true that
                    $\mathcal{U}\cap\mathcal{V}\in\tau$.
            \end{enumerate}
            That is, $\tau$ contains the empty set and the whole set, it is
            closed under arbitrary unions, and the intersection of two elements.
    \end{definition}
    \begin{definition}[\textbf{Topological Space}]
        A topological space is an ordered pair $(X,\,\tau)$ where $X$ is a set
        and $\tau$ is a topology on $X$.
    \end{definition}
    \begin{example}
        If $X$ is a set, then $\mathcal{P}(X)$, the power set of $X$, is a
        topology on $X$. The power set is trivially closed under arbitrary
        unions and finite intersections, and moreover
        $\emptyset\in\mathcal{P}(X)$ and $X\in\mathcal{P}(X)$. This is called
        the \textit{discrete topology} on $X$.
    \end{example}
    \begin{example}
        If $X$ is a set, then the set $\tau=\{\,\emptyset,\,X\,\}$ is a topology
        on $X$. 
    \end{example}
\end{document}
