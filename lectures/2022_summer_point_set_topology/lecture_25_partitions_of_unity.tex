%-----------------------------------LICENSE------------------------------------%
%   This file is part of Mathematics-and-Physics.                              %
%                                                                              %
%   Mathematics-and-Physics is free software: you can redistribute it and/or   %
%   modify it under the terms of the GNU General Public License as             %
%   published by the Free Software Foundation, either version 3 of the         %
%   License, or (at your option) any later version.                            %
%                                                                              %
%   Mathematics-and-Physics is distributed in the hope that it will be useful, %
%   but WITHOUT ANY WARRANTY; without even the implied warranty of             %
%   MERCHANTABILITY or FITNESS FOR A PARTICULAR PURPOSE.  See the              %
%   GNU General Public License for more details.                               %
%                                                                              %
%   You should have received a copy of the GNU General Public License along    %
%   with Mathematics-and-Physics.  If not, see <https://www.gnu.org/licenses/>.%
%------------------------------------------------------------------------------%
\documentclass{article}
\usepackage{graphicx}                           % Needed for figures.
\usepackage{amsmath}                            % Needed for align.
\usepackage{amssymb}                            % Needed for mathbb.
\usepackage{amsthm}                             % For the theorem environment.
\usepackage{float}
\usepackage{tabularx, booktabs}
\usepackage{mathrsfs}
\usepackage[font=scriptsize,
            labelformat=simple,
            labelsep=colon]{subcaption} % Subfigure captions.
\usepackage[font={scriptsize},
            hypcap=true,
            labelsep=colon]{caption}    % Figure captions.
\usepackage{hyperref}
\hypersetup{
    colorlinks=true,
    linkcolor=blue,
    filecolor=magenta,
    urlcolor=Cerulean,
    citecolor=SkyBlue
}

%------------------------Theorem Styles-------------------------%
\theoremstyle{plain}
\newtheorem{theorem}{Theorem}[section]

% Define theorem style for default spacing and normal font.
\newtheoremstyle{normal}
    {\topsep}               % Amount of space above the theorem.
    {\topsep}               % Amount of space below the theorem.
    {}                      % Font used for body of theorem.
    {}                      % Measure of space to indent.
    {\bfseries}             % Font of the header of the theorem.
    {}                      % Punctuation between head and body.
    {.5em}                  % Space after theorem head.
    {}

% Define default environments.
\theoremstyle{normal}
\newtheorem{examplex}{Example}[section]
\newtheorem{definitionx}{Definition}[section]

\newenvironment{example}{%
    \pushQED{\qed}\renewcommand{\qedsymbol}{$\blacksquare$}\examplex%
}{%
    \popQED\endexamplex%
}

\newenvironment{definition}{%
    \pushQED{\qed}\renewcommand{\qedsymbol}{$\blacksquare$}\definitionx%
}{%
    \popQED\enddefinitionx%
}

\title{Point-Set Topology: Lecture 25}
\author{Ryan Maguire}
\date{Summer 2022}

% No indent and no paragraph skip.
\setlength{\parindent}{0em}
\setlength{\parskip}{0em}

\begin{document}
    \maketitle
    \section{Partitions of Unity}
        We're a stone's throw away (no pun intended A. H. Stone) from the
        metrization theorems. It would seem historically that paracompactness
        and local finiteness ideas grew out of the study of metrization
        theorems, but these notions have found enormous use elsewhere. Indeed,
        the use of paracompactness outside of general topology is probably more
        well known than the metrization theorems. In manifold theory and
        geometry paracompactness is important because of its relation to
        \textit{partitions of unity}. Some of the key theorems of differential
        topology (like the Whitney embedding theorem) and Riemannian geometry
        (every smooth manifold is a Riemannian manifold) rely on the fact that
        topological manifolds have partitions of unity. This is a corollary of
        the fact that topological manifolds are paracompact and Hausdorff. In
        this section we introduce the idea and prove some basic theorems.
        \par\hfill\par
        First, some notation. Given two sets $A$ and $B$ we can prove there
        exists a set $\mathcal{F}(A,\,B)$ that is the
        \textit{set of all functions} from $A$ to $B$. Hence if we have two
        topological spaces $(X,\,\tau_{X})$ and $(Y,\,\tau_{Y})$ we can collect
        all \textit{continuous} function from $X$ to $Y$. This is denoted
        $C^{0}(X,\,Y)$, or sometimes just $C(X,\,Y)$. The reason for the
        $0$ is that in some cases (i.e. smooth manifolds and topological vector
        spaces) it is possible to say a function from one topological space to
        another is \textit{differentiable} or \textit{twice differentiable} or
        even \textit{smooth}. We would then use the notations
        $C^{1}(X,\,Y)$,  $C^{2}(X,\,Y)$, and $C^{\infty}(X,\,Y)$, respectively.
        For general topological spaces, however, there is no possible notion
        of derivative, so writing $C(X,\,Y)$ is fine.
        \par\hfill\par
        A word of warning. Some authors use $C(X)$ to denote continuous
        functions from $(X,\,\tau)$ to $\mathbb{R}$, where $\mathbb{R}$ has the
        standard Euclidean topology. Some authors use $C(X)$ to denote
        continuous functions from $(X,\,\tau)$ to $\mathbb{C}$, where
        $\mathbb{C}=\mathbb{R}^{2}$ has the standard topology of the Euclidean
        plane. These notions are not equivalent, and it becomes the
        responsibility of the reader to remember which notation the author is
        using. I won't be doing this. When I want to speak of continuous
        function into $\mathbb{R}$, I'll write $C(X,\,\mathbb{R})$. If I want to
        consider continuous functions into the complex plane, it'll be denoted
        $C(X,\,\mathbb{C})$.
        \begin{definition}[\textbf{Support of a Function}]
            The support of a function $f:X\rightarrow\mathbb{R}$ from a
            topological space $(X,\,\tau)$ to the real line is the set
            $\textrm{supp}_{\tau}(f)\subseteq{X}$ defined by:
            \begin{equation}
                \textrm{supp}_{\tau}(f)
                =\textrm{Cl}_{\tau}\Big(
                    f^{-1}\big[\mathbb{R}\setminus\{\,0\,\}\big]
                \Big)
            \end{equation}
            That is, the closure of the set of all points in $X$ that don't
            map to zero.
        \end{definition}
        We can replace $\mathbb{R}$ with any ring
        (or, more commonly, any field), for those who have studied
        algebra. In analysis one often uses $\mathbb{C}$ instead.
        \begin{definition}[\textbf{Partition of Unity}]
            A partition of unity in a topological space $(X,\,\tau)$ is a
            subset $\mathcal{R}\subseteq{C}(X,\,\mathbb{R})$ of continuous
            functions from $X$ to $\mathbb{R}$ such that:
            \begin{enumerate}
                \item For all $f\in{X}$ and for all $x\in{X}$ it is true that
                    $f(x)\geq{0}$.
                \item The set $\mathcal{O}\subseteq\mathcal{P}(X)$ defined by:
                    \begin{equation}
                        \mathcal{O}=
                        \{\,\textrm{supp}_{\tau}(f)\;|\;f\in\mathcal{R}\,\}
                    \end{equation}
                    is locally finite.
                \item For all $x\in{X}$ we have:
                    \begin{equation}
                        \sum_{f\in\mathcal{R}}f(x)=1
                    \end{equation}
            \end{enumerate}
        \end{definition}
        This last part may be confusing since $\mathcal{R}$ can be uncountably
        big and we've no notion of summing over sets that big
        (It is possible to do this, however. For those who have studied measure
        theory you may recall that if the sum of uncountably many positive
        real numbers converges, then all but countably many are zero). There
        is no issue of convergence because of the second property. Since the
        supports are locally finite, given any $x\in{X}$ there are only finitely
        many $f\in\mathcal{R}$ such that $f(x)\ne{0}$. So the sum in part 3 is
        a finite some for each point.
        \par\hfill\par
        It is useful to attach partitions of unity to open covers. In almost
        all applications we consider partitions of unity that are
        \textit{subordinate} to the cover.
        \begin{definition}[\textbf{Partition of Unity Subordinate to an Open Cover}]
            A partition of unity that is subordinate to an open cover
            $\mathcal{O}$ in a topological space $(X,\,\tau)$ is a partition of
            unity $\mathcal{R}$ such that for all $f\in\mathcal{R}$ there is a
            $\mathcal{U}\in\mathcal{O}$ such that
            $\textrm{supp}_{\tau}(f)\subseteq\mathcal{U}$.
        \end{definition}
        Every paracompact Hausdorff space has a subordinate partition of unity
        for every open cover. Conversely, every Hausdorff space that always
        admits subordinate partitions of unity is paracompact. We will prove
        these two facts in todays notes. The \textit{shrinking lemma} is
        required, which says that paracompact Hausdorff spaces are
        precisely paracompact.
        \begin{definition}[\textbf{Precise Refinement}]
            A precise refinement of a set $\mathcal{A}\subseteq\mathcal{P}(X)$
            in a topological space $(X,\,\tau)$ is a set
            $\mathcal{A}'\subseteq\mathcal{P}(X)$ such that for all
            $A'\in\mathcal{A}'$ there is an $A\in\mathcal{A}$ such that
            $\textrm{Cl}_{\tau}(A')\subseteq{A}$.
        \end{definition}
        With refinements we only required $A'\subseteq{A}$. This allows for the
        possibility that $A=A'$. With precise refinements, unless your sets
        under consideration happen to be closed, to make
        $\textrm{Cl}_{\tau}(A')\subseteq{A}$ work requires
        \textit{shrinking} $A$ a little.
        \begin{definition}[\textbf{Precisely Paracompact Topological Space}]
            A precisely paracompact topological space is a topological space
            $(X,\,\tau)$ such that for all open covers
            $\mathcal{O}\subseteq\tau$ of $X$ there exists a precise locally
            finite open refinement $\mathcal{X}$ of $\mathcal{O}$ that covers
            $X$.
        \end{definition}
        \begin{theorem}[\textbf{The Shrinking Lemma}]
            If $(X,\,\tau)$ is paracompact and Hausdorff, then it is
            precisely paracompact.
        \end{theorem}
        \begin{proof}
            For if not then there is an open cover $\mathcal{O}\subseteq\tau$
            with no precise locally finite open refinement that covers $X$. But
            $(X,\,\tau)$ is paracompact so there is a locally finite open
            refinement $\mathcal{X}$ of $\mathcal{O}$. But $(X,\,\tau)$ is
            Hausdorff and paracompact, so it is regular. Hence for all $x\in{X}$
            and for all $\mathcal{U}\in\tau$ with $x\in\mathcal{U}$ there is a
            $\mathcal{V}\in\tau$ such that $x\in\mathcal{V}$ and
            $\textrm{Cl}_{\tau}(\mathcal{V})\subseteq\mathcal{U}$. Using this,
            since $\mathcal{X}$ is an open cover of $X$, for all $x\in{X}$ there
            is a $\mathcal{U}_{x}\in\mathcal{X}$ such that
            $x\in\mathcal{U}_{x}$. Let $\mathcal{V}_{x}\in\tau$ be such that
            $x\in\mathcal{V}_{x}$ and
            $\textrm{Cl}_{\tau}(\mathcal{V}_{x})\subseteq\mathcal{U}_{x}$.
            Define $\tilde{\mathcal{O}}$ via:
            \begin{equation}
                \tilde{\mathcal{O}}=
                \{\,\mathcal{V}_{x}\;|\;x\in{X}\,\}
            \end{equation}
            This is now a precise open refinement of $\mathcal{O}$, but it is
            probably not locally finite (we've picked an open set for
            \textit{every} point in the space). But $\tilde{\mathcal{O}}$ is an
            open cover of $X$, given $x\in{X}$ it is contained in
            $\mathcal{V}_{x}$ which is an element of $\tilde{\mathcal{O}}$.
            Since $(X,\,\tau)$ is paracompact there is a locally finite open
            refinement $\tilde{\mathcal{X}}$ of $\tilde{\mathcal{O}}$ that is
            an open cover of $X$. But then for all
            $\mathcal{U}\in\tilde{\mathcal{X}}$ there is a
            $\mathcal{V}\in\tilde{\mathcal{O}}$ such that
            $\mathcal{U}\subseteq\mathcal{V}$. But by definition of
            $\tilde{\mathcal{O}}$ there is a
            $\mathcal{W}\in\mathcal{X}$ such that
            $\textrm{Cl}_{\tau}(\mathcal{V})\subseteq\mathcal{W}$. But
            $\mathcal{X}$ is refinement of $\mathcal{O}$, so there is a
            $\mathcal{E}\in\mathcal{O}$ such that
            $\mathcal{W}\subseteq\mathcal{E}$. But then:
            \begin{equation}
                \textrm{Cl}_{\tau}(\mathcal{V})
                \subseteq\textrm{Cl}_{\tau}(\mathcal{U})
                \subseteq\mathcal{W}
                \subseteq\mathcal{E}
            \end{equation}
            and hence $\textrm{Cl}_{\tau}(\mathcal{V})\subseteq\mathcal{E}$.
            But then $\tilde{\mathcal{X}}$ is a precise locally finite open
            refinement of $\mathcal{O}$ that covers $X$, a contradiction. Hence,
            $(X,\,\tau)$ is precisely paracompact.
        \end{proof}
        \begin{theorem}
            If $(X,\,\tau)$ is a Hausdorff topological space, then it is
            paracompact if and only if every open cover has a subordinate
            partition of unity.
        \end{theorem}
        \begin{proof}
            Suppose $(X,\,\tau)$ is Hausdorff and every open cover has a
            subordinate partition of unity. Suppose it is not paracompact,
            meaning there is an open cover $\mathcal{O}\subseteq\tau$ with no
            locally finite open refinement that covers $X$. But since
            $\mathcal{O}$ is an open cover, by hypothesis there is a
            subordinate partition of unity
            $\mathcal{R}\subseteq{C}(X,\,\mathbb{R})$. Define $\mathcal{X}$ via:
            \begin{equation}
                \mathcal{X}=
                \big\{\,f^{-1}\big[\mathbb{R}\setminus\{\,0\,\}\big]
                    \;|\;f\in\mathcal{R}\,\}
            \end{equation}
            Since each element of $\mathcal{R}$ is continuous, and since
            $\mathbb{R}\setminus\{\,0\,\}$ is open, the elements of
            $\mathcal{X}$ are open. $\mathcal{X}$ also covers $X$ since given
            $x\in{X}$, since:
            \begin{equation}
                \sum_{f\in\mathcal{R}}f(x)=1
            \end{equation}
            there must be some $f\in\mathcal{R}$ such that $f(x)\ne{0}$. But
            then $x\in{f}^{-1}\big[\mathbb{R}\setminus\{\,0\,\}\big]$, meaning
            $\mathcal{X}$ is an open cover. It is a refinement of $\mathcal{O}$
            since given $\mathcal{U}\in\mathcal{X}$ we have that there is some
            $f\in\mathcal{R}$ such that
            $\mathcal{U}=f^{-1}\big[\mathbb{R}\setminus\{\,0\,\}\big]$. But
            since $\mathcal{R}$ is subordinate to $\mathcal{O}$ there is a
            $\mathcal{V}\in\mathcal{V}$ such that
            $\textrm{supp}_{\tau}(f)\subseteq\mathcal{V}$. But then:
            \begin{equation}
                \mathcal{U}=f^{-1}\big[\mathbb{R}\setminus\{\,0\,\}\big]
                \subseteq\textrm{Cl}_{\tau}\Big(
                    f^{-1}\big[\mathbb{R}\setminus\{\,0\,\}\big]\Big)
                =\textrm{supp}_{\tau}(f)
                \subseteq\mathcal{V}
            \end{equation}
            and hence $\mathcal{U}\subseteq\mathcal{V}$. Lastly, $\mathcal{X}$
            is locally finite. For let $x\in{X}$. Since $\mathcal{R}$ is a
            partition of unity there is a $\mathcal{U}\in\tau$ such that
            $x\in\mathcal{U}$ and only finitely many $f\in\mathcal{R}$ are
            such that $\textrm{supp}_{\tau}(f)\cap\mathcal{U}$ is non-empty.
            But $f^{-1}\big[\mathbb{R}\setminus\{\,0\,\}\big]\subseteq\textrm{supp}_{\tau}(f)$,
            meaning only finitely many elements of $\mathcal{X}$ are such that
            $f^{-1}[\mathbb{R}\setminus\{\,0\,\}\big]$ intersects $\mathcal{U}$,
            and hence $\mathcal{X}$ is locally finite. So $\mathcal{X}$ is a
            locally finite open refinement of $\mathcal{O}$ that covers $X$,
            a contradiction. So $(X,\,\tau)$ is paracompact.
            Now, suppose $(X,\,\tau)$ is paracompact and Hausdorff. Let
            $\mathcal{O}\subseteq\tau$ be an open cover of $X$. Since
            $(X,\,\tau)$ is paracompact and Hausdorff, it is precisely
            paracompact, so there is a precise locally finite open refinement
            $\mathcal{X}_{0}$ of $\mathcal{O}$ that covers $X$. But then
            $\mathcal{X}_{0}$ is an open cover of $X$, and since $(X,\,\tau)$
            is precisely paracompact there is a precise locally finite open
            refinement $\mathcal{X}_{1}$ of $\mathcal{X}_{0}$. Then for all
            $\mathcal{U}_{1}\in\mathcal{X}_{1}$ there is a
            $\mathcal{U}_{0}\in\mathcal{X}_{0}$ and a
            $\mathcal{U}\in\mathcal{O}$ such that
            $\textrm{Cl}_{\tau}(\mathcal{U}_{1})\subseteq\mathcal{U}_{0}$ and
            $\textrm{Cl}_{\tau}(\mathcal{U}_{0})\subseteq\mathcal{U}$. But
            $\textrm{Cl}_{\tau}(\mathcal{U}_{1})$ is closed, and since
            $\mathcal{U}_{0}$ is open, $X\setminus\mathcal{U}_{0}$ is closed.
            But $\textrm{Cl}_{\tau}(\mathcal{U}_{1})\subseteq\mathcal{U}_{0}$
            so $X\setminus\mathcal{U}_{0}$ and
            $\textrm{Cl}_{\tau}(\mathcal{U}_{1})$ are disjoint. But since
            $(X,\,\tau)$ is paracompact and Hausdorff, by Dieudonn\'{e}'s
            theorem it is normal. But then by Urysohn's lemma, since
            $\textrm{Cl}_{\tau}(\mathcal{U}_{1})$ and
            $X\setminus\mathcal{U}_{0}$ are closed and disjoint, there is a
            continuous function $f_{\mathcal{U}_{1}}:X\rightarrow[0,\,1]$ such
            that
            $f_{\mathcal{U}_{1}}\big[\textrm{Cl}_{\tau}(\mathcal{U}_{1})\big]=\{\,1\,\}$
            and
            $f_{\mathcal{U}_{1}}\big[X\setminus\mathcal{U}_{0}]=\{\,0\,\}$.
            That is, $f_{\mathcal{U}_{1}}$ is continuous, evaluates to $1$ on
            the closure of $\mathcal{U}_{1}$, zero outside of $\mathcal{U}_{0}$,
            and some values between $0$ and $1$ for points in
            $\mathcal{U}_{0}\setminus\textrm{Cl}_{\tau}(\mathcal{U}_{1})$. The
            set:
            \begin{equation}
                \tilde{\mathcal{R}}=
                \{\,f_{\mathcal{U}_{1}}\;|\;
                    \mathcal{U}_{1}\in\mathcal{X}_{1}\,\}
            \end{equation}
            is almost a partition of unity. Since Urysohn's lemma restricts the
            range of the functions to $[0,\,1]$, we have $f(x)\geq{0}$ for all
            $x\in{X}$ and $f\in\tilde{\mathcal{R}}$. The supports are locally
            finite, since the support of
            $f_{\mathcal{U}_{1}}$ is $\textrm{Cl}_{\tau}(\mathcal{U}_{0})$,
            which is locally finite. The last thing we need is that the sums at
            all points evaluate to zero. $\tilde{\mathcal{R}}$ almost certainly
            does not have this property. Given $x\in{X}$, since
            $\mathcal{X}_{1}$ is an open cover of $X$, there is a
            $\mathcal{U}_{1}\in\mathcal{X}_{1}$ such that $x\in\mathcal{U}_{1}$.
            But then $f_{\mathcal{U}_{1}}(x)=1$. Hence the function
            $\Phi:X\rightarrow\mathbb{R}^{+}$ defined by:
            \begin{equation}
                \Phi(x)=\sum_{f\in\tilde{\mathcal{R}}}f(x)
            \end{equation}
            is positive for all $x\in{X}$. It is also continuous since locally
            it is the finite sum of continuous functions. Since $\Phi(x)$ is
            never zero we can define:
            \begin{equation}
                g_{\mathcal{U}_{1}}(x)
                =\frac{f_{\mathcal{U}_{1}}(x)}{\Phi(x)}
            \end{equation}
            The set:
            \begin{equation}
                \mathcal{R}
                =\{\,g_{\mathcal{U}_{1}}\;|\;
                    \mathcal{U}_{1}\in\mathcal{X}_{1}\,\}
            \end{equation}
            is then a partition of unity subordinate to $\mathcal{O}$,
            contradicting the claim that none such partition of unity exists.
            Hence, if $(X,\,\tau)$ is a paracompact Hausdorff space, then
            every open cover has a subordinate partition of unity.
        \end{proof}
        That's about all we'll say about partitions of unity, as far as
        point-set topology is concerned. In the world of differential topology
        and geometry there are two important theorems that are worth noting.
        A topological manifold is a topology space $(X,\,\tau)$ the is
        Hausdorff, second countable, and locally Euclidean. This last property
        means for all $x\in{X}$ there is a $\mathcal{U}\in\tau$ such that
        $x\in\mathcal{U}$ and $(\mathcal{U},\,\tau_{\mathcal{U}})$ is, as a
        subspace, homeomorphic to $\mathbb{R}^{n}$ for some $n\in\mathbb{N}$.
        \par\hfill\par
        A smooth manifold is a topological manifold with a
        \textit{smooth structure} which makes it possible to talk about the
        derivatives of functions, tangent vector, and vector fields, topics
        common in a multivariable calculus course. A Riemannian manifold is a
        smooth manifold with a \textit{Riemannian metric} which makes it
        possible to measure angles of tangent vectors and compute the lengths
        of curves. Notions like geodesics and parallel transport occur in
        Riemannian geometry. So every Riemannian manifold is a smooth manifold
        and every smooth manifold is a topological manifold. Topological
        manifolds are paracompact and Hausdorff, so every open cover has a
        subordinate partition of unity. For smooth manifolds one can improve
        this. Every open cover in a smooth manifold $(X,\,\tau)$ has a
        subordinate partition of unity $\mathcal{R}$ where the functions
        $f\in\mathcal{R}$ can be taken to be \textit{smooth}
        (which now makes sense since in smooth manifolds we can take
        derivatives). Using this it can be shown (without much work,
        surprisingly) that every smooth manifold can be made into a Riemannian
        manifold. It can also be shown that every smooth manifold of dimension
        $n\in\mathbb{N}$ is actually just a topological subspace of
        $\mathbb{R}^{2n+1}$ (one can improve this to $\mathbb{R}^{2n}$, but
        this is harder).
\end{document}
